\documentclass[output=paper]{langscibook}
\ChapterDOI{10.5281/zenodo.17132447}
\author{Friederike Lüpke\orcid{}\affiliation{Helsinki Collegium for Advanced Studies}}
\title[(M)other tongue]{(M)other tongue: How Global South multilingual practices allow uncovering multilingualisms beneath the invented monolingual European Self}
\abstract{This paper, part of a collection on the Othering of multilinguals in linguistics, explores Othering from the perspective of Global South societies, in which multilingualism is commonly perceived through Eurocentric lenses prevalent in cognitive science and linguistics. The paper critically examines these viewpoints through describing the effects of processes of Othering, focusing particularly on the invention of the European Self as a crucial component of Othering and drawing on critical theory and postcolonial studies, especially the works of Édouard Glissant and Gayatri Spivak.
The first process of Othering involves describing the non-European Other using Eurocentric concepts, whose applicability to Global South contexts needs to be challenged. However, merely critiquing these concepts within the Self-Other framework fails to address the need to deconstruct the European Self, invented as part of the Othering process. The constructedness of the European Self itself emphasises the need for a paradigm shift in linguistics that acknowledges the inadequacy of current analytical frameworks. This need is compounded by the effects of Othering on the self-perception of the Other, which compels them to perceive themselves through external lenses, to contribute to the colonial library and to perpetuate colonial viewpoints. The paper advocates a recalibration of ontologies, epistemologies, and methodologies in the description of multilingualism across all settings. By dismantling the notion of the monolingual European Self, the road is paved towards normalising dynamic and fluid multilingualisms, leading to a more comprehensive understanding of multilingualism worldwide based on convivial research paradigms.}
\IfFileExists{../localcommands.tex}{
  \addbibresource{../localbibliography.bib}
  % add all extra packages you need to load to this file

\usepackage{tabularx,multicol}
\usepackage{url}
\urlstyle{same}

\usepackage{listings}
\lstset{basicstyle=\ttfamily,tabsize=2,breaklines=true}

\usepackage{langsci-basic}
\usepackage{langsci-optional}
\usepackage{langsci-lgr}
\usepackage{langsci-osl}
% \usepackage{./langsci/styles/langsci-lgr}
% \usepackage{./langsci/styles/langsci-osl}
% \usepackage{langsci-gb4e}

\usepackage{tikz}
\usetikzlibrary{patterns,calc}
\pgfdeclarepatternformonly{south east lines}{\pgfqpoint{-0pt}{-0pt}}{\pgfqpoint{3pt}{3pt}}{\pgfqpoint{3pt}{3pt}}{
    \pgfsetlinewidth{0.6pt}
    \pgfpathmoveto{\pgfqpoint{0pt}{3pt}}
    \pgfpathlineto{\pgfqpoint{3pt}{0pt}}
    \pgfpathmoveto{\pgfqpoint{.2pt}{-.2pt}}
    \pgfpathlineto{\pgfqpoint{-.2pt}{.2pt}}
    \pgfpathmoveto{\pgfqpoint{3.2pt}{2.8pt}}
    \pgfpathlineto{\pgfqpoint{2.8pt}{3.2pt}}
    \pgfusepath{stroke}}
    
\usepackage{stmaryrd}
\usepackage{wasysym}
\usepackage{multirow}
\usepackage{caption}
\usepackage{subcaption}
\usepackage{mathrsfs}
\usepackage{qtree}

\usepackage{linguex}


  %pminos do not split footnotes
% \interfootnotelinepenalty=10000 %Footnote in Laporte chapters has to be split SN


%\DeclareIndexNameFormat{default}{%
%\nameparts{#1}%
%\usebibmacro{index:name}%
%{\index[names]}%
%{\namepartfamily}%
%{\namepartgiveni}%
% {}% L1
% {}% L2
%{\namepartprefix}% generates spurious space L3
%{\namepartsuffix}% generates spurious space L4
%}

%  {\DeclareIndexNameFormat{default}{%
%     \usebibmacro{index:name}{\index[names]}{#1}{#3}{#5}{#7}}}

%\DeclareIndexNameFormat{default}{%
%  \usebibmacro{index:name}{\sindex[nom]}{#1}{#3}{#5}{#7}}

%\DeclareIndexNameFormat{default}{%
%  \usebibmacro{index:name}{\sindex[person]}{#1}{#3}{#5}{#7}}
%\DeclareIndexNameFormat{default}{%
%\nameparts{#1} \usebibmacro{index:name}{\sindex[person]]}{\namepartfamily}{‌​\namepartgiven}{\nam‌​epartprefix}{\namepa‌​rtsuffix}}

%\newcommand{\smiley}{:)}

%\renewbibmacro*{index:name}[5]{%
%\usebibmacro{index:entry}{#1}%
%{\iffieldundef{usera}{}{\thefield{usera}\actualoperator}\mkbibindexname{#2}{#3}{#4}{#5}}}

% \newcommand{\noop}[1]{}

%remove for final
%\overfullrule=1mm

\newcommand{\tobi}[2]}}
\renewcommand{\S}[1]{\tobi{#1}{\textsc{*}}}

% this volume references
% puts: [this volume]
% already defined: \citetv
%\newcommand{\citepv}[1]{(\citeauthor{#1} \citeyear*{#1} [this volume])}
\newcommand{\citealtv}[1]{\citeauthor{#1} \citeyear*{#1} [this volume]}

%parentheses around example number
\newcommand{\pref}[1]{(\ref{#1})}

% in-text examples

\newcommand{\lnex}[1]{\textit{#1}} %target lang word
\newcommand{\lnlit}[1]{(lit.: `#1')} %literal reading
\newcommand{\lnlat}[1]{(#1)} % latinization
\newcommand{\lntrans}[1]{`#1'} %translation
\newcommand{\lnexl}[2]%
{\lnex{#1}{} \lnlat{#2}} % ex with latinization
\newcommand{\lnexlat}[3]{\lnex{#1}{} \lnlat{#2}{} \lntrans{#3}} % ex with latinization and tranl.

%ch01
\newcommand{\co}[1]{\mbox{\textbf{#1}}}

%ch09

\newcommand{\cyrbulg}[1]{\begin{otherlanguage*}{bulgarian}#1\end{otherlanguage*}}


%ch10
\newcommand{\nlp}{{\small NLP}}
\newcommand{\mwe}{{\small MWE}}
\newcommand{\rae}{{\small RAE}}
\newcommand{\lvc}{{\small LVC}}
\newcommand{\pos}{{\small P}o{\small S}}
%\newcommand{\todo}[1]{ \textcolor{red}{#1} }

%\renewcommand{\labelenumi}{\theenumi}
%\ainamefmt{{vv}{ll}{, ff}{, jj}} % fullname

\newcommand{\biberror}[1]{{\color{red}#1}}

\newcommand{\osenovaitem}{--~} 
  %% hyphenation points for line breaks
%% Normally, automatic hyphenation in LaTeX is very good
%% If a word is mis-hyphenated, add it to this file
%%
%% add information to TeX file before \begin{document} with:
%% %% hyphenation points for line breaks
%% Normally, automatic hyphenation in LaTeX is very good
%% If a word is mis-hyphenated, add it to this file
%%
%% add information to TeX file before \begin{document} with:
%% %% hyphenation points for line breaks
%% Normally, automatic hyphenation in LaTeX is very good
%% If a word is mis-hyphenated, add it to this file
%%
%% add information to TeX file before \begin{document} with:
%% \include{localhyphenation}
\hyphenation{
    Beck-man
    Ngu-yen
    back-chan-nel
    back-chan-nels
    mo-not-o-nous
    ste-reo-typ-i-cal
}

\hyphenation{
    Beck-man
    Ngu-yen
    back-chan-nel
    back-chan-nels
    mo-not-o-nous
    ste-reo-typ-i-cal
}

\hyphenation{
    Beck-man
    Ngu-yen
    back-chan-nel
    back-chan-nels
    mo-not-o-nous
    ste-reo-typ-i-cal
}
 
  \togglepaper[1]%%chapternumber
}{}

\begin{document}
\maketitle 
%\shorttitlerunninghead{}%%use this for an abridged title in the page headers




\section{Introduction: From creating and describing difference to pluriversal perspectives}
\begin{quote}
(l)'Occident n'est pas à l'Ouest, ce n'est pas un lieu, c'est un projet. (The West is not in the West. It is a project, not a place.) \hfill  \parencite[14]{Glissant1981}
\end{quote}

Among the contributions to this collection on Othering of multilinguals in linguistics, this essay is the only one covering geographical settings outside Europe or the Global North. When conceiving it, I struggled to find an angle that would capture salient traits of the hugely diverse multilingual societies in the Global South in non-Othering fashion and  in a short chapter. I found it through looking at all processes associated with Othering and by exploring alternative ways of identity formation and knowledge creation in the Global North and in the Global South. Through investigating in particular one process of Othering, that of the invention of the Self, this endeavour becomes relevant to research in and on Europe and the Global North.

It is now widely accepted that Northern/Western perspectives dominate cognitive science and linguistics. Multilingualism is for its most parts perceived through a lens that takes European monolingually biased ideas of language\footnote{Terms denoting languages and speakers are based on the images of language as an oral medium and often exclude language users communicating in other modalities, for instance users of sign languages. For want of a more inclusive term (even terms denoting sign languages and sign language linguistics are based on a tongue-based metaphor in many languages) I use these terms, but with the explicit intention of including all language users, notably sign language users.} and of multilingualism as the benchmark. This thought process has been described as Othering in philosophy. Through Othering, both Self and Other are constituted based on construing differences between the two, which are, crucially, described in terms of notions grounded in the ideologies of the Self. I draw particularly on notions of Othering as developed in critical theory and postcolonial studies, most notably in the works of Édouard Glissant and Gayatri Spivak. 


Othering in colonial contexts creates the Other as the subaltern native and imposes the epistemological order of the invented European Self on them. The process of Othering is complex and has multiple consequences. In this paper, I focus on the second and third, often overlooked parts of this process and their results. But let me introduce the most widely discussed process of Othering before returning to its other parts and their consequences. First, the non-European Other is described in terms of concepts taken from the ``colonial library“ \parencite[]{Mudimbe1988}  or compared with an imagined unmarked European state of affairs. It is important to critique the validity of these concepts for Global South contexts. At the same time, such critiques do not go beyond the dichotomy Self-Other. Remaining in this framework entails a mere ‘talking back’ to the hegemonic place of theory building, through pointing out differences. I have done so myself in previous research \parencite[]{Luepke2015, Luepke2016, Luepke2016a, Luepke2017, Luepke2018, Luepke2021, Luepke2021a}, which was limited to pointing out the inapplicability of dominant concepts of language, identity and endangerment, to African social contexts.  Through limiting the discussion to the unsuitability of concepts in particular sociopolitical contexts and the pointing out of different practices and language ideas, the European Self is left untouched. Such research unwittingly contributes to the perseverance of a constructed Europe with which other world regions can be contrasted and to exoticising and thus maintaining the Other. My critical stance in this paper is therefore one of self-reflexive practice. 

The second process of Othering goes further than applying Eurocentric concepts in Other places. It refers to the construction of the European Self, building the project of the West, in the process of creating the Other, through constructing the Self as an idealised state of affairs, a deontic modality rather than an appraisal of situations. ‘Europe’, ‘European’, ‘the West’ and ‘Western civilisation’ are canonised idealisations of diverse and dynamic sociopolitical configurations that only emerge from and are ideologically motivated by a desired contrast with the Other. Without the Other, Europe and related notions that are dialectally created cannot stand and give way to complex, entangled multitudes. Decolonial perspectives therefore cannot be limited to deconstructing the former Other; they need to extend critical perspectives to dismantling the ontologies and epistemologies that created the European Self as well. Such lenses enable empirical research combining universal concerns with opennes for the real diversity of situations, wherever they are located. Salikoko \citet[299--300]{Mufwene2020} states succinctly what is needed in order to overcome such Othering in order to develop truly decolonial outlooks in research as follows:

\begin{quote}
However, we should not just present data from, or identify phenomena in, languages in the global South that have traditionally not been accounted for within the same analysis paradigms that have been used to date. We must also show what analytical changes are needed to understand the data and phenomena more adequately. Otherwise, we are still trapped in the conceptual framework and research methods provided by the same ‘colonial linguistics’. Since similar data, phenomena, or speakers’/writers’ behaviours may show up in any part of the world, including the former colonial metropoles, a paradigm shift is in order for a productive decolonial linguistics. The proposed shift […] does not, of course, mean rejecting everything learned from research conducted to date. It entails sorting things out and determining what must be corrected for a better, plural linguistics that sheds light both on universals and on typological peculiarities in ways that do justice to the data, phenomena, and speakers’/writers’ behaviours. The purpose of decolonizing linguistics should not be to foster a global South exceptionalism but to simply account more adequately for the relevant phenomena in ways that improve the practice of linguistics as a discipline and capture diversity more adequately. 
\end{quote}

The third part of Othering is that of forcing the Others to see themselves through external lenses so that they contribute to the colonial library by providing accounts on experiences cast in terminologies and ontologies exterior to them, evident in the persistent coloniality of formal education systems in postcolonial contexts which reproduce such lenses \parencite[]{Mbembe}.

Language ideas dominant in public imagination, institutions and the field of linguistics were formed as the outcome of a centuries-long process of consolidation of ethnoromantic notions of identity culminating in ethnolinguistic nation building, as the contributions to this volume critically discuss. Research on and from the Global South, even if conducted in frameworks that inherited much of this ethnonationalist library, has questioned its basic tenets, uncovering practices and concepts of language that, when put in dialogue with Northern language ideas, enables their rethinking also for these settings. A logical next step is therefore to connect research on alternative language ideas and practice in the Global South to real multilingual complexity and dynamics of language use in Europe, which cannot be described using concepts from the ethnonationalist library either. In this paper, although I mainly reference African and other Global South multilingual settings, I do not do so with the view of describing them as different from the outset. A catalogue of alternative language ideas and practices ‘elsewhere’ would remain inscribed into the logics of Othering, which demands of the Other to conform to the same epistemic scale, even if this scale is extended \parencite[]{Glissant1990}. 

Rather, my aim is to extend an invitation to recalibrate and renew ontologies, epistemologies and methodologies everywhere. All configurations, wherever they are located, benefit from  new perspectives that are not based on primordial differences and  defy the logic of Othering through being mindful of conviality and multiple connections within and between settings. This step contributes to normalising complex multilingualisms beneath the constructed monolingual European Self that consequently becomes dismantled and no longer seen in contrast to an Other. Such multilingualisms also comprise marginalised and minoritised practices impossible to reconcile with idealised monolingual experiences in Global North societies, from so-called heritage languages and new speakers to the use of sign languages and all forms of language use deemed deviant from the perspective of monolingual standard forms of language \parencite[]{FloresRosa2015, MeulderEtAl2019, MoriartyKusters2021,SnellCushing2022, Wiese2014, Wiese2015}.

 In the remainder of this paper, I define central facets of Othering of multilinguals in linguistics and sketch alternative frameworks that can overcome Othering in Section \REF{05:sec:2}. In Section \REF{05:sec:3}, I illustrate how dynamic relationships with people, places and named languages, rather than absolute and fixed roles for languages, characterise linguistic experiences and how recognising this dynamicity can enrich theorisations of multilingualism. In Section \REF{05:sec:4}, I develop ideas for future research that can contribute to making multilingualism research truly global.
 
\section{Othering and alternative frameworks}
\label{05:sec:2}
The terms ‘Othering’ and the ‘Other’ are here used as developed by Gayatri \citet[253]{Spivak1985} in a classic paper on the subject. She evokes Othering as a multidimensional process of social differentiation which resides on the colonial sovereign/Self ``consolidating the self of Europe by constructing it as the sovereign subject and obliging the native to cathect the space of the Other on his home ground.''
To paraphrase, Othering operates through constructing an idea of Europe and imposing it on subjects perceived as different, while simultaneously imposing on them to become invested in these categories, to see themselves as Others. The process unfolds through making ’natives’ perceive themselves through ‘worlding’ practices of the outsider, thus reproducing this external perspective; through writing narratives that, even if they seemingly restore rights, for instance to indigenous languages, frame them as obligations, e.g., in seeing these languages as deficient if not standardised, equipped with a written form and used in formal education. Simultaneously, this process recasts deontic desire of ordered states of affairs – for instance that everybody should have one language that expresses their identity – as law: people have one mother tongue. As hinted at with these examples, in the realm of linguistics, we can illustrate these three Othering processes through the imposition of ontologies and epistemologies residing on European ethnonationalist concepts. To these belong notions such as ‘mother tongue’, the interpretation of identity concepts through ethnonational lenses, as done with ‘ethnicity’; or language right discourses that, though seemingly aiming at protecting linguistic diversity create the obligation to do so on the terms of institutions grounded in ethnonationalism. The perception of language ideologies – such as the one that societies are monolingual, as portraying realities, rather than socio-politically motivated ideals also belongs to these Othering mechanisms. 

\largerpage
Crucial for Spivak’s thinking is that Othering is not simply a process that creates the other through the imposition of European perspectives and worldviews, but also the very process through which Europe consolidates its Self. European ontologies imposed on the Other are thus not to be understood as capturing with any degree of verisimilitude practices on the European continent itself; they are the outcome of language-ideological processes in various European projects of nation building, mirrored in the colonial projects of (mini)nation building. Thus, we need to distinguish between a), Europe, the continent in the sense of an internally complex geographic location of shifting contours and oscillating internal composition, and b), Europe or the West as an ongoing project of narrative creation (in the sense of \citet[]{Glissant1981}). These narratives result in the erasure of practices and speakers disturbing this desired order and in the iconisation of uses and people conform to it \parencite[]{GalIrvine2019}. It is therefore not enough to point out the inapplicability of categories used to describe the Other and to replace them with different notions then contrasted with European concepts. Rather, letting go of Othering also entails that the European Self cannot stand, as explored in several contributions to this volume. However, it is not desirable to simply turn the tables, to make Europe the Other of Global South societies, as Spivak observes. This would be not only impossible but also just result in an inverse exoticisation based on an ideological project of constructing a generisable Global South as Self that then can eroticise the Global North. Recalibrations and enrichments of ideas and concepts of multilingualism are consequently not suggested with the idea of creating Europe as a new Other.\footnote{A radical alternative would be, as suggested by an anonymous reviewer, ``to think about Othering differently; not as the process of creating the Other that is distinct from the Self but the Other who is the other Self''. This invites, as also pondered by the reviewer, to see conviviality not necessarily as an antithesis of Othering, but perhaps to develop a convivial view of the Other. I am very open to explore such a conceptualisation but do not address it in this chapter.} I am particularly inspired by the concepts of relationality and rhizomic networks \parencite[]{Glissant1990}, conviviality \parencite[]{Nyamnjoh2017} and the creation of a pluriversity characterised by ``openness to dialogue among different epistemic traditions'' \parencite[]{Mbembe}. I endorse an epistemology based on the recognition of that knowledge construction is perspectival, that a single perspective is necessarily incomplete, and that conviviality rests on the notion that all knowledge is created in dialogue, as evoked by Francis \citet[262]{Nyamnjoh2017}:

\begin{quote}
Conviviality is recognition and provision for the fact or reality of being incomplete. If incompleteness is the normal order of things, natural or otherwise, conviviality invites us to celebrate and preserve incompleteness and mitigate the delusions of grandeur that come with ambitions and claims of completeness. Not only does conviviality encourage us to recognise our own incompleteness, it challenges us to be open-minded and open-ended in our claims and articulations of identities, being and belonging. Conviviality encourages us to reach out, encounter and explore ways of enhancing or complementing ourselves with the added possibilities of potency brought our way by the incompleteness of others (human, natural, superhuman and supernatural alike), never as a ploy to becoming complete (an extravagant illusion ultimately), but to make us more efficacious in our relationships and sociality. 
\end{quote}

Conviviality can thus be seen as an antithesis to Othering, which denies the situatedness and entanglement of all thinking. Conviviality describes societies constituted on the premise of incompleteness of single entities, and on the importance of bringing together different perspectives, individuals, and, in my interpretation, languages, to create convivial interaction. 

For linguistics, such a framework entails abandoning solitude frameworks \parencite[]{Cummins2008}  which are based on the assumption of the use of a single, non-variable standard language as the normal and ideal state of affairs. When extended to multilingualism, a solitude framework is based on the use of several clearly demarcated, non-variable standard languages, ideally occurring in different and separate individuals, societies, geographical or sociolinguistic spaces. The outcome is regulated multilingualism, a mere multiplication of several solitudes of language. A conviviality view of language and multilingualism is based on radically different premises. Beyond acknowledging multiplicities in and of language, a conviviality framework extends to knowledge production. It invites us to take researchers’ and research participants’ social lives and positionalities into account and compels us to combine multiple perspectives as the only felicitous way of knowledge creation. Such practices involve breaking with ontologies based on a priori and discrete categories, an invention of the Self, and the comparison of this fictional entity with the Other. Rather than insisting on established ontologies, or on solely describing Global South language ideas and practices as different from them, an inquisitive dialogue on how concepts are constructed everywhere creates a convivial pluriversity. The cultivation of hospitable translation \parencite[]{Diagne2022}, rather than imposition and control of conceptual libraries, makes dialogue possible.

\section{Enriching the conceptual library of multilingualism research in convivial ways}
\label{05:sec:3}
Multilingualism research has diversified dramatically over the last decade. Today, its scope is not limited to contemporary industrialised societies in the Global North and postcolonial urban settings in the Global South but also includes a growing research focus on rural situations extending into precolonial times \parencite[]{Luepke2016a, SingerHarris2016, GoodEtAl2019, Stenzel2005, LuepkeEtAl2020, PakendorfEtAl2021}. Yet, an awareness of these (small-scale and rural) multilingual societies as globally widespread and evolutionary old \parencite[]{Evans2017} is slow to enter mainstream linguistics, including multilingualism research. Prevailing tendencies of  foregrounding settings in the anglosphere have been critiqued \parencite[]{BlasiEtAl2022, HenrichEtAl2010} and Othering of Global South practices and research on Global South settings has been pointed out \parencite[]{Kasstan, MeyerhoffNagy2008, Smakman2015}. Yet, it remains exceedingly rare to read a study of multilingualism in a Global North setting that draws inspiration from Southern theories of multilingualism \parencite[]{NdhlovuMakalela2021, PennycookMakoni2019}, and too many studies situated in the Global South exclusively use Eurocentric epistemologies and ontologies \parencite[]{LuepkeStorch2013, Makoni2013}. This marginalisation is exacerbated by the deeply inequal circulation of epistemologies and research outputs that excludes research in and on Global South societies from the global economy of knowledge \parencite[]{AboderinEtAl2023}. Continuing such a focus means that research on settings at the knowledge periphery, most notably in the Global South, is not seen as relevant outside the settings in which its data were collected, echoing \citet[]{Mbembe}’s observation that they (and Africa in particular) are seen as residual entities, places of data collection that have not much to contribute to theory formation. 

As the forgotten and underresearched history of multilingualism in Europe and the Global North \parencite[]{Pavlenko2023} reminds us, pre-nationalist configurations were seen as unsuitable for the construction of the European Self, and their history buried under new identity concepts and linguistic notions of ideally monolingual speakers and societies. Turning to the vast body of research on Global South multilingualisms is therefore of prime importance for decentring European narratives not only in and on Global South societies \parencite[]{NdhlovuMakalela2021, Makalela2018, PennycookMakoni2019, Makoni2013, Adejunmobi2004} but also in and for Global North societies. 

Imaginations of rural societies as monolingual ethnolinguistic communities prior to urbanisation and mass migration and the continuing dominant imagination of multilingualism as mostly happening in ``superdiverse'' Global North societies \parencite[]{Vertovec2007} hinders a full  appreciation of multilingual societies, including in Europe. These assumptions are based on the central tenet that ethnonationalist language ideas are universal AND that they captured the reality of language use in (some) societies prior to recent disturbances of this order. 

In the following, rather than attempting a (futile and inescapably Othering) overview of how multilingual societies outside Europe function, I illustrate how integrating concepts from these settings can serve to dismantle dominant notions in Europe and establish a convivial dialogue as a basis for exploring the meanings of language in a non-deterministic manner. 

I will mainly focus on arguing against the following notions:

\begin{itemize}
\item Language users ideally being speakers and writers of one language, and language as the most prominent marker of (ethnic) identity.
\item Languages having absolute attributes (``indigenous'' vs. ``colonial'', ``mother tongue'', ``ethnic language'', ``lingua franca'', etc.).
\item Roles identified for languages, such as ``mother tongue'' capturing social realities rather than language ideologies.
\item Languages as reified, uniformly territorialised and used by speech communities that are also language communities.
\item Multilingualism, mixing and contact as special (and always identifiable) cases, or as new, urban and young.
\end{itemize}

The categories of mother tongue and native speaker are prime examples of notions originating from a European desire to essentialise identities \parencite[]{Bonfiglio2010} in the wake of the formation of monolingually imagined nation states. Mother tongue is an imprecise signifier in the Lacanian sense, but the \citet{UNESCO} definition ``a language learned in childhood in the home environment, also referred to as mother tongue, first language, or native language'', in turn containing several signifiers that circularly reinforce each other, is representative for contemporary attempts to turn it into a sign with concrete meaning. That the category of mother tongue is not applicable to multilinguals quickly emerges, for instance when sociolinguists and education planners in Senegal and Mali resort to the use the French loanword \textit{langues maternelles} to designate mother tongues for want of a translation equivalent that is not systematically misunderstood as meaning ‘mother’s languages’, that is, the language(s) spoken by one’s mother. Stating their mother tongues presents multilinguals with the conundrum of seeing themselves through the prism of European language ideologies. 

I am often asked to provide alternative notions of speakerhood stemming from colleagues’ and my own research in Global South settings. A number of recent publications offer insight into aspects of identity foregrounded in the relations between speakers and some of their focal languages in a variety of settings from across the globe \parencite[]{GoodEtAl2019, PakendorfEtAl2021, LuepkeEtAl2020}. However, it also emerges from this research that relationships between spaces and speakers, between speakers in interaction, and between researchers and research participants dynamically shape how repertoires are categorised and presented, and that researchers’ focus on language as a central tool for identity influences responses \parencite[]{Goodchild2016, GoodchildWeidl2018,luepke2019, Luepke2021, DiCarlo2016}. This convivial nature of language use and its categorisation is qualitatively incompatible with static and absolute roles for named languages and for the notion of language as a predetermined category that exists outside of an interactional context. Sociolinguistics has undergone three waves bringing it from the assumption of particular speech forms characterising social groups in the first wave and from the recognition of the construction of social groups through the use of particular linguistic features in the second wave to researching how speakers make use of a wide array of styles and features to construct diverse social meanings in interaction \parencite[]{Eckert2012, Tagliamonte2016}, even though this variation often is only studied within one named language, with a standard variety as the benchmark. The time has come for descriptive and documentary linguistics to move from assuming that languages index or construe fixed social identities to studying how language itself is constructed convivially (see \citet[]{LuepkeWatson2020} for an example from Senegal).

A podium discussion on African linguistic diversity at the Kenako Africa Festival 2020 in Berlin illustrates the struggle with European language ideas and the impossibility to simply replace them with alternative categories. Four Afro-Diasporic panellists – linguist Reginald Duah, comparative literature scholar Rémi Tchokothe, writer Elnathan John and political theorist Stephanie Wanga aim at answering the seemingly simple question ``What is your mother tongue?'' asked by the moderator. In the process of answering it many layers of misunderstandings and contradictions connected to this concept are revealed when the panellists try to describe their lived linguistic experiences in Kenya, Nigeria, Cameroon and Ghana through this lens. Elnathan John states his experience as one of growing up without a mother tongue, but with two languages – Hausa and English – which have been imposed on him, and out of which English has become his writing tool. John’s experience illustrates the limits of dichotomies such as ``indigenous'' vs. ``colonial'': Hausa, a language of precolonial empire formation, can be as alienating to speakers on whom it is forced as English, the colonial language, which in turn can become a cherished tool of expression. Rémi Tchokothe mentions Camfranglais and Cameroonian Pidgin English as languages that are seen as nobody’s mother tongue but establish relationships of solidarity among speakers who would be seen as different based on assumptions of a single ethnolinguistic identity. The evocation of non-standard forms of English as horizontal forms of solidarity testifies both of the plurality of forms contained in a named language and of the racialisation of these forms as outside colonial and neo-colonial institutions. Stephanie Wanga dreams of a Kenyan society in which all her father and mother tongues (all the languages and language forms used by her father and mother and their parents in turn) would have equal rights and be invested with resources, testifying of the impossibility of selecting one language as having the all-encompassing importance for identity suggested by the term mother tongue. If cast in mother tongue terms, her vision would mean to allow dynamic, non-regulated multilingualism as a mother tongue. Reginald Duah sets out to provide the named language Akan of Ghana as his mother tongue but confesses that there is no certainty of knowing what Akan means. This label, in circulation for more than 400 years, has been and is used as a changeable umbrella term for languages and lects whose composition in terms of linguistic features and labels differ depending on space, time and perspective, with not all people identifying as Akan speaking an Akan language, and not all speakers of an Akan language necessarily seeing themselves or being categorised as Akan. Even if the orientation here, primed through the question, remains towards named languages, it becomes manifest that they cannot account for the multiple, ambivalent and dynamic communicative practice of the panellists beyond a rough first approximation. That is, meanings of the concept of a named language can be diverse and qualitatively very different from those in a monolingual standard language culture. The following quote, from the autobiography of the Senegalese philosopher Souleymane Bachir \citet[28--29]{Diagne2021} questions the essence of named languages as well as common categorisations of language in terms of sequential acquisition:

\begin{quote}
J’en [des langues, FL] ai parlé assez rapidement quatre; le wolof, le français, le diola, le créole. Le diola, langue à laquelle on identifie la région sud du Sénégal où pourtant tout le monde en parle plusieurs, est une langue que j’ai attrapée assez vite. Je regrette tant de l’avoir laissée s’échapper, plus tard, de la zone de mon cerveau où elle ne s’était pas assez profondément incrustée. Plus tard, c’est-à-dire autour de mes neuf ans, lorsque ma famille a déménagé à Dakar. Le créole portugais, qui était la langue commune de la ville de Ziguinchor, s’est aussi dissipé dans l’atmosphère de Dakar lorsque je ne l’ai plus parlé, même s’il m’en reste des bribes. Le wolof et le français me sont «~langues premières~» autant l’un que l’autre car les circonstances on fait que je les ai parlés en même temps.
\end{quote}

\begin{quote}
[I pretty quickly spoke four languages: Wolof, French, Joola, Creole. Joola, the language with which the Southern region of Senegal is identified, although everybody there speaks several languages, is a language I’ve caught very quickly. I regret very much to have let it escape, later, from the area of my brain where it hadn’t entrenched itself deeply enough. Later, meaning when I was about nine years old, when my family moved to Dakar. Portuguese Creole, which was the communal language of Ziguinchor, also dissipated into the atmosphere of Dakar when I didn’t speak it any longer, even though I still have some pieces of it. Wolof and French are ``first languages'' to me, each as much as the other, because circumstances have made that I have spoken them simultaneously.]
\end{quote}

Are Joola (and if so, which Joola?) and Creole Diagne’s heritage languages? Can French be a mother tongue for a citizen of Senegal? Is a community language different from a lingua franca? How can we account for the fact that Wolof is deeply entangled with French, in a strongly established fluid type of ``unmarked code-switching'' \parencite[]{MyersScotton1993} that has prompted some researchers to characterise it as a ``postcolonial creole'' \parencite[]{Swigart1992}? Is Portuguese Creole an indigenous language? Can Diagne, who has taught at American universities for most of his academic life, claim English as a language he owns on a par with his other languages? The multiplicities, ambivalences, vague referentialities of named languages, changeability of repertoires elucidated in the testimonies presented above cannot be captured with these or any other term, they require thinking away from named languages and their roles in standard language cultures and towards convivial multilingualism.

The shortcomings of the colonial language library and the alternative language worlds that emerge from these testimonies are of relevance not only for Global South societies; comparing them with lived experiences all over the world also reveal hidden multilingualism through enabling the opening of cracks in European and Northern language ideologies. Reacting to linguistic repertoires such as those given above enables Europeans to relate their experiences to them. Rather than seeing them as exotic and alien, the evocation of such complex and dynamic repertoires \parencite[]{LuepkeStorch2013} often empowers Europeans, too, to accept the mismatch between normative stances and actual behaviour and experience and articulate their likewise changeable and multiple links to languages. A recognition of such mismatches allows:

\begin{itemize}
\item seeing  multilingualism as variably categorised rather than associated with fixed functions for clearly determined named languages;
\item recognising  the use of multiple registers within and beyond named languages;
\item perceiving ruptures in language socialisation, for instance between convivial language use in the family and the exposure to a standardised language of education at school, regardless of whether the same named language is used or not \parencite[]{LuepkeCisse2023}.
\end{itemize}

Listening to my examples from Senegal, a Spanish colleague told me that she could relate to them very well, since she had always wondered why she was told that she was receiving mother tongue education when learning standard Galician and Castilian, which were both very removed from language use in her family and out-of-school environment. At home, not only different linguistic varieties of these named languages were used, but they were also used with a fluidity not allowed in the school context. The language of her intimate environment had no place in this mother tongue logic. The American writer Amy Tan has captured the multitude of Englishes and how monolithic notions of a mother tongue and a shared body of linguistic signs as incarnating it fail to grasp this variability in a powerful essay \parencite{Tan1990}. Her narrative captures the multitude of sociopolitical registers and how they connect her to different facets of her identity that are hidden behind the façade of a single named language. Such experiences extend beyond English to everywhere, from Swahili \parencite{Mugane2015} to Finnish \parencite[]{Laakso2022}.


It is not just language users whose dynamic and multiple relations to language(s) and registers means that they cannot be statically and univocally  considered as particular types of speakers. The concepts contained in particular roles associated with named languages, in analogy to those for other socio-political identity concepts such as nationality and ethnicity are indexing imagined, rather than really existing communities \parencite[]{Anderson1983, Silverstein2015}, for which equally imagined named languages serve as signs of difference \parencite[]{GalIrvine2019} through ``lies that bind'' \parencite[]{Appiah2018}. Both the categorisation through language names and language roles and of linguistic practice are subject to perspectival and scalar socio-cognitive perception \parencite[]{Luepke2021}. Investing named languages with fixed social indexicality is only possible through the necessary ``cultivated vagueness'' and ``magical uncertainty'' \parencite[]{BalatonChrimes2021} of seemingly clearly referential ontologies. Fixing languages through a delineated inventory of named languages and associated linguistic forms fails to acknowledge that we all always only use ``bits and chunks of language'' \parencite[]{Blommaert2010}, akin to  Deleuzian assemblages \parencite[]{Pietikaeinen2021}. Dynamic, fluid, and variously categorisable repertoires always defy being cast in terms of finite and objective categories.

\section{Outlook: how to make multilingualism research truly global} %5. /
\label{05:sec:4}
This century has seen a multilingual turn in all areas of linguistics. Too often, this turn is seen as correlated with increasing multilingualism and superdiversity. Such a view is ahistorical in that it only takes into account a short period of imaginary monolingualism (or regulated multilingualism) disturbed by globalisation and increasing migration. It is now necessary to globalise the multilingual turn and free it from Eurocentric language ideologies. Such a turn requires first and foremost a rethinking of language, as \citet[11]{NdhlovuMakalela2021} remind us:

\begin{quote}
as \citet[118]{Ndhlovu2018} puts it, while the invocation of ‘high-sounding metaphors of human rights, anti-imperialism and biodiversity resonate with contemporary international conversations around social justice and equity issues’, they struggle to achieve much because ‘standard language ideology remains ensconced as the only valid and legitimate conceptual framework that informs mainstream understandings of what is meant by ``language''’. This is not a question of language standardisation – the problem that language revival projects generally need to reduce language variety to a much narrower set of options – but that these projects all too often operate with a constricted understanding of what language is and how it operates. 
\end{quote}

Such an opening towards ``free range language'' \parencite[]{Wiese2023} involves ``provincialiser la langue'' \parencite[]{Canut2021}, provincialise language through acknowledging the origin and recognising the limitations of dominant ideas of language not only for Other places, but in the North as well. No longer can we say ``Africans are so multilingual'' and thus contrast them with an imagined mono- or less multilingual European Self. Rather, we should ask what makes African multilingualisms more visible than European multilingualisms, a question that inevitably leads us to investigating the processes that have erased Europe’s and the North’s linguistic diversity from consciousness and banned them from many domains of public life and to researching the social contexts which enable multilingualism to be less censured and more convivial.

Monolingualism is a nationalist invention \parencite[]{Schneider2018, Gramling2016}. If we concede that monolingualism is constructed, this also entails that we need different epistemologies, ontologies and methodologies to approach multilingualism. To paraphrase in arithmetic terms, there is no longer a multiplicand that can be multiplied, nor a code that can be mixed. Without such a base line, what we are left with is fluid (trans)languaging \parencite[]{Mignolo1996, Garcia2009, GarciaWei2014, Wei2018, Makalela2015, Makalela2016} as the phenomenon that can be observed. (Trans)languaging, convivial language use in interaction, constrained in accordance with language ideologies and language regimes at work in particular spaces and exchanges but never a sealed monolingual container. Heteroglossic and variable language use becomes mono- or multilingual through social indexicality \parencite[]{Agha2005, Silverstein2003}. It is thus socio-cognitive categorisation processes that determine how (trans)languaging is categorised as mono- or multilingual. It follows that language IS conviviality: it has no fixed meaning outside of particular places, interactions and observations which are shaped by language ideologies and repertoires \parencite[]{Luepke2021}.

Investigating convivial language use is best done in convivial setups. It starts from exploring interactional spaces, not from researching languages, an endeavour that requires diverse collaborative research groups. In descriptive and documentary linguistics, a conviviality framework paves the way to investigating how different individuals and societies construct, reify and index particular social categories, including language (or not). This requires the development of new models of language and (trans)languaging taking the necessary incompleteness of language in multilingual repertoires and use into account. Such research is necessarily convivial itself, driven by inter- and crossdisciplinarity and aiming at being socially relevant to multilingual speakers. 

Conviviality research therefore requires a political and activist stance. It is not possible to celebrate multilingualism and showcase the creativity of convivial (trans)languaging practices or argue for their introduction into formal education systems without challenging the exclusionary and raciolinguistically motivated practices of these institutions \parencite[]{Cushing2021, FloresRosa2015, Rambukwella2021, LuepkeCisse2023}. Convivial multilingualism is outlawed there, because solitude mono-or multilingualism in standard languages is the prevailing language idea. In postcolonial societies of the Global North and South, these institutions penalise the most convivial multilinguals most, because their presence is either erased or seen as an educational challenge, and their language use does not correspond to solitude ideas of language: they do not have mother tongues corresponding to a clear set of linguistic features and used in standard-based writing. According to mother-tongue-based monolingual ideologies, they are excluded from owning large parts of their repertoires, either because they contain non-valorised Southern forms of languages of European provenance or because they are seen as impure and deficient. Advocating for convivial multilingualism means actively encouraging convivial ideas of language and convivial language use. The very fact that I am constrained by deeply rooted Northern academic traditions to write this article in a strongly reified form of English, which is supposed to instantiate ‘native’ or ‘native-like’ proficiency and keep separate varieties of English that fluidly intermesh in my language use (for instance by choosing either ‘American English’ or ‘British English’) while at the same time remaining excluded from being a ‘native speaker’ or claiming ownership of this language illustrates the enormity of this task. And I am writing as a privileged European steeped in standard language culture and schooled in standard British English. For ``multicephalous'' language users across the globe \parencite[]{Sow2021}, the standard languages imposed on them are the metropolitan forms of colonial languages or (post)colonially created standard version of ‘their’ languages \parencite[]{NgueUm2015}, thereby disowning them and disqualifying their convivial practices, a status quo that we need to unseat. 

\sloppy\printbibliography[heading=subbibliography,notkeyword=this]
\end{document}
