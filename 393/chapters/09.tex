\documentclass[output=paper,colorlinks,citecolor=brown]{langscibook}
\ChapterDOI{10.5281/zenodo.17132455}
\author{Judith Purkarthofer\orcid{0000-0002-2650-2274}\affiliation{University of Duisburg-Essen}}
\title[In/Exclusively addressing the Other in disseminating linguistic results]
	  {Who’s there? In/Exclusively addressing the Other in disseminating linguistic results}

\abstract{Outreach activities, in linguistics and other fields, build on the assumption that scientific knowledge should be made available to members of the larger public. ``Third mission'' activities are prominent in funding schemes, encouraged by universities and evaluated in hiring procedures. Often, rather precise descriptions of target groups are called for and as researchers and educators, we need to negotiate (stereotypical) ideas about target groups with discourses of inclusiveness and opportunities of participation. 
In this paper, I will demonstrate how drawing on audience design helps to understand multilingualisms in practice: by learning about the addressees of transfer activities, we sharpen our own understanding of lived experiences of language. While the effects of research on research participants have been discussed in greater detail, the effects of outreach have been only marginally considered. By drawing on results from one specific transfer project, focusing on family language dynamics, my aim is to reflect on multiple speaker positionalities, negotiations of access and address and finally the effects of research and outreach activities.}


\IfFileExists{../localcommands.tex}{
   \addbibresource{../localbibliography.bib}
   % add all extra packages you need to load to this file

\usepackage{tabularx,multicol}
\usepackage{url}
\urlstyle{same}

\usepackage{listings}
\lstset{basicstyle=\ttfamily,tabsize=2,breaklines=true}

\usepackage{langsci-basic}
\usepackage{langsci-optional}
\usepackage{langsci-lgr}
\usepackage{langsci-osl}
% \usepackage{./langsci/styles/langsci-lgr}
% \usepackage{./langsci/styles/langsci-osl}
% \usepackage{langsci-gb4e}

\usepackage{tikz}
\usetikzlibrary{patterns,calc}
\pgfdeclarepatternformonly{south east lines}{\pgfqpoint{-0pt}{-0pt}}{\pgfqpoint{3pt}{3pt}}{\pgfqpoint{3pt}{3pt}}{
    \pgfsetlinewidth{0.6pt}
    \pgfpathmoveto{\pgfqpoint{0pt}{3pt}}
    \pgfpathlineto{\pgfqpoint{3pt}{0pt}}
    \pgfpathmoveto{\pgfqpoint{.2pt}{-.2pt}}
    \pgfpathlineto{\pgfqpoint{-.2pt}{.2pt}}
    \pgfpathmoveto{\pgfqpoint{3.2pt}{2.8pt}}
    \pgfpathlineto{\pgfqpoint{2.8pt}{3.2pt}}
    \pgfusepath{stroke}}
    
\usepackage{stmaryrd}
\usepackage{wasysym}
\usepackage{multirow}
\usepackage{caption}
\usepackage{subcaption}
\usepackage{mathrsfs}
\usepackage{qtree}

\usepackage{linguex}


   %pminos do not split footnotes
% \interfootnotelinepenalty=10000 %Footnote in Laporte chapters has to be split SN


%\DeclareIndexNameFormat{default}{%
%\nameparts{#1}%
%\usebibmacro{index:name}%
%{\index[names]}%
%{\namepartfamily}%
%{\namepartgiveni}%
% {}% L1
% {}% L2
%{\namepartprefix}% generates spurious space L3
%{\namepartsuffix}% generates spurious space L4
%}

%  {\DeclareIndexNameFormat{default}{%
%     \usebibmacro{index:name}{\index[names]}{#1}{#3}{#5}{#7}}}

%\DeclareIndexNameFormat{default}{%
%  \usebibmacro{index:name}{\sindex[nom]}{#1}{#3}{#5}{#7}}

%\DeclareIndexNameFormat{default}{%
%  \usebibmacro{index:name}{\sindex[person]}{#1}{#3}{#5}{#7}}
%\DeclareIndexNameFormat{default}{%
%\nameparts{#1} \usebibmacro{index:name}{\sindex[person]]}{\namepartfamily}{‌​\namepartgiven}{\nam‌​epartprefix}{\namepa‌​rtsuffix}}

%\newcommand{\smiley}{:)}

%\renewbibmacro*{index:name}[5]{%
%\usebibmacro{index:entry}{#1}%
%{\iffieldundef{usera}{}{\thefield{usera}\actualoperator}\mkbibindexname{#2}{#3}{#4}{#5}}}

% \newcommand{\noop}[1]{}

%remove for final
%\overfullrule=1mm

\newcommand{\tobi}[2]}}
\renewcommand{\S}[1]{\tobi{#1}{\textsc{*}}}

% this volume references
% puts: [this volume]
% already defined: \citetv
%\newcommand{\citepv}[1]{(\citeauthor{#1} \citeyear*{#1} [this volume])}
\newcommand{\citealtv}[1]{\citeauthor{#1} \citeyear*{#1} [this volume]}

%parentheses around example number
\newcommand{\pref}[1]{(\ref{#1})}

% in-text examples

\newcommand{\lnex}[1]{\textit{#1}} %target lang word
\newcommand{\lnlit}[1]{(lit.: `#1')} %literal reading
\newcommand{\lnlat}[1]{(#1)} % latinization
\newcommand{\lntrans}[1]{`#1'} %translation
\newcommand{\lnexl}[2]%
{\lnex{#1}{} \lnlat{#2}} % ex with latinization
\newcommand{\lnexlat}[3]{\lnex{#1}{} \lnlat{#2}{} \lntrans{#3}} % ex with latinization and tranl.

%ch01
\newcommand{\co}[1]{\mbox{\textbf{#1}}}

%ch09

\newcommand{\cyrbulg}[1]{\begin{otherlanguage*}{bulgarian}#1\end{otherlanguage*}}


%ch10
\newcommand{\nlp}{{\small NLP}}
\newcommand{\mwe}{{\small MWE}}
\newcommand{\rae}{{\small RAE}}
\newcommand{\lvc}{{\small LVC}}
\newcommand{\pos}{{\small P}o{\small S}}
%\newcommand{\todo}[1]{ \textcolor{red}{#1} }

%\renewcommand{\labelenumi}{\theenumi}
%\ainamefmt{{vv}{ll}{, ff}{, jj}} % fullname

\newcommand{\biberror}[1]{{\color{red}#1}}

\newcommand{\osenovaitem}{--~}
   %% hyphenation points for line breaks
%% Normally, automatic hyphenation in LaTeX is very good
%% If a word is mis-hyphenated, add it to this file
%%
%% add information to TeX file before \begin{document} with:
%% %% hyphenation points for line breaks
%% Normally, automatic hyphenation in LaTeX is very good
%% If a word is mis-hyphenated, add it to this file
%%
%% add information to TeX file before \begin{document} with:
%% %% hyphenation points for line breaks
%% Normally, automatic hyphenation in LaTeX is very good
%% If a word is mis-hyphenated, add it to this file
%%
%% add information to TeX file before \begin{document} with:
%% \include{localhyphenation}
\hyphenation{
    Beck-man
    Ngu-yen
    back-chan-nel
    back-chan-nels
    mo-not-o-nous
    ste-reo-typ-i-cal
}

\hyphenation{
    Beck-man
    Ngu-yen
    back-chan-nel
    back-chan-nels
    mo-not-o-nous
    ste-reo-typ-i-cal
}

\hyphenation{
    Beck-man
    Ngu-yen
    back-chan-nel
    back-chan-nels
    mo-not-o-nous
    ste-reo-typ-i-cal
}

   \boolfalse{bookcompile}
   \togglepaper[4]%%chapternumber
}

\begin{document}
\maketitle

\section{Introduction}
\label{09:sec:01}
\textit{Working as an academic at times includes train rides, in this case through nocturnal Swedish midlands, slightly whitened by last bits of snow. What a wonderful impression and also what a welcome metaphor to think about addressees of scientific outreach and dissemination. Who is there, out there, in the dark? Do I get to see who might be affected by the specific communicative event that I am about to prepare? Can I find a way to relate my experiences, of living with languages, of crossing borders under specific circumstances, to those of others who are or are not like me?}


Outreach activities, in linguistics and other fields, build on the assumption that scientific knowledge should be made available to members of the larger public. Universities have for the longest time had the task to also cover ``third mission'' activities and these have recently become more prominent in funding schemes, as encouraged by universities and evaluated in hiring procedures. Often, rather precise descriptions of target groups are called for. However, in research on the complexities of lived experience, it seems rather inappropriate to classify speakers by relatively durable categories, like migration background or first languages, as all hold some truth but are also always only covering part of human experience. How can we – as researchers and educators – negotiate (stereotypical) ideas about target groups with discourses of inclusiveness and opportunities of participation? Are there ways to understand how practices of naming and addressing work towards our goals instead of dividing potential target groups and \textit{othering} them in the process? 
	Research in sociolinguistics has focused on speech and language use, including changes in register and a variety of terms of address linked to changing interlocutors in different communicative situations and with diverse goals. However, when focusing more specifically on target groups of outreach activities, we are still in need of models that would help to explain social practices. The motivation of this contribution is thus very specific and linked to project experience: how can we know in which ways we reach our audiences?  At the same time, the attempted model to understand interactions and target groups ideally goes beyond our project. What follows is first a brief introduction to the project and, in Section \ref{09:sec:03}, more theoretical thoughts about audience design and how this helps to understand potential pitfalls. I will come back to the questions in Section \ref{09:sec:04} where I discuss some of our decisions and how we acted upon problems we encountered. Finally, my aim is to close with some ideas of how access can be ensured more easily in Section \ref{09:sec:05}.


\section{Point(s) of departure(s)}
\label{09:sec:02}
My research is situated in the field of sociolinguistics and applied linguistics, both claiming to be interested in ``real-world-problems'' with languages. More precisely, I am interested in language use and language policy in multilingual families: how are expectations between parents negotiated? Which experiences help parents and children to shape their family language policy according to their needs and wants? How do educational and language ideologies contest the use of family languages and include or exclude multilingual speakers? Thinking about speakers and even in a speaker-centred way is thus my specific task and I try to do that based on empirical research that is usually qualitative and situated in diverse and multilingual contexts of post-migrant societies. This term, coined by \citet{Foroutan2019}, conceptualizes society, i.e. in Germany, as deeply entrenched with processes of migration and mobility, in a way that would influence the experiences of every person, not just those with their own migration history \citep{Foroutan2019}.

As a researcher in socially debated field, outreach and education activities have always been part of what I did. However, between 2021 and 2024, these activities became central as our project team successfully completed a project with its main goal of reaching out to multilingual speakers: we worked on a transfer project ``Family language dynamics – empowering speakers of majority and heritage languages'' that was part of the research unit ``Emerging grammars in language contact situations'' (henceforth RUEG, \cite{AllenEtAl2024}). The research group analyzed language change in contact situations on different linguistic levels, looking at Greek, Russian and Turkish as heritage languages in Germany and the USA as well as German and English as majority languages in these countries. Building on the results from this research, our project developed educational resources to be used with parents and educators. We partnered with two institutions for adult education, who were on the one hand running language classes for parents and on the other hand classes aiming at early education and school teachers covering a number of topics. Over the course of three years, we developed videos and texts that are featured on a multilingual website (https://www.ruegram.de) and that are the core of workshop curricula and lesson plans intended for use by educators from pre-school to university and adult education. The educational materials are organized in five categories that are color-coded thoughout the website (see Figure~\ref{fig:09:1}). The proposed contents and activities distribute knowledge and invite reflections about a person's own experiences with languages. The categories cover

\begin{enumerate}
\item Knowledge about languages and language use (green), i.e. about bilingual langauge acquisition,
\item Specific research findings illustrating language ``live'' in language contact situations (blue), e.g. how the use of pronouns is influenced by the speakers' second language German,
\item Experiences of multilingual speakers (purple), e.g. how members of the research group grew up in a multilingual family,
\item Myths/Discourses about languages and multilingual language use (red), e.g. whether or not multilingualism would slow down language acquisition, and finally
\item Citizen Science activities, researching in the own family and community (yellow), e.g. by collecting data about the languages that were used in the grandparents' generation.
\end{enumerate}

Initially, the resources were designed to be used in structured teaching activities such as classes and in the course of parents’ meetings. However, we quickly decided to make them accessible to speakers independently of such activities. This presented the challenge to provide different frames of reference to different audiences, in addition to different linguistic resources, including modes like spoken or written and languages such as German, English, Turkish, Greek and Russian. I will get back to details about the project in Section \ref{09:sec:03}.

\begin{figure}
\includegraphics[width=0.75\linewidth]{figures/five-modules.png}
\caption{Five modules: (1) Knowledge (green), (2) Specific research findings (blue), (3) Experiences (purple), (4) Myths/Discourses (red) as well as (5) Citizen Science (yellow).}
\label{fig:09:1}
\end{figure}

\section{Audience design for speaking and listening subjects}
\label{09:sec:03}

As speaking subjects, both speakers and signers, we interact in different contexts and we experience languages and language use accordingly: \textit{Spracherleben} (lived experience of language, \cite{Busch2012}, \cite{Busch2017}) builds on first-hand experience of using languages and also on the discursive frames associated with these experiences. In the course of a language biography, experiences are re-evaluated and re-framed in the light of new information and new positions. Obvious moments of re-evaluation include political changes (that might entail new status for certain languages), migration and mobility as well as changes in personal and professional relationships, i.e. when entering formal education or when becoming parents. \textit{Multilingual selves} (see also \cite{Kramsch2009}) construct themselves by drawing on their own experiences as well as memories, affects, emotions and social and cultural knowledge: their language learning is governed by imaginations and hope but they might also encounter fears and insecurity.

In linguistics, the speaker has initially been in focus but the \textit{listening subject} has received more attention in recent years. The fact that hearing/seeing as ``perception (whether auditory or visual) is never a natural or unmediated phenomenon but is always already a social practice" has been convincingly demonstrated by the works of Inoue on the perception of Japanese female voices (\cite[157]{Inoue2003}) and the social, historically emerging construction of the listener becomes increasingly relevant. Listening is recognized as a specific practice in therapeutic settings (\cite{MarsilliVargas2022}) and the politics of listening and sounding are taken into account when analyzing language behaviour and sociolinguistic data (\cite{Kozminska2022}). In the influential works of Flores and Rosa (\cite{FloresRosa2015}), they set particular focus on the (white, priviledged) listening subject who holds the power to render perceived language practices ``normal'' or ``deviant'', rather depending on the listener’s own subject position than on the speakers' production. Assuming that privilege is not distributed equally, we are of course concerned with negotiations of (stereotypical) ideas about target groups. Can we use language practices, naming and address terms to unite potential target groups instead of dividing them or othering them in the process?
	In order to discuss the implications of this question, I will go back to a model of audience design that was developed by Allan Bell, an Australian linguist \citep{Bell1984}: he analyzed changes in phonetic features (among others) of one speaker who addressed audiences in different radio stations. The image of the lonesome speaker in the radio studio is in a way reminiscent of the Swedish midlands, filled with lakes and woods, and a surprising number of independent radio stations. But what interests me about Bell’s model are not the features that change in relation to addressees but rather the different options that he offers to be a listening subject a.k.a. audience to utterances and communications. In his seminal paper, Bell distinguished the immediately addressed subjects (addressees) from auditors, meaning those that are present and known, such as the other persons at a table that are following a conversation though not being directly addressed. When there are others in vicinity, they might become overhearers, e.g. as passers-by, or, if they are not even present to the speaker, they can still have a role as an eavesdropper (\cite{Bell1984}, 160).
	In this paper, I will thus speak about scientific transfer and outreach activities and about the immediately addressed as well as the auditors and their role in shaping communications. And I will also take a look at the challenge to reach overhearers, as they are often the speakers with the greatest potential to be interested in our results without necessarily knowing about them. Finally, I will give some consideration to eavesdroppers that become at times only visible when they e.g. comment negatively on the contents of communications.

\subsection{Addressees}

The model of audience design assumes at least one addressee for each utterance – whenever we speak we speak to somebody even if ourselves. The addressee is known and ratified and they are consciously addressed, which might include addressing the inner self in the case of an inner monologue. In light of a project application and implementation, we might be aware of specific addressees in the person of enthusiastic teachers, leaders of parents’ organizations and colleagues at the universities. Those addressees share certain characteristics with us, be it biographically like having children in the same school, or politically in having specific expectations concerning the goals of public education. Sometimes, the shared interests emerge like unexpected halts in the middle of a train ride, in the forms of shared places of residence albeit 15 years apart, of shared loss of languages or other experiences connecting us.
	The need for addressees is a pragmatic one but it is just as important in a conceptual understanding: an addressee is needed in order to find one’s own position in the world, as Butler phrases in her book \textit{Giving an account of oneself}: ``I exist in an important sense for you, and by virtue of you. If I have lost the conditions of address, if I have no ``you'' to address, then I have lost ``myself''.''  (\cite{Butler2005}, 32) In the framework of Spracherleben, the lived experience of languages (\cite{Busch2017}), addressees are seen as an important partner in the speaker-listener relationship, thereby shaping the experiences but also serving as evaluators, enhancers and sounding boards of language experiences. Their presence and accountability is crucial in order to understand the effects of your utterances and ultimately resources.

\subsection{Auditors}

The auditors are within reach of your communications, and they might even follow very closely without being directly addressed. As they are known and ratified according to the model of audience design, they might at any given point turn into speakers and addressees. For our project, the auditors are often those members of schools, early child care institutions and parents’ associations that receive our information materials, might even read some of our texts and watch the occasional video. As they are involved in the conversation, they are actually rather likely to take part in activities and might become direct addressees once a specific language group or target group should be addressed.
	Conceptually, the auditors are probably the target group that is most explicitly mentioned in applications – while we do not name the direct addressees of our activities, the groups they identify with are often named. However, we know that reference is typically made to certain characteristics but not to others: we call for i.e. languages or heritage in the case of parents without mentioning professions and education while we focus on education and professional identity with educators and only secondarily on languages. One challenge in transfer projects is how to turn auditors into participants in the sense that they start to interact and to appropriate the project resources for their own goals. However, at the same time auditors might turn into multipliers for a project when they share the information they are getting – in our project we count on students in teacher education: they can become multipliers when it comes to using project resources in the schools they start to work in.

 On a more critical note, we might find that the role of the auditors is taken up by those that feel that their voices are unlikely to be heard. They might follow interactions but will hardly engage and make themselves heard. It will depend on our listening abilities to provide spaces that can be put to use for those wishing to speak.
 
\subsection{Overhearers}

For a transfer project, the group of overhearers – or rather the many different participants that overhear parts of the conversation – might hold the largest potential. Their existence is partially known to the project team, either individually or as members of named groups, yet they are not considered an active part of the project audience so far. The image of the passers-by that Bell uses in his description is of course temporally situated – in an online world, overhearers go by even less noticed. However, their existence is easily assumed when thinking about the user base of certain contexts, i.e. in the university context. 
	While auditors in my understanding follow a conversation they might not yet take an active part in, overhearers are probably picking up snippets from conversations, see some parts of activities etc. In order to interest them in the project, it seems necessary to communicate the goals and aims as well as the potential ways to interact with the project and its resources. In contrast to auditors, we might not necessarily know about their language preferences or their means of access but we are probably able to extrapolate from earlier experiences.



\subsection{Eavesdroppers}

Finally, the model of audience design includes one category for those who might actually notice parts of the communication but are not present to the speakers and the other participants. The image of the eavesdropper comes to mind for those who, i.e. as technical staff, will engage with the resources but whose existence and influence might not be obvious to the participants in the project.
	On another note, eavesdroppers might be the speakers who come across project activities ``in passing'', reading leaflets and noticing activities without engaging with the members of the project team. If their interest can be raised and their presence becomes known, there is potential in turning eavesdroppers into participants.
	Another group of eavesdroppers will only engage with the project and become visible to the project group when they speak out negatively about the project contents, or in the worst case, engage in harassment. Given the project’s agenda that aligns with a diverse vision of society and favours multilingualism over monolingualism, a certain degree of pushback is expected and so far, we were lucky enough to only experience rather harmless negative comments. We are aware that this is partially mitigated by design decisions, i.e. to opt for a website instead of social media involvement, which might of course also lead to fewer contacts overall. At the same time, colleagues have experienced extended negative comments because their resources were targeted by some influential groups of haters.


\section{Othering-by-design?}
\label{09:sec:04}

Audience design is considered a semi-conscious process in which a speaker orients towards addressees, potentially taking others into account as we have seen in the previous section. The assumption in Bell’s model is that it is the intention of the speaker to be understood by the addressees and that exclusion is not desirable. However, listeners might experience that they are purposely not addressed or even excluded from some conversation, being at times reduced to the role of overhearers. One way how this happens is by means of othering, thereby underlining the  differences between individuals and groups, leading to perceived distances between speakers. Dervin frames this as such: ``Othering means turning the other into an other, thus creating a boundary between different and same, insiders and outsiders'' (\cite{Dervin2016}, 45). The focus on the process that creates distances between some speakers while glossing over distances with others is interesting as it speaks of the constructivist nature of our interactions. Sameness or otherness are not given qualities but are instead done interactionally and conceptually when considering who to reach with any given communication. 
	Othering is often linked to differences that are perceived as stable, i.e. heritage and origin, language, faith, gender etc. However, every trait can be potentially used to construct otherness – one example being accents, as Moylan explains: ``While each of us speaks with an accent, value is conferred upon individual accents within a hierarchy in which accents are invested with different degrees of currency, readability and social capital'' (\cite{Moylan2018}). In this section, I will look more closely into the language ideologies expressed by participants in our transfer project and how they point to risks of othering-by-design.
	
Language ideologies play an important role for speakers: they serve as a reference when evaluating language use, they shape decisions about language transmission in families and oftentimes they are negotiated and questioned by speakers who see their own practices evaluated by others, wishing for alternative realities. ``To comment on languages, or describe them, or recommend policy with respect to them, is to engage in a metadiscourse, a reflexive activity that is at once a practice and a commentary upon that practice, within a realm of alternative possibilities'' (\cite[1]{GalIrvine2019}).
	In this transfer project, our aim was to transmit research-backed knowledge regarding family languages, their transmission and their status in Germany. More general information might be useful for parents and educators of mono- and multilingual children as well as children and adolescents themselves. All activities focus on speakers’ multilingual repertoires and relate to RUEG’s main research question ``What are the linguistic dynamics in heritage speakers’ repertoires?''. I will come back to the specific resources in the end of this section but before, I will use data that we collected as part of our search for audiences to illustrate the potential for othering in this type of research and transfer. 
	When discussing the goals for our transfer activities we quickly realized that those goals were connected to the question of who we would be able to reach. In order to understand more about the potential participants, we tried to address them via our partners and asked them to fill in questionnaires that we had prepared in 2022. We sent the questionnaires to teachers in parents’ classes. These classes are open to persons who are learning German while their children are in school or daycare, and apart from language skills, they also cover ``integration'' content, in the form of knowledge about the German education system, searching for a job and developing plans for the family’s future. The curriculum was recently updated and is innovative in its attempt to focus on the learners and to draw on their experiences both professionally and personally. The participants filled the questionnaires with pen-and-paper and scans were sent to us through the teachers and trainers. A total of 55 two-page questionnaires were completed, each covering nine questions about language use, expectations linked to a speaker’s own repertoire and the future of the children and finally some questions about learning habits and preferences. Overall, the questionnaires were meant to give us insights into preferences to be able to tailor our information and teaching material to the needs of the speakers. 
	In the questionnaires, the complex family repertoires became apparent: For many parents, the languages they used in the family were not the ones that they considered most important for the children: German and English were more often termed ``important for my children'' than for the parents, and the same is true for Spanish, French and Swedish. At the same time, Arabic, Turkish, Italian, Kurdish, Vietnamese, Azeri, Greek and Farsi were seen as more important for the parents than for the children. The distribution of languages is probably influenced by language ideologies that assign more value to certain languages than to others (see e.g. \cite{SchroedlerEtAl2022}). Even if this question asks for individual evaluation, the social evaluation of language resources cannot be ignored. I will focus here on the answers given to two questions that were asked in the end of the questionnaire and intended to open a space for individual comments. They are thus formulated rather openly: (A) Do you talk about language(s) frequently? Which topics do you discuss? Is it important for you that your children are able to speak (and potentially write) the family language? (B) How do you feel when using different languages?
 The responses are given in the German original (including the original choice of phrasing) and in their English translation. Languages in brackets indicate the family languages as written in the questionnaire.

While the parents' and children's own expectations and emotions are central to Spracherleben, there is another way to look at how languages coexist in a social group, i.e. an extended family
\\

\ea \label{09:ex:1}
Parent, question A:\\
Es ist mir sehr wichtig, dass mein Kind die Sprache der Familie verwendet, damit es sich immer mit allen verständigen kann.\\

\trans `It is very important for me that my child is able to speak the language of the family so that [he*she] can always talk to everybody.' (Portuguese)
\ex \label{09:ex:2}
Parent, question A: \\
Ich fand es wichtig, dass mein Kind gut deutsch spricht. Es war genauso wichtig, dass er Vietnamesisch spricht, aber das hat er später von alleine gelernt.\\

\trans `I found it important that my child speaks German well. It was equally important that he speaks Vietnamese but he has learned this later on his own.' (German, Vietnamese)
\z

The focus on family interactions and the extended family makes it relevant to think about the role of audience design: when direct addressees are considered, the parents see themselves as important, but the auditors of these efforts are also members of the extended family, as can be seen in excerpt (\ref{09:ex:1}). In example (\ref{09:ex:2}), the child is initially seen as a speaker of German and it is inferred that efforts were made to speak German with the child. In a second step, the parent mentions that the child acquired Vietnamese 'on his own', probably assuming an overhearer and potentially auditor's role in the family conversation.

In another excerpt (\ref{09:ex:03}), the question of belonging is discussed more urgently, when the use of languages is linked to a feeling of being a (different) person. The affective attachment to a language seems very strong and it is easy to imagine that this feeling can lead to tensions in the families.


\ea
\label{09:ex:03}
Parent, question B:\\
Ich fühle mich wie andere Person. Ich kann nicht hundert prozent meine Meinung oder meine Gefühle ausdrucken.\\

\trans `I feel like a different person. I can not express my opinion or my feelings one hundred percent.' (Turkish, German)
\z

At the same time, the parents are not the only ones who voice their opinions and in some examples, the children are quoted. From the point of view of Spracherleben, the parents feel responsible for language transmission and their goal to raise their children bilingually is at risk through limited exposure and the very prominent status of German (and to some degree English). Three parents quote their children in the questionnaires who address their own and their parents’ language competencies. In Excerpts (\ref{09:ex:4}) to (\ref{09:ex:6}), the parents answer the three prompts successively, all three indicating that they are often talking about language and then quoting topics that are discussed. The children are quoted in all three excerpts, and their questions are: (4) Why is English important? (5) Why should I learn Arabic? and (6) Why don’t you speak proper German yet? The last question is directed to the parent, thereby putting the child in the position of expressing language policing.


\ea
\label{09:ex:4}
Parent, question A:\\
Ja, viel. Warum ist Englisch wichtig? (Kind) Ja, wichtig. Meine Tochter soll erstmal Deutsch lernen.\\

\trans `Yes, a lot. Why is English important? (Child) Yes, a lot. My daughter should learn German first.' (German, Tamil)
\z
  

\ea
\label{09:ex:5}
Parent, question A:\\
Ja, sehr wichtig. Warum soll ich Arabisch lernen? (Kind)\\

\trans `Yes, very important. Why should I learn Arabic?' (Child) (Arabic, German)
\z


\ea
\label{09:ex:6}
Parent, question A:\\
Ja, sehr wichtig. Warum sprichst du immer noch nicht gut Deutsch? (Kind zur Mutter)\\
\trans `Yes, very important. Why don't you speak German well yet?' (Child to mother) (Arabic, German)
\z

In this last excerpt, the temporal aspect of learning and language acquisition comes into play – a topic that is also raised by other parents when they talk about the specific need to focus on certain languages (i.e. when starting schooling in Germany) but also when the process of passing on family languages is described as a two-way street in Excerpt (\ref{09:ex:7}).


\ea
\label{09:ex:7}
Parent, question B:\\
Ich lerne Deutsch mit meinen Kindern. Meine Kinder lernen Arabisch mit mir.\\

\trans `I learn German with my children. My children learn Arabic with me.' (German, Arabic)
\z

In Excerpt (\ref{09:ex:8}), one parent stresses their own desire to understand the children speaking German.

\ea
\label{09:ex:8}
Parent, question B:\\
Ich möchte gerne meine Kinder, wenn sie Deutsch sprechen, verstehen.\\
\trans `I would like to understand my children, when they speak German.' (Spanish, German)
\z
 
The excerpts from parents’ questionnaires are instrumental in understanding some of the challenges that multilingual families face in Germany today, especially in situations where children and parents’ repertoires have larger areas where they do not overlap. For our transfer project, this situation challenged our ideas about families where one family language in addition to German would prevail and where competences would be shared between parents and children. The title of this section is Othering-by-design and building on the feedback from the questionnaires, the concept became relevant for us. Othering as a social process and a set of practices is at times intentional and also unintentional. Othering-by-design is in my understanding the process that constructs the other through poor (or ill-intentioned) services, thereby making the differences between groups relevant where they would not have been otherwise. On a general level, we could think services that require a passport of a certain country for identification when nationality is not a prerequisite. In the same way, language selections might serve as means of othering-by-design when they create bottlenecks, e.g. on websites, that keep persons from using the services that would otherwise benefit them. in our project this could mean to use our preconceived notion of multilingual families and assume that all families have and want access to the same resources. Instead of catering to the needs of specific speakers, we would invite groups who would then find themselves as overhearers at best, not being able to take part in the conversation under equal terms.

Questionnaires were filled in by 55 participants, mentioning 23 languages. This made it quite clear that we would not be able to work with only one or two languages beyond German. One challenge in transfer projects is how to turn auditors into participants in the sense that they start to interact and to appropriate the project resources for their own goals. In light of the questions that we received in the questionnaires, it became clear that parents were dealing with matters of knowledge, as well as matters of confidence (e.g. How can I keep using the languages that I find relevant vis-à-vis a monolingually oriented education system?). But we also saw that the educators and parents had quite some resources at hand and they needed to be taken seriously in their different positions. Many parents had actually quite some expertise in multilingual language transmission and were very capable of describing their preferred ways of learning – others were less sure about the paths to take but had built good support structures. The same was of course true for the instructors of teachers who had some experience of multilingual life of their own, but at times also engaged intensively with multilingual families even though they might not have been multilingual themselves.
From these insights, we realized that we would have to target speakers depending on their different needs but not necessarily in their different roles. We thus opted for a combination of informational resources and activities, inviting speakers to learn but also to reflect on their own resources and potentially to start their own citizen science activities. This also motivated us to inquire and work towards multimodal and multilingual resources and we will discuss this in more detail in the next section.

\section{Designing access}
\label{09:sec:05}
Access takes many forms – I will summarize the main aspects that we have worked with here and then go into more detail for each aspect further along. Initially, designing access is led by an idea about potential audiences and is then driven by linguistic demand and technical affordances. In the process, issues of legal requirements, i.e. licensing, and production choices come into play. All aspects are contributing to the opportunities for information material to be seen, heard and used and for the goal of the transfer project to be reached. While some decisions can be revised, others, mainly regarding the general direction of the project, are harder to change once underway.


\subsection{Potential audiences}
As described in the section on audience design, access is first and foremost shaped by the idea of access – if I am not aware that information is available, useful or entertaining for me, I will not reach the point to actually interact. And I will not find out whether contents are available in language resources that I can interact with.
Designing access so that auditors and overhearers can find their place in the conversation seems thus of greater concern: in our project experience, authors together with adressees are well placed to identify potential meeting grounds with overhearers. Institutions like schools and childcare are potential spaces of access but also parents’ associations and other interest groups can serve as proxies to shift interest from one aspect, i.e. the educational success of a child, towards more general questions of multilingual family life. In light of the answers from the parents’ questionnaires, we see entry points for exchange but we also see that actors from outside of the family are rarely mentioned when it comes to family language decisions. The idea that a family’s language policy could benefit from exchange with research or experts is thus not very prevalent in our sample. While research in Germany has shown that half of the population is generally interested in scientific results, less than one tenth of the population actually takes part in relevant events (\cite[273]{WeitzeHeckl2016}).
Learning from the project, general interest is rarely sparked through singular encounters, but can actually build up over several exchanges. In order to avoid othering, it has proven useful to see those exchanges as opportunities to learn about educational practices, social meaning and language organization in different spaces. Only through these exchanges it is easier to plan meaningful information or intervention.

\subsection{Language use}
\largerpage
Once the interest in family languages or specific aspects of the topic is sparked, access is achieved through the use of languages that are part of speakers’ repertoires. In our case, this includes multilingual formats like audio in one language and subtitles in a second to cater to repertoires in which competences are not completely parallel. After some discussion we opted for German (and some English) as the language that would help navigate the website. We had initially considered parallel versions of the site that would use different languages as matrix languages but decided against it for practical and conceptual reasons: thinking about our target audience, it is rather unlikely that German would not play any role in their language repertoire – and maintaining a site with complete parallel language versions provides a level of complexity that cannot be dealt with without competent staff. It would also call for a limited number of parallel versions (three, four, five?) in order to be able to maintain it further. By opting for German as the language of navigation, we produced a frame that can then be filled with material in an infinite number of languages. Videos are produced in the languages of the speakers and subtitles in several languages are used to ensure understanding. The subtitles are included as separate text files and can be activated according to preference. Linked to the videos are discussion questions in several languages where complete translation can be achieved more easily. The idea of this is that children and adults in group settings could read the questions in different languages of their choice and then reunite to discuss together or else bring the results from their discussions to the plenary meeting and translate together on the way. The subtitle option allows for individual use even while a screening is taking place in a group – speakers with limited proficiency in the video’s language can opt for the written version of the original language to support their learning, or the translations. It goes without saying that subtitles are also an important tool for hard-of-hearing or deaf persons.

\subsection{Technical access}
The language policies on the website are of course also linked to technical affordances. Choices of video formats, subtitling and in general online publishing are influenced by some thoughts about  potential audiences. Opting for a website seemed like a relatively accessible option for those with limited technical skills. No login is required and links can be shared that point to the exact video or tasks that should be accessed. In the process of publishing, we also encountered issues of accessibility linked to different browsers. However, by drawing on the technical framework of the University of Duisburg-Essen, we could also benefit from solutions linked to screen readers etc. that were already set up. In the future, we will also be able to integrate our videos into the public repository of the university that is linked to search engines for open educational resources. In our second project location, Mannheim, a number educational videos were produced with a full studio and we are eager to learn from the differences in production and how they are perceived by different audiences.

\subsection{Legal requirements/Appropriation}
In order to publish, legal requirements have to be considered: legal limitations can be addressed by publishing under creative commons licenses (in the form of open educational resources). Choices can be made whether the information material can be shared in its current form, whether it can be altered or even sold. We have generally opted for a non-commercial licence, meaning that the source should be indicated but that users are free to change contents (i.e. to adopt it to their needs in schools) as long as they are not using the contents commercially.

\subsection{Producing inclusivity}
\label{09:sec:06}
Finally, the impact of underlying language ideologies and societal expectations that are transmitted through teaching and information materials needs to be taken into account. By making certain speakers and their language realities visible,  the risk is that others are excluded and not represented as equally legitimate speakers. For practical as well as ideological reasons, we opted for an approach that would favour speakers’ portrayal as multilinguals, encouraging them to speak the languages they felt more or less competent in. Access in production is further enhanced by working with a variety of speakers, including heritage language speakers and speakers ``with an accent''. However, we see that we have a somewhat conscious overrepresentation of speakers with a high degree of formal education – given that our project is intending to foster transfer from research to other fields of life, this is seen as rather foreseeable as we have asked many researchers to talk about their research but also about their experiences as heritage language speakers and children of migrants to Germany.


\section{Concluding thoughts}

In this paper, I have discussed how drawing on lived experience of language helps to
understand multilingualism in practice: by learning about the addressees of transfer
activities, we sharpen our own understanding of multilingual realities. We encounter
parents in their roles as language teachers and as learners, and we see how they negotiate
confidence and insecurities. These results in dialogue with the writings of scholars working
on inclusion and on decolonizing methodologies provide grounds to question power
hierarchies that put certain experiences and groups in places that are constructed as distant
from scientific knowledge (i.e., speakers of non-standard varieties, migrants and children). In
this way, we discuss attempts to rethink notions of experts and non-experts.
While the effects of research on research participants have been discussed in greater detail,
the effects of outreach have been only marginally considered. By drawing on results from
one specific transfer project focusing on family language dynamics, my aim was to reflect on
multiple speaker positionalities, negotiations of access and address and finally the effects of
research and outreach activities.

\textit{Riding on trains gives access to new places, but it is at times a lesson in patience, when traveling at a slightly slower speed than expected. Or when encountering steps that are quite steep for the amount of baggage one carries. Ideally, outreach takes us to places that we would not have seen otherwise – and we can probably never cover all unforeseen developments beforehand. However, the journey becomes so much easier when we have some experience that tells us that it might be worth bringing duct tape, food and books, and at times, even a tea bag. Or subtitles.}


\section*{Acknowledgements}
My gratitude goes to my amazing colleagues in the project: Rosemarie Tracy, Sofia Grigoriadou and Johanna Tausch, as well as Anne Mölders, Özge Zar, Geylan Ahmed Daud, Serpil Kuzay and Bahar Yilmaz. In addition, MultiTrans' members and guests were so kind to comment on this paper and helped its finalization by providing sea-side views. Two anonymous reviewers as well as the editors should be thanked as their comments prompted the re-shaping of the paper into its present form.

The research was supported through funding by the Deutsche Forschungsgemeinschaft (DFG, German Research Foundation) for the Research Unit ``Emerging Grammars in Language Contact Situations'', project Pt (project number: 313607803).


\printbibliography[heading=subbibliography,notkeyword=this]

\end{document}
