\documentclass[output=paper]{langscibook}
\ChapterDOI{10.5281/zenodo.17132445}
\author{Artemis Alexiadou\orcid{}\affiliation{Leibniz-Zentrum Allgemeine Sprachwissenschaft (ZAS); Humboldt Universität zu Berlin}}
\title{Methodological othering through monolingual controls: How not to}
\abstract{In a substantial body of work on a particular group of multilingual speakers, namely heritage speakers, these are often described as deviating from native-like mastery in a variety of ways. In comparison with monolingual controls, heritage speakers are viewed as lagging behind. This paper addresses this issue and argues that monolingual controls should not be dispensed with, but rather they can be used to gain insights into language variation and the processes that shape monolingual and heritage grammars alike.}
\IfFileExists{../localcommands.tex}{
  \addbibresource{../localbibliography.bib}
  % add all extra packages you need to load to this file

\usepackage{tabularx,multicol}
\usepackage{url}
\urlstyle{same}

\usepackage{listings}
\lstset{basicstyle=\ttfamily,tabsize=2,breaklines=true}

\usepackage{langsci-basic}
\usepackage{langsci-optional}
\usepackage{langsci-lgr}
\usepackage{langsci-osl}
% \usepackage{./langsci/styles/langsci-lgr}
% \usepackage{./langsci/styles/langsci-osl}
% \usepackage{langsci-gb4e}

\usepackage{tikz}
\usetikzlibrary{patterns,calc}
\pgfdeclarepatternformonly{south east lines}{\pgfqpoint{-0pt}{-0pt}}{\pgfqpoint{3pt}{3pt}}{\pgfqpoint{3pt}{3pt}}{
    \pgfsetlinewidth{0.6pt}
    \pgfpathmoveto{\pgfqpoint{0pt}{3pt}}
    \pgfpathlineto{\pgfqpoint{3pt}{0pt}}
    \pgfpathmoveto{\pgfqpoint{.2pt}{-.2pt}}
    \pgfpathlineto{\pgfqpoint{-.2pt}{.2pt}}
    \pgfpathmoveto{\pgfqpoint{3.2pt}{2.8pt}}
    \pgfpathlineto{\pgfqpoint{2.8pt}{3.2pt}}
    \pgfusepath{stroke}}
    
\usepackage{stmaryrd}
\usepackage{wasysym}
\usepackage{multirow}
\usepackage{caption}
\usepackage{subcaption}
\usepackage{mathrsfs}
\usepackage{qtree}

\usepackage{linguex}


  %pminos do not split footnotes
% \interfootnotelinepenalty=10000 %Footnote in Laporte chapters has to be split SN


%\DeclareIndexNameFormat{default}{%
%\nameparts{#1}%
%\usebibmacro{index:name}%
%{\index[names]}%
%{\namepartfamily}%
%{\namepartgiveni}%
% {}% L1
% {}% L2
%{\namepartprefix}% generates spurious space L3
%{\namepartsuffix}% generates spurious space L4
%}

%  {\DeclareIndexNameFormat{default}{%
%     \usebibmacro{index:name}{\index[names]}{#1}{#3}{#5}{#7}}}

%\DeclareIndexNameFormat{default}{%
%  \usebibmacro{index:name}{\sindex[nom]}{#1}{#3}{#5}{#7}}

%\DeclareIndexNameFormat{default}{%
%  \usebibmacro{index:name}{\sindex[person]}{#1}{#3}{#5}{#7}}
%\DeclareIndexNameFormat{default}{%
%\nameparts{#1} \usebibmacro{index:name}{\sindex[person]]}{\namepartfamily}{‌​\namepartgiven}{\nam‌​epartprefix}{\namepa‌​rtsuffix}}

%\newcommand{\smiley}{:)}

%\renewbibmacro*{index:name}[5]{%
%\usebibmacro{index:entry}{#1}%
%{\iffieldundef{usera}{}{\thefield{usera}\actualoperator}\mkbibindexname{#2}{#3}{#4}{#5}}}

% \newcommand{\noop}[1]{}

%remove for final
%\overfullrule=1mm

\newcommand{\tobi}[2]}}
\renewcommand{\S}[1]{\tobi{#1}{\textsc{*}}}

% this volume references
% puts: [this volume]
% already defined: \citetv
%\newcommand{\citepv}[1]{(\citeauthor{#1} \citeyear*{#1} [this volume])}
\newcommand{\citealtv}[1]{\citeauthor{#1} \citeyear*{#1} [this volume]}

%parentheses around example number
\newcommand{\pref}[1]{(\ref{#1})}

% in-text examples

\newcommand{\lnex}[1]{\textit{#1}} %target lang word
\newcommand{\lnlit}[1]{(lit.: `#1')} %literal reading
\newcommand{\lnlat}[1]{(#1)} % latinization
\newcommand{\lntrans}[1]{`#1'} %translation
\newcommand{\lnexl}[2]%
{\lnex{#1}{} \lnlat{#2}} % ex with latinization
\newcommand{\lnexlat}[3]{\lnex{#1}{} \lnlat{#2}{} \lntrans{#3}} % ex with latinization and tranl.

%ch01
\newcommand{\co}[1]{\mbox{\textbf{#1}}}

%ch09

\newcommand{\cyrbulg}[1]{\begin{otherlanguage*}{bulgarian}#1\end{otherlanguage*}}


%ch10
\newcommand{\nlp}{{\small NLP}}
\newcommand{\mwe}{{\small MWE}}
\newcommand{\rae}{{\small RAE}}
\newcommand{\lvc}{{\small LVC}}
\newcommand{\pos}{{\small P}o{\small S}}
%\newcommand{\todo}[1]{ \textcolor{red}{#1} }

%\renewcommand{\labelenumi}{\theenumi}
%\ainamefmt{{vv}{ll}{, ff}{, jj}} % fullname

\newcommand{\biberror}[1]{{\color{red}#1}}

\newcommand{\osenovaitem}{--~} 
  %% hyphenation points for line breaks
%% Normally, automatic hyphenation in LaTeX is very good
%% If a word is mis-hyphenated, add it to this file
%%
%% add information to TeX file before \begin{document} with:
%% %% hyphenation points for line breaks
%% Normally, automatic hyphenation in LaTeX is very good
%% If a word is mis-hyphenated, add it to this file
%%
%% add information to TeX file before \begin{document} with:
%% %% hyphenation points for line breaks
%% Normally, automatic hyphenation in LaTeX is very good
%% If a word is mis-hyphenated, add it to this file
%%
%% add information to TeX file before \begin{document} with:
%% \include{localhyphenation}
\hyphenation{
    Beck-man
    Ngu-yen
    back-chan-nel
    back-chan-nels
    mo-not-o-nous
    ste-reo-typ-i-cal
}

\hyphenation{
    Beck-man
    Ngu-yen
    back-chan-nel
    back-chan-nels
    mo-not-o-nous
    ste-reo-typ-i-cal
}

\hyphenation{
    Beck-man
    Ngu-yen
    back-chan-nel
    back-chan-nels
    mo-not-o-nous
    ste-reo-typ-i-cal
}
 
  \togglepaper[1]%%chapternumber
}{}

\begin{document}
\maketitle 
%\shorttitlerunninghead{}%%use this for an abridged title in the page headers


\section{Introduction}

Typically, in empirical research participants are divided into two groups: the experimental group (the group we are interested in) and the control group. In heritage language ressearch research, the control group is composed of aged-matched monolingual speakers, and also monolingual children, and the experimental group are the heritage speakers. Thus, based on an idealized first language (L1) competence, adopting the inclusion of a control group of monolingual speakers of the language is the default in empirical research, see \citet{VulchanovaEtAl2022} for a recent discussion. However, this perspective leads to othering heritage speakers, viewing them as some sort of special group: as in most cases, such speakers do not behave the exact same way as monolinguals, the explanations that are often given are deficit oriented. The grammar of heritage speakers is non-target grammar in comparison to the target, i.e., monolingual native grammar. This has been argued to be the result of incomplete acquisition or attrition or both depending on the phenomenon \citep{Montrul2016}, the result of a bottleneck \citep{Slabakova2019}, or simply heritage grammar is in some ways defective (\citealt{PolinskyScontras2020}). The focus on non-target properties of heritage grammars have recently led Domínguez et al. (2019: 247ff.) to defend the term incomplete acquisition. As they state, in the field of L1 acquisition, it expresses the procedural character of language acquisition: grammars may fluctuate, change, develop and show sensitivity to maturational constraints. In the field of heritage language grammars, it is suggestive of a situation whereby heritage language speakers have not had the chance to learn all the available cues. The authors explicitly conclude that '{}'investigating the grammatical knowledge of heritage speakers would benefit from including at least two control groups, a group of monolingual speakers and a group of speakers representing the parental/community input that the heritage speakers receive in that context. This could help clarify whether properties not instantiated in heritage speakers’ grammars were present in the input to start with.'{}' \citet[251]{DomínguezEtAl2019}. Thus, from this perspective two and not just one control groups are necessary to assess the properties of heritage grammars.


The so-called deficit framing of heritage speakers has been criticized from a variety of perspectives, see e.g., \citet{KupischRothman2018}, \citet{BousquettePutnam2019}, \citet{HigbyEtAl2023}, \citet{WieseEtAl2021}, \citet{WieseEtAl2022Multilinguals} to mention a few. In this paper, I will align with these perspectives and propose to abandon the idea that the monolingual speaker/native speaker is the model of all different types of language acquisition and competence. My focus is on heritage speakers (henceforth HSs) of Greek. I will adopt the idea that HSs are in fact part of the native speaker range, see e.g., \citet{RothmanTreffers-Daller2014}; while I will not argue that one needs to dispense with the notion of the control group as such, I will suggest that whenever a control group is used, the point of departure should be to show how HSs opt for patterns that are also produced by monolinguals, and thus present in the input in some form, albeit in different settings. I will attempt to identify the mechanisms the different groups of speakers employ, see \citet{HigbyEtAl2023}, \citet{LukBialystok2013}, \citet{WieseEtAl2021} for insightful discussion from different perspectives. I will then discuss differences between HS and monolingual productions from the perspective of \textit{representational economy} and \textit{analyticity}, \citet{ScontrasEtAl2018}, and \citet{PutnamEtAl2021}.

The chapter is structured as follows: in section 2, I offer a definition of heritage speakers and their relation to the notion of the native speaker. In section 3, I report on several studies that show similarities and differences in the productions of HSs and native speakers of Greek. In section 4, I turn to some discussion of the findings, and in section 5 I conclude.

\section{Heritage speakers and the native speaker}

As stated in a post by \citet{GrammonBabel2021}, '{}'the idea of the “native speaker” originated within the context of European nationalism and colonialism in the 19\textsuperscript{th} century. It proved useful both as a way of conceptualizing and labeling a particular linguistic identity tied to a nation and to differentiate between social groups within a colonial hierarchy. A closer examination reveals that the concept of the “native” speaker is tightly connected to discriminatory logics.'{}', see also \citet{HigbyEtAl2023} and references therein.

Why should that be the case? As Grammon \& Babel discuss, linguists and non-linguists use the term native speaker to describe an individual that grew up speaking a particular language and who is fully proficient in that language. The point is, however, that often a certain kind of authority as to how a language should be spoken is bestowed upon native speakers, thus othering multilingual speakers, see also \citet{WieseEtAl2022Multilinguals} for extensive discussion and references.

Typically, monolinguals are seen as native speakers of their language, as in their case one can a priori exclude language interference that may create a more complex linguistic behavior. For this reason, the monolingual speaker has been in the center of investigation in the history of theoretical and experimental linguistics. The origin of this view can be found in the early pages of \textit{Aspects of the Theory of Syntax:} “Linguistic theory is concerned primarily with an ideal speaker-listener, in a completely homogeneous speech-community, who knows its language perfectly and is unaffected by such grammatically irrelevant conditions as memory limitations, distractions, shifts of attention and interest, and errors (random or characteristic) in applying his knowledge of the language in actual performance” \citep[3]{Chomsky1965}.


Thus, for a very long time, as stressed in \citet{Lohndal2013}, only monolingual speakers, treated as producing invariant speech were analyzed. This type of approach, however, leads to a situation where bilingual speakers cannot be viewed as a pure case of study, as there are too many factors interfering with the acquisition of their two grammars. The language of heritage speakers thus might be considered a rather complicated case. In other words, since there is a lot of variation among speakers in this domain, formal models of grammar cannot really be applied to the linguistic behavior of such speakers. However, we know by now that monolingual speakers do exhibit divergent outcomes, see e.g, Dąbrowska (2012) and subsequent work. Thus, if, as rightly suggested by an anonymous reviewer, such gradience is not only characteristic of bilingual populations, but is also found among monolinguals as well, the notion of the ideal speaker-hearer requires a bit of a recalibration anyways (see \citealt{Francis2022} for discussion). \citet{RothmannEtAl2023} point out that focusing on monolingual controls has as a result that the bilingual data remain under-explored. From their perspective, it is rather unfortunate to employ this comparative aspect, especially since nowadays multilingualism is the norm across the globe.


 But who are heritage speaker and are they native speakers or not? According to \citet{LohndalEtAl2019} and references therein, HSs are defined as minority language speakers in a majority language environment, they are bilinguals and by the time they are adults, they tend to be dominant in the language of their larger national community. Lack of formal education in the heritage language is associated with the low status of the heritage language.
 The question of whether HSs are native speakers has by now been addressed by many researchers in our field. In earlier research on HSs, we find terms such “vulnerability / resilience”, “attrition”, and “incomplete acquisition” to describe these speakers, see e.g., \citet{Montrul2016} and \citet{Polinsky2018} for extensive discussion and references. But clearly such terms suggest a perspective that is deficit oriented, as rightly pointed out in \citet{HigbyEtAl2023}, leading us to focus on aspects of heritage speakers’ grammar that are interpreted as lagging behind those of monolingual speakers, rather than the development of novel patterns, cf. the semi-lingualism controversies in the 1970s and 1980s.

  For instance, for \citet{Montrul2016}, HSs display characteristics of second language learners, i.e., not native like in some modules of grammar, while they maintain native language mastery in other modules of grammar. As Montrul (op.cit.) details, HSs may exhibit some sort of ‘not target-like’ development (incomplete acquisition) or they show signs of language loss/decline (attrition). These two processes have rather different developmental paths: if changes are due to attrition, then we are dealing with properties that were acquired in childhood but were later subject to language loss or gradual decline; if changes are due to incomplete acquisition, then a particular area has not been mastered yet. As further stated in \citet{Montrul2016}, a speaker may show attrition in some areas that are acquired in pre-school age (e.g., gender), and incomplete acquisition in others that take several years to develop (e.g., passives).


  \citet[28]{Polinsky2018} opts for the term divergent attainment and concludes: ``the crucial point is that the divergence (or innovation) in the heritage grammar is systematic. In that regard, divergent attainment is different from attrition and transfer; the latter two may be less systematic, whereas the former results in a coherent grammar.’’ The reasons for this divergent attainment maybe diverse. 


 {  \citet{RothmanTreffers-Daller2014} have already pointed out that while most linguists would agree that the terms monolingual and native should not in principle be synonymous, the fact remains that linguists often use these terms interchangeably. But if we understand a native language to be acquired from naturalistic exposure, in early childhood and in an authentic social context/speech community, then the authors stress these criteria of native-hood apply to HS bilingualism as well.}

  More recently, \citet{WieseEtAl2021} provide several arguments as to why HSs are native speakers of both their languages. \citet{HigbyEtAl2023} problematize the way such ideologies appear across frameworks and advocate multicompetence and models that allow nuance and complexity. It is precisely this nuance and complexity that is quite hard to deal with from a generative grammar perspective, see the discussion of \citet{Lohndal2013} above. In \citet{Alexiadou2017}, I have also suggested that we should move away from focusing on what HSs cannot do, but rather examine their grammars as independent linguistic systems. In fact, my point is that what we perceive as 'odd' patterns are changes in the grammar, which emerge whenever there is variability in the realization of particular structures, in the sense of \citet{Adger2014}. While \citet{DomínguezEtAl2019}, as we have already seen, suggest that lack of cues leads to incompleteness, I will argue that nuance emerges when there is more than one option for the realization of a particular feature or structure in the grammar. In such cases, HSs may opt for one of the options driven by general mechanisms, as we will discuss in section 4. The question now is where we find variability. Register variation is a good place to look at and this what I will do next by reviewing recent case studies on heritage-Greek. The point will be that whenever there is an option between two or maybe more strategies, one of the options will be associated with a particular register and because of this perhaps prevail in the heritage grammar. By contrast, in cases the grammar leaves no option, novel patterns may emerge as a result of re-structuring, see also \citet{PutnamEtAl2021} for a summary of finding in different heritage languages.

\section{Case studies}

I will report here on two types of studies carried out within the frame of the research unit 2537 \textit{Emerging Grammars}. I will discuss three case studies involving agreement mismatches in a) restrictive relative clauses (RRCs) (\citealt{AlexiadouRizou2022}), b) gender agreement \citep{AlexiadouEtAl2021}, and c) adjectival agreement \citep{AlexiadouEtAl2023}. All these involve agreement in the context of DPs. I will contrast these results with the use of periphrastic constructions instead of lexical verbs in heritage-Greek (\citealt{AlexiadouRizou2023}). In all those studies, we compared HSs of Greek in the US, in Chicago and New York, to monolingual speakers of Greek residing in Athens. The studies were chosen as with respect to RRCs and periphrastic constructions, we have very good examples of options available to speakers of Greek for the realization of a particular structure that have been argued to be subject to register variation in the language. On the other hand, in the context of pronominal, gender and adjectival agreement, there is no option within the Greek grammar, adjectives, relative pronouns, and determiners all must agree with the head noun in gender and number. Moreover, there is a substantial body of work on heritage languages that shows that gender and DP internal agreement are domains that are affected by restructuring, see e.g., (\citealt{BenmamounEtAl2013,Bolonyai2007,MontrulEtAl2012,AlbiriniEtAl2013,Fuchs2019}, \citealt{PutnamEtAl2021} and references therein). As we will see, the results support the view that heritage grammars do not subsume to one type of re-organization only, see also \citet{BousquettePutnam2019}, and \citet{PutnamEtAl2021} for a similar claim, cf. \citet{Montrul2016}: HSs pursue different directions depending on the domain of investigation of the type described in \citet{BousquettePutnam2019} for Heritage German in the US and in \citet{PutnamEtAl2021} for fusional languages in general, see also \citet{ScontrasEtAl2018}. Thus, since different directions are chosen, the question is what this can be attributed to.


\subsection{Methodology}

The methodology we employed was the following: we elicited naturalistic data in oral and written modality in two distinct communication settings from the aforementioned groups of speakers. This methodology is referred to as “Language situations” methodology, (see \citealt{Wiese2020}). This set-up provides comparable data in both the heritage and majority language in two different levels of formality and in two modalities (data sets 2x2). The data were elicited by native speakers of Greek and American English paying attention to the respective setting. Four communication settings were created (formal spoken, formal written, informal spoken, informal written). Half of our HSs were tested first in their heritage language and afterwards in their majority while the other half of our HSs were tested first in their majority language and then in their heritage. This particular design prevents our participants from biases. Monolingual controls took part only in one language. A short video (00:42 minutes) of a fictional event was shown to every participant. A non-severe car accident was taking place in a parking lot and the task was to retell what happened to a friend and a police officer respectively imagining that they witnessed the incident. They had to produce both an oral (voice messages) and a written narration (\textit{whats app} and written police report) in two distinct communication settings.

\subsection{Restrictive relative clauses in heritage-Greek}
 It has been reported that HSs of several languages encounter difficulties with relative clauses, potentially related to processing difficulties, see \citet{Polinsky2018} for an extensive summary. \citet{AlexiadouRizou2022} examined how HSs of Greek in the US would use RRCs and whether Greek RRCs would show signs of re-structuring. The research question was motivated by discussion in the literature, according to which in both languages there is variation with respect to the realization of the form of the relative pronoun, arguably conditioned by formality.


Greek RRCs appear post-nominally and are introduced either by a) the relative pronoun \textit{o opios} lit. `the who' which agrees in gender and number with the nominal head it modifies or b) the complementizer \textit{pu} `that', which is an un{}-inflected element. It has been argued that in Greek, the choice between \textit{pu} and the agreeing form is regulated by modality and formality. Specifically, \citet{HoltonEtAl1997} signal that Greek RRCs introduced by \textit{pu} appear predominantly in informal, while \textit{o opios} RRCs in formal registers.{Greek RRCs appear post-nominally and are introduced either by a) the relative pronoun \textit{o opios} lit. `the who' which agrees in gender and number with the nominal head it modifies or b) the complementizer \textit{pu} [that], which is an un-inflected element. It has been argued that in Greek, the choice between \textit{pu} and the agreeing form is regulated by modality and formality. Specifically, \citet{HoltonEtAl1997} signal that Greek RRCs introduced by \textit{pu} appear predominantly in informal, while \textit{o opios} RRCs in formal registers.}

 {{  English RRCs in the standard language are introduced by either a) a} {\textit{wh-element, who}}{ for persons, and} {\textit{which}}{ for animals/things, both non-agreeing, or b) the complementizer} {\textit{that}}{. There is also a third option, the zero form, which Greek lacks, e.g.,} {\textit{This is the house which/that/Ø I told you about}}{, from \citet[148]{GuyBayley1995}. \citet{GuyBayley1995} report that the choice of form shows typical patterns of socio-linguistic variation and depends on the type of antecedent (reference to humans), the mode of communication (speech vs. written) and the adjacency of the antecedent. Specifically, they point out that there seems to be an inhibition towards using} {\textit{that}}{ to refer to humans.} {\textit{Wh}}{{}-elements are used mainly in formal writing, while} {\textit{that} }{is favoured in informal settings. In theory then,} {\textit{pu}}{ and} {\textit{that} }{seem to have similar distributional patterns.}}

In \citet{AlexiadouRizou2022}, we found out that Greek monolingual controls produced numerically more \textit{pu} RRCs compared to HSs, see table 1 from \citet[137]{AlexiadouRizou2022}. However, contrary to our expectation and \citet{HoltonEtAl1997}, both groups used more \textit{pu} RRCs in the formal register. Modality did not seem to affect the use of \textit{pu}: 


\begin{table}
\begin{tabularx}{\textwidth}{lYY}
\lsptoprule
 & HSs & Monolingual controls\\
 \midrule
Formal spoken & 135 & 123\\
Formal written & 93 & 129\\
Informal spoken & 101 & 111\\
Informal written & 61 & 85\\
\midrule
& 390 & 448\\
\lspbottomrule
\end{tabularx}
\caption{Production of pu RRCs across registers and modalities}
\label{tab04:1}
\end{table}

With respect to \textit{o opios} RRCs, in \citet{AlexiadouRizou2022}, we noted that monolingual speakers of Greek produce quantitatively more such clauses in contrast to heritage speakers (see table 2, from \citealt{AlexiadouRizou2022}: 138). Monolingual controls seemed to favor the production of this type of clauses in the formal register, while modality again did not seem as influential.


\begin{table}
\begin{tabularx}{\textwidth}{lYY}
\lsptoprule
 & HSs & Monolingual controls\\
 \midrule
Formal spoken & 3 & 88\\
Formal written & 3 & 61\\
Informal spoken & 0 & 39\\
Informal written & 2 & 23\\
\midrule
& 8 & 211\\
\lspbottomrule
\end{tabularx}
\caption{Production of o opios RRCs across registers and modalities}
\label{tab04:2}
\end{table}

 {We concluded that \textit{pu} is not uniquely associated with informal settings, across groups. For the HSs, \textit{o opios} RRCs are avoided, and we attributed this to the fact that the pronoun agrees in phi-features (gender and number) with the head noun it modifies. As we will see in the next case study, Greek HSs in the US avoid producing agreement dependencies within the NP \citep{AlexiadouEtAl2021}, thus, as a result, RRCs introduced by the pronoun \textit{o opios} are avoided so that our speakers do not find themselves in a position, where they need to track agreement. This seems to support e.g., \citegen{Slabakova2019} view that agreement constitutes a potential 'bottleneck'. This is further corroborated by the fact that, as we showed in \citet{AlexiadouRizou2022}, they have no problem using \textit{who} and \textit{which} forms that do not exhibit any (gender/number) agreement in their English productions. In conclusion, in this case study a particular structure is avoided, and a less complex form, which is void of phi features, is used.}
\subsection{Gender agreement in heritage-Greek}

The above results are correlated with the observations made in \citet{AlexiadouEtAl2021} concerning gender agreement in US Greek, which we conducted with the same group of speakers. We noted that some of our speakers fail to assign gender to Greek nouns, Greek being a three-gender language, either by avoiding using the determiner or by resorting to default neuter. Not surprisingly from the perspective of the study on RRCs the most affected category in terms of agreement was the relative pronoun, where basically the form was rarely produced, as detailed in the previous section. Importantly, however, in \citet{AlexiadouEtAl2021} we found no correlation with formality or modality and avoidance of gender assignment. While it can be argued that interference with English, a language that lacks grammatical gender, is the culprit, a surprising finding was the inconsistency in patterns of DP internal agreement. Since, as discussed in \citet{Polinsky2018}, gender is a category subject to re-structuring in heritage grammars, the Greek data further align with results on other languages, which show that the processing of agreement within noun phrases signals a processing difficulty. Monolinguals do not show any gender agreement mismatches.


\subsection{Adjectival agreement in heritage-Greek} %3.4. /

The fact that US speakers of heritage-Greek have problems with DP internal agreement was further corroborated by the results in \citet{AlexiadouEtAl2023}, where we found differences in the production of mismatches between monolinguals and US HSs: prenominal adjectival agreement was indeed more vulnerable than subject verb agreement. Specifically, adolescent speakers in the US show 17.5\% of errors. For those speakers that produced errors, we further noted that some of them repeated the mismatched agreement structure in the second narration and then avoided using the same structure. Others, however, produced a mismatch in their first narration and then produced the correct form in the following narrations. The fact that we do find errors DP internally, points to the same conclusion: once more, these results align with previous literature pointing to issues with agreement within the nominal domain. Monolinguals do not show any agreement mismatches.


\subsection{Periphrastic constructions in heritage-Greek}

A different picture emerges in the domain of the lexicon, discussed in \citet{AlexiadouRizou2023}. Here using the same corpus and the same groups of speakers, the focus was on the question of whether HSs speakers show effects of re-organization in the verbal domain. In particular, the question was whether they use periphrastic constructions (PCs) instead of lexical verbs. 

What prompted this question is the fact that the Greek lexicon has a mixed character, as the result of diglossia \citep{Ferguson1959}. According to \citet{AnastasiadisSymeonidisAsimakis2003}, \citet{Ralli2004} and \citet{Efthymiou2017}, the feature [+learned] characterizes words that either a) come from Ancient Greek or b) constitute artificial formations of ‘katharevusa’. The feature [-learned] applies to words that have a popular origin. [-learned] elements are used in informal or spoken (colloquial) speech, while [+learned] ones are used in refined or written speech. Typically, prefixed verbs and non-active morphology on verbs belong to the so-called [+learned] features and thus verbs that are built on the basis of these prefixes are [+learned]. Precisely these verbs were the focus of our study. We expected that HSs should behave differently from monolinguals due to lack of formal education, in view of the fact that [+learned] features are acquired only via formal education.

\tabref{tab:04:3}, from \citet[98]{AlexiadouRizou2023} shows the raw numbers and the percentages of the speakers who produced PCs across registers.


\begin{table}
\begin{tabularx}{\textwidth}{llYY}
\lsptoprule
 & \textbf{y} & HSs in the US & Control group\\
 \midrule
No PCs &  & 32/ 50.8\% & 40/ 62.5\%\\
Formal &  & 12/ 19.0\% & 6/ 9.4\%\\
Informal &  & 4/ 6.3\% & 9/ 14.1\%\\
Both formal and informal &  & 15/ 23.8\% & 9/ 14.1\%\\
\lspbottomrule
\end{tabularx}
\caption{Use of PCs. No PCs’ indicates the speakers who did not use any PCs in their narratives and the row ``Both formal and informal'' indicates the number of speakers who used PCs in both settings.}
\label{tab:04:3}
\end{table}

We noted that the number of HSs that produced PCs in the formal register is higher than that of monolinguals. Monolingual speakers also use PCs in the formal register, which was not expected. Fewer HSs in the US produce PCs in the informal register, which was not expected, and can be explained by fact that the informal US narratives are much shorter compared to their formal counterparts. Monolinguals use PCs as well but preferably in the informal settings. Some examples of the PCs and the lexical verbs we found are given in table 4, from \citet[99]{AlexiadouRizou2023}:


\begin{table}
\begin{tabularx}{\textwidth}{Ql}
\lsptoprule
\textbf{PCs} & \textbf{Lexical verb}\\
\midrule
\textit{Ekane freno}, \textit{evale ta frena}, \textit{ekane brake=} do brake & \textit{Frenare =} brake\\
\textit{pire attention} = pay attention & \textit{Siniditopiise} = realise\\
\textit{Pire agalia}= take a hug & \textit{Agaliase} = hug\\
\textit{Kanun erevna} = do an investigation & \textit{Erevnun =} investigate\\
\textit{Kanune parking}= do parking & \textit{Ihe stathmefsi} = park\\
\textit{Den ihe ora (na)}= don’t have time to & \textit{Prolavene (na) =} catch up\\
\textit{Ekane stop}= do stop & \textit{Stamatise} = stop\\
\textit{Kano report =} do a report & \textit{(na)katatheso  =} testify\\
\lspbottomrule
\end{tabularx}
\caption{Types of PCs used by HSs vs lexical verbs used by the control group}
\label{tab04:4}
\end{table}

Our prediction was thus borne out: because the Greek lexicon has a mixed character, HSs behave differently from monolinguals due to lack of formal education. We attributed this to the fact that since speakers are exposed to [+learned] elements only in secondary education, HSs receive much less input: their formal education is achieved by attending either bilingual schools or afternoon/ weekend schools, and arguably the hours of instruction are comparatively fewer than in the case of monolinguals. Because of this, as predicted, both the monolingual group and the HSs replace lexical verbs that bear [+learned] features in the informal register, and HSs use PCs in both registers. The importance of formal education has also been recently discussed for Heritage Persian in \citet{GharibiEtAl2023}, who show that literacy seems to be the most important factor for variation in heritage language outcomes. With respect to our analysis of the pattern, we assumed that both the lexical verb and the PC lexicalize the same structure, the difference being that the PC spells-out each subcomponent of the verbal meaning with a dedicated form, the light verb corresponding to a causative layer and the nominal form corresponding the result component, while the lexical verb realizes both sub-parts via a single element.

  In conclusion, in our case studies, monolingual speakers behave differently from HSs in the domain of agreement, where no mismatches are observed. When it comes to the selection of a particular structure, where the choice is guided by register (and/or modality), monolinguals also show the same patterns as HSs in their choice of variant, as is the case with \textit{pu} relatives and PCs. By contrast, register does not seem to play a significant role in the choice of realization of agreement in different contexts.

\section{Discussion}

The above case studies show two different patterns in the same group of speakers, one of which, however, also characterizes the monolingual control group. In the first study, both HSs and monolingual speakers alike use \textit{pu} relatives irrespectively of formality. The difference between the two groups amounts to the avoidance of the agreeing form, found only with HSs. In the agreement mismatches studies, monolinguals do not produce mismatches, while US speakers show agreement restructuring. In other words, in these cases we have suggestive evidence that HSs of Greek use less complex representations. In the case of the lexicon, HSs use PCs, generalizing a pattern found also in monolingual productions, where it is associated with informal settings. Unlike in the first case study, here HSs show effects of use of a more analytic pattern. By contrast, in the case of RRCs, they use a more complex representation. Why should they behave differently in all these case studies?

{To approach this question, I revisit the recent discussion of the mechanisms that shape heritage grammars in \citet{ScontrasEtAl2018}, see also \citet{LohndalPutnam2024a}: according to \citet{ScontrasEtAl2018}, two pressures shape heritage grammars, \textit{representational economy} and \textit{analyticity}. In their terms, in the first case, the heritage grammar somehow eliminates while in the latter it augments structures in comparison to the patterns used by monolingual speakers. Specifically, HSs in some cases prioritize representational economy, restructuring their grammar in favor of lighter-weight linguistic representations: less articulated, more parsimonious structures (e.g., structures with fewer explicit agreement features) could ease the load on working memory and might therefore be preferred to their fully-articulated structures. By contrast, \textit{Analyticity} favors one-to-one correspondence between form and meaning. This pattern suggests that HSs prefer one-to-one form-meaning correspondences, which are more transparent, (see also \citealt{Alexiadou2021}).}

{It is interesting that in our case studies the patterns shown go in very different directions. In the studies which involve agreement, we have use of forms that basically lack features or inconsistent patterns. In other words, as also discussed in \citet{ScontrasEtAl2018}, HSs restructure their agreement categories and favoring fewer feature values and perhaps less structure. The interesting thing in the case of Greek RRCs is that the form with less features is an option in the Greek grammar to begin with, also used by monolinguals and it looks like HSs generalize this. This is not the case in the DP internal agreement patterns, where a restructuring is observed. Here there is no option in the grammar, nouns, adjectives and articles have to agree in number and gender. The same applies also in the lexicon case, where the PC is an option next to the lexical verb. Thus, one may ask, as \citet{ScontrasEtAl2018} do, whether it is possible to predict which domains may deliver more parsimonious representations, and which may increase analyticity in HS grammars but also in general, i.e., beyond the HS debate.}
 {While we might not yet have a full empirical picture of how to answer this yet, an important insight of this paper is that the two mechanisms \citet{ScontrasEtAl2018} discuss are not unique to heritage grammars, but also shape monolingual grammars as well; this becomes evident once we take into consideration how speakers use variants of the same feature/structure. Whenever there is alternation between two (or maybe more) forms, a speaker may prefer one of the two. In the RRC study, it was the form without agreement, the less transparent form, in the case of the lexicon, it was the more explicit one. In the latter case, the choice was influenced by register. This in turn suggests that depending on the domain of grammar, lexicon vs. phi-features, different mechanisms may be exploited, see also \citet{BousquettePutnam2019}, and importantly also by monolingual speakers. Analyticity typically correlates with less formal registers, while representational economy does not seem to be dependent on formality. Importantly, this entails that the use of forms with and without agreement does not correlate with register, at least in Greek RRCs, unlike the situation in English RRCs, where the distribution of wh-elements as opposed to complementizer is regulated by socio-linguistic factors. Rather, it is employed because it is correlates with expression of fewer or rather no features. Agreement is one area, where as \citet{PutnamEtAl2021} state, while robust in the input, shows restructuring across heritage languages. According to \citet{PutnamEtAl2021}, this is particularly clear with fusional languages such as Greek. Finally, note that the agreement mismatches identified are all in the DP domain. A way to explain this is to assume that the mechanisms involved DP internally to produce agreement, namely concord, are more complex that those in the verbal domain, see \citet{AlexiadouEtAl2023} for discussion and references.}

 \section{Conclusions}
 In this paper, I presented a view on monolinguals not as controls in a normative sense, but as a group providing information on speaker variation. In order to better understand the mechanisms involved in heritage grammars that lead to differences between two varieties of the same language or group of speakers, it is important to look at variation within monolingual speech. It is not the case that heritage languages are less complex: they employ patterns also produced by monolinguals. We thus need to investigate a variety of domains of language use to have a more complete picture of what HSs can do and how they differ from monolingual speakers. I showed that if we focus on register comparison, in some cases we are able to present a different narrative about how HSs differ from their monolingual 'controls'. Moreover, it became clear that, at least in Greek, lexicon choice signal register variation, perhaps unsurprisingly. By contrast, use of forms with fewer or less phi-features does not correlate with register variation. In general, our case studies align with the trends \citet{PutnamEtAl2021} identified for fusional languages: towards simplification, on the one hand, and analyticity on the other.

\section*{Acknowledgements}

I am indebted to two reviewers for their extensive comments as well as to the participants to the \textit{Othering} Workshop in July 2022 in Berlin. Special thanks to Vasiliki Rizou for extensive work on and with the RUEG corpus. This research was funded by the Deutsche Forschungsgemeinschaft grants AL 554/13-1, 394836232, and AL 554/15-1, 313607803.

\sloppy\printbibliography[heading=subbibliography,notkeyword=this]
\end{document}
