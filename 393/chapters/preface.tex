\addchap{Preface}
 

The first sparkle of this volume began with intensive discussions within the framework of the research group “Emerging Grammars in Language Contact Situations” (RUEG), funded by the German Research Council (FOR 2537). Three of the editors were involved in RUEG with research projects. Claudio, whose discourse-analytical work on the German migration debate provides an important additional aspect, quickly joined the discussion. The first result of our collaboration was a joint working paper entitled “Multilinguals as Others in society and academia: Challenges of belonging under a monolingual habitus”. We are very grateful to Ben Rampton for constructive comments on a previous version of this paper and for making it possible for us to publish it in the Working Papers in Urban Language \& Literacies (Paper 302, 2022). The positive reactions to the working paper in turn gave us the idea of addressing the topic of “Othering” in the context of multilingualism research in a broader context in a workshop that took place at the Humboldt Forum Berlin on 11 July 2022. The workshop resulted in this volume. All contributions have undergone the elaborate multi-stage review process which LangSci envisages. We thank the internal and external reviewers for their valuable input on earlier versions of the papers, and we are very grateful to all authors for their high-quality contributions. In the final stage of preparation, we gratefully received numerous comments in the crowd review process, and in dealing with these, Lea Coy’s expertise was truly invaluable. Many thanks are also due to İrem Duman Çakır for organizational assistance. Last, but not least, we are grateful to the series editors of the \emph{Contact and Multilingualism} series, Isabelle Léglise and Stefano Manfredi, for accepting the volume and to the publisher, \textit{ad personam} especially Sebastian Nordhoff, for the technical support during its production.


\begin{flushright}
Artemis Alexiadou\\
Claudio Scarvaglieri\\
Christoph Schroeder\\
Heike Wiese
\end{flushright}
