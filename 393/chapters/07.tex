\documentclass[output=paper]{langscibook}
\ChapterDOI{10.5281/zenodo.17132451}
\author{İnci Dirim\orcid{}\affiliation{Universität Wien}}
\title{Multilingualism and identity construction in pedagogical discourse}
\abstract{This chapter inquiries into the relationship between multilingualism and identity construction within pedagogical contexts, particularly in the domain of German as a second language. It traces the evolution of the concept of linguistic identity in educational discourse, highlighting its roots in migration studies and multiculturalism debates of the 1990s. It critiques the paternalistic arguments that often accompany these discussions and underscores the significant impact educational policies and practices have on students' self-conception and societal integration. Using subjectivizational perspectives, and analysing a recent European Commission report, the text scrutinizes the problematic implications of associating language with identity, especially in the context of power dynamics and societal attributions.}
\IfFileExists{../localcommands.tex}{
  \addbibresource{../localbibliography.bib}
  % add all extra packages you need to load to this file

\usepackage{tabularx,multicol}
\usepackage{url}
\urlstyle{same}

\usepackage{listings}
\lstset{basicstyle=\ttfamily,tabsize=2,breaklines=true}

\usepackage{langsci-basic}
\usepackage{langsci-optional}
\usepackage{langsci-lgr}
\usepackage{langsci-osl}
% \usepackage{./langsci/styles/langsci-lgr}
% \usepackage{./langsci/styles/langsci-osl}
% \usepackage{langsci-gb4e}

\usepackage{tikz}
\usetikzlibrary{patterns,calc}
\pgfdeclarepatternformonly{south east lines}{\pgfqpoint{-0pt}{-0pt}}{\pgfqpoint{3pt}{3pt}}{\pgfqpoint{3pt}{3pt}}{
    \pgfsetlinewidth{0.6pt}
    \pgfpathmoveto{\pgfqpoint{0pt}{3pt}}
    \pgfpathlineto{\pgfqpoint{3pt}{0pt}}
    \pgfpathmoveto{\pgfqpoint{.2pt}{-.2pt}}
    \pgfpathlineto{\pgfqpoint{-.2pt}{.2pt}}
    \pgfpathmoveto{\pgfqpoint{3.2pt}{2.8pt}}
    \pgfpathlineto{\pgfqpoint{2.8pt}{3.2pt}}
    \pgfusepath{stroke}}
    
\usepackage{stmaryrd}
\usepackage{wasysym}
\usepackage{multirow}
\usepackage{caption}
\usepackage{subcaption}
\usepackage{mathrsfs}
\usepackage{qtree}

\usepackage{linguex}


  %pminos do not split footnotes
% \interfootnotelinepenalty=10000 %Footnote in Laporte chapters has to be split SN


%\DeclareIndexNameFormat{default}{%
%\nameparts{#1}%
%\usebibmacro{index:name}%
%{\index[names]}%
%{\namepartfamily}%
%{\namepartgiveni}%
% {}% L1
% {}% L2
%{\namepartprefix}% generates spurious space L3
%{\namepartsuffix}% generates spurious space L4
%}

%  {\DeclareIndexNameFormat{default}{%
%     \usebibmacro{index:name}{\index[names]}{#1}{#3}{#5}{#7}}}

%\DeclareIndexNameFormat{default}{%
%  \usebibmacro{index:name}{\sindex[nom]}{#1}{#3}{#5}{#7}}

%\DeclareIndexNameFormat{default}{%
%  \usebibmacro{index:name}{\sindex[person]}{#1}{#3}{#5}{#7}}
%\DeclareIndexNameFormat{default}{%
%\nameparts{#1} \usebibmacro{index:name}{\sindex[person]]}{\namepartfamily}{‌​\namepartgiven}{\nam‌​epartprefix}{\namepa‌​rtsuffix}}

%\newcommand{\smiley}{:)}

%\renewbibmacro*{index:name}[5]{%
%\usebibmacro{index:entry}{#1}%
%{\iffieldundef{usera}{}{\thefield{usera}\actualoperator}\mkbibindexname{#2}{#3}{#4}{#5}}}

% \newcommand{\noop}[1]{}

%remove for final
%\overfullrule=1mm

\newcommand{\tobi}[2]}}
\renewcommand{\S}[1]{\tobi{#1}{\textsc{*}}}

% this volume references
% puts: [this volume]
% already defined: \citetv
%\newcommand{\citepv}[1]{(\citeauthor{#1} \citeyear*{#1} [this volume])}
\newcommand{\citealtv}[1]{\citeauthor{#1} \citeyear*{#1} [this volume]}

%parentheses around example number
\newcommand{\pref}[1]{(\ref{#1})}

% in-text examples

\newcommand{\lnex}[1]{\textit{#1}} %target lang word
\newcommand{\lnlit}[1]{(lit.: `#1')} %literal reading
\newcommand{\lnlat}[1]{(#1)} % latinization
\newcommand{\lntrans}[1]{`#1'} %translation
\newcommand{\lnexl}[2]%
{\lnex{#1}{} \lnlat{#2}} % ex with latinization
\newcommand{\lnexlat}[3]{\lnex{#1}{} \lnlat{#2}{} \lntrans{#3}} % ex with latinization and tranl.

%ch01
\newcommand{\co}[1]{\mbox{\textbf{#1}}}

%ch09

\newcommand{\cyrbulg}[1]{\begin{otherlanguage*}{bulgarian}#1\end{otherlanguage*}}


%ch10
\newcommand{\nlp}{{\small NLP}}
\newcommand{\mwe}{{\small MWE}}
\newcommand{\rae}{{\small RAE}}
\newcommand{\lvc}{{\small LVC}}
\newcommand{\pos}{{\small P}o{\small S}}
%\newcommand{\todo}[1]{ \textcolor{red}{#1} }

%\renewcommand{\labelenumi}{\theenumi}
%\ainamefmt{{vv}{ll}{, ff}{, jj}} % fullname

\newcommand{\biberror}[1]{{\color{red}#1}}

\newcommand{\osenovaitem}{--~} 
  %% hyphenation points for line breaks
%% Normally, automatic hyphenation in LaTeX is very good
%% If a word is mis-hyphenated, add it to this file
%%
%% add information to TeX file before \begin{document} with:
%% %% hyphenation points for line breaks
%% Normally, automatic hyphenation in LaTeX is very good
%% If a word is mis-hyphenated, add it to this file
%%
%% add information to TeX file before \begin{document} with:
%% %% hyphenation points for line breaks
%% Normally, automatic hyphenation in LaTeX is very good
%% If a word is mis-hyphenated, add it to this file
%%
%% add information to TeX file before \begin{document} with:
%% \include{localhyphenation}
\hyphenation{
    Beck-man
    Ngu-yen
    back-chan-nel
    back-chan-nels
    mo-not-o-nous
    ste-reo-typ-i-cal
}

\hyphenation{
    Beck-man
    Ngu-yen
    back-chan-nel
    back-chan-nels
    mo-not-o-nous
    ste-reo-typ-i-cal
}

\hyphenation{
    Beck-man
    Ngu-yen
    back-chan-nel
    back-chan-nels
    mo-not-o-nous
    ste-reo-typ-i-cal
}
 
  \togglepaper[1]%%chapternumber
}{}

\begin{document}
\maketitle 
%\shorttitlerunninghead{}%%use this for an abridged title in the page headers

\section{Introduction}

Identity is a category frequently referred to in pedagogical discourse as well as in the field of German as a second language. When, why and how this category was introduced is not altogether clear. Possibly, it happened parallel to or interwoven with the development of the (German) discourse on migration, especially that of the 1990s. In those years, in educational migration research, but also in sociology, an intense debate about models of coexistence in the migration society started. In contrast to assimilative approaches, ideas of a multicultural society were developed, in the center of which was the ``affirmation of the diversity of cultural ways of life and identities'' (\citealt{CastroVarelaMecheril2010}: 50 f., transl.: İD). \citet{CastroVarelaMecheril2010} summarize the discussion in the following way: ``If assimilationism is concerned with the dissolution of difference in a model that affirms the primacy of the nonimmigrant way of life, multiculturalism advocates a different perspective, namely the recognition of cultural identities, including especially those of immigrant minorities'' (transl.: İD). Language was also quickly identified as a central feature of different cultural ways of life, most probably because it is less abstract than culture – it constitutes a visible and audible diversity.\footnote{Religion was increasingly regarded as another characteristic of culture (\citealt{DonlicYildiz2024}). This domain is not addressed in this article. On the connection between the categories of language, race and religion, see \citealt{Thoma2022,Rühlmann2023}.}  This concept of multiculturalism included the suggestion of cultivating so-called migrant languages. The argument was that the languages of origin of those identified as migrants\footnote{In the nineties, the term ``migrant'' largely replaced the term ``foreigner'', which was perceived as racist. Later, in the course of the growing political recognition of migration as a social fact in Germany, the term ``migration background'' (\citealt{PerchinigTroger2011} in \citealt{KnappikMecheril2018}:169) emerged. However, the term was subsequently strongly criticised, partly due to its imprecision and exclusionary nature (ibid.). In this article, following the criticism, this term is only used when necessary and in a distanced manner.} should be taken into account because of their importance for the identity of these people. This sometimes paternalistic argument was quickly taken up in pedagogical contexts, such as in the field of German as a second language. In this field, there was and still is an interest not only in teaching German, but in languages in general, in migration, the development of the society, and above all in the people of the society (see also Chp. 6). Through their direct contact with the learners, teachers of German as a second language also became and still are experts on migration issues, not least because this area is strongly interwoven with migration policies and educational policies. Thus, it is in the field of German as a second language where the ‘identity argument’ developed, i.e. the idea that origin and thus languages of origin are centrally important and worthy of protection. The argument continues to be used in educational policy as well as in academic contexts. An example for a policy context is the position paper of the Integration Council of the German federal state of North Rhine-Westphalia, where it says: ``The appreciation of a person's language of origin is at the same time the appreciation of a person's identity – this applies in a very special way to children and adolescents'' (\citealt{LandesintegrationsratNRW2022}, transl.: İD). What ideas lie behind these arguments and references? What can it mean for people who are considered to have a migration background, to be associated with these ideas? In this chapter, I will adopt a subjectivizational perspective. The subjectivizational approach encourages thinking about backgrounds and attributions that imply a connection between language and identity (\citealt{Bjegač2020}; \citealt{Pokitsch2022}; \citealt{Rühlmann2023}). In the following, the subjectivizational implications that the claim of the unity of language of origin with identity can have, will be presented and discussed in detail.\footnote{This article contains revised and updated passages from the publication \citealt{DirimHeinemann2016}. I am grateful to the editors and to the anonymous reviewers for helpful suggestions.}

\section{Identity, language and culturalization/lingualization}

In everyday contexts, as well as in educational and didactic models of dealing with bilingualism and multilingualism, ideas of identity are often assumed in which a stable identity, tied to a certain origin or language, forms the center of the personality. According to these conceptions, in the acquisition of a second language, the acquisition of knowledge certainly takes place, but the formation of identity continues to rely on reference back to a speaker’s origin and their language of origin. This notion of origin as the factor determining a person’s identity presumably goes back to Erikson's theory of the development of identity in a so-called life cycle: ```Identity' thus expresses a reciprocal relationship insofar as it encompasses both a permanent being-same-self and a permanent participation in certain group-specific traits'' (\citealt{Erikson1989}: 124; s. also \citealt{Erikson1968,Erikson1982}). Identity in these conceptions is thus perceived of as a trait that is permanently inherent in people and permeates the course of life. It is assumed that there is such a thing as a healthy development of identity, which has the strength to refer back to oneself again and again in the crises of life, and from which one gains the strength to continue to exist with dignity. Identity, in this theoretical orientation, is more or less given from birth, so that there is a natural connection to origin, which includes belonging to a concrete primordial group. Such a teleological view of identity is also evident in approaches to multilingualism which claim that cultural and linguistic (partial) identities need to be fostered so that those fostered can arrive at a balanced development of their total personality. In higher education, one frequently finds the argument in student papers that learners’ mother tongues have to be integrated into the classroom so that the development of their identity can proceed in a healthy way.

Even a quick search on the Internet will yield a large number of articles from different pedagogical approaches that support this approach. An example is the web page ``Erziehungskunst'' (``Art of Education'') on Waldorf education: ``Thus, not only the second language and its cultivation has a fundamental importance for the overall development of children from language minorities, but also the cultivation of the first language. Many members of the third generation of immigrants know the home countries of their parents and grandparents only from vacation, and they often grow up with two languages without being able to identify with either of them. Accommodation to the external ways of life, to the social habits and to the legal norms of the language majority should therefore not be understood as assimilation. The foundation of a healthy personality development can only lie in the harmony between one's own cultural identity and the appreciation and openness towards other cultures – and not in a cultural uprooting'' (\citealt{PerazzoEtAl2012},1, transl.: İD).

Undoubtedly, these arguments are based on a desire to contribute to the well-being of the pupils involved. There is nothing wrong with making room for languages that play almost no role in the monolingual (German-speaking) education system, to address and develop languages that are important to pupils, which they speak in such private contexts as in the family, and to enable education in these languages. However, if such efforts are not reflected carefully, they can have a number of unintended consequences, in particular if the symbolic meaning of language is made central to an identity argument in an overly prescriptive way. This has taken place, and continues to take place, in a very similar way in the discussion of cultural identity, and it has been criticized on a number of occasions (\citealt{Tajmel2013,DoellEtAl2015,Altmayer2023}). Some of the possible effects of a rigid definition of identity can be summarized by referring to Paul Mecheril's discussion of the category of ``cultural identity'' \citep{Mecheril2003}:

When complex social processes of human coexistence are reduced to the category of culture, a ``culturalization'' takes place. People are then no longer perceived in their distinctiveness, but instead, their presumed cultural affiliation is used as a general explanatory pattern. The same applies when the complex process of developing a self-understanding is reduced to a language of origin. In the approach of Niku Dorostkar, this reduction resembles ``culturalization'', and he calles it ``lingualization'' \citep[66]{Dorostkar2014}. Referring to national and supranational EU language discourses, Dorostkar understands lingualization as a reduction of social issues to language (ibid.). One example of lingualization is the language policy, apparent in almost all EU member states, of reducing the complex process of migrants’ integration to language (acquisition). The exclusive and problem-oriented reference to people with a so-called migration background creates the impression that only this group ’suffers’ from ’identity problems’. Conversely, this means that people who are not multilingual and who speak the majority language are at peace with themselves. Furthermore, it is assumed that speaking a language considered to be the language of origin must make the speaker feel comfortable. This overlooks that in particular the multilingual children and adolescents also have their second language and other languages at their disposal, and that identity-relevant confrontations and learning processes take place in these languages, too.

These reifying and one-dimensional notions of identity are gradually being replaced in the academic field by subjectivizational and similar perspectives, largely based on the work of Michel \citet{Foucault1982} and Judith \citet{Butler1990}.\footnote{Both authors have published numerous works on the subject of subjectivation. A very good overview in English is available in \citealt{Wiede2020}.}  In these perspectives, identity is not viewed essentialistically as a relatively closed given, but as an ever dynamically evolving complex system of self-reference. According to these approaches, there is no core identity that is pure and intact and that needs to be protected: it is only through addressings and – concomitant – ascriptions that the reference system identity emerges, and references are always modified and contain contradictions.\footnote{There is a difference between the positions of Foucault and Butler in that Butler grants the subject greater freedom of decision and action than Foucault (\citealt{Foucault1982,Butler1990,Wiede2020}).} It is the discourses that provide the knowledge about roles and affiliations that are attributed to subjects and to which they are more or less at mercy. In his essay on ``Ideology and Ideological State Apparatuses'' \citet{Althusser1977} exemplifies a situation in which someone is called by a policeman with ``Hey, you there!'' \citep[142]{Althusser1977} and turns toward the policeman. According to Althusser, in the moment of this reaction and turning towards the policeman, the individual becomes a subject because he relates the call to himself and thus accepts it. Subjecthood and thus identity are understood in this approaches as radically decentered and always influenced by discourses and practices, through which a subject can be formed. In other words, identity is dependent on invocation and ascription (\citealt{Althusser1977}: 108f.). Stuart Hall critically considers the question of a unitary identity on this basis. He writes that, contrary to prevailing opinion, a reflexive conceptualization of identity cannot hold to a stable core of the self. He considers it impossible for this self to evolve through the twists and turns of life without change and to remain the same through the entire lifetime. He also considers the idea that this self can guarantee cultural unity and clearly emphasizes external differences to be an illusion, as is the idea of having a unified history of one's own self. He considers such a story to be an unrealistic narrative that only takes place because there is a need for it (\citealt{Hall2004}: 170f.). In his opinion, which is also shared by other theorists of cultural studies (for an overview s. \citealt{Marchart2018}, esp. Chap. 1: 17f, and Chap. 5: 169f), identities are constructed, interwoven, and associated with contradictory discourses, practices, and positions, and they are constantly changing and transforming (\citealt{Spies2009} w.p.).

\largerpage[-1]
Let's return to our starting point, language. When language is used in its symbolic function as a marker of origin and identity, the crucial point is the question how origin is conceptualized in this process. References loaded with constructions of origin are brought to subjects, affect them, are carried forward in different ways, and are incorporated into discourses that provide figures of argumentation with which subjects are established. Foucault points out that it is a form of power that turns individuals into subjects \citep[275]{Foucault2005}. However, power here is not thought of as something ``bad'' per se, but as a productive force. Self-conceptions depend on the attributions which are directed at the subjects from positions of higher power in society. The emphasis is on an investigation of the qualities of attributions, as a result of which self-understandings such as ``Spaniard'', ``child'', ``beautiful girl'' etc. are acquired. Subjectivizational approaches do not distinguish between social and personal identity because they question the construct of an identity as such. They understand identity as an ongoing fluid process in which the subject forms itself in the confrontation with a social environment and at the same time shapes this environment, so that personal and social categories cannot be separated. Accordingly, the subject develops in line with the acceptance or rejection of attributions in social space. 

In pedagogical contexts, it is therefore important to take a closer look at the attributions that children and adolescents encounter. In this respect, the work by Vesna \citet{Bjegač2020} is a valuable contribution to the discussion: She analyses pedagogical discourses in the academic field of German as a second language (GSL, see also Chp. 6) on the connection between language and education and relates this to interviews with young people with a so-called migration background. She shows how these adolescents are negatively influenced by negative representations in the academic field. In particular, young people develop negative self-conceptions with regard to the connection between language and origin. It becomes clear that there is a great responsibility of the academic field as well as related pedagogical and educational discourses, as their discourse can have an inferiorizing-subjectivizing effect. This problem is evident, for example, in a study by Heike \citet{Niedrig2015}, where students from African countries report that their French or English is seen as less valuable in the Hamburg schools they attend than the French or English of students who are from Europe, for instance from France. We see here a strong connection between (the attributed) country or continent of origin and language when it comes to recognition, and we understand that recognition differs when countries of origin are judged differently (\citealt{Niedrig2015}, 7). \citegen{Rühlmann2023} dissertation, following studies in the area of racio-linguistics, shows that these hierarchies are related to postcolonial value hierarchies. Thus, the extent to which language competence is recognized also has to do with racial categories.


\largerpage[-1]
Against this background, we can now formulate more precisely the question on the role of language for identity development in migration society contexts: What addresses to multilinguals become visible in connection with language in social discourses and what attributions do these addresses contain? Or, to put it more general: To what extent do ``identity ascriptions'' contain powerful culturalizing or lingualizing connotations through which children, young people, and also adults with a so-called migration background are discursively turned into inferior subjects and are thus disadvantaged? These questions will be discussed in the following, using the example of a European Commission (EC) document that carries a discourse-forming function by mandate.

\section{Language and identity by order of the European Commission}

At the EU level, multilingualism is a programme as well as a requirement. At the supranational level, the idea that multilingualism is of fundamental importance prevails because of the multinationality and multilingualism of the EU. This fundamental attitude has an impact on the view of migration-related multilingualism. In 2008, a newly formed ``Group of Intellectuals for Intercultural Dialogue'' addressed the issue of multilingualism. In the following, I present and analyze a document produced in this context. I carry out this analysis in the frame of the subjectivation approaches outlined above, according to which identity does not emerge on its own, but develops in the course of dealing with attributions, especially when these come from authorities higher in power. The analysis concentrates on the identity-related ideas that the document sends out as a political statement.

\section{Proposals of the council of intellectuals}

At the request of the then President of the European Commission, José Manuel Durão Barroso, and the Commissioner for Multilingualism, Leonard Orban, a group of discourse-shaping actors was set up in 2008.\footnote{The group was led by the author Amin Maalouf and included the following members: Jutta Limbach, president of the Goethe-Institut; Sandra Pralong, communications expert; Simonetta Agnello Hornby, author; David Green, president of EUNIC (European Network of National Cultural Institutes), former director of the British Council; Eduardo Lourenço, philosopher; Jaques de Decker, author, permanent secretary of the Académie Royale de Langue et de Littèrature françaises de Belgique; Jan Sokol, philosopher, former Minister of Education of the Czech Republic; Jan Christoph Grøndahl, author; and Tahar Ben Jelloun, author (Europäische Kommission 2008: 2).} The group was given the task of advising the EU Commission ``on the contribution of multilingualism to intercultural dialogue and mutual understanding among citizens in the European Union''\footnote{ The report was analysed in the German version; all quotes have been translated by the author (İ.D.) into English.}  (European Commission 2008: 2). In 2008, the Council of Intellectuals published a report entitled ``A worthwhile challenge – how multilingualism can contribute to the consolidation of Europe''. As a basis for my analysis, I briefly review the basic statements of this report, especially those that are relevant to our present discussion.


Must we try to define a ‘European identity’? Can this overcome all our differences?'' (ibid.: 2) These questions, classified as ``enormously sensitive'' (ibid.: 2), are then related to multilingualism: ``Our brief was to think about multilingualism and how multilingualism might impact on European integration and intercultural dialogue. Therefore, we decided to leave aside our preconceived opinions – both optimistic and pessimistic – to start from a completely neutral observation: For any society, linguistic, cultural, ethnic or religious diversity brings at the same time advantages and disadvantages, it is a source of richness but also of tensions'' (ibid.: 2). The authors see the solution in taking linguistic diversity into account in a specific way: ``Respecting our linguistic diversity does not only mean taking into account a cultural reality based on history. Even if the majority of European nations were founded on the basis of their identity{}-forming languages, the European Union can only be based on its linguistic diversity. [...{]} We even consider it capable of standing as a model for the whole of humanity of an identity based on diversity" (ibid.: 5). With regard to immigrants, it is stated that a deep knowledge of the national language and the culture they carry is an indispensable condition for integration into the host society, in order to participate in its economic, social, intellectual, artistic and political life. Furthermore, the report underlines that European countries must understand the importance of maintaining the language of origin for any immigrant or person with a migrant background. This is justified by an argumentation that, in turn, ties the loss of identity, first, to the loss of linguistic competence in the so{}-called ‘migration language’, and, second, pathologizes it: ``A young person who loses the language of his or her ancestors also loses the ability to communicate with his or her own parents in an unclouded manner, which is a factor of social destabilization that can, in turn, lead to violence. The exaggerated affirmation of identity often springs from a sense of guilt toward one{}'s culture of origin, a guilt that is sometimes expressed in an overemphasis on religious components. In other words, an immigrant or a person with an immigrant background who is proficient in his or her mother tongue, who can pass it on to his or her children, who feels that his or her language and culture of origin are recognized in the host society, would feel less need to quench his or her thirst for identity in other ways'' (ibid.: 21).]{{The report begins by stating that linguistic diversity is a challenge for Europe. However, the authors argue that it is a worthwhile challenge (ibid. 3), which must be met in an efficient manner. The following question is then brought to the fore: ``How can we make it possible for so many different peoples to live harmoniously together? [...] Must we try to define a ‘European identity’? Can this overcome all our differences?'' (ibid.: 2) These questions, classified as ``enormously sensitive'' (ibid.: 2), are then related to multilingualism: ``Our brief was to think about multilingualism and how multilingualism might impact on European integration and intercultural dialogue. Therefore, we decided to leave aside our preconceived opinions – both optimistic and pessimistic – to start from a completely neutral observation: For any society, linguistic, cultural, ethnic or religious diversity brings at the same time advantages and disadvantages, it is a source of richness but also of tensions'' (ibid.: 2). The authors see the solution in taking linguistic diversity into account in a specific way: ``Respecting our linguistic diversity does not only mean taking into account a cultural reality based on history. Even if the majority of European nations were founded on the basis of their identity-forming languages, the European Union can only be based on its linguistic diversity. [...] We even consider it capable of standing as a model for the whole of humanity of an identity based on diversity" (ibid.: 5). With regard to immigrants, it is stated that a deep knowledge of the national language and the culture they carry is an indispensable condition for integration into the host society, in order to participate in its economic, social, intellectual, artistic and political life. Furthermore, the report underlines that European countries must understand the importance of maintaining the language of origin for any immigrant or person with a migrant background. This is justified by an argumentation that, in turn, ties the loss of identity, first, to the loss of linguistic competence in the so-called ‘migration language’, and, second, pathologizes it: ``A young person who loses the language of his or her ancestors also loses the ability to communicate with his or her own parents in an unclouded manner, which is a factor of social destabilization that can, in turn, lead to violence. The exaggerated affirmation of identity often springs from a sense of guilt toward one's culture of origin, a guilt that is sometimes expressed in an overemphasis on religious components. In other words, an immigrant or a person with an immigrant background who is proficient in his or her mother tongue, who can pass it on to his or her children, who feels that his or her language and culture of origin are recognized in the host society, would feel less need to quench his or her thirst for identity in other ways'' (ibid.: 21).}}

In the course of the text, the authors further concretize the danger of generating violent acts as a consequence of not taking into account the so-called languages of origin, and distinguish it from the category of religion: ``Allowing migrants, European as well as non-European, easy access to their language of origin and allowing them to preserve what could be called their linguistic and cultural dignity seems to us once again an effective antidote to fanaticism. Religious and linguistic affiliation are obviously among the elements that most strongly constitute identity. However, these affiliations function differently, and sometimes even compete with each other. Religious affiliation is exclusive, while linguistic affiliation is not. The decoupling of these two powerful identity-forming factors, the development of a linguistic and cultural belonging not at the expense of religion, but very much at the expense of the identity-forming function of religion, seems to us a salutary undertaking that could contribute to the reduction of tensions both in European societies and elsewhere in the world'' (ibid.: 18).

Overall, the recommendations of the Group of Intellectuals pursue a benevolent goal, namely that of securing peace, and see certain differences between peoples as a possible and historically proven obstacle, namely religious and linguistic differences, which are identified as cultural categories. In relation to language, we find the following underlying argumentation:

\begin{itemize}
\item Language is a factor that creates identity.
\item Everybody has a/one language of origin; language and origin are inseparably connected.
\item People must have opportunities to speak their first language because of its identity-forming function.
\item If young migrants cannot speak `their languages of origin', there is a danger that they will become alienated from their parents and thus become violent.
\item The identity-forming moment of religioness is a problem to be combated. The focus on religion arises from the urge for identity, which can be stopped by taking into account the `mother tongues' of `migrants'.
\end{itemize}

A subjectivizational perspectives reveals the following patterns, among other things:

\begin{itemize}
\item The Council of Intellectuals follows a ``monolingual habitus'' \citep{Gogolin1998}, according to which monolingualism represents the normality and the symbolic means of establishing the togetherness of citizens. \citet{Gogolin1998} shows that in the course of the formation of the German nation-state, German was brought to the fore as the only (legitimate) language of the state; with the help of the educational institution school, in the course of time this position of German became the normality. The Council of Intellectuals seems to follow this idea of the national language as a crucial unifying element, given that it assumes that all people possess an (one) identity language. The document hence equates people and language, as is typical for the development of a certain form of nation-state. Moreover, also in the context of migration, language is understood as a language according to the nation-state model; language contact phenomena and hybridizations that are typical for migration are not taken into account. From the point of view of language ideology theory, a similar picture emerges. The language of origin (often called mother tongue) forms a unity with the so-called fatherland. This establishes a connection that also appears in contemporary negotiations of linguistic (non)belonging: an ``ethnic ownership of language'' \citep[36]{Bonfiglio2013}, with which speakers of a certain language and origin are thought to be indissolubly linked. According to \citet{PokitschDirim2024}, this is linked to three central language-ideological aspects, which are also effective in the document under discussion: ``mother tongue as a singular language category (each person has only one mother tongue), mother tongue as a character-defining feature, and mother tongue as a collectivizing element that determines belonging'' (\citealt{PokitschDirim2024}, also \citealt{PokitschBjegač2022, Bonfiglio2010, Ahlzweig1994}, in detail \citealt{Pokitsch2022}: 50ff.).
\item People regarded as migrants are fixed to their origin and thus positioned as not belonging and having to be integrated. Language is ascribed a central significance in this context. This is hence a pattern of a lingualization \citep{Dorostkar2014}, since no social categories other than religion are taken into account and questions of positive coexistence in migration society are strongly reduced to language. Interestingly, it is not the national language of the country of immigration that is placed in the foreground, but the language of origin. This denies so-called migrants the option to adopt the official language of the country in which they live as \textit{their} language as well. However, in particular for second-generation young people, whom the report focuses on, the official language of their parents’ country of immigration plays an important role as the language of socialization and relationships.
\item There is a perspective on migration understood as a problem which is supposed to be managed with the help of considering migration languages. It is assumed that young people might commit violence should they not be able to speak their first languages. This implies that parents are unable to speak the second language, and that good contact between parents and children can only be established through the first language. The complex parent-child relationship is reduced to the use of the language of origin, which vastly exaggerates the meaning of the first language and ignores the importance of different languages and hybridizations for intra-family communication, and the point that the language of the country is often one of the family languages and frequently one of the first languages for young people. Phenomena of violence are explained with recourse to language, pointing to a clear lingualization of social phenomena that does not take into account central influential factors on the relevant life issues. Taken together, this reifies language and ties a positive identity to a specific language.
\item The category religion is introduced into the discussion in a very problem-oriented way. It is distinguished from language, which is attributed a positive quality in that its identity-creating effect is defined as non-exclusive, in contrast to religion. It remains unclear on what basis the authors argue here. It can only be assumed that they refer to some observed conflicts that are justified by differences between religions. At the same time, they seem to take a naïve view of dealing with diversity through the difference feature of language. Violent processes, such as the long-standing ban on the use of Kurdish in Turkey and Syria and their traumatizing consequences for people who speak these languages, are made invisible. This draws an idealized and, in reality, untenable image of language as a unifying element whose identification possibilities are supposed to serve as a means of conflict reduction. Accordingly, people who do not speak their so-called mother tongue (anymore) are regarded as a potential danger.
\item This does not take into account that in European migration societies, people of different origins live together peacefully for the most part–despite little or no consideration of their so-called languages of origin by state institutions. The care of migration languages is left to the speakers themselves, with a threatening prognosis of what would happen if this is neglected. In the document of the Council of Intellectuals, educational institutions are not assigned any responsibility for the care of multilingualism, which is especially astonishing given that important state functionaries participate in the council.
\end{itemize}
\section{Conclusions} 

In this paper, the recommendations of the Council of Intellectuals have been used as an example to show which lines of argumentation can be found in the current legal and legitimate discourse. I have argued that the connections shown in the EU document are established at discourse-relevant positions between \textit{language}, \textit{origin} and \textit{identity}, and show patterns of culturalization and lingualism. As analyzed above, in pursuing the goal of securing peace, the Council of Intellectuals of the European Commission construct migrants as a threat to peace, thus taking a problem-oriented and even criminalizing view. Language is then not seen as an instrument of communication but as a solution to this threat, reflecting a pattern of lingualizing that reduces (imagined) social phenomena to linguistic ones. The council fixes people to their origin, not allowing for multiple affiliations (\citealt{Mecheril2001}: 44f) and establishes a causal link between language and origin, which cultivates an ethnicizing and nationalizing approach. 

This kind of approach is problematic at different levels. It does not consider the so-called identity of multilingual persons; it puts pressure on speakers to have a positive relation to their language of origin, and it neglects the fact that for many people other languages of the environment which they predominantly stay also in are identity-forming. Furthermore, languages are not thought of independently of countries of origin, which creates a nationalizing and paradoxical argumentation: on the one hand, migrants and their descendants are supposed to be integrated into European countries, but on the other hand, this integration is impossible because their languages of origin, which are seen as central, are symbols of other national contexts that make it virtually impossible to belong to European countries.

Many young people grow up in migration contexts in the midst of multilingualism and languages become symbolic means of creating hierarchized affiliations – perhaps even due to the naturalistic and origin-focused understandings of language and the creation of ethnicizing and nationalizing references. \citet{Pokitsch2022} shows in an analysis of group discussions that young people growing up in Austria associate belonging to Austria with a naturalizing image of ``ownership'' \citep[36]{Bonfiglio2013} of language and classify themselves or others as belonging to Austria or not on the basis of the languages they consider theirs or not \citep{Pokitsch2022}. Pokitsch’s work clearly shows that speaking a migration language does not automatically lead to a sense of belonging to a group with a migrant background, but quite to the contrary, that foreign-ethnicizations and hierarchizing distinctions emerge due to the prescription of identity via these languages. For this reason, it seems important to critically rethink arguments in favor of language maintenance: the identity argument may promote divisive and nationalizing/ethnicizing rather than inclusive ideas.

I maintain that children and young people should be given space to speak and develop the languages that are important to them. However, we need to take into account the danger that language may become a separating, hierarchizing symbolic feature of difference. It seems important to me that speaking a language is not one-sidedly equated with certain attitudes, attachments, countries of origin, interests, affinities, etc.. Rather, processes of subjectivation are highly complex and dependent on a wide variety of influences, including language. The mastery or non-mastery of a language alone is not what determines social ties, family relationships, and even less the affinity of persons to practices of violence. In order to prevent problems of potential violence, we need to take a broad (social-)pedagogical approach. Different languages are probably associated with different spaces of experience, and these may also be national. However, it would be advisable to critically reflect this relationship in order not to let languages become features of difference.

Against this background, we should also reconsider arguments for multilingualism in the academic literature. Subjectivities emerge from the interplay and interlocking of languages with social conditions, and it is perhaps possible to place such languages in the foreground as a symbolic element that creates identity. However, it would then also be necessary to consider how the constructions of difference discussed in this paper should or can be dealt with.

\sloppy\printbibliography[heading=subbibliography,notkeyword=this]
\end{document}
