\documentclass[output=paper,colorlinks,citecolor=brown]{langscibook}
\ChapterDOI{10.5281/zenodo.17132449}
\author{Jana Gamper\orcid{}\affiliation{University of Giessen} and
        Christoph Schroeder\orcid{}\affiliation{University of Potsdam} and Julia Schlauch \affiliation{University of Giessen} and
        Dorotheé Steinbock\orcid{}\affiliation{University of Potsdam}}
        
\title{Constructing a clientele in need: The field of German as a Second Language}
\abstract{When studying multilingual children and adolescents, it is crucial to ask about the age of onset and/or length of exposure to the languages in question before classifying them as L2 learners or not. In our semi-systematic review of 138 papers from the field of research on German as a Second Language (GSL), we ask to what extent this distinction is drawn in the field. GSL is concerned with at least two fundamentally different groups. One group is multilingual children and adolescents who acquire German as one of their first languages or as their early L2. The other group is the first generation of newly immigrated children, who clearly acquire German as their second language. We reviewed a data sample of n=138 papers that were coded according to explicit and implicit features related to GSL speakers. With the help of relative frequencies and association analyses we show that GSL speakers are mainly conceptualized as ``bi- or multilingual'' and speakers with ``languages other than German'' in their repertoire with ``insufficient German competence'' experiencing ``educational disadvantages'' in schools. Common psycholinguistic L2-criteria such as Age of Onset (AoO) occur surprisingly rarely in our data. These findings suggest a deficit-oriented conceptualization of GSL speakers and reinforce tendencies of Othering. In light of these findings, we argue for a narrowing of the concept of GSL and a better disclosure of L2-learner-specific metadata.}


%move the following commands to the "local..." files of the master project when integrating this chapter


\IfFileExists{../localcommands.tex}{
   \addbibresource{../localbibliography.bib}
   % add all extra packages you need to load to this file

\usepackage{tabularx,multicol}
\usepackage{url}
\urlstyle{same}

\usepackage{listings}
\lstset{basicstyle=\ttfamily,tabsize=2,breaklines=true}

\usepackage{langsci-basic}
\usepackage{langsci-optional}
\usepackage{langsci-lgr}
\usepackage{langsci-osl}
% \usepackage{./langsci/styles/langsci-lgr}
% \usepackage{./langsci/styles/langsci-osl}
% \usepackage{langsci-gb4e}

\usepackage{tikz}
\usetikzlibrary{patterns,calc}
\pgfdeclarepatternformonly{south east lines}{\pgfqpoint{-0pt}{-0pt}}{\pgfqpoint{3pt}{3pt}}{\pgfqpoint{3pt}{3pt}}{
    \pgfsetlinewidth{0.6pt}
    \pgfpathmoveto{\pgfqpoint{0pt}{3pt}}
    \pgfpathlineto{\pgfqpoint{3pt}{0pt}}
    \pgfpathmoveto{\pgfqpoint{.2pt}{-.2pt}}
    \pgfpathlineto{\pgfqpoint{-.2pt}{.2pt}}
    \pgfpathmoveto{\pgfqpoint{3.2pt}{2.8pt}}
    \pgfpathlineto{\pgfqpoint{2.8pt}{3.2pt}}
    \pgfusepath{stroke}}
    
\usepackage{stmaryrd}
\usepackage{wasysym}
\usepackage{multirow}
\usepackage{caption}
\usepackage{subcaption}
\usepackage{mathrsfs}
\usepackage{qtree}

\usepackage{linguex}


   %pminos do not split footnotes
% \interfootnotelinepenalty=10000 %Footnote in Laporte chapters has to be split SN


%\DeclareIndexNameFormat{default}{%
%\nameparts{#1}%
%\usebibmacro{index:name}%
%{\index[names]}%
%{\namepartfamily}%
%{\namepartgiveni}%
% {}% L1
% {}% L2
%{\namepartprefix}% generates spurious space L3
%{\namepartsuffix}% generates spurious space L4
%}

%  {\DeclareIndexNameFormat{default}{%
%     \usebibmacro{index:name}{\index[names]}{#1}{#3}{#5}{#7}}}

%\DeclareIndexNameFormat{default}{%
%  \usebibmacro{index:name}{\sindex[nom]}{#1}{#3}{#5}{#7}}

%\DeclareIndexNameFormat{default}{%
%  \usebibmacro{index:name}{\sindex[person]}{#1}{#3}{#5}{#7}}
%\DeclareIndexNameFormat{default}{%
%\nameparts{#1} \usebibmacro{index:name}{\sindex[person]]}{\namepartfamily}{‌​\namepartgiven}{\nam‌​epartprefix}{\namepa‌​rtsuffix}}

%\newcommand{\smiley}{:)}

%\renewbibmacro*{index:name}[5]{%
%\usebibmacro{index:entry}{#1}%
%{\iffieldundef{usera}{}{\thefield{usera}\actualoperator}\mkbibindexname{#2}{#3}{#4}{#5}}}

% \newcommand{\noop}[1]{}

%remove for final
%\overfullrule=1mm

\newcommand{\tobi}[2]}}
\renewcommand{\S}[1]{\tobi{#1}{\textsc{*}}}

% this volume references
% puts: [this volume]
% already defined: \citetv
%\newcommand{\citepv}[1]{(\citeauthor{#1} \citeyear*{#1} [this volume])}
\newcommand{\citealtv}[1]{\citeauthor{#1} \citeyear*{#1} [this volume]}

%parentheses around example number
\newcommand{\pref}[1]{(\ref{#1})}

% in-text examples

\newcommand{\lnex}[1]{\textit{#1}} %target lang word
\newcommand{\lnlit}[1]{(lit.: `#1')} %literal reading
\newcommand{\lnlat}[1]{(#1)} % latinization
\newcommand{\lntrans}[1]{`#1'} %translation
\newcommand{\lnexl}[2]%
{\lnex{#1}{} \lnlat{#2}} % ex with latinization
\newcommand{\lnexlat}[3]{\lnex{#1}{} \lnlat{#2}{} \lntrans{#3}} % ex with latinization and tranl.

%ch01
\newcommand{\co}[1]{\mbox{\textbf{#1}}}

%ch09

\newcommand{\cyrbulg}[1]{\begin{otherlanguage*}{bulgarian}#1\end{otherlanguage*}}


%ch10
\newcommand{\nlp}{{\small NLP}}
\newcommand{\mwe}{{\small MWE}}
\newcommand{\rae}{{\small RAE}}
\newcommand{\lvc}{{\small LVC}}
\newcommand{\pos}{{\small P}o{\small S}}
%\newcommand{\todo}[1]{ \textcolor{red}{#1} }

%\renewcommand{\labelenumi}{\theenumi}
%\ainamefmt{{vv}{ll}{, ff}{, jj}} % fullname

\newcommand{\biberror}[1]{{\color{red}#1}}

\newcommand{\osenovaitem}{--~}
   %% hyphenation points for line breaks
%% Normally, automatic hyphenation in LaTeX is very good
%% If a word is mis-hyphenated, add it to this file
%%
%% add information to TeX file before \begin{document} with:
%% %% hyphenation points for line breaks
%% Normally, automatic hyphenation in LaTeX is very good
%% If a word is mis-hyphenated, add it to this file
%%
%% add information to TeX file before \begin{document} with:
%% %% hyphenation points for line breaks
%% Normally, automatic hyphenation in LaTeX is very good
%% If a word is mis-hyphenated, add it to this file
%%
%% add information to TeX file before \begin{document} with:
%% \include{localhyphenation}
\hyphenation{
    Beck-man
    Ngu-yen
    back-chan-nel
    back-chan-nels
    mo-not-o-nous
    ste-reo-typ-i-cal
}

\hyphenation{
    Beck-man
    Ngu-yen
    back-chan-nel
    back-chan-nels
    mo-not-o-nous
    ste-reo-typ-i-cal
}

\hyphenation{
    Beck-man
    Ngu-yen
    back-chan-nel
    back-chan-nels
    mo-not-o-nous
    ste-reo-typ-i-cal
}

   \boolfalse{bookcompile}
   \togglepaper[23]%%chapternumber
}{}

\begin{document}
\AffiliationsWithIndexing{}
\maketitle

\section{Introduction}
\label{sec:introduction}

We ask here to what extent ``Othering'' applies to studies in the field of German as a Second Language (\textit{Deutsch als Zweitsprache}, henceforth GSL), with a focus on studies related to child and adolescent learners. We combine this with a second question, asking, how academic studies in the academic field of GSL conceptualize their ``clientele'' of learners. We conduct our research in the form of a systematic literature review. Our research questions originate in the observation that GSL is a rather vague concept, tending to take a deficient perspective on its clientele. This is related, in our view, to an unclear differentiation between the psycholinguistic notion of GSL in the frame of second language (henceforth L2) acquisition on the one hand, which has or should have (in our view) certain specific and empirically based implications. On the other hand, the academic field of GSL has a particularly applied tradition in Germany that differs from (established) second language acquisition (henceforth SLA) research in particular ways.

Our chapter is structured as follows: In section \ref{sec:theoretical}, we substantiate our research question. In section \ref{sec:review}, we develop and conduct our review. Section \ref{sec:discussion} discusses our findings. 

\section{Theoretical considerations}
\label{sec:theoretical}

To substantiate our research question, we first briefly introduce our understanding of multilingualism and second language acquisition (\ref{sec:multilingualism}), before turning to the academic field of GSL (\ref{sec:daz}) and some initial qualitative observations (\ref{sec:observations}), which lead us to point out certain dangers of ``Othering'', which may originate in the mismatch between the scope of the academic field and the psycholinguistic notion of GSL.

\subsection{Multilingualism, second language acquisition and age of onset}
\label{sec:multilingualism}

Acquiring an L2 is one way an individual can become multilingual, and in the individual-oriented use, the term ``multilingualism'' is an umbrella term for various forms of language acquisition in the course of an individual's life, as well as for linguistic practices and regulations in everyday life, in working life and institutions (cf. \citealt{Grosjean2010}).

Individuals become multilingual by acquiring more than one language. There is no doubt that the age of onset (AoO) and the length of exposure (LoE) to the language(s) are decisive for the course of the acquisition process, but also for its speed and to some extent for its success (cf. \citealt{GagarinaEtAl2021}, \citealt{GrimmCristante2022}, \citealt{Meisel2009}). Differences in AoO, then, are important indications of the (expected) qualitative course of acquisition. The acquisition of more than one language simultaneously as first languages is commonly called ``bilingual first language acquisition''. The acquisition of a second language after the age of 3 is called ``successive second language acquisition''. Speaking of German as an L2, this is where ``German as a Second Language'' begins.\footnote{This distinction is by no means uncontroversial. There are certainly approaches that place the beginning of successive second language acquisition (in contrast to bilingual first language acquisition) later, namely at the age of four or five (\citealt{Meisel2009, SchulzGrimm2019}). Note that the fact that the exact formulation of the criteria might be controversial does not diminish the importance of clear-cut criteria.}
Successive second language acquisition up to the age of 7 is summarized as ``early childhood second language acquisition''. Finally, a distinction must be made between whether the language(s) are acquired as a child or as an adult. Adult acquisition of another language is considered to start with the onset of puberty (cf. \citealt{HyltenstamAbrahamsson2003}). All in all, language development in German as L2 can be summarised well by saying that the earlier the acquisition of German begins, the more similar its course and dynamics are to monolingual first language acquisition. Work by \citet{ThomaTracy2007} and \citet{Tracy2008} makes it quite clear that children who begin acquiring German before the age of 3-4 go through the same stages of acquisition, at least in the area of syntactic development, as children who acquire German as their first language. A first qualitative break becomes visible at the age of onset into second language acquisition around age 4. In particular, the work of \citet{Haberzettl2014} and \citet{GagarinaEtAl2021} show that children who begin second language acquisition at this age acquire the language differently than children for whom the process begins earlier (see also \citealt{GrimmCristante2022}).

These age-related differentiation of learner groups are no more than approximate guidelines for the connection of onset and course of L2 acquisition, but AoO is a crucial defining factor when it comes to L2 learner identification. However, there are no reliable figures for Germany on how many of the multilingual individuals can be classified as L2 learners. For example, there are no figures available on the proportion of families who speak only German as the family language (=monolingual German language acquisition), or speak German and another language as family languages (=simultaneous bilingual acquisition). Childcare rates in kindergartens could provide a clue, although in official state reports only data on ``migration background'' is available,\footnote{``Migration background'' (Germ. \textit{Migrationshintergrund}) is a sociodemographic characteristic used in demographic statistics in Germany in order to describe the migrant community in the country. According to the Federal Statistical Office in Germany, individuals have a migration background if they themselves or at least one of their parents were born with a non-German citizenship. See for an overview and critical evaluation \citet{Will2020}.}
but not linguistic data. Still, if we take the start of kindergarten attendance as the start of the acquisition of German for children with a ``migration background'', then, the (cautious) conclusions about the age at the start of acquisition are as follows: For the vast majority of children born in Germany with a migration background, we can assume bilingual first language acquisition or early second language acquisition.\footnote{According to the figures of the Federal Statistical Office, just over 20\% of 0-3 year old children with a migration background attend kindergarten and the rate rises to over 80\% for 3-6 year olds (with certain differences between the federal states), see Statistisches Bundesamt, figures for March 2020, at \url{https://www.destatis.de/DE/Themen/Gesellschaft-Umwelt/Soziales/Kindertagesbetreuung/Tabellen/betreuungsquote-migration-unter6jahren-nach-laendern.html;jsessionid=8F81E7EAE4F5065DB0917E1313242370.live742} [accessed 2024/04/09].}

What does this say with regard to expected competencies in German? On the one hand, the idea that second language acquisition is completed when the speaker has mastered the second language at the same level as a monolingual speaker is not tenable: Bilingualism or multilingualism does not mean multiplied monolingualism, as the famous citation from \citet{Grosjean1989} goes. At the same time, the variation in language use and competence among monolingual speakers is so great that a normative assumption of monolingual language use and competence is not possible (\citealt{ShadrovaEtAl2021}). 

However, there is a growing body of research results which compare the academic language competencies of children and adolescents born in Germany with a migration background and/or with a family language other than or besides German with those of children and adolescents without a migration background or with a monolingual acquisition biography. These allow the following conclusion: The vast majority of children and adolescents born and raised in Germany with a family language other than or alongside German do not differ in their linguistic competencies from monolingual children and adolescents when it comes to academic language performance (see among others \citealt{Haberzettl2016}, \citealt{RicartBrede2020}, \citealt{Petersen2014}). If there was successive second language acquisition at all (and not bilingual first language acquisition), then this process reaches completion in primary school (\citealt{GagarinaEtAl2021}). At the same time, this does not necessarily have to mean that the linguistic competencies and practices of multilinguals are indistinguishable from the linguistic competencies or practices of monolinguals. However, potential support needs are no different from those faced by monolinguals and correlate primarily with social factors that limit access to the academic register of school (\citealt{ReissEtAl2019}, \citealt{Rauch2019}).

If we summarise the discussion above for our research questions, two points become evident: First, in the study of multilingual children and adolescents, it is essential to ask for biographical details concerning the exposure to the newly acquired language, before classifying them as L2 learners. On a minimal level, this is best done through inquiring about the age of onset of acquisition (AoO), and/or the length of exposure (LoE), possibly also the length of schooling (LoS) in the language. And second, for those we classify as ``early second language learners'', we can assume that they stop being ``second language learners'' at the latest at primary school age.


\subsection{The academic field of German as a Second Language}
\label{sec:daz}

The academic field of \textit{Deutsch als Zweitsprache} (DaZ / GSL) is relatively new. It emerged from the period of the first ``guest worker'' immigration to Germany in the 1970s. Germany experienced an economic boom in the late 1960s, and the demand for labor exceeded domestic resources. As a result, workers from Southern Europe and North Africa were recruited in large numbers. Initially, these so-called ``guest workers'' were to stay only for a few years, but it soon became clear that the need for labor continued, and companies did not want to let their trained workers leave again. Consequently, the workers were invited to settle for good, together with their families. Research projects focussing on the linguistic aspects of this development (initially carried out under the name of \textit{Deutsch als Fremdsprache (DaF)}, that is, ``German as a foreign language'', and only from the 1980s on ``DaZ'', cf. \citealt{Baur2001}) concentrated, on the one hand, on untutored second language acquisition of (mostly young) adults (\cite{AhrenholzRostRoth2021}). On the other hand, a strong applied orientation developed, which turned to the socially important question of how the school system dealt or should deal with students whose knowledge of German was initially too low to be able to follow the content of instruction (\citealt{Menk2000}). In this line of research, a focal topic of GSL research became the particular school register of academic German (\textit{Bildungssprache}, cf. \citealt{Lange2020}) that is presumed to open doors for social participation.

Almost half a century has passed since the emergence of GSL as an academic field. On the one hand, the original field of activity of GSL has changed considerably. The study of adult second language acquisition has diversified considerably because adult immigration has changed greatly in form and scope. After all, the new German Immigration Act of 2005 has introduced institutionalized language teaching in the form of integration courses (cf. \citealt{GamperEtAl2021}). At the same time, the former ``children of foreigners'' (Germ. \textit{Ausländerkinder}) are now adults ``with a migration background''; they are parents or grandparents; their children and children's children were born in Germany, go through German educational institutions and, as long as at least one parent was born abroad with a foreign citizenship, are again children or young people ``with a migration background''. Let us underline the fact that a history of family migration does not tell us anything about whether German is an L1 or L2, especially for descendants of the first generation of immigrants (see section \ref{sec:multilingualism}).

On the other hand, immigration of children and adolescents continues. Particularly with the large numbers of refugee migration in recent years, first in 2015 following the Syrian civil war, and then after the Russian attack on Ukraine in February 2022, questions about forms of schooling for newcomers with little or no knowledge of German, studies on the second language acquisition of this group at school, and on the conditions and possibilities for support have once again come to the fore (\citealt{GamperEtAl2021}). 

Thus, the academic field of GSL in Germany deals with at least two fundamentally different groups: On the one hand, some multilingual children and adolescents are born in Germany and acquire German as one of their first languages or as their early L2, and, on the other hand, the first generation of immigrants, who acquire German as their L2. It is crucially important to distinguish between these two groups. The second group needs support in learning German, but the first group does not, at least not in the same way. Support here means that newly immigrated children and adolescents need help to acquire basic knowledge of German to gain access to the school-specific registers. Secondly, lumping the two groups together is meaningless if one wants to generalize the dynamics of second language acquisition. And thirdly, not distinguishing between the groups may have the consequence that the first group slips into a ``problem group'' categorization. This is not only wrong but it also has discriminating tendencies, in other words, tendencies of ``Othering''.\footnote{We speak of ``Othering'' as the differentiation of a group or a person from another group by describing the former group or person as different and alien. Language is a possible tool of differentiation here (cf. \citealt{SzymczykEtAl2022}).}

So far, we have seen that the umbrella term GSL is ambiguous. On the one hand, it covers learners who are new immigrants and meet the criterion of having started L2 acquisition after the age of 3. However, due to the historical development of the field of GSL, it also covers learners for whom it is unclear whether German is a (second) L1 or an L2. This conceptual vagueness\footnote{See for a similar discussion in Great Britain \textcite{SzymczykEtAl2022}.}
has consequences for how L2 learners are conceptualized and what attributions are made to this (unclearly defined) group.

\subsection{Observations}
\label{sec:observations}

Given the previous observations, we have the impression that the academic field of GSL does indeed  have a tendency to develop a certain blind spot and treat bilingual young people who were born and raised in Germany as GSL learners \textit{and} in need of special language support. 
Three examples should suffice for a start, and we give them without a reference, because we do not want to put individual authors on a spot:

\begin{itemize}
\item A more didactically oriented introduction to teaching German in multilingual classrooms from 2017 equates being multilingual with acquiring German as the second language.
\item One of the most successful introductions to German as a second language (4th edition, 2020) uses ``migration background'' as the central defining criterion for L2 learners of German. Later, ``insufficient knowledge of German as a second language'' is identified as a decisive factor for the educational failure of this group, in addition to ``lower class membership''.
\item A study on the reading behavior of adolescents, published 2014 by a recognized publisher in the field, defines L2 learners of German on the basis that they have ``non-German mother tongue''. At no point does AoO or LoE play a role, neither are the linguistic competencies of the adolescents topic at any point.
\end{itemize}

Our review is intended to answer the question, whether the above examples represent a general trend in the sense that features like migration background, multilingualism and L2 learner are mixed and additionally attributed with insufficient academic performance, or whether they are exceptions. Let us make quite clear at this point that our study is not intended as just another act of Othering. The authors of this study by no means exclude themselves from this critical (self-)reflection but have in the past engaged themselves in publications which defined German as a L2 through implicit and vague categories. So when asking here, what kind of discursive construction of a clientele emerges in scholarly publications on GSL, we also ask, what kind of discursive constructions we ourselves have engaged in.

\section{Review}
\label{sec:review}
The focus of the empirical work that follows is on the question of whether our selective observations stand up to empirical verification. These observations include on the one hand the hypothesis that different target groups, namely L2 learners in the narrow sense as well as multilingual speakers in general, are mixed under the umbrella term GSL. On the other hand, the construct GSL seems to be associated with insufficient knowledge of German and the need for support. 
To test these assumptions, we conducted a semi-systematic review. Our research question was: How do academic papers on GSL published in Germany, and written in German for a German academic audience, conceptualize child and adolescent GSL learners? 

\subsection{Methods}
\label{sec:methods}

\subsubsection{Sample}
The sample of our review includes 138 papers, of which 92 are empirical studies and 46 are non-empirical (i.e. conceptual). To compile the sample, we conducted an automatic literature search in the database ``FIS Bildung'',\footnote{\url{https://www.fachportal-paedagogik.de/literatur/erweiterte_suche.html?checkFormParams=1&herkunft[]=fis} [accessed 2024/04/09].}
which includes monographs, collective work contributions and journal articles from different fields of acquisition- and learning-related research. The database includes more than 1 million subject-specific titles. 
Within this database, we searched for papers that carried the term ``German as a Second Language'' (in German) in the title in different variants (i.e., German as a Second Language, Second Language German, L2 German, German as a Foreign- and Second Language) and were published between 2011 and 2021. The first sample compiled in this way included 113 titles, which were reduced in a second step excluding works focusing on out-of-school learning, teacher education, development of teaching material, as well as on adult learners. The final sample extracted from the FIS database comprised 79 titles.
In a second step, we included a book series in the sample that has established itself in the German-speaking community as relevant to the topics of GSL, multilingualism, and L2 acquisition. The series is the result of the workshop \textit{Kinder mit Migrationshintergrund} (Children with a Migration Background), which has existed since 2009 and was renamed \textit{Workshop Deutsch als Zweitsprache, Migration und Mehrsprachigkeit} (German as a Second Language, Migration and Multilingualism Workshop) around 2015. The workshop is held annually and invites researchers conducting GSL research. The series was included in the review because it reflects current research projects and discourses and can thus be classified as discourse-shaping. The review includes 14 workshop volumes (volume 1 to double volume 14/15) from the period 2009 to 2021. The volumes comprise a total of 189 contributions, of which we only included the empirical papers in the review (n = 89). Of the 89 empirical workshop papers, 30 were excluded because they did not meet the above mentioned inclusion criteria, leaving 59 titles in the workshop series subsample.  The reason for including only empirical papers from the book series is that we assume that mainly empirical studies can shed light on how GSL learners are operationalized. For the overall sample, however, this means that there is a surplus of empirical studies. 

\subsubsection{Data preparation}
The sample was coded using a double-coding approach (cf. \citealt{O’ConnorJoffe2020}).
The codes were first developed deductively from the psycholinguistic literature and then extended on the basis of a smaller data sample. In this way we were able to identify a total of 19 features, of which we considered three to be explicit and 16 to be implicit or secondary. We recorded both sets of characteristics (i.e., explicit vs. implicit) once with regard to the theoretical description of GSL in the entire sample and once exclusively with regard to the operationalization of subjects in empirical studies. \tabref{tab:features} gives an overview of the features we identified within the sample.

\begin{table}[t]
\caption{Coded features for GSL}
\label{tab:features}
\begin{tabularx}{\textwidth}{lQ}
\lsptoprule
\textbf{explicit features} & \textbf{implicit features}                                               \\
\midrule
age of onset \textit{(AoO)}         & other family language / other L1                                         \\
\tablevspace
Length of Exposure \textit{(LoE)}   & additional language \textit{(i.e., other than German)}                            \\
\tablevspace
Length of Schooling \textit{(LoS)}   & multilingual                                                             \\
\tablevspace
                           & bilingual                                                                \\
\tablevspace
                           & migration background                                                     \\
\tablevspace
                           & foreign                                                                  \\
\tablevspace
                           & migration                                                                \\
\tablevspace
                           & refugee                                                                  \\
\tablevspace
                           & specific ethnical background \textit{(e.g., Turkish)}                             \\
\tablevspace
                           & German as a Foreign Language                                             \\
\tablevspace
                           & specific learning context \textit{(e.g., separate classes)}                       \\
\tablevspace
                           & insufficient competence in German                                        \\
\tablevspace
                           & educational disadvantage \textit{(e.g., poor performance in comparative studies)}                     \\
\tablevspace
                           & born in Germany                                                          \\
\tablevspace
                           & other                                                                    \\
\tablevspace
                           & unspecific                                                               \\
\lspbottomrule
\end{tabularx}
\end{table}

Under ``explicit'', we have grouped features which unambiguously clarify that German is an L2 in the psycholinguistic sense. The AoO, LoE and LoS features can be used to measure the onset and age of acquisition.\footnote{We included Length of Schooling (LoS) under explicit features since we assume that reporting the length of schooling allows us to infer the length of exposure and also the age of onset.}
Under ``implicit'' we have grouped characteristrics that may be a consequence of (e.g., bilingualism or multilingualism, other or additional language spoken at home)\footnote{``Other language'' refers to the assumption that the speakers speak a different language and therefore do not speak German at home or as L1.
``Additional language'', on the other hand, means that German is spoken in addition to another family language.}
or a reason for L2 acquisition (e.g., migration, refuge status), but these are not sufficient criteria to determine whether German is indeed an L2 for the learners described in the papers. Among the implicit features, moreover, there are also numerous ones that have no relation to GSL in the narrow sense, most notably the features ``insufficient competence'' and ``educational disadvantage''. Such characteristics may apply to all learners in the school, regardless of their individual language biographies. The list also includes features that suggest an Othering of GSL speakers (e.g., specific ethnic background or the feature ``foreign''). The term ``unspecific'' was used when GSL learners were mentioned but not described in detail. ``Other'' includes characteristics that are too rare to record separately.

\subsubsection{Coding procedure and interrater reliability}
As mentioned, the data sample was coded by two raters. First, a subsample was coded using the inductively generated lists of features. The coding of the subsample showed that the deductive criteria were not sufficient to capture all definitional features. Based on the first coding step, the feature list was therefore expanded to include additional features derived from the first coding step. The two coders then coded the entire sample with the complete list created in this way.
We distinguished between non-empirical papers on the one hand and empirical papers on the other. Going from this general split we then distinguished between passages and sections within the papers that contain general descriptions and sections that contain an operationalization of participants in a narrow empirical sense. The features listed in \tabref{tab:features} were therefore coded in a twofold manner: Passages and sections containing general descriptions of GSL speakers were assigned the prefix ``th'' for each feature. This applies to all non-empirical papers and to the theoretical sections (and, e.g., discussion of results) in empirical papers. Passages and sections containing information on the operationalization of participants were assigned the prefix ``op'' for each features. This applies only to empirical papers.
To calculate the interrater reliability, a sample of 27 titles (corresponding to 20 per cent of the total sample) was randomly selected. The coding of the data was conducted by Rater 1 and Rater 2, with each rater responsible for half of the data. The calculation of the interrater reliability was achieved by coding the selected titles by the respective other rater. The reliability of the two coding results was determined by calculating Cohen's Kappa \citep{PlonskyDerrick2016}. The calculation resulted in a value of $\kappa = 0.56$, which corresponds to a moderate and thus satisfactory strength of agreement.

\subsection{Results}
\label{sec:results}
In general, we almost always found more than one feature used to describe GSL speakers.  We first present the 15 most frequent features that appear in the entire data set and then differentiate the proportion of identified features according to empirical and non-empirical studies. This should provide a first impression of the characteristics with which GSL can be associated in our sample. In the second step, we use an association analysis to show which features in our dataset occur most frequently together and thus form semantic networks. Through these two analytical steps we want to approach the question of which concepts are associated with the term GSL.

\subsubsection{General overview} 
\label{sec:overview}
The first step is a descriptive presentation of the relative frequencies. It has been calculated how often each feature occurs in the data sample (n=138). At this point, we do not differentiate between empirical and non-empirical papers. The ``th''-features in \figref{fig:relfrequ_all} thus include the general descriptions of GSL speakers both in non-empirical papers and the theoretical sections of empirical papers. A distinction of ``th''-features between empirical and non-empirical  follows in \figref{fig:newrelfreqnonemp} and \figref{fig:relfreqemp}. The ``op''-features in \figref{fig:relfrequ_all}, however, refer only to empirical papers.

\begin{figure}
    \includegraphics[height=.4\textheight]{figures/newrelfreqall.pdf}
    \caption{Relative frequencies of 15 most frequently mentioned features in all papers (n=138)}
    \label{fig:relfrequ_all}
\end{figure}

As \figref{fig:relfrequ_all} shows, the by far most frequently occurring feature (among the 15 most frequent features), both for general descriptions of speakers (``th''-feature) and for operationalization of participants (``op''-feature) is ``other language'' (\textasciitilde 45\% and \textasciitilde 44\% of the sampled papers). Since (real) L2 learners should always be considered multilingual, this finding is not surprising at first. Those who learn German as an L2 automatically have a language other than German in their repertoire. What is striking about the figure, however, is that explicit features for identifying speakers as L2 learners (i.e., AoO, LoE) do not appear until the eighth (\textasciitilde 20\%), ninth (\textasciitilde 17\%), and eleventh (\textasciitilde 14\%) ranks, both about the general description of GSL speakers and operationalization of participants. Instead, we find features like ``bilingual'', ``multilingual'', ``insufficient competence'', ``immigration'', and ``educationally disadvantaged'' much more frequently. They mostly occur as ``th''-features, i.e. they are used for general descriptions of speakers but not for operationalization of participants. About the implicit features of ``bilingual'' and ``multilingual'', it should be noted that, although it can be assumed that all GSL speakers are bi- or multilingual, the reverse is not true for all multi- or bilingual speakers. Not all multi- or bilingual speakers acquire German as an L2, and characteristics such as ``immigration'' cannot be used to infer the onset of L2 acquisition (see section \ref{sec:theoretical}). It is equally surprising that among the 15 most frequent characteristics, the feature ``unspecific'' appears in rank 14 (11\% of the sampled papers; representing general descriptions of speakers) and 15 (10\% of the sampled papers; representing the operationalization of participants). In these cases, we do not find any specification of GSL. Furthermore, we find several implicit features that are more or less associated with GSL: One is the learning context, which includes, for example, features such as unsupervised acquisition or a distinction from German as a Foreign Language. Another is the feature ``refuge status'' in rank 13 (\textasciitilde 11\%), which shows that GSL is used relatively often for those migrants for whom refuge status was identified as the reason for migration. Furthermore, it is surprising that the characteristics ``insufficient competence'' and ``educationally disadvantaged'' occur with the observed frequency. Per se, these features are not an original characteristic of GSL (see section \ref{sec:daz} and \ref{sec:discussion}).

Taken together, Figure \ref{fig:relfrequ_all} suggests that implicit features appear to dominate both the description of GSL, and the operationalisation of participants, with explicit features appearing much less frequently. Within the implicit features, we also find those that can only be loosely associated with GSL speakers. 
In the second step, we distinguished between features in empirical and non-empirical papers. Remember that in Figure \ref{fig:relfrequ_all}, all ``th''-features represent combined findings from both empirical and non-empirical papers. Differentiating between those two types of papers allows two insights: First, we can see whether features in empirical and non-empirical papers differ from each other. Such a difference could be explained by the hypothesis that empirical studies require a clear operationalization of participants and therefore may rely less on implicit features in the general description of GSL speakers. Second, for the empirical studies, we can determine whether the features used in the general description of GSL speakers (=``th'' features) are compatible with the features used to operationalise participants (=``op''-features). We first look at the ten most frequent features in nonempirical papers (see \figref{fig:newrelfreqnonemp}) and then at the empirical ones (see \figref{fig:relfreqemp}).

\begin{figure}
    \includegraphics[height=.4\textheight]{figures/newrelfreqnonemp.pdf}
    \caption{Relative frequencies of the 10 most frequently mentioned features in nonempirical papers (n=46)}
    \label{fig:newrelfreqnonemp}
\end{figure}

As \figref{fig:newrelfreqnonemp} shows, the most frequent features used for a theoretical description of GSL speakers in non-empirical papers are ``other language'', ``bilingual'', ``multilingual'', ``insufficient competence'', and ``educationally disadvantaged''. Explicit features like AoO and LoE do not appear until ranks 6 (\textasciitilde 28\% of sampled nonempirical papers) and 10 (\textasciitilde 15\%). Between those ranks, we find further implicit features like ``immigration'', ``learning context'' (which here means a specific context like tutored or untutored L2-learning), and ``refuge status'' (meaning here that GSL learners are conceptualized as refugees). We thus see a clear dominance of features that refer to possible reasons for becoming a GSL learner or a multilingual speaker (i.e., migration) and possible linguistic outcomes of migration (i.e., multilingualism in a broad sense).

For empirical papers, we looked at the distribution of the 10 most frequent variables from three perspectives: 
1) usage for general description of GSL speakers (= ``th''-features),
2) usage of the same feature for operationalization (= ``op''-feature), and
3) usage of the same feature for both theoretical description and operationalization (= both). We choose this approach to determine whether the general description of GSL speakers is congruent with their operationalization. \figref{fig:relfreqemp} aims at making such potential congruencies (or discrepancies) visible. 

\begin{figure}
    \includegraphics[height=.4\textheight]{figures/newrelfreqemp.pdf}
    \caption{Relative frequencies of the 10 most frequently mentioned features in empirical papers (n=92)}
    \label{fig:relfreqemp}
\end{figure}

In \figref{fig:relfreqemp}, the length of the bars indicates differences in the general frequency of occurrence. Overall, we see several parallels to \figref{fig:newrelfreqnonemp}. Both empirical and non-empirical papers refer to almost the same features (in terms of general frequencies), with slight differences in terms of ranks. The only difference is that empirical studies show a noticeable amount of the feature ``unspecific'' which means that it is more or less open who or what is meant by GSL. An overarching tendency in empirical studies is that several features are frequently mentioned more in theoretical sections of the papers, but are then used less for operationalization. This is the case for ``bilingual'', ``multilingual'', ``immigration'', ``insufficient competence'', and ``educationally disadvantaged''. This points towards the assumption that there seems to be a discrepancy between the general description of GSL speakers and their operationalization. This discrepancy can be seen, for example, in the variables ``multilingual'' and ``bilingual'': Although they are frequently used for theoretical description, they are used much less frequently for operationalization. A relatively small amount of papers use those features for both theoretical description and operationalization. Operationalization is, instead, dominated by the feature ``other language''. Here we also find the highest proportion of overlap between general description and operationalization. Concerning the two explicit features AoO and LoE, we see that, at least in empirical studies, they are used more frequently for operationalization and are mentioned less frequently in the theoretical part, especially for LoE. This contradicts the overarching tendency mentioned above. Nevertheless, it must be kept in mind that explicit variables in general are used surprisingly rarely for operationalization. A particularly high discrepancy between general descriptions and operationalization can be seen in the three features ``insufficient competence'', ``immigration'', and ``educationally disadvantaged''. These are used particularly frequently for general descriptions but are then no longer used for operationalization. There is not a single study that uses the feature ``educationally disadvantaged'' for operationalization; concerning ``insufficient competence'', there are only seven. The feature ``unspecific'' is the most surprising one: Despite its comparatively low overall occurrence, we find studies that remain unspecific when it comes to operationalizing of GSL.

Overall, the data review so far suggests a strong tendency to frame GSL in terms of implicit rather than explicit features. Within these implicit features, there is a tendency toward very general characteristics of speakers and their biographies (e.g., bilingual, multilingual, immigration) that allow little inference about whether German is an L2. Likewise, characteristics such as ``insufficient competence'' and ``educationally disadvantaged'' are often found, which in themselves do not necessarily have any relation to GSL and which in turn are not or hardly ever the subject of empirical studies.
What has been shown based on the general feature frequencies can be consolidated with the help of \figref{fig:newrelfreqnonemp} and \figref{fig:relfreqemp} as follows: Implicit features not only dominate the construction of GSL, they are also found more often for general description of speakers and less often for operationalization of subjects in the context of empirical studies. This discrepancy is particularly pronounced for those features that hardly allow any conclusions to be drawn about whether German is the L2 (i.e., for the features ``insufficient competence'', ``immigration'', and ``educationally disadvantaged'').

So far, we have used relative frequencies to get an exploratory overview of the data. What the frequency calculations do not tell us, however, is how the individual features relate to each other and whether, for example, certain features occur particularly frequently with others. To identify such forms of feature co-occurrence, we performed an association analysis.

\subsubsection{Feature associations} 
\label{sec:associations}

An association analysis \citep{AgrawalEtAl1993} is a statistical method of exploratory data analysis (comparable to data mining techniques) that makes it possible to identify patterns and relationships in large data sets. It is particularly useful for finding associations between different characteristics that are not directly related. The goal of an association analysis is to uncover relationships between different variables and find possible rules or patterns. In the previous section, we have seen that the majority of papers in our data sample use implicit features to describe GSL speakers and operationalize participants. As stated several times, these implicit features are not suitable for making statements about whether German is an L2 for the speakers in question. We have also seen that the general description of potential GSL speakers are often based on features that can be classified as deficient and have no explicit reference to the psycholinguistic notion of L2 acquisition (i.e. features such as ``insufficient competence'' or ``educationally disadvantaged''). For the empirical studies, we also observed a discrepancy between general properties assigned to speakers and the operationalization of participants. We argue that the construction of a speaker group via implicit and also partly deficient features contributes to the construction of Othering. In the third and final step of our analysis, we are therefore interested in whether papers that rely on explicit and psycholinguistically motivated features (i.e., AoO, LoE, LoS) are more likely to refrain from implicit features and deficit attributions than those that do not mention these explicit features at all. To make such potential differences more visible, we divided our data sample into two new subsets: For the first subset (subset 1, n=29), we have selected only those studies that use explicit features to operationalize participants. This subset therefore contains only empirical papers. The second subset (subset 2, n=87), on the other hand, includes all papers that do not use explicit features at all -- neither in the general description of GSL speakers nor for the operationalization of participants. In this subset, we included both empirical and non-empirical papers. Here we would like to identify whether the implicit features used to describe and operationalize speakers or participants are linked to each other in a specific way. We use an association analysis to determine whether implicit (and deficient) attributions are particularly prevalent where explicit defining features are not used.\footnote{A more common data mining technique is cluster analysis. Cluster analyses, however, are based on identifying similarities between observations. In our case, a cluster analysis would have identified papers that share specific features. An association analysis, however, enables us to find co-occurrences across single data points (i.e., papers in our case). Also, cluster analyses usually require a predefinition of clusters to be identified in a data set. Association analysis is not restricted in such a way.}

\largerpage
We performed an association analysis using the Apriori algorithm from the \textit{arules} package in R \citep{HahslerEtAl2005} . Using the algorithm, we identified tendencies that describe which features tend to occur together across studies. Such tendencies of co-occurrences form rules (comparable to different weights) within the data set. Two characteristic values are usually calculated as part of the association analysis: Support and Confidence. ``Support'' is a measure of how often a specific combination of features occurs in the data. If support is high, it means that combination occurs frequently. For instance, if papers within our data sample contain a feature A and also B, then the support for the combination of A and B is high. ``Confidence'' is a measure of how likely it is for B to occur when A occurs. Together, these values help us identify patterns in the data and understand which features often occur together and how strong their relationship is. In addition to Support and Confidence, other parameters can be considered when evaluating association rules (e.g., Coverage and Lift). In our concrete example, we restricted ourselves to Support and Confidence, because these parameters have the highest informative potential.

Association analysis requires the definition of a confidence threshold. We decided on a comparatively high threshold of 0.8 which means that we get an output of those rules whose probability of occurrence is at least 80\%. By setting this, we have ensured that the rules we observe have a high probability of features co-occurring. In addition, an association analysis requires the setting of a support threshold which is a value that is set to determine which combinations of elements in a data set should be considered relevant. It defines the minimum frequency with which a combination must occur for it to be considered. In our case, we initially set a support threshold of 0.2 meaning that we are only interested in combinations of features that occur at least 20\% of the time in our data. In other words, we focus on patterns that occur fairly frequently. For subset 2, however, we found that there is only one rule that meets the Confidence and Support thresholds. To make more rules visible, we therefore lowered the Confidence threshold for subset 2 to 0.6 and Support to 0.1. The two results of subsets 1 and 2 thus differed in the frequency of the rules we observed. In consequence, the rules observed for subset 2 are weaker than those for subset 1.
Below we first see the results of the identified rules in a table format, each for subsets 1 and 2. The ``lhs'' (= left-hand side) column in the \tabref{tab:subset1} and \tabref{tab:subset2} lists the features that were identified as condition- or if-elements of a rule. If more than one feature is listed in this column, it means that these features occur together as a condition for the rule. The order of the variables in the column is not meaningful and has no significance for the interpretation of the rule. The column ``rhs'' (= right-hand side) refers to the variables that are included in the rule as inference- or then-elements. The tables thus show the frequency (= Support) and the likeliness (= Confidence) of if-then-elements (that together form rules) within the data set.  The column ``count'' states the number of papers in which the observed rule occurs.

\begin{table}
\caption{Results association analysis subset 1, n=29 papers which mention explicit features}
\label{tab:subset1}
\fittable{
\begin{tabular}{l@{}c@{}l rr r }
\lsptoprule
\textbf{condition/if-elements}&                 & \textbf{inference/then-elements}& \textbf{support} & \textbf{confidence} & \textbf{count} \\
\midrule
\{op\_LoS\}                         & \rightarrow & \{op\_LoE\}                        & 0.21& 1.00                 & 6              \\
\{th\_disadvanataged\} & \rightarrow & \{th\_biling\}                        & 0.21             & 1.00                 & 6              \\
\{th\_multiling\}                   & \rightarrow & \{op\_otherL\}                        & 0.25             & 1.00                & 7              \\
\{th\_multiling\,op\_LoE\}            & \rightarrow & \{op\_otherL\}                        & 0.21             & 1.00                & 6              \\
\{th\_biling\, op\_otherL\}            & \rightarrow & \{op\_LoE\}                        & 0.25             & 1.00                & 7              \\
\{th\_immigration\}               & \rightarrow & \{op\_LoE\}                        & 0.25             & 0.87                & 7              \\
\{th\_otherL, op\_otherL\}       & \rightarrow & \{op\_LoE\}                        & 0.25             & 0.87                & 7              \\
op\_otherL\, op\_LoE\}                      & \rightarrow & \{op\_otherL\}                        & 0.25             & 0.87                & 7              \\
\{th\_biling\, op\_LoE\}                      & \rightarrow & \{op\_otherL\}                        & 0.25             & 0.87                & 7 \\ 
\{th\_multiling\}             & \rightarrow & \{op\_LoE\}                     & 0.21             & 0.86                & 6              \\ 
\{th\_insufficientcomp\}          & \rightarrow & \{op\_LoE\}                        & 0.21             & 0.86                & 6    
    \\
\{th\_multiling\, op\_otherL\}               & \rightarrow & \{op\_LoE\}                     & 0.21             & 0.86                & 6              \\ 
\lspbottomrule
\end{tabular}
}
\end{table}

\begin{table}
\caption{Results association analysis subset 2, n=87      papers which do not mention explicit features}
\label{tab:subset2}
\fittable{
\begin{tabular}{l@{}c@{}l rr r }
\lsptoprule
\textbf{condition elements (= lhs)} &                 & \textbf{inference element (= rhs)} & \textbf{support} & \textbf{confidence} & \textbf{count} \\
\midrule
\{th\_disadvantaged\}               & \rightarrow & \{th\_biling\}                     & 0.14             & 0.81                & 13             \\
\{op\_multiling\}                   & \rightarrow & \{th\_multiling\}                  & 0.13             & 0.75                & 12             \\
\{op\_multiling\}                   & \rightarrow & \{op\_otherL\}                     & 0.13             & 0.75                & 12             \\
\{th\_multiling, op\_multiling\}    & \rightarrow & \{op\_otherL\}                     & 0.10              & 0.75                & 9              \\
\{op\_otherL, op\_multiling\}       & \rightarrow & \{op\_multiling\}                  & 0.10              & 0.75                & 9              \\
\{th\_multiling, op\_otherL\}       & \rightarrow & \{op\_multiling\}                  & 0.10              & 0.75                & 9              \\
\{th\_biling, op\_otherL\}          & \rightarrow & \{th\_otherL\}                     & 0.12             & 0.68                & 11             \\
\{th\_multiling\}                   & \rightarrow & \{th\_biling\}                     & 0.20              & 0.64                & 18             \\
\{th\_disadvantaged\}               & \rightarrow & \{th\_otherL\}                     & 0.11             & 0.62                & 10             \\
\{th\_immigration\}                 & \rightarrow & \{th\_otherL\}                     & 0.13             & 0.60                 & 12       \\
\lspbottomrule
\end{tabular}
}
\end{table}

As we can see for both \tabref{tab:subset1} and \tabref{tab:subset2}, there is a clear contrast between the Support and Confidence values. We find comparatively high Confidence values, but rather low Support values. This means that the if-then-relations we found in the data are fairly strong in the sense that the likeliness of co-occurrence is very high. At the same time, the co-occurrences do not occur very frequently. This is especially true for subset 2: since we had to lower the Confidence and Support thresholds, the if-then-relations are generally lower. This, in turn, may imply, that there is a certain arbitrariness as to which features are associated with GSL and that there is semantic vagueness and ambiguity in the construct. 

To better understand the numeric results from \tabref{tab:subset1} and \tabref{tab:subset2}, the results of the association analyses are presented in \figref{fig:associ_subset1} and \figref{fig:associ_subset2}.
\figref{fig:associ_subset1} represents the results based on \tabref{tab:subset1} and \figref{fig:associ_subset2} those based on \tabref{tab:subset2}. Both figures were drawn manually, but are based on a visualization of results using the packages \textit{arulesViz} and \textit{gggraph} in R (cf. for application of packages for association analyses \citealt{HahslerKarpienko2017}). The if-conditions are colored in white, and the then-conditions in grey circles. Arrows show the if-then relations between the features. The width of the arrows represents the support values, which are also indicated in the figure legend.

\begin{figure}
    \includegraphics[width=1.3\textwidth, angle =90]{figures/association subset 1.pdf}
    \caption{Associations of features in subset 1}
    \label{fig:associ_subset1}
\end{figure}

\begin{figure}
    \includegraphics[width=1.3\textwidth, angle =90]{figures/association subset 2.pdf}
    \caption{Associations of features in subset 2}
    \label{fig:associ_subset2}
\end{figure}

As can already be seen from \tabref{tab:subset1}, we see in \figref{fig:associ_subset1} two main nodes in those empirical papers that use explicit features to operationalize participants: ``other language'' and ``length of exposure''. It is striking that the two nodes go hand in hand with each other: ``LoE'' is an if-condition for ``other language'' and vice versa. The rather strong connection between these two features is largely plausible: If LoE is recorded, it also seems logical to include other language(s) of participants for operationalization. Also, the ``th''-features used to describe GSL speakers within the theoretical section of the papers follow a largely consistent pattern: we predominantly find ``th''-features such as ``multilingual'' and ``bilingual'' for the general description of GSL speakers. One exception is the ``th''-feature ``insufficient competence''. In addition, the isolated pair of nodes at the bottom right of the figure is also striking: completely detached from the operationalization features LoE and ``other language'', we find a connection between the ``th''-features ``bilingual'' and ``disadvantaged''. Together with the ``insufficient competence'' feature, this is an indication that even studies that make use of explicit and psycholinguistically motivated features for the operationalization of participants use implicit (and deficient-oriented) features to describe GSL speakers. However, such attributions seem to be rather rare in these papers. Where speakers are identified as GSL learners based on the feature LoE, features that are a logical consequence of L2 acquisition (i.e., bi-/multilingualism, presence of other languages in addition to L2 German) predominate. This is different in papers that do not mention the explicit features AoO and LoE at all (i.e. neither for the general description of GSL speakers nor for operationalization of participants). In such papers, we see four main nodes in \figref{fig:associ_subset2}: ``other language'' (as an ``op''-feature), ``multilingual'', ``bilingual'' and ``other language'' (all three as ``th''-features). With respect to the operationalization of participants with the help of the feature ``other language'', we find co-occurrences with ``th''-features like `multilingual’ and ``bilingual''. Although these co-occurences are in line with the papers that refer to more explicit features (mainly LoE, see \figref{fig:associ_subset1}), it remains unclear whether an operationalization of GSL speakers through the feature ``other language'' is justified. Furthermore, we see in \tabref{tab:subset2} and \figref{fig:associ_subset2} that the ``th''-feature ``educationally disadvantaged'' triggers the features ``bilingual'' as well as ``multilingual''. As already pointed out in section \ref{sec:overview}, the feature ``educationally disadvantaged'' is never used for operationalization. Whether the relations of multilingualism, GSL and educational disadvantage suggested by such regularities are empirically verified and tenable remains open.
Overall, the association analysis shows no particularly large differences between papers that use explicit features to operationalize subjects and those that do not mention the features at all. In both data subsets, we see that GSL speakers are predominantly described as bi- or multilingual and as speakers of an ``other language''. There is also a tendency in both subsets to use implicit deficit characteristics such as ``insufficient competence'' and ``educationally disadvantaged'' as part of general descriptions of speakers. Finally, the main difference between the two subsets is that those who rely on explicit features to operationalize participants can state with a high degree of certainty that their participants are real L2 learners. At the same time, this does not justify the deficient attributes that we nevertheless find in the data. For those papers that do not contain any explicit features, we only know that the participants or speakers are bi- or multilingual and speak a language other than German. Whether they are L2 learners in the narrower sense remains completely open.

\section{Discussion}
\label{sec:discussion}

Our review started with the question of how German as a Second Language is conceptualized in scholarly publications on the subject, which concentrate on child and adolescent learners. The research question is based on the observation that GSL seems to be a rather vague concept, implying a deficient perspective on its clientele.
To test our research question, we reviewed a data sample of n=138 papers that were coded according to explicit and implicit features related to GSL. In the first exploratory analysis, we determined which features are generally frequent and whether there are differences between empirical and non-empirical studies. We also assessed the relationship between general descriptions of GSL speakers and the operationalization of participants in the studies. In the second step, we used association analyses to determine whether there are regular relations between features in studies that use explicit features for operationalization (= subset 1) and those that do not mention explicit features at all (= subset 2).
 
Our results can be summarized as follows:

\begin{enumerate}
\item Implicit features occur noticeably more often than explicit ones, regardless of the type of paper (i.e., empirical vs. non-empirical). 

\item The most frequent features are linguistic features, but mainly those that do not allow to conclude whether German is an L2 (i.e., ``bilingual'', ``multilingual'' as well as ``other language''). These features shape the general description of GSL speakers to a large extent and are also frequently used for the operationalization of participant groups in empirical studies.

\item The features ``insufficient competence'' (here concerning German) and ``educationally disadvantaged'' are also frequently used. These non-linguistic parameters are mainly found in the field of theoretical descriptions of GSL but are hardly ever used for operationalization in empirical studies.

\item Both papers that use explicit features for operationalization and those that do not mention them at all base their general description of GSL speakers and the operationalization of participants on the features from point 2. Regardless of whether the papers use explicit features to operationalize participants, we find deficient attributions using features from point 3. 
\end{enumerate}

Based on these results, the initial observation that GSL seems to be a rather vague and unspecific construct can be confirmed, at least for our data sample. It is uncertain whether German can be considered an L2 for the speakers described in the non-empirical papers and the participants in empirical studies, in most cases. This is because only a small number of papers are based on comparatively strict acquisition-theoretical criteria (i.e., AoO, LoE and LoS). In the majority of the papers, GSL speakers are conceptualized as a group of speakers who are bi- and/or multilingual, speak a language other than German at home, are educationally disadvantaged, and/or have insufficient German language skills.   
The fact that explicit criteria for the definition of GSL speakers are rather rare is problematic for two reasons: First, the group's second language proficiency in German is uncertain due to the lack of explicit criteria. Second, the lack of explicit criteria allows for the creation of a construct of GSL speakers that may not exist in reality or may be highly biased. 

Our data analysis suggests an answer to our second research question: The GSL construct, as pursued in the academic field of GSL in Germany indeed has tendencies toward a deficit view of GSL speakers in particular and multilingualism in general. The fact that bi- and multilingualism, inadequate German language skills, and educational disadvantage appear to be entangled in our data in such a dominant way suggests that a discourse is being reiterated here that we originally found in non-expert educational policy debates. Large-scale studies such as PISA or educational trends for German, among others, have suggested, at least in their early results, that migration-related multilingualism is problematic when it comes to achieving adequate German competencies and high educational attainment (cf. \citealt[51ff.]{ArtelEtAl2003}). This initial interpretation has since been significantly weakened (cf. \citealt[158f.]{ReissEtAl2019}). Instead of migration-related multilingualism, social factors (independent of students' linguistic background) seem to influence the attainment of certain language competencies and degrees. GSL research should correct this form of bias with the help of accurate methodological designs, instead of using labels such as GSL or multilingualism to support proxy debates (precisely because important causal factors such as social inequality seem to be more important reasons for different school performance). There is reason to be concerned that such proxy debates stigmatize GSL learners as well as (certain) multilingual speakers. Such an interpretation of our data does not mean that there are no studies that do not address this issue - there certainly are. However, in our data set, we find a clear tendency toward methodological as well as conceptual imprecision coupled with the aforementioned correlation with more deficit education-related features. 
Another tendency that is evident in our data is the conceptualisation of GSL as linguistic Otherness. The dominance of the characteristic "other language" (than German) in our data shows that the mere existence of a language other than German in the family context leads to a classification in which the multilinguals are not competent speakers of German - and thus 'other' from the monolinguals (cf. also \citealt{SzymczykEtAl2022}). Although we did not test whether the characteristic correlates with concepts of Othering, it would be worthwhile investigating such correlations in future studies.
Our findings highlight problematic tendencies in the GSL research community of participating in parts of a deficit discourse and share responsibility for discursively treating multilingual speakers in general and GSL learners in particular as Other. This is a harsh accusation. Let us stress again that the authors of this study by no means exclude themselves from this problem. Our studies, too, often define GSL through implicit and vague categories and engage in a form of Othering.\footnote{We are aware of the fact that it is state of the art to include a monolingual control group in empirical studies. However, there are numerous problems to this, one of them being that monolinguals hardly ever constitute a homogeneous group of speakers but tend to show high degrees of heterogeneity (see also \cite{chapters/04}). It is therefore necessary to carefully evaluate whether a comparison between multilinguals and monolinguals is theoretically grounded and purposeful. In any case, a control group should be well justified and should not be an automatism, as this can contribute to conceptualizing GSL and multilingual speakers as others.}  Therefore, our review in no way intends to defame individuals or research approaches. Rather, it aims to make certain, in our view problematic, approaches in the field of GSL research visible. At the same time, we would like to contribute to renewing our research practice and internal discourses.    

In our view, such a renewal comprises two central points: There needs to be a re-evaluation of which learners or speakers we should label as GSL-learners and which not. And this must have consequences for the compilation and revelation of speaker-related metadata that allow us to precisely describe participant groups (in empirical studies) or learners in general. We elaborate on these two related points below. 

Our review has made clear that in most studies we do not know whether the speakers labelled as GSL learners in the papers are actually L2 learners or not. Explicit and narrowly defined theoretical criteria on AoO and LoE are either not reported and/or not assessed. Instead, GSL is equated with multilingualism or the presence of other languages at home. Thus, it is no longer possible to clearly delineate whether empirical study findings or theoretical presuppositions about GSL refer to L2 learning in the narrow sense at all. It seems reasonable to assume that they do not. This in turn has the consequence that multilingual speakers are automatically assumed to have German as an L2. Coupled with the co-occurrence of the label GSL and partly deficient features such as ``insufficient competencies'', this creates an overall deficient picture of multilingualism.
To enable a differentiated view, it is necessary to return to narrow definition criteria for GSL. As stated in section 2.2, according to the current state of the art, German is a second language only if its acquisition begins after the age of three. Furthermore, studies suggest that early successive L2 acquisition (starting at age 3) does not seem to result in any differences in the acquisition process compared to L1 learners. From an acquisition theory perspective, the fact that German is acquired as an L2 has a qualitative (e.g., concerning acquisition sequences) as well as quantitative (e.g., concerning aquisition speed) effect on the acquisition trajectory at the earliest from the age of four (see section \ref{sec:multilingualism}). The task of GSL research should be to reveal how L2 acquisition proceeds, which factors can influence it, and how. It is not the task of GSL research to capture the linguistic Otherness of multilingual speakers. That this nevertheless happens is, in our view, also due to a lack of clarity when learners leave the status of being learners. For example, are adolescents or adults whose acquisition of German began in childhood still GSL learners? Or are they simply multilingual (if at all)? This question leads us to a discussion on acquisition theory, which is already underway. In a nutshell, this debate can be summarized as follows: Assuming that the age of acquisition is captured, we would expect that being in the process of learning the basic structures of a second language means being in the process of L2 acquisition. The process of the acquisition of academic language however, which builds up upon that, is usually no different from what monolingual L1 speakers have to go through in school-induced language expansion (see \citealt{GamperSchroeder2016}, also \citealt{Hulstijn2019} ``basic language cognition'' vs. ``extended language cognition''). From this point of view, research in the frame of GSL would be focused predominantly on learners who are in early acquisition phases and serve to acquire basic grammatical and lexical structures. We thus argue for a narrowing of the concept of GSL only to learners who are in the early stages of L2 acquisition. GSL would then apply only to learners who are in the early stages of L2 acquisition. In our opinion, this almost always applies to those speakers who are new migrants to Germany. For children born and raised in Germany, this is usually not the case.\footnote{We are grateful to one of the reviewers, who pointed out the parallel to L1 acquisition: Clearly, one would not call a monolingual adult ``L1 learner'', even if one could argue that language learning, at least lexical learning, continues throughout the lifespan.} GSL research and the term ``GSL'' would thus be closely linked to factors in the context of new immigration.\footnote{We are aware of the fact that our proposal comes along with problems when dealing with adult L2 learners. An anonymous reviewer rightly pointed out that adult L2 acquisition takes a comparatively long time when it comes to acquiring basic lexical and grammatical features. Making the term GSL dependent on the language acquisition domain would therefore mean that these learners would have to keep the label GSL for a very long time (although they may achieve a high level of communicative competence in everyday life). One idea could be to distinguish between L2 learners on the one hand and GSL speakers on the other. L2 learners would then (regardless of the AoO) primarily mean those learners who have recently immigrated \textit{and} have to learn basic grammatical and lexical structures. GSL speakers, on the other hand, would primarily refer to those whose acquisition process has been going on for some time, but who do not yet show extensive competence within the domain of basic language cognition. In any case, the label GSL should not be used for speakers who are concerned with the development of so-called extended language cognition.}

We are aware that such narrowing can trigger debates because it questions established research traditions and research questions. However, in our view, it is necessary to significantly revise and overcome deficient concepts of GSL and multilingualism and forms of Othering - also concerning educational policy debates. Our proposal to limit GSL exclusively to newly immigrated learners of German has particular implications for empirical studies in this regard. 

A consensus is needed that studies related to GSL should only be considered GSL if they focus on L2 learners in a narrow sense. That they do so includes both the careful collection of meaningful metadata, which must include at least AoO. Other features such as LoE (or LoS) are useful, and more (such as individual educational and schooling experiences, heritage languages, writing experiences in these languages, and individual language learning situations) are desirable to get a maximally differentiated picture. Such metadata must not only be collected but reported in the studies. In international journals, it has long been a scientific standard to disclose data in the spirit of open data efforts.\footnote{See, e.g., \url{https://www.cambridge.org/core/services/authors/open-data/data-availability-statements} and \url{https://www.dfg.de/foerderung/info_wissenschaft/2022/info_wissenschaft_22_79/index.html} [both accessed 2024/04/09].} It would be desirable to take up this trend in empirical studies in the field of GSL and to transfer it to the disclosure of learner-related metadata. Thus, we argue not only for transparency in the description of GSL learners but also for a (self-)commitment to disclose learner-specific metadata. Such a disclosure would not only be in the spirit of claims for the cross-checking, generalization, and comparability of research data and study findings. Methodological transparency in the sense we call for would also contribute to distancing GSL research from the context of educationally biased and sometimes ideologically driven debates and instead link it to applied linguistics as well as empirically grounded acquisition theory research. 

\section*{Abbreviations}
\begin{tabularx}{.5\textwidth}{@{}lQ@{}}
AoO &Age of onset \\
DaF &\textit{Deutsch als Fremdsprache} \\
DaZ &\textit{Deutsch als Zweitsprache} (GSL) \\
GSL &German as a Second Language \\
\end{tabularx}
\begin{tabularx}{.45\textwidth}{@{}lQ@{}}
LoE &Length of exposure \\
LoS &Length of schooling \\
L2 & Second language \\
\\
\end{tabularx}%



\section*{Contributions}
Aylin Braunewell und Felix Luckau contributed to sampling and coding within the empirical review.

\printbibliography[heading=subbibliography,notkeyword=this]
\end{document}
