\documentclass[output=paper,colorlinks,citecolor=brown]{langscibook}
\ChapterDOI{10.5281/zenodo.15450436}
\author{Junseon Hong\orcid{https://orcid.org/0009-0004-4543-1952}\affiliation{Stanford University}}
\title{Paradigms and discourse effects of English rising declaratives}
\abstract{The aim of this paper is to make a predictable model for \isi{rising declaratives} (\isi{RDs}) based on the division of labor between semantics and pragmatics. Building on the major classification of \isi{RDs} into assertive and inquisitive \citep{jeong2018intonation}, I first explore a wide range of different functions of \isi{RDs}. Then, I show how each subtypes of \isi{RDs} are interpreted by the interaction of conventional meaning and pragmatic reasoning, based on extended Lewisian model. The key proposal lies on the contributions of rising intonation on discourse effects, both on semantic content and projection of commitments.}

%move the following commands to the "local..." files of the master project when integrating this chapter
%\usepackage{tabularx}
%\usepackage{langsci-optional}
%\usepackage{langsci-gb4e}
%\usepackage[linguistics]{forest}
%\usetikzlibrary{arrows.meta,decorations.text, tikzmark}
%\usepackage{langsci-avm}
%\usepackage{listings}
%\usepackage{tikz}
%\usepackage{multirow}
%\usepackage{xcolor}
%\usepackage{ragged2e}
%\usepackage{float}
%\lstset{% general command to set parameter(s)
 %basicstyle=\small,
 %% print whole listing small
 %stringstyle=\ttfamily}
%\bibliography{localbibliography}
%\newcommand{\orcid}[1]{}

\IfFileExists{../localcommands.tex}{
   \addbibresource{../localbibliography.bib}
   % add all extra packages you need to load to this file

\usepackage{tabularx,multicol}
\usepackage{url}
\urlstyle{same}

\usepackage{listings}
\lstset{basicstyle=\ttfamily,tabsize=2,breaklines=true}

\usepackage{langsci-basic}
\usepackage{langsci-optional}
\usepackage{langsci-lgr}
\usepackage{langsci-osl}
% \usepackage{./langsci/styles/langsci-lgr}
% \usepackage{./langsci/styles/langsci-osl}
% \usepackage{langsci-gb4e}

\usepackage{tikz}
\usetikzlibrary{patterns,calc}
\pgfdeclarepatternformonly{south east lines}{\pgfqpoint{-0pt}{-0pt}}{\pgfqpoint{3pt}{3pt}}{\pgfqpoint{3pt}{3pt}}{
    \pgfsetlinewidth{0.6pt}
    \pgfpathmoveto{\pgfqpoint{0pt}{3pt}}
    \pgfpathlineto{\pgfqpoint{3pt}{0pt}}
    \pgfpathmoveto{\pgfqpoint{.2pt}{-.2pt}}
    \pgfpathlineto{\pgfqpoint{-.2pt}{.2pt}}
    \pgfpathmoveto{\pgfqpoint{3.2pt}{2.8pt}}
    \pgfpathlineto{\pgfqpoint{2.8pt}{3.2pt}}
    \pgfusepath{stroke}}
    
\usepackage{stmaryrd}
\usepackage{wasysym}
\usepackage{multirow}
\usepackage{caption}
\usepackage{subcaption}
\usepackage{mathrsfs}
\usepackage{qtree}

\usepackage{linguex}


   %pminos do not split footnotes
% \interfootnotelinepenalty=10000 %Footnote in Laporte chapters has to be split SN


%\DeclareIndexNameFormat{default}{%
%\nameparts{#1}%
%\usebibmacro{index:name}%
%{\index[names]}%
%{\namepartfamily}%
%{\namepartgiveni}%
% {}% L1
% {}% L2
%{\namepartprefix}% generates spurious space L3
%{\namepartsuffix}% generates spurious space L4
%}

%  {\DeclareIndexNameFormat{default}{%
%     \usebibmacro{index:name}{\index[names]}{#1}{#3}{#5}{#7}}}

%\DeclareIndexNameFormat{default}{%
%  \usebibmacro{index:name}{\sindex[nom]}{#1}{#3}{#5}{#7}}

%\DeclareIndexNameFormat{default}{%
%  \usebibmacro{index:name}{\sindex[person]}{#1}{#3}{#5}{#7}}
%\DeclareIndexNameFormat{default}{%
%\nameparts{#1} \usebibmacro{index:name}{\sindex[person]]}{\namepartfamily}{‌​\namepartgiven}{\nam‌​epartprefix}{\namepa‌​rtsuffix}}

%\newcommand{\smiley}{:)}

%\renewbibmacro*{index:name}[5]{%
%\usebibmacro{index:entry}{#1}%
%{\iffieldundef{usera}{}{\thefield{usera}\actualoperator}\mkbibindexname{#2}{#3}{#4}{#5}}}

% \newcommand{\noop}[1]{}

%remove for final
%\overfullrule=1mm

\newcommand{\tobi}[2]}}
\renewcommand{\S}[1]{\tobi{#1}{\textsc{*}}}

% this volume references
% puts: [this volume]
% already defined: \citetv
%\newcommand{\citepv}[1]{(\citeauthor{#1} \citeyear*{#1} [this volume])}
\newcommand{\citealtv}[1]{\citeauthor{#1} \citeyear*{#1} [this volume]}

%parentheses around example number
\newcommand{\pref}[1]{(\ref{#1})}

% in-text examples

\newcommand{\lnex}[1]{\textit{#1}} %target lang word
\newcommand{\lnlit}[1]{(lit.: `#1')} %literal reading
\newcommand{\lnlat}[1]{(#1)} % latinization
\newcommand{\lntrans}[1]{`#1'} %translation
\newcommand{\lnexl}[2]%
{\lnex{#1}{} \lnlat{#2}} % ex with latinization
\newcommand{\lnexlat}[3]{\lnex{#1}{} \lnlat{#2}{} \lntrans{#3}} % ex with latinization and tranl.

%ch01
\newcommand{\co}[1]{\mbox{\textbf{#1}}}

%ch09

\newcommand{\cyrbulg}[1]{\begin{otherlanguage*}{bulgarian}#1\end{otherlanguage*}}


%ch10
\newcommand{\nlp}{{\small NLP}}
\newcommand{\mwe}{{\small MWE}}
\newcommand{\rae}{{\small RAE}}
\newcommand{\lvc}{{\small LVC}}
\newcommand{\pos}{{\small P}o{\small S}}
%\newcommand{\todo}[1]{ \textcolor{red}{#1} }

%\renewcommand{\labelenumi}{\theenumi}
%\ainamefmt{{vv}{ll}{, ff}{, jj}} % fullname

\newcommand{\biberror}[1]{{\color{red}#1}}

\newcommand{\osenovaitem}{--~}
   %% hyphenation points for line breaks
%% Normally, automatic hyphenation in LaTeX is very good
%% If a word is mis-hyphenated, add it to this file
%%
%% add information to TeX file before \begin{document} with:
%% %% hyphenation points for line breaks
%% Normally, automatic hyphenation in LaTeX is very good
%% If a word is mis-hyphenated, add it to this file
%%
%% add information to TeX file before \begin{document} with:
%% %% hyphenation points for line breaks
%% Normally, automatic hyphenation in LaTeX is very good
%% If a word is mis-hyphenated, add it to this file
%%
%% add information to TeX file before \begin{document} with:
%% \include{localhyphenation}
\hyphenation{
    Beck-man
    Ngu-yen
    back-chan-nel
    back-chan-nels
    mo-not-o-nous
    ste-reo-typ-i-cal
}

\hyphenation{
    Beck-man
    Ngu-yen
    back-chan-nel
    back-chan-nels
    mo-not-o-nous
    ste-reo-typ-i-cal
}

\hyphenation{
    Beck-man
    Ngu-yen
    back-chan-nel
    back-chan-nels
    mo-not-o-nous
    ste-reo-typ-i-cal
}

   \boolfalse{bookcompile}
   \togglepaper[23]%%chapternumber
}{}

\begin{document}
\maketitle

\section{Introduction}
\label{sec:intro:Hong}

In all languages, there is a significant correlation between three grammatical moods and \isi{speech acts} they issue \citep{roberts2018speech}: declaratives provide information, interrogatives request information, whereas imperatives issue a direction. Within these three moods, the former two are associated with \isi{speech acts} that are related to changing the discourse context, which are assertions and questions. That is, they are basic direct \isi{speech acts} that are used to exchange information, and each canonically aligns with a distinct syntactic form in English: assertions with falling declaratives and questions with polar interrogatives.\footnote{Note that polar interrogatives are not the only type of interrogatives. There are other major classes of interrogatives, illustrated as follows \citep{ciardelli2018inquisitive}:

\ea
    \ea Is he attractive\textsuperscript{\uparrow} or charming\textsuperscript{\downarrow}? \hfill \textsc{Alternative Interrogatives}
    \ex Is he attractive\textsuperscript{\uparrow} or charming\textsuperscript{\uparrow}? \hfill \textsc{Open Disjunctive Interrogatives}
    \ex Who is attractive? \hfill \textsc{Constituent Interrogatives}
    \z
\z
For the purposes of this paper, I restrict my discussion to polar interrogatives, which specifically ask for the truth value of the expressed proposition, and omit others from consideration.}
However, \isi{rising declaratives} (henceforth, \isi{RDs}), such as (\ref{ex:rd1:Hong}), are non-canonical structures that accompany declarative sentences with rising intonation.

\label{sec:examples:Hong}
\ea \label{ex:rd1:Hong}
 \textit{She's home?}
\z
Due to their linguistic mismatch, \isi{RDs} highlight the complex pattern of phonetic, syntactic, semantic, and pragmatic interface. Compared to canonical clause types, it is widely recognized that \isi{RDs} provide further information beyond the expressed proposition \citep[e.g.,][]{gunlogson2003true, gunlogson2008question, malamud2015three, farkas2017division, jeong2018intonation, rudin2022intonational}. Still, \isi{RDs} share similarities in their \isi{speech acts} with both of the two canonical types: falling declaratives, as in (\ref{ex:2rda:Hong}), and polar interrogatives, as in (\ref{ex:2rdb:Hong}), since they possess properties of both \citep{jeong2018intonation}.

\ea \label{ex:2rd:Hong}
    \ea \label{ex:2rda:Hong} \isi{Assertive rising declaratives}: overlap with assertions\\
    \justifying\noindent{[Context: A asks B where Sally is. B is not sure of Sally’s whereabouts:]}\\
    A: Where's Sally?\\
    B: (Um...) \textit{She's home?} / She's home. / \#Is she home?
    \ex \label{ex:2rdb:Hong} \isi{Inquisitive rising declaratives}: overlap with questions\\
    {[Context: Sally has been skipping school without any specific reason. B has just come back from work and A wants B to speak with Sally immediately about her issues.]}\\
    A: You should speak to Sally right away.\\
    B: \textit{She’s home?} / \#She's home. / Is she home?
    \z
\z
An \isi{RD} in (\ref{ex:2rda:Hong}) functions as an assertion, in which Speaker B responds to the addressee's question by providing information without a concrete certainty. In contrast, an \isi{RD} in (\ref{ex:2rdb:Hong}) functions like a question implicating that the uttered proposition is highly likely. From this contrast, \isi{RDs} can be respectively substituted with either falling declaratives or polar interrogatives. I refer to the former as \textsc{Assertive Rising Declaratives} (henceforth, \isi{ARDs}) and the latter as \textsc{Inquisitive Rising Declaratives} (henceforth, \isi{IRDs}), following \citet{jeong2018intonation}.

The central question that arises from this phenomenon is how to account for the two distinct types. To answer this question, a few previous approaches seek a unified account \citep[e.g.,][]{rudin2022intonational}, while others focus on more specific types of \isi{RDs}, acknowledging the existence of other types \citep[e.g.,][]{farkas2017division}. \citet{jeong2018intonation} is the first approach arguing for two fundamentally different types of \isi{RDs}. This paper further extends her idea. I propose a model that delineates the role of semantics and pragmatics, based on the exploration of diverse types of \isi{RDs} that have not been widely discussed. 

This paper is structured as follows: \sectref{sec:phen:Hong} takes a close look at four specific functions of \isi{RDs} that can convey either assertive or inquisitive meanings. \sectref{sec:back:Hong} introduces the adopted framework and outlines relevant approaches. \sectref{sec:contris:Hong} draws the detailed contributions of rising intonation in the interpretation process of RDs, and \sectref{sec:intris:Hong} explains how each function can be drawn from the interaction of sentence type, conventional intonation, and discourse context. Finally, \sectref{sec:conc:Hong} concludes.

\section{The pattern of rising declaratives}
\label{sec:phen:Hong}

This section provides a brief discussion of the empirical patterns of \isi{RDs}, which perform the speech act of either asserting or questioning. Let us begin with the first type, \isi{ARDs}. I propose two paradigms of \isi{ARDs} in terms of the uncertainty they implicate: epistemic and metalinguistic. 

\ea \label{ex:2ards:Hong}
    \ea \label{ex:2ardsa:Hong} Epistemic Uncertainty \isi{ARD}\\
    A: Where's Sally?\\
    B: (Um...) \textit{She's home?}\\
    \ex \label{ex:2ardsb:Hong} Metalinguistic Uncertainty \isi{ARD}\\
    A: Does he speak Chinese?\\
    B: \textit{He speaks Cantonese?}\\
    \z
\z

\noindent The distinction between the two comes from whether the speaker is conveying uncertainty about the truth condition of the proposition or other aspects. (\ref{ex:2ardsa:Hong}) indicates the speaker’s tentativeness on the truth value of the expressed proposition. In contrast, (\ref{ex:2ardsb:Hong}) conveys an uncertainty of the relevance to the context, which is not directly associated with the truth value of the proposition: the speaker is unsure whether they are giving an adequate answer to the prior question.

Modal substitution and subordination are two pieces of evidence for the two subtypes of \isi{ARDs}. First, Epistemic Uncertainty \isi{ARDs} can be freely substituted with epistemic modals, while Metalinguistic Uncertainty \isi{ARDs} cannot, as in (\ref{ex:ardmod1:Hong}).

\protectedex{
\ea \label{ex:ardmod1:Hong}
    \ea \label{ex:ardmod1a:Hong} Epistemic Uncertainty \isi{ARD}\\
    A: Where's Sally?\\
    B: (Um...) \textit{She's home?} \hfill (necessarily means `She \textbf{may} be home.')
    \ex \label{ex:ardmod1b:Hong} Metalinguistic Uncertainty \isi{ARD}\\
    A: Does he speak Chinese?\\
    B: \textit{He speaks Cantonese?}\\ \hfill (does not necessarily mean `He \textbf{may} speak Cantonese.')\\
    \z
\z
}

\noindent{}Second, parallel behaviors of Epistemic Uncertainty \isi{ARDs} and epistemic modals are also observed in terms of modal subordination \citep{roberts1987modal, roberts1989modal}. An Epistemic Uncertainty \isi{ARD} can be followed by a modal utterance, as in (\ref{ex:ardmod2a:Hong}), while a Metalinguistic Uncertainty \isi{ARD} does not exhibit modal subordination, as in (\ref{ex:ardmod2b:Hong}). 

\ea \label{ex:ardmod2:Hong}
    \ea \label{ex:ardmod2a:Hong} Epistemic Uncertainty \isi{ARD}\\
    A: Where's Sally?\\
    B: (Um...) \textit{She\textsubscript{i}'s home?} She\textsubscript{\textit{i}} \textbf{must} have come from school early.\\
    \ex \label{ex:ardmod2b:Hong} Metalinguistic Uncertainty \isi{ARD}\\
    A: Do you speak Chinese?\\
    B: \textit{He\textsubscript{i} speaks Cantonese?} \#He\textsubscript{\textit{i}} \textbf{must} be born in Hong Kong.\\
    \z
\z

I also follow \citet{jeong2018intonation,jeong2021deriving}'s approach that each type of \isi{ARDs} can bear politeness, often used as a rapport-building process \citep{podesva2011salience, levon2016gender, jeong2018intonation, jeong2021deriving}.\footnote{\isi{RDs} can be used as conversational starters, as shown in (\ref{ex:ardini:Hong}) below.
\ea  \label{ex:ardini:Hong} A: Hello, \textit{my name is David? I'll be your waiter today?}
\z
\citet{jeong2021deriving} suggests that this usage may be allowed because checking in to see if the new topic is relevant or informative appears to be more polite than not checking in. I argue that this is closely related to metalinguistic uncertainty.} Each politeness use in (\ref{ex:ardpola:Hong}) and (\ref{ex:ardpolb:Hong}) corresponds with the paradigm of \isi{ARDs} in (\ref{ex:2ardsa:Hong}) and (\ref{ex:2ardsb:Hong}), respectively.\footnote{An anonymous reviewer points out that, in French, lexical material is necessary to make dialogues like (\ref{ex:ardpola:Hong}) well-formed. While I acknowledge potential differences in the role of intonation in conveying certain meanings or intentions, I leave cross-linguistic generalization for future research.} A more detailed analysis will be provided in \sectref{sec:ard:Hong}.

\ea \label{ex:ardpol:Hong} Politeness effect invoked by \isi{ARDs}
    \ea \label{ex:ardpola:Hong} Epistemic Uncertainty Politeness\\
    A: Do you want a glass of water?\\
    B: \textit{I'll have a wine?}
    \ex \label{ex:ardpolb:Hong} Metalinguistic Uncertainty Politeness\\ 
    A: Which city is Lenny from?\\
    B: (to the supervisor) \textit{She's from Yemen?}
    \z
\z

Two paradigms are contrasted for \isi{IRDs} as well, according to the bias, as illustrated in (\ref{ex:2irds:Hong}).\footnote{The rise of \isi{IRDs} is steeper than that of \isi{ARDs}. See \citet{jeong2018intonation}'s experimental result in \sectref{sec:Jeong:Hong}.}

\ea \label{ex:2irds:Hong}
    \ea \label{ex:2irdsa:Hong} Confirmative \isi{IRD}\\
    {[Context: B is buying a ticket for a flight to Seoul at the airport.]}\\
    A: There's one flight to Seoul.\\
    B: \textit{The flight leaves at 5pm?}\\
    \ex \label{ex:2irdsb:Hong} Contradictory \isi{IRD}\\
    A: I went to the concert last night. Dave is a good singer.\\
    B: \textit{Dave is a good singer?} You must be thinking about Anna.\\
    \ex \label{ex:2irdsc:Hong} Mirative \isi{IRD}\\
    {[Context: B thought that Dave is the only child in his family.]}\\
    A: I met Anna’s brother yesterday.\\
    B: (What?) \textit{She has a brother?}\\
    \z
\z
(\ref{ex:2irdsa:Hong}) and (\ref{ex:2irdsb:Hong}) differ in terms of the speaker’s epistemic bias toward the expressed proposition. The former illustrates the speaker’s high degree of certainty on the expressed proposition. As this type of \isi{IRDs} is used to confirm the speaker’s prediction (\ref{ex:2irdsa:Hong}), they are named \textsc{Confirmative} \isi{IRDs}.\footnote{The overall contrast is first brought up in \citet{gunlogson2003true, gunlogson2008question}, though may not in the exact labels.} In contrast, the latter implicates the speaker’s disbelief in the proposition as it is to ask a question with high suspicion. As this type of \isi{IRDs} contradicts the addressee’s expressed or presupposed content (\ref{ex:2irdsb:Hong}), I call it \textsc{Contradictory} \isi{IRDs}.\footnote{There are also other clause types that indicate bias, which include negative polar questions and tag questions. While space precludes a detailed discussion, interested readers may consult \citet{romero2004negative, malamud2015three, farkas2017division, frana2019attitudes, goodhue2022isn} and references therein.} Additionally, \isi{IRDs} that convey the speaker’s surprise are classified as a subtype of Contradictory \isi{IRDs} \citep[c.f.,][]{ goodhue2021unified}. In (\ref{ex:2irdsc:Hong}), the speaker is surprised by the expressed proposition and a surprised speaker would not have prior knowledge or belief that the proposition is true, which is related to a negative bias.\footnote{An anonymous reviewer points out that Mirative \isi{IRDs} may not be necessarily related to negative bias and could instead be associated with the absence of bias. For a more discussion on the difference between negative bias and the absence of bias, see \citet{sudo2013biased}.} As they implicate mirativity, they are named \textsc{Mirative} \isi{IRDs}.

The negative bias of Contradictory \isi{IRDs} can be emphasized by an overt dissent expression such as \textit{No way} (\ref{ex:irdsnob:Hong}), while Confirmative \isi{IRDs} cannot be followed by such expressions (\ref{ex:irdsnoa:Hong}).

\ea \label{ex:irdsno:Hong}
    \ea \label{ex:irdsnoa:Hong} Confirmative \isi{IRD}\\
    {[Context: Same as (\ref{ex:2irdsa:Hong}).]}\\
    A: There's one flight to Seoul.\\
    B: \textit{The flight leaves at 10am?} \textbf{\#No way.}\\
    \ex \label{ex:irdsnob:Hong} Contradictory \isi{IRD}\\
    A: Dave is a good singer.\\
    B: \textit{Dave is a good singer?} \textbf{No way.}\\
    \z
\z

The distinctive bias of each \isi{IRD} also leads to a difference in the licensed particle responses, analogous to positive and negative polar questions \citep{roelofsen2015polarity}. Bare particle responses to positive polar questions are unambiguous while those to negative polar questions are ambiguous. The same phenomenon is observed for \isi{IRDs}.

%\protectedex{
\ea \label{ex:confird:Hong}
    {[Context: Same as (\ref{ex:2irdsa:Hong}).]}\\
    A: There's one flight to Seoul.\\
    B: \textit{The flight leaves at 10am?}\\
         \ea\label{ex:confirda:Hong} A: \{Yes / \#No\}, it does. \hfill[\textsc{agree}, +]
         \ex\label{ex:confirdb:Hong} A: \{\#Yes / No\}, it doesn’t. \hfill[\textsc{reverse}, --]
    \z
\z
%}

%\protectedex{
\ea \label{ex:contird:Hong}
    A: Dave is a good singer.\\
    B: \textit{Dave is a good singer?}\\
         \ea\label{ex:contirda:Hong} A: \{Yes / No\}, he isn't. \hfill[\textsc{agree}, --]
         \ex\label{ex:contirdb:Hong} A: \{Yes / No\}, he \textsc{is}. \hfill[\textsc{reverse}, +]
    \z
\z
%}
As in (\ref{ex:confird:Hong}) and (\ref{ex:contird:Hong}), particle responses bear two features: [+, --] and [\textsc{agree}, \textsc{reverse}] \citep{roelofsen2015polarity}. The former are absolute features whereas the latter are relative features. Absolute features are responses for being positive or negative about the truth value of the prejacent proposition, while relative features are responses for agreeing or reversing. The bare particle response \textit{yes} signals [\textsc{agree}] and [+], while \textit{no} signals [\textsc{reverse}] and [--]. Aligning with this distinction, in (\ref{ex:confird:Hong}), \textit{yes} is an acceptable response according to the features [\textsc{agree}] and [+] (\ref{ex:confirda:Hong}), while \textit{no} is acceptable with [\textsc{reverse}] and [--] (\ref{ex:confirdb:Hong}), identical to positive polar questions. In contrast, Contradictory \isi{IRDs} can have both (\ref{ex:contirda:Hong}) and (\ref{ex:contirdb:Hong}) as felicitous responses, and thus particle responses are ambiguous like negative polar questions: \textit{no} in (\ref{ex:contirda:Hong}) confirms \textit{¬p}, but \textit{no} in (\ref{ex:contirdb:Hong}) rejects \textit{¬p}.

\section{Background}
\label{sec:back:Hong}
\subsection{Framework}
\label{sec:frame:Hong}

Before turning into the formalization, this section briefly introduces a framework widely accepted for capturing the conventional effects of sentence types and their associated discourse moves. Since \citet{lewis1979scorekeeping} introduced conversational scoreboard, extended and enriched models have been developed in the literature \citep[e.g.,][]{farkas2010reacting, malamud2015three, rudin2022intonational}. I also adopt the \isi{discourse components} proposed by \citet{farkas2010reacting} and \citet{malamud2015three}.

The Common Ground (henceforth, \textit{CG}) is a set of commitments shared by discourse participants and is assumed to play a significant role in tracking participants’ commitments throughout the discourse \citep{stalnaker1978assertion}. Generally, the role of the discourse is often considered as expanding the \textit{CG} and reducing the context set (henceforth, \textit{cs}). However, subsequent works have identified a limitation of Stalnakerian \textit{CG} in its incapability to represent the individual \isi{commitment} of each participant with a single set. For example, \citet{gunlogson2003true} defends the necessity of separate tracking of each participant’s commitments. Following her idea, \citet{farkas2010reacting} set each participant’s discourse \isi{commitment} (henceforth, \textit{DC\textsubscript{X}}) apart from the \textit{CG}, while the \textit{CG} is reserved as a set of propositions that all interlocutors have agreed upon. Each interlocutor has individual \textit{DC\textsubscript{X}} which is a belief of one’s own, having a possibility to be added to the \textit{CG}. Thus, the total \isi{commitment} of speaker \textit{X} throughout the discourse is \textit{DC\textsubscript{X}} $\cup$ \textit{CG}. This \isi{commitment} is doxastic by default and does not need to be true in the world where the conversation takes place.\footnote{
There are different assumptions for other types of \isi{speech acts}. For example, the speaker of an imperative is assumed to have preferential commitments \citep{condoravdi2012imperatives, rudin2018risinga, rudin2018rising}.
}

In the process of discourse, the \textit{Table} records the \isi{Question Under Discussion} \citep[henceforth, \textit{QUD};][]{ginzburg1996dynamics, roberts1996information, roberts2012information}. In other words, the \textit{Table} is a stack that records ‘at-issue’ items. When the item is added to the \textit{Table}, the speaker projects possible future \textit{CG}, which is called the projected Common Ground (henceforth, \textit{CG*}). For example, an assertion projects the expressed proposition \textit{p} to the \textit{CG} (\textit{CG*}= \{\textit{CG} $\cup$ \{\textit{p}\}\}) and a polar question projects each of two possibilities, \textit{p} or \textit{¬p} (\textit{CG*} = \{\textit{CG} $\cup$ \{\textit{p}\}, \textit{CG} $\cup$ \{\textit{¬p}\}\}). The projected commitments of discourse participants (henceforth, \textit{DC\textsubscript{X}*}) are defined as analogous to the \textit{CG*}. \citet{malamud2015three} posit the \textit{DC\textsubscript{X}*} to allow the moves for tentative commitments of the speaker (speaker’s projected \isi{commitment}; henceforth, \textit{DC\textsubscript{sp}*}) or the speaker’s best guess on commitments of the addressee (addressee’s projected \isi{commitment}; henceforth, \textit{DC\textsubscript{ad}*}).\footnote{When the \textit{DC\textsubscript{X}} and the \textit{DC\textsubscript{X}*} are contrasted, I refer to the former as the \textit{present} \isi{commitment}.}

A cooperative discourse participant would remain consistent with their doxastic commitments in a single discourse move \citep{krifka2015bias}. I also suppose that the \textit{DC\textsubscript{X}*} should be consistent throughout the discourse, along with the present ones (i.e., \cap\textit{DC\textsubscript{X}} $\neq$ \emptyset, \cap\textit{DC\textsubscript{X}*} $\neq$ \emptyset, and \{\cap\textit{DC\textsubscript{X}}\} $\cap$ \{\cap\textit{DC\textsubscript{X}*}\} $\neq$ \emptyset). If the \isi{commitment}, whether present or projected, is restricted to worlds where \textit{p} is true, the worlds where \textit{p} does not hold are eliminated. As a result, the intersection with worlds where \textit{p} does not hold is bound to be empty, which is an unexpected outcome considering that the discourse aims to expand the \textit{CG}.

The discussion up to this point is summarized in (\ref{ex:discomp:Hong}).

%\protectedex{
\ea \label{ex:discomp:Hong}
    Discourse Components\\
    \ea \justifying Common Ground (\textit{CG}): the set of propositions that all speakers are publicly committed to \citep{stalnaker1978assertion}.
    \ex Discourse Commitment (\textit{DC\textsubscript{X}}): the set of propositions that the speaker has publicly committed to during the conversation up to the relevant time, and which are not shared by all the other participants \citep{farkas2010reacting}.
    \ex Table (\textit{T}): the stack that records the at-issue content in the conversation \citep{farkas2010reacting}.
    \ex Projected Common Ground (\textit{CG*}): the set of potential \textit{CG}s that gives possible resolutions for the top issue on the \textit{Table} in the next expected stage of the conversation \citep{farkas2010reacting, malamud2015three}.
    \ex Projected Discourse Commitment (\textit{DC\textsubscript{X}*}): the set of propositions that the speaker is expected to become committed to or the best guess of commitments made by other interlocutors \citep{malamud2015three}.
    \z
\z
%}

\subsection{Previous approaches}
\label{sec:prev:Hong}

Numerous accounts of \isi{RDs} have been proposed based on the components introduced above. Although the space precludes comprehensive overview of every approach, I will investigate a few representative analyses. All of these approaches involve both semantics and pragmatics, though the exact allocation between the two varies.

\subsubsection{Gunlogson (2003, 2008)}
\label{sec:Gunlogson:Hong}

\citet{gunlogson2003true} supposes \isi{commitment} requirements for both falling and \isi{rising declaratives}, but the former type locates the \isi{commitment} to the speaker while the latter type locates it to the addressee. Consequently, falling declaratives update the speaker’s \isi{commitment} set, whereas \isi{rising declaratives} update the addressee’s \isi{commitment} set. She also supposes that contextual evidence should support the addressee’s \isi{commitment} to make \isi{RDs} felicitous. By attributing the \isi{commitment} to the addressee while leaving the speaker’s \isi{commitment} open, the speaker can exhibit bias toward or \isi{commitment} to \textit{¬p}. However, it remains unclear for cases where the speaker lacks contextual evidence on the addressee’s \isi{commitment} to the expressed proposition.

Her later work \citep{gunlogson2008question} focuses on biased questions in out-of-blue contexts that function as questions with the speaker’s \isi{commitment} being contingent on the addressee’s. This concept can properly capture Confirmative \isi{IRDs} with a positive bias of the addressee’s \isi{commitment}. However, as pointed out by \citet{jeong2018intonation}, negative epistemic bias that could previously be explained by \citet{gunlogson2003true} cannot be predicted anymore. It is also unclear how her approaches can be expanded to \isi{ARDs} which seem to be more related to speaker-oriented commitments.

\subsubsection{Malamud \& Stephenson (2015)}
\label{sec:Malamud:Hong}

\citet{malamud2015three} develop an analysis of the tentativeness expressed by \isi{RDs} in terms of projected \isi{commitment} sets and metalinguistic issue (henceforth, \textit{MLI\textsuperscript{p}}), which is an inquisitive issue having a non-singleton set. The core effect of \isi{RDs} consists of adding \textit{MLI\textsuperscript{p}} and \textit{p} to the \textit{Table} and adding \textit{p} to the \textit{DC\textsubscript{sp}*}. Since \textit{MLI\textsuperscript{p}} takes the priority to be added to the stack, its two possible resolutions must precede the resolution of \textit{p}.\footnote{Often, only two possible resolutions for \textit{MLI\textsuperscript{p}} (\textit{R}1 and \textit{R}2) are assumed for the sake of simplicity, but there can be more than just two potential resolutions.} That is, the issue regarding \{\textit{p}\} can only be taken into consideration after the resolution of \textit{MLI\textsuperscript{p}}. Moreover, unlike canonical assertions, the proposition \textit{p} is added to the \textit{DC\textsubscript{sp}*} in the first place, and if the addressee uptakes the move and resolves the \textit{MLI\textsuperscript{p}} on the \textit{Table}, it would be moved to the \textit{DC\textsubscript{sp}}. Provided with the addressee's ratification, the resulting effect would be very similar to simply asserted \textit{p} in the first place. This approach is advantageous for predicting \isi{ARDs}. However, it is insufficient to capture \isi{IRDs}, especially Contradictory \isi{IRDs} where the speaker is not committed to the proposition, but its negation (\textit{¬p}). One might attempt to apply the notion of \textit{MLI\textsuperscript{p}} to negatively biased \isi{RDs}, but to the best of my knowledge, it has nothing to do with reversing the interlocutor's epistemic bias.

\subsubsection{Farkas \& Roelofsen (2017)}
\label{sec:Farkas:Hong}

\citet{farkas2017division} present the \isi{discourse effects} of \isi{IRDs}, couched in the Inquisitive Semantics framework \citep{ciardelli2013inquisitive, ciardelli2018inquisitive}. Their approach narrows the scope to \isi{IRDs} and assumes \isi{ARDs} are of a different nature. \isi{IRDs} share the inquisitive sentence radical with rising polar interrogatives, while the former is more marked than the latter. As their special effect, \isi{IRDs} signal the credence level of the speaker, which is zero to low. Their model has an advantage in negative bias with zero evidence of Contradictory \isi{RDs}. That is, the speaker’s negative bias is implied by having low credence at best (i.e., a preference for \textit{¬p} over \textit{p}). 

However, in their analysis it seems difficult to capture the case with a positive bias. In a positively biased case, the speaker assumes that \textit{p} is more probable than \textit{¬p}: the credence level seems to be higher than the average.

\subsubsection{Jeong (2018)}
\label{sec:Jeong:Hong}

\citet{jeong2018intonation} makes a clear distinction between two types of \isi{RDs}, which are classified as either tentative assertions or as biased questions based on intonational structures. Rising intonations, \textsc{rise-a} (assertive rises) and \textsc{rise-i} (inquisitive rises), call for a marked interpretation of morphosyntactically declarative utterances. \isi{ARDs} are marked because they are essentially assertive but are paired with rising intonation, while \isi{IRDs} are marked because they are essentially inquisitive but are paired with declarative syntax. For \isi{ARDs}, \textit{MLI\textsuperscript{p}} added to the \textit{Table} is what makes them differ from canonical falling declaratives. As \textit{MLI\textsuperscript{p}} is at the top of the table, it must be resolved prior to \textit{p}, identical to \citet{malamud2015three}. For \isi{IRDs}, they have the same sentence radical as polar interrogatives which are contributed from \textsc{rise-i}, but update the positive answer \textit{p} to the \textit{DC\textsubscript{ad}*}. This account is assumed to predict both positive and negative bias, the latter resulting from redundancy, which triggers the \isi{pragmatic reasoning} that the speaker has a reason to elicit further explanation or justification from the addressee. 

However, it is not clear how this account can be expanded to cases when the addressee’s present and projected \isi{commitment} sets are not redundant, yet the speaker coveys a negative bias. To properly account the negative bias, the analysis to come entertains an alternative way of relaxing ‘prior addressee utterance that entails \textit{p}’ to ‘prior contextual information that addressee thinks that \textit{p}’.\footnote{Related discussion will be presented in \sectref{sec:ird:Hong}.} Nonetheless, the overall idea that the inference of negative bias arises in cases where the prior context is such that the addressee is pivoted toward \textit{p} (instead of being neutral) still holds.

\subsubsection{Rudin (2018a, 2022)}
\label{sec:Rudin:Hong}

Adopting \citet{jeong2018intonation}'s key distinction between two fundamental types of \isi{RDs}, \citet{rudin2018risinga} presents a formal pragmatic examination, drawing on \citet{farkas2010reacting}. He assumes that falling intonation adds the informative content of a sentence in the speaker’s \isi{commitment} and rising intonation adds \textit{W}, the denotation of \{\textit{p}, \textit{¬p}\} that makes the \isi{commitment} trivial. He also assumes a pragmatic competition between discourse move minimal pairs. Falling declaratives and \isi{RDs} constitute a minimal pair, but the distinction lies in the fact that the former type commits the speaker to \textit{p} while the latter type does not, because of the conventional effect of the rising intonation. \isi{RDs} also constitute a minimal pair with rising polar interrogatives, only differing in whether \textit{¬p} is contained in the issue. The specific convention of \isi{RDs} stems from the competition between their second minimal pair, rising polar interrogatives. If the speaker chooses an \isi{RD}, a polar interrogative would be uncooperative. The only source for being uncooperative is \textit{¬p}, since they both contain \textit{p}. Also, as the speaker chose not to commit to \textit{p} (from the effect of the rising tune), it should be the addressee’s private beliefs that prevent \textit{p} to be added to the \textit{CG}. To put it together, the speaker expects the addressee to say \textit{p} is true, soliciting the addressee to commit to \textit{p}. Meanwhile, the bias of \isi{RDs} arises from the pragmatic competition with the other element of the minimal pair, falling declaratives, which differs in whether the speaker commits to \textit{p}. The speaker has limited evidence (positive bias) or the speaker knows the proposition is false (negative bias). \citet{rudin2022intonational} extends this idea by applying pragmatic analysis using the style of Optimality Theoretic tableaux.

However, expanding his account to \isi{ARDs}, with which the speaker apparently gives new information, is left open. The issue seems to be more complicated when metalinguistic issues are involved. He would say that the rising tune associated with \isi{ARDs} is associated with a different convention, but its exact nature requires further clarification, as he explicitly mentions that the discussion is restricted to \isi{IRDs}.

\section{Contributions of rising intonation}
\label{sec:contris:Hong}

Previous researchers have made various proposals on the effect of rising intonation in \isi{RDs}: (i) eliminating commitments \citep[e.g.,][]{gunlogson2008question, rudin2018risinga, rudin2022intonational}, (ii) adding metalinguistic issues \citep[e.g.,][]{malamud2015three}, (iii) indicating the violation of Gricean Maxims \citep[e.g.,][]{westera2017exhaustivity, westera2018rising}, or (iv) composing \isi{markedness} \citep[e.g.,][]{farkas2017division}. Instead, I propose that rising intonation overrides the convention of falling declaratives as in (\ref{ex:convfd:Hong}) in two ways: (i) conventionally increasing the inquisitive content of the proposition, and (ii) contextually projecting \isi{discourse components}. That is, assertive and inquisitive meanings are conventionally derived, while specific functions are determined by the \isi{pragmatic reasoning}.

\ea \label{ex:convfd:Hong} Convention of falling declaratives\\
    \ea \textit{Table\textsubscript{o}} = \textit{Table\textsubscript{i}} \cup {} \{\textit{p}\}\\
    \ex \textit{DC\textsubscript{sp, o}} = \textit{DC\textsubscript{sp, i}} \cup {} \{\textit{p}\}
    \z
\z

\subsection{Semantic convention}
\label{sec:semconv:Hong}

To elaborate the effect of rising intonation on \isi{semantic content}, I adopt the framework of Inquisitive Semantics (\citealt{ciardelli2013inquisitive, ciardelli2018inquisitive}, and references therein). This framework posits that a sentence not only conveys informative content but also expresses inquisitive content by raising an issue. To illustrate, consider \figref{fig:2content:Hong}. Purely informative propositions are represented on the horizontal axis, where inquisitive content is trivial. For example, falling declaratives are non-inquisitive by default. Meanwhile, propositions on the vertical axis are purely inquisitive. Rising polar interrogatives are basically non-informative with informative content being trivialized. All other propositions, of which informative and inquisitive content are both non-trivial, are located off the axes.

\begin{figure}
\centering
    \begin{tikzpicture}\centering
    \draw[->] (0,0) -- (3,0) node[anchor=north west] {Informative};
    \draw[->] (0,0) -- (0,3) node[anchor=south east] {Inquisitive};
    \draw[densely dotted, <-] (0,2.5) -- (2.5,2.5) node[pos=0.5, above] {?};
    \draw[densely dotted, ->] (2.5,2.5) -- (2.5,0) node[pos=0.5, anchor=west] {!};
    \end{tikzpicture}
\caption{Informative content and inquisitive content}
\label{fig:2content:Hong}
\end{figure}

Since the convention of each \isi{RDs} partially overlaps with the corresponding two canonical sentence types, I argue that both the informative content and inquisitive content of \isi{RDs} are not trivial. Therefore, I place \isi{RDs} off the axes where neither informative content nor inquisitive content is trivial, as illustrated in \figref{fig:fig2:Hong}.\footnote{I suppose other types of non-canonical interrogatives (e.g., tag questions, negative polar questions) can also be located off the axes. Due to space limitations, further discussions on other types are left for future work.} This figure illustrates that \isi{ARDs} are more informative than inquisitive, while \isi{IRDs} are more inquisitive than informative. \isi{ARDs} are located closer to the informative horizontal axis than to the inquisitive vertical axis, demonstrating that they are more informative than inquisitive. While primarily remaining informative, \isi{ARDs} are less informative than canonical falling declaratives as they are tentative assertions. On the other hand, the pattern is reversed for \isi{IRDs}, as they are located closer to the inquisitive vertical axis: \isi{IRDs} exhibit more inquisitiveness than informativeness. Compared with canonical rising polar interrogatives, \isi{IRDs} are more informative since they convey additional information on the bias. 

\begin{figure}
\centering
    \begin{tikzpicture}\centering
    \draw[->] (0,0) -- (3,0) node[anchor=north west] {Informative};
    \draw[->] (0,0) -- (0,3) node[anchor=south east] {Inquisitive};
    \draw[densely dotted, <-] (0,2.5) -- (2.5,2.5) node[pos=0.5, above] {?};
    \draw[densely dotted, ->] (2.5,2.5) -- (2.5,0) node[pos=0.5, anchor=west] {!};
    \draw[-] (0,0) -- (2.5,2.5) node[pos=0.6, anchor=east, xshift=-5pt] {IRDs} node[pos=0.4, anchor=west, xshift=5pt] {ARDs};
    \end{tikzpicture}
\caption{Informative content and inquisitive content of RDs}
\label{fig:fig2:Hong}
\end{figure}

The distinction between two \isi{RDs} reflects that \isi{ARDs} denote a singleton set \{\textit{p}\} like falling declaratives \citep{hamblin1971mathematical}, whereas \isi{IRDs} denote a non-singleton set \{\textit{p}, \textit{¬p}\} like polar interrogatives \citep{karttunen1977syntax}. This difference between singleton and non-singleton sets arises from the effect of rising intonation, which increases the inquisitiveness of the \isi{semantic content}. According to \citet{jeong2018intonation}'s experimental results, lower rise with a high nuclear pitch accent (H*H-H\%) indicates \isi{ARDs}, while a steeper rise with a low nuclear pitch accent (L*H-H\%) is related to \isi{IRDs}. I further propose that a weak rise increases inquisitive content up to the point where it is no higher than informative content, maintaining the proposition primarily informative with a singleton set. In contrast, with a steep rise, inquisitive content surpasses informative content, and thus the content is shifted to a non-singleton set, turning the speech act of a clause into a question. As a result, \isi{ARDs} denote \{\textit{p}\}, whereas \isi{IRDs} denote \{\textit{p}, \textit{¬p}\}, aligning with our intuitive observation that \isi{ARDs} are \textit{assertion-like} while \isi{IRDs} are \textit{question-like}.

\subsection{Discourse components}
\label{sec: discomp:Hong}

The second contribution of rising intonation is projecting \isi{discourse components}. The idea is that the update convention of commitments is not completely restricted syntactically. The notion that a clause type does not deterministically constrain the commitments aligns with the perspective of \citet{beyssade2006speech}. They propose that the default case, in which the speaker's \isi{commitment} aligns with their call on the addressee, can be hybrid. Similarly, I intend to come up with a model that allows a syntactic structure to be involved in a twofold update. However, unlike theirs, my model does not necessarily require that a declarative clause updates the speaker’s \isi{commitment}, and is closer to the approaches of \citet{gunlogson2008question} and \citet{rudin2018rising, rudin2022intonational}, with a few modifications.

\citet{gunlogson2008question} proposes that rising intonation indicates contingent \isi{commitment}, which is dependent on the addressee’s \isi{commitment}. However, \isi{RDs} need not always be dependent on the prior utterance of other interlocutors, as in (\ref{ex:tonight:Hong}).

\ea \label{ex:tonight:Hong} \justifying{[Context: A and B made plans two days ago to get drinks tonight. They haven’t spoken about it since.]}\\
    A: \textit{We’re still on for tonight?}
\z
The \isi{IRD} in (\ref{ex:tonight:Hong}) does not require the addressee’s prior utterance but it is merely the speaker’s expectation toward the addressee (i.e., \textit{DC\textsubscript{ad}*}) which differs from
\citeauthor{gunlogson2008question}'s (\citeyear{gunlogson2008question}) contingent \isi{commitment}. It demonstrates that, although the speaker assumes that the addressee is committed to \textit{p}, it does not necessarily imply that the \isi{commitment} is contingent. 

As summarized in \sectref{sec:Rudin:Hong}, \citet{rudin2018risinga, rudin2022intonational} argues that final rising tone in an \isi{RD} indicates the speaker's lack of \isi{commitment} to its expressed proposition. Following some core ideas of his view, I analyze rising intonation as modifying the status of commitments. However, in my proposal, rising intonation projects commitments, rather than indicating a lack of commitment. For example, the speaker can project \textit{p} to either their own \isi{commitment} (\textit{DC\textsubscript{sp}*}) or the addressee’s (\textit{DC\textsubscript{ad}*}) with rising intonation. I further argue that the application of the projection can be expanded to other \isi{discourse components} as well, especially the projected Table (\textit{Table*}), which will be defined in \sectref{sec:ard:Hong}.

\section{Interpreting rising declaratives}
\label{sec:intris:Hong}

This section focuses on systematizing the specific contextual conditions that determine the projection of a discourse component, leading to a particular interpretation. I then present \isi{discourse effects} for each paradigm of \isi{RDs}.

\subsection{Assertive rising declaratives}
\label{sec:ard:Hong}

\subsubsection{Contextual interpretation}
\label{sec:ardcont:Hong}

For \isi{ARDs}, contextual information about their relevance to the current \isi{QUD} determines their specific paradigm. This relevance can be assessed by comparing the \isi{semantic content} of the current \isi{QUD} with that of the \isi{ARD}. An \isi{ARD} that conveys a proposition that is a subset of the current \isi{QUD} is construed as an Epistemic Uncertainty \isi{ARD}, whereas an \isi{ARD} that conveys a proposition that is not a subset of the current \isi{QUD} is understood as a Metalinguistic Uncertainty \isi{ARD}. To illustrate, compare two \isi{ARDs} in (\ref{ex:2ards:Hong}), repeated in (\ref{ex:2ards2:Hong}).

\ea \label{ex:2ards2:Hong}
    \ea \label{ex:2ards2a:Hong} Epistemic Uncertainty \isi{ARD}\\
    A: Where's Sally?\\
    B: (Um...) \textit{She's home?}
    \ex \label{ex:2ards2b:Hong} Metalinguistic Uncertainty \isi{ARD}\\
    A: Does he speak Chinese?\\
    B: \textit{He speaks Cantonese?}\\
    \z
\z
In (\ref{ex:2ards2a:Hong}), the content of an \isi{ARD} is \{\textit{Sally is home}\}. This proposition is a subset of the current \isi{QUD}, \{\textit{Sally is home, Sally is at school, Sally is at the café, …}\}, which is updated to the topmost stack of the \textit{Table} (i.e., $p \in P$). In contrast, in (\ref{ex:2ards2b:Hong}), \{\textit{He speaks Cantonese}\} is not a subset of \{\textit{He speaks Chinese, He doesn’t speak Chinese}\} (i.e., $p \notin P$).  The result correctly categorizes (\ref{ex:2ards2a:Hong}) as an Epistemic Uncertainty \isi{ARD} and (\ref{ex:2ards2b:Hong}) as a Metalinguistic Uncertainty \isi{ARD}.

\subsubsection{Discourse effects}
\label{sec:ardde:Hong}

Based on the contextual cues discussed above, rising intonation of \isi{ARDs} projects discourse commitments, accordingly. Epistemic Uncertainty \isi{ARDs} are analyzed as updating \textit{p} to the \textit{DC\textsubscript{sp}*}, since the speaker's uncertainty is concerned with the truth value of the expressed proposition.\footnote{To some extent, my account is similar to \citet{malamud2015three} and differs from \citet{jeong2018intonation}  by updating \textit{p} to the \textit{DC\textsubscript{sp}*} instead of the \textit{DC\textsubscript{sp}}, but differs from both in that I do not utilize \textit{MLI\textsuperscript{p}}.} Alternatively, to incorporate Metalinguistic Uncertainty \isi{ARDs} into the Table model, I introduce a modified version of the `projected' \textit{Table} \citep[henceforth, \textit{Table*};][]{malamud2012three, bhadra2020semantics}, which represents a tentative proposal of raising an issue.\footnote{\citet{bhadra2020semantics} defines the \textit{Table*} as an ordered stack which contains tentative issues (i.e., proposals to be added to the Table for future resolution). I follow her account with the formulation of the \textit{Table*} as a stack, but additionally provide a more restrained definition for the tentative issue: the issue which the speaker expects to be relevant to the current \isi{QUD}. A further difference between her approach and mine comes from the \textit{CG*}. She claims that the tentative issue updated to the \textit{Table*} does not update the \textit{CG*}, but I argue that the issue on the \textit{Table*} also affects the \textit{CG*}, remaining consistent with \citet{malamud2012three}. Moreover, in treating the \textit{CG*} projected by questions, my approach aligns with the framework proposed by \citet{farkas2010reacting} and \citet{malamud2015three} with a non-singleton set. In contrast, \citet{bhadra2020semantics} deviates from these and adopts a singleton-set approach to polar interrogatives, which traces its roots back to \citet{bolinger1978yes} and proposes salient alternatives (SalientAlts), provided by the context, to capture the interrogative force of polar interrogatives.}  The \textit{Table*} is defined analogous to the \textit{DC\textsubscript{X}*} and the \textit{CG*}, since they all reflect the expected next stage of conversation. The \textit{CG*} is a set of potential future \textit{CG}s relative to the at-issue content on the \textit{Table}. Likewise, the \textit{DC\textsubscript{X}*} is a tentative \isi{commitment} of the speaker (the \textit{DC\textsubscript{sp}*}) or the speaker’s expectation or guess to the \isi{commitment} of other participants in the discourse (the \textit{DC\textsubscript{ad}*}) and thus the \textit{DC\textsubscript{X}*} also represents the expected next stage of conversation. In the same way, the \isi{propositional content} added in the \textit{Table*} represents the speaker’s expectation on the issue to be relevant to the current \isi{QUD}. Therefore, updating \isi{semantic content} to the \textit{Table*} reflects the speaker’s uncertainty on the relevance to the current \isi{QUD} \citep{roberts1996information, roberts2012information} and their expectation of the information becoming relevant to it. 

To recapitulate, the \isi{discourse effects} of \isi{ARDs} are presented in (\ref{ex:deard:Hong}) where a subscripted \textit{i} stands for ‘input’ and a subscripted \textit{o} stands for ‘output’.

\ea \label{ex:deard:Hong}
    \ea \label{ex:dearda:Hong} Discourse effect of Epistemic Uncertainty \isi{ARDs}\\
    (i) \textit{Table\textsubscript{o}} = \textit{Table\textsubscript{i}} $\cup$ \{\textit{p}\}\\
    (ii) \textit{DC\textsubscript{sp,o}*} = \textit{DC\textsubscript{sp,i}*} $\cup$ \textit{p}\\
    \ex \label{ex:deardb:Hong} Discourse effect of Metalinguistic Uncertainty \isi{ARDs}\\
    (i) \textit{Table\textsubscript{o}*} = \textit{Table\textsubscript{i}*} $\cup$ \{\textit{p}\}\\
    (ii) \textit{DC\textsubscript{sp,o}} = \textit{DC\textsubscript{sp,i}} $\cup$ \textit{p}\\
    \z
\z

The proposed account also provides an explanation for the politeness effect observed in both types of \isi{ARDs} within a unified discourse process \citep[cf., e.g.,][]{jeong2021deriving}. This process involves moving an item from projected components to present ones. Projected components require the addressee’s ratification for the progression of discourse, enhancing the addressee’s face by being indirect, which contributes to politeness. Since both types of \isi{ARDs} update projected components, either \textit{DC\textsubscript{sp}*} or \textit{Table*}, they can both be employed as politeness strategies.

I now move on to the visual representation of updates within the conversational scoreboard model. Consider (\ref{ex:euvis:Hong}) with the proposed analysis in \tabref{tab:euvis:Hong}.\footnote{Following \citet{jeong2018intonation}, I assume \{\textit{q}, \textit{¬q}\} at \textit{t}\textsubscript{1} is retracted and replaced by \{\textit{p}\} at \textit{t}\textsubscript{2}. This can be done as speaker B takes \textit{p} as a partial answer. This retraction process is accepted only when the speaker has assurance on the fact that \textit{p} is relevant to the issue on the \textit{Table}.} At \textit{t}\textsubscript{3}, speaker A ratifies the proposition \textit{p} which speaker B is uncertain about. Then, \textit{p} is automatically moved to \textit{DC\textsubscript{B}} (step 1), allowing the issue to be resolved in a way that expands the \textit{CG} (step 2).\footnote{These two steps take place simultaneously, but are visually separated only for the ease of explanation.}

\ea \label{ex:euvis:Hong}
    {[Context: A and B are sorting paint cans in a store into a ‘red’ bin and an ‘orange’ bin. A points to orangish-red paint.]}\\
    A: What color would you say this is? \hfill \textit{t}\textsubscript{1}\\
    B: \textit{It’s red?}\hfill \textit{t}\textsubscript{2}\\
    A: Yeah, I think so too. \hfill \textit{t}\textsubscript{3}\\

\renewcommand\tabularxcolumn[1]{m{#1}}
\begin{table}
\begin{tabularx}{\textwidth}{>{\arraybackslash}m{1cm}>{\centering\arraybackslash}X>{\centering\arraybackslash}X>{\centering\arraybackslash}X>{\centering\arraybackslash}X>{\centering\arraybackslash}X}
\lsptoprule
      & \multirow{2}{=}{\centering A utters \textit{q?} in \textit{t}\textsubscript{1}} & \multirow{2}{=}{\centering B utters \textit{p?} in \textit{t}\textsubscript{2}} & \multirow{2}{=}{\centering A utters \textit{Yeah} in \textit{t}\textsubscript{3}} & \multicolumn{2}{c}{\centering after \textit{t\textsubscript{3}}} \\ \cline{5-6}
      & & & & step 1 & step 2 \\ \midrule      
    \textit{Table} & $\langle \{q, \neg q\} \rangle$ & $\langle \{\textit{\textbf{p}}\} \rangle$ & $\langle \{\textit{\textbf{p}}\} \rangle$ & $\langle \{\textit{\textbf{p}}\} \rangle$ & \\ 
    \textit{Table*} & & & & & \\ 
    \textit{DC\textsubscript{A}} & & & $\{\textit{\textbf{p}}\}$ & \cellcolor{gray!15}$\{p\}$ & \\ 
    \textit{DC\textsubscript{A}*} & & & & & \\ 
    \textit{DC\textsubscript{B}} & & & & \cellcolor{gray!15}$\{p\}$\tikzmark{a} & \\ 
    \textit{DC\textsubscript{B}*} & & $\{\{\textit{\textbf{p}}\}\}$ & $\{\{p\}\}$ & $(\{\{p\}\})$\tikzmark{b} & \\ 
    \textit{CG} & \textit{s}\textsubscript{1} & \textit{s}\textsubscript{1} & \textit{s}\textsubscript{1} & \textit{s}\textsubscript{1} & \textit{s}\textsubscript{2} = $\{\textit{s}\textsubscript{1} \cup \{p\}\}$\\ 
    \textit{CG*} & \{\textit{s}\textsubscript{1} $\cup$ \{\textit{q}\}, \textit{s}\textsubscript{1} $\cup$ \{\neg \textit{q}\}\} & \{\textit{s}\textsubscript{1} $\cup$ \{\textit{\textbf{p}}\}\} & \{\textit{s}\textsubscript{1} $\cup$ \{\textit{\textbf{p}}\}\} & \{\textit{s}\textsubscript{1} $\cup$ \{\textit{\textbf{p}}\}\} & \\
    \lspbottomrule
\end{tabularx}
  \begin{tikzpicture}[overlay, remember picture]
    \draw [->, thick] ([xshift=-12mm]{pic cs:b}) [bend left] to ([xshift=-9mm]{pic cs:a});
    \end{tikzpicture}
    \caption{Formal analysis of (\ref{ex:euvis:Hong}) with an Epistemic Uncertainty \isi{ARD}}
    \label{tab:euvis:Hong}
\end{table}
\z

Metalinguistic Uncertainty \isi{ARDs} can also be analyzed as automatic movement of \{\textit{p}\} from the \textit{Table*} to the \textit{Table} with the addressee’s ratification. Consider (\ref{ex:muvis:Hong}) with the proposed analysis in \tabref{tab:muvis:Hong}. With the falling \textit{Oh} at \textit{t}\textsubscript{3}, speaker A confirms speaker B’s expectation to update the \textit{Table}. After the automatic move process to the \textit{Table}, the rest of the convention is identical to that of falling declaratives.

\ea \label{ex:muvis:Hong}
    A: Does he speak Spanish? \hfill \textit{t}\textsubscript{1}\\
    B: \textit{He speaks Ladino?}\hfill \textit{t}\textsubscript{2}\\
    A: Oh, I see. \hfill \textit{t}\textsubscript{3}\\
\renewcommand\tabularxcolumn[1]{m{#1}}
\begin{table}[H]
\begin{tabularx}{\textwidth}{>{\arraybackslash}m{1cm}>{\centering\arraybackslash}X>{\centering\arraybackslash}X>{\centering\arraybackslash}X>{\centering\arraybackslash}X>{\centering\arraybackslash}X}
\lsptoprule
      & \multirow{2}{=}{\centering A utters \textit{p?} in \textit{t}\textsubscript{1}} & \multirow{2}{=}{\centering B utters \textit{p?} in \textit{t}\textsubscript{2}} & \multirow{2}{=}{\centering A utters \textit{Oh} in \textit{t}\textsubscript{3}} & \multicolumn{2}{c}{\centering after \textit{t\textsubscript{3}}} \\ \cline{5-6}
      & & & & step 1 & step 2 \\ \midrule      
    \textit{Table} & $\langle \{\textbf{\textit{p}, \textit{\neg p}}\} \rangle$ & & & $\langle \{\textit{\textbf{p}}\} \rangle$ \tikzmark{c} & \\ 
    \textit{Table*} & & $\langle \{\textit{\textbf{p}}\} \rangle$ & $\langle \{\textit{p}\} \rangle$ & $(\langle \{p\} \rangle)$ \tikzmark{d} & \\ 
    \textit{DC\textsubscript{A}} & & & $\{\textit{\textbf{p}}\}$\cellcolor{gray!15} & $\{\textit{\textbf{p}}\}$\cellcolor{gray!15} & \\ 
    \textit{DC\textsubscript{A}*} & & & & & \\ 
    \textit{DC\textsubscript{B}} & & $\{\textit{\textbf{p}}\}$ & \cellcolor{gray!15}$\{p\}$ & \cellcolor{gray!15}$\{p\}$ & \\ 
    \textit{DC\textsubscript{B}*}  & & & & & \\ 
    \textit{CG} & \textit{s}\textsubscript{1} & \textit{s}\textsubscript{1} & \textit{s}\textsubscript{1} & \textit{s}\textsubscript{1} & \textit{s}\textsubscript{2} = $\{\textit{s}\textsubscript{1} \cup \{p\}\}$\\ 
    \textit{CG*} & \{\textit{s}\textsubscript{1} $\cup$ \{\textit{p}\}, \textit{s}\textsubscript{1} $\cup$ \{\neg \textit{p}\}\} & \{\textit{s}\textsubscript{1} $\cup$ \{\textit{\textbf{p}}\}\} & \{\textit{s}\textsubscript{1} $\cup$ \{\textit{\textbf{p}}\}\} & \{\textit{s}\textsubscript{1} $\cup$ \{\textit{\textbf{p}}\}\} & \\
    \lspbottomrule
\end{tabularx}
  \begin{tikzpicture}[overlay, remember picture]
    \draw [->, thick] ([xshift=-12mm]{pic cs:d}) [bend left] to ([xshift=-9mm]{pic cs:c});
    \end{tikzpicture}
    \caption{Formal analysis of (\ref{ex:muvis:Hong}) with a Metalinguistic Uncertainty \isi{ARD}}
    \label{tab:muvis:Hong}
\end{table}
\z


\subsection{Inquisitive rising declaratives}
\label{sec:ird:Hong}

\subsubsection{Contextual interpretation}
\label{sec:irdcont:Hong}

Analogous to \isi{ARDs}, the specific use of \isi{IRDs} is also communicated throughout the close interaction with discourse context. Contradictory \isi{IRDs} are attested when the context indicates that the addressee believes (or is at least biased toward) \textit{p}. Consider the examples below:

\ea \label{ex:irdno:Hong}
    \ea \label{ex:irdnoa:Hong} Confirmative \isi{IRD}\\
    {[Context: Same as (\ref{ex:2irdsa:Hong}).]}\\
    A: There's one flight to Seoul.\\
    B: \textit{The flight leaves at 10am?}\\
    \ex \label{ex:irdnob:Hong} Contradictory \isi{IRD}\\
    A: Dave is a good singer.\\
    B: \textit{Dave is a good singer?}\\
    \z
\z
(\ref{ex:irdnoa:Hong}) is biased toward the expressed proposition \textit{p}, whereas (\ref{ex:irdnob:Hong}) is biased toward its negation \textit{\neg p}. In both contexts, the speaker assumes that the addressee would be committed to \textit{p}, but they differ in contextual cues. In (\ref{ex:irdnoa:Hong}), the context does not directly indicate whether the addressee (speaker A) has a \isi{commitment} on \textit{p}. In other words, there is no explicit evidence in the context to support the addressee’s belief that the flight leaves at 10am. On the contrary, the addressee’s belief on \textit{p} is evident to the speaker in (\ref{ex:irdnob:Hong}) from the explicit expression. It is shown that the contradictory use of \isi{IRDs} is not permitted unless it is supported by the appropriate contextual information. This aligns with the arguments put forth in previous studies \citep{gunlogson2003true, farkas2017division} and experimental results \citep{jeong2018intonation} regarding the necessary contextual condition for contradictory \isi{IRDs}: Contradictory \isi{IRDs} are attested when the context indicates that the addressee believes \textit{p}.

With this account, the cognitive process of deriving negative bias becomes apparent. Similar to \citet{jeong2018intonation}, the negative bias arises from the \isi{pragmatic reasoning} which suggests that the speaker has a reason to seek further explanation or justification from the addressee. The expression of negative bias occurs when the speaker, in a context where the addressee’s \isi{commitment} on \textit{p} is evident, makes a best guess on the addressee’s \isi{commitment} (in accordance with the definition of the \textit{DC\textsubscript{ad}*}). This intentional opening up of the issue regarding \{\textit{p}, \textit{\neg p}\} in order to double-check the addressee’s \isi{commitment} on \textit{p} is unnecessary in typical context. However, as the speaker deliberately brings up the issue of the interlocutor’s assumed \isi{commitment}, this leads to the \isi{pragmatic reasoning} of urging for an additional explanation on the addressee’s \isi{commitment}, due to the speaker’s negative bias.

\subsubsection{Discourse effects}
\label{sec:irdde:Hong}

I propose that Confirmative \isi{IRDs} exhibit a positive bias by adding \textit{p} to the \textit{DC\textsubscript{sp}*}, indicating the speaker’s ‘weaker’ \isi{commitment}. In contrast, Contradictory \isi{IRDs} convey a negative bias by lacking speaker’s commitment and instead adding it to the \textit{DC\textsubscript{ad}*}. This effect stems from the following \isi{pragmatic reasoning}: In a context in which Contradictory \isi{IRDs} are possible, it is assumed that the addressee has either asserted or presupposed the proposition \textit{p}. The fundamental assumption for cooperative discourse is to enhance mutual information, and thus expanding the common ground among participants is the most plausible act. However, the fact that the speaker refrains from adding \textit{p} to their own \isi{commitment} sets, and instead updates it to the \textit{DC\textsubscript{ad}*} may present two potential interpretations regarding bias: either ignorant or negative. Given that an ignorant speaker might have chosen polar interrogatives instead \citep{goodhue2022isn}, the decision to update the \textit{DC\textsubscript{ad}*} signals a negative bias.

Regarding Mirative \isi{IRDs}, I follow \citeauthor{rett2021semantics}'s (\citeyear{rett2021semantics}) concept of illocutionary not-at-issue content. Rett distinguishes illocutionary not-at-issue content (e.g., mirative markers) from descriptive not-at-issue content. A key divergence lies in phenomena related to denial: while denying descriptive not-at-issue content leads to contradiction, denying illocutionary not-at-issue content results in Moore’s Paradox. Similarly, the direct negation of Mirative \isi{IRDs} doesn't yield contradiction but rather invokes Moore’s Paradox. Consider the example in (\ref{ex:mir:Hong}): the last sentence which follows a Mirative \isi{IRD} is infelicitous, but it is not a contradiction. 

\ea \label{ex:mir:Hong}
    \justifying{[Context: A and B are watching a girl give a very professional performance in a school debate. From this, A is thinking that she might be 12 or 13 years old.]}\\
    A: She’s amazing.\\
    B: I know, and she’s only 9 years old.\\
    A: (What?) \textit{She’s nine?} \# I \textsc{knew} that she is nine.
\z
From this evidence, I model Mirative \isi{IRDs} as illocutionary not-at-issue content which updates flavored \isi{commitment} to the \textit{DC\textsubscript{sp}} \citep{rett2021semantics}, as defined in (\ref{ex:dc:Hong}). Flavored discourse \isi{commitment} reflects the speaker's attitude other than belief, including mirativity.\footnote{Other sets of propositions of the form \textit{believes}\textit{\textsubscript{a}}(\textit{p}), \textit{is-pleased}\textit{\textsubscript{a}}(\textit{p}), or \textit{is-not-surprised}\textit{\textsubscript{a}}(\textit{p}) are also proposed by \citet{rett2021semantics}.}

\ea \label{ex:dc:Hong}
    Discourse Commitments \hfill{\citep[][326]{rett2021semantics}}\\
    Let \textit{DC\textsubscript{a}} be sets of propositions of the form \textit{is-surprised}\textit{\textsubscript{a}}(\textit{p}), representing the public commitments with respect to a discourse in which \textit{a} and \textit{b} are the participants, where
    \textit{is-surprised}\textit{\textsubscript{a}}(\textit{p}) is a public \isi{commitment} of \textit{a} iff ‘\textit{a} is surprised with \textit{p}’ is a mutual belief of \textit{a} and \textit{b}.\\
\z

To summarize, the \isi{discourse effects} of \isi{IRDs} are proposed in (\ref{ex:deird:Hong}): in (\ref{ex:deirda:Hong}) for Confirmative \isi{IRDs}, in (\ref{ex:deirdb:Hong}) for Contradictory \isi{IRDs}, and in (\ref{ex:deirdc:Hong}) for Mirative \isi{IRDs}. Note that (\ref{ex:deirdc:Hong}) is identical to (\ref{ex:deirdb:Hong}), except for the update of \textit{DC\textsubscript{sp}} with \textit{is-surprised}\textit{\textsubscript{a}}(\textit{p}). In this respect, Mirative \isi{IRDs} are analyzed as additionally implicating not-at-issue content, i.e., mirativity, on top of Contradictory \isi{IRDs}.

\ea \label{ex:deird:Hong}
    \ea \label{ex:deirda:Hong} Discourse effect of Confirmative \isi{IRDs}\\
        \ea \textit{Table\textsubscript{o}} = \textit{Table\textsubscript{i}} $\cup$ \{\textit{p}, \textit{\neg p}\}\\
        \ex \textit{DC\textsubscript{sp,o}*} = \textit{DC\textsubscript{sp,i}*} $\cup$ \textit{p}\\
        \z
    \ex \label{ex:deirdb:Hong} Discourse effect of Contradictory \isi{IRDs}\\
        \ea \textit{Table\textsubscript{o}} = \textit{Table\textsubscript{i}} $\cup$ \{\textit{p}, \textit{\neg p}\}\\
        \ex \textit{DC\textsubscript{ad,o}*} = \textit{DC\textsubscript{ad,i}*} $\cup$ \textit{p}\\
        \z
    \ex \label{ex:deirdc:Hong} Discourse effect of Mirative \isi{IRDs}\\
        \ea \textit{Table\textsubscript{o}} = \textit{Table\textsubscript{i}} $\cup$ \{\textit{p}, \textit{\neg p}\}\\
        \ex \textit{DC\textsubscript{ad,o}*} = \textit{DC\textsubscript{ad,i}*} $\cup$ \textit{p}\\
        \ex \textit{DC\textsubscript{sp,o}} = \textit{DC\textsubscript{sp,i}} $\cup$ \textbf{is-surprised}\textit{\textsubscript{a}}(\textit{p})\\
        \z
    \z
\z

The procedure for resolving the issue with Confirmative \isi{IRDs} as in (\ref{ex:Seoul:Hong}) is schematized in \tabref{tab:confird:Hong}. The resolution of the issue is rendered by the automatic movement of \textit{p} from \textit{DC\textsubscript{B}*} to  \textit{DC\textsubscript{B}} as presented in step 1. In step 2, speakers have a joint \isi{commitment} to \textit{p}. Thus, the issue is resolved from the \textit{Table} and expands the \textit{CG}.

\ea \label{ex:Seoul:Hong}
    {[Context: B is buying a ticket for a flight to Seoul at the airport.]}\\
    A: There's one flight to Seoul. \hfill \textit{t}\textsubscript{1}\\
    B: \textit{The flight leaves at 10am?} \hfill \textit{t}\textsubscript{2}\\
    A: Yes, from Gate 5. \hfill \textit{t}\textsubscript{3}\\
\renewcommand\tabularxcolumn[1]{m{#1}}
\begin{table}
\begin{tabularx}{\textwidth}{>{\arraybackslash}m{1cm}>{\centering\arraybackslash}X>{\centering\arraybackslash}X>{\centering\arraybackslash}X>{\centering\arraybackslash}X>{\centering\arraybackslash}X}
\lsptoprule
      & \multirow{2}{=}{\centering A utters \textit{q} in \textit{t}\textsubscript{1}} & \multirow{2}{=}{\centering B utters \textit{p?} in \textit{t}\textsubscript{2}} & \multirow{2}{=}{\centering A utters \textit{Yes} in \textit{t}\textsubscript{3}} & \multicolumn{2}{c}{\centering after \textit{t\textsubscript{3}}} \\ \cline{5-6}
      & & & & step 1 & step 2 \\ \midrule      
    \textit{Table} & $\langle \{\textit{\textbf{q}}\} \rangle$ & $\langle \{\textit{p, \neg p}\} \rangle$ & $\langle \{\textit{\textbf{p}}\} \rangle$ & $\langle \{\textit{\textbf{p}}\} \rangle$ & \\ 
    \textit{Table*} & & & & & \\ 
    \textit{DC\textsubscript{A}} & \{\textit{\textbf{q}}\} &  \{\textit{q}\} & $\{\textit{\textbf{p}}\}$ & \cellcolor{gray!15}$\{p\}$ & \\ 
    \textit{DC\textsubscript{A}*} & & & & & \\ 
    \textit{DC\textsubscript{B}} & & \{\textit{q}\} & & \cellcolor{gray!15}$\{p\}$\tikzmark{e} & \\ 
    \textit{DC\textsubscript{B}*} & & $\{\{\textit{\textbf{p}}\}\}$ & $\{\{p\}\}$ & $(\{\{p\}\})$\tikzmark{f} & \\
    \textit{CG} & \textit{s}\textsubscript{1} & \textit{s}\textsubscript{2} = $\{\textit{s}\textsubscript{1} \cup \{q\}\}$ & \textit{s}\textsubscript{2} & \textit{s}\textsubscript{2} & \textit{s}\textsubscript{3} = $\{\textit{s}\textsubscript{2} \cup \{p\}\}$\\ 
    \textit{CG*} & \{\textit{s}\textsubscript{1} $\cup$ \{\textit{\textbf{q}}\}\} & \{\textit{s}\textsubscript{2} $\cup$ \{\textit{\textbf{p}}\}, \textit{s}\textsubscript{2} $\cup$ \{\textit{\neg p}\}\} & \{\textit{s}\textsubscript{2} $\cup$ \{\textit{\textbf{p}}\}\} & \{\textit{s}\textsubscript{2} $\cup$ \{\textit{\textbf{p}}\}\} & \\
    \lspbottomrule
\end{tabularx}
  \begin{tikzpicture}[overlay, remember picture]
    \draw [->, thick] ([xshift=-12mm]{pic cs:f}) [bend left] to ([xshift=-9mm]{pic cs:e});
    \end{tikzpicture}
    \caption{Formal analysis of (\ref{ex:Seoul:Hong}) with a Confirmative \isi{IRD}}
    \label{tab:confird:Hong}
\end{table}
\z

The issue of Contradictory \isi{IRDs} is more difficult to resolve. Due to the empty \textit{DC\textsubscript{sp}} and \textit{DC\textsubscript{sp}*}, the discourse remains in a conversational crisis even after the addressee's ratification. An illustration of the analysis proposed for a Contradictory \isi{IRD} such as in (\ref{ex:answer:Hong}) is given in \tabref{tab:contraird:Hong}. 

\ea \label{ex:answer:Hong}
\justifying
    \noindent A: (student) The answer to this problem is 5 because the square root of 9 is 2 and 2+3 is 5. \hfill \textit{t}\textsubscript{1}\\
    B: (teacher) \textit{The square root of 9 is 2?} \hfill \textit{t}\textsubscript{2}\\
    A: Yes. \hfill \textit{t}\textsubscript{3}\\
\renewcommand\tabularxcolumn[1]{m{#1}}
\begin{table}
\begin{tabularx}{\textwidth}{>{\arraybackslash}m{1cm}>{\centering\arraybackslash}X>{\centering\arraybackslash}X>{\centering\arraybackslash}X}
\lsptoprule
    & A utters \textit{p} in \textit{t}\textsubscript{1} & B utters \textit{p?} in \textit{t}\textsubscript{2} & A utters \textit{Yes} in \textit{t}\textsubscript{3} \\ \midrule
    \textit{Table} & $\langle \{\textbf{\textit{p}}\} \rangle$ & $\langle \{\textbf{\textit{p}, \textit{\neg p}}\} \rangle$ & $\langle \{\textit{p}\} \rangle$ \\ 
    \textit{Table*} & & \\ 
    \textit{DC\textsubscript{A}} & $\{\textbf{\textit{p}}\}$ & $\{\textit{p}\}$ & $\{\textbf{\textit{p}}\}$ \cellcolor{gray!15}  \\ 
    \textit{DC\textsubscript{A}*} &  & $\{\{\textbf{\textit{p}}\}\}$  & \\ 
    \textit{DC\textsubscript{B}} & & & \cellcolor{gray!15}\\ 
    \textit{DC\textsubscript{B}*} & & & \\ 
    \textit{CG} & \textit{s}\textsubscript{1} & \textit{s}\textsubscript{1} & \textit{s}\textsubscript{1} \\ 
    \textit{CG*} & \{\textit{s}\textsubscript{1} $\cup$ \{\textbf{\textit{p}}\}\} & \{\textit{s}\textsubscript{1} $\cup$ \{\textit{p}\}, \textit{s}\textsubscript{1} $\cup$ \{\textbf{\neg \textit{p}}\}\} & \{\textit{s}\textsubscript{1} $\cup$ \{\textbf{\textit{p}}\}\} \\
    \lspbottomrule
\end{tabularx}
\caption{Formal analysis of (\ref{ex:answer:Hong}) with a Contradictory \isi{IRD}}
    \label{tab:contraird:Hong}
\end{table}
\z

In (\ref{ex:answer:Hong}), speaker B does not update any own \isi{commitment} with the Contradictory \isi{IRD} at \textit{t}\textsubscript{2}. Due to the lack of \isi{commitment} from one participant, the issue cannot be resolved even after the other participant utters \textit{Yes} at \textit{t}\textsubscript{3}. The consequence will lead participants to the next stage of the discourse, pursuing to `agree to disagree' \citep{farkas2010reacting} or putting the issue at the bottom of the \textit{Table} to not be discussed unless one of the speakers changes their own commitments. Whatever treatment we may assume, it prevents the issue from expanding the \textit{CG}.

\section{Conclusion}
\label{sec:conc:Hong}

This paper presented a diverse paradigm of \isi{RDs}, focusing on their \isi{speech acts} and how their meanings are acquired throughout the discourse. Given that \isi{RDs} serve both informative and inquisitive functions, they exhibit a multi-functional behaviour. From this observation, I proposed an analysis that delineates the role of semantics and pragmatics by explaining their interface in generating the \isi{discourse effects} observed in different types of \isi{RDs}. Most importantly, rising intonation
affects both \isi{semantic content} and \isi{discourse components}. Depending on its steepness, rising intonation modifies the \isi{semantic content} of the clause. Then, through the interaction with the context, it subsequently projects \isi{discourse components}. By establishing this framework, the proposed analysis provides a predictable model for semantic and pragmatic properties of \isi{RDs}.


\section*{Acknowledgements}

This work is based on my Seoul National University thesis \citep{hong2023semantics} supervised by Jungmee Lee, to whom I am extremely grateful for her wonderful support at various stages of this project. I also thank Sunwoo Jeong and Yong-Yae Park for their insightful comments. Preliminary versions of this paper were presented at CSSP 2023 and LSA 2024. I thank the respective audiences for very useful feedback and discussion. Thanks finally to the editors and reviewers of this volume. All remaining errors are on my own.

\sloppy
\printbibliography[heading=subbibliography,notkeyword=this]
\end{document}

%%% Local Variables:
%%% mode: xelatex
%%% TeX-master: t
%%% End:
