\documentclass[output=paper,colorlinks,citecolor=brown,chinesefont]{langscibook}
\ChapterDOI{10.5281/zenodo.15450430}
\author{Long Chen\orcid{0000-0003-2294-8321}\affiliation{Heinrich Heine University Düsseldorf} and       Laura Kallmeyer\orcid{0000-0001-9691-5990}\affiliation{Heinrich Heine University Düsseldorf} and       Rainer Osswald\orcid{0009-0006-5872-5241}\affiliation{Heinrich Heine University Düsseldorf}}
\title{Primary vs.\ secondary meaning facets of polysemous nouns}
\abstract{Inherently polysemous nouns, or dot-type nouns, have two or more meaning facets that are systematically related to each other and can be addressed simultaneously in copredication constructions. In this paper, we investigate the internal semantic structure of inherently polysemous nouns and argue for a distinction between primary and secondary meaning facets. Evidence is provided by showing that copredication constructions behave asymmetrically for certain polysemous nouns in English and Chinese.   We develop two proposals for modelling restrictions on copredication:  First,     an approach that distinguishes between (i) predicates that pick a specific facet, thereby blocking other facets for subsequent predications,  and (ii) predicates that address a specific facet without blocking other facets. Second, as an alternative, we propose to model the distinction between primary and secondary facets within the representations of dot-type nouns via default constraints, where secondary facets are defeasible and can be blocked in certain predication constructions while primary facets cannot be blocked.
}

%
\IfFileExists{../localcommands.tex}{
   \addbibresource{../localbibliography.bib}
   % add all extra packages you need to load to this file

\usepackage{tabularx,multicol}
\usepackage{url}
\urlstyle{same}

\usepackage{listings}
\lstset{basicstyle=\ttfamily,tabsize=2,breaklines=true}

\usepackage{langsci-basic}
\usepackage{langsci-optional}
\usepackage{langsci-lgr}
\usepackage{langsci-osl}
% \usepackage{./langsci/styles/langsci-lgr}
% \usepackage{./langsci/styles/langsci-osl}
% \usepackage{langsci-gb4e}

\usepackage{tikz}
\usetikzlibrary{patterns,calc}
\pgfdeclarepatternformonly{south east lines}{\pgfqpoint{-0pt}{-0pt}}{\pgfqpoint{3pt}{3pt}}{\pgfqpoint{3pt}{3pt}}{
    \pgfsetlinewidth{0.6pt}
    \pgfpathmoveto{\pgfqpoint{0pt}{3pt}}
    \pgfpathlineto{\pgfqpoint{3pt}{0pt}}
    \pgfpathmoveto{\pgfqpoint{.2pt}{-.2pt}}
    \pgfpathlineto{\pgfqpoint{-.2pt}{.2pt}}
    \pgfpathmoveto{\pgfqpoint{3.2pt}{2.8pt}}
    \pgfpathlineto{\pgfqpoint{2.8pt}{3.2pt}}
    \pgfusepath{stroke}}
    
\usepackage{stmaryrd}
\usepackage{wasysym}
\usepackage{multirow}
\usepackage{caption}
\usepackage{subcaption}
\usepackage{mathrsfs}
\usepackage{qtree}

\usepackage{linguex}


   %pminos do not split footnotes
% \interfootnotelinepenalty=10000 %Footnote in Laporte chapters has to be split SN


%\DeclareIndexNameFormat{default}{%
%\nameparts{#1}%
%\usebibmacro{index:name}%
%{\index[names]}%
%{\namepartfamily}%
%{\namepartgiveni}%
% {}% L1
% {}% L2
%{\namepartprefix}% generates spurious space L3
%{\namepartsuffix}% generates spurious space L4
%}

%  {\DeclareIndexNameFormat{default}{%
%     \usebibmacro{index:name}{\index[names]}{#1}{#3}{#5}{#7}}}

%\DeclareIndexNameFormat{default}{%
%  \usebibmacro{index:name}{\sindex[nom]}{#1}{#3}{#5}{#7}}

%\DeclareIndexNameFormat{default}{%
%  \usebibmacro{index:name}{\sindex[person]}{#1}{#3}{#5}{#7}}
%\DeclareIndexNameFormat{default}{%
%\nameparts{#1} \usebibmacro{index:name}{\sindex[person]]}{\namepartfamily}{‌​\namepartgiven}{\nam‌​epartprefix}{\namepa‌​rtsuffix}}

%\newcommand{\smiley}{:)}

%\renewbibmacro*{index:name}[5]{%
%\usebibmacro{index:entry}{#1}%
%{\iffieldundef{usera}{}{\thefield{usera}\actualoperator}\mkbibindexname{#2}{#3}{#4}{#5}}}

% \newcommand{\noop}[1]{}

%remove for final
%\overfullrule=1mm

\newcommand{\tobi}[2]}}
\renewcommand{\S}[1]{\tobi{#1}{\textsc{*}}}

% this volume references
% puts: [this volume]
% already defined: \citetv
%\newcommand{\citepv}[1]{(\citeauthor{#1} \citeyear*{#1} [this volume])}
\newcommand{\citealtv}[1]{\citeauthor{#1} \citeyear*{#1} [this volume]}

%parentheses around example number
\newcommand{\pref}[1]{(\ref{#1})}

% in-text examples

\newcommand{\lnex}[1]{\textit{#1}} %target lang word
\newcommand{\lnlit}[1]{(lit.: `#1')} %literal reading
\newcommand{\lnlat}[1]{(#1)} % latinization
\newcommand{\lntrans}[1]{`#1'} %translation
\newcommand{\lnexl}[2]%
{\lnex{#1}{} \lnlat{#2}} % ex with latinization
\newcommand{\lnexlat}[3]{\lnex{#1}{} \lnlat{#2}{} \lntrans{#3}} % ex with latinization and tranl.

%ch01
\newcommand{\co}[1]{\mbox{\textbf{#1}}}

%ch09

\newcommand{\cyrbulg}[1]{\begin{otherlanguage*}{bulgarian}#1\end{otherlanguage*}}


%ch10
\newcommand{\nlp}{{\small NLP}}
\newcommand{\mwe}{{\small MWE}}
\newcommand{\rae}{{\small RAE}}
\newcommand{\lvc}{{\small LVC}}
\newcommand{\pos}{{\small P}o{\small S}}
%\newcommand{\todo}[1]{ \textcolor{red}{#1} }

%\renewcommand{\labelenumi}{\theenumi}
%\ainamefmt{{vv}{ll}{, ff}{, jj}} % fullname

\newcommand{\biberror}[1]{{\color{red}#1}}

\newcommand{\osenovaitem}{--~}
   %% hyphenation points for line breaks
%% Normally, automatic hyphenation in LaTeX is very good
%% If a word is mis-hyphenated, add it to this file
%%
%% add information to TeX file before \begin{document} with:
%% %% hyphenation points for line breaks
%% Normally, automatic hyphenation in LaTeX is very good
%% If a word is mis-hyphenated, add it to this file
%%
%% add information to TeX file before \begin{document} with:
%% %% hyphenation points for line breaks
%% Normally, automatic hyphenation in LaTeX is very good
%% If a word is mis-hyphenated, add it to this file
%%
%% add information to TeX file before \begin{document} with:
%% \include{localhyphenation}
\hyphenation{
    Beck-man
    Ngu-yen
    back-chan-nel
    back-chan-nels
    mo-not-o-nous
    ste-reo-typ-i-cal
}

\hyphenation{
    Beck-man
    Ngu-yen
    back-chan-nel
    back-chan-nels
    mo-not-o-nous
    ste-reo-typ-i-cal
}

\hyphenation{
    Beck-man
    Ngu-yen
    back-chan-nel
    back-chan-nels
    mo-not-o-nous
    ste-reo-typ-i-cal
}

   \boolfalse{bookcompile}
   \togglepaper[2]%%chapternumber
}{}

\begin{document}
\maketitle

\section{Introduction}

Inherently polysemous nouns, or \isi{dot-type} nouns, have different meaning facets that can be targeted by different predications over the same occurrence of the noun in a sentence or phrase.
This phenomenon is known as \emph{copredication}.
While there is already a large body of work on the types of meaning facets and their semantic relationships, the internal semantic structure of polysemous nouns and how they license \isi{copredication} is still a topic of ongoing research.
For example, \citet{Pustejovsky:1995} and \citet{asher2011lexical} model the meaning facets of complex type nouns as facets of equal status; \citet{chen2022frame} assume that all the facets are on the same level and related to each other by attributes in the frames of the nouns; \citet{babonnaud2016polysemy} suggest that one of the facets stands for the noun and the other facet is its attribute; \citet{Retore:14} proposes ``flexible'' and ``rigid'' as a feature of the facet to represent the compatibility of a semantic type with other types, which implies that different meaning facets might not be equally prominent; \citet{ortega2019polysemy} introduce the concept of ``activation package'' to indicate the close semantic relationship between the facets that can be jointly addressed by \isi{copredication}; \citet{murphy2021predicate} suggests a ``complexity hierarchy'' of the semantic types which leads to a \isi{copredication} hierarchy.
%\citet{ortega2019polysemy} and \citet{murphy2021predicate} used ``activation package'' or ``coherence'' to indicate the close semantic relationship between facets that can copredicate, but without giving a formal account. %not in a formalized way and without distinguishing primary and secondary facets.

Following \citet{chen2022frame}, we assume that predication over a noun involves either targeting %can only happen when the predicate targets
 one or several  meaning facets of the noun or performing coercion. For example, \emph{book} has an \isi{object} facet and an information facet. In \emph{read the thick book}, \emph{read} targets both facets while \emph{thick} targets the \isi{object} facet.
On the other hand, since \emph{book} cannot be predicated by typical event-targeting verbs such as \emph{perform} or \emph{conduct}, it does not have an event facet.
Consequently, \emph{enjoy the book} is a case of coercion, because \emph{enjoy} targets an event.

The meaning facets of an \isi{inherently polysemous noun} may differ with respect to accessibility and persistence.
This is reflected, among others, by asymmetries in \isi{copredication} constructions.
Such asymmetries have been noticed in previous research, especially in cases where a noun has multiple meaning facets. \citet[63]{asher2011lexical} observed that the felicity of \isi{copredication} related to the polysemous noun \emph{city} in (\ref{ex:asher1:Chen}) is higher than in (\ref{ex:asher2:Chen}).
\ea \label{ex:asher2011:Chen}
\ea [ ]{The city has 500,000 inhabitants and outlawed smoking in bars last year.} \label{ex:asher1:Chen}
\ex [?]{The city outlawed smoking in bars last
year and has 500,000 inhabitants.} \label{ex:asher2:Chen}
\z\z

Similarly, \citet{Retore:14} noticed that for the polysemous noun \emph{Liverpool}, the senses of \type{people} and \type{place} can be subject to \isi{copredication}, but the sense of \type{football team} cannot be elicited together with any of the others. \citet{jezek2011nominals} conducted case studies on deverbal nouns and determined constraints on the \isi{copredication} with action nominals. \citet{chatzikyriakidis2015individuation} investigated the case of \emph{newspaper} and discovered that the \type{organization} reading can occur with the \type{physical object} or \type{information} readings in copredications only under certain conditions. \citet{ortega2019polysemy} investigated the \isi{copredication} of \emph{school} and proposed two ``activation packages'' of \emph{school}, one corresponding to the \type{institution} sense and the other corresponding to \type{object-information} sense, and only meaning facets in the same activation package can occur together in copredications.
\citet{sutton2022restrictions} observed that for the noun \emph{statement}, the \type{physical entity} sense and the \type{eventuality} sense are incompatible for \isi{copredication}, but that both can occur in copredications with the \type{informational content} sense. \citet{murphy2021predicate} investigated the order of \isi{copredication} more systematically and claimed that complexity and coherence are the decisive factors for the order of \isi{copredication}. \citet{michel2024cognitive} further suggest that context is a more fundamental factor than complexity, and explain the order of \isi{copredication} with the notion of ``expectation''.
%(See also \citealp{Retore:14} for observations in a similar direction.) %, who observed that in some cases, predicating over one facet excludes subsequent predications over other facets.)
% For example, the deverbal noun `annotation' has an event facet, an \isi{object} facet and an info facet. In the \isi{copredication} pattern V(erb)+Mod(ifier)+N(oun), if the \isi{modifier} is `correct', which targets the info facet, the verb can be `delete' or `burn', which targets the \isi{object} facet, but it cannot be `attend' or `organize', which target the event facet. In the \isi{copredication} pattern Mod+Mod+N, `correct manual annotation' is more acceptable than `manual correct annotation'.
%These asymmetries indicate that predicating over the \isi{object} or info facet excludes  further predication over the event facet. In other words, the %\isi{object} facet is closer to the info facet than the event facet, and the
%event facet is less prominent %than the other facets
% and might therefore be considered a \isi{secondary facet}.
%\lk{Although this is a nice example, it is rather complex compared to what we do later. It looks like we have facet picking \isi{modifiers} (`correct') that pick a \isi{primary facet}, exclude one \isi{secondary facet} but do not exclude the third facet. We don't talk about that in the subsequent text; all examples we deal with in the rest of the paper, have 2 facets. Shall we keep the example nevertheless?}\cl{My plan was to put a more complex example in the introduction to show that it's interesting and difficult, and also this is quite an uncontroversial example. But if it has to be dealt with in the later part maybe we can replace it with lively vegan dinner.}

The above studies have shown that for nouns with multiple meaning facets, some facets are easier to occur in a \isi{copredication} construction than the others, and it is more natural to predicate over some facets than the others in a \isi{copredication} construction. However, many of them do not distinguish \isi{copredication} from coercion, which are different in the mechanisms of composition. For example, it is debatable whether the reading of \type{managing entity} of \emph{city} or \emph{school} is an inherent meaning or a coerced meaning. We will therefore focus on more unquestionable cases where the nouns have two distinct meaning facets. According to our observation, these asymmetries of \isi{copredication} also occur with nouns with only two meaning facets, and the choice of the predicate also affects the felicity of \isi{copredication}.
For example, in (\ref{ex:introdinner:Chen}), it is easier to target the \isi{object} facet of \emph{lunch} first and target the event facet later (as in \ref{ex:orgdeldin:Chen}) than the opposite order (as in \ref{ex:orderslowdinner:Chen}), and if the event facet is targeted by \emph{quick} rather than \emph{slow}, \isi{copredication} can still happen.\footnote{If not mentioned otherwise, the English examples in this paper are from the English Web 2021 corpus \citep[enTenTen21; see][]{jakubivcek2013tenten} provided by SketchEngine (\url{www.sketchengine.eu}).}

\ea \label{ex:introdinner:Chen}
\ea [ ]{Kerry Weins and Joanne Weins organized a delicious lunch at Bette Axani's home in Union Bay.} \label{ex:orgdeldin:Chen}
\ex [ ]{In one instance I am obsessing over how to improve my blog and in the next I am ordering a quick lunch using Foodpanda!} \label{ex:orderquickdinner:Chen}
\ex [\#]{In one instance I am obsessing over how to improve my blog and in the next I am ordering a slow lunch using Foodpanda!} \label{ex:orderslowdinner:Chen}
\z\z

In this paper we focus on polysemous nouns with only two meaning facets and examine whether the two meaning facets are equally prominent or not. As observed in the previous works (e.g. in \citealt{chatzikyriakidis2015individuation} and \citealt{jezek2011nominals}), the syntactic construction may also affect \isi{copredication} \isi{acceptability}, so we only consider the following two \isi{copredication} patterns: V+Mod+N and Mod1+Mod2+N.\footnote{Constructions of the form Mod1+and+Mod2+N (\emph{delicious and romantic dinner}) or Mod1+,+Mod2+N (\emph{delicious, romantic dinner}) are not exactly the same as the construction we consider, since the two \isi{modifiers} in these constructions are coordinated and the \isi{acceptability} of \isi{copredication} is higher in this case.}
Based on data from English and Mandarin Chinese, we argue that polysemous nouns have primary facets and secondary facets. Primary facets are more prominent and stable in the noun's meaning, which is to say, they cannot be blocked by a predication over another facet.
In particular, they can be targeted in any type of \isi{copredication} pattern.
Secondary facets, by contrast, might be unavailable in certain predication patterns, such as the \isi{object} facet of \emph{lunch} (see above). Concretely, a predication over a different facet might exclude a subsequent predication over a \isi{secondary facet}.
 
We base our analyses on observed possibilities and restrictions for \isi{copredication} constructions. Concretely, we %The theory is based on the observation of the possibility of \isi{copredication}. We
  examine %This paper examines
  three common dot-types, % and
   discuss restrictions on copredications for these types, in particular in relation to the different facets, %their primary and secondary facets, 
    and sketch % in those nouns. We also briefly introduce
    possible ways to model \isi{dot-type} nouns and predications over them.
We will present one analysis that distinguishes between predicates that pick only one of the facets, thereby excluding further predications over the other facets, and predicates that address one of the  facets while keeping the other facets for further predication.
We call the first type of predication \emph{facet-picking} and the second type of predication \emph{facet-addressing}. As an alternative, we sketch a second approach that explicitly encodes primary and secondary facets in the representations of dot types and that models composition during predication in such a way that certain predications exclude further predications over different facets. This second approach assumes that primary facets can never be blocked for further predication. As a formalization for our semantic representations, we choose semantic frames. In the first proposal, \isi{facet-picking} predications roughly move the access point of the noun's frame to the picked facet. In the second proposal, we use \isi{default logic} and model secondary facets as default attributes that can be removed due to conflicting frame constraints. %\lk{I elaborated a little here and mentioned both options (before that, the text said only that we model primary and secondary facets, which fits with the default approach but less with the top-bottom approach).}%the formalization of the \isi{dot-type} nouns with primary and secondary facets in \isi{frame semantics}.
%Both lexical and compositional aspects provide evidence suggesting that, in Mandarin Chinese, in event·food, the event facet is the only \isi{primary facet}; in \isi{object}·info, usually both facets are primary facets; in event·info, usually both facets are secondary facets. This paper also demonstrates how these different \isi{dot-type} nouns and \isi{modifiers} can be modelled in \isi{frame semantics}.
\section{Primary facets, secondary facets and restrictions on copredication}
\label{sec:prim:Chen}

Three common facets of \isi{dot-type} nouns in English and Chinese (as well as other languages) are \type{event}, \type{info(rmation)}and \type{obj(ect)}.
In this section, we investigate the dot-types \type{event}$\bullet$\type{food}, \type{obj}$\bullet$\type{info} and \type{event}$\bullet$\type{info} regarding the restrictions on \isi{copredication}. We provide empirical evidence for \isi{facet-picking} and \isi{facet-addressing} predications over the \isi{dot-type} nouns, which yields hypotheses for the respective patterns of primary and secondary facets.

\subsection{\type{Event}\dott\type{food}}

%\lk{I am wondering whether the asymmetries have to do with the role the obj or info facet plays within the event. For event-food, the food is consumed, i.e., upon finishing the event, it does not exist any longer. For creation events, this is different (speech, annotation). I would really like to pursue this idea as well, i.e., explain asymmetries via a detailed modelling of the respective event structures. What do you think about \ref{ex:more-dinner:Chen}?} 
%\cl{The difficulty to explain the asymmetry with the internal semantics of the nouns is that the two ``speech'' in Chinese (演讲 and 讲话) are different. So in this paper I am explaining it from the perspective of the lexical system, which is the external factor.}
%\cl{I think \ref{ex:more-dinner:Chen} is a felicitous sentence, but in this case I will link ``successful'' to ``cook'' instead of ``dinner''. The event-facet of ``dinner'' is the eating event, but in ``cook a successful dinner'', the success is actually the cooking, not the eating. Still, when the \isi{secondary facet} meets a facet-keeping \isi{modifier} (I should elaborate \isi{facet-picking}/keeping mods later in this paragraph) the \isi{secondary facet} might still stay, such as ``cook yesterday's dinner''.}
%
%\ea \label{ex:more-dinner:Chen}
%\ea My cooking has improved a lot, the guests might stop complaining: Yesterday, I managed to cook a successful dinner.
%\z\z

Nouns for meals in English and Chinese, such as {\cn 午饭} (w\v{u}c\={a}n) `lunch', {\cn 自助餐} (z\`{i}zh\`{u}c\={a}n) `buffet', {\cn 晚饭} (w\v{a}nc\={a}n) `dinner', and {\cn 大餐} (d\`{a}c\={a}n) `feast' have an event and an \isi{object} facet. %An asymmetry in the \isi{copredication} of these two facets can be observed. 
In the \isi{copredication} pattern V+Mod+N, %it is possible that
 the Mod(ifier) can target the \isi{object} facet and the V(erb) the event facet as in (\ref{ex:deldin:Chen}) and (\ref{ex:vegdin:Chen}), but the other direction of \isi{copredication} is only possible for some \isi{modifiers}: in (\ref{ex:livdin:Chen}), %as in (\ref{ex:livdin:Chen}) is not always available, though there might be exceptions to this, see (\ref{ex:earlydin:Chen}). 
 `lively' seems to exclude further predication over the \isi{object} facet, while in (\ref{ex:earlydin:Chen}) `early' is \isi{facet-addressing}.
\ea \label{ex:dinner:Chen}
\ea [ ]{On Saturday night, the INTA had organized a delicious dinner in the CityNorth Hotel.} \label{ex:deldin:Chen}
\ex []{how to organize a vegan dinner\footnote{\href{https://otttimes.ca/life/food/how-to-organize-a-vegan-dinner/}{https://otttimes.ca/life/food/how-to-organize-a-vegan-dinner/}, retrieved 31 May 2024.}} \label{ex:vegdin:Chen}
\ex [\#]{order the lively dinner} \label{ex:livdin:Chen}
\ex []{benefits of eating an early dinner\footnote{\href{https://pharmeasy.in/blog/eat-light-and-feel-light-benefits-of-eating-an-early-dinner/}{https://pharmeasy.in/blog/eat-light-and-feel-light-benefits-of-eating-an-early-dinner/}, retrieved 31 May 2024.}} \label{ex:earlydin:Chen}
\z\z
%We think that this is related to the different meanings of the two \isi{modifiers}. 

The Chinese examples in (\ref{ex:wancan:Chen}) reveal a similar asymmetry in \isi{copredication} possibilities.\footnote{The Chinese examples in this paper are all from the CCL corpus (corpus of the Center for Chinese Linguistics of PKU, see \citealp{zhan2019北京大学}).} %the possibility of \isi{copredication}.
%From the compositional aspect, there is an asymmetry in the \isi{copredication} of event·food dot type. Take \emph{dinner} as a representative for this \isi{dot-type}. The common \isi{modifiers} for \emph{dinner} include \emph{lively}, \emph{delicious}, \emph{abundant}, etc.
\ea \label{ex:wancan:Chen}
\ea [ ]{
\glll {\cn 组织} {\cn 美味}/{\cn 丰盛} {\cn 的} {\cn 晚餐}\\
z\v{u}zh\={i} m\v{e}iw\`{e}i/f\={e}ngsh\`{e}ng de w\v{a}nc\={a}n\\
organize delicious/abundant \textsc{modm}\footnotemark{} dinner\\
\glt `organize a delicious/abundant dinner'} \label{ex:delwancan:Chen}
\ex [\#]{
\glll {\cn 带走} {\cn 热闹}/{\cn 两个小时} {\cn 的} {\cn 晚餐}\\
d\`{a}iz\v{o}u r\`{e}n\`{a}o/li\v{a}ngg\`{e}xi\v{a}osh\'{i} de w\v{a}nc\={a}n\\
take.away lively/two.hours \textsc{modm} dinner\\
\glt `take away the lively/two-hour dinner'} \label{ex:livwancan:Chen}
\z\z

\footnotetext{`\textsc{modm}' stands for \isi{modifier} markers.}
%
%If the noun is modified by the \isi{object} \isi{modifier}, as in \ref{ex:orgdin:Chen}, the phrase can be further predicated by verbs that target the event-facet. On the contrary, if the noun is modified by the event \isi{modifier}, as in \ref{ex:takdin:Chen}, the phrase cannot be predicated by verbs that target the object-facet.
%
%
%
The \isi{copredication} pattern Mod1+Mod2+N in (\ref{ex:conj-dinner:Chen}) also shows the asymmetry of the availability of copredicating the two facets.
There is a rich literature on the preferred order of noun-modifying adjectives and the underlying semantic factors.\footnote{See \citet{Scontras:2023} for a more comprehensive discussion of this topic.}
For example, type-modifying adjectives tend to be closer to nouns than token-modifying adjectives \citep{mcnally2004relational}; individual-level adjectives tend to be closer to nouns than stage-level adjectives \citep{larson1998events}; subsective adjectives tend to be closer to intersective adjectives \citep{morzycki2016modification}. However, the following data imply that these theories are not always able to explain the adjective orders in the case of \isi{copredication}.

%In (\ref{ex:relax-salad-lunch:Chen}), the \isi{object} facet is targeted first and the event facet is targeted second.
%In this paper the order of targeting is based on syntactic structure rather than linear sequence. 
%Linearly the adjective `romantic' comes first, but structurally, `romantic' is modifying `four-course dinner', so in this case `four-course' targets the \isi{object} facet of `dinner' first, and `romantic' targets the event facet later. Contrary to (\ref{ex:relax-salad-lunch:Chen}), in (\ref{ex:salad-relax-lunch:Chen}) where the order of the \isi{modifier} is reversed, when the event facet is targeted first, the \isi{object} facet cannot be targeted again.

\hide{
\ea \label{ex:conj-dinner:Chen} 
\ea [ ]{Couples can indulge in a romantic four-course dinner in the restaurant located 100 metres up the Euromast.} \label{ex:relax-salad-lunch:Chen}
\ex [?]{Couples can indulge in a four-course romantic dinner in the restaurant located 100 metres up the Euromast.} \label{ex:salad-relax-lunch:Chen}
\ex [ ]{The Henderson County Wellness Committee hosted an offsite healthy lunch and Spring into Motion registration opportunity for employees.} \label{ex:off-healthy-lunch:Chen}
\ex [?]{The Henderson County Wellness Committee hosted a healthy offsite lunch and Spring into Motion registration opportunity for employees.} \label{ex:healthy-off-lunch:Chen}
\ex [ ]{Whether using the tool to whip up her own her\footnote{error in corpus} veggie-heavy vegan chili recipe, or an \textit{impromptu} \textit{healthy} dinner for her family, McDonald knows that slow cookers are a no-fuss way to prepare meals even when there's no meat in your recipe.} \label{ex:imp-healthy-dinner:Chen}
\ex [?]{Whether using the tool to whip up her own her veggie-heavy vegan chili recipe, or a \textit{healthy} \textit{impromptu} dinner for her family, McDonald knows that slow cookers are a no-fuss way to prepare meals even when there's no meat in your recipe.} \label{ex:healthy-imp-dinner:Chen}
\ex [ ]{I made this shrimp and pasta salad a couple of weeks ago as a planned cold lunch} \label{ex:plan-cold-lunch:Chen}
\ex [\#]{I made this shrimp and pasta salad a couple of weeks ago as a cold planned lunch} \label{ex:cold-plan-lunch:Chen}
\ex [ ]{I had a quick cold dinner of leftover pizza, much enhanced by a nice glass of wine} \label{ex:quick-cold-dinner:Chen}
\ex [\#]{I had a cold quick dinner of leftover pizza, much enhanced by a nice glass of wine} \label{ex:cold-quick-dinner:Chen}
\ex [ ]{Labrosse, our landlord, had a late hot dinner waiting for us in the smaller room off the large dining-room} \label{ex:late-hot-dinner:Chen}
\ex [\#]{Labrosse, our landlord, had a hot late dinner waiting for us in the smaller room off the large dining-room} \label{ex:hot-late-dinner:Chen}
\z\z}

\ea \label{ex:conj-dinner:Chen}
\ea [ ]{Whether using the tool to whip up her own [\dots] %her\footnote{error in corpus}
veggie-heavy vegan chili recipe, or an impromptu healthy dinner for her family, McDonald knows that slow cookers are a no-fuss way to prepare meals even when there's no meat in your recipe.} \label{ex:imp-healthy-dinner:Chen}
\ex [?]{Whether using the tool to whip up her own her veggie-heavy vegan chili recipe, or a healthy impromptu dinner for her family, McDonald knows that slow cookers are a no-fuss way to prepare meals even when there's no meat in your recipe.} \label{ex:healthy-imp-dinner:Chen}
\ex [ ]{I made this shrimp and pasta salad a couple of weeks ago as a \textit{planned} \textit{cold} lunch.} \label{ex:plan-cold-lunch:Chen}
\ex [\#]{I made this shrimp and pasta salad a couple of weeks ago as a \textit{cold} \textit{planned} lunch.} \label{ex:cold-plan-lunch:Chen}
\ex [ ]{Labrosse, our landlord, had a late hot dinner waiting for us in the smaller room off the large dining-room.} \label{ex:late-hot-dinner:Chen}
\ex [\#]{Labrosse, our landlord, had a hot late dinner waiting for us in the smaller room off the large dining-room.} \label{ex:hot-late-dinner:Chen}
\z\z

In the examples in (\ref{ex:conj-dinner:Chen}), \emph{healthy}, \emph{cold} and \emph{hot} targets the \isi{object} facet and \emph{impromptu}, \emph{planned} and \emph{late} targets the event facet. The \isi{object} facets are always targeted first in these examples.\footnote{In this paper the order of targeting is based on syntactic structure rather than linear sequence. Therefore although linearly \emph{impromptu} comes before \emph{healthy}, structurally \emph{impromptu} is modifying \emph{healthy dinner}, so \emph{dinner} is targeted by \emph{healthy} first, and then by \emph{impromptu}.} The aforementioned theories might not be able to explain this preferred order. For example, \emph{healthy} is an intersective adjective but it is closer to the noun; \emph{hot} and \emph{cold} are stage-level adjectives and are more likely to be modifying the token based on the context, but they are also closer to the noun. Therefore, we propose that the primary and \isi{secondary facet} is also an important factor in deciding the adjective orders.
%\cl{How should I say that although this primary and \isi{secondary facet} theory is necessary, it also cannot explain everything?}

%\cl{Found some puzzling data: it is always ``delicious festive lunch'' instead of ``festive delicious lunch'', but also always ``festive 2-course lunch'' instead of ``2-course festive lunch''. ``delicious romantic dinner'' and ``romantic delicious dinner'' both exist in Google.}

%\footnote{
\hide{There is a rich literature on the preferred order of noun-modifying adjectives and the underlying semantic factors. %(cf., e.g., \citealt{martin1969semantic}, \citealt{futrell-etal-2020}).
For example, \citet{mcnally2004relational} mentioned that adjectives modifying types tend to be closer to nouns, \citet{morzycki2016modification} observed that subsective adjectives are closer to the noun than intersective adjectives, which might be able to explain the order in (\ref{ex:relax-salad-lunch:Chen}). However, we find these distinctions of adjectives rather unclear, and they cannot explain all the adjective order problems in \isi{copredication}. In (\ref{ex:off-healthy-lunch:Chen})-(\ref{ex:healthy-imp-dinner:Chen}), `healthy' seems to be modifying a specific `lunch' instead of a kind of lunch, but still the object-targeting adjective `healthy' needs to be closer to the noun. Also `healthy', `offsite' and `impromptu' are all intersective adjectives\cl{are they?} but they still have a preferred order. \citet{bolinger2013degree} proposed that the distinction between stage-level and individual-level adjectives affects the adjective order. The individual-level adjectives should be closer to nouns. However, in (\ref{ex:plan-cold-lunch:Chen}), (\ref{ex:quick-cold-dinner:Chen}) and (\ref{ex:late-hot-dinner:Chen}), `cold' and `hot' are stage-level adjectives while the other adjectives that target the event facet are individual-level adjectives, but the stage-level adjectives are closer to the noun. These data show that the previous classification on adjectives cannot explain all the adjective ordering problems, and the distinction of primary and secondary facets are necessary for these examples.}
%A thorough comparison has to be postponed to future work but we would expect that the given explanation based on the assumption of primary and secondary facets is at least compatible with existing insights.}
%We are aware that the \isi{modifier} order is complicated and can be decided by multiple factors\footnote{e.g. \citet{vendler1968adjectives} classified more than 20 classes of adjectives by syntactic transformation and concluded their order in modification;  on the other hand \citet{martin1969semantic} based on indicated semantic aspects determine the order of adjectival \isi{modifiers} psycholinguistic experiments},
% \cl{cite sth} 
%but here we are focusing on primary vs.\ secondary facets.
%It might be argued that `four-course' modifies the `lunch' type instead of token, and as suggested by \citet{mcnally2004relational}, type modifying adjectives tend to be closer to the noun, but more examples indicate that this theory cannot explain all the adjective order problem in \isi{copredication}. In (\ref{ex:off-healthy-lunch:Chen})-(\ref{ex:healthy-imp-dinner:Chen}), `healthy' is modifying a specific `lunch' and `healthy lunch' is not a kind of lunch, but still the object-targeting adjective `healthy' needs to be closer to the noun.

The asymmetry also exists in coercion. Usually, in Chinese the \isi{classifier} {\cn 顿} (d\`{u}n) is used for events and {\cn 份} (f\`{e}n) is used for objects.\footnote{In Chinese, different \isi{classifiers} apply to different types of nouns, in general. For example, {\cn 顿} (d\`{u}n), {\cn 次} (c\`{i}), {\cn 场} (ch\v{a}ng) can only modify event nouns but not physical \isi{object} nouns, while {\cn 份}(f\`{e}n), {\cn 只} (zh\={i}) can only modify physical \isi{object} nouns but not event nouns. Here, \isi{classifiers} that modify events are glossed by \textsc{clf}$_e$, and \isi{classifiers} that modify objects by \textsc{clf}$_o$.} However, as shown in (\ref{ex:coerciondun:Chen}), {\cn 顿} (d\`{u}n) can sometimes also be used for food nouns like {\cn 饺子}(ji\v{a}ozi) `dumpling' while {\cn 份} (f\`{e}n) can never be used for event nouns like {\cn 晚宴} (w\v{a}ny\`{a}n) `banquet'. This also indicates that in the context with the event and the food, usually the event is more prominent.

\ea \label{ex:coerciondun:Chen}
\ea \label{ex:yidundump:Chen}
\glll {\cn 一} {\cn 顿} {\cn 饺子}/{\cn 晚宴}\\
y\={i} d\`{u}n ji\v{a}ozi/w\v{a}ny\`{a}n\\
one \textsc{clf}$_e$ dumpling/banquet\\
\glt `a meal of dumplings/a banquet'
\ex \label{ex:yifenbanq:Chen}
\glll {\cn 一} {\cn 份} {\cn 饺子}/\#{\cn 晚宴}\\
y\={i} f\`{e}n ji\v{a}ozi/\#w\v{a}ny\`{a}n\\
one \textsc{clf}$_o$ dumpling/\#banquet\\
\glt `a portion of dumplings/a banquet'
\z\z

One reason for this asymmetry is related to the semantic relation between the meaning facets. In the `dinner' concept, there is an event during which food is % in the evening; during the event there are food served and
 consumed. An interpretation in the other direction is less natural, i.e., %It is less natural to interpret in the other direction, i.e.
  `dinner' is food which is consumed in an eating event. % happening in the evening.
  So conceptually it is more reasonable that the food is embedded under the event, which means the event is more prominent and is the \isi{primary facet}.%\cl{I think although the meaning/concept can't explain everything (like the Chinese speech), it is still one of the factor and can at least explain event$\dott$food; also it is related to the frame of dinner (food is an attribute of event but not the other direction)} \lk{In other words, you think that the fact that there is a functional attribute path from the event to the food but not in the other direction, plays a role? Might be. But why are the two facets of book then not asymmetric in the same way?}\cl{That's why I only said it leads to the fact that the event is the \isi{primary facet}, and didn't mention obj-facet is a \isi{secondary facet}. Also it is ``one reason'', there need other reasons to explain for example in book why both are primary.}

Another possible explanation for the asymmetry is that when the food within a meal is being referred to specifically, it is more accurate and informative to mention the name of the food directly, like answering the question \emph{What did you eat last evening}, \emph{I ate a pizza} would be better than \emph{I ate a dinner}. However, the event facet of a meal seldom has its own name. 
Therefore, when referring to the event, the \isi{dot-type} noun, such as \emph{I will attend the dinner} is used. Otherwise less efficient phrasings like \emph{I will attend the eating activity in the evening} is required to convey the same idea.
%\lk{I find this second explanation rather far-fetched.}\cl{It's true. This is more to be consistent with another far-fetched explanation for the difference between quick dinner and slow dinner. Maybe we can leave them out from the paper?(I really didn't come up with better explanations for quick and slow dinner) And what about the explanation of the two Chinese speech (the last paragraph in section 2)? It is more reasonable to me according to my intuition of these Chinese nouns, but it's in principle similar to these two far-fetched explanations and indeed we can't explain that with content meanings.}

As already observed (see \ref{ex:earlydin:Chen}), despite being a \isi{secondary facet}, the \isi{object} facet is not always unavailable when the event facet is targeted first. Further examples are given in (\ref{ex:lunchspeed:Chen}) where  both \emph{quick} and \emph{slow} target the event facet of \emph{lunch}. In (\ref{ex:quicklunch:Chen}), (\ref{ex:packquicklunch:Chen}) and (\ref{ex:servequicklunch:Chen}) \emph{quick lunch} has the \isi{object} facet and can still be predicated by \emph{cook}, \emph{pack} or \emph{serve}, but \emph{slow lunch} does not have the \isi{object} facet and cannot replace \emph{quick lunch} in those sentences.

\ea \label{ex:lunchspeed:Chen}
\ea [ ]{We prepared the poultice and cooked a quick lunch, waiting for the minnow trap to work for fishing bait.} \label{ex:quicklunch:Chen}
\ex [\#]{We prepared the poultice and cooked a slow lunch, waiting for the minnow trap to work for fishing bait.} \label{ex:slowlunch:Chen}
\ex [ ]{So, we packed a quick lunch and headed out to find... } \label{ex:packquicklunch:Chen}
\ex [\#]{So, we packed a slow lunch and headed out to find...} \label{ex:packslowlunch:Chen}
\ex [ ]{Since 2006, Ronnie's Cafe has been serving hot breakfast and quick lunches to ASU West students and Glendale residents alike.} \label{ex:servequicklunch:Chen}
\ex [\#]{Since 2006, Ronnie's Cafe has been serving hot breakfast and slow lunches to ASU West students and Glendale residents alike.}
\z\z

%The phrase `quick lunch' might seem to be a multi-word expression like `fast food', however, unlike `fast food', in `quick lunch' `quick' can be replaced by `fast', and `lunch' can be replaced by `supper', `dinner', `breakfast' or `meal' and they exhibit the same contrast with `slow lunch', while `fast food' and `quick food' means totally different things. Furthermore, as is shown in (\ref{ex:quickxxxlunch:Chen}), there can still be adjectives targeting the \isi{object} facet between `quick' and `lunch'. Therefore, `quick lunch' is different from `slow lunch' only in the availability of the \isi{object} facet.
\hide{
\ea \label{ex:quickxxxlunch:Chen}
\ea [ ]{Always cook more than you need so you can throw together a quick nutritious lunch.} \label{ex:quicknutrilunch:Chen}
\ex [ ]{I still want a quick healthy lunch , but I would rather it not come between two slices of bread unless they are freshly buttered and grilled!} \label{ex:quickhealthylunch:Chen}
\z\z}

Based on the difference in (\ref{ex:lunchspeed:Chen}), we can categorize adjectival \isi{modifiers} into \isi{facet-addressing} \isi{modifiers} and \isi{facet-picking} \isi{modifiers}. Facet-addressing \isi{modifiers} such as \emph{quick} or \emph{early} tend to keep the secondary facets in the phrase after modifying the \isi{primary facet}, while \isi{facet-picking} \isi{modifiers} such as \emph{slow} and \emph{lively} tend to pick out only the \isi{primary facet}. %For example, the above-mentioned subsective adjectives and type-modifying adjectives tend to be facet-keeping.  %\cl{also ``lively''}

The reasons why some adjectives are \isi{facet-addressing} or \isi{facet-picking} are related to the conceptual relationships between the meanings of the adjective and the dot \isi{object}. For example, in a \emph{lively dinner}, the liveliness is usually not only about eating the food, but also about the eaters talking and other activities not involving food, thus the food is more neglected in a \emph{lively dinner}, making the \isi{object} facet unavailable; on the other hand, an \emph{early dinner} means a dinner where the food is eaten earlier than usual, so the food is still an essential part of the meaning of the phrase, thus the \isi{object} facet is preserved. However this may not be able to explain all the \isi{facet-addressing}/picking adjectives, especially for the case of \emph{quick dinner} and \emph{slow dinner}, where the two adjectives are very close in meaning. The exact reasons why certain adjectives tend to be \isi{facet-addressing} while others are \isi{facet-picking} require further research.

%Whether an adjectival \isi{modifier} is \isi{facet-addressing} or \isi{facet-picking} seems to be related to the informativeness of the \isi{modifier} in terms of the noun. For example, arguments of the events, and the possessors of the objects are usually \isi{facet-addressing} because they are usually expected to be mentioned and can sometimes even be inferred from context. Adjectives that describe the non-inherent properties of a noun are less expected in natural language, so they are adding more information to the noun and getting more \isi{focus}. %In the case of `quick' and `slow', perhaps in English being quick is a more natural state for events, so in a `quick lunch' the \isi{focus} is still on `lunch', so both facets are preserved; while being slow is less common, so in a `slow lunch', `slow' gets more \isi{focus}, so that the \isi{object} facet is ignored.

\subsection{\type{Object}\dott\type{info}}
The two facets in \type{object}\dott\type{info} are usually of equal status and both are primary facets since the \isi{copredication} of V+Mod+N is always available. (\ref{ex:book:Chen}) and (\ref{ex:shu:Chen}) are \isi{copredication} examples of English \emph{book} and Chinese {\cn 书} (sh\={u}) `book'.

\ea \label{ex:book:Chen}
\ea \label{ex:readheavybook:Chen} Others may regard reading thick books as a dull way to learn.
\ex \label{ex:carryinfobook:Chen} Individuals cannot carry their religious books [...] %like the Bible
 with them.
\z
\ex \label{ex:shu:Chen}
\ea
\glll {\cn 我} {\cn 读} {\cn 了} {\cn 一} {\cn 本} {\cn 很} {\cn 厚} {\cn 的} {\cn 书} {\cn 。}\\
w\v{o} d\'{u} le y\={i} b\v{e}n h\v{e}n h\`{o}u de sh\={u} . \\
1SG read ASP one \textsc{clf}$_o$ very thick \textsc{modm} book\\
\glt `I read a very thick book.'
\ex
\glll {\cn 翻开} {\cn 那些} {\cn 难懂} {\cn 的} {\cn 书}\\
f\={a}nk\={a}i n\`{a}xi\={e} n\'{a}nd\v{o}ng de sh\={u}\\
open those difficult \textsc{modm} book\\
\glt `Open those difficult books.'
\z\z

\tabref{tab:book:Chen} presents some common \isi{modifiers} of {\cn 书} (sh\={u}) `book' and the facets the phrases have after application of %being modified by
 these \isi{modifiers}. The second column indicates the target facet of the modification, %the facet of 书(sh\={u}) `book' addressed by the \isi{modifiers},
  and the third column %indicates
   the facets accessible for further predication after modification. %``both'' means the phrase has both facets.
 After the modification, %modified by the \isi{modifiers} in the table,
  the phrases can still be the argument of {\cn 带走} (d\`{a}iz\v{o}u) `take away' (which targets the \isi{object} facet) or {\cn 阅读} (yu\`{e}d\'{u}) `read' (which targets the info facet). %, which targets only the \isi{object} facet and  the info facet respectively. So both facets are kept almost all the time.

\begin{table}[t]
\begin{tabularx}{\textwidth}{lll}
\lsptoprule
               & targeted facet & modified {\cn 书} `book' \\
\midrule
{\cn 两千字} (li\v{a}ngqi\={a}n z\`{i}) `2000 characters'   & info           & \isi{object}, info         \\
{\cn 两百页} (li\v{a}ngb\v{a}i y\`{e}) `200 pages' & \isi{object}            & \isi{object}, info             \\
{\cn 幽默} (y\={o}um\`{o}) `humorous'       & info            & \isi{object}, info             \\
{\cn 厚} (h\`{o}u) `thick'       & \isi{object}            & \isi{object}, info  \\
\lspbottomrule
\end{tabularx}
\caption{Facet availability after modification for {\cn 书} (sh\={u}) `book'}
\label{tab:book:Chen}
\end{table}

\subsection{\type{Event}\dott\type{info}}\label{sec:event.info:Chen}

\emph{Speech} and \emph{lecture} are typical \type{event}\dott\type{info} nouns in English. Neither the event facet nor the info facet is particularly prominent.
 %Copredications of the pattern V~A~N or A~A~N where the leftmost predicate addresses the event facet and the second the info facet are possible (see (\ref{ex:lecture:Chen})--(\ref{ex:shortlecture:Chen})) but it seems that they are not always felicitous, as can be seen in (\ref{ex:attenddetailedspeech:Chen}). In general, we can conclude from (\ref{ex:speech:Chen}) that \isi{facet-addressing} predications over the info facet are possible. Whether the event facet is primary, is less clear. Facet-addressing predications over the event facet seem less acceptable (see (\ref{ex:reschedulespeech:Chen}) and (\ref{ex:speecha:Chen})) but in order to exclude them, we still need to look at more data. For the moment, we assume that both facets are secondary, which means that predicating over them can be excluded due to a predication over the respective other facet. \cl{I also think lecture is a better example, because it is clear it doesn't have an obj-facet. I added two sentences found from sketchengine. ``introductory lecture'' might be argued as a compound, but ``informative lecture'' should be not controversial at all. But I think it's better to keep the speech examples here since below there are Chinese speech nouns. I also composed a new negative example with ``lecture''. I feel it's less acceptable than ``attend detailed speech''. How do you feel?}
 Copredications of the pattern V+Mod+N or Mod+Mod+N where the leftmost predicate addresses the event facet and the second the info facet are possible (see \ref{ex:lecture:Chen}--\ref{ex:shortlecture:Chen}) but it seems that they are not always felicitous, as can be seen in (\ref{ex:attenddetailedspeech:Chen}) and (\ref{ex:reschedulespeech:Chen}). In general, we can conclude from (\ref{ex:speech:Chen}) that \isi{facet-addressing} predications over the info facet are possible. Whether the event facet is primary is less clear. Facet-addressing predications over the event facet seem less acceptable (see 
 %(\ref{ex:reschedulespeech:Chen}) and  
 \ref{ex:speecha:Chen}) but in order to exclude them, we still need to look at more data. For the moment, we assume that both facets are secondary, which means that predicating over them can be excluded due to a predication over the respective other facet. %\cl{I also think lecture is a better example, because it is clear it doesn't have an obj-facet. I added two sentences found from sketchengine. ``introductory lecture'' might be argued as a compound, but ``informative lecture'' should be not controversial at all. But I think it's better to keep the speech examples here since below there are Chinese speech nouns. I also composed a new negative example with ``lecture''. I feel it's less acceptable than ``attend detailed speech''. How do you feel?}

%\lk{This is the old text, it does not really tell us which facet is what. It seems to me that we have to say a lot more here about primary/secondary and facet picking/addressing. Maybe the `recite the crowded speech' example is meant to illustrate that picking the event facet blocks the info facet. Is this correct? Then we should explain all this. And we should add more examples, also with 2 \isi{modifiers}. It seems to me that `short instructive lecture/speech' should also be acceptable. And maybe also in the other order, `instructive short lecture'? Btw., I removed the `attend an amazing speech' example because it is not so clear whether amazing refers to the info facet}For example, in (\ref{ex:speech:Chen}), only (\ref{ex:underoral:Chen}) and (\ref{ex:more-speech:Chen}) is a possible \isi{copredication}.\lk{8a and 8b come without any context; I guess that with an appropriate context, they would be better, at least 8b.}\cl{I added some context that I think would be natural to include the phrases. 8a is still strange, 8b is slightly better but I'm not sure.}

%\lk{Does speech have only event and info facets? Maybe also a obj facet? `deliver a speech' means that you deliver a carrier of the info facet.}
%\cl{``deliver'' and ``give'' are the most common predicates for ``speech'' and from the corpus data I feel they are a bit idiomatic so it is hard to decide what facet are they targeting. But if we try other predicates like ``heavy speech'' and ``burn the speech'' they are not very good. I think it's better to say it doesn't have an obj-facet}

\ea \label{ex:speech:Chen}
\ea %[]{Yesterday, we attended an instructive lecture on climate change.\footnote{Our own example.}}\label{ex:lecture:Chen}
%\ex []{[...]please attend our introductory lecture to find out more about studying History at UCD}\label{ex:introlecture:Chen}
%ex 
 []{I recently attended a most informative lecture delivered by two of Ireland's most prominent obstetricians.}\label{ex:lecture:Chen} %\label{ex:inforlecture:Chen}
\ex [ ]{Pendleton denounced them as unconstitutional, and concluded an elaborate speech against them in these words.}\label{ex:underoral:Chen}
\ex []{[...]after a short instructive lecture on [...]\footnote{\textit{The Musical Magazine} Vol.~2, 1840, p.~210.}}\label{ex:shortlecture:Chen}
%\ex [ ]{Today we attended an amazing speech given by Medea Benjamin, Nobel Peace prize winner and human rights defender.\footnote{Posted 16 April 2014, \url{https://www.flickr.com/photos/bbleekerproject365_2009/13308930654/}, retrieved 04 February 2024.}}\label{ex:more-speech:Chen}
\ex [?]{Yesterday there was a speech, and the speaker elaborated the details on \isi{frame semantics}. I attended the detailed speech.}\label{ex:attenddetailedspeech:Chen}
\ex [?]{Yesterday there was a lecture, and the speaker elaborated the details on \isi{frame semantics}. I missed it because the organizers rescheduled the detailed lecture from Friday to Monday and I wasn't informed.}\label{ex:reschedulespeech:Chen}
\ex [\#]{%The famous scientist made two speeches over the last months dealing with quite different topics. One was held in front of a small audience while the other had a large audience. I liked the content of the latter much better. Therefore I am going to recite that crowded speech to you. \\
Yesterday the famous scientist made a speech. There was a large audience and I liked the content very much. So today I am going to recite that crowded speech.}\label{ex:speecha:Chen}
%\ex [ ]{The captain could read English, but barely understood oral speech.} 
\z\z

%\ea \label{ex:more-speech:Chen}
%\ea Today we attended an amazing speech given by Medea Benjamin, Nobel Peace prize winner and human rights defender.\footnote{Posted 16 April 2014, \url{https://www.flickr.com/photos/bbleekerproject365_2009/13308930654/}, retrieved 04 February 2024.}
%\ex I am trying to sum up the planned speech for the announcement.\footnote{Our own example.}
%\z\z

%\lk{The examples in (\ref{ex:more-speech:Chen}) look fine to me. Maybe use them instead of (\ref{ex:speech:Chen})? One question is however whether the second ex.~does not involve coercion to event. What were again the criteria for distinguishing the two? We should probably talk about that. Btw., does the third predication (`given') in the first example address the \isi{object} facet?}
%\cl{Ideally there should be one or two predicate/construction as a test for whether a noun has a certain facet, but in practice it's never easy to have the specific diagnostics. I usually try several verbs and if most work/fail, then the noun has/hasn't the facet.}

%\lk{The best would be to find data where we use the same \isi{modifiers} addressing the different facets and show that only one order is good. (like the vegan lively dinner)}

In Chinese, however, we can find \type{event}\dott\type{info} nouns with primary facets.
There are two nouns for \emph{speech} in Chinese, namely {\cn 讲话} (ji\v{a}nghu\`{a}) and {\cn 演讲} (y\v{a}nji\v{a}ng).\footnote{The two morphemes in {\cn 讲话} (ji\v{a}nghu\`{a}) mean `speak' and `words', and the two morphemes in {\cn 演讲} (y\v{a}nji\v{a}ng) mean `act/perform' and `speak'. 
%The morphological structures of these two words are not related to the slight difference in meanings discussed here.
}
The former is usually used for speeches given in more serious situations or by important people, while the latter is a more general word for \type{speech}. Both are of type \type{event}\dott\type{info}, but they behave differently in \isi{copredication}.

%\begin{multicolumn}{2}

%\begin{multicols}{2}

\ea \label{ex:classifierspeech:Chen}
\ea \label{ex:pianyanjiang:Chen}
\glll {\cn 背诵}/{\cn 进行} {\cn 一} {\cn 篇} {\cn 演讲}\\
b\`{e}is\`{o}ng/j\`{i}nx\'{i}ng y\={i} pi\={a}n y\v{a}nji\v{a}ng\\
recite/perform one \textsc{clf}$_i$ speech\\
\glt `recite/perform a speech'
\ex \label{ex:pianjianghua:Chen}
\glll {\cn 背诵}/\#{\cn 进行} {\cn 一} {\cn 篇} {\cn 讲话}\\
b\`{e}is\`{o}ng/\#j\`{i}nx\'{i}ng y\={i} pi\={a}n ji\v{a}nghu\`{a}\\
recite/\#perform one \textsc{clf}$_i$ speech\\
\glt `recite/perform a speech'
%\z
%\z
%\end{multicols}
%
%\ea
\ex
\glll {\cn 背诵}/{\cn 进行} {\cn 详细} {\cn 的} {\cn 演讲}/{\cn 讲话} \label{ex:detailedspeech:Chen}\\
b\`{e}is\`{o}ng/j\`{i}nx\'{i}ng xi\'{a}ngx\`{i} de y\v{a}nji\v{a}ng/ji\v{a}nghu\`{a}\\
recite/perform detailed \textsc{modm} speech\\
\glt `recite/perform a detailed speech'
\ex 
\glll \#{\cn 背诵}/{\cn 进行} {\cn 两个小时} {\cn 的} {\cn 演讲}/{\cn 讲话} \label{ex:hourspeech:Chen}\\
b\`{e}is\`{o}ng/j\`{i}nx\'{i}ng li\v{a}ngg\`{e}xi\v{a}osh\'{i} de y\v{a}nji\v{a}ng/ji\v{a}nghu\`{a}\\
recite/perform two.hours \textsc{modm} speech\\
\glt `recite/perform a two-hour speech'
\z\z

%\end{multicolumn}

%In (\ref{ex:classifierspeech:Chen}), 
The Chinese \isi{classifier} {\cn 篇} (pi\={a}n) in (\ref{ex:pianyanjiang:Chen})--(\ref{ex:pianjianghua:Chen}) combines only with nouns that have an info facet. %, such as 文章(w\'{e}nzh\={a}ng) `article' and 报道(b\`{a}od\`{a}o) `report'.
 When it combines with {\cn 演讲} (y\v{a}nji\v{a}ng), the result %noun phrase
  {\cn 一篇演讲} (y\`{i} pi\={a}n y\v{a}nji\v{a}ng) `a speech' still has the event facet and can be predicated by {\cn 进行}(j\`{i}nx\'{i}ng) `perform'; but when it combines with {\cn 讲话} (ji\v{a}nghu\`{a}), the event facet is blocked for subsequent predication. %resulting phrase does not have the event facet anymore.
\hide{
\ea \label{ex:modifyspeech:Chen}
\ea 
\gll {\cn 背诵}/{\cn 进行} {\cn 详细} {\cn 的} {\cn 演讲}/{\cn 讲话} \label{ex:detailedspeech:Chen}\\
recite/perform detailed \textsc{modm} speech\\
\glt `recite/perform a detailed speech'
\ex 
\gll \#{\cn 背诵}/{\cn 进行} {\cn 两个小时} {\cn 的} {\cn 讲话}/{\cn 演讲} \label{ex:hourspeech:Chen}\\
recite/perform two.hours \textsc{modm} speech\\
\glt `recite/perform a two-hour speech'
\z\z
}
If {\cn 讲话} (ji\v{a}nghu\`{a}) and {\cn 演讲} (y\v{a}nji\v{a}ng) are modified by \isi{facet-addressing} \isi{modifiers} such as {\cn 详细} (xi\'{a}ngx\`{i}) `detailed' in (\ref{ex:detailedspeech:Chen}), the resulting phrase will still be a dot \isi{object}. If they are modified by \isi{facet-picking} \isi{modifiers} like {\cn 两个小时} (li\v{a}ng ge xi\v{a}osh\'{i}) `2 hours' in (\ref{ex:hourspeech:Chen}), the phrases only have the event facet and cannot be predicated by {\cn 背诵} (b\`{e}is\`{o}ng) `recite'.

\tabref{tab:speech:Chen} lists the accessible facets of both nouns after modification by some adjectival or nominal \isi{modifiers}. The event facet of {\cn 演讲} (y\v{a}nji\v{a}ng) is always preserved, but when {\cn 讲话} (ji\v{a}nghu\`{a}) is modified by {\cn 两千字} (li\v{a}ngqi\={a}nz\`{i}) `2000 characters', only the info facet remains accessible. %the phrase only has the info facet.
 Therefore, it can be concluded that in {\cn 讲话} (ji\v{a}nghu\`{a}), both facets are secondary facets, whereas in {\cn 演讲} (y\v{a}nji\v{a}ng), the event facet is the \isi{primary facet}.

\begin{table}[t]
%\caption{The ``dotness'' of phrases of 讲话(ji\v{a}nghu\`{a}) and 演讲(y\v{a}nji\v{a}ng)}
\caption{Facet availability after modification for {\cn 讲话} (ji\v{a}nghu\`{a}) `speech' and {\cn 演讲} (y\v{a}nji\v{a}ng) `speech'}
\begin{tabularx}{\textwidth}{lccc}
\lsptoprule
             & targeted & {\cn 讲话} & {\cn 演讲} \\
             & facet & (ji\v{a}nghu\`{a}) & (y\v{a}nji\v{a}ng)\\
\midrule
{\cn 两千字} (li\v{a}ngqi\={a}nz\`{i}) `2000 characters' & info       & info       & info, event        \\
{\cn 两个小时} (li\v{a}ngg\`{e}xi\v{a}osh\'{i}) `2 hours'     & event      & event      & event      \\
{\cn 详细} (xi\'{a}ngx\`{i}) `detailed' & info        &  both          & both        \\
{\cn 盛大} (sh\`{e}ngd\`{a}) `grand'    & event       & event        & event      \\
{\cn 沉闷} (ch\'{e}nm\`{e}n) `dull'     & event           & event      & event   \\
\lspbottomrule
\end{tabularx}
\label{tab:speech:Chen}
\end{table}

The reason why {\cn 演讲} (y\v{a}nji\v{a}ng) has a \isi{primary facet} while {\cn 讲话} (ji\v{a}nghu\`{a}) does not is probably related to the Chinese lexical system. The info and event facets of {\cn 讲话} (ji\v{a}nghu\`{a}) can also be referred to as {\cn 讲话稿} (ji\v{a}nghu\`{a}g\v{a}o) `speech draft' and {\cn 讲话会} (ji\v{a}nghu\`{a}hu\`{i}) `speech meeting', respectively. Neither is a frequent word. On the other hand, although the info facet and the event facet of {\cn 演讲} (y\v{a}nji\v{a}ng) can also be referred to as {\cn 演讲稿} (y\v{a}nji\v{a}ngg\v{a}o) `speech draft' and {\cn 演讲会} (y\v{a}nji\v{a}nghu\`{i}) `speech meeting', {\cn 演讲稿} (y\v{a}nji\v{a}ngg\v{a}o) `speech draft' is much more commonly-used when the info facet is targeted, while {\cn 演讲会} (y\v{a}nji\v{a}nghu\`{i}) `speech meeting' is not frequently used.

%\tabref{tab:speech} also displays the difference between the two \isi{modifiers}, 两千字 (li\v{a}ng qi\={a}nz\`{i}) `2000 characters' and 详细(xi\'{a}ngx\`{i}) `detailed'. The former picks only the info facet of 讲话(speech) and the latter keeps both facets. We can thus categorize \isi{modifiers} into facet-keeping  %\isi{modifiers}
%  and \isi{facet-picking} %\isi{modifiers}
%   with regard to whether or not %the
%    non-targeted secondary facets %which are not targeted
%     are kept. When the \isi{primary facet} of a \isi{dot-type} noun is targeted by a facet-keeping \isi{modifier}, the phrase remains to be a dot-\isi{object}. 

\section{Modelling composition with polysemous nouns}

%The formalization follows the modelling in \citet{chen2022frame} with some modifications in the elementary trees of the polysemous nouns, and is based on \citet{kallmeyer2013syntax}, with the syntactic part replaced by Tree Wrapping Grammars (\isi{TWG})\citep{Kallmeyer/Osswald/vanValin:2013}, which is developed from \isi{Role and Reference Grammar} (\isi{RRG}; \citealt{vanValin:2005}). We propose two options to model the difference between the primary and \isi{secondary facet} and their composition with \isi{modifiers}. One focuses on syntax and uses the top and bottom feature in FTAG, and the other focuses on semantics and uses \isi{default logic} in frames.\cl{also move to the new subsection}

%\cl{I plan to move introduction to \isi{RRG} trees and frames constraint and unification (non-default) here. Is it better to have a new subsection or just as several paragraphs before 3.1?}

%\subsection{A brief introduction to the modelling framework}
\subsection{Background: Syntax-driven composition of semantic frames}

For modelling composition at the syntax-semantics interface, we build on the framework of \citet{kallmeyer2013syntax}, which involves tree rewriting on the syntactic side and frame unification on the semantic side.
The basic representational components of the approach are \emph{elementary constructions}, which are pairs of constituent trees and semantic frames, together with a partial map from constituent nodes to nodes of the associated frame.

Frames are here understood as \emph{generalized feature structures}.\footnote{%
For a formal account of frames and the logic of attribute-value descriptions and formulas see \citet[Sect.~3]{kallmeyer2013syntax} or the more recent version in \citet[Appendix]{chen2022frame}.}
While conventional feature structures require that each node of the frame is accessible from a distinguished root node via a finite attribute sequence, frames relax this condition in that each node is required to be accessible from at least one of a non-empty set of \emph{labeled nodes}. 
A corresponding generalization applies to the unification of frames:
Instead of identifying the designated root nodes of feature structures, \emph{frame unification} relies on the identification of all nodes that carry the same labels.
Similarly, \emph{subsumption} naturally extends from ordinary feature structures to frames.
As in the single-rooted case, the unification of two frames is their least upper bound with respect to subsumption, if existent.

In order to characterize the properties of frames and frame nodes, we make use of \emph{attribute-value formulas} and \emph{descriptions}, which gives us a logical language tailored for the description of frames and for drawing inferences about them.
The vocabulary of this language consists of type symbols (e.g.\ \type{apple}, \type{sweet}, \type{eating}), attribute symbols (e.g.\ \feat{taste}, \feat{actor}, \feat{eater}), relation symbols, node variables (e.g.\ \xvar{a}, \xvar{b}, \xvar{c}, \dots), and node names.
Examples of (primitive) attribute-value descriptions are $\type{apple}$, $\feat{taste}\D\type{sweet}$, and $\feat{actor}\patheq\feat{eater}$.
They can be seen as one-place predicates which might be satisfied at a node of a frame (under a given interpretation of the vocabulary).
For instance, $\feat{taste}\D\type{sweet}$ is satisfied at a node $v$ if $v$ has an attribute denoted by \feat{taste} whose value node is an instance of type $\type{sweet}$.
The description $\feat{actor}\patheq\feat{eater}$ is satisfied at $v$ if $v$ has attributes denoted by \feat{actor} and \feat{eater} whose value nodes are identical.
In contrast to descriptions, attribute-value \emph{formulas} are not satisfied at nodes but by frames.
Formulas include ``grounded'' attribute-value descriptions such as $\xvar{x}\cdot\type{apple}$ and $\xvar{x}\cdot\feat{taste}\D\type{sweet}$, which are satisfied by a frame $F$ if $F$ has a node labeled $\xvar{x}$ at which the respective descriptions are satisfied.
Formulas include also expressions such as $\xvar{x}\cdot\feat{actor}\valeq\xvar{y}$, which is satisfied by $F$ if $F$ has nodes labeled $\xvar{x}$ and $\xvar{y}$ such that the latter node is the value of the attribute denoted by \feat{actor} of the former node. 
Attribute-value formulas and, likewise, descriptions, can be combined by Boolean connectives.

Crucially, frames can be seen as \emph{minimal models} of conjunctive attribute-value formulas.
Conjunctions of attribute-value descriptions or formulas will be often presented in the well-known format of \emph{attribute-value matrices} (AVMs).
%A frame $F$ subsumes a frame $F'$ iff $F'$ is ``as least as informative'' than $F$ in the sense that every attribute-value formula satisfied by $F$ is also satisfied by $F'$.
In addition to descriptions and formulas, we also make use of (universal) \emph{attribute-value constraints}, which are basically universally quantified descriptions.
If $\phi$ is an attribute-value description then the constraint $\forall\phi$ is satisfied if $\phi$ is satisfied at every node.
We write $\phi\implic\psi$ instead of $\forall(\phi\to\psi)$.
Typical use cases are subtyping (e.g.\ $\type{apple}\implic\type{fruit}$), incompatibilities (e.g.\ $\type{fruit}\wedge\type{meat}\implic\bot$), and attribute requirements (e.g.\ $\type{eating}\implic\feat{eater}\D\top$).\footnote{$\bot$ is satisfied by nothing, $\top$ is satisfied by everything.}
Given a finite set of constraints $\phi\implic\psi$ with non-disjunctive $\psi$ and a frame $F$, there exists a unique minimal frame $F'$ subsumed by $F$ that satisfies all constraints (if no inconsistencies occur and $F'$ keeps being finite).

%, where the restriction refers to the fact that the adjoined tree only adds material to one side of the path fromtrees we use for adjunction are such that .\isi{TAG} is a tree-rewriting formalism. A \isi{TAG} consists of a finite set of elementary trees. The nodes of these trees are labelled with non-terminal and terminal symbols, with terminals restricted to leaf nodes. Starting from the elementary trees, larger trees are derived by substitution (replacing a leaf with a new tree) and adjunction (replacing an internal node with a new tree).

The tree rewriting formalism used in this paper is Tree Adjoining Grammars (\isi{TAG}; \citealp{JoshiSchabes:97,AbeilleRambow::00}).%, more precisely, its lexicalized and feature-based version.
\footnote{Formalisms with different tree operations are also possible; e.g., the Tree Wrapping Grammar formalism described in \citet{Kallmeyer/Osswald:2023}.}
A \isi{TAG} is a finite set of elementary trees that can be combined into larger trees via the tree composition operations \textit{substitution} (replacing a leaf with a new tree) and  \textit{adjunction} (replacing an internal node with a new tree). The trees that can be added by adjunction are special in that they have a leaf node that is \isi{marked} as \textit{foot node} and that has the same non-terminal label as the root. These trees are called \emph{auxiliary trees}. The effect of an adjunction is that the root of the adjoined auxiliary tree replaces the target node of the adjunction and the tree below the target node ends up below the foot node. 

As mentioned at the beginning of this section, elementary and derived constructions are understood as pairs of constituent trees and semantic frames together with a partial map that takes constituent nodes to frame nodes.
This mapping is encoded by the feature \feat{I(NDEX)} carried by the constituent nodes.
%As to the syntax-semantics interface, we basically build on approaches which link a semantic representation to an entire elementary tree and which model semantic composition by unifications triggered by substitution and adjunction.
Due to this linking and the resulting identification of the \feat{I} values, syntactic operations, here substitution and adjunction, can give rise to the unification of the associated frames.
Consider \figref{fig:takebook:Chen} below for an example.
%of the composition of `took away' and `the book' for instance in `Kim took away the book'.
%The syntactic tree and the semantic frame are connected through the interface features \feat{i} on the nodes.
The substitution of the tree of `the book' into the \isi{object} NP node of `took away' leads to a unification of the respective frames under the identification of the node labels $p$ and $u$.
The resulting frame is represented at the bottom of the figure.

The example also illustrates the specific grammatical theory, \isi{Role and Reference Grammar} (\isi{RRG}; \citealt{vanValin:2005,Bentley/Mairal/Nakamura/VanValin:2023}), that we use for our syntactic representations.
\isi{RRG} provides an elaborate theory of clause linkage, which is helpful for the analysis of \isi{copredication} constructions.
Moreover, \isi{RRG} assumes a layered structure consisting of \emph{nucleus} (NUC), \emph{core} (CORE) and \emph{clause} (CL).
The nucleus contains the main predicate, the core contains the nucleus and the (non-extracted) syntactic arguments, and the clause includes the core and extracted arguments.
Adjuncts can attach to each layer, and the assumption of a layered structure holds across categories.
Notice that the specific choice of the syntactic inventory does not play an important role for the purposes of the present paper. 
A more common X-bar schema would also do for the compositional mechanisms proposed in the following.

%%%%%%%%%%%%%%%%%%%%%%%%%%%%%%%%%%%%%%%%%%%%%%%%%%%%%%%%%%%%%%
\subsection{The representation of meaning facets and dot type nouns}

Following \citet{chen2022frame}, we assume that frame-semantic representations of polysemous nouns come with facet attributes whose values characterize the individual meaning facets of the nouns.
Predicates and \isi{modifiers} then access meaning facets of the semantic frames contributed by their arguments.
Non-polysemous nouns can be integrated into this picture by assuming that they provide a single \isi{meaning facet} that points to the denoted entity itself.
For example, suppose that our semantic model provides a type \type{apple}, whose instances are denoted by the English word \emph{apple}, and that \type{apple} is a subtype of \type{food}, and  \type{food} is a subtype of \type{physical-object} (\type{phys-obj}).
Moreover, let us assume that each instance of type \type{phys-obj} has an attribute \feat{object-facet} (\feat{obj-fct}) whose value is the instance itself.
A formal presentation of these constraints is given in (\ref{apple-constraints:Chen}), where $\SELF$ stands for the identity function.
%An example are the frame constraints for instances of type \type{apple} in (\ref{apple-constraints:Chen}) that express that \type{apple} is a subtype of \type{food}, which is in turn a subtype of \type{phys-obj} and that a node of type \type{phys-obj} has an attribute \feat{obj-fct} whose value is the \type{phys-obj}  node itself.
\ea\label{apple-constraints:Chen}
\ea\label{apple-physobj:Chen}
$\type{apple}\:\implic\:\type{food}$\,, \quad
$\type{food}\:\implic\:\type{phys-obj}$
\ex\label{self-constraint:Chen}
$\type{phys-obj}\:\implic\footnotemark{}\:\feat{obj-fct}\patheq\feat{self}$ 
%\ea\label{applea:Chen}$\type{apple} \implic \type{food}$
%\ex\label{appleb:Chen}$\type{food} \implic \type{phys-obj}$ 
%\ex\label{applec:Chen}$\type{phys-obj}\implic\feat{obj-fct}\patheq\feat{self}$ 
\z
\z
\footnotetext{`$\implic$' is always the last operator to be combined.}
It follows that each instance \xvar{x} of type \type{apple} is of type \type{phys-obj} and has an attribute \feat{obj-fct} whose value is \xvar{x}, in symbols, $\xvar{x}\cdot\feat{obj-fct}\valeq\xvar{x}$.

Polysemous nouns are assumed to denote instances of a dot type, whose associated facet attributes have values that differ from the \isi{dot-type} instance, in general.
%A polysemous noun, in contrast, as more than one \isi{meaning facet} and includes a node for the dot \isi{object} that is distinct from the individual meaning facets.
This is illustrated by the attribute-value constraint in (\ref{physinfo-facets:Chen}):
Instances of type \type{phys-obj}\dott\type{info} have two attributes \feat{obj-fct} and \feat{info-fct}, whose values
are related by the attribute \feat{cont(ent)} and are of type \type{info-carrier} and \type{info}, respectively.
%Their values are the \isi{object} facet and information facet respectively, % (see \cite{chen2022frame} for the necessity of facet attributes in modelling predications),
 %and they are of type \type{info-carrier} and \type{info} respectively,
\begin{figure}[t]
\small
\begin{tikzpicture}[scale=0.4, minimum size=1.5ex, inner sep=.5ex,  >=stealth, x=9ex, y=8ex, ]
%baseline=(current bounding box.north)]
%\begin{scope}[shift={(0.0,0.0)}]
%\draw (-8.0, 2.0)node[circle](bla){%
\node at (-10.0, 1.5) {
%\avm{[\type{book$\und$}\\\type{phys-obj}\dott\type{info}\\
\xvar{v}\avm{[\type*{book\,$\und$\,phys-obj\dott info}\\
obj-fct & [\type*{info-carrier}\\
cont & \1]\\
info-fct & \1\,[\type{info}]]}};
\draw (-1.0, 1.5)node[scale=.8, draw, circle, label={[label distance=-4.5ex]-135:\textit{\footnotesize\begin{tabular}{c}~\\[-1ex]\type{book}\\[-.5ex]\type{phys-obj}\dott\type{info}\end{tabular}}}](0){\xvar{v}};
  \draw (3.5, 0.0)node[draw, circle, label={[label distance=0.5ex]0:\type{\footnotesize info}}](1){};
  \draw (3.5, 3)node[draw, circle, label={[label distance=0.5ex]0:\type{\footnotesize info-carrier}}](4){};
  \path[->] (0) edge node[above, sloped]{\scriptsize\feat{info-fct}}(1);
  \path[<-] (4) edge node[above, sloped]{\scriptsize\feat{obj-fct}}(0);
  \path[->] (4) edge node[above, sloped]{\scriptsize\feat{cont}}(1);
%\end{scope}
\end{tikzpicture}
\caption{AVM and minimal frame model for the \type{book} frame}
\label{fig:book:Chen}
\end{figure}
%
\ea\label{book-constraints:Chen}
\ea\label{book-dottype:Chen}
$\type{book}~\implic~\type{phys-obj}\dott\type{info}$\\
%\ex\label{bookb:Chen}
%{\tabcolsep0ex
%\begin{tabular}{ll}
%$\type{phys-obj}\dott\type{info}\implic~$
%& $\feat{obj-fct}\D\type{info-carrier}\und\feat{info-fct}\D\type{info}$\\
%& $\und\:\feat{obj-fct}\CP\feat{cont}\patheq\feat{info-fct}$,
%\end{tabular}}\\
\ex\label{physinfo-facets:Chen}
$\type{phys-obj}\dott\type{info}~\implic~\feat{obj-fct}\D\type{info-carrier}\,\und\,\feat{info-fct}\patheq\feat{obj-fct}\CP\feat{cont}$\\
\ex\label{infocarrier:Chen}
$\type{info-carrier}~\implic~\type{phys-obj}\,\und\,\feat{cont}\D\type{info}$
\z
\z
The type \type{info-carrier} is specified in (\ref{infocarrier:Chen}) as a subtype of \type{phys-obj} whose instances have a \feat{cont} attribute of type \type{info}.
In (\ref{book-dottype:Chen}), \type{book} is specified as a subtype of \type{phys-obj}\dott\type{info}.
\figref{fig:book:Chen} shows the minimal frame model for an instance $v$ of \type{book}, subject to the constraints in (\ref{book-constraints:Chen}), together with the corresponding AVM.
\figref{fig:takebook:Chen} illustrates how syntactic argument substitution in the derivation of \emph{took away the book} can give rise to a unification of the `book' frame and the `take away' frame.

%The constraint (\ref{booka:Chen}) indicates that \type{book} is a subtype of a dot type \type{phys-obj}\dott\type{info}; (\ref{bookb:Chen}) indicates that the dot type \type{phys-obj}\dott\type{info} has two attributes: \feat{obj-fct} and \feat{info-fct}. They refer to the \isi{object} facet and info facet respectively (see \cite{chen2022frame} for the necessity of facet attributes in modelling predications), and they are of type \type{info-carrier} and \type{info} respectively, and they are connected by the \feat{cont} (content) attribute. \figref{fig:book} displays the %The AVM in (\ref{bookavm:Chen}) displays the minimal frame for a node of type \type{book} that satisfies the constraints (\ref{booka:Chen}) and (\ref{bookb:Chen}), represented as AVM and also as graph. 

\hide{\ea\label{bookavm:Chen}
\avm{[\type{book$\und$}\\
\type{phys-obj}\dott\type{info}\\
obj-fct & [\type{info-carrier}\\
cont & \1]\\
info-fct & \1[\type{info}]]}
\z
}

The complex types $\type{event}\dott\type{food}$ and $\type{event}\dott\type{info}$, together with their respective subtypes \type{dinner} and \type{speech}, can be specified in a similar vein.
In (\ref{eventfood:Chen}), instances of type $\type{event}\dott\type{food}$ are characterized as having the facet attributes $\feat{ev-fct}$ and $\feat{obj-fct}$, where the value of $\feat{ev-fct}$ is of type \type{eating} and the value of $\feat{obj-fct}$ is the \feat{theme} of the \type{eating} instance, which (\ref{eating:Chen}) requires to be of type \type{food}.\footnote{We omit the fact that \type{dinner} restricts the event time to the evening since it does not play a role for possible predications over facets.}

\ea\label{dinnerconstraints:Chen}
\ea
$\type{dinner}~\implic~\type{event}\dott\type{food}$
%{\tabcolsep0ex
%\begin{tabular}{ll}
%$\type{event}\dott\type{food}\,\implic\,~$
%&$\feat{ev-fct}\D\type{event}\,\und\,\feat{obj-fct}\D\type{food}$\\
%&$\und\,\feat{obj-fct}\patheq\feat{ev-fct}\CP\feat{theme}$,%,
%\end{tabular}}\\
% $\type{food}\implic\type{phys-obj}$,  % has already been introduced before
\ex\label{eventfood:Chen}
%$\type{event}\dott\type{food}~\implic~\feat{obj-fct}\D\type{food}\,\und\,\feat{ev-fct}\CP\feat{theme}\patheq\feat{obj-fct}$
$\type{event}\dott\type{food}~\implic~\feat{ev-fct}\D\type{eating}\,\und\,\feat{obj-fct}\patheq\feat{ev-fct}\CP\feat{theme}$
\ex\label{eating:Chen}
$\type{eating}~\implic~\type{event}\und\feat{theme}\D\type{food}$
\z
\z

According to (\ref{eventinfo:Chen}), an instance of type $\type{event}\dott\type{info}$ has a facet attribute \feat{ev-fct} of type \type{info-event}, which in turn has an attribute \feat{cont} of type \type{info} (\ref{infoevent:Chen}), whose value is identical to the value of the \feat{info-fct} attribute of the \isi{dot-type} instance.
%
\ea\label{speechconstraints:Chen}
\ea
$\type{speech}~\implic~\type{event}\dott\type{info}$
\ex\label{eventinfo:Chen}
$\type{event}\dott\type{info}~\implic~\feat{ev-fct}\D\type{info-event}\,\und\,\feat{info-fct}\patheq\feat{ev-fct}\CP\feat{cont}$
\ex\label{infoevent:Chen}
$\type{info-event}~\implic~\type{event}\und\feat{cont}\D\type{info}$
\z
\z
%
\begin{figure}[tt]
\small
\tikzset{>=stealth}
\begin{forest} for tree={s sep=35pt, l sep=2ex}
%  for tree={parent anchor=south, child anchor=north, align=center, l sep=15mm}
  %[,phantom
  [{CLAUSE},name=coren1
    [{CORE{\footnotesize $ \left[\feat{i}=e\right]$}} %, edge label={node[midway, left, font=\scriptsize]{}}
      [{NP{\footnotesize $ \left[\feat{i}=a\right]$}}]
      [NUC[took away,roof]]
      [{NP{\footnotesize $ \left[\feat{i}=p\right]$}},name=obj
      [,phantom[NP{\footnotesize $\left[\feat{i}=u\right]$},name=rp [the book,roof]]]]
      ]
    ]
  ]
\draw[->,densely dotted] (rp) to[out=90,in=-90] (obj);
\node at (4.5,-0.5){
\avm{\textit{e}[\type*{taking-away}\\
agent & a\\
patient & q]}
};
\node at (4.3,-1.5){
\avm{\textit{p}[obj-fct & q]}
};
\node at (5.5,-4.0){
\avm{\textit{u}[\type{book}\\
obj-fct & [\type*{info-carrier}\\
cont & \1]\\
%time & [\type{evening}]\\
info-fct & \1\,[\type{info}]]}
};
\node at (-3.0,-5.1){Resulting frame:};
\begin{scope}[shift={(-4.0,0.9)}]
\node at (2.73,-8.1){
\tabcolsep0ex
\begin{tabular}{rl}
\xvar{e}&
\avm{[\type*{taking-away}\\
agent & a\\
patient & \xvar{q}[\type*{info-carrier}\\
cont & \1\,[\type{info}]]]}
\\\\[-2ex]
\xvar{u}&
\avm{[\type{book}\\
obj-fct & \xvar{q}\\
%time & [\type{evening}]\\
info-fct & \1]}
\end{tabular}
};
\end{scope}
\begin{scope}[shift={(3.0,-9.0)}]
\tikzset{inner sep=.3ex, minimum size=3ex}
\draw (0.5, 0.5)node[draw, circle, label={[label distance=0.5ex]-180:\footnotesize\type{book}}](0){\scriptsize $u$};
  \draw (3.0, 1.0)node[draw, circle, minimum size=1.5ex, label={[label distance=-0.0ex]-90:\footnotesize\type{info}}](1){};
  \draw (4.0, 0.5)node[draw, circle](a){\scriptsize $a$};
  \draw (3.5, 2.5)node[draw, circle, label={[label distance=-0.0ex]90:\footnotesize\type{taking-away}}](2){\scriptsize $e$};
  \draw (0.5, 2.5)node[draw, circle, label={[label distance=-0.0ex]90:\footnotesize\type{info-carrier}}](4){\scriptsize $q$};
  \path[->] (0) edge node[below, sloped]{\footnotesize\feat{info-fct}}(1);
  \path[->] (2) edge node[above, sloped]{\footnotesize\feat{patient}}(4);
  \path[->] (2) edge node[above, sloped]{\footnotesize\feat{agent}}(a);
  \path[<-] (4) edge node[below, sloped]{\footnotesize\feat{obj-fct}}(0);
  \path[->] (4) edge node[above, sloped]{\footnotesize\feat{cont}}(1);
\end{scope}
\end{forest}
\caption{\label{fig:takebook:Chen}Derivation for \emph{took away the book} with resulting frame}
\end{figure}
%
The following more general constraints encode various type restrictions and incompatibilities for the basic types \type{event}, \type{info} and \type{phys-obj} and the facet attributes \feat{ev-fct}, \feat{info-fct} and \feat{obj-fct}:

\ea\label{typeconstraints:Chen}
\ea
$\type{event}\und\type{info}\:\implic\:\bot$\,,\quad
$\type{event}\und\type{phys-obj}\:\implic\:\bot$\,,\quad
$\type{phys-obj}\und\type{info}\:\implic\:\bot$
\ex
$\feat{ev-fct}\D\true\:\implic\:\feat{ev-fct}\D\type{event}$\,, \quad
$\feat{info-fct}\D\true\:\implic\:\feat{info-fct}\D\type{info}$\,,\\
$\feat{obj-fct}\D\true\:\implic\:\feat{obj-fct}\D\type{phys-obj}$
\ex\label{facetrestrictions:Chen}
$\type{event}\,\und(\feat{info-fct}\D\true\vel\feat{obj-fct}\D\true)~\implic~\false$\,,\\
$\type{info-obj}\,\und(\feat{ev-fct}\D\true\vel\feat{obj-fct}\D\true)~\implic~\false$\,,\\
$\type{phys-obj}\,\und(\feat{ev-fct}\D\true\vel\feat{info-fct}\D\true)~\implic~\false$
\z
\z
In addition to (\ref{facetrestrictions:Chen}), there are constraints on \type{event} and \type{info} that resemble the constraint in (\ref{self-constraint:Chen}) on \type{phys-info}, with \feat{obj-fct} replaced by \feat{ev-fct} and \feat{info-fct}, respectively. 
They ensure that the basic types license exactly one facet attribute, which encodes self-reference.

\hide{Frame representations of a certain domain are usually \isi{subject} to a number of \emph{(universal) AV constraints} that express implicational relations between types and attributes: Types may be (i) subtypes of other types, (ii) certain types may imply the presence of certain attributes (and vice versa), etc. We use the logic from \citet{chen2022frame} to express such constraints.

Our second approach uses %plan is to use
 default constraints in the frames of the polysemous nouns and \isi{modifiers} to distinguish secondary facets from primary facets and enable them to be cancelled during default unification.
 
\ref{booka:Chen} %The first constraint
 indicates that %the type
  \type{book} belongs to the \isi{dot-type} \type{phys-obj}\dott\type{info}. (\ref{bookb:Chen}) %The second constraint indicates
  expresses that anything that is of type %belongs to the \isi{dot-type}
   \type{phys-obj}\dott\type{info} has an \feat{obj-fct} and an \feat{info-fct}, and the two facets are related to each other by a \feat{cont} attribute, which means the \feat{info-fct} is the content of the \feat{obj-fct}.}

%frame and (non-default) unification
\subsection{Using top and bottom features for facet-picking modifiers}

Our first approach to modelling \isi{copredication} restrictions uses feature-structure based \isi{TAG} (FTAG, \citealp{vijay1988feature}). In an FTAG, each node has a top and a bottom feature structure, except for substutition nodes, which only have a top structure. Nodes in the same elementary tree can share feature values. When adjoining an auxiliary tree $\beta$ to a node $\nu$, the top of the root of $\beta$  unifies with the top of $\nu$ while the bottom of the foot of $\beta$ unifies with the bottom of $\nu$.
In the final derived tree, top and bottom feature structures must unify for all nodes. % in the tree. 

We use the capability of FTAG auxiliary trees to separate the top and bottom feature structures of the target adjunction site. This allows adjoining constructions, roughly, to embed the frame of a dot type noun while passing upwards only the frame of the facet that they pick. 
%
As an example, consider the composition of \emph{lively} and \emph{dinner} in \figref{fig:livelydinner2:Chen}. The NUC$_\mathrm{N}$ node of \emph{dinner} has a bottom feature structure with an \feat{i} feature whose value is the label $u$ of the dot type frame. The \feat{i} value in the top feature structure of that node is shared with the bottom of the CORE$_\mathrm{N}$ node. If nothing adjoins, all \feat{i} features in the \emph{dinner} tree will be set to $u$ due to the final top-bottom unification. In \figref{fig:livelydinner2:Chen}, an adjunction takes place at that node. The adjoining tree of \emph{lively} retrieves the \type{dinner} frame via the top \feat{i} feature (value $t$) of its foot node and via final top-bottom unification (which unifies $t$ with $u$), and passes the event facet (\feat{ev-fct}, label $z$) of this frame upwards via the bottom feature structure of its root node. \figref{fig:livelydinnerresult:Chen} shows the result of the adjunction  after the final top-bottom unification. Due to the adjunction, the \feat{i} feature has changed from $u$ (= dot type) on the lower NUC$_\mathrm{N}$ node to $z$ (= event) on the higher NUC$_\mathrm{N}$ node.

\begin{figure}[t]
\small
\tikzset{>=stealth}
\begin{forest} for tree={s sep=30pt, l sep=2ex}
%  for tree={parent anchor=south, child anchor=north, align=center, l sep=15mm}
  [,phantom[{{NUC$_\mathrm{N}$}$_{\mbox{\small [\feat{i}=$z$]}}^{\phantom{\mbox{\small [\feat{i}=$z$]}}}$},name=coren1
    [{AP}, edge label={node[midway, left, font=\scriptsize]{}}
      [lively,roof
      ]
    ] [{NUC$_\mathrm{N}^*$}$_{\phantom{\mbox{\small [\feat{i}=$t$]}}}^{\mbox{\small [\feat{i}=$t$]}}$]
  ]
[NP$_{\mbox{\small [\feat{i}=$\svar{4}$]}}^{\phantom{\mbox{\small [\feat{i}=$\svar{4}$]}}}$
    [CORE$_\mathrm{N}$$_{\mbox{\small [\feat{i}=$\svar{2}$]}}^{\mbox{\small [\feat{i}=$\svar{4}$]}}$
      [{NUC$_\mathrm{N}$$_{\mbox{\small [\feat{i}=$u$]}}^{\mbox{\small [\feat{i}=$\svar{2}$]}}$},name=bnno
        [N[dinner]]
        ]
      ]
    ]]
\draw[->,densely dotted] (coren1) to[out=10,in=150] (bnno);
%\node at (-2.8,-4.5){
%\avm{$t$[ev-fct & \1]}
%};
\node at (-2.0,-4.3){
\avm{$t$[ev-fct & $z$[\type{event}\\
ev-fct & $z$\\
atmo & [\type{lively}]]]}
};
\node at (5.2,-4.5){
\avm{$u$[\type{dinner}\\
ev-fct & [\type{eating}\\
theme & \3\\
%time & [\type{evening}]
]\\
obj-fct & \3[\type{food}]]}
};
\end{forest}
\caption{\label{fig:livelydinner2:Chen}Using top and bottom features for facet picking: The syntactic and semantic composition of \emph{lively dinner}}
\end{figure}
%
%\begin{wrapfigure}{l}{.65\textwidth}   
\begin{figure}[tb]
\small
\tikzset{>=stealth}
\begin{forest} for tree={l sep=2ex}
[NP{\footnotesize $ \left[\feat{i}=z\right]$}
    [CORE$_\mathrm{N}${\footnotesize $ \left[\feat{i}=z\right]$}
      [NUC$_\mathrm{N}${\footnotesize $ \left[\feat{i}=z\right]$}
        [AP
      [lively,roof]]
      [NUC$_\mathrm{N}${\footnotesize $ \left[\feat{i}=u\right]$}
      [dinner,roof]]
        ]
      ]
    ]
%\node at (4.3,-1.0){
%\avm{$z$[ev-fct & \1]}
%};
\node at (4.3,-1.5){
\avm{$u$[\type{dinner}\\
ev-fct & $z$[\type{eating}\\
ev-fct & $z$\\
theme & \2\,[\type{food}]\\
%time & [\type{evening}]\\
atmo & [\type{lively}]]\\
obj-fct & \2]}
};
\end{forest}
\caption{\label{fig:livelydinnerresult:Chen}Derived construction for \emph{lively dinner} after final top-bottom unification}
\end{figure}
%\end{wrapfigure}

\hide{% old \isi{edge feature} figure, frame incorrect ...
\begin{figure}
\begin{forest}
[RP
    [CORE$_\mathrm{R}${\footnotesize $ \left[\feat{i}=z\right]$}
      [{\colorbox{gray!15}{\small [\feat{i}=$z$]}NUC$_\mathrm{R}$},name=bnno
        [{\colorbox{gray!15}{\small [\feat{i}=$z$]}MP\colorbox{gray!15}{\small [\feat{i}=$t$]}},name=b
      [lively,roof]]
      [\colorbox{gray!15}{\small [\feat{i}=$u$]}R[dinner]]
        ]
      ]
    ]
\node at (5.0,-3.0){
\avm{z[\type{dinner}\\
ev-fct & x[\type{eating}\\
theme & y[\type{food}]\\
time & [\type{evening}]\\
atmo & [\type{lively}]]]}
};
\hide{\begin{scope}[shift={(-14.0,-0.5)}]
\draw (14.0, -6.0)node[draw, circle, label={[label distance=0.0ex]-90:\textit{eating}}](0){z};
  \draw (10.0, -8.0)node[draw, circle, label={[label distance=0.0ex]-180:\textit{food}}](1){y};
%  \draw (10.0, -5.0)node[draw, fill, circle, label={[label distance=-0.5ex]180:$z$}](2){};
  \draw (18.5, -8.5)node[draw, circle, label={[label distance=0.5ex]0:$dinner$}](4){u};
  \draw (17.5, -4.5)node[draw, circle, label={[label distance=0.0ex]0:\textit{evening}}](3){};
  \draw (18.5, -6.5)node[draw, circle, label={[label distance=0.0ex]0:\textit{lively}}](5){};
  \path[->] (0) edge node[above, sloped]{\footnotesize\feat{theme}}(1);
  \path[->,out=90,in=180,looseness=15] (0) edge node[above, sloped]{\footnotesize\feat{ev-fct}}(0);
  \path[->] (4) edge node[above, sloped]{\footnotesize\feat{ev-fct}}(0);
  \path[->] (4) edge node[above, sloped]{\footnotesize\feat{obj-fct}}(1);
  \path[->] (0) edge node[above, sloped]{\footnotesize\feat{time}}(3);
  \path[->] (0) edge node[above, sloped]{\footnotesize\feat{atmo}}(5);
%  \path[->] (2) edge node[below, sloped]{\feat{obj-fct}}(1);
\end{scope}}
\end{forest}
\caption{\label{fig:livelydinnerresult:Chen}The final construction for \emph{lively dinner} \cl{there might be one more NUCr depending on which option (edge feature or top-bottom feature) we choose}; fig.6 has the same problem}
\end{figure}
% end of hide old \isi{edge feature} figure with incorrect frame
}

%\figref{fig:livelydinnerresult} shows the final result of the composition of the tree and frames of ``lively'' and ``dinner''.\cl{Is it unified or not? The frame below is the unified frame of u and t but the \isi{edge feature} on the tree seems not unified. Also is this figure needed or not if we will also have ``vegan lively dinner''? Unify it.} Because of the \isi{edge feature} percolation, the i feature of CORE\_R node gets the value $z$, which corresponds to the frame with the type \type{event} without the \feat{obj-fct}.

%On the other hand, if ``dinner'' is not modified by any \isi{modifiers}, the value of the \feat{i}-feature on the CORE node will be $u$, which still contains the obj-facet.%\cl{Maybe also the tree for ``delicious dinner''?}

%If ``dinner'' is modified by an \isi{object} \isi{modifier}, the tree of ``dinner'' remains the same, but the edge features on the MP node of the \isi{modifier} tree are different. As is displayed in \figref{fig:vegandinner} both \feat{i} features have value $v$, which means that the frames $v$ and $u$ unify and the resulting frame still contains the \feat{obj-fct} of ``dinner''.%, is passed as \feat{i} feature to the CORE$_R$ node.

%\begin{wrapfigure}{r}{.4\textwidth}
\begin{figure}[t]
\small
\begin{forest} for tree={s sep=35pt, l sep=2ex}
%  for tree={parent anchor=south, child anchor=north, align=center, l sep=15mm}
  [{{NUC$_\mathrm{N}$}$_{\mbox{\small [\feat{i}=$v$]}}^{\phantom{\mbox{\small [\feat{i}=$z$]}}}$}
    %[AP
      [vegan,roof
      %]
    ]
    [{NUC$_\mathrm{N}^*$}$_{\phantom{\mbox{\small [\feat{i}=$t$]}}}^{\mbox{\small [\feat{i}=$v$]}}$]
    ]
\node at (5.0,-0.5){
\avm{$v$[obj-fct & [\type{food}\\
comp & [\type{non-meat}]]]}
};
\end{forest}
\caption{\label{fig:vegan-top-bottom:Chen}Elementary construction for \emph{vegan}}
%\end{wrapfigure}
\end{figure}
In contrast to a \isi{facet-picking} \isi{modifier} such as \emph{lively}, a \isi{facet-addressing} \isi{modifier} such as \emph{vegan} in \emph{vegan dinner} just adds information to one facet (here the \isi{object} facet, \feat{obj-fct}) but does not change the \feat{i} feature. \figref{fig:vegan-top-bottom:Chen} shows the corresponding elementary construction for \emph{vegan}. After adjoining it to \emph{dinner} and performing a final top bottom unification, we obtain a tree where all nodes on the path from \emph{dinner} to the root have the same \feat{i} feature value, namely the label of the dot type frame (the unification of $u$ and $v$).

\hide{
\begin{figure}
\begin{forest} for tree={s sep=35pt}
%  for tree={parent anchor=south, child anchor=north, align=center, l sep=15mm}
  [,phantom[{{NUC$_\mathrm{N}$}$_{\mbox{\small [\feat{i}=$v$]}}^{\phantom{\mbox{\small [\feat{i}=$z$]}}}$},name=coren1
    [AP
      [vegan,roof
      ]
    ]
    [{NUC$_\mathrm{N}^*$}$_{\phantom{\mbox{\small [\feat{i}=$t$]}}}^{\mbox{\small [\feat{i}=$v$]}}$]
  ]
[NP
    [CORE$_\mathrm{N}$$_{\mbox{\small [\feat{i}=$\svar{2}$]}}^{\phantom{\mbox{\small [\feat{i}=$\svar{2}$]}}}$
      [{NUC$_\mathrm{N}$$_{\mbox{\small [\feat{i}=$u$]}}^{\mbox{\small [\feat{i}=$\svar{2}$]}}$},name=bnno
        [dinner,roof]
        ]
      ]
    ]]
\draw[->,densely dotted] (coren1) to[out=0,in=-190] (bnno);
%\node at (-2.8,-4.0){
%\avm{\3[ev-fct & \1]}
%};
\node at (-1.8,-4.6){
\avm{$v$[obj-fct & [\type{food}\\
comp & [\type{non-meat}]]]}
};
\node at (5.4,-1.8){
\avm{$u$[\type{dinner}\\
ev-fct & [\type{eating}\\
theme & \3\\
%time & [\type{evening}]
]\\
obj-fct & \3[\type{food}]]}
};
\end{forest}
\caption{\label{fig:vegandinner:Chen}The syntactic and semantic composition of ``vegan dinner''}
\end{figure}
}

%\figref{fig:vegandinner} and \figref{fig:vegandinnerresult} show the adjunction of the trees of ``vegan'' and ``dinner'' and the resulting tree and frame after the top-bottom unification. In the composition of ``vegan'' and ``dinner'', the elementary tree of ``dinner'' and its top and bottom feature structures remain the same as in the composition of ``lively'' and ``dinner''; on the elementary tree of ``vegan'', the top and bottom feature structures are different from ``lively'', which shows the difference between facet-keeping \isi{modifiers} and \isi{facet-picking} \isi{modifiers}. Also different from ``lively dinner'', the \feat{i} feature of the CORE$_N$ node is a frame that has both the \feat{ev-fct} and the \feat{obj-fct}.

\hide{
\begin{figure}
\begin{forest}
[NP{\footnotesize $ \left[\feat{i}=u\right]$}
    [CORE$_\mathrm{N}${\footnotesize $ \left[\feat{i}=u\right]$}
      [{NUC$_\mathrm{N}${\footnotesize $ \left[\feat{i}=u\right]$}},name=bnno
        [{AP},name=b
      [vegan,roof]]
      [NUC$_\mathrm{N}${\footnotesize $ \left[\feat{i}=u\right]$}[dinner,roof]]
        ]
      ]
    ]
\node at (5.0,-3.0){
\avm{$u$[\type{dinner}\\
ev-fct & [\type{eating}\\
theme & \svar{3}]\\
%time & [\type{evening}]
obj-fct & \svar{3}\,[\type{food}\\
comp & [\type{non-meat}]]]}
};
\hide{\begin{scope}[shift={(-14.0,-0.5)}]
\draw (14.0, -6.0)node[draw, circle, label={[label distance=0.0ex]-90:\textit{eating}}](0){x};
  \draw (10.0, -8.0)node[draw, circle, label={[label distance=0.0ex]-180:\textit{food}}](1){y};
%  \draw (10.0, -5.0)node[draw, fill, circle, label={[label distance=-0.5ex]180:$z$}](2){};
  \draw (18.5, -8.5)node[draw, circle, label={[label distance=0.5ex]0:$dinner$}](4){u};
  \draw (17.5, -4.5)node[draw, circle, label={[label distance=0.0ex]0:\textit{evening}}](3){};
%  \draw (18.5, -6.5)node[draw, circle, label={[label distance=0.0ex]0:\textit{lively}}](5){};
  \draw (10.0, -6.0)node[draw, circle, label={[label distance=0.0ex]90:\textit{non-meat}}](6){};
  \path[->] (0) edge node[above, sloped]{\footnotesize\feat{theme}}(1);
%  \path[->,out=90,in=180,looseness=15] (0) edge node[above, sloped]{\footnotesize\feat{ev-fct}}(0);
  \path[->] (4) edge node[above, sloped]{\footnotesize\feat{ev-fct}}(0);
  \path[->] (4) edge node[above, sloped]{\footnotesize\feat{obj-fct}}(1);
  \path[->] (0) edge node[above, sloped]{\footnotesize\feat{time}}(3);
%  \path[->] (0) edge node[above, sloped]{\footnotesize\feat{atmo}}(5);
  \path[->] (1) edge node[above, sloped]{\footnotesize\feat{comp}}(6);
%  \path[->] (2) edge node[below, sloped]{\feat{obj-fct}}(1);
\end{scope}}
\end{forest}
    \caption{\label{fig:vegandinnerresult:Chen}Derived construction for ``vegan dinner'' after final top-bottom unification}
\end{figure}
}

\begin{figure}[t]
\small
\tikzset{>=stealth}
\begin{forest} for tree={s sep=35pt, l sep=2ex}
%  for tree={parent anchor=south, child anchor=north, align=center, l sep=15mm}
  [,phantom[{{NUC$_\mathrm{N}$}$_{\mbox{\small [\feat{i}=$z$]}}^{\phantom{\mbox{\small [\feat{i}=$z$]}}}$},name=coren1
    [AP
      [lively,roof
      ]
    ]
    [{NUC$_\mathrm{N}^*$}$_{\phantom{\mbox{\small [\feat{i}=$t$]}}}^{\mbox{\small [\feat{i}=$t$]}}$]
  ]
[NP
    [CORE$_\mathrm{N}$$_{\mbox{\small [\feat{i}=$\svar{2}$]}}^{\phantom{\mbox{\small [\feat{i}=$\svar{2}$]}}}$
      [{NUC$_\mathrm{N}$$_{\mbox{\small [\feat{i}=$v$]}}^{\mbox{\small [\feat{i}=$\svar{2}$]}}$},name=bnno
        [AP[vegan,roof]]
        [{NUC$_\mathrm{N}$$_{\mbox{\small [\feat{i}=$u$]}}^{\mbox{\small [\feat{i}=$v$]}}$}[dinner,roof]]
        ]
      ]
    ]]
\draw[->,densely dotted] (coren1) to[out=10,in=-190] (bnno);
%\node at (-2.8,-4.0){
%\avm{\3[ev-fct & \1]}
%};
\node at (5.7,-1.5){
\avm{$v$[obj-fct & [\type{food}\\
comp & [\type{non-meat}]]]}
};
\node at (6.0,-3.0){
\avm{$u$[\type{dinner}\\
ev-fct & [\type{eating}\\
theme & \3\\
%time & [\type{evening}]
]\\
obj-fct & \3[\type{food}]]}
};
%\node at (-2.8,-4.5){
%\avm{$t$[ev-fct & $z$]}
%};
\node at (-2.0,-4.0){\small
\avm{$t$[ev-fct & $z$[\type{event}\\
ev-fct & $z$\\
atmo & [\type{lively}]]]}
};
\end{forest}
\caption{\label{fig:livelyvegandinner:Chen}Adjunction of \emph{lively} to \emph{vegan dinner}}
\end{figure}

Now we consider combinations of two \isi{modifiers}. 
\figref{fig:livelyvegandinner:Chen} shows the adjunction of the elementary tree of \emph{lively} to the tree of \emph{vegan dinner} (the latter before final top-bottom unification). Throughout the paper, we follow standard practices in \isi{TAG} by assuming that nothing can be adjoined at foot nodes, therefore `lively' can only be adjoined to the higher NUC$_\mathrm{N}$ node in the `vegan dinner' tree.\footnote{Note that this \textit{null adjunction} at foot nodes is only imposed to avoid spurious ambiguities. Different adjunction orders for the same surface order would lead to the same tree.} %the root NUC$_N$ of ``lively'' can only be adjoined to the root NUC$_N$ of ``vegan dinner''. 
\figref{fig:livelyvegandinnerresult:Chen} %and \figref{fig:livelyvegandinnerfinal} #
 shows the resulting derived construction, both before and after % of ``lively vegan dinner'' before and after
  top-bottom unification.

\begin{figure}[tp]
\small
a) Before final top-bottom unification: \hfill ~
\vspace{-2ex}

\tikzset{>=stealth}
\footnotesize
\begin{forest} for tree={s sep=35pt, l sep=1.8ex}
%  for tree={parent anchor=south, child anchor=north, align=center, l sep=15mm}
[NP
    [CORE$_\mathrm{N}$$_{\mbox{\scriptsize [\feat{i}=$\svar{2}$]}}^{\phantom{\mbox{\scriptsize [\feat{i}=$\svar{2}$]}}}$
      [{NUC$_\mathrm{N}$$_{\mbox{\scriptsize [\feat{i}=$z$]}}^{\mbox{\scriptsize [\feat{i}=$\svar{2}$]}}$}
      [AP[lively,roof]]
      [{NUC$_\mathrm{N}$$_{\mbox{\scriptsize [\feat{i}=$v$]}}^{\mbox{\scriptsize [\feat{i}=$t$]}}$},name=bnno
        [AP[vegan,roof]]
        [{NUC$_\mathrm{N}$$_{\mbox{\scriptsize [\feat{i}=$u$]}}^{\mbox{\scriptsize [\feat{i}=$v$]}}$}[dinner,roof]]
        ]
      ]]
    ]
\node at (-4.9,-1.0){\small
\avm{$v$[obj-fct & [\type{food}\\
comp & [\type{non-meat}]]]}
};
\node at (-5.5,-2.5){\small
\avm{$u$[\type{dinner}\\
ev-fct & [\type{eating}\\
theme & \3\\
%time & [\type{evening}]
]\\
obj-fct & \3[\type{food}]]}
};
%\node at (-3.0,-5.8){
%\avm{$t$[ev-fct & \1]}
%};
\node at (-5.0,-4.3){\small
\avm{$t$[ev-fct & $z$[\type{event}\\
ev-fct & $z$\\
atmo & [\type{lively}]]]}
%\avm{$z$\1[\type{event}\\
%ev-fct & \1[\type{event}\\
%atmo & [\type{lively}]]]}
};
\end{forest}
\medskip

\small
b) After final top-bottom unification: \hfill ~
\vspace{-2ex}

\footnotesize
\begin{forest} for tree={s sep=35pt, l sep=1.6ex}
%  for tree={parent anchor=south, child anchor=north, align=center, l sep=15mm}
[NP
    [CORE$_\mathrm{N}${\scriptsize $ \left[\feat{i}=z\right]$}
      [{NUC$_\mathrm{N}${\scriptsize $ \left[\feat{i}=z\right]$}}
      [AP[lively,roof]]
      [{NUC$_\mathrm{N}${\scriptsize $ \left[\feat{i}=v\right]$}},name=bnno
        [AP[vegan,roof]]
        [{NUC$_\mathrm{N}${\scriptsize $ \left[\feat{i}=v\right]$}}[dinner,roof]]
        ]
      ]]
    ]
\node at (-5.5,-2.5){\small
\avm{$v$[\type{dinner}\\
ev-fct & $z$[\type{eating}\\
ev-fct & $z$\\
atmo & [\type{lively}]\\
theme & \3\\
%time & [\type{evening}]
]\\
obj-fct & \3\,[\type{food}\\
comp & [\type{non-meat}]]]}};
\end{forest}
\caption{\label{fig:livelyvegandinnerresult:Chen}Derived construction for \emph{lively vegan dinner} a) before and b) after final top-bottom unification}%\cl{Is this intermediate result needed? I thought it would be clearer and more detailed to have it.}}
\end{figure}

\hide{\begin{figure}[tt]
\begin{forest} for tree={s sep=35pt}
%  for tree={parent anchor=south, child anchor=north, align=center, l sep=15mm}
[NP
    [CORE$_\mathrm{N}${\footnotesize $ \left[\feat{i}=z\right]$}
      [{NUC$_\mathrm{N}${\footnotesize $ \left[\feat{i}=z\right]$}}
      [AP[lively,roof]]
      [{NUC$_\mathrm{N}${\footnotesize $ \left[\feat{i}=v\right]$}},name=bnno
        [AP[vegan,roof]]
        [{NUC$_\mathrm{N}${\footnotesize $ \left[\feat{i}=v\right]$}}[dinner,roof]]
        ]
      ]]
    ]
\node at (-5.5,-4.5){
\avm{$v$[\type{dinner}\\
ev-fct & $z$[\type{eating}\\
ev-fct & $z$\\
atmo & [\type{lively}]\\
theme & \3\\
%time & [\type{evening}]
]\\
obj-fct & \3[\type{food}\\
comp & [\type{non-meat}]]]}};
\end{forest}
\caption{\label{fig:livelyvegandinnerfinal:Chen}Derived construction for ``lively vegan dinner'' after top-bottom unification}
\end{figure}
}

\hide{\begin{figure}
\begin{forest} for tree={s sep=35pt}
%  for tree={parent anchor=south, child anchor=north, align=center, l sep=15mm}
  [,phantom[{{NUC$_\mathrm{N}$}$_{\mbox{\small [\feat{i}=$v$]}}^{\phantom{\mbox{\small [\feat{i}=$z$]}}}$},name=coren1
    [AP
      [vegan,roof
      ]
    ]
    [{NUC$_\mathrm{N}^*$}$_{\phantom{\mbox{\small [\feat{i}=$t$]}}}^{\mbox{\small [\feat{i}=$v$]}}$]
  ]
[NP
    [CORE$_\mathrm{N}$$_{\mbox{\small [\feat{i}=$\svar{2}$]}}^{\phantom{\mbox{\small [\feat{i}=$\svar{2}$]}}}$
      [{NUC$_\mathrm{N}$$_{\mbox{\small [\feat{i}=$z$]}}^{\mbox{\small [\feat{i}=$\svar{2}$]}}$},name=bnno
        [AP[lively,roof]]
        [{NUC$_\mathrm{N}$$_{\mbox{\small [\feat{i}=$u$]}}^{\mbox{\small [\feat{i}=$t$]}}$}[dinner,roof]]
        ]
      ]
    ]]
\draw[->,densely dotted] (coren1) to[out=50,in=-150] (bnno);
%\node at (-2.8,-4.0){
%\avm{\3[ev-fct & \1]}
%};
\node at (6.0,-2.0){
\avm{v[obj-fct & [\type{food}\\
comp & [\type{non-meat}]]]}
};
\node at (6.0,-3.5){
\avm{$u$[\type{dinner}\\
ev-fct & [\type{eating}\\
theme & \3\\
%time & [\type{evening}]
]\\
obj-fct & \3[\type{food}]]}
};
\node at (-2.5,-5.0){
\avm{$t$[ev-fct & $z$[\type{event}\\
ev-fct & $z$\\
atmo & [\type{lively}]]]}
};
\hide{\node at (-2.8,-4.5){
\avm{$t$[ev-fct & \1]}
};
\node at (-2.5,-5.5){
\avm{$z$\1[\type{event}\\
ev-fct & \1[\type{event}\\
atmo & [\type{lively}]]]}
};}
\end{forest}
\caption{\label{fig:veganlivelydinner:Chen}The adjunction of ``vegan'' to ``lively dinner''}% syntactic and semantic composition of ``lively vegan dinner''}
\end{figure}}

\begin{figure}
\small
\tikzset{>=stealth}
\begin{forest} for tree={s sep=35pt, l sep=2ex}
%  for tree={parent anchor=south, child anchor=north, align=center, l sep=15mm}
[NP
    [CORE$_\mathrm{N}$$_{\mbox{\scriptsize [\feat{i}=$\svar{2}$]}}^{\phantom{\mbox{\scriptsize [\feat{i}=$\svar{2}$]}}}$
      [{NUC$_\mathrm{N}$$_{\mbox{\scriptsize [\feat{i}=$v$]}}^{\mbox{\scriptsize [\feat{i}=$\svar{2}$]}}$}
      [AP[vegan,roof]]
      [\colorbox{black!10}{{\textcolor{black}{NUC$_\mathrm{N}$}$_{\mbox{\scriptsize [\feat{i}=$z$]}}^{\mbox{\scriptsize [\feat{i}=$v$]}}$}},name=bnno
        [AP[lively,roof]]
        [{NUC$_\mathrm{N}$$_{\mbox{\scriptsize [\feat{i}=$u$]}}^{\mbox{\scriptsize [\feat{i}=$t$]}}$}[dinner,roof]]
        ]
      ]]
    ]
%\node at (-2.8,-4.0){
%\avm{\3[ev-fct & \1]}
%};
\node (bla) at (2.5,-1.0){\colorbox{black!10}{\begin{tabular}{c}unification\\ failure\end{tabular}}};
\draw[->, thick, color=black!20] (bla) to (bnno);
\node at (-5.0,-1.0){
\avm{$v$[obj-fct & [\type{food}\\
comp & [\type{non-meat}]]]}
};
\node at (-5.5,-2.5){
\avm{$u$[\type{dinner}\\
ev-fct & [\type{eating}\\
theme & \3\\
%time & [\type{evening}]
]\\
obj-fct & \3\,[\type{food}]]}
};
\node at (-5.0,-4.5){
\avm{$t$[ev-fct & $z$[\type{event}\\
ev-fct & $z$\\
atmo & [\type{lively}]]]}
};
\hide{\node at (-3.0,-5.8){
\avm{$t$[ev-fct & \1]}
};
\node at (-3.5,-6.5){
\avm{$z$\1[\type{event}\\
ev-fct & \1[\type{event}\\
atmo & [\type{lively}]]]}
};}
\end{forest}
\caption{\label{fig:veganlivelydinneralt:Chen}Derived construction for \#``vegan lively dinner'' before top-bottom unification}
\end{figure}

When adjoining the two \isi{modifiers} in the reverse order, i.e., adjoining \emph{vegan} higher than \emph{lively}, %first the tree for `lively' to the `dinner' tree and then the tree for `vegan' to the higher NUC$_\mathrm{N}$ node in the `lively dinner' tree,
 we obtain the derived tree in %Similar to the adjunction of ``lively'' to ``vegan dinner'', the adjunction of ``vegan'' to ``lively dinner'' (displayed as \figref{fig:veganlivelydinner}) leads to the derived tree of ``vegan lively dinner'' (displayed as 
 \figref{fig:veganlivelydinneralt:Chen} (before final top-bottom unification). On the NUC$_\mathrm{N}$ node in the middle in the derived tree (shaded in gray), the top feature structure has an \feat{i} feature with value $v$ and the bottom feature structure has an \feat{i} feature with value $z$. In the final top-bottom unification, when $v$ and $z$ unify, the type \type{event} of $z$ is incompatible with the attribute \feat{obj-fct} of $v$, resulting in a unification failure. Thus, the infelicitous phrase \#\emph{vegan lively dinner} is excluded.

This model also excludes infelicitous verbal predications such as \#\emph{take away the lively dinner}. As shown in \figref{fig:livelydinnerresult:Chen}, the value of the \feat{i} feature of the NP of \emph{lively dinner} is the frame node $z$, which is of type \type{event}. If this NP is substituted at the \isi{object} NP node on the elementary tree of \emph{take away}, the frame $z$ will unify with a frame with \feat{obj-fct} (see \figref{fig:takebook:Chen}), which results in a unification failure.%\cl{Maybe I can put an elementary tree of `take away' as an example in 3.1 so that we don't need `take away lively dinner' tree and frames in 3.2 and 3.3.}

\hide{\begin{figure}
\begin{forest} for tree={s sep=35pt}
%  for tree={parent anchor=south, child anchor=north, align=center, l sep=15mm}
[RP
    [CORE$_\mathrm{R}${\footnotesize $ \left[\feat{i}=z\right]$}
      [{\colorbox{gray!15}{\small [\feat{i}=$z$]}NUC$_\mathrm{R}$},name=bnno
        [{\colorbox{gray!15}{\small [\feat{i}=$z$]}MP\colorbox{gray!15}{\small [\feat{i}=$t$]}},name=b
      [lively,roof]]
      [{\colorbox{gray!15}{\small [\feat{i}=$v$]}MP\colorbox{gray!15}{\small [\feat{i}=$v$]}},name=c
      [vegan,roof]]
      [\colorbox{gray!15}{\small [\feat{i}=$u$]}R[dinner]]
        ]
      ]
    ]
%\draw[->,densely dotted] (coren2) to[out=50,in=-150] (bnno);
%\draw[->,densely dotted] (coren1) to[out=50,in=-150] (bnno);
\node at (-3.8,0.0){
\avm{t[ev-fct & \1]}
};
\node at (-4.0,-1.0){
\avm{z\1[\type{event}\\
ev-fct & \1[\type{event}\\
atmo & [\type{lively}]]]}
};
\node at (-4.0,-6.0){
\avm{v[\type{food}\\
obj-fct & [\type{food}\\
comp & [\type{non-meat}]]]}
};
\node at (2.0,-6.0){
\avm{u[\type{dinner}\\
ev-fct & x[\type{event}\\
theme & \type{y}\\
time & [\type{evening}]]\\
obj-fct & y[\type{food}]]}
};
\begin{scope}[shift={(-15.0,-3.5)}]
\draw (14.0, -6.0)node[draw, circle, label={[label distance=0.0ex]-90:\textit{event}}](0){z};
  \draw (10.0, -8.0)node[draw, circle, label={[label distance=0.0ex]-180:\textit{food}}](1){y};
%  \draw (10.0, -5.0)node[draw, fill, circle, label={[label distance=-0.5ex]180:$z$}](2){};
  \draw (18.5, -8.5)node[draw, circle, label={[label distance=0.5ex]0:$dinner$}](4){u};
  \draw (17.5, -4.5)node[draw, circle, label={[label distance=0.0ex]0:\textit{evening}}](3){};
  \draw (18.5, -6.5)node[draw, circle, label={[label distance=0.0ex]0:\textit{lively}}](5){};
  \draw (10.0, -6.0)node[draw, circle, label={[label distance=0.0ex]90:\textit{non-meat}}](6){};
  \path[->] (0) edge node[above, sloped]{\footnotesize\feat{theme}}(1);
  \path[->,out=90,in=180,looseness=15] (0) edge node[above, sloped]{\footnotesize\feat{ev-fct}}(0);
  \path[->] (4) edge node[above, sloped]{\footnotesize\feat{ev-fct}}(0);
  \path[->] (4) edge node[above, sloped]{\footnotesize\feat{obj-fct}}(1);
  \path[->] (0) edge node[above, sloped]{\footnotesize\feat{time}}(3);
  \path[->] (1) edge node[above, sloped]{\footnotesize\feat{comp}}(6);
  \path[->] (0) edge node[above, sloped]{\footnotesize\feat{atmo}}(5);
%  \path[->] (2) edge node[below, sloped]{\feat{obj-fct}}(1);
\end{scope}
\hide{\begin{scope}[shift={(-14.0,-6.5)}]
\draw (14.0, -6.0)node[draw, fill, circle, label={[label distance=0.0ex]-90:\textit{x,z}}](0){};
  \draw (10.0, -8.0)node[draw, circle, label={[label distance=0.0ex]-180:\textit{y}}](1){};
  \draw (10.0, -6.0)node[draw, circle, label={[label distance=0.0ex]90:\textit{non-meat}}](6){};
%  \draw (10.0, -5.0)node[draw, fill, circle, label={[label distance=-0.5ex]180:$z$}](2){};
  \draw (18.5, -8.5)node[draw, circle, label={[label distance=0.5ex]0:$u,v$}](4){};
  \draw (17.5, -4.5)node[draw, circle, label={[label distance=0.0ex]0:\textit{evening}}](3){};
  \draw (18.5, -6.5)node[draw, circle, label={[label distance=0.0ex]0:\textit{lively}}](5){};
  \path[->] (0) edge node[above, sloped]{\footnotesize\feat{theme}}(1);
  \path[->,out=90,in=180,looseness=15] (0) edge node[above, sloped]{\footnotesize\feat{ev-fct}}(0);
  \path[->] (4) edge node[above, sloped]{\footnotesize\feat{ev-fct}}(0);
  \path[->] (4) edge node[above, sloped]{\footnotesize\feat{obj-fct}}(1);
  \path[->] (0) edge node[above, sloped]{\footnotesize\feat{time}}(3);
  \path[->] (0) edge node[above, sloped]{\footnotesize\feat{atmo}}(5);
  \path[->] (1) edge node[above, sloped]{\footnotesize\feat{comp}}(6);
%  \path[->] (2) edge node[below, sloped]{\feat{obj-fct}}(1);
\end{scope}}
\end{forest}
\caption{\label{fig:livelyvegandinner:Chen}The syntactic and semantic composition of ``lively vegan dinner''}
\end{figure}}

%\figref{fig:livelyvegandinner} shows the syntactic and semantic composition of the phrase ``lively vegan dinner''. The trees and frames of ``lively'' and ``dinner'' are the same as the ones in \figref{fig:livelydinner}, and the tree of ``vegan'' is represented in the middle of the figure. The graph representation below is the final result of the frame unification.
\hide{
\begin{figure}
\begin{forest}
[RP
    [CORE$_\mathrm{R}${\footnotesize $ \left[\feat{i}=z\right]$}
      [{\colorbox{gray!15}{\small [\feat{i}=$z$]}NUC$_\mathrm{R}$},name=bnno
        [{\colorbox{gray!15}{\small [\feat{i}=$z$]}MP},name=b
      [lively,roof]]
      [{MP},name=c
      [vegan,roof]]
      [R[dinner]]
        ]
      ]
    ]
\begin{scope}[shift={(-14.0,-0.5)}]
\draw (14.0, -6.0)node[draw, circle, label={[label distance=0.0ex]-90:\textit{event}}](0){z};
  \draw (10.0, -8.0)node[draw, circle, label={[label distance=0.0ex]-180:\textit{food}}](1){y};
%  \draw (10.0, -5.0)node[draw, fill, circle, label={[label distance=-0.5ex]180:$z$}](2){};
  \draw (18.5, -8.5)node[draw, circle, label={[label distance=0.5ex]0:$dinner$}](4){u};
  \draw (17.5, -4.5)node[draw, circle, label={[label distance=0.0ex]0:\textit{evening}}](3){};
  \draw (18.5, -6.5)node[draw, circle, label={[label distance=0.0ex]0:\textit{lively}}](5){};
  \draw (10.0, -6.0)node[draw, circle, label={[label distance=0.0ex]90:\textit{non-meat}}](6){};
  \path[->] (0) edge node[above, sloped]{\footnotesize\feat{theme}}(1);
  \path[->,out=90,in=180,looseness=15] (0) edge node[above, sloped]{\footnotesize\feat{ev-fct}}(0);
  \path[->] (4) edge node[above, sloped]{\footnotesize\feat{ev-fct}}(0);
  \path[->] (4) edge node[above, sloped]{\footnotesize\feat{obj-fct}}(1);
  \path[->] (0) edge node[above, sloped]{\footnotesize\feat{time}}(3);
  \path[->] (1) edge node[above, sloped]{\footnotesize\feat{comp}}(6);
  \path[->] (0) edge node[above, sloped]{\footnotesize\feat{atmo}}(5);
%  \path[->] (2) edge node[below, sloped]{\feat{obj-fct}}(1);
\end{scope}
\end{forest}
    \caption{\label{fig:livelyvegandinnerresult:Chen}The final construction for ``lively vegan dinner''}
\end{figure}}

%However, for the phrase ``\#vegan lively dinner'', as is shown in \figref{fig:veganlivelydinner}, the \isi{edge feature} on the right side of ``vegan'' will unify with the \isi{edge feature} on the left side of ``lively'', which means frame $z$ unifies with frame $v$. The type of $z$ \type{event} is incompatible with the \feat{obj-fct} in the frame $v$, which will lead to the unification failure of the frames $z$ and $v$. Therefore, the composition of ``\#vegan lively dinner'' is impossible.
\hide{
\begin{figure}
\begin{forest} for tree={s sep=35pt}
%  for tree={parent anchor=south, child anchor=north, align=center, l sep=15mm}
  [,phantom[{NUC$_\mathrm{R}$$^*$},name=coren1
    [{\colorbox{gray!15}{\small [\feat{i}=$v$]}MP\colorbox{gray!15}{\small [\feat{i}=$v$]}}, edge label={node[midway, left, font=\scriptsize]{}}
      [vegan,roof
      ]
    ]
  ]
  [{NUC$_\mathrm{R}$$^*$},name=coren2
    [{\colorbox{gray!15}{\small [\feat{i}=$z$]}MP\colorbox{gray!15}{\small [\feat{i}=\svar{3}]}}, edge label={node[midway, left, font=\scriptsize]{}}
      [lively,roof
      ]
    ]
  ]
[RP
    [CORE$_\mathrm{R}${\footnotesize $ \left[\feat{i}=\svar{2}\right]$}
      [{\colorbox{gray!15}{\small [\feat{i}=\svar{2}]}NUC$_\mathrm{R}$},name=bnno
        [\colorbox{gray!15}{\small [\feat{i}=$u$]}R[dinner]]
        ]
      ]
    ]]
\draw[->,densely dotted] (coren2) to[out=50,in=-150] (bnno);
\draw[->,densely dotted] (coren1) to[out=50,in=-150] (bnno);
\node at (-2.8,-4.0){
\avm{\3[ev-fct & \1]}
};
\node at (-2.0,-5.0){
\avm{z\1[\type{\textcolor{red}{event}}\\
ev-fct & \1[\type{event}\\
atmo & [\type{lively}]]]}
};
\node at (-2.0,-6.5){
\avm{v\4[\type{\textcolor{red}{food}}\\
obj-fct & \4[\type{food}\\
comp & [\type{non-meat}]]]}
};
\hide{\node at (6.0,-3.0){
\avm{u[\type{dinner}\\
ev-fct & x[\type{event}\\
theme & \type{y}\\
time & [\type{evening}]]\\
obj-fct & y[\type{food}]]}
};
\begin{scope}[shift={(-14.0,-2.5)}]
\draw (14.0, -6.0)node[draw, fill, circle, label={[label distance=0.0ex]-90:\textit{x,z}}](0){};
  \draw (10.0, -8.0)node[draw, circle, label={[label distance=0.0ex]-180:\textit{y}}](1){};
  \draw (10.0, -6.0)node[draw, circle, label={[label distance=0.0ex]90:\textit{non-meat}}](6){};
%  \draw (10.0, -5.0)node[draw, fill, circle, label={[label distance=-0.5ex]180:$z$}](2){};
  \draw (18.5, -8.5)node[draw, circle, label={[label distance=0.5ex]0:$u,v$}](4){};
  \draw (17.5, -4.5)node[draw, circle, label={[label distance=0.0ex]0:\textit{evening}}](3){};
  \draw (18.5, -6.5)node[draw, circle, label={[label distance=0.0ex]0:\textit{lively}}](5){};
  \path[->] (0) edge node[above, sloped]{\footnotesize\feat{theme}}(1);
  \path[->,out=90,in=180,looseness=15] (0) edge node[above, sloped]{\footnotesize\feat{ev-fct}}(0);
  \path[->] (4) edge node[above, sloped]{\footnotesize\feat{ev-fct}}(0);
  \path[->] (4) edge node[above, sloped]{\footnotesize\feat{obj-fct}}(1);
  \path[->] (0) edge node[above, sloped]{\footnotesize\feat{time}}(3);
  \path[->] (0) edge node[above, sloped]{\footnotesize\feat{atmo}}(5);
  \path[->] (1) edge node[above, sloped]{\footnotesize\feat{comp}}(6);
%  \path[->] (2) edge node[below, sloped]{\feat{obj-fct}}(1);
\end{scope}}
\end{forest}
\label{fig:veganlivelydinner:Chen}
\caption{The syntactic and semantic composition of ``lively vegan dinner''}
\end{figure}}

\hide{
\begin{figure}
\begin{forest} for tree={s sep=35pt}
%  for tree={parent anchor=south, child anchor=north, align=center, l sep=15mm}
[RP
    [CORE$_\mathrm{R}${\footnotesize $ \left[\feat{i}=\svar{2}\right]$}
      [{\colorbox{gray!15}{\small [\feat{i}=\svar{2}]}NUC$_\mathrm{R}$},name=bnno
        [{\colorbox{gray!15}{\small [\feat{i}=$v$]}MP\colorbox{gray!15}{\small [\feat{i}=\textcolor{blue}{$v$}]}},name=b
      [vegan,roof]]
      [{\colorbox{gray!15}{\small [\feat{i}=\textcolor{blue}{$z$}]}MP\colorbox{gray!15}{\small [\feat{i}=$t$]}},name=c
      [lively,roof]]
      [\colorbox{gray!15}{\small [\feat{i}=$u$]}R[dinner]]
        ]
      ]
    ]
%\draw[->,densely dotted] (coren2) to[out=50,in=-150] (bnno);
%\draw[->,densely dotted] (coren1) to[out=50,in=-150] (bnno);
\hide{\node at (-2.8,-4.0){
\avm{\3[ev-fct & \1]}
};
\node at (-3.0,-5.0){
\avm{z\1[\type{\textcolor{blue}{event}}\\
ev-fct & \1[\type{event}\\
atmo & [\type{lively}]]]}
};}
\node at (-3.0,-6.0){
\avm{v[\textcolor{blue}{obj-fct} & [\type{food}\\
comp & [\type{non-meat}]]]}
};
\node at (3.0,-6.5){
\avm{z[\type{\textcolor{blue}{event}}\\
ev-fct & z[\type{event}\\
theme & y[\type{food}]\\
time & [\type{evening}]\\
atmo & [\type{lively}]]]}
};
\hide{\begin{scope}[shift={(-14.0,-2.5)}]
\draw (14.0, -6.0)node[draw, fill, circle, label={[label distance=0.0ex]-90:\textit{x,z}}](0){};
  \draw (10.0, -8.0)node[draw, circle, label={[label distance=0.0ex]-180:\textit{y}}](1){};
  \draw (10.0, -6.0)node[draw, circle, label={[label distance=0.0ex]90:\textit{non-meat}}](6){};
%  \draw (10.0, -5.0)node[draw, fill, circle, label={[label distance=-0.5ex]180:$z$}](2){};
  \draw (18.5, -8.5)node[draw, circle, label={[label distance=0.5ex]0:$u,v$}](4){};
  \draw (17.5, -4.5)node[draw, circle, label={[label distance=0.0ex]0:\textit{evening}}](3){};
  \draw (18.5, -6.5)node[draw, circle, label={[label distance=0.0ex]0:\textit{lively}}](5){};
  \path[->] (0) edge node[above, sloped]{\footnotesize\feat{theme}}(1);
  \path[->,out=90,in=180,looseness=15] (0) edge node[above, sloped]{\footnotesize\feat{ev-fct}}(0);
  \path[->] (4) edge node[above, sloped]{\footnotesize\feat{ev-fct}}(0);
  \path[->] (4) edge node[above, sloped]{\footnotesize\feat{obj-fct}}(1);
  \path[->] (0) edge node[above, sloped]{\footnotesize\feat{time}}(3);
  \path[->] (0) edge node[above, sloped]{\footnotesize\feat{atmo}}(5);
  \path[->] (1) edge node[above, sloped]{\footnotesize\feat{comp}}(6);
%  \path[->] (2) edge node[below, sloped]{\feat{obj-fct}}(1);
\end{scope}}
\end{forest}
\caption{\label{fig:veganlivelydinner:Chen}The syntactic and semantic composition of \#``vegan lively dinner''}
\end{figure}}

%If the \isi{dot-type} noun has two secondary facets, it will have two elementary trees, where the \feat{i} features equal to each facet respectively.


\subsection{Using default constraints for secondary facets}

A shortcoming of the preceding approach %with top and bottom features
 is that the frame %representation
  of the dot type noun does not distinguish %encode the difference
   between primary and secondary facets. In the following, we aim at modelling this distinction in such a way that the semantic representation of the noun restricts possible facet selection patterns.
To this end, we propose to model secondary facets as \emph{default attributes} that will only be present if there is no conflict with other constraints. 
For example, \type{dinner} is of type  \type{event}\dott\type{food} and usually  has an \feat{ev-fct} and an \feat{obj-fct}. However, as discussed above, when \type{dinner} is modified by an event \isi{modifier} such as `lively', its \feat{obj-fct} is no longer available in the resulting frame. % of the resulting phrase.
 In the following, we model this by analyzing the \feat{obj-fct} of \type{event}\dott\type{food} nodes  %In other words, \feat{obj-fct} counts
 as a defeasible attribute.
 
To this end, we introduce \emph{default constraints} (indicated by $\dimplic$) into our logical system.
Given a default constraint $\alpha\dimplic\beta$ and a frame $F$ that satisfies $\alpha$, we may assume that $F$ also satisfies $\beta$ if $\beta$ is compatible with the properties of $F$.\footnote{A default constraint $\alpha\dimplic\beta$ is basically the same as a normal default rule
$\frac{\alpha:\beta}{\beta}$ in the sense of \citet{Reiter:1980}; see also \citet{Osswald:2005}.}
The assertion that $F$ satisfies $\beta$ is defeasible and needs to be retracted if more information about $F$ is known that is not compatible with $\beta$.

\begin{figure}[t]
\small
\begin{tikzpicture}[scale=0.4, minimum size=1.2ex, inner sep=.3ex,  >=stealth, x=10ex, y=10ex] 
\node at (-8.0, 2.0){%
\xvar{v}\avm{[\type{dinner}\\
ev-fct & [\type{eating}\\
theme & \1\\
time & [\type{evening}]]\\
\textcolor{gray}{obj-fct} & \1 [\type{food}]]}
};
%
\draw (0.0, 0.5)node[draw, circle, label={[label distance=0.0ex]-90:\type{eating}}](0){};
  \draw (3.5, 0.5)node[draw, circle, label={[label distance=0.0ex]-90:\type{food}}](1){};
  \draw (3.5, 3)node[draw, circle, label={[label distance=0.5ex]90:\type{dinner}}](4){\xvar{v}};
  \draw (0, 3.0)node[draw, circle, label={[label distance=0.0ex]90:\type{evening}}](3){};
  \path[->] (0) edge node[above, sloped]{\footnotesize\feat{theme}}(1);
  \path[->] (0) edge node[above, sloped]{\scriptsize\feat{time}}(3);
  \path[->] (4) edge node[above, sloped]{\scriptsize\feat{ev-fct}}(0);
  \path[->,densely dashed] (4) edge node[above, sloped]{\textcolor{gray}{\scriptsize\feat{obj-fct}}}(1);
\end{tikzpicture}
\caption{The AVM and graph representations of the frame of \type{dinner} (default constraints are depicted in gray in AVMs and as dashed edges in graph representations)}
\label{fig:dinner:Chen}
\end{figure}
The facet constraints for \type{event}\dott\type{food} in (\ref{eventfood:Chen}) are now replaced by the following set of constraints, where the \feat{obj-fct} attribute, representing a \isi{secondary facet}, is introduced by the default constraint (\ref{constrc:Chen}).

\ea\label{constraints:Chen}
%\ea\label{constra:Chen}$\type{dinner} \implic \type{event}\dott\type{food}$
%\ea\label{constrb:Chen}$\type{event}\dott\type{food}~\implic~\feat{ev-fct}\CP\feat{theme}\D\type{food}$
\ea\label{constrb:Chen}$\type{event}\dott\type{food}~\implic~\feat{ev-fct}\D\type{eating}$
\ex\label{constrc:Chen}$\type{event}\dott\type{food}~\dimplic~\feat{obj-fct}\D\true$
\ex\label{constrd:Chen}$\type{event}\dott\type{food}\und\feat{obj-fct}\D\true~\implic~\feat{obj-fct}\patheq\feat{ev-fct}\CP\feat{theme}$
\z
\z
%
The minimal frame for an instance of type \type{dinner} under the constraints in (\ref{constraints:Chen}) is given in \figref{fig:dinner:Chen}.
%, is the minimal frame that satisfies the constraints in (\ref{constraints:Chen}). 
The attribute depicted in gray in the AVM and as a dashed edge in the graph (\feat{obj-fct}) is a default attribute. 

In addition, we make use of a type \type{event-active} for marking frames of \isi{facet-picking} \isi{modifiers} such as \emph{lively}, which target event facets.
In particular, \type{event-active} holds at nodes whose event facet has been targeted by a \isi{facet-picking} \isi{modifier} at some point of the composition.
The constraint in (\ref{constre:Chen}) ensures the ``picking'' behavior of the \isi{modifier} in that an instance of both \type{event-active} and \type{event}$\dott$\type{food} cannot have the attribute \feat{obj-fct}.
\ea\label{constre:Chen}
$\type{event-active}\und\type{event}\dott\type{food}\und\feat{obj-fct}\D\top~\implic~\bot$
\z
%Note that it does not necessarily mean that this facet still exists (it might have been a default attribute that was removed due to a unification conflict).
%In a similar vein, we will also use new types such as \type{info-active}:
Likewise, we assume that the presence of \type{event-active} in a lexical entry excludes the presence of \feat{info-fct} when combined with nouns of type $\type{event}\dott\type{info}$ (\ref{jianghuae:Chen}).
There are also types that encode the picking of other facets than \type{event-active}.
For instance, the type \type{info-active} excludes the attribute \feat{ev-fct} when combined with nouns of type $\type{event}\dott\type{info}$ (\ref{jianghuaf:Chen}).
%
\ea
\ea\label{jianghuae:Chen}$\type{event-active}\und\type{event}\dott\type{info}\und\feat{info-fct}\D\top~\implic~\bot$
\ex\label{jianghuaf:Chen}$\type{info-active}\und\type{event}\dott\type{info}\und\feat{ev-fct}\D\top~\implic~\bot$
\z
\z
%
%These types serve to formulate constraints that exclude certain facets for further composition, such as the exclusion of the \feat{info-fct} for a node of type $\type{event}\dott\type{info}$. % and of type \type{jianghua} (see (\ref{jianghuaf:Chen})). 

Facet-addressing \isi{modifiers} do not require specific restrictions of this kind.
%we do not use \type{event-active} for nodes whose event facet has been only targeted by \isi{facet-addressing} \isi{modifiers}.\cl{same as above}
%This is because the event facet in this case is preserved, thus the type of the resulting phrase should not be incompatible with the \feat{obj-fct}.
For instance, \isi{modifiers} such as \emph{vegan} target the \isi{object} facet of \type{event}\dott\type{food} nouns without affecting the event facet.
%,\cl{remove this obj-active in vegan and explain} \type{info-active} etc. 
%means the event facet of the frame is targeted, and this type is used to exclude certain (secondary) facet in certain dot type when the event facet of that dot type is targeted.
%
%\ea\label{dinneravm:Chen}\avm{[\type{dinner}\\
%ev-fct & [\type{event}\\
%theme & \1\\
%time & [\type{evening}]]\\
%\textcolor{red}{obj-fct} & \1 [\type{food}]]}
%\z
%
\figref{fig:vegandinnerframe:Chen} and \figref{fig:livelydinnerunified:Chen} illustrate the representations of \emph{vegan} and \emph{lively}, respectively.
%Modifiers such as ``vegan'' and ``lively'' in \figref{fig:vegandinnerframe} and \figref{fig:livelydinnerunified} select a facet (via an attribute \feat{<x>-fct}) and introduce information about the type or the attributes of the value of that facet. 
%The \feat{<x>-fct} attribute in their representation is a default attribute, and the \isi{facet-picking} \isi{modifiers} also add the type \type{<x>-active} to the modified frame.
%They add the type \type{<x>-active} to the modified frame, and the \feat{<x>-fct} attribute in their representation is a default attribute. 
Note that their facets are modelled as default attributes.
In this way, they can target a \isi{primary facet}, in which case the default attribute in the \isi{modifier} construction turns into a non-default attribute, but also a secondary default facet, in which case the facet remains a default attribute.
Furthermore, they can only combine with frames whose type is compatible with the default facet they address.

\begin{figure}[t]
\small
\tikzset{>=stealth}
\begin{forest} for tree={s sep=35pt, l sep=2ex}
%  for tree={parent anchor=south, child anchor=north, align=center, l sep=15mm}
[,phantom[{NUC$_\mathrm{N}${\footnotesize $ \left[\feat{i}=v\right]$}},name=coren1
    [{AP}, edge label={node[midway, left, font=\scriptsize]{}}
      [vegan,roof
      ]
    ]
    [NUC$_\mathrm{N}$$^*$]
  ]
[NP
    [CORE$_\mathrm{N}${\footnotesize $ \left[\feat{i}=u\right]$}
      [{NUC$_\mathrm{N}${\footnotesize $ \left[\feat{i}=u\right]$}},name=bnno
        [N{\footnotesize $ \left[\feat{i}=u\right]$}[dinner]]
        ]
      ]
    ]]
\draw[->,densely dotted] (coren1) to[out=0,in=-190] (bnno);
%\node at (-2.8,-4.0){
%\avm{\3[ev-fct & \1]}
%};
\node at (-2.0,-4.0){
\avm{\textit{v}[%\type*{obj-active}\\
\textcolor{gray}{obj-fct} & [\type{food}\\
comp & [\type{non-meat}]]]}
};
\node at (5.2,-2.5){
\avm{\textit{u}[\type*{dinner}\\
ev-fct & [\type{eating}\\
theme & \1
%time & [\type{evening}]
]\\
\textcolor{gray}{obj-fct} & \1\,[\type{food}]]}
};
\node at (-3.0,-5.2){Resulting frame:};
\node at (1.0,-6.2){
\avm{\textit{u}[\type*{dinner}\\
ev-fct & [\type{eating}\\
theme & \1
%time & [\type{evening}]
]\\
\textcolor{gray}{obj-fct} & \1\,[\type{food}\\
comp & [\type{non-meat}]]]}
};
\hide{\begin{scope}[shift={(0.0,-8.5)}]
\draw (-3.0, 0.5)node[draw, circle, label={[label distance=0.0ex]-90:\textit{x}}](0){};
  \draw (3.0, 0.5)node[draw, circle, label={[label distance=0.0ex]0:\textit{y}}](1){};
  \draw (0.0, 2)node[draw, fill, circle, label={[label distance=0.5ex]90:$dinner$$\und$\type{obj-active}}](4){};
  \draw (4.0,2.0)node[draw, circle, label={[label distance=0.5ex]90:\textit{non-meat}}](6){};
%  \draw (-6.0, 0.0)node[draw, circle, label={[label distance=0.0ex]90:\textit{lively}}](5){};
%  \path[->] (0) edge[out=30, in=155] node[below, sloped]{\footnotesize\feat{content}}(1);
  \path[->] (0) edge[out=-10, in=-170] node[below, sloped]{\footnotesize\feat{theme}}(1);
  \draw (-4, 2.5)node[draw, circle, label={[label distance=0.0ex]90:\textit{evening}}](3){};
  \path[->] (0) edge node[below, sloped]{\scriptsize\feat{time}}(3);
  \path[->] (4) edge node[above, sloped]{\scriptsize\feat{ev-fct}}(0);
  \path[->] (1) edge node[above, sloped]{\scriptsize\feat{comp}}(6);
  \path[->,dashed] (4) edge node[above, sloped]{\scriptsize\feat{obj-fct}}(1);
\end{scope}}
\end{forest}
\caption{\label{fig:vegandinnerframe:Chen}Derivation for \emph{vegan dinner}}
\end{figure}
%
\figref{fig:vegandinnerframe:Chen} shows the derivation of %the phrase
 \emph{vegan dinner}. The frames are unified under the identification of $u$ and $v$, and the \feat{obj-fct} remains a default attribute. The resulting frame %is described by the AVM in the lower part of the figure. It
  has the same facets as the \type{dinner} frame before modification. %, where \feat{ev-fct} and \feat{obj-fct} are both present.
%
%For non-default frames, the unification of two frames are defined as the minimal frame that subsumes both frames. For example, the unification of the frame of \type{lively} (represented in AVM as (\ref{livelyavm:Chen}) \lk{Why are the root and the event-facet nodes in this frame not identical? They are identical in the figure.}\cl{They are not identical. I copied the avm from the edge features and forgot to remove the \svar{1}} and the non-default part of the frame of \type{dinner} will be the frame represented in graph as \figref{fig:livelydinnerframe}. \lk{But this Fig. shows a unification that involves default constraints I think. I don't understand why the frames are called non-default frames. Anyway, I think we don't have default frames, only default constraints.}\cl{Yes I should make another example instead of trying to reuse the existing figure. This part is going to be before section 3.1}
%\ea\label{livelyavm:Chen}\avm{[\type{event-active}\\
%\textcolor{gray}{event-facet} & [\type{event}\\
%atmo & [\type{lively}]]]}
%\z
%
\begin{figure}[tt]
\small
\tikzset{>=stealth}
a) Derivation step: \hfill ~

\vspace{-1em}

\hspace*{\fill}
\begin{forest} for tree={s sep=35pt, l sep=2ex}
%  for tree={parent anchor=south, child anchor=north, align=center, l sep=15mm}
  [,phantom[{NUC$_\mathrm{N}${\footnotesize $ \left[\feat{i}=z\right]$}},name=coren1
    [{AP}, edge label={node[midway, left, font=\scriptsize]{}}
      [lively,roof
      ]
    ]
    [NUC$_\mathrm{N}$$^*$]
  ]
[NP
    [CORE$_\mathrm{N}${\footnotesize $ \left[\feat{i}=u\right]$}
      [{NUC$_\mathrm{N}${\footnotesize $ \left[\feat{i}=u\right]$}},name=bnno
        [N{\footnotesize $ \left[\feat{i}=u\right]$}[dinner]]
        ]
      ]
    ]]
\draw[->,densely dotted] (coren1) to[out=-10,in=170] (bnno);
%\node at (-2.8,-4.0){
%\avm{\3[ev-fct & \1]}
%};
\node at (-2.0,-3.8){
\avm{\textit{z}[\type*{event-active}\\
\textcolor{gray}{ev-fct} & [\type{event}\\
atmo & [\type{lively}]]]}
};
%\node at (-2.0,-6.5){
%\avm{z[\type{event-active}$\wedge$\type{event}\\
%\textcolor{gray}{event-facet} & z\\
%atmo & [\type{lively}]]}
%};
\node at (4.5,-4.0)%(3.0,0.5)
{
\avm{\textit{u}[\type{dinner}\\
ev-fct & [\type{eating}\\
theme & \1]\\
%time & [\type{evening}]]\\
\textcolor{gray}{obj-fct} & \1\,[\type{food}]]}
};
\hide{
\node at (5.5,-5.0){
\avm{$F_N$[\type{dinner}\\
ev-fct & [\type{eating}\\
theme & [\type{food}]]]}
};}
\hide{\begin{scope}[shift={(0.0,-8.5)}]
\draw (-3.0, 0.5)node[draw, circle, label={[label distance=0.0ex]-90:\textit{event}}](0){x};
  \draw (3.0, 0.5)node[draw, circle, label={[label distance=0.0ex]0:\textit{food}}](1){y};
  \draw (0.0, 2)node[draw, fill, circle, label={[label distance=0.5ex]90:$dinner$$\und$\type{event-active}}](4){};
  \draw (-6.0, 0.0)node[draw, circle, label={[label distance=0.0ex]90:\textit{lively}}](5){};
%  \path[->] (0) edge[out=30, in=155] node[below, sloped]{\footnotesize\feat{content}}(1);
  \path[->] (0) edge[out=-10, in=-170] node[below, sloped]{\footnotesize\feat{theme}}(1);
  \draw (-4, 2.5)node[draw, circle, label={[label distance=0.0ex]90:\textit{evening}}](3){};
  \path[->] (0) edge node[below, sloped]{\scriptsize\feat{time}}(3);
  \path[->] (4) edge node[above, sloped]{\scriptsize\feat{ev-fct}}(0);
  \path[->] (0) edge node[above, sloped]{\scriptsize\feat{atmo}}(5);
%  \path[->,dashed] (4) edge node[above, sloped]{\scriptsize\feat{obj-fct}}(1);
\end{scope}}
\end{forest}
%\caption{\label{fig:livelydinnerframe:Chen}The unification of the tree and frame of ``lively'' and ``dinner'' in default logic%\lk{%The lively avm can be simpler in representation, while containing exactly the same information (see my alternative avm). But I don't understand why event-facet is default here. Events necessarily are their own event-facets, this cannot be deleted I think. So it seems to me that lively does not have a default but a normal event-facet attribute.\\ Furthermore, y is used only for structure sharing, right? so use a boxed number instead.\\
%I don't understand why x is used in the dinner frame, it does not appear in the interface. Please delete.\\
%Finally, all base labels should have a uniform font, which is not the case at the moment.}
%\cl{The event-facet does not have to be identical to the frame itself, because other facets might not be excluded. The event-facet is default because we want it to be consistent with "vegan", where the obj-facet is default (otherwise in "vegan dinner" the obj-facet is not default)}\lk{But in this specific frame, we impose identity. And that probably means that event-facet is not default. Btw., why did we impose identity? I lost track of that.}\cl{Yeah we didn't impose identity. I removed it now.}
%}
%\end{figure}
%\begin{figure}
%\smallskip

b) Resulting construction: \hfill ~
\vspace{-1em}

\hspace*{\fill}
\begin{forest} for tree={s sep=35pt, l sep=2ex}
[NP{\footnotesize $ \left[\feat{i}=u\right]$},name=rp
    [CORE$_\mathrm{N}${\footnotesize $ \left[\feat{i}=u\right]$}
      [{NUC$_\mathrm{N}${\footnotesize $ \left[\feat{i}=u\right]$}},name=bnno
        [{AP}
        [lively,roof]]
        [NUC$_\mathrm{N}${\footnotesize $ \left[\feat{i}=u\right]$}[dinner,roof]]%{\footnotesize $ \left[\feat{i}=u\right]$}[dinner]]]
        ]
      ]
    ]
\node at (5.0,-2.0){
\avm{\textit{u}[\type*{event-active$\und$dinner}\\
ev-fct & [\type{eating}\\
theme & [\type{food}]\\
%time & [\type{evening}]\\
atmo & [\type{lively}]]]
}};
%\node at (5.5,-5.0){
%\avm{u[\type{event-active$\und$dinner$\und$event}\\
%ev-fct & u\\
%theme & [\type{food}]\\
%time & [\type{evening}]\\
%atmo & [\type{lively}]]
%}};
\end{forest}
\caption{\label{fig:livelydinnerunified:Chen}Derivation of the construction for  \emph{lively dinner}}
\end{figure}
%
The frames of `lively' and `dinner' specified in the upper part of \figref{fig:livelydinnerunified:Chen}
are not unifiable because of the constraint in (\ref{constre:Chen}).
%, there will be a conflict between the type \type{dinner}\und\type{event-active} and the default attribute \feat{obj-fct}, which results in a unification failure.
The presence of default constraints, however, allows us to retract defeasible information if necessary.
In the case at hand, the default constraint in (\ref{constrc:Chen}) can be taken out of the game, thereby leading to a `dinner' frame without the \feat{obj-fct} attribute, which then unifies with the `lively' frame without problems, as shown in the lower part of \figref{fig:livelydinnerunified:Chen}.

In general, the decision which defaults to retract is non-deterministic since unification clashes may be resolvable in different ways.
In particular, we can consider both of the frames to be unified as candidates for a retraction of defeasible information, or we may take only one of them into account.
For the present purposes, the latter option turns out to be more appropriate.
The two frames to be unified in a derivation step can be distinguished according to the role their syntactic components play in the derivation.
In our model, only the frame of the element which is ``predicated over'' can be subject to a retraction of defeasible information.
In an adjunction, this is the frame of the target tree, in a substitution, it is the frame of the substitution tree.\footnote{
This ``asymmetric'' approach to unification under default constraints is somewhat related to the ``credulous default unification'' proposed in \citet{carpenter1994skeptical}.
%WRONG: In our model, the frame of the target tree can undergo a possible retraction of defeasible information, not the frame of the tree that is substituted or adjoined.
Carpenter assumes for the default unification of $F$ and $G$ that the information in $F$ is strict and all of $G$ is defeasible.}
%
%However, since the \feat{obj-fct} is a defeasible attribute, it can be eliminated from the frame of \type{dinner}, thereby avoiding the conflict. Formally, we follow the idea of ``credulous default unification'' by \citet{carpenter1994skeptical} and define the default unification of frames $F$ and $G$ (symbolized as $F$$\defunify$$G$) as $F$$\unify$$G'$, where $G_N$$\subsum$$G'$$\subsum$$G$, and $G'$ is the maximal frame such that there is no unification failure. $G_N$ is the \emph{non-default} part of a frame $G$, defined as the minimal frame that satisfies the non-default constraints and symbolized as $F_N$. Concerning \type{dinner}, the non-default part of its frame $u$ is the frame $F_N$ where the \feat{obj-fct} edge is removed. \lk{this is a little confusing, $u$ is a base label, $u'$ not; maybe use $F_u$ and $F'_u$ or something like that?}%the graph in \figref{fig:dinner} without the dashed edge. \lk{mention non-associativity and non-commutativity of our unification operation}\cl{I found examples that our default unification is indeed non-associative, but in this paper we actually don't have cases related to that. The ``lively vegan dinner'' unification order problem is rather $F$$\defunify$($G$$\defunify$$H$)$\neq$$G$$\defunify$($F$$\defunify$$H$), so not really about associative. Do I still need to mention non-associativity?} \lk{yes, I think this is important; are the examples you encountered dot type noun related examples?}\cl{Not really, it's an artificial example. In our case the trees are binary so they always have a specific order. But I will mention that anyways.}\lk{but it is a property that follows from our definition of default unification, right? I think we should then mention it.}
%
%In the default unification of the frames of \type{lively} and \type{dinner}, frame $F_N$ \lk{see above} is also the maximal frame that unifies with $z$ without unification failure. Therefore, the result of the default unification of the frames of \type{lively} and \type{dinner} is the unification of $z$ and $F_N$, which is frame $u$ in the lower half of \figref{fig:livelydinnerunified}, where the \feat{obj-fct} is removed.
%
%It is worth mentioning that the default unification defined in this paper, just like in \citet{carpenter1994skeptical}, is non-commutative and non-associative. \lk{How do we then decide the order of unification? The frame-semantic composition triggered by unification or adjunction does not specify an order. Of course we can define an order based on the derivation steps adjunction, substitution etc., but this needs to be defined.}\cl{Here I defined the order in adjunction. In substitution it doesn't matter because the incompatibility happens with a non-default facet. Do I still need to define an order?}\lk{I think we should define an order in all cases, just to cover everything that could potentially occur, even if it is not needed in actual natural language examples.}%The default unification of frames $F$ and $G$ 
%$F$$\defunify$$G$ may not be the same as $G$$\defunify$$F$. In the case of adjunction and substitution, we define the default unification order as $F_a$$\defunify$$F_t$, where $F_a$ is the value of the \feat{i} feature of the adjunct/substitution node and $F_t$ is the value of the \feat{i} feature of the adjunction/substitution target. \lk{Seems slightly imprecise to me. We are unifying wrt. base labels that unify, right? Otherwise, we would not know which nodes in the two frames we have to identify.}
%So, in the frame of `lively', although the \feat{ev-fct} is a default attribute, it is not regarded as defeasible when `lively' modifies a noun. Therefore, when the frame of `lively' unifies with a frame with a type incompatible with the \feat{ev-fct} (as, e.g., in `lively apple'), there will still be a unification failure.

\hide{\begin{figure}
\begin{forest} for tree={s sep=35pt}
%  for tree={parent anchor=south, child anchor=north, align=center, l sep=15mm}
  %[,phantom
  [{CLAUSE},name=coren1
    [{CORE{\footnotesize $ \left[\feat{i}=e\right]$}} %, edge label={node[midway, left, font=\scriptsize]{}}
      [{RP{\footnotesize $ \left[\feat{i}=a\right]$}}]
      [NUC[take away,roof]]
      [{RP{\footnotesize $ \left[\feat{i}=p\right]$}},name=obj
      [,phantom[RP{\footnotesize $\left[\feat{i}=u\right]$},name=rp [the lively dinner,roof]]]]
      ]
    ]
  ]
%[RP{\footnotesize $\left[\feat{i}=u\right]$},name=rp [lively dinner,roof]]
%]
%    [CORE$_\mathrm{R}${\footnotesize $ \left[\feat{i}=u\right]$}
%      [{NUC$_\mathrm{R}${\footnotesize $ \left[\feat{i}=u\right]$}},name=bnno
%        [{MP}
%        [lively,roof]]
%        [NUC$_\mathrm{R}${\footnotesize $ \left[\feat{i}=u\right]$}[R{\footnotesize $ \left[\feat{i}=u\right]$}[dinner]]
%        ]
%      ]
%    ]]
\draw[->,densely dotted] (rp) to[out=90,in=-90] (obj);
%\node at (-2.8,-4.0){
%\avm{\3[ev-fct & \1]}
%};
\node at (-5.2,-1.0){
\avm{\textit{e}[\type{taking-away}\\
agent & a\\
patient & q]}
};
\node at (-5.2,-2.1){
\avm{\textit{p}[obj-fct & q]}
};
\node at (-2.5,-5.5){
\avm{\textit{u}[\type{event-active$\und$dinner}\\
ev-fct & [\type{eating}\\
theme & [\type{food}]\\
%time & [\type{evening}]\\
atmo & [\type{lively}]]]}
};
\hide{
\node at (-2.0,-4.5){
\avm{z\1[\type{event-active}\\
\textcolor{red}{event-facet} & \1[\type{event}\\
atmo & [\type{lively}]]]}
};
\node at (1.0,0.0){
\avm{\textit{u}[\type{dinner}\\
ev-fct & [\type{eating}\\
theme & \type{y}\\
time & [\type{evening}]]\\
\textcolor{red}{obj-fct} & y[\type{food}]]}
};
\begin{scope}[shift={(0.0,-8.5)}]
\draw (-3.0, 0.5)node[draw, circle, label={[label distance=0.0ex]-90:\textit{x}}](0){};
  \draw (3.0, 0.5)node[draw, circle, label={[label distance=0.0ex]0:\textit{y}}](1){};
  \draw (0.0, 2)node[draw, fill, circle, label={[label distance=0.5ex]90:$dinner$$\und$\type{event-active}}](4){};
  \draw (-6.0, 0.0)node[draw, circle, label={[label distance=0.0ex]90:\textit{lively}}](5){};
%  \path[->] (0) edge[out=30, in=155] node[below, sloped]{\footnotesize\feat{content}}(1);
  \path[->] (0) edge[out=-10, in=-170] node[below, sloped]{\footnotesize\feat{theme}}(1);
  \draw (-4, 2.5)node[draw, circle, label={[label distance=0.0ex]90:\textit{evening}}](3){};
  \path[->] (0) edge node[below, sloped]{\scriptsize\feat{time}}(3);
  \path[->] (4) edge node[above, sloped]{\scriptsize\feat{ev-fct}}(0);
  \path[->] (0) edge node[above, sloped]{\scriptsize\feat{atmo}}(5);
%  \path[->,dashed] (4) edge node[above, sloped]{\scriptsize\feat{obj-fct}}(1);
\end{scope}}
\end{forest}
\caption{\label{fig:takelivelydinner:Chen}Derivation for \#`take away the lively dinner': the unification of $u$ and $p$ fails because of the incompatibility of the type \type{event-active$\und$dinner} and the \feat{obj-fct} attribute.}
\end{figure}

\figref{fig:takelivelydinner} represents the unification failure of the infelicitous phrase \#`take away the lively dinner'. The upper part of the figure represents the elementary construction for the verb `take away'. The frame specifies that the patient of `take away' is the \feat{obj-fct} of the \isi{object} RP. \lk{We decided to go for NP, not RP, right?} When the tree of `lively dinner' substitutes the \isi{object} NP node, the frame $u$ and $p$ unify. According to the constraint (\ref{constre:Chen}), the type of $u$ and the \feat{obj-fct} of $p$ are incompatible, leading to the unification failure.\cl{probably remove the figure and only talk about it in the text}}%\cl{I took away the ``take away the lively dinner'' fig and only talk about it in texts (after vegan lively dinner)}

\begin{figure}
\small
\tikzset{>=stealth}
\begin{forest} for tree={s sep=35pt, l sep=2ex}
%  for tree={parent anchor=south, child anchor=north, align=center, l sep=15mm}
  [,phantom [,phantom [,phantom 
  [{NUC$_\mathrm{N}${\footnotesize $ \left[\feat{i}=v\right]$}},name=coren1
    %[{AP}, edge label={node[midway, left, font=\scriptsize]{}}
      [vegan,roof
      ]
   % ]
    [NUC$_\mathrm{N}$$^*$]
  ]]]
[NP{\footnotesize $ \left[\feat{i}=u\right]$},name=rp
    [CORE$_\mathrm{N}${\footnotesize $ \left[\feat{i}=u\right]$}
      [{NUC$_\mathrm{N}${\footnotesize $ \left[\feat{i}=u\right]$}},name=bnno
        %[{MP}
        [lively dinner,roof]]
        %[NUC$_\mathrm{R}${\footnotesize $ \left[\feat{i}=u\right]$}[R{\footnotesize $ \left[\feat{i}=u\right]$}[dinner]]]
      ]
    ]]
\draw[->,densely dotted] (coren1) to[out=0,in=190] (bnno);
%\draw[->,densely dotted] (coren2) to[out=50,in=-150] (bnno);
\hide{\node at (-2.8,-4.0){
\avm{\3[ev-fct & \1]}
};
\node at (-3.0,-5.0){
\avm{z\1[\type{\textcolor{blue}{event}}\\
ev-fct & \1[\type{event}\\
atmo & [\type{lively}]]]}
};}
\node at (-1.7,-1.8){
\avm{\textit{v}[%\type{obj-act}\\
\textcolor{gray}{obj-fct} & [\type{food}\\
comp & [\type{non-meat}]]]}
};
\node at (5.6,-2.6){
\avm{\textit{u}[\type*{event-active$\und$dinner}\\
ev-fct & [\type{event}\\
theme & [\type{food}]\\
%time & [\type{evening}]\\
atmo & [\type{lively}]]]}
};
\hide{\begin{scope}[shift={(-14.0,-2.5)}]
\draw (14.0, -6.0)node[draw, fill, circle, label={[label distance=0.0ex]-90:\textit{x,z}}](0){};
  \draw (10.0, -8.0)node[draw, circle, label={[label distance=0.0ex]-180:\textit{y}}](1){};
  \draw (10.0, -6.0)node[draw, circle, label={[label distance=0.0ex]90:\textit{non-meat}}](6){};
%  \draw (10.0, -5.0)node[draw, fill, circle, label={[label distance=-0.5ex]180:$z$}](2){};
  \draw (18.5, -8.5)node[draw, circle, label={[label distance=0.5ex]0:$u,v$}](4){};
  \draw (17.5, -4.5)node[draw, circle, label={[label distance=0.0ex]0:\textit{evening}}](3){};
  \draw (18.5, -6.5)node[draw, circle, label={[label distance=0.0ex]0:\textit{lively}}](5){};
  \path[->] (0) edge node[above, sloped]{\footnotesize\feat{theme}}(1);
  \path[->,out=90,in=180,looseness=15] (0) edge node[above, sloped]{\footnotesize\feat{ev-fct}}(0);
  \path[->] (4) edge node[above, sloped]{\footnotesize\feat{ev-fct}}(0);
  \path[->] (4) edge node[above, sloped]{\footnotesize\feat{obj-fct}}(1);
  \path[->] (0) edge node[above, sloped]{\footnotesize\feat{time}}(3);
  \path[->] (0) edge node[above, sloped]{\footnotesize\feat{atmo}}(5);
  \path[->] (1) edge node[above, sloped]{\footnotesize\feat{comp}}(6);
%  \path[->] (2) edge node[below, sloped]{\feat{obj-fct}}(1);
\end{scope}}
\end{forest}
\caption{\label{fig:veganlivelydinnerdefault:Chen}Derivation for \#\emph{vegan lively dinner}: the
 unification failure is due to the incompatibility of %the type
 \type{event-active$\und$dinner} and $\feat{obj-fct}\D\true$.}
\end{figure}

\figref{fig:veganlivelydinnerdefault:Chen} illustrates the unification failure that underlies the infelicitous phrase \#\emph{vegan lively dinner}.
Although the \feat{obj-fct} attribute in the frame for \emph{vegan} goes back to a default constraint, it cannot be retracted because the frame belongs to the adjoining tree and not to the target tree.
Therefore, the incompatibility of the attribute \feat{obj-fct} and %the type
 \type{event-active$\und$dinner} given by (\ref{constre:Chen}) leads to a unification failure.

\begin{figure}
\small
\tikzset{>=stealth}
\begin{forest} for tree={s sep=25pt}
%  for tree={parent anchor=south, child anchor=north, align=center, l sep=15mm}
    [,phantom[,phantom[,phantom[{NUC$_\mathrm{N}${\footnotesize $ \left[\feat{i}=z\right]$}},name=coren1
    %[{AP}
    [lively,roof ]
    %]
    [{NUC$_\mathrm{N}$}$^*$]
  ]]]
[NP{\footnotesize $ \left[\feat{i}=u\right]$},name=rp
    [CORE$_\mathrm{N}${\footnotesize $ \left[\feat{i}=u\right]$}
      [{NUC$_\mathrm{N}${\footnotesize $ \left[\feat{i}=u\right]$}},name=bnno
      [vegan dinner,roof]
%        [{AP}
%        [vegan,roof]]
%        [NUC$_\mathrm{N}${\footnotesize $ \left[\feat{i}=u\right]$}[R{\footnotesize $ \left[\feat{i}=u\right]$}[dinner]]]
        ]
      ]
    ]]
%\draw[->,densely dotted] (coren2) to[out=50,in=-150] (bnno);
\draw[->,densely dotted] (coren1) to[out=-10,in=-190] (bnno);
\node at (-2.2,-1.8){
\avm{\textit{z}[\type*{event-active}\\
\textcolor{gray}{ev-fct} & [\type{event}\\
atmo & [\type{lively}]]]}
};
\node at (5.3,-2.5){
\avm{\textit{u}[\type*{dinner}\\
ev-fct & [\type{event}\\
theme & \1%\\
%time & [\type{evening}]
]\\
\textcolor{gray}{obj-fct} & \1[\type{food}\\
comp & [\type{\dots%non-meat
}]]]}
};
\node at (-3.3,-4.8){\begin{tabular}{l}Resulting frame:\end{tabular}};
\node at (1.0,-5.8){
\avm{\textit{u}[\type*{event-active$\und$dinner}\\
ev-fct & [\type{event}\\
theme & [\type{food}\\
comp & [\type{\dots%non-meat
}]]\\
%time & [\type{evening}]\\
atmo & [\type{lively}]]]}
};
\end{forest}
\caption{\label{fig:livelyvegandinnerdefault:Chen}The derivation of \emph{lively vegan dinner}}
\end{figure}

%\lk{We should state that this binarization serves to enforce a unification order by enforcing an adjunction order. And that we have to look at more examples in order to check whether this always works. It might require additional features on the nodes in order to restrict adjunction order. \\ E.g. `\#the lively dinner on the table' or `d\'elicieux d\^{\i}ner somptueux'}

The successful derivation of the construction for \emph{lively vegan dinner} is shown in \figref{fig:livelyvegandinnerdefault:Chen}. Here, %In this derivation, %Same as in \figref{fig:livelyvegandinner}, nothing can be adjoined to the foot node, so the tree of `lively' can only be adjoined to the upper NUC$_N$ node. Therefore, in the phrase `lively vegan dinner',
the frames of \emph{vegan} and \emph{dinner}  unify first, and then the frame of \emph{lively} unifies with the resulting frame. %Here the syntactic composition is achieved by standard adjunction instead of sister adjunction in Tree Wrapping Grammar \cl{Do I need to mention sister adjunction in this paper?} so that the tree of ``lively vegan dinner'' is binarized and the adjunction order is fixed.\cl{More detailed and precise ($NUC_{NA}$)} Thus the frame of ``vegan'' and ``dinner'' will be unified earlier, and the frame of ``lively'' will unify later with the frame of ``vegan dinner'', so that the unification failure in \figref{fig:veganlivelydinnerdefault} does not happen.
Note that the use of binarized trees and null adjunction constraints on foot nodes is crucial for controlling the unification of frames under default constraints:
Lower \isi{modifiers} must be adjoined before higher ones in order to have the frame of the higher \isi{modifier} unified with the unification of the frames of the lower \isi{modifier} and the noun.
In cases where different \isi{modifiers} adjoin at different sides of a noun (e.g., \#`the lively dinner on the table'), additional features might be necessary in order to enforce the correct adjunction order. 

The described approach can handle V+Mod+N \isi{copredication} constructions as well.
For example, the combination \#\emph{take away the lively dinner} is excluded since 
the \isi{object} NP argument of \emph{take away} is required to have an \feat{obj-fct} attribute in its frame (see \figref{fig:takebook:Chen}) while the type of the frame of \emph{lively dinner} is \type{event-active}\und\type{dinner}, which is incompatible with \feat{obj-fct}.
It follows that substituting \emph{the lively dinner} at the \isi{object} NP argument position of \emph{take away} would lead to a frame unification failure.
%The \feat{i} feature of the \isi{object} NP node equals to a frame with \feat{obj-fct} (see \figref{fig:takebook}), and the type of the frame of `lively dinner' is \type{event-active}\und\type{dinner}, which is incompatible with \feat{obj-fct}. This excludes the possibility of unifying the constructions of `take away' and `the lively dinner'.

Let us finally turn to nouns of type \type{event}\dott\type{info}.
According to the working hypothesis proposed in Section~\ref{sec:event.info:Chen}, both facets of  \type{event}\dott\type{info} nouns are secondary, as long as there are no lexical specifications to the contrary.
This can be formalized by replacing the strict constraint (\ref{eventinfo:Chen}) by the constraints in (\ref{event.info:Chen}).
The existence of both facets relies on the default constraints (\ref{event.info.a:Chen}) and (\ref{event.info.b:Chen}), and a further constraint (\ref{event.info.c:Chen}) ensures the presence of at least one of them.
%
\ea\label{event.info:Chen}
\ea\label{event.info.a:Chen}
$\type{event}\dott\type{info}~\dimplic~\feat{ev-fct}\D\top$
\ex\label{event.info.b:Chen}
$\type{event}\dott\type{info}~\dimplic~\feat{info-fct}\D\top$
\ex\label{event.info.c:Chen}
$\type{event}\dott\type{info}~\implic~\feat{ev-fct}\D\top\vel\feat{info-fct}\D\top$
\ex\label{event.info.d:Chen}
$\type{event}\dott\type{info}\und\feat{ev-fct}\D\top~\implic~\feat{ev-fct}\D\type{info-event}$
%\ex\label{event.info.e:Chen}
%$\type{event}\dott\type{info}\und\feat{info-fct}\D\top~\implic~\feat{info-fct}\D\type{info}$
%\ex\label{event.info.f:Chen}
%$\type{info-event}~\implic~\type{event}\und\feat{cont}\D\type{info}$
\ex\label{event.info.g:Chen}
$\type{event}\dott\type{info}\und\feat{ev-fct}\D\top\und\feat{info-fct}\D\top~\implic~\feat{info-fct}\patheq\feat{ev-fct}\CP\feat{cont}$
\z
\z
%
The values of the two facets, if existent, % have the types shown in (\ref{event.info.d:Chen}), (\ref{event.info.e:Chen}) and (\ref{event.info.f:Chen}), and
are related to each other by the \feat{cont} attribute (\ref{event.info.g:Chen}), as before.

Concerning the two Chinese nouns for `speech' discussed in Section~\ref{sec:event.info:Chen}, it was observed that {\cn 演讲} (y\v{a}nji\v{a}ng) has a lexically specified primary event facet.
One option to implement this is by introducing a specific subtype of $\type{event}\dott\type{info}$ that (strictly) implies $\feat{ev-fct}\D\top$.
Alternatively, we could anchor the constraint $\top\implic\feat{ev-fct}\D\top$ locally to the \isi{dot-type} node of the lexical frame.
Instantiation of the lexical item then leads the non-defeasible presence of $\feat{ev-fct}$.
% which could be expressed by $\type{yanjiang}\implic\type{event}\dott\type{info}\und\feat{ev-fct}\D\top$.
The noun {\cn 讲话} (ji\v{a}nghu\`{a}), by contrast, follows the general pattern of $\type{event}\dott\type{info}$ nouns described in (\ref{event.info:Chen}), according to which both of its facets are defeasible.

%In addition, (\ref{jianghuae:Chen}) and (\ref{jianghuaf:Chen}) encode the fact that 讲话(ji\v{a}nghu\`{a}) allows its \isi{modifiers} to suppress one of its facets. (\ref{jianghuaavm:Chen}) gives the minimal frame  for 讲话(ji\v{a}nghu\`{a}) `speech'  that satisfies all the constraints.
%%For nouns that have two secondary facets such as 讲话(ji\v{a}nghu\`{a}) `speech', both facets are due to the satisfaction of default constraints; cf.\ (\ref{jianghuaa:Chen}) and (\ref{jianghuab:Chen}). In addition, there is a constraint (\ref{jianghuac:Chen}) ensuring that at least one of the facets exists.

%\ea
%%\ea\label{jianghuaa:Chen}$\type{jianghua}~\dimplic~\feat{ev-fct}\D\type{event}\und\feat{ev-fct}\CP\feat{cont}\D\type{info}$
%%\ex\label{jianghuab:Chen}$\type{jianghua}~\dimplic~\feat{info-fct}\D\type{info}\und\feat{info-facet}\CP\feat{occa}\D\type{event}$
%%\ex\label{jianghuac:Chen}$\type{jianghua}~\implic~\feat{ev-fct}\D\top\vel\feat{info-fct}\D\top$
%%\ex\label{jianghuad:Chen}$\type{jianghua}\und\feat{ev-fct}\D\top\und\feat{info-fct}\D\top~\implic~\feat{ev-fct}\CP\feat{cont}\patheq\feat{info-fct}\und\feat{info-fct}\CP\feat{occa}\patheq\feat{ev-fct}$
%\ea\label{jianghuaa:Chen}$\type{jianghua}~\implic~\type{event}\dott\type{info}$
%\ex\label{jianghuae:Chen}$\type{event-active}\und\type{jianghua}\und\feat{info-fct}\D\top~\implic~\bot$
%\ex\label{jianghuaf:Chen}$\type{info-active}\und\type{jianghua}\und\feat{ev-fct}\D\top~\implic~\bot$
%\z
%\z
%
%In the above constraints, (\ref{jianghuaa:Chen}) and (\ref{jianghuab:Chen}) are both default constraints, and (\ref{jianghuac:Chen}) is a normal constraint indicating that at least one of the facets should exist. 
%The frame for 讲话(ji\v{a}nghu\`{a}) `speech' in (\ref{jianghuaavm:Chen}) is the minimal frame that satisfies all the constraints.
%Here \feat{CONT} and \feat{OCCA} \lk{this is a solution Rainer and I also briefly discussed yesterday; I am not sure it works because this does not look like a functional relation. You can have several speeches with the same content.}\cl{Yes conceptually it doesn't look like a functional relation. But if we say in a book like the whole Harry Potter series, there are actually 7 different ``books'' inside one physical book. In that case we actually treat the content of the 7 books together as one single piece of info which is the content of the obj-facet. So similarly can we say if the same content is used in three different times, we treat the 3 speech events together as one single event which is the occasion of the info-facet? Still it sounds a bit far-fetched. And what about making \feat{occa} a relation? This should be a reasonable way to keep the two nodes connected.}\lk{Our frames require that every node is accessible from a base labeled node via functional attributes, relations do not count here.} refers to ``content'' and ``occasion''. \lk{how does it satisfy (\ref{jianghuac:Chen})? Furthermore, I think there are constraints missing, for instance that if there is an ev-fact, this should be of type event with an attribute cont of type info ... As it is now, as soon as we delete any of he two facets, constraint (\ref{jianghuad:Chen}) does not apply any longer ..}\cl{Indeed the avm doesn't tell how to satisfy (\ref{jianghuac:Chen}). Is there a way to show it in avm or graph?}
%\lk{The point is, I think, that in the dinner frame, even after deleting the default edge, all other frame nodes are still reachable from the root via functional attributes. That is not the case here.}\cl{I see the problem. I remember we discussed about the relations between facets in different dot types quite some time ago but I don't remember the conclusions. Anyway I changed the frame and constraints so it is consistent with dinner now. btw do we need a bit more than this avm? I feel like stopping at this avm a bit too short but some more figures might be too much.}

%\ea\label{jianghuaavm:Chen}\avm{[\type*{jianghua}\\
%%\textcolor{gray}{ev-fct} & \2 [\type{event}\\ cont & \1]\\
%%\textcolor{gray}{info-fct} & \1 [\type{info}\\ occa & \2]]
%\textcolor{gray}{ev-fct} & [\type*{info-event}\\ cont & \1\,[\type{info}]]\\
%\textcolor{gray}{info-fct} & \1]}
%\z

%\lk{I am against using this occa arttibute, it is definitely not funcitonal and therefore not really helpful. I am in favor of deleting (\ref{jianghuaavm:Chen}) since it only shows the problem that the "one facet has to exist" constraint is not satisfied here, which is not so important. It is rather a problem of representing this in the avm, which is something we don't want to worry about here. On the other hand, we should point out the problem that with what we have here it can happen that after removing default attributes, there are nodes that are not accessible via attributes from a labeled node.}


\section{Discussion and conclusion}

We presented two approaches to modelling \isi{facet-picking} and \isi{facet-addressing} predications over polysemous nouns: One based on top and bottom feature structures on nodes and the possibility to separate these via adjunction, and a second approach based on a \isi{default logic} that allows the removal of defeasible facets.

The first approach builds on an established framework of the syntax-semantics interface \citep{kallmeyer2013syntax,chen2022frame}. Top and bottom feature structures have long been used in the context of \isi{TAG} (see the analyses in the XTAG grammar, \citealp{XTAG:01}) and have been shown to be useful for a wide range of phenomena. The way frame constraints are unified in this approach is order-independent and monotonic. This facilitates the understanding of the semantic contribution of elementary trees.

This approach can capture in an elegant way the fact that in some predications over \isi{dot-type} nouns, one facet is picked while the other facets, even though still present in the semantic representation, are no longer accessible for further predications. A potential disadvantage of this approach is that the choice of \isi{facet-picking} versus \isi{facet-addressing} depends entirely on the predicate (in our examples on the \isi{modifier}). The data considered above have however shown that the \isi{dot-type} nouns clearly influence the result of such a composition in terms of available facets as well.  A large part of the data suggests that there are \isi{dot-type} nouns that distinguish between primary and secondary facets among their meaning components and that allow facet picking combined with the unavailability of other facets only when the other facets are secondary (see for instance the Chinese data concerning the two words for `speech'.) For these cases, the analysis we have so far might require more than one predication tree for the same lexical item depending on the facet properties of the target noun (which can be encoded via the target type). This would mean a multiplication of predication trees. Note, however, that this might be less problematic than it sounds since these trees would be described in a factorized way within a metagrammar, i.e., a principled description of the possible elementary constructions \citep{Crabbe/etal:2013,kallmeyer2013syntax,Lichte/Petitjean:2015}.

%There is, however, evidence suggesting %On the other hand, there are also cases that suggest
% that whether a predicate selecting a \isi{primary facet} is \isi{facet-picking} %comes with a subsequent unavailability of a \isi{secondary facet}
% depends also on the predicate (beyond its capability to select the \isi{primary facet}) and not only on the \isi{dot-type} noun. Compare for instance %the infelicitous
%  (\ref{ex:veganlivelydinner:Chen}) to (\ref{ex:plant-baed:Chen}). The latter %, which
%   seems more acceptable but shows the same pattern of facet selection involving the same \isi{dot-type} noun. This is easy to model with the top-bottom approach. % than with the \isi{default logic} based approach. 

%\begin{multicols}{2}
%\ea\label{ex:vegan-vs-plant-based:Chen}
%\ea [\#]{vegan lively dinner}\label{ex:veganlivelydinner:Chen}
%\ex []{plant-based early dinner}\label{ex:plant-baed:Chen}
%\z
%\z
%\end{multicols}
%A problem with this approach, however, is that the dot type noun does not specify which of its facets is primary and which is secondary. In other words, only the \isi{modifier} decides whether it is \isi{facet-picking} (e.g., `lively' when modifying a \type{event}\dott\type{food} noun) or \isi{facet-addressing} (e.g., `vegan' in the same context).

%\cl{moved from 3.2; different tree->different frame; add multi-facet}
%In general, in this approach the \isi{edge feature} of the tree of the \isi{modifier} decides whether the noun phrase after modification is still polysemous.
%On the tree of facet-keeping \isi{modifiers}, the \feat{i} features on both sides of the edge have the same value; on the tree of \isi{facet-picking} \isi{modifiers}, the edge features have different values: the value of left \isi{edge feature} is the frame of the adjective and the value of the right \isi{edge feature} is the frame which has the \isi{primary facet}.
%The \isi{modifiers} with j-features \textcolor{red}{j-features??} are \isi{facet-picking} \isi{modifiers} and the value of the j-feature is the facet they pick; the \isi{modifiers} with \feat{i} features on both sides of the edge are facet-keeping \isi{modifiers}. 

%This proposal is suitable for the cases where the \isi{modifier} decides the ``dotness'' of the resulting noun phrase. However, in many cases the``dotness'' is decided by the noun and its primary and secondary facets.
%The problem with this proposal is that when the adjective is \isi{facet-picking} it is usually the noun that decides the ``dotness'' of the noun phrase. 
\hide{If we consider the Chinese \isi{modifier} {\cn 两千字}(li\v{a}ngqi\={a}nz\`{i}) `2000 characters', according to \tabref{tab:speech}, if it modifies {\cn 讲话}(ji\v{a}nghu\`{a}), the noun phrase only has an info-facet, but if it modifies {\cn 演讲}(y\v{a}nji\v{a}ng), the noun phrase is still polysemous. In this case there has to be two trees for {\cn 两千字}(li\v{a}ngqi\={a}nz\`{i}) `2000 characters', one used in the modification of {\cn 讲话}(ji\v{a}nghu\`{a}) and the other used in the modification of {\cn 演讲}(y\v{a}nji\v{a}ng). This means that it is sometimes \isi{facet-picking}, sometimes \isi{facet-addressing}. %does not correspond to the fact that it is only a \isi{facet-picking} \isi{modifier}\lk{why is it only \isi{facet-picking}? I think this example shows that it can be both, \isi{facet-picking} or \isi{facet-addressing}.}
%\cl{maybe I should state it more clearly in section 2: \isi{facet-picking}+dot-type->\isi{secondary facet} gone; the \isi{primary facet} won't disappear. So for yanjiang the event-facet is the \isi{primary facet}, so it won't disappear anyway, no matter whether the \isi{modifier} is facet keeping or facet picking. So here 2000 character doesn't need to be facet-addressing}, and also
 Furthermore, additional information is needed to prevent the \isi{facet-addressing} construction of {\cn 两千字}(li\v{a}ngqi\={a}nz\`{i}) `2000 characters' from being used in the modification of {\cn 讲话}(ji\v{a}nghu\`{a}). This could  be achieved by constraining the types of the modified nouns in the \isi{modifier} frame. But the assumption that there are different constructions for the same \isi{modifier} depending on whether its target facet is primary or not is nevertheless not fully satisfying.
Note, however, that we would assume the grammar to be described in a factorized metagrammar, which means that everything the different \isi{modifier} versions have in common would be described only once and then shared by the two.
}

\hide{Another potential problem with this approach might happen when the dot type noun has more than two facets. As introduced above, `correct annotation' has both \feat{obj-fct} and \feat{info-fct} but the \feat{ev-fct} is not accessible. In order to model this, we might need an \feat{obj-fct} and an \feat{info-fct} in the frame of `correct', even though the \feat{obj-fct} is not targeted.
}

The default approach presented here provides a way to model the distinction between primary and secondary facets in \isi{dot-type} nouns. Secondary facets are modelled as default attributes that can be retracted in case of conflicting frame constraints.
%The latter arise from certain frame type combinations, involving frame types that express that certain facets have been targeted already.
For this analysis we extended the frame logic of \citeauthor{kallmeyer2013syntax} (\citeyear{kallmeyer2013syntax}; cf.\ also \citealt{chen2022frame}) by default constraints and made use of a non-commutative and non-associative notion of default unification.
%Note that such a \isi{default logic} might be useful for modelling other phenomena as well. 

%The default-based analysis predicts that the picking/addressing of facets during predication over a \isi{dot-type} noun depends only on the noun, which seems not always to be the case (see  \#`attend the dinner on the table', where the event facet is not available for `attend' to be predicated over, even though it is assumed to be a \isi{primary facet} of `dinner'). One can probably model such  differences between \isi{modifiers} that target the same facet via their types, but it does not follow naturally from the default-based analysis of \isi{dot-type} nouns. 

 %\ea [\#]{attend the dinner on the table}\label{ex:dinner-table:Chen}
 %\z

%The approach using the \isi{default logic} has the advantage that it allows the reduction of information in the frame representation of meanings, which is useful not only in \isi{copredication} asymmetry in this paper but also potentially in other cases where the meaning of a phrase is non-compositional, such as coercion. It also clearly indicates the defeasibility of secondary facets and the condition when the secondary facets are removed.
An advantage of the default approach is that it requires only a single feature structure per syntactic node, and not top and bottom feature structures. (Note, however, that we might need top and bottom feature structures in order to express other constraints.) %, for example the obligatory adjunction of certain functional operators.)
 On the other hand, a  disadvantage of the default approach is that the non-commutativity and non-associativity of unification make meaning contributions less transparent and impose more constraints on the possible syntactic constructions, compared to the top-bottom approach. 

\hide{This approach also has a few disadvantages. First, the default unification defined in this paper is non-commutative and non-associative, which might not be the ideal case for some constructions and imposes more constraints on the choice of the syntactic side of the model. Second, the semantic types of adjectival \isi{modifiers} have to be different from the modified nouns, e.g. the type of `lively' is \type{event-active} instead of \type{event}. This needs to be justified with more linguistic and ontological evidence. Third, this approach makes the assumption that possible \isi{copredication} patterns concerning the subsequent selection of different facets depends only on the type of the noun, which can lead to the exclusion of default facets when combining with types of the form \type{event-active}, \type{obj-facet} etc. This prediction may encounter some difficulties in exceptional cases, examples were given in (\ref{ex:lunchspeed:Chen}). Another example is the phrase %For example, although the event facet of `dinner' is the \isi{primary facet}, in the phrase
 `the dinner on the table', where the event facet is not available even though it is assumed to be a \isi{primary facet} of `dinner' (e.g. \#`attend the dinner on the table'). This cannot be modelled with the default analysis presented above. It could be modeled with the top-bottom feature analysis because a \isi{modifier} as `on the table' could receive a frame that selects the \isi{object} facet and passes it to the higher NP node. 
%our current frame of ``dinner''.
}

%\lk{Question: It seems to me that the lively vegan dinner data changes a little when replacing "vegan" with "plant-based" and "lively" with "early", which also target obj and event facet respectively. "a plant-based early dinner" does not sound so bad, "an early plant-based dinner" is also fine. This might actually be a problem for the default approach, right? But not for the top-bottom approach.}\cl{Yes. In section 2 I mentioned there are more factors that decide \isi{modifier} orders, but the default approach can only deal with facets. More generally speaking, default approach cannot deal with exceptions easily (e.g. ``dinner on the table'' and 一部书(one episode book); I didn't mention these exceptions in section 2)}

In conclusion, this paper develops two promising options for modelling constraints on copredications over \isi{dot-type} nouns.
%We have discussed some properties and implications of the two approaches.
However, in order to decide which of the two approaches is preferable, further empirical and theoretical investigations are needed. In future work, we will examine more data, taking lexical properties beyond the picking/addressing of facets into account, in particular with respect to the internal structure of the \isi{dot-type} frames, and we will also investigate and model further kinds of syntactic \isi{copredication} constructions. % (as the one in (\ref{ex:dinner-table:Chen}) for instance).  

\section*{Acknowledgments}

This work has been carried out as part of the research project `Coercion and Copredication as Flexible Frame Composition' funded by DFG (German Science Foundation). We would also like to thank the anonymous reviewers for their valuable comments and the participants of the CSSP conference for their suggestions and discussions.

%\lk{to be revised}This paper mainly focuses on the internal properties of the inherently polysemous nouns, while the possibility of \isi{copredication} is decided also by various other factors. For future work more data on the other dot types and in other \isi{copredication} constructions will be investigated to justify the distinction of the primary and secondary facets. Also other factors, such as \isi{copredication} in light verb constructions and exception cases like ``the dinner on the table'' will be discussed and explained. The internal structure of dot type nouns with more than two types, e.g. `school' and `annotation', will also be studied.

\hide{
This paper proposes a distinction between primary meaning facets and secondary meaning facets in inherently polysemous nouns based on the observation of the asymmetry of the possibility of \isi{copredication} in \isi{copredication} constructions V~A~N or A~A~N in English and Mandarin Chinese. When another \isi{meaning facet} is targeted, the \isi{primary facet} will be still available, but the \isi{secondary facet} will not be available if the \isi{modifier} that targets the other facet is a \isi{facet-picking} \isi{modifier}. Three types of inherently polysemous nouns are examined in this paper and their \isi{copredication} patterns suggest they have different internal structures with regard to primary and secondary facets.

%This paper describes the asymmetry in the \isi{copredication} of inherently polysemous nouns and proposes an explanation by a distinction of primary facets and secondary facets in inherently polysemous nouns and a distinction of \isi{facet-addressing} and \isi{facet-picking} \isi{modifiers} in adjectives. A \isi{primary facet} is more prominent than a \isi{secondary facet} in a polysemous noun. Copredication is more likely to happen if the \isi{primary facet} is targeted after the \isi{secondary facet}. In \type{event}\dott\type{food} nouns the event facet is the \isi{primary facet} and the \isi{object} facet is the \isi{secondary facet}. In \type{object}\dott\type{info} nouns both facets are primary facets. In some \type{event}\dott\type{info} nouns both facets are secondary facets. When the \isi{primary facet} is modified by a \isi{facet-picking} \isi{modifier}, the \isi{secondary facet} is no longer available and \isi{copredication} is not possible.
%When both facets are secondary facets, the availability of V+Mod+N \isi{copredication} depends on the adjectival \isi{modifiers}. Copredication can happen only if the \isi{modifier} is a facet-keeping \isi{modifier}.

Moreover, this paper proposes two formal models for the possibility and restrictions on \isi{copredication}. The first approach uses top-bottom feature and distinguishes \isi{facet-picking} adjectives from \isi{facet-addressing} adjectives. The second approach models the difference of primary and secondary facets within the frame representation of inherently polysemous nouns by adding default constraints to indicate the defeasible secondary facets.

The benefits and disadvantages of both proposals are discussed. The first proposal makes use of established framework for the syntax-semantics interface and the unification in this approach is order-independent and monotonic. The disadvantage of this proposal is that the primary and secondary facets of the inherently polysemous nouns are not distinguished. The second proposal allows the secondary facets to be removed under certain conditions and is potentially also useful in other cases such as coercion, when there is a need to remove or overwrite the meanings of the components. The disadvantages of this proposal include the non-commutativeness and non-associativeness of the default unification, and that it is less robust and will have some difficulties in exceptional cases and cases where the type of predicates also affects the possibility of \isi{copredication}.

This paper mainly focuses on the internal properties of the inherently polysemous nouns, while the possibility of \isi{copredication} is decided also by various other factors. For future work more data on the other dot types and in other \isi{copredication} constructions will be investigated to justify the distinction of the primary and secondary facets. Also other factors, such as \isi{copredication} in light verb constructions and exception cases like ``the dinner on the table'' will be discussed and explained. The internal structure of dot type nouns with more than two types, e.g. `school' and `annotation', will also be studied.
}
%\lk{To revise}This paper also introduces the formalization of the primary and secondary facets in \isi{frame semantics} and \isi{RRG}. Two different approaches are discussed. The first one uses edge features in Tree Wrapping Grammar. The edge features in the tree of the \isi{modifier} decides whether the frame of the modified phrase has one or two facets. This approach is good at modelling the difference between \isi{facet-addressing} \isi{modifiers} and \isi{facet-picking} \isi{modifiers}. The second one incorporates \isi{default logic} into the attributive-value logic in \isi{frame semantics}. The attributes indicating the secondary facets are generated by default constraints, and will be cancelled when the conflicting constraints apply. This approach describes better the difference between primary and secondary facets.
\hide{
\section{General typesetting}
\label{sec:basic-typsetting:Chen}

We will enforce compliance with the style rules of LSP, which are based on  \href{https://langsci.github.io/gsr/GenericStyleRulesLangsci.pdf}{the generic style rules for linguistics}, and their in-house \href{https://langsci.github.io/guidelines/latexguidelines/LangSci-guidelines.pdf}{guidelines}.

Make sure to have at look at those before starting!

NB: LSP uses \xelatex{} for compiling, and using pdflatex will give you an error.

AVM for ``2000 words'' (without \type{info-prop}):

\avm{[
\textcolor{gray}{info-facet} & [\type{info}\\
word-count & 2000]]}

AVM for ``yanjiang'':

\avm{[\type{yanjiang}\\
ev-fct & [\type{speech}]\\
\textcolor{gray}{info-facet} & [\type{info}]]}

AVM for ``jianghua'':

\avm{[\type{yanjiang}\\
\textcolor{gray}{event-facet} & [\type{speech}]\\
\textcolor{gray}{info-facet} & [\type{info}]]}

If the type of ``2000 words'' is \type{info} instead of \type{info-prop}, there will be a conflict between the type and the event-facet of ``yanjiang''. If the type of ``2000 words'' is
\subsection{Examples}
\label{sec:examples:Chen}

For examples, we use \texttt{langsci-gb4e}. You can find more involved examples in the showcases section in the \href{https://langsci.github.io/guidelines/latexguidelines/LangSci-guidelines.pdf}{LSP guidelines}.

\is{Cognition} %add "Cognition" to \isi{subject} index for this page
\ea \label{ex:descartes:Chen}
\gll cogito                           ergo      sum\\
     think.\textsc{1sg}.\textsc{pres} therefore \textsc{cop}.\textsc{1sg}.\textsc{pres}\\
\glt `I think therefore I am.'
\z
\il{Latin} %add "Latin" to language index for this page

Make sure you use proper crossreferences (with \verb|label| and \verb|ref|) to refer back to examples, figures and tables. That is, to refer to example (\ref{ex:descartes:Chen}), write \verb|(\ref{ex:descartes:Chen})| and not simply `\verb|(1)|'.
\il{Latin} %add "Latin" to language index for this page

\subsection{Trees}
\label{sec:trees:Chen}
In order to make trees, use the \textbf{forest} package.

% LaTeX doesn't care how you arrange the tree in the source code, but for your own sake, you want probably want to write something as follows:
\ea \label{ex:tree1:Chen}\begin{forest}
[S 
  [DP [D [ Ce\\\footnotesize{This} ] ]
      [NP [AP [A [ petit\\\footnotesize{small} ] ] ]
          [N [ arbre\\\footnotesize{tree} ] ] ] ]
  [VP [V [ est\\\footnotesize{is} ] ]
      [AP [AdvP [Adv [ très\\\footnotesize{very} ] ] ]
          [A [ joli\\\footnotesize{nice} ] ] ] ] ]
\end{forest}
\z

You can also make a more involved example, with arrows, as is illustrated in (\ref{ex:rp-structure:Chen}). Ideally, you would want to avoid big spaces (you can adjust the size of the tree, if this is necessary, as we have done in (\ref{ex:another-tree:Chen}) for (\ref{ex:tree1:Chen})). 

\ea \label{ex:another-tree:Chen}\small{
\begin{forest}
[S 
  [DP [D [ Ce\\\footnotesize{This} ] ]
      [NP [AP [A [ petit\\\footnotesize{small} ] ] ]
          [N [ arbre\\\footnotesize{tree} ] ] ] ]
  [VP [V [ est\\\footnotesize{is} ] ]
      [AP [AdvP [Adv [ très\\\footnotesize{very} ] ] ]
          [A [ joli\\\footnotesize{nice} ] ] ] ] ]
\end{forest}
}
\z

\ea  \label{ex:rp-structure:Chen} 
\footnotesize{%
\begin{forest} for tree={l=10mm, l sep=0}
   [AuxP, name=top 
      [beP [NP
              [DP [John] ] 
              [R$'$ [withP  
                       [RP [DP [hair] ] 
                           [R$'$ [AP [VP ] 
                                     [A [ colored, name=cellar ] ] ] 
                                 [R] ] ] 
                        [with, name=source] ] 
                    [R, name=c1] ] ] 
            [be, name=c2]   ]  
      [Aux [ has ] ] ]
    % and now, we draw some arrows  
   \draw[->,color=red,very thick,dotted] (source) to[out=east,in=south east] (c1);
   \draw[->,color=blue,ultra thick,dashed] (c1) to[out=east,in=south east] (c2);
   \draw [-{Latex[length=2.5mm]},color=purple,ultra thick,%
          postaction={decorate,decoration={text along path,%
          text align=center,raise=-2.5ex, text color = purple,%
          text={{Because why the hell not}}}}] (cellar) to[out=east,in=east] (top);
\end{forest}}
\z

\subsection{AVMs}
\label{sec:avms:Chen}

For avms, we use \textbf{langsci-avm}.

\ea \avm{[ attr1 & \1\\
attr2 & \2[attr3 & val3\\
attr4 & val4] ]}
\z

You can combine \textbf{forest} and \textbf{langsci-avm}:

\ea \begin{forest}
[A [B] [{\avm{[attr1 & val1\\
attr2 & val2\\
attr3 & val3]}} ] ]
\end{forest}l
\z

\subsection{Tables}
\label{sec:tables:Chen}

Make sure to use the following format for tables:


\begin{table}
\caption{Frequencies of word classes}
\label{tab:myname:frequencies:Chen}
 \begin{tabularx}{.8\textwidth}{X rrrr}
  \lsptoprule
            & nouns & verbs  & adjectives & adverbs\\
  \midrule
  absolute  &   12  &    34  &    23      & 13\\
  relative  &   3.1 &   8.9  &    5.7     & 3.2\\
  \lspbottomrule
 \end{tabularx}
\end{table}



\section{Bibliography management}
\label{sec:bibl-manag:Chen}

LSP uses biblatex, rather than the older BibTex. As a consequence, you can use unicode encoding in your BIB file, and you are not restricted to the ASCII-range.

Therefore, you can write directly:

\begin{lstlisting}[language=tex]
title =  {Über die Sprache und Weisheit der Indier},
\end{lstlisting}

rather than the older, escaped version:

\begin{lstlisting}[language=tex]
title =  {\"{U}ber die Sprache und Weisheit der Indier},
\end{lstlisting}

The latter will continue to work.

LSP uses biblatex with the \texttt{natbib} switch, and thus, all \texttt{natbib} commands  should work and behave as expected. You can continue to use the following and more:

\begin{lstlisting}[language=tex]
  \cite{chomsky1995}   % discouraged by natbib
  \citet[12]{chomsky1995}
  \citep[see, e.g.][12]{chomsky1995}
  \citealt[12]{chomsky1995}
  \citealp*{chomsky1995}
\end{lstlisting}

\subsection{No biblatex warnings}

When running BibLaTeX on your BIB file, there should be no warnings or errors whatsoever. Please pay attention to the BibLaTeX compilation log to verify this.

\subsection{Don't abbreviate names}

Every entry in your BIB file should include the names of the authors and editors as they appear on the work in question -- please don't abbreviate any author's or editor's name yourself.

For example, the author of the book \emph{Semantic interpretation in generative grammar} is given as "Ray S. Jackendoff". If you want to cite this book, then the corresponding entry in your BIB file should contain the following author field:

\begin{lstlisting}[language=tex]
author = {Jackendoff, Ray S.},
\end{lstlisting}

In this example, you \emph{shouldn't} abbreviate ''Ray S.'' as ''R. S.'' or ''Ray'' or ''R.''.

\subsection{Don't force unnecessary capitalization in titles}

Since there are different bibliographic styles for how titles of works of different types are capitalized, please don't force unnecessary capitalization in titles of works in your BIB file, for otherwise a bibliographic style (which is determined by a bibliographic style file) won't be able to change this.

For example, if you want to cite the book \emph{The sound pattern of English}, then the corresponding entry in your BIB file should contain the following title field (following the convention known as \href{https://en.wikipedia.org/wiki/Letter_case#Title_case}{Title case}):

\begin{lstlisting}[language=tex]
title = {The Sound Pattern of {English}},
\end{lstlisting}

If the title is given in this way, then ''English'' will be capitalized, whereas ''Sound'' and ''Pattern'' may but need not be capitalized, which allows the bibliographical style file to decide, depending on the chosen bibliographic style. 

If you enclose in brackets only the first letter of the word you wish to capitalize, this will interfere with correct kerning between the first letter and the rest of the word. So, avoid the following:

\begin{lstlisting}[language=tex]
title = {The Sound Pattern of {E}nglish},
\end{lstlisting}

\subsection{References in languages other than English}
\label{sec:refer-lang-other:Chen}

Languages such as French don't have a tradition of title case, and in the case of German, the orthography requires all nouns to be capitalized.

In such cases, do not force the capitalization of all common nouns by enclosing it in brackets, but rather add a special \texttt{LANGID} identifier, and let biblatex do the rest:

\begin{lstlisting}[language=tex]
 title =  {Über die Sprache und Weisheit der Indier},
 langid = {german},
\end{lstlisting}

\begin{lstlisting}[language=tex]
 title =  {Cours de linguistique générale},
 langid = {french},
\end{lstlisting}

\section*{Abbreviations}
\label{sec:abbreviations:Chen}


\begin{tabularx}{.5\textwidth}{@{}lQ@{}}
... & \\
... & \\
\end{tabularx}%
\begin{tabularx}{.5\textwidth}{@{}lQ@{}}
... & \\
... & \\
\end{tabularx}
%}
\section*{Acknowledgements}
\citet{Nordhoff2018} is useful for compiling bibliographies.
%\section*{Contributions}
%John Doe contributed to conceptualization, methodology, and validation. 
%Jane Doe contributed to writing of the original draft, review, and editing.
}
\sloppy
\printbibliography[heading=subbibliography,notkeyword=this]
\end{document}

%%% Local Variables:
%%% mode: xelatex
%%% TeX-master: t
%%% End:
