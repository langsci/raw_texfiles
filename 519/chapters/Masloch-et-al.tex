\documentclass[output=paper,colorlinks,citecolor=brown]{langscibook}
\ChapterDOI{10.5281/zenodo.15450442}
\author{Simon Masloch\orcid{0000-0002-9292-4395}\affiliation{Ruhr-University Bochum} and         Johanna M. Poppek\orcid{0000-0003-0353-1109}\affiliation{Ruhr-University Bochum} and         Tibor Kiss\orcid{0000-0003-1857-3686}\affiliation{Ruhr-University Bochum}}

\title[Reflexive binding into the subject of German experiencer-object verbs]{On the (im-)possibility of reflexive binding into the subject of German experiencer-object verbs}

\abstract{This paper presents an acceptability rating study on the possibility of reflexive binding into the subject of German experiencer-object psych verbs.
Experiencer-object verbs are claimed to license exceptional binding patterns in many languages, but analyses differ in whether they relate this behaviour to a peculiar syntactic structure of the verbs or independently available logophoric binding. An explanation in terms of logophoricity is not viable in German, since the German reflexive \om{sich} does not allow a logophoric interpretation. The study shows that reflexive binding into the subject of German experiencer-object verbs is only possible in the midfield if the antecedent precedes the reflexive in linear order and -- given the binary-branching structure of the midfield -- thus c-commands it. The pattern observed poses a problem for predicate-based theories of binding and it is only explainable if sentence-level constituents in German are base-generated in their surface positions or scrambling does not reconstruct for binding.}

%\newcommand{\orcid}[1]{}

\IfFileExists{../localcommands.tex}{
   \addbibresource{../localbibliography.bib}
   % add all extra packages you need to load to this file

\usepackage{tabularx,multicol}
\usepackage{url}
\urlstyle{same}

\usepackage{listings}
\lstset{basicstyle=\ttfamily,tabsize=2,breaklines=true}

\usepackage{langsci-basic}
\usepackage{langsci-optional}
\usepackage{langsci-lgr}
\usepackage{langsci-osl}
% \usepackage{./langsci/styles/langsci-lgr}
% \usepackage{./langsci/styles/langsci-osl}
% \usepackage{langsci-gb4e}

\usepackage{tikz}
\usetikzlibrary{patterns,calc}
\pgfdeclarepatternformonly{south east lines}{\pgfqpoint{-0pt}{-0pt}}{\pgfqpoint{3pt}{3pt}}{\pgfqpoint{3pt}{3pt}}{
    \pgfsetlinewidth{0.6pt}
    \pgfpathmoveto{\pgfqpoint{0pt}{3pt}}
    \pgfpathlineto{\pgfqpoint{3pt}{0pt}}
    \pgfpathmoveto{\pgfqpoint{.2pt}{-.2pt}}
    \pgfpathlineto{\pgfqpoint{-.2pt}{.2pt}}
    \pgfpathmoveto{\pgfqpoint{3.2pt}{2.8pt}}
    \pgfpathlineto{\pgfqpoint{2.8pt}{3.2pt}}
    \pgfusepath{stroke}}
    
\usepackage{stmaryrd}
\usepackage{wasysym}
\usepackage{multirow}
\usepackage{caption}
\usepackage{subcaption}
\usepackage{mathrsfs}
\usepackage{qtree}

\usepackage{linguex}


   %pminos do not split footnotes
% \interfootnotelinepenalty=10000 %Footnote in Laporte chapters has to be split SN


%\DeclareIndexNameFormat{default}{%
%\nameparts{#1}%
%\usebibmacro{index:name}%
%{\index[names]}%
%{\namepartfamily}%
%{\namepartgiveni}%
% {}% L1
% {}% L2
%{\namepartprefix}% generates spurious space L3
%{\namepartsuffix}% generates spurious space L4
%}

%  {\DeclareIndexNameFormat{default}{%
%     \usebibmacro{index:name}{\index[names]}{#1}{#3}{#5}{#7}}}

%\DeclareIndexNameFormat{default}{%
%  \usebibmacro{index:name}{\sindex[nom]}{#1}{#3}{#5}{#7}}

%\DeclareIndexNameFormat{default}{%
%  \usebibmacro{index:name}{\sindex[person]}{#1}{#3}{#5}{#7}}
%\DeclareIndexNameFormat{default}{%
%\nameparts{#1} \usebibmacro{index:name}{\sindex[person]]}{\namepartfamily}{‌​\namepartgiven}{\nam‌​epartprefix}{\namepa‌​rtsuffix}}

%\newcommand{\smiley}{:)}

%\renewbibmacro*{index:name}[5]{%
%\usebibmacro{index:entry}{#1}%
%{\iffieldundef{usera}{}{\thefield{usera}\actualoperator}\mkbibindexname{#2}{#3}{#4}{#5}}}

% \newcommand{\noop}[1]{}

%remove for final
%\overfullrule=1mm

\newcommand{\tobi}[2]}}
\renewcommand{\S}[1]{\tobi{#1}{\textsc{*}}}

% this volume references
% puts: [this volume]
% already defined: \citetv
%\newcommand{\citepv}[1]{(\citeauthor{#1} \citeyear*{#1} [this volume])}
\newcommand{\citealtv}[1]{\citeauthor{#1} \citeyear*{#1} [this volume]}

%parentheses around example number
\newcommand{\pref}[1]{(\ref{#1})}

% in-text examples

\newcommand{\lnex}[1]{\textit{#1}} %target lang word
\newcommand{\lnlit}[1]{(lit.: `#1')} %literal reading
\newcommand{\lnlat}[1]{(#1)} % latinization
\newcommand{\lntrans}[1]{`#1'} %translation
\newcommand{\lnexl}[2]%
{\lnex{#1}{} \lnlat{#2}} % ex with latinization
\newcommand{\lnexlat}[3]{\lnex{#1}{} \lnlat{#2}{} \lntrans{#3}} % ex with latinization and tranl.

%ch01
\newcommand{\co}[1]{\mbox{\textbf{#1}}}

%ch09

\newcommand{\cyrbulg}[1]{\begin{otherlanguage*}{bulgarian}#1\end{otherlanguage*}}


%ch10
\newcommand{\nlp}{{\small NLP}}
\newcommand{\mwe}{{\small MWE}}
\newcommand{\rae}{{\small RAE}}
\newcommand{\lvc}{{\small LVC}}
\newcommand{\pos}{{\small P}o{\small S}}
%\newcommand{\todo}[1]{ \textcolor{red}{#1} }

%\renewcommand{\labelenumi}{\theenumi}
%\ainamefmt{{vv}{ll}{, ff}{, jj}} % fullname

\newcommand{\biberror}[1]{{\color{red}#1}}

\newcommand{\osenovaitem}{--~}
   %% hyphenation points for line breaks
%% Normally, automatic hyphenation in LaTeX is very good
%% If a word is mis-hyphenated, add it to this file
%%
%% add information to TeX file before \begin{document} with:
%% %% hyphenation points for line breaks
%% Normally, automatic hyphenation in LaTeX is very good
%% If a word is mis-hyphenated, add it to this file
%%
%% add information to TeX file before \begin{document} with:
%% %% hyphenation points for line breaks
%% Normally, automatic hyphenation in LaTeX is very good
%% If a word is mis-hyphenated, add it to this file
%%
%% add information to TeX file before \begin{document} with:
%% \include{localhyphenation}
\hyphenation{
    Beck-man
    Ngu-yen
    back-chan-nel
    back-chan-nels
    mo-not-o-nous
    ste-reo-typ-i-cal
}

\hyphenation{
    Beck-man
    Ngu-yen
    back-chan-nel
    back-chan-nels
    mo-not-o-nous
    ste-reo-typ-i-cal
}

\hyphenation{
    Beck-man
    Ngu-yen
    back-chan-nel
    back-chan-nels
    mo-not-o-nous
    ste-reo-typ-i-cal
}

   \boolfalse{bookcompile}
   \togglepaper[8]%%chapternumber
}{}

\begin{document}
\maketitle

\section{Introduction}
\label{sec:introduction:Masloch}

There is a long-standing debate about so-called \enquote{backward binding} into the \isi{subject} of experiencer-\isi{object} (EO) verbs, i.e. psych-verbs whose \isi{experiencer} is realised as an \isi{object}, \citep[see i.a.][]{belletti_psych-verbs_1988,pollard_anaphors_1992,pesetsky_zero_1995,landau_locative_2010,cheung_psych_2015}.
In \citeauthor{belletti_psych-verbs_1988}'s (\citeyear{belletti_psych-verbs_1988}) Italian examples in (\ref{ex:questipettegolezzi:Masloch}), only in the example containing an EO verb (\ref{ex:questipettegolezzipreoccupano:Masloch}) may the \isi{anaphor} be bound although it is (superficially) not c-commanded by its \isi{antecedent}.

\protectedex{
\ea Italian \citep[312]{belletti_psych-verbs_1988} \label{ex:questipettegolezzi:Masloch}
    \ea [] {\gll Questi pettegolezzi {su di} sé preoccupano Gianni più di ogni altra cosa.\\
                these rumours about \textsc{refl} worry Gianni more than every other matter\\
        \glt ‘These gossips about himself worry Gianni more than anything else.’\label{ex:questipettegolezzipreoccupano:Masloch}
        }
    \ex [*] {\gll Questi pettegolezzi {su di} sé descrivono Gianni meglio di ogni biografia ufficiale.\\
            these rumours about \textsc{refl} describe Gianni better than every biography official \\ 
        \glt ‘These gossips about himself describe Gianni better than any official biography.’ \label{ex:questipettegolezzidescrivono:Masloch}}
    \z
\z
}

Some authors take such examples to provide evidence for the \isi{unaccusativity} of (certain classes of) EO verbs \citep[e.g.][]{belletti_psych-verbs_1988,cheung_psych_2015}. Accordingly, the \isi{subject}/nominative originates in a position below the \isi{object}, such that \isi{c-command} does hold at some point during the derivation. Others claim that such cases represent instances of \isi{logophoric} or point-of-view-based \isi{binding}, a phenomenon that extends beyond the domain of psych verbs \citep[e.g.][]{pollard_anaphors_1992,bouchard_semantics_1995}.

In this paper, we will present evidence from an \isi{acceptability} judgment study that \isi{binding} into the \isi{subject} of EO verbs in the German \isi{midfield} is possible only if the \isi{object} precedes (and given the binary-branching structure of the \isi{midfield} thus c-commands) the \isi{subject} in surface order.
In this regard, German is of special interest for multiple reasons:  
First, the overall \isi{grammaticality} of examples analogous to \pex{ex:questipettegolezzipreoccupano:Masloch} is disputed \citep[cf.][]{kiss_reflexivity_2012,platzack_backward_2012,fischer_theories_2015,temme_backward_2017}, with \citet{fischer_theories_2015} claiming that there is an effect of \isi{linear order}. 
Secondly, despite the widespread assumption that scrambling disables \isi{binding} possibilities in German \citep[and enables new ones, see e.g.][]{haider_mittelfeld_2017}, \citet{temme_backward_2017} claim to have found experimental evidence for \isi{backward binding} with EO verbs in German using examples involving quantificational \isi{binding}.
Thirdly, German does not license \isi{logophoric} \isi{binding} \citep{kiss_reflexivity_2012}, so if \isi{backward binding} \emph{is} possible, a \isi{logophoric} interpretation of the \isi{reflexive} cannot account for it.
In the absence of an explanation relying on \isi{logophoricity}, \isi{unaccusativity} may be suggested to account for backward-\isi{binding} patterns. 
However, what we find is that \isi{binding} into the \isi{subject} of an EO verb is possible in German only if it is \emph{not} backward. 
The patterns observed can be explained by assuming that surface orders of the type A B imply that A asymmetrically c-commands B and that the German \isi{reflexive} \om{sich} requires a c-commanding \isi{antecedent}.
There is thus no need to return to \isi{unaccusativity}, nor does an analysis suggest itself that is based on the concept of (lexical) predicates.

The structure of this paper is as follows: We will introduce some necessary background on German clause structure and \isi{linearisation} as well as on \isi{binding} peculiarities with EO verbs in \sectref{sec:background:Masloch}.
This will lead us to expectations about the \isi{acceptability} of \isi{reflexive} \isi{binding} into the subjects of German EO verbs.
\sectref{sec:study:Masloch} describes the experimental study, the results of which are discussed in \sectref{sec:discussion:Masloch}.
\sectref{sec:conclusion:Masloch} concludes the paper.

\section{Background}
\label{sec:background:Masloch}

We will now briefly discuss some relevant aspects of German syntax and provide some background about \enquote{backward binding} with EO verbs.

\subsection{German clause structure and the unmarked argument order with experiencer-object verbs}
\label{sec:background_German:Masloch}

German is a verb-second language.
The finite verb is placed after the first constituent in matrix clauses, but a verb-final order can be observed in embedded clauses.
Placing a constituent in the pre-field (the area in front of the verb in verb-second clauses) may have interpretational effects \citep{frey_contrast_2006}. 
Thus, in an experimental study all constituents of relevance should -- if possible -- be placed in the so-called \isi{midfield}, i.e. the area between C (the position of the finite verb in verb-second clauses) and the verbal complex at the end of the clause.

Usually different linearisations of constituents in the \isi{midfield} are grammatical, but there is a normal (information-structure-wise most neutral, see \citeauthor{hohle_explikationen_2019} \citeyear{b1ca882e12e014096a64e8713c144a59}/\citeyear{hohle_explikationen_2019}) order that is at least partially dependent on the predicate (we will use the terms \term{normal} and \term{unmarked} interchangeably).
Deviations from the \isi{normal order} outside a licensing context may influence \isi{acceptability} judgments independently of \isi{binding} constraints, making it necessary to consider their effects here.
One prominent approach to German clause structure takes the unmarked order(s) to be base-generated while other orders are derived via scrambling (viewed as movement; see i.a. \citealp{frey_syntaktische_1993,haider_mittelfeld_2017}).
Other approaches favour base-generation of the different orders \parencite[i.a.][]{fanselow_features_2001} or assume a fixed base-generated order and movement, but do not equate it with the unmarked order \parencite[i.a.][]{mueller}.

In the spirit of \textcite{belletti_psych-verbs_1988}, the literature on unmarked \isi{word order} with EO verbs in German usually draws a distinction between those with an accusative \isi{object} and those with a dative \isi{object}.
Although the unmarked order with EO verbs is debated in the literature \parencite[cf. i.a.][]{lenerz_zur_1977,scheepers_linking_2000,haider_scrambling_2003,ellsiepen_constraints_2018}, recent experimental evidence points to a preference for \isi{object} before \isi{subject} with (most) dative-\isi{object} EO verbs and a preference for \isi{subject} before \isi{object} with (most) accusative-\isi{object} EO verbs if subjects are inanimate and all other factors potentially influencing \isi{linear order} are controlled for \citep{temme_verb_2016, masloch_not_2024}.

\textcite{masloch_not_2024} assume \isi{base generation} of sentence-level constituents and violable linear precedence constraints to account for their \isi{linearisation} data.
They treat linear precedence constraints as weighted constraints within Maximum Entropy Grammar \parencite{goldwater_learning_2003}, a probabilistic variant of Optimality Theory.
There is much research on the factors influencing the \isi{linear order} of elements in the German \isi{midfield} \parencite[see i.a.][]{lenerz_zur_1977,uszkoreit_word_1987,hoberg_linearstruktur_1997,keller_gradience_2000,ellsiepen_constraints_2018}.
As \textcite{masloch_not_2024} argue, most accusative-\isi{object} EO verbs have a causer \isi{subject}, while this is not the case for most dative-\isi{object} EO verbs (their \isi{subject} being an \isi{object} of emotion in \citeauthor{pesetsky_zero_1995}'s \citeyear{pesetsky_zero_1995} terms).
They do not assume constraints making reference to case or grammatical function, but (among others) a constraint \lpc{causer}{non-causer}, which places causers before non-causers and has more weight than the constraint \lpc{animate}{inanimate}.
Because in our experimental setting subjects will be inanimate but objects animate, these two constraints will lead to a preference for a \isi{subject} before \isi{object} \isi{linearisation} with accusative-\isi{object} EO verbs and a preference for \isi{object} before \isi{subject} with dative-\isi{object} EO verbs.
Since the constraints are violable, these preferences are not absolute and the reverse order is strictly speaking syntactically well-formed, although it may be less acceptable.
We will follow this account.
Furthermore, we follow \citet{haider_syntax_2010} in assuming a binary-branching structure and the absence of functional projections between V and C.
Thus, the schematic structure in \figref{fig:structural-schema-masloch} emerges.

\begin{figure}
    \centering
    \begin{forest}
        [CP, align = center
            [C, tier = final ]
            [V$'$, align = center
                [$\alpha$, tier = final ]
                [V$'$, align = center 
                    [$\beta$, tier = final ]
                    [V, align = center, tier = final ]
                ]
            ]
        ]
        \end{forest}
    \caption{(Simplified) structural schema of German clauses}
    \label{fig:structural-schema-masloch}
\end{figure}

We will abstract away from certain phenomena, which are orthogonal to our analysis, such as the fronting of (\isi{reflexive}) pronouns, the possibility to place constituents in the prefield, exceptional scope and \isi{binding} options under a rise-fall intonation, and extraposition.
Then, a phrase $\alpha$ dependent on a verbal head precedes another phrase $\beta$ dependent on the same head in \isi{linear order} iff $\alpha$ asymmetrically c-commands $\beta$.
All orders of dependents of a verbal head are strictly speaking syntactically well-formed (unless they violate some other constraint, of course), but not all of them are equally acceptable in every context.

We will base the predictions for our experiment on this view of German clause structure. 
An approach assuming a fixed \isi{base order} plus scrambling conceived as movement to the left \parencite[as e.g. in][]{haider_mittelfeld_2017} will produce the same predictions as long as \isi{binding} constraints are evaluated at the target position only, i.e.,
\begin{inparaenum}[i)]
    \item scrambling does not \isi{reconstruct} for \isi{reflexive} \isi{binding}; and
    \item it is not the case that \isi{binding} constraints can apply at any point.
\end{inparaenum}
 

\subsection{Experiencer-object verbs and \enquote{backward binding}}
\label{sec:log_bind_German:Masloch}

It has been observed for many languages that an \isi{anaphor} contained in the \isi{subject} of an EO verb may precede its experiencer-\isi{object} \isi{antecedent} \parencites[see, among many others, for Italian][]{belletti_psych-verbs_1988}[for English][]{pesetsky_zero_1995}[for Chinese][]{cheung_psych_2015}. %[for Japanese][]{fujita_1993_object}.
Examples like \pex[]{ex:questipettegolezzipreoccupano:Masloch}, repeated here, pose a problem for theories of \isi{binding} that require an \isi{anaphor} to be c-commanded by its \isi{antecedent}.
\begin{exe}
    \exr{ex:questipettegolezzipreoccupano:Masloch} Italian \citep[312]{belletti_psych-verbs_1988}\\
    \gll Questi pettegolezzi {su di} sé preoccupano Gianni più di ogni altra cosa.\\
                these rumours about \textsc{refl} worry Gianni more than every other matter\\
        \glt ‘These gossips about himself worry Gianni more than anything else.’
\end{exe}
The solutions proposed in the literature can be divided into two broader classes: Those that take the \isi{backward binding} pattern to relate to a peculiar syntactic structure of EO verbs (and possibly some wider class, e.g. \citealp{belletti_psych-verbs_1988,pesetsky_zero_1995,cheung_psych_2015}) and those that relate it to \isi{logophoric} or point-of-view-based \isi{binding}, which is available more generally \parencite[e.g.][]{pollard_anaphors_1992,bouchard_semantics_1995}.
On \citegen{belletti_psych-verbs_1988} account of the syntax of psych verbs, the nominative argument of EO verbs is not an external argument but originates in a position where it is c-commanded by the \isi{experiencer}.
It is then assumed that Principle A can be satisfied before the stimulus moves to a position above the \isi{experiencer}.
While analyses that take the (surface) \isi{subject} to originate in a position below the \isi{object} are still widely assumed for dative-\isi{object} EO verbs and sometimes for accusative-\isi{object} EO verbs \citep[see the overview in][]{rozwadowska_psych_2020}, using \isi{backward binding} to argue for it has somewhat fallen out of fashion \citep[see, e.g.][]{landau_locative_2010,hirsch_german_2018}.
This is so because various authors have shown that \isi{backward binding} can be licit even if it is impossible to establish a \isi{c-command} relationship between the \isi{antecedent} and the putative \isi{anaphor} at any given syntactic level \citep[see e.g.][]{pollard_anaphors_1992,bouchard_semantics_1995,cancado_exceptional_1999}.
\pex[]{ex:cancadofranchi28:Masloch} is an illustrative example.
Logophoric \isi{binding} can account for such cases, so one needs to assume it anyway.

\ea Brazilian Portuguese \citep[140]{cancado_exceptional_1999}\\
    \gll Rumores sobre si explicam a insegurança mostrada por João.\\
    rumors about himself explain the insecurity shown by John\\ \label{ex:cancadofranchi28:Masloch}
\z

Logophoric \isi{binding} or exemption from Principle A, however, is not attested with the German \isi{reflexive} \om{sich} \parencite{kiss_reflexivity_2012}.
Picture-NPs do neither allow intersentential \pex{ex:kiss5a:Masloch}, nor non-c-commanding \pex{ex:kiss5b:Masloch}, nor split antecedents of embedded reflexives \pex{ex:kiss5c:Masloch}, unlike in English \parencite[245]{pollard_head-driven_1994}.
The examples in \pex[]{ex:kiss5:Masloch} are not only somewhat degraded but grossly unacceptable on the coindexations given. 

\begin{exe}
    \ex \parencite[158]{kiss_reflexivity_2012} \label{ex:kiss5:Masloch}
    \begin{xlist}
        \ex[*] {\gll Ulrich\textsubscript{i} war sauer. Ein Bild von sich\textsubscript{i} war beschädigt worden.\\
        Ulrich was upset a picture of \textsc{refl} had damaged been\\
        \glt `Ulrich was upset. A picture of himself had been damaged.'\label{ex:kiss5a:Masloch}}
        \ex[*] {\gll [Schumachers\textsubscript{i} Reklamevertrag] verlangte eine Nacktaufnahme von sich\textsubscript{i}.\\
        Schumacher.\textsc{gen} promotion.contract required a nude.photo of \textsc{refl}\\
        \glt `Schumacher's promotion contract required that nude photos of himself be taken.'\label{ex:kiss5b:Masloch}}
        \ex[*] {\gll Ulrich\textsubscript{i} zeigte Klaus\textsubscript{j} einige Bilder von sich\textsubscript{i+j}.\\
        Ulrich showed Klaus some pictures of \textsc{refl}\\
        \glt `Ulrich showed Klaus some pictures of themselves.'\label{ex:kiss5c:Masloch}}
    \end{xlist}
\end{exe}

% \begin{table}[t]
%     \begin{tabularx}{\textwidth}{L{6cm}rr}
%     \lsptoprule
%      & \isi{unaccusative} & not \isi{unaccusative} \\
%     \midrule
%     all orders base-generated or scrambling does not \isi{reconstruct} for \isi{reflexive} \isi{binding} & $\times$ & $\times$ \\
%     fixed base-order + scrambling may \isi{reconstruct} for \isi{reflexive} \isi{binding} & OK & $\times$ \\
%     \lspbottomrule
%     \end{tabularx}
%     \caption{Expectations about the acceptability of reflexive binding into a subject that linearly precedes the object with German EO verbs depending on theoretical assumptions about the verbs (unaccusative: the nominative argument originates in a position c-commanded by the accusative/dative; not unaccusative: this is not the case) and the overall clausal syntax. OK: Should be acceptable; $\times$: Should not be acceptable.}
%     \label{tab:whenisbackwardbindingpredicted:Masloch}
% \end{table}

If \isi{backward binding} with EO verbs is possible only due to exemption, it should thus \emph{not} be possible in German.
If it \emph{is} possible, one could use this fact as an argument for \isi{unaccusativity}:
On the assumption that scrambling reconstructs for \isi{binding} or that a \isi{c-command} requirement can be fulfilled at an early point or at any point in a derivation \parencite[as assumed by i.a.][]{grewendorf_scrambling_1999,mueller}, the \isi{experiencer} could \isi{c-command} the \isi{reflexive} before the latter moves across it.
If (dative) EO verbs have an \isi{unaccusative} structure, \isi{binding} into their \isi{subject} should be possible irrespective of the \isi{linear order} of the arguments.
By contrast, only the orders in which the \isi{object} precedes the \isi{subject} containing the \isi{reflexive} should be grammatical if there is a \isi{c-command} requirement and all orders are base-generated as we assume here, or scrambling as movement destroys \isi{binding} possibilities and creates new ones \parencite[as is frequently assumed, see e.g.][]{haider_mittelfeld_2017}.
%These expectations are summarised in Table~\ref{tab:whenisbackwardbindingpredicted:Masloch}.

The \isi{acceptability} of German examples analogous to \pex{ex:questipettegolezzipreoccupano:Masloch} is disputed in the literature \parencite[cf.][]{kiss_reflexivity_2012,platzack_backward_2012,fischer_theories_2015,temme_backward_2017}. 
\textcite{fischer_theories_2015} claims that there is an effect of \isi{linear order}. 
According to her, \isi{binding} into the \isi{subject} of an EO verb is possible if the \isi{antecedent} object-\isi{experiencer} precedes it, and \pex[]{ex:kiss13d:Masloch} is acceptable to her.

\begin{exe}
    \ex \parencites[161]{kiss_reflexivity_2012}[b. acceptable according to][1390]{fischer_theories_2015}\label{ex:kiss13:Masloch} \begin{xlist}
        \ex[*] {\gll Ich glaube, dass die Bilder von sich den Kindern gefielen.\\
        I believe that the.\textsc{nom} pictures.\textsc{nom} of \textsc{refl} the.\textsc{dat} children.\textsc{dat} appealed.to\\} \label{ex:kiss13c:Masloch}
        \ex[*/\langscicheckmark{}] {\gll Ich glaube, dass den Kindern die Bilder von sich gefielen.\\
        I believe that the.\textsc{dat} children.\textsc{dat} the.\textsc{nom} pictures.\textsc{nom} of \textsc{refl} appealed.to\\
        \glt `I believe that the children liked the pictures of themselves.' \label{ex:kiss13d:Masloch}}
    \end{xlist}
\end{exe}

While \textcite{kiss_reflexivity_2012} judges \pex{ex:kiss13d:Masloch} as unacceptable, it is grammatical on his theory if his assumption that the \isi{subject} must be the last argument to combine with the verbal projection is dropped and the \isi{linear order} of constituents translates to \isi{c-command} in the way we assume here (see Section~\ref{sec:binding:Masloch}).

The only experimental study on \isi{backward binding} with EO verbs in German known to us is \citeauthor{temme_backward_2017}'s (\citeyear{temme_backward_2017}) and it claims that \isi{backward binding} is more acceptable with EO than with action verbs.
For reasons to be discussed below, the authors chose a configuration that does not involve \isi{reflexive}, but quantificational \isi{binding}.
They report the results of two experiments (one for accusative-\isi{object} verbs, one for dative-\isi{object} verbs), in which they compared the \isi{acceptability} of \isi{backward binding} into the subjects of EO verbs and agentive verbs in two conditions: particular and generic.
Since they argue that apparent \isi{binding} possibilities on the latter reading are only illusory, we will focus on the former.
Participants were asked to provide binary \isi{acceptability} judgments.
In both experiments, Temme and Verhoeven found a significant (and non-negligible) effect of verb class to the extent that \isi{backward binding} was more acceptable with EO verbs.
However, the overall \isi{acceptability} within the EO conditions was still not high (30\,\% for accusative-\isi{object}, 40\,\% for dative-\isi{object} EO verbs, which compare to an average \isi{acceptability} of around 20\,\% for Principle-C violations and ca. 83\,\% for backward coreference across their experiments).
\textcite{temme_backward_2017} rightfully argue that it is the observed difference between the conditions in the controlled experiment that counts and that the relatively low acceptance rate does not imply ungrammaticality.
They propose that it may be due to processing difficulties that arise with quantificational \isi{binding} and the backward dependency as well as the fact that the reading of the stimuli they asked their participants to evaluate is not the most prominent one.
Ultimately, they take their findings to show that \isi{backward binding} is a peculiar property of psych verbs in German after all.
Yet, we take it to be possible that other factors are responsible for the effect.
In particular, \textcite{webelhuth_command_2022} recently showed in a \isi{corpus study} that quantificational \isi{binding} in German is possible \emph{without} \isi{c-command}.
He concludes that \enquote{[t]he overall picture that emerges from the corpus evidence is thus that topicality motivates wide scope and scope rather than \isi{c-command} licenses [\dots] bound pronouns} \parencite[387]{webelhuth_command_2022}.
In an article about argument \isi{linearisation}, \textcite{temme_verb_2016} argue that experiencers are more likely to be aboutness-topics than patients.
Thus, it may be the case that \citeauthor{temme_backward_2017}'s (\citeyear{temme_backward_2017}) results are due to the \isi{experiencer} being more likely to be interpreted as the topic than a patient and taking wide scope in turn, which licenses quantificational \isi{binding}.
Since \textcite{webelhuth_command_2022} does not claim \isi{reflexive} \isi{binding} to be licensed by topicality, we consider it preferable to rely on \isi{reflexive} \isi{binding} to test for the availability of \isi{backward binding} in German.

To sum up: The data on the possibility of \isi{backward binding} in German is murky. 
The \isi{acceptability} of pertinent examples is disputed in the literature and the only experimental study finds an effect, but it is weaker than one may expect and may be caused by an independent factor.
Logophoricity is not a factor in German.
The possibility of \isi{binding} into a \isi{subject} preceding the \isi{object} could be explained by assuming \isi{unaccusativity} and a \isi{c-command} condition that either allows \isi{reconstruction} or may be fulfilled at any or an early point.
Both examples in \pex{ex:kiss13:Masloch} should be grammatical then.
If arguments are base-generated in their surface positions or scrambling does not \isi{reconstruct} for \isi{binding}, only \pex{ex:kiss13d:Masloch} should be grammatical.
\textcite{masloch_not_2024} argue that the latter view explains the \isi{linearisation} preferences they observe with German EO verbs less naturally.

A potential problem for an experimental investigation into the possibility of \isi{reflexive} \isi{backward binding} in German pointed out by \citet[286]{temme_backward_2017} concerns the subjects themselves:
Since German lacks a genitive \isi{reflexive}, a \isi{reflexive} can only be embedded in the \isi{subject} within a PP.
However, the usage of such a PP can be functionally overshadowed by a considerably more frequent construction involving a possessive, as in \pex{ex:possessive:Masloch}. 

\ea \parencite[based on an in-text example by][286]{temme_backward_2017} \label{ex:possessive:Masloch}\\
\gll Er betrachtete seine Möbel / \textsuperscript{??}die Möbel von sich.\\
     he beheld    his   furniture {} ~~\,the furniture of \textsc{refl}\\
\glt `He looked at his furniture.'
\z

Such overshadowing may lead to reduced \isi{acceptability} of the stimuli in an experimental setting and should thus be avoided.
However, not all [N [P \textsc{refl}]] structures are equally unacceptable: e.g. \omt{Bilder von sich}{pictures of \textsc{refl}} as in \pex[]{ex:kiss13:Masloch} is not generally unacceptable as shown by sentences like \omt{Warum hat Claude Cahun\textsubscript{i} die Bilder von sich\textsubscript{i} zurückgehalten?}{Why has Claude Cahun withheld the pictures of herself?} \parencite[156]{kiss_reflexivity_2012}.
In this sentence, \omt{ihre Bilder}{her pictures} would mean something like `pictures she made/owns' rather than `pictures of herself / pictures depicting herself':
It is one of the cases often noted in the literature on possessives \parencite[see e.g.][]{barker2019} where the exact relation holding between the possessor and the possessed is provided by context (Claude could e.g. own the pictures or it could be pictures she has taken).
\omt{Bilder}{pictures} being a relational noun, the question arises why the interpretation \omt{}{pictures depicting her} is not salient (or perhaps unavailable) for \om{ihre Bilder}.
We will tentatively assume that the preposition \omt{von}{of} is not devoid of meaning and that NPs containing such PPs compete with NPs containing a possessive.
Which variant is preferred for a given meaning will then depend on the exact literal meaning of the candidates, the factors involved in the competition and general pragmatic principles.
For the purposes of experiments on \isi{reflexive} \isi{binding}, the exact analysis of this phenomenon does not matter as long as there is a way to ensure that the [N [P \textsc{refl}]] structures are not overshadowed by [\textsc{poss} N] structures (but see \sectref{sec:discussion:Masloch}).

%Such functional overshadowing may lead to reduced \isi{acceptability} of the stimuli in an experimental setting, so stimuli where this may happen are to be avoided.
%This is possible since not all combinations of noun, preposition and \isi{reflexive} are equally unacceptable: e.g. \omt{Bilder von sich}{pictures of \textsc{refl}} as in \pex[]{ex:kiss13:Masloch} is not generally unacceptable as shown by sentences like \omt{Warum hat Claude Cahun\textsubscript{1} die Bilder von sich\textsubscript{1} zurückgehalten?}{Why has Claude Cahun withheld the pictures of herself?} \parencite[156]{kiss_reflexivity_2012}.
%In this sentence, \omt{ihre Bilder}{her pictures} would mean `her pictures' (pictures that she took, owns, etc.) rather than `pictures of \textsc{refl}'. 

\section{Experimental Study}
\label{sec:study:Masloch}

We aimed to answer the question if \isi{reflexive} \isi{binding} into the \isi{subject} of EO verbs is possible in German by conducting an \isi{acceptability} rating study.
The study has been preregistered with OSF (\url{https://doi.org/10.17605/OSF.IO/EV7MA}).
All scripts and materials are available via \url{https://doi.org/10.17605/OSF.IO/VNWFQ}.

\subsection{Design}
\label{sec:design:Masloch}

The design reflects the two factors \factor{order} (\level{\isi{subject} before object}, \level{SO} or \level{\isi{object} before subject}, \level{OS}) and \factor{case} (of the \isi{object}, \level{accusative} or \level{dative}).
While \factor{case} is tested between items (since there is no synchronic object-case alternation with EO verbs having a \isi{subject} in German), each item is presented in both ordering conditions.
Participants only see each item in one ordering condition, but each of them rates the same number of \level{SO} and \level{OS} sentences.
Answers are provided on a 5-point scale ranging from \omt{vollkommen unnatürlich}{completely unnatural} to \omt{vollkommen natürlich}{completely natural}.
All points had a natural language name.

\subsection{Materials}
\label{sec:materials:Masloch}

Test items were constructed according to examples (\ref{ex:testitemsacc:Masloch}--\ref{ex:testitemsdat:Masloch}), presented here without any \isi{acceptability} judgment:

\begin{exe}
\ex Accusative \isi{object} \label{ex:testitemsacc:Masloch} 
\begin{xlist}
    \ex Subject before \isi{object}:\\
    \gll Es ist offensichtlich, dass das Gerücht über sich den Professor genervt hat.\\
        it is obvious that the.\textsc{nom} rumour.\textsc{nom} about \textsc{refl} the.\textsc{acc} professor.\textsc{acc} annoyed has\\
    \ex Object before \isi{subject}:\\
    \gll Es ist offensichtlich, dass den Professor das Gerücht über sich genervt hat.\\
        it is obvious that the.\textsc{acc} professor.\textsc{acc} the.\textsc{nom} rumour.\textsc{nom} about \textsc{refl} annoyed has\\
    \glt `It is obvious that the rumour about himself annoyed the professor.'
\end{xlist}
\protectedex{
\ex Dative \isi{object} \label{ex:testitemsdat:Masloch} 
\begin{xlist}
    \ex Subject before \isi{object}:\\
    \gll Es ist allgemein bekannt, dass die Meldung über sich dem Opernsänger gefallen hat.\\
        it is commonly known that the.\textsc{nom} report.\textsc{nom} about \textsc{refl} the.\textsc{dat} opera.singer.\textsc{dat} appealed.to has\\
    \ex Object before \isi{subject}:\\
    \gll Es ist allgemein bekannt, dass dem Opernsänger die Meldung über sich gefallen hat.\\
    it is commonly known that the.\textsc{dat} opera.singer.\textsc{dat} the.\textsc{nom} report.\textsc{nom} about \textsc{refl} appealed.to has\\
    \glt `It is commonly known that the opera singer liked the report about himself.'
\end{xlist}}
\end{exe}

Test items contained the clause of interest embedded in a matrix clause to ensure a verb-final sentence. 
In total, we used eight test items containing an accusative-\isi{object} EO verb and eight test items containing a dative-\isi{object} EO verb, each of them in both \factor{order} variants.\footnote{A complete list of all items used in the study can be found in the OSF directory. Dative-\isi{object} verbs: \omt{auffallen}{to strike}, \omt{behagen}{to please}, \omt{einleuchten}{to be evident}, \omt{gefallen}{to appeal to}, \omt{imponieren}{to impress}, \omt{missfallen}{to displease}, \omt{nahegehen}{to afflict}, \omt{widerstreben}{to have an aversion against};
accusative-\isi{object} verbs: \omt{anekeln}{to sicken}, \omt{ärgern}{to anger}, \omt{ängstigen}{to frighten}, \omt{beeindrucken}{to impress}, \omt{befremden}{to alienate}, \omt{faszinieren}{to fascinate}, \omt{nerven}{to bother}, \omt{verärgern}{to annoy}.} 
We created two lists so that each participant rated only one ordering variant per item.
In addition to the test items, the questionnaires contained 64 filler items, so that each participant judged 80 sentences (plus one sentence in the instructions).
Among the fillers, there were six unannounced calibration items included to familiarise participants with the task and the scale, sixteen control items used to identify uncooperative or distracted participants, and five attention items used to detect inattentive participants.
Filler items varied in expected \isi{acceptability} so that participants would see roughly the same number of clearly acceptable, clearly unacceptable, and somewhat degraded sentences.
Within each subcategory, half of the filler items were related to the test items either by containing \om{sich}, by containing a noun-preposition-noun structure, or by containing a psych verb, while the other half was unrelated.
All items were presented in pseudo-randomised order \isi{subject} to some constraints.\footnote{The presentation of items was constrained as follows: calibration items had to come first, test and control item had to be separated by fillers, controls had to occur in the last 66\,\% of the questionnaire, and the last item had to be a filler.}

The verbs used in the test items were chosen based on their syntactic behaviour in corpus data -- essentially following the procedure of and using the materials from \textcite{masloch_not_2024} -- so that a preference for inanimate subjects, the frequency of non-psych readings and other potential confounding factors were taken into account. 
In all test items, the \isi{subject} was an NP containing an embedded PP whose internal argument was the third person \isi{reflexive} \om{sich}, while the embedded verbs' \isi{object} was the only possible \isi{antecedent} for the \isi{reflexive}.
The noun-preposition sequences are frequent collocates and we ensured that the use of the PP is not overshadowed by a possessive construction (as in \ref{ex:possessive:Masloch}).
In order to do so, nouns that frequently have a preposition as their right neighbour were extracted from DeReKo \parencite{kupietz_German_2010} using KorAP \parencite{diewald_korap_2016}.
From these, 327 nouns were manually chosen. We then calculated collocation scores with 81 prepositions (as direct right neighbours) and possessive pronouns (maximally three words to the left of the noun) for each of them and
chose noun-preposition combinations from the pairs with a high logDice (an association score defined by \citealt{rychly_lexicographer_2008}). Afterwards we manually checked whether the use of an embedded \isi{reflexive} is overshadowed by a possessive construction (cf. \sectref{sec:log_bind_German:Masloch}).
The noun-preposition-\isi{reflexive} combinations used in the items were chosen such that there is no overshadowing in our judgment.
Additionally we avoided psych nouns for potential confounding effects.

\subsection{Participants and Procedure}
\label{sec:procedure:Masloch}
  
Participants (monolingual native speakers of German, residents of Germany, Austria, or Switzerland) were recruited via \textit{Prolific} (\href{https://www.prolific.com}{prolific.com}).
79 participants completed the questionnaire and received a compensation of \textsterling\,3.5.
A typical run lasted ca. 15 minutes.
The experiment was conducted using a web-based infrastructure using jsPsych \citep{deleeuw2015} and JATOS \citep{jatos} on a university server – where participants' individual reaction times were automatically measured.
Taking the control- and attention checks specified in the pre-registration as well as possible topic awareness (checked with an open question at the end of the survey) into account, data from 48 participants was included in the analysis.\footnote{A version of the analysis script where all participants are included is available via the OSF directory and the results are interpretation-wise the same as the ones presented here. The reason for the comparatively large number of participants whose data did not enter the analysis were the rather strict predefined exclusion criteria, namely: 1. the participant guessed the topic of the study correctly or displayed significant linguistic knowledge (we asked participants to guess the topic),
2. the participant did not complete the questionnaire,
3. the participant did not judge at least 80\,\% of the attention items correctly,
4. the participant did not judge at least 80\,\% of the related control items correctly,
5. the participant did not judge at least 80\,\% of the unrelated control items correctly,
6. the participant had unusually long or short answering times as determined by \citegen{pieper_identifying_2023} method,
7. the participant self-reported residing in a country or area where German is not the official language.
The OSF directory contains the script that was used in the exclusion process, where all exclusions are discussed.
We slightly deviated from \citegen{pieper_identifying_2023} criteria for reaction times, which appeared to be too strict given the overall very fast reaction times.
Overall, these criteria are quite strict because participants have to fulfill all of them.
We think that this is a desirable property because participants recruited via web-based participant recruitment platforms tend to rush through studies and the decisive manipulation was rather small and could easily be overlooked.}
After giving their informed consent to participate in the experiment, participants read written instructions asking them to rate how natural the sentences sound to them as sentences of their mother tongue.
They saw an example item on the instructions page.
The experiment started with the six unannounced calibration items.
Each item was presented on its own page together with the answer options.
There was no time limit for providing an answer.

\subsection{Hypotheses and predictions}
\label{sec:hypotheses:Masloch}

As discussed in \sectref{sec:background_German:Masloch}, we follow \citeauthor{masloch_not_2024}'s (\citeyear{masloch_not_2024}) account of argument \isi{linearisation} in the \isi{midfield} for German EO verbs, which takes surface-order at face value: A constituent $α$ in the \isi{midfield} is taken to \isi{c-command} a constituent $β$ if and only if $α$ precedes $β$ in \isi{linear order}.
Combined with the assumption that the German \isi{reflexive} \om{sich} must be c-commanded by its \isi{antecedent}, it follows that sentences in which the \isi{subject} precedes the \isi{object} are ungrammatical (because the \isi{reflexive} cannot be c-commanded by its \isi{antecedent}).\footnote{As mentioned in \sectref{sec:background_German:Masloch}, we do not aim at covering pronoun fronting, which is a different mechanism than the \isi{word order} freedom we look at here \parencite{haider_mittelfeld_2017}. We assume that \om{sich} is fronted in many apparent counterexamples to a \isi{c-command} condition. For our purposes, the precise mechanism behind such cases is irrelevant since in our items the \isi{reflexive} is embedded.}

\begin{exe}
    \ex Main hypothesis \label{ex:mainhypothesis:Masloch}\\
    In the German \isi{midfield}, the \isi{object} of an experiencer-\isi{object} verb cannot bind a \isi{reflexive} embedded in a \isi{subject} preceding it.
\end{exe}

Thus, an item is ungrammatical if the \isi{subject} precedes the \isi{object} in the target clause.
As mentioned in \sectref{sec:log_bind_German:Masloch}, theories of German clausal syntax that take scrambling not to \isi{reconstruct} for \isi{reflexive} \isi{binding} will share this hypothesis irrespective of the base structure assumed.
Sentences in which the \isi{object} precedes the \isi{subject} are strictly speaking grammatical but may violate linear precedence constraints, possibly leading to different degrees of \isi{acceptability}.
An OS \isi{linearisation} will violate \lpc{causer}{non-causer} with (most) accusative-, but not with (most) dative-\isi{object} EO verbs (see \sectref{sec:background_German:Masloch}).
An SO \isi{linearisation} violates \lpc{animate}{inanimate}, which is outweighed by \lpc{causer}{non-causer} with accusative-\isi{object} verbs, but not with dative-\isi{object} verbs.
Thus, without any violations of \isi{binding} constraints OS should be more acceptable with dative-\isi{object} verbs, SO with accusative-\isi{object} verbs.

Based on these prerequisites, we expect dative-\isi{object} verbs to receive high ratings in OS \isi{linearisation} while the same order is \isi{marked} with accusative-\isi{object} verbs, which should result in lower ratings.
In SO order, the \isi{reflexive} is not c-commanded by its \isi{antecedent} and the order is \isi{marked} for dative-\isi{object} verbs. 
However, we may hypothesise that some participants, especially when they try to behave like cooperative discourse-participants (remember that we asked participants to rate how natural the sentences sound to them) or parse the sentences only superficially, will correct the ungrammatical sentence.
Since the resulting OS order is unmarked with dative EO verbs, 
participants may rate the items containing them higher than one would expect based on their theoretical \isi{grammaticality} status.
In contrast, since SO already is the unmarked order with accusative-\isi{object} verbs, this effect is not possible for them. 
Consequently, sentences in this condition should be rated as unnatural.
This state of affairs is summarised in \tabref{tab:predictions:Masloch}.

\begin{table}[t]
    \begin{tabularx}{0.42\textwidth}{lrr}
    \lsptoprule
     & dative & accusative \\
    \midrule
    OS & high & medium \\
    SO & low/medium & low \\
    \lspbottomrule
    \end{tabularx}
    \caption{Predicted acceptability of the test items within the different conditions (see main text for qualification)}
    \label{tab:predictions:Masloch}
\end{table}

Given the model we will use for the analysis (see \sectref{sec:results:Masloch}), our assumptions lead to the following expected effects:\footnote{This is \emph{not} the model mentioned in the pre-registration. 
A reviewer for CSSP suggested that a sum-coded model may be easier to understand than the dummy-coded one we were using originally. 
We think that they are right and only discuss the sum-coded model here.
The dummy-coded model is still available in the analysis script on the OSF directory.}
\begin{exe}
    \ex Expectations fixed effects:
    \begin{itemize}
        \item \factor{Case}: Mildly positive
        \item \factor{Order}: Medium/strong positive
        \item \factor{Case} $\times$ \factor{Order}: marginal or non-existent
    \end{itemize}
\end{exe}

\subsection{Results}
\label{sec:results:Masloch}

\figref{fig:emp_dist:Masloch} displays the empirical distribution of ratings in all four conditions.
\begin{figure}[t]
    \centering
    \includegraphics{../figures/distribution_judgments_binding_subjects_Masloch}
    % Underlying values:
    % case,order,ANSWER,count
    % dative,\isi{object} < \isi{subject},1,27
    % dative,\isi{object} < \isi{subject},2,55
    % dative,\isi{object} < \isi{subject},3,19
    % dative,\isi{object} < \isi{subject},4,56
    % dative,\isi{object} < \isi{subject},5,35
    % dative,\isi{subject} < \isi{object},1,68
    % dative,\isi{subject} < \isi{object},2,84
    % dative,\isi{subject} < \isi{object},3,15
    % dative,\isi{subject} < \isi{object},4,18
    % dative,\isi{subject} < \isi{object},5,7
    % accusative,\isi{object} < \isi{subject},1,44
    % accusative,\isi{object} < \isi{subject},2,57
    % accusative,\isi{object} < \isi{subject},3,26
    % accusative,\isi{object} < \isi{subject},4,50
    % accusative,\isi{object} < \isi{subject},5,15
    % accusative,\isi{subject} < \isi{object},1,69
    % accusative,\isi{subject} < \isi{object},2,65
    % accusative,\isi{subject} < \isi{object},3,27
    % accusative,\isi{subject} < \isi{object},4,27
    % accusative,\isi{subject} < \isi{object},5,4
    \caption{Empirical distribution of ratings. \enquote{5} stands for \enquote{completely natural}, \enquote{1} for \enquote{completely unnatural}. All choice options were presented to the participants with a natural language label.}
    \label{fig:emp_dist:Masloch}
\end{figure}
We see that sentences in which the \isi{subject} containing the \isi{reflexive} precedes the \isi{object} (condition \level{SO}) received very low ratings, although ratings (unexpectedly) improve slightly with accusative-\isi{object} verbs.
In the \level{OS} condition, in which the \isi{reflexive} is preceded by its \isi{antecedent}, sentences receive overall better judgments, although there is still a large number of lower ratings.
Ratings are higher with dative-\isi{object} verbs there.
These descriptive results speak in favour of the main hypothesis \pex[]{ex:mainhypothesis:Masloch}.

To model the data and test our hypotheses, we fitted a Bayesian cumulative logit generalised linear mixed model with flexible thresholds using the \texttt{brms} package \citep{burkner_brms_2017} in R \citep{r_core_team_r_2020}.
Cumulative models are a type of model that can be used to analyse ordinal data (see \citealt{buerkner_ordinal_2019} for an introduction): 
Since one cannot assume that the intervals between response options of Likert items have equal size, it is not appropriate to use a metric model \parencite[see i.a.][]{liddell_2018_analyzing}.
Both factors were sum-coded with \level{dative} and \level{OS} coded $1$ and \level{accusative} and \level{SO} coded $-1$.
The model includes fixed effects for \factor{case}, \factor{order}, and their interaction, varying intercepts for participants and items, a varying slope for \factor{order} for items, and varying slopes for \factor{case}, \factor{order}, and their interaction for participants, as well as all possible correlation parameters between them.\footnote{The analysis script on the OSF directory contains several additional models as well as a comparison between them. We will focus on this model here.}
% Bayesian cumulative generalised mixed models differ from vanilla linear regression models in several ways: 
% In a mixed model, the strength of effects may vary per e.g. participant or item.
% In a generalised mixed model, a link function is applied to the linear predictor.
% Cumulative models are a type of model that can be used to analyse ordinal data (see \citealt{buerkner_ordinal_2019} for an introduction): 
% Since one cannot assume that the intervals between response options of Likert items have equal size, it is not appropriate to use a metric model \parencite[see i.a.][]{liddell_2018_analyzing}.
% In a cumulative model, the response is taken to relate to a latent variable (in our case: perceived naturalness) that can be modelled as linear and is partitioned into ordered bins corresponding to the response options via thresholds that are estimated in the model.
% The probability that a response option is chosen then depends on the linear predictor and the thresholds.
% We use a logit model, so the probability that a response option $r$ of an ordered variable $Y$ is chosen is modelled as $P(Y = r\mid \eta) = logistic(\tau_{r\mid r+1}-\eta) - logistic(\tau_{r-1\mid r}-\eta)$, where $\eta$ is the linear predictor, $\tau_{r\mid r+1}$ is the threshold between $r$ and the next higher response option, $\tau_{r-1\mid r}$ is the threshold between $r$ and the next lower response option, and $logistic$ is the logistic distribution (for the first response option the subtrahend will be $0$, for the last one, the minuend will be $1$).
% Figure~\ref{fig:thresholds:Masloch} shows the (point) estimates for the thresholds from the model reported below.
% While the estimates for the thresholds are model-dependent and thus uninteresting in themselves, it is obvious that the response options correspond to portions of the latent variable of unequal size.
% A metric model would treat the distances as being equal.
In Bayesian models, parameters are random variables, so one can talk about the credibility of different values (an introduction aimed at a linguistic readership is provided by \citealt{nicenboim_introduction_2024}).
One starts off with a prior distribution across the parameters, which reflects one's prior knowledge, and updates it using the data to receive a posterior distribution, which reflects the uncertainty about the parameter values.
Given the lack of previous comparable studies and quantitative predictions for the effect sizes, we use mildly informative, theory-neutral regularising priors.\footnote{We performed prior predictive checks and additionally fitted models with different (and more informative) priors, which can be found in the analysis script on the OSF directory. The results are interpretation-wise the same.}
For hypothesis testing, we will use Bayes factors: The Bayes factor (BF) is the ratio of the marginal likelihoods of two models and tells us under which of them the data are more likely.
We will compare the model we report here against a model in which the effect of interest is set to 0.
According to \citeauthor{jeffreys_theory_1939} (\citeyear{jeffreys_theory_1939}, cited after \citealt{nicenboim_introduction_2024}), a BF of 3 indicates moderate evidence in favour of the first model, a BF of 10 strong evidence and a BF of 100 extreme evidence.
1 is the neutral value, ⅓ indicates moderate evidence in favour of the second model etc.\footnote{The Bayes factor depends on the prior. The analysis script contains a sensitivity analysis for the Bayes factors as well as a detailed look at the posterior distributions of different models (including models with more informative priors). For our theoretical purposes, the results are similar.}

% \begin{figure}[t]
%     \begin{tikzpicture}
%         \draw[->] (-6,0) -- (6,0);
%         \foreach \p / \l in {-5.6/-2,-3.8/-1,-2/0,-0.2/1,1.6/2,3.4/3,5.2/4}
%             \draw (\p cm,3pt) -- (\p cm,-3pt) node[anchor=north] {$\l$};
%         \draw (-4.88 cm,5pt) -- (-4.88 cm,-13pt) node[anchor=north] {$\tau_{1\mid 2}$};
%         \draw (-0.81 cm,5pt) -- (-0.81 cm,-13pt) node[anchor=north] {$\tau_{2\mid 3}$};
%         \draw (0.68 cm,5pt) -- (0.68 cm,-13pt) node[anchor=north] {$\tau_{3\mid 4}$};
%         \draw (4.66 cm,5pt) -- (4.66 cm,-13pt) node[anchor=north] {$\tau_{4\mid 5}$};
%         \node[text width = 2cm] at (-5.2,0.7) {completely unnatural};
%         \node at (-2.9,0.7) {rather unnatural};
%         \node at (-0.1,0.7) {no tendency};
%         \node at (2.5,0.7) {rather natural};
%         \node[text width = 2cm] at (5.4,0.7) {completely natural};
%     \end{tikzpicture}
% \caption{Point estimates for thresholds and proportions of the latent variable corresponding to the response options.}\label{fig:thresholds:Masloch}
% \end{figure}

There is an effect of \factor{order} in the expected direction ($\hat{\beta} = 0.79$, 95\,\% credible interval (CrI) $= [0.47, 1.12]$)\footnote{We use hats ($\hat{\circ}$) to indicate estimates. When describing the marginal posterior distribution of a parameter in Bayesian models, we use the mean as the measure of central tendency (estimate) and the standard deviation for variability (CrIs).}, which is, however, not quite as strong as expected as it roughly corresponds to the difference between two levels of the ordered response variable.\footnote{Due to the sum-coding used, the distance between \level{OS} and \level{SO} would be $0.79\times 2 = 1.58$, which is smaller than the distance between the first and the second or between the third and the fourth threshold, but larger than the distance between the second and the third on the latent variable within the model, see the full model summary in the analysis script on the OSF. In a cumulative model, the response is taken to relate to a latent variable -- in our case: perceived naturalness -- that can be modelled as linear and is partitioned into ordered bins corresponding to the response options via thresholds that are estimated in the model.}
Nevertheless, the Bayes factor shows that there is extreme evidence for this effect (Bayes factor computed as Savage-Dickey density ratio between the model and a model where $\beta$ is set to 0; BF\textsubscript{10} $=126.7$).
Changing from \level{SO} to \level{OS} leads to an increased naturalness.
The point estimate for the effect of \factor{case} ($\hat{\beta} = 0.17$, CrI $= [-0.24, 0.57]$, BF\textsubscript{10} $=0.074$) is slightly positive, but we do not have enough evidence to postulate its existence. 
Indeed, the Bayes factor shows that one may do better without it. 
In any case, the effect is only small.
The reason we expected a mildly positive effect of \factor{case} was that OS is unmarked for dative-\isi{object} EO verbs, but not for the accusative-\isi{object} ones.
While descriptively sentences are rated higher in \level{dative OS} than in \level{accusative OS}, they are rated lower in \level{dative SO} than in \level{accusative SO}.
%it is the other way round for the \level{SO} sentences.
Judments of naturalness may correspond to \isi{normal order} in a more straightforward way, such that a sentence perceives higher ratings irrespective of \isi{binding} constraints if it is normally ordered (this may be a caused by performance mistakes).
The interaction effect would capture such a pattern: a positive value of it would correspond to a preference for the \isi{normal order} irrespective of the other factors including \isi{binding} constraints because -- given the encoding chosen -- \level{dative SO} and \level{accusative OS} get the value $-1$ and \level{dative OS} and \level{accusative SO} get the value $1$. 
Although the posterior distribution hints at a small positive effect ($\hat{\beta} = 0.23$, CrI $= [-0.06, 0.52]$), the Bayes factor indicates that the data provide evidence against it (BF\textsubscript{10} $=0.138$).

The model assumes a comparatively large standard deviation of the participants' varying intercepts ($\hat{SD} = 1.57$, CrI $= [1.22, 2]$), so there is variability between participants.
The standard deviation of the participants varying slope for \factor{order} is considerable ($\hat{SD} = 0.5$, CrI $= [0.26, 0.75]$), while the ones for \factor{case} ($\hat{SD} = 0.16$, CrI $= [0.01, 0.39]$) and for the interaction ($\hat{SD} = 0.22$, CrI $= [0.01, 0.48]$) are not.
There is variation between items ($\hat{SD} = 0.74$, CrI $= [0.45, 1.18]$), also for the varying slope for \factor{order} ($\hat{SD} = 0.47$, CrI $= [0.23, 0.81]$).
The estimated correlations between varying effects are rather unremarkable.

An exploratory look into the responses of individual participants shows that twelve out of the 48 participants whose responses entered the analysis assign low scores across conditions.
Additionally, three of the test items received almost only low scores in both ordering conditions.
Taken together, these observations may explain the overall lower level of \isi{acceptability} (and hence the surprisingly small effect size of \factor{order}) and in part also the variation among participants and items.
If the relevant items and participants are excluded, accusative-\isi{object} EO verbs receive mixed judgments in the \level{OS} condition, dative-\isi{object} EO verbs rather good ones.

\section{Discussion}
\label{sec:discussion:Masloch}

\subsection{General discussion}

We take these results to support our main prediction: Reflexive \isi{binding} into the \isi{subject} of German EO verbs is licit only if it is \emph{not} backward.\footnote{For a discussion of the surprisingly low overall \isi{acceptability} in the OS condition, see below. One may object that both orderings could be ungrammatical and that enhanced \isi{acceptability} of the OS order is due to priming effects: Since participants see the \isi{experiencer} \isi{object} first, they are already familiar with the possible \isi{antecedent} and will be able to correct the grammar violations instantly, whereas this is not the case in the SO variant. An account along such lines is not tenable since the examples in \pex{ex:kiss5:Masloch}, where the possible \isi{antecedent} precedes the \isi{reflexive}, too, are clearly unacceptable.}
This holds for dative-\isi{object} as well as for accusative-\isi{object} EO verbs.
As discussed in \sectref{sec:log_bind_German:Masloch}, this is expected on different accounts of German clausal syntax and the syntax of EO verbs, but it is \emph{not} expected on an account that 
\begin{inparaenum}[i)]
\item takes (dative) EO verbs to be \isi{unaccusative} (which should translate into an OS \isi{base order}); and
\item takes scrambling to \isi{reconstruct} for \isi{reflexive} \isi{binding}.    
\end{inparaenum}
Indeed, the results are incompatible with the idea that scrambling reconstructs for \isi{binding} because if it did so, there should be no difference between the ordering conditions. Thus, our data do not speak against the \isi{unaccusativity} hypothesis (taken to imply an OS \isi{base order}) directly because the base order does not matter for \isi{reflexive} \isi{binding} if scrambling does not reconstruct for it; but they also do not speak for it.
%Indeed, the results are incompatible with the idea that scrambling reconstructs for \isi{binding} because if it did so, there should be no difference between the ordering conditions.
%If we take the \isi{unaccusativity} hypothesis to imply that the \isi{base order} of the respective verbs is \isi{object} before \isi{subject}, our data are compatible with it if scrambling in German does not \isi{reconstruct} for \isi{binding}, but then they are equally compatible with an SO \isi{base order}.
Both a base-generation approach as described in \sectref{sec:background_German:Masloch} and a scrambling-as-movement account without \isi{reconstruction} for \isi{reflexive} \isi{binding} are compatible with the data.

We expected a mildly positive effect of \factor{case}, which seems to be inexistent.
The reason we expected this effect was that we assumed that participants may rate a sentence better if the \isi{binding} condition is fulfilled in the unmarked order.
This does not seem to be the case.
% Rather, our model suggests that there is a slight tendency for participants to rate an item in which the order of \isi{subject} and \isi{object} is unmarked higher than one in which it is not irrespective of overall \isi{grammaticality} (interaction effect), even though we cannot be sure about its existence.
% If it is there, it is small and we may suspect it to be reducible to performance mistakes.

Two remarkable aspects are the unexpectedly low level of \isi{acceptability} (also in comparison to filler and control items categorised as acceptable a priori, which received high ratings, see the analysis script on the OSF directory) and the substantial individual variation.
We do not have a full explanation for this at the moment, but regarding both, one has to consider that the test items were complex sentences that had to fulfill highly specific criteria and contained a relatively infrequent phenomenon, namely the PPs containing reflexives discussed in \sectref{sec:log_bind_German:Masloch}.
The overshadowing process mentioned there presumably involves competition between a PP containing the \isi{reflexive} and the possessive, but a PP containing a personal pronoun will be involved as well.
We suspect that there are differences both between and within speakers (varieties, register) with respect to the weighting of the factors involved. 
These will influence how natural the noun-preposition-\isi{reflexive} combinations sound to them in the given setting because (as one may assume) the losing candidate in a competition will be judged as less natural.
Items may differ in which factors involved in the competition are relevant.
To the best of our knowledge, there is no in-depth investigation of the phenomenon yet.\footnote{The EISS reviewers asked for baselines involving PPs containing full NPs or personal pronouns. We did not include items containing such NPs to keep the number of conditions low and because we did not expect the overall rather low \isi{acceptability}. On our explanation of the latter, such items could not function as real baselines (because they either stand in competition with the variant we used or lack a competing candidate) but would be very interesting for further investigation.}
The fact that the standard deviation of the participants' varying slope for \factor{order} is not as low as one may expect may be due to the participants who rate all test items as unnatural.
Something similar may happen with the standard deviation of the items' varying slope for \factor{order} since there are some test items that received low ratings across conditions.
A radical alternative explanation for the rather low overall \isi{acceptability} and the individual variation would be lectal variation in \isi{binding} domains such that for some speakers the \isi{reflexive} has to be bound only at clause level while for others the \isi{binding} domain is narrower, so that the \isi{reflexive} needs an \isi{antecedent} within the NP or PP.
However, given the availability of a plausible alternative, we do not want to pursue this path.

A reviewer asks us to disentangle the effects of \isi{linear order}, \isi{c-command} and topicality, which all seem to make the same prediction here.
While we only looked at the \isi{midfield}, where \isi{linear order} and \isi{c-command} (largely) correspond to each other, a constituent situated in the prefield linearly precedes the \isi{midfield}.
It is widely assumed that the prefield constituent moves there from the \isi{midfield}.
It may thus (at some point) be c-commanded by a constituent that follows it in surface order.
\pex{ex:cahunveröffentlicht:Masloch} is acceptable even though the \isi{reflexive} precedes its \isi{antecedent}.
\begin{exe}
    \ex \gll [Bilder von sich$_i$ veröffentlicht]\textsubscript{1} hat [Claude$_i$ [fast nie \textit{t}\textsubscript{1}]].\\
    pictures of \textsc{refl} published has Claude almost never\\
    \glt `Claude almost never published pictures of herself.'\label{ex:cahunveröffentlicht:Masloch}
\end{exe}
In \pex{ex:fotografin:Masloch}, the \isi{reflexive} in the accusative \isi{object} may be bound by the \isi{subject} or the dative \isi{object}, although only the \isi{subject} is topical.
\begin{exe}
    \ex \textit{Was ist mit der Fotografin?} `What about the photographer?'\\
    \gll Die Fotografin$_i$ zeigte einem Kunden$_j$ Bilder von sich$_{i/j}$.\\
    the.\textsc{nom} female.photographer.\textsc{nom} showed a.\textsc{dat} customer.\textsc{dat} pictures.\textsc{acc} of \textsc{refl}\\
    \glt `The photographer showed a customer pictures of herself/himself.'\label{ex:fotografin:Masloch}
\end{exe}
Both \isi{linear order} and topicality are thus unlikely to be the decisive factor for licensing \isi{reflexive} \isi{binding} by themselves.

\subsection{Theories of binding}
\label{sec:binding:Masloch}

So far, we assumed that the German \isi{reflexive} \om{sich} has to be c-commanded by its \isi{antecedent}, which  aligns with Principle A of classical \isi{binding} theory.
Theories that try to capture \isi{binding} data using tree-configurational notions such as \isi{c-command} are rivaled by predicate-based theories of \isi{binding} \parencite[i.a.][]{pollard_anaphors_1992,reinhart_reflexivity_1993}, in which co-argumenthood is decisive.
On \citegen{reinhart_reflexivity_1993} account, only heads with an external argument count as syntactic predicates.
According to their condition A, \isi{reflexive} syntactic predicates (i.e., predicates that have two co-indexed arguments) need to be \isi{reflexive} \isi{marked}, which can either happen lexically or via a \textsc{self}-\isi{anaphor}.
German \om{sich} can be a \textsc{self}-\isi{anaphor} according to \citet{reuland_pronouns_1995}.
\citet{pollard_anaphors_1992} define \isi{binding} conditions in terms of relative obliqueness.
In their analysis of English \isi{anaphora}, an \isi{anaphor} has to be co-indexed with a less oblique co-argument if there is one.
In our test items, the predicate relevant to determining co-argumenthood is the noun or the preposition.
In both cases, there is no co-argument/\isi{subject}. 
Thus, it is not a syntactic predicate for \citet{reinhart_reflexivity_1993} and condition A does not apply. Thus, the \isi{anaphor} does not have a less oblique co-argument so that \citegen{pollard_anaphors_1992} Principle A does not apply. 
As a result, the \isi{reflexive} should be licensed without being bound.\footnote{\textcite{pollard_anaphors_1992} do not claim their theory to be applicable to languages other than English. Thus, our data do not speak against their theory directly, they only show that it cannot trivially be extended to German.}
Even if there were some further principle requiring \om{sich} to have a co-argument or if one were to assume that there is an unpronounced external argument of the noun or preposition, one could not capture our data, since it is the positioning of the whole syntactic predicate (the NP/PP) that makes the difference. 
Hence some structural condition must be at play.\footnote{According to \textcite{reuland_pronouns_1995} and \textcite{reuland_2011}, \om{sich} may also be a \textsc{se}-\isi{anaphor}, in which case it would not mark the predicate as \isi{reflexive}. However, it should not be possible to stress \textsc{se}-\om{sich} \parencites[see][249\psqq]{reuland_pronouns_1995}[][275\psqq]{reuland_2011}, but in our judgment stressed \om{sich} is perfectly fine in examples like \pex{ex:kiss13d:Masloch}. In order to be interpreted as bound, \textsc{se}-\om{sich} would have to be in a chain with its \isi{antecedent}. This cannot be the case (if only because \textcite[167\psqq]{reuland_2011} takes D to block the attraction process that would be necessary on his account). If \textsc{se}-\om{sich} cannot enter a chain, one may expect a \isi{logophoric} interpretation to occur, but 1. German \om{sich} does not have a \isi{logophoric} interpretation as shown in \sectref{sec:log_bind_German:Masloch} and 2. one would get the same problems as mentioned for \textsc{self}-\om{sich} in the main text then.}

This structural condition need not be a universal principle: 
On \citeauthor{kiss_reflexivity_2012}' (\citeyear{kiss_reflexivity_2012}) account of \isi{reflexive} \isi{binding}, anaphoric dependencies are introduced in syntax.
A \isi{reflexive} pronoun will receive a feature $\mathbf{D}(n)$, where $n$ is an index, if it is an argument of a head with an articulated argument structure ($\approx$ having an external argument, see \citeauthor{kiss_reflexivity_2012}~\citeyear{kiss_reflexivity_2012}), which will bear the feature $+\textsc{Arg-S}$.
$\mathbf{D}(n)$ is projected upwards until $n$ is identified with the index of a sister of a phrase bearing the feature.
In German, a feature $\msout{\mathbf{D}}(n)$ representing an inactive dependency is introduced if the head does not have an articulated argument structure ($-\textsc{Arg-S}$), and projected in the same way until it is activated (= leads to the phrase having the feature $\mathbf{D}(n)$) when meeting a $+\textsc{Arg-S}$ head.\footnote{English is taken to lack inactive dependencies, so something like a predicate-based \isi{binding} theory emerges.}
A local resolution condition requires dependencies to be resolved within the clause.
In effect, this means that German \om{sich} has to be co-indexed with a c-commanding phrase within the same clause.\footnote{Note that on \citeauthor{kiss_reflexivity_2012}' (\citeyear{kiss_reflexivity_2012}) theory there may also be differences between different lexical items within a language, so our results for \om{sich} may not be directly transferable to reciprocal \omt{einander}{each other}.}
\figref{fig:analysisbinding:Masloch} illustrates how this works for \pex{ex:kiss13d:Masloch}.
Since P and N do not have an articulated argument structure there, the $\msout{\mathbf{D}}(n)$ introduced with \om{sich} becomes active only once \textsubscript{2}NP%\footnote{We use natural numbers to distinguish nodes in the tree and small letters for indices.} 
becomes a daughter of the verbal projection, i.e. \textsubscript{2}NP bears $\mathbf{D}(n)$.
$\mathbf{D}(n)$ is projected upwards to \textsubscript{1}V$'$, where $n$ can then be identified with the index of \textsubscript{1}NP. 
Indeed, it must do so in order for the local resolution condition to be fulfilled.
By contrast, local resolution cannot be fulfilled if the NP containing the \isi{reflexive} is the last one to combine with the verbal projection as in \pex{ex:kiss13c:Masloch} and our SO items, leading to ungrammaticality.

\begin{figure}
    \centering
   \begin{forest}
            [CP, align = center, for tree={parent anchor=south, child anchor=north}
                [{C\\dass\\that} ] 
                [\textsubscript{2}V$\textbf{}'$\textsubscript{{$[+\textsc{Arg-S}]$}}, align = center
                    [\textsubscript{1}NP$_i$ [ {den Kindern\\the.\textsc{dat} children}, roof ] ]
                    [\textsubscript{1}V$'$\textsubscript{{$[+\textsc{Arg-S}, \textbf{D}(n=i)]$}}, align = center 
                        [\textsubscript{2}NP\textsubscript{$[\textbf{\msout{D}}(n), \textbf{D}(n)]$}, align = center 
                            [{D\\die\\the} ]
                            [N'\textsubscript{$[-\textsc{Arg-S}, \textbf{\msout{D}}(n)]$}, align = center
                                [{N\textsubscript{$[-\textsc{Arg-S}]$}\\Bilder\\pictures} ] 
                                [PP\textsubscript{$[-\textsc{Arg-S}, \textbf{\msout{D}}(n)]$}, align = center
                                    [{P\textsubscript{$[-\textsc{Arg-S}]$}\\von\\of} ]
                                    [{NP\textsubscript{$[\textbf{\msout{D}}(n)]$}\\sich\\\textsc{refl}} ] 
                                ] 
                            ] 
                        ]
                        [{V\textsubscript{[$+\textsc{Arg-S}$]}\\gefielen\\appealed.to} ]  
                    ] 
                ] 
            ] 
        \end{forest}
    \caption{Analysis for \pex{ex:kiss13d:Masloch}. Numbers are used to distinguish nodes in the tree, small letters for indices}
    \label{fig:analysisbinding:Masloch}
\end{figure}

% Note that nothing in this account hinges on the fact that the verb in \pex[]{ex:analysisbinding} is an EO verb.
% Binding should be possible in the same way with verbs of other classes, although additional factors may influence \isi{acceptability}.
% With agent-patient verbs, \isi{acceptability} may be reduced by the fact that the variant containing the possessive is strongly preferred over the PP variant with most animate head nouns in the standard register and by the strong preference for an SO \isi{linearisation} in a neutral context.
% This is a topic for future research.

\section{Summary}
\label{sec:conclusion:Masloch}

Our study shows that \isi{reflexive} \isi{binding} into the \isi{subject} of experiencer-\isi{object} verbs is licensed in the German \isi{midfield} only if the \isi{subject} is preceded -- and thus c-commanded -- by the \isi{antecedent} in surface structure.    
The results are in principle compatible with both free base-generation and movement-based accounts of \isi{linearisation} in the \isi{midfield}, but with the latter only if scrambling is taken not to \isi{reconstruct} for \isi{binding}.
Analysing German EO verbs as \isi{unaccusative} is not necessary to explain their \isi{reflexive} \isi{binding} patterns (although they are not incompatible with \isi{unaccusativity}).
Because the positioning of the constituent containing the embedded \isi{reflexive} influences \isi{acceptability}, the results are problematic for predicate-based \isi{binding} theories.

\section*{Acknowledgements}

We would like to thank three anonymous reviewers for CSSP and three anonymous reviewers for EISS as well as the audience at CSSP and the editors of this volume, Gabriela Bîlbîie and Gerhard Schaden, for their valuable feedback and suggestions!
A special thanks goes to Tamara Stojanovska for performing the collocation analyses needed for item construction.
The research reported here was supported by the Deutsche Forschungsgemeinschaft
(DFG) grant no. 437144413 to Tibor Kiss.

%\section*{Contributions}
%John Doe contributed to conceptualization, methodology, and validation. 
%Jane Doe contributed to writing of the original draft, review, and editing.
\newpage
\section*{Abbreviations}
\begin{multicols}{2}
\begin{tabbing}
  CrI \hspace{1em} \= \isi{object} before subject\kill
  BF \> Bayes factor\\
  CrI \>  95\,\% credible interval\\
  EO \>  experiencer-object\\ 
  SO \>  subject before object\\
  OS \> object before subject\\
\end{tabbing}
\end{multicols}

\sloppy
\printbibliography[heading=subbibliography,notkeyword=this]
\end{document}

%%% Local Variables:
%%% mode: xelatex
%%% TeX-master: t
%%% End:
