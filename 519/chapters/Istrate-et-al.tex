\documentclass[output=paper,colorlinks,citecolor=brown]{langscibook}
\ChapterDOI{10.5281/zenodo.15450438}
\author{Fabian Istrate\orcid{}\affiliation{Université Paris Cité} and         Ruxandra Ionescu\orcid{}\affiliation{Université Paris Cité} and         Barbara Hemforth\orcid{}\affiliation{CNRS;Université Paris Cité}}
\title{Next mention biases predict the choice of null and pronominal subjects}
\abstract{In this paper we investigate the production of null and pronominal subjects in Romanian. Data from corpus and experimental studies show that several factors need to be taken into account for this alternation beyond the syntactic role of the antecedents. Subject pronouns in Romanian seem to be sensitive to semantic and pragmatic factors influencing accessibility or predictability, induced among others by the interaction between verb semantics and coherence relations. Our results contribute to the larger debate regarding the extent to which predictability affects the choice of referring expressions, suggesting that null subjects are preferred not only for a subject antecedent, but also for a referent that is more predictable in the context. }


\IfFileExists{../localcommands.tex}{
   \addbibresource{../localbibliography.bib}
   % add all extra packages you need to load to this file

\usepackage{tabularx,multicol}
\usepackage{url}
\urlstyle{same}

\usepackage{listings}
\lstset{basicstyle=\ttfamily,tabsize=2,breaklines=true}

\usepackage{langsci-basic}
\usepackage{langsci-optional}
\usepackage{langsci-lgr}
\usepackage{langsci-osl}
% \usepackage{./langsci/styles/langsci-lgr}
% \usepackage{./langsci/styles/langsci-osl}
% \usepackage{langsci-gb4e}

\usepackage{tikz}
\usetikzlibrary{patterns,calc}
\pgfdeclarepatternformonly{south east lines}{\pgfqpoint{-0pt}{-0pt}}{\pgfqpoint{3pt}{3pt}}{\pgfqpoint{3pt}{3pt}}{
    \pgfsetlinewidth{0.6pt}
    \pgfpathmoveto{\pgfqpoint{0pt}{3pt}}
    \pgfpathlineto{\pgfqpoint{3pt}{0pt}}
    \pgfpathmoveto{\pgfqpoint{.2pt}{-.2pt}}
    \pgfpathlineto{\pgfqpoint{-.2pt}{.2pt}}
    \pgfpathmoveto{\pgfqpoint{3.2pt}{2.8pt}}
    \pgfpathlineto{\pgfqpoint{2.8pt}{3.2pt}}
    \pgfusepath{stroke}}
    
\usepackage{stmaryrd}
\usepackage{wasysym}
\usepackage{multirow}
\usepackage{caption}
\usepackage{subcaption}
\usepackage{mathrsfs}
\usepackage{qtree}

\usepackage{linguex}


   %pminos do not split footnotes
% \interfootnotelinepenalty=10000 %Footnote in Laporte chapters has to be split SN


%\DeclareIndexNameFormat{default}{%
%\nameparts{#1}%
%\usebibmacro{index:name}%
%{\index[names]}%
%{\namepartfamily}%
%{\namepartgiveni}%
% {}% L1
% {}% L2
%{\namepartprefix}% generates spurious space L3
%{\namepartsuffix}% generates spurious space L4
%}

%  {\DeclareIndexNameFormat{default}{%
%     \usebibmacro{index:name}{\index[names]}{#1}{#3}{#5}{#7}}}

%\DeclareIndexNameFormat{default}{%
%  \usebibmacro{index:name}{\sindex[nom]}{#1}{#3}{#5}{#7}}

%\DeclareIndexNameFormat{default}{%
%  \usebibmacro{index:name}{\sindex[person]}{#1}{#3}{#5}{#7}}
%\DeclareIndexNameFormat{default}{%
%\nameparts{#1} \usebibmacro{index:name}{\sindex[person]]}{\namepartfamily}{‌​\namepartgiven}{\nam‌​epartprefix}{\namepa‌​rtsuffix}}

%\newcommand{\smiley}{:)}

%\renewbibmacro*{index:name}[5]{%
%\usebibmacro{index:entry}{#1}%
%{\iffieldundef{usera}{}{\thefield{usera}\actualoperator}\mkbibindexname{#2}{#3}{#4}{#5}}}

% \newcommand{\noop}[1]{}

%remove for final
%\overfullrule=1mm

\newcommand{\tobi}[2]}}
\renewcommand{\S}[1]{\tobi{#1}{\textsc{*}}}

% this volume references
% puts: [this volume]
% already defined: \citetv
%\newcommand{\citepv}[1]{(\citeauthor{#1} \citeyear*{#1} [this volume])}
\newcommand{\citealtv}[1]{\citeauthor{#1} \citeyear*{#1} [this volume]}

%parentheses around example number
\newcommand{\pref}[1]{(\ref{#1})}

% in-text examples

\newcommand{\lnex}[1]{\textit{#1}} %target lang word
\newcommand{\lnlit}[1]{(lit.: `#1')} %literal reading
\newcommand{\lnlat}[1]{(#1)} % latinization
\newcommand{\lntrans}[1]{`#1'} %translation
\newcommand{\lnexl}[2]%
{\lnex{#1}{} \lnlat{#2}} % ex with latinization
\newcommand{\lnexlat}[3]{\lnex{#1}{} \lnlat{#2}{} \lntrans{#3}} % ex with latinization and tranl.

%ch01
\newcommand{\co}[1]{\mbox{\textbf{#1}}}

%ch09

\newcommand{\cyrbulg}[1]{\begin{otherlanguage*}{bulgarian}#1\end{otherlanguage*}}


%ch10
\newcommand{\nlp}{{\small NLP}}
\newcommand{\mwe}{{\small MWE}}
\newcommand{\rae}{{\small RAE}}
\newcommand{\lvc}{{\small LVC}}
\newcommand{\pos}{{\small P}o{\small S}}
%\newcommand{\todo}[1]{ \textcolor{red}{#1} }

%\renewcommand{\labelenumi}{\theenumi}
%\ainamefmt{{vv}{ll}{, ff}{, jj}} % fullname

\newcommand{\biberror}[1]{{\color{red}#1}}

\newcommand{\osenovaitem}{--~}
   %% hyphenation points for line breaks
%% Normally, automatic hyphenation in LaTeX is very good
%% If a word is mis-hyphenated, add it to this file
%%
%% add information to TeX file before \begin{document} with:
%% %% hyphenation points for line breaks
%% Normally, automatic hyphenation in LaTeX is very good
%% If a word is mis-hyphenated, add it to this file
%%
%% add information to TeX file before \begin{document} with:
%% %% hyphenation points for line breaks
%% Normally, automatic hyphenation in LaTeX is very good
%% If a word is mis-hyphenated, add it to this file
%%
%% add information to TeX file before \begin{document} with:
%% \include{localhyphenation}
\hyphenation{
    Beck-man
    Ngu-yen
    back-chan-nel
    back-chan-nels
    mo-not-o-nous
    ste-reo-typ-i-cal
}

\hyphenation{
    Beck-man
    Ngu-yen
    back-chan-nel
    back-chan-nels
    mo-not-o-nous
    ste-reo-typ-i-cal
}

\hyphenation{
    Beck-man
    Ngu-yen
    back-chan-nel
    back-chan-nels
    mo-not-o-nous
    ste-reo-typ-i-cal
}

   \boolfalse{bookcompile}
   \togglepaper[23]%%chapternumber
}{}

\begin{document}
\maketitle
\section{Introduction} \label{sect:intro:Istrate}
\subsection{Pronoun resolution and predictability}
Pronoun resolution (i.e., the question of how pronouns retrieve their antecedents in discourse) is one of the most studied phenomena in psycholinguistics (see, for recent studies, \citealt{holler2016empirical,  arnold2010speakers, kehler2019prominence, schulz2021corpus, colonna2012information}). While much of the empirical work concerns processes in comprehension, a closely related question is which type of referential expression a speaker will use to refer back to a specific \isi{antecedent} \citep{Fukumuraetal2022a}. In this paper, we aim to add both corpus and experimental evidence about the division of labor between null and \isi{pronominal subjects} in a pro-drop language like Romanian. In particular, we are interested in investigating which factors affect speakers' choices between these referential expressions and how next mention biases (i.e., the probability of mentioning a particular referent next) influence their \isi{production} \citep{arnold2001effect}.

As a general principle of communication, when referring back to a previously mentioned and highly accessible entity in discourse, speakers tend to use shorter or less specific forms of reference, while making their exchanges as informative as necessary. This tendency to make the communication as economical as possible was captured by \citet{grice1975logic} in his \textit{Maxim of Quantity}. Communication becomes more efficient as it requires less speaker effort without generating significant communicative cost (see also \citealt{ jaeger2006speakers}). To ensure that a referent is still accessible to the listener, the entity should be in the focus of attention, making it highly predictable to be mentioned next. A current debate in the literature addresses the question whether a more predictable referent will be recovered by a pronoun or a more complex referential expression (such as proper names, definite descriptions, etc.). In this respect, previous studies offer contradictory results: while some of them show no evidence  that pronoun \isi{production} is affected by referent \isi{predictability} (\citealt{ferretti2009verb, fukumura2010choosing, kehler2013probabilistic, rosa2015semantic, patterson2022bayesian}, among others), others suggest the existence of such an effect (\citealt{arnold2001effect, rosa2017predictability, lindemann2020thematic}, among others).
Theories about the choice of a given \isi{referential form} can be traced back to earlier hypotheses proposed in the literature on pronoun resolution. \citet{givon1983topic}’s \textit{Topic Continuity Theory} posits that the degree of topicality that a given entity has in discourse is correlated with the usage of referential forms. In subsequent work, \citet{ariel1994interpreting} proposed in her \textit{Accessiblity Theory} that referential forms can be ranked on an explicitness scale, ranging from full names to zero expressions. The more accessible an entity is in discourse, the less complex its corresponding anaphoric expression will be. In line with these hypotheses, \citet{von2019discourse} put forward the concept of \textit{prominence} aiming to characterize \textit{a highlighted entity} in discourse. The properties of a prominent referent are based on the following criteria of prominence in grammar, given by \citet{himmelmann2015prominence}: (i) linguistic units of equal rank compete for the status of being in the center of attention, (ii) their status may shift, (iii) prominent units act as structural attractors in their domain. They suggest that the focus of attention can be shifted by linguistic means, thus updating the prominence structure. Similar observations postulate that semantic/coherence-driven \isi{predictability} of referents in discourse may come from multiple sources, inducing next mention biases (see \citealt{gernsbacher1988accessing,gordon1993pronouns,grosz1995centering,fukumura2011effect}): intrinsic properties like animacy, grammatical factors like subjecthood, information structural factors like topichood, or semantic and pragmatic factors like verb semantics or \isi{coherence relations}.

\subsection{Null and pronominal subject alternation in pro-drop languages}

A prominent account of the alternation between null and \isi{pronominal subjects} is the \textit{Position of Antecedent Hypothesis} (henceforth, \textit{PAH)} proposed by \citet{carminati2002processing} for intra-sentential anaphoric subjects in Italian. The \textit{PAH} posits that null \isi{subject} pronouns generally refer back to \isi{subject} antecedents (\ref{null_carm:Istrate}) whereas \isi{pronominal subjects} usually prefer a non-\isi{subject} \isi{antecedent} (\ref{pron_carm:Istrate}).

\ea \ea \label{null_carm:Istrate}
\gll Marta scriveva frequentemente a Piera quando era negli Stati Uniti. \\ 
Marta write.\textsc{pst.ipfv.3sg} frequently to Piera when was in States United\\
\glt`Marta wrote to Piera often when she was in the United States.'    
\ex \label{pron_carm:Istrate}
\gll Marta scriveva frequentemente a Piera quando \textbf{lei} era negli Stati Uniti. \\ 
Marta write.\textsc{pst.ipfv.3sg} frequently to Piera when she was in States United\\
\glt `Marta wrote to Piera often when she was in the United States.' 
\z
\z

A similar pattern has also been observed by \citet{de2013effects} for Spanish, showing the impact of grammatical role of the antecedents. However, in their experimental study, tendencies in Spanish seem to be less strong than those found for Italian, i.e. \isi{null subjects} can also take \isi{object} antecedents. Moreover, aside from the \isi{subject} function of antecedents, \citet{de2013effects} point out that left-dislocation of the antecedents increases their discourse accessibility, thus being mostly retrieved by \isi{null subjects} (see \citealt{runner2016information} for similar observations about \isi{information structure}). In a \isi{production} and a comprehension experiment on Greek and Italian, \citet{torregrossa2020variation} confirmed the left-dislocation effect by arguing that \isi{null subjects} have a bias for left-dislocated objects compared to \textit{in-situ} objects. In line with the previous studies, \citet{contemori2021microvariation} postulate that Italian and Spanish display a distinct division of labor between null and \isi{pronominal subjects}: whereas in Italian the \isi{production} of the \isi{referential form} seems to be very distinct according to the grammatical role of the \isi{antecedent}, in Spanish the division of labor is less clear. The results of their \isi{production} experiment show that \isi{verb bias} and causal \isi{coherence relations} might play a role in Italian and in Mexican Spanish, although to a lesser extent in the latter: speakers produce a null \isi{subject} to corefer to the \isi{object} when an object-biased verb is present in the context. However, \citet{chamorro2018offline} suggests different tendencies for Spanish in an offline judgment task and an online eye-tracking study showing that \isi{null subjects} do not exhibit a clear preference while \isi{pronominal subjects} mostly prefer \isi{object} antecedents. She also postulates that clause order (i.e., main-subordinate vs. subordinate-main) may also be responsible for the \isi{antecedent} preferences of pronouns in the case of intra-sentential \isi{anaphora}.
Regarding the choice between a null and a pronominal \isi{subject} in Romanian, \citet{lindemann2020thematic} and \citet{istrate2022position} have shown that grammatical role plays an important role in this alternation, the tendencies being close to those found for Spanish \citep{de2013effects}. Moreover, \citet{lindemann2020thematic} argues that \isi{null subjects} are the most preferred choice when referring back to a more prominent or accessible referent, i.e. a \isi{subject} \isi{antecedent} which is also the \textit{goal}, in terms of thematic roles. Divergent results concerning the \isi{production} of these referential expressions might be due to different experimental tasks (online vs. offline), but also to existing differences in the experimental material. Further parallel corpus and experimental studies are necessary in order to make a proper crosslinguistic comparison between pro-drop languages.

\subsection{The current study}
The goal of the current study is to add evidence about the \isi{production} of intra-sentential anaphoric subjects combining corpus and experimental data on Romanian, a pro-drop language in which the choice of null and \isi{pronominal subjects} has not been extensively studied from a quantitative perspective so far. In particular, we are interested in investigating which role \isi{predictability} and prominence may play in pronoun \isi{production}, by shedding light on the potential influence of grammatical and semantic-pragmatic factors. In \sectref{sect:intro:Istrate}, we present some background about pronoun resolution and \isi{predictability}, followed by empirical evidence on the choice of referential expressions in pro-drop languages. Then, we focus on Romanian as a consistent null \isi{subject} language and we put forward our hypothesis regarding Romanian based on previous studies. In \sectref{corpus:Istrate}, we present a \isi{corpus study} on complex sentences (sampling, annotation, results and discussion). Given the results we found in particular with respect to discourse relations, we ran a follow-up experiment in order to test to what extent the choice of referential expressions is sensitive to \isi{predictability} invoked by implicit \isi{causality} verbs in causal relations (\sectref{exp:Istrate}). In \sectref{sect:discussion:Istrate}, we put together the corpus and experimental results, by pointing out that preferences in the \isi{production} of null and \isi{pronominal subjects} in Romanian go beyond the grammatical role of the antecedents and must include \isi{predictability} and discourse relations.  

\subsection{Hypotheses about Romanian as a pro-drop language}

Romanian is a language that licenses the presence of \isi{null subjects} \citep{dobrovie2013reference}. In this subsection, we firstly present the classification of pro-drop languages proposed by
 \citet{holmberg2010null}, which underpinned the predictions made for Romanian. Consistent null \isi{subject} languages (Italian, Spanish, European Portuguese\footnote{The pro-drop status of Brazilian Portuguese is controversial (see \citealt{duarte1996perda, duarte2000loss} for \textit{ongoing parameter change} of Brazilian Portuguese).}, Greek, etc.) permit the use of a null \isi{subject} irrespective of number, person or verb tense. A second category includes partial pro-drop languages (such as Russian), which limit \isi{null subjects} to the 1st and 2nd person in finite clauses, and 3rd person pronouns \textit{bound by a higher argument}. A third category consists of expletive null \isi{subject} languages (such as German), allowing null expletive subjects but not referential ones.\footnote{German has more recently also been described as a topic drop language, see \citet{Schäfer2021}.} The last category is represented by radical pro-drop languages (or \textit{discourse pro-drop languages}, such as Japanese or Chinese), which permit other nominal arguments (e.g. objects) to be null, in addition to \isi{null subjects}.  
According to \citet{holmberg2010null}, Romanian falls into the category of consistent null \isi{subject} languages. In Romanian, there are several ways in which speakers may refer to an entity, including \isi{null subjects} (\ref{null1:Istrate}), but also overt subjects realized as personal pronouns (\ref{pron1:Istrate}), \isi{demonstratives} (\ref{dem1:Istrate}), proper names (\ref{alex:Istrate}), definite descriptions (\ref{coleg:Istrate}), etc.\footnote{In this paper, we will make the distinction between \isi{pronominal subjects} (personal pronouns) and lexical subjects (proper names, definite descriptions) for methodological reasons regarding the annotation of the collected data.}

\ea \ea \label{null1:Istrate}
\gll A ajuns la petrecere. \\ 
has arrived at party\\
\glt`He/she has arrived at the party.'    
\ex \label{pron1:Istrate}
\gll \textit{El} a ajuns la petrecere.\\ 
he has arrived at party\\
\glt `He has arrived at the party.' 
\ex \label{dem1:Istrate}
\gll \textit{Acesta} a ajuns la petrecere.\\ 
this has arrived at party\\
\glt `This one has arrived at the party.' 
\ex \label{alex:Istrate}
\gll \textit{Alexandru} a ajuns la petrecere.\\ 
Alexandru has arrived at party\\
\glt `Alexandru has arrived at the party.'
\ex \label{coleg:Istrate}
\gll \textit{Colegul} \textit{nostru} a ajuns la petrecere.\\ 
colleague \textsc{poss.1pl} has arrived at party\\
\glt `Our colleague has arrived at the party.'
\z
\z

With respect to consistent null \isi{subject} languages, the alternation between null and \isi{pronominal subjects} has attracted particular attention in the linguistic as well as the psycholinguistic literature \citep{carminati2002processing, chamorro2018offline,torregrossa2020variation, lindemann2020thematic, contemori2021microvariation}. According to \citeauthor{ariel1994interpreting}'s (\citeyear{ariel1994interpreting}) \textit{Accessibility Theory}, null and personal pronouns (either stressed or unstressed) are very close on the accessibility scale, compared to definite descriptions or proper names. Alternation between these two referential expressions therefore needs quantitative research to shed light on the sometimes fine-grained distinctions determining their choice.
Thus, we will only focus on these two referential expressions, in line with previous work.

Based on previous work on Romanian (see \citealt{lindemann2020thematic}), we hypothesize that \isi{null subjects} will be preferred not only for \isi{subject} antecedents, but also for more prominent or predictable referents in the context. We seek to establish: (i) in how far verb semantics and discourse relations render antecedents more predictable, and (ii) to what extent the referent \isi{predictability} impacts on pronoun resolution in Romanian (see \citealt{demberg2023systematic}).
From a comparative perspective, we expect to observe a similar pattern in Romanian as in Spanish, Italian \citep{contemori2021microvariation} and Catalan \citep{mayol2018asymmetries}: \isi{null subjects} should mostly be produced when the choice of the referent in the upcoming subordinate clause is in line with next mention biases. More precisely, we predict that the implicit bias of a verb for an upcoming referent makes the choice of a null \isi{subject} for this referent more likely. 
However, if our predictions are on the right track, Romanian should be different from Mandarin Chinese, \textit{a discourse pro-drop language}, where no evidence has been found for next mention biases affecting the \isi{production} of referential expressions \citep{hwang2022role}.

\section{Corpus study} 
\label{corpus:Istrate}
%-----------------------------------
\subsection{Details about Romanian corpora}

We used two corpora in our studies: the \textit{Parseme-ro 1.2} corpus for written Romanian and the \textit{CoRoLa} corpus for spoken Romanian. 
The \textit{Parseme-ro 1.2} corpus \citep{savary2018parseme} is a written corpus of texts collected from the \textit{Agenda} newspaper (containing 56,703 sentences and 1,015,624 words). Although the corpus has no subsections, it is a homogeneous journalistic corpus. Some texts included in \textit{Parseme-ro 1.2} are also part of the \textit{Romanian Universal Dependencies} corpus. 


The \textit{CoRoLa} corpus (\textit{The Reference Corpus of the Contemporary Romanian Language}, \citealt{mititelu2018reference}) comprises a written part and an oral part. In order to compare the \isi{production} of null and \isi{pronominal subjects} in Romanian, we extracted data from the oral part of the \textit{CoRoLa} corpus (covering 151 hours, 57 minutes and 21 seconds). The oral texts in \textit{CoRoLa} are mainly professional recordings from various sources (radio stations, recordings) for which transcriptions are available. Another part of the oral corpus is represented by texts read by various speakers in various circumstances: news read on radio stations, texts read by people close to them and texts read by professional speakers recorded in studios. However, we focused on spontaneous speech, taking into account only extracts from radio news and interviews. The reason for this choice was to have two sufficiently different sub-corpora so that we could expect to find interesting effects. We did not find any a priori reason to believe that read texts should be particularly different from written texts.

\subsection{Corpus sampling}
% ------ Corpus sampling ------------------
As the \textit{Parseme-ro 1.2.} corpus is morpho-syntactically annotated, we used SQL query formulas to collect the data, both for null and \isi{pronominal subjects}. However, for the \textit{CoRoLa} corpus there is currently no such automatic annotation. We extracted occurrences with null and \isi{pronominal subjects}, using the most common verbs  in Romanian (a total of 552 verbs).\footnote{While lexical subjects certainly play a role for referential expressions, we only focused on null and pronominal occurrences in this study in line with much of the experimental work on this topic.}
We thus constructed a sample of 368 complex sentences.\footnote{The \isi{production} of null and \isi{pronominal subjects} in simple clauses is part of a separate study.} Following \citet{oakhill1989line}, who point out the role of semantic and pragmatic effects of main clause factors in pronoun resolution, and in line with previous studies \citep{soares2020effect, costa2004semantic, de2013effects}, we analyzed the \isi{production} of null and \isi{pronominal subjects} occurring in  subordinate clauses. 
Moreover, since main sentences generally did not provide information about the previous \isi{antecedent} in the context (such as syntactic function), we decided to analyze the choice of null and \isi{pronominal subjects} in subordinate sentences.
We excluded a number of occurrences in which the alternation between null and \isi{pronominal subjects} was not possible in the context or in which annotation of our factors of interest in this study was not possible. More precisely, we did not retain in our study sentences that met the following criteria: (i) when a null \isi{subject} was impossible in the context, i.e. \isi{pronominal subjects} of non-finite verbal forms (infinitives and gerunds), as in (\ref{gerund:Istrate}); (ii) when \isi{null subjects} or \isi{pronominal subjects} were discourse persons (1st or 2nd person) as in (\ref{disc_pers:Istrate}), since it is often impossible to establish information about the factors of interest such as the grammatical role of the \isi{antecedent};\footnote{The person factor was shown to influence the \isi{production} of referential expressions (see \citealt{soares2020effect} for Brazilian Portuguese). However, for the goal of this paper, we will only study anaphoric subjects.} (iii) when \isi{null subjects} were the only option in the context, i.e. \isi{null subjects} of impersonal \isi{reflexive} verbs (\ref{impers:Istrate}).

\ea
\ea \label{gerund:Istrate}  
\gll Când	ne	întâlnim,	vorbim	pe	ungurește,	el	fiind	pe jumătate	maghiar. \\
when \textsc{cl.1pl}.\textsc{acc}
meet.\textsc{prs.1pl} talk\textsc{prs.1pl} \textsc{prep} Hungarian he being \textsc{prep} half Hungarian \\
\glt `When we meet, we speak Hungarian, as he is half-Hungarian.' (corola-38914)      
\ex 
\gll Dar	când	trebuie	să fiu	eu,		atunci	roșesc,	ne	mărturisește.\\
but when have \textsc{sbjv} be.\textsc{sbjv.1sg} I then  blush\textsc{.prs.1sg} \textsc{cl.1pl.dat}
confess.\textsc{prs.3sg}  \\
\glt `But when I have to be myself, then I blush, he confesses.'  
\label{disc_pers:Istrate} 
\ex 
\gll Deși		starea		carosabilului	este	deplorabilă,	nu	se	pot	face lucrările de reabilitare	 necesare,		din lipsa fondurilor. \\
although state\textsc{.def} road.\textsc{def.gen} is deplorable \textsc{neg} \textsc{refl.3} can do work.\textsc{pl.def} of repair necessary of lack funds.\textsc{def.gen}  \\
\label{impers:Istrate} 
\glt `Although the state of the road is deplorable, the necessary repair work cannot be carried out due to a lack of funds.' (corola-56647)
\z
\z
During the data collection and annotation process of intra-sentential anaphoric subjects, we faced a number of problematic cases. Firstly, for both corpora we had to manually extract the complex sentences we were interested in in this study. With the data available to us, we unfortunately had to limit our analysis to a relatively small number of \isi{pronominal subjects} for the written corpus (68 occurrences). In order to have a balanced set of observations for statistical analyses, we applied the \textit{upSample} function from the \textit{caret} package \citep{Kuhn2008} that allows to add simulated observations without changing the general distribution.
Annotating antecedents was constrained at some points by limited access to the previous context (the sentence preceding the target sentence we annotated). This concerned in particular antecedents of pronouns in main clauses and is the main reason why, while we annotated each factor both for main clauses and subordinate clauses, we decided to analyze only the \isi{production} of null and \isi{pronominal subjects} occurring in subordinate sentences for the variables where properties of the \isi{antecedent} were at stake.

\subsection{Annotated factors}

Comparative studies on pro-drop languages (see \citealt{contemori2021microvariation, torregrossa2020variation}, among others) have revealed that crosslinguistic variation may exist in pronoun resolution as well in the choice of referential expressions in \isi{production}. It is, thus, not necessarily the case that results from previous studies can be taken for granted for Romanian. Following our main research question, we annotated a list of factors which have been shown to affect the choice of antecedents or referential expressions across several languages.  \tabref{tab:myname:annot_fact:Istrate} shows the 15 factors we manually annotated for complex sentences. 

\begin{table}
\caption{Factors used for annotation}
\label{tab:myname:annot_fact:Istrate}
\begin{tabularx}{1.0\textwidth}{XX}
 \lsptoprule
 Annotated element & Factors\\
  \midrule
Sentence & modality, polarity\\ %\hline
Subordinate clause & position (right, left) \\ %\hline
Adverbial clause & discourse relation, connectives \\ %\hline
Verb & agentivity, mood, tense, voice\\ %\hline
Subject & number, gender, animacy, type (null vs. pronominal), place \\  
& (main vs. subordinate) \\ %\hline
Antecedent & syntactic function \\   \lspbottomrule
\end{tabularx}
\end{table}

While we  annotated a variety of factors that may influence the choice of null and \isi{pronominal subjects}, our general question mainly concerns factors influencing the next mention probability, that means in particular factors affecting the prominence of a referent as well as discourse relations. We generally assume that higher prominence will increase the probability of null \isi{subject} choices. Following \citet{carminati2002processing}, we annotated the syntactic function of antecedents in order to test possible preferences for \isi{subject} antecedents. 
Animacy has been shown to have a strong influence on pronoun choice in corpus studies on Brazilian Portuguese (\citealt{soares2020effect, duarte2000loss} a.o.). Agentivity seems to predict preferences for pronoun antecedents aside from topicality and subjecthood in German \citep{schumacher2016thematic}.
Voice also seems to impact pronoun resolution. \citet{colonna2018effects} show that the use of passives increases the  accessibility of the \isi{subject} \isi{antecedent} (see also \citealt{d1973some, burmester2018sensitivity}). We predict that the \isi{salience}-enhancing effect of passives may also increase the probability of \isi{null subject} choices in \isi{production}. 
Moreover, as shown by \citet{rohde2014grammatical}, pronoun resolution might be sensitive to discourse relations. We therefore also annotated  discourse relations.
Next mention probability may also be affected by 
gender, which has been shown to affect choices of \isi{subject} types in Italian \citep{cacciari2011pronoun} but also \isi{antecedent} choices in English \citep{ferstl2011implicit}.
Beyond these factors, verb mood, tense and number have been found to play a role in the frequency of null and \isi{pronominal subjects} in Granada Spanish \citep{manjon2016factores}. 

\subsection{Results}

All data were analyzed using logistic regressions (\textit{glm} function in the \textit{lme4} package, cf. \citealt{bates2015package}, \textit{lmerTest} function for \textit{p}-values, cf. \citealt{kuznetsova2017lmertest}). Our data did not allow us to calculate a single general model with all factors, due to overfitting problems. We therefore analyzed the choice of  \isi{subject} type (dependent variable) based on the annotated factors in the main clause (independent variable) one by one. All factors were mean centered such that \textit{p}-values reflect main effects.\footnote{m = glm(response $\sim$ 
 FactorC, data=data, family = ``binomial"), where \textit{response} corresponds to \textit{\isi{subject} type}.} 
\tabref{raw_results:Istrate} shows the full set of results in raw numbers. For better comparability with data from previous studies, the variables \textit{Antecedent function}, \textit{Animacy}, \textit{Agentivity}, \textit{Voice}, \textit{Gender}, \textit{Number} and \textit{Discourse relations} reflect \isi{subject} choices in the subordinate clause. Descriptive (\tabref{raw_results:Istrate}) and inferential statistics (\tabref{tab:myname:results_fact:Istrate}) are calculated for subordinate clauses only for these variables. Effects of 
\textit{Mood}, \textit{Polarity} and \textit{Tense} were calculated across main and subordinate clauses.


\begin{table}
\caption{Descriptive statistics (raw numbers)}
\label{raw_results:Istrate}
\fitpagewidth{
\begin{tabularx}{1.0\textwidth}{XXXXX}
  \lsptoprule  
Factors & Values & Null subjects & Pronominal subjects & Total \\
 \midrule
%  Antecedent function & \isi{subject} & 155 & 69 & 224 \\ %\hline
%  & non-\isi{subject} & 34 & 38 & 72 \\ %\hline
Antecedent function & \isi{subject} & 149 & 31 & 180 \\ %\hline
& non-\isi{subject} & 30 & 30 & 60 \\ %\hline
  Discourse relations & temporal & 57 & 12 & 69 \\ %\hline
  & causal  & 47 & 48 & 95 \\ %\hline
  & condition & 11 & 7 & 18 \\ %\hline
  & concession & 18 & 2 & 20 \\ %\hline
  & result & 14 & 1 & 15 \\ %\hline
  & other & 35 & 2 & 37 \\ %\hline
  Animacy & animate & 151 & 56 & 207 \\ %\hline
  & non-animate & 31 & 16 & 47 \\ %\hline
  Agentivity & agentive & 93 & 15 & 108 \\ %\hline
  & non-agentive & 89 & 57 & 146 \\ %\hline
  Voice & active & 139 & 63 & 202 \\ %\hline
  & non-active & 43 & 9 & 52 \\ %\hline
  Gender (animates) & feminine & 43 & 44 & 87 \\ %\hline
  & masculine & 110 & 71 & 181 \\ %\hline
  & other & 14 & 16 & 30 \\ %\hline
  Number & singular & 136 & 48 & 184 \\ %\hline
  & plural & 46 & 24 & 70 \\ %\hline
  Polarity & affirmative & 179 & 158 & 337 \\ %\hline
  & negative & 21 & 10 & 31 \\ %\hline
  Mood & indicative & 181 & 153 & 334 \\ %\hline
  & conditional & 4 & 0 & 4 \\ %\hline
  & subjunctive & 15 & 15 & 30 \\ %\hline
  Tense & present & 122 & 109 & 231 \\ %\hline
  & compound past & 57 & 38 & 95  \\ %\hline
  & imperfect and pluperfect & 10 & 11 & 21 \\ %\hline
  & future & 10 & 10 & 20 \\  %\hline
 % Discourse \isi{connective} & \textit{pentru că} 'because' & 42 & 44 & 86 \\ %\hline
%  & \textit{când} 'when' & 49 & 24 & 73 \\ %\hline
%  & other & 109 & 100 & 209  \\
 \\      \lspbottomrule
\end{tabularx}
}
\end{table}


Different from previous studies, we did not find significant effects of animacy or number of the main clause \isi{subject}, or of polarity, mood, and tense.

\begin{table}
\caption{Inferential statistics}
\label{tab:myname:results_fact:Istrate}
\begin{tabularx}{1.0\textwidth}{XXXXX}
 \lsptoprule
Factors & Estimate & Std. error & z value & Pr (|z|)\\
  \midrule
Antecedent function & 1.570 & 0.325 & 4.831 & $<$ 0.001\\ %\hline
%Antecedent: spoken & -1.979 & 0.593 & 3.336 & p $<$ 0.001 \\ %\hline
%Antecedent: written& 2.379 & 0.7 & -3.397 & p $<$ 0.001 \\ %\hline
Discourse relations & 1.579 & 0.378 & 4.177 & $<$ 0.001 \\ %\hline
%Discourse marker & -0.666 & 0.275 & -2.423 & p $<$ 0.01 \\ %\hline
Animacy & 0.058 & 0.278 & 0.211 & 0.832 \\ %\hline
Voice & -0.773 & 0.397 & -1.947 & 0.052 \\ %\hline
Agentivity & 1.379 & 0.326 & 4.231 & $<$ 0.001 \\ %\hline
Gender (animates) & 0.522 & 0.249 & 2.097 & $<$ 0.05 \\ %\hline
%Number & -0.391 & 0.38 & 3.192 & p $<$ 0.001 \\ %\hline
Number & -0.391 & 0.303 & -1.292 & 0.197 \\ %\hline
Polarity & 0.617 & 0.399 & 1.545 & 0.122 \\ %\hline
Mood & 0.068 & 0.362 & 0.189 & 0.850 \\ %\hline
Tense & 0.166 & 0.217 & 0.767 & 0.443 
\\   \lspbottomrule
\end{tabularx}
\end{table}

\figref{ModSub:Istrate} and \figref{ModMain:Istrate} show the distribution of null and \isi{pronominal subjects} in main and subordinate clauses. Null subjects are more frequent in subordinate clauses while \isi{pronominal subjects} are more frequent in main clauses (Est. = -2.889, std. error = 0.288, z=-10.043, p $<$ 0.001). For both main and subordinate clauses, \isi{null subjects} are more frequent in the written modality (main clauses: Est. = 0.806, std. error = 0.328, z=2.46, p $<$ 0.05; subordinate clauses: Est. = -1.333, std. error = 0.333, z=-4.008, p $<$ 0.001).


\begin{figure}[h]
\centering
%\includegraphics[width=0.7\textwidth,height=0.5\textwidth]{../figures/ModalityEmbedded-Istrate.pdf}
\includegraphics[width=0.6\textwidth]{../figures/ModalityEmbedded-Istrate.pdf}
\caption{Modality subordinate clauses}
\label{ModSub:Istrate}
\end{figure}

\begin{figure}[h]
\centering
%\includegraphics[width=0.7\textwidth,height=0.5\textwidth]{../figures/ModalityMain-Istrate.pdf}
\includegraphics[width=0.6\textwidth]{../figures/ModalityMain-Istrate.pdf}
\caption{Modality main clauses}
\label{ModMain:Istrate}
\end{figure}

For the syntactic function of the \isi{antecedent}, we only analyzed antecedents for the subordinate clauses. Antecedents for main clauses were not possible to identify in more than 30\% of the cases. For the subordinate clauses, we found a significantly higher  frequency of \isi{null subjects} with a \isi{subject} \isi{antecedent} (see \tabref{tab:myname:results_fact:Istrate} and \figref{CorpusAnt:Istrate}). However, in the case of non-\isi{subject} antecedents, there is no preference for a null or pronominal \isi{subject}.

\begin{figure}[h]
\centering
%\includegraphics[width=0.7\textwidth,height=0.5\textwidth]{../figures/Antecedent-Istrate.pdf}
\includegraphics[width=0.6\textwidth]{../figures/Antecedent-Istrate.pdf}
\caption{Subject distribution (Antecedent function)}
\label{CorpusAnt:Istrate}
\end{figure}


With respect to factors that may increase the \isi{salience} of an \isi{antecedent}, we looked at the effects of animacy of the \isi{subject} and agentivity as well as voice of the verb in the main clause on \isi{subject} choices in the subordinate clause. Animacy did not show a reliable effect. For voice, we found that non-active voice in the main clause (most often passives) marginally favors null \isi{subject} pronouns in the upcoming subordinate clause (see \tabref{tab:myname:results_fact:Istrate} and \figref{voice:Istrate}).

\begin{figure}[h]
\centering
%\includegraphics[width=0.7\textwidth,height=0.5\textwidth]{../figures/Voice-Istrate.pdf}
\includegraphics[width=0.6\textwidth]{../figures/Voice-Istrate.pdf}
\caption{Subject distribution (Voice in main clause)}
\label{voice:Istrate}
\end{figure}

Another variable we assumed to be related to \isi{salience} is the agentivity of the verb in the main clause. Agents have a higher next mention probability. In line with this hypothesis, we found that agentive verbs in the main clause lead to significantly more \isi{null subjects} in the subordinate clause (see \tabref{tab:myname:results_fact:Istrate} and \figref{agentivity:Istrate}).

\begin{figure}[h]
\centering
%\includegraphics[width=0.7\textwidth,height=0.5\textwidth]{../figures/Agentivity-Istrate.pdf}
\includegraphics[width=0.6\textwidth]{../figures/Agentivity-Istrate.pdf}
\caption{Subject distribution (Agentivity in main clause)}
\label{agentivity:Istrate}
\end{figure}

Gender could be a further variable related to \isi{antecedent} \isi{salience}. Looking only at animate antecedents, we found that \isi{null subjects} were reliably more frequent for male antecedents while null and \isi{pronominal subjects} were equally distributed for female antecedents (see \tabref{raw_results:Istrate} and \tabref{tab:myname:results_fact:Istrate}). This result could be interpreted as evidence that male antecedents are seen as more salient (although more detailed analyses would be necessary to confirm this hypothesis). 

Finally, we looked at the distribution of null and overt subjects in the subordinate clauses based on discourse relations (\figref{subj_dist_disc:Istrate}). Temporal and causal relations were the most frequent in our data, thus we compared only these two relations through statistical analysis. In adverbial temporal subordinates, we found an increased tendency for producing \isi{null subjects} compared to \isi{pronominal subjects} (see \tabref{tab:myname:results_fact:Istrate}), whereas in causal subordinates the distribution between the two types of subjects is roughly balanced.

\begin{figure}[h]
%\begin{figure}[ht!]
\centering
%\includegraphics[width=0.7\textwidth,height=0.5\textwidth]{../figures/rel_discourse1-Istrate.pdf}
\includegraphics[width=0.6\textwidth]{../figures/rel_discourse1-Istrate.pdf}
\caption{Subject distribution (Discourse relations)}
\label{subj_dist_disc:Istrate}

\end{figure}




\subsection{Discussion: corpus study}
Overall, we observed that modality (written vs. spoken) has a clear influence on referential expression \isi{production} in Romanian. The lower frequency of \isi{pronominal subjects} in the written corpus might be due to the role of normative grammar reserving the use of \isi{pronominal subjects} solely for emphatic contexts (cf. \citealt{avram2001gramatica}). Moreover, we suggest that the higher frequency of \isi{pronominal subjects} in spoken corpora might be explained by some version of a noisy channel model \citep{gibson2013noisy}. According to this model, given the increased noise in oral communication, a rational speaker might opt to be more redundant in order to ensure the message is understood, thus using more \isi{pronominal subjects}. Regarding \isi{antecedent} function, \isi{null subjects} were mostly produced when referring  back to a \isi{subject} \isi{antecedent}, with an even higher proportion in the written modality. This tendency found for Romanian seems to be in line with previous findings in the literature, postulating that subjects make particular good antecedents \citep{crawley1990use, arnold1998reference}. Further, in a controlled experimental study using items with transitive verbs, \citet{istrate2022position} found a clear preference of \isi{null subjects} for \isi{subject} antecedents, although less categorical than in our \isi{corpus study} (see \citealt{de2013effects} for similar tendencies in Spanish). Moreover, in the \isi{corpus study}, the \isi{production} of \isi{pronominal subjects} did not show a clear preference for \isi{subject} or non-\isi{subject} antecedents. Grammatical properties of the antecedents, such as \textit {PAH} \citep{carminati2002processing}, seem to not fully capture the division of labor in null and \isi{pronominal subjects}. Moreover, following \citeauthor{givon1983topic}'s (\citeyear{givon1983topic}) \textit{Topic Continuity Hypothesis}, a preference for \isi{null subjects} with \isi{subject} antecedents may be explained by \isi{information structure} as suggested by \citet{mayol2010refining}. Subjects as default topics have a higher probability of being mentioned next in the discourse. 

Factors influencing the \isi{salience} of a referent in the discourse, such as agentivity, voice and possibly gender, were shown to play a role in our \isi{corpus study}. As argued before, we assume that higher \isi{salience} increases the probability of being mentioned next in the discourse and, thus, the probability of a null \isi{subject}.

Discourse relations also play a role in the present \isi{corpus study}. In temporal subordinates, \isi{null subjects} were more frequent, while for causal relations the results show no clear preference for a \isi{referential form}. How can these results be related to next mention probabilities or \isi{predictability}? 

The high frequency of \isi{null subjects} in temporal relations might be explained by the \textit{Topic Continuity Hypothesis} \citep{givon1983topic, runner2016information}. Temporal relations are typically part of a narration where topics rarely shift (see \ref{strasbourg:Istrate}). 

For causal relations, we suggest that next mention probabilities may play a role as observed in examples from our corpus data. Implicit \isi{causality} biases of the verbs may play a role here. In (\ref{save:Istrate}), a null \isi{subject} in the causal clause retrieves an \isi{object} \isi{antecedent} (\textit{doamnei Mihaela} `to Mrs. Mihaela'). The \isi{object} in this case is the most predictable referent due to the implicit \isi{causality} bias of a verb like \textit{a mulțumi} `thank' that induces an expectation for a reason why Mihaela should be thanked. 
Null subjects will  refer back to the \isi{object} \isi{antecedent} when it is foregrounded by the implicit \isi{causality} bias of the verb. The foregrounded \isi{antecedent} becomes the most predictable in the discourse. Subject-biased verbs like \textit{fascinate} would predict the \isi{subject} of the main clause to be mentioned next and to be referred to with a null \isi{subject} in the causal subordinate clause.

\ea \ea \label{strasbourg:Istrate}
\gll Procedând în acest fel, Martin Șluț și-a încălcat promisiunea făcută anul trecut, {atunci când} a fost ales în fruntea Parlamentului {de la} Strasburg.\\
doing in this manner Martin Șluț \textsc{refl.dat.3-aux.3sg} broke.\textsc{pst} promise.\textsc{def} made.\textsc{sg.f} year.\textsc{def} past {when} \textsc{aux.3sg} be.\textsc{pst} elected in head.\textsc{def} parliament.\textsc{def.gen} of Strasbourg \\
\glt`By doing so, Martin Sluț has broken the promise he made last year when he was elected to lead the Strasbourg Parliament.'
\ex \label{save:Istrate}
\gll  Mă numesc Nicolae Maria, doresc să-i mulțumesc doamnei Mihaela {pentru că} mi-a dezlegat cununia. \\
\textsc{cl.1sg.acc} name.\textsc{prs.1sg} Nicolae Maria wish.\textsc{prs.1sg} \textsc{sbjv-cl.3sg.dat} thank.\textsc{prs.1sg} madam.\textsc{gen} Mihaela because \textsc{cl.1sg.dat-aux.3sg} save.\textsc{pst} marriage.\textsc{def}\\
\glt `My name is Nicolae Maria, I would like to thank Mihaela because she saved my marriage.' (corola-32168)
\z
\z

The overall results of the \isi{corpus study} suggest that semantic-pragmatic factors seem to affect the \isi{production} and interpretation of null and \isi{pronominal subjects}. Their distribution was shown to be influenced by causal relations which differ considerably compared to temporal relations. However, while the role of implicit \isi{causality} biases for the choice of null and \isi{pronominal subjects} in the causal subordinate clause is plausible \citep{mayol2018asymmetries}, we cannot confirm it based on the corpus data alone. This is why we ran the controlled experimental study reported in the next section.

\section{Experiment: sentence completion task}
\label{exp:Istrate}

\subsection{Implicit causality verbs}

Our \isi{corpus study} showed no clear preference for null or \isi{pronominal subjects} for sentences with causal relations. In the following experimental study, we want to shed light on why this may be the case. In previous studies taking into account semantic-pragmatic factors, two main classes of verbs were tested for potential \isi{predictability} effects: implicit \isi{causality} verbs (see \citealt{caramazza1977comprehension, costa2004semantic, fukumura2010choosing, rohde2014grammatical, holler2016empirical, mayol2018asymmetries, weatherford2021semantic, bott2023production}) and transfer-of-possession verbs (\cite{stevenson1994thematic, rohde2008coherence, vogels2019both, lindemann2020thematic}). We will only focus on implicit \isi{causality} verbs in the following. So-called implicit \isi{causality} verbs possess an inherent causal meaning introducing a semantic bias towards continuations referring back to the entity related to the underlying causer of the event, which can appear in either \isi{subject} position (subject-biased implicit \isi{causality} verbs) or \isi{object} position (object-biased implicit \isi{causality} verbs). In (\ref{subj_bias:Istrate}), the \isi{verb bias} increases the prediction of the upcoming cause to be attributed to the \isi{subject} \textit{Mary} while in (\ref{obj_bias:Istrate}) the upcoming cause is predicted to be attributed to the \isi{object} \textit{Peter}.

\ea \ea \label{subj_bias:Istrate}
\textit{Mary} fascinated Peter because \ldots $\rightarrow$ \textit{Mary} more likely continuation
\\
\ex \label{obj_bias:Istrate} Mary criticized \textit{Peter} because \ldots $\rightarrow$ \textit{Peter} more likely continuation
\\
\z \z

\citet{ferstl2011implicit} moreover suggest that, beyond the \isi{next mention bias} invoked by the verb, the gender of the antecedents may play a role in that male antecedents have a slightly higher probability of being seen as the causer of an event. The general gender effect we found in our \isi{corpus study} makes a similar prediction.

\subsection{Methods}

\subsubsection{Hypotheses} 
With a sentence \isi{completion task}, we tried to answer the following hypotheses that are, to some extent, interconnected. 
According to the \textit{PAH} (\citealt{carminati2002processing}, and see also \textit{topic continuity} in \citealt{givon1983topic} or similar approaches), \isi{null subjects} have a strong tendency to go with a \isi{subject} \isi{antecedent}, which is compatible with the \isi{corpus study} presented in the previous section. This hypothesis predicts that participants produce more \isi{null subjects} when referring back to a \isi{subject} \isi{antecedent}. Pronominal subjects should be used more when participants refer back to non-subjects. If, however, \isi{null subjects} prefer more predictable antecedents that are likely to be mentioned next, verb biases may change the picture: Null subjects should be more frequent when the continuation is in line with the \isi{verb bias}.

The gender of the antecedents might also be affected by next mention biases, as suggested by \citet{ferstl2011implicit}. Our results from the previous \isi{corpus study} revealed that masculine antecedents are more prominent, thus we also expect to observe a higher preference for \isi{null subjects} with masculine antecedents compared to feminine ones. 

To sum up our hypotheses:
\begin{itemize}
\item A null \isi{subject} will be produced more often when retrieving a \isi{subject} \isi{antecedent}.
\item Based on the implicit \isi{causality} biases of the verbs, continuations should refer to the \isi{antecedent} foregrounded by the verb (the more predictable \isi{antecedent}). The choice of the referential expression (null vs. pronominal \isi{subject}) will then be influenced by the \isi{verb bias}.
\item Following results from \citet{ferstl2011implicit}, we may also find a gender effect with a preference to choose male antecedents as the causer of an event, leading to a preference for \isi{null subjects} to refer to male antecedents.
\end{itemize}

\subsubsection{Participants}
Thirty-one native Romanian speakers (age range 19 to 32 years, mean age: 27 years) participated in our experiment. All of the participants spent their childhood in Romania. They were students recruited at the University of Bucharest. Given that the participants are enrolled in an institution of higher education, their level of instruction is fairly homogeneous (a minimum of 12 years of instruction). Thus, the participants had no difficulty in reading, understanding, or continuing the sentences. Participation was voluntary and participants were not paid for their contribution. The experiment was run on a version of Ibex farm installed on a local server at Université Paris Cité. Participants’ data were immediately anonymized. At no moment was identifying information stored.

\subsubsection{Materials}
The experiment focuses on testing the \isi{production} of referential expressions (lexical vs. pronominal vs. null \isi{subject}) in Romanian as well as the preference for an \isi{antecedent} using a free passage \isi{completion task} with a paradigm similar to \citet{kehler2019prominence}. 
In order to increase the \isi{predictability} of an \isi{antecedent}, we chose implicit \isi{causality} verbs  that increase the next mention probability of the \isi{subject} as in (\ref{dezamm:Istrate})-(\ref{dezamv:Istrate}) with subject-biased verbs, or the \isi{object} as in (\ref{aleador:Istrate})-(\ref{albador:Istrate}) with object-biased verbs.

\ea \label{romanian_ic:Istrate}
\ea \gll Maria îl dezamăgește pe Victor {pentru că} \ldots \\
Maria \textsc{cl.3sg.m.acc} disappoint.\textsc{prs.3sg} \textsc{dom} Victor because \\
\glt `Maria disappoints Victor because...’ \label{dezamm:Istrate}
\ex \gll Victor o dezamăgește pe Maria {pentru că} \ldots \\
Victor \textsc{cl.3sg.f.acc} disappoint.\textsc{prs.3sg} \textsc{dom} Maria because \\
\glt `Victor disappoints Maria because \ldots’ \label{dezamv:Istrate}
\ex \gll Alexandra îl adoră pe Albert {pentru că} \ldots \\
Alexandra \textsc{cl.3sg.m.acc} adore.\textsc{prs.3sg} \textsc{dom} Albert because \\
\glt `Alexandra adores Albert because \ldots’ \label{aleador:Istrate}
\ex \gll Albert o adoră pe Alexandra {pentru că} \ldots \\
Albert \textsc{cl.3sg.f.acc} adore.\textsc{prs.3sg} \textsc{dom} Alexandra because \\
\glt `Albert adores Alexandra because \ldots’ \label{albador:Istrate}
\z \z

One of the antecedents was always a feminine first name, the other a masculine first name. We created two conditions for each sentence switching the gender of the \isi{subject} and the \isi{object} to test for possible gender effects as they were found in \citet{ferstl2011implicit}. Participants were asked to continue sentences following the pattern in (\ref{fem:Istrate}) using a plausible continuation of their choice (freely  choosing a referential expression for the \isi{subject} of the causal clause). According to the literature, participants continue with either a pronominal, a lexical or a null \isi{subject} in more than 85\% of the cases (see \citealt{kehler2019prominence}).

\ea Female/Male first name + implicit \isi{causality} \isi{subject}/object-bias verb + Male/Female first name + \textit{because}
\label{fem:Istrate}
\z

In order to create our experimental items, we selected a total of 48 implicit \isi{causality} verbs (24 subject-biased verbs and 24 object-biased verbs) chosen from the database created by \citet{ferstl2011implicit}. Given the fact that there is no similar database in Romanian, we based the choice of our verbs on the English verbs with the highest implicit \isi{causality} biases (above 70\%) which were then  translated and adapted to Romanian. The verbs as well as the experimental items were reviewed by an independent native Romanian speaker (other than the creators and annotators of the experiment). We selected approximately 50\% of verbs with a positive connotation (e.g. \textit{impress, congratulate}) and 50\% with a negative connotation (e.g. \textit{disappoint, envy}). The names used for our items were very common, well-known, typical Romanian names to limit other potential biases as much as possible. 


\subsection {Procedure}

Completions in the kind of task we use here can be free or constrained. In a constrained \isi{completion task}, participants are invited to write completions referring back to an entity that is somehow marked. For the unconstrained or free sentence \isi{completion task} that we applied,  participants were asked to complete the items without any constraints for the \isi{antecedent}, providing likely continuations to the given sentences.  Relying on the strength of the implicit \isi{causality} biases, we opted for a free \isi{completion task} (for a discussion of advantages and disadvantages of both paradigms, see \citealt{demberg2023systematic}). The task was conducted online on the Ibex Farm platform at Université Paris Cité (created by Alex Drummond and maintained by Achille Falaise). The experiment began with instructions for the task as well as a series of demographic questions (age, gender, first language, i.e. language spoken since early childhood). Participants gave their informed consent to the use of their anonymized data for research purposes. A total of 1071 completions were annotated excluding ambiguous or inappropriate answers. The continuations were independently annotated separately by two native Romanian speakers (both coauthors of the paper) with respect to the intended \isi{antecedent} of the continuation as well as with respect to the referential expression used for the \isi{subject} of the causal subordinate sentence introduced by \textit{because}.

\subsection {Results} 
 In the annotation process, the \isi{antecedent} of a null \isi{subject} cannot be determined by syntactic markers given the nature of a null pronoun.  
Hence, when \isi{null subjects} were produced, \isi{antecedent} choice was determined by the meaning of the causal subordinate clause. The two annotators agreed on all decisions.\footnote{E.g., in a sentence like \textit{Peter thanked David because he proofread the thesis}, it is highly plausible to assume that the null \isi{subject} refers back to the \isi{object} of thanking.} All data were analyzed using logistic regressions (\textit{glmer} function in the \textit{lme4} package, cf. \citealt{bates2015package}, \textit{p}-value being estimated using \textit{lmerTest}, cf.  \citealt{kuznetsova2017lmertest}). We first analyzed the effect of implicit \isi{causality} and gender on \isi{antecedent} choice. Gender of the \isi{subject} of the root clause as well as \isi{verb bias} were added as mean centered fixed factors and participants and items as random factors. Random slopes could not be added due to convergence failure. This is true for all models presented here. As shown in \figref{object_ant:Istrate}, participants’ continuations were highly consistent with the \isi{verb bias}. They mostly chose a continuation consistent with a \isi{subject} \isi{antecedent} after subject-biased verbs and with the \isi{object} \isi{antecedent} after object-biased verbs (Est. = -6.90, std. error = .6176, z=-11.179, p $<$$ $.001). There was also a small numeric effect of gender with slightly more \isi{subject} choices (less \isi{object} choices) when the \isi{subject} \isi{antecedent} was male (Est. = -.5938, std. error = .3586, z=-1.656, p = .0977).

\begin{figure}[h] 
%\includegraphics[width=0.7\textwidth,height=0.5\textwidth]{../figures/Fig1-exp-Istrate.pdf}
\includegraphics[width=0.6\textwidth]{../figures/Fig1-exp-Istrate.pdf}
\caption{Next mention and verb bias}
\label{object_ant:Istrate}
\end{figure}

Participants chose null or \isi{pronominal subjects} in more than 95\% of the cases. We therefore excluded other referential expressions from our analyses. Participants overwhelmingly produced \isi{null subjects} in the \textit{because}-subordinate clause independently of \isi{verb bias} (Est. = 4.8020, std. error = .6262, z=7.669, p < .001). As shown in \figref{vbias_nullsubj:Istrate}, \isi{null subjects} were moreover chosen more frequently for sentences with subject-biased verbs where \isi{verb bias} and the general preference of \isi{null subjects} for \isi{subject} antecedents align (Est. = 1.6934, std. error = .5866, z=2.887, p $<$.01). We finally looked at the frequencies of null and \isi{pronominal subjects} depending on the \isi{antecedent} choice made by the participants. Logistic regressions included \isi{antecedent} choice and \isi{verb bias} as fixed factors (both mean centered) and participants and items as random factors. \figref{vbias_antec:Istrate} shows that participants chose \isi{null subjects} more often in cases where the \isi{verb bias} and the \isi{antecedent} choice aligned, i.e. when they produced a continuation consistent with an \isi{object} \isi{antecedent} in sentences with object-biased verbs or a continuation consistent with a \isi{subject} \isi{antecedent} in sentences with subject-biased verbs (Est. = 4.3433, std. error = 1.2946, z=3.355, p $<$ .001).

\begin{figure}[h]
%\includegraphics[width=0.7\textwidth,height=0.5\textwidth]{../figures/Fig2-exp-Istrate.pdf}
\includegraphics[width=0.6\textwidth]{../figures/Fig2-exp-Istrate.pdf}
\caption{Subject antecedents and verb bias}
\label{vbias_nullsubj:Istrate}
\end{figure}

\begin{figure}[h]
%\includegraphics[width=0.7\textwidth,height=0.5\textwidth]{../figures/Fig3-exp-Istrate.pdf}
\includegraphics[width=0.6\textwidth]{../figures/Fig3-exp-Istrate.pdf}
\caption{Subject choice and verb bias}
\label{vbias_antec:Istrate}
\end{figure}

\subsection {Discussion} 
%-------------------------------------------------------------------------------------------
All in all, the results of our free passage \isi{completion task} in Romanian confirm the hypothesis that \isi{null subjects} are the most preferred \isi{referential form} to retrieve \isi{subject} antecedents. However, \isi{null subjects} can also easily be produced for non-\isi{subject} antecedents when they are highly predictable in the context. 
This can be interpreted as is clear evidence against at least a simple version of the \textit{PAH} \citep{carminati2002processing} that stipulates a general preference of null \isi{subject} pronouns for \isi{subject} antecedents. While it might be argued that the experimental situation could lead to non-natural productions from the participants, the continuations produced by the participants with \isi{object} antecedents in the case of object-biased verbs (see \ref{perm:Istrate} and \ref{premiu:Istrate} for examples from the experiment) were judged as highly natural by both native speakers of Romanian who annotated them.

\ea
\ea \gll Laura îl felicită pe Ionuț {pentru că} a luat permisul.\\
Laura \textsc{cl.3sg.m.acc} congratulate.\textsc{prs.3sg} \textsc{dom} Ionuț because \textsc{aux.3sg} take.\textsc{pst} license.\textsc{def.m.sg}\\
\glt 
`Laura congratulates Ionuț because he got his driving license.’ \label{perm:Istrate}
\ex \gll Ionuț o felicită pe Laura {pentru că} a câștigat un permiu.\\
Ionuț \textsc{cl.3sg.f.acc} congratulate.\textsc{prs.3sg} \textsc{dom} Laura because    \textsc{aux.3sg} win.\textsc{pst} a prize\\
\glt `Ionuț congratulates Laura because she won a prize.’ \label{premiu:Istrate}
\z \z


In an experimental study on pronoun choice and thematic roles, \citet{lindemann2020thematic} suggest a very similar pattern for Romanian. Their results also show a general preference of \isi{null subjects} for \isi{subject} antecedents with no clear preference for \isi{pronominal subjects}. Interestingly, in their study, thematic roles (goal vs. source) affected the \isi{production} of referring expressions alongside the grammatical role of the antecedents. Null subjects were more often used to retrieve \textit{goal} referents, i.e. more prominent or predictable referents.

In our experiment, we observe that participants produced \isi{pronominal subjects} more often when the \isi{antecedent} was less salient or predictable. However, other cases of \isi{pronominal subjects} (as in \ref{dezam:Istrate} and \ref{invid:Istrate}) included continuations that contained a contrast between antecedents (see \citealt{dobrovie2013reference} and \citealt{mayol2010refining} for similar suggestions). The role of contrast has more recently also been confirmed in experimental studies by \citet{istrate2024subject}.

\ea 
\ea \gll Maria îl dezamăgește pe Victor {pentru că} și el a făcut același lucru.\\
Maria \textsc{cl.3sg.m.acc} disappoint.\textsc{prs.3sg} \textsc{dom} Victor because also he    \textsc{aux.3sg} do.\textsc{pst} same thing\\
\glt `Maria disappoints Victor because he also did the same thing.’ \label{dezam:Istrate}
\ex \gll Victor o invidiază pe Maria {pentru că} ea are note {mai bune} decât el.\\
Victor \textsc{cl.3sg.f.acc} envy.\textsc{prs.3sg} \textsc{dom} Maria because she has grades better than him\\
\glt `Victor envies Mary because she has better grades than him.’ \label{invid:Istrate}
\z
\z

With respect to the \isi{predictability} effect induced by implicit \isi{causality} verbs, tendencies in Romanian are fairly similar to those suggested by \citet{bott2023production} for German in which the referent \isi{predictability} was shown to have a strong impact on pronoun \isi{production}. Moreover, from a crosslinguistic perspective, Romanian seems to align with preferences found for other pro-drop Romance languages (see \citealt{contemori2021microvariation} for Italian and Spanish, \citealt{mayol2018asymmetries} for Catalan), i.e. null \isi{subject} pronouns will be favoured for more predictable referents. For European Portuguese, \citet{costa2004semantic} showed similar preferences for \isi{null subjects}, but a different pattern in the case of \isi{pronominal subjects}, which were used more often for \isi{object} antecedents foregrounded by an object-biased verb.
Regarding gender biases, we found a small numeric gender effect (following \citealt{ferstl2011implicit}) in our experimental data.

Unlike \citet{bott2023production} and \citet{rosa2017predictability}, who suggest that \isi{predictability} effects are stronger with same-gender antecedents, we found that next-mention bias can also play a strong role when using different-gender antecedents in an implicit \isi{causality} experiment. 


\section{General discussion}
\label{sect:discussion:Istrate}

In our \isi{corpus study}, we found that, while \isi{null subjects} have a strong preference for \isi{subject} antecedents as predicted by the \textit{PAH} \citep{carminati2002processing}, other more semantic-pragmatic factors also play a role. In particular, prominence enhancing factors such as \textit{voice} or \textit{agentivity} (and potentially \textit{gender}) but also \textit{modality} affect the choice of the referential expression. Moreover, discourse relations seem to play a role in that null and \isi{pronominal subjects} are equally distributed in causal relations but not in temporal relations. 

In our experimental study, we tried to better understand the specific pattern for causal relations. Despite a slight general preference for \isi{subject} antecedents, \isi{null subjects} were shown to be strongly preferred as \isi{referential form} for non-\isi{subject} antecedents as well when they were predictable enough in the context. More concretely, \isi{object} antecedents were retrieved mostly by a null \isi{subject} in the context of causal \isi{coherence relations}, with an implicit \isi{causality} verb biased towards the \isi{object} (e.g., \textit{congratulate}). While \isi{null subjects} were preferred when the continuations aligned with the \isi{verb bias}, \isi{pronominal subjects} were generally used by participants in continuations which go against the \isi{verb bias} (for example, in contrastive contexts; for similar suggestions see \citealt{mayol2010refining}). 

Moreover, putting our \isi{corpus study} and experimental results together, we contribute to a broader research question which is under significant debate, more specifically, whether \isi{predictability} influences the choice of referring expressions (cf. \citealt{arnold2001effect, fukumura2010choosing, rohde2014grammatical, holler2016empirical, modi2017modeling, rosa2017predictability} a.o.). Both corpus and experimental evidence in Romanian suggest that higher \isi{predictability} \citep{tily2009refer} triggers a clear preference for null \isi{subject} pronouns. This effect of referent \isi{predictability} is also in line with previous hypotheses in the literature on the role of \isi{salience} or accessibility \citep{givon1983topic, ariel1994interpreting, grosz1995centering, chafe1996inferring} or prominence \citep{von2019discourse}. As suggested by \citet{demberg2023systematic}, the \isi{next mention bias} may be triggered by a complex interaction of two semantic-pragmatic factors, i.e. the \isi{verb bias} and \isi{coherence relations}. 

In general, our data on Romanian replicate results from previous studies \citep{costa2004semantic, mayol2018asymmetries, contemori2021microvariation} on  languages from the Romance family, while they are inconsistent with previous data from Mandarin Chinese \citep{hwang2022role}. More crosslinguistic studies are needed to establish in how far general pro-drop patterns in a language may play a role here.

Data and materials are accessible here:
\href{https://osf.io/fmjnq/}{https://osf.io/fmjnq/}.

\section*{Acknowledgements}

We want to thank Anne Abeillé and Gabriela Bîlbîie for their ideas concerning the annotated factors in the corpus study. We also thank Verginica Barbu-Mititelu for her assistance in collecting data from Romanian corpora. And finally we would like to thank Labex EFL and Smarts-Up (Idex Université Paris Cité) for co-funding our work.


\sloppy
\printbibliography[heading=subbibliography,notkeyword=this]
\end{document}

%%% Local Variables:
%%% mode: xelatex
%%% TeX-master: t
%%% End:
