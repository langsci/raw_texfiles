\documentclass[output=paper,colorlinks,citecolor=brown]{langscibook}
\ChapterDOI{10.5281/zenodo.15450440}
\author{Jacques Jayez\orcid{} \affiliation{ENS de Lyon \& LORIA, CNRS}}
\title{Discourse markers are not special (but they can be complicated)}
\abstract{This chapter discusses the identification and representation of \textit{discourse markers} (\isi{DMs}), indexical elements which contribute to the organization of discourse, the manifestation of the speaker or the interaction between speaker and addressee(s). I argue that, in spite of their diversity, \isi{DMs} fall into two broad classes, \textit{connective} \isi{DMs}, which trigger \isi{presuppositions}, and \textit{Hic et Nunc Particles}, which trigger \isi{conventional implicatures} and, like \citeauthor{Potts:2007}'s (\citeyear{Potts:2007}) expressives, are anchored to the utterance situation. In order to lay the foundations for a representation, I extend the interactional framework of \citet{Ginzburg:2012} to incorporate the contribution of \isi{DMs} in a flexible way.
}


\IfFileExists{../localcommands.tex}{
	\addbibresource{../localbibliography.bib}
	% add all extra packages you need to load to this file

\usepackage{tabularx,multicol}
\usepackage{url}
\urlstyle{same}

\usepackage{listings}
\lstset{basicstyle=\ttfamily,tabsize=2,breaklines=true}

\usepackage{langsci-basic}
\usepackage{langsci-optional}
\usepackage{langsci-lgr}
\usepackage{langsci-osl}
% \usepackage{./langsci/styles/langsci-lgr}
% \usepackage{./langsci/styles/langsci-osl}
% \usepackage{langsci-gb4e}

\usepackage{tikz}
\usetikzlibrary{patterns,calc}
\pgfdeclarepatternformonly{south east lines}{\pgfqpoint{-0pt}{-0pt}}{\pgfqpoint{3pt}{3pt}}{\pgfqpoint{3pt}{3pt}}{
    \pgfsetlinewidth{0.6pt}
    \pgfpathmoveto{\pgfqpoint{0pt}{3pt}}
    \pgfpathlineto{\pgfqpoint{3pt}{0pt}}
    \pgfpathmoveto{\pgfqpoint{.2pt}{-.2pt}}
    \pgfpathlineto{\pgfqpoint{-.2pt}{.2pt}}
    \pgfpathmoveto{\pgfqpoint{3.2pt}{2.8pt}}
    \pgfpathlineto{\pgfqpoint{2.8pt}{3.2pt}}
    \pgfusepath{stroke}}
    
\usepackage{stmaryrd}
\usepackage{wasysym}
\usepackage{multirow}
\usepackage{caption}
\usepackage{subcaption}
\usepackage{mathrsfs}
\usepackage{qtree}

\usepackage{linguex}


	%pminos do not split footnotes
% \interfootnotelinepenalty=10000 %Footnote in Laporte chapters has to be split SN


%\DeclareIndexNameFormat{default}{%
%\nameparts{#1}%
%\usebibmacro{index:name}%
%{\index[names]}%
%{\namepartfamily}%
%{\namepartgiveni}%
% {}% L1
% {}% L2
%{\namepartprefix}% generates spurious space L3
%{\namepartsuffix}% generates spurious space L4
%}

%  {\DeclareIndexNameFormat{default}{%
%     \usebibmacro{index:name}{\index[names]}{#1}{#3}{#5}{#7}}}

%\DeclareIndexNameFormat{default}{%
%  \usebibmacro{index:name}{\sindex[nom]}{#1}{#3}{#5}{#7}}

%\DeclareIndexNameFormat{default}{%
%  \usebibmacro{index:name}{\sindex[person]}{#1}{#3}{#5}{#7}}
%\DeclareIndexNameFormat{default}{%
%\nameparts{#1} \usebibmacro{index:name}{\sindex[person]]}{\namepartfamily}{‌​\namepartgiven}{\nam‌​epartprefix}{\namepa‌​rtsuffix}}

%\newcommand{\smiley}{:)}

%\renewbibmacro*{index:name}[5]{%
%\usebibmacro{index:entry}{#1}%
%{\iffieldundef{usera}{}{\thefield{usera}\actualoperator}\mkbibindexname{#2}{#3}{#4}{#5}}}

% \newcommand{\noop}[1]{}

%remove for final
%\overfullrule=1mm

\newcommand{\tobi}[2]}}
\renewcommand{\S}[1]{\tobi{#1}{\textsc{*}}}

% this volume references
% puts: [this volume]
% already defined: \citetv
%\newcommand{\citepv}[1]{(\citeauthor{#1} \citeyear*{#1} [this volume])}
\newcommand{\citealtv}[1]{\citeauthor{#1} \citeyear*{#1} [this volume]}

%parentheses around example number
\newcommand{\pref}[1]{(\ref{#1})}

% in-text examples

\newcommand{\lnex}[1]{\textit{#1}} %target lang word
\newcommand{\lnlit}[1]{(lit.: `#1')} %literal reading
\newcommand{\lnlat}[1]{(#1)} % latinization
\newcommand{\lntrans}[1]{`#1'} %translation
\newcommand{\lnexl}[2]%
{\lnex{#1}{} \lnlat{#2}} % ex with latinization
\newcommand{\lnexlat}[3]{\lnex{#1}{} \lnlat{#2}{} \lntrans{#3}} % ex with latinization and tranl.

%ch01
\newcommand{\co}[1]{\mbox{\textbf{#1}}}

%ch09

\newcommand{\cyrbulg}[1]{\begin{otherlanguage*}{bulgarian}#1\end{otherlanguage*}}


%ch10
\newcommand{\nlp}{{\small NLP}}
\newcommand{\mwe}{{\small MWE}}
\newcommand{\rae}{{\small RAE}}
\newcommand{\lvc}{{\small LVC}}
\newcommand{\pos}{{\small P}o{\small S}}
%\newcommand{\todo}[1]{ \textcolor{red}{#1} }

%\renewcommand{\labelenumi}{\theenumi}
%\ainamefmt{{vv}{ll}{, ff}{, jj}} % fullname

\newcommand{\biberror}[1]{{\color{red}#1}}

\newcommand{\osenovaitem}{--~}
	%% hyphenation points for line breaks
%% Normally, automatic hyphenation in LaTeX is very good
%% If a word is mis-hyphenated, add it to this file
%%
%% add information to TeX file before \begin{document} with:
%% %% hyphenation points for line breaks
%% Normally, automatic hyphenation in LaTeX is very good
%% If a word is mis-hyphenated, add it to this file
%%
%% add information to TeX file before \begin{document} with:
%% %% hyphenation points for line breaks
%% Normally, automatic hyphenation in LaTeX is very good
%% If a word is mis-hyphenated, add it to this file
%%
%% add information to TeX file before \begin{document} with:
%% \include{localhyphenation}
\hyphenation{
    Beck-man
    Ngu-yen
    back-chan-nel
    back-chan-nels
    mo-not-o-nous
    ste-reo-typ-i-cal
}

\hyphenation{
    Beck-man
    Ngu-yen
    back-chan-nel
    back-chan-nels
    mo-not-o-nous
    ste-reo-typ-i-cal
}

\hyphenation{
    Beck-man
    Ngu-yen
    back-chan-nel
    back-chan-nels
    mo-not-o-nous
    ste-reo-typ-i-cal
}

	\boolfalse{bookcompile}
	\togglepaper[23]%%chapternumber
}{}

\begin{document}
	\maketitle

	\section{Introduction}

	This chapter deals with the semantic status of \textit{discourse markers} (\isi{DMs}) in French. As for instance in \citet[151]{Degand:2014}, I take them to be words or expressions whose main function is to suggest \textit{discourse relations} (\isi{DRs}) between semantic objects such as propositions, to refer to epistemic or affective states of the speaker or to her interaction with other conversational agents. Some typical examples in French are \isi{DMs} which suggest consequence, like \textit{donc} `therefore', \textit{du coup} `so', etc., or contrast, like \textit{mais} `but', \textit{pourtant} `yet', etc., and \isi{DMs} like \textit{ah} or \textit{bon} `well'. \isi{DMs} have a paradoxical status. On the one hand, they have spawned an abundant literature in many languages. On the other hand, the mosaic of specific meanings and monographic studies makes it difficult to discern a more global image, if any, and relate it to general semantic categories. Although I do not deny the obvious fact that specific \isi{DMs} must be distinguished by fine-grained and sometimes very subtle aspects of meaning, I contend that, as a group, \isi{DMs} are not special, since they fit pretty closely into a well-known classification of three general semantic kinds, namely \isi{modifiers} participating in the \isi{propositional content}, \isi{presupposition} triggers and \isi{conventional implicature} triggers.
    
	In \sectref{sec-dimensions:Jayez}, I present a number of relevant distinctions, starting from the basic ones (\sectref{sec-basic-distinctions:Jayez}) and indicating my position about the identity of \isi{DMs} (\sectref{sec-DM-identity:Jayez}). In \sectref{sec-connective-DM:Jayez}, I show how to type \isi{connective} \isi{DMs} (\sectref{sec-unique-DR:Jayez}) and present their presuppositional status (\sectref{sec-DM-as-PP-triggers:Jayez}) and their representation (\sectref{sec-representation1:Jayez}). Next, I show in \sectref{sec-HN-particles:Jayez} how to apply the same kind of analysis to the second major class of \isi{DMs}, the \textit{Hic et Nunc Particles} (\isi{HNPs}).

\section{Introducing DM}
\label{sec-dimensions:Jayez}

\subsection{Basic distinctions}
\label{sec-basic-distinctions:Jayez}


In the 90's, Gisela Redeker and Eve Sweetser independently proposed to classify \isi{DM} uses by the kind of \isi{discourse relations} (\isi{DRs}) or \isi{reference types}, i.e. the objects they refer to. \citet{Redeker:1990} sees \isi{DMs} as denoting two different types of \textit{coherence} relations (in her terminology): \textit{ideational} or \textit{pragmatic}. Ideational \isi{DRs} correspond to relations which hold in the real world\footnote{The term is borrowed from Halliday, see for instance \citep[84]{Halliday:2003}.} while pragmatic \isi{DRs} concern either the beliefs and intentions motivating the related utterances (\textit{rhetorical} \isi{DRs}) or the management of discourse evolution (\textit{sequential} \isi{DRs}). Temporal, causal, elaborative, reason and consequence relations are examples of ideational \isi{DRs}. Concession, justification, and contrast illustrate pragmatic-rhetorical \isi{DRs}. Topic change, correction and digression illustrate pragmatic-sequential \isi{DRs}.

For \citet{Sweetser:1990}, a \textit{domain} is the type of the entity a \isi{DM} refers to in a given context (a \isi{reference type}). For instance, in (\ref{late:Jayez}), the \textit{parce que} `because' \isi{DM} connects two states of affairs. Other examples show that French \textit{parce que}, like its English counterpart \textit{because}, can involve belief states (\ref{car:Jayez}) or \isi{speech acts} (\ref{watch:Jayez}). Sweetser focuses on what she calls ``pragmatic ambiguity'', the idea that the same \isi{semantic content} can concern various \isi{reference types}, typical examples for English being \textit{because} or \textit{if} \citep{Sweetser:1990,Dancygier:1998}. For instance, the same causal relation associates states of affairs in (\ref{late:Jayez}) and belief states in (\ref{car:Jayez}). The main difference between Redeker's and Sweetser's perspectives is the fact that Redeker is not concerned with \isi{reference types}, as apparent from her comment on \textit{because} examples parallel to (\ref{car:Jayez}) \citep[373]{Redeker:1990}.


	\ea
	\label{late:Jayez}
	\gll Paul est {en retard} {parce qu'} il a raté son train.\\
	Paul is late because he has missed his train\\
 \glt `Paul is late because he missed his train.'
%	\glt `I think therefore I am.'
	\z
%	\il{Latin} %add "Latin" to language index for this page

\ea
\label{car:Jayez}
\gll Paul est chez lui {parce que} sa voiture est devant la maison.\\
Paul is at him because his car is {in-front} the house\\
\glt `Paul is at home because his car is in front of the house.'
\z
\ea
\label{watch:Jayez}
\gll Tu as l' heure? {Parce que} j' ai oublié ma montre.\\
you have the time because I have forgotten my watch\\
\glt `Do you have the time? Because I forgot my watch.'
\z

It is difficult to arbitrate between these two views on general and abstract grounds and I will follow the cautious policy of more recent work in distinguishing domains (\isi{reference types}) and functions \citep{Crible:2018,CribleDeGand:2019}, which essentially amounts to (i) keeping \isi{DR} types and \isi{reference types}, and (ii) determining whether these two dimensions are independent on an empirical, non \textit{a priori}, basis.

Another relevant distinction concerns the semantic contribution of \isi{DMs}. Some \isi{DMs} contribute to the \isi{propositional content} (\isi{PC})\footnote{Other labels include \textit{at-issue content}, \textit{main content}, \textit{main point} and \textit{proffered content}.} of the utterance whereas others don't. Again, this raises the question of dimension independence: is the contribution dimension of (non-)participation to the \isi{PC} separated or not from domain and function dimensions? For instance, should we claim that \isi{DMs} can contribute to the \isi{PC} only when they express a causal relation between states of affairs (an ideational relation), as in (\ref{late:Jayez})?  We will see in the next section that contribution to the \isi{PC} is indeed limited to the ideational interpretation of \isi{DMs}.

Finally, there is the question of displaceability, a criterion that \citet{Potts:2007} borrowed from \citet{Hockett:1960} and applied to expressives like swearwords or evaluative/emotional adjectives in non-predicative constructions (\textit{my stupid neighbor}, etc.). The linguistic tests for displaceability may be sometimes difficult to deploy but they prove useful to isolate the important category of \isi{HNPs} \citep{Dargnat:2024} in \sectref{sec-HN-particles:Jayez}.

In the next section I address the question of the contribution of \isi{DMs} to the \isi{PC}, in relation to the problem of the boundaries of the category ``\isi{discourse marker}''.

\subsection{DM or not DM?}
\label{sec-DM-identity:Jayez}
\largerpage[1]
Many books or papers on \isi{DMs} start with enumerating a list of features supposed to characterize \isi{DMs}. Classic references include works by \citet{Schiffrin:1987}, \citet{Brinton:1996}, \citet{Schourup:1999}, \citet{Aijmer:2002,Aijmer:2013}, \citet{Beeching:2016}. The general impression is tersely summarized by \citet[30]{Brinton:1996}: ``The definitions of pragmatic particles found in the literature seem to bear little resemblance to one another.''
More recent works \citep{Crible:2017,FedrianiandSansopaper:2017,Heineetal:2021} tend to take a step back and keep only the most frequently mentioned features or those which are supported by corpus data. A typical example is the list\footnote{For a thorough discussion of various features attributed to \isi{DMs}, see \citet[section 2.1]{Brinton:1996}.} of \citet[1209]{Heine:2013}, where \isi{DMs} are presented as in (\ref{list:Jayez}):

\ea
\label{list:Jayez}
\ea syntactically independent from their environment,\\
\ex typically set off prosodically from the rest of the utterance,\\
\ex \label{list3:Jayez}
non-restrictive in their meaning (``not part of the propositional meaning of a clause''),\\
\ex procedural rather than conceptual-propositional,\\
\ex non-compositional and, as a rule, short.
\z
\z

A simple illustrative example is (\ref{ex:sick:Jayez}), given by \citet[3]{Heineetal:2021}. In (\ref{ex:still-sick:Jayez}), \textit{still} is a constituent of the sentence and contributes to its meaning, whereas the same item does not show these two properties in (\ref{ex:is-sick:Jayez}). Moreover it is much easier to prosodically isolate \textit{still} in (\ref{ex:is-sick:Jayez}) than in (\ref{ex:still-sick:Jayez}). 

\ea \label{ex:sick:Jayez}
\ea \label{ex:still-sick:Jayez}
John is still sick.\\
\ex \label{ex:is-sick:Jayez}
Still, John is sick.
\z
\z
\largerpage
\noindent In \citet[6]{Heineetal:2021}, the function of \isi{DMs} is defined as ``metatextual, being anchored in the situation of discourse and serving the organization of texts, the attitudes of the speaker, and/or speaker–hearer interaction.'' Consider (\ref{late:Jayez}) in this perspective. The \textit{parce que} `because' conjunction can contribute to the \isi{PC}, as evidenced by the usual tests of negation, interrogation and \textit{it}-clefting:\footnote{I use \textit{it}-clefting whenever possible because it is less susceptible to variation in semantic intuition.} \textit{Paul was not late because he missed his train, but for another reason}, etc. Therefore, according to (\ref{list3:Jayez}), it does not behave as a \isi{DM}. This limitation is operative in a number of approaches, in particular \isi{cooptation} \citep{Heine:2013,Heineetal:2021} or some versions of Construction Grammars, as illustrated by \citeauthor{Traugott:2018}'s (\citeyear[27]{Traugott:2018}) definition: ``By a \isi{DM} I mean a metatextual marker that signals some kind of relationship between clauses/utterances.'' Traugott acknowledges the fact that this is a ``restrictive definition''.

In other approaches \citep[e.g.,][]{Webberetal:2017}, items contributing to the \isi{PC} are not kept apart from other resources involved in discourse coherence. Given that (i) such items serve one of the central functions ascribed to \isi{DMs}, namely structuring discourse, and (ii) in a number of cases, there is an interesting similarity between the \isi{PC} and non-\isi{PC} uses, I adopt the liberal view that items relevant to the \isi{PC} must be included in the family of \isi{DMs}, and show that a basic semantic function can underlie \isi{PC} and non-\isi{PC} contributions via typing constraints.

\section{Typing connective DMs}

\label{sec-connective-DM:Jayez}

Connective \isi{DMs} are those \isi{DMs} which express a \isi{discourse relation} between two semantic objects. In French, they can be subordinating conjunctions like \textit{parce que} `because' or \textit{si} `if' or adverbials like \textit{donc} `therefore/so' or \textit{après} `after'. This prompts two questions: what kind of semantic objects can such \isi{DMs} relate and what kind of contribution to the discourse do they make? I address these questions in turn in the next two sections.



\subsection{Unique discourse relation}

\label{sec-unique-DR:Jayez}

In this section, I consider the kind of semantic configuration studied by \citet{Sweetser:1990}, that is, cases where, while there is some variation on the type of semantic objects the \isi{DM} connects, there is a strong intuition that the connection corresponds in every case to the same \isi{discourse relation}, for instance \isi{causality}, consequence or opposition.

In examples (\ref{late:Jayez}), (\ref{car:Jayez}) and (\ref{watch:Jayez}), we met the ubiquitous \isi{DM} \textit{parce que} `because', which was mentioned by Sweetser (\citeyear{Sweetser:1990}) as open to pragmatic ambiguity: one meaning (causal) but several possible \isi{reference types}. When the \isi{DM} \textit{parce que} contributes to the \isi{PC}, it can only relate states of affairs. Using our \textit{it}-cleft test once more, we observe that, when the host clause of \textit{parce que} is in focus, the belief \isi{reference type} is not possible. So, the following variant of (\ref{car:Jayez}) in (\ref{car-var:Jayez}) can only mean that there is some causal, non-epistemic, relation between the location of the car and the presence of Paul. By turning the second sentence into a psychological state of affairs we preserve the causal link (\textit{It's because $\dots$ that I think that he is at home}).


\ea
\label{car-var:Jayez}
\gll C'est {parce que} la voiture de Paul est devant la maison qu' il est chez lui.\\
it's because the car of Paul is {in-front} the house that he is at him\\
\glt `It's because Paul's car is in front of the house that he is at home.'
\z

It is not very surprising that contributing to the \isi{PC} and connecting states of affairs coincide if we assume that the role of the \isi{PC} is precisely to describe states of affairs, whether physical, psychological, social or mathematical. This is precisely one of the properties that theories which distinguish between, for instance, force and thought (Frege), mood and radical (Stenius), modus and dictum (Meillet), illocutionary force and \isi{PC} (Searle), try to capture. It does not follow that \textit{not} contributing to the \isi{PC} disallows reference to states of affairs. Let us consider again the abductive interpretation of \textit{parce que} `because' illustrated in (\ref{car:Jayez}). Given the general structure \textit{A parce que B} `A because B', there are nine possible denotation choices and as many paraphrases. For instance, the choice of a connection \textsc{soa}--\textsc{soa}\footnote{\textsc{soa} is the type of states of affairs, \textsc{bel} that of belief states and \textsc{sa} that of \isi{speech acts}.} produces the paraphrase \textit{the fact that Paul's car is in front of the house is the cause of the fact that he is at home}, which, as noted above for (\ref{car-var:Jayez}), is implausible. In fact, there are only four plausible patterns. The <\textsc{bel} cause \textsc{sa}> and the <\textsc{bel} cause \textsc{bel}> ones have paraphrases of the form \textit{the belief that Paul's car is in front of the house is the cause of the belief/assertion that he is at home}. The paraphrases for the patterns <\textsc{soa} cause \textsc{bel}> or <\textsc{soa} cause \textsc{sa}> are \textit{the fact that Paul's car is in front the house is the cause of the belief/assertion that he is at home}. These relations are all enthymematic in some way. For instance, the presence of the car in front of the house can trigger the belief that Paul is at home only if we supplement the relation with the additional premiss that the speaker is aware of the car's location.

The abductive interpretation for (\ref{car:Jayez}) sheds light on the nature of the difference between Redeker's and Sweetser's approaches. For \citet{Redeker:1990}, an abductive interpretation would count as a justification, a non-ideational relation which does not hold in the real world. For \citet{Sweetser:1990}, there is a general relation of causation, which forms the semantic core of \textit{because} and can be specified in difference domains (\isi{reference types}). If we accept that cognitive (beliefs) and communicative (\isi{speech acts}) processes are grounded in the real world, it would probably be more accurate to say that Redeker's ideational relations concern those causal processes which do not involve intentional behavior.\footnote{The intentionality perspective is central to \isi{discourse relation} models such as \textit{Rhetorical Structure Theory} \citep{TaboadaandMann:2006}.}


Given the abductive distribution for \textit{parce que}, one may wonder whether the exclusion of <\textsc{soa} relation \textsc{soa}> is a general property of non-\isi{PC} interpretations or a specificity of causation and possibly other relations. A natural follow-up is then to consider other types of relation. The concession relation is \textit{a priori} a good candidate since it has been described as a negative counterpart of \isi{causality} \citep{Verhagen:2005}. Concessive \isi{DMs} like \textit{bien que} `although', \textit{mais} `but', \textit{pourtant} `yet', etc., do not contribute to the \isi{PC} and have no corresponding verb expressing concession, unlike \textit{because} with its partners \textit{to cause}, \textit{to bring about} or \textit{to trigger}. In order to force the \textsc{soa} reading, I use perceptual verbs, which prevent the abductive reading, since one cannot infer that one sees (observes, witnesses) something.\footnote{More precisely, one can infer that what one sees is such or such object, but, under normal circumstances, an agent cannot infer the existence of the seeing event itself.} While the possible \isi{reference types} for (\ref{bien-que-a:Jayez}) cannot be determined, (\ref{bien-que-b:Jayez}) forces the \textsc{soa} attribution. Examples like (\ref{bien-que-b:Jayez}) indicate that non-PC-contributing \isi{DMs} can associate \textsc{soa}-denoting elements. Adverbial elements like \textit{mais} `but', \textit{pourtant} `yet', etc. exhibit the same behavior. Therefore, contributing to the \isi{PC} and denoting \textsc{soa} are not equivalent.\footnote{For similar findings, see \citet{JayezandRossari:2001} for consequence \isi{DM}.} When a \isi{DM} contributes to the \isi{PC}, it must relate \textsc{soa}s, when it does not, it can also target \textsc{soa}s.

\ea
\label{bien-que:Jayez}
\ea \label{bien-que-a:Jayez}
\gll Paul n' est pas chez lui {bien que} sa voiture soit devant la maison.\\
Paul \textsc{neg} is not at him although his car be.\textsc{sbjv.3sg} {in-front} the house\\
\glt `Paul {is not} at home although his car is in front of the house.'\\
\ex \label{bien-que-b:Jayez}
\gll Je constate que Paul n' est pas chez lui {bien que} j' aie vu que sa voiture était devant la maison.\\
I observe that Paul \textsc{neg} is not at him although I have.\textsc{sbjv.1sg} seen that his car was {in-front} the house\\
\glt `I observe that Paul is not at home although I saw that his car is in front of the house.'
\z
\z


%\subsubsection{Similarity of function}
%\label{sec-function-similarity:Jayez}
%In  a number of cases the putative non-\isi{DM} uses serve the organization of texts, exactly like `genuine' \isi{DM} uses. It has been repeatedly noted that lexical elements like \textit{because} help hearers or readers to detect discourse relations and assign structure to discourse (see \cite{Yungetal:2017} for a recent reappraisal). In French, the same is true for temporal markers like \textit{après} (\textit{after}) or \textit{ensuite} (\textit{next}) or conditional markers like \textit{si} (\textit{if}), which can contribute to the \isi{PC} \textit{and} constitute a resource for text structuring.
%same function,proximity of meaning, distinction present for \isi{DM} that cannot contribute to the \isi{PC}, diachronic 'enrichement' ('alors que')

\subsection{Proximity of meaning}
\label{sec-meaning-proximity:Jayez}
\largerpage[-1]

There are some items which, like \textit{parce que} `because', (i) feature more than one of Sweetser's domains (\isi{reference types}) and can contribute to the \isi{PC} or not, but which (ii) do not exhibit the same apparent identity of meaning across their uses. In French, this is for instance the case of temporal markers like \textit{après} `after', \textit{ensuite} `then', \textit{maintenant} `now', \textit{en même temps} `simultaneously' and some conditional markers like \textit{si} `if' or \textit{au cas où} `in case'. For example, in the dialogue given in (\ref{theatre:Jayez}), B uses \textit{après} as a concessive marker. The exchange in (\ref{theatre:Jayez}) takes place in the context of a discussion where B complains about her financial limitations. Speaker A notes that there are relatively cheap theater seats available. In a probabilistic setting \citep{Jayez:2024,Winterstein:2010} based on the confirmation approach \citep{CrupiandTentori:2016}, A's assertion makes the probability that B books a seat increase whereas B's answer makes it decrease. With a pattern $d_1 \mbox{ \textit{après} } d_2$, where $d_1$ and $d_2$ are two discourse chunks, \textit{après} can signal either that an eventuality described by $d_2$ temporally follows an eventuality described by $d_1$ (the \isi{PC} use) or that there is a proposition $p$ such that (i) $d_1$ can be interpreted as an argument in favor of $p$, by making its probability rise, and (ii) $d_2$ makes the probability of $p$ decrease or cancels the raising effect of $d_1$ (the non-\isi{PC} use).\footnote{In the language of modern argumentation theory, $d_2$ can either \textit{attack} $p$ or constitute an \textit{undercut}, that is attack the relevance of $d_1$ or its force of connection with $p$ \citep[see for instance][]{Walton:2013}.}

\ea
\label{theatre:Jayez}
\gll A: mais tu peux avoir des places de théâtre [\dots] qui valent vachement le coup.\\
{A:} but you can have some seats of theater [\dots] which worth really the deal\\
\glt `But you can get theater seats which are a bloody great deal.'\\
\gll B: ouais mais après faut voir le truc qu' y a aussi.\\
{B:} yeah but after need see.\textsc{inf} the thing that there has also\\
\glt `Yeah but you have also to see what the program is.'\\
\gll C: après il y en a beaucoup?\\
     {C:} after it \textsc{cl.loc} \textsc{cl.gen} is many?\\
\glt `There are many of them?'\\
(ESLO, \url{http://eslo.huma-num.fr/CorpusEslo/html/fiche/enregistrement?id=1332})
\z


As noted by a reviewer, \textit{mais} `but' would also be appropriate in (\ref{theatre:Jayez}). However, the two \isi{DMs} are not synonymous. One can directly reject $p$ with \textit{mais} whereas the same move would be awkward with \textit{après}, as shown in (\ref{mais-apres:Jayez}).\footnote{The sentence would be much more natural with \textit{il peut se tromper} `he might be wrong' instead of \textit{il se trompe} `he is wrong'.} As proposed by \citet{LeDraoulecandRebeyrolle:2018}, I assume that the concessive use of \textit{après} retains something of its temporal anaphoric use. The fact that other \isi{DMs} can also express concession but not direct rejection -- as is the case with \textit{ensuite} `then' (expressing succession), \textit{en même temps} `simultaneously' (expressing anaphoric simultaneity) or \textit{maintenant} `now' (expressing deictic simultaneity) -- being thus very similar to \textit{après}, suggests that there is some non-accidental connection between the temporal and concessive uses.\footnote{See \citet[chapter 2]{Aijmer:2002} for a detailed analysis of \textit{now} which converges in many respects with the remarks made here and parallel observations for Romanian \citep{Walters:2023}. The potential of originally temporal items for marking contrast is pointed out by \citet[189]{Kortmann:1997}, see \textit{tandis que} and \textit{alors que} `while/whereas' analyzed in \citet{CombettesandDargnat:2024}. However, the type of concession under consideration is not mentioned by \citet{Kortmann:1997} nor by \citet{Heberlein:2011} for Latin.} I have no convincing explanation to offer for the emergence of a concessive value in those cases, but the temporal aspect can explain the difference with \textit{mais}. Concretely, if saying $\alpha$ is an argument for $p$ and saying $\beta$ counteracts this move, emphasizing a temporal location with a concession makes sense only if we keep the potential conclusions of both moves `alive'. The marking of temporal organization is used in enumerations to mark the limits between different elements or in contrasts (\textit{on the one hand / on the other hand}, etc.) to mark the balancing elements. It has then in general a demarcating function. The same function serves the presentation of two opposite arguments in concession. If the second move completely neutralized the argumentative impact of the first, there would be no need to give prominence to any kind of temporal demarcation since, with only one effective argument, there would be nothing to distinguish. While there is no temporal succession for \textit{maintenant} `now' and \textit{en même temps} `simultaneously', the demarcation function is still salient. To summarize, since all four items introduce a concessive move, the demarcation function associated with their temporal meaning makes direct rejections infelicitous. These items can also introduce a non-assertive speech act, as shown by the alternative answer C in (\ref{theatre:Jayez}), which hints at the possibility that the most advantageous seats are perhaps not so many.\footnote{The coexistence of a temporal and argumentative function for the \isi{DM} \textit{après} might correspond to the trajectory described in \isi{cooptation} from intra-sentential function to discourse function but also to the role assigned to \textit{subjectification} in some semantic approaches \citep{Verhagen:2005}. Concerning subjectification, while the existence of a path from external events to internal psychological events seems well-attested, one may wonder why it leads to a concessive and not also to an additive interpretation. For example, why is \textit{après}  semantically different from \textit{de plus} `moreover/in addition/besides', which has very likely followed a similar path? The \isi{DM} \textit{de plus}, like the cognate expressions \textit{en plus} and \textit{qui plus est}, is used to add an argument in the same direction. This might reflect the Latin comparative origin, the Latin \textit{plus} marking a higher degree on a qualitative or quantitative scale. When shifting to an argumentative interpretation, it is possible that the initial comparative value is transposed as the mention of additional argument, which reinforces the plausibility or relevance of the conclusion.}

\largerpage
\ea
\label{mais-apres:Jayez}
\gll
Paul pense que Marie est chez elle, (\{mais / ??après / ??maintenant / ??ensuite / ??{en même temps}\}) il se trompe.\\
Paul thinks that Mary is at her but / after / now / next / simultaneously he \textsc{refl.3} is-wrong\\
\glt `Paul thinks that Mary is at home, but he is wrong.'
\z

In some cases, an item does not contribute to the \isi{PC} but otherwise shows the same type of variation as \textit{après} and similar expressions. A well-studied item of this type is \textit{déjà} `already' \citep{MH:2008,Squartini:2014}. Less well-known cases include iterative items, such as \textit{encore une fois} lit. `again once', \textit{une fois encore} `once again' or \textit{pour la dernière fois} `for the last time', and additive ones like \textit{aussi} `too/also'. In (\ref{iterative-additive1:Jayez}), the \isi{presupposition} trigger \textit{encore une fois} `once again' is not part of the \isi{PC}. For instance, in contrast to a temporal adverb, it cannot be clefted. It signals that Paul already failed an exam (perhaps the same type of exam as the current one), exactly like \textit{encore} `again'. In (\ref{iterative-additive2:Jayez}), the same item is preferentially interpreted as meaning that the speaker repeats something which has been said before or paraphrases a belief that has been around. In clause-internal position, \textit{pour la dernière fois} `for the last time' signals that Paul will not have the possibility to fail the exam again. In clause-initial position, it indicates that the speaker is not willing to repeat what she says. \textit{Aussi} in clause-internal position has the same possible interpretations as \textit{too} in English. In spoken French, in clause-initial or clause-final position, it communicates either that the speaker adds something to the ongoing discourse (like \textit{also} in spoken English) or that she lets the addressee infer from the host sentence that there is some condition which (i) might have prevented some harmful event (generally clear from the context) and (ii) has not or might have not occurred. In (\ref{iterative-additive3:Jayez}), A mentions that Paul failed his exam (an adverse event) and B lets A infer that it is possible that Paul did not fulfill a standard precondition for not failing an exam, i.e. studying for it.


\ea
\label{iterative-additive1:Jayez}
\gll Paul a encore {une fois} échoué à son examen.\\
Paul has again once failed to his exam\\
\glt `Paul once again failed his exam.'
\z

\ea
\label{iterative-additive2:Jayez}
\gll Encore {une fois}, Paul a échoué à son examen\\
again once Paul has failed to his exam\\
\glt `Once again, Paul failed his exam.'
\z


\ea
\label{iterative-additive3:Jayez}
\gll A: Paul a encore échoué à son examen.\\
A: Paul has again failed to his exam\\
\glt `Paul again failed his exam.'\\
\gll B: Aussi est-ce qu' il a révisé?\\
B: also is-this that he has revised\\
\glt `But did he study for it?'
\z


\citet{Heineetal:2021} give numerous examples of items which begin their life as ideational indicators before extending their scope over discourse units and developing a discourse-oriented meaning. The association of \isi{DMs} with peripheries is in fact a very general phenomenon \citep{vanolmenandSink}, which is not limited to items contributing to the \isi{PC} in some of their uses. Another path of evolution from \isi{PC} uses to non-\isi{PC} uses is a categorial change, like when, for instance, a prepositional phrase becomes an adverbial \isi{discourse marker} (see \citealt{FagardandCharolles:2018} for the French \textit{d'ailleurs} `besides'), or a verb in the imperative loses its complement(s) and can scope over a whole clause which it relates argumentatively to a previous discourse sequence (see \citealt{Brinton:1996,Traugott:2018a,Dostie:2004} for English \textit{look} and its French counterpart \textit{regarde}).\footnote{A similar move is observed for some verbs in the indicative (see \citealt{Andersen:2007,Brinton:2008} for French \textit{tu sais} and its English counterpart \textit{you know}). However, the resulting \isi{DMs} are more naturally analyzed as ``particles'' in the sense of \sectref{sec-HN-particles:Jayez}.} In the next two subsections, I focus on the presuppositional status of certain \isi{DMs} and its consequences on their representation.




\subsection{DMs as presupposition triggers}
\label{sec-DM-as-PP-triggers:Jayez}

In this section, I argue that a number of \isi{DMs}, often called \textit{connectives}, are \isi{presupposition} (\isi{PSP}) triggers and I introduce the basic elements of a formal representation. The \isi{DMs} in question express \isi{DRs} and connect two objects, with the ability to pick up non-discursive elements of the utterance situation as terms of the relation -- at least for some of them. We mentioned in \sectref{sec-meaning-proximity:Jayez} the case of \textit{encore} `again', \textit{encore une fois} `once again' and \textit{aussi} `also/too', which are parallel to their English counterparts. The idea that \isi{connective} \isi{DMs} are \isi{PSP} triggers was introduced in \citet{Jayez:2004a} and, subsequently, developed or rediscovered in \citet{DargnatandJayez:2020,Pavese:2023,Stokke:2017}.

There are in fact two possible ways of presupposing for \isi{DMs}. Trivially, \isi{connective} \isi{DMs} need some \isi{antecedent} to make sense. One uses \textit{afterwards}, \textit{therefore} or \textit{although} with respect to some element of the discourse or utterance situation. This corresponds to the concept of ``cohesion'' of \citet{HallidayandHasan:1976} and has been explicitly assimilated to \isi{anaphora} by \citet{Berrendonner:1993}, whose connection with \isi{PSP} is well-known \citep{vanderSandt:1992,Geurts:1999}. In addition, for some \isi{DMs}, the relation with the \isi{antecedent} (= the \isi{DR}) is itself presupposed. Consider the following example, a variant of \citet[example 13]{DargnatandJayez:2020}. Suppose that a company has decided to hire an employee and set up a recruiting committee for this goal. According to the recruiting regulations, any member of the committee who knows a candidate personally is supposed to not participate in the interview. We observe that, like any \isi{PSP} trigger, \isi{DMs} like \textit{donc} `therefore', \textit{du coup} `so', \textit{alors} `then', etc., are not affected by negation (\ref{PP-neg:Jayez}) or interrogation (\ref{PP-interr:Jayez}): the consequence \isi{discourse relation} holds in the scope of the negative and interrogative operators. The conditionalization test in (\ref{PP-condit:Jayez}) makes the consequence relation dependent on the existence of the rule. Moreover, example (\ref{local-effect:Jayez}) shows that the \isi{DMs} have ``local effect'' \citep{Tonhauseretal:2013}, that is, the consequence relation can be endorsed by Mary. Examples (\ref{PP:Jayez})--(\ref{local-effect:Jayez}) are different from more usual examples with pronouns or definite descriptions, where the existence of the \isi{antecedent} is preserved or suspended, like in \textit{The committee member does not know Annie}, which presupposes that there is an identifiable particular committee member or \textit{if a committee is set up they will examine the applications}, where the existence of a committee is not presupposed. In the case at hand, we could manipulate two different \isi{PSPs}. One is the trivial \isi{antecedent}, which is the premiss of the rule and the target of the \isi{anaphora} (= the \isi{DM}). When this premiss is suspended, the conclusion is also suspended: if Paul does not know Annie X, he may attend her interview.\footnote{Barring other circumstances, the spontaneous extension from \textit{If Paul knows Annie then he may not attend the interview} to \textit{If he doesn't know her then he may attend the interview} is, of course, an instance of the invalid but natural move known as \textit{conditional perfection}, an enrichment of the consequence relation (see \citealt{CarianiandRips:2023} for an evaluation of its robustness).} In that case, the rule still takes effect. The other \isi{PSP} is the rule itself, which is explicitly suspended in (\ref{PP-condit:Jayez}). We observe similar effects with concessive \isi{DMs} like \textit{mais} `but' or \textit{pourtant} `yet'.

\ea
\label{PP:Jayez}
\ea \label{PP-neg:Jayez}
Il n'est pas vrai que Paul connaît personnellement Annie X et (\{donc / du coup\}) ne peut assister à son audition.\\
`It is not the case that Paul knows Annie X personally and (\{therefore / so\}) may not attend her interview.'
\ex \label{PP-interr:Jayez}
Est-il vrai que Paul connaît personnellement Annie X et (\{donc / du coup\}) ne peut assister à son audition?\\
`Is it the case that Paul knows Annie X personally and (\{therefore / so\}) may not attend her interview?'
\ex \label{PP-condit:Jayez}
Si réellement il y a cette règle sur les membres du comité qui connaissent un(e) candidat(e), Paul ne peut (\{donc / du coup / alors\}) pas assister à l'audition d'Annie X\\
`If, really, there is that rule about committee members who know personally a candidate, (\{therefore / so / then\}) Paul may not attend Annie X's interview.'
\z
\z

\ea \label{local-effect:Jayez}
Marie pense que Paul ne peut (\{donc / du coup / alors\}) pas assister à l'audition d'Annie X.\\
`Mary thinks that (\{therefore / so / then\}), Paul may not attend Annie X's interview.'
\z

Consequence and concessive \isi{DMs} have two main features. They do not contribute to the \isi{PC} and they suggest some dependence between semantic objects, for instance \textsc{soa}. Items like \textit{parce que} `because' can contribute to the \isi{PC}. When they do, the trivial \isi{PSP}, that is, the \textsc{soa} for which \textit{parce que} introduces a cause, is still there: negating (\ref{late:Jayez}) gives \textit{Ce n'est pas parce que Paul a raté son train qu'il est en retard} (`It's not because Paul missed his train that he is late'), which presupposes that he is late. For syntactic reasons, things are more complicated for non-PC-contributing uses but we can exploit the conditionalization test of (\ref{PP-condit:Jayez}).\footnote{A veterinary to an inexperienced assistant: \textit{If they become edgy when you enter their cage, stay away \upshape{(\textsc{sa})} \textit{because we risk a problem}}.}

In dialogues, another complication is the possibility, for causal and consequence connectives, to target preconditions of or inferences drawn from the speech act of another participant. In (\ref{connectives-in-dialogue:Jayez}), B targets a \isi{PSP}. Other possibilities concern temptative inferences to the cause of the asserted \isi{PC} (C's answer) or its social legitimacy (D's answer).

\ea \label{connectives-in-dialogue:Jayez}
A: Paul a encore raté son examen.\\
A: `Paul has failed his exam again.'\\
B: (\{Parce qu' / donc\}) il l'avait déjà raté ?\\
B: (lit. `because / so') `he had already failed it?'\\
C: (\{Parce qu' / donc\}) il n'avait pas révisé?\\
C: (lit. `because / so') `he had not revised?'\\
D: (\{Parce que / donc\}) tu crois que ça me regarde?\\
D: (lit. `because / so') `You think it's my business?'
\z


\subsection{Representation}
\label{sec-representation1:Jayez}


The ingredients we need for a basic representation of \isi{connective} \isi{DMs} all have to do with the kind of context update that such \isi{DMs} bring about. The approach in \citet{Ginzburg:2012} provides a good starting point for two reasons. It uses the well-understood, visually clear and flexible technique of \textit{records} and \textit{record types},\footnote{Records are finite sets of <attribute = value> pairs and record types finite sets of <attribute : type> pairs, where the colon indicates a typing judgment. I will not be concerned here with some technical aspects of records, in particular the stratified type construction detailed in \citet{Cooper:2023}.} familiar from computer science, and it gives a central role to the notion of discourse move, intended to capture the dynamic effect of discourse actions. This said, my use of the Ginzburg's framework is opportunistic and there are a number of aspects (in particular, \isi{commitment} structure and the typing of \isi{DM} arguments) which have been added to the various record structures, since, in their initial form, they were not intended as tools for describing \isi{DM}. When studying discourse moves, there are a number of general points to take into account.

	Firstly, there  is the the well-known notion of \textit{Question Under Discussion} (\isi{QUD}), the current topic of discourse \citep{Roberts:2012}. \isi{QUD} is used in particular to separate elements of the \isi{PC} from \isi{PSP} or other non-\isi{PC} elements (in particular, \isi{conventional implicatures}). This is crucial for \isi{DMs} like \textit{parce que} `because' or \textit{après} `after' which have \isi{PC} and non-\isi{PC} uses.

	Secondly, we also need to accommodate the notion of \textit{commitment}, whose impportance has been demonstrated by \citet{Krifka:2023} and \citet{MacFarlane:2011} for assertion, and which is generalized here to other facets of discourse moves. Although there is some debate about the status of \isi{commitment} and intention in communication \citep{Geurts:2019,Harris:2019}, I will adopt the view that expressing certain parameters in terms of \isi{commitment} is more neutral than using intentions, and that it is always possible to make intentions enter the picture at a later stage if it is deemed useful. For instance, assertions will be minimally associated with a \isi{commitment} to the truth of the \isi{PC}, not with an intention to be believed, even though this intention can be attributed to the speaker in most contexts.
	
	Thirdly, the process of updating discourse states may be represented in several ways, but there are two general principles to be observed: (i) respecting the cumulative nature of discourse, that is, the fact that previous discourse moves remain theoretically accessible (discourse information is not lost at every fresh move); (ii) participants can react to various `aspects' or `facets' of moves, for instance they can raise an issue about the \isi{PC}, a \isi{PSP}, a \isi{conventional implicature} or the (in)appropriateness of an expression. The \isi{QUD} records only what the speaker considers as the default trend of discourse, that is, the contents to which participants are supposed to react and which they are supposed to elaborate, in the absence of evidence to the contrary. As shown in \sectref{sec-DM-as-PP-triggers:Jayez}, presuppositional \isi{DMs} display a dependence on previous discourse. By using such a \isi{DM}, the speaker gets committed to the existence of a possible \isi{antecedent} and, for non-PC-contributing \isi{DMs}, to that of an appropriate \isi{discourse relation}. However, these commitments are not reflected in the \isi{QUD}, since the speaker does not intend (by default) to propose the presupposed elements to the discussion.

A basic idea in Ginzburg's framework is to develop a library of moves as transitions between \textit{Dialogue Game Boards} (\isi{DGBs}), which record the state of the conversational agents at some point in discourse. \isi{DMs} contribute to such transitions in a way which is specified in the \isi{DGB} representations to follow. In order to illustrate the general architecture of the \isi{DGB}, I will present a simple case in a stepwise fashion, commenting on (\ref{pcq-pc-move:Jayez}), which corresponds to a (partial) \isi{DGB} capturing the transition associated with the PC-use of \textit{parce que}, as in (\ref{late:Jayez}). Each attribute (\textsc{cmt}, for \isi{commitment}, \textsc{qud} and \textsc{moves}) gets a value which is the result of an update ($\oplus$).

\ea \label{pcq-pc-move:Jayez}
\avm[attributes=\scshape,values=\upshape]{
			[
			spkr & $a$\\
			addr & $b$\\
			cmt & $sit \models pc \wedge \exists x !(! x : \textbf{pc} \wedge \textit{fant}_\text{pc}!(!b,m,pc,\textit{parce que}!)! =$\\
			& $\textit{fant}_\text{pc}!(!a,m,pc,\textit{parce que}!)! = x \wedge cause!(!pc,x!)!!)! \oplus c$\\
			qud & $sit \models pc \wedge \textit{cause}\mbox{!(!}pc,\textit{fant}_\text{\textbf{pc}}\mbox{!(!}a,m,pc,\textit{parce que}\mbox{!)!!)!} \oplus q$\\
			moves & $\mbox{\textit{assert}!(!}a,b,cause\mbox{!(!}pc,\textit{fant}_\text{\textbf{pc}}\mbox{!(!}a,m,pc,\textit{parce que}\mbox{!)!!)!},sit\mbox{!)!} \oplus \phantom{x}$\\
			& $!(!\mbox{\textit{assert}!(!}a,b,pc,sit\mbox{!)!} \oplus m!)!$
			]
		}
        \z


%\begin{table}[H]
	%\caption{The record for \textit{parce que} in the PC}
	%\label{pcq-pc-move:Jayez}
%\end{table}

Let us start with \textsc{moves}. The existing list of moves ($m$) is updated with
a compact form of \textit{IllocProp} object (an element of Ginzburg's ontology), using a pattern \textit{rel}(\textit{spkr},\textit{addr},\textit{pc},\textit{sit}), meaning that a speaker communicates to an addressee that a certain illocutionary force (\textit{rel}) applies to a \isi{PC} \textit{pc} in the situation \textit{sit}. In the case at hand, the $m$ update is decomposed into two elements: $\mbox{\textit{assert}}(a,b,pc,sit)$ denotes that the speaker ($a$) communicates to $b$ an assertion about a certain \isi{PC} ($pc$) in the discourse situation ($sit$). $pc$ is the \isi{PC} of the sentence introduced by \textit{parce que}, i.e. \textit{il a raté son train} `he missed his train' in (\ref{late:Jayez}). The other element is an assertion by $a$ to $b$ that the same \isi{PC} is a cause of a certain \textsc{soa}, the value of the cryptic function \textit{fant}.

To clarify what \textit{fant} is supposed to mean, I return to the issue of presuppositional \isi{DMs}. One might require that \isi{PSPs} be recorded in \textsc{facts} (a standard attribute of \isi{DGBs}). However, the situation of \isi{DMs} is more subtle. \citet{Debaisieux:2004} shows that, in natural conversation, the \isi{antecedent} of \textit{parce que} is constructed through a fast inferential process and is not necessarily available before the \isi{DM} occurrence nor accommodated out of cooperativeness. Even in more simple cases, where the \isi{DR} is relatively transparent, it is perhaps not as salient as an object picked up by a pronoun or a definite description. So, I use a more flexible constraint than the presence inside \textsc{facts}, specifically a \isi{commitment} of the speaker to the possibility for the addressee to identify \textit{some} plausible referent (weak version) or the same referent as the speaker's chosen one (strong version). For space reasons, I will only mention the strong version here. The \textit{fant$_\text{type}$} function (for `fetch \isi{antecedent} of type \textit{type}') function provides a black box for general \isi{antecedent} recovery procedures. It denotes the result of extracting an \isi{antecedent} of type \textit{type} from the previous moves (so from $m$) and has four parameters: the agent (speaker or hearer in general) looking for an \isi{antecedent}, the list of moves to explore ($m$), the \isi{PC} and the \isi{DM} on which the exploration is based.
As a result, the $\textit{assert}(a,b,cause(pc,\textit{fant}_\text{pc}(a,m,pc,\textit{parce que},sit)))$ stands for an assertion by $a$ to $b$ that the \isi{PC} is a cause of whatever propositional \isi{antecedent} $a$ recovers from $m$ in $sit$. Note that the \isi{antecedent} must be a \isi{PC}, the basic ingredient of a \textsc{soa}, because we want the causal link to connect two PCs. However, the fact that the utterance situation satisfies the \isi{PC} is not declared in \textsc{moves} but in \textsc{qud}, since \textsc{moves} contains only \isi{speech acts}.

The \textsc{qud} update material is quite simple: the discourse situation satisfies the \isi{PC} $(sit \models pc)$ and the causal relation asserted in the update of $m$. Finally, the \textsc{cmt} attribute is updated with the fact that $sit$ satisfies the \isi{PC} (as for the \textsc{qud}) and also the existence of a suitable \isi{antecedent} $x$, shared by $a$ and $b$ and such that the \isi{PC} is a cause of $x$. In the simplest case of an \textit{A parce que B} structure, the \isi{antecedent} is just the \isi{PC} of \textit{A}, to be recovered from the last element of $m$.


In (\ref{pcq-other-move:Jayez}) and in an abductive configuration (\ref{pcq-other-move:Jayez-1}), the causal relation is not in the \isi{PC}, which gives us two presupposed elements: the \isi{antecedent} (like in (\ref{pcq-pc-move:Jayez})) and the causation. The type of the \textit{fant} function is changed to \textsc{bel} in order to take into account the fact that the \isi{antecedent} should be a belief. The same configuration is used for a speech act \isi{antecedent} when \textit{parce que} introduces an explanation/justification of the speech act (\ref{pcq-other-move:Jayez-2}). (\ref{pcq-other-move:Jayez-3}) corresponds to example (\ref{connectives-in-dialogue:Jayez}) and $a$'s move consists in asking $b$ to confirm the truth of a felicity condition of a previous move by $b$.


%$\exists x !(! x : \textbf{pc} \wedge \textit{fant}_\text{pc}!(!b,m!)! = \textit{fant}_\text{pc}!(!a,m!)! = x!)!$

\ea
\label{pcq-other-move:Jayez}
\ea \label{pcq-other-move:Jayez-1}
\avm[attributes=\scshape,values=\upshape]{
	[
	\textbf{\upshape{belief}} &\\
	cmt & $sit \models pc \wedge \exists x !(! x: \textbf{pc} \wedge \textit{fant}_\text{\textbf{bel}}!(!b,m,pc,\textit{parce que}!)! =$\\
	& $\textit{fant}_\text{\textbf{bel}}!(!a,m,pc,\textit{parce que}!)! = x \wedge cause!(!pc,x!)!!)! \oplus c$\\
	qud & $sit \models pc \oplus q$\\
	moves & $\mbox{\textit{assert}!(!}a,b,pc,sit\mbox{!)!} \oplus m$
	]
    }
    \ex \label{pcq-other-move:Jayez-2}
    \avm[attributes=\scshape,values=\upshape]{
		[
		\textbf{\upshape{sp.\ act}} &\\
		cmt & $sit \models pc \wedge \exists x !(! x: \textbf{pc} \wedge \textit{fant}_\text{\textbf{sa}}!(!b,m,pc,\textit{parce que}!)! =$\\
		& $\textit{fant}_\text{\textbf{sa}}!(!a,m,pc,\textit{parce que}!)! = x \wedge cause!(!pc,x!)!!)! \oplus c$\\
		qud & $sit \models pc \oplus q$\\
		moves & $\mbox{\textit{assert}!(!}a,b,pc,sit\mbox{!)!} \oplus m$
		]
	}
\ex \label{pcq-other-move:Jayez-3}
\avm[attributes=\scshape,values=\upshape]{
	[
	\textbf{\upshape{sp.\ act}} &\\
	cmt & $\exists x !(! x: \textbf{pc} \wedge \textit{fant}_\text{\textbf{sa}}!(!b,m,pc,\textit{parce que}!)! =$\\
	& $\textit{fant}_\text{\textbf{sa}}!(!a,m,pc,\textit{parce que}!)! = x \wedge precond!(!pc,x!)!!)! \oplus c$\\
	qud & $sit \models pc \oplus q$\\
	moves & $\mbox{\textit{question}!(!}a,b,pc,sit\mbox{!)!} \oplus m$
	]
}  \\
    \z
    \z
    

%\begin{table}[H]
%\caption{The record for \textit{parce que} outside the PC}
%\label{pcq-other-move:Jayez}
%\begin{tabular}{l}
%	{
%\avm[attributes=\scshape,values=\upshape]{
%	[
%	\textbf{\upshape{belief}} &\\
%	cmt & $sit \models pc \wedge \exists x !(! x : \textbf{pc} \wedge \textit{fant}_\text{\textbf{bel}}!(!b,m,pc,\textit{parce que}!)! =$\\
%	& $\textit{fant}_\text{\textbf{bel}}!(!a,m,pc,\textit{parce que}!)! = x \wedge cause!(!pc,x!)!!)! \oplus c$\\
%	qud & $sit \models pc \oplus q$\\
%	moves & $\mbox{\textit{assert}!(!}a,b,pc,sit\mbox{!)!} \oplus m$
%	]
%}
% remember to put <, >, [, ], and (, ) between !!
%	}
%\\%[5mm]
%{
%	\avm[attributes=\scshape,values=\upshape]{
%		[
%%%		& $\textit{fant}_\text{\textbf{sa}}!(!a,m,pc,\textit{parce que}!)! = x \wedge cause!(!pc,x!)!!)! \oplus c$\\
%		qud & $sit \models pc \oplus q$\\
%		moves & $\mbox{\textit{assert}!(!}a,b,pc,sit\mbox{!)!} \oplus m$
%		]
%	}
%}\\
%{
%\avm[attributes=\scshape,values=\upshape]{
%	[
%	\textbf{\upshape{sp.\ act}} &\\
%	cmt & $\exists x !(! x : \textbf{pc} \wedge \textit{fant}_\text{\textbf{sa}}!(!b,m,pc,\textit{parce que}!)! =$\\
%	& $\textit{fant}_\text{\textbf{sa}}!(!a,m,pc,\textit{parce que}!)! = x \wedge precond!(!pc,x!)!!)! \oplus c$\\
%	qud & $sit \models pc \oplus q$\\
%	moves & $\mbox{\textit{question}!(!}a,b,pc,sit\mbox{!)!} \oplus m$
%	]
%}
%}
%\end{tabular}
%\end{table}

In \sectref{sec-unique-DR:Jayez}, we saw that \textit{bien que} `although' can connect \textsc{soa}. However, the connection between \textsc{soa}s is not necessarily a sort of inverted causation: communicating that \textit{A although B} does not imply that \textit{B} normally \textit{prevents} \textit{A} from happening. In an abductive example like (\ref{car:Jayez}), it is unclear whether there is a causal link between the two \textsc{soa}s. I use a more neutral relation of being counter to expectation: \textit{cntr-expect}(x,y) means that $x$ being true should not normally lead us to observe or think that $y$ is also true. The record for \textit{bien que} `although' would then be as in (\ref{bien-que-move:Jayez}).

\ea \label{bien-que-move:Jayez}
\avm[attributes=\scshape,values=\upshape]{
				[
				cmt & $sit \models pc \wedge \exists x !(! x: \textbf{soa} \wedge \textit{fant}_\text{\textbf{soa}}!(!b,m,pc,\textit{bien que}!)! =$\\
				& $\textit{fant}_\text{\textbf{soa}}!(!a,m,pc,\textit{bien que}!)! = x \wedge \textit{cntr-expect}!(!pc,x!)!!)! \oplus c$\\
				qud & $sit \models pc \oplus q$\\
				moves & $\mbox{\textit{assert}!(!}a,b,pc,sit\mbox{!)!} \oplus m$
				]
			}
            \z

%%\setlength{\tabcolsep}{1pt}
%\begin{table}[H]
%	\caption{The record for \textit{bien que}}
%	\label{bien-que-move:Jayez}
%	\begin{tabular}{c}
%		{
%			\avm[attributes=\scshape,values=\upshape]{
%				[
%				cmt & $sit \models pc \wedge \exists x !(! x : \textbf{soa} \wedge \textit{fant}_\text{\textbf{soa}}!(!b,m,pc,\textit{bien que}!)! =$\\
%				& $\textit{fant}_\text{\textbf{soa}}!(!a,m,pc,\textit{bien que}!)! = x \wedge \textit{cntr-expect}!(!pc,x!)!!)! \oplus c$\\
%				qud & $sit \models pc \oplus q$\\
%				moves & $\mbox{\textit{assert}!(!}a,b,pc,sit\mbox{!)!} \oplus m$
%				]
%			}
			% remember to put <, >, [, ], and (, ) between !!
%		}
%	\end{tabular}
%\end{table}

%	doi = {https://doi.org/10.4000/discours.9997},
%	url = {https://journals.openedition.org/discours/9997?lang=en},

As a final illustration, I turn to the temporal vs.\ concessive uses of \textit{après} `after', defined in (\ref{apres-move:Jayez}). The temporal use defined in (\ref{apres-move:Jayez-1}) parallels (\ref{pcq-pc-move:Jayez}) for \textit{parce que}, with a temporal relation instead of a causal one. The concessive move defined in (\ref{apres-move:Jayez-2}) is different in two respects. First, the concession relation is presupposed and put in the \isi{commitment} store. Second, the last assertive move, corresponding to the host sentence, is declared in \textsc{cmt} as temporally following the initial value of \textsc{moves} ($m$). The \textit{after-dem} predicate corresponds to the demarcation function introduced in \sectref{sec-meaning-proximity:Jayez} in the analysis of example (\ref{theatre:Jayez}).

\ea
\label{apres-move:Jayez}
\ea \label{apres-move:Jayez-1}
\small{
\avm[attributes=\scshape,values=\upshape]{
				[
				\upshape{\textbf{temporal}} &\\
				cmt & $sit \models pc \wedge \exists x !(! x : \textbf{pc} \wedge \textit{fant}_\text{\textbf{pc}}!(!b,m,pc,\textit{après}!)! =$\\
				& $\textit{fant}_\text{pc}!(!a,m,pc,\textit{après}!)! = x \wedge \textit{after}!(!pc,x!)!!)! \oplus c$\\
				qud & $sit \models pc \wedge \textit{after}\mbox{!(!}pc,\textit{fant}_\text{\textbf{pc}}\mbox{!(!}a,m,pc,\textit{après}\mbox{!)!!)!} \oplus q$\\
				moves & $\mbox{\textit{assert}!(!}a,b,pc,sit\mbox{!)!}$ \oplus \\
                & $\mbox{!(!}\mbox{\textit{assert}!(!}a,b,\textit{after}\mbox{!(!}pc,\textit{fant}_\text{\textbf{pc}}\mbox{!(!}a,pc,m,\textit{après}\mbox{!)!!)!},sit\mbox{!)!}\oplus m\mbox{!)!}$
				]
			}
            }
\ex \label{apres-move:Jayez-2}
\small{
\avm[attributes=\scshape,values=\upshape]{
				[
				\upshape{\textbf{concessive}} & \\
				cmt & $sit \models pc \wedge \exists x !(! x : \textbf{pc} \wedge \textit{fant}_\text{\textbf{sa}}!(!b,m,pc,\textit{après}!)! =$
               \\ & $ \textit{fant}_\text{\textbf{sa}}!(!a,m,pc,\textit{après}!)! =$\\
				& $ x \wedge \textit{concession}!(!pc,x!)!!)! \wedge \textit{after-dem}!(!\mbox{\textit{assert}!(!}a,b,pc,sit\mbox{!)!},m!)! \oplus c$\\
				qud & $sit \models pc \oplus q$\\
				moves & $\mbox{\textit{assert}!(!}a,b,pc,sit\mbox{!)!} \oplus m$
				]
			}
            }
\z
\z


%\setlength{\tabcolsep}{3pt}
%\begin{table}[H]
%	\caption{The record for \textit{après}}
%	\label{apres-move:Jayez}
%	\begin{tabular}{l}

%		{\small
%			\avm[attributes=\scshape,values=\upshape]{
%				[
%				\upshape{\textbf{temporal}} &\\
%				cmt & $sit \models pc \wedge \exists x !(! x : \textbf{pc} \wedge \textit{fant}_\text{\textbf{pc}}!(!b,m,pc,\textit{après}!)! =$\\
%				& $\textit{fant}_\text{pc}!(!a,m,pc,\textit{après}!)! = x \wedge \textit{after}!(!pc,x!)!!)! \oplus c$\\
%				qud & $sit \models pc \wedge \textit{after}\mbox{!(!}pc,\textit{fant}_\text{\textbf{pc}}\mbox{!(!}a,m,pc,\textit{après}\mbox{!)!!)!} \oplus q$\\
%				moves & $\mbox{\textit{assert}!(!}a,b,pc,sit\mbox{!)!} \oplus \mbox{!(!}\mbox{\textit{assert}!(!}a,b,\textit{after}\mbox{!(!}pc,\textit{fant}_\text{\textbf{pc}}\mbox{!(!}a,pc,m,\textit{après}\mbox{!)!!)!},sit\mbox{!)!}\oplus m\mbox{!)!}$
%				]
%			}
%			% remember to put <, >, [, ], and (, ) between !!
%		}
%		\\
%		{\small
%			\avm[attributes=\scshape,values=\upshape]{
%				[
%				\upshape{\textbf{concessive}} & \\
%				cmt & $sit \models pc \wedge \exists x !(! x : \textbf{pc} \wedge \textit{fant}_\text{\textbf{sa}}!(!b,m,pc,\textit{après}!)! =\textit{fant}_\text{\textbf{sa}}!(!a,m,pc,\textit{après}!)! =$\\
%				& $ x \wedge \textit{concession}!(!pc,x!)!!)! \wedge \textit{after-dem}!(!\mbox{\textit{assert}!(!}a,b,pc,sit\mbox{!)!},m!)! \oplus c$\\
%				qud & $sit \models pc \oplus q$\\
%				moves & $\mbox{\textit{assert}!(!}a,b,pc,sit\mbox{!)!} \oplus m$
%				]
%			}
%		}

%	\end{tabular}
%\end{table}


\section{Hic et Nunc Particles}

\label{sec-HN-particles:Jayez}

\subsection{Description}

\textit{Hic et Nunc particles} (\isi{HNPs}) have been recently examined in \citet{DargnatandJayez:2020,Dargnat:2024}. They represent an extension of the traditional grammatical category of ``interjections''. Their central property is that, like presuppositional \isi{DMs}, they remain outside the \isi{PC}, but, unlike them and like \citeauthor{Potts:2007}'s (\citeyear{Potts:2007}) expressives, they are anchored to the utterance situation. This is most easily seen in a local effect context like in (\ref{HNP:Jayez}). It is not possible to attribute the \isi{HNP} to Mary, unless a direct quotation interpretation is forced (\textit{Mary thought: ``heck! Paul missed his train''}).

\ea
\label{HNP:Jayez}
Marie pense que (\{zut / merde / bon / tiens\}) Paul {a raté} son train.\\
`Mary thinks that (\{heck / shit / well / look\}) Paul missed his train.'\\
\z

Three points have to be clarified at the outset. Firstly, not all \isi{HNPs} are expressives. Many of them are, hence the frequent confusion with interjections, which are often assimilated to emotional expressions. Items like \textit{bon} `good/well', \textit{hein} `right?' or \textit{quoi} `what' are not intrinsically expressives. Of course they can be colored by intonation, but this is the case with other parts of speech as well.

Secondly, \isi{HNPs} are more radically anchored to the utterance situation than some expressives. Consider the contrast in (\ref{utterance-sit:Jayez}). (\ref{utterance-sit-c:Jayez}) is a bit strange out of context. It would be natural in a situation where I happen to remember that I had broken the vase, for instance if I just bought some flowers, was looking to put them in the vase and realized too late that the vase does not exist anymore. In that case, \textit{oops} would point to my present disappointment. In (\ref{utterance-sit-b:Jayez}), \textit{fucking} also indexes my present irritation but this feeling is not overtly caused by elements of the utterance situation, such as an external or internal event. Maybe I just dislike the vase in general, not because of \textit{hic et nunc} circumstances.

\ea
\label{utterance-sit:Jayez}
\ea \label{utterance-sit-a:Jayez}
Oops! I dropped the vase.\\
\ex \label{utterance-sit-b:Jayez}
I dropped the fucking vase yesterday.\\
\ex \label{utterance-sit-c:Jayez}
\#Oops I dropped the vase yesterday.\\
\z
\z

Thirdly, there is the question of ineffability or inscrutability.  One can easily develop the impression that \isi{HNPs} are more difficult to analyze than other \isi{DMs}, in particular the \isi{connective} \isi{DMs} I have briefly described in \sectref{sec-DM-as-PP-triggers:Jayez}. This feeling is probably justified in part, because, even though many connectives remain outside the \isi{PC}, they have a \isi{propositional content}, i.e. a content which can be described in a propositional language, by saying, for instance, that two semantic objects are related by a consequence or concession relation. However, this is not the case for every \isi{connective}. For instance, \textit{en fait} `in fact/actually' or \textit{de toute façon, en tout cas} `anyway' involve more complex constraints \citep{Rossarietal:2018}.

Lastly, it has been repeatedly proposed that interjections occupy intermediate positions in a continuum, whose exact nature varies (e.g., a degree of conventionalization or a showing-meaning\footnote{The intuition that Wharton's expression intends to capture is that of a contrast between two extreme points: displaying some internal state by one's behavior (`showing') vs. articulating the same fact by means of linguistic expressions. He calls the second scenario `saying'. He relates this distinction to the problem of effability in chapter 4 of his book.} scale, see \citealt[chapter 4]{Wharton:2009}). Although quite plausible, this possibility should not hide the fact that \isi{HNPs} share two common basic features: (i) they are indexical elements whose reference is restricted to aspects of the utterance situation, and (ii) these aspects fall into two broad classes: internal events (emotional, attentional or cognitive variations) and speaker-hearer interaction. \isi{HNPs} can be very vague, and their \isi{reference type} can depend on context. For instance,  many \isi{HNPs} which index emotional states, like \textit{ah}, \textit{oh}, \textit{putain} `damn', can correspond to surprise, admiration, anger, relief, etc., depending on context and intonation. Their common function is to signal a rise or decrease along some emotional dimension, as proposed by \citet{Potts:2007} for expressives. \isi{HNPs} like \textit{tiens} `look', \textit{ah} or \textit{eh bien} can signal that the speaker has focused her attention on some particular event or object. The speaker-hearer interaction is apparent in items like \textit{écoute} `listen', \textit{tu sais} `you know' or \textit{tu parles} `you bet'.

\subsection{Representation}
\label{sec-representation2:Jayez}

It seems that the \isi{HNP} information should be located in the \isi{commitment} store. However, if I am to follow this route, I will have to explain why a sentence like \textit{Ah, I am really suprised} is not redundant. If we insert (i) the effect of the assertion that the speaker is surprised (a \textsc{soa}) in \textsc{qud}, and (ii) whatever conveys the same information in \textsc{cmt}, we will end up storing in \textsc{cmt} the effect of the assertion (copied from \textsc{qud}) and the same information (independently added to \textsc{cmt}). One might suggest that the explicit phrasing \textit{I am really surprised} allows the speaker to resolve the potential ambiguity of \textit{ah} for the hearer, but I doubt that this would explain away examples like \textit{Ouch! it hurts}, in particular if the hearer observes a situation where the speaker hits or cuts herself. Another option would be to get rid of the idea of a \isi{commitment} but then it would be hard to explain why responses such as \textit{Oh come on, you just fake it} make perfect sense.

Intuitively, \isi{HNPs} are manifestations of events (emotions, attention focusing, etc.). They are linguistic signs indicating that something is happening at the moment of the utterance. The speaker is committed to letting the addressee believe that a plausible associate of the sign is happening, but she does not declare that this associate is occurring. The proximity of interjections to involuntary reactions is often noted in the literature and suggests that they are perceived as reactions to stimuli. Intuitive as this view might be, it tends to mix up two different things, the event that causes a certain reaction and the reaction indexed by the \isi{HNP}: \isi{HNPs} are not directly about external events but about the internal events that are elicited by them. Moreover, \isi{HNPs} can be used even if there is no corresponding external event, for instance if I say \textit{ah!} because I happen to remember something totally unrelated to the current situation. \isi{HNPs} which have to do with interactions are not reactive in the sense that many interjections are supposed to be. They index an intention to act on the hearer, in general by orienting her attention in a certain direction or by emphasizing for her the epistemic status of some piece of information, as is typically done by \textit{tu sais} `you know', which suggests that the host sentence conveys something which is not obvious to the hearer. In contrast to \isi{connective} \isi{DMs}, \isi{HNPs} do not necessarily make use of the content of the host clause (if any), which aligns with the fact that they can function as holophrases, even under non-exclamative interpretations.

The basic structure for \isi{HNPs} is shown in (\ref{tu-sais-and-bon:Jayez-1}), the \isi{commitment} is the existence of an event (i) whose spatio-temporal trace, given by the \textit{loc} function, coincides with the current situation and (ii) which has properties appropriate to the \isi{HNP} which indexes it (the dummy $X$) and is used in the \textsc{moves} update. (\ref{tu-sais-and-bon:Jayez-2}) illustrates the case of \textit{tu sais}. The generic intended effect of \textit{tu sais} is, as with a number of other \isi{HNPs}, to raise the degree of attention of the hearer to something, specifically, with \textit{tu sais}, the \isi{PC} of the host sentence. \textsc{cmt} is also updated with the proposition that the speaker has no evidence that the hearer already believes what the host sentence conveys, a proposition which reflects the fact that it would be odd to use \textit{tu sais} when confirming what another speaker has just declared (\ref{tu-sais:Jayez}).

\ea \label{tu-sais:Jayez}
\gll A: Marie doit être chez elle.\\
A: Mary must be at her\\
\gll B: Oui, ??tu sais, elle est chez elle.\\
B: yes you know she is at her\\
\glt `A: Mary must be at home. B: Yes, she is at home.'
\z

For \textit{bon}, the indexed event in (\ref{tu-sais-and-bon:Jayez-3}) is considered by the speaker to be the final element in a sequence of events. This is the basic `termination effect' of \textit{bon} (\textsc{cmt1}), which is frequently used to mark some stage in a process or to signal an attempt to interrupt the addressee. When \textit{bon} has a conversational import and is used to mark a major or minor speaking turn change or an attempt to interrupt, the list $s$ which the speaker considers to be ended by \textit{bon} is a sublist of the list of moves $m$ (\textsc{cmt2}). The $s \sqsubseteq_{<} m$ notation indicates that $s$ respects the ordering of $m$.

%\begin{table}[H]
%	\caption{HNP items}
%	\label{HNP-move:Jayez}
%	{\small
%			\avm[attributes=\scshape,values=\upshape]{
%				[
%				\upshape\textbf{general form}
%				cmt & $\exists e !(!\textit{loc}!(!e!)! = \textit{sit} \wedge \textit{index}!(!\textit{hnp},e!)! \wedge X!(!e!)!!)!$\\
%				qud & \dots\\
%				moves & $\textit{use}!(!\textit{hnp},sit!)! \oplus m$
%				]
%			}
%		}
%\end{table}

\ea
\label{tu-sais-and-bon:Jayez}
\ea \label{tu-sais-and-bon:Jayez-1}
{\small
		\avm[attributes=\scshape,values=\upshape]{
			[
			\upshape\textbf{\isi{HNP} skeleton}\\
			cmt & $\exists e !(!\textit{loc}!(!e!)! = \textit{sit} \wedge \textit{index}!(!\textit{hnp},e!)! \wedge X!(!e!)!!)!$\\
			qud & \dots\\
			moves & $\textit{use}!(!\textit{hnp},sit!)! \oplus m$
			]
		}
	}
\ex \label{tu-sais-and-bon:Jayez-2}
{\small
	\avm[attributes=\scshape,values=\upshape]{
		[
		\upshape{\textit{\textbf{tu sais}}} &\\
		cmt & $\exists e !(!\textit{loc}!(!e!)! = sit \wedge \textit{index}!(!hnp,e!)! \wedge e =$\\
		& $\textit{intend}!(!a,\textit{raise-att}!(!b,pc!)!!)!!)! \wedge \neg \textit{Bel}!(!a,\textit{Bel}!(!b,\textit{sit} \models \textit{pc}!)!!)!$ $\oplus c$\\
		qud & $\textit{sit} \models pc \oplus q$\\
		moves & $\textit{use}!(!\textit{tu sais},sit!)! \oplus!(!\textit{assert}!(!a,b,pc,sit!)! \oplus m!)!$
		]
	}
	% remember to put <, >, [, ], and (, ) between !!
}
\ex \label{tu-sais-and-bon:Jayez-3}
{\small
	\avm[attributes=\scshape,values=\upshape]{
		[
		\upshape{\textit{\textbf{bon}}} &\\
		cmt1 & $\exists e !(!\textit{loc}!(!e!)! = sit \wedge \textit{index}!(!hnp,e!)! \wedge  $\\
        & $ \exists s !(!s:\textbf{list of events} \wedge \textit{Bel}!(!a,e = \textit{last}!(!s!)!!)!!)! \oplus c$\\
		cmt2 & $\exists e !(!\textit{loc}!(!e!)! = sit \wedge \textit{index}!(!hnp,e!)! \wedge $\\
        & $ \exists s !(!s:\textbf{list of events} \wedge s \sqsubseteq_{!<!} m \wedge \textit{Bel}!(!a,e = \textit{last}!(!s!)!!)!!)! \oplus c$\\
		qud & \dots\\
		moves & $\textit{use}!(!\textit{bon},sit!)!  \oplus m$
		]
	}
	% remember to put <, >, [, ], and (, ) between !!
}
\z
\z


%\begin{table}[H]
%\caption{The record for \textit{tu sais} and \textit{bon}}
%\label{tu-sais-and-bon:Jayez}
%\begin{tabular}{l}
%	{\footnotesize
%		\avm[attributes=\scshape,values=\upshape]{
%			[
%			\upshape\textbf{\isi{HNP} skeleton}\\
%			cmt & $\exists e !(!\textit{loc}!(!e!)! = \textit{sit} \wedge \textit{index}!(!\textit{hnp},e!)! \wedge X!(!e!)!!)!$\\
%			qud & \dots\\
%			moves & $\textit{use}!(!\textit{hnp},sit!)! \oplus m$
%			]
%		}
%	}
%	\\
%{\footnotesize
%	\avm[attributes=\scshape,values=\upshape]{
%		[
%		\upshape{\textit{\textbf{tu sais}}} &\\
%		cmt & $\exists e !(!\textit{loc}!(!e!)! = sit \wedge \textit{index}!(!hnp,e!)! \wedge e =$\\
%		& $\textit{intend}!(!a,\textit{raise-att}!(!b,pc!)!!)!!)! \wedge \neg \textit{Bel}!(!a,\textit{Bel}!(!b,\textit{sit} \models \textit{pc}!)!!)!$ $\oplus c$\\
%		qud & $\textit{sit} \models pc \oplus q$\\
%		moves & $\textit{use}!(!\textit{tu sais},sit!)! \oplus!(!\textit{assert}!(!a,b,pc,sit!)! \oplus m!)!$
%		]
%	}
%	% remember to put <, >, [, ], and (, ) between !!
%}
%\\
%{\footnotesize
%	\avm[attributes=\scshape,values=\upshape]{
%		[
%		\upshape{\textit{\textbf{bon}}} &\\
%		cmt1 & $\exists e !(!\textit{loc}!(!e!)! = sit \wedge \textit{index}!(!hnp,e!)! \wedge \exists s !(!s:\textbf{list of events} \wedge \textit{Bel}!(!a,e = \textit{last}!(!s!)!!)!!)! \oplus c$\\
%		cmt2 & $\exists e !(!\textit{loc}!(!e!)! = sit \wedge \textit{index}!(!hnp,e!)! \wedge \exists s !(!s:\textbf{list of events} \wedge s \sqsubseteq_{!<!} m \wedge \textit{Bel}!(!a,e = \textit{last}!(!s!)!!)!!)! \oplus c$\\
%		qud & \dots\\
%		moves & $\textit{use}!(!\textit{bon},sit!)!  \oplus m$
%		]
%	}
%	% remember to put <, >, [, ], and (, ) between !!
%}

%\end{tabular}
%\end{table}

\section{Conclusion}
In this paper, I have shown that, in spite of their individual variability, already apparent from the few examples I presented, \isi{DMs} are not semantically special, since they are members of the \isi{PC}, trigger \isi{PSPs} or behave like \isi{conventional implicatures}. That said, although I offered a number of guidelines for their description and representation, I barely scratched the surface of certain problems or set them aside. I will briefly discuss two of them. First, concerning the \isi{PSP} triggers, I did not discuss the (im)possibility of organizing their records into a hierarchical network, as is done in \isi{HPSG} \citep{HPSG:2021} or Construction Grammars \citep{Diessel:2023}. In particular, the diachronic motivations offered in many cases invite us to examine whether we can construct a coherent and large scale picture of these derivations and the resulting meaning similarities (see \citealt{Kortmann:1997} for an enterprise of this kind). For \isi{HNPs}, it is desirable to describe not only what would count as their core meaning, which I illustrated here, but also the interactional scenarios in which they occur \citep{Aijmer:2002}, which is particularly appropriate in a framework like \citeauthor{Ginzburg:2012}'s (\citeyear{Ginzburg:2012}).


%	\section*{Abbreviations}
%	\begin{tabularx}{.5\textwidth}{@{}lQ@{}}
%		... & \\
%		... & \\
%	\end{tabularx}%
%	\begin{tabularx}{.5\textwidth}{@{}lQ@{}}
%		... & \\
%		... & \\
%	\end{tabularx}
%
%	\section*{Acknowledgements}
%
%	OK
	%\section*{Contributions}
	%John Doe contributed to conceptualization, methodology, and validation.
	%Jane Doe contributed to writing of the original draft, review, and editing.

\section*{Acknowledgements}

I gratefully acknowledge the contribution of four anonymous reviewers which helped me to clarify, deepen or modify a number of points in the initial version of this chapter, as well as to give the text a more balanced structure. I also thank Mathilde Dargnat for heated and, as a result, very useful discussions about \textit{Hic et Nunc Particles}.

	\sloppy
	\printbibliography[heading=subbibliography,notkeyword=this]
\end{document}

