\documentclass[output=paper,colorlinks,citecolor=brown]{langscibook}
\ChapterDOI{10.5281/zenodo.15450434}
\author{Achille Fusco\orcid{0000-0002-5389-8884}\affiliation{NeTS Lab, IUSS Pavia} and        Cristiano Chesi\orcid{0000-0003-1935-1348}\affiliation{NeTS Lab, IUSS Pavia} and        Valentina Bianchi\orcid{0000-0002-4441-063X}\affiliation{University of Siena}}
\title[Assessing the meaning of Subjective Attitude Verbs]{Using belief-perception mismatch to assess the meaning of Subjective Attitude Verbs}
\abstract{Subjective Attitude Verbs (SAVs) like \textit{find} and \textit{consider} are known to embed only propositions that express a subjective judgement or opinion of the attitude holder. A point of contention, however, is whether both of the subjective attitudes involve a doxastic component. In this study, we empirically assess the doxastic status of SAVs by testing the acceptability rating of attitude reports with the two SAVs in contexts of potential doxastic conflict (PDC). Results showed that attributions with \textit{find} in PDC contexts produced significantly lower acceptability rates than their counterparts with \textit{consider}, suggesting that \textit{find}, but not \textit{consider}, lacks a doxastic component in its denotation.}

\IfFileExists{../localcommands.tex}{
   \addbibresource{../localbibliography.bib}
   % add all extra packages you need to load to this file

\usepackage{tabularx,multicol}
\usepackage{url}
\urlstyle{same}

\usepackage{listings}
\lstset{basicstyle=\ttfamily,tabsize=2,breaklines=true}

\usepackage{langsci-basic}
\usepackage{langsci-optional}
\usepackage{langsci-lgr}
\usepackage{langsci-osl}
% \usepackage{./langsci/styles/langsci-lgr}
% \usepackage{./langsci/styles/langsci-osl}
% \usepackage{langsci-gb4e}

\usepackage{tikz}
\usetikzlibrary{patterns,calc}
\pgfdeclarepatternformonly{south east lines}{\pgfqpoint{-0pt}{-0pt}}{\pgfqpoint{3pt}{3pt}}{\pgfqpoint{3pt}{3pt}}{
    \pgfsetlinewidth{0.6pt}
    \pgfpathmoveto{\pgfqpoint{0pt}{3pt}}
    \pgfpathlineto{\pgfqpoint{3pt}{0pt}}
    \pgfpathmoveto{\pgfqpoint{.2pt}{-.2pt}}
    \pgfpathlineto{\pgfqpoint{-.2pt}{.2pt}}
    \pgfpathmoveto{\pgfqpoint{3.2pt}{2.8pt}}
    \pgfpathlineto{\pgfqpoint{2.8pt}{3.2pt}}
    \pgfusepath{stroke}}
    
\usepackage{stmaryrd}
\usepackage{wasysym}
\usepackage{multirow}
\usepackage{caption}
\usepackage{subcaption}
\usepackage{mathrsfs}
\usepackage{qtree}

\usepackage{linguex}


   %pminos do not split footnotes
% \interfootnotelinepenalty=10000 %Footnote in Laporte chapters has to be split SN


%\DeclareIndexNameFormat{default}{%
%\nameparts{#1}%
%\usebibmacro{index:name}%
%{\index[names]}%
%{\namepartfamily}%
%{\namepartgiveni}%
% {}% L1
% {}% L2
%{\namepartprefix}% generates spurious space L3
%{\namepartsuffix}% generates spurious space L4
%}

%  {\DeclareIndexNameFormat{default}{%
%     \usebibmacro{index:name}{\index[names]}{#1}{#3}{#5}{#7}}}

%\DeclareIndexNameFormat{default}{%
%  \usebibmacro{index:name}{\sindex[nom]}{#1}{#3}{#5}{#7}}

%\DeclareIndexNameFormat{default}{%
%  \usebibmacro{index:name}{\sindex[person]}{#1}{#3}{#5}{#7}}
%\DeclareIndexNameFormat{default}{%
%\nameparts{#1} \usebibmacro{index:name}{\sindex[person]]}{\namepartfamily}{‌​\namepartgiven}{\nam‌​epartprefix}{\namepa‌​rtsuffix}}

%\newcommand{\smiley}{:)}

%\renewbibmacro*{index:name}[5]{%
%\usebibmacro{index:entry}{#1}%
%{\iffieldundef{usera}{}{\thefield{usera}\actualoperator}\mkbibindexname{#2}{#3}{#4}{#5}}}

% \newcommand{\noop}[1]{}

%remove for final
%\overfullrule=1mm

\newcommand{\tobi}[2]}}
\renewcommand{\S}[1]{\tobi{#1}{\textsc{*}}}

% this volume references
% puts: [this volume]
% already defined: \citetv
%\newcommand{\citepv}[1]{(\citeauthor{#1} \citeyear*{#1} [this volume])}
\newcommand{\citealtv}[1]{\citeauthor{#1} \citeyear*{#1} [this volume]}

%parentheses around example number
\newcommand{\pref}[1]{(\ref{#1})}

% in-text examples

\newcommand{\lnex}[1]{\textit{#1}} %target lang word
\newcommand{\lnlit}[1]{(lit.: `#1')} %literal reading
\newcommand{\lnlat}[1]{(#1)} % latinization
\newcommand{\lntrans}[1]{`#1'} %translation
\newcommand{\lnexl}[2]%
{\lnex{#1}{} \lnlat{#2}} % ex with latinization
\newcommand{\lnexlat}[3]{\lnex{#1}{} \lnlat{#2}{} \lntrans{#3}} % ex with latinization and tranl.

%ch01
\newcommand{\co}[1]{\mbox{\textbf{#1}}}

%ch09

\newcommand{\cyrbulg}[1]{\begin{otherlanguage*}{bulgarian}#1\end{otherlanguage*}}


%ch10
\newcommand{\nlp}{{\small NLP}}
\newcommand{\mwe}{{\small MWE}}
\newcommand{\rae}{{\small RAE}}
\newcommand{\lvc}{{\small LVC}}
\newcommand{\pos}{{\small P}o{\small S}}
%\newcommand{\todo}[1]{ \textcolor{red}{#1} }

%\renewcommand{\labelenumi}{\theenumi}
%\ainamefmt{{vv}{ll}{, ff}{, jj}} % fullname

\newcommand{\biberror}[1]{{\color{red}#1}}

\newcommand{\osenovaitem}{--~}
   %% hyphenation points for line breaks
%% Normally, automatic hyphenation in LaTeX is very good
%% If a word is mis-hyphenated, add it to this file
%%
%% add information to TeX file before \begin{document} with:
%% %% hyphenation points for line breaks
%% Normally, automatic hyphenation in LaTeX is very good
%% If a word is mis-hyphenated, add it to this file
%%
%% add information to TeX file before \begin{document} with:
%% %% hyphenation points for line breaks
%% Normally, automatic hyphenation in LaTeX is very good
%% If a word is mis-hyphenated, add it to this file
%%
%% add information to TeX file before \begin{document} with:
%% \include{localhyphenation}
\hyphenation{
    Beck-man
    Ngu-yen
    back-chan-nel
    back-chan-nels
    mo-not-o-nous
    ste-reo-typ-i-cal
}

\hyphenation{
    Beck-man
    Ngu-yen
    back-chan-nel
    back-chan-nels
    mo-not-o-nous
    ste-reo-typ-i-cal
}

\hyphenation{
    Beck-man
    Ngu-yen
    back-chan-nel
    back-chan-nels
    mo-not-o-nous
    ste-reo-typ-i-cal
}

   \boolfalse{bookcompile}
   \togglepaper[4]%%chapternumber
}{}

\begin{document}
\maketitle

\section{Introduction}
\label{sec:introduction:Fusco}

Since the seminal papers by \citet{lasersohn_2005} and \citet{kolbel_2004}, there has been much interest, both in philosophy of language and in natural language semantics, in linguistic phenomena related to \isi{subjectivity}, in particular with regards to faultless disagreement and \isi{predicates of personal taste} (\isi{PPT}; \cite{stephenson_judge_2007,stojanovic_2007,moltmann_2010,barker_2013,bylinina_thesis_2014, gunlogson_carlson_2016,munoz2019}). Such phenomena seem to suggest that the interpretation of certain natural language expressions does not depend on matters of fact alone, but also on the assessment of some relevant individual or group.

It comes by no surprise, then, that a considerable attention has been devoted to Subjective Attitude Verbs (SAVs), such as \textit{find} and \textit{consider}, which are known to embed only subjective clauses, e.g., clauses containing a \isi{PPT} (e.g., \textit{tasty}) or, more generally, a gradable predicate (e.g., \textit{heavy}) that allows for a subjective interpretation (\cite{saebo_2009,kennedy_2013, bouchard_thesis_2012, fleisher_2013,coppock_2018,kennedy_willer_2022}). This is illustrated by the contrast between (\ref{ex:dinos:Fusco}) and (\ref{ex:chili:Fusco}-\ref{ex:luggage:Fusco}):

\ea[\#]{\label{ex:dinos:Fusco}I find/consider dinosaurs extinct.}
\ex[]{\label{ex:chili:Fusco}I find/consider this chili tasty.}
\ex[]{\label{ex:luggage:Fusco}I find/consider this luggage heavy.}
\z

%Due to their peculiar selectional properties, embeddability under SAVs is used as a diagnostic for \isi{subjectivity}. 
\noindent{}However, SAVs do not form a uniform class, as asymmetries have been observed between \textit{find} and \textit{consider}. In particular, \textit{find} seems to require some first-person experience relevant to the judgement expressed by the complement clause, while \textit{consider} may elude such a requirement. The question then arises whether \textit{find} really resembles its cognates \textit{consider} and \textit{believe} in expressing, at its core, a belief of the \isi{subject} (\cite{stephenson_thesis_2007, fleisher_2013, kennedy_willer_2022, vardomskaya_2018, korotkova_anand_find_2021}).
On the other hand, non-doxastic analyses of \textit{find} include so-called radical reductionist theories (\cite{saebo_2009, bouchard_thesis_2012}), the theory of \citet{munoz2019} based  on perceptual alternatives and the expressivist proposal by \citet{franzen2020evaluative}.

This investigation is not only crucial for precisely characterizing the semantics of \textit{consider} and \textit{find}, but also for contributing to the broader debate about \isi{subjectivity} in natural language. Indeed, given that embedding under SAVs has been taken as a diagnostic test for \isi{subjectivity}, we think it is important to disentangle what it is exactly that SAVs select for and what they contribute, compositionally, to the meaning of a subjective attitude report. Therefore, the aim of the present work will be to assess the presence of a doxastic component in the semantics of \textit{consider} and \textit{find}.

It is important to emphasize here that we are not assuming that \textit{consider}, when paired with an appropriate (i.e., subjective) complement clause, behaves exactly like \textit{believe}. In fact, there are good empirical reasons to question such an assumption: \textit{believe} has in fact been argued to encode a weaker \isi{commitment} of the attitude holder (\citealt{koev2019strong}) and to be incompatible with direct evidence (\citealt{charlow2021experiential}).\footnote{We thank an anonymous reviewer for bringing up this important point.} Rather, our focus is on the presence or absence of a doxastic component in the denotation of the two SAVs -- specifically, whether the content of the complement clause must be interpreted as compatible with the beliefs of the attitude holder. In possible world terms, we set to investigate whether \textit{find} should be regarded as an intensional operator that quantifies over doxastic alternatives. So in the following, we will use the term ``belief'' to refer to an abstract attitude of a \isi{subject} \textit{s} towards a proposition \textit{p} such that \textit{s} regards \textit{p} to be true. Thus, if \textit{s} has a belief that \textit{p}, it does not follow that it should be felicitous to utter ``s believes that p''. Again, since the scope of this work is the strict comparison between \textit{consider} and \textit{find}, we leave aside more fine-grained discussion about \textit{believe} reports.\footnote{`Believe reports' are attitude reports with \emph{believe}.}

To address the issue presented above, we investigated experimentally how speakers evaluate 3rd-person \isi{SAV} ascriptions in contexts of potential doxastic conflict (PDC), where a mismatch between belief and perception arises. By comparing the \isi{acceptability} of \textit{find} and \textit{consider} reports in such contexts, we aim to elucidate the nature of the attitudes they describe. 
The remainder of this paper is organized as follows: \sectref{sec:observations:Fusco} presents the main observations about SAVs, focusing on the asymmetries between \textit{find} and \textit{consider};  \sectref{sec:accounts:Fusco} discusses major accounts that have been proposed in the literature;  \sectref{sec:pdc:Fusco} introduces the idea of Potential Doxastic Conflict as a possible way to disentangle belief from perception; \sectref{sec:present_study:Fusco} reports the experimental study we conducted; \sectref{sec:discussion:Fusco} discusses the implications of the results for existing theories and suggests future research directions.

\section{Empirical observations}
\label{sec:observations:Fusco}

SAVs are a cross-linguistic class of propositional attitude verbs which require that their complement clause express a subjective judgement or opinion of the attitude holder, i.e., the matrix \isi{subject}. Specifically, embedding objective clauses under SAVs like English \textit{find} results in strong infelicity (\ref{find_fermented:Fusco}). This behavior contrasts with plain doxastic attitude verbs like \textit{believe} or \textit{think}, which felicitously embed objective and subjective clauses alike (\ref{believe:Fusco}).

\ea 
\ea []{Alice finds the drink delicious.} 
\ex [\#]{Alice finds the drink fermented.} \label{find_fermented:Fusco}
\z
\ex \label{believe:Fusco}
\ea []{Alice believes/thinks that the drink is delicious.}
\ex []{Alice believes/thinks that the drink is fermented.}
\z
\z

Similar patterns are attested with German \textit{finden} (\cite{saebo_2009,umbach_2016}), French \textit{trouver} (\cite{bouchard_thesis_2012}), Norwegian \textit{synes} (\cite{saebo_2009}), Swedish \textit{tycka} (\cite{saebo_2009,coppock_2018}) and Italian \textit{trovare} (\cite{fusco2022sav}). 

Apart from these verbs, however, the class of SAVs also include other clause embedding verbs, like English \textit{consider} and Italian \textit{considerare}. Such verbs show a somewhat more nuanced behavior: they are like \textit{find} verbs in banning clearly objective complement clauses, as in (\ref{consider_equal:Fusco}), 
but generally display more relaxed embedding requirements. More specifically, \emph{consider}, but not \emph{find}, can embed propositions expressing judgements or opinions that allow for coordination by stipulation \citep{kennedy_willer_2016,kennedy_willer_2022}, i.e., propositions whose truth can be established if speakers could reasonably set, by stipulation, the arbitrary criteria for the meaning of the main predicate.
%but display more relaxed embedding requirements, especially when it is made clear that some amount of discretion is being applied. 
For example, although the predicate \textit{vegetarian} may strike one as objective, the \textit{consider} report is acceptable in (\ref{consider_vegetarian:Fusco}), where the continuation makes it clear that its meaning is deliberately ``stretched'' as to include anyone whose meat consumption is restricted to shellfish. By contrast, this is not possible with \textit{find}, as shown in (\ref{find_vegetarian:Fusco}).

\ea[\#]{I consider the sum of two and two equal to four.} \label{consider_equal:Fusco}
\ex \label{consider_find_vegetarian:Fusco}
\ea[]{I consider Nick vegetarian...} \label{consider_vegetarian:Fusco}
\ex[\#]{I find Nick vegetarian...} \label{find_vegetarian:Fusco}
\z
{... because the only animals he eats are shellfish.} 
\z


The two verbs can be distinguished by another important feature, already mentioned in \sectref{sec:introduction:Fusco}: Although both seem to enforce some familiarity requirement with respect to the content of the complement clause (\cite{kennedy_willer_2022}), it appears that \textit{find} requires some relevant  first-hand experience, whereas \textit{consider} may not have this requirement. First, note that a \textit{find} report is infelicitous if it is made explicit that the attitude \isi{subject} lacks relevant direct experience: 

\ea \#Although I haven’t tried it, I find this chili tasty (because everyone seems 
to enjoy it).
\z

This has been taken by some authors to be related to the fact that PPTs like \textit{tasty} already trigger an inference of acquaintance, even when unembedded, as evidenced in (\ref{acq_inf:Fusco}) (\cite{ninan2014taste,ninan2020projection}; see also \cite{gunlogson_carlson_2016} on Predicates of Experience and  \cite{crespo2019tasting} on the Experience Principle). However, it has been observed that the acquaintance inference is triggered by \textit{find} even when the embedded predicate does not give rise to it when unembedded (\cite{korotkova_anand_find_2021,kennedy_willer_2022}), as shown in (\ref{authentic:Fusco}), taken from \citet{korotkova_anand_find_2021}:

\ea This chili is tasty, \# although I haven't tried it.\label{acq_inf:Fusco}
\ex Context 1 (direct): The speaker has eaten at this restaurant. \label{authentic:Fusco}

Context 2 (indirect): The speaker read reviews about this restaurant on TripAdvisor.
\ea Food in this restaurant is authentic.\hfill \textsc{ok} Context 1, \textsc{ok} Context 2
\ex I find food in this restaurant authentic. \hfill \textsc{ok} Context 1, \# Context 2
\z
\z

On the contrary, \citet{kennedy_willer_2022} argue that \textit{consider} also comes with some familiarity with relevant facts, but not necessarily related to direct experience. Consider the example in (\ref{familiarity:Fusco}), taken from \citet{kennedy_willer_2022}, in which a relevant comparison is made with a \textit{believe} report:


\ea \label{familiarity:Fusco}Context 1 (direct): Kim doesn’t know the ingredients that went into this cake, but based on its taste...

Context 2 (indirect): Kim hasn’t tried the cake, but she knows that it was made using rice flour from a mill that also produces wheat flour, so ...
\ea ... she doesn’t consider it gluten-free.\hfill \# Context 1, \textsc{ok} Context 2
\ex ... she doesn’t believe that it’s gluten-free. \hfill \textsc{ok} Context 1, \textsc{ok} Context 2
\z
\z

A different but related asymmetry regards evaluativity, i.e., the feature of certain natural language expressions of conveying a positive or negative attitude, especially in aesthetic and moral domains. In a series of works on the topic (\cite{mcnally_stojanovic_2017,stojanovic_2019,stojanovic_mcnally2023}), Isidora Stojanovic and Louise McNally have argued that \textit{find} anti-selects for pure evaluativity. As evidence in favor of this hypothesis they report the contrast in (\ref{mediocre:Fusco}):

\ea \label{mediocre:Fusco}
\ea [?]{I find Miró’s mosaic on the Rambles mediocre.} \label{find_mediocre:Fusco}
\ex[]{I consider Miró’s mosaic on the Rambles mediocre.}
\z
\z

According to the authors, the oddness of (\ref{find_mediocre:Fusco}) would imply that an (aesthetic) evaluation cannot be made on experiential grounds alone. This point was confirmed by a search on the BNC corpus, which showed that \textit{find} rarely embeds prototypically evaluative predicates like \textit{good}, \textit{bad} and \textit{beautiful}.

Similarly, \citet{stojanovic_mcnally2023} presented a more systematic study on the COCA corpus investigating the subjective nature of moral predicates by examining their combination with the two \isi{subjective attitude verbs}, \textit{find} and \textit{consider}. The study found that, although moral predicates can naturally occur with both verbs, they exhibit a clear preference for \textit{consider} over \textit{find}. Conversely, PPTs frequently and naturally embed under \textit{find}, but rarely under \textit{consider}. This, again, suggests that the \isi{subjectivity} observed in moral judgments may differ from the \isi{subjectivity} of personal taste. Interestingly, the authors claim that, when embedded under \textit{find}, aesthetic and moral judgements are coerced into being not purely evaluative, but rather introduce an experiential component. 

We will now turn our attention to how these observations have been explained by previous accounts of SAVs.

\begin{table}
\caption{Summary of empirical observations on \textit{find} and \textit{consider}}
\label{tab:summary:Fusco}
\begin{tabularx}{.8\textwidth}{X rr}
  \lsptoprule
     & \textsc{find} & \textsc{consider}\\
  \midrule
  \textit{objective complement} & \# &  \#\\
  \textit{complement supporting stipulation} & \# &   \textsc{ok}  \\
  \textit{no direct experience} & \# &  \textsc{ok}  \\
  \textit{evaluative complement} & ? &  \textsc{ok}  \\
  \lspbottomrule
 \end{tabularx}
 \end{table}

\section{Accounts of subjective attitude verbs}
\label{sec:accounts:Fusco}

\citet{stephenson_thesis_2007} first recognized that \textit{find} implies a knowledge of a particular kind with respect to the content of the complement clause. In her account, this intuition was implemented as a requirement of direct experience built right into the denotation of \textit{find}. \citet{saebo_2009} criticized this analysis and proposed a radical reductionist approach, whereby the only requirement for the embedded clause is to be subjective, i.e., dependent on a ‘judge’ contextual parameter (\cite{lasersohn_2005, stephenson_judge_2007, stojanovic_2007} a.m.o.), and the only semantic contribution of \textit{find} is to shift the judge to the matrix \isi{subject} (see also \cite{bouchard_thesis_2012}).

One of the problems with these earlier theories is that they did not account for the similarities and differences between \textit{find} and \textit{consider}. In particular, in Stephenson’s framework, it is hard to imagine a corresponding requirement for \textit{consider} that would explain its embedding patterns. Similarly, on a radical reductionist account, it is not clear whether \textit{consider} should also shift the judge parameter and whether it should contain a doxastic component (see also other problems with these accounts, discussed in detail in \citealt{anand2022theorize}).

Later approaches have attempted to tackle the issue in terms of different patterns of \isi{presupposition} and assertion. \citet{kennedy_willer_2016,kennedy_willer_2022} take both SAVs to express, at their core, a plain doxastic attitude, like \textit{believe}. To explain the embedding patterns and the asymmetries of \textit{find} and \textit{consider}, however, they propose that SAVs trigger specific presuppositions of contingency of the complement clause. More specifically, they propose that \isi{SAV} ascriptions generate a set of contextual alternatives to the \isi{subject}'s doxastic state, named \textit{counterstances}. All of the counterstances agree on the objective facts but disagree on the judgements about those facts, reflecting speakers' choices about language use. 

Thus, a \textit{consider} ascription would carry the \isi{presupposition} that the content of the embedded clause is counterstance contingent, in the sense that its truth is not constant across all the counterstances generated. This captures the contrast in (\ref{consider_find_vegetarian:Fusco}), p. \pageref{consider_find_vegetarian:Fusco}.
%\sectref{sec:observations:Fusco}. 
On the other hand, embedding under \textit{find} would trigger the stronger \isi{presupposition} of radical counterstance contingency, which implies that the proposition expressed by the complement clause is contingent with respect to each of the counterstances. In other words, not even single counterstances are able to establish the truth of the embedded proposition.
 
Therefore, in Kennedy \& Willer's account, the issue raised by the complement of \textit{consider} is not resolved by matters of fact alone, but also by decisions about linguistic norms. By contrast, the issue raised by the complement of \textit{find} is not resolved even when such decisions are fixed (see \citealt{coppock_2018} for a similar strategy reframed in an outlook-based semantics).

\citet{munoz2019} proposes that \textit{find} and \textit{consider} differ with respect to both presupposed and asserted content. Casting his analysis in terms of hyperintensions sensitive to patterns of relevant linguistic behaviour, Muñoz’s definition of \textit{consider} reports is close in spirit to the counterstance approach: They presuppose that the agent’s beliefs about the world falsify the hyperintension of the complement at some model, meaning that a different linguistic behavior may determine a content which is false at all the doxastic alternatives. 

\textit{Find}, on the other hand, presupposes that there is no other agent and world pair (different from the ones to which the report is anchored) at which there is direct evidence verifying the hyperintension as it is anchored to the agent: In other words, no agent other than the \isi{subject} of the attitude verb can have the relevant direct evidence. The asserted content, in turn, involves quantification over perceptual alternatives, defined as the set of worlds compatible with the agent’s direct perceptions. By default, doxastic alternatives are a subset of perceptual ones: In other words, speakers generally believe what they perceive. However, Muñoz admits situations in which they could not take themselves as accurate perceivers, perhaps because perception conditions are not ideal: In this case, the entailment of belief would be blocked.

A crucial difference between the two latter approaches is that in Kennedy \& Willer's proposal, both SAVs have a doxastic base; in Muñoz's approach, instead, \textit{find} differs from \textit{consider} in that it does not involve a doxastic component.

\section{Potential doxastic conflict}
\label{sec:pdc:Fusco}

Concerning the presence or absence of the doxastic component in SAVs, the evidence discussed in the literature is limited and controversial: In particular, it revolves around situations of potential doxastic conflict, i.e., a conflict between the \isi{subject}'s perceptions and their beliefs about the referent of the small clause.

%Given the theoretical landscape, a crucial empirical test for assessing the various theories on the market would be to check whether judgements embedded in \isi{SAV} reports can express content which is not a belief of the matrix \isi{subject}.

%In fact, there have been some observations which give some initial plausibility about the hypothesis, although the data are rather scarce and quite controversial.

%In discussing the Acquaintance Inference (or Experience Principle, as they called it), Crespo \& Veltman (2019) claim that, ordinary gradable adjectives like \textit{tall} and \textit{long} are not \isi{subject} to it. 


Firstly, \citet{munoz2019} claims that while a sentence like (\ref{consider_licorice:Fusco}) would always be infelicitous, (\ref{find_licorice:Fusco}) would be acceptable in two kinds of situations: i) one in which Alphonse has forgotten what licorice tastes to him, and mistakenly believes it is tasty to him, or ii) one in which he is abstracting away from his own taste, maybe taking into account how licorice tastes to people in general (therefore evaluating it non-autocentrically; see \cite{lasersohn_2005,stephenson_judge_2007}).

\ea \label{licorice:Fusco}
\ea[]{Alphonse doesn’t find licorice tasty, but he thinks that it is.} \label{find_licorice:Fusco}
\ex[\#]{Alphonse doesn’t consider licorice tasty, but he thinks that it is.} \label{consider_licorice:Fusco}
\z
\z

Secondly, \citet{munoz2019} also reports that a first-person \textit{find} report like (\ref{find_movie:Fusco}) would be felicitous when uttered by someone who has seen the movie numerous times and is therefore not interested in it anymore, but who nonetheless thinks that it would be interesting to others in ideal conditions. 
Again, the felicity of an analogous \textit{consider} report (\ref{consider_movie:Fusco}) would not be saved in such circumstance.



\ea \label{movie:Fusco}
\ea[]{I don’t find the movie interesting (anymore), but it is interesting.{\footnotemark{}}} \label{find_movie:Fusco}
\ex[\#]{I don’t consider the movie interesting (anymore), but it is interesting.} \label{consider_movie:Fusco}
\z
\z
\footnotetext{Note that (\ref{find_movie:Fusco}) would be bad with an overt \isi{experiencer} PP referring to the \isi{subject} in the second clause:
%\newline\newline
\ea[\#]{I don't find the movie interesting (anymore), but it is interesting to me.}
\z
%\newline\newline
This shows that the second clause must indeed be interpreted non-autocentrically. We thank an anonymous reviewer for providing this example.}

Regarding interpretation (i) for the contrast in (\ref{licorice:Fusco}), \citet[fn. 9]{kennedy_willer_2022} note that there may be nothing irrational in believing a proposition by hearsay while believing its negation through direct acquaintance. Presumably, however, this would be possible only when the \isi{subject} is not aware of the contradiction. 

However, the same objection does not seem to apply to (\ref{find_movie:Fusco}), where the report is in the first-person.
One possible worry with (\ref{find_movie:Fusco}) is that, in the complement of \textit{find}, the predicate \textit{interesting} seems to be anchored to the speaker, while in the unembedded clause it does not. Therefore, (\ref{find_movie:Fusco}) is compatible with believing autocentrically the proposition expressed by the complement clause while believing its negation exocentrically. These observations would also apply to the interpretation (ii) of (\ref{licorice:Fusco}).

Finally, \citet{crespo2019tasting} provide evidence that a proposition embedded in a first-person report can, in fact, be believed to be false even without taking into account exocentric interpretations. They report that, when looking at the picture of a Müller-Lyer illusion (Fig. \ref{fig:muller-lyer:Fusco}), a rational speaker may utter (\ref{find_longer:Fusco}) even when realizing that the two arrows are of equal length.

\begin{figure}
\includegraphics[height=.09\textheight]{../figures/Fusco_Muller-Lyer.png}
\caption{The Müller-Lyer illusion}
\label{fig:muller-lyer:Fusco}
\end{figure}

\ea \label{find_longer:Fusco} {I find the upper arrow longer than the lower one.} 
\z

Before proceeding, we wish to address a potential concern that may arise here. The relevance of (\ref{find_longer:Fusco}) becomes doubtful if one takes into account that dimensional predicates like \textit{long} or \textit{heavy} often display an ambiguity between ``quantitative" and ``qualitative" interpretations (see \cite{kennedy_2013} for discussion).\footnote{We thank an anonymous reviewer for highlighting this point.} This ambiguity is well illustrated in the following example (taken from \cite{kennedy_2013}):

\ea \label{flight_longer:Fusco} {The flight from Chicago to Tokyo is longer than the one from Chicago to Hong Kong.}
\z

According to \citet{kennedy_2013}, there is one reading in which (\ref{flight_longer:Fusco}) refers to the actual amount of time of the flight and is, therefore, objectively false. However, in a second reading, (\ref{flight_longer:Fusco}) may refer to the subjective experience of the flight time, which could be affected by other factors such as comfort, weather, travel companions, etc. For clarity, we can indicate the two readings of \textit{longer} as \textit{longer}$_{\textit{quant}}$ and \textit{longer}$_{\textit{qual}}$, respectively. Once this potential ambiguity is acknowledged, one may worry that (\ref{find_longer:Fusco}) is not surprising after all, since two distinct propositions may be at stake:

\ea \label{longer_quant:Fusco} {The upper arrow is longer$_{\textit{quant}}$ than the lower one.}

\ex \label{longer_qual:Fusco} {The upper arrow is longer$_{\textit{qual}}$ than the lower one.}
\z

According to this line of reasoning, (\ref{longer_qual:Fusco}) would be the content of the \textit{find} attitude in (\ref{find_longer:Fusco}), while the belief of the \isi{subject} would involve either (\ref{longer_quant:Fusco}) or its negation: Consequently, no contradiction ensues. Although well-founded, there are other empirical facts that show that this analysis is not to be preferred for (\ref{find_longer:Fusco}). First of all, notice that, if this were a genuine ambiguity of the predicate \textit{long}, we would expect it to arise also in unembedded contexts, as in (\ref{flight_longer:Fusco}). However, this does not seem to be the case:

\ea \label{arrow_longer:Fusco} {The upper arrow is longer than the lower one.}
\z

To our judgement, (\ref{arrow_longer:Fusco}), when referring to a picture of a Müller-Lyer illusion, can only be taken to mean (\ref{longer_quant:Fusco}), not (\ref{longer_qual:Fusco}): in other words, the ambiguity does not seem to arise here. The same goes, apparently, for a doxastic attitude report:

\ea {I believe/think that the upper arrow is longer than the lower one.}
\z

Thus, even postulating the ambiguity, the subjective or qualitative reading of the predicate \textit{long} in the Müller-Lyer scenario would arise only when embedded under \textit{find}, while it would not be available in both unembedded judgements and doxastic reports.\footnote{Notice that \citet{kennedy_2013} argued that \textit{find} only selects qualitative readings, however, nothing in his theory should prevent qualitative readings to show up in other contexts, as testified by (\ref{flight_longer:Fusco}).} Therefore, we assume that no ambiguity arises with \textit{longer} in (\ref{find_longer:Fusco}) and that the reason for its \isi{acceptability} is, indeed, the lack of a doxastic meaning. 

According to \citet{crespo2019tasting}, however, this would be possible only with ordinary gradable adjectives like \textit{tall} and \textit{long}. In contrast, they claim that a similar pattern would not be possible with adjectives like PPTs. In the following, we would like to propose that it is, in fact, possible to create contexts analogous to the Müller-Lyer illusion for PPTs.

Consider, for example, the scenario in (\ref{pdc_taste:Fusco}):
\ea \label{pdc_taste:Fusco}Context: Mary has to take some medicines that alter her taste for a little while. Nonetheless, one day, after taking her medicines, she can’t help taking a slice of her mum’s apple pie, her favorite. The taste is terrible due to the medicines.

Mary (to her mother): Your apple pie is tasty, but I have just taken my medicines so it doesn’t taste good to me right now.
\z

The sentence uttered by Mary is perfectly natural. However, if we make the assumption that something that tastes good to Mary is tasty to Mary, we observe it is not possible to express the second conjunct of (\ref{pdc_believe:Fusco}) with a doxastic attitude:

\ea [\#]{Your apple pie is tasty, but I have just taken my medicines so I don’t believe it’s tasty.\footnotemark{}\label{pdc_believe:Fusco}}
\footnotetext{As mentioned in \sectref{sec:introduction:Fusco}, there may be another reason why (\ref{pdc_believe:Fusco}) is infelicitous: \textit{believe} reports are typically not felicitous  with direct evidence \citep{stephenson_thesis_2007, charlow2021experiential}. Having said that, no such incompatibility is present with \textit{consider}, so we can reasonably attribute the unacceptability of (\ref{pdc_consider:Fusco}) to a genuine doxastic clash.}
\z
%\label{pdc_believe:Fusco}

Indeed, (\ref{pdc_believe:Fusco}) gives rise to the well-known Moore's paradox. Note that, \textit{ex hypothesi}, Mary is acquainted with the taste of the pie under normal conditions, so the apple pie is actually tasty according to her: In other words, the first conjunct is naturally interpreted as being anchored to Mary's judgement. Moreover, we are taking Mary as not changing her belief about the pie.

In order to avoid a doxastic conflict, (\ref{pdc_believe:Fusco}) would be felicitously paraphrased using a verb of perception, since this would not entail a belief, as in (\ref{pdc_perceive:Fusco}):

\ea []{Your apple pie is tasty, but I have just taken my medicines so I don’t perceive/feel it (as) tasty.} \label{pdc_perceive:Fusco}
\z

Turning now to SAVs, we observe that \textit{consider} gives rise to the same contradiction as (\ref{pdc_believe:Fusco}), thus confirming that \textit{consider} does have a doxastic component, as expected:
\ea [\#]{Your apple pie is tasty, but I have just taken my medicines so I don’t consider it tasty.}\label{pdc_consider:Fusco}

\sn ⇒ \# I believe your apple pie is tasty ∧ I don’t believe your apple pie is tasty.
\z

If \textit{find} implied a belief,  embedding under \textit{find} should be equally unacceptable, since Mary would still be in a doxastic conflict. 

\ea Your apple pie is tasty, but I have just taken my medicines so I don’t find it tasty. \label{pdc_find:Fusco}
\z

\citet{fusco2022sav} set up a forced-choice task in Italian, assessing speakers’ preferences between \textit{trovo} (‘find’) and \textit{considero} (‘consider’) in sentences like (\ref{pdc_find:Fusco}). The results gave partial support to the hypothesis that \textit{trovo} lacks a doxastic component, however no significant effect of the PDC factor was found with PPTs, since \textit{trovo} was the preferred option across both conditions. This was possibly due to the limitations of the forced-choice design, which doesn’t allow to measure fine-grained variations in \isi{acceptability} among conditions. Moreover, one potential confound is that in sentences like (\ref{pdc_find:Fusco}), the referential noun phrase in the introductory clause refers to a kind (“the kind of pie that you usually make”), but in the second clause, the anaphoric pronoun it refers to a specific instance of that kind in the relevant situation.

\section{Experimental study}
\label{sec:present_study:Fusco}

In order to assess the empirical adequacy of various theoretical proposals about the precise characterization of SAVs, this study investigates the effect of PDC contexts, i.e. contexts of belief-perception mismatch, on the perceived \isi{acceptability} of \isi{SAV} reports in English. 
%This study investigates the effect of PDC contexts on the perceived \isi{acceptability} of \isi{SAV} reports in English, to assess the empirical adequacy of various theoretical proposals about the precise characterization of SAVs. 
More specifically, building on the experimental setup adopted by \citet{fusco2022sav}, we ask whether the presence of a PDC affects the perceived \isi{acceptability} of sentences where \textit{find} or \textit{consider} embed a small clause complement. The present study, then, had a 2x2, within-subjects design, with “Context” (PDC vs no-PDC) and “\isi{SAV}” (\textit{find} vs. \textit{consider}) as main factors. To overcome the limitations of the previous study, we devised a 7-point \isi{acceptability} task, in which participants were presented with contexts followed by a monoclausal target sentence. Also, by separating the context and the target sentence, the potentially problematic anaphoric relation (discussed above) was avoided. Subjects were then asked to rate the target sentence as a possible description of the preceding context. 

\subsection{Materials and procedure}
\label{sec:materials:Fusco}

Each experimental item contained a context and a target sentence. In every context a \isi{subject} has a general evaluation about an \isi{object} (or a type of \isi{object}), but, while in situations with no PDC her perception gives rise to an evaluation that is consistent with the general one, contexts with PDC give rise to a contrasting evaluation. Therefore, all the contexts, presented in short narrative texts, described a situation in which a \isi{subject} \textit{S} has a certain belief \textit{p} about some \isi{object}, with which \textit{S} is already acquainted. However, contexts varied in whether a PDC, due to a belief-perception mismatch, arises. 

The target sentences consisted of present tense 3rd-person attitude attributions, having \textit{S} as matrix \isi{subject} and \textit{p} as (reduced) complement clause, and were manipulated as to have \textit{find} or \textit{consider} as their matrix verb. To present a complete example, (\ref{contexts:Fusco}) describes a scenario in the PDC and no-PDC condition respectively, while (\ref{find_consider:Fusco}) lists the two possible target sentences.

\ea \label{contexts:Fusco}
\ea Context (PDC): Nora has to take some medicine that alters her taste for a little while. Nonetheless, one day, after taking her medicine, she can’t help having a slice of her mother’s apple pie, her favorite. The taste is terrible due to the medicine, but she knows it is tasty. \label{context_pdc:Fusco}
\ex Context (no-PDC): Nora is at her parents’ house. She’s on a diet but she can’t help having a slice of her mother’s apple pie, her favorite one. The taste is delicious, but Nora regrets eating the cake afterwards.
\z
\ex \label{find_consider:Fusco} \ea Nora finds the pie tasty.
\ex Nora considers the pie tasty.
\z
\z

In this design, target sentences are in the affirmative form, in order to avoid a potential scope ambiguity of negation. Thus, assuming that \textit{find} lacks a doxastic component and tracks direct experience instead, reports with \textit{find} in PDC scenarios are expected to be less acceptable. 16 scenarios were created, each involving a different vague adjectives/PPTs. The adjectives used as main predicates of the complement clauses are the following: 

\ea \textit{annoying}, \textit{dangerous}, \textit{easy}, \textit{fragrant}, \textit{frightening}, \textit{funny}, \textit{light},  \textit{long}, \textit{moving}, \textit{pleasurable}, \textit{shabby}, \textit{small}, \textit{tall}, \textit{tasty}, \textit{warm},  \textit{worrying}
\z

Every scenario was developed in the 4 factor combinations described above, yielding a total of 64 context-target pairs. The total number of pairs was split into 4 lists, so that each list contained 4 items for each factor combination. 16 filler items were also added to all lists, resulting in a total of 32 items per list.

\subsection{Procedure}
\label{sec:procedure:Fusco}

Subjects were automatically assigned to one of the four lists as to ensure equal distribution among lists. Therefore, each participant saw a total of 32 items in random order. Participants were asked to rate, on a 7-point Likert scale, how much they agreed with the target sentence as a possible description of the preceding context. The test was administered online on the NeTS Lab experimental platform. \figref{fig:sample_item:Fusco} illustrates an experimental item in the PDC – \textit{find} condition, as shown on the online platform.

\begin{figure}
%\includegraphics[height=.27\textheight]{../figures/Fusco_clip_image016.png}
\includegraphics[width=1.0\textwidth]{../figures/Fusco_clip_image016.png}
\caption{Sample item as viewed from the experimental platform.}
\label{fig:sample_item:Fusco}
\end{figure}

A total of 51 self-reported native speakers of English (21 American, 3 Australian, 27 British; Age range=24-74, M=38.16, SD=13.71) took part in the experiment.

\begin{figure}
%\includegraphics[height=.35\textheight]{../figures/Fusco_ratings.png}
\includegraphics[width=1.0\textwidth]{../figures/Fusco_ratings.png}
\caption{Acceptability ratings according to the various experimental conditions. Acceptability ratings (from 1 = unacceptable to 7 = completely acceptable) for SAVs (\textit{consider} and \textit{find}) after contexts triggering a potential doxastic conflict (PDC) or not (no-PDC). Gray bars indicate response counts, solid lines kernel density estimation of the distribution, dotted lines the mean values.}
\label{fig:ratings:Fusco}
\end{figure}

\subsection{Results}
\label{sec:results:Fusco}

\figref{fig:ratings:Fusco} shows the distribution of Likert responses (1-7) for each combination of context and \isi{SAV}. \tabref{tab:descriptive_statistics:Fusco} provides descriptive statistics.

\begin{table}
\caption{Descriptive statistics for raw ratings across conditions}
\label{tab:descriptive_statistics:Fusco}
 \begin{tabularx}{.8\textwidth}{XX rrr}
  \lsptoprule
    Context & SAV & mean & sd & median \\
  \midrule
  no-PDC & \textit{consider} &   5.74  & 1.40 & 6\\
    & \textit{find} &  5.89  &  1.39 & 6\\
  PDC & \textit{consider} &   4.83  & 1.65 & 5\\
    & \textit{find} &   3.93 &   1.78 & 4\\
  \lspbottomrule
 \end{tabularx}
\end{table}

The results were analyzed with the R programming language using the \textit{ordinal} package (\cite{christiansen_ordinal_2023}). Data were fitted against a mixed effects ordinal logistic regression model, with a fixed effect structure of the form CONTEXT×\isi{SAV} and a random effect structure allowing for by-\isi{subject} and by-item intercepts. Therefore, the model estimated 2 fixed effect parameters and 1 interaction parameter. A type III analysis of variance based on mixed ordinal logistic regression indicated no statistically significant effect on \isi{acceptability} of CONTEXT ($χ^2$(1) =4.34×10$^{−10}$, n.s.) or of \isi{SAV} ($χ^2$(1) = 1.74×10$^{−8}$, n.s.). However, there was a statistically significant CONTEXT×\isi{SAV} interaction ($χ^2$(1) = 27.067, p < .001). 
%Reference points for the model were set to no-PDC for context and \textit{consider} for \isi{SAV}. 
%Pairwise comparisons using Tukey’s HSD test indicated that Likert scores for ‘a’ vs. ‘c’ were statistically significantly different (Z = -2.78, p < .05), but not for ‘a’ vs. ‘b’ (Z = -0.85, n.s.) or for ‘b’ vs. ‘c’ (Z = -1.82, n.s.).
%The data were fitted using a range of linear mixed-effect models, including one or both factors (“Context” and “\isi{SAV}”) as fixed effects; the random structure for all the models included random by-\isi{subject} and by-item intercepts. The analysis revealed a significant main effect both of “Context” (χ2=183.07, df=1, p<0.0001 ) and “\isi{SAV}” (χ2=11.96, df=1, p<0.001). Crucially, the analysis also highlighted a significant interaction between the two factors (χ2= 43.97, df=2, p<0.0001). 

To further understand the interaction, a pairwise comparison using Tukey’s HSD test was performed. The comparison revealed that the PDC condition triggered significantly lower \isi{acceptability} rates both with \textit{consider} (no-PDC consider – PDC consider: estimate = -1.258, SE = 0.187, z = -6.739  p < .0001) and with \textit{find} (no-PDC find – PDC find: estimate = -2.623, SE = 0.204, z = -12.877, p < .0001). However, in the no-PDC condition, \isi{acceptability} is not significantly affected by the kind of \isi{SAV} present (estimate = 0.292, SE = 0.190, z = 1.534, n.s.), while in the PDC condition, \isi{acceptability} with \textit{find} is significantly lower than with \textit{consider} (PDC consider - PDC find: estimate = -1.073, SE = 0.181, z = -5.916, p < .0001). This indicates that when the context did not trigger a PDC, the type of \isi{SAV} used for the attitude report did not significantly affect the ratings. In contrast, contexts that triggered a PDC, although leading to generally lower ratings, asymmetrically impacted the \isi{acceptability} of the two SAVs in the target sentences. Specifically, the decrease in \isi{acceptability} was greater for \textit{find} reports than for \textit{consider} reports.


\section{Discussion}
\label{sec:discussion:Fusco} 
In this study, the doxastic status of SAVs was assessed by eliciting \isi{acceptability} judgements of 3rd-person attributions preceded by relevant contexts, which were manipulated in order to license or not license a PDC. We hypothesized that a \textit{find} ascription reports the \isi{subject}'s experience rather than a doxastic attitude. Therefore, in contexts that allow a PDC, a judgment compatible with belief but not with direct experience would be less acceptable under \textit{find} than under \textit{consider}. In fact, results seem to confirm this hypothesis: attributions with \textit{find} after PDC contexts produced significantly lower \isi{acceptability} ratings than their counterparts with \textit{consider}. 

This outcome clearly favours some accounts of SAVs over others. A doxastic theory, such as the one proposed by \citet{kennedy_willer_2022}, fails to account for the contrast observed, since in both cases the corresponding belief ascription is true and, additionally, being felicitously embedded under \textit{find} in the no-PDC condition, the content of the ascription qualifies as counterstance contingent: therefore, the lower \isi{acceptability} of \textit{find} ascriptions could not be attributed to \isi{presupposition} failure. 

On the other hand, the analysis proposed by \citet{munoz2019} is better suited to explain the results: The absence of a doxastic component in the denotation of \textit{find} allows the system the necessary flexibility to represent experiential content independently enough from belief. More generally, we take the results to support the idea that direct experience is indeed essential for the kind of \isi{subjectivity} that licenses \textit{find} reports, as in the original proposal by \citet{stephenson_thesis_2007}.

Our results also seem to point towards a more substantial difference between the two attitude verbs, such that one is not just more restrictive than the other in terms of embedding behaviour. Rather, as it was suggested by \citet{stojanovic_mcnally2023}, the two verbs would represent different kinds of attitudes. In this perspective, we think that a promising direction for future work would be to place the discussion about SAVs in the broader context of attitude predicates, and thus to assess similarities and differences with a wider range of verb classes (see e.g., \cite{anand_hacquard_2013epistemics}).

Before concluding, we would like to address a few concerns about the study presented here, which will highlight further directions for future research. The first one, which has been raised by Caroline Heycock (p.c.), is that the present tense form of both \textit{find} and \textit{consider} may be associated with both generic and episodic readings. This ambiguity could represent a potential confounding factor in the experiment: Indeed, even after PDC contexts, mean ratings for \textit{find} ascriptions are still around the scale midpoint, suggesting not complete unacceptability. This could be seen as an unwelcome result, given our hypothesis about the lack of a doxastic component for \textit{find}. We acknowledge the possible ambiguity of our target sentences, which could be avoided in future studies by using adverbial markers like \textit{now} that would force an episodic reading.

Apart from this recommendation, we would like to offer a few observations about the potential ambiguity and the interpretation of the present results. On one hand, under a generic reading,  doxastic and non-doxastic theories would both predict the \isi{acceptability} of \textit{find} reports after PDC: Recall that the scenarios made clear that the \isi{subject} is already acquainted with the \isi{object} of the attitude and that she already has a belief about it. On the other hand, under the episodic reading, a doxastic analysis would still predict \isi{acceptability}, while a non-doxastic account would predict unacceptability, because the situation does not prompt the \isi{subject} to revise her previous beliefs. This becomes clear if, from a \textit{find} ascription, we generate different paraphrases to reflect the relevant interpretations, as in (\ref{gen_think_tasty:Fusco}--d):

\ea Context: as in (\ref{context_pdc:Fusco})
\z
\ea Nora finds the pie tasty.
\ea[\approx]{Generally, Nora thinks that the pie is tasty.} \label{gen_think_tasty:Fusco} 
\ex[\approx] {Generally, the pie tastes good to Nora.} \label{gen_taste_good:Fusco}
\ex[\approx]{Right now, Nora thinks that the pie is tasty.} \label{now_think_tasty:Fusco}
\ex[\approx]{\# Right now, the pie tastes good to Nora.} \label{now_taste_good:Fusco}
\z
\z

Therefore, after a context triggering a PDC, a doxastic analysis would always predict \isi{acceptability}, whereas a non-doxastic analysis would predict \isi{acceptability} only with the generic reading.\footnote{Of course, all the readings would be available in no-PDC notexts, given that the current experience and belief about the pie would be in agreement with the past experiences and the beliefs formed upon them.} Consequently, even if the ambiguity is taken into account, the contrast observed in the data in the PDC condition seems to be best predicted by a non-doxastic account.\footnote{A more compelling argument could be made if we could demonstrate that the observed contrast does not arise from an aversion of \textit{consider} to episodic readings. While we do not have data on this matter, future research would benefit from examining sentences like (\ref{ex:crazy:Fusco}):%\newline

\ea ?Today/this morning everyone considers me crazy.\label{ex:crazy:Fusco}%\newline\newline
\z
If such examples are deemed acceptable, it would suggest that \textit{consider} indeed allows for both readings while still producing the observed contrast.
} 

Of course, the issue will need further scrutiny. One way to address it would be to disentangle the various readings using suitable adverbial \isi{modifiers}, such as \textit{generally} and \textit{(right) now}, thus producing a paradigm along the following lines:

\ea Generally, Nora finds the pie tasty. \label{gen_find:Fusco}
\ex Generally, Nora considers the pie tasty. \label{gen_consider:Fusco}
\ex Right now, Nora finds the pie tasty. \label{now_find:Fusco}
\ex Right now, Nora considers the pie tasty. \label{now_consider:Fusco}
\z

Our prediction would be that, after PDC contexts, both (\ref{gen_find:Fusco}) and (\ref{gen_consider:Fusco}) would be fine, since, presumably, previous generalizations are not affected by the present experience. Although there is reason to believe that both (\ref{now_find:Fusco}) and (\ref{now_consider:Fusco}) would be degraded to some extent (since the adverbial modification would make the present experience more salient)  we would expect an asymmetry to arise, with (\ref{now_find:Fusco}) more acceptable than (\ref{now_consider:Fusco}). 

The second concern that we would like to tackle pertains to an aspect of \isi{SAV} embedding patterns that we have ignored so far, namely the syntactic structure of their complement clauses. In the foregoing discussion and in the experiment, we have considered only patterns of SAVs involving small clauses. However, the literature also reports the possibility for \textit{find} to embed finite clauses (see e.g., \cite{bouchard_thesis_2012,kennedy_willer_2022}), while in some languages this is the only option available for \textit{find}-verbs (\cite{saebo_2009,korotkova_anand_find_2021}). The question, then, is whether the two structures are equivalent with respect to the phenomena discussed so far and, in particular, with respect to the results obtained from the experiment. To that regard, while small clauses under SAVs are generally treated as analogous to finite clausal complements in expressing a proposition, \citet{crespo2019tasting} provide evidence against such a simplistic view. They report the following contrast:

\ea Context: Anna is a newborn baby.
\ea[]{\label{ex:infant-formula:Fusco}Look! Anna finds infant formula tasty!}
\ex[\#]{\label{ex:infant-formula-alt:Fusco}Look! Anna finds that infant formula is tasty!}
\z
\z

% This asymmetry suggests, at the very least, that syntactic differences may correlate with differences in semantic interpretation.
This asymmetry suggests, at the very least, that syntactic differences may correlate with differences in semantic interpretation. Specifically, (\ref{ex:infant-formula:Fusco}) is compatible with our proposal that \textit{find} does not involve a doxastic component, assuming that the speaker does not intend to attribute a doxastic attitude to the baby. From this perspective, however, (\ref{ex:infant-formula-alt:Fusco}) is unexpected and suggests that \textit{find} $+$ finite clause may convey a slightly different attitude.\footnote{A possible interpretations could be that, with \textit{find} $+$ finite clause, the embedded proposition $p$ must be accessible to the \isi{subject}, in the sense of \citet{yalcin2018}, i.e., the \isi{subject} must be able to ask herself \textit{whether}-$p$. A thorough examination of this hypothesis is deferred to future research efforts.}
Thus, it is not at all obvious that the results we obtained with small clause complements would apply to \textit{find}  ascriptions with finite complements. Determining this, however, is beyond the scope of the present work and will have to be addressed in future research.

A third limitation, which has been raised by several anonymous reviewers, is the lack of a direct comparison with \textit{believe} reports in the relevant conditions. As we noted in \sectref{sec:introduction:Fusco}, we deliberately restricted this study to \textit{find} and \textit{consider}. One reason for this choice was that, as seen above, \textit{believe} has other implications regarding weak commitments and/or lack of direct evidence. The other is that including such verbs into our experimental setting would require us to test items in a slightly different syntactic form: Instead of using a small clause, which is not ideal for the complement clause of \textit{believe}, we should adopt an infinitive \textit{to be} clause. However, we are not sure that our predictions would be borne out in such a syntactic environment, for the reasons also mentioned above. Therefore, we opted for a more minimal comparison between \textit{consider} and \textit{find}. However, future research into this topic should also include a more comprehensive comparison with other clause embedding verbs, including doxastics such as \textit{believe} and \textit{think} but also perception predicates like \textit{perceive} or \textit{feel}. This will surely shed more light on the \isi{subject} under investigation. 

\section*{Acknowledgements}
We would like to express our gratitude to the audiences at CSSP 2023 and the DGfS 2024 workshop ``(De-)composition and Modification of Attitude Predicates" for their valuable feedback and insightful discussions, which greatly contributed to shaping the ideas presented in this work. We also wish to thank the three anonymous reviewers for their constructive critiques, which helped us refine our arguments and highlight avenues for future research.

\clearpage
\section*{Abbreviations}
\begin{multicols}{2}
\begin{tabbing}
SAV \hspace{1em} \= predicates of personal taste\kill
SAV \> subjective attitude verb\\
PPT \> predicates of personal taste\\
PDC \> potential doxastic conflict\\
\end{tabbing}
\end{multicols}
%John Doe contributed to conceptualization, methodology, and validation. 
%Jane Doe contributed to writing of the original draft, review, and editing.

\sloppy

\printbibliography[heading=subbibliography,notkeyword=this]
\end{document}

%%% Local Variables:
%%% mode: xelatex
%%% TeX-master: t
%%% End:
