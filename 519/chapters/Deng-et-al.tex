\documentclass[output=paper,colorlinks,citecolor=brown]{langscibook}
\ChapterDOI{10.5281/zenodo.15450432}
\author{Chenyuan Deng\orcid{0009-0004-0490-5250}\affiliation{Humboldt-Universität zu Berlin} and 	Antonio Machicao y Priemer\orcid{0000-0001-7321-0795}\affiliation{Humboldt-Universität zu Berlin} and  	Giuseppe Varaschin\orcid{0000-0003-1446-2700}\affiliation{Humboldt-Universität zu Berlin}
        }

\title[Unifying modifiers, classifiers and demonstratives]{Unifying modifiers, classifiers and demonstratives}


\abstract{We argue that the distributional properties of modifiers, classifiers and demonstratives with respect to nouns in Mandarin Chinese motivate a head-functor  approach \citep{VanEynde06a}, where numeral-classifier-noun sequences are analyzed as left branching structures with the noun as the head. This approach explains the distributional similarities between all prenominal categories by unifying their combinatorial properties under a single phrasal schema  while also accounting for their differences by means of selectional constraints and a hierarchy of \textsc{marking} values. We also analyze classifiers themselves as special kinds of noun. All in all, our analysis entails that nominal complexes in Mandarin Chinese are fundamentally different from those found in languages with dedicated specifiers (e.g.\ determiners in a language like English), suggesting a two-way typology that is parallel to the NP/DP parameter proposed in the minimalist tradition (\citealt{boskovic2008will}, i.a.).}

%move the following commands to the "local..." files of the master project when integrating this chapter
%\usepackage{tabularx}
%\usepackage{langsci-optional}
%\usepackage{langsci-gb4e}
%\usepackage[linguistics]{forest}
%\usetikzlibrary{arrows.meta,decorations.text}
%\usepackage{langsci-avm}
%\usepackage{multicol}
%\usepackage{listings}
%\lstset{% general command to set parameter(s)
 %basicstyle=\small,
 %% print whole listing small
 %stringstyle=\ttfamily}
%%\bibliography{localbibliography,bib-chinese,cssp}
%\bibliography{bib-chinese}
%\newcommand{\orcid}[1]{}

\IfFileExists{../localcommands.tex}{
   \addbibresource{../localbibliography.bib}
   % add all extra packages you need to load to this file

\usepackage{tabularx,multicol}
\usepackage{url}
\urlstyle{same}

\usepackage{listings}
\lstset{basicstyle=\ttfamily,tabsize=2,breaklines=true}

\usepackage{langsci-basic}
\usepackage{langsci-optional}
\usepackage{langsci-lgr}
\usepackage{langsci-osl}
% \usepackage{./langsci/styles/langsci-lgr}
% \usepackage{./langsci/styles/langsci-osl}
% \usepackage{langsci-gb4e}

\usepackage{tikz}
\usetikzlibrary{patterns,calc}
\pgfdeclarepatternformonly{south east lines}{\pgfqpoint{-0pt}{-0pt}}{\pgfqpoint{3pt}{3pt}}{\pgfqpoint{3pt}{3pt}}{
    \pgfsetlinewidth{0.6pt}
    \pgfpathmoveto{\pgfqpoint{0pt}{3pt}}
    \pgfpathlineto{\pgfqpoint{3pt}{0pt}}
    \pgfpathmoveto{\pgfqpoint{.2pt}{-.2pt}}
    \pgfpathlineto{\pgfqpoint{-.2pt}{.2pt}}
    \pgfpathmoveto{\pgfqpoint{3.2pt}{2.8pt}}
    \pgfpathlineto{\pgfqpoint{2.8pt}{3.2pt}}
    \pgfusepath{stroke}}
    
\usepackage{stmaryrd}
\usepackage{wasysym}
\usepackage{multirow}
\usepackage{caption}
\usepackage{subcaption}
\usepackage{mathrsfs}
\usepackage{qtree}

\usepackage{linguex}


   %pminos do not split footnotes
% \interfootnotelinepenalty=10000 %Footnote in Laporte chapters has to be split SN


%\DeclareIndexNameFormat{default}{%
%\nameparts{#1}%
%\usebibmacro{index:name}%
%{\index[names]}%
%{\namepartfamily}%
%{\namepartgiveni}%
% {}% L1
% {}% L2
%{\namepartprefix}% generates spurious space L3
%{\namepartsuffix}% generates spurious space L4
%}

%  {\DeclareIndexNameFormat{default}{%
%     \usebibmacro{index:name}{\index[names]}{#1}{#3}{#5}{#7}}}

%\DeclareIndexNameFormat{default}{%
%  \usebibmacro{index:name}{\sindex[nom]}{#1}{#3}{#5}{#7}}

%\DeclareIndexNameFormat{default}{%
%  \usebibmacro{index:name}{\sindex[person]}{#1}{#3}{#5}{#7}}
%\DeclareIndexNameFormat{default}{%
%\nameparts{#1} \usebibmacro{index:name}{\sindex[person]]}{\namepartfamily}{‌​\namepartgiven}{\nam‌​epartprefix}{\namepa‌​rtsuffix}}

%\newcommand{\smiley}{:)}

%\renewbibmacro*{index:name}[5]{%
%\usebibmacro{index:entry}{#1}%
%{\iffieldundef{usera}{}{\thefield{usera}\actualoperator}\mkbibindexname{#2}{#3}{#4}{#5}}}

% \newcommand{\noop}[1]{}

%remove for final
%\overfullrule=1mm

\newcommand{\tobi}[2]}}
\renewcommand{\S}[1]{\tobi{#1}{\textsc{*}}}

% this volume references
% puts: [this volume]
% already defined: \citetv
%\newcommand{\citepv}[1]{(\citeauthor{#1} \citeyear*{#1} [this volume])}
\newcommand{\citealtv}[1]{\citeauthor{#1} \citeyear*{#1} [this volume]}

%parentheses around example number
\newcommand{\pref}[1]{(\ref{#1})}

% in-text examples

\newcommand{\lnex}[1]{\textit{#1}} %target lang word
\newcommand{\lnlit}[1]{(lit.: `#1')} %literal reading
\newcommand{\lnlat}[1]{(#1)} % latinization
\newcommand{\lntrans}[1]{`#1'} %translation
\newcommand{\lnexl}[2]%
{\lnex{#1}{} \lnlat{#2}} % ex with latinization
\newcommand{\lnexlat}[3]{\lnex{#1}{} \lnlat{#2}{} \lntrans{#3}} % ex with latinization and tranl.

%ch01
\newcommand{\co}[1]{\mbox{\textbf{#1}}}

%ch09

\newcommand{\cyrbulg}[1]{\begin{otherlanguage*}{bulgarian}#1\end{otherlanguage*}}


%ch10
\newcommand{\nlp}{{\small NLP}}
\newcommand{\mwe}{{\small MWE}}
\newcommand{\rae}{{\small RAE}}
\newcommand{\lvc}{{\small LVC}}
\newcommand{\pos}{{\small P}o{\small S}}
%\newcommand{\todo}[1]{ \textcolor{red}{#1} }

%\renewcommand{\labelenumi}{\theenumi}
%\ainamefmt{{vv}{ll}{, ff}{, jj}} % fullname

\newcommand{\biberror}[1]{{\color{red}#1}}

\newcommand{\osenovaitem}{--~}
   %% hyphenation points for line breaks
%% Normally, automatic hyphenation in LaTeX is very good
%% If a word is mis-hyphenated, add it to this file
%%
%% add information to TeX file before \begin{document} with:
%% %% hyphenation points for line breaks
%% Normally, automatic hyphenation in LaTeX is very good
%% If a word is mis-hyphenated, add it to this file
%%
%% add information to TeX file before \begin{document} with:
%% %% hyphenation points for line breaks
%% Normally, automatic hyphenation in LaTeX is very good
%% If a word is mis-hyphenated, add it to this file
%%
%% add information to TeX file before \begin{document} with:
%% \include{localhyphenation}
\hyphenation{
    Beck-man
    Ngu-yen
    back-chan-nel
    back-chan-nels
    mo-not-o-nous
    ste-reo-typ-i-cal
}

\hyphenation{
    Beck-man
    Ngu-yen
    back-chan-nel
    back-chan-nels
    mo-not-o-nous
    ste-reo-typ-i-cal
}

\hyphenation{
    Beck-man
    Ngu-yen
    back-chan-nel
    back-chan-nels
    mo-not-o-nous
    ste-reo-typ-i-cal
}

   \boolfalse{bookcompile}
   \togglepaper[3]%%chapternumber
}{}

%\input{tex-commands}

%%%%%%%%%%%%%%%%%%%%%%%%%%%%%%%%
%% Subscript & Superscript: no italics inside math mode
%\newcommand{\down}[1]{\textsubscript{#1}}
%\newcommand{\val}[1]{\emph{#1}}		%Values & Types
%\newcommand{\nix}[1]{#1}
%\newcommand{\sizefig}{.65}	% f.e. resize trees, use with "scalebox"
%\newcommand{\sizeavm}{.7}	% f.e. resize trees, use with "scalebox"
%\newcommand{\obj}[1]{\emph{#1}}                 %Emphasising
%%\newcommand{\term}[1]{\textsc{#1}}              %for abbreviated terminology
%\newcommand{\feat}[1]{\textsc{#1}}	%Features
%\forestset{
	%%% aligns every terminal node with the tree's lowest terminal node:			
	%bottom word/.style={for tree={parent anchor=south, child anchor=north, align=center, base=bottom, where n children=0{tier=word,inner xsep=0pt,outer sep=0pt}{}}}, 
%}
%
%%%%%%%
%%% BibLaTeX:
%%% BibLaTeX Options
%\AtEveryBibitem{%
	%\clearfield{url}%
	%\clearfield{urlyear}%
	%%	\clearfield{number}%	
%}
%


\begin{document}
\maketitle

%\input{chapters/01-intro}
%\input{chapters/02-standard}
%\input{chapters/03-puzzles}
%\input{chapters/04-head-functor}
%\input{chapters/05-con}
%\input{chapters/07-figs}

\section{Introduction}\label{sec:int:deng}
The structure of Mandarin Chinese nominal complexes (CNCs) is \isi{subject} to extensive research, mostly centering on their implications for  the NP/DP debate and right and \isi{left-branching} analyses \citep{cheng1999bare, bovskovic2013word, Her&Tsai20, jiang_jenks_jin_2022}.%
%
\footnote{We employ the term \emph{nominal complex} in order to not beg the question of the categorial status of these structures. }  %
%
However, there has been little work on CNCs in the tradition of \isi{HPSG} \citep{Pollard&Sag94a}. In this paper, we attempt to bridge this gap by focusing on the combination of prenominal elements (demonstratives,
modififiers, and classifiers) %(DEM, MOD, CLF) 
with the head N. (\ref{pre-n:deng}) illustrates the kind of structure we will deal with.
		
\ea

	\gll na da de san ben guanyu yuyanxue de shu\\
	\textsc{dem} \nix{big} \nix{\textsc{de}} three \textsc{clf}  \nix{about} \nix{linguistics} \nix{\textsc{de}} \nix{book}\\
	\glt  `those three big books about linguistics'  \label{pre-n:deng}
\z 			

As we will see when we examine other examples, the combinatorial properties of prenominal elements in CNCs pose several puzzles for both traditional NP accounts as well as for DP-based theories of nominal complexes. %For instance, the fact that \isi{demonstratives}, possessives and CLFPs are not obligatory, as we see in (\ref{bare:deng}), suggests that these categories behave in a way that very different from what we see for specifier elements in languages like English. In fact, in these languages, optionality a trait typically associated with \isi{modifiers}. 


%\ea \label{bare:deng}
%	\gll wo mai-le shu.\\
%	1.\textsc{sg} buy-\textsc{pfv} book\\
%	\glt `I bought \{a/the book/$\emptyset$/the books\}.'
%\z

%\ea[*]{I bought book. \label{eng-bare:deng}}
%\z


We argue that the distributional properties of modifiers (MODs) and demonstratives (DEMs) 
%We argue that the distributional properties of MODs and DEMs with respect to Ns and CLFs 
motivate a head-functor approach (HFA; \citealp{VanEynde06a}) with respect to Ns and classifiers (CLFs), where NUM-CLF-N sequences are analyzed as \isi{left-branching} NPs, and where the sequence NUM-CLF forms a CLFP that acts as a functor over the noun. The notion of functor replaces the more specific grammatical relations of specifier and \isi{modifier}, which are prevalent in NP approaches to the nominal domain. It also stands in contrast to DP analyses, which treat prenominal elements as heads of extended nominal projections.\footnote{The term \emph{functor} comes from categorial grammar, where it refers to any selecting category, i.e.\ heads, \isi{modifiers} and specifiers (\citealt[23]{Bouma88a}). We use the term here in the more restricted way to refer only to non-head daughters that impose selectional requirements on their sisters.}
The HFA allows us to unify the combinatorial properties of CLFs, MODs and DEMs, thus explaining their similarities in Mandarin Chinese, while also accounting for their differences by means of selectional constraints and a hierarchy of \textsc{marking} values. %In order to capture the behavior of Mods, we also propose an inheritance hierarchy where  \textsc{cl}s and lexical nouns are represented as subtypes of \textit{noun}s. %we examine interpretive and distributional properties of \isi{modifiers} and \isi{demonstratives} within CNSs. On the basis of the similar behavior of Mods and Dems, we propose a head-functor NP approach, where [Num Cl N] sequences are analyzed as \isi{left-branching} constituents. 
%All in all, our analysis confirms the validity of the NP analysis (\citealt{MyP&Mueller21a}), preserving its main advantages. But our treatment also entails
Our treatment also entails that CNCs are fundamentally different from nominal structures in languages with dedicated specifiers, suggesting a two-way typology of languages (\isi{head-specifier} or head-functor) that is parallel to the \isi{NP/DP parameter} proposed in the minimalist tradition (\citealt{boskovic2008will, BovskovicCo2011a, Despic2011a, Phan2015a}, i.a.). %n contrast to typical specifier-head languages like English or German %a language without dedicated determiners,

In \sectref{sec:stand:deng},  we briefly review two standard theories of nominal complexes: the \isi{head-specifier} NP approach and the head-complement DP approach.   In \sectref{sec:puz:deng}, we discuss the puzzles CNCs pose for both of these approaches. In \sectref{sec:h-f:deng}, we summarize the HFA alternative  and propose an analysis of the CNC data. Lastly, \sectref{sec:conc:deng} summarizes our approach and situates it in the context of a broader typological hypothesis about nominal complexes across languages.%(including about HFA)



\section{Standard views about nominal complexes}\label{sec:stand:deng}

In this section, we summarize two main approaches to the combinatorics of nominal complexes (NCs): a standard NP approach, which posits that NCs are headed by nouns that take determiners and other functional prenominal elements as their specifiers; and a DP approach, which assumes that NCs are headed by determiners that take nouns as their complement.  For the sake of commensurability with our own theory,  we focus on simple \isi{HPSG} variants of  these approaches, but most of what we say here is independent of choice of formalism. 

Let us start with the standard NP approach, which is rooted in classic X$'$ theory (\citealt{Chomsky1970a, Jackendoff77a}) and found its way to \isi{HPSG} through works like \citet[139--143]{Pollard&Sag87a}, \citet[34]{Ginzburg2000a}, \citet[102]{Sag&Co03a} and \citet[84]{Levine17a}. Its defining assumption is the idea that NCs are formed by first combining a nominal head with all of its complements/\isi{modifiers} and subsequently by adding a specifier that closes off the nominal projection to further combinations and allows it to serve as an argument to other heads. Specifiers of N typically belong to the functional category D, which encompasses definite articles (\textit{the}), \isi{demonstratives}  (\textit{this}), genitives (\textit{the queen's}), quantifiers  (\textit{some}, \textit{each}), and measure phrases (\textit{a lot of}). Some examples are given in (\ref{spec-ex:deng}).%
%
\footnote{Semantically, it is generally assumed that specifiers are functions that map the property meaning of N (of type $\langle e,t\rangle$) into a semantic type that can be combined with the property expressed by the VP (\citealt{Heim1998a}). This can be either an entity or a generalized quantifier (of type $\langle\langle e,t\rangle, t\rangle$).  In this paper, we \isi{focus} mostly the syntactic combinatorics of CNCs, leaving an account of their semantic composition for future work (see \citealt{Krifka95a} for a proposal).} %As the genitive and measure phrase examples show, specifiers can be syntactically complex. 
%

\ea \label{spec-ex:deng}
	a.  $\left\{\begin{array}{l}
		\text{The}\\
		\text{These}\\
		\text{The queen's}\\
		\text{Some}\\
		%\text{Two}\\
		\text{A lot of}\\
		\end{array}\right\}$ books are bad.  
	\hspace{2em} b. $\left\{\begin{array}{l}
		\text{The}\\
		\text{This}\\
		\text{The queen's}\\
		\text{Some}\\
		\text{Each} \\
		\end{array}\right\}$ book is bad. 
\z


In \isi{HPSG}, combinations of these functional prenominal elements with N are licensed by the schema in (\ref{head-spec-rule:deng}) as  structures of the type \val{\isi{head-specifier} phrase}.

	
\ea \label{head-spec-rule:deng}
%\scalebox{\sizeavm}{
	\val{head-specifier-phrase} $\Rightarrow$ \\\avm{
[\type*{headed-phrase}
synsem|loc|cat|spr & <> \\
\punk{head-dtr|synsem|loc|cat}{[spr <\1> \\ comps <>]}
non-head-dtrs   <[synsem  & \1]> 
]} 
\z 

\noindent The schema requires the head daughter to have the \val{synsem} of the non-head daughter in its \textsc{spr} list and the mother to have an empty \textsc{spr}, blocking the combination with further specifiers. The head daughter is also constrained to have an empty \textsc{comps} list, to ensure that a noun can only combine with its specifier after its complements. Furthermore, heads that select NCs require their N arguments to have empty \textsc{spr} lists, so a specifier will always be necessary to integrate Ns in a larger phrasal structure. This  ensures, for instance, that a verb selecting a noun can only combine with the noun after it has already been saturated by a specifier.\footnote{So structures like *\textit{Amy read book} are predicted to be ungrammatical.  For bare plurals and mass nouns  (which can combine with other heads without an overt determiner) it is necessary to assume either an empty determiner that only combines with Ns of the appropriate type or a unary rule that identifies this exact class of Ns and removes the determiner from their \textsc{spr} list. Similar provisions are necessary under  DP analyses, which make exactly the same predictions.}
Since \isi{head-specifier} structures are required to be of type \textit{headed-phrase}, they also have to abide by the Head-Feature Principle, given in (\ref{head-feat-principle:deng}) (\citealt[34]{Pollard&Sag94a}). A schematic depiction of the effects of (\ref{head-spec-rule:deng})--(\ref{head-feat-principle:deng})  is shown in \figref{fig:sch-spr:deng}. %\footnote{The only alternative is to posit a unary a unary rule that empties the \textsc{spr}  list of N.} 


\ea \label{head-feat-principle:deng}
%\scalebox{\sizeavm}{
	\val{headed-phrase} $\Rightarrow$ \\\avm{
[\type*{headed-phrase}
synsem|loc|cat|head & \1 \\
\punk{head-dtr|synsem|loc|cat|head}{\1}
]} 
%}
\z 

%\ea
\begin{figure}
    \centering
    \begin{forest}
	[\val{head-specifer-phrase}\\
	\avm{
		[head & \1\\
		spr & <> ] $\approx$ NP
	}, name=phrase
	%%%
	[\val{specifier}\\
	\avm{
		\2[head & det]	$\approx$ D
	}]
	%%%
	[\val{head}\\
	\avm{
		[head & \1 noun\\
						spr & <\2>
						] $\approx$ N$'$
	}, name=head]
	]
	\draw[->] (head) to[out=east,in=east] (phrase);
\end{forest}	
    \caption{NP analysis (\emph{head-specifer-phrase})}
    \label{fig:sch-spr:deng}
\end{figure}
%\ex \label{sch-spr:deng}
%\scalebox{\sizefig}{



Crucially, \isi{modifiers} of N are introduced by the separate combinatorial schema in (\ref{head-adjunct-schema:deng}), which defines a distinct phrasal type \val{head-adjunct-phrase} (\citealt[56]{Pollard&Sag94a}). According to this schema, \isi{modifiers} select their heads by means of their \textsc{mod} feature, which is structure-shared with the head's \textit{synsem}. The combination with a \isi{modifier} has no effect over the categorial properties of the phrasal head.  
%
A simplified example of a structure satisfying these schemas is given in \figref{fig:head-adj:deng}.

\ea \label{head-adjunct-schema:deng}
%\scalebox{\sizeavm}{
\val{head-adjunct-phrase} $\Rightarrow$ \\
\avm{
[\type*{headed-phrase}
head-dtr|synsem \1 \\ %[spr & nelist] 
non-head-dtrs  <[synsem|loc|cat & [head|mod & \1]]>
]} 
%}
\z

%\ea 
%\scalebox{.75}{
%\begin{forest}
%	[
%	\avm{
%		[head & \1\\
%		spr & <> ] 
%	} 
%	%%%
%	[
%	\avm{
%		\2[spec & \3]	
%	} [these] ]
%	%%%
%	[
%	\avm{
%		\3[head & \1 noun \\
%						spr & <\2>
%						] 
%	}
%	[\avm{
%		[head & adj \\
%						mod & \3
%						] 
%	} [red] ] [\avm{
%		\3[head & \1 noun \\
%						spr & <\2>
%						] 
%	}  [books] ]
%	]
%	]
%\end{forest}	
%}
%\z

%\ea 

\begin{figure}
  \centering
\begin{forest}
	[
	\avm{
		[head & \1\\
		spr & <> ] 
	} 
	%%%
	[
	\avm{
		\2[head & det]	
	} [the, tier=word] ]
	%%%
	[
	\avm{
		[head & \1 \\
						spr & <\2>
						] 
	}
	[\avm{
		[head & [\type*{adj} mod & \3] ] 
	} [red, tier=word] ] [\avm{
		\3[head & \1 [\type*{noun} concord & pl] \\
						spr & <\2>
						] 
	}  [books, tier=word] ]
	]
	]
\end{forest}	
  \caption{Modified NP}
  \label{fig:head-adj:deng}
\end{figure}


On this analysis the NC is headed by the noun, which, therefore, determines both the internal combinatorics of the phrase (e.g.\ which kinds of arguments can be introduced in the NC) as well as its distribution when combining with other heads. This seems to be intuitively correct if we look back at (\ref{spec-ex:deng}), where we see that, for at least some of the specifiers (e.g.\ \textit{the}, \textit{some}), the inflection on V seems to depend only on the number value of N. Strikingly,  in (\ref{spec-ex:deng}a), the measure phrase specifier \textit{a lot of} is singular, but V must be plural, in conformity with the number value of \textit{books}. This makes sense under an NP approach, where many of the categorial properties of NCs (all of the \textsc{head} features, including the agreement values under \textsc{head}|\textsc{concord}) are inherited from N due to (\ref{head-feat-principle:deng}).

	
The opposite holds on the DP approach, proposed in \isi{HPSG} by \citet{Netter94a} on the basis of GB work by \citet{Abney87a}. Under this analysis, the combination of prenominal elements like D with nominal projections is viewed as a  \val{head-complement-phrase}.  The principle constraining structures of this sort is (\ref{head-comp-schema:deng}).


\ea \label{head-comp-schema:deng}
%\scalebox{\sizeavm}{
	\val{head-complement-phrase} $\Rightarrow$ \\\avm{
[\type*{headed-phrase}
synsem|loc|cat|comps & \1 \\
\punk{head-dtr|synsem|loc|cat|comps}{<\2> \+ \1}
non-head-dtrs  <[synsem  & \2]> 
]} 
%}
\z 


\noindent According to (\ref{head-comp-schema:deng}), the prenominal element selects a nominal projection via its valence feature \textsc{comps}. After both combine, the resulting phrase no longer has an NP in its \textsc{comps} list. Non D-heads always select  Ds with empty \textsc{comps} lists (i.e.\ DPs), ensuring that D is obligatory to integrate NCs in larger structures, which rules out bare N arguments \textit{*boy sings}. The requirement that D be fully saturated for it to be selected by other heads rules out structures like \textit{*a sings}, where a verb combines directly with a N-less determiner. A  NC licensed by (\ref{head-feat-principle:deng}) and (\ref{head-comp-schema:deng}) is shown in \figref{fig:dp-ex:deng}. Modifiers of N are assumed to be introduced by (\ref{head-adjunct-schema:deng}), as in the NP analysis. 

\begin{figure}
  \centering
  \begin{forest}
	[\val{head-complement-phrase}\\
	\avm{
		[head &  \1\\
		comps & <> ] $\approx$ DP
	}, name=phrase
	%%%
	[\val{head}\\
	\avm{
			[head &  \1 \\
				comps & <  \2>] $\approx$ D
	}, name=head] 
	%%%
	[\val{complement}\\
	\avm{
		  \2[			spr & <> \\ comps & <>  ] $\approx$ NP
	}] 
	]
	\draw[->] (head) to[out=west,in=west] (phrase);
  \end{forest}	
  \caption{DP analysis (\emph{head-complement-phrase})}
  \label{fig:dp-ex:deng}
\end{figure}

On this approach,  D determines the internal structure and the distributional properties of the NC as a whole. As we saw, this seems  counterintuitive in light of some of the English data in (\ref{spec-ex:deng}), where the agreement information seems to flow from N alone. However, it is a natural assumption if one considers  constructions like (\ref{ger-ex:deng}) in German, where the inflection on V appears to to be coming from D alone, given that the person values on D and N do not coincide (\citealt[301]{Netter94a}). 


\ea[]{\gll Du Idiot \{\! hast / *\! hat\} das Brot vergessen. \label{ger-ex:deng} \\
	you.\textsc{2sg} idiot.\textsc{3sg} {} have.\textsc{2sg} {} {} have.\textsc{3sg} the bread forgotten \\
	\glt `You idiot forgot the bread.'}
\z


Both (\ref{ger-ex:deng}) and (\ref{spec-ex:deng}) can be  accommodated to NP and DP approaches by appealing to underspecification. For instance, in the case of (\ref{ger-ex:deng}), a proponent of the standard NP treatment could say that German Ns are underspecified for a \textsc{person} value and simply inherit it from the \textsc{person} value of the D in their \textsc{spr} list.\footnote{\citet[47--51]{Pollard&Sag94a} propose a modified NP approach that combines aspects of the NP and DP analyses by incorporating a mechanism of mutual selection between the specifier and N. On their theory,  N selects its specifier via \textsc{spr} and the specifier also selects its N head through a dedicated \textsc{spec} feature. \citet[371--373]{Pollard&Sag94a} motivate this NP analysis on the basis of facts about declension classes and adjectival inflection in German (see also \citealt{MyP&Mueller21a}).  %. 
%A modified statement of the \isi{head-specifier} schema is shown in (\ref{head-spec-modified:deng}).
%
%\ea \label{head-spec-modified:deng}
%\scalebox{.75}{
%	\val{head-specifier-phrase} $\Rightarrow$ \avm{
%[\type*{headed-phrase}
%synsem|loc|cat|spr  <> \\
%%head-dtr \2 [synsem|loc|cat|spr  & <\1>]  \\
%\punk{head-dtr|synsem}{\2[loc|cat & [spr <\1> \\ comps <>]]}\\
%non-head-dtrs  <[synsem  & \1 [spec & \2]]> 
%]} 
%}
%\z 
%This account reclaims some of the advantages of the DP approach insofar as the \textsc{spec} feature allows prenominal functional elements to have a say in the internal combinatorics of NCs. Nonetheless, the theory is still an instance of the standard NP approach, because the head of the resulting phrase is a noun that selects its prenominal element through its \textsc{spr} feature, as (\ref{spec-approach:deng}) illustrates.
Since nothing here hinges on the difference between this and the more standard NP approach outlined above, we assume the latter for ease of exposition.
%
%\ea \label{spec-approach:deng}
%\scalebox{.75}{
%	\input{graphics/tree-specifier}
%}
%\z 

%\citet[371--373]{Pollard&Sag94a} motivate this NP analysis on the basis of facts abut declension classes and adjectival inflection in German which are also cited by \citet[333--339]{Netter94a} as arguments for his DP approach. 
}

Overall, the main difference between the NP and DP theories concerns \isi{locality} of selection (\citealt{Sag2010a}).  Assuming a head can only select the properties of its sister,  each approach makes different predictions about what features should be accessible for selection after an N is combined with a prenominal element. Treating NCs as instances of \textit{head-specifier-phrase} implies that the properties of the prenominal element (e.g.\ D) are not accessible after it is removed from \textsc{spr}. Treating NCs as objects of the sort \textit{head-complement phrase} implies, in turn, that the properties of N are not accessible after NP is removed from \textsc{comps}. 

In spite of these differences, the standard NP and DP analyses outlined above share many assumptions about the structure of NCs. Both approaches assume that there is a unique element -- typically, a determiner -- that has to be combined in the right order to complete NCs, turning them into suitable arguments for other selecting categories.  Once this special element is combined, the NC is closed off to further prenominal elements and its combinatorial history is no longer accessible  to other selectors. Consider the examples in (\ref{bad-ncs:deng})--(\ref{bad-ncs-4:deng}).

%\begin{minipage}[t][][t]{}

%\begin{multicols}{2}
%\ea 
	\ea[]{Amy bought [this (big) bed].} \label{bad-ncs:deng}
	\ex[*]{Amy bought [(big) bed].} \label{bad-ncs-2:deng}
	\ex[*]{Amy bought [this my big bed].} \label{bad-ncs-3:deng}
	%\ex[*]{Robin bought [red book]}
	\ex[*]{Amy bought [big this bed].} \label{bad-ncs-4:deng}
	\z
%\z
%\end{multicols}
%\end{minipage}

\noindent The standard NP and DP analyses outlined above are all unanimous in licensing (\ref{bad-ncs:deng}) and ruling out (\ref{bad-ncs-2:deng})--(\ref{bad-ncs-4:deng}). However, as we will see in \sectref{sec:puz:deng}, analogues of these structures are all acceptable in Mandarin Chinese. 



\section{Puzzles} 
\label{sec:puz:deng}

In \sectref{sec:stand:deng}, we provided a review of the standard theories about NCs (as proposed within \isi{HPSG}).  
%MyP
When taking Mandarin Chinese into account, the empirical data appears to not perfectly align with either of the proposals.  
%
%When drawing Mandarin Chinese into the discussion, empirical data appears to not perfectly align with either of them. 
%There are corresponding puzzles that need to be solved for both theories. We will begin with data that pose puzzles for the DP approach. 
In this section, we summarize the puzzles that arise when applying these theories to CNCs.


\subsection{Challenges for the standard DP approach}\label{sec:con-dp:deng}

As it is well known (see \citealt{Chierchia98a}, a.o.), Mandarin Chinese lacks articles and NCs can appear in argument positions with or without DEM or CLF(P).%
%
\footnote{At this point of our argument, we are using CLFP as a mere descriptive convenience to refer to NUM-CLF sequences. We will see in \figref{fig:tree:ambig:deng} and \sectref{sec:solutionpuzzles:deng}, there is good evidence for a \isi{left-branching} structure where CLF and NUM build a constituent headed by CLF -- i.e.\ a CLFP (\citealt{Bale2019a, Her&Tsai20, Wkagiel2020a, Wkagiel2021a}).} %
%
That is, bare Ns, CLFP-N, DEM-N and DEM-CLFP-N are all grammatical combinations, cf.\ (\ref{ex:argall:deng}). In other words, DEM and CLF, traditionally analyzed as D-heads in DP analyses, %\todoMyP{sources?}
are not obligatory in CNCs. The optionality of D is a problem for the DP approach, which requires D to project a DP.\footnote{Optionality is even more problematic in theories like \citet{Cheng&Sybesma99a} and \citet{Cinque05b, Cinque23a}, which decompose D into multiple heads (NUM$^{0}$, DEM$^{0}$, CLF$^{0}$). Either all of them would have to have phonetically empty variants (which arguably generates unwanted readings for cases like \ref{ex:barenp:deng}), or the \textsc{comps} lists of each of them would have to be disjunctively specified: e.g.\ DEM$^{0}$ can select a ModP, or a CLFP, or an NP, etc.} This leads to the first puzzle: an adequate analysis of CNCs needs to make both, bare and complex NCs, selectable by a verb.


\settowidth\jamwidth{(DEM-CLFP-N)I} 
\ea \label{ex:argall:deng}	
\ea \label{ex:barenp:deng}	
\gll wo mai-le \obj{shu}.\\
1.\textsc{sg} buy-\textsc{pfv} book\\ 
\jambox{(bare NP)}
\glt `I bought \{a book / the book / $\emptyset$ books / the books\}.'

\ex 
\gll wo mai-le \obj{san} \obj{ben} \obj{shu}.\\
1.\textsc{sg} buy-\textsc{pfv}  three \textsc{clf} book\\
\jambox{(CLFP-N)}
\glt `I bought three books.' 

\ex  
\gll wo mai-le \obj{na} \obj{shu}.\\
1.\textsc{sg} buy-\textsc{pfv} \textsc{dem} book\\
\jambox{(DEM-N)}
\glt `I bought that book.' \\

\ex
\gll wo mai-le \obj{na} \obj{san} \obj{ben} \obj{shu}.\\
1.\textsc{sg} buy-\textsc{pfv} \textsc{dem} three \textsc{clf} book\\
\jambox{(DEM-CLFP-N)}
\glt `I bought those three books.' 

\z 
\z 


%%% EARLIER
%In addition to these candidates for head D, multiple positions of MODs also challenge the DP analysis.%
%%
%\footnote{In this paper, we limit ourselves to phrasal MODs (see \citealp{Paul05a,Sun15a,XuZ18a}, i.a.).  We assume that non-\obj{de}-\isi{marked} MODs like \obj{bai-zhi} `white paper' are introduced morphologically.} %
%%
%MODs can have flexible positions within NCs (\ref{ex:allposition:deng}).  Comparing (\ref{ex:position1:deng}) and (\ref{ex:position2:deng}), we can see that CLFPs and MODs can be ordered freely without affecting the truth condition of the NC, which already predicts that the syntactic status of the CLFP is similar to a MOD, i.e.\ it looks more like an adjunct than a D-element. Even more problematic is the fact that MODs can appear before  \textsc{dem} (\ref{ex:position3:deng}). In a standard DP approach, the properties of the noun are not accessible after NP is removed from \textsc{comps}. To enable the adjective \obj{da de} `big' to modify the N \obj{shu} `book' locally, the information of nominal head needs to be projected, which is the second puzzle of Mandarin Chinese NCs for DP approach.  

%% MyP:
Furthermore, the various possible positions for MODs%
%
\footnote{\label{fn1:deng}In this paper, we limit ourselves to phrasal \obj{de}-marked MODs like \obj{bai de} `white \obj{de}' in \obj{bai} \obj{de} \obj{zhi} `white paper' (\citealp[cf.][]{Paul05a,Sun15a,XuZ18a}, i.a.).  
In the interest of simplifying the exposition, we mostly refer to adjectives in this paper, but the \obj{de}-marker can attach to different types of nominal \isi{modifiers}, such as relative clauses, possessives, PPs, etc. 
Furthermore, we assume that non-\obj{de}-marked MODs, e.g.\ \obj{bai} in \obj{bai-zhi} `white paper' or `blank paper' , are introduced morphologically through a compounding rule that forms a new lexeme out of two, being, therefore, outside the scope of our investigation.} %
%
within CNCs represent a further challenge for the DP analysis, cf.\ (\ref{ex:allposition:deng}). 
%%
For instance (\ref{ex:position1:deng}) and (\ref{ex:position2:deng}) show that the order of CLFPs and MODs can be reversed without affecting the meaning of the NC. This suggests that the syntactic relation of the CLFP with respect to\ N is similar to MODs, i.e.\  more like an adjunct than a D head. 
%
Even more challenging is the possibility to realize MODs preceding a DEM, cf.\ (\ref{ex:position3:deng}). 
% 
In a standard DP approach, N is not accessible after NP is removed from the \textsc{comps} list of D;
but to enable \obj{da de} `big' to modify the N \obj{shu} `book' locally in (\ref{ex:position3:deng}) the information of N needs to be projected above DEM. This leads us to the second puzzle: an adequate analysis of CNCs needs to account for the different attested positions of MODs.   


%\begin{multicols}{2}

\ea \label{ex:allposition:deng}
\ea[]{
	\gll %wo mai-le 
	na san ben  \emph{da} \emph{de} shu\\
	%	1.\textsc{sg} buy-\textsc{pfv} 
	\textsc{dem}  three \textsc{clf} \nix{big} \nix{\textsc{de}}  \nix{book}\\
		\glt  `those three big books' 
%	\jambox{[\textsc{dem} Num \textsc{cl} Mod N]}	
	%	\glt  `these three big books' 
	%	`I bought three big boxes of books about linguistics' 
	\label{ex:position1:deng}
}


%\columnbreak


\ex[]{
	\gll %wo mai-le 
	na \emph{da} \emph{de} san ben shu\\
	%	1.\textsc{sg} buy-\textsc{pfv} 
	\textsc{dem}  \nix{big} \nix{\textsc{de}} three \textsc{clf}  \nix{book}\\
%	\jambox{[\textsc{dem} Mod Num \textsc{cl} N]}	
		\glt  `those three big books' 
	%	\glt  `these three big books' 
	%	`I bought three big boxes of books about linguistics' 
	\label{ex:position2:deng}
}

\ex[]{
	\gll %wo mai-le 
	\emph{da} \emph{de} na san ben shu\\
	%	1.\textsc{sg} buy-\textsc{pfv} 
	\nix{big} \nix{\textsc{de}}  \textsc{dem}  three \textsc{clf}  \nix{book}\\
	\glt  `those three big books' 
	%	`I bought three big boxes of books about linguistics' 
	\label{ex:position3:deng}
}
\z 
\z 

%\end{multicols}

\noindent Cases like (\ref{ex:position3:deng}) also seem to be at odds with Greenberg's  (\citeyear{Greenberg63c}) Universal 20, which says that, in pre-N position, DEM has to precede ADJ. DP theories like Cinque's (\citeyear{Cinque05b}), which hardwire Universal 20 in the form of a rigid  sequence of functional heads, have to say that MOD can only appear before DEM if it undergoes movement. However, we will see that this generates unwanted readings.

%% PREVIOUS
%Mandarin Chinese CLFs can be subcategorized into sortal and measure \isi{classifiers}  (CLF\down{s}, CLF\down{m}) depending on their lexical contribution (\citealt{Cheng&Sybesma99a,Her12a,Her12b,Li13a} a.o.).  CLF\down{m} have extra lexical meaning that can be modified, CLF\down{s} do not allow modification, as  (\ref{ex:34con:deng}), adapted from \citet[7]{Zhang11a}, shows.%
%%
%\footnote{In \isi{HPSG} terms, we can think of CLF\down{m} as introducing their own \textsc{index} (which can appear as an argument to elementary predications introduced by MODs), while CLF\down{s} simply inherit the \textsc{index} of their corresponding Ns, akin to the analysis of parasitic heads in \citet{Levine10a}.}%
%%

%% MYP
A further challenge arises from the behavior of CLF.  As proposed in the literature (\citealt{Cheng&Sybesma99a,Her12a,Her12b,Li13a}, a.o.), the class of CLF can be further subcategorized into sortal and measure CLF  (CLF\down{s}, CLF\down{m}), where CLF\down{m} has extra lexical meaning (e.g.\ `box' in \ref{ex:moddemnumclmn2:deng}) allowing modification, and CLF\down{s} is purely functional, hence not allowing modification, as shown with  (\ref{ex:moddemnumclsn3:deng}--\ref{ex:moddemnumclmn2:deng}),  
%(\ref{ex:34con:deng}), 
adapted from \citet[7]{Zhang11a}.%
%
\footnote{In \isi{HPSG} terms, we can say that a CLF\down{m} is able to introduce its own \textsc{index}, which can appear as an argument to elementary predications introduced by MODs; while a CLF\down{s} always inherits the \textsc{index} of its corresponding N, akin to the analysis of parasitic heads in \citet{Levine10a}; see also \citet{Ben&Sie04} for a similar approach to Japanese \isi{classifiers}.} %
%
%%NEW MYP
For instance, in (\ref{ex:moddemnumclsn3:deng}), since `triangular' and `square' both have to modify `chips',  the intended reading runs into a contradiction -- the chips are claimed to be triangular and square at the same time. In contrast, with a CLF\down{m} in (\ref{ex:moddemnumclmn2:deng}), one \isi{modifier} can take scope over CLF\down{m} --  yielding `square box', and the other over the noun, yielding `triangular chips'.



%\ea \label{ex:34con:deng}
	\ea[\#]{
	\gll %wo mai-le {[\mins} 
	\obj{sanjiaoxing} \obj{de} na fangfangzhengzheng de \obj{san} \obj{pian} shupian\\
	%	1.\textsc{sg} buy-\textsc{pfv} {} 
	triangular \textsc{de} \textsc{dem} square \textsc{de} three \textsc{clf}\down{s} chips\\
	\glt Intended: `those triangular square chips'
	\label{ex:moddemnumclsn3:deng}  
}

\ex[ ]{
	\gll %wo mai-le {[\mins} 
	\obj{sanjiaoxing} \obj{de} na fangfangzhengzheng de \obj{san} \obj{xiang} shupian\\
	%	1.\textsc{sg} buy-\textsc{pfv} {} 
	triangular  \textsc{de} \textsc{dem} square \textsc{de} three \textsc{clf}\down{m $\approx$ `box'}  chips\\
	\glt `those three square boxes of triangular chips' \\ %or \\
	%	`three boxes of these big books.' or\\
	%	`these three boxes of big books.'\\
	%	Not possible: `three big boxes of these books.'
	\label{ex:moddemnumclmn2:deng}  
}

%	\z 
\z 	
%\z 
%\hfill (adapted from \citealt[ex.(8a)]{Zhang11a})\\

\noindent The modifiability of (some) \isi{classifiers} makes them more similar to Ns than Ds, blurring the distinction between functional and lexical categories.  
%
It also implies that the scope of MODs preceding CLF\down{m} can be both CLF or N. %
%% PREVIOUS
%For pragmatic reasons,  the only thing in (\ref{ex:ambig1:deng}) that can be delicious are cherries. Thus \obj{haochi de} `delicious' can only have the scope over the N \obj{yingtao} `cherry'. In (\ref{ex:ambig2:deng}), given that \obj{yingtao} `cherry' is required to be small, \textit{dada de} `big' must be appilied to \textit{xiang} `box'. In cases like (\ref{ex:ambig3:deng}), MODs preceding \textsc{clf}s can have both readings, leading to ambiguity. 
%%
%% MyP
This is shown in (\ref{ex:ambig3:deng}), where the MOD preceding CLF\down{m} leads to an ambiguity.

	
%%\ea	\label{ex:ambig:deng}
%\ea \label{ex:ambig1:deng}
%\gll haochi de yi xiang yingtao\\
%delicious \textsc{de} one \textsc{cl}\down{m $\approx$ `box'} cherry\\
%\glt `a box of delicious cherries' 
%
%%MyP
%%However, it is also observed that the scope of pre-Num \isi{modifiers} can include \isi{classifiers}: %in some cases:
%\settowidth\jamwidth{zhan2222} 
%\ex \label{ex:ambig2:deng}
%\gll  dada de yi xiang xiao yingtao \\
%big \textsc{de} one \textsc{cl}\down{m $\approx$ `box'} small cherry\\ \jambox{(adapted from \citealt[7]{Zhang11a})}
%\glt `a big box of small cherries' 

\ea \label{ex:ambig3:deng}
\gll  \obj{haokan} \obj{de} yi xiang shu\\
nice \textsc{de} one \textsc{clf}\down{m $\approx$ `box'} book\\
\glt `a nice box of books' or `a box of nice books'
\z 
%\z 


%%% PREVIOUS
%Assuming \isi{locality}, these facts make it necessary to posit two positions for MODs preceding CLF\down{m}: one inside the CLFP and another to its left, as in (\ref{tree:ambig:deng}).  This kind of structure is needed to predict why is (\ref{ex:ambiggra:deng}) acceptable but (\ref{ex:ambigungra:deng}) is not. However, (\ref{tree:ambig:deng}) is  problematic for DP approaches because if D (CLF) is the head, the most peripheral adjective (MOD\down{1}) should not be able to locally modify the NP.  

%% MyP
\noindent  Assuming \isi{locality} \citep[cf.][8]{MyP&Mueller21a}, these facts suggest two possible attachment positions for MODs preceding CLF\down{m}: (i) adjunction inside CLFP and (ii) adjunction to the N projection, cf.\ \figref{fig:tree:ambig:deng}.  This structure predicts the \isi{acceptability} of (\ref{ex:ambiggra:deng}) ruling out (\ref{ex:ambigungra:deng}) (with the intended reading).  However, (\ref{ex:ambiggra:deng}) and (\ref{ex:ambigungra:deng}), adapted from \citet[7]{Zhang11a}, pose a problem for the DP approach since MOD\down{1} in \figref{fig:tree:ambig:deng} should not be able to locally modify N. In other words, for a right-branching DP approach, any MOD preceding CLF (i.e. the D) should only have scope over CLF. %
%
%\footnote{
	
One could try to derive pre-CLF MODs with scope over N (e.g.\  \textit{chao haochi de} in \ref{ex:ambiggra:deng}) by moving them to the left-periphery over the CLF and reconstructing their interpretation at their original N-adjacent position. However,  if this kind of movement/\isi{reconstruction} is allowed for MODs, it becomes a mystery why  MODs with CLF scope cannot be at the leftmost position inside NCs. E.g.\  (\ref{ex:ambigungra:deng}) could be licensed by  generating and interpreting  \textit{dada de} in a CLF-adjacent position and subsequently moving it to the left edge of the NC. Thus, in the absence of a restrictive theory of which MODs can move and \isi{reconstruct}, this movement alternative would overgenerate.%} %\todoMyP{We can add: without assuming movement, and assuming movement nothing prevents (19) to be licensed} % to rule out the MOD of a CLF to be moved to a higher NC-initial position. This should not be possible, because the leftmost MODs in a NC are always take scope over N.


\settowidth\jamwidth{(DEM-CLFP-N)I} 
\ea[]{
	\gll \emph{chao} \emph{haochi} \textit{de} dada de yi wan chao xiao de yingtao \\
	very delicious \textsc{de} big \textsc{de} one \textsc{clf}\down{m $\approx$ `bowl'} very small \textsc{de} cherry\\
	\glt Intended: `a big bowl of very delicious small cherries' } \label{ex:ambiggra:deng}
%\z
%Our proposal accounts for the Mod ambiguity in Mandarin Chinese CNCs.
\ex[\#]{
	\gll  \textit{dada} \textit{de} chao haochi de yi wan chao xiao de yingtao \\
	big \textsc{de} very delicious \textsc{de} one \textsc{clf}\down{m $\approx$ `bowl'} very small \textsc{de} cherry\\
	\glt Intended: `a big bowl of very delicious small cherries' } \label{ex:ambigungra:deng}
	
\z

\begin{figure}
  \centering
  \begin{forest}
	%	bottom word,
	[NP
	[MOD\down{1}]
	[NP, draw=black
	[CLFP
	[MOD\down{2}]
	[CLFP, draw=black
	[NUM]
	[CLF]
	]
	]
	[N]
	]
	]
  \end{forest}
  \caption{ Possible positions for modifiers}
  \label{fig:tree:ambig:deng}
\end{figure}



%%% PREVIOUS
%In general, whenever CLF\down{m}  can be modified by a MOD, then specification by \textsc{dem} is also possible. This CLF\down{m}P-internal \textsc{dem} position is further supported by the observation that \textsc{dem}s can act as semantic functors over CLF\down{m} meanings. Suppose three identical books are placed respectively in a box and in a basket. In the case of contrast, the \isi{classifier} is emphasized: \obj{na XIANG shu, bushi na LAN shu} `that box book not that basket book'. Similarly, stress can be placed on N to distinguish between the contents of the identical boxes. \textsc{dem}s appearing before NUM-CLF\down{m} are also structurally ambiguous, as (\ref{pre-dem:deng}) shows.

%% MyP
Furthermore, it is not only the case that CLF\down{m} can be modified with CLF$_{\text{m}}$; specification by DEM is also possible in these cases. 
This CLF\down{m}P-internal DEM is supported by the observation that DEMs can act as semantic functors over CLF\down{m} meanings. 
For instance, assume that three books are placed respectively in a box and in a basket. In case of contrast, the CLF can be contrastively emphasized: \obj{na XIANG shu, haishi na LAN shu} `that BOX of books or that BASKET of books'.  
%Similarly, stress can be placed on N to distinguish between the contents of the identical boxes. 
Hence, DEMs appearing before CLF\down{m}P are also structurally ambiguous, cf.\ (\ref{pre-dem:deng}).

\ea
{\gll \obj{na} san xiang shu\\
 \textsc{dem} three \textsc{clf}\down{m $\approx$ `box'} book\\
\glt `those three boxes of books' or `three boxes of those books' \label{pre-dem:deng}
}
\z

%%% PREVIOUS
%Therefore, our analysis needs to account for the modifiability of CLF\down{m} as well as the ambiguous scope of  MODs and \textsc{dem}s preceding CLFPs.

%%MyP
Therefore, the next puzzle: an analysis of CNCs needs to account for the modifiability of CLF\down{m} (vs.\ CLF\down{s}) and the ambiguity of MODs and DEMs preceding CLFPs.



\subsection{Challenges for the standard NP approach}


%%% PREVIOUS
%Compared to the standard DP approach, standard NP (\isi{head-specifier}) approach seems to be less problematic for our purpose, but there are still puzzles that need to be considered. Referring to (\ref{pre-n:deng}),  we can see that there are at least two candidates for specifiers in Mandarin Chinese NCs: \textsc{dem}s and CLFs. As with standard specifiers, neither \textsc{dem} nor CLF can be iterated, as (\ref{ex:iteration:deng}a-b) show.\footnote{We limit ourselves in the paper to sortal and measure CLFs excluding kind CLFs, see \citet{Liao&Wang11a} for further information about iterated CLF phrases. However, the fact that some CLFs can be iterated also suggests that the standard NP analysis is problematic for Mandarin Chinese.} Crucially, the impossibility of iterating \textsc{dem}s applies even when the \textsc{dem}s have different scopes, with one applying to N and the other to CLF\down{m}, as in (\ref{ex:iteration:deng}b). This is already problematic for standard DP and NP approaches, because it appears that any left-peripheral \textsc{dem} which scopes over N would need to to look into the CLFP that it c-commands to see if there is any other \textsc{dem} contained within it -- a clear violation of \isi{locality}. 

%% MyP
Compared to the DP analysis, the NP (\isi{head-specifier}) approach faces less difficulties with respect to\ the data, but there are still issues to be considered. 
%
For instance, taking (\ref{ex:modnmodclnumclmn:deng}) into account, there are at least two candidates for specifiers in CNCs: DEM and CLFP, which have to be realized in a specific order DEM-CLFP, cf.\ (\ref{ex:modnmodclnumclmn1:deng}) vs.\ (\ref{ex:modnmodclnumclmn2:deng}).%
%
\footnote{Examples like (\ref{ex:kindcl:deng}) are also possible. But in this case, CLF-DEM-N can only be interpreted as \obj{this kind of book} (singular). Here, 
we limit ourselves to CLF\down{s} and CLF\down{m}, excluding \emph{kind} CLF, cf.\ \citet{Liao&Wang11a} for  iteration of CLFPs.	
Beyond these cases, DEM always precedes CLF. 
	\ea[]{
		\gll %wo mai-le 
		san ben zhe (zhong) shu		\label{ex:kindcl:deng}\\
		three \textsc{clf} \textsc{dem} ~\textsc{clf} book\\
		\glt  `three books of this kind' 
	}
	\z 
} % 
%
Similar to ``standard'' specifiers, neither DEM nor CLFP can be iterated, as  (\ref{ex:iteration1:deng}--\ref{ex:iteration2:deng}) show. %
%
%\footnote{We limit ourselves to CLF\down{s} and CLF\down{m} excluding \emph{kind} CLF, see \citet{Liao&Wang11a} for further information on iteration of CLFPs. However, the fact that some CLFs can be iterated is problematic for a standard \isi{head-specifier} analysis of CNCs.} %
%
Crucially, the impossibility of iterating DEMs applies even when DEMs have different scopes, with one applying to N and the other to CLF\down{m}, as in (\ref{ex:iteration2:deng}). 
%
This is problematic for DP and NP approaches, since any left-peripheral DEM which scopes over N would need to look into the CLFP that it c-commands to see whether there is any other DEM contained in it -- a clear violation of \isi{locality}. 


\ea\label{ex:modnmodclnumclmn:deng}
\ea[]{
	\gll %wo mai-le 
	\obj{na}  \obj{san} \obj{ben} guanyu yuyanxue de shu\\
	%	1.\textsc{sg} buy-\textsc{pfv} 
	\textsc{dem} three \textsc{clf}  \nix{about} \nix{linguistics} \nix{\textsc{de}} \nix{book}\\
	\glt  `those three  books about linguistics' 
	%	`I bought three big boxes of books about linguistics' 
}\label{ex:modnmodclnumclmn1:deng}

\ex[*]{
	\gll %wo mai-le 
	\obj{san} \obj{ben} \obj{na} guanyu yuyanxue de shu\\
	%	1.\textsc{sg} buy-\textsc{pfv} 
	three \textsc{clf} \textsc{dem}  \nix{about} \nix{linguistics} \nix{\textsc{de}} \nix{book}\\
	\glt  `those three books about linguistics' 
	%	`I bought three big boxes of books about linguistics' 
}\label{ex:modnmodclnumclmn2:deng}
\z
%\z 


\ex  \label{ex:iteration:deng}	
\ea[*]{
	\gll %wo mai-le 
	\nix{yi} \obj{xiang} \nix{yi} \obj{ben} \nix{shu}\\
	%	1.\textsc{sg} buy-\textsc{pfv} 
	\nix{one} \nix{\textsc{clf}\down{m $\approx$ `box'}} \nix{one} \textsc{cl} \nix{book}\\
	%	\glt  Intended: `those three boxes of these two books' 
	%	`I bought three big boxes of books about linguistics' 
	%	\label{ex:demnumcldemnumcln:deng}
}\label{ex:iteration1:deng}	

\ex[*]{
	\gll %wo mai-le 
	[\down{NP}\! \obj{na}  [\down{CLFP}\! \obj{zhe} liang xiang] shu]\\
	%	1.\textsc{sg} buy-\textsc{pfv} 
	{} \textsc{dem} {} \textsc{dem} \nix{two} \textsc{clf}\down{m $\approx$ `box'} \nix{book}\\
	\glt  Intended: `these two boxes of those books' 
	%	`I bought three big boxes of books about linguistics' 
	%	\label{ex:demnumcldemnumcln:deng}
}\label{ex:iteration2:deng}	
%\hfill \size{(adapted from \citealt[ex.(8a)]{Zhang11a})}\\
\z 
\z 


%A specific problem for the standard NP approach is, on the one hand, the combination of two different specifier-like elements \textsc{dem} and CLF, as  (\ref{ex:modnmodclnumclmn1:deng}) shows.  That is, more than one ``specifier'' can be selected by N, contrary to what the version of the \isi{head-specifier} schema in (\ref{head-spec-rule:deng}) requires. On the other hand, as shown in (\ref{ex:barenp:deng}), it is also possible to have neither \textsc{dem} nor CLF, since bare NCs are grammatical in argument position. This suggests that \textsc{dem}s and CLFs diverge from the properties as specifiers and are behaving more like MODs.\footnote{Note that a restriction on the iteration of \textsc{dem}s and CLFs as in (\ref{ex:iteration:deng}) also applies to MODs of the same type (e.g.\ *\textit{the book about linguistics about psychology}) and is not specific to specifiers.}  If \textsc{dem}s and CLFs are treated as MODs, some information about the presence of these pre-N elements needs to be projected \emph{locally} to higher projections in order to rule out multiple \textsc{dem}s (\ref{ex:iteration2:deng}) or combinations like CLF-\textsc{dem} (\ref{ex:modnmodclnumclmn2:deng}). 

A specific problem for the standard NP approach concerns the valency of the \textsc{spr} feature. On the one hand, it is possible to combine two different specifiers (DEM and CLF) with N, cf.\ (\ref{ex:modnmodclnumclmn1:deng}). That is, more than one ``specifier'' can be selected by N, contrary to (\ref{head-spec-rule:deng}). On the other hand it is also possible for N to not select either DEM nor CLF (\ref{ex:argall:deng}a), or to select only one of them (\ref{ex:argall:deng}b--\ref{ex:argall:deng}c). That is, the standard NP approach would require either disjunctive lists as the value of \textsc{spr} (with DEM, with DEM and CLF, with CLF, without DEM and CLF) or it would require lexical rules adding/removing \textsc{spr} elements. This suggests that DEMs and CLFs diverge from the properties of specifiers and behave more like MODs.\footnote{Note that a restriction on the iteration of DEMs and CLFs as in (\ref{ex:iteration:deng}) also applies to MODs of the same type (e.g.\ *\textit{the book about linguistics about psychology}) and is not specific to specifiers.}  If DEMs and CLFs are treated as MODs, some information about the presence of these pre-N elements needs to be projected locally to higher projections in order to rule out multiple DEMs (\ref{ex:iteration2:deng}) or combinations like CLF-DEM (\ref{ex:modnmodclnumclmn2:deng}), cf. \sectref{sec:solutionpuzzles:deng}. 


That brings us to the puzzle related to the NP approach: 
an adequate analysis of CNCs needs to account, on the one hand, for the similar behavior between DEMs, CLFs, and MODs, and, on the other hand, it needs to explain the different possible syntactic configurations with and without DEM and CLF. 

%What is the function of \textsc{dem}s and CLFs that exhibit characteristics of both specifiers and \isi{modifiers}? How does the combination between such prenominal elements  with N work in Mandarin Chinese?





%%%%%%%%%%%%%%%
\subsection{The singular demonstrative paradox}\label{sec:sing:deng}


There is a further puzzle for DP and NP approaches that is related to the number interpretation of NCs.  
%
When an overt CLFP is present, the number interpretation of DEM-CLFP-N comes from the cardinal relation encoded by the NUM inside the CLFP, as in (\ref{ex:cardinal:deng}). %the cardinal has the value \obj{three}, thus the NC is plural. 
 %Following this, the number of CNCs with the combination of \textsc{dem} and CLFP is logically determined by the value of the cardinal.  In (\ref{ex:cardinal:deng}), the cardinal has the value \obj{three}, thus the NC is plural. 
 %
Without an overt CLFP, bare Ns have number-neutral interpretations, i.e.\ both singular or plural readings are possible (\ref{ex:bareboth:deng}). 
%
A combination of DEM-MOD-N in (\ref{ex:demred:deng}) is also underspecified for number, since neither DEM nor MOD express number information. %anything about the numbe the value of the number is underspecified since the adjective does not change anything.  
%
However, the paradox emerges when DEM is directly combined with a bare N, as in (\ref{ex:dem-n:deng}). In that case, the NC can only be interpreted as singular. This raises the question: what is it about the presence  of MOD in (\ref{ex:demred:deng}) that allows the NC to retain the number-neutral interpretation lexically associated with bare Ns? Why should the absence of MOD have any influence over the absence of a plural interpretation in (\ref{ex:dem-n:deng}), given that MOD does not encode plurality, as the singular reading of (\ref{ex:demred:deng})  shows?% given that both MODs and \textsc{dem}s are in principle compatible with plural, as (\ref{ex:cardinal:deng}) and (\ref{ex:demred:deng}) show?


\ea \label{ex:sing-dem-para:deng}

	\ea { 
	\gll wo mai-le na san ben shu.\\
	1.\textsc{sg} buy-\textsc{pfv} \textsc{dem} three \textsc{clf} book\\
	\glt `I bought those three books.'  \label{ex:cardinal:deng}
}	

	\ex {
	\gll wo mai-le shu. \label{ex:bareboth:deng}\\
	1.\textsc{sg} buy-\textsc{pfv} book\\
	\glt `I bought \{a book / the book / $\emptyset$ books / the books\}.' }

	\ex { 
	\gll wo mai-le na da  de shu.\\
	1.\textsc{sg} buy-\textsc{pfv} \textsc{dem} big \textsc{de} book\\
	\glt `I bought \{that big book / those big books\}.'\label{ex:demred:deng}
}
	\ex { 
	\gll wo mai-le  na shu. \label{ex:dem-n:deng}\\
	1.\textsc{sg} buy-\textsc{pfv} \textsc{dem} book\\
	\glt `I bought that book.'
}
\z\z 


\section{The Head-Functor Approach in Mandarin Chinese}\label{sec:h-f:deng}


In this section, we introduce the Head Functor Approach (HFA) to NCs (\citealt{Allegranza98a,Allegranza07a,VanEynde06a,VanEynde20a,VanEynde21a}) and  propose a solution to the puzzles outlined in \sectref{sec:puz:deng} based on this theory.  


\subsection{The Head-Functor Approach}



The HFA can be defined by its rejection of two distinctions that are basic to NP and DP theories: the distinction between lexical and functional categories and, most importantly, the distinction between specifiers and \isi{modifiers}.% within the nominal domain.


%Departing from  standard NP and DP approaches, the HFA eliminates the distinction between lexical and functional categories and, most importantly, the distinction between specifiers and \isi{modifiers} within the nominal domain. %in order to deal with the syntactic similarities between these two categories, for instance: 


The first of these contentions is motivated by the observation that the notion of a determiner does not correspond to a morphosyntactically uniform functional category. Expressions that typically fall under the class of determiners share characteristics with lexical parts of speech, such as N and ADJ.  An example of the former  is CLF in Mandarin Chinese, which, as we saw in (\ref{ex:moddemnumclmn2:deng}), 
can be modified like an ordinary N. The structural parallelism between functional and lexical categories (e.g.\ CLFP and NP in Mandarin Chinese) is especially problematic for the  DP theory, whose defining feature is precisely the idea that NCs are the combinatorial yield of selectional properties of a dedicated functional element (D), which takes a lexical N as its complement. If such Ds do not exist as a category distinct from N, then neither do DPs, by definition (\citealt[157]{VanEynde06a}).\footnote{Another example of N-like Ds are adnominal pronoun constructions like \textit{du Idiot} in (\ref{ger-ex:deng}), where the prenominal position is filled by a pronoun. Conversely, definite determiners in German (\textit{der}, \textit{die}, \textit{das}) can function as pronouns without an accompanying N (\citealt{Postal69a, Wiltschko98a}). This can also be taken to suggest an N status, though something has to be said to explain the contrast in the dative plural (\textit{den} vs. \textit{denen}), see \citet[155]{Wiltschko98a}. %Something similar can be said  about English  interrogative Ds  (e.g.\ \textit{What (book) do you like?}). 
Other putative determiners seem to function more like ADJs, like possessives in Mandarin Chinese. %In Portuguese, some determiners can occupy both pre- and post-nominal positions (\ref{bp-ds1:deng})--(\ref{bp-ds2:deng}), exactly as ADJs do (\ref{bp-ds3:deng}).%(Many Ds also inflect for gender like the majority ADJs in Portuguese.) 
%\ea\label{bp-ds:deng}
%	\ea{\gll meu livro  / livro meu \\
%	my.\textsc{m.sg} book.\textsc{m} {} book.\textsc{m} my.\textsc{m.sg} \\} \label{bp-ds1:deng}
%%	\ex{\gll este livro que comprei / livro este que comprei\\
%%	that.\textsc{m.sg} book.\textsc{m} that bought-\textsc{1.sg} {} book.\textsc{m} that.\textsc{m.sg}  that bought-\textsc{1.sg} \\ }
%	\ex{\gll nenhum livro / livro nenhum\\
%	no.\textsc{m.sg} book.\textsc{m} {} book.\textsc{m} no.\textsc{m.sg} \\ } \label{bp-ds2:deng}
%%	\ex{\gll qualquer livro / livro qualquer\\
%%	any.\textsc{sg} book.\textsc{m} {} book.\textsc{m} any.\textsc{sg} \\ }
%	\ex{\gll bom livro  / livro bom \\
%	good.\textsc{m.sg} book.\textsc{m} {} book.\textsc{m} good.\textsc{m.sg} \\} \label{bp-ds3:deng}
%	\z
%\z
}%which hardwires the idea that prenominal elements belong to functional category D distinct from N as part of the combinatorics of NCs. In a world without Ds, DPs  cannot exist either%\footnote{The categorial heterogeneity of Ds can also be seen in inflectional paradigms (\citealt{VanEynde06a}). In Portuguese, most of the Ds inflect for gender in the same way as ordinary  ADJs (\ref{bp-ds:deng}a), but others  (\ref{bp-ds:deng}b), like common gender Ns, do not.

%\ea\label{bp-ds:deng}
%	\ea{\gll \{a / minha / esta / alguma\} \{mesa / *livro\}\\
%	the.{f} my.\textsc{f}  {} that.\textsc{f} {} some.\textsc{f}  table.\textsc{f} {} book.\textsc{m} \\}
%	\ex{\gll \{que / qual / tal\} \{mesa / livro\}\\
%	what {} which {} such  table.\textsc{f} { } book.\textsc{m} \\}
%	%\ex{\gll \{a / essa / minha / alguma\} mesa\\
%	%the.\textsc{fem} {} this.\textsc{fem} {} my.\textsc{fem} {} some.\textsc{fem} table \\}
%	\z
%\z
%
%The gendered Ds share other properties with ADJs:  e.g.\ they can be post-nominal, at least in some contexts, where they may also acquire different meanings, as (\ref{bp-inv:deng}c) shows. 
%
%\ea\label{bp-inv:deng}
%	\ea{\gll mesa minha\\
%	table.\textsc{f}  my.\textsc{f} \\
%	\glt `my table' }
%	\ex{\gll mesa esta que ganhei \\
%	table.\textsc{f} that.\textsc{f} which  won-\textsc{1.sg} \\
%	\glt `that table I won' }
%	\ex{\gll mesa alguma\\
%	table.\textsc{f} some.\textsc{f} \\
%	\glt `no table' }
%	\z
%\z } 

%\ea[]{\gll Hannah hat den (Mann) gesehen. \label{d-pro:deng} \\
%Hannah has \textsc{d.m.sg.acc} man seen \\
%\glt `Hannah saw him/the man.' 
%}
%\z


%
%\ea I like that (book). \label{that:deng}
%\z
%
%\ea What (book) do you like? \label{what:deng}
%\z
%

The elimination of the specifier-\isi{modifier} distinction, in turn, is motivated by the fact that bearers of these grammatical relations share  more syntactic properties than either DP or NP theories typically ascribe to them. First, both specifiers and \isi{modifiers} can occupy prenominal positions. Second, both add information to the NC that must be projected  for the purposes of semantic interpretation and selection by other heads.\footnote{In the case of specifiers, this can be information about the cardinality of the plurality denoted by N (e.g.\ \textit{two books}) or about definiteness (e.g.\ \{\textit{the} / \textit{a}\}  \textit{book}). In the case of MODs, the information is more diverse. It can be an expression of  size  (e.g.\ \textit{big books}), color  (e.g.\ \textit{red books}), quality (e.g.\ \textit{good books}), as well as cardinality (e.g.\ \{\textit{single} / \textit{unique} / \textit{numerous} / \textit{various}\} \textit{books}).} For Mandarin Chinese, as we saw in (\ref{ex:iteration:deng}), the information about whether a head has previously combined with a DEM or a CLF at some point is relevant for selection because DEM/CLF cannot be iterated. Third, specifiers of different types can be stacked, much like \isi{modifiers} (e.g.\ \textit{the \textsc{big} \textsc{red} book}).  We saw this in the case of CLF and DEM in Mandarin Chinese in (\ref{ex:modnmodclnumclmn1:deng}).\footnote{Similar examples may also exist in English, depending on how one analyzes sequences like  \textit{\textsc{all} \textsc{these} \textsc{three} books}. Though \citet{VanEynde06a} assumes the HFA approach is necessary for English, we believe that languages like Mandarin Chinese provide much stronger support for it. In spite of that, we cite some data from English and German to facilitate the exposition.} %The colloquial Brazilian Portuguese examples in  (\ref{stack:deng}) show how definite articles and possessives can be stacked and, moreover, how any of them can be attested as bearing  number inflection. % --  (\ref{stack:deng}b-d) are typical of colloquial varieties. 
%\ea \label{stack:deng}
%	%\ea[]{\gll o meu livro\\
%	%the.\textsc{m} my.\textsc{m} book.\textsc{m} \\}
%	\ea[]{os meus livro\\
%	the.\textsc{m.pl} my.\textsc{m.pl} book.\textsc{m} \\}
%	\ex[]{os meu livro\\
%	the.\textsc{m.pl} my.\textsc{m} book.\textsc{m} \\}
%	\ex[]{o meus livro\\
%	the.\textsc{m} my.\textsc{m.pl} book.\textsc{m} \\}
%	\z
%\z
%The stackability of prenominals is especially problematic for the NP approach, which assumes that the \textsc{spr} of Ns is a list consisting of a unique element.
Fourth, specifiers like \isi{modifiers} can be omitted, as bare singular Ns in Mandarin Chinese (\ref{ex:barenp:deng}) show. This is unexpected if specifiers (or \isi{modifiers})  are treated as heads.
 %-- a property that is unusual for syntactic heads. An example of this are the bare singular Ns in Mandarin Chinese in (\ref{ex:argall:deng}a). 


The projectability of prenominals is especially problematic for the standard NP theory because, after the specifier is combined with the head, only the properties of the latter project. The only residue of the presence of a specifier inside a NC is an empty \textsc{spr} list. Stackability and omissibility are  properties that the classic DP and NP theories both take to be exclusive to \isi{modifiers}. Therefore, the fact that elements taken to be specifiers/Ds exhibit these properties cannot be  expressed under either of these accounts without special stipulations. For instance, in order to derive omissibility (i.e.\ bare Ns), the DP approach would have to appeal to an empty D head, while the NP approach would need to posit an empty specifier or a unary rule that eliminates the element in the \textsc{spr} list of the head.%, making the resulting semantics number-neutral. 

Crucially, the HFA strives to capture all of the properties of specifiers and \isi{modifiers} while preserving \isi{locality} of selection and endocentricity within NCs (\citealt{Chomsky07a,Chomsky&Co19b,Bruening09a,Bruening20a,MyP&Mueller21a}). This leads the HFA to view N as the heads of NCs and prenominal elements previously analyzed as specifiers, \isi{modifiers} or Ds as functors over N. %In the next subsection, we introduce the HFA proper. We then  discuss how it applies in Mandarin Chinese. 


Following  the rejection of the distinction between specifiers and \isi{modifiers}, the selection features \textsc{spr}, \textsc{mod} and \textsc{spec} are replaced by a single feature \textsc{select}. It is through this feature that functors (ADJs, DEMs, CLFPs, possessives, etc.) impose selectional constraints on the \textit{synsem} of their corresponding N-heads. %As a valence feature, \textsc{spr} was supposed to encode constraints that a head imposes on its arguments, \textsc{select}, by contrast, is similar to \textsc{mod} and \textsc{spec} in that it encodes the constraints that a non-head imposes on its head.

%At the core of the HFA lies, therefore, 

 The HFA assumes with previous theories the Head Feature Principle in  (\ref{head-feat-principle:deng}) as a constraint on objects of the sort \val{headed-phrase}. Corresponding to the abandonment of  \textsc{spr}, \textsc{mod} and \textsc{spec}, the schemas for \textit{head-specifier-phrase} (\ref{head-spec-rule:deng}), and \textit{head-adjunct-phrase} (\ref{head-adjunct-schema:deng})  are replaced by the more general Selector Principle in (\ref{selp:deng}) (based on \citealt[164]{VanEynde06a}), which constraints objects of sort \val{head-functor-phrase}. This allows the (non-head) functor to select the head daughter.%
 %
 \footnote{In comparison to \citeauthor{VanEynde06a}'s Selector Principle, we enforce the necessity of the functor's \textsc{comps} list to be empty. This parallels the constraints generally imposed on \textit{head-specifier-phrase} and \textit{head-adjunct-phrase} \citep[cf.][333 \& 335]{Mueller&MyP19a}. This condition is necessary to ensure that CLF (without NUM) cannot be a functor of N, cf.\ (\ref{num-cl:deng}).} %
 %


\ea \label{selp:deng}
%\scalebox{\sizeavm }{
	\val{head-functor-phrase} $\Rightarrow$ \\\avm{
[\type*{headed-phrase}
head-dtr|synsem & \1 \\
\punk{non-hd-dtr} {
	<[synsem|loc|cat
		[head|select & \1\\
		comps & <>
		]
	]>
	}  
]} 
%}
\z 

To register the effects of prenominal elements on N the \textsc{marking} feature proposed by \citet[46]{Pollard&Sag94a} is used. \textsc{marking} values can be used to state selectional constraints of functors or heads over NCs. We propose a \textsc{marking} value \textit{weak} (for bare N and CLF) and \textit{strong}  (for DEM). These values play a similar role to \textsc{spr} lists, however,  as we will see in the next section, \textsc{marking}  gives the HFA more flexibility than the distinction between empty and non-empty \textsc{spr}, since different subsorts of \textit{weak} and  \textit{strong} can be postulated (e.g.\  \textit{n(oun)-marked}) with the purpose of modeling more fine-grained selectional constraints on  NCs. %In the following section, we propose a specific hierarchy of  \textsc{mrk} values that 
E.g. some functor or V might be constrained in such a way that it can only select for NCs that have a specific subsort of \textit{weak} (e.g.\ \textit{n-marked}) as its value.%
%
%\todoMyP{For example a Chinese ``there is'' construction that doesn't allow dem-\isi{marked}?}
%


The projection of \textsc{marking} is governed by the Generalized Marking Principle in (\ref{gmp:deng}), which ensures that the \textsc{marking} values of a head-functor phrase come from its functor \citep[166]{VanEynde06a}. % the functor get projected to its mother. 

\ea \label{gmp:deng}
%\scalebox{\sizeavm }{
\val{head-functor-phrase} $\Rightarrow$ \\
\avm{
[\type*{headed-phrase}
synsem|loc|cat|marking & \1 \\
non-hd-dtr  <[synsem|loc|cat|marking \1]>  
]} 
%}
\z 

Taken collectively, the Head-Feature Principle in  (\ref{head-feat-principle:deng}), the Selector Principle  (\ref{selp:deng}) and the Generalized Marking Principle  (\ref{gmp:deng}) yield the structure depicted in \figref{fig:head-functor-structure:deng}.

\begin{figure}
  \centering
		\begin{forest}
		[\val{head-functor-phrase}\\
		\avm{
			[marking & \1\\
			head & \2 ] $\approx$ NP
		}, name=phrase
		%%%
		[\val{functor}\\
		\avm{
			[select & \3\\
			marking & \1]	$\approx$ \{\hspace{1pt}  \textnormal{MOD} \hspace{1pt} \\\hspace{1pt}  \textnormal{CLFP} \hspace{1pt} \\\hspace{1pt}  \textnormal{\textsc{DEM}} \hspace{1pt} \}
		}, name=functor]
		%%%
		[\val{head}\\
		\avm{
			\3[head & \2\\
%			spr & <\green{\2}>
			] $\approx$ N
		}, name=head]
		]
		\draw[->] (head) to[out=east,in=east] (phrase);  
		\draw[->] (functor) to[out=west,in=west] (phrase);
		\end{forest}	
  \caption{NP analysis (\emph{head-functor-phrase})}
  \label{fig:head-functor-structure:deng}
\end{figure}

The structures licensed by the HFA combine aspects of both NP and DP theories. On the one hand, the HFA is unequivocally an NP approach, since the category D is not assumed and N functions as the head of NCs. On the other hand, however, the HFA departs from most NP analyses  in two crucial respects: (i) it rejects the notion that N imposes selectional requirements on its prenominal dependents; and (ii) a nominal projection can, through its \textsc{marking} value,  encode fine-grained categorial information about its prenominal daughters, providing a local record of its ``combinatorial history'', much like a DP encodes information about D. As David Adger (p.c.) points out to us, the idea that a  phrase may simultaneously inherit categorial properties of its lexical head and its non-head daughter makes the HFA  essentially a version of \citeauthor{Grimshaw91a}'s (\citeyear{Grimshaw91a})  theory of extended projections encoded in the formalism of typed feature structures.


\subsection{Accounting for the Mandarin Chinese puzzles}
\label{sec:solutionpuzzles:deng}

%The features of the HFA are useful in accounting for the Mandarin Chinese data discussed in Section \ref{sec:puz:deng}. In particular, we want to account for the following properties, which are puzzling from the perspective of prior NP and DP theories: 

 In the case of Mandarin Chinese, we want the HFA to account especially for the following properties, which lie at the heart of the puzzles discussed in \sectref{sec:puz:deng}: 
(i) the modifiability of CLF\down{m}; 
(ii) the similar prenominal distribution of CLF, DEM, and MOD;
(iii) the differences between MOD, CLF and DEM with respect to iteration; 
(iv) the optionality of CLF and DEM; %, which suggests that CLFs and \textsc{dem}s are not needed to complete the NCs
(v) the singular DEM paradox.
 
%\begin{itemize}
%	
%	\item the modifiability of CLF\down{m}
%
%	\item the similar prenominal distribution of CLF, \textsc{dem}, and MOD
%	
%	\item the differences between MOD, CLF and \textsc{dem} with respect to iteration 
%	
%	\item the optionality of CLF and \textsc{dem} (i.e.\ bare N arguments)%, which suggests that CLFs and \textsc{dem}s are not needed to complete the NCs
%	
%	%\item the fact that, when a \textsc{dem} appears inside a CLFP, there cannot be another \textsc{dem} attaching to the NP
%
%	\item the singular \textsc{dem} paradox 
%\end{itemize}


Though each of these puzzles taken individually could be solved by adding some stipulation to standard NP or DP approaches, we are not aware of any version of these theories that could address all four of them simultaneously. We will show that  the HFA is essential to provide a unified and coherent solution to all of these problems, while also raising interesting typological hypotheses. 

The first ingredient of our theory is a hierarchy of nominal \textsc{head} values, as depicted in \figref{fig:head-value-hierarchy}.  Due to the different positions and interpretations MODs and DEMs can have inside NCs according to the subtypes of CLF (CLF\down{s} and CLF\down{m}), we assume that N and CLF constitute a natural class of type \textit{noun}, which is divided in different subtypes, with measure \isi{classifiers} (\textit{m-cl-noun}) being at the same time subtypes of \textit{n(ouny)-noun}, like lexical nouns (\textit{lex-noun}), and of \textit{cl(assifier)-noun}, like sortal-\isi{classifiers} (\textit{s-cl-noun}).\footnote{Further evidence for this hierarchy comes from the historical development of \isi{classifiers} out of nouns, see \citet{huang1964cong}. On this picture, we can understand this diachronic development as a grammaticalization of elements of type \textit{lex-noun} as  \textit{m-cl-noun} and, subsequently as \textit{s-cl-noun}.} Prototypical Ns like \textit{shu} `book' have \textsc{head} values of type \textit{lex-noun}. %This process is arguably driven by semantic bleaching -- i.e.\ the loss of relational lexical content.

\begin{figure}
  \centering
\begin{forest}
	[\textit{noun}
	[\textit{cl-noun}
	[\textit{s-cl-noun}]
	[\textit{m-cl-noun}, name=mcln]
	]
	[\textit{n-noun}, name=nn
	[\textit{lex-noun}]
	]
	]
	\draw[black] (mcln.north)--(nn.south);
\end{forest}
  \caption{Hierarchy of nominal \textsc{head} values}
  \label{fig:head-value-hierarchy}
\end{figure}

We posit that \isi{modifiers} (i.e.\ DePs) and DEMs can attach only to elements of type \val{n-noun}. For this to work, we also need to assume a \isi{left-branching} analysis, where NUM and CLF combine to form a CLFP before combining with the head N. This account explains why DePs and DEMs have a similar distribution and, moreover, why measure \isi{classifier} phrases can  be modified by DeP and DEMs just like ordinary lexical Ns.\footnote{As a reviewer notes, the class of \textit{m-cl-noun} is not semantically homogeneous, given that some of its members can have both counting and measuring readings, as in (\ref{ex:c&m:deng}) adapted from \citet[135]{Li13a}:
	
\settowidth\jamwidth{(measure reading) T}  	
\ea \label{ex:c&m:deng}	
\ea \label{ex:coungting:deng}	
\gll wo ling-le \obj{liang} \obj{ping} \obj{hongjiu}.\\
1.\textsc{sg} lift-\textsc{pfv} two \textsc{clf}\down{m $\approx$ `bottle'} red.wine\\ 
\jambox{(counting reading)}
\glt `I carried two bottles of red wine.'

\ex 
\gll ta-de jiuliang shi \obj{liang} \obj{ping} \obj{hongjiu}.\\
3.\textsc{sg}-\textsc{de} drinking.capacity \textsc{be} two \textsc{clf}\down{m $\approx$ `bottle'} red.wine\\
\jambox{(measuring reading)}
\glt `The most he can drink (i.e.\ his `drinking capacity') is two bottles of red wine.'%`His drinking-capacity is two bottles of red wine.' 
\z\z

 \noindent \citet{Li&Rothstein12} and \citet{Li13a} attribute these two readings to different syntactic structures. This is not necessary under our analysis. Making use of the flexible syntax-semantics interface of \isi{HPSG}, we can distinguish the two readings purely at the level of \textsc{content}. In the counting reading, \obj{ping} conveys a nominal semantic relation \textit{bottle-rel}, which is the semantic head of the whole NP, taking the relation expressed by N as an argument and the relation expressed NUM as essentially a \isi{modifier}.  In the measuring reading, \obj{ping} conveys a measure function \textit{bottle-measure-rel}, which relates the \textsc{index} of the head N to the number expressed by NUM (see \citealt[400]{Krifka95a} for a similar idea). The two readings are depicted in (\ref{ping-sem:deng}).
 
\settowidth\jamwidth{(measure reading) T}  
 \ea\label{ping-sem:deng}
 \ea %
% \scalebox{\sizeavm}{
 \avm{[\textsc{index} & \1 \\ \textsc{rels} & $\langle$\textit{bottle-rel}(\1, \2), \textit{card-rel}(\1, 2), \textit{wine-rel}(\2)$\rangle$] }
% }
 \jambox{(counting reading)}

 \ex %
% \scalebox{\sizeavm}{
 \avm{[\textsc{index} & \1 \\ \textsc{rels} & $\langle$\textit{bottle-measure-rel}(\1, 2),\textit{wine-rel}(\1)$\rangle$] } 
%}
\jambox{(measuring reading)}
 \z
 \z
 
% we can distinguish the two readings purely at the level of \textsc{content} by sharing the \textsc{index} values of the CLFPs with different semantic relations. In the counting reading, the \textsc{index} of the CLFP is identified with  the argument of the semantic relation  expressed by the head N \textit{bottle-rel}, and in the measuring reading, the \textsc{index} of the CLFP is identified with the argument of a measure function \textit{measure-rel} (see \citealt[400]{Krifka95a} for more on measure functions).

 
% the semantic relation \textit{measuring} would dominate the content of CLFPs. 

The idea would be that the \textsc{content} value of \obj{ping} is lexically underspecified with respect to whether it encodes \textit{bottle-rel} or \textit{bottle-measure-rel}. The advantage of this treatment is that the uniform syntactic structure is preserved while the different readings are resolved at the semantic level. We leave a more explicit formalization of the semantics of CLFPs for future work.}  Since any DeP or DEMs can in principle combine as a functor with  CLF\down{m} or  N, cases where DeP (\ref{ex:ambig3:deng}) and DEMs (\ref{pre-dem:deng}) precede NUM-CLF\down{m}-N sequences are predicted to be structurally ambiguous. No such ambiguity should exist for sortal \isi{classifiers}, because they are not a subsort of \textit{n-noun}.  That is, our account derives the range of positions in Figures~\ref{fig:npwclm:deng}--\ref{fig:npwcls:deng} for DePs and DEM in NCs with measure \isi{classifiers}  and for NCs with sortal \isi{classifiers}.

%In general, whenever CLF\down{m}  can be modified by a DeP, then specification by \textsc{dem} is also possible. This CLF\down{m}P-internal \textsc{dem} position is further supported by the observation that \textsc{dem}s can act as semantic functors over CLF\down{m} meanings. Suppose three identical books are placed respectively in a box and in a basket. In the case of contrast, the \isi{classifier} is emphasized: \obj{na XIANG shu, bushi na LAN shu} `that box book not that basket book'. Similarly, stress can be placed on N to distinguish between the contents of the identical boxes. %Hence, like NPs, CLF\down{m}P can have its own \textsc{dem}, in addition to its own MODs. Considering all the possibilities mentioned above, we obtain the analysis displayed in Fig.\ \ref{fig:npwclm:deng}.





This proposal correctly predicts non-spurious structural ambiguities in (\ref{cl-rep:deng}). 

\ea
{\gll wo mai-le da de na san xiang shu.\\
1.\textsc{sg} buy-\textsc{pfv} big \textsc{de} \textsc{dem} three \textsc{clf}\down{m $\approx$ `box'} book\\
\glt `I bought those three big boxes of books' or \\`I bought those three boxes of big books'  or \\`I bought three boxes of those big books' \label{cl-rep:deng}
}
\z


  \begin{figure}[b]
    \centering
		\begin{forest}
		bottom word,
		[NP
			[(DeP) [,phantom]]
			[NP
			[(DEM) [,phantom]]
			[NP
			[(DeP) [,phantom]]
			[NP
			[({CLF\down{m}P})
			[({DeP}) [,phantom]]
			[({CLF\down{m}P})
			[({DEM}) [,phantom]]
			[({CLF\down{m}P})
			[({DeP}) [,phantom]]
			[({CLF\down{m}P})
			[NUM [,phantom]]
			[{CLF\down{m}} [,phantom]]
			]
			]
			]
			]
			[NP
			[(DeP) [,phantom]]
			[N [,phantom]]
			]
			]
			]
			]
		]
		\end{forest}
    \caption{\isi{NP structure} with CLF\down{m}}
    \label{fig:npwclm:deng}
  \end{figure}


\noindent The interpretation `three big boxes of those books' is ruled out by the structure in  \figref{fig:npwclm:deng} because, if \textit{da de} `big'  is predicated of \textit{xiang} `boxes', then so must be the DEM to its right appearing before the CLF\down{m} -- i.e.\ given their position, these functors have to be the topmost DePs and DEMs inside the CLFP in \figref{fig:npwclm:deng}. An approach that allows DeP to raise over DEM (as we suggested above in connection to \citealt{Cinque05b}) would have trouble capturing this restriction, since there could be a lower copy of \textit{da de} `big'  inside the CLF\down{m}P that is chosen for \isi{reconstruction}, while DEM is attached high directly to the NP, above the CLF\down{m}P. %. Given the linear position and interpretation of DePand \textsc{dem},

%\todoMyP{Should we change MOD into DePalso in earlier chapters? to make our notation more consistent?} %lower copy of DePchosen for interpretation could be inside the CLF\down{m}P.




The structures in Figures~\ref{fig:npwclm:deng}--\ref{fig:npwcls:deng} model quite well the distribution of DEM, DeP, and CLFP observed in \sectref{sec:puz:deng}. In particular, the structural ambiguity of DePs and DEMs preceding NUM--CLF\down{m} is derived because DePs and DEMs can combine either with  CLF\down{m}P or higher up with NP. CLFPs are also similar to DePs and DEMs in the sense that they are prenominal and optional. Overall, DePs, DEMs, and CLFPs in Mandarin Chinese have similar collocational properties, which is a natural consequence of the unification of \isi{modifiers} and specifiers in the HFA.

%\begin{minipage}[b][][b]{.48\textwidth}
%	\vspace{0.25cm}
%	\vspace{0pt}

%\end{minipage}
%%

  \begin{figure}
    \centering
    
		\begin{forest}
		bottom word,
		[NP
		[(DeP) [,phantom]]
			[NP
			[(DEM) [,phantom]]
			[NP
			[(DeP) [,phantom]]
			[NP
			[({CLF\down{s}P})
			[NUM [,phantom]]
			[{CLF\down{s}} [,phantom]]
			]
			[NP
			[(DeP) [,phantom]]
			[N [,phantom]]
			]
			]
			]
			]
		]
		\end{forest}
   \caption{\isi{NP structure} with CLF\down{s}}
   \label{fig:npwcls:deng}
   % \caption{t}
 \end{figure}

 %\begin{figure}
  %\floatsetup{style=plain}
  %\begin{minipage}{0.48\textwidth}
    %\centering
    %\begin{forest}
    %bottom word,
    %[NP
      %[(DeP) [,phantom]]
      %[NP
      %[(DEM) [,phantom]]
      %[NP
      %[(DeP) [,phantom]]
      %[NP
      %[({CLF\down{m}P})
      %[({DeP}) [,phantom]]
      %[({CLF\down{m}P})
      %[({DEM}) [,phantom]]
      %[({CLF\down{m}P})
      %[({DeP}) [,phantom]]
      %[({CLF\down{m}P})
      %[NUM [,phantom]]
      %[{CLF\down{m}} [,phantom]]
      %]						
      %]
      %]
      %]
      %[NP
      %[(DeP) [,phantom]]
      %[N [,phantom]]
      %]
      %]
      %]
      %]
    %]
    %\end{forest}
    %\caption{\isi{NP structure} with CLF\down{m}}
    %\label{fig:npwclm:deng}
  %\end{minipage}
  %\hfill
  %\begin{minipage}{0.48\textwidth}
    %\centering
    %\begin{forest}
    %bottom word,
    %[NP
    %[(DeP) [,phantom]]
      %[NP
      %[(DEM) [,phantom]]
      %[NP
      %[(DeP) [,phantom]]
      %[NP
      %[({CLF\down{s}P})
      %[NUM [,phantom]]
      %[{CLF\down{s}} [,phantom]]
      %]
      %[NP
      %[(DeP) [,phantom]]
      %[N [,phantom]]
      %]
      %]
      %]
      %]
    %]
    %\end{forest}
    %\caption{\isi{NP structure} with CLF\down{s}}
    %\label{fig:npwcls:deng}
  %\end{minipage}
%\end{figure}%\begin{minipage}[b][][b]{.48\textwidth}
%	\vspace{1.8cm}
%	\vspace{0pt}
%	\centering
%	\scalebox{\sizefig }{
%	}
%\end{minipage}


%These structures model quite well the distributional behavior of \textsc{dem}s, DePs and CLFPs observed in Section \ref{sec:puz:deng}.   In particular, the structural ambiguity of DePs and \textsc{dem}s preceding NUM-CLF\down{m} is derived  because DePs and \textsc{dem}s can attach either to the  CLF\down{m}P  or higher up to the NP. Overall, it is clear that DePs and \textsc{dem}s in Mandarin Chinese have a very a similar collocational properties, which is a natural consequence of the unification of \isi{modifiers} and specifiers in the HFA.  



%There is, however, a difference in the behavior of DePs and \textsc{dem}s that we also observed in Section \ref{sec:puz:deng} \textcolor{red}{(REF)}, namely, the fact that DePs  can be indefinitely iterated and \textsc{dem}s can only appear once per NP, regardless of whether they combine inside the  CLF\down{m}P or directly with the NP. In order to explain why two \textsc{dem}s cannot appear at the same time and DePs can, we need to invoke another ingredient of the HFA. Elements in the NC (N, \textsc{dem}, CLF) have different \textsc{mrk} values. We propose that objects of sort \textit{marking} are  arranged  the following  hierarchy:\footnote{We believe that the hierarchy in (\ref{mark-hie:deng}) provides a more natural encoding for the Mandarin Chinese data than the  one in \citet[167]{VanEynde06a}, which is motivated for European languages. That said, for ease of exposition, one can think of \textit{n-marked} as being equivalent to Van Eynde's \textit{bare}, and \textit{weak} and \textit{strong} as being parallel to his \textit{unmarked} and \textit{marked} sorts, respectively. }




There is, however, a difference between DEMs, CLFPs, and DePs that we also observed in \sectref{sec:puz:deng}, namely, the fact that DePs  can be indefinitely iterated and CLFPs and DEMs can only appear once per NP -- in the case of DEMs, regardless of whether they combine inside the CLF\down{m}P or directly with the NP. 

To explain why multiple DEMs/CLFPs cannot appear at the same time and DePs can, we need to invoke another ingredient of the HFA. Elements in the NC  carry different \textsc{marking} values, only some of which can be selected by CLF and DEM. %and DEM and CLF only select heads that do not carry their own marking values
We propose that objects of sort \textit{marking} are  arranged as in the  hierarchy in \figref{fig:mark-hie:deng}, where  \textit{n-marked} is the \textsc{marking} value of N and \textit{cl-marked} is  the \textsc{marking} value of CLF.\footnote{We believe that the hierarchy in \figref{fig:mark-hie:deng} provides a more natural encoding for the Mandarin Chinese data than the  one in \citet[167]{VanEynde06a}, which is motivated for European languages. That said, for ease of exposition, one can think of \textit{n-marked} as being equivalent to Van Eynde's \textit{bare}, and \textit{weak} and \textit{strong} as being parallel to his \textit{unmarked} and \textit{marked} sorts, respectively. }


\begin{figure}
  \centering
\begin{forest}
[\textit{marking}
	%%%
	[\textit{weak}
%		[\dots ]
		%%%	
		[\textit{n-marked}]
		%%%
		[\val{cl-marked}] 
	]
	%%%
	[\textit{strong}
		%%%
%		[\dots ] 
		%%%
		%[\textit{dem-marked}] 
	]
]
\end{forest} 
  \caption{Hierarchy of \textsc{marking} values}
  \label{fig:mark-hie:deng}
  %\label{fig:u}
\end{figure}

%\vbox{
%\ea {Hierarchy of \textsc{marking} values  \\
%\scalebox{\sizefig }{
%
%}}
%\z
%}

%Since the \textsc{marking} value of a head can only be modified when it combines with a functor (as per  the Generalized Marking Principle in (\ref{gmp:deng})), it is capable of registering the combinatorial history of a head (e.g.\ whether it is a \textit{complete} projection), in a more fine-grained way than  the \textsc{spr} feature in \textit{spec-head} approaches.  In other words, the formalism itself does not embody the assumption that special kinds of prenominals (namely, \textit{functional preonominals} like determiners) are always needed to complete a nominal projection. Consider the two English examples below:
%
%
%\ea
%\scalebox{\sizefig}{\begin{forest} bottom word,
%[{[\textsc{head} \inx{1}, \textsc{marking} \inx{2}]}
%	[{[\textsc{head}$\mid$\textsc{sel} \inx{4}, \textsc{marking} \inx{2}\textit{marked}]}
%	[that]
%	]
%	[{\inx{4} [\textsc{head} \inx{1}, \textsc{marking} \inx{5}\textit{incomplete}]}
%		[{[\textsc{head}$\mid$\textsc{sel} \inx{3}, \textsc{marking} \inx{5}\textit{unmarked}]} [red]] [{\inx{3}[\textsc{head} \inx{1}\textit{noun}, \textsc{marking} \inx{5}\textit{incomplete}]} [book]]
%	]
%]
%\end{forest}}
%
%\z

Recall that, in virtue of the Generalized Marking Principle in (\ref{gmp:deng}),  the \featD{marking} value of the combination between a functor and a head is the \textsc{marking} value of the functor. To explain why DeP functors like the adjectival \textit{haokan de} (`nice') can be freely iterated and combined with either N or CLF\down{m} in any order, we propose  the general structure in (\ref{de-str:deng}). We assume that DePs are  head-complement phrases, with the dependent-marking particle \textit{de} serving as the head daughter. What \textit{de} effectively does, according to (\ref{de-str:deng}), is take a predicate of some sort (an adjective, in that case) and map it into a functor that selects an N. 
% Problem here: no matter if I do newpage or not, I will have in any case the separation
%\newpage
%\newpage

\protectedex{
\ea Sample \textit{DeP} (for adjectives) \\
%\scalebox{\sizeavm }{
\oneline{%
  \avm{
		[\type*{head-complement-phrase} phon & $\langle$haokan,de$\rangle$ \\
		synsem|loc|cat & 
		[head & \2 
		[\type*{de} 
		select & [head & n-noun \\ comps & <>  \\ marking & \1] 
		] \\
		comps & <> \\
		marking & \1] \\
		hd-dtr|synsem|loc|cat & [head & \2 \\ comps & <\3> \\ marking \1] \\
		non-head-dtrs &< 
		\3 [ synsem|loc|cat & [head & adj \\ comps & <> \\ 
		%marking & marking 
		]]> 
		]
	}} 
        % }
        \label{de-str:deng}
\z
}
%\largerpage[3]

\noindent Crucially, DeP does not impose any requirement on the \textsc{marking} value of the head -- it inherits its own \textsc{marking} value from the \val{n-noun} (N or CLF\down{m})  it selects. This \textsc{marking} value is, by virtue of (\ref{gmp:deng}), projected the mother of DeP. %Since the selected element is of type \val{n-noun}, DePs can combine with projections of N or CLF\down{m}, giving rise to \ref{fig:npwclm:deng}.


The non-iterability of CLFPs is due to the constraints on the \textsc{marking} values of the heads they select. We treat CLFPs as signs of type \textit{num-cl-phrase}, a subtype of \textit{head-complement-phrase}, \isi{subject} to (\ref{head-comp-schema:deng}). What defines this specific type of head-complement structure is the fact that it is headed by a non-phrasal CLF with a non-phrasal NUM as its complement. The non-phrasal status of the CLF and NUM daughters of CLFP is represented by a positive specification for the boolean \textsc{lex} feature (\citealt[72--73]{Pollard&Sag87a}).%We treat CLFPs as subtypes of head-complement structures headed by a non-phrasal CLF with a non-phrasal NUM as their complement%as a complex word consisting of a non-phrasal CLF with a non-phrasal NUM as its daughters to ensure that no \isi{modifiers} can be inserted into them, as mentioned in footnote \ref{fn1:deng}.%
%
\footnote{The fact that the daughters of these structures have to be non-phrasal likens \textit{num-cl-phrase} to compounds. Unlike compounds, however, the output of the combination is \textsc{lex} $-$ (this allows DEMs to select CLFPs, cf. \ref{min-dem:deng}). We assume that, due to constraints on the sort \textit{phrase}, the result of any phrasal combination (\textit{head-complement}, \textit{head-functor}, etc.)  is always \textsc{lex} $-$. The fact that the \textsc{phon} of the complement of CLF (i.e.\ NUM) precedes that of CLF follows from a general linear precedence rule for all nominal head-complement structures in Mandarin Chinese.} 
%
This ensures that no \isi{modifiers} can be inserted between NUM and the CLF head -- i.e.\ it rules out structures like \textit{*san da de xiang shu} `three big \textsc{de} box book'. Crucially, all CLFP functors select a \textit{n-marked} head of type \textit{lex-noun} (in virtue of their CLF head) and have the \textsc{marking} value \textit{cl-marked}, which (\ref{gmp:deng}) ensures is transmitted to their mother. (\ref{num-cl:deng}) exemplifies this. 
%
%\todoMyP{I remember there were problems with that (maybe compounding?) But this is not our problem \shrug }
%
%Any  such CLFP structure will thus minimally instantiate   (\ref{num-cl:deng}).
%Regarding CLFs, we also assume that idiosyncractic selection

\ea Sample structure for \textit{Classifier Phrase}
% \scalebox{\sizeavm }{
\oneline{%
	\avm{
		[\type*{num-cl-phrase} %phon & $\langle$haokan$\rangle$ \\
		synsem|loc|cat & 
		[head & \2 
		[\type*{cl-noun} 
		select & [head & lex-noun \\ marking & n-marked] 
		] \\
		comps & < > \\
		lex & $-$ \\
		marking & \1 cl-marked] \\
		hd-dtr|synsem|loc|cat  & [head & \2 \\ comps & <\3> \\  lex & $+$ \\ marking \1]  \\
		non-head-dtrs &< 
		\3 [ synsem|loc|cat & [head & num \\ lex & $+$ %\\  marking & n-\isi{marked} 
		]]
		> 
		]
	} 
   }     % }
        \label{num-cl:deng}
\z

%\ea Sample structure for \textbf{Classifier Phrase}\\
%\scalebox{\sizeavm }{%\val{head-classifier-word} $\Rightarrow$
%	\avm{
%		[\type*{num-classifier-phrase} \\
%		phon & <\1 $\oplus$ \2>\\
%		synsem|loc|cat & 
%		[head & 
%		[\type*{cl-noun} 
%		select & [head & lex-noun \\ marking & n-\isi{marked}] 
%		] \\
%		lex & $-$ \\
%		marking & cl-\isi{marked}] \\ 
%		dtrs & <[phon & \1\\head & num\\lex & $+$], [phon & \2\\head & cl-n\\lex & $+$]>\\
%		%non-hd-dtrs & <[phon & \1\\head & num\\lex & $+$]> \\
%		] } 
%} \label{num-cl:deng}
%\z




\noindent Since \textit{cl-marked}  is not a subtype of \textit{n-marked}, a second CLFP cannot be added to a NC after a CLFP has already marked the nominal projection -- even if there is an intervening DeP between them, as \figref{fig:npw2cl:deng} illustrates (the offending phrases are signaled with an asterisk, the remainder of the structure is well-formed).

%	\centering
%	\scalebox{.5}{
%		\begin{forest}
%			bottom word,
%			%		[NP
%			%		[Mod\down{5} [x]]
%			%		[NP
%			%		[Dem\down{2} [x]]
%			[(NP\down{\val{cl-marked}})
%			[(*CLF\down{s}P) \\ \textsc{marking} \val{\inx{6} cl-marked} \\
%			\textsc{select} \inx{7} [ben \\ \textsc{cl}]]
%			[NP \\ \textsc{marking} \val{\inx{4} cl-marked} 
%			[DeP\\ \textsc{marking} \val{\inx{4}} \\
%			\textsc{select} \inx{5} [de de\\ big \textsc{de}]]
%			[\inx{5} NP \\ \textsc{marking} \val{\inx{4} cl-marked} 
%			[ CLF\down{m}P \\ \textsc{marking} \val{\inx{4}} \\
%			\textsc{select} \inx{3}
%				[Num [san \\ three]]
%				[CLF\down{m} [xiang \\ box]]
%			]
%				[\inx{3} NP \\  \textsc{marking} \val{\inx{1} n-marked} 
%					[DeP\\ \textsc{marking} \val{\inx{1}} \\
%			\textsc{select} \inx{2}   [haokan de\\ nice \textsc{de}]]
%					[\inx{2} N \\ \textsc{marking} \val{\inx{1}}   [shu\\book]]
%				]
%			]
%			]
%			]
%			%		]
%			%		]
%		\end{forest}
%	}
\begin{figure}
  \centering
		\begin{forest}
			bottom word,
			%		[NP
			%		[Mod\down{5} [x]]
			%		[NP
			%		[Dem\down{2} [x]]
			[*NP\down{\val{cl-marked}}
			[*CLF\down{s}P\down{\val{cl-marked}} [ben \\ \textsc{clf}]]
			[NP\down{\val{cl-marked}}
			[DeP\down{\val{cl-marked}} [da de\\ big \textsc{de}]]
			[NP\down{\val{cl-marked}}
			[CLF\down{m}P\down{\val{cl-marked}}
				[Num [san \\ three]]
				[CLF\down{m} [xiang \\ \textsc{clf}\down{m $\approx$ `box'}]]
			]
				[NP\down{\val{n-marked}}
					[DeP\down{\val{n-marked}} [haokan de\\ nice \textsc{de}]]
					[N\down{\val{n-marked}} [shu\\book]]
				]
			]
			]
			]
			%		]
			%		]
		\end{forest}
  \caption{Ill-formed \isi{NP structure} with two CLFPs}
  \label{fig:npw2cl:deng}
\end{figure}

DEMs in Mandarin Chinese have minimally the structure in (\ref{min-dem:deng}). What is essential is that DEMs select a phrasal \textit{n-noun} head with a \textit{weak} \textsc{marking} value. % and mark the resulting mother with their own \textit{strong}  \textsc{marking} value. 
%
%We can account for the the structural ambiguity of pre-CLF\down{m}  DEM (\ref{pre-dem:deng}) because both  CLF\down{m} or N, as subsorts of \textit{n-noun}, can be selected by DEM. 
%
This predicts  that DEM can precede a CLFP  (\ref{ex:modnmodclnumclmn1:deng}), because the \textsc{marking} value of the latter is \textit{cl-marked}, i.e.\ a subsort of \textit{weak} -- the value selected by DEM.  Therefore, we effectively solve one of the puzzles that Mandarin Chinese NCs posed for  \textit{head-specifier} approaches (and to some extent also DP theories): the fact that the same structure can have multiple D-like elements. 


\ea Minimal structure for \textit{demonstratives}  \\
%\scalebox{\sizeavm }{
\avm{
	[ 
	cat & 
		[head & 
			[\type*{demonstrative} select & 
				[head & n-noun \\ marking & weak \\ lex & $-$] 
			] \\
	marking & strong] \\
	]
      }
      % }
      \label{min-dem:deng}
\z

\begin{figure}
  \centering
			\begin{forest}
				bottom word,
				%		[NP
				%		[Mod\down{5} [x]]
				%		[NP
				%		[Dem\down{2} [x]]
				[NP\down{\val{strong}}
				[DEM\down{\val{strong}} [na \\ \textsc{dem}]]
				[NP\down{\val{cl-marked}}
				[CLF\down{m}P\down{\val{cl-marked}}
				[Num [san \\ three]]
				[CLF\down{m} [xiang \\ \textsc{clf}\down{m $\approx$ `box'}]]
				]
				[NP\down{\val{n-marked}}
				[DeP\down{\val{n-marked}} [haokan de\\ nice \textsc{de}]]
				[N\down{\val{n-marked}} [shu\\book]]
				]
				]
				]
				%		]
				%		]
			\end{forest}
  \caption{\isi{NP structure} with DEM and \textsc{clf}P}
  \label{fig:npw2spec:deng}
\end{figure}

After a DEM  functor combines with CLF\down{m} or N, the resulting phrase inherits the \textit{strong}  \textsc{marking} of DEM. An NP marked with \textit{strong} has no problem being selected by a DeP, assuming the latter has a structure like (\ref{de-str:deng}). This allows us to solve the puzzle for the DP approach pointed out in connection to (\ref{ex:position3:deng}) -- namely, the fact that DePs can precede DEMs, as \figref{fig:de-dem:deng} illustrates.\footnote{The fact that prenominal DeP can appear both before and after DEM, apparently contradicting Greenberg's (\citeyear{Greenberg63c}) Universal 20, arguably follows from the fact that both DEMs and DeP in Mandarin Chinese are subsumed under the general category of functors. We can hypothesize that Universal 20 only applies to languages where DEMs function as specifiers.} 

\begin{figure}
  \centering
			\begin{forest}
				bottom word,
				%		[NP
				%		[Mod\down{5} [x]]
				%		[NP
				%		[Dem\down{2} [x]]
				[NP\down{\val{strong}}
				[DeP\down{\val{strong}} [haokan de\\ nice \textsc{de}]]
				[NP\down{\val{strong}}
				[DEM\down{\val{strong}} [na \\ \textsc{}]]
				[NP\down{\val{cl-marked}}
				[CLF\down{m}P\down{\val{cl-marked}}
				[Num [san \\ three]]
				[CLF\down{m} [xiang \\ \textsc{clf}\down{m $\approx$ `box'}]]
				]
				[N\down{\val{n-marked}} [shu\\book]]
				]
				]
				]
				%		]
				%		]
			\end{forest}
  \caption{\isi{NP structure} with DeP preceding DEM}
  \label{fig:de-dem:deng} 
\end{figure}

The iteration of DEMs is ruled out because DEM is \textit{strong}-marked but selects \textit{weak}-marked elements. Note that \textit{strong} marking projects regardless whether the most embedded DEM is attached to  N or CLF\down{m}. In the latter case, the \textsc{marking} value \textit{cl-marked} that is lexically associated with CLF\down{m} is overwritten by \textit{strong} as soon as DEM is merged. This becomes the \textsc{marking} value of CLF\down{m}P, so no further DEM can be added to N to the left of the CLF\down{m}P, as \figref{fig:npwdems:deng} shows. %makes it impossible to integrate any further DEMs into the structure, 

\begin{figure}
  \centering
		\begin{forest}
			bottom word,
			%		[NP
			%		[Mod\down{5} [x]]
			%		[NP
			%		[Dem\down{2} [x]]
	[*NP\down{\val{strong}}
		[*DEM\down{\val{strong}} [zhe \\ \textsc{dem}]]
		[NP\down{\val{strong}}
		[CLF\down{m}P\down{\val{strong}}
			[DEM\down{\val{strong}} [na \\ \textsc{dem}]]
			[CLF\down{m}P\down{\val{cl-marked}}
				[NUM [san \\ three]]
			[CLF\down{m} [xiang \\ \textsc{clf}\down{m $\approx$ `box'}]]
			]
		]	
			[NP\down{\val{n-marked}}
				[DeP\down{\val{n-marked}} [haokan de\\ nice \textsc{de}]]
				[N\down{\val{n-marked}} [shu\\book]]
			]
		]
	]
			%		]
			%		]
		\end{forest}
  \caption{Ill-formed \isi{NP structure} with two DEMs}
  \label{fig:npwdems:deng} 
\end{figure}


As we hinted at in \sectref{sec:puz:deng}, in the absence of a \textsc{marking} feature, it would not be possible to capture this effect without somehow relaxing \isi{locality} of selection. On a DP theory, one would have to say that  DEM can probe inside the non-immediate daughters of its complement to make sure that no other DEM was combined inside of them. On standard NP theories, one would have to first allow N to have two optional specifiers (see for instance \citealt{Ng97}) and then impose a constraint to ensure that, if the second specifier contains DEM somewhere among its non-head daughters, DEM cannot appear as the first specifier. %to account for the possibility of DEM-CLFP-N

This account also explains why CLFPs cannot precede DEMs within the NP, as we saw in (\ref{ex:modnmodclnumclmn2:deng}).   Given (\ref{min-dem:deng}), any NP mother of DEM will have \textit{strong}  \textsc{marking} value. This is incompatible with the selectional requirements imposed by CLFPs  (\ref{num-cl:deng}), which require \textit{n-marked} NPs as their sisters, as \figref{clp-dem:deng} illustrates.


%Though CLFPs are  similar to DEMs and DePs in that they are optional prenominal categories (the major reason why we treat CLFPs as functors), they cannot be iterated either (with the exception of kind \isi{classifiers}). To constraint the iterability of CLFP, we use \textsc{mrk} values as well.

With these ingredients, we can also envision a solution to the two of the remaining puzzles: the optionality of  CLF and DEM (\sectref{sec:con-dp:deng}) and the singular demonstrative paradox (\sectref{sec:sing:deng}). The optionality of prenominal elements like CLF and DEM can be explained  by positing that verbs and other heads that can take NP as their valents in Mandarin Chinese simply do not care about the \textsc{marking} values on NP \footnote{Verbs in Mandarin Chinese do impose requirements on their valents beyond what we formalize in (\ref{mai-entry:deng}), depending on the grammatical function of arguments (\citealt{Huang&Co09a}). Subjects, for instance,  can only be definite NPs, whereas some combinations like  CLFP-N (with a null NUM proform meaning \obj{one}, see below) are only possible as objects. These semantic restrictions can be encoded as constraints on the \textsc{content} values of the elements in the \textsc{subj} or \textsc{comps} of verbs.}. That is, the \textsc{marking} values of NP valents are underspecified, as illustrated by the partial entry for \textit{mai} `buy' in (\ref{mai-entry:deng}).

\begin{figure}
  \centering
		\begin{forest}
			bottom word
	[*NP\down{\val{cl-marked}}
			[*CLF\down{m}P\down{\val{cl-marked}}
				[NUM [san \\ three]]
			[CLF\down{m} [xiang \\ \textsc{clf}\down{m $\approx$ `box'}]]
			]
		[NP\down{\val{strong}}
		[DEM\down{\val{strong}} [zhe \\ \textsc{dem}]] 	
			[NP\down{\val{n-marked}}
				[DeP\down{\val{n-marked}} [haokan de\\ nice \textsc{de}]]
				[N\down{\val{n-marked}} [shu\\book]]
			]
		]
	]
			%		]
			%		]
		\end{forest}
  \caption{Ill-formed structure with CLFP preceding \textsc{dem}}
  \label{clp-dem:deng}
\end{figure}


\ea 
%\scalebox{\sizeavm }{
	\avm{
		[%\type*{word} 
		phon & $\langle$mai$\rangle$ \\
		synsem|loc|cat & 
					[head & verb \\
					subj & <!\textup{NP\down{\val{marking}}}!>\\
					comps & <!\textup{NP\down{\val{marking}}}!>]
		]
	} 
        % }
        \label{mai-entry:deng}
\z



We can also solve the singular demonstrative paradox in (\ref{ex:sing-dem-para:deng})  -- i.e.\ the observation that  the combination of a DEM and a bare N can only be interpreted as singular, while the combination of a DEM with a modified N can be singular or plural. %The paradox stems from the fact that DePmodifiers do not seem to encode anything about NUM in other contexts, but its presence in DEM-DeP-N sequences seems to add a plural interpretation which is absent from DEM-N alone. 
The key to solving this puzzle lies in the requirement that DEM select an element of type \val{n-noun} with the \featD{lex}  value $-$ (\ref{min-dem:deng}).%
%
\footnote{With respect to the \textsc{lex} feature, see \citet{Pollard&Sag87a} and \citet{Arnold&Sadler92a}.}
%
% Make here a paragraph break?
 A bare N is of type \textit{lex-noun} (i.e.\ a subtype of \textit{n-noun}, along with \textit{m-cl-noun}) and has the \featD{lex} value $+$, as in (\ref{nnoun-lex:deng}). But after combining with a DeP, the resulting phrase becomes \featD{lex} $-$. The number neutrality of the resulting phrase follows from the fact that N and DeP are both underspecified for number.

\ea Minimal structure for \emph{lex-noun}s (a subtype of \textit{n-noun}) \label{nnoun-lex:deng}\\
%\scalebox{\sizeavm }{
\avm{
	[ 
	synsem|loc|cat & 			[head & lex-noun  \\
	marking & n-marked \\
	lex & $+$ ] 
	]
} 
%}
\z


\noindent A consequence of (\ref{nnoun-lex:deng}) is that DEM cannot directly combine with a bare N.\footnote{This also predicts that DEM cannot combine directly with a CLF\down{m}  head (a subtype of \textit{n-noun}), before CLF\down{m}  combines with its NUM complement giving rise to a phrasal structure like (\ref{num-cl:deng}).}  We propose that when DEM combines with a seemingly bare N, there is actually a covert singular CLF\down{s}P. We must posit, therefore, a phonologically empty NUM expressing a  singular cardinality relation and a null CLF$_{s}$ selecting this singular NUM via \val{head-complement-phrase}, giving rise to a structure like \figref{fig:sing-dem:deng}.%\footnote{There is independent evidence for a null proform for `one'  in Mandarin Chinese.  Phrases where NUM does not overtly occur can only be interpreted as singular (e.g.\ \textit{na ben shu} `that \textsc{cl} book'), indicating the presence of a covert singular NUM  (see also \citealt[183--187]{Zhang2019a}). 		
%\ea \label{ex:demsg:deng}
%\gll na ben shu\\
%\textsc{dem} \textsc{cl}  book\\
%\glt `that book.' 
%\z 
%}

\begin{figure}
  \centering
		\begin{forest}
			bottom word,
			%		[NP
			%		[Mod\down{5} [x]]
			%		[NP
			%		[Dem\down{2} [x]]
	[NP\down{\val{strong}}{[\textsc{lex $-$}]}
		[DEM\down{\val{strong}} [zhe \\ \textsc{dem}]]
		[NP\down{\val{cl-marked}}{[\textsc{lex $-$}]}
			[CLF\down{s}P\down{\val{cl-marked}}
				[NUM [$\emptyset$ \\ one]]
			[CLF\down{s} [$\emptyset$ \\ \textsc{clf}\down{s}]]
			]
			[N\down{\val{n-marked}}{[\textsc{lex $+$}]}
				[shu\\book]
			]
		]
	]
			%		]
			%		]
		\end{forest}
  \caption{\isi{NP structure} with covert singular CLF\down{s}P}
  \label{fig:sing-dem:deng}
\end{figure}

There is independent evidence for a null proform for `one'  in Mandarin Chinese.  Phrases where NUM does not overtly occur can only be interpreted as singular (e.g.\ \textit{na ben shu} `that \textsc{clf} book'), indicating the presence of a covert singular NUM  (see \citealt{Heratal15a} and \citealt[183--187]{Zhang2019a} for more discussion).%
%\largerpage[-5]
%
\footnote{Similarly, the existence of a null CLF\down{s} can be supported by structures where CLF\down{s} is optional, like \textit{wuqian (ge) ren} `5000 (\textsc{clf}) person'  \citep[1669]{Her12a}. In some northern Chinese dialects and formal registers, NUM-N sequences are even more productive and can occur with other kinds of Ns (\citealt{Heratal15a}).  
 %\ea \label{ex:without:deng}
%\gll wuqian ren \\
%5000 person \\
%\glt `5000 people' 
%\z
The fact that the null CLF\down{s} has a more widespread distribution in some forms of speech follows from variation in its selectional properties. We can speculate that, unlike in standard Mandarin Chinese, the null CLF\down{s} in these varieties does not require the element in its \textsc{comps} list to be the null NUM proform. The association between lexemes and particular registers or situational parameters can be modeled using an extension of the \isi{HPSG} formalism, see \citet{MyP&Co22a}  and \citet{Varaschin&Co24a}.}  
%
%\ea \label{ex:demsg:deng}
%\gll na ben shu\\
%\textsc{dem} \textsc{cl}  book\\
%\glt `that book' 
%\z

Our theory can, thus, derive these seemingly paradoxical facts from independently motivated assumptions, without  attributing contradictory properties to DEM (singular number in DEM-N, number neutrality in DEM-DeP-N sequences).


\section{Conclusion and typological hypothesis}\label{sec:conc:deng}

In this study we proposed a Head-Functor Analysis (HFA) for NCs in Mandarin Chinese. This is motivated by the observation that, as opposed to what we see in languages with dedicated determiners, in Mandarin Chinese bare Ns can appear as arguments in all contexts and typical \isi{modifiers} (e.g.\ APs) and  specifiers (DEMs) can be freely interweaved and co-occur. The structural ambiguity of elements that appear before NUM also supports this approach and provides evidence for a \isi{left-branching} [[NUM CLF] N] structure headed by N.

All in all, the HFA allows us to solve the puzzles that NCs in Chinese pose for standard NP and DP theories. Bare and complex NCs are licensed as full NPs, without the need to posit empty specifiers or Ds. DEM, CLF,  and MODs can be unified under the category of functors which select and mark N-heads, which explains their prenominal status and optionality. Constraints on the \textsc{marking} values of the heads selected by each of these functors account for the distributional differences between them, e.g.\ for the fact that DEM can precede CLFP (but not vice versa), and that MODs can be iterated but DEM and CLFP cannot. We also proposed that CLF\down{m} and CLF\down{s} have different structures and that DEM and MOD can attach only to elements of type \val{n-noun}, i.e.\ CLF\down{m} or lexical Ns. The singular demonstrative paradox is solved by postulating a lexicon containing only one phonologically empty CLF\down{s} and an independently motivated a singular NUM.%, needed to specify the meaning of N to `one'.


%This \textsc{marking} hierarchy allows us to have a finer-grained understanding of the effects of prenominal elements on the category of resulting phrases than standard NP approaches, where the effects of prenominal non-heads on their mothers can only be seen through the distinction between saturated  (i.e.\  [\textsc{spr} $\langle \rangle$]) and unsaturated (i.e.\ [\textsc{spr} $\langle \textit{synsem} \rangle$]).  

The behavior of prenominal elements in Mandarin Chinese is very different from what we see in languages like German or English, both of which have dedicated determiners (\citealt{Pollard&Sag94a, MyP&Mueller21a}). In our view, both types of languages favor an NP account, but  require different combinatorial mechanisms: in German and English, prenominal elements fall more naturally into two different categories (specifiers vs. \isi{modifiers}) while in Mandarin Chinese, all prenominal elements seem to behave similarly and preserving the \isi{modifier}/specifier distinction seems unworkable.

This general outlook suggests a typology of languages which is parallel to the \isi{NP/DP parameter} proposed in the minimalist tradition, from which  many  properties have been argued to follow (\citealt{cheng1999bare, boskovic2008will, bovskovic2013word}). Rather than interpreting this in terms of the presence of DPs, these properties could be due to the presence of structures of the sort  \textit{specifier-head-phrase}  or  \textit{head-functor-phrase} in each language. D-less languages like Mandarin Chinese, Polish and Turkish would only have the latter, while English and German would only have the former. It remains to be seen the extent to which Mandarin Chinese conforms to these generalizations and how they can be derived from the differences between \textit{head-functor} and \textit{specifier-head} phrases. 

\section*{Acknowledgements}

We would like to thank the audiences at the GLiF Seminars (Formal Linguistics Research Group at the Universitat Pompeu Fabra in Barcelona), the BerlinBrnoVienna Workshop (Masaryk University in Brno), the Kolloquium Syntax und Semantik (Humboldt-Universität zu Berlin), the CSSP 2023 (École Normale Supérieure in Paris), and at the Constraining Linearization workshop at the DGfS 2024 conference (Ruhr Universität Bochum) for their helpful feedback. We profited from useful conversations with Anne Abeillé, David Adger, Sascha Alexeyenko, Zehui Guo, One-Soon Her, Zi Huang, Jean-Pierre Koenig, Xuping Li, Jialing Liang, Yanru Lu, Stefan Müller, Louise McNally, Adam Przepiórkowski, Viola Schmitt, Marcin Wągiel and Martina Wiltschko. We are very grateful to three anonymous reviewers for their detailed and constructive comments, which have led to substantial improvements. Of course, all remaining errors are our responsibility. The research reported here was partially funded by a Ph.D.\ scholarship from the Chinese Scholarship Council and Deutsche Forschungsgemeinschaft (DFG) – SFB 1412, Project A04, ID 416591334.

%\citet{Nordhoff2018} is useful for compiling bibliographies.
%\section*{Contributions}
%John Doe contributed to conceptualization, methodology, and validation. 
%Jane Doe contributed to writing of the original draft, review, and editing.
\newpage
\section*{Abbreviations}
\begin{multicols}{2}
\begin{tabbing}
distr \hspace{1em} \= arguments/modifiers of N \isi{marked} with \textit{de}\kill
N(P) \> noun (phrase)\\
D(P) \> determiner (phrase) \\
NUM \> numeral\\
CLF(P) \> classifier (phrase) \\
DEM \> demonstrative \\
\textsc{V} \> verb  \\
\textsc{ADJ} \> adjective \\
\textsc{MOD} \> modifier\\
\textsc{PFV} \> perfective\\
\textsc{SG} \> singular \\  
CLF\down{m} \> measure classifier \\
CLF\down{s} \> sortal classifier\\
DeP \> arguments/modifiers of N \\
\textcolor{white}{DeP} \> marked with \textit{de}\\%We also use following abbreviations for \isi{HPSG} analyses: 
% give explanation in normal caps
\textsc{synsem} \> syntax-semantics \\ 
\textsc{dtrs} \> daughters\\
\textsc{cat} \> category\\ 
\textsc{spr} \> specifier\\ 
\textsc{spec} \> specified \\ 
\textsc{comps} \> complements\\
\textsc{subj} \> subject\\
\textsc{phon} \> phonology\\ 
\textsc{cont} \> content\\ 
\textsc{loc} \> local\\
\textsc{lex} \> lexical \\  
\textsc{mod} \> modified\\
\end{tabbing}
\end{multicols}


\sloppy
\printbibliography[heading=subbibliography,notkeyword=this]
\end{document}

%%% Local Variables:
%%% mode: xelatex
%%% TeX-master: t
%%% End:
