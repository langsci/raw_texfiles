\chapter{Introduction}\label{chapter01}

The complexities of speech production, perception, and comprehension are enormous. This circumstance has led to the development of numerous models and theories of language production, perception, and comprehension since the dawn of the modern study of language structures in the early twentieth century. Especially since the rise of psycholinguistics in the 1960s, psycholinguistic methods and findings contributed to the development of pertinent theoretical approaches to language structure. As has been shown repeatedly in recent linguistic research, however, it remains a challenge for most if not all established approaches to account for findings on more and more intricate features of language such as differences in subphonemic detail.

In research on speech production, it was shown that homophonous lexemes differ in their acoustic duration due to differences in frequency (e.g. \cite{Jurafsky2002, Lavoie2002, Gahl2008, Drager2011, Lohmann2018}). Such findings indicate that phonologically identical lemmas may differ in their phonetic realisation. Similarly, fine phonetic differences were also found, for example, for bound versus free bases (e.g. \cite{Kemps2005a, Kemps2005b}), for final segments of a mono-morphemic stem versus the final segments of the same stem if followed by a suffix (e.g. \cite{Sugahara2004, Sugahara2009}), and for prefixes in prefixed versus so-called pseudo-prefixed words (e.g. \cite{Smith2012}). A popular case for the research of such fine-grained phonetic detail below the word level is word-final /s/ and /z/ in English. Previous corpus studies (\cite{Zimmermann2016, Plag2017, Tomaschek2019}) showed that the acoustic duration of word-final /s/ depends on its morphological make-up, with non-morphemic /s/ being longest and auxiliary clitic /s/ being shortest in duration. However, previous experimental studies (\cite{Walsh1983, Hsieh1999, Seyfarth2017}) found effects in the opposite direction. It is thus the first general aim of this book to investigate by means of a production task whether such durational differences between different types of word-final /s/ really exist, and to find a potential explanation for the contradictory nature of previous results. I will introduce relevant theoretical approaches of speech production such as feed-forward formal theories of morphology-phonology interaction (e.g. \cite{Chomsky1968, Kiparsky1982}), the framework of Prosodic Phonology (e.g. \cite{Booij1983, Nespor2007}), psycholinguistic theories of speech production (e.g. \cite{Levelt1999, Roelofs2019}), exemplar-based models (e.g. \cite{Goldinger1998, Pierrehumbert2001, Gahl2006}), and discriminative learning (e.g. \cite{Rescorla1988, Ramscar2007, Ramscar2010}) to discuss their respective explanatory limits. As the approach of discriminative learning can only be meaningfully discussed in light of an implementation of such an approach, a linear discriminative learning network (e.g. \cite{Baayen2019}) is implemented. This implementation not only allows for a discussion of the general approach itself, but potentially offers insight into the nature of the durational differences in word-final /s/.

Research on the perception of fine phonetic detail found that listeners make use of segment durations as a cue for word boundaries (\cite{Shatzman2006}) and to assist in differentiating phonologically similar lemmas (\cite{Warner2004}). Findings on bare versus suffixed stems indicate that listeners make use of acoustic duration as a cue for distinguishing such stems (\cite{Kemps2005a, Kemps2005b, Blazej2015}). Yet, there is barely any research on the question of how small such subphonemic durational differences may be to remain perceptible. This question is answered for individual segments by a rather dated study by \citet{Klatt1975}. That is, to be perceptible, a durational difference in fricatives should be of 25 ms or more. However, these authors found that perceptibility is worse in fricatives and word-final position. Hence, the second general aim of this book is to explore how small a durational difference in word-final /s/ is perceptible in a same-different perception task. I will discuss the findings taking into account abstractionist approaches (e.g. \cite{Klatt1979, McClelland1986, Norris1994, Norris2008}), approaches relying on fine phonetic detail (e.g. \cite{Goldinger1996}), approaches combining abstract representations and fine phonetic detail (e.g. \cite{Hawkins2001, Pierrehumbert2002}), and computational models of speech perception (e.g. \cite{tenBosch2015, Baayen2019}).

\largerpage
For an account of the influence of subphonemic detail on comprehension, one can consider the same results which have been brought forward to describe the perception of such fine phonetic detail. As subphonemic durational differences are used as a cue for word boundaries (\cite{Shatzman2006}), they are not only perceptible but also used in the comprehension of words. Differentiating between unsuffixed and suffixed stems by means of acoustic durations (\cite{Kemps2005a, Kemps2005b, Blazej2015}) does not only indicate that such differences are perceived, but also that such differences are made use of in comprehension. In general, however, there is little research available which directly asks the question of whether subphonemic durational differences significantly influence comprehension (e.g. \cite{Blazej2015}). Thus, it is the third general aim of this book to investigate this question. This is done by means of two number-decision tasks in a mouse-tracking paradigm. Taking into account the findings from these experiments, I will discuss the same set of theoretical approaches as are taken into account for the results of the perception study: abstractionist approaches, approaches relying on fine phonetic detail, approaches combining abstract representations and fine phonetic detail, and computational models of speech perception. 

\largerpage
The overarching goal of this book, then, is to draw a more detailed, intricate, and exhaustive picture of the production, perception, and comprehension of subphonemic detail. To achieve this goal, two important methodological decisions were taken. First, where applicable, pseudowords as well as real words are used as target items to account for potentially confounding effects of lexical properties (e.g. effects of frequency, e.g. \cite{Gahl2008, Lohmann2018}; effects of storage, e.g. \cite{Caselli2016}). Second, sound statistical analyses are performed, relying on novel statistical techniques where appropriate. Overall, the findings presented in this book are the results of a thorough methodological approach to item design and statistical analysis, offering a reliable account of the nature of subphonemic detail and a strong foundation for future research.

This book is structured as follows. In Chapter \ref{chapter02}, I will give a detailed overview of previous findings on the production, perception, and comprehension of subphonemic durational differences and introduce pertinent theoretical approaches. Taking these approaches as a starting point, hypotheses to be investigated in the subsequent chapters are derived. Chapter \ref{chapter03} will introduce the general method used in this book. It will discuss pseudowords as a type of item (Section \ref{section03_1_1}) and present the pseudoword and real word items used across all studies of this book (Section \ref{section03_1_2}). Statistical methods and procedures are described in Section \ref{section03_2}. Then, the approach of linear discriminative learning is introduced (Section \ref{section03_3}). Chapter \ref{chapter04} presents the production study investigating the production of subphonemic durational differences in word-final /s/, while Chapter \ref{chapter05}, relying on the introduction to linear discriminative learning in Section \ref{section03_3}, presents the implementation of such a linear discriminative learning network to account for the nature of the reported subphonemic durational differences. In Chapter \ref{chapter06}, I will present the perception study, which consists of a same-different task to investigate the perceptibility of durational differences in word-final /s/. Chapters \ref{chapter07} and \ref{chapter08} introduce and discuss the two number-decision tasks used to examine the influence of subphonemic durational differences on comprehension. In Chapter \ref{chapter09}, I will bring together the results of the individual studies presented in Chapters \ref{chapter04} to \ref{chapter08} and discuss them in light of the general aims set in the present and the hypotheses given the following chapter. Chapter \ref{chapter10} concludes this book.
