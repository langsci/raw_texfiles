\documentclass[output=paper]{langscibook}
\ChapterDOI{10.5281/zenodo.6811466}
\author{Emmanuelle Labeau\orcid{}\affiliation{Aston University} and Raquel Tola Rego\orcid{}\affiliation{Hackney Education}}
\title[CLIL to make primary pupils click for languages]
      {CLIL to make primary pupils click for languages: Lessons from Hackney}
\abstract{In this paper, we provide a brief overview of the Primary Languages landscape in England and highlight the issues it is facing. We present the best practice case of Hackney Education with special attention to its recent Content and Language Integrated Learning (CLIL) developments. We argue that a CLIL approach has the potential of addressing the challenges of implementing the national Modern Foreign Languages (MFL) entitlement at primary level by reducing the timetabling constraints, expanding the teacher pool, and adapting to children’s cognitive development. Indeed, the delivery of disciplinary contents in another language maximises language input without extra pressure on the timetable. This format can also help address the shortage of qualified language teachers by extending the delivery of primary languages to non-specialist teachers, supported by specialists providing scheme of work, upskilling and contributing to delivery. Finally, the CLIL approach fits the cognitive development -- as identified by the Research in Primary Languages (RiPL) project -- of primary school pupils who learn implicitly, by being immersed in the language and using it.}

\begin{document}
\maketitle 

\section{Introduction}

From September 2014, the primary Modern Foreign Language (MFL) entitlement made Key Stage 2 (KS2)\footnote{Years 3 to 6 of primary education in England.} statutory programmes of second language (L2) study and attainment targets a legal requirement in English primary schools. The range of L2s was extended to include any modern or ancient language, the focus being on “enabling pupils to make substantial progress in one language.” (\citealt[2]{DfE2013key}).

In September 2018, the first cohort made the transition to secondary school and the Research in Primary Languages (RiPL) project reviewed the extent to which the statutory requirement had been met, and with which results. A white paper (\citealt{HolmesMyles2019}) identified a number of challenges:

\begin{quote}
The principal problems in schools relate to time allocation, teacher subject knowledge and language proficiency, limited access to professional development and a lack of shared and agreed understanding of pupil progress at the point of transfer from primary to secondary schools. Given the central importance of subject knowledge to good teaching, the variability of initial teacher training in subject knowledge development is a cause of concern. The current infrequency of Ofsted inspection of primary languages is a further cause of concern. (\citealt[9]{HolmesMyles2019})
\end{quote}

The call to better implement the transition between primary and secondary levels, and to support teacher\footnote{\citegen{GrahamEtAl2017} longitudinal study of learners of French in the transition between primary (years 5 and 6) and secondary (year 7) highlights the importance of “teachers with sufficient pedagogical and linguistic expertise” (\citeyear[954]{GrahamEtAl2017}).} training has been echoed in the \textit{Policy Briefing on Modern Languages Educational Policy in the UK} by \citet{Ayres-BennettCarruthers2019} and \textit{Towards a National Languages Strategy: Education and Skills} (\citealt{British2020}), a joint report published by the British Academy, the Arts and Humanities Research Council, the Association of School and College Leaders, the British Council and Universities UK.

In this paper, we will comment on the implementation of the MFL entitlement through the \textit{Spanish First Initiative} in the East London borough of Hackney (for a general presentation, see \citealt{Baldwin2021}), with special consideration of its recent Content and Language Integrated Learning (CLIL) developments. We will start with a reminder of present challenges for MFL in England and a brief historical overview of the MFL provision in primary education before presenting the \textit{Spanish First Initiative} launched in Hackney in 2014. We will then focus on the emerging CLIL developments in Hackney schools and discuss the advantages of a CLIL approach for addressing the challenges around language teaching in primary schools. 

\section{The national picture}
\subsection{Current challenges for MFL in England}

Anglo-American culture heavily features in Western and world media, thanks to the dominant position of English in enterprise, business, and the creative industries. By contrast, the mainstream portrayal of other languages and cultures remains limited in the UK, propagating ``insular or Eurosceptic attitudes'' that lead to negative perceptions of MFL (\citealt[903]{TaylorMarsden2014}). This perhaps explains the reluctance in reporting home languages in the census. Indeed, only 90 languages were reported for London in the 2011 Census. By contrast, a survey of 896,700 children in London (\citealt{BakerEversley2000}) reported over 300 home languages. Shying away from languages has only been exacerbated by Brexit \citep{Adams2019}. First, leaving the European Union and focusing on business with English-speaking countries fosters a devaluation of language skills that is feared will impact the uptake of MFL, in constant regression since 2004, when languages stopped being compulsory post-14. Enrolment in languages examinations at GCSE (General Certificate of Secondary Education taken by 16 year olds) and A-level (advanced levels for 18 year olds) in England has steadily declined overall, despite the rise of Spanish (see Figures~\ref{fig:labeau:1}--\ref{fig:labeau:2}).

  
\begin{figure}
%\includegraphics[width=\textwidth]{figures/a10LabeauTolaRego-img001.png}
\pgfplotstableread{
GCSE	French	German	Spanish
2002	315071	122053	54050
2003	308342	121679	56790
2004	295970	118014	59588
2005	251706	101466	57731
2006	216481	86680	57561
2007	197774	77671	59121
2008	184813	73318	62015
2009	173604	70195	62029
2010	163283	67084	62580
2011	141472	58382	60773
2012	141044	55083	67791
2013	165127	60649	85954
2014	157175	57513	87836
2015	147356	51986	85217
2016	134420	47756	87136
2017	121095	41762	85184
2018	117925	42509	89577
2019	122803	41222	96811
}{\datatable}
\begin{tikzpicture}
\begin{axis}[
	 sharp plot,
	 ylabel = {Number of candidate entries in England\\(Thousands of entries)},
	 ylabel style={align=center},
     ymin = 0,
     ymax = 350,
     axis lines*=left,
     y filter/.code={\pgfmathparse{#1/1000}\pgfmathresult},
     xtick = data,
     xticklabel style= {/pgf/number format/1000 sep=,rotate=60,anchor=east},
     width=.9\textwidth,
     height=7cm
    ]
    \addplot+[lsMidOrange,mark options={fill=lsMidOrange}] table [y=French] from \datatable;
    \addplot+[lsMidBlue, mark options={fill=lsMidBlue}] table [y=German] from \datatable;
    \addplot+[lsYellow, mark options={fill=lsYellow}] table [y=Spanish] from \datatable;
    \legend{French, German, Spanish};
\end{axis}
\end{tikzpicture}
\caption{{\label{fig:labeau:1}Evolution of GCSE uptake for French, German and Spanish in England between 2002 and 2019 (adapted from \citealt[6]{Churchward2019})}}
\end{figure}

  
\begin{figure}
%\includegraphics[width=\textwidth]{figures/a10LabeauTolaRego-img002.png}
\pgfplotstableread{
A-level	French	German	Spanish
2002	13680	6435	4938
2003	13685	6429	5136
2004	13295	5869	5309
2005	12716	5481	5601
2006	12867	5775	5844
2007	12808	5856	6215
2008	13234	5829	6386
2009	12794	5379	6648
2010	12422	5184	6946
2011	11979	4866	6885
2012	11298	4478	6645
2013	10249	3999	6923
2014	9422	3953	6998
2015	9332	3791	7941
2016	8646	3573	7702
2017	8539	3422	7813
2018	7874	2859	7591
2019	7607	2864	7932
}{\datatable}
\begin{tikzpicture}
\begin{axis}[
	 sharp plot,
	 ylabel = {Number of candidate entries in England\\(Thousands of entries)},
	 ylabel style={align=center},
     ymin = 0,
     ymax = 15,
     axis lines*=left,
     y filter/.code={\pgfmathparse{#1/1000}\pgfmathresult},
     xtick = data,
     xticklabel style= {/pgf/number format/1000 sep=,rotate=60,anchor=east},
     width=.9\textwidth,
     height=7cm
    ]
    \addplot+[lsMidOrange,mark options={fill=lsMidOrange}] table [y=French] from \datatable;
    \addplot+[lsMidBlue, mark options={fill=lsMidBlue}] table [y=German] from \datatable;
    \addplot+[lsYellow, mark options={fill=lsYellow}] table [y=Spanish] from \datatable;
    \legend{French, German, Spanish};
\end{axis}
\end{tikzpicture}
\caption{{\label{fig:labeau:2}Evolution of A-level uptake for French, German and Spanish in England between 2002 and 2019 (adapted from \citealt[18]{Churchward2019})}}
\end{figure}

These downwards trends have caused the closure of university language departments, while compromising the training of local MFL teachers needed to make up for the diminished supply of EU teachers post-Brexit \citep{Savage2019}. Acceptances onto modern language degrees have decreased by 36\% in the last ten years -- from 6,005 in 2011 to 3,830 in 2020, including a 13\% drop last year \citep{Mcquillan2021}. The country is already experiencing a shortage of language teachers:

\begin{quote}
The rates of state-funded secondary school modern language teachers without a ``relevant'' post-A Level qualification in their subject, are higher than the average across all subjects (around 34\% compared to 25\% on average) (\citealt{LongDanechi2021})
\end{quote}

Almost 70\% of state schools and 90\% of independent schools have at least one teacher of languages who is a citizen of a European Union Member State (excluding Ireland) (\citealt{Collen2020}), but restriction of the freedom of movement is likely to affect that external supply of educators, potentially leaving the country unable to meet its requirements for primary languages entitlement. A recent report has shown that schools failed to recruit enough language teachers (only 72\% of posts filled) to support their Ebacc ambitions (\citealt{LongDanechi2021}). Meanwhile, the Language Trends 2021 \citep[4]{Collen2021} reports that “in 53\% of primary schools in England, language teaching was discontinued during the first national lockdown from 23  {March 2020} to late  {June 2020}”. This may be due in part to the lack of in-house linguistic skills and the dependence on peripatetic teachers who were not available during the pandemic.

\subsection{MFL provision in primary education}

Between 1964 and 1974, a large-scale experimental introduction of French in primary schools was attempted \citep{Burstall1974}. The study aimed (i) to investigate the long-term development of pupils’ attitudes towards foreign-language learning; (ii) to establish a potential correlation between pupils’ levels of achievement in French and their attitudes towards foreign-language learning; (iii) to consider the impact of personal variables (e.g. age, sex, socio-economic status, parental support, contact with France, employment perspectives, etc) on achievement in French and attitude towards language learning; (iv) to consider the effect of teachers’ attitudes and expectations on those of their pupils; (v) to establish whether the early introduction of French positively impacted other areas of the curriculum. It concluded that no substantial progress was achieved by children starting French at KS2, although they displayed a more positive attitude towards the language than beginners at KS3. The conclusions of the report were widely debated, given its shortcomings in representing the opinion of all stakeholders and debatable methodological choices \citep{Buckby1976}. Nevertheless, the overwhelmingly ``pessimistic'' report (\citealt{HuntEtAl2005}) was considered insufficient to extend MFL to primary schools, despite study flaws and alleged inconsistencies between the conclusions and the evidence. As a result, government support and funding were withdrawn. However, the efforts of lobbyists for languages managed to turn the tide and, in 2013, the educational system formally ensured the teaching of MFL across KS2 (7--11 years) in primary state schools. The national curriculum (NC) states: ``Learning a foreign language is a liberation from insularity and provides an opening to other cultures'' (\citealt{DfE2013key}, quoted by \citealt[7]{Zefi2021}).

Implementing the decision entails various challenges including (i) the transition between primary and secondary languages, (ii) the lack of suitably qualified teachers and (iii) the constraints of timetabling in an already overcrowded curriculum. Those problems are being successfully addressed in Hackney schools, and we will now see what makes the “Spanish First Initiative” a good practice case.

\section{The Spanish First initiative}

The ``Spanish First'' language project was launched in 2013 by Martin Buck, the Hackney Learning Trust’s (now Hackney Education) former Head of Secondary Service. He was concerned about developing a closer relationship between local primary and secondary schools in Hackney to ensure an easier transition between levels. The initiative pursued three aims:

\begin{enumerate}
\item To implement and promote the teaching of Spanish in all primary and secondary schools in Hackney.
\item To enable pupils to attain high standards in that language.
\item To ensure a coherent and smooth transition from primary to secondary school.
\end{enumerate}

We will discuss in the next few sub-sections how those objectives have been achieved.

\subsection{Implementation and promotion of Spanish in all Hackney primary and secondary schools}

\begin{sloppypar}
When it became a government requirement to teach a language in primary schools in 2014, Spanish was chosen in Hackney on the grounds of its position as a world language deemed easier to acquire for English-speaking learners at the earliest stages than others (such as French)  due to the relative simplicity of its phonology for English speakers.\footnote{\citet[8]{McLelland2018} also mentions that “Spanish was also consistently presented by its advocates as easy to learn”.} A steering committee was set up to oversee and implement the project, comprising representatives from primary schools, a secondary school Headteacher, and Hackney Learning Trust languages specialists Bernadette Clinton and Anushka Sonpal. The objective was to get all primary schools in the borough on board with the introduction of Spanish as part of the KS2 curriculum. The project was launched in seven primary schools with the support of a handful of secondary language teachers, one dedicated primary teacher and the Trust language specialists, and benefited from the statutory requirement to teach a foreign language in Key Stage 2 from  {September 2014}. Virtually all primary schools in Hackney (with the exception of two Jewish schools) started teaching some Spanish from this time. As for secondary schools, they progressively joined the initiative, but continue to offer to this day a range of languages (including French, German, Italian, Mandarin, Arabic, Polish, Portuguese, Turkish, Latin and Modern Hebrew.)\footnote{Most of these languages are spoken in the community and -- alongside English -- belong to the top 10 most spoken languages: Turkish (4.5\%), Polish (1.7\%), Spanish (1.5\%), French (1.4\%), Yiddish (1.3\%), Bengali (1.3\%), Portuguese (1.2\%), Gujarati (0.8\%) and German (0.7\%).} In 2020, 51 out of 58\footnote{Three schools that already teach Modern Hebrew are not part of the project, and several newly created schools do not yet host a Year 6 cohort.} primary schools delivered Spanish to their year 6 pupils.\end{sloppypar}

In Parkwood school, which received the International Spanish School Accreditation in {July 2019}, all classes from Nursery to Year 6 (Y6) are taught Spanish. Early Years Foundation Stage’s (EYFS) and Y1 and Y2 pupils are taught Spanish for 30 minutes a week, Y3 and Y4 for 45 and Y5 and Y6 for 50 minutes.

\subsection{Pupils’ attainment through a wider community initiative}\largerpage[2]

Pupils’ attainment in Spanish is supported by a range of actions. First, resources have been developed in support of the initiative. These include schemes of work guiding delivery and transition data to facilitate continuity between primary and secondary teaching. Second, the quality of teaching is ensured by the employment of peripatetic language specialists, the linguistic upskilling of Primary school teachers and continuous professional development for teachers at all levels. Finally, the scheme is supported by external stakeholders including national bodies such as the British Council and the \textit{Consejería de Educación} (e.g. upskilling of teachers), cultural institutions such as the Cervantes Institute (e.g. bringing a Spanish author to Hackney for a storytelling workshop), and local stakeholders such as the Río Cinema, local Spanish-speaking artists and businesses. The external involvement of these organisations has contributed to making Hackney a Spanish borough, where the language is visible and its learning valued. For instance, some schools have held Spanish Community events such as a Hispanic Week with the contribution of Spanish-speaking families. In addition, parents have been encouraged to join cultural events (e.g. cooking, dancing etc.), and Spanish language classes run by Parkwood Primary school in collaboration with Hackney Adult Education, Hackney Council.

\subsection{Transition from primary to secondary}

The transition from primary to secondary school is achieved through several means. First, data regarding the Spanish language development of each pupil is transferred electronically to their chosen secondary school, which enables language teachers to ensure continuity and avoid the demotivating situation where pupils restart from scratch. The information provided by primary schools includes the length of Spanish study and the level of achievement across Reading, Writing, Speaking and Listening.\footnote{In summer 2021, a transition document listing the knowledge and grammatical terminology that Y6 pupils have covered has been added to the information transmitted to help with the post-Covid recovery curriculum (see Appendix \ref{app:labeau:a}).} The inclusion of data for Spanish highlights the improved status of the language in the borough, where it is considered a core subject like English, Mathematics and Science. Second, stronger personal links are established between primary and secondary schools. For instance, continuous professional development brings together teachers from both levels. The exchanges also take place between pupils as young Language Leaders from secondary schools teach primary pupils, while primary pupils from pilot schools have been teaching secondary partners in years 7 and 8.

Evidence from the Spanish First Initiative in Hackney suggests that cooperation between secondary schools and their feeder schools combined with appropriate teachers’ continuous professional development has already resulted in an improved uptake of languages at GCSE and A-levels, in stark contrast with the national declining trends. Tables~\ref{tab:labeau:1} and~\ref{tab:labeau:2} show the growing uptake in Hackney schools for GCSEs and A-levels in French (the most popular language prior to the Spanish First Initiative) and Spanish. It must be noted that the pupils who have followed the entire Spanish First programme (in Year 3 in 2014) are only starting their GCSEs, so the programme seems to be impactful beyond the direct beneficiaries.

This rise has occurred even though the primary curriculum, designed by Bernadette Clinton in association with primary and secondary teachers in Hackney for the Spanish First Language Initiative, was intentionally not designed around progression towards GCSE examinations. It aims to teach Spanish as ``a language rather than a subject''. As a result, teachers can be creative and design enjoyable lessons using ``useful language'', including topics such as cooking and art.

\begin{quote}
Sue Roberts [former chair of the Spanish First Language Initiative steering group and secondary school representative for Hackney Education] commented that some of the best practice in primary schools includes the use of ``a few Spanish phrases'' in multiple other subject areas, such as P[hysical] E[ducation], which means that all the staff are learning some Spanish alongside their pupils and acknowledging that Spanish is a positive part of their school’s culture. (Spanish first, \citealt[10]{Baldwin2021})
\end{quote}

For instance, at Parkwood School, all members of staff are involved in the teaching of Spanish in different ways. Teachers and teaching assistants may support the specialist teacher in class, reinforce Spanish with daily routines, support the use of Spanish in the lunch hall or run Spanish playground games. This integrated approach can be considered as CLIL, and we will now present it in more detail.


\begin{table}
\caption{{\label{tab:labeau:1}Uptake for GCSEs in Hackney schools (Data from Hackney Learning Trust for 2017--2020, and Bernadette Clinton for 2015, 2016 and 2021)}}
\begin{tabular}{l *7{r}}
\lsptoprule
& {2015} & {2016} & {2017} & {2018} & {2019} & {2020} & {2021}\\
\midrule
Spanish & 525 & 653 & 677 & 808 & 1,009 & 1,213 & 1,319\\
French & 485 & 441 & 347 & 403 & 311 & 301 & 285\\
\lspbottomrule
\end{tabular}
\end{table}


\begin{table}
\caption{\label{tab:labeau:2}Uptake for A-levels in Hackney schools (Data provided by Hackney Learning Trust for 2017--2020, and by Bernadette Clinton for 2021)}
\begin{tabular}{l *5{c}}
\lsptoprule
& {2017\footnote{According to Bernadette Clinton, “A level French and Spanish numbers were extremely small before 2017” (email communication 22/02/2022). For more information on education in Hackney, see \citet{BoyleHumphreys2012}.}} & {2018} & {2019} & {2020} & {2021}\\
\midrule
Spanish & 32 & 40 & 33 & 62 & 80\\
French & 15 & 12 & 10 & 26 & 25\\
\lspbottomrule
\end{tabular}
\end{table}

\section{The CLIL approach}

In this section, we will provide a short description of the CLIL approach, highlight its benefits, and sketch its implementation in Hackney schools.

\subsection{What is CLIL?}

CLIL stands for Content (or Curriculum) and Language Integrated Learning. This model for the teaching of languages was developed in Europe in the mid-1980s, inspired by the Canadian immersion model in which disciplines are taught through the medium of the target language. The approach takes several forms: ``There is neither one CLIL approach nor one theory of CLIL'' \citep[101]{Coyle2008}, but it relies on what \citet{Coyle2006} has called the 4Cs framework~illustrated below (see \figref{fig:labeau:3}).

\begin{figure}
%\includegraphics[width=\textwidth]{figures/a10LabeauTolaRego-img003.png}
\begin{tikzpicture}[>={Triangle[]}]
	\node(culture) at (0,0) {Culture};
	\node[draw](content) at (-2.5,-1.5){\strut Content};
	\node[draw](cognition) at (2.5,-1.5){\strut Cognition};
	\node[draw](communication) at (0,1.5){\strut Communication};
	
	\draw[<->] (content) -- (cognition);
	\draw[<->] (content) to[out=105,in=180] (communication);
	\draw[<->] (communication) to[in=75,out=0] (cognition);
	\draw (content.0) -- (communication.270);
	\draw (communication.270) -- (cognition.180);
	\draw (content.0) -- (culture.270);
	\draw (culture.270) -- (cognition.180);
	\draw (communication.270) -- (culture.90);
	
	\draw[bend left]  (content) to  (communication);
	\draw[bend right]  (cognition) to (communication);
	\draw[bend right]  (content.south) to (cognition.south);
\end{tikzpicture}
\caption{\label{fig:labeau:3}The CLIL 4Cs Framework (\citealt{Coyle2006}, in \citealt[551]{Coyle2007})}
\end{figure}

\textit{Content} refers to the discipline (history, biology…) taught in class. Communication of the discipline is conveyed in another language creating meaning on the content, which needs to be interpreted through \textit{Cognition}. For instance, language supports understanding, critical analysis or memorising of the content. Those processes take place within a \textit{Culture} that includes ways of interacting socially (e.g. the target culture conventions for personal introductions), expected learning behaviours (e.g. classroom conventions) and discipline-based expectations (e.g. the typical structure of a science experiment report).

\subsection{Benefits of a CLIL approach}\largerpage

Research has shown that a CLIL approach entails enhanced learner engagement attributable to the authentic contents covered (\citealt{CoyleEtAl2010, MehistoEtAl2008}). As a result, learners in a CLIL context seem to achieve greater proficiency in the target language \citep{Wesche2002} and, perhaps more surprisingly, perform similarly or better in their first language (\citealt{Alberta2010,Baker2006}) without any loss in the acquisition of contents \citep{Dalton-Puffer2008}. In addition, CLIL learners display a superior intercultural competence and develop more positive attitudes towards other cultures (\citealt{LasagabasterSierra2009}; \citealt{RodriguezPuyal2012}; \citealt{Sudhoff2010}).

\subsection{CLIL in the Spanish First Language Initiative}

The statutory requirement to teach a foreign language at both Key Stages 2 and 3 states the learning aims but it does not lay out how those aims should be achieved or how much time should be dedicated to achieving them (\citealt{DfE2013key}). We will thus contend that Content and Language Integrated Learning -- the teaching of contents through the means of an additional language (e.g. geography through French, physical education [PE] through German \ldots) -- may offer a practical effective solution to the language deficit in primary schools. Indeed, the potential delivery of additional language input by non-language specialists helps counteract the dearth of language teachers in English schools. In addition, the exposure to another language during periods dedicated to other subjects contributes to addressing competition -- often detrimental to languages -- in a crowded curriculum.

The CLIL approach is an exciting development of the overall Spanish First project, and it has been piloted in a few Hackney schools. The CLIL element was introduced in Hackney as the project developed in association with the Spanish Department of Education at the Spanish Embassy (\textit{Consejería de Educación}), which seeks to promote the learning of the Spanish language internationally. To promote the use of Spanish in authentic communication, teachers started conducting Art lessons in Spanish in two primary schools and one secondary school. Art was not considered to be very taxing with respect to contents, especially at primary level where it is an overwhelmingly practical subject, so the language needed to run the classes was expected to remain limited. For instance, at Parkwood Primary school, CLIL Art is taught for one hour a week in all year groups from Y1 to Y6. An extension of the scheme to ten more schools was planned in the school year 20/21, but the plans had to be put on hold due to COVID-19 restrictions. Some schools have adopted an CLIL approach in other subjects too, such as Mathematics. Spanish phrases have also been piloted to give instructions during PE lessons in Reception and Year 1 at Parkwood school, and there are plans to extend the practice to Nursery years from 2021--22. Pupils are also encouraged to use their knowledge outside the classroom. This has translated into school trips to Spain for Year 5 pupils, organising fundraising events related to Spanish, teaching Spanish to Y7 and Y8 pupils in link secondary schools, helping class teachers with daily Spanish revision within school or using Spanish at lunchtime to earn house points (pupils have posters to help them talk in Spanish at lunch, e.g. to give opinions about the food or ask for things like bread or water), chatting to locals in Spain during school visits. Importantly, integrated learning of content and language proves an enjoyable experience, in stark contrast to the demotivating experience of traditional language teaching, as illustrated by the following pupil testimonies:\largerpage

\begin{quote}
I like learning Spanish because it is fun and easy language to learn. When we do Spanish Art, we get to use our Spanish skills to ask for things we need. If I ever go to a Spanish speaking country, I should be able to understand what they are saying. Doing Spanish on its own is fun but doing Art as well makes it even more fun. We use our Art skills to help our Spanish and our Spanish skills to help our Art. I love doing Spanish Art. It’s by far one of my favourite subjects! (Year 6 pupil).
\end{quote}

\begin{quote}
Spanish Art is very enjoyable because I love Art and I love Spanish! We have to say what we need in Spanish. One example is: necesitamos papel, acuarelas y pincel. This has also improved my Spanish skills because now I know how to describe what I need easily. Without help! In Spanish Art we make art based on our topic in English. I find this very interesting because we are learning about the Vikings and in Spanish Art we are creating shield patterns. In conclusion I feel Spanish Art is important for all ages! (Year 6 pupil).
\end{quote}

\noindent In addition, Spanish learners perceive the social importance of language learning:

\begin{quote}
I love being taught Spanish because it will help me in the future. If I went to Spain I would be able to communicate with others easily. Our Spanish lessons are very fun and interactive because we learn Spanish through games and songs. Our Spanish work is cross-curricular so, if we are learning about the Vikings in English, that means we would make stories and comic strips about the Vikings in Spanish. In conclusion, I feel Spanish is a very exciting language to learn. (Year 6 pupil).
\end{quote}

\noindent Schools in Hackney thus teach Spanish as a life skill rather than a subject, and this is perceived as beneficial to the whole learning experience:

\begin{quote}
As the Chair of Governors at Parkwood, I have seen the curriculum develop over the past years. Spanish is now an integral part of how we learn at Parkwood, and this is further underpinned by its constant use within our creative subjects. The Spanish lessons integrate the wider curriculum topics and help drive the creative programme. We observe a variety of lessons across the years and topics. It is always great to see the children learn in Spanish. The value for our children through their whole education (not just primary) is critical. They are not just learning a language in a classroom environment but are using Spanish as an integral part of their learning and education. (Karen Willey, Chair of Governors at Parkwood Primary School).
\end{quote}

Martin Buck, former Head of Secondary Service for Hackney Learning Trust, sees CLIL as a way of ``culturally as well as linguistically broadening a more diverse approach to language teaching'' which also seems to improve achievement throughout the curriculum as evidenced by \citegen{Woodfield2021} experience at KS3. 

Therefore, CLIL could provide an effective framework for the redefinition of language learning in England’s schools, and we will discuss practical ways through which CLIL could help solve the language deficit in UK primary schools and beyond.

\section{How do you solve a problem like language learning in England?}

Since the turn of the century, regular calls have been issued by cultural and political institutions (e.g. The British Academy or the All Party Parliamentary Group for Languages) to overturn the trends for languages in the country. They have also suggested ways of remedying the current situation. In addition, recent concrete initiatives have tried to reverse the decline of language learning in England. While valuable, they fail to address the specific needs of primary languages for which CLIL constitutes a much more relevant approach.

\subsection{Current attempts}

In {December 2017}, the Department for Education (DfE) announced the creation of MFL hubs aiming ``to improve access to high quality modern foreign languages subject teaching, particularly for disadvantaged pupils, drawing on the findings of the Bauckham review\footnote{Ian Bauckham led a review of MFL pedagogy, published by the Teaching Schools Council in 2016, which states the “vast majority” of young people should study a modern foreign language up to the age of 16.} -- building expert hubs to share best practice, targeted in disadvantaged areas.'' (\url{https://ncelp.org/about/background/}). In {Spring 2018}, 9 schools, each leading a hub made of 5 schools, were identified. A National Centre of excellence for Language Pedagogy (NCELP) hosted by the University of York was launched in  {December 2018} and has been producing freely available teaching materials ever since. As valuable as they are, those resources present two major shortcomings for supporting primary languages. First, the proportion of materials targeted at primary ages is limited. At the end of {July 2021}, only 3.44\% of the published resources were aimed at primary school pupils (limited to Year 3 to 6), while the bulk (95.51\%) targeted at KS3 pupils (Years 6 to 9) (see \tabref{tab:labeau:3}).

\begin{table}
\begin{tabular}{lr}
\lsptoprule
Target age range & Published resources\\
\midrule
11--12 (Y7) & 660\\
12--13 (Y8) & 199\\
13--14 (Y9) & 112\\
10--11 (Y6) & 29\\
\phantom{1}9--10 (Y5) & 17\\
\phantom{1}7--8\phantom1 (Y3) & 7\\
\phantom{1}8--9\phantom1 (Y4) & 6\\
14--15 (Y10) & 6\\
15--16 (Y11) & 6\\
16--17 (Y12) & 3\\
17--18 (Y13) & 2\\
\lspbottomrule
\end{tabular}
\caption{\label{tab:labeau:3}Distribution of NCELP teaching resources (accessed 27/07/21)}
\end{table}

A second limitation of the materials resides in their strong grammatical and metalinguistic focus, which reduces their accessibility to non-language specialists who may have to deliver primary languages.

Another governmental initiative is the consultation on new GCSE contents carried out in {Spring 2021}. The proposal, revolving around a list of common vocabulary and the removal of set topics such as families, holidays or hobbies deemed middle-class and alienating for lower socio-economic backgrounds, was widely criticised. The main issues included the failure to acknowledge the pitfalls of identifying most frequent words and of letting word lists take over topics. Indeed, the most frequently used words depend on the medium of communication and the type of corpora taken as reference; they are also susceptible to change (e.g. the vocabulary related to COVID-19 was unknown prior to 2020), which could compromise the sustainability of teaching materials. In addition, topics currently covered in the curriculum are familiar to large proportions of the pupil population, and it may be challenging to identify meaningful and motivating topics to convey the programme.

\subsection{A CLIL approach to primary languages?}

\citegen[9]{HolmesMyles2019} white paper highlighted several challenges to the successful implementation of MFL in primary schools. These included limited time allocation, the lack of qualified teachers, and the transition from primary to secondary. We would argue that an inadequate teaching approach, leading to a lack of motivation and of support, also hinders it. We will contend in the remainder of this paper that a CLIL approach could greatly contribute to solving the issues raised by the MFL entitlement

\subsubsection{Time allocation}

\citegen{HolmesMyles2019} white paper noted the restricted time allocation to languages in the curriculum, ranging from 30 minutes\footnote{Which represents as little as half of the average time spent internationally on languages in primary schools (\citealt{OECD2014}).} to 1 hour a week. In addition to this limited provision, language classes were also subject to cancellation, particularly in Year 6 due to preparations for the SATS (Standard Assessment Tests):

\begin{quote}
Due to emphasis on SATS year 6 tend to have less language teaching however after SATS the intention is to complete more. Some teachers aren’t confident in teaching MFL and give it less priority than other subjects.

Sometimes if the curriculum demands are high, or there is a testing week or other events such as Harvest, language teaching is often dropped for that week. (quotes from \citealt[9]{Collen2021})
\end{quote}

The COVID-19 crisis has further highlighted the fragility of school language delivery as a recent British Council survey reports that more than half of UK primary schools,\footnote{This led one of us to start a YouTube Channel with her son based on the CLIL approach to support his class’ learning of French during the first lockdown. Resources can be found at this address: \url{https://www.youtube.com/channel/UCwySblarKsO0gNoFn1vTNuA}.}  and 40\% of secondary schools did not teach languages during the first lockdown \citep{Bawden2021}. 

Moreover, some pupil groups are removed from language teaching for educational support. Those include Special Education Needs pupils and L2 speakers of English (English as an additional language or EAL pupils) although the MFL classes were likely to benefit them, particularly the latest, given their experience in functioning in another language:

\begin{quote}
There is research evidence that EAL children are at an advantage when it comes to foreign language learning outcomes, and that the language classroom might be the only context in which they are not at a communicative disadvantage when compared to their monolingual peers. Anecdotal evidence suggests, however, that EAL learners are often withdrawn from the language class to receive additional English-language support. This seems to be misguided, when language lessons can play an important role in enhancing EAL children’s metalinguistic understanding and give them confidence. (\citealt[11--12]{HolmesMyles2019})
\end{quote}

By integrating MFL in other parts of the curriculum, schools increase exposure to languages and fill the gap with other countries while widening participation in languages by involving pupils who might usually be removed from language classes for remedial support.

In Parkwood school, Spanish receives the full support of senior management and Headteacher Paul Thomas: Language classes are not subject to cancellation, which vitally contributes to the success of the initiative.

\subsubsection{Teacher subject knowledge and language proficiency}

Another obstacle identified by the white paper was the shortage of suitably qualified language teachers. The problem results from a variety of circumstances. First, a lack of governmental support to MFL teaching has exacerbated the traditional resistance by English speakers towards taking up foreign languages. This attitude has been reinforced by the reluctance of schools to enrol anybody but their strongest pupils on language GCSEs and A-Levels as those ``harder'' topics generating comparatively lower marks were feared to impact school rankings negatively. Even if the primary MFL entitlement attempts to reverse the trend, it encounters a number of conjectural obstacles. {Since 1992}, when the Treaty of Maastricht allowed freedom of movement within the European Union for its citizens, UK schools have become over-reliant on language teachers trained abroad or on EU residents who trained as language teachers in the UK. Post-Brexit, this supply is reducing, and alternative ways of staffing schools need to be found. However, language learning post-14 has steadily declined since 2004 entailing a significant reduction in the numbers of pupils enrolling for language degrees. Moreover, the dual language degrees traditionally pursued by would-be teachers have declined because of the diminution of university applicants with A-Levels in two languages (a result both of the reduced offering in most schools and the reduction of A-levels from 4 to 3 in 2016) and because of the development of combined honours degrees (e.g. business with a language) deemed more conducive to employability. These combined trends make England unable to implement its modest language ambitions.

CLIL relies on carefully scaffolded language, for instance with pictures. It also focuses on content-obligatory language (i.e. the vocabulary, grammatical structures and functional language for specific subjects) and on subject-specific language (e.g. imperative forms in instructions). These features allow for a more predictable use of language, which is likely to enhance the confidence of teachers with a limited command of it.

\subsubsection{Adequate teaching approach}

\citegen{HolmesMyles2019} white paper also highlights that, during primary education, learning progressively shifts from implicit to explicit:

\begin{quote}
Input plays a particularly important role in middle childhood (from ages 6/7 to 11/12). During much of this phase, children learn implicitly, by being immersed in the language and using it. However, for implicit learning to take place, rich and plentiful input, as well as opportunities to use the language meaningfully, are necessary. The balance between implicit learning and more explicit forms of learning starts to shift gradually during middle childhood. (\citealt[10]{HolmesMyles2019})
\end{quote}

Given that CLIL provides a meaningful learning experience with immersion in the target language and culture, it is a particularly suitable way of introducing children to languages and of fostering a positive response to MFL, thereby preserving motivation that is supported by meaningful scaffolded communication:

\begin{quote}
KS2 children are generally highly motivated when starting to learn a language, and are primarily interested in learning languages as a means of face-to-face communication, e.g. for holidays and travel, particularly enjoying encounters with language assistants, link schools abroad etc. (\citealt[11]{HolmesMyles2019})
\end{quote}

\section{How could CLIL be supported?}

In that context, a CLIL approach could help address the deficit in primary teachers’ linguistic knowledge and their limited language proficiency in the L2. By adopting a CLIL approach, non-specialist language teachers could benefit from a confidence boost as they deliver contents they are familiar with, rather than metalinguistic knowledge, and use targeted and limited language resources that can be acquired through continuous professional development (CPD) or twin-teaching with language specialists in the short-term. This approach has proved successful in overcoming language prejudice and upskilling teachers in Hackney, even if the CLIL sessions are delivered by language specialists.\footnote{This does not need to be the case. Judith Woodfield, a geography teacher, delivered CLIL sessions in French \citep{Woodfield2021}.} Extra benefits include a closer cooperation between primary teachers and secondary language specialists that enhances pupils’ progression and maintains the enthusiasm for Spanish awoken at primary level. In parallel, projects led by cultural associations\footnote{\emph{Le niveau bleu} by the Institut Français or the \emph{German and STEM} project developed by the Goethe Institut.} or universities through pan-European projects\footnote{For instance ADiBE or the Erasmus+ project, ``Gamifying CLIL within a mathematical context''.} as well as the actions by the Learning through Languages UK consortium offer teachers ready-to-use materials and training in CLIL methodology. However, such local initiatives remain fragile without governmental endorsement and, longer-term, training in CLIL principles and language learning should become part of initial teaching training to ensure the national recovery of languages. A recent report (\citealt{British2020}) commissioned by the British Academy, the Arts and Humanities Research Council, the Association of School and College Leaders, the British Council and Universities UK recommends the following:

\begin{quote}\sloppy
Universities and colleges, through their Institution-Wide Language Programmes, should facilitate language learning for primary education trainees, to ensure that there is an opportunity for all primary teachers to attain at least the equivalent of Common European Framework of Reference for Languages (CEFR) A1/A2 level in a language. \\\hbox{}\hfill\citep[16]{British2020}\hbox{}
\end{quote}

Partnerships with schools would enhance the trainees’ confidence. Indeed, Parkwood school hosts PGCE students from two London universities. They are all encouraged to observe Spanish lessons and teach Spanish or other language lessons during which they are mentored and advised by a language specialist. In 2020--2021, 6 PGCE students took up the challenge and taught Spanish, Japanese and Korean lessons. Two of them taught CLIL art lessons. This proved a great experience for the trainee teachers and the primary pupils. Teaching MFL during a placement should thus become an expectation rather than a choice, as even those who did not speak other languages found the experience very positive.

Finally, support from individual schools’ management is essential to facilitate teachers’ upskilling as it allows freeing up time and emphasises the importance of the MFL input. This has been crucial to the success of the Spanish First initiative at Hackney, and particularly at Parkwood, its flagship school, but also in the schools on which \citet{Woodfield2021} reports.

\section{Conclusion}

In this paper, we have provided a brief overview of the Primary Languages landscape in England and highlighted the issues it is facing. We have presented the best practice case of Hackney Education with special attention to the recent CLIL developments. We have then argued that a CLIL approach had the potential of addressing the challenges of implementing the national MFL entitlement at primary level by reducing the timetabling constraints, expanding the teacher pool, and adapting to children’s cognitive development. While language teachers with high MFL proficiency have consistently been identified as conducive to success in MFL, the current staff deficit exacerbated by the post-Brexit context prevents all schools in England from benefiting from the support of such highly qualified professionals. Schools with a lower socio-economic profile are disproportionately affected by the situation, which reinforces the elitism traditionally attached to languages in England. 

We have argued that the adoption of a CLIL approach could help address the issue by extending the delivery of languages in schools to non-specialist teachers, supported by specialists providing scheme of work, upskilling and contributing to delivery. In addition, the CLIL approach fits the cognitive development of primary school pupils as identified by research, and makes children click for languages.

Yet, implementing CLIL is not easy. It takes time for specialist and language teachers to plan and develop lessons. It takes commitment from a whole school and a whole community to uphold multilingualism. Above all, it takes support from school leadership to make the time for teachers to collaborate and the place for language skills to develop within the curriculum. Therefore, we hope that decision makers at the highest level will consider the potential benefits of CLIL for language recovery in the country, especially when initiatives such as Spanish First show that an integrated approach succeeds in enthusing youngsters for languages, which traditional methods have failed to achieve.


\appendixsection{2020--2021 post-COVID-19 recovery curriculum transfer form}\label{app:labeau:a}
Hackney Year 6 Spanish Transition Document 2021.\footnote{Adapted from a version produced by \url{https://ascl.org.uk}.} Highlight the items in the left-hand column which you have covered well in KS2 and email to
\texttt{bernadette.clinton@hackney.gov.uk} by 28 May 2021

\bigskip
\noindent
Name of primary school \hrulefill

\bigskip
\noindent
\begin{longtable}{ *2{p{.5\textwidth}} }
\lsptoprule Knowledge \& grammatical terminology & Examples only\\\midrule\endfirsthead
\midrule Knowledge \& grammatical terminology & Examples only\\\midrule\endhead
\endfoot\lspbottomrule\endlastfoot
GRAMMAR	&\\
\tablevspace
Gender

Awareness of gender as a concept and use of terminology masculine \& feminine &	\textit{Un perro} is a masculine noun

\textit{Una tortuga} is a feminine noun\\
\tablevspace
Common letter patterns which show gender
(although not always the case) & Words ending in \textit{-o} masculine

Words ending in \textit{-a} feminine\\
\tablevspace
Nouns \& Determiners

Understand that a determiner introduces a noun, and that it can be an indefinite article, a definite article or a numeral & Indefinite article

\textit{Un gato}, \textit{una araña}, \textit{unos gatos}, \textit{unas arañas}

Definite article

\textit{El gato}, \textit{la serpiente}, \textit{los perros}, \textit{las arañas}

Numeral

\textit{Un gato}, \textit{una serpiente}, \textit{tres perros}, \textit{cinco serpientes}\\
\tablevspace
Rules for capitalisation &	No capitalisation for days/months -- unless they start a sentence\\

Plurals

An ability to recognise \& form nouns in the plural & Plurals of nouns ending in a vowel just add \textit{-s}

\textit{Una casa} -- \textit{dos casas}; \textit{un gato} -- \textit{tres gatos}

Where a noun ends in a consonant add \textit{-es}

\textit{El árbol} -- \textit{los árboles}; \textit{una televisión} -- \textit{unas televisiones}\\

Agreement

Awareness of agreement as a concept, i.e. the matching of words by number \& gender & Noun, determiner \& adjective

\textit{La regla roja y pequeña} 

\textit{Los zapatos negros}

Subject \& verb

The “I” person of the verb in the present tense ends in \textit{-o}

\textit{Hablo}, \textit{como}, \textit{miro}, \textit{bebo}

The “tu” person of the verb in the present tense in -\textit{ar} verbs ends in -\textit{as} \& for -\textit{er} verbs ends in -\textit{es}

\textit{Hablas}, \textit{miras}, \textit{escuchas}

\textit{Comes}, \textit{bebes}\\
\tablevspace
Position \& agreement of adjectives of colour

Know that most adjectives follow the noun and have to agree in number and gender with the noun

Know that some colour adjectives are invariable with m and f nouns & \textit{Una regla roja}, \textit{unos zapatos negros}, \textit{las gomas blancas}

\textit{Una regla verde}, \textit{un lápiz verde}, \textit{unas camisetas verdes}, \textit{unos lápices verdes}\\
\tablevspace
Regular verbs

Be familiar with some parts of regular verbs in present tense & \textit{Cantar} -- \textit{canto}, \textit{cantas}, \textit{cantamos}, \textit{cantan}

\textit{Comer} -- \textit{como}, \textit{comes}, \textit{comemos}, \textit{comen}\\


High frequency regular verbs &	\textit{Hablar}, \textit{cantar}, \textit{bailar}, \textit{nadar}, \textit{saltar}, \textit{trabajar}, \textit{mirar}, \textit{caminar}, \textit{tocar}, \textit{escuchar}, \textit{practicar}, \textit{viajar}, \textit{tomar}, \textit{beber}, \textit{comer}, \textit{leer}\\
\tablevspace
Irregular verbs

Know parts of the verb ser, estar \& understand the difference in usage;
Tener, ir &	\textit{Ser} -- `to be'

\textit{Soy}, \textit{eres}, \textit{es}, \textit{somos}, \textit{sóis}, \textit{son}

\textit{Estar} -- `to be'

\textit{Estoy}, \textit{estás}, \textit{está}, \textit{estamos}, \textit{estáis}, \textit{están}

\textit{Tener} -- `to have'

\textit{Tengo}, \textit{tienes}, \textit{tiene}

\textit{Ir}

\textit{Voy}, \textit{vas}\\

High frequency irregular verbs

Know some in the “I” form & \textit{Jugar} -- \textit{juego}

\textit{Venir} -- \textit{vengo}

\textit{Poder} -- \textit{puedo}

\textit{Hacer} -- \textit{hago}

\textit{Dormir} -- \textit{duermo}\\
\tablevspace
Core structures

Be able to use these in sentences & \textit{Hay}

\textit{No hay}\\
\tablevspace
Opinion phrases	& \textit{Me gusta(n)}, \textit{No me gusta(n)}, \textit{Me encanta(n)}, \textit{Odio}, \textit{Detesto}

\textit{Prefiero}\\
\tablevspace
Opinion adjectives & \textit{Excelente}, \textit{fantástico}, \textit{genial}, \textit{guay}, \textit{interesante}, \textit{fatal}, \textit{fenomenal}, \textit{mal}, \textit{regular}, \textit{aburrido}, \textit{pequeño}, \textit{grande}\\
\tablevspace
Conjunctions &	\textit{Y}, \textit{pero}, \textit{con}, \textit{porque}, \textit{también}, \textit{pues}, \textit{entonces}\\

Knowledge \& grammatical terminology &	Examples only\\
\midrule
Intensifiers &	\textit{Muy}, \textit{bastante}, \textit{más grande}, \textit{más pequeño}, \textit{demasiado}\\
\tablevspace
Prepositions &	\textit{En}, \textit{sobre}, \textit{debajo de}, \textit{enfrente de}, \textit{al lado de}, \textit{a la derecha}, \textit{a la izquierda}, \textit{delante}, \textit{detrás}\\
\midrule
PHONOLOGY	& \\
\tablevspace
Key phonemes &	Vowel sounds, + j, ll, v, rr, ge/gi, ga/go, qu=k, ce/ci, ca/co, z, silent h\\
\tablevspace
Accents -- be aware when used &	\textit{Bufón}, \textit{árbol}, \textit{fantástico}, \textit{fantasía}\\
\midrule
VOCABULARY	& \\

Core phrases

Teacher classroom instructions & Greetings \& polite phrases

\textit{Escucha(d)}, \textit{repeti(d)}, \textit{mira(d)}, \textit{siéntate}, \textit{sentáos}, \textit{de pie}, \textit{levanta(d)}, \textit{levanta(d) la mano}, \textit{silencio}, \textit{abre(d) el libro}, \textit{coge(d) el lápiz}\\

Question words & \textit{¿cómo?}, \textit{¿qué?}, \textit{¿cuántos?}, \textit{¿cuándo?}, \textit{¿cuál?}, \textit{¿quién?}, \textit{¿dónde?  ¿cómo te llamas?}, \textit{¿cómo estás?}, \textit{¿qué tal?}, \textit{¿qué fecha es hoy?}, \textit{¿cuántos años tienes?}, \textit{¿cuándo es tu cumpleaños?}, \textit{¿cuál es tu color favorito?}, \textit{¿quién es la mujer?}, \textit{¿dónde está mi cuaderno?}, \textit{¿qué haces?}, \textit{¿qué hora es?}\\

Basic vocabulary

Days of the week \& Months

Colours

Numbers 0 -- 31 \& dates \& Time

Family members

Animals

Weather and seasons

Geographical features

Modes of transport

Sports and hobbies

Clothing

Parts of the body

Food

The planets

Places in town & \\	
\tablevspace
Knowledge about the Spanish-speaking world \& intercultural understanding	& The geography of Spain -- main cities

Where in the world Spanish is spoken

Spain \& the Hispanic World:

Important festivals \& traditions

Art \& artists

Music \& dance

Food \& menus\\
\tablevspace
Raúl el súper cocinero \& cuaderno & List the chapters you have read with your pupils\\
\tablevspace
Anything else you want to add? & \\

\end{longtable}

\printbibliography[heading=subbibliography,notkeyword=this]
\end{document}
