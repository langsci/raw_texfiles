\documentclass[output=paper]{langscibook}
\ChapterDOI{10.5281/zenodo.6811464}
\author{Rowena Kasprowicz\orcid{}\affiliation{University of Reading} and Karen Roehr-Brackin\orcid{}\affiliation{University of Essex} and Gee Macrory\orcid{}\affiliation{Manchester Metropolitan University}}
\title{Metalinguistic awareness in early foreign language learning}
\abstract{The introduction of foreign language teaching at younger ages in schools around the world has prompted debate about the role of explicit teaching and learning in children. In particular, there is discussion regarding the extent to which form-focused instruction can effectively develop young learners’ metalinguistic awareness and the usefulness of this knowledge for early foreign language learning. Findings to date suggest that contrary to the common assumption that children’s language learning is implicit, primary-school age pupils can and do learn explicitly at least to some extent, provided that certain conditions are met. We present results from a classroom-based, quasi-experimental study with 9 to 11-year old learners of German as a foreign language in primary schools in England. The study explored the effectiveness of input-based explicit grammar instruction for developing learners’ metalinguistic knowledge of nominative and accusative case marking on masculine definite articles in German. Pre- and post-test data indicate that the learners were able to consistently and accurately discuss the grammatical role of the target structures and make use of appropriate metalinguistic terminology when doing so. In contexts such as England, children starting a foreign language at the age of 7 have already been exposed to extensive explicit training in their first language, including in relation to their understanding and use of core metalinguistic terminology. Therefore, the findings highlight the value of harnessing young learners’ existing metalinguistic knowledge when introducing new second language structures.}

\begin{document}
\maketitle

\section{Introduction}
\subsection{Early foreign language learning}

The introduction of languages within England’s primary school curriculum since 2014 has, unsurprisingly, prompted much discussion about appropriate pedagogy for children in the 7 to 11 age range (\citealt{HolmesMyles2019}). In the field of classroom second language (L2) teaching and learning, a long-standing issue is the place of form-focused instruction and the related debate about the role of metalinguistic awareness in child learners.

In this chapter, we present a brief overview of the theoretical conceptualisation of the notion of metalinguistic awareness, followed by a summary of key empirical studies investigating child learners’ metalinguistic abilities. We then present data that speaks to the question of whether explicit grammar instruction can effectively develop young learners’ verbalisable metalinguistic knowledge. We consider to what extent such knowledge is retained over time as well as the issue of young learners’ ability to make use of metalinguistic terminology when talking about the L2. Both quantitative and qualitative results pertaining to these questions are discussed. In the concluding section, we integrate our findings with previous work in order to highlight the level of metalinguistic awareness which primary-school children are able to develop in instructed settings and to point towards the potential benefits of metalinguistic ability in children’s L2 learning. 

\subsection{Theoretical background}

The notion of \textit{metalinguistic awareness} is closely related to the concepts of \textit{metalinguistic knowledge} and \textit{metalinguistic ability}. Metalinguistics refers to linguistic activity which focuses on language as an object in its own right (\citealt{Gombert1992}). Metalinguistic knowledge can be regarded as analysed knowledge about language; it is distinguishable from linguistic knowledge by virtue of its greater level of generality, including knowledge of general principles applicable to more than one language (\citealt{Bialystok2001}). Metalinguistic ability can be defined as “the capacity to use knowledge about language as opposed to the capacity to use language” (\citealt[124]{Bialystok2001}), i.e. linguistic ability, while metalinguistic awareness suggests that the language user’s attention is “focused on the domain of knowledge that describes the explicit properties of language” (\citealt[127]{Bialystok2001}).

In applied linguistics research concerned with instructed L2 learning, metalinguistic awareness is typically conceptualised in terms of \textit{explicit knowledge about language}. In this research tradition, a distinction between explicit knowledge on the one hand and implicit knowledge on the other hand is made. Explicit knowledge “is knowledge about language and about the uses to which language can be put” (\citealt[229]{Ellis2004}). It is knowledge an individual is consciously aware of, and, memory permitting, is potentially able to articulate (\citealt{Ellis2004}). Explicit knowledge is represented declaratively (\citealt{Hulstijn2005}) and is subject to controlled processing (\citealt{EllisEtAl2009}).

Knowledge of technical \textit{metalinguistic terminology} or \textit{metalanguage} such as ``subject'', ``co-ordinating conjunction'', ``accusative case'' or ``intransitive verb'' is not seen as an essential component of metalinguistic awareness, knowledge or ability, nor indeed of explicit knowledge about language, though it is often acquired in parallel (\citealt{Ellis2004}). Accordingly, some researchers make a point of distinguishing between analysed knowledge on the one hand and knowledge of metalanguage on the other hand (\citealt{Gutierrez2016}), equating analysed knowledge with knowledge that is available to consciousness but not necessarily for verbal report. Conversely, knowledge of metalanguage comprises knowledge of technical terminology and entails the ability to verbalise analysed knowledge. Thus, analysed knowledge can be held independently of knowledge of metalanguage, in the sense that learners may be aware of a grammatical systematicity and may be able to deliberately draw on this analysed knowledge to inform their language use, but may be unable to articulate or describe it. For instance, a speaker may know that the sentence \textit{Jane rarely goes to the zoo} is acceptable while the sentence \textit{Jane goes rarely to the zoo} is dispreferred in English, but they may not be able to express the reasons for this. Of course, in instructed settings, metalinguistic labels such as ``subject'', ``object'' or ``pronoun'' are often taught alongside the concepts they denote, although this is arguably more common with cognitively mature than with young learners. It is immediately obvious that knowledge of metalinguistic terminology is useful for the purpose of description, explanation and hence the verbalisation of metalinguistic knowledge. In other words, if a learner is to be made aware of a pattern, or if they have discovered a regularity themselves, the existence of a commonly understood label to name the pattern or regularity is of practical benefit.

Metalinguistic awareness in the sense of explicit knowledge about language can be measured by means of tests and/or verbal reports. As quantitative instruments, tests allow for relatively fast measurement in a single administration session, and measures that are suitable for primary school-age children are available (e.g. \citealt{Hakes1980,PintoEtAl1999,Tellier2013}). Verbal reports as evidence of metalinguistic awareness can take the form of task-concurrent think-aloud protocols or retrospective stimulated recall protocols. Both approaches ask learners to verbalise any patterns, systematicities or rules they have noticed in the input during an experimental treatment or while performing a task. Responses are then analysed in order to establish the learner’s level of awareness and/or their use of metalanguage.

\subsection{Empirical background}

At first glance, one might wish to discount more qualitative verbal reports for use with young learners, since children may lack the terminology to articulate their metalinguistic awareness. However, at least two studies have successfully used guided group discussions and interviews to investigate the metalinguistic awareness of young learners in this way (\citealt{AmmarEtAl2010,BouffardSarkar2008}).

\citet{BouffardSarkar2008} trained 8 to 9-year-old first language (L1) English children to notice and repair L2 French errors, identify the language features involved, negotiate form and perform grammatical analyses. The setting for the study was a French immersion programme in Canada, where English-speaking children are educated in a French environment from ages 5 to 6 onwards. According to the researchers, children typically achieve good levels of reading and listening comprehension, but their productive skills remain weaker. As a possible remedy to this situation, the researchers trialled a form-focused approach aimed at improving children’s oral and written language development via prior enhancement of their metalinguistic awareness.

Children from two intact classes took part in a three-stage training programme over three months. First, communicative classroom activities were video-re\-cord\-ed on 23 occasions. Corrective feedback, mostly in the form of elicitation, metalinguistic clues and repetition, was provided for lexical, phonological, grammatical errors and errors that could be directly attributed to transfer in order to prompt self-repair. Second, the footage was edited to obtain 287 isolated clips of error-feedback-repair sequences, amounting to 167 minutes in total. Third, children were audio-recorded over 28 sessions in which they were prompted to analyse the videotaped error sequences under teacher guidance. Each session involved four to seven children, with a total of 38 participants. The aim was “to push participants to achieve grammatical analysis through collaborative discussion” (\citealt[8]{BouffardSarkar2008}).

The results demonstrated an improvement over time in children’s metalinguistic abilities regarding the discussion of errors, that is, children gradually became more adept at labelling and analysing errors featuring in the taped episodes. Lexical errors often occurred when children used light verbs such as \textit{faire} `make, do' instead of choosing more precise alternatives. In the teacher-led discussion sessions, the children proved able to use strategies to enhance their metalinguistic awareness. They acknowledged differences between English and French and were able to attend to the negotiation of form. Grammatical errors were analysed in terms of noun phrase and verb phrase errors. Children demonstrated knowledge of the gender of French nouns and determiners, and they were able to pinpoint the absence of grammatical gender in English. Verb phrase errors proved to be more challenging. Towards the end of the data collection period, instances of successful metalinguistic analysis began to appear. In the area of transfer, lexical mapping errors occurred when an L1 word corresponded to more than one L2 word, e.g. ``know'' and \textit{savoir/connaître}. With prompting, children were able to compare L1 and L2 and thus showed facility in identifying the likely cause of such errors. Word order errors proved challenging and required teacher guidance in order to be identified and labelled with appropriate metalinguistic terminology. 

In sum, the findings suggested three consecutive phases of metalinguistic development. In the earliest phase, children were able to correct errors, but required extensive prompting to achieve identification. In the second phase, the young learners began to make metalinguistic guesses and tried to use metalinguistic terminology. These strategies led to the realisation that error analysis was possible. Negotiation of form came more easily, and children moved into the final stage, in which they used metalinguistic terminology appropriately. They were able to identify, correct and analyse errors and occasionally were able to propose explanations. Thus, over the three months of the study, the teacher-led small-group discussions enabled the children not only to develop considerable metalinguistic awareness, but also to articulate it. 

More recently, \citet{BellEtAl2020} investigated young learners’ spontaneous use of cross-linguistic connections without teacher intervention in two groups of francophone children, also in a Canadian context. They described these as verbalised, metalinguistic reflections comparing two or more languages. Their study was carried out with nine primary (aged 11 to 12) and 16 secondary (aged 15 to 16) students who were following a regular L2 English programme comprising 1--2 hours of instruction per week. The task required the students, working in dyads, to edit an English paragraph containing 19 errors whilst justifying each change. A number of linguistic features were chosen that differ between English and French, such as adverb placement and choice of preposition. Learners’ discussions were analysed for cross-linguistic connections, operationalised as justifications including references to the L1. For example, one participant noted that ``There's no S in \textit{their} because it’s like in French \textit{leur face}'' (\citealt[103]{BellEtAl2020}). In fact, out of a total of 195 metalinguistic reflections, only 28 were categorised as cross-linguistic in nature, leading the researchers to conclude that participants infrequently used cross-linguistic connections when completing a metalinguistic task in the L2. Of the 28 episodes, 15 contained no rule and of the 13 that did, only six included a verbalisation that demonstrated the participants were aware they were contrasting English and French. Primary-school students made fewer cross-linguistic connections than secondary-school students, which was attributed to the greater focus on explicit grammatical knowledge in the secondary school system. The researchers nevertheless argue that explicit knowledge about the L1 has a potentially important role to play in the L2 classroom and suggest that encouraging the use of such knowledge may demonstrate to learners the value of understanding their L1 system. The nature of the particular grammatical feature may, however, be pertinent (\citealt{McManus2019}).

Despite the comparative approach taken in the above studies, many teachers still tend to avoid the L1 in the classroom (\citealt{HorstEtAl2010}), possibly because of the perceived failure of approaches informed by classic contrastive analysis (\citealt{BellEtAl2020}). However, as recent research demonstrates, interest has grown in a more holistic approach that seeks to draw on learners’ ability to reflect on language, including the relationship between L1 and L2 (see \citealt{HallCook2012}).

\citet{HorstEtAl2010} developed a series of cross-linguistic awareness activities for 48 francophone learners of English in Québec, Canada, based on a range of linguistic features, including ones that tend to be problematic for French-speaking learners, such as the possessives ``his'' and ``her''. Having demonstrated that many of the young learners were able to compare the two languages and note useful points of similarities and differences, the authors concluded that raising cross-linguistic awareness is “a viable pedagogy with demonstrable advantages for learners” (\citealt[347]{HorstEtAl2010}). \citet{WhiteEtAl2007} also investigated the acquisition of English possessive determiners, albeit with slightly older learners aged 13 to 14, and with French or Spanish/Catalan language backgrounds. The research consisted of two parallel studies carried out in schools in Québec and Catalonia, Spain. The instructional treatment involved providing learners with two types of explicit information about \textit{his} and \textit{her}. They were given a rule of thumb (``whose \ldots\,is it?'') and then a comparison between possessive determiners in English and in their first languages (French or Catalan/Spanish). The five-week intervention showed that the students were able to verbalise their choices, using metalinguistic terminology. The researchers found that not only was explicit instruction effective in developing the learners’ ability to use and understand the possessive determiners, but that this was the case for both language backgrounds. In a Scottish context, \citet{Kanaki2020} carried out an ethnographic study of 53 monolingual English-speaking primary-school children aged 10 to 11 who were learning French. She found that the young participants were able to express reflections on language analysis and on their own learning strategies but were more likely to focus on similarities between the two languages than differences.

In summary, findings to date suggest that even young learners are capable of metalinguistic reflection involving a comparative analysis of L1 and L2. However, research is still somewhat limited in terms of the languages and the learning settings being investigated, since most studies have focused on English and French in a Canadian context, where both languages have an equivalent status. The study reported in this chapter was conducted in a UK context with German as the target L2, i.e. a foreign language that is not present in the children’s everyday lives.

\section{Research issues and methodology}

The data presented below is taken from a wider study (\citealt{Hanan2015,KasprowiczMarsden2018}) investigating the effectiveness of explicit grammar instruction for young L1 English learners of L2 German in a primary-school context in England. In the present chapter, the following research questions are addressed:

\begin{enumerate}\sloppy
\item To what extent is explicit grammar instruction effective in developing young learners’ verbalisable metalinguistic knowledge?
\item  To what extent is this knowledge durable over time?
\item To what extent are learners able to make use of metalanguage (i.e. technical terminology) when talking about the L2?
\end{enumerate}

Two types of input-based explicit grammar practice, that is, task-essential form-meaning connection practice versus task-essential form-spotting practice, were investigated to establish their effectiveness for learning definite article case marking in German (see \citealt{KasprowiczMarsden2018} for detailed analysis and discussion) and for developing learners’ metalinguistic knowledge related to this grammatical structure, which is the focus of the current chapter. The target structure -- nominative (\textit{der} `the-\textsc{nom}') and accusative (\textit{den} `the-\textsc{acc}') case-marking for masculine definite articles in German -- can be problematic for L1 English learners due to their tendency to rely on word order (the more reliable cue in English) when interpreting and assigning grammatical roles (subject/object) in German sentences (\citealt{CulmanEtAl2009,Jackson2007,VanPattenBorst2012}).

A classroom-based experimental study involving pre-test (week 1), a five-week teaching intervention, post-test (week 7), and delayed post-test (week 16) was conducted across three primary schools in England. Participants (aged 9 to 11) from four classes were randomly assigned to two experimental groups who received explicit information on the target structure followed by either task-es\-sen\-tial practice requiring attention to the target structure’s form-meaning connection (TE-FM group, $n = 45$) in line with \citegen{VanPatten2002} referential activities, or enriched input practice requiring learners to spot the target form only (TE-F group, \textit{n} = 41), in line with \citet{ReindersEllis2009}.

The explicit information consisted of: 

\begin{enumerate}
\item  a short explanation of the terms ``subject'' and ``object'' with example sentences in the learners’ L1 English (in weeks 1 and 2 only)
\item  an explanation of the function of the masculine articles \textit{der} and \textit{den}, alongside example sentences in L2 German (in weeks 4 and 5, it was also highlighted that the feminine and neuter articles do not change in this context), and
\item  a reminder of the importance of paying attention to the articles in German sentences due to the flexibility of word order, alongside example object-verb-subject sentences.
\end{enumerate}

As exemplified in \figref{fig:kasprowicz:1}, the TE-FM activities were designed in such a way that learners were required to make the connection between the target form and its meaning in order to correctly complete the activity. For example, both nouns are missing from the sentence; therefore, in order to identify the correct position for each noun, the learner had to notice the case marking on each article and the corresponding meaning conveyed (\textit{der} `the-\textsc{nom}' indicating subject; \textit{den} `the-\textsc{acc}' indicating object). Further, the word order was manipulated (items varied between subject-verb-object and object-verb-subject order) to ensure that learners were unable to rely on a default ``first noun is the subject'' strategy (\citealt{VanPatten2002}).



\begin{figure}
\fbox{\parbox{\linewidth-4\fboxsep}{\raggedright Decide which noun fits in each gap, so that the sentence matches the picture.\medskip\\
\includegraphics[width=\linewidth]{figures/a9Kasprowiczetal-img001.png}}}
\caption{\label{fig:kasprowicz:1}Example item from TE-FM intervention activity}
\end{figure}

In contrast, the aim of the TE-F activities was to draw learners’ attention to the grammatical form only and did not push learners to make the additional step of connecting form with meaning. The TE-F activities provided enriched input (i.e. exemplars of the target structure); however, the primary focus was vocabulary practice. For example, as shown in \figref{fig:kasprowicz:2}, only one noun is missing from the sentence, and the learner must choose which of the two nouns provided completes the sentence. One of the nouns appears in the corresponding picture and one does not. Learners then completed the ``form spotting'' task, in which they had to identify the target structure within each sentence.


  
\begin{figure}
\fbox{\parbox{\linewidth-4\fboxsep}{\raggedright Decide which noun fits in the gap, so that the sentence matches the picture. Then circle the different words for \textit{the}.\medskip\\
\includegraphics[width=\linewidth]{figures/a9Kasprowiczetal-img002.png}}}
\caption{\label{fig:kasprowicz:2}Example item from TE-F intervention activity}
\end{figure}



Three intact classes formed a test-only control group ($n = 52$), who completed the pre- and immediate post-test, but continued with their normal German lessons during the intervention period, including practice of the vocabulary used in the test and intervention materials, but no explicit instruction on the target structure. 

A battery of five outcome measures was developed to test learners’ written and oral receptive and productive knowledge of the target structure (see \citealt{Hanan2015} and \citealt{KasprowiczMarsden2018} for detailed discussion of these tests and associated results). In addition, a one-to-one think-aloud Sentence Reconstruction task was developed to measure the extent to which the learners were able to verbalise their knowledge and understanding of the target structure. The quantitative and qualitative results from this measure are the focus here.

The Sentence Reconstruction task was designed to measure learners’ ability to make use of and verbalise the target grammatical rules, i.e. their metalinguistic knowledge. The task was completed one-to-one with the researcher and consisted of three items. For each item, participants were presented with a picture and five words on individual pieces of paper. Participants were asked to create a sentence to match the picture by placing the words into the correct order. Each sentence was a simple noun-verb-noun construction, as shown in \figref{fig:kasprowicz:3}. Participants were asked to explain why they had chosen that order for the words, with particular emphasis on the positioning of the articles.
  
\begin{figure}
\includegraphics[width=.8\textwidth]{figures/a9Kasprowiczetal-img003.png}
\caption{\label{fig:kasprowicz:3}Example item from Sentence Reconstruction task}
\end{figure}

Item 1 included a masculine subject and a masculine object to test learners’ knowledge of the nominative (\textit{der}) and accusative (\textit{den}) case-marked masculine articles. Item 2 included a masculine subject and a feminine or neuter object, and item 3 included a feminine or neuter subject and a masculine object. In German, feminine and neuter articles do not change between the nominative and accusative cases; therefore, items 2 and 3 gave the opportunity for learners to demonstrate metalinguistic reasoning by applying their knowledge of the masculine articles to work out the grammatical roles of nouns in sentences containing a non-case-marked feminine or neuter article.

Participants’ explanations were scored; one point was awarded for correctly explaining the function and position of each article within an item (e.g. for the item in \figref{fig:kasprowicz:3}, the explanation “\textit{der} is placed in front of \textit{Mann} because the man is doing the writing” would receive one point). Across the three items within the task, a total of six points was available. The data were non-normally distributed; therefore, non-parametric statistical tests were employed. Friedman’s ANOVA followed by pairwise comparisons with Bonferroni correction was used to analyse changes in the TE-FM and TE-F groups’ performance over time. A Kruskal-Wallis test followed by pairwise comparisons with Bonferroni correction was used to compare the performance of the three groups at pre- and post-test. The control group did not complete the delayed post-test; therefore, a Mann Whitney U-test compared the performance of the TE-FM and TE-F groups only. Cohen’s~$d$ effect size was calculated to indicate the magnitude of the observed effects and interpreted using \citegen{PlonskyOswald2014} field-specific benchmarks for between-group contrasts (small, $d = 0.40$; medium, $d = 0.70$; large, $d = 1.00$) and within-group contrasts (small, $d = 0.60$; medium, $d = 1.00$; large, $d = 1.40$).

Additionally, data-driven thematic coding provided a more in-depth, qualitative analysis of the content of participants’ explanations. Participants were not required to use metalinguistic terminology within their explanations; however, as can be seen from the results presented below, many participants were able to utilise relevant terminology. 

\section{Results}
\subsection{Quantitative analysis of the sentence reconstruction task}

As reported in \citet{Hanan2015} and \citet{KasprowiczMarsden2018}, there was a significant change over time in both the TE-FM ($\chi^2 (2) = 65.790,\allowbreak p = 0.001$) and TE-F ($\chi^2(2) = 59.842,\allowbreak p = 0.001$) groups’ scores.

Pairwise comparisons revealed a significant increase in scores between pre- and post-test for both groups (TE-FM, $ p = 0.001,\allowbreak d =5.12$; TE-F, $p = 0.001,\allowbreak d = 4.28$), reflecting substantial improvement in the learners’ ability to provide accurate explanations relating to the function and position of the target structure. Notably, however, a significant decrease in both groups’ performance was observed between post- and delayed post-test (TE-FM, $ p=0.015,\allowbreak d = -0.69$; TE-F, $p=0.015,\allowbreak d=-0.74$), suggesting a decline in their ability to articulate the target grammatical rules, although for both groups performance remained significantly higher at delayed post-test than at pre-test (TE-F, M, $p=0.001,\allowbreak d = 2.71$; TE-F, $p = 0.001,\allowbreak d =3.20$).


\begin{table}
\begin{tabular}{lrrrrr}
\lsptoprule
&  \multicolumn{5}{c}{Pre-test}\\\cmidrule(lr){2-6}
& \textit{n} & \textit{M} & \textit{SD} & \textit{Min} & \textit{Max}\\
\midrule
 TE-FM & 45 & 0.07 & 0.33 & 0 & 2\\
 TE-F & 41 & 0 & 0 & 0 & 0\\
 Control & 52 & 0 & 0 & 0 & 0\\
 \midrule
&  \multicolumn{5}{c}{Post-test}\\\cmidrule(lr){2-6}
& \textit{n} & \textit{M} & \textit{M} & \textit{M} & \textit{M}\\
\midrule
 TE-FM & 45 & 4.78 & 4.78 & 4.78 & 4.78\\
 TE-F & 41 & 4.32 & 4.32 & 4.32 & 4.32\\
 Control & 52 & 0.04 & 0.04 & 0.04 & 0.04\\
 \midrule
& \multicolumn{5}{c}{Delayed}\\\cmidrule(lr){2-6}
& \textit{n} & \textit{M} & \textit{M} & \textit{M} & \textit{M}\\
\midrule
 TE-FM & 45 & 3.46 & 3.46 & 3.46 & 3.46\\
 TE-F & 41 & 2.9 & 2.9 & 2.9 & 2.9\\
 Control & 52 & {}- & {}- & {}- & {}-\\
\lspbottomrule
\end{tabular}
\caption{Descriptive statistics\label{tab:kasprowicz:1}}
\end{table}

In terms of between-group comparisons, there was no significant difference between the TE-FM, TE-F and control groups’ performance at pre-test ($H(2) = 3.90,\allowbreak p = 0.143$). Examination of the descriptive statistics revealed that none of the groups were able to provide accurate explanations at this time point. At post-test, however, a significant difference between the three groups was observed, which pairwise comparisons indicated was due to both the TE-FM ($p = 0.001,\allowbreak d = 4.56$) and the TE-F ($p = 0.001,\allowbreak d = 3.17$) groups significantly outperforming the control group. At delayed post-test, there was no significant difference between the TE-FM and TE-F groups’ scores ($U = 748.000,\allowbreak z = -1.527,\allowbreak p = 0.127,\allowbreak d = 0.27$), indicating that there was an equivalent decline in both groups’ scores at this time point.

To examine the proportion of learners in the TE-FM and TE-F groups who were able to provide correct explanations at post- and delayed post-test, the learners were divided into sub-groups according to their score on this task: ``High-Scorers'' who scored 5 or more out of 6, ``Mid-Scorers'' who scored 3 or 4 out of 6 and ``Low-Scorers'', who scored 2 or less. 

At post-test, approximately two thirds of the learners in both the TE-FM and TE-F groups were able to consistently provide correct explanations relating to the position and function of the articles in the three items within the task (``High-Scorers''). A further 31\% in the TE-FM group and 20\% in the TE-F group were able to provide correct explanations but showed some inconsistency in their responses (``Mid-Scorers'' who scored 3 or 4 out of 6, indicating insufficient and/or incorrect explanation(s) for at least one of the test items). Additionally, a small number of learners (TE-FM, 9\%; TE-F, 20\%) demonstrated limited verbalisable knowledge of the target structures (``Low-Scorers''). In contrast, at delayed post-test, there was a decline in the proportion of ``High-Scorers'' on this task (TE-FM, 42\%; TE-F, 39\%) and a corresponding increase in the proportion of ``Low-Scorers'' in both groups (TE-FM, 33\%; TE-F, 41\%).   

In terms of the nature of learners’ responses on this task, the explanations were examined to explore the extent to which correct explanations included the use of metalinguistic terminology (i.e. relevant grammatical terms such as ``subject'' and ``object''). 

\vfill
\begin{table}[H]
\begin{tabular}{llrrrr}
\lsptoprule
&  &  &  & \multicolumn{2}{c}{Delayed}\\
&  & \multicolumn{2}{c}{Post-test} & \multicolumn{2}{c}{post-test}\\\cmidrule(lr){3-4}\cmidrule(lr){5-6}
Group & Sub-group & \textit{n} & \% & \textit{n} & \%\\
\midrule
TE-FM
(\textit{n} = 45) & High scorers (>5) & 27 & 60 & 19 & 42\\
& Mid-scorers (3–4) & 14 & 31 & 11 & 25\\
& Low scorers (<2) & 4 & 9 & 15 & 33\\
TE-F
(\textit{n} = 41) & High scorers (>5) & 25 & 60 & 8 & 20\\
& Mid-scorers (3–4) & 8 & 20 & 16 & 39\\
& Low scorers (<2) & 8 & 20 & 17 & 41\\
\lspbottomrule
\end{tabular}
\caption{Proportion of ``High-Scorers'', ``Mid-Scorers'', and ``Low-Scorers''}
\label{tab:kasprowicz:2}
\end{table}

\begin{table}[H]
\begin{tabular}{lrrrrrrrrrr}
\lsptoprule
Group & \multicolumn{10}{c}{Post-test}\\\cmidrule(lr){2-11}
& \multicolumn{2}{c}{Subject} & \multicolumn{2}{c}{Object} & \multicolumn{2}{c}{Masc.} & \multicolumn{2}{c}{Fem.} & \multicolumn{2}{c}{Neut.}\\
\cmidrule(lr){2-3}\cmidrule(lr){4-5}\cmidrule(lr){6-7}\cmidrule(lr){8-9}\cmidrule(lr){10-11}
& \textit{n} & \% & \textit{n} & \% & \textit{n} & \% & \textit{n} & \% & \textit{n} & \%\\
\midrule
TE-FM (\textit{n} = 45) & 40 & 89 & 39 & 87 & 21 & 47 & 28 & 62 & 22 & 49\\
TE-F (\textit{n} = 41) & 32 & 78 & 33 & 80 & 25 & 61 & 26 & 63 & 15 & 37\\
\midrule
 & \multicolumn{10}{c}{Delayed post-test}\\\cmidrule(lr){2-11}
& \multicolumn{2}{c}{Subject} & \multicolumn{2}{c}{Object} & \multicolumn{2}{c}{Masc.} & \multicolumn{2}{c}{Fem.} & \multicolumn{2}{c}{Neut.}\\
\cmidrule(lr){2-3}\cmidrule(lr){4-5}\cmidrule(lr){6-7}\cmidrule(lr){8-9}\cmidrule(lr){10-11}
& \textit{n} & \% & \textit{n} & \% & \textit{n} & \% & \textit{n} & \% & \textit{n} & \%\\
\midrule
TE-FM (\textit{n} = 45) & 25 & 56 & 30 & 67 & 21 & 47 & 18 & 40 & 17 & 38\\
TE-F (\textit{n} = 41) & 22 & 54 & 22 & 54 & 24 & 59 & 16 & 39 & 16 & 39\\
\lspbottomrule
\end{tabular}
\caption{Proportion of learners correctly employing grammatical terminology\label{tab:kasprowicz:3}}
\end{table}
\vfill\pagebreak

As detailed in \tabref{tab:kasprowicz:3}, when learners made use of grammatical terminology (i.e. technical metalanguage) within their explanations, these tended to centre on terms related to describing the grammatical function and/or grammatical gender of the articles and nouns within each sentence. For each grammatical term, one correct usage was counted per learner at each time point. At post-test, the majority of learners in both the TE-FM (89\%) and TE-F (78\%) groups were able to utilise the terms ``subject'' and ``object'' correctly on at least one occasion during completion of the task. In addition, up to two thirds of the learners were able to correctly make use of at least one relevant term related to grammatical gender (masculine, feminine, and/or neuter). At delayed post-test, a drop in the use of grammatical terminology was observed, although just over half of the participants were still correctly utilising the terms ``subject'' (TE-FM, 56\%; TE-F, 54\%) and ``object'' (TE-FM, 67\%; TE-F, 54\%). It is important to note that some learners were still able to provide correct explanations relating to the target structure without the use of grammatical terminology, as detailed below.

\subsection{Qualitative analysis of participants’ explanations}\largerpage[2]

In order to provide a complementary picture of learners’ verbalisable metalinguistic knowledge of the target structure, responses on the Sentence Reconstruction task were analysed thematically. The findings at each time point are presented in turn to illustrate changes in learners’ verbalisable knowledge between pre-, post- and delayed post-test.

\subsubsection{Explanations provided at pre-test}

At pre-test, there were no instances of learners discussing the function of the target structure (\textit{der} and \textit{den}) in assigning grammatical roles (subject and object respectively) for masculine nouns in German sentences. This was as expected, given that the learners had received no instruction on this grammatical structure prior to the study. Rather, the learners utilised a range of strategies to work out and explain the word order chosen for each sentence. Often this would be based on translation into English, with many learners able to recognise the role of \textit{der}, \textit{den}, \textit{die}, and \textit{das} as articles, although there were no instances of learners using the grammatical term ``article'' in their explanations:

\begin{quote}
  \textbf{R:} And why did you put der [the-\textsc{nom}] with Mann [man] and   den [the-\textsc{acc}] with Brief [letter]?   

  \textbf{P:} Because Brief [letter] means letter and in English we would   say the letter or a letter so den [the-\textsc{acc}] would go next to it. 

(Participant 34, TE-F, School 2)
\end{quote}

\begin{sloppypar}
At pre-test, learners’ explanations also tended to centre on discussion of the grammatical gender of the nouns in each sentence. In many cases, learners were able to utilise appropriate metalinguistic terminology (masculine, feminine, neuter) in their responses:
\end{sloppypar}

\begin{quote}
  \textbf{R:} OK so we’ve got \textit{\ul{der}} Hund [the-\textsc{nom} dog]. Why did you   decide to put those two next to each other?

  \textbf{P:} Because (.) the dog is (.) masculine and (.) \textit{\ul{die}} Katze [the-\textsc{nom/acc} cat] is feminine. 

(Participant 133, Control, School 2)
\end{quote}

In other cases, learners utilised more colloquial terms (e.g. male/female) to express their understanding of grammatical gender, whilst some learners associated grammatical gender with the biological gender of the associated referent:

\begin{quote}
  \textbf{P:} Because um (.) \textit{\ul{der}} [the-\textsc{nom}] wouldn’t go with Frau   [woman] because (.) \textit{\ul{der}} [the-\textsc{nom}] is for male and (.) \textit{\ul{die}} [the-\textsc{nom/acc}] is for female. (Participant 45, TE-FM, School 1)

  \textbf{P:} \ldots\,And I knew \textit{\ul{die}} [the-\textsc{nom/acc}] goes with woman   because um (.) die [the-\textsc{nom/acc}] goes with (.) woman (.) no,   yeah like woman and girls. And \textit{\ul{der}} [the-\textsc{nom}] goes with boys   and men. (Participant 80, TE-F, School 2)
\end{quote}

Another common explanation related to the animacy of the referent involved:

\begin{quote}
   \textbf{R:} Yes and why did you put \textit{\ul{den}} [the-\textsc{acc}] with Frisbee   [frisbee]?

  \textbf{P:} Because it’s like (.) with the letter it’s like a thing. And then   Jungen [boy] is a boy. And \textit{\ul{das}} [the-\textsc{nom/acc}] goes with   that. (Participant 97, Control, School 3)
\end{quote}

Finally, there were also instances of learners relying on guesswork or intuition (“it sounds right”) when deciding on the position of the articles within each sentence. As demonstrated by the extracts and observations above, at pre-test learners across all three groups did not express any awareness of the function of \textit{der} and \textit{den} in assigning grammatical roles within sentences. This finding is consistent with the learners’ baseline performance on the receptive and productive outcome measures, indicating that they had no knowledge of these structures prior to the intervention. 

\subsubsection{Explanations provided at post-test}

As reflected in the quantitative analysis above, a substantial change was observed in the explanations given by many of the TE-FM and TE-F learners at post-test. The majority of learners in these groups expressed a clear understanding of the function of \textit{der} and \textit{den} in assigning subject and object roles, respectively. As shown in \tabref{tab:kasprowicz:3} above, many of the learners were able to utilise appropriate metalanguage in their explanations:

\begin{quote}
  \textbf{P:} Because I knew that \textit{\ul{der}} [the-\textsc{nom}] is for the subject of the   sentence, the thing that does the action. And \textit{\ul{den}} [the-\textsc{acc}] is   for the object, the thing being done to. And the dog is being   chased by the bird. So der Vogel verfolgt den Hund [the-\textsc{nom}   bird chases the-\textsc{acc} dog], and Hund [dog] is dog. 

(Participant 33, TE-F, School 2)
\end{quote}

Many learners were also able to articulate their understanding that for feminine and neuter nouns the same article (\textit{die} and \textit{das} respectively) is used for both the subject and the object of a sentence. The feminine and neuter articles had been briefly introduced during the explicit information provided in weeks 4 and 5 of the intervention. The learners’ explanations demonstrated that they were able to utilise this information as well as apply their knowledge of the case-marked masculine articles to deduce the function of the ``non-case-marked'' article in sentences containing one masculine noun alongside a feminine or neuter noun.

\begin{quote}
  \textbf{P:} I mean \textit{\ul{die}} [the-\textsc{nom/acc}] is a feminine noun and \textit{\ul{den}}   [the-\textsc{acc}] is (.) used for object, masculine. And \textit{\ul{die}} [the-\textsc{nom/acc}] can be used for subject and object. But because   \textit{\ul{den}} [the-\textsc{acc}] is used for the object, then \textit{\ul{die}} [the-\textsc{nom/acc}]   will be used for the subject of the sentence. 

(Participant 50, TE-FM, School 1)
\end{quote}
\begin{quote}
  \textbf{P:} Well the kid is hugging the teddy bear and \textit{\ul{den}} [the-\textsc{acc}] is   um (.) the masculine word that’s   used as the object. So I   thought \textit{\ul{das}} [the-\textsc{nom/acc}] must be the subject since \textit{\ul{den}}   [the-\textsc{acc}] is the object. 
  
  (Participant 25, TE-F, School 1)
\end{quote}

Despite the successful use of grammatical terminology by many learners this was not a requirement for successful completion of the task. Some learners expressed their understanding of the function of the target structures in their own words, without the use of terms such as ``subject'' or ``object'':

\begin{quote}
  \textbf{P:} Um because I know the Vogel [bird] was a bird and it was   chasing the dog so I put \textit{\ul{der}} [the-\textsc{nom}] there in front of Vogel   [bird] and it was chasing (\textit{verfolgt [chasing]}) um (.) and then   the dog is being chased so it’s \textit{\ul{den}} Hund [the-\textsc{acc} dog].

  \textbf{R:} Yes the dog was being chased so it’s \textit{\ul{den}} [the-\textsc{acc}].   Anything else you can tell me about \textit{\ul{der}} [the-\textsc{nom}] or \textit{\ul{den}}   [the-\textsc{acc}] in that sentence?

  \textbf{P:} \textit{\ul{dern}} [the-\textsc{nom}] means it’s doing the action and \textit{\ul{den}} [the-\textsc{acc}] means it’s receiving the action. 

(Participant 14, TE-FM, School 2)
\end{quote}

A number of learners also took the opportunity to express their awareness that word order is flexible in German:

\begin{quote}
  \textbf{P:} Because \textit{\ul{der}} [the-\textsc{nom}] is the (.) subject. \textit{\ul{Der}} [the-\textsc{nom}] is   to describe what the subject is. And \textit{\ul{das}} [the-\textsc{nom/acc}] is to   describe what the object is. (.) Or you could do (.) um it that   way round. (\textit{Pupil swaps \ul{der} Vater [the-\textsc{nom} father] and \ul{das}   Baby [the-\textsc{nom/acc} baby]})

  \textbf{R:} ok, \textit{\ul{das}} Baby küsst \textit{\ul{der}} Vater [the-\textsc{nom/acc} baby kisses   the-\textsc{nom} father]. Why can you have it that way round?

  \textbf{P:} You can have it that way round because (.) you’ll still know   which way round it goes (.) because \textit{\ul{der}} [the-\textsc{nom}] is the   subject (.) and \textit{\ul{das}} [the-\textsc{nom/acc}] is the object. (.) 

(Participant 59, TE-FM, School 2)
\end{quote}

Such responses demonstrated that these learners were no longer primarily relying on the word order cue from their L1 English and were able to correctly interpret object-verb-subject sentences by relying on the masculine articles (\textit{der} and \textit{den}) to assign grammatical roles to the nouns within the sentences. Nevertheless, some learners continued to associate the subject with the ``first thing'' in the sentence, and the object with the ``second thing'' in the sentence: 

\begin{quote}
  \textbf{P:} Well \textit{\ul{der}} [the-\textsc{nom}] would go first because it’s the subject   and \textit{\ul{den}} [the-\textsc{acc}] is the object. And because the ball is hitting   the football player, then you would know that the ball goes   there (\textit{next to \ul{der} [the-\textsc{nom}]}) (.) first, and den Fuβballspieler   [the-\textsc{acc} footballer] would go afterwards because that’s the   thing being done to it. 
  
  (Participant 69, TE-F, School 1)
\end{quote}

Finally, as reflected in \tabref{tab:kasprowicz:2}, there were a small number of learners who did not express any awareness of the role-assigning function of \textit{der} and \textit{den} at post-test. Rather, their explanations continued to focus on gender, animacy, and/or guesswork, as at pre-test.

\subsubsection{Explanations provided at delayed post-test}

As noted above, at delayed post-test, a majority of the learners continued to demonstrate at least some awareness of the target structure (\textit{der} and \textit{den}) and their grammatical role-assigning function. As at post-test, many learners were able to provide appropriate explanations, either with or without metalinguistic terminology. In addition, learners in both the TE-FM and TE-F groups continued to demonstrate awareness of how to use the case-marked masculine articles to interpret sentences which also contained a non-case-marked feminine or neuter article, as well as how to interpret sentences in object-verb-subject word order. Nevertheless, as reflected in \tabref{tab:kasprowicz:2}, there was a decline in some learners’ ability to verbalise their knowledge of the target structures at delayed post-test. In particular, analysis of some learners’ responses at delayed post-test suggested that their metalinguistic knowledge may be less reliable than at post-test.

\begin{quote}
  \textbf{Post-test} 

  \textbf{P:} [\ldots] So I know \textit{\ul{den}} [the-\textsc{acc}] is for the object so I know   this (\textit{das [the-\textsc{nom/acc}]}) is going to be the subject. And the   kid is the subject because he’s doing (.) it’s cuddling the teddy. 

  \textbf{Delayed} \textbf{post-test} 

  \textbf{P:} Well the father is kissing the baby. These (\textit{der [the-\textsc{nom}]} and \textit{das [the-\textsc{nom/acc}]}) (.) \textit{\ul{das}} [the-\textsc{nom/acc}] can either   go at the start or at the end, because if it’s \textit{\ul{die}} [the-\textsc{nom/acc}]   or \textit{\ul{der}} [the-\textsc{nom}] (.) I think (.) they go at the start. But if it’s uh   (.) I can’t remember the other one (.) \textit{\ul{den}} [the-\textsc{acc}] or   something, then that one (\textit{das [the-\textsc{nom/acc}]}) goes at the   start. 

(Participant 15, TE-FM, School 2)
\end{quote}

Additionally, at delayed post-test, there was a greater level of inconsistency in individuals’ responses to the different items. This finding is reflected in the increase in the number of participants within the ``Mid-'' and ``Low-scorer’ groups at delayed post-test (see \tabref{tab:kasprowicz:2}). There were many instances of individuals providing correct explanations, often utilising appropriate terminology, for one item, but then being unable to provide an appropriate explanation on another item. Where students were unable to provide correct explanations, often their responses would centre on a discussion of the gender of the referents, as at pre-test:

\begin{quote}
  \textbf{Item} \textbf{1:}

  \textbf{P:} Because um (.) \textit{\ul{das}} [the-\textsc{nom/acc}] means (.) um I don’t   know how to say it (.) if it’s a baby, then \textit{\ul{das}} [the-\textsc{nom/acc}]   would go with the baby. And \textit{\ul{der}} [the-\textsc{nom}] would go with (.)   in front of a male. So I put \textit{\ul{der}} Vater [the-\textsc{nom} father] and \textit{\ul{das}}   [the-\textsc{nom/acc}] in front of Baby [baby] and I put küsst   [kisses] in the middle because the father was kissing the baby.

  \textbf{Item} \textbf{3:}

  \textbf{P:} Because the ball is what’s doing the hitting (.) and the   football player is the one that’s getting it done to them

  \textbf{R:} Ok, so why did you put \textit{\ul{der}} [the-\textsc{nom}] uh (.) at the   beginning, or with Ball [ball]?

  \textbf{P:} Because the ball is the one that is hitting. Because you can   (.) some German people put it that way (\textit{swaps} \textit{order to \ul{Den}   Fuβballspieler trifft \ul{der} Ball [the-\textsc{acc} footballer hits the-\textsc{nom}   ball]}) and say it like that.

  \textbf{R:} So \textit{\ul{den}} Fuβballspieler trifft \textit{\ul{der}} Ball [the-\textsc{acc} footballer hits   the-\textsc{nom} ball]. Ok

  \textbf{P:} And it so you know (.) \textit{\ul{der}} [the-\textsc{nom}] tells you (.) that’s   what’s doing it and \textit{\ul{den}} [the-\textsc{acc}] tells you who is receiving it.

  \textbf{R:} Ok, so that (\textit{new order}) means the same as the other way   round?

  \textbf{P:} Yes. 

(Participant 22, TE-FM, School 2)
\end{quote}

As shown in the example above, learners tended to be more consistent in their provision of correct explanations for items involving two masculine (\textit{m}) nouns (and therefore both \textit{der} and \textit{den}). Where items included a feminine (\textit{f}) or neuter (\textit{n}) noun, the learners tended to have more difficulty consistently providing correct explanations for the positions of the articles within the sentences; see also \tabref{tab:kasprowicz:4} for the mean score (out of 2) for each item type at post- and delayed post-test. Such inconsistencies contributed to the significant decline in learners’ performance on the Sentence Reconstruction task at delayed post-test.


\begin{table}
\begin{tabular}{l *4{c}}
\lsptoprule
& & \multicolumn{3}{c}{Post-test \textit{M (SD)}}\\\cmidrule(lr){3-5}
~ & \textit{n} & m+m & m+f & m+n\\\midrule
 TE-FM & 45 & 1.69 (0.67) & 1.51 (0.66) & 1.58 (0.72) \\
 TE-F  & 41 & 1.51 (0.84) & 1.29 (0.78) & 1.51 (0.78) \\\midrule
& & \multicolumn{3}{c}{Delayed \textit{M (SD)}} \\\cmidrule(lr){3-5}
~ & \textit{n} & m+m & m+m & m+m\\\midrule
 TE-FM & 45 & 1.33 (0.90) & 1.09 (0.85) & 1.03 (0.86) \\
 TE-F  & 41 & 1.33 (0.91) & 0.95 (0.76) & 0.62 (0.75) \\
\lspbottomrule
\end{tabular}
\caption{Descriptive statistics by item at post- and delayed post-test (max. score per item = 2)\label{tab:kasprowicz:4}}
\end{table}

\section{Discussion}

In response to Research Question 1, the quantitative and qualitative analysis of the Sentence Reconstruction task demonstrated that, following explicit grammar instruction, the majority of learners who received explicit information followed by either task-essential form-meaning connection practice (TE-FM group) or task-essential form spotting practice (TE-F group) were able to consistently and accurately discuss the function of the target grammatical structure, the masculine definite articles \textit{der} and \textit{den}. The explanations provided by learners at post-test indicated that they had robust verbalisable metalinguistic knowledge. Additionally, the majority of learners (>80\%) were able to accurately employ appropriate metalinguistic terminology in their explanations. These findings add to existing research findings which have demonstrated that young learners (in this study aged 9 to 11) can express their awareness of the form and function of linguistic structures and engage in language analysis (e.g. \citealt{BouffardSarkar2008}; \citealt{HorstEtAl2010}).

It is important to note that a decline was observed in the learners’ performance at delayed post-test along with a corresponding decline in the proportion of learners who were utilising metalinguistic terms in their explanations. With regard to Research Question 2, this finding suggests that without additional reinforcement and revisiting, learners’ ability to verbalise their knowledge of grammatical rules is susceptible to decay over time. Furthermore, across the other receptive and productive outcome measures, learners in both the TE-FM and TE-F groups maintained their learning gains between post-test and delayed post-test (see \citealt{Hanan2015,KasprowiczMarsden2018}). Principal component analysis revealed that at delayed post-test all outcome measures were loading onto one component, suggesting that by this time point all of the tasks were likely tapping into the same type of knowledge (see \citealt{Hanan2015} for a detailed discussion of this analysis). Therefore, it seems that it was specifically learners’ ability to verbalise their explicit knowledge that had declined, rather than the knowledge underpinning their ability to accurately interpret and use the target structure.

\begin{sloppypar}
In terms of Research Question 3, the analysis of the terminology learners utilised in their responses suggests that many learners had successfully developed both analysed knowledge (i.e. awareness of the relevant grammatical rules) as well as knowledge of metalanguage (i.e. the technical terminology needed to talk about language). \citet{Ellis2004} notes that, whilst metalanguage in and of itself is not essential for the development of explicit knowledge, developing learners’ knowledge of metalanguage (i.e. grammatical terminology) may help to strengthen their understanding of the linguistic constructs being learnt. In the present study, an association between performance on the Sentence Reconstruction task and use of technical terminology was observed, with the majority of High- and Mid-Scorers utilising relevant metalinguistic terminology in their responses. Notably, there were a small number of learners at post- and delayed post-test (four High-Scorers and one Mid-Scorer at each time point respectively), who were able to provide accurate explanations for the target structure without any use of technical terminology. However, it is not possible to determine whether this was due to a lack of knowledge or understanding of the relevant terminology or the learners simply choosing to express their understanding of the grammatical rules in their own words. 
\end{sloppypar}

During the intervention, the TE-FM and TE-F learners were exposed to the metalanguage related to the target grammatical structures (e.g. subject, object, masculine, feminine, neuter) within the brief explicit information provided in weeks 1, 2, 4, and 5, prior to completion of the practice activities in each session (see description above and in \citealt{Hanan2015}). The rationale for including an explanation of the terms ``subject'' and ``object'' in the context of the learners’ L1 was to ensure that the learners had a clear understanding of these terms, prior to using them in the explanations related to the L2. Indeed, recent research indicates that provision of L1 explicit information (and practice) alongside L2 explicit information and practice may be beneficial in clarifying key concepts and establishing form-meaning mappings in the L1, prior to the application of these concepts for learning of target L2 structures (\citealt{McManusMarsden2017}). Further, research has also indicated a relationship between learners’ awareness of L1 and L2 differences and learners’ performance on tasks requiring use of relevant grammatical structures (\citealt{AmmarEtAl2010,WhiteRanta2002}), as well as the potential usefulness of tasks that employ cross-linguistic comparisons (\citealt{WhiteEtAl2007}). Therefore, whilst the present study did not seek to investigate the effectiveness of providing explicit information relating to terms and concepts in the learners’ L1, the finding that the majority of TE-FM and TE-F learners were able to explain the function of the target structures and accurately used relevant terminology and/or their own words to do so, suggests that the explicit information which discussed the core concepts in the L1 prior to the application of these terms in the L2 is likely to have at the very least reinforced, if not established, learners’ understanding of the core metalanguage and how this relates to particular grammatical structures in both the L1 and L2.

With regard to the relationship between L1 and L2 knowledge about language, some existing research has indicated that young learners are unlikely to spontaneously make cross-linguistic comparisons when engaging in tasks requiring language analysis in the L2 (e.g. \citealt{BellEtAl2020}). Notably, \citet{BellEtAl2020} observed that older learners (aged 15 to 16) were more likely to make cross-linguistic comparisons than younger learners (aged 11 to 12), which the authors attributed to the greater focus on explicit grammatical knowledge in language instruction for older learners. The present study has demonstrated that explicit grammar instruction, involving comparison with the L1, can also successfully develop younger learners’ (aged 9 to 11) L2 metalinguistic knowledge and adds to existing studies which have demonstrated that younger learners are “mature enough to attend to form if they are taught how to” (\citealt[22]{BouffardSarkar2008}).

\section{Conclusion} 

The present study sought to investigate the extent to which young learners can develop verbalisable metalinguistic knowledge, as part of a larger study exploring the efficacy of explicit grammar instruction (see \citealt{Hanan2015,KasprowiczMarsden2018}). The findings revealed that, following instruction which combined L1 and L2 explicit information, learners were able to consistently and accurately discuss the grammatical role of the target L2 structures, in a majority of cases, drawing on appropriate metalinguistic terminology to do so. Some decline in learners’ metalinguistic knowledge and use of metalanguage was observed at delayed post-test (although not to baseline levels), suggesting that regular revisiting is needed to reinforce and maintain such knowledge. Notably, this study was conducted in England, that is, in an educational context where the development of learners’ L1 metalinguistic knowledge is prioritised from an early age (\citealt{DfE2013curriculum}). Therefore, the findings support the suggestion that in such contexts the foreign language classroom can usefully draw on learners’ developing L1 metalinguistic knowledge and harness the metalanguage that learners are expected to be familiar with when introducing new L2 structures. This would help to reinforce learners’ understanding of cross-linguistic similarities and differences, supporting their L2 development, as well as underpin the value of their developing L1 knowledge (\citealt{BellEtAl2020}).

\printbibliography[heading=subbibliography,notkeyword=this]
\end{document}
