\documentclass[output=paper,chinesefont]{langscibook}
\ChapterDOI{10.5281/zenodo.6811468}
\author{Kevin McManus\orcid{}\affiliation{Penn State University} and Brody Bluemel\orcid{}\affiliation{Delaware State University}}
\title[Instructional practices in  dual language immersion classrooms]
      {Instructional practices in English-Chinese and English-Spanish kindergarten dual language immersion classrooms}
\abstract{Dual language immersion (DLI) programs in the United States have been rapidly increasing in recent years. However, very little research to date has investigated what DLI instruction looks like and what opportunities for learning are available in DLI classrooms. The current study contributes to understanding in these areas by investigating teachers’ instructional practices in English-Chinese and English-Spanish kindergarten DLI classrooms. Video, audio, and observation data were collected from eight kindergarten DLI classrooms using a 50/50 model in which 50\% of instruction was delivered in English and 50\% in Chinese or Spanish. Results indicated important differences and similarities for (i) teachers’ language use in the different classrooms and (ii) teachers’ instructional practices in the different languages. Teachers’ instructional practices, the availability and type of instructional input, and their impact on opportunities for learning are discussed as ways to inform decisions about subject content teaching and language development in DLI classrooms.~}

\begin{document}
\maketitle 

\section{Introduction}

Exposure to the target language is understood to play an essential role in explaining the rates and routes of second language (L2) learning, a claim central to many theories of second language acquisition (for reviews, see \citealt{GassEtAl2021}, \citeauthor{MitchellEtAl2019} \citeyear{MitchellEtAl2019}). In instructed contexts, a critical source of exposure to the target language includes teachers’ language use, especially in English-dominant national contexts where access to languages other than English can be difficult (\citealt{LanversEtAl2021,MitchellMyles2019,PorterEtAl2020}). To date, research has shown that exposure to the target language in foreign language (FL) classrooms can be variable and, in some cases, infrequent (\citealt{DuffPolio1990,Wilkerson2008}). Studies documenting the instructional input indicate that experience, L2 proficiency, and pedagogical context can play important roles in shaping how teachers use the target language (\citealt{CollinsEtAl2012,Huensch2019,Macaro2001}).

In the U.S., some teaching organisations have expressed concern about the quantity of target language use in FL classrooms. The \textit{American Council on the Teaching of Foreign Languages} (ACTFL), for instance, considers it to be is insufficient (\citealt{ACTFL2021}; see also \citealt{VanPatten2014}). As a result, ACTFL has for a long time now recommended that ``learning take place through the target language for 90\% or more of classroom time […] The target is to provide immersion in the target language unless there is a specific reason to NOT use the target language'' (\citealt{ACTFL2021}). Even though these ``guiding principles for language learning'' likely constitute an important step in supporting language teaching in the U.S., especially given that no federal policy currently exists, the recommended practice of more than 90\% target language use is only loosely based on research evidence. This is because this advice is not based on research that has investigated relationships between the amounts and/or functions of language use in classrooms and L2 learning outcomes. In addition, teachers are encouraged to use the target language ``unless there is a specific reason'' not to do so. However, it remains unclear what such a reason would look like.

One challenge to making evidence-based recommendations about language use in FL classroom contexts, however, is that (i) not all FL programs share the same aims and objectives and (ii) very little research has actually examined the instructional input in FL classrooms (\citealt{CollinsEtAl2012,Huensch2019,Macaro2001}). While some research has calculated the amount of L2 use compared to L1 use in classrooms (\citealt{DuffPolio1990}), for example, very little is known about \textit{how} FL teachers use the target language in the classroom (i.e., what are the purposes of the instructional input?). Investigating this question in a variety of classroom types (e.g., intensive, dual language, and ``traditional'' language learning contexts) is critically needed in order to develop evidence-based recommendations for teachers and administrators in FL programs.

In the current study, we addressed these gaps in understanding about teachers’ use of the target language in an understudied pedagogical context, dual language immersion (DLI) classrooms. Given that this is a growing pedagogical context in the U.S. (see \citealt{Commission2017,Valdes1997}), our aim was to better understand what DLI instruction looks like and what opportunities for learning are present in DLI classrooms. Such findings are needed to develop appropriate, evidence-based recommendations that are appropriate for FL teachers in DLI classrooms. Although some research has focused on student learning outcomes in these contexts (e.g., \citealt{BurkhauserEtAl2016,FortuneTedick2015}), much less is known about the instructional input and pedagogical activities available in DLI classrooms.

\section{Language use in classroom contexts}

Documenting the availability of target language input in FL classrooms is important for understanding the potential for L2 learning in instructed settings (\citealt{CollinsEtAl2012,DuffPolio1990,Huensch2019}). However, very few studies have actually investigated what the instructional input in FL learning contexts looks like, especially when compared to studies of L2 learning outcomes, for example. While this is particularly the case for classrooms in DLI programs (\citealt{Jia2017,LiEtAl2016}), a small body of research has provided critical insights into questions about the availability of target language input in FL classrooms.

In \citet{DuffPolio1990}, for example, target and non-target language use in thirteen FL classrooms, including both commonly taught (e.g., French) and less commonly taught languages (e.g., Slavic languages), was assessed by audio recording the classroom content and conducting observations. Group results that averaged language use across the thirteen classes indicated that target language use represented approximately 68\% of the classroom input. However, considerable variation in the amount of target language use was found across the different classes, ranging from 10\% to 100\%. Even though the authors expressed surprise that ``over half of the teachers observed here used the L2 less than ninety percent of time'' (ibid., p. 162), we should be careful to note that these results reflect ``the amount of English and the amount of [the target language] spoken by the teacher and the students'' (ibid., p. 156). That is, then, these results combine student and teacher usage into a single analysis and also ignore other types of target language input in the classroom (e.g., textbook materials, videos, audio recordings). Also, the authors do not make a case for why 90\% should be the goal at which to evaluate language use in classroom contexts. Even though quantity of target language input is important, it is arguably just as important to understand  \textit{how} the L2 was used (e.g., for classroom procedures, explanations, group discussions). 

In sum, while Duff and Polio’s account provides a useful starting point for thinking about language use in FL classrooms, which likely acted as a catalyst for recommendations such as ACTFL’s 90\% or more target language use, more research is needed to contextualise these findings. That is, in addition to documenting the quantities of language use in FL contexts, research is needed that seeks to document \textit{how} teachers use the target language. In the remainder of this section, we review studies that have examined the functions of language use in FL classrooms to better understand how teachers use the target language. 

In one such study, \citet{CollinsEtAl2012} investigated the functions of teachers’ language use in an intensive English elementary school in Canada. Data were collected from three sixth grade classes (i.e., students aged 11--12 years old) in areas outside Montreal, in which students had little to no contact with English outside of the classroom. Video and audio recordings resulted in an instructional corpus of approximately 40 hours. Recordings were transcribed to examine the functions of teachers’ language use in the classrooms. The teachers were ``native or highly proficient speakers of English'' (ibid., p. 70). In order to understand how teachers used the target language in the classroom, subsets of the instructional corpus were examined to understand the range of purposes the teacher input served, which ``yielded a number of precise functions such as modeling a tongue twister, preparing and monitoring an activity, explaining specific aspects of language (grammar, vocabulary, pronunciation, etc.), reading aloud, and so on'' (ibid., p. 76). These functions of the teacher input were then grouped into five broad categories for understanding the functions of teacher talk in this instructional context: classroom procedures, language related episodes, text-based input, text-related discussion, and personal anecdotes. 

\begin{sloppypar}First, the most frequent function of the instructional input was for classroom procedures, accounting for 75\% of all teacher talk. Classroom procedures included teacher talk that organised classroom activities, routines, and student behavior. In one example, \citet[76]{CollinsEtAl2012} show the teacher interrupting an activity to provide further guidance to students: ``okay guys, can I have your attention a moment? The papers, the scrap paper that you’re using is just for you to write some ideas, to invent the name of your restaurant and to write, you know […]''.\end{sloppypar}

The second most frequent function of teachers’ language use included lan\-guage-re\-lat\-ed episodes, accounting for 17\% of the aural input in the classrooms. This is instructional input that focused on language, such as grammar, pronunciation, and vocabulary. For example: ``Okay, so here it's not he needs \textit{a} glue. He needs \textit{some} glue because glue is like liquid and you can't count. You see? That's why you put \textit{some} glue. You understand?'' (ibid., p. 77).

Although the data were also coded for text-based input for text read by the teacher, discussion of text-based input, and personal anecdotes were all relatively infrequent in the instructional input (less then 10\% for all three categories). For example, personal anecdotes, when teachers discussed or shared stories or experiences, accounted for 1\% of teacher talk. One example of this involved the teacher telling a story related to a classroom discussion of the idiom ``break a leg'': 

\begin{quote}
[my husband] was playing in a tennis tournament and he was known to jump over the net […] instead of going on the other side, around--he would jump over the net, okay? So before the tournament I told him, I said “break a leg” […] So, of course, he jumped over the net and what do you think happened?\hfill\hbox{(\citealt[78]{CollinsEtAl2012})}
\end{quote}

Taken together, \citegen{CollinsEtAl2012} results indicate that the majority of teach\-er talk in these intensive English elementary school classes in Canada was for classroom procedures. 

A useful contextualisation for these findings can be found in \citegen{Huensch2019} study of teacher talk in university-level FL classrooms in the U.S. In this study, classroom data were collected from graduate teaching assistants of French and Spanish. Audio recordings resulted in a classroom corpus of approximately 22.5 hours. Usefully for purposes of the comparison with \citet{CollinsEtAl2012}, both studies investigated the functions of language use by analysing the corpus using the same coding procedures. At the same time, it is important to note that the students in these classrooms were quite different (11--12 years olds in \citealt{CollinsEtAl2012}, but undergraduate students in \citealt{Huensch2019}).

First, in line with \citet{CollinsEtAl2012}, \citet{Huensch2019} reported that classroom procedures accounted for the most frequent type of instructional input in the FL classes, at 37\%, followed by language related episodes at 28\%. Although these proportions are lower than that reported by \citet{CollinsEtAl2012}, they are likely reflective of the different student populations, especially given that the younger students studied by \citet{CollinsEtAl2012} were aged 11--12 years. In addition, \citet{Huensch2019} reported some variation across the classes in terms of the proportion of instructional input dedicated to classroom procedures. For example, even though the average amount of teacher talk dedicated to classroom procedures was 37\%, these proportions ranged from 27\% to 61\% across the different classes. Similarly, in some classrooms, the proportion of language-related episodes that focused on grammar ranged from 5\% to 46\%. In line with \citet{CollinsEtAl2012}, personal anecdotes represented a very small proportion of the teacher talk (2\%). 

Taken together, \citegen{Huensch2019} findings indicate two important trends: (i) the instructional input across multiple FL classes was not the same and (ii) classroom procedures represented a frequent function of the instructional input (a finding also reported by \citealt{CollinsEtAl2012}). These results are important to consider going forward, especially since one argument for aiming for 90\% or more target language use in FL classrooms is to provide rich and varied exposure to the target language (see \citealt{ACTFL2021}). If the most frequent function of teacher talk is to organise classroom activities and student behavior, teacher talk might not be the richest source of language input to foster L2 learning.

\section{Teachers’ language use in dual language contexts}

Turning now to studies of language use in DLI classrooms, although such accounts are rare compared to accounts in FL contexts, two studies have provided rich accounts. For example, \citet{LiEtAl2016} reported on a large-scale study of the implementation of DLI across a large, urban school district in the state of Utah in the Western U.S. Classroom observations were used to study teaching practices and language use in DLI classrooms. Even though this approach is different from the previous studies discussed in this chapter given that audio/video data were not collected, this approach provides a broad account of teachers’ language use in this relatively under-researched context, which is an insightful approach given that very little is known about what DLI instruction looks like in US contexts. The observation protocols included a range of instructional practices (e.g., ``lesson objectives clearly defined, displayed, and reviewed with students'') that were rated using a 5-point scale: 4~=~completely evident, 3~=~mostly evident, 2~=~somewhat evident, 1~=~slightly evident, 0~=~not at all evident.  

In total, 56 teachers from 18 schools were observed for one class period from kindergarten through to the 12th grade (students aged 5 to 18 years old). The languages from those dual language programs included English, Russian, Spanish, Japanese, and Mandarin.

Overall, the classroom observations indicated that lesson plans were clearly defined, displayed, and reviewed with students. The classroom input was ``made comprehensible'' with explanations and activities (e.g., use of visuals, gestures, modeling). A variety of different learning strategies were used that included frequent opportunities for interaction. Just over half of the teachers were L1 speakers of the language they taught (57\%). In addition, the instructional practices did not appear to vary systematically across the languages. It was also found that the target language was used in very high proportions. Furthermore, the majority of teachers used the target language 100\% of the time. It should be noted that a key focus of this study was to provide a broad understanding of what DLI instruction looks like, achieved by studying a large number of teachers, in a variety of different schools, with students of varying language abilities. Clearly, fine-grained accounts are also needed to understand how these different pedagogical activities were implemented and to what end. 

Similar findings were reported in \citegen{Jia2017} study of two Chinese-English dual language classrooms in a southwestern city of the U.S. In line with \citet{LiEtAl2016}, data about teaching practices and language use came from observations, but this time of a small number of classrooms. These classroom observations were described as follows: ``I wrote down what was orally produced by students as well as teachers, recorded (by hand) activities the class was engaged in, types of written exercises carried out in class, etc'' (ibid., p. 49). Overall, \citet{Jia2017} found that a focus on language explanations, as in the ``language related episodes'' from \citet{CollinsEtAl2012}, for example, constituted a very small part of the classroom activity. Instead, teachers encouraged output activities. Indeed, interviews with teachers indicated a strong preference for a communicative approach that encouraged spontaneous output from the learners. In addition, no instances of using English to discuss grammar were found. That said, English was present in the classroom. In each observation, the instructor used Chinese 75\% or more of the time. When English was used by the teacher, it ``was limited to one word expressions or short sentence explanations'' (ibid., p. 73).

Taken together, this review of teachers’ language use in different FL contexts indicates some differences but also some important similarities among the classes. A difficulty drawing comparisons among the different pedagogical contexts is that different methodologies and ways of accounting for instructional input were used. While such an approach is of course complementary, as seen in the review of DLI classrooms, the different methodologies do not make it possible to draw meaningful comparisons across contexts. For example, observations were the primary data source used in the studies of teaching in DLI classrooms, but classroom input was audio/video recorded, transcribed, and then analysed in the studies of \citet{CollinsEtAl2012} and \citet{Huensch2019}. Not only do these methodological differences make comparisons across classrooms difficult, but the exclusive use of observation methodologies limits our understanding of the instructional input in the DLI contexts. Complementing these observations with some type of video/audio accounts, even if just partial accounts, would provide richer insights into the instructional input in this context. A further consequence of these methodological decisions is that the studies of \citet{CollinsEtAl2012} and \citet{Huensch2019} provide richer insights into the functions of teacher talk in those pedagogical contexts. In contrast, the observation findings from the dual language contexts seem to indicate a strong focus on promoting target language use, but it is not always clear how that was achieved. For example, it seems likely that teachers may have developed specific strategies to use in the target language and to encourage target language use with the students. Some account of these strategies would be useful for understanding what DLI teacher talk looks like. To address these limitations in our understanding of the pedagogical activities and the instructional input in DLI classes, research is needed that more comprehensively documents what DLI instruction looks like and what opportunities for learning are present in DLI classrooms using a variety of methodologies. In so doing, such research can make a critical contribution to developing evidence-based recommendations for language teaching in these relatively newer and under-researched pedagogical contexts. 

\section{Current study}

This study addressed the aforementioned gaps in previous research by examining the instructional input and different pedagogical activities used by teachers in Chinese, English, and Spanish DLI classrooms and the extent to which these  differed as a function of the target language of the DLI classes. One particular motivation for this study is that compared to our understanding of instruction in (post-secondary) FL contexts, relatively little is known about opportunities for language learning in DLI classrooms. 

\begin{sloppypar}
In the current study, two research questions were investigated. Following \citet{MitchellMyles2019} as well as the data collections procedures and conventions established by the French Learner Language Oral Corpora project (see \citealt{MylesMitchell2021}), Chinese, English, and Spanish kindergarten DLI classrooms were video recorded in order to capture the entirety of the language input in those classrooms. To achieve this aim, multiple video cameras were used to capture different perspectives on what was happening in the classrooms. Regular visits were made to these classrooms over the course of one year to document instruction and opportunities for learning over time. In addition, classroom observations using carefully designed and piloted protocols were used to complement the video recordings. The classroom video data were then transcribed and analysed in using the CHILDES software (\citealt{MacWhinney2000}), with CHAT and CLAN. Using these data, we examined what the most frequent pedagogical activities were in the English, Chinese, and Spanish DLI classrooms by following the coding conventions created by \citet{CollinsEtAl2012}. In so doing, we sought to understand the frequency of the following activities in the different dual language classrooms: classroom procedures, language related episodes, text-based input, text-related discussion, and personal anecdotes. The following research questions were investigated:
\end{sloppypar}

\begin{description}
  \item[RQ1:]What are the most frequent pedagogical activities in the English, Chinese, and Spanish DLI classrooms?
  \item[RQ2:]  To what extent does the frequency of pedagogical activities in the different DLI classrooms differ as a function of target language?
\end{description}




\section{Method}
\subsection{Context: The Delaware dual language immersion model}

Data for the current study were collected in Delaware, a state in the Mid-Atlantic region of the U.S. The Delaware dual language immersion model was established in 2011 through the then governor’s World Language Expansion Initiative. The model was first implemented during the 2012--2013 academic year in three school districts throughout the state. Since then, the number of programs and participating districts and schools have expanded annually and there are now immersion programs in almost 60 schools in twelve out of the sixteen school districts in the state with the number growing annually. Currently, there are Spanish-English and Chinese-English program options that begin in kindergarten and continue through high school. The program is structured as a 50/50 model from kindergarten (ages 5--6) through to fifth grade (ages 10--11), wherein students spend half of their day learning through the target language (e.g., Chinese, Spanish) and half of their day learning through English. In middle school, grades 6--8 (ages 11--14), students continue with intensive language learning opportunities with approximately 30\% of their studies being conducted through the target language. For high school, grades 9--12 (ages 14--18), the Delaware state department of education has established an agreement with partnering universities to provide dual enrollment course options.

The elementary 50/50 model is structured slightly differently for lower grades than the higher elementary grade levels. In both models, students have two teachers, an immersion language teacher (Spanish or Chinese) as well as an English teacher. Students switch between classrooms and instructors at the midpoint of every school day. In the instructional split from kindergarten through to third grade, the half of the day that is spent in the target language includes foreign language arts classes, maths, and science. The target language literacy and language arts class lasts for 50 minutes, science in the target language lasts for 40 minutes, and maths in the target language lasts for 60 minutes. When students transition to the English classroom, they then spend 120 minutes doing English language arts mixed with social studies content followed by a 20--30 minute bridge lesson. During the bridge lesson, the English instructor uses the time to reinforce content learned in science and math delivered through the target language. It is important to note that the bridge lesson is not used to reteach the content, but rather to complete exercises and activities that apply the content learned to reinforce learning and assess student development.

The instructional split for content taught in grades 4 and 5 is the same, but with adjustments made to the time spent in each area. The half of the day spent in the target language sees an increase in target language use for literacy\slash language arts instruction (60 minutes) and a decrease for science (30 minutes). Similarly, on the English side of instruction, English language arts is scheduled for 100 minutes and a 30-minute time block is designated for social studies. Further, the bridge lesson is also slightly decreased to 20 minutes as students in these grade levels have now established a higher level of language proficiency in the target language. 

As students transition to middle school, most school districts include seven class periods during the school day. For the language immersion students, two of the seven courses are taught in the language. One class period is for Spanish or Chinese language arts and the other class period is a content course taught through the target language. The state immersion model designates social studies as the content area course to be offered; however, in practice, some school districts have instead offered science through the target language. A driving factor behind this variation has been the availability of qualified instructors who are able to teach both the content and the language. 

The continuing model as immersion students transition into high school has just recently been established because the oldest cohort of immersion program students are in ninth grade during the 2021--2022 academic year. The high school model enrolls all students in the Advanced Placement (AP) language course during the ninth-grade year. Students who earn a score of 4 or 5 on the AP exam, as well as students who score a 3 and have instructor recommendation may then continue into dual-enrollment language courses for their remaining high school years. The state department of education established a Memorandum of Agreement with two Delaware universities that outlines the courses offered and guarantees that students who complete the course will earn credit that is transferrable to either university. The courses include advanced composition, speech, and civilisation courses. Only one course option is offered annually, but with a rotating schedule from year to year. This approach guarantees a unique course offering to students in each year of their high school experience. It also means that a student who continues in the program all four years (grades 9--12) will earn both high school credit and up to 15 credit hours of college credit in the target language by the time they graduate from high school. 

\subsection{Data}

Several sources of data were collected for the current study including classroom observation and recordings, instructor surveys and questionnaires, and stake holder surveys. The expansive classroom observation data includes recordings of both Chinese immersion and Spanish immersion classrooms at all grades in the elementary immersion program. At the time of data collection there were only a few students in the oldest cohorts in middle school and thus the data collection was targeted on the elementary school populations. The data analysed for this study came from kindergarten classes at two different schools. One from a Chinese immersion program in central Delaware and the other from a Spanish immersion program in northern Delaware. The Spanish immersion program that was observed had slight variation from the outlined state immersion model. The program still follows a 50/50 immersion model but with slightly different breakdown in content area instruction focus.

In the Chinese immersion program, the researchers were able to observe and record the instructor’s classroom. At least two cameras and an additional audio recording device were used to record the classroom interaction. One camera focused on students and student interaction and the second camera was focused on the instructor(s) during the class.  Additional audio recording devices were used as needed for improved sound quality. In the Spanish immersion program, both the Spanish immersion classroom and the English partner classroom were observed and recorded. The same approach was taken in classroom recording and data collection. Researchers arrived before the students arrived for school and spent the entire day recording the classrooms. Each classroom was observed at least two to three times. The researchers collecting data also took observation notes to identify any notable events or exceptionalities that occurred during recording.

Characteristic of the immersion programs in the state of Delaware, the Chinese immersion classrooms observed were unidirectional whereas the Spanish immersion classrooms were bidirectional. This design is semi-intentional, but primarily determined by the enrolled student body: Almost all students in the Chinese immersion program are English L1 speakers learning Chinese as L2, with only a few instances of students who are English L2 speakers (either Chinese L1 or who speak a language other than Chinese or English as their L1). Further, the Chinese immersion classrooms function as a cohort within a school where there are two classrooms of Chinese immersion students in the school and all other students and classes in the school are not immersion, but English monolingual.

The Spanish immersion programs in the state include both unidirectional and bidirectional programs. While data was collected from both programs, the excerpts analysed for this study came from the bidirectional one. The observed school is a 100\% immersion school where all students are Spanish immersion. This school also strives to balance classroom cohorts with approximately 50\% Spanish L1 speakers learning English as an L2 and 50\% English L1 speakers learning Spanish as an L2. There is variation from this targeted structure dependent upon the background of student enrollment.

Following data collection, classroom observation data was then uploaded to a database. Members of the research team with advanced language expertise in the target languages (Spanish and Chinese) then transcribed all data. These transcriptions were compiled into a corpus for analysis and evaluation. 

\subsection{Data preparation and analysis}

The classroom data from kindergarten classrooms were video recorded and transcribed following CHAT conventions (\citealt{MacWhinney2000}). Transcribers were L1 speakers or advanced-level L2 speakers of Mandarin Chinese, English, or Spanish. Important for the analysis, line breaks in the transcripts were introduced at the start of each new Analysis of Speech unit (ASU). ASUs were defined following \citet[365]{FosterEtAl2000}: ``a single speaker’s utterance consisting of an independent clause or subclausal unit, together with any subordinate clause(s) associated with either''. Transcription accuracy was checked by at least two members of the research team before analysis.

The speech provided by teachers in each classroom was coded using the categories created by \citet{CollinsEtAl2012} and used in subsequent research to understand the functions of teacher talk in instructional contexts (e.g., \citealt{Huensch2019}). Following this previous research, only teacher speech was coded. As previously noted, the kindergarten classrooms included two teachers and the talk from both is analysed here. This coding included the following pedagogical activities: classroom procedures (for teacher talk that involved organizing activities and managing student behavior), language-related episodes (for talk that focused on features of the language, such as grammar, pronunciation), text-based input (for scripted language, such as reading from a PowerPoint presentation or a book), discussion of text-based input (for any discussion related to scripted language), and personal anecdotes (for any talk involving personal information and stories). Using these categories, an independent tier was created called \%TTC (teacher talk code). Each ASU unit of teacher speech (identified at the beginning of each line with the ID code *TEA) was then coded according to its function, as follows: CPR for classroom procedures, LRE for language related episodes, TBI for text-based input, DTB for discussion of text-based input, and PAN for personal anecdotes. For example:

\ea
\begin{tabular}[t]{@{}lll@{}}
12  &  *TEA: & {Good morning everybody .}\\
13  &  \%TTC: & CPR\\
14  &  *STU: & {Good morning miss .}\\
15  &  *TEA: & {Let's see who are we missing?}\\
16  &  \%TTC: & CPR\\
17  &  *STU: & {NAME .}\\
18  &  *TEA: & {He's still over there ?}\\
19  &  \%TTC: & CPR\\
20  &  *STU: & {NAME is missing .}\\
\\
135  & *TEA: & did you have a bad day ?\\
136  & \%TTC: & PAN\\
137  & *STU: & xxx .\\
138  & *TEA: & NAME what about the baseball game?\\
139  & \%TTC: & PAN\\
140  & *TEA: & anything you want to add NAME ?\\
141  & \%TTC: & PAN\\
142  & *TEA: & Nothing today ?\\
143  & \%TTC: & PAN\\\tablevspace
144  & *TEA: & How's the baby doing ?\\
145  & \%TTC: & PAN\\
146  & *STU: & good .\\
147  & *TEA: & getting big ?\\
148  & \%TTC: & PAN\\
149  & *STU: & he's nine months old .\\
150  & *TEA: & tell us what you ate for breakfast .\\
151  & \%TTC: & PAN\\
\end{tabular}
\z

Using this analytical procedure, we were then able to quantify the different functions of teacher talk in the kindergarten corpus of DLI classrooms.

\section{Results}

In terms of the different types of pedagogical activities in the English, Chinese, and Spanish classrooms, \figref{fig:mcmanus:1} shows proportions for the five pedagogical activities used by teachers in the respective classrooms. These proportions show important similarities and differences between the classes. On the one hand, classroom procedures are the most frequent pedagogical activity in each of the different language classrooms: 70.3\% in Chinese, 54.9\% in English, and 47\% in Spanish. This means that a significant proportion of the teacher talk in these classrooms involved providing instructions to students and managing classroom behaviours, especially in the Chinese classroom. \tabref{tab:mcmanus:1} presents examples of classroom procedures in the different classrooms.

\begin{figure}[p]
\begin{tikzpicture}
\begin{axis}[
	ybar,
	ylabel = {\%},
    ymin = 0,
    ymax = 100,
    height = .3\textheight,
    width = \textwidth,
    axis lines*=left,
    xtick = data,
    xticklabels = {CPR,LRE,TBI,DTB,PAN},
    xlabel = {Function},
    ]
    \addplot+[lsMidOrange] coordinates {
    (1,54.8532731376975)
    (2,1.12866817155756)
    (3,4.74040632054176)
    (4,11.7381489841986)
    (5,27.5395033860045)
    };
    \addlegendentry{English}
    
    
    \addplot+[lsMidBlue] coordinates {
    (1,70.3230653643877)
    (2,0.901577761081893)
    (3,4.58302028549962)
    (4,13.2231404958678)
    (5,10.969196093163)
    };
    \addlegendentry{Chinese}
    
    \addplot+[lsRichGreen] coordinates {
    (1,47.6973684210526)
    (2,0)
    (3,19.7368421052632)
    (4,32.2368421052632)
    (5,0.328947368421053)
    };
    \addlegendentry{Spanish}
\end{axis}
\end{tikzpicture}
\caption{Functions of teacher language in the English, Chinese, and Spanish classrooms in percent\label{fig:mcmanus:1}}
\end{figure}


\begin{table}[p]
\caption{Examples of classroom procedures provided by teachers from the English, Chinese, and Spanish classrooms}
\label{tab:mcmanus:1}

\begin{tabularx}{\textwidth}{XX}
\lsptoprule
 Organizing activities & Managing behavior\\
 \midrule
I need a volunteer to come up here and show us what a retell really looks like and then you'll get to choose ready ? & Just raise your hand NAME.\\
\tablevspace
You can use your own whiteboards to play this game & Everybody crisscross those legs I hope you're sitting on your butts for a minute.\\
\tablevspace
{\cn 蓝色的小鸟组来} NAME {\cn 老师} \newline `blue bird group, come to Ms. NAME' & {\cn 安静地坐在地毯上面}\newline `Sit quietly on the carpet'\\
\tablevspace
{\cn 把你的笔记本放在柜子里}\newline `Put your notebook in your cubby' &{\cn  你在干嘛呢在那儿}\newline `what are you doing there?'\\
\tablevspace
Quien va a empezar?\newline `who is going to start?' & Para dónde vas?\newline `where are you going?'\\
\tablevspace
Escucha a tu amiguito\newline `Listen to your friend' & Vamos a ver quién está listo\newline
`Let’s see who’s ready'\\
\lspbottomrule
\end{tabularx}
\end{table}

In addition to identifying classroom procedures as being the most frequent type of pedagogical activity, we examined the language used to give those instructions. This analysis indicated a number of differences in the language of classroom procedures. In the Chinese classrooms, all CPR was delivered in Chinese, but, in the Spanish classrooms, 59\% of CPR was delivered in Spanish, 36\% in English (e.g., ``NAME you need to sit near me'') and 5\% in a combination of Spanish and English.


A second notable difference between the classrooms is that personal anecdotes appear to play a relatively important role in the English classrooms, but less so in the Chinese or Spanish classrooms: 30\% of teacher talk involves personal anecdotes in the English classrooms, but that proportion is 10\% in the Chinese classrooms and 1\% in the Spanish classrooms. The extracts in (2) are examples of such anecdotes from the English classrooms:

\ea
\begin{tabularx}{\linewidth}[t]{@{}lQ@{}}
*TEA: & Does anybody have something fun to share from the weekend?\\
*TEA: & We had two days off from school.\\
*TEA: & Try to remember what you did on Saturday and Sunday hmm or maybe yesterday.\\
*TEA: & Let me tell you where I went after school yesterday.\\
*TEA: & I went to a softball game and I watched PLACE very first softball team.\\
*TEA: & Did you know we had a softball team and a baseball team?\\
*TEA: & How's the baby doing?\\
*TEA: & let's look at NAME he's got something to say what do you want to share today?\\
*TEA: & tell us what you ate for breakfast.\\
*TEA: & Who did your hair today?\\
*TEA: & Did your sister have fun at PLACE yesterday?\\
\end{tabularx}
\z

As the extracts indicate, personal anecdotes function as a way for the teachers to engage with students to talk informally about their weekends and out-of-class activities. This takes the form of both teachers asking direct questions to students based on some shared information. For example, the teachers asked one student ``how’s the baby doing'' based on a previous conversation about a new sibling in the child’s family. The teachers also appear to use this type of interaction as a check in with the students (e.g., to find out if and what students had eaten for breakfast that morning). In addition to teachers asking direct questions to the students, students also comment on each other's stories. (For example, one student was describing a birthday gift they received from their parents and a second student commented about a new toy truck they received.) 

\begin{figure}
\includegraphics[width=\textwidth]{figures/a11McManusBluemel-img004.png}
\caption{Video still of students in the English class sitting on the carpet to share personal stories\label{fig:mcmanus:2}}
\end{figure}

It is also important to note how the different classes use the same space to carry out the different functions. Each morning, the teacher and the students sit in a circle on the mat. It is interesting to note that while all the DLI classes begin their school days in this way, only the English classes use this set-up to discuss and share personal stories. In the Spanish classrooms, there tends to be more singing and story-telling activities rather than sharing personal stories. For example, on one day, students entered with a song they had used in class on a previous day. The song contained key expressions for greetings: ``Buenos días, buenos días; Buenas tardes, buenas tardes; Buenas noches, buenas noches''. After everyone had entered the classroom and they had been singing the song, students stayed on the carpet to practice greeting each other. Then, students practiced new greetings with teacher scaffolding:

\ea
\begin{tabular}[t]{@{}ll@{}}
*TEA: & Buenos días me llamo señora Name.\\
*TEA: & Buenos días me llamo.\\
*TEA: & Let me hear you.\\
*TEA: & How do you say it?\\
*STU: & Name.\\
*STU: & Name.\\
*TEA: & Me llamo Name.\\
\end{tabular}
\z

\section{Discussion}

The current study examined the functions of teacher talk in Chinese, English, and Spanish kindergarten DLI classrooms. In so doing, we sought to better understand the types of language input that young learners in this relatively new pedagogical context can be exposed to. We contextualised our understanding by reviewing previous research involving teacher talk in a variety of FL classrooms (\citealt{CollinsEtAl2012,Huensch2019,Jia2017,LiEtAl2016}). Taken together, our findings indicate both considerable overlap with this previous research as well as important differences.

First, our finding that classroom procedures account for a large proportion of teacher talk in DLI classrooms is consistent with previous research on this topic (\citealt{CollinsEtAl2012,Huensch2019}). Although some differences are visible across the different languages (e.g., greater use of classroom procedures in the Chinese versus the Spanish classes), a key take-away is that organizing classroom activities and managing behaviors appear to constitute an important function of teacher talk in these classes. Of course, given the age and experience of the students, this is perhaps unsurprising. Indeed, \citegen{CollinsEtAl2012} findings involving 11--12-year-olds similarly show high proportions of teacher talk focused on classroom procedures. One interesting finding about the DLI classes, however, was that this type of teacher talk was always delivered in Chinese in the Chinese classes, but it was delivered in Spanish and English in the Spanish classrooms (e.g., 59\% in Spanish).

\citet{CollinsEtAl2012} discuss the frequency of classroom procedures in their data in terms of the richness of exposure to the language. One consideration at play here is to what extent classroom procedure talk can provide rich, engaging, and meaningful exposure to the target language. This is an important reflection point, given that classroom procedure talk is the most common type of language input for learners in DLI classrooms and FL classrooms more generally. To this end, \citet[81]{CollinsEtAl2012} noted that ``when the role of the teacher went beyond facilitating oral interaction among students to include interacting with them herself, her own speech became a richer source of input''. This is one way that the sharing of personal anecdotes might be a very useful source of language exposure for young learners. However, the experience and proficiencies of students can be expected to shape the extent to which students and teachers can engage in the sharing of stories. At the same time, though, it is likely that these relatively informal uses of the target language can provide a useful resource for language development, as suggested by \citet{Jia2017}.

Second, a key difference found among the DLI classes is the extent to which teachers engage students with personal stories and anecdotes. For example, this pedagogical activity accounted for approximately 30\% of teacher talk in the English classes, but it remained relatively infrequent in the Chinese and Spanish classes. Indeed, comparisons with previous research from FL classrooms found the sharing of personal stories to be a relatively infrequent pedagogical activity (e.g., 1\% of the teacher input in \citealt{CollinsEtAl2012}). One explanation for this difference could be that it is a feature of specific teachers’ approaches to kindergarten learning. For example, even though all kindergarten classes started the day with time on the mat, only the English teachers used this time for sharing stories. It is possible that students’ greater proficiency in English is one reason for why this pedagogical activity was particularly common, however, this can only be part of the reason. It also appears that the teachers used this time, as previously mentioned, to check on the wellness of students (e.g., had they eaten breakfast that day). Nonetheless, this function of teacher talk was rich and diverse, involved a variety of topics and a variety of different constructions. It also allowed students the opportunity to interact with each other in a more informal way (e.g., compared to the practicing of greetings).\largerpage

We should also acknowledge the cultural impact on immersion instruction. All instructors observed in the Chinese immersion programs and most instructors observed in Spanish immersion programs are L1 speakers of their respective languages with a variety of citizenship and cultural backgrounds. These international educators come with strong qualifications, having undergone teacher training and necessary certification, and, at the same time, bring along culturally diverse content learning experiences and backgrounds. In many instances, there are likely differences in the expectations of student and teacher roles and behaviors. One result of this is a bringing together of educational cultures in the immersion programs as teachers bring with them their culturally embedded understanding of classroom interaction. The result is that the immersion teachers learn and adapt to a new culture of learning and teaching. Indeed, some cultural practices persist that also push the students to adapt and collectively form a merged culture of learning.

\section{Conclusion}

The current study set out to better understand teachers’ use of the target language in an understudied pedagogical context, DLI classrooms. To achieve this goal, DLI classes in Chinese, English, and Spanish were video recorded and analysed. Our results showed that a common function of teacher talk in these contexts is to organise classroom activities and manage student behavior, consistent with previous research from FL classrooms. Although a number of differences between the different language classrooms were found (e.g., frequent discussion and sharing of personal stories in the English classes), we also found considerable similarities among the different languages. Taken together, these findings provide rich accounts of the instructional input in DLI classrooms.

\printbibliography[heading=subbibliography,notkeyword=this]
\end{document}
