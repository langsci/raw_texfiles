\documentclass[output=paper]{langscibook}
\ChapterDOI{10.5281/zenodo.6811456}
\author{Li Wei\orcid{}\affiliation{UCL Institute of Education, University College London}}
\title{Preface}
\abstract{\noabstract}

\begin{document}
\AffiliationsWithoutIndexing{}
\maketitle 

\noindent Vivian Cook, who throughout the years was a close colleague of Florence Myles' in various roles and capacities, was meant to be writing this Preface. Sadly, Vivian passed away after a battle with illness on 10\textsuperscript{th} December 2021. I have no doubt that Vivian would have endorsed the tributes friends and colleagues of Florence Myles’ are paying through their contributions to this special volume. He would also, I am certain, have recalled the ``golden age of SLA at Newcastle''. Both Florence and Vivian joined Newcastle University in 2004. I was Head of School of Education, Communication and Language Sciences at the time, and recruited Vivian from Essex. I was also on the interview panel that appointed Florence to the professorship in French linguistics in the School of Modern Languages. With their appointments, as well as several others at different levels and in different departments, Newcastle had an impressive critical mass of researchers in second language acquisition. And Florence, who later led the cross-school Centre for Research in Linguistics and Language Sciences, was instrumental in not only spearheading SLA research at Newcastle but also facilitating much more interactions and collaborations across a range of areas of theoretical and applied language research.

What has become a hallmark of Florence’s work is her ability to connect linguistic theories with empirical and pedagogical issues in language teaching and learning. It is clear from the numerous publications she has produced over the years that Florence is especially interested in second language learning in the school context. She has researched the development of second language vocabulary, morpho-syntax, formulaic expressions, narrative through classroom-based teaching, and analysed the data with a range of different theoretical approaches including generative, corpus, and psycholinguistic approaches. Few scholars in SLA have shown the breadth that Florence has in her work. But she has done so without shouting out what has become a rather meaningless word ``interdisciplinary''. Florence has also researched task complexity and variability, as well as the impact of technology, the study abroad experience and the development of criticality through language learning. 

Florence has led many research projects, funded by competitive grants from agencies such as the ESRC, AHRC and the British Academy, as well as the European Commission. Through these projects, she has built large online databases of learner language oral corpora, especially French and Spanish as second languages, which have become key resources for research and curriculum design. Her work is characteristically collaborative, and she has worked with researchers in many different institutions and countries.

A considerable amount of Florence’s work focuses on young learners and the school curriculum. Here, her work not only deals with key theoretical topics such as age, individual differences, cross-linguistic transfer and morpho-syntactic processing, but also import policy and practice issues such as learner motivation and creativity and culture in the language curriculum. Florence’s work reinforces the importance of research in developing policies and practices that work in schools and classrooms, including developing language pedagogies that are appropriate for young learners.  

Second language acquisition research is a thriving field across the world. A great deal of work, though, is on acquiring English as a second or foreign language by speakers of other languages. Research on the acquisition of other languages by English-speaking learners that is simultaneously theoretically motivated, deeply-rooted in data, and has a focus on implications for policy and practice, as exemplified in Florence’s career-long work, is rather rare and so needed. We owe Florence a big debt of gratitude for her inspiring work. May the present volume be a small but good token of that! 

\printbibliography[heading=subbibliography,notkeyword=this]
\end{document}
