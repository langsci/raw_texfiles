\documentclass[output=paper]{langscibook}
\ChapterDOI{10.5281/zenodo.6811458}
\author{Bernardette Holmes\orcid{}\affiliation{Co-chair of the Research in Primary Languages Network} and Angela Tellier\orcid{}\affiliation{University of Essex}}
\title[The role of research in primary languages policy in the UK]
      {The role of research in primary languages policy in the UK: The journey from policy to practice}
\abstract{To celebrate the extensive professional achievements of Professor Florence Myles, we are seeking to illustrate how languages policy can benefit from research. In particular, we want to highlight the major contribution that she has made to our understanding of cognitive development in middle childhood, and how this affects children’s learning of a second or foreign language in instructed settings. We will argue that languages education policy should involve regular government funded research both to inform policy design and to monitor and support its implementation. In the earlier parts of the chapter, we will revisit key periods in the history of language policy over the last hundred years, and review a number of political documents on language policy between 1918 and 2014. We will show that the synergy between research and policy has rarely been optimised, and that perennial questions about the why, when, what, and how of primary languages have yet to be fully addressed taking full account of research findings. We will assert that this disconnection between policy and what we know about language learning can lead to false starts and unnecessary setbacks. We will then illustrate how the specific contribution of Myles and her vision for how to develop research-informed practice for primary languages, have contributed to creating a positive dialogue between researchers and policy makers in recent years, in particular through the establishment of the Research in Primary Languages Network (RiPL) which draws together leading academics and policy developers with practitioners and decision makers at local and national levels.}

\begin{document}
\maketitle 

\section{Lessons from the past}

Revisiting the history of languages policy in UK primary schools\footnote{\textrm{For international readers who may be less familiar with the organisation of schools in England and Wales, it may be helpful to explain in advance some of the terminology that will be used in our discussions. Maintained schools in England and Wales follow a national curriculum that is divided into key stages. These are legal terms that describe blocks of years that relate to the children’s age: key stage 1 refers to the period of two years of schooling (Years 1 and 2) when children are aged between five and seven; key stage 2 refers to the period of four years (Years 3, 4, 5 and 6) when children are aged between seven and eleven; key stage 3 refers to the period of three years (Years 7, 8 and 9) when pupils are between eleven and fourteen; and key stage 4 refers to the two years of schooling (Years 10 and 11) leading up to GCSE and other public examinations, when pupils are aged between fourteen and sixteen. Key stages 1 and 2 constitute primary/elementary education; key stages 3 and 4 refer to secondary education. Readers should bear in mind that Wales, Northern Ireland and Scotland are now devolved administrations responsible for their own national curricula and hence modern language provision. All three administrations have their own priorities which differ to some extent from those in England, in particular with regard to indigenous languages, Welsh, Irish, Scots and Gaelic.} } reveals the inherent risks of policy makers repeating the same mistakes, if account is not taken of past experience, research and scholarship. The key periods and publications that demand particular attention are (please see bibliography for further details):

\begin{itemize}\sloppy
\item the Leathes report \citep{Leathes1918}
\item the Annan Committee Report \citep{Annan1962}
\item the Nuffield Pilot Scheme for the teaching of French (1964--1974) \citep{Burstall1975}, a longitudinal cohort study of children learning primary French in England and Wales, and the Burstall Report NFER \citep{BurstallEtAl1974}
\item the Plowden report \citep{Plowden1967}
\item the Education Reform Act \citep{DfES1988} and the introduction of the national curriculum
\item the Nuffield Languages Inquiry \citep{NuffieldFoundation2000}
\item Languages for all: Languages for Life -- A Strategy for England \citep{DfES2002}\footnote{The Department for Education (DfE) is her Majesty’s government department for child protection, education, apprenticeships and wider skills. It has been named variously at different periods of time, including the Department for Education and Employment (DfEE), the Department for Education and Skills (DfES) and the Department for Children, Schools and Families (DCSF).}
\item the Key Stage 2 Framework for Languages \citep{DCSF2005}
\item the Pathfinder projects (2003--2005) DfES \citep{MuijsEtAl2005}
\item the introduction of the statutory requirement to teach a foreign language from the age of seven \citep{DfE2014}
\end{itemize}

It is curious, but, in hindsight, predictable, that the earliest of these, the Leathes Report \citep{Leathes1918} on the role of modern languages in the modernisation of education, published more than a century ago and described as the Magna Carta of language teaching \citep{Byram2021}, should reflect a striking similarity with concerns that continue to preoccupy the current administration of England and, indeed, to a greater or lesser extent, those of the devolved nations of the UK. Further scrutiny of policy initiatives relating to primary education show that over the last hundred years, policy makers appear to have been vexed by the same questions with regard to primary languages policy and its implementation. These questions break down into four distinct but interrelated categories. In simple terms, they address the why, when, what, and how of primary languages in national curricula. 


\subsection{Why teach primary languages?}

The first of these questions: Why teach primary languages? invites us to re-examine the rationales given by particular administrations for the perceived importance, or otherwise, of the early introduction of the learning of a language other than English, which we will reference in this chapter using the terms ``primary languages'' or ``modern languages''. It will be seen that modern languages education in general has been closely affected over the last century by its specific political context and has tended to be influenced by the nature of the UK’s relations with other countries of Europe and around the world.



In the midst of the First World War, Herbert H. Asquith, Liberal politician and Prime Minister 1908--1916, commissioned a committee, chaired by the First Commissioner of the British civil service, Stanley Leathes: 


\begin{quote}
to enquire into the position occupied by the study of Modern Languages in the educational system of Great Britain, especially in Secondary Schools and Universities, and to advise what measures are required to promote their study, regard being had to the requirements of a liberal education, including an appreciation of the history, literature and civilisation of other countries, and to the interests of commerce and public service \citep[1]{Leathes1918}.
\end{quote}


These terms of reference thus distinguish two purposes of language teaching, a ``liberal education'' on the one hand and the ``interests of commerce and public service'' on the other \citep{Byram2014}. It can be argued that this duality of purpose in language study still bedevils the languages community to this day and raises a number of unresolved questions. Is language study largely instrumental and transactional, serving the needs of ``commerce and public service'', terms that would probably be replaced today by ``the needs of the economy, and of diplomacy''? Or is language study much broader in its reach, more aligned to the humanities, providing insight into other cultures, their history, beliefs and values, referenced in today’s curricula as intercultural understanding or cultural capital? Can both ambitious purposes be met? And if these are the rationales given for language study in secondary education and the universities, what are the implications for early language learning? Is there a place for languages in the primary school curriculum, and if there is, what are the parameters and expectations for the subject? And, more importantly, do they take full account of the most relevant research into how early language learners of a new or foreign language learn most effectively?


\subsubsection{Languages in the curriculum: Communication skills versus humanities?}

The inherent tension between these different rationales is evident in the Leathes Report, a tension that has created confusion over the position and purpose of languages as a subject of study in school and university curricula for more than a century. As language educators, it is important to answer the question whether the study of a modern language is a skill responding to the instrumental agenda, increasingly driven by the language needs of globalisation, or a discipline, responding more closely to the humanities \citep{Canning2009} critically exploring the development of cultures and societies, or both. Evidence over time shows that decisions made in formulating answers to this question deeply affect primary languages policy and curriculum development, and are affected by the historical, social and cultural context.

This becomes evident from the Leathes Report: the practical and instrumental advantages of language study as a skill had been thrown into sharp relief in 1916 by the First World War, when it was suggested that poor language skills might have been responsible for the failure to understand the reasons for discontent which had led to the conflict in the first place \citep{Bayley1991}. Among other persuasive arguments for the practical and diplomatic value of languages skills, Leathes furthermore provided a strong case why the business world should take modern languages seriously. Having consulted relevant government departments, the Leathes committee turned its attention to the business community, including the Chambers of Commerce. Firms contributing case studies to the Leathes commission reported hiring international recruits in a range of key positions when dealing with international trade due to a lack of language skills among British employees. The report cites an example of German firms securing the distribution trade in South America, even when British goods were concerned, because the Germans took the pains to learn Spanish, and concludes that ``[w]ith such examples before us we can hardly afford to wait till all the world has formed the habit of talking English.'' \citep{Leathes1918}

A century later, in 2016, a similar landscape was still in evidence. Born Global, a policy research project of the British Academy into languages and employability, found that international recruits speaking multiple languages had a distinct advantage in the global labour market over their monolingual British peers. It would seem that no one in the UK heeded Leathes' timely warning in 1918. Leathes could not have predicted the rapid rise of English as a global language over recent decades which has paradoxically adversely and advantageously influenced the teaching of modern languages in Anglophone countries. The enduring message remains that learning English as an additional language is an advantage; speaking only English is a significant limitation. 

\subsubsection{Political imperatives driving policy change}

It was not until the 1960s and the context of the Cold War that we see a renewed government interest in languages capability and the need for a deeper understanding of other international forces in the global arena, which led to the commissioning of a report on the teaching of Russian by a committee chaired by Lord Annan. Based on the recommendations of this report that ``it would be advantageous if the regular teaching of a first foreign language were started in good conditions and by the right methods in the primary school'' \citep{Annan1962}, a Pilot Scheme, henceforth referred to as the Nuffield pilot scheme, for teaching French in primary schools was established in England and Wales. Three cohorts of pupils were followed for ten years in order to assess the feasibility and educational desirability of introducing foreign language teaching to a broader range of pupils. 

It is pertinent to note that at the time of the Nuffield Pilot Scheme, the UK was embroiled in its efforts to join the EEC (European Economic Community), to which it first applied in 1961. The UK applications had been vetoed twice by the then French president, Charles de Gaulle, in 1963 and again in 1967. With De Gaulle’s departure from the presidency in 1969, the UK made its third application for membership. Georges Pompidou proved to be more amenable, and in 1973, Sir Edward Heath took the UK into the European Community, and, by public referendum in 1975, the British public voted to remain in the EEC. At that time, the general public had a keen interest in Europe. With this interest came support for early language learning; parents wanted their children to learn languages (\citealt{Burstall1975}).  The final report on this scheme (\citealt{BurstallEtAl1974}), however, concluded that there was no evidence for any advantage conferred by an early start to language learning, resulting in the withdrawal of government funds \citep{Tellier2019}. The reasons for the perceived failure of the pilot and the lessons that should have been learned will be discussed later. 

\begin{sloppypar}
Similar political imperatives surrounding entry into Europe drove policy change in Scotland. Throughout the late sixties and seventies, Scotland, like England, had experienced challenges in its efforts to introduce primary languages \citep{Johnstone1996}. Clark observes that a report by HM Inspectorate on the introduction of French into primary schools in Scotland in the 1960s noted that there was a lack of continuity on transition to secondary school and that many primary teachers lacked sufficient competence in the language \citep{Clark1998}. This did not deter the government backing a further national pilot programme in Scotland on Modern Languages in Primary Schools in the late 1980s \citep{Johnstone1996}. This interest was stimulated by the prospect of entering the European Single Market, which was established in 1992, and underpinned by the belief shared by politicians in Scotland that the introduction of early language learning would in time improve the competitiveness of Scottish businesses \citep{Johnstone1996} -- an interest shared across England and Wales.  
\end{sloppypar}

\subsubsection{Building national capability in language skills}

The Education Reform Act \citep{DfES1988} had already established the study of modern foreign languages as a foundation subject for children aged between 11 and 16, in key stages 3 and 4 for England and Wales. The Department for Education and Skills (DfES) published its intentions for modern languages in the curriculum for England and Wales in 1988. These stated ambitions had much in common with current policy in their aim to build national capability in language skills. The rationale for language learning was ``founded on the belief that education at school can and should have lasting and beneficial effects on the prosperity and well-being of individuals and the nation'' \citep[1]{DfES1988}. The focus was firmly set on language as skills that are ``worthwhile'' for individuals and that can be ``put to use by people at work or in their personal lives, at home and abroad''. In the same spirit, a National Curriculum Working Group for England and Wales recommended that primary languages should be more widespread (while holding back from recommending including them in the national curriculum):

\begin{quote}
We firmly believe that it is now desirable to identify the steps which need to be taken to make widespread teaching of modern foreign languages in primary schools possible, and we have noted the recommendations of the House of Lords Select Committee \ldots\ to this effect \citep[para 3.13]{DfES1990}.
\end{quote}

\begin{sloppypar}
Head teachers were also largely in favour of this recommendation and at the National Association of Head Teachers’ conference in 1992, a resolution was made calling for the introduction of primary languages into the curriculum \citep{Satchwell1996}. The ``groundswell of renewed interest'' described by \citet{Satchwell1996} was engendered by the prospect of Britain in Europe and the need to build capacity in language skills from an early age.
\end{sloppypar}

\subsubsection{Proficiency in language skills versus intercultural competence}

In recent policy decisions in England, it would seem that little regard has been taken of research into the cognitive development of children in middle childhood in relation to the most appropriate approach to early language learning and to whether proficiency or intercultural competence should be the main drivers of curriculum content \citep{Myles2017}.

Considerations of the rationale for learning modern languages over time, reveal that the apparent duality of purpose of the subject discipline risks pitting the development of language skills against the study of cultural content. History shows that it is the strength of the instrumental argument, resting on the economic case and the employability agenda that policy makers at national and local levels have found most persuasive. As a result, the study of languages, particularly in the school curriculum, tends to be positioned firmly as a skill. Based on this argument, the impetus for early language learning has rested on the assumption that an early start will provide a faster route to language competence in one or more new languages in addition to the mother tongue, and that this will increase our national capability in languages, and arising out of this, we will see increases in GDP (Gross Domestic Product), international trade and wider UK engagement in international relations. 

To some extent, the rationales for language learning across the UK over many decades have made some attempt to rebalance the stated purposes of language study, and in addition to the value of languages to the economy and international trade, have commonly referred to the importance of languages in developing positive attitudes to others and greater openness to cultural diversity (\citealt{Leathes1918, HMI1987, DES1991, NuffieldFoundation2000, DfES2002, DCSF2005}). We see this purpose firmly enshrined in the current programmes of study for languages in England, Wales, Scotland and Northern Ireland. 

Recall that for Leathes, the role of language learning in the development of culture was a strong feature in his policy recommendations \citep{Leathes1918}. Leathes argued compellingly for modern languages ``as a means to general education and culture'' as well as a practical skill \citep[(v) 53]{Leathes1918}. Making no apology for putting ``practical ends first'', the report makes clear that the study of modern foreign languages should be ``the study of modern peoples in any and every aspect of their national life.'' It went further in its definition, stating that ``the study of languages is, except for the philologist, always a means and never an end in itself'' \citep[1 Definitions (b)]{Leathes1918}. 

Within this context, it is instructive to reflect on the breadth of Leathes’ definition of ``culture'', which prompts us to clarify how national programmes of study have defined and addressed culture, and how it fits into the conceptualisation of language learning. This has implications for how and what teachers teach in the languages curriculum.

Such reflections suggest that despite early and continued recognition of the cultural contribution of languages in language policy statements, it has remained very much a secondary consideration.  This is reflected in the National Curriculum for England and Wales \citep{DfES1988}, which strongly emphasised the economic benefit of supporting the UK as a member of the European Union and stated that ``opportunities will be opened up for trade, tourism, international relations, science and other fields'' \citep[1]{DfES1988}. Similarly, in 2000, the Nuffield Inquiry focused firmly on the assumed benefits of an early start to language learning to improving standards and national capability, assigning a similarly secondary role to the value of language learning to personal, social and cultural development and intercultural understanding. The Inquiry Committee, under the joint chairmanship of Sir Trevor McDonald OBE and Sir John Boyd, KCMG, had been given ``the mandate from the Nuffield Foundation to look at the UK’s capability in languages and to report on what we needed to do as a nation to improve it.'' The Inquiry Committee made a number of ambitious and timely recommendations. It concluded: 

\begin{quote}
In spite of parental demand, there is still no UK-wide agenda for children to start languages early. There is a widespread public perception, backed by research, that learning another language needs to start earlier if the next generation is to achieve higher standards. An early start to language learning also enhances literacy, citizenship and intercultural tolerance. \citep[6]{NuffieldFoundation2000}
\end{quote}

\subsubsection{The emergence of intercultural understanding as part of the primary languages curriculum}

The duality of purpose of language learning, referred to earlier, as both a skill and a means to develop cultural awareness and intercultural understanding was likewise at the heart of the national strategy,  Languages for All: Languages for Life. A Strategy for England \citep{DfES2002}. The vision statement recognised languages as a lifelong skill with both economic and personal benefits, in particular those of instilling a broader cultural understanding. It emphasised that these were essential skills in the 21\textsuperscript{st} century and recognised the past failures of the UK to develop capabilities of multilingualism and cultural awareness. The statement points to the dangers of both cultural impoverishment and economic disadvantages as a result of the lack of foreign language skills among the UK population and workforce. 

The Key Stage 2 Framework for Languages published in \citeyear{DCSF2005} certainly presented a fresh conceptualisation of language learning which challenged the traditional discrete four skill approaches to teaching, based on Listening, Speaking, Reading and Writing, replacing these classifications with the integrated strands of Oracy and Literacy. This was the first national document in England to identify learning objectives for intercultural understanding linked to language learning. The Framework was organised in five interrelated strands: three progressive strands -- Oracy, Literacy and Intercultural Understanding (ICU), and two cross-cutting strands, Knowledge about Language and Language Learning Strategies. Learning objectives and learning opportunities were defined for each year group for the progressive strands. The Framework intended to illustrate how to integrate ICU within language lessons and across the wider curriculum. Knowledge about Language and Language Learning Strategies were by their nature recursive, and although there was a clear read across to the learning objectives for the progressive strands, it was understood that knowledge about language and language learning strategies would be relevant to language learning in all four years of key stage 2 at different levels of complexity.

The definition of Intercultural Understanding was ambitious, framing the ability to conceptualise the child’s world from the perspective of other cultures and traditions. ICU was considered an essential component of citizenship, integrated with language learning -- both inside the language classroom and across the wider curriculum. These objectives are retained in the 2014 National Curriculum for England \citep{DfE2014}, which conceptualises language learning as ``a liberation from insularity'', fostering curiosity, deepening the understanding of the world, learning new ways of thinking, and reading great literature in the original language. Written in pre-Brexit Britain, the tone of the purpose of study for key stages 2 and 3 was expansive and optimistic:

\begin{quote}
Language teaching should provide the foundation for learning other languages, equipping pupils to study and work in other countries. \citep{DfE2014}
\end{quote}

In common with \citet{Leathes1918}, the current rationale for learning languages thus extends beyond the transactional to include cultural empathy and intercultural understanding, together with ``history, literature and civilisation of other countries'', fostering an international outlook and supporting personal development and concepts of global citizenship. However, the same duality of purpose prevails: languages as a skill; languages as ``liberal education''.  

\begin{sloppypar}
So, if there is consensus that rationales for language learning attempt to achieve a duality of purpose, developing language competence and to some extent cultural capital, with particular emphasis on developing intercultural understanding contributing to citizenship, we have defined the Why question, which leaves us to investigate the other key questions: What? When? And How? Policy decisions about why languages are included in the national curriculum should logically affect what programmes of study define and what teachers are required to teach. Decisions about when to introduce languages into the national curriculum and the appropriate starting age affect how they are taught, pedagogic principles and methodology, and should be guided by what researchers and practitioners understand about how children learn at different stages of development. The How question has another dimension relating to how to implement policy, and includes a range of challenges surrounding provision of suitably qualified teachers, appropriate resources, time in the curriculum, and effective transition arrangements, addressing how to ensure continuity and progression at points of transfer from primary to secondary education.  All of these decisions can (and should) be supported by research. 
\end{sloppypar}

In the next section of this chapter, we will revisit key moments in the history of language policy-making, and investigate the extent to which policy decisions took account of available research findings, or whether political and socio-political factors took precedence in the decisions taken with regard to the when, what and how questions. 

\subsection{When? Is younger better?}

The appropriate starting age for language learning in school curricula has always proved contentious, and has more recently received particular attention from \citeauthor{MylesEtAl2012}. It becomes particularly contentious if the main driver for the early introduction of primary languages is linguistic competence and the perceived advantages of an early start to building national capability in language skills. The risks of the instrumental agenda for languages based on age-related attainment outcomes overriding all other considerations are illustrated starkly by the government response to the National Foundation for Educational Research (NFER) evaluation of the Nuffield Pilot Scheme for the teaching of French in primary schools \citep{BurstallEtAl1974}. 

\begin{sloppypar}
The Pilot Scheme ran from 1964--1974 in England and Wales, introducing French into the primary school curriculum on an experimental basis from September 1964. The scheme took the form of a longitudinal study of three cohorts of pupils aged between 8 and 11, each cohort involving five to six thousand pupils. The main purpose of the experiment was to discover whether it would be both feasible and educationally desirable to extend the teaching of a foreign language to pupils who represented a wider range of age and ability than those to whom foreign languages had traditionally been taught \citep{Burstall1975}. It was agreed that the experiment would be subject to a ten-year period of evaluation by NFER. Main findings from the study were produced and published in two interim reports (\citealt{Burstall1968, Burstall1970}) and a final report \citep{BurstallEtAl1974}. 
\end{sloppypar}

The goals of the Nuffield Pilot Scheme were far-reaching and sought to investigate a wide range of academic, socio-cultural and socio-economic factors and their impact on early language learning, and also intended to explore the effect of language learning on other subjects. The study was conducted with the aims: 

\begin{enumerate}\sloppy
\item[(i)] to investigate the long-term development of pupils’ attitudes towards foreign-language learning;
\item[(ii)] to discover whether pupils’ levels of achievement in French were related to their attitudes towards foreign-language learning;
\item[(iii)] to examine the effect of certain pupil variables (such as age, sex, socio-economic status, perception of parental encouragement, employment expectations, contact with France, etc) on level of achievement in French and attitude towards foreign language learning;
\item[(iv)] to investigate whether teachers’ attitudes and expectations significantly affected the attitudes and achievement of their pupils;
\item[(v)] to investigate whether the early introduction of French had a significant effect on achievement in other areas of the primary school curriculum. \citep{Burstall1975}
\end{enumerate}

The conclusions of the NFER final report (\citeyear{Burstall1977}) were unequivocal that there was no perceived long-term advantage to progress made in language learning by virtue of the early start. This was in stark contrast to the prevailing view of the time. The final report stated that ``other things being equal, the older children tended to learn French more efficiently than the younger ones did'' \citep[247f.]{Burstall1977}. By age 16, there were no noteworthy differences in proficiency between early-starter children and later-starter non-project participants, except for minimal differences in listening comprehension which ``although statistically significant, were hardly of a substantial nature … a fairly minimal return for the extra years spent learning French in the primary school'' \citep[248]{Burstall1977}. Younger was not better. 

All other findings relating to the wider research purposes in particular those relating to ability, socio-economic factors, attitudes, achievement and motivation were given less attention. This was deeply regrettable, as empirical evidence about learning outcomes from children across the full ability range, and a study of the most appropriate methods to teach children from different socio-economic backgrounds would have been of considerable value in planning future initiatives for primary languages. It was also significant that Burstall suggested a link between positive attitudes generated in early language learning and greater L2 proficiency at a later stage: ``the development of attitudes towards foreign-language learning during later years may be powerfully influenced by the learner’s initial and formative experience of success or failure in the language learning situation'' \citep[235]{BurstallEtAl1974}. The central conclusion from the experiment that there was a lack of convincing evidence that younger was better curtailed further expansion of the Pilot Scheme and set back the progress of primary languages policy development for forty years \citep{Tellier2019}. 

\subsection{Who? What? and How? The challenges of teaching a specialist subject in the primary curriculum}

The lack of evidence of the advantage of the early start on linguistic outcomes was, indeed, the major factor in the government decision to withdraw funding in 1974, but was not the only factor at play. The socio-political context at the time of the Nuffield Pilot Scheme, despite firm advocacy and encouragement for languages emanating from the Annan Report, was not entirely favourable towards primary languages, as is evidenced by the publication of \citet{Plowden1967}. The Plowden Committee, while reserving judgement until evidence from the NFER evaluation of the ``experiment'' became available, did not give wholesale support to the introduction of primary languages. A number of organisational factors, similar to those raised by Leathes, were of concern to the Committee, and there was also an underlying pedagogical question about the place of language learning in the primary curriculum. Questions were asked about who would teach primary languages, what they would be teaching and how they would do it. 

The late sixties were a time of social transition when there was much debate about the relative rights of society and the individual \citep[493]{Plowden1967}. Would approaches to pedagogy for primary languages be at variance with the prevailing philosophy of teaching advocated by the Plowden Committee that prioritised the individual needs of the child? The Committee regarded fitting children for the society in which they would grow up as one obvious purpose of education \citep[494]{Plowden1967}. Education and pedagogic principles would inevitably change with the focus on child-centred education. The Plowden Report advocated a move away from formal class teaching to group work, projects and learning through social interaction, play and creativity. These arguments were influenced by Piaget and his findings on the late emergence of powers of abstract thought \citep[371]{Plowden1967}. 

There was concern that if formal teaching and specialisation were to be introduced too early, it would interfere with the development of the individual child. The teaching of a modern language was seen to present such a risk:

\begin{quote}
The introduction of a modern language into primary schools raises acutely the question of specialisation. It will be easier when many more primary teachers are qualified to teach French, but that time is still a long way off. In the meantime there is bound to be some anxiety lest the methods used in teaching French vary sharply from those used for the rest of the curriculum. The developing tradition in primary education since 1945 has been away from class teaching and from formal lessons, but the early stages of learning a modern language inevitably involve some class teaching and many teachers fear that much hard-won ground will have to be given up \citep[617 (iv)]{Plowden1967}.
\end{quote}

\subsubsection{The central importance of teacher supply and specialist subject knowledge}

The most hard-hitting argument mustered by Plowden against primary lang\-uages related to the conditions of success and the lack of appropriately qualified staff, taking account of subject knowledge in its broadest definition, encompassing language competence and pedagogic knowledge and understanding. 

\begin{quotation}
It is unfortunate that many schools and areas which are outside the experiment have chosen to add French to the curriculum without ensuring reasonable conditions for success. [...]The fact remains, that far too many schools have introduced French without having a teacher who possesses even minimum qualifications, without consideration of what constitutes a satisfactory scheme and timetable and without any consultation with receiving secondary schools. This can only be deplored. No good purpose can possibly be served by it. Without a teacher who is well qualified linguistically and in methods suitable for primary schools, it is better to have nothing to do with French. The presence of a native French speaker, while it guarantees the former, often fails to provide the latter \citep[617 (v)]{Plowden1967}\end{quotation}

The Committee furthermore had concerns over less able pupils and the suitability of teaching languages to the full ability range. The Plowden Report left little room for doubt that Committee members remained unconvinced by the progress of the implementation of the Nuffield Pilot at the time, and strongly counselled against expanding primary languages provision until the outcomes were fully known \citep[618]{Plowden1967}.

\subsubsection{Implicit versus explicit learning}

The Plowden Committee was not the first group of experts to question the wisdom of the early start. The relationship between literacy levels in the child’s first language and the learning and acquisition of a second language, together with considerations about stages of development that we may now refer to as meta-cognition, were under close scrutiny over a hundred years ago as the Leathes report shows. Leathes set out arguments for and against an early start at the age of nine or twelve \citep[114--120]{Leathes1918}, in other words either an early start in primary school where learning is more implicit or in secondary school where teaching approaches are more explicit. The proponents of an early start argued strongly for the advantages of the ``imitative faculty'' (implicit learning) which would enable younger children to acquire the new language readily by exposure and imitation \citep[115]{Leathes1918}. The counterargument refuted the value of pure imitation and made the case that pupils should not attempt the difficult task of learning a foreign language until they had acquired a reasonable mastery of their own \citep[117]{Leathes1918}. 

The substance for the counterargument relied on three main observations: first, that pupils should be familiar with elementary notions of grammar which would be necessary for the systematic study of a new medium of expression; second, that the mind of younger learners was not yet ripe for the serious study of a foreign language; and third, that early beginners would soon cover the whole of the content accessible to them and teachers would have to fall back upon a monotonous repetition of the rudimentary type of instruction. Leathes does not hesitate to point out the negative consequences of what would be regarded today as demotivation due to an unchallenging programme of learning: 

\begin{quote}
At first, the children may respond readily and brightly. Before long they grow weary of what they regard as nothing more than a singularly uninteresting form of game. In the end they become stale; and when they are old enough to have their work arranged on a system that is regularly progressive, they have lost the keenness which a new study should call forth. \citep[117]{Leathes1918} 
\end{quote}

\subsubsection{Planning progression and the problem of transition}

In common with Leathes, the Plowden Committee was plainly aware of the risks of repetitious learning arising from the introduction of primary languages. It was also clear that transition from primary to secondary schools could be problematic and that progression would not automatically continue cross-phase. The Committee saw this as a challenge shared by foreign languages, science and mathematics \citep[446]{Plowden1967}. Professor Eric Hawkins, who had served on the Plowden Committee, later reflected that the government had only itself to blame for the shortcomings of the Nuffield Pilot Scheme. Citing the Annan Report \citep{Annan1962}, \citet{Hawkins1996} remarked that ``the perceptive Annan Report not only prodded Government to take action but put down marker buoys on the very rocks on which the national Pilot Scheme, launched in 1963, was to founder in 1974.''

Annan had highlighted that: 

\begin{quote}
the attractions in starting to teach a modern language early are that pupils become familiar with the foreign idiom at an age when their imitative faculties are perhaps at their peak. It is of course, a prerequisite of success that the teachers themselves should have really fluent command of the spoken language and the methods they use should be up to date. To find or create such a body of teachers would take a long time and care would be needed to avoid the undesirable complications in the presentation of the language in secondary schools, which generally draw their pupils from a multiplicity of primary schools. \citep[para 63]{Annan1962}
\end{quote}

Issues over methodology, consistency of provision, teacher supply, teacher training, effective transition from primary to secondary schools, were all contributory factors in the demise of the Nuffield Pilot Scheme. These ``marker buoys'' warning of the dangers ahead are as relevant today as they were in 1918.

\citet{MuijsEtAl2005} raised similar concerns in their Evaluation of the Key Stage 2 Language Learning Pathfinders (2003--2005). They reported that some schemes of work in the Pathfinder project schools showed some evidence of differentiation and progression across the four years of key stage 2 , but others did not. Inconsistency in planning and delivery in some cases led to a repetition of the same content from one year to the next with no planned progression from year to year \citep{MuijsEtAl2005}. \citet{CableEtAl2010}, following a three-year longitudinal study of pupils in key stage 2, reported similar findings showing that there was very little assessment of pupils’ progress in their learning.

The seminal study by \citet{MylesMitchell2012} which explored the learning of French from ages five, seven and eleven brought together several of the challenges highlighted in previous policy initiatives such as starting age, motivation, attitudes, progression and attainment. The authors were particularly interested in the rates and routes of language learning, and addressing the question of grammar. Plainly, similar challenges continue to affect the implementation of primary languages in the national curriculum of 2014, and these are brought together and addressed by research-informed recommendations in the RiPL White Paper (\citealt{HolmesMyles2019}), drawing substantially on Myles’ research. 

Very little seems to have changed over time and the introduction of the statutory requirement in 2014 to teach a foreign language from the age of seven in England seems to have made scant difference. Due to a lack of coherent cross-phase planning, secondary schools often take little account of prior learning and start from scratch, meaning that pupils can find themselves repeating what they have already learnt which can lead to long-term lack of curiosity and interest, that may be a contributory factor to low uptake when languages become optional for pupils at key stage 4 at the age 14 (\citealt{TinsleyDoležal2018}; \citealt{HolmesMyles2019}).

It can be argued that history shows that policy makers rarely appear to learn lessons from the past and tend to be rather selective about lessons from research, seeking findings that are the most comfortable fit with political intentions. However, it is equally valid to argue that when policy decisions are preceded by government-funded research and recommendations from the evaluation of pilot studies are taken into account, there is a greater likelihood of successful policy implementation.

\section{Conditions for success}

There is clear evidence from history, that if the ``marker buoys'' are observed and the right conditions for success are put in place, positive outcomes can be achieved. One such example is the Scottish initiative launched in 1989, which served to inspire future policy initiatives in England that were to follow in the late 1990s and into the new millennium \citep{NuffieldFoundation2000}. This initiative was implemented in a manageable way. It started out with a small number of schools based on a cluster model where the secondary school would work with all of its primary feeder schools. The intention was to avoid problems at the age of transfer by ensuring that all of the pupils would share similar experiences of language learning in their schools. From the outset, it was made clear that an expansion to all primary schools would not be automatic but would be decided on the progress made. Gradually, further pilots were added and these were followed by regional initiatives. By 1992, there was sufficient confidence in the results of the Scottish pilots for the Secretary of State for Scotland to announce the intention to introduce primary languages into all primary schools in Scotland over the next five-year period \citep{Johnstone1996}. 

Encouraged by the Scottish pilots, the late 1990s saw government support for primary languages accelerating in England and Wales (cf. \citealt{MorganNeil2001}). Key developments were taking place through government-funded classroom projects and online support. Central to these initiatives in England was the cooperation between government, the Qualifications and Curriculum Authority (QCA), the Teacher Training Agency (TTA) and the Centre for Information on Language Teaching and Research (CILT). 

In 1999, the Good Practice Project, funded by government and run by CILT was established in England and Wales. It involved eighteen primary schools representing different types of school, a diverse range of pupils and different areas of the country. Each school was assigned a language teaching adviser, who would visit and support the development of classroom practice and curriculum planning, observing lessons and providing feedback, modelling practice and giving advice on resources. Professional learning was two-way; the language teaching advisers and the classroom teachers were project partners, co-constructing best practice and evaluating what worked most effectively.  The Good Practice Project ran for two years (September 1999 to March 2001) and published an evaluation report for government in 2001.

Support structures for the implementation of primary languages were put in place. In the late 90s, the National Advisory Centre for Early Language Learning was set up in CILT, London, with access to library facilities and online support. Regular bulletins were produced disseminating best practice and sharing case studies from schools participating in the Good Practice Project, and from other schools where primary practice was developing successfully. Initiatives were being put in place to steadily build capacity in primary languages teaching and learning, and they needed time to grow and time to embed. As the Good Practice Project continued and developed, a major shock took place in language policy for England that was to shape the future of primary languages policy, putting pressure on both primary and secondary schools in England.

\subsection{U-turn for fourteen-year-olds puts primary languages at the forefront of government policy}

The \citeyear{Morris2002} publication, ``Extending opportunities, raising standards, Green Paper'', by Estelle \citeauthor{Morris2002}, then Secretary of State for Education, illustrates how the tectonic plates had shifted for language policy: government attention was firmly focused on introducing an entitlement to language learning from the age of 7, while removing the statutory requirement for all pupils to study a modern language from 14 to 16. The flagship policy of ``languages for all'' from eleven to sixteen, brought in by the national curriculum from 1992, had largely failed, and the commitment to the introduction of primary languages was seen as the solution and counterbalance to that failure. The reasons for the policy U-Turn are of interest, since there is a degree of overlap in certain factors that affect language policy implementation in both primary and secondary phases. The expansion, both vertically in relation to the age group that were required to learn a modern language up to the age of sixteen, and horizontally in terms of offering language courses to the full ability range led to a shortage in the supply of adequately qualified teachers. There was also the need for intensive professional development to cater for the needs of a far broader pupil demographic, including pupils with special educational needs and disability.

There were other pressures affecting decisions at secondary school level about curriculum priorities affecting languages that also have resonance with decisions later to be made in primary schools. Alongside the national curriculum, the government had introduced performance league tables from 1992, primarily to monitor schools’ examination performance at secondary level and children’s performance in standard assessment tests in the core subjects of English, Maths and Science at the primary level. Concern surrounding overall school performance led headteachers in secondary schools to overuse disapplication procedures that allowed pupils to be removed from the study of particular subjects, like modern languages, to make room for support in English and mathematics or for vocational courses. Of the foundation subjects affected by disapplication, the teaching of modern foreign languages was undoubtedly the most severely compromised \citep{Morris2002}. We will see that later in the implementation of the primary languages policy from 2014, headteachers and class teachers will choose not to teach primary languages for lengthy periods of time in order to prepare pupils for Standard Assessment Tasks (SATs). 

\subsection{Government-funded research supporting the national languages strategy (2002--2010)}

Following the publication of \citet{Morris2002}, the implementation of primary languages had to accelerate as primary languages had assumed far greater priority in national language policy. Lessons from the Nuffield Pilot Scheme were not entirely ignored and were reexamined to some extent. Encouraging progress from the Scottish initiatives served to shape decisions around government funding to support the national strategy for England. There was also a serious commitment to research. In addition to the Evaluation of the Key Stage 2 Pathfinder Projects by \citeauthor{MuijsEtAl2005} (2003--2005), the government commissioned two three-year longitudinal studies. The first of these, which was conducted by the NFER between 2006 and 2009, intended to assess the nature and extent of language learning provision at key stage 2 in primary schools in England, and to evaluate progress toward the implementation of the national strategy target that all children from the age of seven should have an entitlement to language learning in class time by 2010. The focus of the NFER research was quantitative. It comprised an annual survey of primary schools, using a longitudinal sample~(including a representative sub-sample of 500 schools, selected to eliminate any possible bias), of all local authorities representing all of the different local government areas in England. 

During the same period, the Open University, the University of Southampton and Canterbury Christ Church University were commissioned to carry out a qualitative longitudinal study of languages learning at key stage 2. This study was to explore provision, practice and developments over three school years: 2006/2007, 2007/2008 and 2008/2009 in a sample of primary schools. The focus was on children’s oracy, literacy and intercultural understanding, as well as to identify possible broader cross-curricular impact of the introduction of languages learning at this stage. 

Both studies overall reported favourably on progress towards the implementation of the primary languages entitlement, while indicating research-based priorities for further development and investment over time, and highlighting areas for concern. By 2008, 92 per cent of schools were offering pupils in key stage 2 the opportunity to learn a language within class time and 69 per cent of schools were fully meeting the entitlement for all four years of key stage 2. Progress was being made toward full implementation, but nonetheless, there were warnings that around 18 per cent of schools were unlikely to be in a position to offer the entitlement by the target deadline of 2010. Typically, the most frequent language offered was French, followed by Spanish and then German. The common pattern of provision favoured a single lesson per week of around forty minutes, less than the recommendation in the Key Stage 2 Framework for Languages of one hour \citep{WadeEtAl2009}.

\citet{CableEtAl2010} reported similar findings, although the sample of 40 schools in this study showed that teaching time varied from 30 minutes to one hour per week, already providing a warning that finding sufficient time in the congested primary curriculum would continue to be an issue. Professional development was having a positive impact on provision, and schools were drawing increasingly on the Key Stage 2 Framework for Languages to plan lessons and to develop mid-to-long term curriculum plans. Teachers tended to concentrate on the oracy strand and to a lesser extent on literacy, but intercultural understanding was under-represented. There was very little assessment of pupils’ progress in their learning. Yet, empirical evidence from lesson observation and assessment tasks completed by a smaller sample of eight case study schools clearly demonstrated that children were making progress and could achieve the learning objectives set out in the Key Stage 2 Framework in oracy (listening and speaking) and some of those objectives related to reading set out in literacy strand. Children showed good knowledge of topic vocabulary, nouns, and set phrases, but they knew very few verbs and writing was underdeveloped. 

Evidence from NFER’s nationwide survey found that the majority of schools were choosing to provide language learning in discrete lessons, but the sample of 40 schools in the longitudinal study by \citeauthor{CableEtAl2010} found four distinct approaches to the delivery of the primary entitlement. These were lessons teaching the language, sensitisation to language(s) (tasters), language awareness, and language teaching through another subject (curriculum embedding/CLIL Content and Language Integrated Learning). Both NFER and \citeauthor{CableEtAl2010} reported that transition and transfer from key stage 2 to key stage 3 were proving to be challenging, and that planning for progression in the absence of developed assessment practices was variable.

\subsection{2010: A new government, a national consultation and a long-awaited policy decision}

This was the position for primary languages in 2010, when the general election returned a hung parliament to the House of Commons, resulting in a change of political leadership. The centrepiece of the National Languages Strategy to give every child between the ages of seven and eleven the entitlement to learn a language by 2010, promised by Andrew Adonis, Parliamentary Under-Secretary of State for Schools in 2005, was subject to a policy hiatus over a period of some four years, while decisions were made about the wisdom of introducing a statutory requirement. During this period of uncertainty, much of the significant national investment into training teachers and building the infrastructure to support primary languages was lost. Elizabeth Truss, then Education Minister, requested the Department for Education to conduct a national consultation in the summer of 2012 on the proposal to make foreign languages compulsory for primary school pupils aged seven to eleven. In its press release published on 17  November 2012, the DfE reported overwhelming support for the plan with nine out of ten respondents in favour. The Minister announced that the government would now make foreign languages a statutory subject at key stage 2 from 2014. The reasons behind this decision were influenced not only by the public consultation, but also in the belief that the early start could prevent the slide in standards and in uptake at key stage 4. In 2010, uptake at GCSE had fallen to an all time low of 40 per cent. England, still a member state of the European Union, had suffered humiliation in the First European-wide survey of language competences of teenage learners conducted by the European Commission, being ranked bottom of the table, underscoring the need for the government to prioritise modern languages. Once again, the early start was thought to provide the solution to the challenge of improving national capability in language skills. 

The earlier parts of this chapter have documented the political, socio-political and educational factors that over the last 100 years have influenced, and ultimately provided the impetus for, the introduction of statutory foreign language teaching in primary schools in England by the UK Government in 2014. The Programme of Study, however, was published in 2014 without explicit reference to previous primary policy initiatives or to relevant research into primary language policy and primary language pedagogy. In the next section, we will highlight the central role of Professor Florence Myles in raising the profile and relevance of research in influencing and informing policy formation in the current educational context. Myles’ research into language learning in middle childhood, illustrates why policy decisions should address the why, what, when and how questions, taking full account of research findings and practitioner experience and expertise. 

\section{Research on language learning in middle childhood}
\begin{sloppypar}
Research shows that younger learners learn differently from older learners in classroom contexts. Younger learners are enthusiastic and receptive to new sounds, new words and new worlds \citep{Myles2017}. If, however, the question of whether younger is better is framed purely with regard to attainment outcomes, then research consistently shows that younger learners are less efficient than older learners (\citealt{MylesMitchell2012}; Barcelona Age Factor (BAF) project, \citealt{Munoz2006}). \citet{MylesMitchell2012} found that older children learned faster, and this was related to their use of cognitive strategies to support their learning and to their more advanced literacy skills (\citealt{MylesMitchell2012,Myles2017}). They also found in the same study, that the younger learners were particularly enthusiastic and receptive to new language and new cultural input. Younger learners learn implicitly, and rich and plentiful input plays a key role in language learning in middle childhood (from ages 6/7 to 11/12). This means that younger learners require a greater amount of curriculum time and quality of input than are currently being provided by typical classrooms in England (\citealt{HolmesMyles2019}). These research findings thus have implications for national expectations of progress and also for national rationales for the early introduction of languages to the curriculum (\citealt{MitchellMyles2019}). Myles raises awareness that if proficiency is the only driver for the early introduction of languages, then research evidence suggests that it is not the strongest argument for policy change (\citealt{Myles2017,MitchellMyles2019}). She puts forward strong arguments in favour of the motivational, cultural and cognitive benefits of early language learning \citep{Myles2020}. There are clearly linguistic benefits from learning another language in addition to the first language, and Myles recommends that links with L1 literacy and all the languages children know and are learning need to be strengthened (\citealt[10]{HolmesMyles2019}). But it is the benefits to the personal, social and cultural development of children that Myles believes are undervalued in favour of the focus on proficiency in language skills \citep{Myles2017}. There are certainly implications of how and what teachers are required to teach, if our rationale for primary languages changes its emphasis and focuses more robustly onto cultural aspects as well as linguistic content. The teaching of modern languages in a country that believes that it speaks the global language presents particular challenges \citep{PorterEtAl2020}. Securing and sustaining the motivation of learners are crucial to raising standards and uptake, and cultural input is seen to interest and inspire young learners. As Myles has pointed out, [i]t seems that even an hour per week has the potential to awaken a lifelong interest in foreign languages, which must be welcome in a country where foreign language learning is undervalued and in crisis \citep{Myles2017}.
\end{sloppypar}

\section{The development of the Research in Primary Languages (RiPL) network}

Myles’ research findings into second language acquisition prior to 2014 (e.g. \citealt{Myles2014}) contextualised how research into second language acquisition (SLA) over previous decades fed into the understanding of language learning and teaching in classroom contexts and, more latterly, has contributed to identifying the implications for the introduction of compulsory foreign language learning at key stage 2, highlighting issues and questions surrounding these implications, namely: a lack of guidelines and adequate training for teachers to implement the new policy, especially important for practitioners with no previous experience of language teaching; a lack of adequate and age-appropriate teaching materials; time allocation given to the provision of primary languages; curriculum content; assessment; and transition (\citealt{HolmesMyles2019}).

It was clear from the outset that primary schools were faced with many challenges surrounding the implementation of the statutory order to teach primary languages from the age of seven introduced in 2014, and did not have the resources or guidelines to ensure that they could deal with these challenges successfully. This disconnection between research findings, the primary practitioner context, and government policy prompted Myles to realise her vision of a re\-searcher--prac\-ti\-tion\-er stakeholder network which would provide a forum for combined stakeholders to consider and explore the central role of research in developing age-appropriate teaching methods and the kind of language pedagogy appropriate for children of primary-school age. Building on her own research and in collaboration with other leading academics, Myles planned to mobilise research findings and support policy-making at national and local levels by providing access to research and to researchers which could inform and influence policy and practice. 

Myles thus established the Research in Primary Languages (RiPL) Network in 2018, bringing together active researchers, prominent in their field, with policy developers and practitioners, to inform, influence and develop primary languages policy and its implementation in whatever ways were open to them. The intention was to address, and hopefully to counter, the false starts of history. By serendipity, the launch of the RiPL network and the subsequent publication of its White Paper in 2018 coincided with the centenary of the publication of the Leathes Report in 1918. This simple coincidence becomes a cogent reminder of how little appears to have changed in the fortunes of primary languages in over a hundred years.

The network grew out of a successful ESRC bid developed by Myles and Mitchell for a seminar series in 2015 on the topic of Early Foreign Language Learning in an Anglophone context. From its inception, the idea was to build a network of researchers and practitioners in the UK, to identify and address some of the issues facing primary languages following their introduction into the statutory curriculum from September 2014, and provide research-informed solutions for them. 

Building on Myles’ vision and drawing on her own considerable expertise and experience in the field, strengthened by that of her network partners, the RiPL Network rapidly became a central line of communication between practitioners, teacher educators, other professional stakeholders, and policy makers. The uniqueness and success of the network lie in its informed approach underpinned by published research, experience of policy development and active collaboration with classroom practitioners. 

The Network’s publicly accessible website (\href{http://www.ripl.uk/}{www.ripl.uk}) has become a hub of information on all things relating to and about language learning in primary schools. For example, it features overviews of state-of-the-art research central to the field authored by university-based researchers, leaders in their field and policy advisers: themes include the role of age in learning; cultural competence and intercultural understanding; curriculum models and curriculum policy; transition from primary to secondary school; pedagogy and teacher expertise; linguistic development and expectations; multilingualism and additional language learning; literacy, foreign language learning and wider academic achievement; and Content and Language Integrated Learning (CLIL). 

Additionally, researchers, under the direction of Myles, have produced one-page summaries of research articles of interest and relevance to practitioners in Primary education and Secondary education, with particular reference to transition in the latter case. These have been written in non-academic language to ensure that they are also easily accessible to a non-academic audience. From feedback given by practitioners following the very successful MOOCs -- ``Teaching and Learning Languages in Primary Schools: Putting Research into Practice'', the summaries have proved a valuable resource for teachers, teacher trainers, journalists, and policy makers, and have lived up to their aim in helping to reshape thinking and training. They are complemented by a section on resources for teachers, a ``School Focus'' section which features inspiring examples of best practice, and a regular blog which features articles by practitioners and researchers to keep the network abreast of relevant events and developments. The website also hosts policy documents and policy reviews.

\section{Synergy between research and policy development}

The collaboration between Myles and Holmes, respectively chair and co-chair of RiPL, was instrumental in bringing together leading research and policy expertise. Under the direction of Myles and Holmes and RiPL collaborators, workshops and summits brought together a comprehensive range of stakeholders interested in primary language learning. Discussions arising from the Primary Languages Policy Summit, which took place on Friday 23rd November 2018 at the British Academy, subsequently fed into the RiPL White Paper authored by \citet{HolmesMyles2019} which summarised and evaluated the state of primary languages provision and issues and the challenges that practitioners and schools faced in implementing government policy at that time. In conclusion the White Paper put forward ten recommendations, providing research-informed solutions to some of the problems and questions surrounding effective implementation of primary languages. The recommendations focused on specific key priorities that should be addressed if appropriate conditions for the success were to be put in place to support the implementation of policy. These were: 

\begin{itemize}
\item allocation of a minimum teaching time to ensure progression; 
\item primary pedagogy developed through initial teacher training and CPD provision; 
\item curriculum planning across phases; 
\item transition arrangements between primary and secondary school; 
\item assessment and reporting to ensure continuity; 
\item use of digital technology; 
\item importance of school accountability; 
\item role of school leadership; 
\item research programme to address gaps in understanding of age-appropriate pedagogy;
\item the necessity of creating a national taskforce to address the challenges faced by schools and to coordinate the implementation of the national policy. 
\end{itemize}

The White Paper and its recommendations were endorsed by the British Association for Applied Linguistics (BAAL). It has been cited in \textit{Language Trends} 2020 \citep{Collen2020} and \textit{Ofsted Research Review} 2021 \citep{Ofsted2021} among other publications. Myles, representing and liaising with the RiPL Network, was invited to advise on the development of primary languages for Oak National Academy, the online resource set up to support teaching and learning throughout the pandemic. Thanks to Myles’ initiative, research on primary languages and other school-centred issues and challenges in early language learning have become more mainstream and, as such, increasingly harder for policy makers at local and national levels to ignore.

The most recent impact of the RiPL White Paper and its emphasis on the interrelationship of literacy in the first language, other languages that the children speak, and the learning of new languages, can be seen in the information document published by the Association of School and College Leaders \citep{OFarrellAndersonHolmes2022}, and distributed to 19,000 schools and trusts. This seminal document provides guidance for mainstream primary schools in how to develop an inclusive curriculum. 

Furthermore, the combined focus of research and policy activities of RiPL, including the RiPL White Paper, became a central part of the Research Excellence Framework (REF 2021) submission from the University of Essex Department of Language and Linguistics. The REF is the national assessment of research quality and impact carried out every four years at UK universities, it has national implications for both funding and reputation. A panel of experts assesses submissions for originality, significance, and rigour of research activity, evaluating these against criteria, and awarding these as 4* world-leading, 3* internationally excellent, 2* recognised internationally and 1* recognised nationally. The RiPL case study was deemed to be world-leading (4*), therefore contributing to the overall result of joined 1st in research impact in modern languages and linguistics in the Times Higher Education ranking of the REF 2021 results.


\section{Conclusion}

The summary of the historical debates, initiatives and recurring issues presented in this chapter clearly underscores that, were greater attention given to available research findings, and if new research to accompany primary languages policy were more systematically commissioned, then policy implementation would be more effective. It is crucial that a deeper research-informed understanding of how learners learn a new language most effectively should underpin policy decisions and should be included in initial education and in continuing professional development. All these challenges are comprehensively addressed in the White Paper through research-based recommendations, based largely on the body of work undertaken by Myles, and also benefits from the collaborative work of researchers, practitioners and policy makes brought together by Myles in the RiPL Network. 

In sum, we attest that where research and policy-making work in synergy, progress towards the implementation of primary languages has been strengthened, but the journey from policy to practice has been interrupted many times, and is often overly influenced by political expediency and other socio-political factors which can obscure lessons from research. The why, what, when, and how questions still require research-informed guidance to ensure that the ``marker buoys'' are in position to help us avoid the hazards of history and that the right ``conditions of success'' called for by \citealt{Leathes1918}, \citealt{Annan1962}, \citealt{Plowden1967}, \citealt{MuijsEtAl2005}, \citealt{CableEtAl2010} and \citealt{HolmesMyles2019} are put in place. This is a journey worth continuing. 

\sloppy\printbibliography[heading=subbibliography,notkeyword=this]
\end{document}
