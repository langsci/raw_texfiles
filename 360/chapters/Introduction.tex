\documentclass[output=paper]{langscibook}
\author{Kevin McManus\orcid{}\affiliation{Penn State University} and Monika S. Schmid\orcid{}\affiliation{University of York}}
\title{Introduction and overview of the volume}
\abstract{\noabstract}
\ChapterDOI{10.5281/zenodo.6882173}

\begin{document}
\maketitle

\noindent This volume is a celebration of the academic achievements and scholarship of Professor Florence Myles as a world-leading scholar in the fields of Second Language Acquisition (SLA) and French Linguistics, and in particular for her work in corpus-based SLA and, more recently, language policy in primary school education. As Li Wei highlights in his Preface to the volume, Florence is a prolific researcher and leader whose work has helped change and shape the field of SLA for decades (see also Roger Hawkins’ Postscriptum). For example, the ground-breaking work on the creation and development of the \textit{French Learner Language Oral Corpora} (FLLOC)\footnote{\url{http://www.flloc.soton.ac.uk}} project which Florence has conducted with her long-standing collaborator Professor Rosamond Mitchell has provided a crucial resource to support many studies into the development of French interlanguage. This impressive collection of openly accessible learner corpora now houses nine rich datasets, many of which are the direct result of grants that Florence held at various times (e.g., \textit{Linguistic Development Corpus}, \textit{Newcastle Corpus}, \textit{Young Learners Corpus}). The compilation and publication of these corpora constitutes a major innovation in the field. Indeed, through this work, Florence and Ros showed how making research data openly available and easily accessible can support collaboration and help our understanding of learner languages to develop. Florence’s dedication to making research data accessible to all has not only been a hallmark of her work but inspired many others to follow suit. Florence's collaborations with researchers working on the acquisition of other languages (e.g.,~Laura Domínguez) have paved the way for other repositories of open data, including the \textit{Spanish Learner Language Oral Corpora} (SPLLOC)\footnote{\url{http://www.splloc.soton.ac.uk}} project. 

A critical impact of Florence’s work on language learning in primary schools, primed by her ESRC grant “Learning French from ages 5, 7 and 11: An investigation into starting ages, rates and routes of learning amongst early foreign language learners”, can be seen in her work on language learning policy. A recent example of this is a joint response article to the Ofsted curriculum research review, which she co-authored with Alison Porter, Suzanne Graham, and Bernardette Holmes \citep{PorterEtAl2022}. This paper contains evidence-based recommendations advocating for a more holistic and nuanced approach to language education that is centred around opportunities to actively communicate in the language and supported by rich and plentiful input. The evidence base underpinning these recommendations for supporting primary languages education arose from the Research in Primary Languages (RiPL)\footnote{\url{https://ripl.uk/}} network, which Florence founded, and is chair of. Today, RiPL is an impressive collaboration of researchers, teachers, teacher educators, and policy makers with interests in research in primary languages. Florence's dedication to community building is further evident in her pioneering role in establishing research centres at various institutions (such as the \textit{Centre for Research in Linguistics and Language Sciences} at Newcastle University, and the \textit{Centre for Research in Language Development throughout the Lifespan} at the University of Essex) have provided a framework and structure in which scholars across disciplines come together, work together, discover synergies and commonalities in their research and disseminate their findings to experts and laypeople alike.

Being a research leader is a continuous characteristic of Florence's career, and her activities in these areas include such roles as president of the European Second Language Association (EuroSLA), the editor of the \textit{Journal of French Language Studies} (published by Cambridge University Press), and membership on the Editorial Board of journals such as \textit{Second Language Research}, \textit{Language Teaching}, the EuroSLA \textit{Yearbook} series, and the \textit{Revue Française de Linguistique Appliquée}. She is also co-author of the internationally best known and most widely used textbook on SLA, \textit{Second Language Learning Theories} (with Rosamond Mitchell and Emma Marsden), now in its 4th edition. In addition, Florence has supervised a large number of PhD students over the years, conducting theses in a range of topics and areas in the field of language learning, and provided support and mentorship to many more young scholars. 

All of these activities underline the one characteristic that probably defines Florence most clearly, alongside her brilliance as a researcher on language learning and language policy and her dedication towards her teaching, and that is her immense generosity of spirit. Throughout her career, she not only has been a team player, but she has never shied away from taking on the hard and sometimes thankless tasks that were needed – by the departments in which she worked, by the associations and publications supporting and disseminating research, and by the field at large – for others to thrive. Most of the contributors to this volume, and probably a fair proportion of its readers, will at one time or another have benefitted from her marvelous hospitality and been uplifted by her cheerful kindness. 

As the chapters in this volume demonstrate, Florence’s impact on the field of language learning are far-reaching and considerable. This collection speaks not only to the diversity of her scholarship, but also to the ways in which she has always gone the extra mile in her efforts to mentor and collaborate with others.

\begin{sloppypar}
The volume begins with a chapter authored by Bernardette Holmes and Angela Tellier, who show how languages policy can benefit from research. In particular, they highlight the major contribution that Florence Myles has made to our understanding of cognitive development in middle childhood, and how this affects children’s learning of a second or foreign language in instructed settings. In addition to presenting an overview of language policy over the last century, Holmes and Tellier show that the synergy between research and policy has rarely been optimised, and that perennial questions about the why, when, what, and how of primary languages have yet to be fully addressed. The chapter concludes by illustrating how Myles' specific contribution and her vision for research-in\-formed practice for primary languages have created dialogue between researchers and policy makers in recent years, in particular through the establishment of the RiPL network.
\end{sloppypar}

In the next chapter, Rosamond Mitchell and Sarah Rule, two of Florence’s long-standing collaborators, bring together evidence from the ESRC-funded study ``Learning French from ages 5, 7 and 11", (\citealt{MylesEtAl2012, Myles2017}) regarding the development of target language vocabulary knowledge by early learners over a year’s instruction in French as a foreign language. The data analysed includes lesson plans, video recordings and transcriptions of complete lessons, as well as receptive vocabulary tests constructed to systematically sample the vocabulary actually taught. The chapter draws conclusions about that the rate of progress in vocabulary learning in a constrained classroom context and highlights the factors which seem to promote development most consistently.

With a focus on the contributions of research to the professional development of language teachers, Alison Porter and Suzanne Graham outline what is currently known about primary school teachers’ knowledge and beliefs about language pedagogy and what research-informed principles might be important for them to know and understand. They then present data from an online training initiative designed to develop teachers’ understanding of primary languages pedagogy and practice. The chapter concludes by considering the implications of the study for models of primary school language teacher development and areas for future research.

In the chapter by Rowena Kasprowicz, Karen Roehr-Brackin, and Gee Macrory, the key questions are about the place of form-focused instruction and the related debate about the role of metalinguistic awareness among young learners. The authors begin by outlining conceptualisations of metalinguistic awareness, followed by a summary of key empirical studies investigating child learners’ metalinguistic abilities. Then, analyses are presented that speak to the question of whether explicit grammar instruction can effectively develop young learners’ verbalisable metalinguistic knowledge. This chapter concludes by integrating these findings with previous work in order to highlight the level of metalinguistic awareness which primary-school children are able to develop in instructed settings.

In the following chapter, Emmanuelle Labeau and Raquel Tola Rego provide an overview of the Primary Languages landscape in England and present the best practice case of Hackney Education with special attention to its recent Content and Language Integrated Learning (CLIL) developments. The authors argue that a CLIL approach has the potential to address the challenges of implementing the national Modern Foreign Languages (MFL) entitlement at primary level by reducing the timetabling constraints, expanding the teacher pool, and adapting to children’s cognitive development. In concluding, this chapter shows how the CLIL approach fits the cognitive development -- as identified by the Research in Primary Languages (RiPL) project -- of primary school pupils who learn implicitly, by being immersed in the language and using it.

Kevin McManus and Brody Bluemel present findings from a recent study on dual language immersion programs in the United States. That study investigated teachers’ instructional practices in English-Chinese and English-Spanish kindergarten DLI classrooms using video, audio, and observation data. Their results indicated important differences and similarities for (i) teachers’ language use in the different classrooms and (ii) teachers’ instructional practices in the different languages. Using these data, teachers’ instructional practices, the availability and type of instructional input, and their impact on opportunities for learning are discussed as ways to inform decisions about subject content teaching and language development in dual language classrooms.

\begin{sloppypar}
In the next chapter, Victoria Murphy and Hamish Chalmers provide an overview of recent discussions about the benefits of being bilingual, including improved executive function and inter-cultural understanding. In the field of foreign language education, related research has, quite understandably, focused on the implications of being bilingual on the teaching and learning of a foreign language itself. Their chapter explores the small body of work that speaks to whether and to what extent knowing more than one language can impact academic achievement, literacy, metalinguistic awareness, employment opportunities, and so on. Using this basis, the authors identify some of the methodological issues that make interpreting work of this type problematic, and set forth a research agenda to more rigorously address this important area of inquiry.
\end{sloppypar}

In the final chapter, Laura Domínguez and María J. Arche present ongoing work on the emergence and development of null and overt subjects in L2 Spanish, by English-speaking learners. The oral data for this study were collected using a paired discussion task and a story retell and are freely available from the SPLLOC project.\footnote{\url{http://www.splloc.soton.ac.uk}} The authors claim that the cline of difficulty suggested by \citet{ChoSlabakova2014}, based on whether L1-L2 form-meaning mismatches require reassembly and whether a dedicated morpheme is available, makes appropriate predictions for these structures. It is also argued that the type of task used to elicit the oral data and the overall linguistic and narrative abilities of the learners are also likely to influence the rate of use of these forms.

Taken together, the work presented in this volume speaks to the influence of Florence Myles’s work in many different areas of SLA research, including theory-building, corpus-based investigations, studies of language development, as well as informing teacher professional development through research. We invite readers to learn more about the fascinating research presented here as inspired by Florence’s dedication to field!

\printbibliography[heading=subbibliography,notkeyword=this]
\end{document}
