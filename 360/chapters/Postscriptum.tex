\documentclass[output=paper]{langscibook}
\ChapterDOI{10.5281/zenodo.6811476}
\author{Roger Hawkins\orcid{}\affiliation{University of Essex}}

\title{Post scriptum}
\abstract{}

\begin{document}
\AffiliationsWithoutIndexing{}
\maketitle

\begin{flushright}
\textit{La connaissance des langues est la porte de la sagesse.} --- Roger Bacon (13\textsuperscript{th} c.)
\end{flushright}

\noindent Approaches to studying how people learn to speak more than one language, whether simultaneously from birth or sequentially (learning one or more ``foreign'' languages after a single mother tongue has been established), have always been diverse. Some researchers have focused on the nature of the linguistic knowledge involved, some on the psychological concomitants of language knowledge (like processing in real time or interactions between linguistic and non-linguistic cognition), some on the social settings in which languages are learned, and yet others on the effects of different kinds of input on learning outcomes.

The work of Florence Myles shows a remarkable openness to this diversity of approach. In contrast to some of her fellow researchers who show a certain tunnel vision in pursuing their own corner of the field, Florence actively seeks answers through interdisciplinary investigation. Her book with colleagues Ros Mitchell and Emma Marsden, \textit{Second language learning theories}, bears witness to this openness to different perspectives, an openness clearly appreciated by its readers, given that the book is now in its fourth edition. But it is also in some of the roles that Florence has taken on during her career that we get a sense of her commitment to interdisciplinarity as a route to understanding bilingualism/second language acquisition. She was instrumental in creating the Multi-disciplinary Centre for Research in Linguistics and Language Sciences at Newcastle University. She was the driving force behind, and founding director of, the Centre for Research in Language Development Throughout the Lifespan at Essex University. She has been president of the European Second Language Association, a body that embraces diversity in approaches to the study of second languages. And most recently she was the founding Chair of the Research in Primary Languages Network, bringing together researchers and language teachers to find practical solutions to the challenge of teaching foreign languages to children of primary school age.

The present volume is a fitting tribute to Florence’s work over more than three decades to increase our understanding of what it means to know more than one language, presenting as it does original studies of teaching methods used in instructed language learning settings, aspects of learner language and the role of social and personal factors in learning.

If knowledge of languages is the door to wisdom, as Bacon proposed, Florence’s work has shone a particular light on that portal. She has inspired researchers and teachers alike to look more closely at the factors involved in learning other languages and she has put in place structures that will foster collaborative work and greater understanding far into the future.

\end{document}
