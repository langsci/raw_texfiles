\documentclass[output=paper]{langscibook}
\ChapterDOI{10.5281/zenodo.6811470}
\author{Victoria Murphy\orcid{}\affiliation{University of Oxford} and Hamish Chalmers\orcid{}\affiliation{University of Oxford}}
\title{The impact of language learning on wider academic outcomes}
\abstract{Much is claimed about the benefits of being bilingual. From improving executive function, through promoting inter-cultural understanding, to mitigating the effects of aging on cognitive decline, bilingualism is framed as something of a panacea for the human condition. In the field of foreign language education, related research has, quite understandably, focused on the implications of being bilingual on the teaching and learning of a foreign language itself. Little of this body of research has expanded its purview to explore the causal implications that learning a foreign language (as opposed to already knowing one) may have in terms of the benefits observed in the bilingualism research. In this chapter, we explore the small body of work that speaks to whether and to what extent knowing more than one language can impact on outcomes like academic achievement, literacy, metalinguistic awareness, employment opportunities, and so on. We identify some of the methodological issues that make interpreting work of this type problematic, and we set forth a research agenda to more rigorously address this important area of inquiry. We conclude by arguing in support of a more prominent position for foreign languages in the primary classroom, but temper this with a caveat to the research community that we need more systematic and rigorous research about the effects of foreign language learning if we are to bring the rest of society along with us.}

\begin{document}
\SetupAffiliations{mark style=none}
\maketitle 


\section{Introduction}

Are there advantages to being bilingual? This is a straightforward yes or no question with no agreed-upon straightforward answer. From one perspective, the answer is contained in the question itself, and is obvious. Regardless of whether bilingualism is defined as full competence in two languages, or more modestly, some knowledge of another language, the very fact of bilingualism is self-ev\-i\-dent\-ly advantageous because as the world gets ever smaller, and transmigration introduces ever richer linguistic tapestries to our daily lives, having an expanded linguistic repertoire with which to engage with that world is clearly personally and societally beneficial. A credible answer to the question then, is yes! 


However, when academics pose this question, they are typically asking about something else. Not the self-evident, and perhaps arguably run of the mill, advantage of bilingualism itself (after all, more people in the world are bilingual than not (\citealt{Grosjean2010,Grosjean2014,BialystokEtAl2012})). Rather, they are likely referencing research over the past few decades that explores the cognitive effects of knowing and using more than one language. For example, whether bilingualism is associated with improved executive functioning. This body of research has been hotly debated (\citealt{BialystokEtAl2012,PaapGreenberg2013,DeBruinEtAl2015,Friedman2016}), and remains contested, despite the significant amount of work that has aimed to provide clarity on this question. As far as this body of research is concerned, at best, the only thing that one can state with confidence is that the evidence is ``inconsistent'' (e.g. \citealt{WollWei2019}).

While the question of potential cognitive advantages in bilingual children is interesting (albeit somewhat belaboured by now), in this chapter we focus on a broader concern. Namely, what evidence exists to inform us about whether learning other languages has outcomes that go beyond the ability to use more than one language, and which addresses less ephemeral aspects of cognition than reaction times on Stroop, Simon, and Flanker tests.\footnote{Stroop, Simon and Flanker tasks are commonly used in experiments to assess cognitive functioning. They involve asking participants to respond differently to different types of stimuli, such as pressing a red button when seeing a green colour, a green one when seeing a red colour, and so on. The speed and accuracy with which they respond is used as a measure of cognitive functioning.} In particular, we are interested not only in whether bilingualism in children might have an impact on other aspects of their cognitive or socio-affective performance, but whether learning a foreign language (FL) in taught, input-limited contexts like classrooms, might have knock-on effects in other domains of learning. Increasingly, FL learning is being introduced as a compulsory subject in primary schools \citep{Murphy2014}. In Europe, for example, the age at which compulsory FL learning begins is as low 3 (Belgium), and most European children have started learning an FL formally by the age of 9 \citep{Devlin2015}. While in Europe and other non-anglophone countries, the emphasis tends to be on learning English -- though note that all but five European countries require children to learn \textit{two} FLs \citep{Devlin2015} -- in the UK, this policy change is chiefly in response to concerns about Britain’s global competitiveness, in view of waning interest in foreign languages at all stages of formal education \citep{Chen2018}. However, this shift in policy has also sparked interest in whether FL learning might be ``sold'' on the basis of other potential benefits, too \citep{FoxEtAl2020}. It is timely, therefore, to consider what the evidence can tell us about the wider impacts of learning or knowing more than one language. 

This is a particularly relevant question given the shift of formal language learning to stages of the learners’ lives where aspects of the first language (L1) are still in development (e.g., literacy). The L1 system might therefore be more susceptible to some form of influence by the introduction of new languages. There have, to our knowledge, been remarkably few studies that have investigated this, despite the prevalence of FL instruction around the world at increasingly younger ages. We will discuss the small but burgeoning work in this area in this chapter. 

The foundation for our discussion stems from a piece of work that we, along with colleagues at the University of Oxford, carried out in 2019/2020 \citep{MurphyEtAl2020}. We were commissioned by the Education Endowment Foundation (EEF), a UK-government-funded educational research centre, to carry out a Rapid Evidence Assessment (REA),\footnote{A rapid evidence assessment is a type of systematic review, where literature addressing a specific question or set of questions is systematically searched for, located, appraised, synthesised and transparently reported, to give an unbiased account of the findings of the body of research as a whole. In an REA, some of the elements of a standard systematic review are omitted to speed up the process. This allows for a quick ``temperature check'' of the field before a more thorough review can be conducted. Our full report is free to download at \url{https://educationendowmentfoundation.org.uk/education-evidence/evidence-reviews/foreign-language-learning}.} reviewing research that addresses the issue of the wider academic outcomes of foreign language learning. The objectives of this REA were to summarise evidence on the effective teaching of a FL and the impact of learning a FL on other academic subjects. The aim was to provide some practical recommendations on best practice in teaching FLs with a view to maximizing benefits of wider academic outcomes, should any be found to exist. Evidence from across the globe was reviewed in three major areas: best practice in FL teaching; the impact of learning a FL on wider outcomes; and the impact of using non-native languages as the medium of instruction. For the purposes of this chapter, we focus on the second of these areas, namely, what the research has so far revealed about wider impacts of learning other languages. 

The EEF, who commissioned the original report, are particularly focused on research that speaks to effective classroom practice. They are the designated ``what works'' centre for education in the UK and have commissioned a large number of evaluations of teaching interventions. These have adopted research designs most well suited to supporting causal inferences; randomised control trials (RCTs) in the main, but also quasi-experimental designs (QEDs). These designs have for some time been considered a gold-standard in effectiveness research. 

This is because inferring causality depends on making fair comparisons. To evaluate any causal relationship between a teaching approach and student outcomes, the teaching approach must be compared with an alternative. The \textit{comparison} provides the context for any claims about ``what works''. However, comparisons can be more or less \textit{fair} in the way they are made. It is common in FL research to find studies that make unfair comparisons or no formal comparison at all. One way in which comparisons can be unfair is in the way that the groups being compared are generated. If groups of students being compared are systematically different when the research begins (at baseline), we cannot be sure that any difference in outcomes at the end of the research is attributable to the teaching approach or the characteristics of the groups being compared. In QEDs, researchers attempt to create groups that are similar in terms of characteristics that are considered important, through statistical matching. For example, a researcher might ensure that the average age between groups is similar, or that the average scores on FL tests are similar. However, while we can make informed guesses about what characteristics are likely to be important and then match for them, we can never be sure that we have accounted for all the possible influences on student outcomes. RCTs help to address this by randomly allocating individuals or groups of individuals (cluster RCTs) to comparison groups. Random allocation ensures that any differences between groups at baseline is only a result of the play of chance and not a result of systematic differences between them. Any difference in outcome can then be more confidently attributed to the intervention and not the characteristics of the groups. RCTs are thus the most trustworthy design for ensuring that like is being compared with like \citep{Chalmers2018}. Whether outcome measures are academic achievement, scores on a vocabulary test, levels of student engagement, teacher attitudes, or anything else, we benefit when researchers have endeavoured to minimise the potential for biases to mislead us when we make those comparisons.

In the UK, the uptick in interest in RCTs since the EEF’s inception, after which the term more firmly established itself in the discourse around educational research, has led many to advocate for the increased use of these research designs in educational settings (\citealt{ConnellyEtAl2017,Chalmers2018}). Other designs have their place, and there is a lot to be learned from more descriptive, exploratory, and correlational type research. However, in line with the EEF’s own research, the focus of our REA was specifically on available evidence from RCTs and QEDs. Hence research of this type will be the focus of our chapter. In the following sections  we present the major findings of this REA in relation to the effects of knowing, learning, and/or using more than one language on outcomes beyond just the self-evident one.

\section{Bilingualism and cognitive abilities}\label{sec:murphy:1.1}

As we’ve already mentioned, the relationship between bilingualism and cognitive skills has been much investigated over the past few decades. Some researchers have extolled numerous perceived virtues of being bilingual, arguing that bilingualism confers specific advantages related to greater mental flexibility -- in the form of task switching, attentional control, staving off dementia, and the like (e.g., \citealt{BialystokEtAl2012,Costa2020}). Other researchers are clear that there is a great difficulty in replicating these findings and that there is little compelling evidence to support this claim (e.g. \citealt{PaapGreenberg2013,WollWei2019}). Many of the studies we found in in our REA compared already bilingual participants with monolingual participants on specific outcome measures, rather than comparing the effects of learning a FL, specifically. We will, therefore, mention some of this work in our discussion, but we will try to be clear about whether we are referring to research on bilingualism or to research on FL learning. Defining the construct of \textit{bilingualism} is a challenging endeavour in and of itself and (yet) another area where there is little agreement \citep{Murphy2014}. For the purposes of our discussion, we distinguish between children who are learning two (or more) languages in naturalistic settings and those who are receiving instruction in another language in the context of formal education. The former we will refer to as bilingual; the latter we will refer to as FL learners.\footnote{Of course, some FL learners will also be bilingual in that they may have more than one language in the home, and then be taught a foreign language in the classroom. To our knowledge, there has been little research on this area and hence at this stage we will not discuss this particular sub-population.}

\section{Wider outcomes of learning languages}\label{sec:murphy:2}

In this section, we examine some of the main areas that our REA identified as having been investigated using either RCTs or QEDs specifically addressing the question of whether being bi/multilingual, or having learned another language in instructed settings, has a wider impact beyond being able to speak two or more languages. 

\subsection{Metalinguistic awareness}

Metalinguistic awareness refers to the ability to think about language independently of its literal meanings and use. It relates to understanding how a language works as a system. When we are metalinguistically aware, our tacit knowledge of language (i.e., being able to use it without explicit awareness of how we are manipulating it to make meaning) becomes more salient. That is, we are able to talk about how we use language and understand language as a process (see \citealt{Roehr-Brackin2018} for a review). Several studies (see reviews by \citealt{FoxEtAl2019,FoxEtAl2020}) over the past few decades have investigated whether learning more than one language contributes to the development of metalinguistic awareness. They tend to find that knowing more than one language can result in understanding the similarities, differences, and ultimately the arbitrariness of language (e.g., it’s \textit{apple} in English but \textit{pomme} in French, but these are essentially arbitrary sounds that could just as easily be \textit{wug} or \textit{zibber}, or anything else). For example, \citet{MurphyPine2003} compared English-German bilinguals against monolingual English-speaking children. They asked children to complete a wug-type task \citep{Berko1958} where they had to generate the past tense forms of nonsense verbs (e.g., \textit{This is Graham. Graham is plinking. Yesterday he \_\_\_\_?}). \citegen{MurphyPine2003} findings suggested that the bilingual children tapped into different, more explicit, representations of their knowledge of verb structure relative to their monolingual peers. The evidence on the associations between bilingualism and metalinguistic awareness, considered in our REA tends to agree that bilingualism is associated with better metalinguistic awareness compared to monolingualism. This was true for phonological, phonotactic, morphological and syntactic awareness. There are also examples of research where children were becoming bilingual through attending bilingual schools. For example, \citet{RederEtAl2013} recruited French children in France who had been learning German since age four. These children were not in a typical instructed FL setting but rather were in a partial immersion programme where half the instructional time was in French, the other half in German. These French-German bilinguals had higher scores than the monolinguals on tests of compounds, morphological awareness, and syntactic awareness. Similarly, \citet{LaurentMartinot2010} found that children learning another language by attending a bilingual education programme showed superior phonological awareness in comparison to their monolingual peers. They also suggest that these differences strengthen over time.

There are too many studies investigating metalinguistic awareness within populations of children who know and use more than one language to review in this chapter.\footnote{See also \textcitetv{chapters/Kasprowicz} for further, more detailed discussion of metalinguistic awareness in language learning.} That said, it is important to note that the preponderance of these kinds of studies examines metalinguistic awareness among children who have developed bilingualism in naturalistic settings. A much smaller proportion assess the effects on metalinguistic awareness of instructed FL learning, and those which do rarely adopt designs that can support confident causal inferences, as discussed above. With those caveats in mind, we see most research converging on the same general conclusion: learning more than one language can increase metalinguistic awareness. In other words, having knowledge of two or more languages is likely to lead to these learners being better language processers and analysers. This is an important set of findings. It matters for educational contexts because we know that metalinguistic skills underpin literacy (e.g. \citealt{Murphy2018}), and literacy underpins academic achievement. 

\subsection{Academic achievement}

There tends to be a strongly held belief among advocates for bilingualism and FL learning that learning or knowing another language has substantive academic benefits (e.g. \citealt{Holster2005}). In reality, we have very little evidence, particularly from studies adopting experimental designs, which speak directly to the wider academic achievement of children who are either bilingual and/or learning a FL in school. Perhaps because of this paucity of evidence, we have anecdotal evidence that many schools might drop the FL class in favour of other more ``academic'' subjects when there are looming pressures from national assessments (e.g. \citealt{ELLiE2011,Murphy2014}). On the one hand, then, we have positive moves from governments about the importance of learning FLs, and, on the other, anecdotal evidence that FL learning is often not afforded the same status in the curriculum as more traditional academic subjects. Given that governments are mandating the inclusion of FL in primary curricula, it is increasingly important to have a better understanding of what this means, if anything, in terms of academic achievement. Some examples of studies which speak to this issue are presented here.

\citet{ZaunbauerMöller2010} examined maths attainment of German grade 1 and 2 children in a bilingual school (where all subjects including maths were taught in English), against monolingual peers in an all-German medium of instruction school. Maths attainment was similar in both groups, but the children in the English medium school made faster progress in maths than the monolingual children. These results were consistent with a related study by \citet{KuskaEtAl2010} also showing that German speaking children in bilingual vs. monolingual programmes tended to show better learning and memory performance, skills which underpin academic attainment.

Another study from Germany, carried out by \citet{GunzenhauserEtAl2019}, compared verbal competence of 21 third grade bilingual and monolingual children. The monolingual children in this study performed slightly better than the bilingual children on measures of verbal competence, and they found no differences between the groups on other measures of reasoning skills assessed. This study, then, did not provide any evidence that bilingualism confers advantages on these important academic skills.

In the USA, \citet{TaylorLafayette2010} compared children in the Louisiana State bilingual programme with monolingual children. They found that children in the bilingual schools outperformed monolinguals on measures of maths, science and social studies. They also found differences in favour of bi/multilingual groups on a test of basic reasoning skills. Also in the USA, \citet{CooperEtAl2008} analysed the SAT\footnote{The SAT is a university entrance exam used routinely in the USA.} reasoning scores of more than 9,000 children in Atlanta Georgia and compared the scores of children who had taken a FL course with those who had not. Those who had taken a FL course outperformed those who had not, a difference that was more pronounced for students who had spent more time learning FL. 

It is difficult to establish causal relationships in most of these studies, even though they attempt to carefully compare those with multilingual competence against those without on measures of academic achievement. This is because, in most of these studies, the comparison groups are systematically different at baseline from each other. One is bilingual, the other is not. Because it is practically and ethically impossible to randomly assign children to become bilingual and others to not (and in so doing create unbiased comparison groups), differences observed between these pre-existing groups could be explained by something common to these groups other than knowing and/or using another language. For example, the attitudes and support for education in general among parents who have deliberately sought out relatively scarce bilingual education programmes for their children may differ in important ways from their peers who do not seek out bilingual education opportunities. Nonetheless, these findings are useful because they hint to a potential that needs to be explored more systematically with more carefully controlled designs, and most especially in the context of instructed FL learning. 

\subsection{Language and literacy skills}

Research has compared bilinguals and monolinguals on measures of literacy, most of which has suggested that bilinguals have advantages in this area (see \citealt{MurphyEtAl2020} for more detailed discussion). For example, \citet{SilvenRubinov2010} have suggested that bilinguals respond better to early literacy teaching than monolinguals. \citet{Modirkhamene2006} indicates that bilinguals have higher scores on English reading comprehension than monolinguals. \citet{KnellEtAl2007} demonstrated that bilinguals have superior oral language skills. And \citet{BialystokFeng2009} argued that bilinguals have better vocabulary recall than monolinguals. These studies are interesting, important, and certainly indicative that bilinguals can show superior performance relative to monolinguals in these areas. They focus on examining inherent traits (being bilingual or not) in relation to their different outcome measures. They do not explicitly compare a particular approach to pedagogy with alternatives. Nonetheless, they are suggestive that children who know more than one language might be advantaged in relation to aspects of literacy, with obvious implications for academic achievement. 

There are several studies which have compared children in bilingual education programmes with those in monolingual programmes, in relation to language and literacy scores. Such studies have demonstrated higher levels of linguistic performance \citep{Lazaruk2007}; superior phonological skills (\citealt{LaurentMartinot2010}) and more rapid growth in vocabulary knowledge (\citealt{LoMurphy2010}). Other work has suggested superior performance on reading comprehension for students in bilingual programmes (e.g. \citealt{DeSousa2012}). As above, these studies are important as they allow us to gauge the nature and extent of any effects of participation in bilingual programmes. But again, the children in these studies were not randomly assigned to participate in bilingual education or not, and therefore systematic differences between the participants at baseline may explain the differences in outcomes between the groups. Indeed, children or families typically self-select to participate in bilingual education programmes \citep{Murphy2014}, so many of these studies are de facto comparisons between families who choose bilingual education and those who do not -- a subtle but important distinction. Causal relationships then cannot convincingly be inferred from these studies, as indicative as they may be.

One study that does support more confident causal inferences examined the question of whether learning a FL confers advantages in the domain of literacy. \citet{MurphyEtAl2015} report on a small-scale RCT of a group of seven-year-old English-speaking children (in England) who were randomly assigned to one of three groups: Italian, French or Control. The Italian and French groups received 15 hours of instruction in either Italian or French, respectively. The control group received no FL instruction, and at that stage in their academic progression had never been taught a FL. All children were pretested on measures of English (the L1) reading and spelling, including measures of phonological awareness. After the 15 weeks of FL instruction, all children were tested again on the same measures. The 15 weeks’ worth of FL instruction had a positive impact on some aspects of children’s developing L1 literacy skills, particularly in phonological processing. Furthermore, there were some advantages for the children in the Italian group over the French which might indicate that learning a FL with a transparent mapping between sounds to graphemes (as in Italian) can be particularly helpful in shaping developing L1 literacy skills where the mapping is considerably more opaque (as in English). This finding, while derived from a small-scale study and only 15 hours’ worth of FL learning, is indicative that learning a FL in input-limited contexts can have positive impacts on developing L1 literacy, possibly through increasing metalinguistic awareness, which in turn could support academic achievement. 

Another type of design which is lacking in this area is longitudinal. Longitudinal designs tend to be resource intensive and, consequently, tend to be less prevalent in applied linguistics research. However, there are some important, if rare, examples of longitudinal research that also adopts RCT or QED methodology. One such example is \citet{JaekelEtAl2017}. They investigated linguistic development in German children learning English at school from age 6 and compared these to those who had been learning English since age 8. Children were tested on various tasks including picture recognition, sentence completion and reading comprehension when they were aged 10 to 11, and then again when they were aged 12 to 13. At the first assessment point in this longitudinal study, there was an advantage for the children who had begun learning English at age 6. However, when re-tested the following year, the later-start group (who had begun learning English at age 8) began to overtake the early-start group. This is an important finding as it demonstrates the truism that learning takes time. While children are sometimes regarded as natural language learners who ``pick up language like a sponge'', we know from decades of research that this is an inaccurate characterisation of how language development proceeds (in both L1 and L2/L3/Ln) in children. 

\subsection{Creativity}

Creativity, defined variously in this body of literature as including divergent thinking, structured imagination, innovative thinking, and non-verbal creativity and flexibility, is considered a desirable educational outcome \citep{Harris2016}. Proponents of the bilingual advantage idea have argued that knowing more than one language leads to greater creativity and, thus, FL learning/bilingualism may help to support this aspect of educational development. There are studies that support this claim. In the review by \citet{FoxEtAl2019}, seven studies adhering to an experimental design were found that addressed this issue. Two studies \citep{Kharkhurin2009,Kharkhurin2010} suggested that bilinguals have higher scores on tests of divergent thinking and structured imagination than monolinguals \citep{Kharkhurin2009}; and that bilinguals have an advantage on non-verbal creativity \citep{Kharkhurin2010}. \citet{KostandyanLedovaya2013} also reported advantages for simultaneous bilinguals relative to monolinguals on measures of nonverbal flexibility. However, other work studying bilinguals has not found such an advantage on measures of creativity. For example, \citet{LeeKim2010} investigated Korean-English bilingual students (between 7--18 years old) and found no difference between groups of balanced, unbalanced and/or monolingual students on tests of innovative and adaptive creativity. A recent study by \citet{BootonEtAlInpress} recruited 111 bilingual children in the UK who completed three separate measures of divergent thinking alongside measures of nonverbal IQ, vocabulary and exposure to English. This study, like \citegen{LeeKim2010}, found no advantage for bilingual children over monolinguals. This work on creativity in bilingualism is aligned with the work mentioned earlier on cognitive advantages (\sectref{sec:murphy:1.1}), as it has been argued by some that it is the alleged advantages bilinguals have with executive functions (e.g. \citealt{SampedroPeña2019}) that leads to improvements in creativity. Other accounts suggest that bilinguals are more creative because they have more diverse life experiences (e.g. \citealt{RitterEtAl2012}). As with the research on putative cognitive advantages in bilinguals, this area is fraught with inconsistent findings.

Some research investigating creativity in FL learners has also suggested associated advantages. \citet{FurstGrin2018}, in a study looking at adult FL learners, demonstrated a positive correlation between FL learning and divergent thinking. Research by \citet{GhonsoolyShowqi2012} assessed the creativity of Iranian secondary school girls who had been learning EFL for at least six consecutive years and compared it with similar students who had never formally studied English. The EFL learners demonstrated better scores on tests of creative thinking. 

In both the bilingual and FL contexts, we can see then that some research has employed experimental designs to investigate whether knowing more than one language leads to superior performance on measures of creativity. However, as with much of the research on cognitive advantages in bilinguals, the findings are relatively mixed and/or thin on the ground. To address this question properly, we need many more experimental studies with appropriately matched control groups on agreed-upon measures of creativity. Until this work has been carried out, the jury is still out on whether creativity is fostered through knowing more than one language.

\subsection{Communicative and intercultural competence}

Communicative competence refers to a language user’s knowledge of the mechanical aspects of language, such as syntax, morphology, and phonology, as well as the user’s social knowledge of how and when to use particular aspects of a language (i.e. pragmatics). When we learn another language, we also develop knowledge of the culture(s) that use that language. For example, communicative and intercultural competence in an East-Asian language like Thai will include knowledge of the hierarchical honorifics and personal pronouns considered important for the Thai social script, and how to use them appropriately. One of the stated reasons for supporting FL instruction in England is to ``\ldots\,provide[s] an opening to other cultures'', and to ``\ldots\,deepen their understanding of the world'' (\citealt{DfE2013curriculum}: no page). Being able to communicate with others using another language and having increased intercultural awareness and understanding are laudable outcomes one might predict would naturally stem from knowing more than one language. Having communicative competence in another language is expected to bring an individual closer to the culture associated with the language that is being learned. Some studies have, not surprisingly, looked at communicative and/or intercultural competence trying to understand the nature of this relationship. 

In terms of communicative competence, \citet{SiegalEtAl2010} studied the effect of L2 learning on conversational understanding in German-Italian and English-Japanese bilinguals. German-Italian children (living in Italy with German as an L1) were statistically significantly better able to identify violations of conversational maxims (cf. \citealt{Grice1975}) than Italian monolinguals when assessed using the Conversational Violations Test in Italian. Results were similar for English-Japanese bilinguals (living in England), who demonstrated greater sensitivity to conversational maxim violations than Japanese monolinguals in Japan. \citet{SiegalEtAl2010} argue that these results support the claim that L2 learning contributes to better conversational competence.

In an experimental study on young FL learners in Portugal, findings suggested that participating in a three-month programme called \textit{Awakening to Languages}, compared to normal FL instruction, led to superior oral comprehension and attitudes towards language and cultural diversity \citep{CoelhoEtAl2018}. In another study looking at oral competence in FL, \citet{DominguezPessoa2005} suggest that early learning of a FL supports oral skills and confidence in using the language. Both of these studies, therefore, suggest FL learning supports communicative and intercultural competence.

\citet{HeiningBoyntonHaitema2007} examined students’ attitudes towards FL learning over ten years from elementary to high school, in North Carolina. They found that students had long-lasting positive attitudes towards language learning, FL speakers, and foreign cultures. They attribute this to engagement in instructed FL learning from the early grades. Note, though, that they did not compare the attitudes of this group of learners with others who had not studied FL from the early grades, so causal relationships are tentative at best. Nonetheless, these findings are consistent with a study by \citet{Merisuostorm2007} in Finland where more positive attitudes towards FL learning were reported in children in a partial immersion programme relative to monolingual children. Similarly, \citet{Hood2006} reported that early FL learning led to raised positive attitudes towards other cultures. 

In an example of a relatively rare longitudinal study in the UK, which included a focus on intercultural awareness, \citet{DriscollEtAl2013} suggest that FL programmes need to be more focussed on delivering ``intercultural awareness'' as an outcome, particularly given the UK government’s view that this is an important by-product of FL learning (see above). In their three-year study, they investigated teaching approaches, staff attitudes towards intercultural understanding, and students’ attitudes in UK primary schools. Over 50\% of the teachers in their study reported that developing intercultural sensitivity was a core aim of learning and teaching at primary level (as per curricular guidance), however, there was a general lack of medium to long-term planning to develop it, and there was an observed ``mismatch'' between the stated importance of this outcome, and the actual practice devoted to developing it in classrooms. Pupils also mentioned they did not have sufficient time in class to develop an understanding of cultural issues that were of interest to them. 

Despite these studies suggesting that learning a FL and/or being bilingual can improve intercultural awareness, there is surprisingly little relevant research in this area, despite being a stated aim in many governmental FL policies, such as in the UK (see \citealt{DfE2013curriculum}). Of those studies that exist, even fewer have employed experimental designs in which to explore this important question. 

\section{Methodological issues \& research agenda}

In \sectref{sec:murphy:2} we briefly reviewed some of the research that has emerged examining the impacts of bilingualism and/or FL learning on general academic achievement, metalinguistic awareness, language and literacy, creativity and communicative and intercultural competence. There are other effects that have been studied that are beyond the scope of this chapter. One is employability, where it has been argued that we need (particularly in English-speaking contexts like the UK) to support FL learning and bilingualism, so we will have an internationally competitive workforce (\citealt{MitchellMyles2019}). Our REA reviewed a handful of studies that touch upon this, which generally converge on the conclusion that employers and employees alike view the ability to use more than one language as advantageous \citep{MurphyEtAl2020}. However, these studies do not tell us anything about whether this translates into actual competitiveness in the job market for bilinguals.

Compared to the vast body of work that has investigated the so-called cognitive benefits of bilingualism, there hae only been a handful of studies that have examined wider impacts of FL learning. This is lamentable because more children around the world are exposed to another language in instructed FL settings than any other context \citep{Murphy2014}. Critically, we need more work in this area, and particularly more work that adopts designs that are better equipped to help us untangle causal relationships between what we do in the classroom and the effects on outcomes that we value. Before we articulate the research agenda we would like to see developed, we will explain why we are concerned with the overall quality of research thus far.

The studies we were looking at in the REA were focused on a specific and narrow type of research design -- namely RCTs or QEDs. Many researchers in the behavioural social sciences agree that such designs represent the best way of identifying ``what works'' and if we want to establish a shared understanding of best practice in the language classroom, we need more research which follows this experimental approach. A major finding of our REA was that there were really very few RCTs or QEDs which examined this question. Within the context of FL learning, there were only a handful. Indeed, most of the work we found in the REA related to bilingual vs monolingual comparisons, where the basic trait of being bilingual (or not) was not something that was randomly assigned but rather, something the participants brought with them. This is not to suggest that work of this type is uninformative. But it only takes us so far. If we want to really understand whether learning a language has reliable and predictable impacts on other domains of learning (apart from linguistic ones), we need many more RCTs, where participants are allocated to alternative teaching interventions on the basis of chance and outcomes that are meaningful to them and their teachers are compared. Reducing biases in this way allows us more confidence in the causal inferences we can subsequently make. In an odd way, such a design is becoming more difficult because finding a true monolingual comparison of children who have not been taught a FL is becoming increasingly difficult. Nonetheless, there is still scope for these designs within the area of FL learning, particularly for young learners. We believe that without such experimental designs we will at best be able to say that X is associated with Y (learning a language is \textit{associated} with higher metalinguistic awareness, for example) instead of X leads to Y (learning a language increases metalinguistic awareness). If we are to effectively and positively influence and support FL education policy, reliable evidence on the effects of that policy is crucial. In educational contexts we’d like to be able to confidently recommend specific ways of teaching to teachers, specific ways that have been demonstrated to work through rigorous, unbiased, comparisons. Until we have more RCTs in this area, we will not be able to do so. This lack of RCT research is not unique to this area within applied linguistics. A systematic review in the field of EAL, carried out by \citet{MurphyUnthiah2015} and subsequently replicated and extended by \citet{OxleyDeCat2019}, conducted an exhaustive search for eligible literature from around the world. It revealed that vanishingly little relevant research had adopted this design. 

Another challenge with much of the work that we found in our REA, and reported here, particularly in relation to the debate around the so called bilingual advantages, is that many of the outcome measures are proxies for things we are actually interested in. This research may well show that bilinguals outperform monolinguals on psycholinguistic tasks like Stroop, Flanker and Simon, but knowing this only has utility to teachers and educational policy makers if it is a harbinger of (or at least related to) an improvement in a real-world educational task (such as solving word problems in maths). Some of the work on creativity cited earlier uses divergent thinking measures, but these are not necessarily immediately transferrable into the classroom. Indeed, even agreeing upon what would constitute a reliable measure of the arguably rather subjective construct that is creativity is a challenge. Research in these areas rarely takes the extra step to investigate whether these proxy measures relate to specific outcomes that teachers are interested in. While not all work in this area claims to inform pedagogy, there are nonetheless good arguments as to why proxy measures for educational success may be misleading. We would like to see researchers take that extra step more frequently to assess whether proxy outcomes translate into outcomes that are more informative to teachers, schools, and policy makers.

Finally, a major difficulty that we face in this area, regardless of who the participants are (i.e., bilinguals or FL learners), relates to publication bias. This kind of bias refers to the difficulty in finding published studies that contradict the orthodoxy of the day in a particular field, not because there are none, but either because research that contradicts the orthodoxy are more frequently rejected by journal editors or just not submitted for publication in the first place. We know that publication bias is a real problem in the field of bilingualism research. For example, \citet{DeBruinEtAl2015} and \citet{LehtonenEtAl2018} found through their analyses that research on bilingualism and cognitive function was much more likely to be published if it confirmed an advantage. They also found evidence for selective outcome reporting. Researchers tend to use a battery of tests to tap into the constructs under investigation, but analyses suggest that only those tasks where advantages were found end up being reported. It should be obvious why this is a misleading practice. Picking and choosing to report only those cases when statistically significant findings emerge is tantamount to flipping a coin ten times, it coming down tails on the first eight flips and heads on the final two, then presenting only the two final flips to argue the coin is double-headed. Many of these issues in research are discussed in \citet{Bishop2020} and solutions she advocates within experimental psychology are equally relevant for the field of applied linguistics. It is now a well-worn cliché, but we really do need more research in this area. We need to be much more willing to publish null results, and in an ideal world, pre-register our research designs and/or manuscripts, through organisations like the Centre for Open Science (\url{www.cos.io}) to enable greater clarity and transparency throughout all stages of research and feed into productive replication.

It will not come as a surprise, then, that as a research agenda we would like to see more, and more diverse, forms of research that tap into the wider impacts of FL learning on other outcomes. From an educational standpoint those other outcomes should realistically be focused on factors that we know contribute to academic attainment (language, literacy, engagement, motivation, self-regulation, and the like). 

We would also like to see more experimental work in the form of RCTs. The areas discussed in this chapter are so pertinent for educational contexts, and have such direct (potential) impacts on practice, that we really need to take these stakeholders into account when thinking of what areas we need to research and what designs to employ. As researchers, we often have good ideas about what works in the FL classroom, but these ideas must be seen as feasible by teachers and must address outcomes that teachers and learners regard as important if they are ever to be implemented in the classroom. Collaborating with teachers to find out what questions around FL pedagogy they would like answered and to work with them to design and implement experiments to evaluate these questions can be an effective way to meet this aspiration. One approach to this is through Priority Setting Partnerships (PSPs). PSPs employ a well-established method where end users of research (in this case teachers) work together to set and publicise research priorities with a view to maximising the chance that new research addresses demonstrated needs and interests (\citealt{StaleyCrowe2019,ChalmersEtAl2021}). The outcome of a PSP is a set of prioritised research questions. Some of which may have already been addressed by the literature, but some may yet require systematic investigation. Following from the PSP, the next step is to prepare a systematic review on the topics that have not yet been investigated. A systematic review enables researchers to set \textit{a priori} inclusion and exclusion criteria in addressing specific research questions relating to extant published literature. It enables a researcher to gather a more comprehensive picture of a current state of research within a particular domain, and furthermore, mitigates some of the bias mentioned earlier that can creep into research. Where possible, meta analyses can be carried out to assess the magnitude and direction of any effects of a pedagogical intervention on the basis of all of the relevant research. This allows us to better understand the likely impact of these interventions. The third step in our ideal research agenda would be to work with teachers to design an RCT, informed by the findings of the PSP and systematic review, to evaluate the effects of a teaching intervention on outcomes considered important and meaningful. As indicated earlier, this cycle of PSP, systematic review ormeta analysis, intervention study is not the only way to go about research. But we feel it deserves a more prominent place in the current research ecosystem. This, not to displace other forms of research, but to complement them. We believe that the type of research we have described here is missing, to the detriment of a holistic understanding of the relationships between learning other languages and outcomes beyond learning the language itself.

Finally, in terms of language and education, we would like to see more joined up thinking in respect of how the FL classroom is likely to consist of more than monolinguals. As referenced earlier, many estimates suggest more people (and this includes children) are bilingual than not. Many such bilinguals (or at least, emergent bilinguals) are in classrooms and are being taught foreign languages. Yet surprisingly, there is barely any research that has examined what FL means for an already bilingual student, what advantages or challenges that student may face, how a teacher should approach FL learning in already bilingual children, and the like. Furthermore, in governmental policy and curricular guidance relating to FL education (at least in England) the FL documents seem to completely ignore the fact that a significant proportion of children at primary school are learning EAL and hence bring knowledge of at least two languages to the FL classroom. This is just beginning to be explored (e.g. \citealt{CostleyEtAl2020}) and we need to see much more research examining questions around this area. 

\section{Summary and conclusion}

In this chapter we have presented a relatively brief summary of some of the main findings of a larger REA asking questions about what available research says about the potential wider outcomes associated with learning more than one language \citep{MurphyEtAl2020}. We presented some of the studies looking at whether bilinguals and/or FL learners show superior gains in metalinguistic awareness, academic achievement, language and literacy, as well as creativity and communicative and intercultural competence. A background for this discussion was the wider work on cognitive advantages in bilingualism. In general, the research we discussed, which largely adhered to either randomised control trials or quasi-experimental designs, has shown positive impacts of learning more than one language on these outcome measures. However, the work is not without its limitations, and we discussed in this chapter some of the methodological challenges inherent in asking these questions, and in the specific extant research. We also talked about research that we would like to see in terms of specific research designs and approaches to tackling research to make it more useful to stakeholders within education. These questions concerning wider outcomes of language learning have the potential to have a significant impact on educational attainment, so more, and more rigorous, work needs to be done in this area. Despite the methodological problems we discussed in this chapter, we are optimistic that once this research agenda has been taken up and further developed, we are likely to gather more credible evidence about the importance of language learning, both within and outside educational settings. 

We began this chapter with a simple yes/no question: Are there advantages to being bilingual? Our answer is: Yes. Being able to understand and/or use more than one language is clearly an advantage in a world where more people can do this than cannot. We believe that current evidence, such as that presented in this chapter, equally suggests that being bi/multilingual is likely to have wider (largely beneficial) impacts on other aspects of learning and/or performance. However, what those advantages are precisely, and whether and when they manifest themselves in different types of learners and contexts of learning still requires greater exploration and understanding. Ideally, this  work would employ research designs and methods better suited to establishing causal relationships. In this way, we would be able to clearly articulate to educational policy makers and teachers more precisely the ways in which learning other languages can benefit young, developing children. We are optimistic that in pursuing research further we will understand how integral language learning should be in the school curriculum. We believe it would lead to a greater understanding of the importance of promoting language development in children who come to school with knowledge of more than one language. We also believe that it would enable us to consider more critically, and from an evidential base, why we should be promoting foreign language learning for young children in primary school settings. This should lead to advances in teacher education and support for this important venture.

\sloppy\printbibliography[heading=subbibliography,notkeyword=this]
\end{document}
