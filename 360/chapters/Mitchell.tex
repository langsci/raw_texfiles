\documentclass[output=paper]{langscibook}
\ChapterDOI{10.5281/zenodo.6811460}
\author{Rosamond Mitchell\orcid{}\affiliation{University of Southampton} and Sarah Rule\orcid{}\affiliation{University of Southampton}}
\title[Learning vocabulary in the primary languages classroom]
      {Learning vocabulary in the primary languages classroom: What corpus analysis can tell us}
\abstract{The national curriculum for primary schools in England now includes foreign language learning for children aged 7–11. However, organised instructional time is very limited, and must be used as effectively as possible, taking account of children’s characteristics as learners to develop and consolidate their target language knowledge. 
  
This chapter draws on evidence from the ESRC-funded study ``Learning French from Ages 5, 7 and 11'', (\citealt{MylesEtAl2012,Myles2017}) regarding the development of vocabulary knowledge by early learners over a year’s instruction in French. Data include video recordings and transcriptions of all lessons, as well as receptive vocabulary tests sampling systematically the vocabulary actually taught. We examine the influence on vocabulary development of factors including word frequency and word functions in classroom talk, the status of lexical items as cognates/non-cognates, the provision of multimodal support for new vocabulary, activity types in which new words were encountered and practised, and the relationship between spoken and written input. We further examine briefly how far variation in vocabulary learning is due to individual learner characteristics. We draw tentative conclusions regarding the rate of progress in vocabulary learning which can be expected in a constrained classroom context, and highlight the factors which seem to promote development most consistently.}

\begin{document}
\SetupAffiliations{mark style=none}
\maketitle 

\section{Introduction}

The significance of input and interaction is generally recognised by theorists of language acquisition. In first language (L1) acquisition research, there is a long tradition of building and analysing corpora of caretaker and child talk, which has been central to our developing understanding of child language development and the contribution of interaction and of environmental language to this process. However, in second language acquisition (SLA), the corpus-based study of input is still rare, and where it is undertaken, reference corpora rather than those capturing the actual experience of individual learners are commonly used \citep{Mitchell2021}. 

Over many years, Florence Myles has championed the use of corpora in SLA research, and has promoted the creation and analysis of a range of learner corpora in French and Spanish. In the project ``Learning French at ages 5, 7 and 11'', she took corpus research in a new direction, leading the creation of a longitudinal audiovisual corpus which captured the total second language (L2) learning experience of young children over the equivalent of a year’s worth of lessons in an authentic classroom setting (\citealt{MylesEtAl2012,Myles2017}). This chapter draws on that corpus to explore the vocabulary to which learners were exposed, how far they succeeded in learning it, and the factors which influenced that success. It was a privilege to work with Florence in the creation of the corpus, and it is a continuing privilege to work with data with such rich potential to contribute to policy formation in language education as much as to our understanding of instructed SLA.

\section{Literature review}
\subsection{The early instructed learner: Key issues}

Much of what we know about child L2 acquisition comes from studies in naturalistic contexts or immersion settings \citep{Murphy2014}. However, in recent years, there has been more research on young instructed L2 learners, including vocabulary acquisition (see edited volumes by \citealt{Nikolov2009,Pinter2011,Garcia2017}). There is consensus that vocabulary learning is crucial for other areas of language development; for example, in speaking fluency (\citealt{DeJongEtAl2013}), it is a robust indicator of L2 proficiency \citep{Cummins2000}, and an essential component of competence \citep{Alexiou2009}. For L1 acquisition, the child has extensive exposure to the target language and L1 acquisition of vocabulary is characteristically fast (\citealt{QianLin2020}). In one view, it is thought to take place through a fast-mapping process, isolating words from the input, creating potential meanings and mapping meanings onto forms (\citealt{RohdeTiefenthal2000}). For instructed SLA, the task is more onerous, with limited input in the formal classroom setting. Fast mapping is more difficult in L2 acquisition, and is often mediated through the L1. In addition, there is also the fact that the early L2 learner has to deal with unfamiliar sounds. Overall, the young L2 learner needs to acquire semantic, orthographic and phonological representations (\citealt{ZhaoMurphy2017}) and develop the ability to retrieve these stored representations. It has been observed in studies of young L2 learners that they tend to learn nouns and adjectives more easily \citep{CableEtAl2010} and recognition is more successful than production. In contrast to naturalistic child L2 acquisition, L2 learning in classrooms is considered to be explicit and conscious, and to involve memory systems.

The learners in this study are young, instructed learners of French who only received target language input in the classroom. These learners show a wide diversity in outcomes and make relatively slow progress, but they can and do successfully acquire vocabulary (and grammar: \citealt{MitchellMyles2019}). In the literature on L2 vocabulary acquisition, certain factors are identified that can predict the success of these young learners in acquiring vocabulary items. These include word-related factors, for example, modality of input, frequency of encounters in the input, word class of the item and whether the word is a cognate. Factors related to individual differences in the learners also have an influence, for example cognitive differences in working memory (WM) and attention, differences of motivation and engagement, and differences in L1 literacy levels. Finally, how much the word is practised by the learner, in what different contexts it is practised, and how (and how often) it is retrieved, have an impact on learning and retention of vocabulary items.

\subsection{Word-related factors: Input, cognates, multimodality}

Frequency, saliency and similarity with L1 are all important word-related factors influencing vocabulary acquisition. Young instructed L2 learners learn nouns and adjectives with a concrete referent more easily than verbs \citep{CableEtAl2010}. Frequency in the input also correlates positively with success in vocabulary learning (\citealt{Szpotowicz2009,VanZeelandSchmitt2013,PetersWebb2018,DeWildeEtAl2021}). Additionally, learners’ attention tends to focus on features in the input ``that are consistent with their L1 systems'' (\citealt{Ellis2006}, cited in \citealt[214]{Kormos2020}). This importance of the relationship with the learners’ L1 entails that cognates seem to be learned more successfully than non-cognates (\citealt{Szpotowicz2009,PetersWebb2018,DeWildeEtAl2021}). However, for learners of French, cognate status may be unclear, for example \textit{la table} and \textit{le silence} are cognates orthographically but only partly phonologically, and for young L2 learners of French reliant on oral input their cognate status may not be noticed.

Differences in input modality may also play a role in what is learned and remembered. It has been found that multimodal input can enhance the comprehension and recall of information, indicating that a combination of images and verbal information may improve L2 learning and memorisation \citep{Syodorenko2010}. However, multimodality of input seems to have varying success. While audio-visual input, images and gestures have all been shown to have a positive impact (\citealt{Allen1995,Tellier2008,KellyEtAl2009,Porter2016,MuñozEtAl2021}), some studies report that words accompanied by iconic gestures have no impact on memorisation \citep{MorettEtAl2012}. Another type of input prevalent in child L2 classrooms, the occurrence of vocabulary in songs, does not necessarily lead to better learning (\citealt{CoyleGómezGracia2014,RuleMitchell2014}).

From a psycholinguistic perspective, multimodality could enhance encoding and lead to deeper memory traces which are longer lasting and more easily retrieved. We adopt a current information processing view of memory and assume that WM can deal simultaneously with both sound and visual input \citep{BaddeleyEtAl2015}. There are a number of different arguments supporting the positive effect that using images and gestures alongside text or spoken input can have on acquisition. For \citet{ClarkPaivio1991}, learning is reinforced when both verbal and non-verbal modalities co-occur, and different modalities can enable deeper processing. In a similar perspective, enrichment of input leaves richer traces that in turn help memorisation and retrieval \citep{Baddeley1997}. Based on the assumption that WM includes independent auditory and visual working memories, multimedia learning is claimed to be efficient because it conveys both auditory and visual information (\citealt{MorenoMayer2000}). Yet for some children with limited attention and reduced WM capacity, this multimodal input can be distracting because it is difficult for them to focus on the critical information \citep{MatuszEtAl2014}.

Concerning gesture, it is thought that enactment during encoding improves memory performance (\citealt[100]{Kormi-NouriNilsson2001}), as motor encoding is thought to be more durable, more accessible, and highly resistant to forgetting (\citealt[785]{KnopfNeidhardt1989}). In a study by \citet{MorettEtAl2012}, it was found that, although gestures alone did not aid memorisation, if the participants enacted the gestures themselves, then gains were made. In a multimodal input study with young French children, \citet{Tellier2008} found that although pictures and teacher gestures gave the children equal gains in L1 word acquisition, gestures were more advantageous when enacted by the children themselves. 

Orthography can also affect the word learning process; seeing the word written down seems to ensure deeper levels of processing (\citealt{CraikLockhart1972}). Thus it has been claimed that the presence of the written form during word learning leads to better learning of a word’s meaning \citep{RickettsEtAl2009}, though others have concluded it instead supports the learning of pronunciation \citep{KrepelEtAl2021}. In a similar way to enactment, another factor that may improve acquisition and memorisation is the physical act of writing the word down, asking the question whether this motor action adds another layer of processing and a stronger memory trace. 

\subsection{Focused attention and practice}\largerpage

Exposure is critical for vocabulary acquisition and, with severely limited class time, practice plays a crucial role in skill acquisition \citep{DeKeyser2012}. If there are no opportunities for repeated practice, then words may go unnoticed, or the meaning may not be inferred \citep{Laufer2020}. Multiple retrieval opportunities are needed, and successful retrieval will strengthen the representation of the vocabulary item \citep{Nakata2020}. There are many researchers who stress the importance of both repetition and variation (\citealt{Lightbown2008,Kersten2011}). This ``repeated exposure to novel lexical items can help develop rich lexical representations so word recognition and recall can proceed quickly and without effort'' (\citealt{Perfetti2007} cited in \citealt[215]{Kormos2020}). This in turn leads to better memorisation and longer lasting encoding and retention in long term memory.

The importance of practice and repetition can be accounted for in \citeauthor{Barcroft2002}’s model of word learning Type of Processing Resource Allocation (TOPRA) (\citeyear{Barcroft2002, Barcroft2013} cited in \citealt[262]{Newton2020}). In this model, one of the key implications for vocabulary learning is that exposure to repeated occurrences of new words is beneficial for word learning. One of the other proposals of this model is that the way we process information determines the aspects we will remember, and a crucial factor is limited attentional capacity. This seems to be particularly significant for children as they have a reduced capacity for selective attention \citep{Fougnie2008} and processing L2 input is cognitively demanding.

In classroom settings with limited input, repeated practice is essential for vocabulary learning to take place. Adopting a skill acquisition theoretical framework, recently, researchers have investigated the optimum spacing between practice sessions and the importance of distributed practice (\citealt{LiDeKeyser2019,SuzukiEtAl2019,RogersCheung2020}). This is particularly important research for the interface between learning theories and pedagogy. Contrasting results were given by \citet{KasprowiczEtAl2019} who studied young L1 learners of L2 French grammar, drawing on research in cognitive psychology and the premise that temporally spaced sessions lead to better learning and retention. The spacing of practice differed between 3.5 and 7 days. The results of their study indicated that there was “limited impact” of the different distributions of practice and individual cognitive factors were more significant.

\subsection{Individual differences}

With L2 learning in classrooms being at least partly explicit and conscious there is the implication that individual differences between learners will be significant. One area that predicts L2 success is L1 skills. L1 literacy is seen as a foundation for successful instructed L2 learning \citep{SparksEtAl2006}. \citet{DufvaVoeten2001}, who studied 7-year-old Finnish learners of English, used measures of native language literacy (word recognition and comprehension skill) along with phonological memory measures, and this significantly predicted L2 outcomes. Level of attainment in L2 is seemingly moderated in instructed settings by level of attainment in L1 \citep{Sparks2012}.

There are also important cognitive differences that affect language comprehension and vocabulary learning and one of these is WM capacity. Children have reduced capacity when compared to adults, and not all children have the same WM capacity. There are also cognitive changes in children as they develop, and this includes WM. These affect the extent to which children are able to store, access, retain and recall target knowledge (\citealt[4]{KasprowiczEtAl2019}).

As described above, we take Baddeley’s view that WM is a multicomponent system consisting of a central executive and two slave systems, the phonological loop (sound based storage system) and the visuo-spatial sketchpad (related to visual imagery). During their processing, lexical items need to be encoded, organised and consolidated and this all happens within WM. This then interacts with Long Term Memory, where lexical items and their meanings are stored. Within WM, the Verbal Short Term Memory (VSTM) is related to vocabulary learning (\citealt{GathercoleEtAl1992,VerhagenLeseman2016}). \citet{EngeldeAbreuGathercole2012}, using nonword repetition and digit recall tests, found that VSTM is related to both L1 and L2 vocabulary. A further study found that phonological awareness predicts both spelling and L2 proficiency \citep{Sparks2012}. Linked to WM is the concept of attention, and attention paid to input is vital for L2 development (\citealt{IndrarathneKormos2018}). Attention is a controversial topic in psychology and applied linguistics but where there is consensus, is that it is subject to intentional control and is selective \citep[212]{Kormos2020}.

Sometimes differences in L2 learning cannot be attributed to individual cognitive differences or even differences in L1 skills. As reported in a previous study of the same group of students, some participants had similar WM scores and L1 literacy scores but exhibited different vocabulary learning outcomes (\citealt{MitchellRule2016}). Thus it is necessary to examine the phenomenon of engagement. There is a history of studies that have linked academic engagement to positive outcomes in student learning \citep{ChristensonEtAl2012}. It is a multidimensional phenomenon involving aspects of students’ behaviour, emotion and cognition \citep{FredricksEtAl2004}. Recent studies have highlighted two particular aspects of engagement as being crucial for learning: mental effort and concentration. It is also believed that thoughtful engagement can lead to deeper learning (\citealt{BrysonHand2007}). While motivation is a necessary condition for engagement, motivation is also linked to attention, but again, one does not necessarily entail the other \citep{Baddeley1997}. Observation is a recognised method of studying engagement in classrooms (\citealt{FredricksMcColskey2012}), and the video corpus discussed in this chapter made it possible to track the engagement of individual case study children, as reported elsewhere (\citealt{MitchellRule2016,MitchellMyles2019}).

\section{Research questions}

The research questions addressed in this study are:

\begin{enumerate}
\item What is the impact of individual learner characteristics on early classroom L2 vocabulary learning? (WM, L1 literacy)
\item What is the impact of classroom input and lexical characteristics on early classroom L2 vocabulary learning? (item frequency, item distribution, cognate/non-cognate status)
\item What pedagogical practices are most supportive of enhancing L2 vocabulary learning?
\end{enumerate}

\section{Methodology}
\subsection{The \emph{Learning French} project}

The data for this study are drawn from the longitudinal research project ``Learning French at ages 5, 7 and 11'' (\citealt{MylesEtAl2012,Myles2017}). This project has provided exceptional in-depth insights into the processes and learning outcomes for foreign languages in the UK primary (elementary) school context, and contributed research insights to policy discussions on the place of languages in the curriculum, and the conditions under which such a curriculum initiative could be successful (\citealt{HolmesMyles2019}). In England, the teaching of a foreign language of the school’s choice was promoted within primary schools on a voluntary basis from the early 2000s, and became a compulsory part of the curriculum in 2014. In this project, where Myles was Principal Investigator, three intact classes of Year 1, Year 3 and Year 7 children were tracked through their first 38 hours of lessons in French. All lessons were taught by the same specialist teacher, following the same oracy-led approach and the same lesson content, with minor age-appropriate adaptations. Most lessons (\textit{n} = 33) were video-recorded and subsequently transcribed using the CHAT system \citep{MacWhinney2000}. Later analyses using ELAN (\citealt{LausbergSloetjes2009}) coded the teacher’s gestural behaviour and use of multimodal resources, as well as the cognitive, behavioural and emotional engagement of a subset of children. Children’s individual differences relevant to classroom L2 learning were identified as a) L1 literacy, and b) WM. L2 development was tracked through a series of specially developed instruments: among them were group role plays, a story retelling task, an Elicited Imitation test (EIT: \citealt{Tracy-VenturaEtAl2014}), and receptive vocabulary tests (RVT) which predominantly sampled lexical items found in teacher input. These tests were administered on three occasions: mid-instruction, immediately post-instruction, and delayed post-instruction. Previous publications arising from the Learning French project have reported the key role in vocabulary learning of frequency in teacher input and learners’ individual characteristics (WM, L1 literacy: \citealt{MylesEtAl2012}) and of learner engagement (\citealt{MitchellRule2016,MitchellMyles2019}). Here we briefly summarise findings concerning children’s individual differences, and then expand previous published analyses to include closer examination of the contribution of cognates, of teacher input and children’s L2 output, and of the teacher’s multimodal pedagogy. We focus on the Year 3 dataset, i.e. the sequence of French lessons with 26 children aged 7--8 (15 girls, 11 boys), and related assessments.

\subsection{Instruments and procedures}\largerpage
\subsubsection{Individual differences}

In this study, we draw on the Year 3 findings from the 28-item non-word repetition (NWR) test developed by \citet{GathercoleBaddeley1996}, which was used in the project to measure children’s WM. L1 literacy scores indicating Year 3 children’s progress with reference to the English National Curriculum, on a scale from 1 to 9, were provided by the school.

\subsubsection{Lexical development}

To measure lexical development, we use findings from the 50-item RVT used at Posttest (PT) and Delayed Posttest (DPT). This test was specially created; it drew on the set of lesson recordings to sample vocabulary items from the detailed curriculum developed by the teacher, and used by her in class with varying frequency. Altogether, 44 items were selected from the 670 lexemes used by the teacher in the course of the 33 recorded lessons. A small number of items not found in the classroom input were also included (\textit{n} = 6), as were a number of cognates (phonological and/or orthographic, \textit{n} = 10). Two word classes were sampled: nouns (\textit{n} = 32) and verbs (\textit{n} = 18). The test was computer based, and was administered individually; for each multiple choice item, the participant heard the target word, and had to click on the most appropriate image from a selection of four (images were drawn from the Peabody vocabulary test). As well as individual students’ scores, facility values were calculated for each item on the test.

\subsubsection{Lexical frequency}

To analyse lexical frequencies in teacher L2 input and in learner L2 output, across the 33 lessons recorded in the study, we have used the CLAN programs of TalkBank \citep{MacWhinney2000} to conduct counts of all occurrences of individual lemmas as well as the number of lessons in which these occurred, and to examine contexts of use. 

To explore relationships among lexical frequency and item cognate status, participants’ individual characteristics and test scores, and the item facility of the 50 “target words”, we used various statistical tests available in SPSS 27. To document the pedagogical strategies attached to a subsample of target words (some “well learned”, others “less well learned”), including teaching activities and multimodal practices, we conducted qualitative analysis of lesson videos and accompanying lesson observation notes. Children’s classroom engagement, not discussed in detail here, was also analysed and annotated using ELAN.

\section{Findings}\largerpage[2]
\subsection{A descriptive overview}

The results for the 50-item vocabulary Posttest and the Delayed Posttest are summarised in \figref{fig:mitchell:1}. The Posttest findings represent participants’ learning immediately following the lesson sequence, and the Delayed Posttest was administered three months later. They show moderate levels of achievement on the 50-item test.%%%, with overall scores as in \tabref{tab:mitchell:posttest}.

% % % \begin{table} % Table removed as per author's request 18/07/22
% % % \caption{\color{red}Please provide a caption\label{tab:mitchell:posttest}}
% % % \begin{tabular}{lcc}
% % % \lsptoprule
% % %      & mean score & SD\\\midrule
% % % PT   &  27.96 & 8.33\\
% % % DPT  &  27.38 & 7.79\\
% % % \lspbottomrule
% % % \end{tabular}
% % % \end{table}

A paired samples $t$ test showed no significant difference between the two sets of results, so the learning achieved during instruction was effectively maintained. (As previously reported, there was also no significant difference between scores on these tests and those on the Midtest, which used a slightly different selection of vocabulary items, \citealt{MylesEtAl2012}.)

\begin{figure}
\includegraphics[height=.4\textheight, align=t]{figures/a7MitchellRule-img001.png}
\includegraphics[height=.42\textheight, align=t]{figures/a7MitchellRule-img002.png}
\caption{\label{fig:mitchell:1}Results for 50-item RVT, used as Posttest (PT) and Delayed Posttest (DPT)}
\end{figure}

\tabref{tab:mitchell:1} provides an overview of the 50 lexemes targeted in the RVT, with their facility values on PT and DPT, their status as cognates, and their frequency of production by the teacher and by the children, in all video-recorded lessons ($n= 33$). (Counts of individual and group productions by children have been merged.) All frequency counts ignored morphological variation, e.g. \textit{lève, levez} are both subsumed under \textit{lever} `to raise', and \textit{main, mains} are subsumed under \textit{main} `hand'. The 10 words judged to be cognates recognisable to participants are indicated with \textit{(cog)}. The table also shows the number of lessons in which each lexeme occurred.

The 50 target items are ranked in \tabref{tab:mitchell:1} according to their facility value on the PT. They have been grouped into 3 clusters: ``Well learned'', ``Moderately learned''  and ``Poorly learned'' words. Given that the children’s scores on the multiple-choice receptive vocabulary test likely involved some guessing, the thresholds for these categories have been set quite high. To count as ``Well learned'' ($n = 18$), an item must have a mean facility score of 60+, on both PT and DPT; to count as ``Moderately learned'' ($n = 15$), it must have a mean facility score of 40+ on both test occasions. The remaining words were placed in the ``Poorly learned'' category ($n = 17$).


\begin{longtable}{lrrrrr}
\caption{Target lexical items (ranked by PT facility). $n$ lessons: 33. Cognates are indicated with \textit{(cog)}.\label{tab:mitchell:1}}\smallskip\\
    \lsptoprule
           & \multicolumn{2}{c}{Frequency} & & \multicolumn{2}{c}{Facility}\\\cmidrule(lr){2-3}\cmidrule(lr){5-6}
     Item  &  T input & child output & Lessons & PT & DPT\\\midrule\endfirsthead
     \midrule
           & \multicolumn{2}{c}{Frequency} & & \multicolumn{2}{c}{Facility}\\\cmidrule(lr){2-3}\cmidrule(lr){5-6}
     Item  &  T input & child output & Lessons & PT & DPT\\\midrule\endhead
     \endfoot\lspbottomrule\endlastfoot
     \multicolumn{6}{c}{Well learned words}\\\midrule
    \textit{silence} `silence' (cog) &297	& 70 &	32 &	96.15 &	88.46\\
    \textit{poisson} `fish' &	124 &	35	& 9	& 96.15 &	76.92\\
    \textit{glace} `icecream' &	35 &	28 &	5 &	92.31 &	69.23\\
    \textit{lever} `to raise, to lift' &	365 &	121 &	31 &	88.46 &	88.46\\
    \textit{trois} `three' &	518 &	164	& 33 &	88.46 &	88.46\\
    \textit{écouter} `to listen' &	215 &	30 &	27 &	88.46 &	73.08\\
    \textit{danser} `to dance' (cog) &	42 &	2 &	11	& 84.62 &	96.15\\
    \textit{chien} `dog'	& 124 &	58 &	16 &	84.62 &	84.62\\
    \textit{fleur} `flower' (cog) &	64 &	46 &	5 &	84.62 &	69.23\\
    \textit{bébé} `baby' (cog) &	2 &	0	& 1	& 80.77	& 88.46\\       
    \textit{crayon} `pencil' (cog)  &	8	& 0	& 5	& 80.77	& 65.38\\
    \textit{fraise} `strawberry'	& 62 &	29	& 9	& 80.77	& 65.38\\
    \textit{dix} `ten' &	245	& 152	& 28	& 76.92	& 88.46\\
    \textit{skier} `to ski' (cog)	& 0	& 0	& 0	& 76.92 &	76.92\\
    \textit{frapper} `to knock, clap' &	37 &	18 &	10	& 69.23	& 73.08\\
    \textit{vert} `green' &	80	 & 49	& 15	& 65.38 &	73.08\\
    \textit{chanter} `to sing' &	95 &	13 &	24	& 65.38 &	69.23\\
    \textit{fille} `girl'	& 65	& 0	& 15	& 65.38 &	69.23\\   
    
    \midrule
    \multicolumn{6}{c}{Moderately learned words}\\
    \midrule
    
    \textit{garçon} `boy' &	59 &	0	& 17	& 65.38	& 57.69\\
    \textit{regarder} `to look' &	339 &	31	& 30	& 65.38 &	34.62\\
    \textit{ballon} `ball' (cog)	& 41 &	20 &	8 &	61.54 &	53.85\\
    \textit{grimper} `to climb' &	20 &	14 &	8	& 53.85 & 	61.54\\
    \textit{serpent} `snake' (cog) &	14 &	1 &	3 &	53.85 &	61.54\\
    \textit{table} `table' (cog) &	9 &	0 &	6 &	53.85 &	46.15\\
    \textit{déchirer} `to tear' &	0 &	0 &	0 &	53.85 &	46.15\\
    \textit{manger} `to eat' &	29 &	0 &	10 &	53.85 &	42.31\\
    \textit{escargot} `snail' &	0 &	0 &	0 &	53.85 &	19.23\\
    \textit{bisou} `kiss' &	43 &	17 &	6 &	50\phantom{.00} &	61.54\\
    \textit{mélanger} `to mix'	& 4	& 0	& 3	& 50\phantom{.00} &	50\phantom{.00}\\
    \textit{attraper} `to catch' & 	3 &	0 &	3 &	50\phantom{.00} &	46.15\\
    \textit{drapeau} `flag' &	4 &	0 &	1 &	46.15 &	42.31\\
    \textit{épaule} `shoulder' &	89 &	76 &	4 &	46.15 &	42.31\\
    \textit{yeux} `eyes'	& 142 &	75 &	11 &	42.31 &	46.15\\
    
    
    \midrule
    \multicolumn{6}{c}{Poorly learned words}\\
    \midrule

    \textit{papier} `paper' (cog) &	29 &	0	& 9	& 38.46 &	50\phantom{.00}\\
    \textit{gagner} `to win' &	18 &	11 &	6 &	38.46 &	42.31\\
    \textit{donner} `to give' &	75 &	0 &	28 &	38.46 &	38.46\\
    \textit{chaîne} `chain' (cog) &	0 &	0 &	0 &	34.62 &	61.54\\
    \textit{nager} `to swim'	& 39 &	1 &	3	& 34.62 &	46.15\\
    \textit{flèche} `arrow'	& 0 &	0 &	0 &	34.62 &	38.46\\
    \textit{chou} `cabbage' &	34 &	33 &	2 &	34.62 &	34.62\\
    \textit{coeur} `heart' &	53 &	32 &	3	& 34.62 &	26.92\\
    \textit{lancer} `to throw' &	8 &	0	& 5	& 34.62 &	23.08\\
    \textit{parler} `to speak' &	187	& 84 &	18 &	34.62 &	34.62\\
    \textit{main} `hand'	& 160 &	52 &	20 &	30.77 &	19.23\\
    \textit{église} `church' &	0 &	0 &	0 &	29.63 &	38.46\\
    \textit{maison} `house' &	36 &	7 &	13 &	26.92 &	26.92\\
    \textit{roi} `king'	& 0	& 0	& 0 &	23.08 &	50\phantom{.00}\\
    \textit{sauter} `to jump' &	47 &	4 &	4 &	23.08 &	38.46\\
    \textit{feuille} `leaf' &	20 &	0 &	9 &	23.08 &	15.38\\
    \textit{écrire} `to write' &	30 &	0 &	9 &	11.54 &	38.46\\
\end{longtable}

\subsection{Individual learner characteristics and attainment in L2 vocabulary}

Statistical analysis (Pearson correlation) showed that the Year 3 learners’ L1 literacy level and WM scores as measured by the NWR test were quite closely related ($r = 0.630,\allowbreak p < 0.001$). A standard multiple regression was used to explore the strength of these two variables as predictors of L2 vocabulary learning (as reflected in the RVT PT results). In combination, they explained a substantial amount of the variance in learners’ test scores ($R^2 = 0.499$, Adjusted $R^2 = 0.456,\allowbreak p < 0.0005$). Of the two variables, only L1 literacy level had significant independent influence (L1 literacy level, $\beta = 0.620,\allowbreak p < 0.003$; NWR score, $\beta = 0.126$, n.s.). Learners’ age and gender also had no significant relationship with their vocabulary learning. However, our earlier qualitative research has shown that levels of classroom engagement can also influence achievement, in individual cases. \citet{MitchellRule2016} and \citet{MitchellMyles2019} showed that learners with relatively low L1 literacy and NWR scores but high behavioural and cognitive engagement could achieve better than predicted, and vice versa.

\subsection{The contributions of item frequency and of cognate status to L2 vocabulary learning}

We have seen from the literature review that the frequency of encounters is well established as playing a powerful role in vocabulary learning. However, \tabref{tab:mitchell:1} shows that the frequency of the individual target items in classroom talk was highly variable, with some items offering many more exposures than the literature suggests are necessary for acquisition, and others falling well below any likely acquisition threshold. The literature also suggests that the spacing of vocabulary encounters and vocabulary practice may influence learning; again, the distribution of target items across the 33 recorded lessons was quite variable.

\tabref{tab:mitchell:2} shows the most frequent items (200+ occurrences, including teacher and child output). It is clear that most of these very high frequencies can be explained primarily by these items’ general functions in classroom management, e.g. in expressions such as \textit{levez la main} `put up your hand' or \textit{regardez le tableau} `look at the board' found throughout the lesson corpus. Exceptions are the two numerals \textit{trois} and \textit{dix}, sometimes used in classroom management (e.g. in countdowns), but also explicitly taught and practised as part of a numbers series.

\begin{table}
\begin{tabular}{lcc}
\lsptoprule
{Items} &	Output\footnote{T and child combined}	& Lessons\\\midrule
\textit{trois} `three' &	682 &	33\\
\textit{lever} `to raise, get up' &	486 &	31\\
\textit{dix} `ten' &	397 &	28\\
\textit{regarder} `to look' &	370 &	30\\
\textit{silence} `silence' &	367 &	32\\
\textit{parler} `to speak' &	271	& 18\\
\textit{écouter} `to listen'	& 245 &	27\\
\textit{yeux} `eyes'	& 217	& 11\\
\textit{main} `hand'	& 212 &	20\\
\lspbottomrule
\end{tabular}
\caption{\label{tab:mitchell:2}Lexical items with frequencies of 200+ in all output}
\end{table}

Lexical frequency could also be boosted by an item’s inclusion in a song or a story, e.g. the expression \textit{frappe les mains} `clap your hands' was sung many times in ``\textit{Si tu aimes parler français}'' (a French version of ``If you’re happy and you know it \ldots''), and \textit{épaules} `shoulders' occurred with high frequency in the songs \textit{Tête, épaules, genoux et pieds} \ldots\ and \textit{Si tu aimes parler français} (155 occurrences, though clustered in only 4 lessons). The fruit \textit{fraise} `strawberry' was encountered in frequent repetitions of the story \textit{La chenille qui fait des trous} (a French version of `The very hungry caterpillar'). Other items relating to specific lesson themes might be the focus for targeted multimodal practice and some incidental exposure, again in lesson clusters, e.g. \textit{poisson} `fish' and \textit{chien} `dog' during a sequence of lessons on pets.\largerpage

\begin{sloppypar}
Lexical frequency tended to be lower for items used only incidentally in teacher’s management language, without any systematic instructional focus or expectations that children would produce them (e.g. \textit{attraper} `to catch', \textit{lancer} `to throw', \textit{papier} `paper', \textit{feuille} `leaf; piece (of paper)', \textit{écrire} `to write', \textit{table} `table', each used less than 30 times, by the teacher only). However, some such items could be moderately frequent (e.g. \textit{fille} `girl', \textit{garçon} `boy', each used around 60 times).
\end{sloppypar}

So far, we have reviewed item frequency without distinguishing between the productions of the teacher and the participating children. \tabref{tab:mitchell:3} reports Pearson correlations between the different types of item frequency, and item facility on both PT and DPT. The relationship between the two frequency types is extremely close ($r = 0.837, p < 0.0005$), so that both are significantly related to test performance.

\begin{table}
%\small
\begin{tabular}{l *4{r}}
\lsptoprule
&\multicolumn{2}{c}{Output frequency} & \multicolumn{2}{c}{Facility}\\\cmidrule(lr){2-3}\cmidrule(lr){4-5}
&	\multicolumn{1}{c}{T} &	\multicolumn{1}{c}{Child} &	\multicolumn{1}{c}{PT} &	\multicolumn{1}{c}{DPT}\\\midrule
{T Output Frequency}        &	1\phantom{.837**}   & 0.837**	& 0.441**   & 0.364**\\
{Child Output Frequency}    &	0.837** & 1\phantom{.837**}     & 0.387**   & 0.385**\\
{PT Facility}               &	0.441** & 0.387**	& 1\phantom{.837**}     & 0.828**\\
{DPT Facility}              &	0.364** & 0.385**   & 0.828**   & 1\phantom{.837**}\\
\lspbottomrule
\end{tabular}
\caption{\label{tab:mitchell:3}Pearson correlations between target item frequencies in teacher and child output, and results of PT and DPT. ** Correlation is significant at the 0.01 level (2-tailed).}
\end{table}

Cutting across the role of item frequency in our expectations for acquisition, is the role of the items' cognate status. Again, the literature suggests that cognates recognisable to the learner, aurally and/or in writing, should be easier to learn. Some cognates, which had not occurred at all in teacher output, were accordingly included in the test design (\textit{skier, chaîne}), as well as some which had occurred with very low frequencies (\textit{bébé, crayon, table}).

To explore the relative contributions of item frequency and cognate status to L2 vocabulary learning, we ran a standard multiple regression with teacher frequency and cognate status as predictor variables, and facility values on the PT as the criterion variable. The regression analysis results showed that the two variables jointly explained a statistically significant 35.3\% of variance in test scores ($R^2 = 0.353$, adjusted $R^2 = 0.323,\allowbreak p < 0.001$). While both contributions were significant, teacher frequency made a greater contribution ($\beta = 0.492,\allowbreak p < 0.001$) than cognate status ($\beta = 0.402,\allowbreak p < 0.001$).

\subsection{Multimodality and other pedagogic factors} %5.4 /

As we have just seen, input frequency and cognate status account for a significant proportion of vocabulary learning in this study. However, a glance at \tabref{tab:mitchell:4} will confirm that some non-cognate items are quite well learned, despite not being of the highest frequency, and that some frequent items are poorly learned. In this section, we examine more closely the pedagogic treatment of a selection of these items (both nouns and verbs), to seek to identify those pedagogic strategies which were most effective in supporting learning. We consider the degree of focused attention given to an item, the range of activity types in which it occurred, the nature of multimodal support provided, and (to reflect the rhythm of practice) the number of lessons in which the item occurred.

\vfill
\begin{table}[H]
\caption{\label{tab:mitchell:4}Selection of well learned and poorly learned items (non-cognates)}
    \begin{tabular}{lrrrrr}
    \lsptoprule
    & \multicolumn{2}{c}{Facility} & \multicolumn{2}{c}{Frequency} & \\\cmidrule(lr){2-3}\cmidrule(lr){4-5}
    Item &	\multicolumn{1}{c}{PT} &	\multicolumn{1}{c}{DPT} &	\multicolumn{1}{c}{T} &	\multicolumn{1}{c}{Child} & {Lessons}\\\midrule
    \textit{poisson} `fish' &	96.15 &	76.92 &	124 &	35 &	9\\
    \textit{glace} `ice cream' &	92.31 &	69.23 &	35 &	28 &	5\\
    \textit{frapper} `to clap' &	69.23 &	73.08 &	37 &	18 &	10\\
    \textit{parler} `to speak' &	34.62 &	34.62 &	187 &	84 &	18\\
    \textit{main} `hand' &	30.77 &	19.23 &	160 &	52 &	20\\
    \lspbottomrule
    \end{tabular}
\end{table}
\vfill\pagebreak

The noun \textit{poisson} `fish' was among the very best known at PT, and was still well known at DPT. This word was introduced in Lesson 14 as part of a new curriculum theme ``Pets''. During three lessons (14--16), pet vocabulary was the main focus of attention, with intensive oral rehearsal supported by images on flashcards, by text (word labels) and an iconic gesture for each animal. In a variety of worksheet- and whiteboard-based activities and games, children produced pet names in response to images and gestures, and linked labels and drawings; further drawing and labelling followed in Lesson 18. A pets-themed song was introduced in Lesson 14, and repeated in several later lessons (15, 23, 33); a petshop story was introduced in Lesson 15 (with text and images projected on the whiteboard, plus teacher narration), and repeated in other lessons (17, 20, 21, 23, 33). Both these texts included further incidental exposure to the item \textit{poisson}, as did a short film shown in Lesson 16. Success in learning this particular item seems connected to the initial focused practice, supported with multimodal variations, followed by regular incidental encounters in song and story.

\begin{table}
\caption{Target lexical items (ranked by PT facility)}
    \begin{tabularx}{\textwidth}{lQQ}
    \lsptoprule
    {Item} &	{Pedagogic activities} & 	{Multimodal support}\\
    \midrule
    \textit{poisson} `fish' &	Focused oral practice

                                Metacomment
                                
                                Incidental use (song, film, story, game)
                                
                                Drawing and labelling &	Iconic gesture (swimming)
        
                                                        Image (flashcards, story)

                                                        Text (image labels)
                                
                                                        Text (story)\\
    \tablevspace
    \textit{glace} `ice cream' &	Focused oral practice
    
                                    Incidental use (games, role play)
                                    
                                    Drawing and labelling &	Image (flashcards, whiteboard images)

                                                            Imitation foods

                                                            Text (image labels)\\
    \tablevspace
    \textit{frapper} `to clap' &	Incidental use (song, game)	& Action (handclapping)\\
    \tablevspace
    \textit{parler} `to speak' &	Incidental use (song, classroom management)	& None\\
    \tablevspace
    \textit{main} `hand' &	Focused oral practice
    
                Incidental use (song, classroom management, game) &	Actions (handclapping, hand raising)

                                                            Pointing/touching own body\\
    \lspbottomrule
    \end{tabularx}
\end{table}

The well learned noun \textit{glace} `ice cream' was attached to the last theme of the lesson sequence, which was to create and perform in a café role play. In Lesson 27, relevant food and drink vocabulary was introduced and intensively practised using flashcards, and drawing and labelling activities. In Lesson 29, there was similar practice, varied with two guessing games. The children also received sets of cards with food and drink vocabulary items (text only), and had to sort them according to their personal preferences, and then according to which were ``healthy''\slash``unhealthy''. By Lesson 31, the emphasis was on producing full sentences referring to food preferences and choices (e.g. \textit{je voudrais un hamburger} `I would like a hamburger'), which were intensively practiced individually and in a game. Café menus were also written. In Lesson 32, there was further intensive practice of food lexis and of full sentences through 3 different competitive games supported by flashcards and imitation food toys, including a plastic ice cream cone. Finally a café role play script was read and rehearsed in pairs, and then performed in groups. Songs and a story were featured in this lesson sequence, though these did not happen to include the item \textit{glace}. Success in learning this item seems to be connected to the initial focused practice, plus somewhat increased learner engagement, sustained through attractive games and role play. However, recency could also have supported the children’s strong performance on PT, which declined somewhat at DPT.

The action verb \textit{frapper} `to clap' was well recognised in PT, and its facility score even increased slightly in DPT. However, children’s exposure to this item was completely different from the two nouns just examined, since \textit{frapper} occurred only in the phrase \textit{frappe(z) des/tes mains}, and nearly always within the song ``\textit{Si tu aimes parler français} \ldots\,''(sung in 10 lessons and spaced from Lesson 1 to Lesson 33). It was also always accompanied by the action of handclapping, which may have been significant in enhancing its processing.

The next item to be considered is the verb \textit{parler} `to speak', which despite being of very high frequency (271 occurrences altogether), was poorly learned. A few instances of \textit{parler} were found in the teacher’s classroom management instructions ($n =11$), e.g. \textit{bon vous allez parler ensemble} ``ok you’re going to talk together'' (Lesson 11). However such instructions were typically embedded with more extensive instructions in English, so it was not essential for the children to process them in detail. They also occurred only in the earlier lessons (the expression just quoted from Lesson 11 was the last). Apart from these few instances, all the remaining examples of \textit{parler} again occurred in the song phrase \textit{si tu aimes parler français}; the children only ever produced this item in this song. And unlike \textit{frapper, parler} was not accompanied by any distinctive action or gesture, nor any other form of attention-getting practice. It seems the learners were not yet able to distinguish individual lexical items within the phrase \textit{si tu aimes parler} \ldots. Of course, in the last few lessons they were beginning to construct sentences with verbs of liking such as \textit{je voudrais} `I would like', \textit{je déteste} `I hate', \textit{j’aime} `I like' (in the café role play). It is interesting to speculate whether in due course, this could have led to better analysis of the lexical components of \textit{si tu aimes parler français}.

Finally, we consider the noun \textit{main} `hand', again poorly learned though of high frequency (212 instances) and found in the majority of lessons ($n = 20$). \textit{Main} was used with some regularity in classroom management, usually through the expression \textit{levez la main} `raise your hand', which occurred in 10 lessons. Occasionally, more elaborate expressions were used, e.g. \textit{je veux voir les actions avec les mains et les doigts} `I want to see the actions with your hands and fingers' (Lesson 31). \textit{Main} also received some focal attention in a small group of lessons with a ``Parts of the body'' theme (Lessons 19--21). Here, the teacher used the children’s own bodies rather than images to convey meaning (rather distractingly for at least some children), and the words were presented and practised orally only. Following oral repetition, the game \textit{Jacques a dit} `Simon says' was played, with varied instructions including \textit{touchez la main} `touch the hand'. And, of course, the plural \textit{les mains/tes mains} was part of the commonly sung phrase \textit{frappe les mains} discussed above.

It is not entirely clear why \textit{main} was poorly learned and not clearly extracted from expressions such as \textit{levez la main} or \textit{frappe les mains} in spite of a reasonable range of uses including some multimodal support. The amount of focused attention \textit{main} received in Lessons 19--21 was fairly limited by comparison with other related items, and -- perhaps more importantly -- it was never seen in writing. Incidental uses in classroom instructions and in song may be useful in promoting the acquisition of formulaic expressions, but less so in promoting parsing and the precise identification of individual lexical items within these, given children’s relatively limited capacity for selective attention.

\section{Discussion and conclusion}

The project findings summarised in this chapter regarding children’s individual differences (Research Question 1) confirm past research that shows how different  the influences on L2 learning among school aged instructed young learners are from those applying to naturalistic L1 acquisition. These 7--8 year old children’s overall learning success was strongly related to overall academic attainment, as reflected in L1 literacy level, which is in turn related to the development of WM (\citealt{Sparks2012,CourtneyEtAl2017,KasprowiczEtAl2019}).

\begin{sloppypar}
Our investigations of item facility also confirm the influence of input frequency and of the cognate status of new words on acquisition (Research Question 2: \citealt{Szpotowicz2009,PetersWebb2018,DeWildeEtAl2021}). It seemed that at age 7--8, children could benefit from orthographic as well as phonological presentations of cognates, so that, for example, words like \textit{table} or \textit{serpent} could be learned well. The nature of practice was also key to learning; our study shows that word learning benefited from both focused input and intensive output practice \citep{DeKeyser2012}, and incidental encounters, for example in songs, were generally insufficient by themselves. While the nature of our dataset does not allow for strong conclusions on the distribution of practice, unlike e.g. \citet{KasprowiczEtAl2019}, it was certainly beneficial for focused practice to be followed up by later incidental encounters.
\end{sloppypar}

This study also illustrates the ongoing significance of multimodality in supporting both focused, explicit vocabulary instruction and extensive, implicit exposure (Research Question 2). As suggested by \citet{Syodorenko2010}, for example, the use of gestures, images and text were helpful in engaging attention to particular lexical items during intensive practice, in clarifying meaning, and in sustaining engagement. The contribution of particular types of support (and writing in particular) to focused vocabulary instruction is an obvious area for further research. Teacher management language consistently supported with gesture, games, songs and stories supported with images, gesture, toy objects and texts for stories, provided richer and more varied exposure to French, in which formulaic language could be consolidated, and new vocabulary could also be encountered meaningfully in more varied contexts: see, for example, the balanced treatment of \textit{poisson} which led to highly successful learning of this item. 

Finally, what can be learned more broadly from this study, regarding the likely outcomes of L2 learning under UK conditions? With a specialist teacher, during 38 hours of instruction, these Year 3 children were exposed to a corpus of almost 700 French lexemes (types) in teacher speech. Of these, around 330 were heard on 10 occasions or more, and 190 of this particular subgroup were nouns and verbs. If these words were learned in similar proportions to performance on the 50-item RVT test, then the children could be expected to have receptive knowledge of c100 nouns and verbs (plus an unknown number from other word classes). This is soberingly small compared with L1 acquisition rates, but nonetheless a not insignificant first step on the ladder (and at least comparable with findings e.g. of \citealt{CableEtAl2010}). To maintain and possibly increase progress, it is clear that vocabulary development requires a strategic approach, including both intensive focused practice of target items with multimodal support, and ongoing extensive exposure in engaging meaning-focused activities. Use of target language for classroom management, and use of repetitive formulaic language, are an important enrichment, provided lexical content is also supported in other ways (e.g. through extraction and practice of items within formulas). The learners will need multiple retrieval opportunities that reinforce the range of encounters that they have with a word (\citealt{KasprowiczEtAl2019,Nakata2020,Newton2020}). These retrieval opportunities will also need to be mirrored in any test conditions for the learners \citep{Nakata2020}. The regular linking of spoken and written word forms, from the very beginning, is helpful to literate learners for word isolation, identification and memorisation, as well as (in due course) for cracking new phoneme-grapheme correspondences.

Overall, the ``French from 5, 7 and 11'' research programme of which this study is a part has demonstrated the intense value of longitudinal documentation of instruction for a better understanding of the pace and mechanisms of classroom learning. Following this pioneering path set by Florence, more studies of this kind, extending into other areas of learning (grammar, pronunciation \ldots), and to other ages and stages of learning, will be essential to underpin more effective delivery of language learning in the challenging conditions of the Anglophone UK environment.

\printbibliography[heading=subbibliography,notkeyword=this]
\end{document}
