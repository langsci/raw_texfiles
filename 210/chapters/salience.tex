\chapter{A few words on salience and exemplar theory}
\label{ch.sal}

This chapter contains some thoughts on the notion of \isi{salience} and its role within the framework of \isi{exemplar} theory.
Both concepts are of prime importance for this study, and it is therefore vital that some basic assumptions pertaining to these notions be defined before we move on to the empirical results that they will help interpret and explain.

	\section{Salience}
	\label{sec.sal.sal}

In this book, the concept of \isi{salience} has already been brought up several times by now, without, however, having received a definition of any kind.
Since the term is omnipresent in sociolinguistic research chances are that most readers will have a pretty good idea of what `\isi{salience}' is, but it is not at all unlikely that there will not be just one idea, but several ideas.
This is because sociolinguistic \isi{salience} is a notoriously vague concept that is defined in a number of different ways by different researchers.
I do not intend to partake in the discussion as to which of the various definitions of \isi{salience} is the most useful one, since -- as I hope to make clear below -- the question of what \emph{makes} a linguistic variable salient\is{salience} is largely irrelevant to the present study.
This study is rather interested in what \isi{salience} \emph{does}, primarily in perception.
A short review of some relevant literature is nevertheless necessary in order to avoid confusion as to what exactly is meant when the term `\isi{salience}' is used in this work.
However, this account will deliberately be as brief as possible; more detailed analyses of \isi{salience}, its history, and use in sociolinguistics can, for example, be found in \cite{kerswillwilliams2002}, \cite{racz2013}, and \cite{auer2014} -- all three of which are also the primary sources of what is to follow below.

		\subsection{Salience and circularity}
		\label{sec.sal.sal.circle}

Strictly speaking, providing a basic definition of \isi{salience} that all or at least the majority of researchers can agree on should be a rather straightforward and uncontroversial task.
As the \emph{Oxford English Dictionary} puts it, \isi{salience} (in psychology) is the ``quality or fact of being more prominent in a person's aware\is{awareness}ness or in his memory\is{memory structure} of past experience'' -- in simpler terms, \isi{salience} is the quality of `sticking out' from the rest.
\textcite[81]{kerswillwilliams2002} stay very close to this general description when they define (socio-)linguistic \isi{salience} as ``a property of a linguistic item or feature that makes it in some way perceptually and cognitively prominent\is{foregrounding}''.
While the two definitions are very similar, there is actually a crucial difference, because \citeauthor{kerswillwilliams2002} talk about \isi{salience} as something that \emph{makes} a feature stick out, not just the simple fact that it \emph{does}.
This type of definition can easily lead to what \textcite[cf.][9]{auer2014} criticises as mixing \emph{criteria} that allow us to identify salient\is{salience} features with the \emph{causes} of \isi{salience}, i.e. the traits that \emph{make} a variable salient\is{salience} in the first place.
He does, however, acknowledge that criteria and causes often \emph{are} difficult to distinguish because they can actually be dependent on each other.
His example is based on overt corrections, which are not only evidence for the \isi{salience} of the corrected feature, but which also have their share in \emph{making} the feature salient\is{salience} within the \isi{speech community}.

A more serious problem ensues when \emph{criteria} and \emph{effects} of \isi{salience} (on language \isi{change}) are not strictly kept apart.
This issue is addressed by \textcite[82]{kerswillwilliams2002} as well, who argue that when \isi{salience} is used as ``a potential explanatory factor, (\ldots) the concept all too easily lapses into circular\is{circularity}ity and mere labelling'', a point that is illustrated very well by their critique of \textcite{trudgill1986}.
According to \citeauthor{trudgill1986}, salient\is{salience} \isi{marker}s can be distinguished from non-salient\is{salience} \isi{indicator}s (see \sectref{sec.sal.sal.study}) by the fact that, among other things, the former are stigmatise\is{stigmatisation}d and undergoing \isi{change} while the latter are not.
The problem is that \isi{stigmatisation} and the \isi{change} that it often entails (for example, when people start avoiding the stigmatise\is{stigmatisation}d variant) are not only the prerequisites of \isi{marker} status, but also its outcome -- people are aware\is{awareness} of non-standard variants because they are stigmatise\is{stigmatisation}d, and the variants are stigmatise\is{stigmatisation}d because people are aware\is{awareness} of them.
This essentially boils down to saying that a variable is salient\is{salience} because it is salient\is{salience}, which means that `\isi{salience}' loses any explanatory potential altogether.

In the present study, this would correspond to
\begin{inparaenum}[(1)]
	\item hypothesising that only salient\is{salience} variables will create a \isi{priming} effect,
	\item running a perception experiment directly, and then
	\item claiming that the presence of a \isi{priming} effect for some variables but not for others is evidence for their \isi{salience},
	\item which in turn explains their behaviour in the perception test.
\end{inparaenum}
To avoid this kind of circular\is{circularity}ity it is therefore absolutely crucial to establish the \isi{salience} status of the test variables \emph{independently}, which is why the production data were collected.

\newpage 
Research based on the notion of \isi{salience} is perhaps particularly prone to falling victim to the circular\is{circularity}ity trap because ``\isi{salience} \emph{attempts} to combine both structural (language-internal) factors with sociolinguistic and psychological (ex\-tra-linguistic) factors in a single explanatory concept'' \parencite[83, my emphasis]{kerswillwilliams2002}, but many researchers actually focus primarily on one particular aspect only.
However, if \isi{salience} is to have any explanatory value (which necessitates avoiding circular\is{circularity}ity), ``it \emph{must} have recourse to extra-linguistic factors, which will be a combination of cognitive, social psychological or pragmatic fac\-tors'' \parencite[83, my emphasis]{kerswillwilliams2002}.

		\subsection{Cognitive vs. social salience}
		\label{sec.sal.sal.cog}

The way it is commonly used, sociolinguistic \isi{salience} is thus a concept that combines cognitive and social components.
However, as \textcite[cf.][11]{racz2013} points out, it actually makes sense to distinguish cognitive and social \isi{salience}.
The cognitive aspect is at least implicitly present in the most basic definition of \isi{salience}: for something to `stick out' it needs to have some quality that makes it more prominent in perception, and since this is inevitably linked to processing it is part of the cognitive domain.
Social \isi{salience}, according to \textcite[cf.][10]{auer2014}, is based on the fact that a particular feature can be linked to a certain (social) type of speaker, who, in turn, is associated with social and emotional evaluations, which are then transferred to the linguistic feature itself.
The stronger these negative or positive evaluations are, the more (socially) salient\is{salience} the feature will be.
Naturally, a feature has to be noticed first before it can be socially evaluated and judged, so cognitive \isi{salience} is in fact a prerequisite of social \isi{salience}.
If a feature is \emph{cognitively} salient\is{salience} it can acquire social meaning and thus become \emph{socially} salient\is{salience}, too -- crucially, though, it does not have to \parencite[cf.][11]{racz2013}.
Cognitive \isi{salience} is thus a necessary, but not a sufficient condition for social \isi{salience}.

Distinguishing cognitive from social \isi{salience} can potentially help in sorting out some of the apparent confusion in \isi{salience} research, because it allows to separate problems concerned with, for example, the interplay of social \isi{salience} and language \isi{change}, from a discussion that is more focussed on the primary causes of \isi{salience} in the cognitive domain, irrespective of whether or not they result in social \isi{salience} in a particular context.
However, researchers are not really agreed on what makes something \emph{cognitively} salient\is{salience}, either.
While he does not claim that this is the only source of \isi{salience}, \textcite[cf.][9]{racz2013} largely equates cognitive \isi{salience} with surprise and operationalises it by means of transitional probabilities: a feature is surprising if it is unexpected in a particular context, i.e. when it has a low probability of occurrence.

\textcite[cf.][37]{jaegeretal2017} embrace the same idea of \isi{surprisal} as a function of unexpectedness, or low probability of occurrence, in a given context and equate it with \isi{informativeness} -- the more surprising an input, the more information is gained by processing it.
They champion this operationalisation of \isi{salience} not only because it is relatively easy to quantify, but also because \isi{surprisal} has been found to play a role in research looking at reading times and implicit learning \parencite[cf.][37]{jaegeretal2017}.
Crucially, \citeauthor{jaegeretal2017} see \isi{surprisal} as (one of) the cause(s) of \emph{initial} \isi{salience}, when the listener first encounters a given variant.
Long-term \isi{salience}, as the result of cumulative exposure, on the other hand, is based on \enquote{\isi{informativeness} about social group membership} \parencite[38]{jaegeretal2017}, i.e. on the association of a feature with a group of speakers, in whose speech it is usually frequent and thus not unexpected any more.

This account may well be able to explain the diverging levels of \isi{salience} reported in the literature for the four variables analysed in this book.
Lenition of /k/ and fronted \textsc{nurse} are largely limited to Liverpool English, while velar nasal plus and happ\textsc{y}-tensing are also found in other accents.
From the point of view of the \isi{speech community} as a whole, the former two have thus a lower probability of occurrence, and are also more informative with respect to their association (only) with Liverpool speakers.

Conceiving of salient\is{salience} features as surprising (and ``informative'') ones is thus in line with research in psycholinguistics and the cognitive sciences, and this approach may also go some way to explaining the \isi{salience} of certain sociolinguistic variables. 
But at least in sociolinguistics, \isi{surprisal} is by no means the only option.
Many other factors have also been proposed as potential sources of cognitive \isi{salience}, for example (high) \isi{frequency} or phoneme status \parencite[cf.][8]{auer2014}.
Furthermore, it seems quite clear that \isi{attention}, as a top-down factor, interferes with the bottom-up stimulus property of unexpectedness, for example when subjects are asked to count passes in a basketball video and fail to notice a person in a weird costume (a highly surprising event) crossing the scene \parencite[cf.][8]{zarconeetal2017}.
It can thus be said that \isi{attention} \enquote{weights \isi{surprisal} effects from one level or another, depending on the current goals and on perceived rewards} \parencite[8]{zarconeetal2017}.

With regard to the effects of \isi{salience} on linguistic behaviour -- usually \isi{change}, convergence, and divergence are the focus of interest -- I agree with \textcite[17]{auer2014} who claims that sociolinguistic \isi{salience} is ``hierarchically organised'' in the sense that ``cognitive [causes of \isi{salience}] are subordinate to social ones'' (my translation).
He argues that cognitive aspects do contribute to the sociolinguistic \isi{salience} of a variable, but much less so than social ones, and explains that this is because cognitive factors of \isi{salience} are ``filtered'' by the social layer \parencite[cf.][18]{auer2014}.
What this means in practical terms is that only certain cognitively salient\is{salience} features are selected for social evaluation (i.e. they receive social \isi{attention}) while others do not acquire social meaning.
In the first case cognitive factors merely ``reinforce'' sociolinguistic \isi{salience} (which is nonetheless dominated by social evaluations), while in the latter (i.e. when cognitively salient\is{salience} features are not used to do social work), the resulting \isi{salience} of the feature is ``markedly'' lower \parencite[cf.][18]{auer2014}.
Moreover, \enquote{from a sociolinguistic perspective, the choice of features which become [sociolinguistically] salient\is{salience} is in large part an arbitrary one} and seems to depend primarily on \enquote{community consensus} \parencite[56]{llamasetal2017}, which is why I would argue that, \emph{for a sociolinguist}, the question of what makes something cognitively salient\is{salience} can be considered secondary to the (descriptive) knowledge about which features the community agreed to pay \isi{attention} to.

		\subsection{Salience in this study}
		\label{sec.sal.sal.study}

Since the primary hypothesis of this study is that only variables having a very high degree of \emph{sociolinguistic} \isi{salience} are capable of creating \isi{priming} effects in perception experiments (cf. \sectref{sec.intro.intent}), it follows that the focus in independently assessing the \isi{salience} of the variables presented in Chapter \ref{ch.var} should be on social aspects.
I will, therefore, only be interested in \emph{if} a variable is sociolinguistically salient\is{salience} for speakers, but not in \emph{why} it is.
It is, for instance, quite possible that a variable that is found to be socially salient\is{salience} is so because it is more informative than others with respect to unambiguously indexing a particular \isi{speech community}.
Given the fact that I am interested in the \emph{effects} of \isi{salience} rather than its \emph{causes}, however, this piece of information, while interesting, is irrelevant to the present study.
For this reason, cognitive aspects of \isi{salience} will largely remain unaddressed in this book.

In very general terms, the question of interest in the present study is thus simply ``[w]hether a variable is recognised in any way'', which means that this book is in line with many other sociolinguistic studies, where this is ``what researchers (\dots) usually mean when they talk about \emph{\isi{salience}}'' \parencite[4, emphasis in original]{racz2013}.
In contrast to \textcite{racz2013}, who explicitly includes his own work in the above statement, I will not, however, regard a feature as salient\is{salience} if it is recognised in \emph{any} way, but only if it is ``recognised'' as socially meaningful.
The next question, then, is of course how we know that a variable is socially meaningful for speakers.
While Chapter \ref{ch.var} provides a rough distinct\is{distinctness}ion into salient\is{salience} and non-salient\is{salience} variables of Liverpool English as they are presented in the literature, these classifications are
\begin{inparaenum}[(1)]
	\item primarily based on the observations of experts (dialectologists) or laypersons with a special interest in linguistic phenomena (e.g. the authors of the \emph{Lern Yerself Scouse} series), and/or
	\item grounded on databases that are often several decades old (\citealt{watsonclark2013} and \citealt{watsonclark2015} are notable exceptions to both points).
\end{inparaenum}
An additional, up-to-date assessment of \isi{salience} among the speakers of the variety themselves therefore seems desirable to make sure the conclusions drawn in the literature are still valid, and, if possible, to arrive at a more fine-grained ordering of variables on the \isi{salience} scale.

Unfortunately, uncovering social \isi{attitude}s towards a particular phonetic-pho\-nol\-o\-gical feature is seldom a straightforward task.
This is because ``language users are usually very much aware\is{awareness} of particular words or \isi{intonation} patterns \emph{other} people use (\ldots), but are much less attentive to phonetic differences'' \parencite[3, emphasis in the original]{racz2013}.
Directly asking subjects about phonetic or phonemic characteristics of an accent is still an option, but one that, for the majority of speakers, will only work in the case of the most heavily stigmatise\is{stigmatisation}d features.
A more indirect measure is required to capture the middle ground of variables that do carry some social meaning, but not enough to attract overt commentary\is{overt commentary}.
In this study, as in many others, this indirect measure is based on the \emph{effects} of social \isi{salience}, the most important of which include \emph{\isi{social stratification}}, \emph{hypercorrect\is{hypercorrection}ion}, and, above all, \emph{\isi{style shifting}}.

Social stratification is based on the idea that ``the normal workings of society have produced systematic differences between certain (\ldots) people'', which can be thought of in terms of status or \isi{prestige}, and assumes that these social differences are mirrored in linguistic behaviour: when two people can be ranked with respect to a social status criterion, they will be ranked identically with respect to their use of a non-standard feature \parencite[44--45]{labov1972}.
What this means in practical terms is that, for instance, middle-class speakers will usually have lower frequencies of usage than working-class speakers.
In this work, the term will also be extended to gender differences, but certainly not because I wish to imply a social ranking between women and men.
Rather, this is because, in numerous sociolinguistic studies, women have been shown to be more sensitive to linguistic forms that are socially relevant \parencite[cf.][290--291]{labov2001a}, so if women use a variant in a different way than men then this suggests that said variant has acquired at least a certain degree of social meaning.

As a general term, hypercorrect\is{hypercorrection}ion refers to the ``misapplication of an imperfectly learned rule'' \parencite[126]{labov1972}.
In sociolinguistics, the term is traditionally used to describe cases where a particular group of speakers (sub-)conscious\is{awareness}ly tries to approximate the linguistic usage of a (prestigious) target \isi{speech community}, but fails in their endeavour because the speakers actually `overshoot the mark' and end up with realisational rates that are beyond the model set by the target group \parencite[cf.][126]{labov1972}.
In the present study, the term hypercorrect\is{hypercorrection}ion will mostly be used in the more general sense, which extends its scope to any case where a given rule has been learned ``imperfectly'', e.g. when speakers use an even more non-standard variant in more formal speech styles (compared to spontaneous speech) or when they correct the ``wrong'' member of a merger.
Both applications of hypercorrect\is{hypercorrection}ion imply (sub-)conscious\is{awareness} aware\is{awareness}ness of socially meaningful variation, as both the target (in the Labovian definition) or the rule (in the more general reading) have a social component.

Style shifting, finally, is similar to \isi{social stratification} (in fact, another term that is used by Labov is stylistic stratification).
However, in \isi{style shifting}, use of linguistic features is not correlated with social status of the speakers, but with the degree of formality of the communicative situation.
A non-standard variant will thus be used most in very informal (e.g a conversation among friends), less in more formal (e.g. a job interview), and least in the most formal speaking registers (e.g. reading out a written text) -- of course, the reverse is true for standard, prestigious variants.
The presence of \isi{style shifting} presupposes (sub-)conscious\is{awareness} evaluation of the linguistic feature, which results in it being considered more or less appropriate in a given, socially loaded, communicative situation.
In consequence, ``social aware\is{awareness}ness of a given variable corresponds to the slope of \isi{style shifting}'' \parencite[196]{labov2001a}.

Based on \isi{social stratification}, hypercorrect\is{hypercorrection}ion, and \isi{style shifting}, \citeauthor{labov1972}'s \citeyear{labov1972} hierarchy of \emph{\isi{indicator}s}, \emph{\isi{marker}s}, and \emph{\isi{stereotype}s} is a convenient way of categorising linguistic variables according to their sociolinguistic \isi{salience}.
An \isi{indicator} is a (non-standard) linguistic feature which is shared among a particular group of speakers and can therefore act as a defining characteristic of that \isi{speech community} (which it indexes, i.e. `points to'), particularly to outsiders.
The \isi{speech community} itself is, however, completely unaware\is{awareness} of the feature and uses it to the same degree in all communicative situations, so there is no \isi{style shifting}.
When a \isi{speech community} starts to become (sub-conscious\is{awareness}ly) aware\is{awareness} of a feature it is increasingly invested with social meaning and associated with a particular degree of (non-)formality.
These \isi{marker}s show \isi{social stratification} (i.e. they are used more by some social groups and less by others) and \isi{style shifting}: frequencies of non-standard realisations decrease systematically in more formal speaking styles.
A \emph{\isi{stereotype}} finally, does not only exhibit \isi{social stratification} and \isi{style shifting}, but has actually crossed the threshold to conscious\is{awareness} aware\is{awareness}ness, and is explicitly commented on by members of the \isi{speech community} \parencite[cf.][178--180]{labov1972}.
Speakers are thus completely unaware\is{awareness} of \emph{\isi{indicator}s}, only sub-conscious\is{awareness}ly aware\is{awareness} of \emph{\isi{marker}s}, and fully conscious\is{awareness} of \emph{\isi{stereotype}s}.

Originally, \citeauthor{labov1972} conceived of this hierarchy as a sort of sociolinguistic life cycle that every linguistic feature invariably went through: starting out as an \isi{indicator}, acquiring social meaning and turning into a \isi{marker}, before finally becoming the object of \isi{stigmatisation} which eventually leads to disappearance.
He later on corrected this interpretation, however, after several decades of sociolinguistic research had shown that some \isi{indicator}s do not seem to ever turn into \isi{marker}s and that heavily stigmatise\is{stigmatisation}d variants can nevertheless survive \parencite{labov1994}, for instance when they enjoy covert \isi{prestige} as \isi{marker}s (this time in the every day sense of the word) of a local \isi{identity}.
In any case, this question does not affect the usefulness of the \isi{indicator}-\isi{marker}-\isi{stereotype} hierarchy as a means of categorising variables according to how aware\is{awareness} speakers are of them.

Based on the work of \citeauthor{silverstein2003} (cf., for instance, \citealt{silverstein2003}) \textcite[cf.][78]{johnstoneetal2006} have introduced new terminology centred around \emph{first-}, \emph{second-}, and \emph{third-order \isi{indexicality}}.
There is a large degree of overlap between these terms and Labov's \isi{indicator}-\isi{marker}-\isi{stereotype} distinction, while, of course, the two frameworks are not completely identical.
Notable differences can, for example, be found between \isi{stereotype}s and third-order \isi{indexicality}: the former is (traditionally, at least) closely linked to \isi{stigmatisation} and a higher chance of disappearance of the feature, while the latter term focusses on the conscious\is{awareness} use of these features in perform\is{accent performance}ances of local \isi{identity} and presumes that the relevant linguistic variants are, at this stage, primarily associated with place, and less with other social categories such as class \parencite[cf.][81--84]{johnstoneetal2006}.
As we will see later, features of Scouse that can be classified as Labovian \isi{stereotype}s \emph{are} actually used in accent perform\is{accent performance}ances, and do not seem to be disappearing either, so it might seem preferable to use \citeauthor{johnstoneetal2006}'s terminology.
However, with respect to a hierarchical ordering of variables according to how conscious\is{awareness} speakers are of them, \isi{indicator}, \isi{marker}, and \isi{stereotype} -- on the one hand -- and first-, second-, and third-order \isi{indexicality} -- on the other hand -- can be regarded as synonyms.
Since the degree of sociolinguistic aware\is{awareness}ness is what this study is interested in, I will therefore stick to the more traditional Labovian terminology.

\textcite[6]{racz2013} criticises the \isi{indicator}/\isi{marker} distinct\is{distinctness}ion as ``impl[ying] a complete absence of gradience'' while linguistic aware\is{awareness}ness should be conceived of as having ``many levels, very few categorical''.
I agree with the second part of this statement, but I do not see why one would have to give up on the convenience of Labov's classification (which is indeed rather categorical in nature) just because one believes that \isi{salience} is gradient.
It seems to me that it is quite possible to distinguish, for example, different \emph{degrees} of \isi{style shifting} (How many styles are kept distinct? How significant are the differences?), \emph{in addition to} a simple binary assessment of whether \isi{style shifting} is present or not, and the same should hold for \isi{social stratification} or hypercorrect\is{hypercorrection}ion.
Such an approach should allow us to arrive at more fine-grained classifications of variables such as, for example, `solid \isi{marker} close to \isi{stereotype} status' or `\isi{indicator} showing the beginnings of style-shifting'.

No matter how fine-grained the classification, however, what I intend to do is clearly what \textcite{kerswillwilliams2002} have called using \emph{\isi{salience}} as a label, which means that it is ``no more than another term for the \isi{indicator}/\isi{marker} distinct\is{distinctness}ion'' \parencite[32]{racz2013}.
This statement is certainly true with respect to the present study, but, as I hope to have made clear, the use of \isi{salience} as a `mere label' should not constitute a problem against the backdrop of what this book is interested in.
I do not, in fact, \emph{need} more than a convenient label that describes how much social meaning a particular variable carries for its users.
In contrast to the argument presented by \textcite{kerswillwilliams2002} -- who mainly talk about research investigating \isi{change} and contact\is{dialect contact} -- \isi{salience} \emph{will} nevertheless have an explanatory value in the present study when it is linked up to how sociolinguistic variables behave in \isi{priming} experiments.
Salience\is{salience}, in this book, will thus be understood as meaning the amount of (social) aware\is{awareness}ness speakers have of a sociolinguistic variable.
As such, it will be measured by the presence and, if applicable, degree of \isi{social stratification}, hypercorrect\is{hypercorrection}ion, \isi{style shifting}, and explicit comments\is{overt commentary} and evaluations.

	\section{Exemplar theory}
	\label{sec.sal.exemplar}

Any linguistic study that deals with the perception of speech is faced with the theoretical problem of how listeners process the range of intra- and inter-speaker variation that is abundant in naturalistic language data.
Sociophonetic studies in particular have largely turned away from traditional accounts which assume that variation in the speech signal is normalise\is{normalisation}d away to make the input fit into highly abstract and idealised mental categories.
Most researchers explain their results against the backdrop of \isi{exemplar} theory, and the present study is no exception in this respect.
I will therefore provide a short overview of the assumptions and principles of this theory before addressing the place of \isi{salience} in this model.
Just as in \sectref{sec.sal.sal}, my account (which is inspired by the one presented in \citealt{juskanma}) must be considered nothing but a brief summary, albeit one that should be more than sufficient for the purposes of the present study.
The reader is referred to \citealt{pierrehumbert2006} for a more detailed discussion.

		\subsection{Basic principles}
		\label{sec.sal.exemplar.basic}
		
Exemplar theory has its origins in psychology \parencite[cf.][]{medinschaffer1978}, where it was conceived as a general theory to model how information is stored, organised, and accessed in long-term memory\is{memory structure}.
The basic tenet of this model is that every stimulus, or sensory experience, leaves a memory\is{memory structure} trace in the perceiver's mind.
Crucially, now, these traces, or `\isi{exemplar}s', are specific in nature, so what is remembered is not a (single) abstract and idealised prototype of a category, but rather there will be a whole number of similar, but still slightly different \isi{exemplar}s.
The information that is stored for any episodic memory\is{memory structure} is not restricted to the single feature of an \isi{exemplar} that is most useful (or maybe even sufficient) to distinguish different mental categories.
Instead, the memory\is{memory structure} trace is poly-dimensional and can include several characteristics \parencite[cf.][517]{pierrehumbert2006}.
For a visual stimulus, for instance, this might include shape, colour, size, and others, even when only the shape is relevant in that hypothetical context.
In fact, \isi{exemplar}s are even ``indexed'' with additional information that is not directly linked to physical or sensory properties of the stimulus itself, but which pertains to the situation or the circumstances under which the experience in question was made.
It would, for example, be remembered that the hypothetical visual stimulus from above was encountered in an experimental setting as part of a \isi{categorisation} task -- and possibly also whether the stimulus was categorise\is{categorisation}d correctly or not \parencite[cf.][210--212]{medinschaffer1978}.
The outcome in long-term memory\is{memory structure} of a number of similar sensory experiences will thus be a cloud of specific \isi{exemplar}s which are indexed with all sorts of additional information.

This should not be taken to imply that there are no mental categories, because \isi{exemplar} theory by no means denies their existence.
It assumes, however, that they are created on the basis of -- and in addition to -- the individual \isi{exemplar}s that are stored in memory\is{memory structure} in full detail.
Categorisation happens via the process of \isi{indexation} just explained.
When perceivers are confronted with stimuli as representatives of a particular category (for example that of ``circle'') then the concrete realisations will be remembered as detailed individual \isi{exemplar}s, but each of them will \emph{also} be indexed as being a member of that category.
A mental ``bin'' in \isi{exemplar} theory thus consists not of a single idealised prototype, but rather of a cloud of individual instantiations that all share a given label.

Newly encountered input will then be perceived (and categorise\is{categorisation}d) with respect to how similar it is to the traces that have already been acquired.
If, for example, a perceiver has remembered a cloud of small blue triangle \isi{exemplar}s, which are indexed as belonging to category A, and a second cloud of big red circles (indexed as instances of category B), then a newly encountered small blue circle is likely to be categorise\is{categorisation}d as a kind of A because the stimulus is more similar to the A \isi{exemplar}s than it is to the B \isi{exemplar}s (provided shape, colour, and size all have equal weighting) \parencite[cf.][210--212]{medinschaffer1978}.
A stimulus `activate\is{activation}s' all remembered \isi{exemplar}s that are similar to it, which essentially means that they are cognitively foregrounded\is{foregrounding} and therefore more ``accessible'' (compared to other \isi{exemplar}s) for help in categorising the new input.
Once an \isi{exemplar} is stored in memory\is{memory structure} it can also act ``as a retrieval cue to access information \emph{stored with} stimuli similar to the probe'' \parencite[210, my emphasis]{medinschaffer1978}.
This means that a stimulus that is similar to one particular \isi{exemplar} X will not only activate\is{activation} this one memory\is{memory structure} trace (and possibly a few others that are also extremely similar), but in fact the whole memory\is{memory structure} cloud of \isi{exemplar}s that share a particular label with X, for example category membership or context in which the \isi{exemplar} was acquired.
It will become clear in the following paragraphs that \isi{activation} of \isi{exemplar}s via indexed information is a crucial aspect of \isi{exemplar} theory for any sociolinguistic \isi{priming} study.

  \subsection{Application in (socio-)linguistics}
  \label{sec.sal.exemplar.socio}

According to \textcite[cf.][517]{pierrehumbert2006}, \textcite{goldinger1996} and \textcite{johnson1997} were the first to interpret linguistic findings (from speech processing) with the help of \isi{exemplar} theory.
In traditional approaches, variation in the speech signal is normalise\is{normalisation}d away to reduce different phones to idealised, essentialist forms which correspond to the abstract phoneme categories in the perceiver's mind.
In an \isi{exemplar} theoretic account, speech sounds enter long-term memory\is{memory structure} as phonetically detailed \isi{exemplar}s, so ``the lowest level of description is a parametric phonetic map rather than a set of discrete categories'' \parencite[519]{pierrehumbert2006}.
Phonemes do exist as mental categories, but as just explained for episodic approaches more generally, they have to be viewed as ``clusters of similar experiences'', whose centres of gravity are malleable and can be changed by ``incremental updating'' of remembered \isi{exemplar}s \parencite[cf.][519]{pierrehumbert2006}.
A phoneme is thus a collection of phonetic variants (the memory\is{memory structure} traces) which are all indexed as being realisations of one particular phoneme \parencite[cf.][113]{pierrehumbert2002}.

Indexation\is{indexation} is, however, not restricted to phoneme assignment, but can also extend to other bits of linguistic information such as the immediate phonetic context.
And of course any \isi{exemplar} can be indexed with information that is somehow related to the wider context the experience was made in (cf. \sectref{sec.sal.exemplar.basic}).
Sociophonetic studies usually assume that phonetically detailed \isi{exemplar}s are primarily indexed with social information about the speaker who uttered them, e.g. their regional origin, gender, age, etc. \parencite[cf.][370]{hayetal2006a}.

Activation\is{activation} of remembered \isi{exemplar}s is conceived of in the same way as in psychology.
When speech sounds are perceived they activate\is{activation} any \isi{exemplar}s stored in long-term memory\is{memory structure} which are phonetically similar to the new input.
The foregrounded\is{foregrounding} memory\is{memory structure} traces then form the basis the input is processed and classified against.
Activation\is{activation} can also be triggered indirectly via social information that the episodic memories are indexed with, a process which is actually very useful in dealing with variation in the speech signal.

Consider, for instance, the perception of vowels.
It is a well-known fact that the formant structure of vowel realisations differs between women and men due to differences in vocal tract length.
A perceiver who has been exposed to both female and male vowel articulations will therefore have two separate clouds of \isi{exemplar}s in long-term memory\is{memory structure}: one indexed with ``female'', one with ``male''.
When this perceiver now engages in conversation with a person they have never met before a non-linguistic perception (such as a visual cue that the interlocutor is female) will activate\is{activation} the memory\is{memory structure} cloud indexed with the appropriate gender before the other person has uttered a single sound.
Thanks to this pre-\isi{activation} of potentially similar \isi{exemplar}s subsequent perception of new material should be easier and more successful.
The two types of \isi{activation} (via similarity and via \isi{indexation}) can reinforce each other: In cases where perceived social information about the speaker and the phonetic shape of the input activate\is{activation} the same group of \isi{exemplar}s, full \isi{activation} will be reached faster \parencite[cf.][370--371]{hayetal2006a}.
If, however, social cues and the phonetics of the stimulus are at odds (for example when a woman has an unusually deep voice), the ``wrong'' \isi{exemplar}s will be activate\is{activation}d via \isi{indexation} and misperception becomes more likely.

Social \isi{indexation} of phonetically detailed memory\is{memory structure} traces is not merely a theoretic assumption of \isi{exemplar} theory but something that has been tested empirically.
\textcite{strandjohnson1996} had participants classify synthesis\is{resynthesis}ed vowels from a \textsc{foot}-\textsc{strut} continuum, which were presented together with photos of female and male faces.
One and the same audio stimulus was classified differently depending on whether it had been accompanied by a photo of a woman or a man.
This non-linguistic bit of information (gender of the speaker) was thus used in perception and biased subjects towards using ``female'' or ``male'' vowel boundaries when classifying the stimuli.
The same effect could be achieved when confronting perceivers with a range of (consonantal) s-ʃ variants.
These two fricatives are primarily distinguished by their central frequency, and the boundary between the two phonemes (i.e. the point where, perceptually, a /ʃ/ becomes a /s/) is typically lower for male than for female realisations.
When subjects assumed a speaker to be male (because they had been shown a photo of a male) the threshold for categorising an auditory stimulus as an instantiation of /ʃ/ was lower \parencite[cf.][]{strandjohnson1996,strand1999}.

Of course, the sex/gender distinction is a rather crucial one in language perception, as men have vocal tracts that are physiologically different from those of women, which results in markedly lower resonance frequencies for the former.
Since the difference is -- at least to a degree -- biologically determined and thus phylogenetically precedes other social categories such as class or occupation, it could be that gender of speaker is a piece of information that enjoys a particular status in linguistic processing.

\textcite{niedzielski1999} has shown, however, that effects of social information on the perception of linguistic material are not limited to gender.
She tested perception of Canadian Raising in Detroit.
Many Canadian speakers have a raised onset in the /ɑʊ/ diphthong, so that the realisation of this vowel is often [əʊ].
These raised variants can also be found in the speech of Detroiters, but while Canadian Raising is a firm part of the stereotypical\is{stereotype} believes people from Detroit hold about Canadians, they are completely unaware\is{awareness} of raised onsets in their \emph{own} speech, which they consider to be standard US English \parencite[cf.][63]{niedzielski1999}.
\citeauthor{niedzielski1999} played her participants recordings of a female Detroit speaker, who naturally produced Canadian raising, presented them with 6 resynthesis\is{resynthesis}ed vowels (ranging from hyper-low to hyper-raised onsets), and asked them to indicate which one sounded most like the one they had heard in the stimulus.
All perceivers listened to the same voice, but half of them had ``MICHIGAN'' written at the top of their answer sheet, while in the other group the corresponding label was ``CANADIAN''.
These labels had a significant effect: although everyone received the same acoustic input, subjects who had been prime\is{priming}d for ``Canada'' were significantly more likely to perceive Canadian Raising than those who had been prime\is{priming}d for ``Michigan'' \parencite[cf.][64--68]{niedzielski1999}.

While \citeauthor{niedzielski1999} does not do so herself, these results can be interpreted as evidence for the existence of social \isi{indexation} of phonetically detailed \isi{exemplar}s.
When the concept ``Canada'' is invoked (via the label on the answer sheet) participants activate\is{activation} memory\is{memory structure} traces that are marked (``indexed'') as having been produced by speakers of Canadian English.
These \isi{exemplar}s contain raised onsets of the /ɑʊ/ diphthong and, since they are cognitively foregrounded\is{foregrounding}, they bias the perceptual system towards hearing these variants in the new input as well.
If the prime\is{priming} is ``Michigan'', however, perceivers activate\is{activation} \isi{exemplar}s that are indexed with `US standard English' (because Detroiters consider themselves to be speakers of standard English).
The centre of gravity in this \isi{exemplar} cloud is, of course, shifted towards lower onsets, so subjects are more likely to perceive non-raised variants of /ɑʊ/ when these memory\is{memory structure} traces bias perception \parencite[cf.][372]{hayetal2006a}.

\textcite{hayetal2006a} later successfully replicate\is{replication}d \citeauthor{niedzielski1999}'s findings.
They had an essentially identical methodology, but used the New Zealand-Austra\-lia opposition to prime\is{priming} participants, instead of Michigan-Canada as in \citeauthor{niedzielski1999}'s study.
Their experiment was concerned with the perception of short front vowels, particularly /ɪ/.
This phoneme is often realised as a raised [i] by Australians, and as a centralised [ə] by New Zealanders.
Speakers in both countries frequently comment on this feature under the label of the ``fish 'n' chips'' \isi{stereotype}, as this is a common phrase that can be used to illustrate the differences in realisation \parencite[cf.][354]{hayetal2006a}.
Participants were asked to match synthesis\is{resynthesis}ed vowels to the ones they had heard in recordings of a female New Zealand speaker.
The only difference between the experimental groups was once again the label at the top of the answer sheet.
Results were comparable to \citealt{niedzielski1999}: subjects prime\is{priming}d for New Zealand were more likely to perceive centralised tokens, while subjects prime\is{priming}d for Australia were more likely to report more Australian percepts \parencite[cf.][359--363]{hayetal2006a}.
\textcite{jannedyetal2011} have shown that a perceptual bias can even be generated when the \isi{priming} categories are (socially and ethnically stratified) districts of one and the same city.

\largerpage
Whether subjects actually \emph{believed} that the speaker was Australian turned out to be irrelevant: once \isi{exemplar}s indexed with ``Australia'' had been activate\is{activation}d by the prime\is{priming} they biased perception, irrespective of conscious\is{awareness} evaluations of the prime\is{priming} \parencite[cf.][374]{hayetal2006a}.
In a follow-up study \textcite{haydrager2010} furthermore demonstrated that such \isi{priming} effects can be generated by much more subtle and less direct cues.
Instead of an explicit label on an answer sheet they used stuffed toys commonly associated with Australia (kangaroo, koala) and New Zealand (kiwi) to prime\is{priming} perceivers.
The toys were merely present in the room where the participant was seated, but they were not directly linked to the experiment.
All the same, they generated a \isi{priming} effect that was comparable to the one found in the \isi{replication} of the \citeauthor{niedzielski1999} study \parencite[cf.][871--872 and 874--875]{haydrager2010}.
Previous research has thus clearly shown that information about the regional origin of speakers is part of long-term phonetic memory\is{memory structure}, and that \isi{exemplar}s activate\is{activation}d on the basis of this type of extra-linguistic information can bias subjects towards perceiving variants that are typically associated with the prime\is{priming}d group of speakers.

		\subsection{Frequency and salience in exemplar theory}
		\label{sec.sal.exemplar.freq}
		
My hypothesis that \isi{exemplar} \isi{priming} in sociolinguistics is a phenomenon that only occurs with (highly) salient\is{salience} variables is not a purely exploratory one.
Rather, it is actually directly motivated by the framework of \isi{exemplar} theory, where \isi{salience} has been suggested to play a role from the very beginnings.

For one thing, \isi{salience} is believed to structure long-term memory\is{memory structure} to a certain degree by ``ranking'' different aspects of a given \isi{exemplar}.
With respect to (indexical) information that is stored with a particular memory\is{memory structure} trace, for instance, \textcite[cf.][210--212]{medinschaffer1978} already pointed out that not all bits need to be equally important, but that the different dimensions an \isi{exemplar} is associated with can, in fact, be weighted.
They use the example of a mannequin, a stimulus which, for almost any perceiver, will share many features with remembered \isi{exemplar}s of the category ``human'' (e.g. overall shape, size, proportions, number of limbs\ldots).
However, the mannequin stimulus differs from the ``human'' \isi{exemplar}s in a very `salient\is{salience}' category, viz. that of animacy.
As a consequence, no subject will cognitively include (i.e. `perceive as') a mannequin among the \isi{exemplar} cloud of humans, despite the large degree of overlap in features related to physical appearance.
In perception, the difference in a salient\is{salience} feature category (animacy) thus overrides more numerous similarities in less salient\is{salience} ones.

While interesting, this is not the effect of \isi{salience} that is most important for the study at hand, because it can, by definition, only unfold in this way once a stimulus has been remembered.
Salience\is{salience} is, however, already a crucial factor during the act of perception \emph{before} the stimulus enters long-term memory\is{memory structure} as an \isi{exemplar}.
Although humans do seem to be able to store quite an impressive amount of information (cf. \citealt{johnson2005}, cited in \citealt[44]{racz2013}) -- meaning that our memory\is{memory structure} \emph{could} theoretically contain all experiences ever made -- we do not, in practice, remember every single stimulus we have encountered during our lifetime.
Rather, \isi{exemplar}s fade over time if they are not activate\is{activation}d, just like any other kind of memory\is{memory structure}, which results in ``[d]ifferent \isi{exemplar}s hav[ing] different strengths'' \parencite[cf.][115]{pierrehumbert2002}.
For this reason, \isi{exemplar} theory has ``\isi{frequency} effects everywhere'' \parencite[524]{pierrehumbert2006}.
Variants that are encountered more often than others can be memorised more often, and will dominate memory\is{memory structure} structure in one of two ways.

Firstly, frequent remembrance\is{memory structure} of similar stimuli results in denser memory\is{memory structure} clouds, i.e. mental categories which simply contain more \isi{exemplar}s than others.
By their sheer numbers, these \isi{exemplar}s develop a ``cumulative force'' that biases the processing of new material: subsequent input is likely to be categorise\is{categorisation}d as a member of this dense cloud as well \parencite[cf.][524]{pierrehumbert2006}.
Secondly, a new experience can be so similar to an already remembered one that it will not be stored as a separate \isi{exemplar}.
Instead, it will ``impact the same [neural] circuits'', which ``involves updating or strengthening'' of the extremely similar \isi{exemplar} already stored in memory\is{memory structure} \parencite[525]{pierrehumbert2006}.
There is thus not an increase in the number of \isi{exemplar}s in a category, but -- at least up to a certain extent -- the existing memory\is{memory structure} traces themselves enjoy a ``cumulative effect of exposure'' \parencite[525]{pierrehumbert2006}, i.e. they become more prominent or foregrounded\is{foregrounding} due to a higher degree of remnant \isi{activation} from the last exposure.

A crucial aspect here is that we are talking about \emph{\isi{frequency} of remembrance\is{memory structure}}, and not simply \emph{\isi{frequency} of occurrence}, of a particular variant.
It is therefore not sufficient to consider the frequencies of certain tokens in, say, a corpus in order to model the memory\is{memory structure} structure of subjects who are exposed to these tokens.
The reason for this is that long-term memory\is{memory structure} is not a mirror image of ``undifferentiated raw experience'' \parencite[525]{pierrehumbert2006}.
Instead, ``a process of \isi{attention}, recognition, and coding which is not crudely reflective of \isi{frequency}'' intervenes between the physical, sensory input on the one hand, and the act of actually storing an \isi{exemplar} on the other \parencite[525]{pierrehumbert2006}.
As a general rule of thumb, research in psychology has shown that perceivers seem to pay more \isi{attention} to ``informative'' events (cf. also the discussion in \citealt{racz2013}) and ``[e]vents that are attended to are in turn more likely to be remembered'' \parencite[525]{pierrehumbert2006}.
\textcite[cf.][525]{pierrehumbert2006} stresses the fact that informative events are often infrequent.
If one passes a particular shop every day, this event will soon not be informative any more and will (no longer) be attended to, resulting in an inability to remember details like specials of the day even a short time after the experience.
If, however, on one occasion, there is a hot-air balloon in the car park next to the store, then this rare event will probably be remembered for a long time and in vivid detail.

Two points need to be mentioned here:
\begin{inparaenum}[(1)]
	\item the tendency \citeauthor{pierrehumbert2006} describes should not be taken to mean that high \isi{frequency} and high \isi{informativeness} are, a priori, mutually exclusive, and
	\item even if they were, the general statement that events that attract \isi{attention} are more likely to be remembered would still hold -- and high \isi{frequency} tokens could very well be attended to by perceivers for reasons other than their \isi{informativeness} (particularly in terms of \isi{surprisal}).
\end{inparaenum}
The bottom line is that which (and how many) \isi{exemplar}s are retained in long-term memory\is{memory structure} is not simply a matter of raw \isi{frequency} in the linguistic input a person receives, but rather one of ``\emph{effective} exposure'', which is ``a function of actual exposure as well as cognitive factors such as \isi{attention} and memory\is{memory structure}'' \parencite[519, my emphasis]{pierrehumbert2006}.

It is not really surprising that \textcite{pierrehumbert2006} discusses the whole issue under the sub-heading \emph{Salience\is{salience}}, because salient\is{salience} features are features that stand out in perception (whatever the exact cause for this may be), which is essentially the same as saying they attract above average degrees of \isi{attention}.
The way \isi{salience} is understood in the present study (cf. \sectref{sec.sal.sal.study}) ties in with this: if speakers are (sub-)conscious\is{awareness}ly aware\is{awareness} of a linguistic feature because it carries social meaning, and this aware\is{awareness}ness shows in production differences (i.e. \isi{attention} paid to their own speech), then it only makes sense to assume that they also pay more \isi{attention} to these features in perception.\footnote{In fact, several studies have produced evidence for a connection between production and perception. \cite{hayetal2006b}, for instance, found that New Zealanders' perception of /ɪə/-/ɛə/ pairs depends on whether the listeners merge these two vowels in their own production. In another study using synthesis\is{resynthesis}ed vowel continua, \textcite{kendallfridland2017} showed that perceptual discrimination of /æ/ and /ɑ/ is influenced not by the absolute position of these vowels in US subjects' realisational spaces, but actually by the degree to which they produced a merger of the low back vowels /ɑ/ and /ɔ/ -- which suggests that the link between production and perception can also have a more indirect base in the relations between vowels instead of their absolute positions.}
If, in turn, salient\is{salience} variants receive more \isi{attention} then it follows that they will be remembered more often, meaning that long-term memory\is{memory structure} will either contain more of these \isi{exemplar}s or it will be biased to a degree by salient\is{salience} memory\is{memory structure} traces that are cognitively more prominent\is{foregrounding}.
In both cases, \isi{exemplar}s containing salient\is{salience} variants should activate\is{activation} considerably faster and more strongly than less- or non-salient\is{salience} ones, and, as a consequence, the resulting \isi{priming} effects should be more powerful for the former than for the latter.

Existing research in sociophonetics has, in fact, collected some evidence that hints at the possibility that \isi{exemplar} \isi{priming} might only work for highly salient\is{salience} variables.
\textcite[cf.][69--75]{niedzielski1999}, for instance, found that the \isi{priming} effect discovered in the perception of Canadian Raising was not statistically robust for vowels undergoing the Northern Cities Chain Shift (which served as secondary test variables).
The \citeyear{hayetal2006a} study of \citeauthor{hayetal2006a}, in turn, produced two secondary findings which are also of considerable interest for the present study:
\begin{inparaenum}[(1)]
	\item the \isi{priming} effect was particularly strong for stimuli containing the word \emph{fish} (which also occurs in the label commonly used to denote this \isi{shibboleth}) \parencite[cf.][363]{hayetal2006a}, and
	\item \isi{priming} with the two secondary dependent variables /æ/ and /ɛ/ was statistically less robust or even completely non-significant \parencite[cf.][367]{hayetal2006a}.
\end{inparaenum}
Both experiments have thus unearthed \isi{priming} effects exclusively, or at least primarily, for linguistic variables that can be classified as sociolinguistic \isi{stereotype}s.

While \textcite{hayetal2006a} do hint at a possible connection between \isi{exemplar} \isi{priming} and the \isi{salience} of the test variable, this is clearly not the primary concern of their study.
Understandably, their discussion of this issue is therefore very brief and also somewhat speculative.
To my knowledge, there is no study to date that has thoroughly and systematically investigated the impact that (social) \isi{salience} has on the presence and strength of \isi{exemplar} \isi{priming} effects.
It is the intention of the present study to start closing this very gap.

\section{Summary}

Salience\is{salience} is defined in a number of ways by different researchers and there is a particularly high degree of disagreement with respect to what causes a feature to be salient\is{salience}.
This book does not partake in this discussion, but is merely interested in the \emph{effects} of \isi{salience} in perception, not its \emph{causes}.
Sociolinguistic \isi{salience} will be understood as a scale of (sub-)conscious\is{awareness} aware\is{awareness}ness.
Features will be classified with respect to Labov's \isi{indicator}-\isi{marker}-\isi{stereotype} hierarchy which, in turn, will be based on the presence and extent of \isi{social stratification}, \isi{style shifting}, and hypercorrect\is{hypercorrection}ion.
For perception, \isi{exemplar} theory (a model which assumes that long-term memory\is{memory structure} contains phonetically detailed \isi{exemplar}s indexed with social information) predicts that -- thanks to the \isi{attention} filter -- salient\is{salience} features will be stored in memory\is{memory structure} more often and/or will be more prominent than non-salient\is{salience} ones.
As a consequence, \isi{activation} of salient\is{salience} \isi{exemplar}s should be easier, faster, and stronger.
It is therefore to be expected that \isi{exemplar} \isi{priming} effects either do not occur at all or are at least considerably weaker when the test variable does not enjoy a high degree of (conscious\is{awareness}) aware\is{awareness}ness among perceivers.