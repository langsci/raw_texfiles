
\makeatletter
\let\thetitle\@title
\let\theauthor\@author
\makeatother

\newcommand{\togglepaper}[1][0]{
   \bibliography{../localbibliography}
   \papernote{\scriptsize\normalfont
     \theauthor.
     \titleTemp.
     To appear in:
     E. Di Tor \& Herr Rausgeberin (ed.).
     Booktitle in localcommands.tex.
     Berlin: Language Science Press. [preliminary page numbering]
   }
   \pagenumbering{roman}
   \setcounter{chapter}{#1}
   \addtocounter{chapter}{-1}
}

\newbool{bookcompile}
\booltrue{bookcompile}
\newcommand{\bookorchapter}[2]{\ifbool{bookcompile}{#1}{#2}}

\SetupAffiliations{output in groups = false, 
                   separator between two = {\bigskip\\},
                   separator between multiple = {\bigskip\\},
                   separator between final two = {\bigskip\\}
                   }


\providecommand{\indi}{$_{i}$ }
\providecommand{\temporal}{[+\textsc{temporal}]}
\providecommand{\indj}{$_{j}$ }
\providecommand{\indk}{$_{k}$ }
% \providecommand{\marg}{\marginpar}
\providecommand{\cl}{\textsubscript{\textsc{cl}} }
\providecommand{\lra}{$\longrightarrow$ }
\providecommand{\Lra}{$\Longrightarrow$ }
\providecommand{\down}{$\downarrow$}
\providecommand{\zero}{\textsuperscript{0} \space }
\providecommand{\zerot}{\textsuperscript{0} }
\providecommand{\arb}{\textsubscript{ARB} }
% \providecommand{\bin}{\B{dl}\B{dr}}
\providecommand{\GN}{[\textsubscript{GN} }
\providecommand{\GV}{[\textsubscript{GV} }
\providecommand{\phrase}{[\textsubscript{P} }
\providecommand{\Su}{[\textsubscript{S} }
\providecommand{\Ve}{[\textsubscript{V} }
\providecommand{\COD}{[\textsubscript{COD} }
\providecommand{\COI}{[\textsubscript{COI} }
\providecommand{\COId}{[\textsubscript{COIdatif} }
\providecommand{\AS}{[\textsubscript{AS} }
\providecommand{\AO}{[\textsubscript{AO} }


% \NewDocumentCommand{\Next}{}{(N)}
% \NewDocumentCommand{\NNext}{}{(NN)}
% \NewDocumentCommand{\LLast}{}{(LL)}
% \NewDocumentCommand{\Last}{}{(L)}

% \newcommand{\oldae}{$\stackrel{\textrm{\tiny e}}{\textrm{a}}$}
% \newcommand{\oldoe}{$\stackrel{\textrm{\tiny e}}{\textrm{o}}$}
% \newcommand{\oldue}{$\stackrel{\textrm{\tiny e}}{\textrm{u}}$}
% \newcommand{\oldao}{$\stackrel{\textrm{\tiny o}}{\textrm{a}}$}
% \newcommand{\oldoo}{$\stackrel{\textrm{\tiny o}}{\textrm{o}}$}
% \newcommand{\olduo}{$\stackrel{\textrm{\tiny o}}{\textrm{u}}$}

\newcommand{\oldae}{aͤ}
\newcommand{\oldoe}{oͤ}
\newcommand{\oldue}{uͤ}
\newcommand{\oldao}{aͦ}
\newcommand{\oldoo}{oͦ}
\newcommand{\olduo}{uͦ}

\newcommand{\DrawPercentageBar}[1]{%
  \begin{tikzpicture}
    \fill[color=black]   (0.0 , 0.0) rectangle (#1*4ex , 1.5ex );
    \fill[color=gray] (#1*4ex  , 0.0) rectangle (4.0ex, 1.5ex);
  \end{tikzpicture}%
}

% A command to cite the shorthand of a bib entry, used in Bloom
\DeclareCiteCommand{\CiteShortHand}
  {\usebibmacro{cite:init}\usebibmacro{prenote}}
  {\usebibmacro{cite:shorthand}}
  {}
  {\usebibmacro{postnote}}

\newcommand{\fnref}[1]{Footnote~\ref{#1}}
