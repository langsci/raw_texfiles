\documentclass[output=paper,colorlinks,citecolor=brown]{langscibook}
\ChapterDOI{10.5281/zenodo.17077194}

\IfFileExists{../localcommands.tex}{
% add all extra packages you need to load to this file

\usepackage{tabularx,multicol}
\usepackage{url}
\urlstyle{same}

\usepackage{listings}
\lstset{basicstyle=\ttfamily,tabsize=2,breaklines=true}

\usepackage{langsci-basic}
\usepackage{langsci-optional}
\usepackage{langsci-lgr}
\usepackage{langsci-osl}
% \usepackage{./langsci/styles/langsci-lgr}
% \usepackage{./langsci/styles/langsci-osl}
% \usepackage{langsci-gb4e}

\usepackage{tikz}
\usetikzlibrary{patterns,calc}
\pgfdeclarepatternformonly{south east lines}{\pgfqpoint{-0pt}{-0pt}}{\pgfqpoint{3pt}{3pt}}{\pgfqpoint{3pt}{3pt}}{
    \pgfsetlinewidth{0.6pt}
    \pgfpathmoveto{\pgfqpoint{0pt}{3pt}}
    \pgfpathlineto{\pgfqpoint{3pt}{0pt}}
    \pgfpathmoveto{\pgfqpoint{.2pt}{-.2pt}}
    \pgfpathlineto{\pgfqpoint{-.2pt}{.2pt}}
    \pgfpathmoveto{\pgfqpoint{3.2pt}{2.8pt}}
    \pgfpathlineto{\pgfqpoint{2.8pt}{3.2pt}}
    \pgfusepath{stroke}}
    
\usepackage{stmaryrd}
\usepackage{wasysym}
\usepackage{multirow}
\usepackage{caption}
\usepackage{subcaption}
\usepackage{mathrsfs}
\usepackage{qtree}

\usepackage{linguex}


%pminos do not split footnotes
% \interfootnotelinepenalty=10000 %Footnote in Laporte chapters has to be split SN


%\DeclareIndexNameFormat{default}{%
%\nameparts{#1}%
%\usebibmacro{index:name}%
%{\index[names]}%
%{\namepartfamily}%
%{\namepartgiveni}%
% {}% L1
% {}% L2
%{\namepartprefix}% generates spurious space L3
%{\namepartsuffix}% generates spurious space L4
%}

%  {\DeclareIndexNameFormat{default}{%
%     \usebibmacro{index:name}{\index[names]}{#1}{#3}{#5}{#7}}}

%\DeclareIndexNameFormat{default}{%
%  \usebibmacro{index:name}{\sindex[nom]}{#1}{#3}{#5}{#7}}

%\DeclareIndexNameFormat{default}{%
%  \usebibmacro{index:name}{\sindex[person]}{#1}{#3}{#5}{#7}}
%\DeclareIndexNameFormat{default}{%
%\nameparts{#1} \usebibmacro{index:name}{\sindex[person]]}{\namepartfamily}{‌​\namepartgiven}{\nam‌​epartprefix}{\namepa‌​rtsuffix}}

%\newcommand{\smiley}{:)}

%\renewbibmacro*{index:name}[5]{%
%\usebibmacro{index:entry}{#1}%
%{\iffieldundef{usera}{}{\thefield{usera}\actualoperator}\mkbibindexname{#2}{#3}{#4}{#5}}}

% \newcommand{\noop}[1]{}

%remove for final
%\overfullrule=1mm

\newcommand{\tobi}[2]}}
\renewcommand{\S}[1]{\tobi{#1}{\textsc{*}}}

% this volume references
% puts: [this volume]
% already defined: \citetv
%\newcommand{\citepv}[1]{(\citeauthor{#1} \citeyear*{#1} [this volume])}
\newcommand{\citealtv}[1]{\citeauthor{#1} \citeyear*{#1} [this volume]}

%parentheses around example number
\newcommand{\pref}[1]{(\ref{#1})}

% in-text examples

\newcommand{\lnex}[1]{\textit{#1}} %target lang word
\newcommand{\lnlit}[1]{(lit.: `#1')} %literal reading
\newcommand{\lnlat}[1]{(#1)} % latinization
\newcommand{\lntrans}[1]{`#1'} %translation
\newcommand{\lnexl}[2]%
{\lnex{#1}{} \lnlat{#2}} % ex with latinization
\newcommand{\lnexlat}[3]{\lnex{#1}{} \lnlat{#2}{} \lntrans{#3}} % ex with latinization and tranl.

%ch01
\newcommand{\co}[1]{\mbox{\textbf{#1}}}

%ch09

\newcommand{\cyrbulg}[1]{\begin{otherlanguage*}{bulgarian}#1\end{otherlanguage*}}


%ch10
\newcommand{\nlp}{{\small NLP}}
\newcommand{\mwe}{{\small MWE}}
\newcommand{\rae}{{\small RAE}}
\newcommand{\lvc}{{\small LVC}}
\newcommand{\pos}{{\small P}o{\small S}}
%\newcommand{\todo}[1]{ \textcolor{red}{#1} }

%\renewcommand{\labelenumi}{\theenumi}
%\ainamefmt{{vv}{ll}{, ff}{, jj}} % fullname

\newcommand{\biberror}[1]{{\color{red}#1}}

\newcommand{\osenovaitem}{--~}
%% hyphenation points for line breaks
%% Normally, automatic hyphenation in LaTeX is very good
%% If a word is mis-hyphenated, add it to this file
%%
%% add information to TeX file before \begin{document} with:
%% %% hyphenation points for line breaks
%% Normally, automatic hyphenation in LaTeX is very good
%% If a word is mis-hyphenated, add it to this file
%%
%% add information to TeX file before \begin{document} with:
%% %% hyphenation points for line breaks
%% Normally, automatic hyphenation in LaTeX is very good
%% If a word is mis-hyphenated, add it to this file
%%
%% add information to TeX file before \begin{document} with:
%% \include{localhyphenation}
\hyphenation{
    Beck-man
    Ngu-yen
    back-chan-nel
    back-chan-nels
    mo-not-o-nous
    ste-reo-typ-i-cal
}

\hyphenation{
    Beck-man
    Ngu-yen
    back-chan-nel
    back-chan-nels
    mo-not-o-nous
    ste-reo-typ-i-cal
}

\hyphenation{
    Beck-man
    Ngu-yen
    back-chan-nel
    back-chan-nels
    mo-not-o-nous
    ste-reo-typ-i-cal
}

\boolfalse{bookcompile}
\togglepaper[23]%%chapternumber
}{}

% % % \addbibresource{../../../Latex/ref.bib}

\author{Barthe Bloom\affiliation{Friedrich-Alexander Universität Erlangen-Nürnberg}}
\title[Cohesion and/or disruption?]      {Cohesion and/or disruption? The discourse functions of Early New High German preverbal \textit{so}}
      
\abstract{This study investigates the discourse functions of the left peripheral \textit{so} in Early New High German narratives and evaluates how it is used as tools for cohesion. It is argued that sequences with a preverbal adverbial +  \textit{so} are primarily, but by no means exclusively used to signal a shift in frame-setting, connecting the following discourse explicitly to the previous.

Based on data extracted from Early New High German prose texts, the paper evaluates whether [adverbial + \textit{so}] is used as a tool for cohesion and/or disruption. Cohesion is a well-established concept and disruption is defined as instances in which the expected flow of cohesion is interrupted. It is argued that it is used for both.

This hypothesis is evaluated by considering the distribution of [adverbial + \textit{so}] within the larger text and its co-occurrence with referential continuity. For this purpose, a distinction is made between constructions with adverbial clauses and those with adverbs. The sequence [adverb + \textit{so}] is exploited as a cohesive tool linking the following proposition explicitly to the previous discourse, whereas sequences containing an adverbial clause or a prepositional phrase may be more disruptive and do not necessarily set up this link themselves.

% \keywords{Early New High German, verb-second, cohesion, referential continuity, adverbial clauses, adverbs}
}

\begin{document}
\maketitle


\section{Introduction}\label{sec:bloom:1}
This study investigates the discourse functions of the  Early New High German (henceforth: \textit{ENHG}) patterns in which an adverbial combines with preverbal \textit{so} in narrative prose and evaluates how it is used as a tool for cohesion or disruption. The examples in \REF{ex:1} illustrate the pattern, which is referred to as \textit{\textsc{advl} + \textit{so}}.

\begin{exe}
\ex\label{ex:1}
\begin{xlist}
\ex \label{ex:1a}
\gll frauwe wolt ir yne sehen \textbf{so} wil ich yne uch bringen\\
woman wanted you him see \textsc{so} want I him you bring\\
\glt `Woman, if you want to see him, I will bring him to you.' (Pontus, 11ra)

\ex \label{ex:1b}
\gll Er ordinert sin schiff in das hoge mere vnd wolt nit das sie sin gewar wurden Dar vmb det er das \ul{Vort} \textbf{so} nam er vl schyff vnd fore ans lant \\
He commands his ship in the high sea and wanted not that they his aware became there for did he that further \textsc{so} took he 45 ship and rode to.the land\\
\glt `He commanded his ship to the high sea and did not want that they would become aware of them. That is why he did that. In addition, he took 45 ships and rode to land.' (Pontus, 85va)\footnote{Punctuation is represented as in the source.}
\end{xlist}
\end{exe}

This use of \textit{so} was more productive in ENHG than it is in Present-Day German \citep{ThimMabrey1987}. Not only did \textit{so} occur following conditional clauses \REF{ex:1a} and concessives – contexts in which it is attested nowadays \citep[31--32]{Catasso2021Generalized}  -- it could also follow adverbial clauses with other functions and is, remarkably, even attested following short adverbs \REF{ex:1b}.
%It is argued first that \textsc{advl} + \textit{so} is primarily, but not exclusively, used to signal a shift in frame-setting. Such shifts in frame-setting on the one hand connect the following discourse explicitly to the previous but simultaneously present a break in the discourse. Thus, the pattern seems to be associated with both cohesion and disruption. 

\largerpage
In previous literature, \textit{so} following adverbials -- both clausal and non-clausal ones -- has been analyzed as a (adverbial) resumptive  \citep{Meklenborg2020}, see for Middle High German \citep{Catasso2021Generalized}, meaning that it is used as a placeholder for a previous constituent within the clause \citep{Haegeman2023}. Its main function is to syntactically and/or semantically integrate a constituent into the upcoming utterance \citep{ThimMabrey1987} and thereby it seems to be a tool used for cohesion.\footnote{But cf. \citet{Axel-Tober2023}, who analyzes this use of \textit{so} as an expletive.} The story of preverbal \textit{so} is, however, a bit more complex. Resumptives in general are thought of as elements that facilitate processing, reducing the load processing takes on the working memory \citep{Hawkins2004, Links2017}. Therefore, they are especially useful with heavy elements, which are difficult to process within a sentence. In addition, they are useful whenever an element introduces new information that would otherwise occur in an unexpected position in the clause \citep{Prince1997}, as for example in \textit{A young man, barely 20 years old, \textbf{he} had jelly from a doughnut on his face}. The dislocated constituent introduces the new information in the discourse. The resumptive subsequently can refer to the same information, which is now given, within the clause. In those cases, the resumptive construction breaks the discourse-flow. Further, an item can be resumed whenever it has not been relevant for some time \citep[11]{Givon1983}. Hereby, resumption may function both as a disruptive element -- activating an element outside of the active realm -- and as a tool for cohesion, since it takes up an earlier item from the discourse. In this light, certain uses of preverbal \textit{so} are unexpected from what is known about resumptives. For example, in \REF{ex:1b} we find \textit{so} following a light element -- \textit{vort} -- that does not refer to new or newly introduced information and \textit{so}, if it is an adverbial resumptive, seems to pick up an adjacent constituent. A similar use of apparent resumptives is found in the modern Scandinavian languages, where the cognate of \textit{so} -- \textit{s{\aa}} -- is likewise found following adverbs \citep{Elmquist1945, Nordstrom2010, Eide2011, Meklenborg2020}. This is exemplified for Swedish in \REF{ex:2}.\footnote{The Swedish examples in the paper originate from the \"A{}ldre svenska romaner section of the Språkbanken corpus \citep{Pettersson2022}.}

\begin{exe}
\ex \label{ex:2}
\gll Och därmed \textbf{så} dansade jag. \\
and there.with \textsc{so} danced I\\
\glt `And thus, I danced.' \citet{Almqvist18421843}
\end{exe}

\largerpage
For \textit{så}, it has been argued that it is used to mark a new point of departure \citep{Nordstrom2010} and has also been connected to a type of topic-shift in Norwegian \citep{Eide2011}. Both analyses show a large degree of overlap with analyses of scene-setting and frame-setting elements \citep{Speyer2008Doppelte}. Due to a similarity in distribution between the Scandinavian \textit{så} and ENHG \textit{so} \citep{Meklenborg2020}, such an analysis may be extendable to ENHG.  In addition, other patterns with an adverbial in non-initial but preverbal position in Present-Day Dutch have been argued to mark topic-shift as well \citep{Breindl2011}. It is therefore very well possible that a similar analysis can be preposed for the ENHG \textsc{advl} + \textit{so}.

It should be noted, however, that in contexts like \REF{ex:1b}, it has been argued that it is not in fact the adverb preceding \textit{so} that introduces the new topic or presents a new point of departure, since the adverb is connective in this context \citep{BloomJouGe}. When \textit{so} follows an adverb, it then does not function as a resumptive taking up the adverb per se but instead seems to collaborate with the adverb to connect the following proposition to another discourse segment \citep[see e.g.][]{Patocka1998}.\footnote{There are several arguments against a unified analysis of \textit{so} with adverbs and adverbial clauses, in particular their position in relation to vocatives and their backward-orientation \citep{BloomJouGe, BloomTT}. The combination of the adverb and \textit{so} may have formed a chunk \citep[see][165ff.]{Traugott2022}, as the two elements typically serve a joint function. However, the most frequent adverbs are not necessarily adjacent to \textit{so} in these contexts \citep{BloomJouGe}, which makes there status as one unit a bit less convincing but of course not improbable (see e.g., particle verbs).} The results in this paper do support this, as adverb + \textit{so} is more strongly a tool for cohesion than adverbial clause + \textit{so}.

On a terminological note, I refer to the preposed adverbial, be it clausal or not, as the precursor and to the sentence that hosts it, as the host, or host sentence. The use of \textit{so} in this context is abbreviated as PV \textit{so}, which is short for preverbal \textit{so} in order to remain agnostic to its precise function. The construction in which an adverbial and \textit{so} combine is referred to as \textsc{advl} + \textit{so}.

This paper first argues that by and large \textsc{advl} + \textit{so} is associated with a shift in frame-setting but notices that it is also found in other contexts. This is primarily the case for patterns with non-clausal adverbials. Based on data extracted from ENHG prose texts, the paper evaluates whether \textsc{advl} + \textit{so} is used as a tool for cohesion and/or disruption. It is argued that it is used for both. This hypothesis is evaluated by considering the distribution of \textsc{advl} + \textit{so} within the larger text as well as its co-occurrence with referential continuity. Some differences emerge between sentences with clausal and non-clausal adverbial clauses, as the data indicate disruptive contexts for the former but not the latter. Combinations of an adverb with \textit{so} in particular are argued to link the following predication to the previous discourse, whereas adverbial clauses and prepositional phrases do not necessarily set up this link themselves and regularly co-occur with a coordinating conjunction.

\sectref{sec:bloom:2} presents the background to the current study and introduces the concepts of cohesion and disruption and of frame-setting. \sectref{sec:bloom:3} introduces the data. In \sectref{sec:bloom:4}, the question whether and to what degree \textsc{advl} + \textit{so} signals a (shift in) frame-setting is evaluated. \sectref{sec:bloom:5} focuses on the issue of cohesion and disruption, evaluating the distribution of the construction within the narrative (at the beginning of chapters and turns, versus within narrative sequences) and the extent of its co-occurrence with referential continuity. \sectref{sec:bloom:6} then considers the different types of adverbials (adverbs, clauses, and prepositional phrases) that precede \textit{so} and presents a more detailed argument as to their differential behavior in relation to discourse cohesion and disruption. Finally, \sectref{sec:bloom:7} concludes the paper.

\section{Background}\label{sec:bloom:2}
\subsection{Cohesion \& disruption}
Central in this study are the concepts of cohesion and disruption. Going back to the basic definitions, the term cohesion can be defined as ``the act or state of sticking together" \citep[cohesion]{MerriamWebster}, whereas disruption is ``a break or interruption in the normal course or continuation of some activity, process, etc.'' \citep[disruption]{MerriamWebster}.

For texts, cohesion is created by what \citet{Halliday1976} call \textit{ties}, i.e., elements that partake in dependency relations. These cohesion devices contribute to ``the continuity of meaning in the text.'' \citep{Griffith1986}. Well-known types of ties are replacement \REF{ex:ties1}; ellipsis \REF{ex:ties2}; conjunction \REF{ex:ties3}; and lexical cohesion \REF{ex:ties4} and \REF{ex:ties5}.

\begin{exe}
\ex \label{ex:ties}
\begin{xlist}
\ex \label{ex:ties1} \textit{They wait for her to go. She \textbf{does}.}
\ex \label{ex:ties2} \textit{You've shared your stories. Well, she has \textbf{\_\_} .}
\ex \label{ex:ties3} \textit{I came to Caldera, \textbf{because} I wanted experiences.}
\ex \label{ex:ties4} \textit{In the \textbf{library} he lingered over the \textbf{books} and pored over rare Japanese \textbf{prints}.} (COCA, \citealt{Davies2009})
\ex \label{ex:ties5} \textit{These \textbf{tracks} here are from a male running, just the toe of his \textbf{shoe} leaves a \textbf{print}, takes all his weight.}
\end{xlist}
\end{exe}

\noindent In each of those cases, ``the interpretation of some element in the discourse is dependent on that of another'' \citep[4]{Halliday1976}. To be explicit, in \REF{ex:ties1}, the interpretation of \textit{does} depends on the \textit{to go} in the preceding sentence; for \REF{ex:ties2} \textit{she has} is interpreted on the basis of the preceding \textit{`ve shared your stories}, in \REF{ex:ties}, \textit{I wanted the experiences} is interpreted in relation to \textit{I came to Caldera}, and in \REF{ex:ties4}, the interpretation of \textit{prints} is guided by the preceding \textit{library} and \textit{books} different than e.g., in \REF{ex:ties5}, where \textit{print} is interpreted against \textit{tracks} and \textit{shoe}.

Disruption should be a closely related term to talk about cases in which there is a break or interruption in the expected flow of cohesion. This concept has been used in studies on e.g., narration in psychology \citep{Jepsen2022} but is generally not used in contexts of narration by non-impaired speakers. Yet, breaks or interruptions in the expected continuation of a discourse or text are commonly signaled explicitly by linguistic means.
For example, unpredictable referents are frequently explicitly signaled \citep{Haspelmath2021}: Low accessible referents, i.e., unexpected ones, are realized in a different form (e.g., by long definite descriptions) than highly accessible referents (which may be left unexpressed or are realized by pronouns) \citep{Ariel2001}. Another example is differential object marking, in which animate and/or definite objects -- which are unexpected since prototypical grammatical objects are inanimate and indefinite -- are cross-linguistically frequently indicated by an extra morpheme \citep{Aissen2003}. This is illustrated for Spanish in \REF{ex:DOM}. 

\begin{exe}
\ex \label{ex:DOM}
\begin{xlist}
\ex Encontré un problema.
\glt `I found a problem.'
\ex Encontré \textbf{a} un superviviente.
\glt `I found a survivor.' \citep{Fabregas2013}
\end{xlist}
\end{exe}

Other markers associated with disruption are the various discourse particles.  Some concrete examples are that the beginnings and endings of a segment are often accompanied by a discourse particle, such as the German \textit{ach} and \textit{achso} as used in conversation \citep[e.g.,][]{Golato2008, Barske2010, Steensig2013}. Moreover, \textit{by the way} in English is used as a so-called digressive marker, indicating a shift away from the current topic \citep[46]{Traugott2022}. Discourse particles may also be used to indicate that the information conveyed is surprising or unexpected, like the English \textit{actually} \citep[e.g.,][74]{FoxTree2015} or Dutch \textit{eigenlijk} \citep{VanBergen2018, Rasenberg2020}. 

In each of these examples, an interruption of the cohesive flow is mitigated by a signal that indicates that there is something different than what the normal course or continuation of the discourse or narration would lead one to expect. Disruption is thus signaled by the use of additional linguistic material to modify the predictions made by the addressee of what follows.


\subsection{Frame-setting}
Frame-setters are elements that ``limit the applicability of the main predication to a certain domain'' \citep[50]{Chafe1976}. In an earlier study, I referred to these domains as spaces  \citep{BloomJouGe}. However, in order to relate the current study to the literature on frame-setting, where the definition of a frame is distinct from its definition in the literature on mental spaces
\citep{Fauconnier1994, Fauconnier1996, Dancygier2005} and to avoid terminological confusion, I refer to these domains as realms, in which propositions are predicted or claimed to hold.

Adverbial clauses in particular have been described as elements that serve as so-called “guideposts”, ``providing a temporal, conditional, causal, or other such orientation for the information in the upcoming main clause'' \citep[444]{Chafe1984}. They provide an orientation, viz. a ground. For example, a conditional clause \REF{ex:ENcon} or a temporal clause \REF{ex:ENtemp} may activate a realm in which a certain proposition holds.

\begin{exe}
\ex \label{ex:ENcon} If Hiro reaches out and takes the hypercard, then the data it represents will be transferred from this guy’s system into Hiro’s computer. \citep[33]{Dancygier2005}
\ex \label{ex:ENtemp} When she awakens she will think a magician has been there. \citep[46]{Dancygier2005}
\end{exe}

In \REF{ex:ENcon}, the data will be transferred under the condition, that is, in a realm in which Hiro reaches out and takes the hypercard. In \REF{ex:ENtemp}, the temporal clause similarly presents a realm in which it is claimed that she will think that a magician has been there. 

 Frame-setters are thought to imply contrast: ``if there is no alternative perspective to be considered, (...) there is no need for an explicit frame setter either'' \citep{Krifka2008}. In this way, they are similar to contrastive topics. The difference between them is, according to \citet{Krifka2008}, that contrastive topics mark that the topic of a sentence diverges from the given expectation, but frame-setters mark that the information that is provided is restricted to the specified dimension.  The content of the clause following a conditional is claimed to apply in the realm in which the content of the \textit{if}-clause holds, and there is an implication that if Hiro does not reach out, the data also won't be transferred. There is thus limited applicability. Similarly, in the temporal adverbial clause in \REF{ex:ENtemp} there is the implication that she won't think that a magician has been there before she awakens.

The limited applicability of the main predication implies that frame-setters have a direct effect on the truth of the main predication, which has been recognized by \citet[656]{Jacobs2001}. That is, according to him, the main predication should not hold outside of the constraints, as the frame-setter directly affects the truth of the comment. This presents a rather straightforward operationalization of the notion of frame-setter: If the information that is provided applies outside the realm signaled by the precursor, we are not dealing with a frame-setter in the narrow sense. However, as will become more clear throughout the paper, in particular in \sectref{sec:bloom:4}, frame-setting is not always used in this strict sense but has also been applied to contexts in which the truth of the comment is not necessarily invalidated outside of the specified realm \citep{Maienborn2001, Maienborn2012}.

What this broader sense of frame-setting may include can be elaborated by making reference to work on the Scandinavian cognate of PV \textit{so}, \textit{så}. This element has been proposed to mark a shift in frame-setting \citep{Nordstrom2010, Eide2011}. \citet{Nordstrom2010} focuses on the use of \textit{så} after non-clausal constituents in modern Swedish. An example can be found in (\ref{ex:SVso}). \citet{Nordstrom2010} argues that this use of \textit{så} is used to signal a new point of departure. This term can be understood as the landmark \citep{Langacker1987, Langacker2008} or ground \citep{Talmy1975}, i.e., the reference point ``with respect to which the figure's path or site receives characterization" \citep[419]{Talmy1975}. Crucially, the idea is that \textit{så} is used in contexts where the ground for the following proposition is not the same as the one in which the preceding utterance should be understood. To put it differently, where the realm in which the following predication is to be understood changes. In the example in \REF{ex:SVso}, this constitutes a temporal shift, but note that the proposition `I froze it in' is nevertheless entailed.

\begin{exe}
\ex \label{ex:SVso}
\gll Och \textbf{sen} \textbf{så} fr\"os jag in det\\
and then so froze I in it\\
\glt `And then I froze it in.' \citep[47]{Nordstrom2010}
\end{exe}

\citet{Sollid2008} and \citet{Eide2011} present a similar analysis for Norwegian \textit{så} and analyze the Norwegian \textit{så} as an Aboutness-Shift topic, in the sense of \citet{Bianchi2010}.
 They argue that \textit{så} is used to signal a new topic, or to return to a previously abandoned topic.

Frame-shifting constructions are associated with both cohesion and disruption. Between the precursor and the host, frame-setter constructions are cohesive: The interpretation of the host depends on the precursor. This is rather obvious, as the sentence forms a structural unit, which are thought to be cohesive by definition \citep[8]{Halliday1976}. Beyond this structural relation, however, the construction disrupts cohesion between the host and the discourse that precedes the precursor: By the explicit introduction of a new frame, the applicability of the proposition or the ground upon which the proposition in the host  is to be interpreted is no longer determined by the preceding discourse, thereby disrupting the discursive flow.

\section{Data}\label{sec:bloom:3}
 The central data set used in the current study consists of 898 adverbial + \textit{so} + finite verb sequences that are extracted from a preliminary version of the Roko.UP \citep{Roko}, a corpus of ENHG narrative texts. In addition, these texts have been used to extract data and examples to illustrate phenomena going beyond the scope of the central data set. Whenever this is the case, it is mentioned explicitly.
 
\begin{table}
\caption{Early New High German narratives}\label{tab-1}
\begin{tabular}{lll} 
\lsptoprule
\multicolumn{3}{l}{Middle High German}\\
\quad  Wilhelm& Upper German  & 1481\\
\quad Tristrant& Upper German & 1484\\
\quad  Wigalois & Upper German & 1519\\
\multicolumn{3}{l}{French and Latin}\\ 
\quad Pontus & Rhine-Franconian& between 1450 and 1476 \\
\quad Melusine & Upper German & 1474[1456]\\
\quad Huge Scheppel & Rhine Franconian & between 1455 and 1472 \\
\quad Magelone & Upper German & 1535\\
\quad Goldener Esel & Upper German & 1538\\
\multicolumn{3}{l}{No model text}\\
\quad Fortunatus& Upper German & 1509\\
\quad Rollwagenbüchlein& Upper German & 1555\\
\lspbottomrule
\end{tabular}
\end{table}

These texts originate from different narrative traditions: First, \emph{Tristrant und Isalde}, \emph{Wigalois vom Rade} and \emph{Wilhelm von \"O{}sterreich} are reworkings of Middle High German poetic novels. 
Second, the corpus contains adaptations from French texts: \emph{Pontus und Sidonia}, \emph{Huge Scheppel}, \emph{Melusine}, and \emph{Die Schöne Magelone}, and one text translated from Latin, the \emph{Goldene Esel}. 
Third, \emph{Fortunatus} and \emph{Das Rollwagenb\"uchlein} are not based on a model text. \emph{Fortunatus} was originally written in ENHG and is a long running narration, just like the other texts of the corpus. \emph{Das Rollwagenb\"uchlein} is a collection of farces. For \emph{Huge Scheppel} and the \emph{Goldene Esel} the data has been extracted from excerpts of ca. 25,000 words; for the other sources, the entire text was used.
All but two texts come from the Upper German dialect area; only \emph{Pontus} and \emph{Huge Scheppel} are written in the Rhine Franconian dialect. This is summarized in Table {\ref{tab-1}}. 
For a general overview of the frequency distribution of the construction, see \citet{BloomJouGe}.

Narrative prose is a particularly useful genre to investigate issues of cohesion and disruption, as they are cohesive texts that typically discuss multiple characters and report on a plurality of events. Moreover, narrative prose is per definition temporally organized, with many deviations from iconic experientiality \citep{Enkvist1981}, i.e., the order and balance in temporal sequence in the discourse coincides with that of the story \citep{Genette1972, Genette1980}. For this reason, they lend themselves well to a transparent analysis of frame\hyp shifting.
%Moreover, the division of most of the narratives into chapters allows for a seemingly straightforward analysis of whether an adverbial may be forwards-oriented or must connect to the previous discourse. 


\section{Frame-setting}\label{sec:bloom:4}
As has been noted previously, PV \textit{so} typically follows conditional clauses, especially V1-conditionals \citep[96]{Meklenborg2020}. This is exemplified in (\ref{ex:rep1a}) below.

\begin{exe}
\ex \label{ex:rep1a} \emph{Context:} Ponthus and the children are eating hastily, because they fear they will soon be brought to death. Ponthus declares that that lies in the hands of God.\\
\gll Ist is sin wille \textbf{so} sterben wir.\\
is it his wish \textsc{so} die we\\
\glt `If it is his wish, we will die.' (Pontus, 4rb)
\end{exe}

In conditional contexts like in (\ref{ex:rep1a}), the content of the precursor affects the applicability of the proposition of the host clause. In other words, the event expressed in the host, in this case `we will die', is only true if the adverbial clause also presents a true statement, i.e., if `it (= `that we will die') is his wish' is true. In the implied alternative realm, where it is not his wish, or something else is his wish, it is also not the case that `we will die'. The applicability of the content of the host is thus explicitly delimited, and this is a neat instance of frame-setting in the narrow sense. The wider context furthermore indicates that, prior to Ponthus uttering this sentence, the children had the impression that they were going to die anyway, i.e., independent of a specific condition. The use of the conditional clause thus explicitly delimits the realm in which it is true, thereby shifting the realm in which the statement holds.

Preposed temporal and causal adverbial clauses also occur with this narrow definition of frame-setting, as in  \REF{ex:4a} and \REF{ex:4b} below.


\begin{exe}
\ex \label{ex:4}
 \begin{xlist}
\ex \label{ex:4a} \emph{Context:} Someone has been ordered to come to court and understands that he will have to pay a fine. He complains about this and is advised the following:\\
\gll wann er mit im f\"ur das gericht keme, \textbf{so} solt er kein ander antwort geben, (...) dann das einig wort ``blee".\\
when he with him for the court come.\textsc{sbjv} \textsc{so} shall he no other answer give (...) than the single word blee\\
\glt `When he will bring him in front of the court, he should answer nothing (...) but the single word ``blee".' (Rollwagen, 60)

\ex \label{ex:4b}  \emph{Context:} There is a helmet that can test whether one has six virtues. If one lacks one virtue and puts the helmet on, the gemstone that signifies that virtue will disappear. Joraffin suspects that Rial might not have all the virtues needed. Rial asks what the six virtues are. Joraffin answers him:\\
\gll seid yr edle wird s\olduo{}chent \textbf{so} will ich sy ewch auch hie ze wissen th\olduo{}n der erst stein rubin bedeitet daz er fest sol sein. vnd tugent vol. (...)\\
since you gem's worth search \textsc{so} want I them you also here to know do the first stone ruby signifies that he firm will be and virtue full\\
\glt {`}{``}Since you seek the significance of the gems, I want to let you know them here: The first stone, ruby, signifies that he will be firm and full of virtue (...)''.' (Wilhelm, 224/26v)
\end{xlist}
\end{exe}


\noindent In \REF{ex:4a}, the expression in the host clause only holds when `he will bring him in front of the court'. At other times and under different circumstances `he' should not only answer ``blee''. One may suggest that this example contains a conditional clause instead. However, the context indicates that the court-appearance is a given, and thus, it is a matter of `when', not `if'. The advice is restricted to this time; at other times the addressee of course can answer with other words.

Similarly, in \REF{ex:4b}, the speaker, Joraffin, only wants to let Rial know about the meaning of each gem, because Rial is looking for this. Since it is the reason for Joraffin's willingness, in realms in which it is not true, Joraffin would not want to share the gems' significance with Rial.

In this way, the examples support the fact that \textit{so} occurs with frame-setters in the strict sense: The proposition of the host is confined to the scenario expressed by the adverbial clause that precedes it. Since there is an explicit frame-setter in these examples, it is expected, or perhaps even necessary, that there is a shift in frame-setting, because with frame-setting continuation, the frame-setting is left implicit \citep{Krifka2008}. This is clearly contextually supported for \REF{ex:4a}; there is a shift in temporal setting, from speaker-here-and-now, to a point in the future. For \REF{ex:4b}, the shift is a bit more complex, since the realm is in fact opened by Rial requesting this information (\textit{Do sprach Rial Herr sagent mir bald w\oldoe{}lcherley mugent die sechs tugenden sein} `Rial said then ``Lord, tell me quickly whatever may those six virtues be?'''). The shift happens within Joraffin: As he fears that Rial doesn't have the needed qualities, he initially advises him to just let the helmet be, so that his honor stays intact. Rial decides that he is interested despite this, which then shifts Joraffin's perspective on things.

\textit{So} does not necessarily follow adverbial clauses but may also occupy the preverbal position on its own. In those constellations, it is not considered as a resumptive -- because there is nothing to be resumed -- but instead is typically viewed as a demonstrative adverb. Interestingly, even there it regularly occurs in contexts in which frame-setting is being negotiated. These examples are not part of the core data set, but they are found in the texts and are worth a specific mention. In these narrative texts, PV \textit{so} is attested in contexts where the interaction between two characters results in the limitation of the applicability of the proposition in the host. Consider \REF{ex:7} as an example.

\begin{exe}
\ex \label{ex:7} 
\gll Vnd sprach Agripina begerest du hye in disem closter dein wonung zu haben\\
and said Agripina desire you here in this monastery your living.space to have\\
\glt `And said, ``Agripina, do you desire to live here in this monastery?'''\\
\gll sy sprach gar dem\oldue{}tiklich \textbf{ja} gnedige fraw \oldae{}bti\ss{}in.\\
she said quite humbly yes mercifull miss abbess\\
\glt `She spoke quite humbly,  ``Yes, merciful miss abbess."'\\
\gll Sy sprach \textbf{so} würst du mir gehorsam sein (...)\\
she spoke \textsc{so} will you me obedient be (...)\\
\glt `She said, ``Then you should obey me (...).''' (Fortunatus, 17/555)\footnote{There are a couple of reasons why a reading like `She said thus ...' is unlikely. In general,  \textit{so} is rarely used cataphorically in the ENHG Roko.UP and is instead, unlike the English \textit{so}, backward-oriented. Moreover, I checked Fortunatus, Tristrant and Melusine, and it would mean that these texts would contain no V2-clauses following such an inquit. V2 is at this time already the most dominant word order for declarative main clauses and often follows other inquit-formulae. Finally, the use of \textit{so} in these contexts systematically occurs at the beginning of consequences, which is fully in line with its typical use following conditionals in which it likewise introduces apodoses.}
\end{exe}

One of the characters, the abbess, asks Agripina whether she wants to live in the monastery. Agripina's affirmation of this question is taken as the frame-setting for the host: Only in a realm in which Agripina desires to live in the monastery is it true that she has to obey the abbess. In the implied alternative realm where Agripina would have reacted negatively ti the question, she also would not have to obey the abbess.

In the context of narrative texts, we are of course not in actuality dealing with two different people working together to construct a frame-setting, but one writer presenting the situation from two different perspectives. As such, there is no true negotiation involved but an imitation thereof. Yet, such segments in which two characters interact are reflective of interaction between two individuals, and thus we may here speak of negotiation. It serves an additional goal to engage the reader and structure the discourse in a dynamic manner. Examples like \REF{ex:7}  indicate that frame-setting is not necessarily dependent on the presence of a preposed adverbial but can be associated with the element before \textit{so} regardless of its syntactic status, or, following \citet{BloomJouGe}, with a slot within a construction -- in the construction grammar sense (i.e., a conventionalized form-function pair) -- in which \textit{so} is lexically  specified (e.g., of the type [[X]\textsubscript{frame-setter} \textit{so} ... ]).

Other support for this analysis is that there are also other cases in which we find the frame-setting function with non-adverbials, if they combine with and precede \textit{so}. In such cases, they may even be positioned such that they cannot be considered to be a syntactic component of the `host clause'. This is illustrated in \REF{ex:8}.

\begin{exe}
\ex \label{ex:8}
\gll Peter lieber sun du wayst das wir keyn anderenn sun haben (...) dann alleyn dich einigenn auch sunst keynenn erbenn dann dich \textbf{so} stehet auch alle vnser hoffnunng vnnd trost zu dir.\\
Peter dear son you know that we no other son have (...) than only you alone also else no heir than you \textsc{so} stands also all our hope and consolation to you \\
\glt `Peter, dear son, you know that we have no other son than just you, also no other heir than you. Thus, all our hope and consolation is with you.' (Magelone, 596--597)
\end{exe}

The restricting information in \REF{ex:8} is not expressed by an adverbial but by a complement clause that is a constituent of a previous sentence. The state that is denoted by this complement clause frames the proposition expressed by the following clause: `all our hope and consolation is with you' is only true in the realm in which Peter is their sole heir. In the implied alternative realm, in which there are other heirs, not all hope would be on Peter. Thus, a non-adverbial constituent that serves another function is here repurposed as a frame-setter limiting the applicability of the following proposition. While the element that activates the realm is explicit in the discourse, it initially serves a different function and is not part of the same sentence. As such, what can be the precursor in the \textit{so}-construction may have to be expanded, as it seems to be not limited to linguistic units that have the form of a prototypical frame-setter (i.e., adverbial clause, prepositional phrase, or adverb).

Above we have seen three contexts in which the element that \textit{so} follows can be construed as a frame-setting element. First, PV \textit{so} occurs with unequivocal frame-setters \REF{ex:4}; second, \textit{so} is used in contexts in which frame-setting is negotiated \REF{ex:7}; and third, \textit{so} can be used to repurpose a constituent that was used with a different function as a frame-setter \REF{ex:8}. In these contexts, the applicability of the main predication is conditioned or restricted by a previous element. This constituent, the frame-setter, is prototypically adjacent and takes the form of an adverbial clause, but this is not necessarily the case.  The last  two uses of PV \textit{so} indicate that \textit{so} itself may be the element that signals that there is a shift in the activated realm or flags a construction with a frame-setter in the first slot.
While this nicely supports an analysis of PV \textit{so} as an element signaling a shift in frame-setting, it cannot be the whole story, as we will see in what follows.

Even when PV \textit{so} follows temporal adverbial clauses -- frame-setters par excellence -- the precursor does not necessarily need to impact the applicability of the following proposition. Consider as an example \REF{ex:10}.

 \begin{exe}
\ex \label{ex:10}% Context: They found a dead nobleman and didn't know what to do out of fear. A new chapter starts -- titled 
% \textit{Wie der bo:eßwicht Andrean ainen edelman ermort und yn in ain prifet wurff / und darvon kam} `How the malevolent Andrean killed a nobleman and throws in him a toilet and ran away'.
\gll VNd als sy also st\oldoe{}nden \textbf{so} kommpt der schalck geloffen\\
and as they also stood \textsc{so} comes the rascal walking\\
\glt `And when they stood like that, the rascal comes walking.' (Fortunatus, 30/415)
\end{exe}

The two events expressed in \REF{ex:10} coincide temporally, but whether the proposition of the host holds does not depend on whether the proposition of the precursor holds. Moreover, there is no implied alternative realm in which there is a prediction about whether `the rascal comes walking'. That is, the rascal comes walking regardless of whether `they' stood there like that. Instead, the relation between the two events is a matter of non-causal, incidental temporal sequentiality. In this regard, the initial temporal clause is not a frame-setter in the strict sense. The event expressed by the precursor is presented as being relevant to the second and technically presents the time in which the second event is claimed to occur. Nevertheless, there is a sense of frame-setting in a wider sense, in that the precursor grounds the following expression temporally -- not delimiting its application but providing an anchor in the previous discourse to which the following expression can be related.

This is more broadly applicable to PV \textit{so} following \textit{als-} and \textit{so-}clauses.\footnote{The conjunctions \textit{als} and \textit{so} are historically related \citep[102--154]{Vandenberghe2006}.} For Dutch, these patterns in which clauses with \textit{als} and \textit{so} as a conjunction combine with \textit{so} in preverbal position have been treated as an independent construction, separate from other configurations with adverbial clauses by \citet{Vandenberghe2006}. She refers to this as \textit{kopie\hyp correlatieven} (`copy\hyp correlatives') and \textit{expressieve correlatieven} (`expressive correlatives'), in which the similarity between the two expressed propositions is highlighted. Such comparative correlatives are also found in ENHG, this is illustrated by the examples in \REF{ex:11}.

\begin{exe}
\ex \label{ex:11}
\begin{xlist} 
 \ex \label{ex:11a} \emph{Context:} A lady has given Fortunatus a wallet from which he can get an infinite amount of money. In return, she requests that Fortunatus take one day off each year in order to find a man who has a daughter but is too poor to pay her dowry. Fortunatus should give them clothes and money. \\
 \gll  \textbf{als} du heüt erfreüwet bist worden von mir
 \textbf{so} erfrew du alle iar ain arme iungfraw\\
 \textsc{conj} you today delighted am became from me \textsc{so} delight you every year a poor lady\\
 \glt `And as you have been delighted by me today, you will delight a poor lady every year.' (Fortunatus, 47/431)

\ex \label{ex:11b}
\gll \textbf{So} du nun schuldig bist. 
\textbf{so} bist du auch schuldig vnd gepunden mir widerumb z\olduo{} helffen au\ss{} meiner so grossen nott. \\
\textsc{conj} you now guilty are \textsc{so} are you also guilty and bound me again to help out my so big need \\
\glt `As you are now guilty, you are also guilty and bound to help me out of my big need.' (Tristrant, 54)
	\end{xlist}
\end{exe}

In \REF{ex:11a}, the speaker does not restrict the applicability of the main predication -- whether Fortunatus delights a poor lady depends on that he should give them clothes and money, not that the speaker has pleased him today. Instead, both propositions are asserted within the same realm.  Again, there is a sense of frame-setting in a wider sense; the precursor grounds the following expression by highlighting the similarities to an experience familiar to Fortunatus. % the speaker is urging Fortunatus to do something in the future -- relating it to familiar, pleasant experiences, tool to convince him to actually do it.
A similar example is presented in \REF{ex:11b}, where the precursor introduced by \textit{so} presents the standard of comparison and the host presents the item that is compared to this. 
%The difference between these structures lies in the absence or presence of prediction \citep{Dancygier2005}. 

In a different context, whenever PV \textit{so} occurs with a concessive clause, the precursor cannot function as a frame-setter in the narrow sense either. With concession, the proposition of the host-clause holds despite something else. As such, there is no restricted applicability. Moreover, just like with the comparative correlatives in \REF{ex:11a} and \REF{ex:11b}, both propositions are claimed to hold true and are asserted. Consider the two examples \REF{ex:conc}.

\begin{exe}
\ex \label{ex:conc}
\begin{xlist}
\ex \label{ex:cV2} \emph{Context:}  The narrator is describing an animal called a \textit{forthaspinacht}, which must be feared by all animals, because it has a very sharp and sturdy hide so that it cannot be harmed by any hand weapons.\\
\gll  \textbf{Doch} \textbf{so} th\olduo{}t es niemant nich{\ss}.\\
still so does it no.one nothing\\
\glt `Yet, it does not harm a fly.' (Wilhelm, 261/56v)

\ex \label{ex:cV1} 
\gll wie er jung sy \textbf{so} ist er doch werstendlich wyse mennlich vnd von gro\ss{}em gelucke\\
how he young be \textsc{so} is he still responsible wise manly and of great fortune\\
\glt `How young he is, he is still responsible, wise, manly and of great fortune.' (Pontus, 33va)
\end{xlist}
\end{exe}

\noindent In the first example, (\ref{ex:cV2}), \textit{doch} does not function as a frame-setter, nor does the preceding main clause or any other preceding constituent: The proposition expressed by \textit{es thut niemand nich\ss{}} `it doesn't harm a fly' holds true in any realm. What is expressed here is that this proposition even holds true despite the expectation that has been set by the \textit{forthaspinacht} being feared and having a hard and pointy skin. 
In the second example, \REF{ex:cV1}, the applicability of the main predication is likewise not contingent on the first: That he is responsible, wise, manly, and lucky does not depend on how young he is; on the contrary, it is the case despite the fact that he is young. Again, the proposition holds true despite an implication set by the preceding text. 
There is however one crucial difference between \REF{ex:cV2} and \REF{ex:cV1}: While the expectation despite which the proposition holds is set by the preceding adverbial in the second example, in the first example, the expectation is set by a larger discourse segment, namely the entire preceding context. The initial adverbial \textit{doch} specifies more precisely how the following clause connects to this larger segment. It is thus a connective instead of a frame-setting element.

When the link to the previous discourse is one of addition, as is the case in \REF{ex:add}, the initial adverbial cannot be construed as a frame-setter in the narrow sense either. Unlike with concessives, there is no frame-shift in any sense. This additive function is only attested when PV \textit{so} combines with a non-clausal adverbial and does not combine with constructions with adverbial clauses.

\begin{exe}
\ex \label{ex:add}
\begin{xlist}
\ex \label{ex:adda}
\emph{Context:} Er ordinert sin schiff in das hoge mere vnd wolt nit das sie sin gewar wurden Dar vmb det er das \\
`He commanded his ship to the high sea and did not want that they would become aware of them. That is why he did that.' \\
\gll \textbf{Vort} \textbf{so} nam er vl schyff vnd fore ans lant\\
further \textsc{so} took he 45 ship and rode to.the land\\
\glt `In addition, he took 45 ships and rode to land.' (Pontus, 85va)

\ex \label{ex:addb}
\emph{Context:} dem küng wz aber s\oldoe{}llichs nit vermeint. das er sein tochter seinem trucks\oldae{}ssen solt geben. \\
`The king did not think any such thing, that he should give his daughter to his bailiff.'\\
\gll \textbf{Auch} \textbf{so} wisset er vormalen s\oldoe{}llicher manheyt von jm nicht.\\
also \textsc{so} knew he before such manliness of him not\\
\glt `Also, he had not known him to be so brave.' (Tristrant, 29) 
\end{xlist}
\end{exe}

\noindent In \REF{ex:adda}, the host-clause is simply asserted without any conditions or restrictions posed on the realm in which it holds. As such, \textit{vort} does not function as a frame-shifting element. Instead, as was the case with \textit{doch} in \REF{ex:cV2}, it connects the following predication to what has been said before. Differently, it is not an expectation set in the preceding discourse that the host-clause relates to, but the more general state of affairs that has been established. The host-clause further builds upon this. The realm thus explicitly remains the same as in the previous context.

The same is true for the host in \REF{ex:addb} where \textit{auch} is a connective element, signaling that the following predication builds upon the realm that is established before. The context here is that the king has promised to let his daughter marry the person who slays the serpent. The king's bailiff hears that the serpent has been slain, and he goes to claim the king's daughter by pretending that he was the one who completed this quest. For the reader, this is part of the common ground at this stage. The two sentences in \REF{ex:addb} both give the reader hope that the king sees through this deceit and does not approve the marriage. Thus \textit{auch so} provides an additive connection at discourse-level not at story-level.

In this section it becomes clear that \textsc{advl} + \textit{so} typically correlates with the setting of a specific frame. The frame-setting happens both in the narrow sense, with limited applicability of the host due to the predictive relation between the two propositions \citep{Dancygier2005}, or in a broader sense, in which the proposition in the host is more loosely related to the background content provided by the precursor. It should be noted that this frame-setting function does not need to be realized by a preposed adverbial but can attach to other elements as well. Finally, PV \textit{so} also follows additive adverbs, where the realm is explicitly continuous. The frame-setting remains expressly the same and such patterns do not have to involve any kind of shift.

The data suggest that PV \textit{so} dominantly combines with adverbials that provide a local shift, being frame-setters, albeit not in a strict sense. They are thereby forward-oriented, as they pertain to their host.  We also saw that PV \textit{so} follows adverbs that mark additive relations, in which the elements in front of PV \textit{so} are tied more strongly to the preceding discourse. As such, \textsc{advl} + \textit{so} with \textit{so} may be both discourse disruptive -- activating a different realm for the following discourse -- or cohesive -- explicitly connecting what follows to what has gone before.


\section{Cohesion and/or disruption?}\label{sec:bloom:5}
In the previous section, it was indicated that PV \textit{so}  occurs with elements that are both discourse connective and shifting. This section further evaluates  whether patterns with PV \textit{so} are more strongly associated with cohesion or disruption, or with both to a similar degree. For this purpose, I first look at the position at which \textsc{advl} + \textit{so} occurs in the narration: On the one hand, \textsc{advl} + \textit{so} can be found at the beginning of chapters and at the beginning of turns of speaking that are taken by the characters within the story. Such contexts are typically associated with a disruption of the main flow of the narration. On the other hand, it often supports larger sequences of discourse, e.g., contributes to the main argument, or progresses a narrative sequence. In these contexts, disruption is generally absent and instead a cohesive function is expected. Thereafter, the section considers the type of subjects, the length of the subjects, and the referential continuity in the host sentence in terms of cohesion and disruption. Throughout this section, a distinction between clausal and non-clausal adverbials is made as the latter may be associated with cohesion to a higher degree than the former.

\subsection{In the narrative}
This section considers three different contexts within the narrative in which \textsc{advl} + \textit{so} occurs: At the beginning of chapters, at the beginning of turns, and within sequences. The first two are associated with discourse disruption, as the beginning of chapters and turns starts a new sequence. Differently, the use of \textsc{advl} + \textit{so} within or at the end of a sequence is considered as a cohesive context. 

\subsubsection{Chapter initial}
The first distributional property of elements associated with discourse disruption is a chapter initial position. A new beginning of a chapter, typically coincides with a shift in topic and/or location or time in which the reported events occur. 
Earlier research has indicated that specific elements are recruited for this function and consequently often occur at the beginning of new episodes \citep[42--44]{Brinton1996}. 
If \textsc{advl} + \textit{so} is found frequently at the beginning of chapters, it would be an indication that this is a function that may be specifically associated with the construction.  Therefore, it is predicted that if \textsc{advl} + \textit{so} does occur at the beginning of chapters, it is at the very least compatible with a disruption of the discourse.

Only those texts that use explicit chapter titles are included here. The main reason for this is that these headers provide an optical disruption in the text and function as prospective segments, summarizing that what follows. Thereby, they deliberately break up the presentation of the story and disrupt the cohesive flow. Data from the Pontus, Huge Scheppel, and Wigalois are not considered here, as they lack chapter headings in the reported editions.

The data indicate that it is very rare for \textsc{advl} + \textit{so} to occur at the beginning of chapters: Only 1.08\% of the clausal adverbials + PV \textit{so} are found here (once in  Fortunatus, Melusine, and Rollwagen, and twice in Wilhelm) and none of the adverb + PV \textit{so} sequences are used as the first segment of a chapter. In the rare case that it occurs, the clauses have a temporal function: Four of the five adverbial clauses have an unambiguous temporal meaning. This is illustrated in \REF{ex:chini1}. The \textit{dieweil}-clause + PV \textit{so} in Rollwagen is either temporal, causal, or both \REF{ex:chini2}.

 \begin{exe}
\ex \label{ex:chini} 
\begin{xlist}
\ex \label{ex:chini1} \textbf{Wie der marschalck Wegrich den knaben mit jm f\oldue{}ret gen zwingen vnd in dem künig schanckt.}\\
`How the marshal Wegrich leads the lad along with him to Zwingen and gives him to the king.'\\
\gll Do jm nun der marschalck wegrich z\olduo{} jm in sein schif nam/ vnnd z\olduo{} dem gestadt des m\oldoe{}res mit jm kommen was/ und yecz hinder sich sicht. \textbf{so} sicht er das (...)\\
when him now the marschal Wegrich to him in his ship took and to the place of.the sea with him came was and now behind \textsc{refl} sees \textsc{so} sees he that (...) \\
\glt When the marshal Wegrich took him to his ship and had come with him to the place of the sea and now looks behind, he sees that (...).' (Wilhelm, 201/9v)
\ex \label{ex:chini2} \textbf{Wie ein g\olduo{}t frumm mann am Kochersperg einem g\olduo{}ten einfaltigen ein walfart verdinget z\olduo{} Sant Veiten z\olduo{} wallen.}\\
`How a good pious man at the Kochersberg sells a good simple man a pilgrimage to travel to Sant Veit.'\\

\gll Dieweil wir yetzund auch auff einer fart oder rei\ss{} sind, \textbf{so} manet mich gleych ein g\olduo{}ter schwanck\\
while we now.and also on a trip or travel are \textsc{so} reminded me immediately a good tale \\
\glt `While we are now on a journey or a trip, I am reminded of a good tale.' (Rollwagen, 11) 
\end{xlist}
\end{exe}

Thus, there is only evidence for temporal adverbial clauses + PV \textit{so} to occur at the beginning of chapters. Other adverbial functions as well as adverb + \textit{so} are unattested. How does this relate to other elements that are found in this context? For the Fortunatus (F), Melusine (M), Wilhelm (O), and  Rollwagen (R), I have, in addition to the data set described in \sectref{sec:bloom:3}, collected the types of elements that are found in chapter initial position. The results are presented in Table \ref{tab-chini}.

\begin{table}
\caption{Chapter initial element}\label{tab-chini}
\begin{tabular}{lrrrrr}
\lsptoprule
&  {R} & {M} & {F} & {O} & {total} \\
\midrule
    character & 63 &37 & 17 & 19 & 136\\
      \quad whereof definite & 5 & 24 & 16 & 16 & 61\\
    temporal setting  & 7 & 27 & 26  &  36  & 96\\
    other & 41 & 4 & 4& 3& 52\\
      \quad whereof place & 39 & 0 & 0 & 0 & 39\\\midrule
    total & 111& 68& 47 & 58 & 284\\
\lspbottomrule
\end{tabular} 
\end{table}


In general, the majority of the chapter initial elements are references to characters of the story (48\%), as exemplified in \REF{ex:charchini}. These are primarily expressed by proper names or definite noun phrases in each text except R. R is a collection of short narrations each with different characters, while M, F, and O present one main story and thus contain multiple recurrent characters. For this reason, definite subjects are more likely to occur at the beginning of chapters in M, F, and O than in R.

\begin{exe}
\ex \label{ex:charchini}
\gll \textbf{REÿmond} kam in diser grossen klag zů einem brunnen genant der turst brunn\\
Reymond came in this great mourning to a well called the thirst well\\
\glt `Reymond came, in this great mourning, to a well called the thirst well.' (Melsune, 22)

\ex \label{ex:tempchini} 
\gll \textbf{DA} \textbf{Agley} \textbf{den} \textbf{brief} \textbf{gela\ss{}} schreib sy einen anderen\\
when Agley the letter read wrote she an other\\
\glt `When Agley read the letter, she wrote another.' (Wilhelm, 248/45r)

\ex \label{ex:locchini} 
\gll \textbf{Z\olduo{}} \textbf{Bern} haben gewont zwen g\olduo{}t freünd mit nammen Mathias Apiarius der ein und Hans Ypocras der ander.\\
to Bern have lived two good friends with names Mathias Apiarius the one and Hans Ypocras the other.\\
\glt `Two good friends, one named Mathias Apiarius and the other Hans Ypocras, lived in Bern.' (Rollwagen, 24)
\end{exe}

The texts differ from each other. While references to characters are the main chapter starters in R and M  (37\% and 54\% respectively), in F and O, reference to a temporal setting is dominant (55\% and 62\% resp.), as illustrated in \REF{ex:tempchini}. This type of chapter initial element is frequent in M as well (40\%) but rarely attested in R (6\%), which instead more often has locative frame-setting elements, as in \REF{ex:locchini}. This can again be accounted for by M, F \& O being long running narratives in which a temporally progressing story is presented, whereas R's stories only progress temporally within a chapter. As these are tales that were often read during travels, the dominant strategy of the locative grounding of stories is furthermore fitting.

The uses of PV \textit{so} after temporal adverbial clauses at the beginning of chapters fall into a larger pattern of starting chapters by specifying the temporal setting of what follows. In this way, the realm is not merely specified for the following proposition, but it may set the stage for a segment of discourse that encompasses more events.

Despite the low frequency of adverbial clause + PV \textit{so} at the beginning of chapters -- both in comparison to other elements that occur there and to chapter internal use of the pattern -- the attested examples do illustrate that temporal adverbial clauses + PV \textit{so} \textit{can} be used here. As such, the sequence temporal adverbial clause + PV \textit{so} is likely forward-oriented and compatible with a disruption of the discourse. For other patterns, no chapter initial position is attested and hence, no such conclusions can be drawn.

\subsubsection{Turn initial}
The second diagnostic for a construction's discourse disruptive use is its use in turn initial position. In narrative prose, parts of the story are presented by letting the characters speak, either in monologue or dialogue. When \textsc{advl} + \textit{so} occurs at the beginning of a turn, it coincides with a change in voice, viz. `who speaks' \citep{Genette1980}. 

Present-Day German clause initial \textit{so} -- especially without a preceding adverbial -- is known to be an action-shift marker \citep{Burkhardt1987, Barske2010}. % Mazeland 1984, Brunner 1987, Ehlig 1987, Sandig 1987, Meier 2002
It marks either new discourse moves or new (non-verbal) actions, and as such it lends itself well as a turn-entry device  \citep[719f.]{Sacks1974}, that is, an element with which a discourse participant takes over the floor and starts speaking. Noteworthy is that with this use, time adverbials like \textit{dann} may accompany PV \textit{so} \citep[256]{Barske2010}. Turn-taking in narrative prose of course differs from turn-taking in actual conversation in that the former is not spontaneous behavior. Features of real conversation are stylized and represented with the audience in mind \citep[14--17]{Morgan2019}. A dialogue is merely a sketch of ``the remains of a conversation that never took place" \citep[147]{Morgan2019}. Consequently, elements that are used by speakers to negotiate who is speaking can be used to signal to readers transitions between characters that are speaking but are likely less frequently represented. If turn-taking is a use of \textit{adverb + so} in ENHG as well, it is nevertheless expected that we find this sequence regularly at the beginning of turns, especially when the reported speech is not mediated by other means.

In addition, the alternation between segments of reported speech and narration defines a narrative rhythm \citep[96--97]{Genette1972, Packard2008}. A shift from narration to direct reported speech is typically one from summary to isochrony (viz. a coincidence of the temporal progression of the narrated and narrating events), slowing down the narrative speed. Moreover, the perspective on the represented events changes: In the main narration, we are listening to the narrator speaking, but in segments of direct speech, we are listening to the voice of a character \citep{Jong2004}. This is another way in which the beginning of a turn may be associated with a form of  disruption. It should be noted that the voice of a character was not individualized and disconnected from the narrator in the same manner we see in modern novels  \citep[see also][178--196]{Louviot2016}, but this does not speak against a shift in who speaks.\footnote{Depending on the text, the form of the dialogue may strongly differ. For example, reported speech in Ponthus includes many long monologues with a lot of imagery, which underscore the virtues and eloquence of particular characters. In others, e.g., Tristrant, the speaker in reported speech changes more rapidly.}  For these reasons, if \textit{adverbial + so} does regularly occur at the beginning of turns, it may be associated with a narrative disruption in voice and pace. 

Example of turn initial adverbial clauses followed by PV \textit{so} are illustrated in \REF{ex:tinia}. In \REF{ex:tini1}, the segment of direct speech is explicitly introduced by a verbum dicendi, but the turn  in \REF{ex:nonmed} is not mediated. It is quite regularly the case that there is no indication at all whether it is the character who speaks, which would lead one to expect the presence of other signals of turn-transition.

\begin{exe}
\ex  \label{ex:tinia}
\begin{xlist}
\ex  \label{ex:tini1}
\gll dann sie sprach so mein vater solchs innen wurd \textbf{so} wurd er vns nachuolgen\\
then she spoke \textsc{conj} my father such in becomes \textsc{so} becomes he us follow\\
\glt `Then she said, ``If my father notices it, he will follow us."' (Magelone, 637)
\ex \label{ex:nonmed}
\gll Herr jch hei\ss{} Tristrant vnd bin küng marcksen schwester sun. \textbf{Seyd} jr her tristrant so hab jch vor malen dick von euch vernommen. vil grosser frümkeyt vnd manlicher gethat.\\
lord I am.called Tristrant and am king Marck's sister son \textbf{are} you lord Tristrant so have I before times such from you heard many great virtues and manly deeds\\
\glt {`}{``}Lord, my name is Tristrant and I am king Marck's nephew." ``If you are Lord Tristrant, I have heard a lot about you: Many great virtues and manly deeds."' (Tristrant, 119)
\end{xlist}
\end{exe}

Table \ref{tab-tini} presents the distribution of turn initial position of clausal and non-clausal precursors. The data show that adverbial clause + PV \textit{so} recurrently occurs at the beginning of turns: 17\% is found in this position. Non-clausal adverbials are used less frequently than expected as turn initial elements, while clausal adverbials occur more often than expected. This difference is significant with $p < 0.001$, $\chi^2 = 34$; the primary cells that are responsible for this is “clausal -- turn initial” ($\chi^2$-residual = 2.91) and “non-clausal -- turn initial” ($\chi^2$-res. = $-4.73$). In other words, only adverbial clause + \textit{so} is disruptive in this way.

\begin{table}
\caption{Turn initial element}\label{tab-tini}
\begin{tabular}{lcrrr}
    \lsptoprule
    & {perc.bar} & {turn initial} & {not turn initial} & {total}\\
    \midrule
    clausal &\DrawPercentageBar{0.17} &  111 & 549 & 660\\
    non-clausal &\DrawPercentageBar{0.02}&  4 & 235 & 239\\\midrule
    total &\DrawPercentageBar{0.13}& 115 & 784 & 899\\
    \lspbottomrule
\end{tabular} 
\end{table}

\noindent Even if one disregards those with a vocative or affirmative/negative particle preceding the adverbial clause in the turn (e.g., \textit{Frauwe/ja/nein}) 70 adverbial clauses and 4 non-clausal adverbials are turn initial. This difference is still significant ($p < 0.001$, $\chi^2 = 19$; $\chi^2$-residual = $2.23$ and $\chi^2$-residual = $-3.62$).

Contra the expectation one would have from a Present-Day German perspective, non-clausal precursors are not used as a tool to signal transitions between characters that are speaking: It occurs only a few times,  and even then, it is mostly clearly indicated that there is a change in who is speaking, exemplified in \REF{ex:tiniadv}.

\begin{exe}
\ex \label{ex:tiniadv}
\gll Do \textbf{sprach} Andrean \textbf{morgen} \textbf{zu} \textbf{mittag} so solt ir mit mir essen in Jeronimus roberti hau\ss{}\\
then spoke Andrean morgen to midday \textsc{so} shall you with me eat in Jeronimus Roberti house\\
\glt `Then Andrean said: ``Tomorrow afternoon you will eat with me in Jeronimus Roberti's house."' (Fortunatus, 29/413)
\end{exe}

What should be noted is that the majority of turn initial adverbial clauses followed by PV  \textit{so} express conditionality or causality; no prototypical example of a temporal clause are found in this position; even the conjunctions that in other contexts often introduce temporal adverbial clauses -- \textit{dieweil} and \textit{seid} --  receive here a causal reading, as exemplified in \REF{ex:22}.

\begin{exe}
\ex \label{ex:22}
\gll Mein edler herr vnd freünd dieweyl jm also ist  als jr anzaigt  so rath ich (...) \\
My noble lord and friend while him also is as you show \textsc{so} advise I (...)\\
\glt `My noble lord and friend, while it is to him as you show, I advise you (...).' (Magelone, 637)
\end{exe}

The turn initial position is not in the same way associated with a break in the discourse as a chapter initial position. Instead of a shift of temporal setting, it is more common to begin new turns opening a hypothetical realm or to re-frame previously understood information as a cause or reason for the character to want or do something. Only evidence for disruptive use of adverbial clause + \textit{so} has been provided in this section, as was the case with chapter initial use. For non-clausal adverbials, no disruptive use has thus far been identified.

\subsubsection{Within sequences}
In the previous sections, evidence for adverbial clause + \textit{so} in narrative positions that are associated with disruption was discussed. The absence of non-clausal adverbials + \textit{so} in these contexts may suggest that it is more strongly associated with cohesion. Besides in the sequence-initial positions, \textsc{advl} + \textit{so} is also attested within textual sequences.  \textsc{advl} + \textit{so} supports a broader communicative purpose when used in such context. As such, it is here thought to be associated with cohesion rather than disruption. I first consider the patterns with non-clausal adverbials, which are argued to occur even in absence of disruption. After this, clausal adverbials are discussed.

An example of non-clausal  \textsc{advl} + \textit{so} within a sequence is illustrated in \REF{ex:adda} with its additive function, which is repeated in \REF{ex:addaREP} below.

\begin{exe}
\ex \label{ex:addaREP}
\emph{Context:} Er ordinert sin schiff in das hoge mere vnd wolt nit das sie sin gewar wurden Dar vmb det er das \\
`He commanded his ship to the high sea and did not want that they would become aware of them. That is why he did that.' \\
\gll \textbf{Vort} \textbf{so} nam er vl schyff vnd fore ans lant\\
further \textsc{so} took he 45 ship and rode to.the land\\
\glt `In addition, he took 45 ships and rode to land.' (Pontus, 85va)
\end{exe}

\noindent In \REF{ex:addaREP}, the narrator is describing what Ponthus did. The narrator explains the motivations behind Ponthus' decision and afterwards returns to the description of Ponthus' actions, adding more information about the temporal unfolding of events.  Instead of introducing a new setting, the construction with \textit{so} explicitly links the following description to a previous one that still is being built upon. The narrative sequence is continued.

While in \REF{ex:addaREP}, the sequence is interrupted by the narrator commentary `that's why he did that', such disruptions are not necessary for non-clausal \textsc{advl} + \textit{so} to occur in otherwise cohesive contexts. This is exemplified in \REF{ex:coh2}. 

\begin{exe}
\ex \label{ex:coh2}
\gll allso stach man zw\olduo{} oder drey stund vnd \textbf{denn} \textbf{so} tantzet man  \textbf{denn} \textbf{so} a\ss{} man\\
thus jousted one two or three hours and then \textsc{so} danced one then \textsc{so} ate one\\
\glt `They jousted like that for two or three hours. And then they danced, then they ate.' (Fortunatus, 95/479)
\end{exe}

\textsc{Advl} + \textit{so} in \REF{ex:coh2} occurs without any disruptions. The function of the construction is not to return to a previously introduced topic but simply to connect the following proposition to the preceding sentence. As a whole, it is a fully cohesive, narrative sequence, in which the temporal progression of what happened at the event is presented. This example very strongly indicates that adverb + PV \textit{so} may occur without shifts and even occurs in a context in which exactly the same sequence is used in the preceding clause.

Besides in narrative sequences, non-clausal  \textsc{advl} + \textit{so} can be used  within argumentative sequences when the proposition simply supports the main argument. This is illustrated in \REF{ex:cohSUP}, in which a character gives a lengthy answer to the king.

\largerpage
\begin{exe}
\ex \label{ex:cohSUP} \emph{Context:} The king asks Ponthus what he should do about the war against all the heathen in the country and what Ponthus would advise him to do. Ponthus answers that one does not have to fear the heathens, even though there are so many of them, because God is on our side.\\
\gll Dieser krieg berurt den Cristenglauben vnd gett an all Cristenmentschen \textbf{Dar} \textbf{vmb} \textbf{so} hoff ich das alle Cristen die da von vernemen vns zu hulff kommen sollen \\
this war touches the Christian.faith and gets on all Christian.people there for \textsc{so} hope I that all Christians who there of hear us to help come shall\\
\glt `This war touches Christianity and concerns all Christians. Therefore I hope that all Christians who hear of this will come to help us.' (Pontus, 23ra)
\end{exe}

The main advice that Ponthus gives  to the king has yet to follow, namely \textit{Dar vmb ratt ich das ir schribent uwer brieffe (...)} `Therefore I advise that you write your letters (...).' The segment presented in \REF{ex:cohSUP} builds up to this, and the sentence with \textsc{advl} + \textit{so} plays a supporting role in the argumentation. There is no shift or disruption of the cohesive flow, and \textit{darumb} directly relates to the preceding sentence.

In the examples above, the adverbial position before PV \textit{so} is filled by a non-clausal adverbial. Even adverbial clauses with \textit{so} are attested as sentences continuing or concluding a sequence. Consider as an example \REF{ex:coh1}.

\begin{exe}
\ex \label{ex:coh1} \emph{Context:} Fortunatus is match-making. He asks the young woman how she likes the neighbour's son. She answers:\\
\gll ich wil nit w\oldoe{}llen. dann w\oldoe{}lchen mir mein vater und m\olduo{}ter gebent den wil ich haben \textbf{vnd} \textbf{solt} \textbf{ich} \textbf{on} \textbf{man} \textbf{sterben} \textbf{so} wil ich kainen selber nemen.\\
I want not want than which me my father and mother give that want I have and must I without man die \textsc{so} want I none self take\\
\glt `I won't want anyone, because whomever my father and mother give, him I want to have. And even if I have to die without a man, I do not want to take one myself.' (Fortunatus, 72/456)
\end{exe}

Clausal \textsc{advl} + \textit{so}'s function in the answer of the young woman in \REF{ex:coh1} is not to return to a previously introduced topic but to further strengthen her just stated position -- that she will not decide whether the neighbor's son would make a good husband for her. This reinforcement is appropriate here, since the young woman is in fact lying; she is very much in love with the boy, and her mother knows this. To hide her inappropriate feelings, the young woman feigns her conviction and overdoes it. The clause following PV \textit{so} continues to add to the same realm as the preceding discourse.

Unlike the previous variables, whether or not \textsc{advl} + \textit{so} is used within a sequence depends on the level of analysis. Consider for example \REF{ex:discl}.

\begin{exe}
\ex \label{ex:discl} 
\emph{Context:} Wigoleys arrives at the door of the castle, touches the golden ring and is let in. When he crosses himself and goes in, the door disapppears. In its place, there is an extremely small window.\\
\gll Er erschrack vnd gedacht in im selbs was will dises bedeüten \textbf{die} \textbf{weil} \textbf{er} \textbf{also} \textbf{in} \textbf{gedencken} \textbf{st\olduo{}nd} so kompt ein grosser donerschlag mit eim so lauten krach  als ob die burg z\olduo{} hauf fallen wolt.\\
 he startled and thought in him self what will this mean the time he thus in thought stood \textsc{so} comes a great thunderclap with a so loud crack as if the castle in heaps fall would \\
 \glt `He was startled and thought "What does this mean?" While he stood in thought, a great thunderclap came with such a loud crash as if the castle would fall apart.' (Wigalois, 73)
\end{exe}

On the one hand, the temporal simultaneity of Wigoleys thinking and the sound of the thunderclap is expressed as part of one narrative sequence in which there is unity of time and place. On the other hand, the discourse moves from a representation of Wigoleys thoughts -- `What does this mean?' -- to an event that happens independently. In other words, there is a perspective-shift. \REF{ex:discl} may simultaneously be analyzed as being internal to a narrative sequence and as shifting the perspective, thereby serving a different communicative purpose.

 In general,  \textsc{advl} + \textit{so} may occur within narrative sequences to simply temporally progress the narration, or in argumentative sequences to support a larger argument. Specfically, in structures in which a non-clausal adverbial precedes PV \textit{so}, a shift or disruption is not necessary. Such convincing contexts are not found for \textit{so} with clausal adverbials as shifts are present on a different level of analysis.


\subsection{Subjects and referential continuity}
This section considers the correlation between preposed adverbials and referential continuity, i.e., the repeated occurrence of expressions that refer to the same participant(s) \citep{Garnham1982, Givon1983}. Two aspects are reported on: First, the realization of the subject of the host clause is investigated. This is taken as a grammatical reflection of whether or not the participant is an accessible part of the previous discourse -- thereby facilitating cohesion -- or not. 

 Second, whether the subject of the host clause  is co-referential with the subject of the preceding sentence is discussed. The participant that is most crucially involved in the action sequence is most likely to be coded as the grammatical subject \citep{Givon1983a}. Furthermore, the subject of a sentence is typically topical \citep{Givon1983}. If the two subjects of the sentences are co-referential, it can be reasonably assumed that the most central participant is kept the same and thus that we are dealing with topic continuity, as in \REF{ex:coh1}. If not, there is a shift, see \REF{ex:discl} for an example.

\begin{figure}
 	\caption{Subject of the host sentence}
 	\label{fig:bloom:1}
 	\begin{tikzpicture}
 	  \begin{axis}[xbar,
        axis lines*=left,
        ytick=data,
        y dir=reverse,
        symbolic y coords = {pronoun,∅,definite NP,name,bare nominal,clausal,quantified,negated NP,indefinite NP,affix},
        xmin=0,
        nodes near coords,
        width=8cm,
        height=8cm
 	  ]
 	  \addplot [fill=black!80,draw=black] coordinates {
 	  (615,pronoun)
      (94,∅)
      (84,definite NP)
      (25,name)
      (16,bare nominal)
      (15,clausal)
      (14,quantified)
      (13,negated NP)
      (13,indefinite NP)
      (10,affix)
 	  };
 	  \end{axis}
 	\end{tikzpicture}
% %    	\includegraphics[width=0.8\linewidth]{graphParS.PNG}
 \end{figure}
 
In \figref{fig:bloom:1}, the frequency of the different realizations of subjects of the host sentence are presented. By far the majority of the subjects of the hosts are pronominal and thus seem to provide given information, connecting the current expression to a participant that has been mentioned previously. This is true both for sentences with clausal preposed adverbials and sentences with non-clausal preposed adverbials. There are no striking differences between the two regarding the type of subject they co-occur with.  Pronouns, zero-subjects and definite noun phrases signal accessible referents, either highly accessible ones (pronouns and zero-subjects), or referents lower on the accessibility scale (definite noun phrases) \citep{Ariel2001}. New referents, as seen in example \REF{ex:discl}, occur relatively seldom following \textsc{advl} + \textit{so}, which suggests that referential continuity and not referential shifts accompany the construction. 

The second way in which referential continuity is operationalized is by testing whether the subject of the previous main clause is the same or different to the subject of the host sentence. This is annotated for all clauses, except for the six that occur in chapter initial position. The results are visualized in Table \ref{tab-scon}.
 
\begin{table}
\caption{Subject continuation}\label{tab-scon}
\begin{tabular}{lcrrr}
\lsptoprule
& {perc.bar} & {same subject} & {different subject} & {total}\\
\midrule
clausal &\DrawPercentageBar{0.39} &  253  & 401 & 654\\
non-clausal &\DrawPercentageBar{0.33}&  79 & 160& 239\\
\midrule
total &\DrawPercentageBar{0.37}& 332 & 561 & 893\\
\lspbottomrule
\end{tabular} 
\end{table}

On average, 37\% of the hosts have a subject that is co-referential with the previous main clause. For hosts that are preceded by a non-clausal adverbial, subject-continuity is found in 39\% of the sentences. For constructions that have an adverbial clause as precursor, 33\% show subject-continuity. This distinction between the two groups is not significant ($p = 0.14$). Overall, thus a shift in subject, as for example seen in \REF{ex:cohSUP} and \REF{ex:discl}, is more common than subject-continuity, \REF{ex:coh2} and \REF{ex:coh1}. However, with 1/3 of the subjects being the same as in the previous main clause, it is by no means rare for \textsc{advl} + \textit{so} to occur in contexts of referential continuity.

%interrim summary
In the previous sections, the different signs of cohesion and/or disruption were discussed. For turn initial position, it was noted that there is a significant difference between clausal and non-clausal adverbials, with clausal adverbials occurring more than expected in a turn initial position. In positions within a sequence, non-clausal adverbial clauses seem to be more strongly associated with cohesion than disruption;  they are even able to be used without any type of shift. For clausal adverbial clauses both continuation and discontinuation are present, depending on the level of analysis. For the type of subject in the host and subject continuation, however, no quantitative distinction was found between clausal and non-clausal preposed adverbials.
The question rises whether there is a systematic difference between clausal and non-clausal adverbials regarding their use as a discourse cohesive and disruptive tool. This is evaluated in the following section.

\section{Types of adverbials}\label{sec:bloom:6}
The signs of discourse disruption discussed in the previous section are the discontinuation  and newness (approximated by indefiniteness marking) of the referent expressed by the subject as well as the occurrence in chapter and turn initial position. Of these, only turn initial position was statistically differentiating between clausal and non-clausal adverbials. 
Signs of discourse cohesion are the use of short, pronominal subjects or zero subjects and the occurrence within a sequence. Whether something is part of a sequence is, however, problematic to systematically annotate, as this depends on the discourse level one looks at. Therefore, to evaluate the differences between clausal and non-clausal adverbials in the \textsc{advl} + \textit{so}-pattern, the use of the construction with \textit{vnd} `and' is taken as a signal for occurrence in a cohesive context. 

\begin{exe} 
\ex  \label{ex:und}
\begin{xlist}
  \ex \label{ex:und1}
  \gll Aber du pist selbs schuldig daran. \textbf{vnd} von deiner grossen vnwarheit vnd vntrew wegen \textbf{so} wirdest du dein herczliebes liep verlieren\\
but you are selfs guilty there.on and from your  great dishonesty and untrue ways \textsc{so} will you your beloved love lose\\
  \glt `But you are yourself guilty of that. And because of your great dishonesty and unfaithfulness, you will lose your beloved love.' (Melusine, 120)

\ex \label{ex:und2}
\gll Der g\olduo{}t gesell was geschwindt auff den f\oldue{}ssen, lieff dem beinhau\ss{} z\olduo{}, \textbf{und} als er jetzund die stiegen hinnabkumpt, \textbf{so} h\"o{}rt er den dieb nü\ss{} krachen.\\
the good companion was quick on the feet walked the ossurary to and when he now the stairs descends \textsc{so} hears he the thief nut crack\\
\glt `The good companion was quick on his feet, walked towards the ossuary, and when he then descends the stairs, he hears the thief cracking nuts.' (Rollwagen, 56)
\end{xlist}
\end{exe}

In both sentences in \REF{ex:und}, \textit{vnd} explicitly  links the host to the preceding discourse. The examples show that this is possible both for \textsc{advl} + \textit{so}-patterns with a clausal and non-clausal adverbial. However, the data show a significant difference between them. This is reflected in Table \ref{tab-coord}.

\begin{table}
\caption{Coordination}\label{tab-coord}
\begin{tabular}{lcrrr}
\lsptoprule
& {perc.bar} & {with \textit{vnd}} & {no \textit{vnd}} & {total}\\
\midrule
clausal &\DrawPercentageBar{0.26} &  172  & 488 & 660\\
non-clausal &\DrawPercentageBar{0.14}&  33 & 206& 239\\
\midrule
total &\DrawPercentageBar{0.23}& 205 & 694 & 899\\
\lspbottomrule
\end{tabular} 
\end{table}
 
This distinction is significant with  $p < 0.001$.  Specifically, non-clausal adverbials have a lower than expected frequency in contexts with coordinating \textit{vnd} ($\chi^2$-residual = $-2.91$). This seems slightly conflicting in light of the previous results: A turn initial position is viewed as a signal for discourse disruption, whereas co-occurrence with \textit{vnd} `and' is an explicitly cohesive tool. The results then seem to suggest that \textsc{advl} + \textit{so} with adverbial clauses is both more and less disruptive than \textsc{advl} + \textit{so} with non-clausal adverbials. How is this possible?
 
What seems to be the case is that many of the non-clausal adverbials in \textsc{advl} + \textit{so} serve a similar function as  \textit{vnd}, since they themselves are connectives \citep{BloomJouGe}. This is primarily the case for adverbs, while prepositional phrases, such as the one in \REF{ex:und}, combine at least as easily with \textit{vnd} as adverbial clauses, as can be read off from Table \ref{tab-coord2}. Since these adverbs are typically connectives themselves, would the combination with PV \textit{so} not constitute a construction in which the clause connection is redundantly marked? Probably, yes. \citet{FabriciusHansen2005} concludes that connectives are redundant if the intended interpretation can be assigned to the sentence with and without the connective being present.

\begin{table}
\caption{Coordination}\label{tab-coord2}
\begin{tabular}{lcrrr}
\lsptoprule
& {perc.bar} & {with \textit{vnd}} & {no \textit{vnd}} & {total}\\
\midrule
adverb &\DrawPercentageBar{0.10} &  19  & 175 & 194\\
adverbial clause &\DrawPercentageBar{0.26}&  169 & 474& 643\\
PP & \DrawPercentageBar{0.41}&  11 & 16& 27\\
other & \DrawPercentageBar{0.16}&  6 & 29& 35\\
\midrule
total &\DrawPercentageBar{0.23}& 205 & 694 & 899\\
\lspbottomrule
\end{tabular} 
\end{table}

In contexts like the one in \REF{ex:undadvso}, one may even suggest \textit{vnd, darumb,} and PV \textit{so} are all redundant. Mere juxtaposition of \textit{Wi\ss{} vnd hab keinen zweyfel daran nit} should not change the interpretation, because the clauses are connected by \textit{daran} and the effect-cause relation between the clauses can be inferred: It quite naturally follows that the addressee should not doubt his answer, because it was a good one.
 
\begin{exe}
\ex \label{ex:undadvso}
\gll  Aber du gabest jm gar ein g\olduo{}tte antwurt Vnd darumb so wi\ss{} vnd hab keinen zweyfel daran nit\\
But you gave him quite a good answer and therefore \textsc{so} know and have no doubt there.on not\\
\glt `But you gave him quite a good answer. And therefore, thus, know and don't doubt it.' (Melusine, 41)
\end{exe}

\begin{sloppypar}
This potential redundancy is not problematic. As \citet{Zehentner2021} have argued, syntagmatic redundancy is widespread and highly frequent, although typically only with the double marking of the same thing. Trine expression, as in \REF{ex:undadvso}, is considerably more rare but not unheard of. If one considers it as redundant marking, it should come as no surprise that the conjunction \textit{vnd} is considerably less frequent in contexts in which the clause connection is already signaled by a connective adverb and PV \textit{so}.
\end{sloppypar}

It should be noted that PV \textit{so}  is one of the most frequent adverbs in ENHG. In the Baumbank.UP \citep{Demske2020},\footnote{\url{https://www.uni-potsdam.de/en/guvdds/potsdam-treebank-of-early-new-high-german/corpus}, accessed via Tündra at August 1st, 2023.} it is the most frequent one: with 2500 occurrences as adverb  (AVD), compared to 2129 \textit{auch} and 1070 \textit{wol}, the second and third most token frequent adverbs. Moreover, in the sequence adverb-finite verb, it is likewise the most frequent type to occur as adverb, making up 16\% of the instances with a lexical finite verb and 27\% of those with an auxiliary.\footnote{The queries \texttt{[pos= "AVD"] . [pos = "VVFIN"]} (adverb followed by a finite lexical verb) and \texttt{[pos= "AVD"] . [pos = "VAFIN"]} (adverb followed by a finite auxiliary) were used.  The results are 648 of the 4159 sequences with a lexical verb have \textit{so} as the adverb and 530 of the 1903 with an auxiliary.} This relatively high frequency of \textit{so} in front of a finite verb in comparison to other adverbs supports the idea that PV \textit{so} is not the element whose occurrence should be questioned. Being the most frequent adverb in this position, it can be taken as a typical signal or flag of a construction that connects the following proposition to the previous discourse. This is also the case for sentence in which there is an additional adverb preceding PV \textit{so}, as in \REF{ex:undadvso}, where the knowing is related to giving a good answer (see also \citealt{BloomJouGe}). Since \textit{so} is a highly polysemous element \citep[e.g.,][918]{Paul2002}, the use of an additional connective adverb with more specific semantics -- like \textit{darumb} -- would then more precisely signal the semantic relation of the following proposition to the previous. 

There is thus a distinction between \textsc{advl} + \textit{so} with adverbial clauses and prepositional phrases vs. with adverbs, in that adverb + \textit{so} as a whole functions as a tool serving discourse cohesion, while adverbial clauses and prepositional phrases typically set a frame (in the narrow or broader sense) in which the proposition expressed in the following clause is to be interpreted.


\section{Conclusion}\label{sec:bloom:7}
The primary issue addressed in this paper is whether \textsc{advl} + \textit{so} in ENHG is associated with discourse cohesion or disruption, leaving open the option that it is associated with both. Cohesion is a well-established concept \citep{Halliday1976} and disruption is defined as instances in which the expected flow of cohesion is interrupted. The two are thus not mutually exclusive but in fact often go hand in hand. This was hypothesized for \textsc{advl} + \textit{so} in ENHG as well, because PV \textit{so} functions both as an adverbial resumptive \citep[e.g.,][]{Meklenborg2020, Catasso2021Generalized} and is attested in contexts where it cannot function as such, i.e., following adverbs. For a similar construction in Swedish and Norwegian, it has been proposed that it is used to indicate a topic/frame-shift \citep{Nordstrom2010, Eide2011}, which indicates disruption.

The paper first evaluates to which degree the element before PV \textit{so}, the precursor, is prototypically a frame-setting element. Despite the fact that frame-setting in the narrow sense, i.e., where the truth value of the proposition depends on the truth value of the frame-setter, is not very prevalent, frame-setting in the broader sense, viz., where the proposition is interpreted in relation to the information provided by the frame-setter, is well attested. Yet, the presence of a precursor does not always indicate a shift, as for example when the adverbial has an additive meaning. In this regard then, \textsc{advl} + \textit{so} is compatible both with cohesive and slightly disruptive discourse segments. This hypothesis was further explored in \sectref{sec:bloom:1}.

The position of the \textsc{advl} + \textit{so}-patterns within the narratives was considered as well as its co-occurrence with referential continuity.
The data indicate that temporal adverbial clauses + \textit{so} are attested in chapter initial position. For other precursors, no evidence was found. Concerning the turn initial occurrence of the construction, the results show that clausal precursors are regularly attested here, but non-clausal ones + \textit{so} are not. Finally, although all types of precursors are attested within sequences, the lack of a shift is more convincing for structures in which there is a non-clausal adverbial. All in all, these results suggest that there is a distinction between clausal and non-clausal adverbials, with non-clausal precursors being less associated with discourse disruption than clausal ones. 

For referential continuity, the realization of the subject in the host was considered and whether it was co-referential with the subject of the preceding main clause. In the majority of the cases, the subject was pronominal, which is suggestive of a strong discourse cohesion and high accessibility of the subject referent. While the subject of the host was more often not co-referential with the subject of the preceding main clause, it was not rare for the two to be co-referential. This means that both a subject-shift and continuation are fully compatible with the pattern. In this regard, there was no significant difference to be found between clausal and non-clausal precursors. This was somewhat unexpected in light of the earlier differences between the constructions. Is there a systematic difference between clausal and non-clausal adverbials regarding their use as a discourse cohesive and disruptive tool? \sectref{sec:bloom:1} addressed this issue and argues for the affirmative. Specifically, the sequence adverb + \textit{so} is exploited as a cohesive tool linking to the following proposition explicitly to the previous discourse, whereas constructions with adverbial clauses and prepositional phrases as precursors may be more disruptive and do not necessarily set up this link themselves. Yet, they are fully compatible with such contexts, which is why co-occurrence with a coordinating conjunction is quite widespread.

\section*{Acknowledgements}
\begin{sloppypar}
This research was funded by the Deutsche Forschungsgemeinschaft (Project 456973946, “Wortstellung und Diskursstruktur in der Frühen Neuzeit”).
\end{sloppypar}

\section*{Primary sources}
\begin{description}[font=\normalfont]
\sloppy
\item[E -- der Goldene Esel (1538).] \textit{Ain Schoen Lieblich, auch kurtzweylig gedichte Lucij Apuleij von ainem gulden Esel...:}\textit{lustig zu lesen /} \textit{mit sch\"o{}nen figuren zugericht grundtlich verdeutscht, durch Johan Sieder.} \textit{1538. 71 pages.} Freiburg: Universitätsbibliothek Freiburg. Available online: \url{http://dl.ub.uni-freiburg.de/diglit/apuleius1538}.

\item[F -- Fortunatus (1509).] Schmitt, Ludwig Erich \& Renate Noll-Wiemann (eds.) 1974. \textit{Fortunatus:} \textit{Von Fortunato und seynem Seckel auch Wünschhütlein.} Hildesheim: Georg Olms Verlag. verified against the first edition published in Müller, Jan-Dirk (ed.). 1990. \textit{Romane des 15. und 16. Jahrhunderts. Nach den Erstdrucken mit sämtlichen Holzschnitten,} 383--586. Frankfurt am Main: Deutscher Klassiker Verlag.

\item[H  -- Huge Scheppel (1455--1472).] \textit{Huge Scheppel; K\"onigin Sibille Saarbrücken. 1455--1456. 76 pages.} Hamburg: Staats-und Universtitätsbibliotek. Codex 12 in scrinio. pp. 1ra-57vb. Available online: \url{https://resolver.sub.uni-hamburg.de/kitodo/HANSh495}.

\item[M  -- Melusine (1474).] Müller, Jan-Dirk (ed.). 1990. \textit{Romane des 15. und 16. Jahrhunderts.} \textit{Nach den Erstdrucken mit sämtlichen Holzschnitten,} 9--176. Frankfurt am Main: Deutscher Klassiker Verlag.\footnote{The later discovered first print dates from 1456.}

\item[O -- Wilhelm von \"Osterreich (1481).] Podleiszek, Franz (ed.). 1936. \textit{Volksbücher von Weltweite und Abenteuerlust,}  191--284. Leipzig: Verlag von Philipp Reclam jun. verified against the printed manuscript
\textit{Historie von Herzog Leopold und seinem Sohn Wilhelm von Österreich}. 1481. BSB Bayerische Staatsbibliothek/MDZ Münchener Digitalisierungszentrum Digitale Bibliothek, 2 Inc c.c. 1067. Augsburg: Anthonius Sorg

\item[P -- Pontus und Sidonia (between 1450 and 1476).] \textit{Pontus und Sidonia. Tandareis und Flordibel - BSB Cgm 577.} München, Bayerische Staatsbibliothek. \url{https://www.digitale-sammlungen.de/en/view/bsb00014877}. Transcribed within the DFG project Wortstellung und Diskursstruktur in der Frühen Neuzeit.

\item[R -- Das Rollwagenbüchlein (1555).] Georg Wickram. 1979. \textit{Das Rollwagenb\oldue{}chlin.} Stuttgart: Philipp Reclam jun.

\item[S -- Die Sch\"o{}ne Magelone (1535).] Müller, Jan-Dirk (ed.). 1990. \textit{Romane des 15. und 16. Jahrhunderts. Nach den Erstdrucken mit sämtlichen Holzschnitten,} 587--678. Frankfurt am Main: Deutscher Klassiker Verlag.

\item[T -- Tristrant und Isalde (1484).] \textit{Tristrant und Isalde: Historia Tristant} 1484. Staatsbibliothek zu Berlin - Preußischer Kulturbesitz, 8 Inc 138, Augsburg: Anton Sorg.  The same edition is published in Elsner, Helga (ed.). 1989. \textit{Tristan und Isolde (Augsburg bei Antonius Sorg, 1484).} Hildesheim: Georg Olms Verlag.

\item[W -- Wigalois vom Rade (1519).] Melzer, Helmut (ed.). 1973. \textit{Wigalois.} Hildesheim: Georg Olms Verlag.
\end{description}

% % % {\sloppy
% % % \printbiblist[heading=subbibliography,keyword={BloomPrimary},title={Primary sources}]{shorthand}}

{\sloppy\printbibliography[heading=subbibliography,notkeyword=this]}
\end{document}
