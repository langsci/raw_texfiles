\documentclass[output=paper]{langscibook}
\ChapterDOI{10.5281/zenodo.17077188}
\author{Nicholas Catasso\orcid{}\affiliation{Bergische Universität Wuppertal}}
\title[A cartographic model for the derivation of non-V2 in German]{“No cracks in the bottleneck!”: A cartographic model for the derivation of non-V2 in German}
\abstract{This paper discusses the derivation of syntactic patterns of present-day German that appear to linearly violate the V2 constraint within the cartographic model and draws insights into the inner structure of the left periphery in this language. It is proposed that these apparent non-V2 arrangements are derivable by means of a limited number of syntactic operations (combining Merge and Move) that do not put into question the classical account of German as a structural V2 language. In particular, the article makes the following claims: (i) the attested V3 and other Verb-Later orders are to be categorized as variants of V2 that are derived according to principles that are compatible with the a/m operations; (ii) German also exhibits a Split CP that is similar~– mutatis mutandis~– to that of Modern Romance, but makes use of different derivational strategies to achieve the relevant linearization(s); (iii) yhis insight can only be gained if (at least) comparable methods and (at least) comparable theoretical assumptions about the syntax of German are used to interpret the data.}
\IfFileExists{../localcommands.tex}{
  \addbibresource{../localbibliography.bib}
  % add all extra packages you need to load to this file

\usepackage{tabularx,multicol}
\usepackage{url}
\urlstyle{same}

\usepackage{listings}
\lstset{basicstyle=\ttfamily,tabsize=2,breaklines=true}

\usepackage{langsci-basic}
\usepackage{langsci-optional}
\usepackage{langsci-lgr}
\usepackage{langsci-osl}
% \usepackage{./langsci/styles/langsci-lgr}
% \usepackage{./langsci/styles/langsci-osl}
% \usepackage{langsci-gb4e}

\usepackage{tikz}
\usetikzlibrary{patterns,calc}
\pgfdeclarepatternformonly{south east lines}{\pgfqpoint{-0pt}{-0pt}}{\pgfqpoint{3pt}{3pt}}{\pgfqpoint{3pt}{3pt}}{
    \pgfsetlinewidth{0.6pt}
    \pgfpathmoveto{\pgfqpoint{0pt}{3pt}}
    \pgfpathlineto{\pgfqpoint{3pt}{0pt}}
    \pgfpathmoveto{\pgfqpoint{.2pt}{-.2pt}}
    \pgfpathlineto{\pgfqpoint{-.2pt}{.2pt}}
    \pgfpathmoveto{\pgfqpoint{3.2pt}{2.8pt}}
    \pgfpathlineto{\pgfqpoint{2.8pt}{3.2pt}}
    \pgfusepath{stroke}}
    
\usepackage{stmaryrd}
\usepackage{wasysym}
\usepackage{multirow}
\usepackage{caption}
\usepackage{subcaption}
\usepackage{mathrsfs}
\usepackage{qtree}

\usepackage{linguex}


  %pminos do not split footnotes
% \interfootnotelinepenalty=10000 %Footnote in Laporte chapters has to be split SN


%\DeclareIndexNameFormat{default}{%
%\nameparts{#1}%
%\usebibmacro{index:name}%
%{\index[names]}%
%{\namepartfamily}%
%{\namepartgiveni}%
% {}% L1
% {}% L2
%{\namepartprefix}% generates spurious space L3
%{\namepartsuffix}% generates spurious space L4
%}

%  {\DeclareIndexNameFormat{default}{%
%     \usebibmacro{index:name}{\index[names]}{#1}{#3}{#5}{#7}}}

%\DeclareIndexNameFormat{default}{%
%  \usebibmacro{index:name}{\sindex[nom]}{#1}{#3}{#5}{#7}}

%\DeclareIndexNameFormat{default}{%
%  \usebibmacro{index:name}{\sindex[person]}{#1}{#3}{#5}{#7}}
%\DeclareIndexNameFormat{default}{%
%\nameparts{#1} \usebibmacro{index:name}{\sindex[person]]}{\namepartfamily}{‌​\namepartgiven}{\nam‌​epartprefix}{\namepa‌​rtsuffix}}

%\newcommand{\smiley}{:)}

%\renewbibmacro*{index:name}[5]{%
%\usebibmacro{index:entry}{#1}%
%{\iffieldundef{usera}{}{\thefield{usera}\actualoperator}\mkbibindexname{#2}{#3}{#4}{#5}}}

% \newcommand{\noop}[1]{}

%remove for final
%\overfullrule=1mm

\newcommand{\tobi}[2]}}
\renewcommand{\S}[1]{\tobi{#1}{\textsc{*}}}

% this volume references
% puts: [this volume]
% already defined: \citetv
%\newcommand{\citepv}[1]{(\citeauthor{#1} \citeyear*{#1} [this volume])}
\newcommand{\citealtv}[1]{\citeauthor{#1} \citeyear*{#1} [this volume]}

%parentheses around example number
\newcommand{\pref}[1]{(\ref{#1})}

% in-text examples

\newcommand{\lnex}[1]{\textit{#1}} %target lang word
\newcommand{\lnlit}[1]{(lit.: `#1')} %literal reading
\newcommand{\lnlat}[1]{(#1)} % latinization
\newcommand{\lntrans}[1]{`#1'} %translation
\newcommand{\lnexl}[2]%
{\lnex{#1}{} \lnlat{#2}} % ex with latinization
\newcommand{\lnexlat}[3]{\lnex{#1}{} \lnlat{#2}{} \lntrans{#3}} % ex with latinization and tranl.

%ch01
\newcommand{\co}[1]{\mbox{\textbf{#1}}}

%ch09

\newcommand{\cyrbulg}[1]{\begin{otherlanguage*}{bulgarian}#1\end{otherlanguage*}}


%ch10
\newcommand{\nlp}{{\small NLP}}
\newcommand{\mwe}{{\small MWE}}
\newcommand{\rae}{{\small RAE}}
\newcommand{\lvc}{{\small LVC}}
\newcommand{\pos}{{\small P}o{\small S}}
%\newcommand{\todo}[1]{ \textcolor{red}{#1} }

%\renewcommand{\labelenumi}{\theenumi}
%\ainamefmt{{vv}{ll}{, ff}{, jj}} % fullname

\newcommand{\biberror}[1]{{\color{red}#1}}

\newcommand{\osenovaitem}{--~} 
  %% hyphenation points for line breaks
%% Normally, automatic hyphenation in LaTeX is very good
%% If a word is mis-hyphenated, add it to this file
%%
%% add information to TeX file before \begin{document} with:
%% %% hyphenation points for line breaks
%% Normally, automatic hyphenation in LaTeX is very good
%% If a word is mis-hyphenated, add it to this file
%%
%% add information to TeX file before \begin{document} with:
%% %% hyphenation points for line breaks
%% Normally, automatic hyphenation in LaTeX is very good
%% If a word is mis-hyphenated, add it to this file
%%
%% add information to TeX file before \begin{document} with:
%% \include{localhyphenation}
\hyphenation{
    Beck-man
    Ngu-yen
    back-chan-nel
    back-chan-nels
    mo-not-o-nous
    ste-reo-typ-i-cal
}

\hyphenation{
    Beck-man
    Ngu-yen
    back-chan-nel
    back-chan-nels
    mo-not-o-nous
    ste-reo-typ-i-cal
}

\hyphenation{
    Beck-man
    Ngu-yen
    back-chan-nel
    back-chan-nels
    mo-not-o-nous
    ste-reo-typ-i-cal
}
 
  \togglepaper[1]%%chapternumber
}{}

\begin{document}
\maketitle 

\section{Introduction}
\label{sec:catasso:1}

German is a so-called “asymmetric Verb-Second (V2) language” in which (declarative) main clauses exhibit leftward verb movement from their base-generation site in the VP to some C-head and – typically – raising of a further constituent into a left-peripheral specifier \REF{ex:catasso:1}.\footnote{I would like to thank two anonymous reviewers for very insightful comments on a previous version of this paper.} The latter operation is related to the satisfaction of an \mbox{EPP(-like)} feature (\citealt{Chomsky2000, Chomsky2001}), which, in its most general formulation, corresponds to a formal mechanism requiring that the specifier of the C-domain (in languages like German) be filled by a maximal projection.\footnote{The exact formulation of the EPP and its structural locus in German are still a matter of controversy in the literature. Here, I follow, e.g., \citet{Grewendorf2002Minimalistische}, \citet{Fanselow2004Cyclic}, \citet{Frey2006}, \citet{Holmberg2015} in assuming that movement to Spec,CP (or to one of the CP-specifiers) is brought about by a feature of C° that is saturated by lexicalizing the corresponding spec-position in declarative main clauses by means of a movement operation or external merge.} Meanwhile, in embedded clauses introduced by an overt subordinating element, instead, the finite verb remains in I° and no XP base-generated in the middle field is moved into the left periphery \REF{ex:catasso:2a}. This is generally captured in terms of a complementary distribution between the finite verb and a subordinator in C°: if the head of the CP is lexicalized by a complementizer, the presence of this element blocks V-movement into the same position \citep{Höhle1986} \REF{ex:catasso:2b}:

\ea%1
    \label{ex:catasso:1}
    \gll Hans hat gestern Maria geküsst.\\
         Hans \textsc{aux.prs.3sg} yesterday Mary \textsc{ptcp}{}-kiss-\textsc{ptcp}\\
    \glt `Hans kissed Mary yesterday.'
\ex%2
    \label{ex:catasso:2}
\ea[]{\label{ex:catasso:2a}
  \gll …dass Hans gestern Maria geküsst hat.\\
        that Hans yesterday Mary \textsc{ptcp}{}-kiss-\textsc{ptcp} \textsc{aux.prs.3sg}\\}
        
\ex[*]{\label{ex:catasso:2b}
    \gll …dass Hans hat gestern Maria geküsst.\\
         that Hans \textsc{aux.prs.3sg} yesterday Mary \textsc{ptcp}{}-kiss-\textsc{ptcp}\\
    \glt (int.:)  `…that Hans kissed Mary yesterday.'}
\z
\z

While the V2 status of present-day German is basically uncontroversial, this is not the case for the internal makeup of the left periphery in this language assumed to accommodate the corresponding structures. In the generative literature, a number of proposals have been put forward that seek to combine the above-mentioned generalization and the general principles of transformational grammar in order to answer the question of how complex this clausal domain can be assumed to be in German. Sticking to the key facts sketched in \REF{ex:catasso:1} and \REF{ex:catasso:2}, three of these models are presented in \sectref{sec:catasso:2} before addressing some problematic data and spelling out a theoretical proposal for their derivation in the following sections. 

\section{How to derive V2?}
\label{sec:catasso:2}
\subsection{Three models for the derivation of German main clauses}
\label{sec:catasso:2.1}

As mentioned above, over the past decades, three models have been particularly successful in the generative literature as possible theoretical models for the derivation of the word orders shown in \REF{ex:catasso:1} and \REF{ex:catasso:2} for present-day German (in what follows: M1, M2, and M3). These approaches yield different implications that correspond to different ways of conceiving of the structure of the left periphery in general, as well as the mechanisms leading to surface arrangements in specific languages.

M1 is the classic generative model advocated (in this form) at least since Government \& Binding and still fairly popular in the treatment of bona fide V2 languages (cf., e.g., \citealt{Haider1993, Lohnstein2000}, \citealt[165f.]{Frey2000, Fuß2008, Walkden2017Language}). In M1, the left periphery of German consists of only one projection, i.e. one head and one specifier. This representation implies that the CP is understood as an unspecified domain that is compatible with different grammatical and information\hyp structural features. At the same time, the idea of a one-projection CP is in line with the fact that a well-behaved V2 language by definition does not allow for “exceptions to the rule”, that is, arrangements in which the preverbal position is occupied by more than one constituent. In a non\hyp Kaynian (\citeyear{Kayne1994}) approach to linearization, the verb is base\hyp generated in V°, moved to the sentence\hyp final head of the IP to acquire its inflectional features and then moved to C°.\footnote{As pointed out by an anonymous reviewer, it should be acknowledged that for German, the very existence of the IP layer has been called into question in the literature, since v-to-T movement in a verb-final clause could be deemed string-vacuous. The reader is referred to Haider (e.g., \citealt{Haider1993, Haider2010}) for discussion. In what follows, I adopt a cartographic approach and label the clausal domain below the CP “IP”. However, this detail is not central to the focus of the present study, which primarily centers on the makeup of the left periphery.} Any type of constituent – bearing e.g. a topic (\textit{α}), a focus (\textit{ß}), a \textit{wh} (\textit{γ}) feature, etc.) appearing to the left of the finite verb in the surface syntax is first-merged in the middle field and raised to Spec,CP, as illustrated in \figref{fig:catasso:1}.

\begin{figure}
% % % \includegraphics[width=\textwidth]{figures/a04catassoNocracksinthebottleneckpaperVersion2-img001.png}
\begin{forest} for tree={fit=band,align=center,minimum width=2cm}
[CP
    [{Spec\\
    $\{\alpha, \beta, \gamma, ...\}$\\
    \textbf{XP}}, name=XP
    ]
    [C$'$
	[{C°\\{}[+\textsc{fin}]\\\textbf{Vfin}},name=Vfin]
	[IP
	    [(\st{XP}),name=XPbar]
	    [I$'$
		[VP
		    [{\st{\textbf{XP}}~\dots~\st{\textbf{Vfin}}\\
			~$\{\alpha\}$ ~~[+\textsc{fin}]}, roof,name=VP]
		]
		[{I°\\\st{\textbf{Vfin}}},name=Vfinbar]
	    ]
	]
    ]
]
\begin{scope}[>={Triangle[]}]
  \draw [dashed, ->] (XPbar.250) |- ++(-2ex, -2ex) -| (XP);
  \draw [dashed, ->] (VP.195) -| (XPbar.290);
  \draw [dashed, ->] (VP.345) -| (Vfinbar.250);
  \draw [dashed, ->] (Vfinbar.290) |- ++(-2ex,-3\baselineskip) -| (Vfin);
\end{scope}
\end{forest}
\caption{\label{fig:catasso:1}M1: One-projection model for the representation of V2}
\end{figure}


M2 is the one developed within the cartographic model (\citealt{Rizzi1997,Rizzi2001,Rizzi2004Structures}, and much subsequent work) (\figref{fig:catasso:2}). While this paradigm was originally proposed for the representation of the left edge of languages like present-day Italian, it has also been adopted for many other languages and language stages (cf., inter alia, \citealt{Axel2007, Axel-Tober2018, Speyer2008Doppelte, BideseEtAl2012, Petrova2012, Cognola2013Syntactic, Wolfe2016Left,Wolfe2018, HaegemanGreco2018West, Hinterhölzl2017, SpeyerWeiß2018, DeClercqHaegeman2018, Samo2019, Eska2020, GrecoHaegeman2020, Catasso2021Large,Catasso2021Verbspäterstellungen, Catasso2021Theoretical}). In this model, the left edge of the clause is made up of a cascade of functional projections, each encoding a particular pragmatic, grammatical or information\hyp structural feature. In a nutshell, each projection postulated in Rizzi’s Split CP corresponds to one of the functional traits licensed by the underspecified Spec,CP in the model addressed above. A relevant innovation introduced by the idea of a multi-projectional left periphery is that elements carrying these features occur in a fixed word order in the clause and that the corresponding regularities are cross\hyp linguistically observable (all necessary changes having been made). In the original version given below, ForceP is responsible for clause typing and the interface between sentence and discourse, the two TopPs host (different types of) topics, FocP encodes focus and \textit{wh}\hyp interrogativity, and FinP regulates the relation between the left periphery and the inflectional domain of the clause.

\begin{figure}
% % % % \includegraphics[width=.6\textwidth]{figures/a04catassoNocracksinthebottleneckpaperVersion2-img002.png}
\begin{forest}for tree={align=center,minimum width=1.5cm}
[ForceP
    [Spec]
    [Force$'$
	[Force°]
	[TopP*
	    [Spec]
	    [Top$'$
		[Top°]
		[FocP
		    [Spec]
		    [Foc$'$
			[Foc°]
			[TopP*
			    [Spec]
			    [Top$'$
				[Top°]
				[FinP
				    [Spec]
				    [Fin$'$
					[Fin°]
					[IP
					    [{\hspace*{1cm}}, roof]
					]
				    ]
				]
			    ]
			]
		    ]
		]
	    ]
	]
    ]
]
\end{forest}
\caption{\label{fig:catasso:2}M2: Cartographic model for the CP domain \citep{Rizzi1997}}
\end{figure}

Based on this model, the recent literature has increasingly supported the proposal of a so-called “ForceP V2” in asymmetric V2 languages (cf., inter alia, \citealt{Poletto2013, Wolfe2016Left, BiberauerRoberts2015,BiberauerRoberts2016, Hinterhölzl2017}). This implementation of main-clause syntax implies obligatory cyclic movement of the finite verb to Fin° and to Force° and raising of \textit{minimally} and \textit{maximally} one constituent to the left periphery via Spec,FinP – in order to derive the so-called “bottleneck effect” \citep{Haegeman1996} that blocks further movement into the CP domain from the middle-field – followed by movement operations to some higher specifier and eventually to Spec,ForceP, as illustrated in \figref{fig:catasso:3}. This analysis represents an elegant attempt to combine the idea that, even in strict V2 languages, the CP layer has a complex inner makeup and the fact that the preverbal specifier can only be targeted by one maximal projection in these systems.

\begin{figure}
% % % % \includegraphics[width=\textwidth]{figures/a04catassoNocracksinthebottleneckpaperVersion2-img003.png}
\begin{forest} for tree = {fit=band}
[ForceP
  [Spec\\XP,name=XP]
  [Force'
    [Force°\\Vfin,name=Vfin]
    [...
      [,nice empty nodes]
      [FinP
        [\sout{XP},name=XPbar]
        [Fin'
          [Fin°\\\sout{Vfin},name=Fin]
          [IP/VP
              [\sout{XP} ... \sout{Vfin},roof,name=IPVP]
          ]
        ]
      ]
    ]
  ]
]
\begin{scope}[>={Triangle}]
\draw [->, dashed] (XPbar.250) |- ++(-2ex, -2ex) -| (XP);
\draw [->, dashed] (Fin.250)   |- ++(-2ex, -2ex) -| (Vfin);
\draw [->, dashed] (IPVP.210)  |- ++(-2ex, -2ex) -| (XPbar.290);
\draw [->, dashed] (IPVP.320)  |- ++(-2ex, -4ex) -| (Fin.290);
\end{scope}
\end{forest}
\caption{\label{fig:catasso:3}Derivation of ForceP V2}
\end{figure}

A third option for a formalization of the left periphery (of German) is M3 in \figref{fig:catasso:4}. This approach results from a revision of the original one-projection model (\figref{fig:catasso:1}) and has been very successful at least since \citegen{Chomsky1995} minimalist turn (also cf. \citealt{Bobalijk1999, Chomsky2000, Richards2001}; for German, \citealt{Grewendorf2002Minimalistische, Grewendorf2002Left, Rezac2004, Lahne2009}, G. \citealt{Müller2010}; for a similar implementation with respect to historical stages and urban varieties of German, \citealt{Fuß2008,Walkden2014,Walkden2017Language}; for a similar idea within a theory of CP recursion, also cf. \citealt{IatridouKroch1992, Vikner2017}, etc.).

\begin{figure}
% % % % \includegraphics[width=\textwidth]{figures/a04catassoNocracksinthebottleneckpaperVersion2-img004.png}
\begin{forest}for tree={align=center,minimum width=2cm}
[CP
    [Spec$_1$]
    [C$'$
	[Spec$_2$]
	[(\dots)
	    [Spec$_3$]
	    [C$'$
		[C°]
		[IP
		    [{\hspace*{1cm}}, roof]
		]
	    ]
	]
    ]
]
\end{forest}
\caption{\label{fig:catasso:4}M3: CP-recursion model of the left periphery}
\end{figure}

M3 is based on the idea that the CP layer can be recursively expanded to make room for additional material, but that this operation only involves the Spec position, so that the left edge of a clause consists of \textit{one} projection, viz. one head, and as many specifiers as are needed. An important implication of this model is that each non-head position within the CP resulting from recursive applications of Merge, is specialized in the encoding of one specific feature (say, α = Frame, ß = Familiar Topic), which ensures that only some linearizations are possible, while others are excluded, and that the finite verb is attracted to the head position C°.\footnote{It is to be noted, however, that not all proposals based on the idea of CP recursion treat feature encoding in the same way: in \citegen{Walkden2017Language} model, which includes two C-layers, each CP has its own head and the lower CP roughly corresponds to FinP, while the higher one includes all features from FocP upwards.} This is schematically illustrated in \figref{fig:catasso:5}.

\begin{figure}
% % % % \includegraphics[width=\textwidth]{figures/a04catassoNocracksinthebottleneckpaperVersion2-img005.png}
\begin{forest} for tree={align=center,minimum width=2cm}
[CP
    [Spec\\$\{\alpha\}$,name=alpha]
    [C$'$
	[Spec\\$\{\beta\}$,name=beta]
	[C$'$
	    [C°]
	    [IP/VP
		[{\hspace*{1cm}}, roof]
	    ]
	]
    ]
]
\begin{scope}[every node/.style={single arrow, rotate=90, minimum height=1cm}]
\node [anchor=north,below=\baselineskip of beta.255, fill=lsRed!50, draw=lsRed] {};
\node [anchor=north,below=\baselineskip of alpha.255, fill=lsMidDarkBlue!50, draw=lsMidDarkBlue] {};
\end{scope}
\end{forest}
\caption{\label{fig:catasso:5}Feature encoding in CP recursion}
\end{figure}

M1, M2 and M3 originate from divergent conceptions of the internal makeup of the left periphery and the derivation of syntactic structures, but are in principle all perfectly compatible with the simple asymmetric-V2 system illustrated for present-day German in \REF{ex:catasso:1}. 

What seems to be problematic in relation to the derivation of syntactic structures in this language is the fact that it allows for a number of linear phenomena that seem to challenge each of the models presented in this section. In order to account for every possible canonical and non-canonical word order, it is therefore necessary to consider how maximally complex (and still grammatical) structures can be derived in German without violating the V2 constraint.

\subsection{Scope of the paper}
\label{sec:catasso:2.2}

In consideration of the options presented above, the aim of this paper is to draw attention on so-called “non-V2” or “Verb-Late(r)” phenomena of German, i.e. linear V3, V4, etc., and to propose a model for the derivation of a number of phenomena superficially incompatible with the descriptive notion of V2. 

In what follows, it will be proposed that the most appropriate model to account for the data observable in German is M2, i.e. that this language has~– mutatis mutandis~– a Split CP à la Rizzi. It will be shown that V2 is (the product of) an all-structural, not linear constraint, and does not result from the lexicalization of ForceP, but from V-to-Fin movement implying a simultaneous derivation of the bottleneck effect and satisfaction of an EPP(-like) feature in this projection resulting from a combination of Move and Merge. 

In a nutshell, all V3, V4 etc. patterns attested at different levels of the language represent linear variants of V2 that are operationalized according to the very same principles and do not require \textit{ad hoc} postulations about the makeup of the left periphery or the status of the individual phenomena.

The rest of the paper is structured as follows: in \sectref{sec:catasso:3}, the relevant German data linearly violating the V2 restriction described above are presented. In \sectref{sec:catasso:4}, a uniform model is proposed on the basis of M2 that attempts to derive all possible Verb-Late(r) patterns by considering not only the single phenomena, but also their interaction in the left periphery of the clause. \sectref{sec:catasso:5} concludes.

\section{V3 as a window into the structure of the left periphery}
\label{sec:catasso:3}


As mentioned in \sectref{sec:catasso:2}, German exhibits a number of main-clause phenomena which – at least from a linear perspective – deviate from the V2 constraint. In particular, there exist constructions in which the prefield, namely the intrasentential portion of the clause preceding the finite verb, is occupied by more than one element, as has been noticed in a considerable amount of literature in the last decades (cf., among many others, \citealt{Lee1975, vandeVelde1978, Lühr1985, Cardinaletti1989, Auer1996,Auer1997}, St. \citealt{Müller2003,Müller2005}, 2023 \citealt{Frey2004, Eggs2004, Speyer2008Doppelte, Breindl2011, MüllerBildhauerCook2012, Schalowski2015}, Speyer \& Weiß 2018, S. \citealt{Müller2019, Bunk2020, WieseEtAl2020, Catasso2021Large, Breitbarth2022,Breitbarth2023, Axel-Tober2023}). This does not seem to be in line with the very nature of V2 itself, which imposes a limit of one (moved or merged) full XP per CP domain.

As the examples in \REF{ex:catasso:3} and \REF{ex:catasso:4} show, V3 is attested in different configurations. The sentences in \REF{ex:catasso:3} instantiate correlative Verb-Late(r) constructions, i.e. sequences in which the finite component of the verbal predicate formally surfaces in third position and the two elements in the prefield are co-referential. The structure in \REF{ex:catasso:3a} is a left dislocation: a fronted DP is taken up by a pronominal resumptive (glossed as \textsc{p.res} here and in what follows) in preverbal position replicating its $\varphi $-features. \REF{ex:catasso:3b} illustrates adverbial resumption: similarly to DP left dislocation, an adjunct occurs in first clause position and is resumed by an adverbial element (\textsc{a.res}) to the left of the finite verb in C. Both patterns raise the question as to how it is possible to accommodate the left-dislocated phrase and the resumptive in the preverbal area:

\ea%3
    \label{ex:catasso:3}
    \ea \label{ex:catasso:3a} 
      \gll Den Hans, den kenne  ich seit langem.\\
           the-\textsc{acc.sg} Hans \textsc{p.res.acc.sg} know.\textsc{prs.1sg} I-\textsc{nom.sg} since long\\
      \glt `I have known Hans for a long time.'  \citep[9]{Cardinaletti1989}
    \ex \label{ex:catasso:3b} 
     \gll Wenn Sie den Job haben wollen, dann müssen Sie ein bisschen auf den Punkt kommen.\\
          if you-\textsc{nom.sg.pol} the-\textsc{acc.sg} job have-\textsc{inf} want-\textsc{prs.3pl} \textsc{a.res} must.\textsc{prs.3pl} you-\textsc{nom.sg.pol} a little to the-\textsc{acc.sg} point come-\textsc{inf}\\
      \glt `If you want the job, you need to get to the point.' (adapted from: \citealt{Auer2000}: 5)  
    \z
\z

\newpage
A further group of V3 phenomena includes different non-correlative patterns, like the ones exemplified in \REF{ex:catasso:4}. In \REF{ex:catasso:4a}, a clausal (here: conditional) adjunct precedes the \textit{wh}{}-interrogative \textit{warum} (‘why’) in preverbal position. \REF{ex:catasso:4b} features an argument to the left of a so-called “post-initial particle” marking the fronted constituent – depending on the information-structural context – as an aboutness or contrastive topic. In \REF{ex:catasso:4c} and \REF{ex:catasso:4d}, two variants of what seems to be the same pattern are illustrated: in \REF{ex:catasso:4c} a frame(-like) adjunct\footnote{Frame setters are expressions that provide the setting within which the predicate of the clause is to be interpreted (\citealt{Chafe1976, Krifka2006, Jacobs2001}). This notion can be understood in different ways. In the present discussion, I will refer to frame setters as to any type of adjunct (i.e., phrase that is not part of the verb’s argument structure) occurring in the CP domain that introduces a background. This background can be realized by a prototypical local or temporal adverbial, but also by a more abstract category (conditionality, concessiveness, causality, finality, etc.). A conditional clause, for instance, does not introduce a deixis localized in the actual world, but it projects the content of the clause into a parallel world, thus providing an abstract local/temporal reference for the sentence. Similarly, a fronted causal adjunct encloses the action or condition described by the predicate in a certain (presupposed) state of affairs resulting from a cause-and-effect relation, and so on.} appears before an argument (the subject of the sentence), in \REF{ex:catasso:4d} the argument surfaces in a higher position than the adjunct (here: a temporal adverbial clause). In \REF{ex:catasso:4e}, two distinct adverbials, the temporally-interpreted PP \textit{am Nachmittag} (‘in the afternoon’) and the locative PP \textit{in einem Gang…} (‘in an alley…’), which clearly have different references, both occur in preverbal position:

\ea%4
    \label{ex:catasso:4}

\ea \label{ex:catasso:4a}\citep[346]{Eggs2004}\\\relax
   \gll [Wenn sie     schon so … viel Geld gewonnen                   haben,] [warum] spendieren sie nicht wenigstens eine Runde?\\
        if             they-\textsc{nom} already       so          {}   much        money        \textsc{ptcp}{}-win-\textsc{ptcp} \textsc{aux.prs.3pl} why            pay-\textsc{prs.3pl} they-\textsc{nom} \textsc{neg} at-least one-\textsc{acc.sg} round\\
    \glt `If they have won so much money, why don’t they at least pay for a round?'

  
\ex  \label{ex:catasso:4b}\citep[18]{Breindl2011}\\
 \gll [Dieses Raster] [nun] entspricht in einer gewissen Weise einem Enzym.\\
      this-\textsc{nom.sg} grid now correspond-\textsc{prs.3sg} in a-\textsc{dat.sg} certain-\textsc{dat.sg} way a-\textsc{dat.sg} enzyme\\
 \glt ‘In a way, this grid corresponds to an enzyme.’

 \ex \label{ex:catasso:4c}\citep[8]{Breitbarth2022}\\
 \gll [Heute,] [die Goethe-Institute in Indien] heißen alle Max Müller Bhavan.\\
      today the\textsc{{}-nom.pl} Goethe-institute-\textsc{nom.pl} in India be-called-\textsc{prs.3pl} all-\textsc{nom-pl} Max Müller Bhavan\\
 \glt `Today, all Goethe Institutes in India are called “Max Müller Bhavan”.'
  
\ex \label{ex:catasso:4d}\citep[120]{Lee1975}\\
 \gll [Piachi,] [als ihm der Stab gebrochen war,] verweigerte sich hartnäckig der Absolution.\\
      Piachi when he-\textsc{dat.sg} the-\textsc{nom.sg} staff \textsc{ptcp}{}-break-\textsc{ptcp} \textsc{aux.pst.3sg} refuse-\textsc{pst.3sg} \textsc{refl.acc.3sg} stubbornly the-\textsc{dat.sg} absolution\\
 \glt `Piachi, when the staff was broken over him, stubbornly refused absolution.'

\ex \label{ex:catasso:4e}(\citealt{Müller2003}: 38, quoted from: \citealt{vandeVelde1978}: 35)\\
   \gll [Am Nachmittag] [in einem Gang zwischen Ladenhäusern] ließ B sich das Flugticket geben.\\
         in-the-\textsc{dat.sg} afternoon in a\textsc{{}-dat.sg} alley between shop-house-\textsc{dat.pl} let-\textsc{pst.3sg} B \textsc{refl.dat.3sg} the-\textsc{acc.sg} flight-ticket give-\textsc{inf}\\
   \glt `In the afternoon, in an alley between buildings, he received the flight ticket.'
\z
\z

In light of the current debate, both the correlative and the non-correlative patterns exemplified above are problematic for a number of reasons. From a theoretical point of view, not only is the internal makeup of the left periphery of German a hotly debated subject (see \sectref{sec:catasso:2.1}); there does also not seem to be consensus on the syntactic and semantic status of the elements occupying the prefield in these sentences – for instance, on how to derive resumptive structures like (\ref{ex:catasso:3a}--\ref{ex:catasso:3b}), on whether adverbials like those in (\ref{ex:catasso:4a}--\ref{ex:catasso:4c}) occur in an intra- or extrasentential projection and whether they are moved or merged into their surface position, and on which criteria should be considered in order to determine if element like post-initial \textit{nun} in \REF{ex:catasso:4b} are maximal projections or heads.  

Moreover, most studies investigating V3 in German include analyses of single patterns – e.g., only the derivation of left dislocation, or only the derivation of contrastive topics marked by a post-initial element – and do not consider their interaction within the left periphery. Given that in the majority of cases neither the derivational mechanisms leading to the relevant linearizations nor the individual analyses are shared by the linguists’ community, it follows that the theoretical results are not necessarily comparable.

In \sectref{sec:catasso:4}, the attempt will be made to derive the different phenomena considered in \REF{ex:catasso:3} and \REF{ex:catasso:4} within the same theoretical model and to propose a comprehensive model for the formalization of all observable V2 and Verb-Late(r) data in German, including those involving multiple V3 phenomena simultaneously. This proceeds from the idea that only an approach that makes use of the same assumptions and tools of analysis to account for the relevant facts can reduce all linear instances of V2, V3, V4, etc. to the same common denominator.

\section{A cartographic model for the left periphery of German}
\label{sec:catasso:4}
In this section, a theoretical model will be presented that aims to allow for an exhaustive treatment of unmarked V2 linearizations, as well as of non-V2 patterns. It will be proposed that the only sensible way of dealing with this kind of variability is to assume a Split CP à la \citet{Rizzi1997} (the cartographic M2 in \sectref{sec:catasso:2}) and to adapt this to the observable regularities of German.

The following technical details will be assumed in what follows for reasons to become apparent below:\largerpage[2]

\begin{itemize}
    \item in present-day German, V2 is basically an instantiation of V-to-C   movement. There is no IP V2 (pace \citealt{Travis1984});
    \item it is FinP that carries the EPP(-like) feature, thus attracting the finite verb from its base-generation site in V°;
    \item V2 in well-behaved V2 languages is subject to a bottleneck effect (\sectref{sec:catasso:2.1} and (\ref{ex:catasso:5}--\ref{ex:catasso:6})): only one XP can move into the left periphery, and this operation is mediated by FinP, whose specifier is \textit{always} filled by a trace in declarative main clauses (\ref{ex:catasso:5a}--\ref{ex:catasso:5b}). If it is the case that the preverbal position is occupied by an expletive \REF{ex:catasso:6a}, this element is first-merged into Spec,FinP \REF{ex:catasso:6b}. For ease of representation, in \REF{ex:catasso:5} and \REF{ex:catasso:6} only the derivation of the preverbal phrase is considered.
\end{itemize}

\ea%5
    \label{ex:catasso:5}
\ea \label{ex:catasso:5a}
    \gll Den Keks hat Peter schon gegessen.\\
         the-\textsc{acc.sg} biscuit \textsc{aux.prs.3sg} Peter already \textsc{ptcp}{}-eat-\textsc{ptcp}\\
    \glt `Peter has already eaten the biscuit.'
\ex \label{ex:catasso:5b}\relax  
    [\textsubscript{YP} \textit{den Keks}\textsubscript{i} [\textsubscript{FinP}  t\textsubscript{i} [\textsubscript{IP/VP} t\textsubscript{i} ]]]]
\z
\z

\ea%6          
\label{ex:catasso:6}
    \ea  \label{ex:catasso:6a}
    \gll Es regnet.\\
         \textsc{expl} rain-\textsc{prs.3sg}\\
    \glt `It’s raining.'
    \ex \label{ex:catasso:6b} 
    [\textsubscript{FinP}  \textit{es} [\textsubscript{IP/VP}]]
    \z
\z

Before going into the details of the interplay between the non-V2 patterns illustrated in the previous section, the derivation of left dislocation and constructions involving post-initial markers will be addressed in \sectref{sec:catasso:4.1}.

\subsection{Derivation of left dislocation and post-initial particles}
\label{sec:catasso:4.1}
Two linear non-V2 phenomena that provide crucial evidence for identifying the positions in which information structure is mapped onto the syntactic representation in the CP domain are left dislocation and post-initial markers.

\subsubsection{Pronominal and adverbial left dislocation}
\label{sec:catasso:4.1.1}
\largerpage

With respect to the former, it can be assumed that resumptive constructions with a fronted DP \REF{ex:catasso:7a} and a fronted adverbial \REF{ex:catasso:7b} have the same underlying derivation, since these two patterns are united by  a number of common properties: they both realize a strategy to introduce a topic in the discourse (an aboutness or contrastive topic in nominal left dislocation, a frame-setting topic in adverbial resumption), the construction includes a left-peripheral resumptive in preverbal position that is only possible in main clauses and in embedded clauses with root-like properties,{\interfootnotelinepenalty=10000\footnote{As for unintroduced embedded clauses, \citet{Bayer2001} provides the following example in which the left-dislocated constituent tops an object clause that is not selected by a complementizer. Such a structure can be assumed to be “root-like” for at least two reasons: (i) from a semantic point of view, it complements a doxastic predicate (\emph{glauben}  ‘believe’) compatible with a semi\hyp assertive reading of the subordinate clause (cf., e.g., \citealt{Reis1997, Auer1998, Meinunger2004, Truckenbrodt2006, Freywald2013, Petrova2020}); (ii) syntactically, it does not contain a lexicalized complementizer; if that were the case, the conjunction would block any movement or merge into the left periphery of the clause:
  \ea \citep[24]{Bayer2001}\\
      \gll Ich glaube, den Hans, den kennt er kaum.\\
          {I-\textsc{nom}}  {believe-\textsc{prs.1sg}}  {the-\textsc{acc.sg}}  {Hans}  {\textsc{p.res.acc.sg}}  {know-\textsc{prs.3sg}}  {he-\textsc{nom}} {barely}\\
      \glt ‘I think he barely knows Hans.’
   \z}} 
and this element is fully optional. In the case of \textit{d}{}-resumptives \REF{ex:catasso:7a}, which take up the reference of a preposed nominal expression, and of the specialized resumptive \textit{da} \REF{ex:catasso:7b} (\citealt{Meklenborg2020, MeklenborgLohndal2023}), which is compatible with local and temporal antecedents, this optionality is purely register-based: the variant with the resumptive is typically produced in spoken\slash colloquial German, while the resumptiveless one characterizes the standard language. There is no interpretive difference between the two structures:

\ea%7
    \label{ex:catasso:7}
    \ea \label{ex:catasso:7a}
        \gll  Den Hans(, den) kann ich nicht leiden.\\
              the-\textsc{acc.sg} Hans \textsc{p.res.acc.sg} can-\textsc{prs.1sg} I-\textsc{nom} \textsc{neg} suffer\textsc{{}-inf}\\   
        \glt `I can’t stand Hans.'
    
    \ex \label{ex:catasso:7b}
        \gll Damals(, da) hatten wir Wichtigeres  zu tun.\\
             back-then \textsc{a.res} have-\textsc{pst.1pl} we-\textsc{nom} important-\textsc{comp.acc.sg} to do-\textsc{inf}\\               
         \glt `Back then, we had more important things to do.'
     \z
\z

It is reasonable to assume the following derivation for these patterns (in the spirit of \citealt{Ross1967, Cinque1977, Grohmann1997, Grohmann2000, Grohmann2003, Grewendorf2002Minimalistische,Grewendorf2002Left, SpeyerWeiß2018, Catasso2021Large}): the phrase surfacing on top of the syntactic tree is base-generated in the relevant middle-field position and cyclically moved to Spec,FinP, where it derives the bottleneck effect, and to some higher specifier, in which it is pronounced, at the end of the process. The latter phase of the derivation will be addressed in more detail in the next paragraphs. What is crucial here is that the resumptive that the moved constituent leaves in Spec,FinP is treated as a trace spell-out, viz. as a trace whose phonetic realization is optional and which replicates the same referential features as the fronted XP. Whether it is pronounced or not, the trace blocks any further movement operation from a lower position and prevents merge of lexical material in the left periphery. The finite verb is moved to Fin°.{\interfootnotelinepenalty=10000\footnote{As is the case for V3 phenomena in general, a number of alternative implementations have been proposed in the literature for left dislocation. \citet{Ott2014}, for instance, assumes left dislocation to derive from a biclausal structure with two juxtaposed CPs in which the higher one is deleted at PF (i). \citet{Haider2010}, instead, makes the case that the left-dislocated phrase is pre-adjoined to the clause, i.e., base-generated to the left of the CP (ii):

\ea (adapted from \citealt{Ott2014}: 270)\\{} 
[{\textsubscript{CP1}}  {[XP]} {\textsubscript{i} } {[… t} {\textsubscript{i} } {…]] [} {\textsubscript{CP2} } {\textsc{res}} {\textsubscript{k}}  {… t} {\textsubscript{k} }  {…]}

\ex (adapted from \citealt{Haider2010}: 3)\\{} 
[{\textsubscript{FP}}  {XP} {\textsubscript{j}} [ {\textsubscript{FP}}  {\textsc{res}} {\textsuperscript{i}} {\textsubscript{j}}  {[Vfin [ … t} {\textsubscript{i}}  {…]]]]}
\z
In the present discussion, I will stick to a movement analysis, which I think does better justice both to the parallelism between topicalization without resumption and left dislocation and to the interaction between left-dislocation constructions and other V3 patterns simultaneously occurring in the left periphery (see below).}} In this sense, the linear V3 sequence resulting from these operations does not violate the V2 constraint \textit{structurally}, since it involves movement of only one constituent into the prefield. A sketchy representation is given in \REF{ex:catasso:8}:

\ea%8
    \label{ex:catasso:8}\relax [\textsubscript{ForceP} [\textsubscript{YP} [XP]\textsubscript{i} … [\textsubscript{FinP} [\textsc{res}]\textsubscript{i} [\textsubscript{Fin°} Vfin\textsubscript{j} ] [\textsubscript{IP/VP} t\textsubscript{i} … t\textsubscript{j} ]]]]
\z

\subsubsection{Particles in post-initial position}
\label{sec:catasso:4.1.2}

As far as markers in post-initial position are concerned, it is to be noticed that elements like \textit{aber} (‘however’), \textit{hingegen} (‘instead’), \textit{beispielsweise} (‘for instance’), etc. typically mark fronted contrastive topics \citep{Breindl2011}. In \REF{ex:catasso:9}, \textit{Maria} and \textit{Hans} are two referents that are selected from a set of alternatives, but are not part of the focus domain of the respective sentences. In this utterance, the contrastive topic in the second conjunct is disambiguated by means of the marker \textit{aber}:

\ea%9
    \label{ex:catasso:9}
    \gll (Die) Maria war auf der Party. (Den) Hans aber habe ich nicht gesehen.\\
         die-\textsc{nom.sg} Maria be-\textsc{pst.3sg} at the\textsc{{}-dat.sg} party the-\textsc{acc.sg} Hans however \textsc{aux.prs.1sg} I-\textsc{nom} \textsc{neg} \textsc{ptcp}{}-see-\textsc{ptcp}\\
     \glt `Maria was at the party. However, I didn’t set eyes on Hans.'
\z

There is unbiased evidence that contrastive topics are moved constituents in German (cf., e.g., \citealt{Catasso2021Large}: 786--787). For instance, in some (western and northern) varieties of the language, prepositional adverbs of the \textit{da}+\textsc{prep} type (\textit{dafür} ‘for that’, \textit{dagegen} ‘against that’, \textit{damit} ‘with that’, etc.) can optionally be split. Therefore, in contexts like \REF{ex:catasso:10}, which imply a contrastive reading of the preposed constituent (\textit{da} anaphorically refers back to renting an apartment in Valletta in the pre-context), the referential component of the adverb can surface in the prefield either independently \REF{ex:catasso:10a} or together with the preposition \REF{ex:catasso:10b}:

\ea%10
    \label{ex:catasso:10}
    \emph{Context:} For our holiday in Malta, Maria has proposed that we could rent an apartment together in the city center of Valletta.

\ea \label{ex:catasso:10a}
    \gll Da habe ich nichts gegen (…aber ich möchte mein eigenes Schlafzimmer haben)\\
         \textit{da} have-\textsc{prs.1sg} I-\textsc{nom} nothing against but I-\textsc{nom} want-\textsc{sbj.1sg} my own\textsc{{}-acc.sg} bedroom have-\textsc{inf}\\\
\ex \label{ex:catasso:10b} 
    \gll Dagegen habe ich nichts (…aber ich möchte mein eigenes Schlafzimmer haben)\\
         \textit{da-}against have-\textsc{prs.1sg} I-\textsc{nom} nothing but I-\textsc{nom} want-\textsc{sbj.1sg} my own\textsc{{}-acc.sg} bedroom have-\textsc{inf}\\
     \glt `I have no problem with that (…but I would like to have my own bedroom).'
\z
\z

In sentences like \REF{ex:catasso:9} and \REF{ex:catasso:10}, the fronted phrase arguably moves into the specifier of a left-peripheral projection in which contrastiveness is encoded (a position that has already been independently proposed for German by \citealt{Frey2004}, who calls the corresponding projection “KontrP”). It is suggested here, following \citet{Catasso2015, Catasso2021Large}, that the post-initial marker (\textit{aber} in the second conjunct in \REF{ex:catasso:9}) is not an XP, but a lexicalization of the head position of this projection, in whose specifier the contrastive topic surfaces at PF. This means that it is base-generated in the position in which it is pronounced. Note that the post-initial particle is fully optional, cannot occupy the left periphery independently\footnote{This can be taken to show that “this” {\textit{aber}}  is not to be categorized as the head of an AdvP. This would imply not only that this element could appear in the left periphery independently of a fully\hyp fledged phrase, but also that it could be the constituent triggering V2 in a main clause. However, this is not the case:

\ea[*]{
   \gll … aber habe                  ich              den                   Hans     nicht gesehen.\\
        {} but \textsc{aux.prs.1sg}  I-\textsc{nom}   the-\textsc{acc.sg}  {Hans}  \textsc{neg} \textsc{ptcp}-see-\textsc{ptcp}\\
    \glt (int.:) `…but I did not see Hans.”'}
 \z

In some approaches (e.g. \citealt{Weisser2022}), it is proposed that items like \emph{aber} in \REF{ex:catasso:9} are coordinating conjunctions surfacing in second position as a result of the application of a prosodic displacement rule. However, while this would in principle be possible for \emph{aber}  (which can also be an adversative coordinator meaning ‘but’), the class of elements that can perform this function in post-initial position includes words such as – among many others – \emph{hingegen}  ‘instead’ and \emph{freilich}  ‘indeed’, which cannot be coordinating conjunctions.} and is only licensed by the presence of a maximal projection to its left. A provisional derivation for a structure like \REF{ex:catasso:9} is given in \REF{ex:catasso:11}. The contrastive projection in the left periphery is neutrally labeled “YP” here:

\ea%11
    \label{ex:catasso:11}\relax       
    [\textsubscript{ForceP}   [\textsubscript{YP} [\textit{den Hans}]\textsubscript{j} [\textsubscript{Y°}  \textit{aber} … [\textsubscript{FinP}  t\textsubscript{j} [\textsubscript{Fin°} \textit{habe}\textsubscript{i} [\textsubscript{IP/VP}  t\textsubscript{j}  \textit{ich nicht gesehen}  t\textsubscript{i} ]]]]]]
\z

Also in this case, the base generation of the marker in the left periphery prevents a violation of the structural V2 constraint.\footnote{{The syntactic derivation of sentence structures featuring post-initial particles has remained a contentious topic and continues to ignite passionate debate within the syntactic community to this day. Some works (e.g. \citealt{VolodinaWeiß2010} propose a derivation in which the post-initial marker is base-generated in the middle field and raised to Spec,CP together with the fronted phrase. The approach advocated here seems to be preferable for a number of reasons. One of the problems related to merging an element like} {{aber}}  {in \REF{ex:catasso:9} in the middle field concerns the categorial status of the constituent in first clause position. In \citegen{Chomsky2013} theory of labeling, when a head and a phrase merge, the head projects, i.e., provides the label for the resulting object. In order to achieve the relevant word order (XP >} {\textit{aber}} {; *}{\textit{aber}}  {> XP), the phrase interpreted contrastively is assumed to move into the specifier of} {\textit{aber}} {, the latter lexicalizing a PrtP (Particle Phrase). This would entail that the pronounced linearization would result from movement of a deficient phrase into the left periphery, which is not a licit operation in German. For further arguments in favor of a base generation of the post-initial particle within the CP, cf. \citet{Catasso2015}.}}

\subsubsection{Co-occurrence of dislocation and post-initial particles}
\label{sec:catasso:4.1.3}

The two non-V2 patterns illustrated in \sectref{sec:catasso:4.1} can co-occur in the same sentence. Given the existence of a bottleneck effect and the derivational mechanisms proposed above, the corresponding operations do not violate the structural V2 constraint preventing more than one XP from entering the CP domain. In a structure like \REF{ex:catasso:12a}, the direct object \textit{den Hans} is first-merged in the relevant middle-field position and cyclically moved to Spec,FinP, where it leaves a trace spelled out as the pronominal correlate, to derive the bottleneck effect and – at the end of the derivation – to Spec,YP (YP being the contrastive projection discussed above), in whose head the post-initial particle is generated.

\ea%12
    \label{ex:catasso:12}        
    \ea \label{ex:catasso:12a}  
        \gll Den Hans aber, den habe ich nicht gesehen.\\
             the-\textsc{acc.sg} Hans however \textsc{p.res.acc.sg} \textsc{aux.prs.1sg} I-\textsc{nom} \textsc{neg} \textsc{ptcp}{}-see-\textsc{ptcp}  \\
    \ex \label{ex:catasso:12b}\relax  
        [\textsubscript{ForceP} [\textsubscript{YP} [\textit{den Hans}]\textsubscript{j} [\textsubscript{Y}°  \textit{aber}  … [\textsubscript{FinP}  [\textit{den}]\textsubscript{j} [\textsubscript{Fin°} \textit{habe} [\textsubscript{IP/VP} \textit{ich} t\textsubscript{j} \textit{nicht gesehen}  t\textsubscript{i} ]]]]]]
    \z
\z

At this point, a further observation should be made. So far, in the discussion of the derivation of left dislocation and post-initial particles, it has been remarked that contrastively interpreted constituents target Spec,YP as their landing site, i.e. that they occur in that position at PF. A closer look at possible constructions featuring such elements, however, suggests that FinP and YP are not the only left-peripheral projections involved in the derivation. Consider, for instance, the following sentence:

\ea%13
    \label{ex:catasso:13}
    Context: Der Traum, dass Gott auf die Erde komme, ist wahr geworden in Jesus von Nazareth, dem Kind in der Krippe, dem Wanderprediger, Heiler und Helfer. (`The dream of God coming unto the Earth came true in Jesus of Nazareth, the child in the crib, the evangelist, healer and helper.')
    \gll Das Ende aber dieser Geschichte war kein Traum mehr. Das Ende war am Kreuz.\\
         the-\textsc{nom.sg} end however this-\textsc{gen.sg} story be-\textsc{pst.3sg} no-\textsc{nom.sg} dream more the-\textsc{nom.sg} end be-\textsc{pst.3sg} at-the-\textsc{dat.sg} cross\\
    \glt `The end of this story, however, was not a dream anymore. The end was on the cross.' (\textit{johanniter.de}, March 09\textsuperscript{th}, 2021)
\z

In \REF{ex:catasso:13}, two phases of a story (the beginning, which is presented in the context, and the end in the relevant clause) are contrasted to each other. The contrastive topic opening the second conjunct (\textit{das Ende dieser Geschichte} ‘the end of this story’) appears in a split configuration in which the post-initial particle \textit{aber} intervenes between the DP and its genitival modifier.\footnote{Note that the same configuration can be obtained with any other NP/DP modifier or complement, as exemplarily illustrated in the following examples:

\ea {(PP modifier)}\\
 \gll Die Mutter aber von Hans war nicht damit einverstanden.\\
     {the-\textsc{nom.sg}}  mother however of Hans  {be-\textsc{pst.3sg}} \textsc{neg} {with-that} {happy}\\
  \glt `Hans’ mother, however, was not happy with that.'

\ex {(CP modifier)}\\
  \gll Die Frau aber, die Hans am meisten liebte, war nicht damit einverstanden.\\
       {the-\textsc{nom.sg}}  woman however  {\textsc{rel.acc.sg}}  Hans at-the-\textsc{dat.sg}  {more-\textsc{sup.dat.sg}}  {love-\textsc{pst.3sg}} be-\textsc{pst.3sg} \textsc{neg} {with-that} {happy}\\
  \glt `The woman whom Hans loved most, however, was not happy with that.”'  
  
\ex (CP complement)\\
 \gll Die Tatsache aber, dass Hans nicht damit einverstanden war, gab uns Hoffnung.\\
      {the-\textsc{nom.sg}}  fact however that Hans  {\textsc{neg}} {with-that} happy be-\textsc{pst.3sg} give-\textsc{pst.3sg} we-\textsc{dat}  {hope}\\
 \glt `The fact that Hans was not happy with that, however, gave us hope.'
\z} 
Since this sequence occurs preverbally and along the lines of what has been said above, it is proposed that the constituent realizing the contrastive topic moves through a further CP-specifier before reaching its landing site in Spec,YP. The corresponding projection is the one in which the constituent acquires a [+\textsc{top}] feature qualifying it as an aboutness topic. Further movement into the higher specifier of YP ensures that the constituent checks an additional [+\textsc{contr}] feature that enables its reading as a contrastive topic. From this, it follows that contrastive topics are not information-structural simplexes, but result from the addition of two features cyclically acquired in the syntax. The projection encoding the [+\textsc{top}] feature is the position in which aboutness topics (i.e., constituents that refer to “what the sentence is about” and are not interpreted contrastively (\citealt{Reinhart1981, Frey2000,Frey2007, Krifka2006, FrascarelliHinterhölzl2007})) appear at PF, that is, their landing site in a sentence like \REF{ex:catasso:14a} (assuming the relevant pragmatic reading). It corresponds to one of the TopPs assumed by \citet{Rizzi1997} in the clausal left periphery. Of course, since genuine V2 languages only allow for one XP to enter the CP area, (some of) the different TopPs that are simultaneously activatable in languages like Italian are conflated into one projection in German.\footnote{This very assumption derives from the premise that in languages like Italian, the same type of topic can apparently be realized in more than one left-peripheral position. This arguably correlates with the fact that Italian allows for multiple movement from the IP/VP area, and the instantiations of the possible configurations exhibiting more than one, say, aboutness topic in the CP domain depend on a conspiracy of intricate principles at the interface between syntax and information structure that I will not go into here for reasons of space (for an overview, see \citealt{RizziBocci2017}). Cf., e.g., the following examples, which represent syntactic minimal pairs of the same sentence (the capitalization of one segment of the non-finite verb signals focus stress):

\ea \gll [A Gianni,] perché, [la macchina] gliela volete regaLAre?\\
         to Gianni why the car  {\textsc{cl.dat.3sg.m-cl.acc.3sg.f}}  {want-\textsc{prs.2pl}}  {give-\textsc{inf}}\\
\ex \gll Perché [a Gianni,] [la macchina] gliela volete regaLAre?\\
         why to Gianni the car {\textsc{cl.dat.3sg.m-cl.acc.3sg.f}}  {want-\textsc{prs.2pl}}  {give-\textsc{inf}}\\
\ex \gll Perché [la macchina,] [a Gianni,] gliela volete regaLAre?\\
         why the car to Gianni  \textsc{cl.dat.3sg.m-cl.acc.3sg.f} {want-\textsc{prs.2pl}}  {give-\textsc{inf}}\\ 
    \glt `Why do you want to give the car to Gianni as a gift (and not, for instance, sell it to him)?'
\z
(adapted from: \citealt{RizziBocci2017}: 5).
This type of configuration is not present in German, which seems to feature only one genuine aboutness\hyp topic projection in its left periphery.} A simplified representation of the corresponding derivation, in which the intermediate CP projection encoding aboutness topichood is neutrally labeled “ZP”, is given in \REF{ex:catasso:14b}:

\ea%14
    \label{ex:catasso:14}
\ea \label{ex:catasso:14a}\emph{Context:} The interlocutor wants to know something about Hans.\\
    \gll Hans hat letztes Jahr seinen ersten Roman veröffentlicht.\\
         Hans \textsc{aux.prs.3sg} last-\textsc{acc.sg} year his-\textsc{acc.sg} first-\textsc{acc.sg} novel publish-\textsc{ptcp}\\
    \glt `Hans published his first novel last year.'
\ex \label{ex:catasso:14b} \relax [\textsubscript{ForceP} [\textsubscript{YP} [\textsubscript{Y}°] [\textsubscript{ZP} \textit{Hans}\textsubscript{j} [\textsubscript{Z°}] [\textsubscript{FinP}  [\textsubscript{Fin°} \textit{hat}\textsubscript{i} [\textsubscript{IP/VP} \textit{letztes Jahr} t\textsubscript{j} \textit{seinen ersten Roman veröffentlicht}  t\textsubscript{i} ]]]]]]
\z
\z

\largerpage
With respect to more complex structures like \REF{ex:catasso:13}, the relevant assumption made here is that, given a modified or complemented XP moved to Spec,ZP, if this constituent is interpreted contrastively in its entirety, it can further move to Spec,YP (\ref{ex:catasso:15a}--\ref{ex:catasso:16a});{\interfootnotelinepenalty=10000\footnote{Note that this configuration does not instantiate a violation of Criterial in the strict sense (\citealt{Rizzi2006,Rizzi2010}), since the further leftward movement of the XP is driven by its need to check a (contrastive) feature that is different from the (topical) one that has already been acquired in Spec,ZP. Indeed, this movement operation only takes place if a topic features an additional contrastive feature.}} if, instead, only one part of it (systematically corresponding to the left edge in German) receives a contrastive reading, this segment can move to Spec,YP independently of its modifier\slash complement, which can be left behind in Spec,ZP (\ref{ex:catasso:15b}--\ref{ex:catasso:16b}). If the latter is the case, the corresponding linearization can be obtained by extraposing the modifier/complement, which is obligatorily non-stressed, DP-internally. As expected, post-initial particles can only be spelled out in one position, namely the head of YP:

\ea%15
\label{ex:catasso:15}
\ea  \label{ex:catasso:15a}
\gll Das Ende dieser Geschichte aber war kein Traum mehr.\\
     the-\textsc{nom.sg} end this\textsc{{}-gen.sg} story however be-\textsc{pst.3sg} no-\textsc{nom.sg} dream more\\
     
\ex \label{ex:catasso:15b}
\gll Das Ende aber dieser Geschichte war kein Traum mehr.\\
     the-\textsc{nom.sg} end however this-\textsc{gen.sg} story be-\textsc{pst.3sg} no-\textsc{nom.sg} dream more\\
\glt `The end of this story, however, was not a dream anymore.'
\z
\z

\ea%16
    \label{ex:catasso:16}        
    \ea \label{ex:catasso:16a}\relax   [\textsubscript{ForceP} [\textsubscript{YP} [\textit{Das Ende dieser Geschichte}]\textsubscript{j} [\textsubscript{Y}° \textit{aber}] [\textsubscript{ZP} t\textsubscript{j} [\textsubscript{Z°}] [\textsubscript{FinP} t\textsubscript{j} [\textsubscript{Fin°} \textit{war}\textsubscript{i} [\textsubscript{IP/VP} t\textsubscript{j} \textit{kein Traum mehr} t\textsubscript{i} ]]]]]]
    \ex \label{ex:catasso:16b}\relax [\textsubscript{ForceP} [\textsubscript{YP} [\textit{Das Ende}]\textsubscript{x} [\textsubscript{Y}° \textit{aber}] [\textsubscript{ZP} [[t\textsubscript{x}] \textit{dieser Geschichte}]\textsubscript{j} [\textsubscript{Z°}] [\textsubscript{FinP} t\textsubscript{j} [\textsubscript{Fin°} \textit{war}\textsubscript{i} [\textsubscript{IP/VP} t\textsubscript{j} \textit{kein Traum mehr} t\textsubscript{i} ]]]]]]
\z
\z

Also in this case, the trace left in Spec,FinP by the constituent \textit{das Ende dieser Geschichte} on its way to Spec,YP can optionally have a phonetic form, thereby generating the structure in \REF{ex:catasso:17a}, whose syntactic representation is given in \REF{ex:catasso:17b}. For the purposes of the present discussion, in \REF{ex:catasso:17} only the version of the sentence with the split nominal expression is given. It is tacitly understood that the more unmarked structure with an unsplit DP like in \REF{ex:catasso:15a} is always possible:\largerpage

\ea%17
    \label{ex:catasso:17}
    \ea\label{ex:catasso:17a} Das Ende aber dieser Geschichte, das war kein Traum mehr.
    \ex\label{ex:catasso:17b}\relax  [\textsubscript{ForceP} [\textsubscript{YP} [\textit{Das Ende}]\textsubscript{x} [\textsubscript{Y}° \textit{aber}] [\textsubscript{ZP} [[t\textsubscript{x}] \textit{dieser Geschichte}]\textsubscript{j} [\textsubscript{Z°}] [\textsubscript{FinP} [\textit{das}]\textsubscript{j} [\textsubscript{Fin°} \textit{war}\textsubscript{i} [\textsubscript{IP/VP} t\textsubscript{j} \textit{kein Traum mehr}  t\textsubscript{i} ]]]]]]
    \z
\z

Although the resulting linearization is rather complex, it does not involve any violation of the V2 constraint as it was characterized above. The analysis advocated here is not only that the syntactic acquisition of the information-structural feature of contrastiveness is independent of topichood (an idea which is already found in \citealt{Frey2004}, although with quite a different implementation), but also that the representation of the left periphery in German is necessarily bound to the technicalities of the cartographic model: different information\hyp structural features such as [+\textsc{top}] and [+\textsc{contr}] are checked in different Split-CP positions \REF{ex:catasso:18}, and the inner complexity of the left periphery of German is both confirmed by the empirical data and compatible with the notion of a well-behaved V2 language:

\ea%18
    \label{ex:catasso:18}
    \relax [\textsubscript{ForceP}   [\textsubscript{YP} …   [\textsubscript{ZP} …   [\textsubscript{FinP}   [\textsubscript{IP/VP} ]]]]]      
           \textsubscript{[+}\textsc{\textsubscript{contr}}\textsubscript{]   [+}\textsc{\textsubscript{top}}\textsubscript{]}
\z

This featural split of contrastiveness and other grammatical or information-structural attributes is also visible if the fronted constituent belongs to a non-aboutness-topical category, that is, if it is, for instance, a frame-setter \REF{ex:catasso:19a}, an interrogative phrase \REF{ex:catasso:19b} or a non-topic (here: a negated expression) \REF{ex:catasso:19c}. Note that in all three examples, the fronted constituents consists of an XP (a PP in (\ref{ex:catasso:19a}--\ref{ex:catasso:19b}), a pronoun in \REF{ex:catasso:19c}) modified by an attribute (a relative clause in \REF{ex:catasso:19a}, a genitival phrase in \REF{ex:catasso:19b}, a PP in \REF{ex:catasso:19c}) and that the post-initial particle intervenes between these two components. This gives rise to the very same configuration as those addressed above.\largerpage[2]

\ea%19
    \label{ex:catasso:19}
    \ea \label{ex:catasso:19a} (\textit{lr-online.de} (news portal), Jun. 17\textsuperscript{th}, 2005)\\
    \gll Am ersten Tag aber, an dem die „Hängematte“geschlossen war, standen sie wieder vor der Tür.\\
         on-the-\textsc{dat.sg} first-\textsc{dat.sg} day however on \textsc{rel.dat.sg} the-\textsc{nom.sg} \textit{Hängematte} \textsc{ptcp}{}-close-\textsc{ptcp} be-\textsc{pst.3sg} stand-\textsc{pst.3pl} they-\textsc{nom} again in-front-of the-\textsc{dat.sg} door\\
    \glt `However, on the first day that the \textit{Hängematte} was closed, they were already back in front of the door.'


\ex  \label{ex:catasso:19b}
    Context: Das Urteil: Schavan hat getäuscht. (`The verdict: Schavan has cheated (in her PhD).')\\
    \gll In welchem Kapitel aber ihrer Dissertation hat Schavan nun eindeutig plagiiert?\\
         in which chapter however her.\textsc{gen.fem} dissertation \textsc{aux.prs.3sg} Schavan now clearly plagiarize-\textsc{ptcp}\\
    \glt `But which parts of her PhD dissertation are unequivocally plagiarized?'
    (adapted from: \textit{handelsblatt.com} (news portal), Oct. 15\textsuperscript{th}, 2012){\interfootnotelinepenalty=10000\footnote{{The original text features a simple interrogative adverb:} \textit{Wo aber hat Schavan nun eindeutig plagiiert?}  {(lit. `where however has Schavan now unequivocally plagiarized'). Irrespective of whether the} {{wh}} {{}-constituent is split, one may wonder how an interrogative element can receive a contrastive interpretation and how this reading differs from its standard interpretation. For contexts like \REF{ex:catasso:19b}, it can be assumed that the question of the “where” (‘in which chapter of her dissertation?’) is extracted out of a set of alternatives that are potentially exhausted by all possible} {{wh}} {{}-elements questioning different aspects of the content introduced in the utterance. In this case, the pre-context provides one further piece of information that is contrasted to the “where” in the second clause, namely the “what” (“Schavan has plagiarized”).}}}

\ex \label{ex:catasso:19c}Context: Wir kennen wohl die Psychologie des Kindes aus Büchern und haben seine Sprache belauschen gelernt, … (`We arguably know the child’s psychology from books, and we have learnt how to eavesdrop on their language…')\\
 \gll …keiner aber von uns kennt die Physiognomie des Kindes.\\
      none-\textsc{nom.sg} however of we-\textsc{dat} know-\textsc{prs.3sg} the-\textsc{acc.sg} physiognomy the-\textsc{gen.sg} child-\textsc{gen.sg}\\
 \glt `…none of us, however, knows the child’s physiognomy.'   (Oskar Kalbus 1935: 130, scientific monograph)
\z
\z

If the analysis proposed for aboutness and contrastive topics is on the right track and in light of the fact that other categories can appear in very similar constructions, it is compelling to assume that the derivation of the patterns in \REF{ex:catasso:19} occurs cyclically, i.e., that the left periphery of the clause contains dedicated projections in which the corresponding features are acquired. For a sentence like \REF{ex:catasso:19a}, for instance, the analysis in \REF{ex:catasso:20} follows: the frame is base-generated in the middle field, then moved to Spec,FinP (bottleneck effect) and to Spec,WP (here: the clause-internal projection assumed to encode frame\hyp setting features). This projection is similar to (or, mutatis mutandis, the same as) the ModP assumed in \citet{Rizzi2004Structures} (also cf. \citealt{Samo2022Criterial}, and in particular the discussion in \citealt{Samo2022Moved}, where a FrameP and a ModP are considered separately). In the final phase of the derivation, only the segment \textit{am ersten Tag}, which is the contrastively interpreted part of the constituent, is raised to Spec,YP. The particle \textit{aber} is merged in the head of YP:

\ea%20
    \label{ex:catasso:20}\relax    
    [\textsubscript{ForceP} [\textsubscript{YP[+}\textsc{\textsubscript{contr}}\textsubscript{]} [\textit{Am ersten Tag}]\textsubscript{x} [\textsubscript{Y°} \textit{aber}] … [\textsubscript{WP}\textsc{\textsubscript{[+frame]}} [[t\textsubscript{x}] \textit{an dem die „Hängematte“ geschlossen war}]\textsubscript{j}… [\textsubscript{FinP} t\textsubscript{j} [\textsubscript{Fin°} \textit{standen}\textsubscript{i}] [\textsubscript{IP/VP} t\textsubscript{j} \textit{sie wieder vor der Tür} t\textsubscript{i} ]]]]]
\z

It comes as little surprise that, just as in \REF{ex:catasso:17}, the trace left by the frame setter in Spec,FinP can optionally be pronounced as an adverbial resumptive even in the complex construction in \REF{ex:catasso:19a} and \REF{ex:catasso:20}:\largerpage[2]

\ea%21
    \label{ex:catasso:21}
    \gll Am ersten Tag aber, an dem die „Hängematte“ geschlossen war, da standen sie wieder vor der Tür.\\
         on-the-\textsc{dat.sg} first-\textsc{dat.sg} day however on \textsc{rel.dat.sg} the-\textsc{nom.sg} \textit{Hängematte} \textsc{ptcp}{}-close-\textsc{ptcp} be-\textsc{pst.3sg} \textsc{a.res} stand-\textsc{pst.3pl} they-\textsc{nom} again in-front-of the-\textsc{dat.sg} door\\
     \glt `However, on the first day that the \textit{Hängematte} was closed, they were already back in front of the door.'
\z

At this point, at least two questions arise: (i) in which relative order do the projections encoding the different grammatical and information-structural features (contrastiveness, topichood, interrogativity) occur in the extended left periphery?; and (ii) which position do non-topics target in the left periphery? These two issues are addressed in the following sections.

\subsection{Derivation of non-dislocated frames}\label{sec:catasso:4.2}

In the previous section, it was argued that frame(-like) phrases surfacing in a left-peripheral specifier that can be resumed by a preverbal adverbial element are base-generated in the middle field and cyclically moved into the left edge of the clause and that there is a dedicated position in which the feature [+\textsc{frame}] is acquired. The facts presented above entail that the canonical frame-setter position (i.e., the site in which non-contrastively interpreted frames are pronounced, as in \REF{ex:catasso:22}) must be positioned in a lower specifier than the one encoding contrastiveness, since WP-to-YP movement is supposed to be implemented only if the frame setter receives a contrastive reading:

\ea%22
    \label{ex:catasso:22}
    \ea \label{ex:catasso:22a} 
        \gll Gestern\textsubscript{[}\textsc{\textsubscript{non-contr}}\textsubscript{]} habe ich Hans getroffen.\\
             yesterday \textsc{aux.prs.1sg} I-\textsc{nom} Hans \textsc{ptcp-}meet-\textsc{ptcp}\\
         \glt `Yesterday I met Hans.'
    \ex \label{ex:catasso:22b}\relax
         [\textsubscript{ForceP} [\textsubscript{YP} … [\textsubscript{WP} \textit{gestern}\textsubscript{j} [\textsubscript{FinP} t\textsubscript{j} [\textsubscript{IP/VP} … t\textsubscript{j} ]]]]]
    \z
\z

However, it still has to be clarified where this projection exactly occurs, viz., how it interacts with the other categories addressed above.

\subsubsection{Topics and frames}\label{sec:catasso:4.2.1}

As shown below, moving a frame-setting constituent into the left periphery from its base-generation site in the IP/VP area is not the only possible option. In cases like \REF{ex:catasso:4a}, \REF{ex:catasso:4c} and \REF{ex:catasso:4d} (replicated in \REF{ex:catasso:23} for the reader’s convenience), this XP seems to be base-generated in the CP area.

\ea%23
    \label{ex:catasso:23}          
\ea \label{ex:catasso:23a}
    \gll  Wenn sie              schon so … viel Geld gewonnen haben, warum spendieren sie nicht wenigstens eine Runde?\\
          if they-\textsc{nom} already so {} much money \textsc{ptcp}{}-win-\textsc{ptcp} \textsc{aux.prs.3pl} why pay-\textsc{prs.3pl} they-\textsc{nom} \textsc{neg} at-least one-\textsc{acc.sg} round\\
    \glt `If they have won so much money, why don’t they at least pay for a round?'
    
\ex \label{ex:catasso:23b} 
    \gll Heute, die Goethe-Institute in Indien heißen alle Max Müller Bhavan.\\
         today the\textsc{{}-nom.pl} Goethe-institute-\textsc{nom.pl} in India be-called-\textsc{prs.3pl} all-\textsc{nom-pl} Max Müller Bhavan\\
    \glt `Today, the Goethe institutes in India are all called “Max Müller Bhavan”.'
\ex \label{ex:catasso:23c} 
    \gll Piachi, als ihm der Stab gebrochen war, verweigerte sich hartnäckig der Absolution.\\
         Piachi when he-\textsc{dat.sg} the-\textsc{nom.sg} staff \textsc{ptcp}{}-break-\textsc{ptcp} \textsc{aux.pst.3sg} refuse-\textsc{pst.3sg} \textsc{refl.acc.3sg} stubbornly the-\textsc{dat.sg} absolution\\
    \glt `Piachi, when the staff was broken over him, stubbornly refused absolution.'
\z
\z

In \REF{ex:catasso:23a} and \REF{ex:catasso:23b} (respectively \REF{ex:catasso:4a} and \REF{ex:catasso:4c} in \sectref{sec:catasso:3}), the frame linearly precedes a \textit{wh}{}-interrogative (\textit{warum}) or an aboutness topic (\textit{die Goethe-Institute}). These two examples feature different interpretations of the constituent in first clause position: in \REF{ex:catasso:23a}, the \textit{wenn}{}-clause introduces a fully presupposed content (“they have won a lot of money”), while \textit{heute} in \REF{ex:catasso:23b} has a temporal reference that modifies the predicate \textsc{be} \textsc{called} and seems to imply a contrastive meaning. Since it is intuitively clear that in these two sentences it is not the frame but the XP in second position that interacts with the V2 syntax of the clause, it is compelling to postulate that the \textit{wenn}{}-clause and the adverb \textit{heute} are first-merged in their surface position.\footnote{An anonymous reviewer raises the point about how one can justify not treating both the utterance-initial topic and the wh-phrase in structures like (i) (discussed in the literature, e.g., by \citealt{BoeckxGrohmann2005, Ott2014, Samo2019}) as both moved to the CP layer, given that the former is case-marked:  

\ea (Adapted from \citealt{Samo2019}: 153, reported as “Hubert Haider, p.c.”)\\
    \gll Seinen                 neuen               Aufsatz, wann sollen wir den / ihn lesen?\\
         his-\textsc{acc.sg}  {new-\textsc{acc.sg}}  article when shall\textsc{.prs.3sg}  we-\textsc{nom} \textsc{p.res.acc.sg} / {he-\textsc{acc}} read-\textsc{inf}\\
    \glt ‘His new article, when shall we read it?’ 
\z

It should be considered that the assumption that two XPs can be independently raised to the left periphery in a given structure would amount to the undesirable implication that this should also be the case in other configurations, which clearly does not align with the empirical observations generally made for present-day German. For structures like (i), I adopt \citegen{Samo2019} idea of what he calls “Case Hanging Topics”, namely hanging vPs first-merged above ForceP and replicating the structure of the clause-internal vP that exhibit a portion deleted at PF. For (i) above, Samo assumes the presence of a partially elliptical vP in the outer left periphery interpreted as ‘as for reading his new article’ and proposes the analysis in (ii). The interested reader is referred to \citet[146ff.]{Samo2019} for the technical details of the derivation:

\ea \citep[156]{Samo2019}\\
  \gll [{\textsubscript{Hanging Topic}} Seinen               neuen                Aufsatz lesen,          [\textsubscript{ForceP}  {…} wann sollen                  wir                  \textit{den}     / ihn              lesen]]?\\
       {}                               his-\textsc{acc.sg}  new-\textsc{acc.sg}  article read\textsc{-inf} {} {}                      when shall\textsc{.prs.3sg}  we-\textsc{nom} \textsc{p.res.acc.sg} / he-\textsc{acc}  read-\textsc{inf}\\
  \glt   
\z
\vskip\baselineskip
} This is also corroborated by the fact that an obligatory prosodic pause intervenes between them and the constituent that linearly follows them in preverbal position. Given the different interpretations of the relevant constituents, a reasonable assumption would be that the \textit{wenn}{}-clause and \textit{heute} are base-generated in different specialized projections either extra-sententially (KP and JP in \REF{ex:catasso:24}) or very high CP-internal positions (GP and HP in \REF{ex:catasso:24}).\footnote{{Note that post\hyp syntactic merger is starkly dispreferred, which is why this construction occurs less frequently than other non-V2 patterns. It is to be noted, however, that the sequence Adverbial > Argument > Vfin is not limited to varieties like Kiezdeutsch \citep{Wiese2012}, but is also present in standard spoken German, as shown, e.g., in the empirical studies by \citet{Breitbarth2022, Breitbarth2023}.}} This assumption is in line with \citegen{Breitbarth2022} analysis, which is grounded, inter alia, on the empirically observable absence of reconstruction effects and deep embedding in the sentence structure.\footnote{{Cf. \citet[21]{Breitbarth2022}: “The absence of reconstruction and deep embedding indicates that there is no trace in the subsequent clause from which the prepositioning could have originated, which... only allows for the possibility that the clause-initial adverbial modification must have been base-generated outside the subsequent clause,” (translation mine).}} For reasons of space, I leave the technical details of this intricate differentiation and the corresponding typology for future research, thereby placing the stress on the idea that the surface position of both types of XPs does not result from movement:\largerpage

\ea%24
    \label{ex:catasso:24}\relax
    [\textsubscript{KP} (\textit{wenn}{}-clause) [\textsubscript{JP} (\textit{heute}) [\textsubscript{ForceP} [\textsubscript{GP} (\textit{wenn}{}-clause) [\textsubscript{HP} (\textit{heute}) […] [\textsubscript{FinP} [\textsubscript{IP/VP}]]]]]
\z

In the pattern illustrated in \REF{ex:catasso:23c}, instead, the frame in second position modifies the predicate by providing a temporal reference for its interpretation and can \textit{under no circumstances} be interpreted contrastively. This aspect is crucial in our discussion, since it excludes that the temporal clause might be positioned in Spec,YP in this sentence (whether its position is supposed to result from movement or not). I propose that in this case, the frame setter is directly base\hyp generated in the specifier that may otherwise host either a moved frame interpreted non\hyp contrastively or the trace of a frame further raised to Spec,YP to receive a contrastive reading. The two assumptions that (i) the position of this XP must be lower than that of topics; and that (ii) this constituent must be base-generated can be tested. In fact, such frames can occur to the right of a contrastive topic (\REF{ex:catasso:25a}, where the context and the particle in post\hyp initial position concur to disambiguate the interpretation of the DP in first position) or of an aboutness topic (\REF{ex:catasso:25b}, in this case assuming a corresponding context). What is more, in a complex non-V2 structure like this, the immediate preverbal position (in the present approach: Spec,FinP) can host a pronominal resumptive replicating the features of the DP (here: \textit{der} ‘he\textsc{.nom.sg}’), but not an adverbial resumptive (here: \textit{dann} ‘then’) taking up the reference of the temporal CP \REF{ex:catasso:25c} – irrespective of whether the DP in first position has a contrastive reading. This can be taken to show that the moved element in this linearization must be the DP and cannot be the adverbial clause: 

\ea%25
    \label{ex:catasso:25}
    \ea \label{ex:catasso:25a}  
        Context: Im Märchen kommt der unbeschwerte Jüngstgeborene, der von den Älteren nicht ganz voll genommen wird und in der Familie leer ausgeht, durch sein gutes Herz draußen in der Welt zum Erfolg. (`In fairytales, the lighthearted youngest born, who is not taken seriously by their older siblings and is left out in the cold by their family, is successful in the external world by means of his good behavior.')\\
    \gll Der Erstgeborene hingegen, wenn er den Weg des Glücklichen in krampfhaftem Ehrgeiz nachzuahmen versucht, erreicht das gerade Gegenteil.\\
         the-\textsc{nom.sg} first-born-\textsc{nom.sg} instead if he-\textsc{nom} the-\textsc{acc.sg} way the-\textsc{gen.sg} happy-\textsc{gen.sg} in frantic-\textsc{dat.sg} eagerness to-emulate-\textsc{inf} try-\textsc{prs.3sg} reach-\textsc{prs.3sg} the-\textsc{acc.sg} exact-\textsc{acc.sg} opposite\\
    \glt `When the firstborn instead tries to emulate the path of the happy one with frantic eagerness, he gets the exact opposite.' (DeReKo, N95/MAI.19855)
    \ex\label{ex:catasso:25b}
    \gll …aber Kinder\textsubscript{[+aboutness]}, wenn sie es von ihren Eltern anders vorgelebt bekommen, können da sehr grausam untereinander sein.\\
        but kid-\textsc{nom.pl} if they-\textsc{nom} it-\textsc{acc} from their-\textsc{dat.pl} parent-\textsc{dat.pl} differently \textsc{v.prt-ptcp-}live-\textsc{ptcp} receive-\textsc{prs.3pl} can-\textsc{prs.3pl} there very cruel among-each-other be-\textsc{inf}\\
    \glt `…but children can be cruel to each other if their parents set a different example for them.' (baby.gofeminin.de (online forum), Feb. 23\textsuperscript{rd}, 2023) 
\ex \label{ex:catasso:25c} Der Erstgeborene (hingegen), wenn er … nachzuahmen versucht, \textsuperscript{OK}der/*dann erreicht das gerade Gegenteil.
\glt `When the firstborn (instead) tries to emulate the path of the happy one with frantic eagerness, he gets the exact opposite.'
\z
\z

The data suggest that the projection labeled “WP” or “canonical frame position” above and exemplified in (\ref{ex:catasso:20}--\ref{ex:catasso:22}) and \REF{ex:catasso:25} must be positioned to the right of both YP and ZP. This projection serves different functions, all linked to framing: run-of-the-mill frames are generated in the middle field and moved to Spec,WP, where they appear at PF unless they are contrastive, in which case they are further moved to Spec,YP. Other types of frames are extrasentential or merged post-syntactically in a very high left-peripheral position. A third strategy consists in generating a frame directly in Spec,WP. In the latter two cases, an adverbial resumptive taking up the reference of the frame cannot be spelled out in Spec,FinP. The derivation of \REF{ex:catasso:25c} is sketched in \REF{ex:catasso:26}:

\ea%26
    \label{ex:catasso:26}\relax
    [\textsubscript{ForceP} [\textsubscript{YP} [\textit{Der Erstgeborene}]\textsubscript{j} [\textsubscript{Y°} \textit{hingegen,} [\textsubscript{ZP} t\textsubscript{j} [\textsubscript{WP} \textit{wenn er … nachzuahmen versucht,} [\textsubscript{FinP} [\textit{der}]\textsubscript{j} [\textsubscript{Fin°} \textit{erreicht}\textsubscript{i} [\textsubscript{IP/VP} t\textsubscript{j} \textit{das gerade Gegenteil} t\textsubscript{i}]]]]]]]]
\z

In the discussion so far, only contrastive and aboutness topics have been mentioned. Familiar topics, i.e., given, unstressed, often pronominally realized constituents expressing topic continuity (\citealt{Pesetsky1987, Givón1983, FrascarelliHinterhölzl2007}), occupy a lower CP position. It is proposed here that familiar topics, unlike contrastive topics, do not need to acquire an aboutness\hyp topical feature before moving to a dedicated specifier encoding a putative [+\textsc{familiar}] feature. After all, familiar topics are nothing but the spell-out of topic continuity. In non-pro-drop languages like German, such topics are bundles of features phonetically spelled out as weak copies of a topic present in the discourse (viz., in the pre-context). In pro-drop languages, they are generally realized as \textit{pro}s. Therefore, they can be reasonably assumed to move from the middle field to Spec,FinP to derive the bottleneck effect and not to undergo any further leftward movement operations. This is independent of the status of the XPs positioned to their left in the CP area. What is crucial with respect to the present discussion, however, is that the sequence “familiar topic > (non-contrastive) frame setter” is not possible. If a (pronominally realized) topic appears to the left of a frame, it is obligatorily interpreted as a non-familiar, preferably as a contrastive topic, as in \REF{ex:catasso:27}. This is arguably due to the fact that contrastive, unlike aboutness topics, are associated with a recognizable prosodic contour that disambiguates their reading. Of course, the left-peripheral frame in second position need not be an adverbial clause, but can be also realized by other phrasal categories (for instance by a PP, as in \REF{ex:catasso:27b}):

\ea%27
    \label{ex:catasso:27}
    \ea \label{ex:catasso:27a} 
     Context: Auf Sam wird in diesem Buch herumgetreten bis zum Gehtnichtmehr, sei dies von den Vampiren, vom Rudel oder Alan. (`Sam is heavily trampled on in this book – by the vampires, by the troop, and by Alan.')\\
    \gll Aber sie, sobald man wieder nett zu ihr ist, vergisst dies alles …\\
         but she-\textsc{nom} as-soon-as one-\textsc{nom} again nice to she-\textsc{dat} be-\textsc{prs.3sg} forget-\textsc{prs.3sg} that-\textsc{acc} all\\
    \glt `But when she is treated decently again, she forgets that all…' (amazon.de, Sept. 17\textsuperscript{th}, 2014 (book review))
    \ex  \label{ex:catasso:27b} 
    \gll Und ich, trotz der vier Beine, stürze, falle, fliege durch Raum und Zeit zurück schnurstracks zu der Expertin hin.\\
         and I\textsc{{}-nom} despite the-\textsc{gen.pl} four leg-\textsc{gen.pl} tumble-\textsc{prs.1sg} fall-\textsc{prs.1sg} fly-\textsc{prs.1sg} through space and time back directly to the-\textsc{dat.sg} expert up-to\\
    \glt `And despite walking with four legs, I tumble, fall, fly through space and time directly back to the expert.' (DWDS, WK\_1995\_2020, und\_ich\_','\_trotz\_15, aus: Coaching-Blogger, Jul. 05\textsuperscript{th}, 2012)
\z
\z

\subsubsection{Interrogatives and frames}\label{sec:catasso:4.2.2}

Similar observations can be made with respect to the interplay between interrogative pronouns/adverbs and frame setters.

A frame-setting constituent co-occurring left-peripherally with a \textit{wh}{}-element can surface to the left or to the right of it. In both cases, the presence of an adverbial resumptive in preverbal position is excluded (\ref{ex:catasso:28a}--\ref{ex:catasso:28b}), which suggests that the adjuncts have not been moved into the CP area from a middle-field position. Along the lines of what has been proposed in relation to the interaction between topics and frames, non-dislocated frames can be first-merged either clause\hyp externally\slash in a very high CP-internal position or in a specifier to the right of the sites encoding topichood and interrogativity. Note that for independent reasons, the trace left by the interrogative element in Spec,FinP cannot be pronounced in languages like German: this possibility is limited to non-focused referential elements (contrastive, aboutness and frame\hyp setting topics moving through Spec, FinP). It is assumed. however, that run-of-the-mill \textit{wh}-elements are not base-generated in German:

\ea%28
    \label{ex:catasso:28}        
    \ea \label{ex:catasso:28a}
    \gll Als du hier an der Kilbi angekommen bist, was (*da) war das Erste, das dich begeisterte?\\
         when you-\textsc{nom} here at the-\textsc{dat.sg} Kilbi \textsc{v.prt-ptcp}{}-come-\textsc{ptcp} \textsc{aux.prs.2sg} what \textsc{a.res} be-\textsc{pst.3sg} the-\textsc{nom.sg} first-\textsc{nom.sg} \textsc{rel.nom.sg} you-\textsc{acc.sg} thrill-\textsc{pst.3sg}\\
    \glt `What was the first thing that thrilled you when you arrived at Kilbi [= name of a music festival]?' (wavebuzz.ch (blog), May 31\textsuperscript{st}, 2015)
\ex \label{ex:catasso:28b} 
    \gll Warum, wenn der Ofen nicht benutzt wird, (*dann) ist der Luftstrom oben?\\
         why when the-\textsc{nom.sg} oven \textsc{neg} use-\textsc{ptcp} \textsc{aux.prs.3sg} \textsc{a.res} be-\textsc{prs.3sg} the-\textsc{nom.sg} airflow above\\
    \glt `Why is the airflow of the oven even on when it is not used?' (manuall.de, Aug. 08\textsuperscript{th}, 2020)
\z
\z

If the interrogative element is contrastive, it can be adjacent to a post-initial particle even when the constituent in second position is a frame setter, as in \REF{ex:catasso:29}:

\ea%29
    \label{ex:catasso:29}
    \gll Warum aber, wenn doch das Anliegen aller ein wirksamer Kampf gegen Antisemitismus sein sollte, wird dem Phänomen des BDS hier überhaupt so viel Aufmerksamkeit eingeräumt?\\
    why however if \textsc{prt} the-\textsc{nom.sg} concern all-\textsc{gen.pl} a-\textsc{nom} effective-\textsc{nom.sg} fight against antisemitism be-\textsc{inf} shall-\textsc{sbj.3sg} \textsc{aux.prs.3sg} the-\textsc{dat.sg} phenomenon the-\textsc{gen.sg} BDS here at-all so much attention \textsc{v.prt-ptcp-}concede\textsc{{}-ptcp}\\
    \glt `But why, if everybody’s concern should be the fight against antisemitism, does the BDS phenomenon receive so much attention?' (\textit{Frankfurter Rundschau}, May 13\textsuperscript{th}, 2021)
\z

This is in line with the derivational mechanisms assumed so far: the \textit{wh}\hyp element is merged in the IP/VP domain and moved to Spec,FinP, where its trace blocks any further movement from the lower area, then to a dedicated specifier in which interrogativity is encoded (FocP in Rizzi’s 1997, 2001 model for main clauses; LP in \REF{ex:catasso:30} below) and eventually to Spec,YP, in whose head the particle \textit{aber} is spelled out. The \textit{wenn}{}-clause is base-generated in the lower Spec,WP:

\ea%30
    \label{ex:catasso:30}\relax
    [\textsubscript{ForceP} [\textsubscript{YP} \textit{Warum}\textsubscript{j} [\textsubscript{Y°} \textit{aber,} [\textsubscript{LP[+}\textsc{\textsubscript{Foc/Int}}\textsubscript{]} t\textsubscript{j} [\textsubscript{WP} \textit{wenn doch … sein sollte,} [\textsubscript{FinP} t\textsubscript{j} [\textsubscript{Fin°} \textit{wird}\textsubscript{i} [\textsubscript{IP/VP} \textit{dem Phänomen} … t\textsubscript{j} \textit{so viel Aufmerksamkeit} \textit{eingeräumt} t\textsubscript{i}]]]]]]]]
\z

Although clause-internal (i.e., moved) topics and foci/\textit{wh}{}-elements can never co-occur in the same sentence in German given that only one constituent can enter the left periphery, for formal reasons I assume two distinct projections for these features (neutrally labeled “ZP\textsc{\textsubscript{[+top]}}\textsc{”} and “LP\textsc{\textsubscript{[+wh/int]}}\textsc{”)} which both occur to the left of WP\textsc{\textsubscript{[+frame]} }and match\textsubscript{} the serialization in Rizzi’s model of the Split CP:

\ea%31
    \label{ex:catasso:31}\relax
    [\textsubscript{ForceP} [\textsubscript{YP}\textsc{\textsubscript{[+contr]}} [\textsubscript{ZP}\textsc{\textsubscript{[+top]}} [\textsubscript{LP}\textsc{\textsubscript{[+wh/int]}} [\textsubscript{WP}\textsc{\textsubscript{[+frame]} }[\textsubscript{FinP} [\textsubscript{IP/VP}]]]]]]]
\z

In the non-V2 structures addressed in this section so far, the featural distinction between the two phrases simultaneously occupying the CP area is evident. It is undeniable that, say, an aboutness topic and a frame-setting constituent must be attracted to the left periphery by different features. In the next section, it is shown that some non-V2 patterns including a frame are more ambiguous with respect to their syntactic interpretation.

\subsubsection{Potentially ambiguous patterns}\label{sec:catasso:4.2.3}

In some cases, the structural relation between two (frame-like) adjuncts both positioned in the prefield and thus generating a linear non-V2 configuration is syntactically ambiguous.

To clarify this, let us firstly consider a linear non-V2 pattern in which the derivation is \textit{not} inherently ambiguous. Often, two or more frames occupying the CP area belong to the same or to an analogous semantic cline, like in \REF{ex:catasso:32}, where temporality (\textit{gestern Abend} ‘yesterday evening’) and locality (\textit{auf dem Weg}~… ‘on my way to~…’) concur to localize the general time-space coordinates of the daring described in the sentence. 

\ea%32
    \label{ex:catasso:32}
    \gll Gestern Abend auf dem Weg zum Sport habe ich es gewagt, am Sahlkamp in Hannover auf der Fahrbahn zu fahren.\\
         yesterday evening on the-\textsc{dat.sg} way to-the-\textsc{dat.sg} sport \textsc{aux.prs.1sg} I-\textsc{nom} it-\textsc{acc} \textsc{ptcp}{}-dare-\textsc{ptcp} at-the-\textsc{dat.sg} Sahlkamp in Hanover on the-\textsc{dat.sg} road to go-\textsc{inf}\\
    \glt `Last night, on my way to sports practice, I decided to travel on the road at Sahlkamp (= name of a city district) in Hanover.' (X, LaSuze, Jun. 27\textsuperscript{th}, 2023)
\z

It is compelling to assume that the two phrases build a “frame-setting complex” generated in the middle field and assembled there (or in a separate workspace) at some point of the derivation. This sentence can be modeled in different ways to demonstrate that the “big frame” resulting from this operation can appear as a block in Spec,YP \REF{ex:catasso:33a}, in a split configuration within the CP with the temporal XP being raised to Spec,YP and the local one left behind in Spec,WP \REF{ex:catasso:33b}, as two distinct constituents surfacing respectively in the left periphery and in the IP/VP area \REF{ex:catasso:33c} or a complex in the middle field \REF{ex:catasso:33d}. Moreover, in examples in (\ref{ex:catasso:33a}--\ref{ex:catasso:33c}) it is shown that if at least one of the two phrases is moved into the CP domain, it can optionally be taken up by an adverbial resumptive. If it is true that the two XPs form a big frame when surfacing together in the prefield, it can be assumed that this is a standard case of adverbial dislocation (\sectref{sec:catasso:4.1.1}) and that the resumptive has the entire frame as an antecedent (cf. the arguments in \citealt{Catasso2021Large}):

\ea%33
    \label{ex:catasso:33}
    \ea\label{ex:catasso:33a} Gestern Abend auf dem Weg zum Sport aber, (da) habe ich es gewagt, am Sahlkamp in Hannover auf der Fahrbahn zu fahren.
    \ex\label{ex:catasso:33b} Gestern Abend aber auf dem Weg zum Sport (da) habe ich es gewagt, am Sahlkamp in Hannover auf der Fahrbahn zu fahren.
    \ex\label{ex:catasso:33c} Gestern Abend (aber) (,) (da) habe ich es auf dem Weg zum Sport gewagt, am Sahlkamp in Hannover auf der Fahrbahn zu fahren.                
    \ex\label{ex:catasso:33d} Ich habe es gestern Abend auf dem Weg zum Sport gewagt, am Sahlkamp in Hannover auf der Fahrbahn zu fahren. 
\glt `Last night, on my way to sports practice, I decided to travel on the road at Sahlkamp (= name of a city district) in Hanover.'
\z
\z

In other cases, however, the relation between the two phrases in not as clear, as already pointed out in several works by St. \citet{Mueller2003, Müller2005, Muller2018}. In \REF{ex:catasso:34}, six sentences are presented that exemplarily illustrate some possible combinations (and serializations) of adverbials in this pattern: modal > temporal \REF{ex:catasso:34a}, explicative\slash subject\hyp oriented > concessive \REF{ex:catasso:34b}, temporal > causal \REF{ex:catasso:34c}, local > concessive \REF{ex:catasso:34d}, temporal > final \REF{ex:catasso:34e}, temporal > concessive \REF{ex:catasso:34f}:

\ea%34
    \label{ex:catasso:34}
    \ea \label{ex:catasso:34a} 
      \gll Aber plötzlich, im Jahre achtzehnhundertzwölf, gab es eine große Überraschung.\\
          but suddenly in-the-\textsc{dat.sg} 1812 give-\textsc{pst.3sg} \textsc{expl} a-\textsc{acc.sg} big-\textsc{acc.sg} surprise\\
      \glt `But suddenly, in 1812, there was a big surprise.' (M. Leblanc. 2022. \textit{Arsène Lupin heiratet} (novel), p. 66)
    \ex \label{ex:catasso:34b}  
        \gll Schließlich, trotz allem, was dein Abschlussball-Date dir sagt, zählen die ersten Eindrücke.\\
          after-all despite all-\textsc{dat.sg} \textsc{rel.acc.sg} your prom-date you-\textsc{dat.sg} tell-\textsc{prs.3sg} count-\textsc{prs.3pl} the-\textsc{nom.pl} first-\textsc{nom-pl} impression-\textsc{nom.pl}\\
      \glt `After all, despite what your prom date will tell you, it is the first impressions that count.' (demodia.com, 2023, company website)
    \ex \label{ex:catasso:34c}  
     \gll am Ende aufgrund der Mehrzahl an klaren Torchancen gewinnen wir das … Viertelfinale ...\\
          at-the-\textsc{dat.sg} end because-of the-\textsc{gen.sg} multitude at clear-\textsc{dat.pl} goal-chance-\textsc{dat.pl} win-\textsc{prs.1pl} we-\textsc{nom} the-\textsc{acc.sg} {} quarterfinal {}\\
      \glt `In the end, thanks to a number of goal-scoring opportunities, we won [lit. ‘win’] the quarterfinal …' (Facebook, ASC GW 49 - 1. Herren, Oct. 31\textsuperscript{st}, 2019)
    \ex \label{ex:catasso:34d}  
    \gll Aber hier, obwohl Verkehrssicherheit ja angeblich so wichtig ist, sieht man die Herren der Ordnungspolizei nicht in Aktion.\\
          but here although road-safety \textsc{prt} allegedly so important be-\textsc{prs.3sg} see-\textsc{prs.3sg} one-\textsc{nom} the\textsc{{}-acc.pl} gentleman-\textsc{acc.pl} the-\textsc{gen.sg} Ordnungspolizei \textsc{neg} in action\\
    \glt `But although road safety is supposed to be very important, you don’t get to see the \textit{Ordnungspolizei} gentlemen here.' (op-online.de, May 16\textsuperscript{th}, 2022 (online newspaper))
    \ex \label{ex:catasso:34e}  
    \gll Und heute damit der tag nicht langweilig ist spielen wir „HOT OR NOT“.\\
          and today so-that the-\textsc{nom.sg} day \textsc{neg} boring be-\textsc{prs.3sg} play-\textsc{prs.1pl} we-\textsc{nom} hot or not\\
     \glt `And today, we will play “HOT OR NOT”, so the day will not get boring.' (Facebook, Frankfurter Twins, May 24\textsuperscript{th}, 2014 [orthography and punctuation unmodified])
    \ex \label{ex:catasso:34f}  
     \gll damals, trotz 68er-Elan, durfte man beide als nichtheterosexuell nicht denken oder wenigstens nur heimlich.\\
          back-then despite 1968 vigor can-\textsc{pst.3sg} one-\textsc{nom} both-\textsc{acc.pl} as non-heterosexual \textsc{neg} think\textsc{{}-inf} or at-least only secretly\\
      \glt `Back then, despite the 1968 spirit, one was not allowed to think of them as non-heterosexual men, at least not publicly.' (taz.de, January 07\textsuperscript{th}, 2011 (online newspaper))\footnotemark
    \z
\z
\footnotetext{An anonymous reviewer notes that YP, ZP, LP and WP refer to positions that could be identified, respectively, as a ContrP, an AboutTopP, a WhP, and a FrameP. To be sure, the use of letters for labeling these projections is an act of prudence. With respect to the projection called “WP” here, for instance, the examples in \REF{ex:catasso:34} show that the adverbial in second position in this pattern need not be a frame-setting topic (at least in the mainstream sense). In the following sentence, WP hosts a modal XP:
      
\ea 
  \gll  [Ihr Anwalt,] [mit umständlicher                 Begründung, dringlich  und in aller                Form beantragte Wahrheitsbeweis\\
        their lawyer  with sophisticated\textsc{-dat.sg} motivation pressingly] and in all-\textsc{dat.sg}  form request.\textsc{pst.3sg}  truth-proof\\
  \glt ‘With a sophisticated argumentation, pressingly and in due form, their lawyer requested a proof of the truth.’ \citep[18]{Lühr1985}  
\z
        
The question of whether German also exhibits an independent ModP or the projection that is labeled WP here can be assumed to incorporate the functions of a FrameP and a ModP is not addressed in the present paper (cf. the discussion of ex. \REF{ex:catasso:20} above) and is left to future research. Thus, for the time being, a rather neutral labeling is chosen over a more descriptive one.}

The derivation of structures like those in \REF{ex:catasso:34} is apparently ambiguous in the sense that the fact that the two adjuncts belong to distinct semantic clines, i.e., modify the predicate in different ways, makes the idea of a “big frame” assembled in the middle field intuitively less plausible (but, at least in principle, not impossible). If the analysis developed above is on the right track, sentences like these (in particular, their left-peripheral serializations) can possibly be derived in different ways. Let us exemplarily consider \REF{ex:catasso:34a} and \REF{ex:catasso:34f}. These sentences can be modeled such that one (e.g., \ref{ex:catasso:35a}--\ref{ex:catasso:36a}, \ref{ex:catasso:35d}--\ref{ex:catasso:36d}), or both XPs (e.g., \ref{ex:catasso:35b}--\ref{ex:catasso:36b}, \ref{ex:catasso:35e}--\ref{ex:catasso:36e}) in the CP domain is/are clearly positioned in Spec,YP and that one or both is/are taken up by a preverbal resumptive: 

\ea%35
\judgewidth{\textsuperscript{??}}
\label{ex:catasso:35}
\ea[]{plötzlich aber, im Jahre achtzehnhundertzwölf gab es …\label{ex:catasso:35a}}
\ex[]{plötzlich, im Jahre achtzehnhundertzwölf aber gab es …\label{ex:catasso:35b}}
\ex[]{plötzlich, im Jahre achtzehnhundertzwölf, da gab es …\label{ex:catasso:35c}}
\ex[]{plötzlich aber, im Jahre achtzehnhundertzwölf, da gab es …\label{ex:catasso:35d}}
\ex[]{plötzlich, im Jahre achtzehnhundertzwölf aber, da gab es …\label{ex:catasso:35e}}
\z

\ex%36
\label{ex:catasso:36}
\ea[]{damals aber, trotz 68er Elan durfte man beide …\label{ex:catasso:36a}}
\ex[\textsuperscript{?}]{damals, trotz 68er-Elan aber durfte man beide …\label{ex:catasso:36b}}
\ex[]{damals, trotz 68er Elan, da durfte man beide …\label{ex:catasso:36c}}
\ex[]{damals aber, trotz 68er Elan, da durfte man beide …\label{ex:catasso:36d}}
\ex[\textsuperscript{??}]{damals, trotz 68er-Elan aber, da durfte man beide …\label{ex:catasso:36e}}
\z
\z

If one considers these two combinations of adverbials, it seems that the resulting variants are not all grammatically acceptable in the same way. In particular, the simultaneous raising of both XPs to Spec,YP produces a less-than-optimal result in \REF{ex:catasso:36}, while it is unproblematic with the sequence modal~> temporal. This suggests that \textit{plötzlich} and \textit{im Jahre achtzehnhundertzwölf} plausibly do build a big frame here and can appear either in the same clausal domain or in different positions. This is probably due to the fact that although \textit{plötzlich} is a modal adverb, its proximity to a time-related meaning implying perfectivity makes it cognitively relatable to – viz. readily interpretable as – a temporal element. This does not seem to be the case instead for \textit{damals} and \textit{trotz 68er Elan}: in \REF{ex:catasso:36}, the temporal phrase is arguably the one which is first-merged in the middle field and undergoes cyclical movement to the CP area (via Spec,FinP, Spec,WP and – if a contrastive reading occurs – to Spec,YP) and is optionally resumed by an overt trace spell-out (\textit{da}) in preverbal position. Indeed, concessive antecedents are generally not compatible with adverbial resumption. The phrase \textit{trotz 60er Elan} is better analyzed as a segment base\hyp generated in Spec,WP. 

In light of this insight, which is not at all self-evident, all the other patterns are easily derivable by means of the mechanisms illustrated in the previous sections. The proposed derivations of \REF{ex:catasso:35d} and \REF{ex:catasso:36d} (the most complex variants among the examples discussed here) are given in \REF{ex:catasso:37} (where the intermediate middle-field operations are not considered for ease of exposition): 

\ea%37
    \label{ex:catasso:37}
    \ea \label{ex:catasso:37a}   [\textsubscript{ForceP} [\textsubscript{YP} \textit{Plötzlich}\textsubscript{x} [\textsubscript{Y°} \textit{aber} [\textsubscript{WP} [[t\textsubscript{x}] \textit{im Jahre 1812},]\textsubscript{j}, [\textsubscript{FinP} [\textit{da}]\textsubscript{j} [\textsubscript{Fin°} \textit{gab}\textsubscript{i} [\textsubscript{IP/VP} t\textsubscript{j} \textit{es eine große Überraschung} t\textsubscript{i}]]]]]]]]
    \ex \label{ex:catasso:37b}  [\textsubscript{ForceP} [\textsubscript{YP} \textit{Damals}\textsubscript{j} [\textsubscript{Y°} \textit{aber} [\textsubscript{WP} \textit{trotz 68er Elan}, [\textsubscript{FinP} [\textit{da}]\textsubscript{j} [\textsubscript{Fin°} \textit{durfte}\textsubscript{i} [\textsubscript{IP/VP} t\textsubscript{j} \textit{man beide als nichtheterosexuell nicht denken} t\textsubscript{i}]]]]]]]]
    \z
\z

The general idea conveyed here is that if it is the case that in a given linear non-V2 pattern two adjuncts simultaneously surface in the prefield and these two phrases are semantically divergent, it cannot be established a-priorily whether they constitute a big frame (i.e., they belong to the class of topicalized XPs addressed in \sectref{sec:catasso:4.1}) or only one of them is a moved element, while the other is better analyzed as a non-dislocated adjunct generated in Spec,WP. In order to determine this, factors such as the ones considered with respect to \REF{ex:catasso:35} and \REF{ex:catasso:36} are to be tested. 

\subsection{Non-topics}\label{sec:catasso:4.3}

The last category to be reviewed is that of what will be referred to as “non-topics”. This class includes constituents that: (i) are not interrogative\slash focused, and; (ii)~can under no conditions function as sentence topics (in the widest sense), i.e., can be neither frame-setting nor familiar, aboutness or contrastive topics. In \REF{ex:catasso:19c} above, an example belonging to this category was illustrated that contains a negated phrase (\textit{keiner von uns} ‘none of us’). Besides negatives \REF{ex:catasso:38a}, further representatives of this group are quantified \REF{ex:catasso:38b} and non-specific indefinite \REF{ex:catasso:38c} expressions. 

Let us now consider two peculiar characteristics of non-topics. In the first place, as all examples in \REF{ex:catasso:38} show, a non-topic appearing in the left periphery cannot be adverbially or pronominally resumed in preverbal position:

\ea%38
    \label{ex:catasso:38}
    \ea  \label{ex:catasso:38a}
     \gll An keinem Ort {(*, da)} konnte man sich verstecken.\\
          at no-\textsc{dat.sg} place \textsc{a.res} can-\textsc{pst.3sg} one-\textsc{nom} \textsc{refl.acc.sg} hide-\textsc{inf}\\
     \glt `We could not hide anywhere.'
    \ex  \label{ex:catasso:38b}
     \gll Alle Studierenden {(*, die)} haben die Klausur bestanden.\\
          all-\textsc{nom.pl} student-\textsc{nom.pl} \textsc{p.res.nom.pl} \textsc{aux.prs.3pl} the-\textsc{acc.sg} exam pass-\textsc{ptcp}\\
     \glt `All students passed the exam.'
    \ex  \label{ex:catasso:38c}
      \gll Einem Hund {(*, dem)} würde ich so etwas nicht zumuten.\\
           a-\textsc{dat.sg} dog \textsc{p.res.dat.sg} \textsc{aux.sbj.1sg} I-\textsc{nom} so something \textsc{neg} demand-\textsc{inf}\\
      \glt `I would never demand something like this from a dog.'
    \z
\z

Secondly, we have already seen that the corresponding constituents can appear in the specifier of the left-peripheral projection encoding contrastiveness. If the non-topic is a complex XP (including, for instance, a modifier), the two parts of the phrase can appear in a split configuration in the CP area:

\ea%39
    \label{ex:catasso:39}
    \gll Einem Menschen aber, der so etwas tut, würde ich nicht vertrauen.\\
     a-\textsc{dat.sg} person-\textsc{dat.sg} however \textsc{rel.nom.sg} so something do-\textsc{prs.3sg} \textsc{aux.sbj.1sg} I-\textsc{nom} \textsc{neg} trust-\textsc{inf}\\
    \glt `I could not trust someone who says something like that.'
\z

Superficially, non-topics might be thought to have these two properties in common e.g., with \textit{wh}-phrases surfacing in the C-domain. However, a crucial difference between these two categories is that a non-topic cannot plausibly be assumed to acquire a [+\textsc{non-top}] feature in a left-peripheral projection, since this would amount to the absurd assumption that the CP layer hosts a position in which \textit{non-features} are checked. What is more, the semantics of non-topics is too various to allow for a characterization of a putative projection in this sense. 

Therefore, I will propose that non-topics, differently from \textit{wh}{}-interrogatives, undergo Formal Movement to Spec,FinP (and appear there) in the standard case and are further raised to Spec,YP only if the constituent (or a part of it) is interpreted contrastively. In other words, while interrogatives derive the bottleneck effect in Spec,FinP (leaving a non-pronounceable trace) and are then attracted to Spec,LP (cf. \ref{ex:catasso:30}) and – if the relevant conditions apply – to Spec,YP, non-topics surface in Spec,FinP unless they are attracted by some higher head in order to check an information-structurally-driven feature. Non-topics have this position in common with familiar topics (which are moved there) and expletives (which are base-generated), but the latter two can never be raised to Spec,YP (or to another position to the left of Spec,FinP) since they are not compatible with a contrastive (or some other) reading. Of course, if the left segment of a complex non-topic moves to Spec,YP, the unstressed modifier remains in Spec,FinP and is spelled out in that position. The derivation of \REF{ex:catasso:39} is given in \REF{ex:catasso:40}:

\ea%40
\label{ex:catasso:40}\relax
[\textsubscript{ForceP} [\textsubscript{YP} [\textit{Einem Menschen}]\textsubscript{x} [\textsubscript{Y°} \textit{aber} [\textsubscript{FinP} [[t]\textsubscript{x} \textit{der so etwas tut}]\textsubscript{j} [\textsubscript{Fin°} \textit{würde}\textsubscript{i} [\textsubscript{IP/VP} \textit{ich} t\textsubscript{j} \textit{nicht vertrauen} t\textsubscript{i}]]]]]]
\z

As a gate into the left periphery, FinP is thus the most crucial of all left\hyp peripheral projections in the present analysis: its specifier serves the derivation of the bottleneck effect, carries the EPP-like feature, hosts some types of constituents first-merged there or raised to that position by Formal Movement; its head attracts the finite verb.

\subsection{The spatial limits of the German left periphery}\label{sec:catasso:4.4}

If one tries to realize most of the left-peripheral categories addressed above in one and the same sentence simultaneously, thereby defying the limits of the occupability of the German C-domain, one may produce a structure like the following:

\ea%41
    \label{ex:catasso:41}
\gll Dem Kellner aber, der da arbeitet, als wir angekommen sind, dem habe ich’s gezeigt!\\
     the-\textsc{dat.sg} waiter however \textsc{rel.nom.sg} there work-\textsc{prs.3sg} when we-\textsc{nom} \textsc{v.prt-ptcp}{}-come-\textsc{ptcp} \textsc{aux.prs.1pl} \textsc{p.res.dat.sg} \textsc{aux.prs.1sg} I-\textsc{nom-}it-\textsc{acc} \textsc{ptcp}{}-show-\textsc{ptcp}\\
\glt `But the waiter who works there, when we arrived… I owned him!'
\z

Constructions like \REF{ex:catasso:41} are of course marked, but still possible, especially in spoken interaction. This sentence makes the lexicalization of the specifier and head positions discussed in this paper visible, as illustrated in the derivation in \REF{ex:catasso:42}:

\ea%42
\label{ex:catasso:42}\relax
[\textsubscript{ForceP} [\textsubscript{YP} [\textit{Dem Kellner}]\textsubscript{x} [\textsubscript{Y°} \textit{aber,} [\textsubscript{ZP} [[t]\textsubscript{x} \textit{der da arbeitet}]\textsubscript{j} [\textsubscript{WP} \textit{als wir angekommen sind} [\textsubscript{FinP} [\textit{dem}]\textsubscript{j} [\textsubscript{Fin°} \textit{habe}\textsubscript{i} [\textsubscript{IP/VP} \textit{ich’s} t\textsubscript{j} \textit{gezeigt} t\textsubscript{i}]]]]]]]]
\z

In this structure, the finite verb is attracted to Fin° and the indirect object \textit{dem Kellner, der da arbeitet} (‘the waiter who works there’) is base-generated in the middle field and cyclically moved to Spec,FinP (bottleneck) and then to ZP to acquire a [\textsc{+top}] feature. In Spec,FinP, the spell-out of its trace is pronounced as a pronominal resumptive matching the $\varphi $-features of the dislocate (but~– crucially~– could also be silent). Subsequently, \textit{dem Kellner} is further raised to Spec,YP, where it checks a [+\textsc{contr}] feature. In doing this, it leaves its unstressed (DP-internally extraposed) modifier in Spec,ZP. Note that this modifier is a restrictive relative clause, which excludes the assumption that this constituent could have a parenthetical status. The head position of the projection in whose specifier \textit{dem Kellner} surfaces spells out the post-initial particle \textit{aber}. The frame-like \textit{als}{}-clause is base-generated in Spec,WP.

\section{Conclusions}
\label{sec:catasso:5}

In this paper, a cartographically-oriented model of the left periphery of German has been proposed that allows the syntactic derivation of a number of complex prefield serializations. In order to test the hypotheses formulated throughout the discussion, non-V2 patterns – linear V3, V4, V5, etc. – have been considered. 

It has been argued that V2 is primarily a \textit{structural}, not (only) a \textit{linear} constraint, and Verb-Late(r) – when it is possible - is a (superficial) variant of this pattern. In other words, the linear component of V2 can be violated, but its structural core is inescapable. 

The existence (and the formal investigation) of complex prefields shows that even the left periphery of a well-behaved V2 language may contain multiple projections encoding different features. A combination of Move and Merge warrants the derivability of all possible word orders. German can thus be assumed to have a Split CP à la \citet{Rizzi1997} (with the respective differences having been considered), but its occupation is regulated by stricter rules than in languages like Italian: for instance, it exhibits obligatory V-to-C movement in all run-of-the-mill main clauses and its CP layer shows a so-called “bottleneck effect” and an EPP-like feature which is arguably positioned in a lower position in systems like English and Romance.

What is more, it has been suggested on the basis of data featuring Verb-Late(r) orders that the finite verb moves to Fin, not to Force in German declarative main clauses.

\section*{Sources}
\begin{description}[font=\normalfont,noitemsep]
\sloppy
\item[amazon.de,] September 17\textsuperscript{th}, 2014. Kundenrezension von \textit{Homo sapiens movere}. \url{https://www.amazon.de/review/RPVZFNGNN76U6} (last accessed on September 02\textsuperscript{nd}, 2023)

\item[baby.gofeminin.de,] February 23\textsuperscript{rd}, 2023. Wie lange stillen? \url{https://baby.gofeminin.de/forum/wie-lange-stillen-fd1049277} (last accessed on September 30\textsuperscript{th}, 2023)

\item[demodia.com,] 2023. Homepage. \url{https://www.demodia.com/de/resources/b2b-website-gestaltung} (last accessed on August 15\textsuperscript{th}, 2023).

\item[Facebook,] May 24\textsuperscript{th}, 2014. Frankfurter Twins. \url{https://www.facebook.com/Frankfurtertwins/?locale2=fr\_FR} (last accessed on October 04\textsuperscript{th}, 2023).

\item[Facebook,] October 31\textsuperscript{st}, 2019. ASC GW 49 - 1. Herren. \url{https://www.facebook.com/profile.php?id=100057523025946\& locale=ar\_AR} (last accessed on October 04\textsuperscript{th}, 2023).

\item[\textit{Frankfurter Rundschau},] May 13\textsuperscript{th}, 2021. Über den Antisemitismus-Diskurs in Deutschland.  \url{https://www.fr.de/kultur/gesellschaft/warum-sich-nichts-bewegt-90572891.html} (last accessed on August 29\textsuperscript{th}, 2023)

\item[handelsblatt.com,] October 15\textsuperscript{th}, 2012. Dr. Annette Schavan. Die Stunde der Plagiatsjäger. \url{https://www.handelsblatt.com/politik/deutschland/dr-annette-schavan-die-stunde-der-plagiatsjaeger/7255298.html} (last accessed on September 04\textsuperscript{th}, 2023).

\item[johanniter.de,] March 09\textsuperscript{th}, 2021. Kunstpause: Der Gnadenthron – eine Bildbetrachtung. \url{https://www.johanniter.de/presse/nachricht/kunstpause-der-gnadenthron-eine-bildbetrachtung-2735/} (last accessed on September 01\textsuperscript{st}, 2023).

\item[Kalbus, Oskar.] 1935. \textit{Vom Werden deutscher Filmkunst}. Berlin: Cigaretten-bilderdienst.

\item[Leblanc, Maurice.] 2022. \textit{Arsène Lupin heiratet}. Göttingen: LIWI.

\item[lr-online.de,] June 17\textsuperscript{th}, 2022. Elsterwerda. Das Thema „Hängematte“ steht vorerst nicht zur Diskussion.
https://www.lr-online.de/lausitz/elsterwerda/\_das-thema-\_haengematte\_-steht-vorerst-nicht-zur-diskussion\_-33422878.html (last accessed on August 19\textsuperscript{th}, 2023).

\item[manuall.de,] August 05\textsuperscript{th}, 2020. Bedienungsanleitung Bosch HBG635BB1 Backofen. \url{https://manuall.de/bosch-hbg635bb1-backofen/} (last accessed on September 15\textsuperscript{th}, 2023).

\item[op-online.de,] May 16\textsuperscript{th}, 2022. Autofahrer missachten neues Verkehrszeichen und kassieren Strafen. \url{www.op-online.de/region/seligenstadt/verkehrszeichen-ueberholverbot-auto-fahrrad-motorrad-neu-zweirad-schild-seligenstadt-polizei-kontrolle-news-91541446.html} (last accessed on August 30\textsuperscript{th}, 2023).

\item[Schömann, Lena (Produzentin) \& Bora Dagtekin (Regisseurin).] 2019. \textit{Das perfekte Geheimnis}. München: Constantin Film.

\item[taz.de,] January 07\textsuperscript{th}, 2011. Große Liebe: Ernie und Bert \url{https://taz.de/Grosse-Liebe-Ernie-und-Bert/!526510/} (last accessed on September 15\textsuperscript{th}, 2023)

\item[wavebuzz.ch,] May 31\textsuperscript{st}, 2015. Interview: The Slow Show (UK). \url{https://wavebuzz.ch/2015/05/31/interview-the-slow-show-uk-kilbi-bad-bonn-2015/} (last accessed on September 30\textsuperscript{th}, 2023).
\end{description}

\section*{Digital corpora}
\begin{description}[font=\normalfont,noitemsep]
\sloppy
\item[DWDS-Textkorpora:] Supplied by the Digitale Wörterbuch der deutschen Sprache (Berlin-Brandenburgische Akademie der Wissenschaften). \url{https://www.dwds.de/d/korpora} 

\item[IDS 2022:] Deutsches Referenzkorpus\slash Archiv der Korpora geschriebener Gegenwartssprache 2022-I. Mannheim: Leibniz-Institut für Deutsche Sprache. PID: 00-04B6-B898-AD1A-8101-4.
\end{description}
\sloppy\printbibliography[heading=subbibliography,notkeyword=this]
\end{document}
