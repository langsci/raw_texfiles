\documentclass[output=paper]{langscibook}
\ChapterDOI{10.5281/zenodo.17077190}
\author{Elise Louviot\orcid{}\affiliation{Université Reims Champagne-Ardennes} and Thérèse Robin\orcid{}\affiliation{Université Paris Est Créteil}}
\title[Verb-third with initial \textit{þa}/\textit{thô} in Old English, Old Saxon and OHG verse]{Verb-third with initial \textit{þa}/\textit{thô} in Old English, Old Saxon and Old High German verse}
\abstract{This paper is an empirical study of instances of V3 in clauses with the marker \textit{þa}/\textit{thô} in initial position in Old English, Old Saxon and Old High German verse. It examines whether such instances can be considered variants of V2 licensed when the second element is a subject pronoun and/or a given element. It concludes that such is not the case. In the corpus assembled for the sake of our study, V3 tends to share more affinities with V-late than with V2, while in the Old English section of the corpus V2 cannot be said to be the standard word order for independent clauses starting with \textit{þa.}}
\IfFileExists{../localcommands.tex}{
  \addbibresource{../localbibliography.bib}
  % add all extra packages you need to load to this file

\usepackage{tabularx,multicol}
\usepackage{url}
\urlstyle{same}

\usepackage{listings}
\lstset{basicstyle=\ttfamily,tabsize=2,breaklines=true}

\usepackage{langsci-basic}
\usepackage{langsci-optional}
\usepackage{langsci-lgr}
\usepackage{langsci-osl}
% \usepackage{./langsci/styles/langsci-lgr}
% \usepackage{./langsci/styles/langsci-osl}
% \usepackage{langsci-gb4e}

\usepackage{tikz}
\usetikzlibrary{patterns,calc}
\pgfdeclarepatternformonly{south east lines}{\pgfqpoint{-0pt}{-0pt}}{\pgfqpoint{3pt}{3pt}}{\pgfqpoint{3pt}{3pt}}{
    \pgfsetlinewidth{0.6pt}
    \pgfpathmoveto{\pgfqpoint{0pt}{3pt}}
    \pgfpathlineto{\pgfqpoint{3pt}{0pt}}
    \pgfpathmoveto{\pgfqpoint{.2pt}{-.2pt}}
    \pgfpathlineto{\pgfqpoint{-.2pt}{.2pt}}
    \pgfpathmoveto{\pgfqpoint{3.2pt}{2.8pt}}
    \pgfpathlineto{\pgfqpoint{2.8pt}{3.2pt}}
    \pgfusepath{stroke}}
    
\usepackage{stmaryrd}
\usepackage{wasysym}
\usepackage{multirow}
\usepackage{caption}
\usepackage{subcaption}
\usepackage{mathrsfs}
\usepackage{qtree}

\usepackage{linguex}


  %pminos do not split footnotes
% \interfootnotelinepenalty=10000 %Footnote in Laporte chapters has to be split SN


%\DeclareIndexNameFormat{default}{%
%\nameparts{#1}%
%\usebibmacro{index:name}%
%{\index[names]}%
%{\namepartfamily}%
%{\namepartgiveni}%
% {}% L1
% {}% L2
%{\namepartprefix}% generates spurious space L3
%{\namepartsuffix}% generates spurious space L4
%}

%  {\DeclareIndexNameFormat{default}{%
%     \usebibmacro{index:name}{\index[names]}{#1}{#3}{#5}{#7}}}

%\DeclareIndexNameFormat{default}{%
%  \usebibmacro{index:name}{\sindex[nom]}{#1}{#3}{#5}{#7}}

%\DeclareIndexNameFormat{default}{%
%  \usebibmacro{index:name}{\sindex[person]}{#1}{#3}{#5}{#7}}
%\DeclareIndexNameFormat{default}{%
%\nameparts{#1} \usebibmacro{index:name}{\sindex[person]]}{\namepartfamily}{‌​\namepartgiven}{\nam‌​epartprefix}{\namepa‌​rtsuffix}}

%\newcommand{\smiley}{:)}

%\renewbibmacro*{index:name}[5]{%
%\usebibmacro{index:entry}{#1}%
%{\iffieldundef{usera}{}{\thefield{usera}\actualoperator}\mkbibindexname{#2}{#3}{#4}{#5}}}

% \newcommand{\noop}[1]{}

%remove for final
%\overfullrule=1mm

\newcommand{\tobi}[2]}}
\renewcommand{\S}[1]{\tobi{#1}{\textsc{*}}}

% this volume references
% puts: [this volume]
% already defined: \citetv
%\newcommand{\citepv}[1]{(\citeauthor{#1} \citeyear*{#1} [this volume])}
\newcommand{\citealtv}[1]{\citeauthor{#1} \citeyear*{#1} [this volume]}

%parentheses around example number
\newcommand{\pref}[1]{(\ref{#1})}

% in-text examples

\newcommand{\lnex}[1]{\textit{#1}} %target lang word
\newcommand{\lnlit}[1]{(lit.: `#1')} %literal reading
\newcommand{\lnlat}[1]{(#1)} % latinization
\newcommand{\lntrans}[1]{`#1'} %translation
\newcommand{\lnexl}[2]%
{\lnex{#1}{} \lnlat{#2}} % ex with latinization
\newcommand{\lnexlat}[3]{\lnex{#1}{} \lnlat{#2}{} \lntrans{#3}} % ex with latinization and tranl.

%ch01
\newcommand{\co}[1]{\mbox{\textbf{#1}}}

%ch09

\newcommand{\cyrbulg}[1]{\begin{otherlanguage*}{bulgarian}#1\end{otherlanguage*}}


%ch10
\newcommand{\nlp}{{\small NLP}}
\newcommand{\mwe}{{\small MWE}}
\newcommand{\rae}{{\small RAE}}
\newcommand{\lvc}{{\small LVC}}
\newcommand{\pos}{{\small P}o{\small S}}
%\newcommand{\todo}[1]{ \textcolor{red}{#1} }

%\renewcommand{\labelenumi}{\theenumi}
%\ainamefmt{{vv}{ll}{, ff}{, jj}} % fullname

\newcommand{\biberror}[1]{{\color{red}#1}}

\newcommand{\osenovaitem}{--~} 
  %% hyphenation points for line breaks
%% Normally, automatic hyphenation in LaTeX is very good
%% If a word is mis-hyphenated, add it to this file
%%
%% add information to TeX file before \begin{document} with:
%% %% hyphenation points for line breaks
%% Normally, automatic hyphenation in LaTeX is very good
%% If a word is mis-hyphenated, add it to this file
%%
%% add information to TeX file before \begin{document} with:
%% %% hyphenation points for line breaks
%% Normally, automatic hyphenation in LaTeX is very good
%% If a word is mis-hyphenated, add it to this file
%%
%% add information to TeX file before \begin{document} with:
%% \include{localhyphenation}
\hyphenation{
    Beck-man
    Ngu-yen
    back-chan-nel
    back-chan-nels
    mo-not-o-nous
    ste-reo-typ-i-cal
}

\hyphenation{
    Beck-man
    Ngu-yen
    back-chan-nel
    back-chan-nels
    mo-not-o-nous
    ste-reo-typ-i-cal
}

\hyphenation{
    Beck-man
    Ngu-yen
    back-chan-nel
    back-chan-nels
    mo-not-o-nous
    ste-reo-typ-i-cal
}
 
  \togglepaper[1]%%chapternumber
}{}

\begin{document}
\maketitle 
%\shorttitlerunninghead{}%%use this for an abridged title in the page headers

\section{Introduction}
\label{sec:louviot:1}

The marker \textit{þa/thô} is extremely common in Old Germanic languages, where it is often found at the beginning of a clause. When the verb is in late or final position, \textit{þa/thô} is usually interpreted as a subordinating conjunction (“when”), whereas when the verb is in second position, initial \textit{þa/thô} is usually interpreted as a discourse marker (derived from the temporal adverb “then” or “at that time”), marking a new step in a narrative and foregrounding action (\citealt{Breivik2003, Enkvist1972, Enkvist1986, EnkvistWårvik1987, Foster1975, Schulz2014, Wårvik1995,Wårvik2011, Wårvik2013}).

According to Walkden, when \textit{þa} occurs in initial position in Old English, V3 does not occur:

\begin{quote} A first glance at the syntax of OE main clauses “suggests a strong parallelism” between OE and modern Germanic V2 languages such as Dutch and German (…) In all of these examples the verb follows the first constituent; in \textit{wh}{}-questions (…) and where an adverb such as \textit{þa} or \textit{þonne} is initial, this pattern is essentially exceptionless \citep[293]{Eythórsson1995}, and it is the majority pattern in main clauses in general. \citep{Walkden2015}
\end{quote}

It is worth noting, however, that Eythórsson’s actual words are somewhat more measured than Walkden’s:

\begin{quote} In \textit{wh}{}-questions the verb occurs in second position between the wh-word and a subject pronoun (…). There are virtually no exceptions to this in Old English prose \citep{Mitchell1985}. The same pattern is also regularly found with fronted negations (…), and frequently with fronted adverbs such as \textit{þa}, \textit{þonne} ‘then’, and \textit{nu} ‘now’ as well (…). \citep[293]{Eythórsson1995}
\end{quote}

Eythórsson makes no claim for when the subject is not a pronoun or when the sentence is in verse rather than prose, while “frequently” is not quite the same as “essentially exceptionless”. 

In fact, in the verse corpus examined here, cooccurrences of V3 with initial \textit{þa/thô} can be found in Old English, but also Old Saxon and Old High German, though the phenomenon is much rarer in the latter two languages.

Two main hypotheses have been put forward to explain V3 in Old Germanic languages. The first hypothesis is the so-called clitic hypothesis, whereby it is assumed that apparent V3 clauses are typically made up of a fronted constituent, a subject pronoun and a finite verb (which may then be followed by other constituents) and that the subject pronoun is actually cliticized to the verb and should therefore not be regarded as occupying a separate position. The argument has been made for Old English by \citet{vanKemenade1987} and \citet{Pintzuk1996, Pintzuk1999} and for Old High German by \citet[356--357]{Tomaselli1995}. In \citeyear{Eythorsson1995}, \citeauthor{Eythorsson1995} could claim that there was a consensus on the issue (\citeyear[295]{Eythorsson1995}). 

However, objections have been put forward since. \citet{Bech2001}, in particular, shows that interpreting the data according to the clitic hypothesis increases the proportion of Old English clauses which can be considered V2, but cannot explain away many remaining instances of V3, and tends to obscure some aspects of the evolution from a heterogeneous word order in Old English to a more homogeneous one in Late Middle English.

Accordingly, \citet{Bech2001} offers a new hypothesis: that V3 is not the result of exclusively syntactic factors, but also of pragmatic ones, and in particular that it is influenced by information structure. This hypothesis has been taken up and explored further, in Old and Middle English (\citealt{KemenadeLos2006}), but also in other Old Germanic languages (\citealt{Walkden2009, Walkden2015, HinterhölzlPetrova2009, HinterhölzlPetrova2011}). These studies concur in suggesting that, in the \textit{þa/thô X V} pattern which interests us here, the X position is most likely to be available for discourse\hyp given elements. This hypothesis would explain the high number of pronoun subjects found in this position (which prompted the clitic hypothesis in the first place), as well as the relatively high number of subjects in that position, whether pronominal or not (\citealt{Allen1990, Haeberli2002}).

The aim of the present paper is to test these hypotheses on a verse corpus comprising Old English, Old Saxon and Old High German. While many studies of V3 and of the syntax of early Germanic languages have been conducted within the generative framework, such is not the case of this paper, whose approach is deliberately empirical.

The second section of this article introduces our corpus, the third explains our methodology, and the fourth presents our results, which are analysed in more detail in the fifth section.

\section{Corpus}
\label{sec:louviot:2}
\subsection{Why Old English, Old Saxon and Old High German poetry?}
\label{sec:louviot:2.1}

Very little has been written about the marker \textit{þa/thô} in a V3 context, except to claim that it never happens (see above). V3 and \textit{þa/thô} (considered separately) have been studied extensively, but with a clear imbalance: Old English has received much more attention than other Old Germanic languages and has mainly been considered in isolation.\footnote{Which is not to say that neither issue has been examined in other Germanic languages (on the contrary, see the introduction to this paper). However, the three languages have not received equal attention and very few studies compare all three West Germanic languages.}

However, the marker \textit{þa/thô} is widely used in all West Germanic ancient languages and V3 does occur in all of them as well (albeit less frequently in other languages than in Old English), so it seems worthwhile to examine the interaction of \textit{þa/thô} and V3 in a wider linguistic context.

Our choice to focus on poetry for such a study may surprise some readers. As aptly summarized by \citet[84]{SomersDubenionSmith2014}:

\begin{quote} The received wisdom regarding the use of poetry for a syntactic analysis is, of course, that one should not do it.\footnote{A typical representative of that approach is provided in the following statement: “Word order in poetry is very different from that in prose. Therefore, poetry cannot be considered a reliable source of information on the standard of O[ld] E[nglish]” (\citealt{vanKemenade1987}: 4).}
\end{quote}

While West Germanic narrative poetry does not mistreat syntax as enthusiastically as skaldic poetry often does, it is nevertheless true that metrical requirements are likely to interfere with syntax to some extent, which is why poetry is avoided in many syntactic studies (e.g. \citealt{Bech2001}: 6). However, as noted by \citet{SomersDubenionSmith2014}, studying prose instead is hardly better: Old Germanic prose texts are mostly translations and therefore likely to be affected by interferences from Latin, even if they do not mimic Latin word order slavishly (not to mention that for Old Saxon, no prose text of any significant length exists anyway). Some scholars actually find poetry a more reliable guide for that very reason. After an extensive study of word order in prose and poetic texts in Old English and Old High German, \citet[207]{Cichosz2010} concludes:\footnote{Likewise, Somers argues (about the \textit{Evangelienbuch} in comparison with other extant Old High German texts) that “in some respects, Otfrid might even be considered one of the better texts on which to base syntactic analyses” \citep[57]{Somers2018}.}

\begin{quote} 
Poetry (…) is the only truly native text type without any foreign influences, with traditionally Germanic structures relatively undisturbed by external factors.
\end{quote}

Likewise, \citet[40]{Schulz2014} rejects the notion that poetry is inherently less reliable because it must obey metrical constraints:

\begin{quote} 
\relax
[Such a view] implies that the metrical structure exists outside and independently of OE poetry forcing the language into a particular form. Instead, it is the metrical patterns that are based on the natural structure of the language.
\end{quote}

\largerpage
However, it is worth asking what a completely “reliable” text devoid of any kind of interference would be anyway. It is well known that medium and genre both have a significant impact on grammar (see e.g. \citealt{Halliday1985}). Even if we had original prose texts in Old High German or Old Saxon, they would not give us a true reflection of how people spoke those languages in the 9\textsuperscript{th} century~-- they would only give a reflection of how people in a certain professional milieu wrote a specific type of texts.

For this reason, it seems important to keep in mind what we can and what we cannot do. Comparing a text translated from Latin with its Latin source and with original texts can help us isolate some aspects of Latin influence, but they cannot give us access to a “pure” text devoid of any Latin influence, given that anyone with access to writing in the early medieval period would likely have been living in a multilingual community and exposed to Latin influence in various ways. In other words, religious prose is unlikely to be entirely devoid of Latin influence, even when it is not directly translated from the Latin. Likewise, examining how metre can constrain syntax is useful so that a constraint emerging from metre is not mistaken for a strictly syntactic one. However, reconstructing a syntax free from metrical influence cannot be the aim: the poets write in the conventional style appropriate to their medium and the constructions they favour are influenced by those conventions in multiple ways that go beyond strictly metrical concerns. The language of poetry is different from the language of prose, and both are different from the language of spontaneous speech in their own ways. This does not mean that either of them should be disqualified as linguistic evidence.

Cumulative evidence from studies attempting different methodologies on different corpora can help us extrapolate credible hypotheses regarding the features of early Germanic syntax more broadly, but a single corpus-based study will necessarily shed light primarily on the corpus considered rather than on the whole language(s).

The aim of this study, therefore, is to examine the contexts in which V3 appears in a set of comparable texts which have much in common (so that differences attributable to medium or subject matter will only have a limited impact), but which may very well display features that would set them apart from other language productions in the same languages.

\subsection{Why these poems in particular?}
\label{sec:louviot:2.2}
\largerpage[1.5]

The texts chosen for this study are the \textit{Heliand} (Old Saxon), Otfrid’s \textit{Evangelienbuch} (Old High German) and a selection of Old English poems: \textit{Genesis A}; \textit{Elene}, \textit{Fates of the Apostles}, \textit{Juliana} and \textit{Christ II}; the \textit{Paris Psalter}. Our objective was to examine three comparable poems of sufficient length, one for each language, and this is the closest approximation we were able to attain. In the following paragraphs, we explain the reasons for this choice (and for discarding other options).

A common strategy, when compiling a corpus, is to gather a sample of (usually shorter) texts from multiple sources.\footnote{For instance, \citet{Cichosz2010} uses excerpts from Ælfric and Wulfstan’s homilies as well as the \textit{Anglo-Saxon Chronicle} and the Laws of Alfred to yield a total of 774 clauses in her original Old English prose category, and similar samples of various texts for the other categories; \citet{Bech2001} uses excerpts from four or five different texts to yield 1,250 clauses for each of the period she studies (with a total of 19 texts for four periods).} The use of multiple sources diminishes the risk that the personal preferences of an individual author get mistaken for a more general trend in the language examined. However, this strategy has its own pitfalls: it is simply impossible for Old Saxon (only two poems, and indeed only two continuous texts of any significant length survive) and quite difficult for Old High German, where the extant texts are extremely different from each other in terms of style, medium (prose, verse or something in between),\footnote{Among major Old High German texts, the translation of Isidore’s \textit{De Fide Catholica} is in prose, whereas the translation of Tatian’s \textit{Diatessaron} is an intermediate form between prose and verse.} dialect, and relationship to a Latin source, so that lumping them together is not very satisfactory.

We have opted for the opposite strategy: choosing only three comparable texts, one for each language. Strictly speaking, such a strategy only allows for the comparison of three texts, not of three languages. However, in doing so we compare three authentic linguistic objects (however limited they might be) rather than three artificial linguistic constructs.

Unfortunately, only an imperfect version of such a strategy can be implemented for the languages considered. For Old Saxon, the choice naturally falls on \textit{Heliand}, which is the only long text available (5983 lines). Adding the \textit{Saxon Genesis} (only 337 lines) would make the Old Saxon corpus slightly unbalanced, while not contributing much additional data. The poem uses the alliterative style typical of Old Germanic poetry and was composed by an anonymous author in the first half of the 9\textsuperscript{th} century. It is the transposition of the life of Jesus Christ in a Germanic context, based on (a presumably Latin version of) the \textit{Diatessaron}, a Harmony of the Gospels.\footnote{\textit{Heliand} should more accurately be considered a paraphrase than a translation: “A poetic paraphrase is quite a different thing from a prose translation, for reasons which go beyond merely formal considerations. In both Old and New Testament poetry in Old English the biblical text is mediated not only by the voice of the poet and the style of the poem but also by the logic of interpretation – the work is no substitute for the original, but an aesthetic response to it, a form of commentary.” \citep[8--9]{Liuzza1998}. Liuzza comments on the Old English corpus, but his remarks are equally applicable to \textit{Heliand} and the \textit{Saxon Genesis}.}

\largerpage[1.5]
No alliterative poem of any significant length survives for Old High German, so we have chosen to use Otfrid’s \textit{Evangelienbuch} instead. Like \textit{Heliand}, it is a kind of Harmony of the Gospels and therefore contains much of the same narrative material. It was written around 870, in a newly-minted poetic style using both internal and final rhymes, apparently devised by Otfrid himself. The poem therefore obeys metrical rules quite different from those of \textit{Heliand},\footnote{In his preface, Otfrid explains that his aim was to write a work in Frankish that would be as good as those in Latin \citep{Robin2013}.  It is the only Old High German text that refers to its own use of the vernacular language and it is considered the first German poetic text (distinctive for its use of rhymes, in particular). Its original poetic style is influenced by Latin poetry, but it is not wholly exempt from influence from Germanic alliterative poetry.} but, like \textit{Heliand}, it is narrative poetry composed by a native speaker relatively independently from its source material. Additionally, it is a very long text, which is convenient for our purpose (7104 lines).\footnote{The \textit{Evangelienbuch} has many sources in Latin (the Gospels, obviously, but also a great deal of commentary on the Gospels), but it is a new text rather than a close translation of an existing one.}

The Old High German \textit{Tatian} – a text based on the same Latin source as the \textit{Heliand} – might also have been an option, but its status as a poem and indeed as an independent text is rather more ambiguous.\footnote{The Old High German text of the \textit{Tatian} was composed by multiple translators and it seems to have been meant to be read alongside the Latin source. For Franz \citet{Simmler2008, Simmler2015}, the work should be regarded as a bilingual Latin-Old High German text and not as an independent Old High German text.} 

Since both \textit{Heliand} and the \textit{Evangelienbuch} can be dated to the 9\textsuperscript{th} century, choosing an Old English alliterative poem loosely based on the Gospels and composed in the same period would have been ideal, but no such poem exists. Old English poetry is notoriously hard to date and most poems are considerably shorter than those selected for Old Saxon and Old High German. The closest approximation we managed was to pick three large homogeneous bodies of religious texts likely centred around the 9\textsuperscript{th} century: \textit{Genesis~A} (2319 lines), an anonymous translation of Genesis presumed early by most and sometimes dated to the 8\textsuperscript{th} century; all four signed poems by Cynewulf (\textit{Elene}, \textit{Fates of the Apostles}, \textit{Juliana} and \textit{Christ II}: three hagiographical texts and a narrative of the Ascension, totalling 2601 lines), usually presumed later than \textit{Genesis A}, and sometimes ascribed to the 9\textsuperscript{th} century; and the \textit{Paris Psalter} (a verse translation of Psalms 51 to 150, totalling 5085 lines), known to be late, probably composed in the 10\textsuperscript{th} century.

This selection may seem skewed in favour of Old English, but a larger quantity of text was necessary to yield a comparable number of instances: our corpus contains 699 instances of \textit{þa/thô} in Old Saxon, 700 in Old High German and only 559 in Old English.\footnote{The \textit{Paris Psalter}, which is the only Old English poem comparable in length with \textit{Heliand}, contains very few instances of \textit{þa}.}

\largerpage
While this corpus is unavoidably imperfect, it has an interesting property in that its three subsets group differently depending on which criterion is considered. Syntactically, Old Saxon and Old High German group together because they are considered relatively straightforward V2 languages, whereas Old English is only “verb-second-ish” \citep{Bruening2016}. Metrically, on the other hand, Old English and Old Saxon group together, as our poems are written in an alliterative style obeying very similar traditional constraints, whereas Otfrid’s innovative verse form is the odd one out. This should help us assess the respective impact of metrical and syntactic constraints on our corpus.

\section{Methodology}
\label{sec:louviot:3}

V3 can co-occur with \textit{þa/thô} in a position other than the initial one:\footnote{All translations are our own.}

\ea
\ea (\textit{Genesis A} 2632b--2633a)\\
\gll Him                   þær \textbf{fylste}                \textbf{þa} / ece drihten,\\
     pron.\textsc{dat.sg} there \textbf{help-\textsc{pst.sg}} \textbf{þa} {} eternal-\textsc{nom.sg} lord-\textsc{nom.sg}\\
\glt ‘In that situation, the eternal lord helped him then’

\ex (\textit{Heliand} 3136b--3137a)\\
\gll Petrus                 \textbf{thô}     \textbf{gimahalde}, / helið hardmôdig\\
      Peter-\textsc{nom.sg} \textbf{thô}     \textbf{speak-\textsc{pst.sg}}, {} hero-\textsc{nom.sg} strong-minded-\textsc{nom.sg}\\
\glt ‘Peter then spoke, a strong-minded hero’

\ex (\textit{Evangelienbuch} V, 4: 36)\\
\gll ther         éngil                  bi    einen               líbon   {\textbackslash}   spráh \textbf{tho} sar zen wíbon\\
     \textsc{dem} angel-\textsc{nom.sg} about one-\textsc{dat.sg} live-\textsc{prs.ptcp}\\
\glt ‘the angel spoke then at once to the women’
\z
\z

However, such instances are comparatively rare in our corpus (12 instances in Old English, 6 in Old Saxon and 19 in Old High German). Even more importantly, those instances seem to correspond to very different patterns (with \textit{þa/thô} either before or after the verb and different kinds of phrases before the verb), so we have chosen to exclude them from the scope of this study and to focus exclusively on cooccurrences where \textit{þa/thô} is found in initial position.\footnote{For a discussion of V3 clauses in Old High German with particles such as \textit{thô} in second position, see \citet[22--23]{CatassoBastiani2024}.} 

At first glance, it would seem that subordinate clauses are naturally excluded from a study of V3, since subordinate clauses are supposed to be V-final (or V-late) in Old Germanic language and since V3 is usually seen as an alternative to V2 (i.e. with verb fronting or, from the perspective of the topological field model, with the verb in the left bracket) which is quite distinct from V-final (without fronting, i.e. with the verb in the right bracket, see e.g. \citealt{Walkden2015}).

However, telling main and subordinate clauses apart in early Germanic languages is not nearly as straightforward as it is for Present-Day Dutch or German, whether in verse or in prose. As noted for instance by Jolivet and Mossé: V-final in Old High German is often actually V-late (with some further elements found after the verb, see also \citealt{Catasso2021}: 6); V-final is rare in main or independent clauses but can occur; V1 or V2 can occasionally be found in its stead in what is clearly a subordinate clause (\citealt[206--207]{JolivetMosse1965}; see also \citealt{Behaghel1932}: 12--16).\footnote{\citet{Fourquet1938} carried out one of the first in-depth comparative analyses of word order in Old Germanic languages (which includes a review of much of the scholarship already published at that time). His driving thesis is that much of the difficulty that arises in the study of syntax in Old Germanic languages comes from the use of anachronistic categories, such as the opposition between initial, second and final position for the verb, which applies to Present-Day German, but is not necessarily relevant for earlier stages of the language (Fourquet posits several successive stages for Old Germanic languages, where the rules governing word order differ significantly, not just in what each position signifies, but also in what kind of positioning is meaningful: at some stage what matters may be the absolute position of the verb in the clause, but at another it may be its position relative to another element for instance).} According to \citet[3]{Catasso2021}, “the formal differentiation between main and embedded clauses in terms of asymmetric Verb-Second (henceforth: V2) is still not complete and becomes systemic, i.e. \textit{obligatory}, only from the 12\textsuperscript{th} century onwards”.

Likewise, for Old English prose, Mitchell regards the idea that a clause starting with \textit{þa} and ending with the verb is likely to be a subordinate clause a good “rule of thumb” \citep[\S3922]{Mitchell1985}, but he notes many possible exceptions, which do not necessarily impede interpretation, once context is considered, but which certainly prevent any automatic handling of the clauses. Elsewhere, he goes even further in suggesting “that there may have been an intermediate stage between ‘Then’ and ‘When’” (\citealt[\S2536]{Mitchell1985}; see also \cites[390--394]{Mitchell1978}[395--412]{Mitchell1980}) so that some clauses should best be regarded as somewhat subordinate, but not quite.\footnote{The case for an intermediate stage has also been made from a different perspective by Jean \citet{Haudry1973}: Haudry examines the phenomenon of correlation in Latin and concludes that it lies between parataxis and hypotaxis, and that it is not peculiar to Latin, but something that likely existed in other Indo-European languages as well, including Germanic languages. This type of structure (some of them using \textit{thô}) is well-attested in \textit{Heliand}, for instance.} For Old English poetry, the matter is even more controversial, with some scholars arguing that word order is a reliable guide while others are doubtful. \citet[41]{Schulz2014} reviews all the competing arguments in her dissertation and concludes:

\begin{quote} At any rate, it has been demonstrated that the position of finite verbs within the metrical line is indeed dependent on various factors, i.e. lexicality of the verb, syllable count and clause structure. These influences hardly leave room for the verb position to carry any grammatical information about the clause and the warning that any attempt on disambiguating clauses introduced by \textit{þa} on the basis of word order cannot but fail is conclusively justified.
\end{quote}

Since in some cases a clause may equally be regarded as independent or subordinate on the basis of its meaning and since there is no reliable syntactic feature that would allow us to automatically identify subordinate clauses starting with \textit{þa/thô}, excluding such clauses from analysis a priori is imprudent. Consequently, all clauses starting with \textit{þa/thô} are included in this study, and their potential subordinate status is examined in the qualitative analysis and whenever an example is quoted instead of being determined a priori.

Likewise, clauses starting with a coordinating conjunction such as \textit{ond/endi} have not been removed a priori, although they are likely to exhibit a somewhat different behaviour (see \citealt[\S3889]{Mitchell1985}). In such clauses, the conjunction is considered to be occupying a null position (i.e. the first element after the conjunction occupies the first position).

Since negative particles are systematically placed before the verb and can be fused to the verb (e.g. \textit{nolde} for \textit{ne wolde}), they are treated as genuine clitics here and are therefore not counted as occupying a separate position.

Occasionally, in Old English, \textit{þa} appears with another adverb, which seems to intensify its meaning somewhat rather than to constitute a separate constituent (the most frequent types are \textit{þa gien} `then again' and \textit{þa gyt} `then still'). In such cases, the two words are interpreted as occupying one position together. There is no equivalent phenomenon in Old High German or Old Saxon, so the issue does not arise there.

\section{Preliminary results}
\label{sec:louviot:4}
\subsection{Prevalence of V3}
\label{sec:louviot:4.1}

As mentioned above, our corpus contains 699 instances of \textit{þa/thô} in Old Saxon, 700 in Old High German and 559 in Old English. Of those, 429, 203 and 330 respectively are found in initial position (\figref{fig:louviot:1}).

\begin{figure}
\caption{Prevalence of \textit{þa/thô} in initial position}
\label{fig:louviot:1}
\small
\begin{tikzpicture}
\begin{axis}[
    ybar stacked,
    axis lines*=left,
    symbolic x coords={Heliand,Evangelienbuch,Old English},
    xtick=data,
    ylabel=\%,
    ylabel near ticks,
%     nodes near coords,
    bar width=1cm,
    enlarge x limits={0.2},
    height=5cm,
    width=9cm,
    legend pos=outer north east,
    reverse legend,
    ymin=0,
    ymax=100
    ]
    \addplot [lsMidDarkBlue,fill=lsMidDarkBlue!60] coordinates {(Heliand,61) (Evangelienbuch,29) (Old English,59)};
    \addlegendentry{initial position}
    \addplot [lsMidOrange,fill=lsMidOrange!60] coordinates {(Heliand,39) (Evangelienbuch,71) (Old English,41)};
    \addlegendentry{other positions}
\end{axis}
\end{tikzpicture}
\end{figure}


Old English is therefore closest to Old Saxon, in terms of proportion of \textit{þa/thô} in initial position, and within the Old English corpus, \textit{Genesis A} is closest (59\%) whereas Cynewulf is above (67\%) and the Psalter much below (45\%). In other words, it looks as if, on the basis of this single criterion at least, \textit{Genesis A} and \textit{Heliand} are closest stylistically, whereas the Paris Psalter and the \textit{Evangelienbuch} are outliers with perhaps a more “modern” style less reliant on the repetition of initial \textit{þa/thô}.\footnote{Furthermore, the proportion of initial \textit{thô} is highest in book I and lowest in book III of \textit{Evangelienbuch} (respectively the first and the last to have been composed), which aligns well with what has been observed elsewhere regarding Otfrid’s style using fewer traditional features (i.e. fewer features found elsewhere in Germanic alliterative poetry) as his work progresses \citep{Louviot2021Direct}.}Cynewulf’s tendency to use initial \textit{þa} even more than \textit{Genesis~A} (which holds across all four poems) could be related to his taste for archaism observed elsewhere \citep[78]{Louviot2016}. Among these clauses with initial \textit{þa/thô}, a minority have their verbs in the third position (see \tabref{tab:louviot:1}).\footnote{In the table, the question mark corresponds for the most part to instances where the third position is also the final position in the clause and occasionally to elliptic clauses without an expressed verb.}

\begin{table}
\begin{tabular}{l@{\quad}l *4{r@{~}l} r}
\lsptoprule
    &       &            &               &           &                & \multicolumn{2}{c}{V-late} &  & \\
    &       &  \multicolumn{2}{c}{V2}    &   \multicolumn{2}{c}{V3}   & \multicolumn{2}{c}{or final} & \multicolumn{2}{c}{?}   &    {Total}    \\\midrule
\multicolumn{2}{l}{Old English}             & 93  & (28\%) & 76 & (23\%) & 144 & (44\%) & 17 & (5\%)   & 330\\
& \textit{Genesis A}                        & 50  & (30\%) & 36 & (21\%) & 73  & (43\%) & 10 & (6\%)   & 168\\
& \textit{Cynewulf}                         & 38  & (33\%) & 34 & (30\%) & 37  & (32\%) & 5  & (4\%)   & 114\\
& \textit{Paris Psalter}                    & 5   & (10\%) & 7  & (15\%) & 34  & (71\%) & 2  & (4\%)   & 48\\
\multicolumn{2}{l}{\textit{Heliand}}        & 307 & (72\%) & 23 & (5\%)  & 90  & (21\%) & 9  & (2\%)   & 429\\
\multicolumn{2}{l}{\textit{Evangelienbuch}} & 124 & (61\%) & 15 & (7\%)  & 63  & (31\%) & 1  & (0.5\%) & 203\\
\lspbottomrule
\end{tabular}
\caption{Verb position in clauses with initial \textit{þa/thô}. Percentages are rounded.}
\label{tab:louviot:1}
\end{table}

\textit{Heliand} and the \textit{Evangelienbuch} display a clear predilection for V2 in clauses starting with \textit{þa/thô}, but the trend is much more subdued in Old English. It would be tempting to assume that the lower proportion of V2 in our Old English corpus is directly linked to a higher proportion of V3, but the reality is more nuanced than that. V3 \textit{is} more common in Old English (23\% of all instances against only 5\% in \textit{Heliand} and 7\% in the \textit{Evangelienbuch}), but so is V-late/final (44\% against 21\% and 31\% in the Old Saxon and Old High German poems), and, though it cannot be observed in \tabref{tab:louviot:1}, so is V1.

Indeed, in Old English, the pattern \textit{V þa X} (e.g. \textit{Genesis A} 144b, \textit{Heht þa lifes weard} ‘Then, the guardian of life ordered) is almost as common as \textit{þa V X} (e.g. \textit{Juliana} 66, \textit{þa reordode rices hyrde}, ‘Then the guardian of the kingdom spoke’) and may compete with it.\footnote{\textit{V þa X} is 0.7 times as frequent as \textit{þa V X}, with 69 instances against 93.} That pattern also exists in Old Saxon (25 instances, e.g. \textit{Heliand} 3689b, \textit{sprak thô uuordo filu}, ‘then [he] spoke many words’) and especially in Old High German (55 instances, e.g. \textit{Evangelienbuch} IV, 7:20 \textit{Gíang tho drúhtin thánana}, ‘Then the lord went away’), but in those languages it is much less common than \textit{þa V X} (12 times less in Old Saxon and 2.3 in Old High German). 

\subsection{Prevalence of subject pronouns in second position with V3}
\label{sec:louviot:4.2}

An examination of the constituents which can be found in second position with initial \textit{þa/tho} and V3 shows that subject personal pronouns (i.e. pronouns in the nominative case) are indeed a relatively common option, but not necessarily a dominant one in our corpus (\tabref{tab:louviot:2}). In the table, “intensified forms” refer to instances where the personal pronoun takes a form such as “he himself” (e.g. \textit{Heliand} 5518b \textit{Thuo hie selƀo sprak,} “then he himself spoke”).


\begin{table}
\small
\tabcolsep=.6\tabcolsep
\begin{tabular}{l *7{c}}
\lsptoprule
% Initial þa/tho  & V3
                    & \multicolumn{4}{c}{Old English} & \\\cmidrule(lr){2-5}
                    &  Subtotal     & Genesis & Cynewulf &  Psalms & Heliand                                    &  Otfrid &  Total\\\midrule
\textsc{nom} personal \textsc{pron}   
                    &  8            & 4       & 3        &  1      & 6\footnote{Including 3 intensified forms.} &  5      &  19   \\
\textsc{acc \& dat} pronouns 
                    &  8            & 3       & 3        &  2      & 1                                          &  3      &  12   \\
\textsc{nom} lexical NP      
                    &  34           & 14      & 19       &  1      & 4                                          &  6      &  44   \\
Other lexical NPs   &  11           & 7       & 3        &  1      & 1                                          &  1      &  13   \\
Non-finite verb     &  6            & 1       & 5        &  0      & 1                                          &  0      &  7    \\
Other               &  9            & 6       & 1        &  2      & 10                                         &  0      &  19   \\
Total               &  76           & 35      & 34       &  7      & 23                                         &  15     &  114  \\
\lspbottomrule
\end{tabular}
\caption{Nature of the second element (X) in the pattern ÞA /THO X V}
\label{tab:louviot:2}
\end{table}

Across all three languages, the subject is the element most likely to occur in this pattern (63 out of 114 occurrences or 56\%), but (nominative) personal pronouns are not dominant, so the clitic hypothesis is unlikely to account for the data in this corpus. That being said, pronouns do constitute a significant minority (31 out of 114 occurrences or 27\%), suggesting that some factors pertaining to weight and lack of stress (both relevant metrically) or to information structure (since pronouns normally have recoverable referents) may be relevant.

The appearance of a few non-finite verbs (past participles or infinitives) is also interesting. The pattern \textit{V\textsubscript{non-fin}}\textit{. + V\textsubscript{fin.}}\textsubscript{} normally implies a V-late or V-final pattern, which suggests the possibility that some of our occurrences (not just the 7 with non-finite verbs in second position but also others) should be seen as instances of a V-late pattern rather than instances of V3 as a variant of V2.

\section{Qualitative analysis}
\label{sec:louviot:5}
\subsection{Old English poems}
\label{sec:louviot:5.1}

In our Old English corpus, the choice of V3 over V2 seems to be affected by several factors. One of them is obviously the choice of the verb. Some verbs, in particular the light verbs \textit{wesan} (“to be”) and \textit{weorþan} (“to become”) are much more likely to be found with V2 than with V3.\footnote{\label{fn:louviot:18}\textit{Wesan} and \textit{weorðan} can be considered light verbs both semantically and prosodically. Semantically, both verbs are either copulas or auxiliaries and therefore carry very little meaning if any. Prosodically, \textit{wesan} is the only Old English verb which is more likely to be unstressed than stressed in verse \citep[174]{Momma1997}. \textit{Weorðan} (together with \textit{cuman}, \textit{cweðan}, \textit{habban}, \textit{hatan}, \textit{magan}, \textit{onginnan}, \textit{þurfan}, \textit{willan} and \textit{gewitan}) is among the group of verbs which are least likely to be stressed, though they are more often stressed than not in their finite form \citep[175]{Momma1997}.} This suggests that, in some cases at least, metrical factors are at play. Finite verbs are the trickiest word class in Old English metre in that (together with non-derived adverbs) they may receive primary, intermediate or no stress at all \citep[56]{Yakovlev2008}. High-frequency verbs with a low semantic content are especially likely to be unstressed, and \textit{wesan} more so than any other.\footnote{See \fnref{fn:louviot:18}.} According to Kuhn’s Law, such unstressed finite verbs must be placed within the first unstressed portion of the clause with other unstressed particles (such as \textit{þa} itself), which is likely to generate a V2 pattern when \textit{þa} is initial. This affinity of V2 with frequent monosyllabic verbs in Old English poetry, and particularly in \textit{Beowulf}, has been noted before \parencites[161]{Bliss1981}[321]{Getty2002}. \citet{TangelderLos2017} go as far as to suggest that in clauses starting with \textit{þa}, V2 is only possible with such verbs:\footnote{Tangelder \& Los allow for some exceptions, but they consider that they are few and limited to stock phrases.}

\begin{quote} In the absence of a full subject as the first constituent of the clause, the finite verb in second position is limited to mostly monosyllabic copulas and auxiliary verbs that appear unstressed as part of verse that leaves the [first prominent metrical] position empty, such as light verse. (\citealt{TangelderLos2017}: 204)
\end{quote}

It is worth looking at our own corpus in more detail, to see if it conforms to that trend or if other factors may also play a part.

\subsubsection{Lexical and metrical factors}
\label{sec:louviot:5.1.1}

A closer look at our corpus confirms the expected affinity of V2 with light verbs, but shows a slightly more nuanced picture (\figref{fig:louviot:2}). 

\begin{figure}
\small
\begin{tikzpicture}
  \begin{axis}[
    ybar,
    axis lines*=left,
    symbolic x coords={GA,GB,CA,CB,PA,PB},
    xtick=data,
    nodes near coords,
    enlarge x limits={0.2},
    height=5cm,
    width=\textwidth,
    legend style={at={(0.5,-0.2)},anchor=north},
    legend columns = -1,
    legend cell align=left,
    ymin=0
    ]
    \addplot [lsMidDarkBlue,fill=lsMidDarkBlue!60] coordinates {(GA,21) (GB,12) (CA,22) (CB,13) (PA,4) (PB,0)};
    \addlegendentry{unstressed monosyllabic}
    \addplot [lsMidOrange,fill=lsMidOrange!60] coordinates {(GA,14) (GB,2) (CA,3) (CB,0) (PA,1) (PB,0)};
    \addlegendentry{lexical carrying the alliteration}
    \addplot [black,fill=black!60] coordinates {(GA,1) (GB,0) (CA,0) (CB,1) (PA,0) (PB,0)};
    \addlegendentry{other}
  \end{axis}
\end{tikzpicture}
\caption{Verb-type in V2 clauses starting with \textit{þa}.
         GA: Genesis, a-verse,
         GB: Genesis, b-verse,
         CA: Cynewulf, a-verse,
         CB: Cynewulf, b-verse,
         PA: Psalms, a-verse,
         PB: Psalms, b-verse.}
\label{fig:louviot:2}
\end{figure}

Cynewulf almost only ever uses V2 with light verbs, especially \textit{wesan}, \textit{weorþan} and \textit{cuman} (“to come”), which together account for 92\% of all occurrences (\tabref{tab:louviot:3}). The Paris Psalter only has 5 occurrences in total, but \textit{wesan} and \textit{weorþan} account for 4 of them. Light verbs, especially \textit{wesan}, \textit{weorþan} and \textit{cuman}, are also very common with V2 in \textit{Genesis A}, but the \textit{Genesis A}{}-poet has a broader palette: not only do they use other light verbs in similar contexts (the \textit{verba dicendi} \textit{cweðan} and \textit{sprecan}, as well as \textit{onginnan}, literally “to start”), but they also use various full lexical verbs (typically class 2 weak verbs which can fill up the whole half-line with \textit{þa}: \textit{reordian}, \textit{gebletsian}, \textit{sceawian}, etc.), which carry the alliteration, a pattern known to Cynewulf, but which he uses much more infrequently.\footnote{A similar pattern in also found once in the Psalms (104: 46--47a), but with a class 1 weak verb (\textit{bebycgan}, “to sell”).}

Light verbs are also compatible with V3. In fact, in this corpus, the verbs most frequently found with V3 are all light verbs: \textit{onginnan} (9 occurrences), \textit{hatan} (“to command”, “to promise”), \textit{wesan} and \textit{weorþan} (6 occurrences each). Other light verbs found with V3 include \textit{gewitan} (“to depart”, 2 occurrences), \textit{cweðan} (“to say”, 2), \textit{habban} (“to have) and arguably \textit{sprecan} (“to speak”, 4 occurrences).\footnote{\textit{Onginnan}, \textit{hatan}, \textit{habban}, \textit{cweðan} and \textit{gewitan} are all listed by \citeauthor{Momma1997} (\citeyear[175]{Momma1997}, see \fnref{fn:louviot:18} above) as being frequently unstressed. In the third person past singular form, they are all “monosyllabic” in the sense intended by \citet[161]{Bliss1981} in that for those with an unstressed prefix, the prefix has no impact on their metrical value (except for \textit{habban}). They all have a relatively low semantic content (the essential information tends to be provided by the complement), but this is especially true for \textit{onginnan}, which is known to have undergone grammaticalization by the end of the Old English period, at least when it is used with a bare infinitive, which is systematically the case in this corpus \citep{Los2000}. \textit{Sprecan} is not mentioned by Momma, but it is monosyllabic, it is a frequent, general \textit{verbum dicendi} much like \textit{cweðan}, and it does occur in unstressed position in our corpus (but not with V3).}

For such verbs, then, two patterns are theoretically possible (V2 and V3), but the pattern chosen has an impact on metrical stress: light verbs are almost systematically unstressed with V2 and stressed with V3 (in which case they do not carry the alliteration, even in a-verses). Additionally, while V2 would theoretically seem available to all light verbs, in practice only \textit{wesan}, \textit{weorðan} (and to a lesser extent \textit{cuman}) are frequently found with that pattern. Other light verbs (i.e. \textit{onginnan} or \textit{hatan}) are much more likely to be found with V1 when they are unstressed (with \textit{þa} placed immediately after the verb).

\begin{table}
\tabcolsep=.85\tabcolsep
\fittable{\begin{tabular}{l rrrr}
\lsptoprule
  & \textit{Genesis A} & Cynewulf & Psalter & Total\\\midrule
\multicolumn{5}{l}{V2}\\
\quad Verb tokens & 50 & 38 & 5 & 93\\
\quad Verb types & 20 & 6 & 3 & 23\\
\quad Type-Token Ratio & 0.40 & 0.16 & 0.60 & 0.25\\
\quad 3 most common types & 30/50 (60\%) & 35/38 (92\%) & 5/5 (100\%) & 69/93 (74\%)\\
\quad Alliterating verbs  & 16/50 (32\%) & 4/38 (11\%) & 2/5 (40\%) & 22/93 (24\%)\\
\multicolumn{5}{l}{V3}\\
\quad Verb tokens & 35 & 34 & 7 & 76\\
\quad Verb types & 26 & 16 & 7 & 42\\
\quad Type-Token Ratio & 0.74 & 0.47 & 1 & 0.55\\
\quad 3 most common types & 11/35 (31\%) & 17/34 (50\%) & 3/7 (43\%) & 21/76 (28\%)\\
\quad Alliterating verbs  & 5/35 (14\%) & 4/34 (12\%) & 4/7 (57\%) & 13/76 (17\%)\\
\lspbottomrule
\end{tabular}}
\caption{\label{tab:louviot:3}Distribution of verbs with V2 and V3}
\end{table}

Another element worth mentioning, regarding Tangelder \& Los’s assessment quoted above and how it compares with our own data, is that unstressed monosyllabic verbs are not only compatible with V2 in light verses or in anacruses (which are only licit in the a-verse). They can also occur in straightforward B or C types, which can be found in the b-verse :

\ea\label{ex:louviot:2}
\ea 
Light verb with type B :    /x     x     /x           x     x       /      x  x   /\\
\gll \textbf{drun}con and \textbf{drym}don. Þa cwæð \textbf{driht}lecu mæg (\textit{Genesis A} 2782)\\
drink-\textsc{pst.pl} and experience.joy-\textsc{pst.pl}. Þa say-\textsc{pst.sg} noble-\textsc{nom.sg} girl-\textsc{nom.sg}\\
\glt ‘[they] drank and enjoyed themselves. Then the noble girl said’

\ex    
(light verb with type C) / x / {\textbackslash} x x x / \\
\gll \textbf{þrun}gon \textbf{þræc}hearde. ~ ~ ~ ~ Þa wæs \textbf{þuf} hafen (\textit{Elene} 123)\\
press forward-\textsc{pst.pl} bravely. Þa be-\textsc{pst.sg} banner-\textsc{nom.sg} raise-\textsc{pst.ptcp}\\
\glt ‘[they] pressed forward bravely. Then the banner was raised’
\z
\z

In other words, in our corpus, V2 is more often found in the a-verse and more often with light verbs (especially \textit{wesan}, \textit{weorþan} and \textit{cuman}), but it can be found in other configurations as well: the metrical constraints are real, but they are not nearly as restrictive as suggested by \citet{TangelderLos2017}.

In this respect, comparison with Old Saxon is interesting. Old Saxon alliterative poetry is subject to roughly the same metrical constraints as its Old English counterpart \citep{Suzuki2004}, and indeed everything suggests that written Old Saxon poetry was heavily influenced by Old English models \citep{Green2003}. As a consequence, Old Saxon poetry faces similar metrical constraints as regards the position of its verbs. If the metrical constraints of Old English poetry prevented V2 from occurring except in a very restricted set of circumstances, one would expect Old Saxon poetry either to have few occurrences of V2 or to violate the metrical constraints of Old English poetry. Instead, what we find is that \textit{Heliand} overwhelmingly favours V2 and that it does so by taking advantage of patterns that also exist in Old English poetry, but which are much less frequent there and restricted to a smaller set of verbs.

Most V2 clauses in \textit{Heliand} use short unstressed verbs as in \REF{ex:louviot:3}, which can easily be scanned according to Old English metrical rules:\footnote{A random sample of 50 clauses with \textit{þa V} in \textit{Heliand} showed that 47 of them were unstressed (94\%) and 90\% occurred in the b-verse (as most clauses starting with tho do}

\ea\label{ex:louviot:3}
 (light verb with type B)\\
/     x    x     /   ͜  x   x            x    x  x    /       x     / \\
\gll \textbf{uuî}son thes \textbf{uue}rodes.       Thô gesah \textbf{uual}dand Krist (\textit{Heliand} 3683)\\
visit-\textsc{inf} \textsc{dem} people-\textsc{gen.sg} Thô look-\textsc{pst.sg} ruling-\textsc{nom.sg} Christ-\textsc{nom.sg}\\
\glt ‘... to visit the people. Then the Ruling Christ looked...’
\z

What sets \textit{Heliand} apart from Old English poetry is the great variety of verbs it will accept in such a position. \tabref{tab:louviot:4} sets the data presented in \tabref{tab:louviot:3} in comparison with data from \textit{Heliand}.\footnote{Figures from the \textit{Paris Psalter} have been omitted, since occurrences are so few.} Direct comparison with V3 in \textit{Heliand} is not possible, given the paucity of occurrences. Instead, we have examined verb diversity in V-late and V-final clauses, in order to provide some perspective.\footnote{Since type variety is affected by the size of the corpus and there are many more occurrences of V2 clauses in \textit{Heliand}, we have chosen to examine five random samples of 50 occurrences of \textit{thô/thuo V} in Old Saxon (the same number as those found in \textit{Genesis A}), rather than all instances. The median values for the 5 samples were then calculated and the figures provided in \tabref{tab:louviot:4} are those median values. We also calculated the mean values, but it made very little difference, so we have chosen not to include them here, as it would merely make the table more unwieldy without contributing anything of note.}

The three most common verbs in \textit{Heliand} are only partially the same as in the Old English corpus: in V2 clauses, the two most common are \textit{uuerðan} and \textit{sprekan}, and the third is often (but not systematically) \textit{uuesan}; in V-late/final clauses, \textit{sprekan} is once again the most common verb, this time followed by \textit{kuman} (`come'), \textit{gisehan} (`see'), and \textit{gangan} (`go'), not necessarily in that order.


\begin{table}
\begin{tabular}{ll rrr}
\lsptoprule
 &  & \textit{Genesis A} & Cynewulf & \textit{Heliand}\\\midrule
\multicolumn{5}{l}{V2}\\
   & Verb tokens & 50 & 38 & 50\\
   & Verb types & 20 & 6 & 20\\
   & Type-Token Ratio & 0.40 & 0.16 & 0.40\\
   & 3 most common types (\%) & 30/50 (60\%) & 35/38 (92\%) & 24/50 (48\%)\\
\multicolumn{5}{l}{V3 (OE) or -late/final (OS)}\\
   & Verb tokens & 35 & 34 & 32\\
   & Verb types & 26 & 16 & 7\\
   & Type-Token Ratio & 0.74 & 0.47 & 0.64\\
   & 3 most common types (\%) & 11/35 (31\%) & 17/34 (50\%) & 16/50 (32\%)\\
\lspbottomrule
\end{tabular}
\caption{Verb Diversity in \textit{Heliand}, Cynewulf and \textit{Genesis A}}
\label{tab:louviot:4}
\end{table}

Several conclusions can be drawn from the figures in \tabref{tab:louviot:4}. At first sight, if we only look at the type/token ratios, it looks like verb diversity in V2 clauses is the same in \textit{Heliand} and \textit{Genesis A}, and that there is a similar gap between V2 clauses and clauses with the verb in a later position, although the gap is more pronounced in \textit{Genesis A}.

If we take a closer look at the actual distribution, however, a different picture emerges. On the whole, the \textit{Genesis A} poet uses a very rich vocabulary, seldom reusing the same verb, hence the high ratio of V3 clauses. This is true for V2 clauses as well, except for \textit{wesan}, \textit{weorðan} and to a lesser extent \textit{cuman}, which are used very frequently with that pattern.

By contrast, \textit{Heliand}’s vocabulary is less varied, with several verbs used two or three times in a 50-clause sample, but V2 clauses do not show a massive concentration of tokens on just a few verbs.

As for Cynewulf, its vocabulary is even less varied, and it shows a much higher concentration of tokens on the most common verbs, hence the staggering proportion of 92\% of tokens for the three most common types.

The data observed in \textit{Heliand} is compatible with a light semantic constraint on V2 clauses. If we accept that V2 clauses with initial \textit{thô} are more likely to be independent clauses where \textit{thô} is a discourse marker signalling a new narrative episode, whereas V\hyp late\slash final clauses are more likely to be subordinate clauses (even though not all V-late/final clauses will be subordinate, see \sectref{sec:louviot:3} above), then it makes perfect sense for verb diversity to be slightly reduced in V2 clauses: some actions are simply more suitable to start a new episode than others.

The kind of concentration observed in the Old English corpus is much harder to fully account for. Metrical constraints obviously play a significant part, but they cannot explain why, unlike the \textit{Genesis~A} poet, Cynewulf rarely ever uses V2 with any full lexical verb or with any light verb other than \textit{wesan}, \textit{weorþan} and \textit{cuman}. They do not explain either why he will easily use other light verbs in initial position followed by \textit{þa} (which \textit{Heliand} rarely ever does), but not the other way round. Prosodic factors may very well play a part here, but not metrical stress since the adverb and the verb are both unstressed.\footnote{Whether one should grant a weak stress to a light verb which does not carry the alliteration as in examples 4 and 5 is a matter of debate, since the scansion of finite verbs is one of the trickiest aspects of Old English metre, as mentioned above. Some scholars, such as Bliss, would never stress a finite verb in such a position \citep{Bliss1967}; most grant that some finite verbs can be stressed, particularly lexical verbs which carry the alliteration or which are found to carry it in some cases, but would not stress the finite verbs in examples 4 and 5 (e.g. \cites[61--62]{Getty2002}{Russom2022}); \citet{Bredehoft2005}, conversely, would grant a weak stress to all finite verbs except \textit{wesan} and \textit{weorðan}. However, Bredehoft’s foot-based scansion system treats both \textit{þa com} and \textit{com þa} as acceptable kinds of “s-feet”, which can be found in the same metrical positions, so the two alternatives are functionally equivalent in the examples above, even if they are scanned xs and sx respectively \citep[25]{Bredehoft2005}.} From a metrical point of view, the initial dips in (a) and (b) seem equivalent in both languages:

\ea\label{ex:louviot:4}
\ea
x    x      /    /    x\\
\gll  þa com \textbf{soð} metod (\textit{Genesis A} 2793b)\\
   then come.\textsc{pst.sg} true\textsc{.nom.sg} creator.\textsc{nom.sg}\\
'Then the true creator came'
\ex
x     x   / ͜  x x   /\\
\gll Com þa \textbf{wi}gena hleo (\textit{Elene} 150 b)\\
then come.\textsc{pst.sg} fighter-\textsc{gen.pl} protector-\textsc{nom.sg}\\
\glt 'Then the protector of fighters came'
\z
\z

\ea\label{ex:louviot:5}
\ea
   x     x    x   /   x x   /\\
\gll  Thô hêt the \textbf{hê}lago Crist (\textit{Heliand} 4076b)\\
then order-\textsc{pst-sg} \textsc{dem} holy-\textsc{nom.sg} Christ-\textsc{nom.sg}\\
\glt 'Then the holy Christ commanded'
\ex
x    x    x  /      x  /   x\\
\gll   hêt thô is \textbf{rin}kos faran (\textit{Heliand} 728b)\\
order-\textsc{pst-sg} then \textsc{poss} soldier-\textsc{acc.pl} travel-\textsc{inf}\\
\glt 'Then [he] ordered his soldiers to travel'
\z
\z


However, the two languages behave very differently. In \textit{Heliand}, (b) occurs occasionally, but the preferred pattern is overwhelmingly (a), whereas in our Old English corpus, there is no such general preference: some verbs favour (a) (\textit{wesan}, \textit{weorþan}, \textit{cuman}) while others favour (b) (\textit{hatan}, \textit{onginnan}).

Clearly then, metrical factors cannot be ignored, but they cannot be considered the sole driving factor behind the relative marginality of V2 with initial \textit{þa} in our Old English poems.

Another interesting question here is whether a diachronic evolution might be discerned. Conventional wisdom would suggest that V2 might become more common as time goes by, since many scholars assume that Old English prose respected a V2 constraint, but that Old English verse (in particular earlier poems such as \textit{Beowulf}) reflected an earlier stage of the language, before that constraint became prevalent (see e.g. \citealt{Stockwell1977, Kemenade1987, StockwellMinkova1991}). Accordingly, one might expect the syntactic patterns of later poems to be more similar to prose than those of the earliest poetry.

However, the pattern \textit{þa V X} rather seems to become less productive over time. It is beyond the scope of this paper to study the entirety of the Old English poetic corpus, but since it is impossible to make a guess regarding diachronic trends based on only two authors, a look at other texts is welcome.\footnote{I do not include the author of the Psalms, since the number of occurrences is too limited to draw any conclusion.} Among the earliest poems, \textit{Beowulf} uses \textit{þa V X} predominantly with \textit{wesan} (48 occurrences), and only occasionally with other light verbs (\textit{cuman}, 5 occurrences, \textit{sculan}, “shall”, 1 occurrence) or with verbs carrying the alliteration (3 occurrences).\footnote{Interestingly, \citet[334]{FulkEtAl2008} note that in \textit{Beowulf} it is “not out of the ordinary” for a finite verb to occur in alliterating position in the b-verse, and several of their examples show V2 after an unstressed particle, though only one of them (2980b) after \textit{þa}. The other instances occur after \textit{þonne} (‘then’ or ‘when’, 1327b, 2544b) and after \textit{þæt} (‘that’, 1548b). The verbs in that pattern are naturally all full lexical verbs.} \textit{Daniel}, on the other hand, which is presumed to be about as early as \textit{Beowulf} and \textit{Genesis A} \citep[348]{Fulk1992}, regularly uses the pattern with other light verbs.\footnote{In \textit{Daniel}, this pattern is found once each with the verbs \textit{gan}, \textit{habban}, \textit{cweþan}, \textit{gewitan} and \textit{geseon}, as well as 7 times with \textit{wesan} and 5 with \textit{weorþan}. It is also found once with a full lexical verb carrying the alliteration (116a).} Late poems such as \textit{Maldon} and \textit{Judith} show very few instances of the pattern, almost all with either \textit{weorþan} or \textit{wesan}.\footnote{The Battle of Maldon related in the poem took place in 991 and the poem was likely composed relatively soon after. Judith consistently places among the latest poems in \citegen{Fulk1992} assessment of linguistic criteria for dating and is usually attributed to the 10\textsuperscript{th} century as well. The two poems are therefore roughly contemporary with the \textit{Psalms}.} Another noteworthy feature of those poems is that, like the Psalms, they do not seem to use \textit{þa} in initial position very much at all.\footnote{In \textit{Maldon}, 45 clauses use \textit{þa}. It is found in initial position in only 24 clauses and the verb is found in second position in only 4 of them (i.e. only 17\% for V2): twice with \textit{weorþan}, once with \textit{wesan} and once with \textit{standan}. In \textit{Judith}, 50 clauses use \textit{þa}. It is found in initial position in only 14 clauses, and the verb is found in second position in 8 of them (which makes for a higher proportion, but out of a much smaller number of occurrences). All occurrences use \textit{weorþan} or \textit{wesan} (6 and 2 respectively).} 

One way of accounting for this evidence is to suppose that \textit{þa V X} was relatively productive with various verbs at an earlier period (though already showing a strong preference for light verbs), and that the most productive patterns fossilized around that time. Then, productivity decreased but the fossilized patterns (with \textit{weorþan}, \textit{wesan} and to a lesser extent \textit{cuman}) remained frequent for a time, before gradually falling out of use as initial \textit{þa} became less common in Old English poetry. \footnote{On this typical trajectory for linguistic constructions, see for instance \citet{Bozzone2014}: 83--88.} In Old Saxon, conversely, if a similar initial situation ever prevailed, it would seem that the pattern gained in productivity, accepting a much broader range of verbs.\footnote{Of course, there is no certainty that such a situation ever prevailed. Suzuki considers that, whether the \textit{Heliand} poet was massively dependent on Old English models or whether he also relied on a significant native alliterative tradition, “the \textit{Beowulf} metre would have represented the substance of a metrical knowledge that the \textit{Heliand} poet inherited and made innovative use of for his own work” \citep[7]{Suzuki2004}. However, as shown above, word order preferences can only partially be accounted for by metre, so if the \textit{Heliand} poet could rely on native oral models rather than just on imported Old English poetry, they may very well have been familiar with different patterns.}Given the limited evidence available, however, such a hypothesis can only be tentative.

\subsubsection{Choosing between V2 and V3}
\label{sec:louviot:5.1.2}
The section above has shown that some verbs were found with V2 but not with V3 (a few class 2 weak verbs), while others were found with V3, but not with V2 (many lexical verbs). However, a number of light verbs can occur with both patterns. Comparing them might help us better understand what might lead a poet to opt for V2 rather than V3 and whether non-metrical factors may also be at play.

Verbs occurring with both patterns in our corpus are \textit{cuman}, \textit{cweðan}, \textit{geseon}, \textit{onginnan}, \textit{sprecan}, \textit{standan}, \textit{weorþan} and \textit{wesan}. We will not look at \textit{standan} since the two occurrences are found in two different subsections of our corpus, but will instead compare occurrences of the other verbs to try and determine what factors might be at play.

\subsubsection{Cuman}
\label{sec:louviot:5.1.3}
The verb \textit{cuman} is found only once with V3 and 4 times with V2 in Cynewulf.\footnote{V2: \textit{Elene} 549b, \textit{Juliana} 242b, 563b and 614b; V3: \textit{Christ II} 449b.} In all four instances, \textit{cuman} is used straightforwardly as a verb of movement. In this case, the main factor at play seems to be metrical: the one instance with V3 has \textit{se æþeling} (“the prince”) as its subject and this can only yield a metrical line with \textit{þa se æþeling cwom} (verse of type B, with the second stress falling on \textit{cwom}) or with \textit{cwom þa se æþeling} (light type A, only used in the a-verse).\footnote{\textit{Beowulf}, which is usually taken as the gold standard of early Old English poetry, never uses light verses of type A in the b-verse \citep[437]{Doane1991}.} The first pattern, which can equally be found in the a-verse or the b-verse, seems to have been preferred by Old English poets (5 occurrences of \textit{þa se æþeling} \textit{V} with various verbs in the Anglo-Saxon Poetic Records, including this one, against only one with the latter).

\subsubsection{Cweðan}
\label{sec:louviot:5.1.4}
There are 2 instances of V2 and 2 of V3 in \textit{Genesis A} with \textit{cweðan}.\footnote{V2: 2389b, 2782b; 1327a, 1510b.} They are potentially interesting because most of them have God as their subject, which means it would be extremely easy to use a synonym to achieve metrical conformity with a different word order. Thus we have \textit{þa cwæð halig god} (“then spoke holy God”, \textit{Genesis A} 2389b) which could just easily have been \textit{þa se halga cwæð} (“then the holy one spoke”), and \textit{ða to Noe cwæð       nergend user} (“then to Noah spoke our saviour”, \textit{Genesis A} 1327), which could easily have been \textit{ða cwæð wuldres helm     word to Noe} (“then spoke the protector of glory some words to Noe”). In such cases, the poet clearly has the choice between V2 and V3 and is not constrained by rhythm or alliteration.\footnote{The other two relevant occurrences are \textit{Genesis A} 2782b--2783a and 1510b--1511.}

Whether the element following \textit{þa} in V3 is easily recoverable in context cannot be a factor either: in all four instances the element which could occur or does occur in that position is well-known to the reader and has been mentioned recently.

One possible difference between the two sets of occurrences might be the “shock” factor of the speech being introduced. The two V3 instances occur when God speaks to Noah at length about their covenant (before and after the Flood). There is no antagonism and no surprise here. By contrast, the two instances with V2 can be seen as displays of righteous anger. In one case, God hears Sarah laughing at his promise that she will bear a child and he denounces her error (though he admittedly does so benignly and addresses Abraham rather than Sarah herself).  In the other, Sarah sees Ishmael playing in front of her husband and can no longer tolerate his presence within the household, which leads her to speak to her husband to demand his departure.

\subsubsection{Geseon}
\label{sec:louviot:5.1.5}
There are only two instances with \textit{geseon} in \textit{Genesis A} (1270 and 2877b--2878). They are unlikely to reveal much because the clause with V3 is under significant syntactic constraints, so that V2 would be difficult:

\ea\label{ex:louviot:6}
\gll þa se eadega wer geseah hlifigan         hea dune (\textit{Genesis A} 2877b--2878)\\
þa \textsc{dem} blessed-\textsc{nom.sg} man-\textsc{nom.sg} see-\textsc{pst.sg} rise-\textsc{inf} high-\textsc{acc.sg} hill-\textsc{acc.sg}\\
\glt ‘then the blessed man saw rise up a high hill’
\z

The sentence contains two verbs, one in the past tense, the other in the infinitive. Placing \textit{se eadega wer} before the finite verb makes it much clearer that it has nothing to do with \textit{hlifigan}.

\subsubsection{Onginnan}
\label{sec:louviot:5.1.6}
There is one instance of V2 with \textit{onginnan} in \textit{Genesis A} (217--2718a) and 3 instances of V3 (1497--1498a, 1555--1557 and 1898b--1900a). Here again, metre seems to be a factor: short subjects (\textit{Noe} and \textit{se halga}, “the holy one”) favour a V3 pattern, whereas the long name Abimelech can only fit with V2 (or V-late), but not V3:

\ea\label{ex:louviot:7}
\ea
 x   x  x     / ͜  x {\textbackslash}   x        / x    x     /     x   /     x    x  /     x\\
\gll  þa ongan \textbf{A}bimæleh \textbf{A}braham \textbf{swi}ðan \textbf{wo}ruldgestreonum (2717--2718a)\\
þa begin-\textsc{pst.sg} Abimelech-\textsc{nom.sg} Abraham-\textsc{acc.sg} enrich-\textsc{inf} world-treasure-\textsc{dat.pl}\\
\glt ‘then Abimelech began to enrich Abraham with worldly treasures’ \footnote{The scansion of 2717 is delicate, given the fact that Abimelech and Abraham are not native names. Evidence from the rest of the poem suggests that they are treated in a way similar to compound native names, and that their secondary stress can go unstressed in some metrical contexts. As suggested above (footnote 26), most scholars would consider that \textit{ongan} is unstressed in 2. In \citegen{Bredehoft2005} scansion system, \textit{ongan} would receive a secondary stress, but both \textit{þa ongan} \textbf{\textit{A}}\textit{bimæleh} (xxs {\textbar} Ssx) and \textit{ongan þa} \textbf{\textit{A}}\textit{bimæleh} (xsx{\textbar} Ssx) would yield a licit sC type-verse according to his system.}


\ex     x   /      x    /             /   x    /   x
 x     /      /     x              /      /     {\textbackslash}   x\\
\gll  ða \textbf{Noe} ongan         \textbf{ni}wan stefne mid hleomagum         ham staðelian (1555--1557)\\
ða Noe-\textsc{nom.sg} began new-\textsc{dat.sg} time-\textsc{dat.sg} with relative-\textsc{dat.pl} home-\textsc{acc.sg} establish-\textsc{inf}\\
\glt ‘then Noe began to build a new home with his relatives’
\z
\z

If there were a strong imperative against using V2, V3 would relatively easily be achievable through the use of a shorter periphrasis instead of the proper noun (such as \textit{se eorl}, \textit{se æðela}, or \textit{se æðeling}, if one wanted to retain the same alliteration, or \textit{se beorn}, \textit{se rinc}, \textit{se maga}, etc. if one were to change the b-verse), but if both patterns were equally available, then choosing V2 makes sense on metrical grounds.\footnote{The periphrases \textit{se eorl} (“the man of rank”), \textit{se æðela} (“the nobleman”), \textit{se æðeling} (“the nobleman”, “the prince”), \textit{se beorn} (“the man”, “the warrior”), \textit{se rinc} (“the man”, “the warrior”), and \textit{se maga} (“the man”, “the kinsman”) are all used at some point in the poem to designate a male protagonist.}

Semantically and pragmatically, there is nothing obvious that sets the V2 occurrence apart from the others, apart from the fact that it follows direct speech directly: all four occurrences have a subject lexical noun phrase as one of the first three elements in the clause and it refers to well-known characters in all cases.

\subsubsection{Sprecan}
\label{sec:louviot:5.1.7}
\textit{Sprecan} is similar to \textit{onginnan} in that all occurrences but one favour the V3 pattern.\footnote{V2 clauses: 1294b--1295a, 1483, 1744--1745, 2880. V3 clause: 2123b--2125a.} Most take the subject noun phrase as their second element (one takes an indirect object instead, with the subject located immediately after the verb) and in all cases the element located in second position is well-known at this stage of the narrative (God, Noah or Abraham).

However, in two instances with V3 (1483 and 1744--1745), the element in second position has not been referred to recently and is being reintroduced in the narrative.

Conversely, the only clause with V2 has its subject straight after the verb and the subject refers to a person who has just delivered a speech (the prince of Sodom) and is therefore still fresh in the audience’s memory. In other words, givenness does not seem to be a factor in privileging V3 over V2 with that verb:

\newpage
\ea\label{ex:louviot:8}
\ea
... /   x     /    x   x            x     x       /      {\textbackslash}   x  \\
\gll and þæs hereteames ealles teoðan sceatAbraham sealde \textsc{go}des bisceope.\\
and \textsc{dem} plunder-\textsc{gen.sg} all-\textsc{gen.sg} tenth-\textsc{acc.sg} part-\textsc{acc.sg} Abraham-\textsc{nom.sg} give{}-\textsc{pst.sg}  God-\textsc{gen.sg} bishop-\textsc{dat.sg}\\
\glt ‘and Abraham gave a tenth part of all the plunder to God’s bishop.’

\ex 
\gll \textbf{þa} \textbf{spræc} \textbf{g}uðcyning, ~ ~ ~ ~ Sodoma aldor,  secgum befylled,  ~ ~ ~ ~ to Abrahame (\textit{Genesis A} 2121b--2125a)\\
 þa speak-\textsc{pst.sg} battle-king-\textsc{nom.sg} ~ ~ ~ ~ inhabitant.of.Sodom-\textsc{gen.pl} prince-\textsc{nom.sg}, man-\textsc{dat.pl} deprive-\textsc{pst.ptcp} ~ ~ ~ ~ to Abraham-\textsc{dat.sg}\\
\glt \textbf{Then} \textbf{spoke} the battle-king, the prince of Sodom, deprived of warriors, to Abraham (…).’
\z
\z 
Metrical factors likely play a part in this passage: a compound such as \textit{guðcyning} (which provides the alliterative key) offers limited options and V3 is simply not workable metrically. If the poet was keen on using this compound to designate the prince of Sodom, then using V2 was a handy solution, whereas V3 would have required a noun phrase with only one metrical stress (e.g. \textit{þa se gamola spræc}, “then the old one spoke”, cf. 1295a \textit{þa waldend spræc} “then the ruler spoke”, 1483a \textit{þa to Noe spræc}, “then to Noah spoke”, 1744a \textit{ða se halga spræc} “then the holy one spoke”, 28880a \textit{ða Abraham spræc} “then Abraham spoke). 

It is worth noting, however, that the metrical constraint is not very onerous. In all instances, both V2 and V3 are possible provided a noun phrase of appropriate length is chosen and the wealth of synonyms for male protagonists in Old English is such that this is relatively easily achievable.\footnote{Thus, we could have \textit{ða spræc halig god,         heofonrices weard} “then spoke holy God, the guardian of the kingdom of heaven” instead of \textit{ða se halga spræc,         heofonrices weard}, “then the holy one spoke, the guardian of the kingdom of heaven” (1744); \textit{þa spræc waldend god}, “then spoke God the ruler” instead of \textit{þa waldend spræc} (1294b); \textit{ða spræc Abraham} instead of \textit{ða Abraham spræc} (2880a cf. 2717a above, \textit{þa ongan Abimæleh}, which displays a similar pattern). 1483a would be more difficult to replace without changing the line more substantially.}

The fact that in 4 out of 5 instances the poet favoured V3 over V2 suggests that, unlike the \textit{Heliand} poet, he felt no particular inclination toward V2. At the same time, the fact that we do have one instance of V2 with \textit{sprecan} in an unstressed position where alternatives with V3 (or V1 or V-late, for that matter) were not very difficult to come by suggests that the poet did not consider that this pattern was best avoided with all verbs except \textit{wesan} and \textit{weorðan}. The pattern exists, but in most instances the poet simply chooses not to use it.

In terms of meaning, there is not much that sets the V2 instance apart from its V3 counterparts. If there is an odd one out here, it would rather be 2880 (which uses V3: \textit{ða Abraham spræc         to his ombihtum}, “then Abraham spoke to his servants”), which is a much simpler and straightforward speech introduction than the other three. Otherwise, all instances are very similar in terms of formality, “shock factor” or givenness of information.

\subsubsection{Weorþan}
\largerpage
\label{sec:louviot:5.1.8}
\textit{Genesis A} and the Paris Psalter only use \textit{weorþan} with V2, never with V3, but Cynewulf uses both word orders with this verb.\footnote{Occurrences with V2: \textit{Christ II} 491--492a and 743b--74, \textit{Elene} 69--70a and 584b--585a, \textit{Juliana} 607--608. Occurrences with V3: \textit{Fates of the Apostles} 82b--83, \textit{Juliana} 58--59, 684b--585a, 594b--595a and 669b--671a, \textit{Elene} 988b--990a. Total number of occurrences with V2 in our corpus: 15; with V3: 6.} In 2 instances, the 2\textsuperscript{nd} element in a V3 clause is a past participle (with the subject located after the verb), so the clause is probably better interpreted as V-late than V3. Three of the four other V3 clauses also have a past participle (\textit{weorþan} being used as an auxiliary for a passive construction), but in these it is further down the clause (after the finite verb). The fourth one is used with predicative adjectives instead.

Arguably, there might be a difference in “shock factor” among those instances, though it must be granted such differences are hard to assess objectively. Instances of V3 in \textit{Juliana} are very straightforward in that the events depicted would be shocking in real-life, but are expected and indeed necessary steps in this type of narrative (the villain grows enraged and the saint is finally killed after several bouts of torture). The instance of V3 in \textit{Elene} depicts the joy of Constantine when he receives news that the Cross has been found: again, the event is important, but wholly expected.

One instance of V2 is likewise expected to some extent, but it is a kind of apotheosis and therefore spectacular in that sense (\textit{Christ II} 743b--744): it is the conclusion of a passage on the Ascension and it describes the joy of the inhabitants of Heaven on this glorious occasion. Another instance of V2 describes the terror felt by the wise men after Elene’s threats (584b--585a), which is a logical reaction, but which brings more narrative tension, since it leads them to change tack and betray Judas, when before they had been silent. Two other instances with V2 concern miraculous apparitions (a vision of the cross for Constantine, \textit{Elene} 69--70a and angels lifting Christ from earth to bring him to Heaven, \textit{Christ II} 491--492a). The remaining one depicts the moment when Juliana reacts to her death sentence (607--608): it marks an important step in the narrative (the torture sequence is over and her execution starts) and a moment of intense reversal of emotions, underlined by the use of the \textit{geniwad}{}-formula, which symbolizes the triumph of Christ over death in Cynewulfian texts \citep{Louviot2021Grief}. Therefore, it is arguably more emotionally loaded than the sentence depicting her death, which is expressed rather straightforwardly and with minimum fuss (and with V3, see above).

Another thing that sets the V3 clauses apart is the nature of their second constituent. In three of the four true V3 clauses, the second element is a definite subject lexical noun phrase (with the demonstrative \textit{se} or with a possessive adjective). In the fourth one, it is also a definite lexical noun phrase, but used as an indirect object: however, semantically that object corresponds to the experiencer of the feeling conveyed by the participle, and it is therefore extremely similar to some of the other instances where the syntactic subject has the same semantic role.

In Momma’s terminology, the demonstrative and possessive adjectives are an attached unstressed elements and may be found anywhere in the clause \citep[76]{Momma1997}. However, in our corpus, when such a determinative adjective is used, V3 is much more common than V2, not just with \textit{weorðan}, but with other verbs as well.

To ascertain this, we have examined the first three constituents in all V2 and V3 clauses in our corpus (93 and 76 clauses, so 169 in total). By definition, one of the three constituents is \textit{þa} and another is the verb. The third one can be any number of things (noun phrase, prepositional phrase, adverb, etc.). The first word of that constituent is a demonstrative adjective in 34 instances (20\%). However, there are huge discrepancies in our corpus, among poems and especially among clause types.

Even though \textit{Genesis A} accounts for 50\% of all clauses examined, it only accounts for 32\% of demonstrative adjectives (11 out of 34) whereas Cynewulf accounts for 43\% of all clauses, but 68\% of all demonstrative adjectives (23 out of 34).\footnote{The Paris Psalter never uses the demonstrative adjective in either context, which may or may not be significant, given the small number of clauses overall (12 out of 169, so only 7\% of our V2 and V3 sub-corpus).}This is likely due to the more archaic style of \textit{Genesis A}, which represents a stage in the language where the demonstrative adjective has yet to become a genuine definite article, and is therefore comparatively rarer.

\largerpage
More interestingly, 27 of our 34 instances are found with V3 rather than V2 (79\% overall: 91\% for \textit{Genesis A} and 74\% in Cynewulf), even though V2 is slightly more common overall.

Given that the demonstrative adjective is the ancestor of the definite article, it might be tempting to regard this as evidence of the givenness of the element in second position. However, analysis of clauses with other verbs have shown many instances where the first noun phrase in the sentence referred to given information, whether the demonstrative was used or not, and no correlation with V3 was observed.

In fact, it is much more likely that the high number of noun phrases with a demonstrative in V3 clauses owes more to metrical and prosodic factors than to information structure. Almost all of our instances (33 out of 34) are short lexical noun phrases, which carry only one metrical stress (\textit{þære halgan}, \textit{seo fæmne}, \textit{se snotra}, \textit{se rinc}, \textit{seo cwen}, etc.). Such short noun phrases are relatively rare in Old English poetry, but can be quite convenient fillers in some metrical contexts, particularly when the finite verb occupies the second stressed position in a half-line:

\ea\label{ex:louviot:9}
x   x   /     /    x            x   /     /    x\\
\gll þa se wer \textbf{hyrde}         his waldende (\textit{Genesis A} 2804)\\
then \textsc{dem} man-\textsc{nom.sg} obey-\textsc{pst.sg} \textsc{poss} ruler-\textsc{dat.sg}\\
\glt ‘then the man obeyed his ruler’
\z

\ea
x  x    /  x    /             /        x     / ͜  x   x\\
\gll þa se dema \textbf{wearð} hreoh ond hygegrim (\textit{Juliana} 594b--595a)\\
then \textsc{dem} judge-\textsc{nom.sg} become-\textsc{pst.sg} savage-\textsc{nom.sg} and cruel-\textsc{nom.sg}\\
\glt ‘then the judge became savage and cruel’
\z

V-late can also be achieved in a similar way (metrically speaking), for instance with the addition of a personal pronoun:

\ea\label{ex:louviot:10}
x   x   x    /      x  /            x    /   x   /\\
\gll Þa he his folc \textbf{genam}      swa fæle sceap (Paris Psalter 77, 145)\\
Then he \textsc{poss} people-\textsc{acc.sg} take-\textsc{pst.sg} like faithful-\textsc{acc.sg} sheep-\textsc{acc.sg}\\
\glt ‘Then he took his people like faithful sheep’
\z

It seems likely, then, that the affinity of V3 with this type of noun phrase stems from the fact that such noun phrases make it easier for the poet to achieve V3 or V-late within a single half-line and is not attributable to pragmatic or semantic factors.

\subsubsection{Wesan}
\label{sec:louviot:5.1.9}
\textit{Wesan} is the most frequent verb among those accepting both word orders (40 occurrences with V2 and 6 with V3). Here, V3 is used only in a minority of instances, something that is clearly connected to \textit{wesan}’s status as a the lightest of light verbs. Not only can \textit{wesan} occur in the same contexts where other unstressed light verbs are found, but it can also be found in additional contexts, including with definite lexical noun phrases of the type discussed above:

\ea\label{ex:louviot:11}
\ea
x    x      x   /      x           x     x  x  / ͜   x  /   x\\
\gll ða wæs sio fæmne         mid hyre fæder willan (\textit{Juliana} 32)\\
then be-\textsc{pst.sg} \textsc{dem} young.woman-\textsc{nom.sg} with \textsc{poss} father-\textsc{gen.sg} approval-\textsc{dat.sg}\\
\glt ‘then was the young woman, with her father’s approval...’

\ex
x   x      x   /    x            /   x  / ͜  x   x\\
\gll ða wæs se \textbf{snot}ra         sunu Lamehes (\textit{Genesis A} 1543)\\
then be-\textsc{pst.sg} \textsc{dem} clever-\textsc{nom.sg} son-\textsc{nom.sg} Lamech-\textsc{gen.sg}\\
\glt ‘then was the clever son of Lamech...’
\z
\z

As seen above, such short noun phrases (made up of a demonstrative or a possessive adjective and a non-compound noun) are relatively uncommon and only occur in peculiar metrical contexts. With such noun phrases, other verbs will almost always be stressed and found in 3\textsuperscript{rd} or later position. Only \textit{wesan} (and to a lesser extent \textit{weorþan}) can be found with V2 if the next constituent is such a noun phrase.\footnote{There are occurrences of V2 with such a noun phrase as the third element, most of them in Cynewulf’s poems. 6 of them use the verb \textit{wesan} (\textit{Juliana} 32a, 38a, 287a; \textit{Elene} 894a; \textit{Genesis A} 1543a) and 1 \textit{weorþan} (\textit{Juliana} 607a)}

Conversely, \textit{wesan}’s light status makes it more likely to be found in third position if the other constituent is a personal pronoun, but such instances are rare. In such a case, \textit{wesan} will still be unstressed. According to Momma, pronouns may be found either before or after an unstressed finite verb, but there are marked preferences for some verbs. \textit{Wesan} is five times more likely to be found after rather than before the pronoun \citep[113--126]{Momma1997}. A quick survey of the Anglo-Saxon Poetic Records suggests that the trend might be even stronger in clauses starting with \textit{þa}: we have found no instance of \textit{þa wæs} + personal pronoun, whereas the order \textit{þa} + personal pronoun + \textit{wæs} is attested a few times (five in total). This happens twice in our corpus: \textit{Genesis~A} 1591a and 2237b.

When the second element is longer than such a pronoun, \textit{wesan} is stressed, which is an infrequent occurrence (\textit{Elene} 883b--884a, 1062b--1063, \textit{Genesis~A} 162b--163a, 1719), though by no means an extraordinary one. Three of the four instances are found in the b-verse, so that simply inverting the 2\textsuperscript{nd} and the 3\textsuperscript{rd} constituent, which would yield a light verse with only one stressed position, is not an option. For the instance located in the a-verse, on the other hand, such a manipulation would easily be possible:

\ea\label{ex:louviot:12}
\ea
    x   x      /    x    /               /         x /    x\\
\gll þa þæs mæles wæs         mearc agongen (\textit{Genesis A} 1719)\\
then \textsc{dem} appointed.time-\textsc{gen.sg} be-\textsc{pst.sg} term-\textsc{nom.sg} pass-\textsc{pst.ptcp}\\
\glt ‘Then the right time had come...’
a’\\
x    x      x      /    x           /         x /    x\\
þa wæs þæs mæles         mearc agongen
\z
\z

\textit{Wesan} is unambiguously used as an auxiliary here, so it cannot have a greater semantic weight than in other instances. It seems more likely that the poet is deliberately manipulating word order to make the beginning of this new fitt in the poem more stately and ponderous.\footnote{The new fitt starts on p. 83 of the Junius 11 manuscript: the previous page has been left incomplete to leave room for an illustration and a new section starts at the beginning of this page, marked with a large ornamental capital letter.} The placement of the finite verb between the genitive and the noun it complements slows down the rhythm of this sentence, and delays the announcement of Abraham’s marriage. The announcement of the name of Abraham’s bride is itself delayed by other rhetorical devices, so that it only appears on line 1723.\footnote{This can be compared with the very terse and direct account provided in the Vulgate: \textit{Duxerunt autem Abram et Nachor uxores: nomen uxoris Abram, Sarai: et nomen uxoris Nachor, Melcha} (“Then Abram and Nahor took wives: the name of Abram’s wife was Sarai and the name of Nahor’s wife Milcah”). By comparison, the Old English poet avoids references to secondary characters and focuses all his attention on Abraham and Sarah, to celebrate the rightness and the solemnity of their union.}

This exception to a general trend only confirms \textit{wesan} peculiar status: whereas other light verbs can often be promoted to a stressed position at the end of a half-line, \textit{wesan} will very rarely appear in such a position unless it serves a specific rhetoric effect.

\subsubsection{Conclusion on Old English}
\label{sec:louviot:5.1.10}
This comparison of V2 and V3 clauses shows that multiple factors are at play in our corpus. Metrical and prosodic factors are undeniably important: not only do light verbs behave differently from other verbs but, within that category, there is a gradient from very (\textit{wesan}) to slightly light verbs (\textit{sprecan}), showing preferences for different patterns. Since several patterns do occur with the same verb, but with different frequencies, it seems that the preference has more to do with prosody than with metre strictly speaking: several patterns may be metrically licit, but some of them apparently sound better to the poets.

Additionally, in any given context, local metrical constraints may influence the choice of the poet in either direction. That being said, metre is no straightjacket. With all the (highly frequent) verbs examined in this section, both V2 and V3 can yield perfectly legitimate metrical patterns in most contexts, provided phrases of appropriate length are selected to complete the line, something a competent poet would have had no problem doing, as demonstrated in the examples above.

In this corpus, opportunities to use V2 are limited by the metrical constraints, but they are not that rare. However, these opportunities are far from being seized every time they present themselves, which suggests that not only is V2 not the standard word order for declarative clauses in this corpus, but there is no obvious push towards V2, as one might expect if there were a conflict between the poet’s sense of grammaticality and metrical constraints inherited from an earlier language stage.

While there is no doubt that information structure and discursive aims are important factors in determining the order in which constituents appear in a clause generally, there is no evidence in our corpus of givenness of the referent of the first noun phrase increasing the likelihood of V3 being used over V2.

What we observe here is a heterogeneous word order, where the preferred patterns of certain verbs, which seem largely determined by their weight, can be manipulated for effect. Thus, verbs with a preference for a later position seem more likely to appear with V2 when conveying spectacular or surprising events. Conversely, \textit{wesan}, which has the strongest preference for V2, can appear with V3 to create a stately, ponderous effect.

\subsection{Heliand}
\label{sec:louviot:5.2}
As mentioned above, in \textit{Heliand}, \textit{thô} followed by V2 is by far the dominant pattern, though several alternatives exist, including patterns where \textit{thô} is not in initial position (\tabref{tab:louviot:5}).


\begin{table}
\begin{tabular}{llrr}
\lsptoprule
 &  &  Number of   &      \\
 &  &  occurrences & Total\\
\midrule
Initial \textit{thô} & \textit{Thô} V…\textit{V2} & 307 & 429\\
& \textit{Thô} X V…\textit{V3} & 23  & \\
& \textit{Thô} X… V (…) \textit{V-final} & 90 & \\
& \textit{Thô} X V \textit{(ambiguous)} & 9  & \\
\addlinespace
Post-verbal \textit{thô} & V \textit{thô}…\textit{V1} & 25 & 212\\
& V… \textit{thô}…\textit{V1} & 99 & \\
& X V \textit{thô}…\textit{V2} & 40 & \\
& X V… \textit{thô} (…)\textit{V2} & 35 & \\
& Other & 13 & \\
\addlinespace
Non-initial, but  & X \textit{thô} V…\textit{V3} & 6 & 58\\
preverbal         & Other & 52 & \\
\addlinespace
Total&  &  & 699\\
\lspbottomrule
\end{tabular}
\caption{Syntactic patterns in \textit{Heliand}}
\label{tab:louviot:5}
\end{table}

Among the 23 occurrences of V3 with initial \textit{thô} in this corpus, 6 have the subject as their second element, and among them 3 are simple pronominal subjects (without the intensifier \textit{selƀo}), thereby corresponding in appearance to the prototypical V3 pattern identified in Old English prose, and which inspired the clitic hypothesis. However, examination of the 3 clauses quickly reveals that something very different is going on here. In all three, the clause should be interpreted as a subordinate clause and V3 as V-final, since what follows the verb is admittedly its direct object, but this direct object is in fact an infinitive clause, as in the following example:

\ea\label{ex:louviot:13}
\gll thuo hie gihôrda   \textbf{thia} \textbf{man} \textbf{sprecan} (5336b)\\
when he hear-\textsc{pst.sg} \textsc{dem} man-\textsc{acc.pl} speak-\textsc{inf}\\
\glt ‘when he heard the men speak’
\z

If we add to this the occurrence where the second element is a past participle and therefore indicative of a V-final word order (\textit{Heliand} 4865b), that brings our total number of genuine V3 occurrences down to only 19.

\subsubsection{Verb diversity}

Given the significant differences observed in Old English regarding the syntactic patterns favoured by individual verbs, it makes sense to examine this parameter in this section of the corpus as well.

However, \textit{Heliand} yields very different results. As shown above, for clauses with initial \textit{thô}, the most common word order overall is V2, followed by V-late\slash final, with V3 occurring quite rarely. The same hierarchy can be observed for each verb considered individually and the only verbs occurring more than once with V3 are verbs which have many more occurrences with V2 (\tabref{tab:louviot:6}). Thus, the verbs most commonly used with V3 are \textit{sprekan} and \textit{uuerðan}, but the proportion of occurrences of those verbs using V3 is perfectly in line with the proportion of V3 overall in clauses with initial \textit{thô}.\footnote{There are 23 clauses with V3 out of 429 with initial \textit{thô} or 5\%. 4 instances of \textit{sprekan} out of 73 amounts to 5\%, and 5 occurrences out of 77 for \textit{uuerðan} amounts to 6\%.}


\begin{table}
\begin{tabular}{l *4{r}}
\lsptoprule
                 &    &    & V-late/ & \\
                 & V2 & V3 & final & Total\\\midrule
\textit{uuerðan} & 70 & 5 & 2 & 77\\
\textit{sprekan} & 55 & 4 & 14 & 73\\
\textit{gangan} & 19 & 0 & 9 & 28\\
\textit{uuesan} & 21 & 1 & 3 & 25\\
\textit{giuuîtan} & 21 & 0 & 3 & 24\\
\textit{kuman} & 8 & 2 & 7 & 17\\
\textit{hebbian} & 13 & 1 & 2 & 16\\
\textit{biginnan} & 14 & 0 & 0 & 14\\
\textit{hêtan} & 8 & 0 & 4 & 12\\
\textit{gisehan} & 6 & 0 & 6 & 11\\
\textit{faran} & 4 & 1 & 3 & 8\\
\textit{gifregnan} & 7 & 0 & 0 & 7\\
\textit{uuillian} & 6 & 0 & 1 & 7\\
\textit{gihôrian} & 2 & 1 & 2 & 5\\
\textit{môtan} & 6 & 0 & 0 & 6\\
\textit{gibiodan} & 0 & 3 & 2 & 5\\
\textit{niman} & 4 & 0 & 1 & 5\\
\textit{sehan} & 4 & 0 & 1 & 5\\
\textit{skulan} & 4 & 0 & 1 & 5\\
\lspbottomrule
\end{tabular}
\caption{Patterns favoured by verbs with at least 5 occurrences in clauses with initial \textit{thô}}
\label{tab:louviot:6}
\end{table}

The only exception to that trend is the verb \textit{gibiodan} (“to command”), which occurs 3 times with V3 and twice with V-late/final, even though it is not a common verb overall.\footnote{Aside from the 5 occurrences with initial \textit{thô} examined here, there is only one other instance in our corpus, where the pattern is [subject noun phrase + verb + \textit{thô}] and where \textit{thô} is directly followed by a noun clause reporting a speech indirectly (l. 520).}

In all five instances, the verb is located at the very end of the half-line, while a lexical noun phrase denoting the speaker is found in the following half-line (or the one after, in one case):

\ea\label{ex:louviot:14}
\ea
\gll  Thô he selƀo \textbf{gibôd}, \textbf{uualdand} mid is uuordun (4502b--4503a)\\
Then he \textsc{intensive} command-\textsc{pst.sg}, ruler-\textsc{nom.sg} with \textsc{poss} word-\textsc{dat.pl}\\
\glt ‘Then he himself commanded, the ruler with his words’

\ex  
\gll Thô sô stillo \textbf{gebôd} \textbf{mahtig} \textbf{barn} \textbf{godes} (2037b--2039a)\\
Then so quietly command-\textsc{pst.sg} mighty-\textsc{nom.sg} child-\textsc{nom.sg} God-\textsc{gen.sg}\\
\glt ‘Then so quietly commanded the mighty son of God’

\ex  
\gll Thô he sô hardo \textbf{gibôd}, \textbf{Erodes} oƀar is riki (727b--728a)\\
Then he so strictly command-\textsc{pst.sg} Herod-\textsc{nom.sg} over \textsc{poss} kingdom-\textsc{acc.sg}\\
\glt ‘Then he very strictly commanded, Herod throughout his kingdom’
\z
\z

It seems that what we have here is a conventional formula to report a command, and that it requires placing the verb beyond the second position, either in third or fourth position, depending on the weight of the intervening elements.

That this formula was conventional is all the more likely that it can also be found in the \textit{Evangelienbuch}, even though the poem uses a style usually quite different from the traditional formulaic style of alliterative poetry. In the \textit{Evangelienbuch}, the formula typically uses \textit{so} instead of \textit{thô} and the subject is placed in second position (without any additional subject noun phrase located after the verb), but the verb is also usually in third position. The most common form is \textit{so druhtin sélbo gibót} (“as the lord himself ordered”: first dedicatory letter, line 71; book II, 9, l. 50; II, 12, l. 64), but other close variants can be found, such as \textit{so ther wízzod gibot} (“as the commandment required”, book III, 15, l. 6 and III, 16, l. 42) or \textit{Er in sélbo gibot} (“he in person ordered”, III, 13, l. 57).

In itself, this formula is interesting because it requires a form of V-late (V3 or later) and in \textit{Heliand}, it is never used in a way suggesting it could be interpreted as a subordinate clause (as opposed to the way it is used in the \textit{Evangelienbuch}). This implies that V3 was likely a legitimate word order for independent clauses starting with \textit{thô} at some point in Old Saxon, though it may have become less productive and started to become fossilized by the time \textit{Heliand} was composed. However, it is important to bear in mind that V3 appears to be a kind of V-late in this pattern (since V4 also occurs), so this is quite different from the V3 patterns identified in Modern Germanic languages.

The way the formula is used in \textit{Heliand} also provides us with important clues which may help us account for most remaining instances. The point of the formula seems to be the emphatic representation of a speech act by a powerful speaker. In all instances, the subject is either split or delayed, with the result that a noun phrase denoting the speaker appears in a marked position.

\subsubsection{Subject in marked position}

Something very similar occurs with most other instances of V3 in our corpus. In 15 of our 19 instances, V3 is immediately followed by a subject lexical noun phrase (the second position being occupied by an adjunct in most cases); in one more instance, V3 is followed by the direct object, but a subject lexical noun phrase is used in fifth position; and in one more instance a genitive noun referring to the subject is used in fourth position instead, so that we have a split subject once again:

\ea\label{ex:louviot:15}
\gll thuo \textbf{thiu} \textbf{seola}       quam                  / \textbf{Iudases} an grund grimmaro helliun\\
     when \textsc{dem}  soul-\textsc{nom.sg} come-\textsc{pst.sg} {} Judas-\textsc{gen.sg} on bottom-\textsc{acc.sg} bitter-\textsc{gen.sg} hell-\textsc{gen.sg}\\
\glt ‘when the soul of Judas came to the bottom of bitter Hell’ (5428b--5429)
\z

Whatever the nature of the second constituent (either a subject noun phrase itself or a prepositional phrase), the overall effect of the pattern is the same: the subject noun phrase located after the verb receives maximum focus, which can be reinforced by various rhetorical devices, as in this instance:

\ea\label{ex:louviot:16}
\gll Thô te thes uuatares staðe \textbf{samnodun} \textbf{thea} \textbf{gesîðos} \textbf{Cristes}, \textbf{the} he imu habde selƀo \textbf{gicorane,} \textbf{sie} \textbf{tueliƀi} thurh iro treuua gôda (2902--2903)\\
Then to \textsc{dem} water-\textsc{gen.sg} place-\textsc{dat.sg} gather-\textsc{pst.pl} \textsc{dem} companion-\textsc{nom.pl} Christ-\textsc{gen.sg} \textsc{rel} he \textsc{refl} have-\textsc{pst-sg} \textsc{ints} choose-\textsc{pst.ptcp}, \textsc{dem} twelve-\textsc{acc.pl} through \textsc{poss} true-\textsc{acc.sg} faith-\textsc{acc.sg}\\
\glt ‘Then to the riverside the companions of God came together, those that he had chosen in person, the twelve for their good faith’
\z

The alliterating \textit{th}{}-sound in \textit{thô, thes, thea}, the rhythmical interplay of light and heavy elements line 2902, the relative clause highlighting the elect status of the apostles and the two noun phrases placed in late (and thus marked) position (\textit{thea gesîðos Cristes} after V3, and \textit{sie tueliƀi thurh iro treuua gôda} after the relative clause) all contribute to the same rhetorical strategy which puts maximum focus on the apostles.

\subsubsection{Conclusion on Heliand}

In \textit{Heliand}, V2 is extremely dominant, but word order is flexible enough that other options can occasionally be chosen. In clauses starting with \textit{thô}, the poet has the option of delaying the verb until the third or a later position and of placing a subject noun phrase after the verb. The effect of that marked word order is to place maximum focus on the subject, whereas the pattern \textit{thô V subject} would be unmarked and therefore much less emphatic. The verb \textit{gibiodan}, which denotes an emphatic speech act, is almost systematically found with that pattern, but otherwise it does not seem to be associated with or restricted to any particular verb.

\subsection{Evangelienbuch}

In the \textit{Evangelienbuch} as in \textit{Heliand}, V2 is very much dominant with initial \textit{thô}, though once again competing with other patterns, including patterns where \textit{thô} is not initial (\tabref{tab:louviot:7}).


\begin{table}
\begin{tabular}{ll rr}
\lsptoprule
&                                             & $n$ & \% \\\midrule
\multicolumn{2}{l}{Initial \textit{thô}}      & 203\\
& \textit{Thô} V… \textit{V2}                 & 124 & 61\%  \\
& \textit{Thô} X V… \textit{V3}               & 15  & 7\% \\
& \textit{Thô} X… V (…) \textit{V-late/final} & 63  & 31\% \\
& No verb                                     & 1   & <1\% \\
\addlinespace
\multicolumn{2}{l}{Post-verbal \textit{thô}}  & 312\\
& V \textit{thô}… \textit{V1}                 & 55  & 18\%\\
& V… \textit{thô}… \textit{V1}                & 142 & 46\%\\
& X V \textit{thô}… \textit{V2}               & 33  & 11\%\\
& X V… \textit{thô} (…) \textit{V2}           & 65  & 21\%\\
& Other                                       & 17  & 5\%\\
\addlinespace
\multicolumn{2}{l}{Non-initial, but preverbal}& 179\\
& X \textit{thô} V… \textit{V3}               & 9   & 5\%\\
& Other                                       & 170 & 95\%\\
\addlinespace
\multicolumn{2}{l}{Non-initial, other}        & 6   & \\
& No explicit verb                            & 6   & \\
Total&                                        & 700\\
\lspbottomrule
\end{tabular}
\caption{Syntactic patterns in the \textit{Evangelienbuch}. Percentages are rounded and may add up to more than 100\%.}
\label{tab:louviot:7}
\end{table}

\subsubsection{Verb diversity}

As in \textit{Heliand}, there seems to be very little restriction as regards verb diversity with V2. Comparison of random sets comprised of 50 verbs shows that verb diversity is slightly higher in V-late/final clauses than in V2 clauses, but not hugely so.\footnote{The median type/token ratio of the samples tested was 0.62 for V2, against 0.76 for V-late/final, to be compared with \textit{Heliand}’s 0.40 for V2 and 0.64 for V-late/final and \textit{Genesis A}’s 0.40 for V2 and 0.74 for V3. This means that the \textit{Evangelienbuch} is both the text with the greatest lexical variety (at least where verbs are concerned) and the text where V2 shows the least restriction. Likewise, the 3 most common verbs in samples from the \textit{Evangelienbuch} only accounted for 34\% of all tokens in V2 clauses (against 48\% in \textit{Heliand} and 60\% in \textit{Genesis~A}).} Again, this is consistent with a slight semantic restriction with \textit{thô} more likely to be used as a discourse marker signalling a new stage in the narrative with V2, but not with a limitation on V2’s productivity as observed in Old English.

An examination of verbs totalling at least 5 occurrences in clauses starting with \textit{thô} shows that most verbs favour V2, with the exception of \textit{biginnan} (which favours V-late/final) and \textit{willon}, which favours both V3 and V-late/final over V2, which is all the more surprising that V3 is quite rare overall.{\interfootnotelinepenalty=10000\footnote{3 instances of \textit{biginnan} with V-late/final against 2 with V2 and 0 with V3; 3 instances of \textit{willon} with V3, against 1 with V2 and 2 with V-late/final.}}

One thing that these two verbs have in common is that they typically take an infinitive verb or a full infinitive clause as their object. We have already seen instances above, when V3 or V-late/final was favoured over V2 in such contexts, so as to avoid confusion regarding which clause a noun phrase belongs to.

The occurrences of \textit{wellen} in our corpus are quite short and straightforward, so such confusion is unlikely here. They are also strikingly similar, suggesting the influence of a potential conventional formula (\textit{tho drúhtin wolta INFINITIVE}), though it has proved impossible to find further evidence of such a formula:

\ea\label{ex:louviot:17}
\ea \gll tho drúhtin \textbf{wolt} es \textbf{wáltan},      fon themo grábe \textbf{irstántan}; (III, 7, l. 6)\\
tho lord-\textsc{nom.sg} want-\textsc{pst.sg} \textsc{pron-gen.sg} wield-\textsc{inf}, from \textsc{dem} grave-\textsc{dat.sg} rise\textsc{{}-inf}\\
\glt ‘Then/when the lord \textbf{decided} \textbf{to} \textbf{take} \textbf{control} of it, \textbf{to} \textbf{rise} from the grave’

\ex  
\gll Thaz ward állaz so gidán,      tho selbo drúhtin \textbf{wolta} \textbf{irstán} (IV, 34, l. 11)\\
\textsc{dem} \textsc{aux-pst.sg} all-\textsc{nom.sg} so do-\textsc{pst.ptcp}, tho \textsc{dem} \textsc{ints} lord-\textsc{nom.sg} want-\textsc{pst.sg} resurrect-\textsc{inf} \\
\glt ‘It was all done so, when the lord himself \textbf{decided} \textbf{to} \textbf{resurrect’}

\ex  
\gll Tho drúhtin \textbf{wolta} \textbf{réison},      sin selbes ríches \textbf{wison} (V, 16, l. 1)\\
Tho lord-\textsc{nom.sg} want-\textsc{pst.sg} travel-\textsc{inf}, \textsc{poss} \textsc{ints} kingdom-\textsc{gen.sg} visit-\textsc{inf}\\
\glt ‘Then the lord \textbf{wanted} \textbf{to} \textbf{travel}, \textbf{to} \textbf{visit} his own kingdom’
\z
\z

Another reason why clauses using \textit{willon} might be overrepresented with V3 is simply that such clauses tend to contain very few constituents outside the subordinate infinitive clause. Whenever a V-final clause using \textit{willon} appears, it is likely to \textit{look like} a V3 clause, simply because it contains so few constituents. However, this does not explain why \textit{biginnan} and \textit{willon} are often found with V-late/final in the first place.

\subsubsection{Second element in thô X V}

Our first findings suggest that \textit{thô X V} is a very marginal pattern in the \textit{Evangelienbuch}, which may be triggered in certain peculiar contexts.

Still, it is worth taking a closer look to the second constituent in this pattern, because it tends to match the prototypical features identified in prose corpora much more closely than in \textit{Heliand} or in the Old English section of our corpus.

In our 15 relevant clauses, the second constituent is often a pronoun (8 occurrences) or a subject noun phrase (11 occurrences), and sometimes both \REF{ex:louviot:5}.

By nature, pronouns point to a referent that is recoverable, as in this example where the pronoun \textit{sie} (“they”) clearly refers anaphorically to \textit{thie mán} (“the people”).

\ea\label{ex:louviot:18}
\gll \textbf{Thie} \textbf{mán} thoh thie thar scanktun,      iz fílu wola irkántun, theiz wazzar lútaraz was,      tho \textbf{sie} fúltun thiu vaz.   (II, 8, l. 41--42)\\
\textsc{dem} person-\textsc{nom.pl} however \textsc{rel} there pour-\textsc{pst.sg} pron-\textsc{acc.sg} very completely recognize-\textsc{pst.pl}, \textsc{rel-dem} water-\textsc{nom.sg} clear-\textsc{nom.sg} be-\textsc{pst.sg}, tho they fill-\textsc{pst.sg} \textsc{dem} barrel-\textsc{acc.sg}\\
\glt ‘\textbf{The} \textbf{people}, however, who poured (the drinks) there wholly recognized this, / that this was clear water when \textbf{they} filled up the barrel.’
\z

The noun phrases used in the same position also mostly refer to elements known to the audience: “the lord” (three occurrences), “the week” (referring to the Feast of Tabernacles, already referred to in the text), “all this world”.\footnote{Respectively, III, 7, l. 6; IV, 34, l. 11; V, 16, l. 1; III, 16, l. 1; I, 23, l. 1.} The only new element is “the seventh hour of the (previous) day”, which introduces a new time setting at the beginning of a speech (III, 2, l. 31). 

However, the example quoted above (II, 8, l. 41--42) reveals another important feature of those clauses: most often, they are best interpreted as subordinate clauses. Out of the 15 clauses considered, at least 10 make most sense if the clause is interpreted as subordinate and only 3 would make no sense if interpreted as subordinate:

\ea\label{ex:louviot:19}
\ea 
\gll Tho drúhtin wolta réison,      sin selbes ríches wison (V, 16, l. 1)\\
Tho lord-\textsc{nom.sg} want-\textsc{pst.sg} travel-\textsc{inf,} \textsc{poss} \textsc{ints} kingdom-\textsc{gen.sg} visit-\textsc{inf}\\
\glt ‘Then the lord wanted to travel, to visit his own kingdom’

\ex
\gll Tho er déta thaz sih zárpta      ther hímil sus io wárpta (II, 1, l. 21)\\
Tho he do-\textsc{pst.sg} \textsc{conj} \textsc{refl} turn-\textsc{pst.sg} \textsc{dem} sky-\textsc{nom.sg} thus so twist-\textsc{pst.sg}\\
\glt ‘Then he did (so) that on itself turned the sky (and) thus twisted’

\ex 
\gll Tho thisu wórolt ellu      quám zi theru stúllu, ouh zi theru zíti,      thaz Kríst sih iru iróugti (I, 23, l. 1--2)\\
Tho \textsc{dem} world-\textsc{nom.sg} all come-\textsc{pst.sg} to \textsc{dem} hour-\textsc{dat.sg}, also to \textsc{dem} time-\textsc{dat.sg,} \textsc{conj} Christ-\textsc{nom.sg} \textsc{refl} \textsc{pron-dat.sg} reveal-\textsc{pst.sg}\\
\glt ‘Then all this world came to this hour and this time when Christ revealed himself to it (= to this world)’
\z
\z

The first instance has already been briefly commented above. The second pre\-sents a similar pattern in that the verb is immediately followed by a subordinate clause which is the complement of the verb. As noted above, it is not impossible for such structures to show greater tolerance for V-late as a way to clarify both where the subordinate clause starts and what the connection between the two clauses is, through keeping the verb and its complement together. Either way, the clauses are V-final as much as they are V3. There is no obvious explanation for the third instance, except poetic license: the placement of the verb in third position, the use of a very heavy prepositional phrase (\textit{zi theru stullu ouh zi theru zíti,      thaz} \textit{Kríst sih iru iróugti}) and the placement of the verb denoting the main action inside the relative clause (in final position) all contribute to a very ponderous and solemn opening.


\subsubsection{Conclusion on the \textit{Evangelienbuch}}

A superficial survey of the \textit{Evangelienbuch} might lead us to believe that \textit{thô X V} as an alternative to \textit{thô V X} exists in that text, particularly when X is a subject noun phrase. However, careful examination of all occurrences suggests that such is not the case. Rather, V-late/final clauses with a subordinate meaning may occasionally take the form V3 rather than V4 or beyond. Additionally, some independent clauses starting with \textit{thô} may also have their verb beyond the second position (third or later) in some contexts (for instance when the verb takes a subordinate clause as a complement or when the poet is striving for a particular rhetorical effect). Nothing in this corpus suggests the existence of a specific V3 pattern distinct from V-late.

\section{General Conclusion}

This comparative study largely confirms that V3 is significantly more common in Old English than in Old Saxon or Old High German. This is hardly surprising in itself, but it does show that metrical constraints only have a limited impact on word order: they exert some influence, but they do not impose an alien word order on a given language. Old English poetry, and \textit{Genesis~A} in particular, may be extremely close to \textit{Heliand} in terms of metre and poetic style, but when it comes to word order they behave very differently. Conversely, the \textit{Evangelienbuch} uses a kind of verse very different from \textit{Heliand}, but it shares a marked preference for V2 across (almost) all verbs. This confirms that poetic texts do not behave randomly or according to idiosyncratic rules independent from the syntactic system of the language in which they operate. Like any textual genre, poetic texts have their own specificities, but they provide valuable material for syntactic analysis.

Another important conclusion of this work is that the existence of V3 as an alternative to V2 in Old Germanic languages is a thorny issue, which cannot ignore the fact that those languages show much more variety and flexibility than present-day languages where word order is concerned. Even in the texts where V2 is most dominant, it competes with V3, but also with V1, V-late and V-final, including in independent clauses, and in some cases it is far from obvious that V3 should be considered alongside V2 rather than V-late.

In \textit{Heliand} and in the \textit{Evangelienbuch}, there is significant evidence that V3 should be regarded as one possible form taken by V-late rather than as a variant of V2. In our Old English corpus, the likelihood of V3 as a (slight) deviation from V2 is even smaller, because V2 is not the default pattern with initial \textit{þa}.

Of course, the significance of the evidence provided here should not be overstated. Our study concerns clauses with initial \textit{þa/thô} exclusively, in which it has been claimed that V3 does not exist, so in a sense this paper confirms that claim: although clauses with the verb in third position do occur with initial \textit{þa/thô} in our corpus, they do not seem to be V3 clauses in the sense meant by Walkden, Eythórsson, van Kemenade and others.
\sloppy\printbibliography[heading=subbibliography,notkeyword=this]
\end{document}
