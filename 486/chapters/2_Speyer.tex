\documentclass[output=paper]{langscibook}
\ChapterDOI{10.5281/zenodo.17077184}
\author{Augustin Speyer\orcid{}\affiliation{Universität des Saarlandes}}
\title{Verb-third in German: A diachronic perspective}
\abstract{German follows the V2 constraint rather strictly which is a consequence of all prefield phrases (i.e. phrases standing before the finite verb in main clauses) having to pass through SpecFinP. Considerations such as movement being feature-driven lead to the assumption that in German a richly articulated CP-structure is active nevertheless. Data from Early New High German suggest that independent movement of more than one constituent to the prefield is a grammatical option in that period. This is analyzed best as an articulated CP-architecture with violation of the bottleneck effect. At the same time, some examples are only analyzable as movement of a partially emptied VP (remnant-VP-movement). In Modern German, most instances of multiply filled prefields are in fact instances of remnant-VP-movement, although a residue of independent movement of more than one constituent is still active. This change of frequency is interpreted as the result of further strengthening of strict V2 syntax.}
\IfFileExists{../localcommands.tex}{
  \addbibresource{../localbibliography.bib}
  % add all extra packages you need to load to this file

\usepackage{tabularx,multicol}
\usepackage{url}
\urlstyle{same}

\usepackage{listings}
\lstset{basicstyle=\ttfamily,tabsize=2,breaklines=true}

\usepackage{langsci-basic}
\usepackage{langsci-optional}
\usepackage{langsci-lgr}
\usepackage{langsci-osl}
% \usepackage{./langsci/styles/langsci-lgr}
% \usepackage{./langsci/styles/langsci-osl}
% \usepackage{langsci-gb4e}

\usepackage{tikz}
\usetikzlibrary{patterns,calc}
\pgfdeclarepatternformonly{south east lines}{\pgfqpoint{-0pt}{-0pt}}{\pgfqpoint{3pt}{3pt}}{\pgfqpoint{3pt}{3pt}}{
    \pgfsetlinewidth{0.6pt}
    \pgfpathmoveto{\pgfqpoint{0pt}{3pt}}
    \pgfpathlineto{\pgfqpoint{3pt}{0pt}}
    \pgfpathmoveto{\pgfqpoint{.2pt}{-.2pt}}
    \pgfpathlineto{\pgfqpoint{-.2pt}{.2pt}}
    \pgfpathmoveto{\pgfqpoint{3.2pt}{2.8pt}}
    \pgfpathlineto{\pgfqpoint{2.8pt}{3.2pt}}
    \pgfusepath{stroke}}
    
\usepackage{stmaryrd}
\usepackage{wasysym}
\usepackage{multirow}
\usepackage{caption}
\usepackage{subcaption}
\usepackage{mathrsfs}
\usepackage{qtree}

\usepackage{linguex}


  %pminos do not split footnotes
% \interfootnotelinepenalty=10000 %Footnote in Laporte chapters has to be split SN


%\DeclareIndexNameFormat{default}{%
%\nameparts{#1}%
%\usebibmacro{index:name}%
%{\index[names]}%
%{\namepartfamily}%
%{\namepartgiveni}%
% {}% L1
% {}% L2
%{\namepartprefix}% generates spurious space L3
%{\namepartsuffix}% generates spurious space L4
%}

%  {\DeclareIndexNameFormat{default}{%
%     \usebibmacro{index:name}{\index[names]}{#1}{#3}{#5}{#7}}}

%\DeclareIndexNameFormat{default}{%
%  \usebibmacro{index:name}{\sindex[nom]}{#1}{#3}{#5}{#7}}

%\DeclareIndexNameFormat{default}{%
%  \usebibmacro{index:name}{\sindex[person]}{#1}{#3}{#5}{#7}}
%\DeclareIndexNameFormat{default}{%
%\nameparts{#1} \usebibmacro{index:name}{\sindex[person]]}{\namepartfamily}{‌​\namepartgiven}{\nam‌​epartprefix}{\namepa‌​rtsuffix}}

%\newcommand{\smiley}{:)}

%\renewbibmacro*{index:name}[5]{%
%\usebibmacro{index:entry}{#1}%
%{\iffieldundef{usera}{}{\thefield{usera}\actualoperator}\mkbibindexname{#2}{#3}{#4}{#5}}}

% \newcommand{\noop}[1]{}

%remove for final
%\overfullrule=1mm

\newcommand{\tobi}[2]}}
\renewcommand{\S}[1]{\tobi{#1}{\textsc{*}}}

% this volume references
% puts: [this volume]
% already defined: \citetv
%\newcommand{\citepv}[1]{(\citeauthor{#1} \citeyear*{#1} [this volume])}
\newcommand{\citealtv}[1]{\citeauthor{#1} \citeyear*{#1} [this volume]}

%parentheses around example number
\newcommand{\pref}[1]{(\ref{#1})}

% in-text examples

\newcommand{\lnex}[1]{\textit{#1}} %target lang word
\newcommand{\lnlit}[1]{(lit.: `#1')} %literal reading
\newcommand{\lnlat}[1]{(#1)} % latinization
\newcommand{\lntrans}[1]{`#1'} %translation
\newcommand{\lnexl}[2]%
{\lnex{#1}{} \lnlat{#2}} % ex with latinization
\newcommand{\lnexlat}[3]{\lnex{#1}{} \lnlat{#2}{} \lntrans{#3}} % ex with latinization and tranl.

%ch01
\newcommand{\co}[1]{\mbox{\textbf{#1}}}

%ch09

\newcommand{\cyrbulg}[1]{\begin{otherlanguage*}{bulgarian}#1\end{otherlanguage*}}


%ch10
\newcommand{\nlp}{{\small NLP}}
\newcommand{\mwe}{{\small MWE}}
\newcommand{\rae}{{\small RAE}}
\newcommand{\lvc}{{\small LVC}}
\newcommand{\pos}{{\small P}o{\small S}}
%\newcommand{\todo}[1]{ \textcolor{red}{#1} }

%\renewcommand{\labelenumi}{\theenumi}
%\ainamefmt{{vv}{ll}{, ff}{, jj}} % fullname

\newcommand{\biberror}[1]{{\color{red}#1}}

\newcommand{\osenovaitem}{--~} 
  %% hyphenation points for line breaks
%% Normally, automatic hyphenation in LaTeX is very good
%% If a word is mis-hyphenated, add it to this file
%%
%% add information to TeX file before \begin{document} with:
%% %% hyphenation points for line breaks
%% Normally, automatic hyphenation in LaTeX is very good
%% If a word is mis-hyphenated, add it to this file
%%
%% add information to TeX file before \begin{document} with:
%% %% hyphenation points for line breaks
%% Normally, automatic hyphenation in LaTeX is very good
%% If a word is mis-hyphenated, add it to this file
%%
%% add information to TeX file before \begin{document} with:
%% \include{localhyphenation}
\hyphenation{
    Beck-man
    Ngu-yen
    back-chan-nel
    back-chan-nels
    mo-not-o-nous
    ste-reo-typ-i-cal
}

\hyphenation{
    Beck-man
    Ngu-yen
    back-chan-nel
    back-chan-nels
    mo-not-o-nous
    ste-reo-typ-i-cal
}

\hyphenation{
    Beck-man
    Ngu-yen
    back-chan-nel
    back-chan-nels
    mo-not-o-nous
    ste-reo-typ-i-cal
}
 
  \togglepaper[1]%%chapternumber
}{}

\begin{document}
\maketitle 
%\shorttitlerunninghead{}%%use this for an abridged title in the page headers


\section{Introduction}
\label{sec:speyer:1}

German is known to be a language that follows rather strictly the Verb\hyp second\hyp constraint (in the following: V2) in declarative and wh\hyp interrogative main clauses. This means that the German declarative main clause is built in such a way that some constituent comes first, regardless of its syntactic function, and is followed by the finite part of the verb \REF{ex:speyer:1}. This state of affairs is implemented in the topological field model, in that the position of the finite verb (the “left sentence bracket”) is preceded by the so-called “prefield” which normally hosts exactly one constituent (e.g. \citealt{Wöllstein2010}). The generative term “left periphery” corresponds to prefield plus left sentence bracket in the topological field model. 

\ea \label{ex:speyer:1}
\ea \label{ex:speyer:1a}
\gll Gestern hat Uller ein Buch für Gwendolyn gekauft.\\
       yesterday has Uller a book for Gwendolyn bought\\
\glt ‘Yesterday, Uller bought a book for Gwendolyn.’
\ex\label{ex:speyer:1b}
\gll Uller hat gestern ein Buch für Gwendolyn gekauft. \\
       yesterday has Uller a book for Gwendolyn bought\\
\glt ‘Yesterday, Uller bought a book for Gwendolyn.’
\ex\label{ex:speyer:1c}
\gll Für Gwendolyn hat Uller gestern ein Buch gekauft. \\
       for Gwendolyn has Uller yesterday a book bought\\
\glt ‘Yesterday, Uller bought a book for Gwendolyn.’
\ex\label{ex:speyer:1d}
\gll Ein Buch hat Uller gestern für Gwendolyn gekauft. \\
       a book has Uller yesterday for Gwendolyn bought\\
\glt ‘Yesterday, Uller bought a book for Gwendolyn.’
\z
\z

Of course, the question arises whether German follows a structural version of V2 or merely a linear version. In other words: Is V2 the result of a strict structural configuration that allows nothing but V2, or is it merely an epiphenomenon, a way of using whatever structure is underlying in such a way that the verb always follows the first non-verbal constituent.

Interesting in this respect are violations of the V2 constraint, often referred to as “multiply filled prefield” in German linguistics. Some cases from the literature are presented in \REF{ex:speyer:2}. The constituents are indicated by square brackets.

\ea\label{ex:speyer:2}
\ea\label{ex:speyer:2a}(\citealt{Müller2005}: 299)\\
\gll [Zum zweiten Mal]    [die Weltmeisterschaft]               errang  Clark 1965.\\
     [to=the second time] [the world-championship].\textsc{acc} gained  Clark 1965\\
\glt ‘Clark won the world championship for the second time in 1965.’ 
\ex\label{ex:speyer:2b}(\citealt{Müller2005}: 301)\\
\gll [Den Kürzungen]            [zum Opfer]        fiel auch das [...] Magazin [...].\\
     [the cutbacks.\textsc{dat} [to=the sacrifice] fell also the {}    magazine\\
\glt ‘The journal fell prey to the cutbacks.’ 
\z
\z

Such structures offer the possibility to investigate the question as to how the left periphery in the German sentence is structured. In \sectref{sec:speyer:2}, possible analyses of the German left periphery are presented and evaluated with respect to multiple prefield-filling. \sectref{sec:speyer:3} offers data from Early New High German (in the following: ENHG) and pursues the question whether the evidence speaks in favor of either of the analyses presented in \sectref{sec:speyer:2}. A synthesis between Early New High German and Modern German (in the following: ModG) is given in \sectref{sec:speyer:4}, addressing the question whether the changes between ENHG and ModG are the result of syntactic change \textit{stricto sensu}. With this term I mean real structural change in the sense of e.g. \citet{Lightfoot1999} or \citet{RobertsRoussou2003} and not only change in frequency of competing variants, which might precede but not necessarily lead to syntactic change \textit{stricto sensu}. 

\section{Analyses of the German left periphery}
\label{sec:speyer:2}
\subsection{Unarticulated CP vs. split CP}
\label{sec:speyer:2.1}

There are mainly two generative approaches on the market pertaining to the German left periphery. The classical analysis originally goes back to \citet{denBesten1983} and \citet{Thiersch1978} and is pursued e.g. by \citet{Dürscheid1989}, \citet{Sabel2000} and \citet{Bacskai-Atkari2021}. It states that the German left periphery is an unarticulated CP, the prefield corresponding to SpecCP and the left sentence bracket corresponding to C° (\figref{fig:fromex:3a}). The second analysis takes the cartographic approach into account \citep{Rizzi1997}. It is basically argued for by e.g. \citet{Haegeman1996}, \citet{Grewendorf2002Minimalistische}, \citet{Frey2006}, \citet{FrascarelliHinterhölzl2007}, \citet{Speyer2008Doppelte}, \citet{Catasso2021Large, chapters/4_Catasso}, to name but a few, although these analyses differ quite dramatically as to the exact structure of the left periphery. In \figref{fig:fromex:speyer:3b}, a slightly modified version of the approach by \citet{Speyer2008Doppelte}, which basically goes back to \citet{Grewendorf2002Minimalistische}, is presented as an example of a cartographic approach. The prefield corresponds consequently to everything left of Fin°.

\begin{figure}
\begin{forest}
[CP
  [XP\textsubscript{2}] 
    [C’
      [C°\\\textit{finite verb}\textsubscript{1}]
      [TP
        [... t\textsubscript{2} ... t\textsubscript{1}, roof]
      ]
    ]
]
\end{forest}
\caption{The German left periphery as unarticulated CP}
\label{fig:fromex:3a}
\end{figure}


\begin{figure}
\begin{forest}
[ForceP
  [XP\textsubscript{2} Force°]
  [Top\textsuperscript{Scene}P
    [XP\textsubscript{2} Top\textsuperscript{Sc}°]
    [ContrP
      [XP\textsubscript{2} Co°]
      [Top\textsuperscript{about}P
        [XP\textsubscript{2} Top\textsuperscript{ab}°]
        [FinP
          [t\textsubscript{2}]
          [Fin'
            [Fin°\\\textit{finite verb}\textsubscript{1}]
            [TP
              [... t\textsubscript{2} ... t\textsubscript{1}, roof]
            ]
          ]
        ]
      ]
    ]
  ]
]
\end{forest}
\caption{The German left periphery as split CP}
\label{fig:fromex:speyer:3b}
\end{figure}

In \figref{fig:fromex:speyer:3b}, all prefield phrases pass through SpecFin before being moved on to a specialized position in the C-architecture. This accounts for the bottleneck effect, i.e. that only one phrase is normally moved, although there are several potential landing sites in the C-architecture available. The bottleneck effect comes about by the fact that any phrase moved to the C-architecture lands first in SpecFin, the lowest possible landing site in the C-architecture, before potentially being moved further upwards. If the phrase is moved further to some higher position in the C-architecture, it leaves a trace in SpecFin which prevents movement of any other phrase in that position. Multiple prefield filling by movement is thus excluded. If there is no bottleneck effect (that is, if not all phrases moved to the left periphery need to pass through SpecFin), the many potential landing sites in the C-architecture would allow multiple movement to the left periphery, as it is implemented in languages like e.g. Italian and Hungarian \citep{Rizzi1997}.  Note also that all of the above theories equate the landing site of the finite verb with the lowest projection in the C-architecture, Fin°. So, German would be a Fin-V2-language in the sense of \citet{Wolfe2016Left}. In theory, the landing site of the finite verb could also be the highest projection, Force°, in which case the bottleneck effect would come for free, ForceP being the highest phrase with only one specifier for prefield movement left. Usually, however, movement of the verb and the prefield constituent through Fin° and SpecFin, respectively, is assumed which gives rise to the bottleneck effect \parencite{DeClerqHaegeman2023}. An analysis of German as a Force-V2-language is upheld by e.g. \citet{Poletto2013}; \citet{Wolfe2016Left}; \citet{Hinterhölzl2017}; \citet{DeClerqHaegeman2018}, to name but a few. We will concentrate on the two approaches of the left periphery (a.) as an unarticulated CP (in the following: u-CP-analysis) and (b.) as a split CP in the cartographic tradition with FinP as bottleneck (in the following: s-CP-analysis). 

A question that arises when trying to account for V3 phenomena is how the two approaches fare in that respect. Under the u-CP-analysis, basically the only possibility to derive V3 clauses is by “partial verb phrase fronting” or “remnant movement”, i.e. the VP from which parts have been removed is moved to SpecCP (4a; cf., inter alia, \citealt{denBestenWebelhuth1987}; G. \citealt{GereonMüller1998}; S. \citealt{Mueller2003, Müller2005}, \citealt{Hinterhölzl2006, Hinterhölzl2009Predicate}). In the s-CP-analysis, there would be in principle many landing sites available (indicated by underscores in the tree in \figref{fig:fromex:speyer:4b}), if FinP did not function as a bottleneck. Either way, the constituents in the prefield are positioned independent of each other, be it by independent movement of the constituents (which leads to at least one violation of the generalized movement through SpecFin; cf. \citealt{Speyer2008Doppelte}) or by base\hyp generation of constituents in some positions and movement of (optimally one) constituent to another position (cf. \citealt{Catasso2021Large, chapters/4_Catasso}).

\begin{figure}
\begin{forest}
[CP
  [VP\textsubscript{3}
    [t\textsubscript{2} zum 2. Mal die WM t\textsubscript{4} t\textsubscript{1}, roof]
  ]
  [C'
    [C°\\\textit{errang}\textsubscript{1}]
    [TP
      [DP\\Clark\textsubscript{2}]
      [T'
        [DP\\1964\textsubscript{4}]
        [T'
          [VP\\t\textsubscript{3}]
          [T\\t\textsubscript{1}]
        ]
      ]
    ]
  ]
]
\end{forest}
\caption{Multiple prefield movement in an unarticulated CP}
\label{fig:fromex:speyer:4a}
\end{figure}

\begin{figure}
\begin{forest}
[ForceP
  [\_ Force°]
  [Top\textsuperscript{Scene}P
    [PP\textsubscript{4}]
    [ContrP]
  ]
]
\end{forest}
\caption{Multiple prefield movement in a split CP}
\label{fig:fromex:speyer:4b}
\end{figure}

Both analyses have their problems. The main problems with the u-CP-analysis are twofold. First: The final derivation shows many ungoverned traces in the VP-remnant (at least the subject, the verb, plus potentially material that has been moved out of the VP prior to raising it to SpecCP). Second: Movement out of VP is not necessarily warranted by information-structural features like focus. In \figref{fig:fromex:speyer:4a}, which is an analysis of \REF{ex:speyer:2a}, one could argue that the DP \emph{1965} bears a focus feature by virtue of it being new information. But the clause could easily be extended by a phrase that is information-structurally inert and which would have to be moved out regardless of its lacking information-structural features \REF{ex:speyer:5}. This would violate the ban on movement that is not feature-driven.

\ea\label{ex:speyer:5}(\citealt{Müller2005}: 299)\\
\gll [Zum 2. Mal] [die Meisterschaft] errang Clark schließlich 1965.\\
     [to=the 2\textsuperscript{nd} time] [the championship].\textsc{acc} gained Clark finally 1965\\
\glt ‘Clark finally won the world championship for the second time in 1965.’ 
\z

\begin{sloppypar}
The main problem with the s-CP-approach is that the bottleneck is obviously circumvented somehow. Several attempts to explain this circumvention have been proposed: \citet{Speyer2008Doppelte}, for instance, assumes an optimality-theoretic module on top of the regular generative module. In his model, phrases can be moved directly to a landing-site in the C-architecture, if SpecFin is already blocked; since this is, however, a “costly” operation (as it violates some optimality\hyp theoretic constraints, among others a constraint that bans locality violations) it is done only rarely. \citet{Catasso2021Large, chapters/4_Catasso} sees some phrases as being base\hyp generated in the left periphery (in his model, the scene\hyp setter “zum zweiten Mal” would be base-generated in the suitable specifier) and only one to be moved; thus, the bottleneck effect that constrains movement would be unharmed. 
\end{sloppypar}

Let us now look at the advantages and predictions of these approaches for multiple prefield-filling. With the u-CP-analysis, the V2 effect comes for free, as there is only one specifier available for movement. V3 is derivable via remnant movement of a partially emptied VP. The predictions are that multiple prefield elements should always appear in the base-generated order (see \cites[54]{Mueller2003}[307--309]{Müller2005}). A second prediction would be that the subject should never be part of multiple prefield filling, as the subject is obligatorily moved out of the VP to SpecTP (\citealt{BorsleySuchsland1997}, \citealt[309--310]{Müller2005}, \citealt{Winkler2014, Winkler2017}).\footnote{\citet{Winkler2014, Winkler2017} offers a constraint-based approach to the multiple filling of the prefield}  An exception are non-canonical subjects, like those of passive or medial constructions \citep[299--300]{Müller2005}. The movement of the subject out of the VP is, however, a matter of dispute, as some researchers (e.g. \citealt{Haider2010}) do not assume an independent TP in German. 

On the other hand, the s-CP-analysis has the advantage that cases of multiple prefield filling can be accounted for in principle (although by circumventing the bottleneck effect somehow), as there are several landing sites available. The prediction here would be that the relative order of the multiple prefield elements should correspond to the hierarchical order of the cascade of split-CP\hyp projections. So both analyses lead to clear predictions that can be checked. 

\subsection{Predictions of the u-CP-analysis}
\label{sec:speyer:2.2}

What would be the exact order predicted by the u-CP-analysis? As we might assume that only the unscrambled VP can be moved to the prefield, the order would have to conform to the base-generated order that is the outcome of the flattened-out hierarchical structure of VP. We already said that we would not expect subjects to appear in a VP remnant, as they are moved out of the VP for independent reasons. The base-generated relative order of dative and accusative objects is a matter of debate (see e.g. \citealt{Speyer2015}): Dative before Accusative (Dat~> Acc) is the unmarked order (e.g. \citealt{Lenerz1977}, \citealt{Höhle1982}), but there are good arguments in favor of a base\hyp generated serialization Acc~> Dat. First: In cases in which the dative and the accusative object refer to the same entity, it is the accusative which binds the dative reflexive/reciprocal and not vice versa (6; e.g. G. \citealt{GereonMüller1999}, \citealt{Speyer2015}).

\ea\label{ex:speyer:6}
\ea[]{\label{ex:speyer:6a}
\gll Gestern hat Gwendolyn die Gäste einander vorgestellt. \\
       yesterday has Gwendolyn the.\textsc{acc} guests.\textsc{acc} each=other.\textsc{dat} introduced \\
\glt ‘Yesterday, Uller Gwendolyn introduced the guests to each other.’}
\ex[*]{\label{ex:speyer:6b}
\gll Gestern hat Gwendolyn den Gästen einander vorgestellt. \\
       yesterday has Gwendolyn the.\textsc{dat} guests.\textsc{dat} each=other.\textsc{acc} introduced \\
\glt ‘Yesterday, Uller Gwendolyn introduced the guests to each other.’}
\ex[]{\label{ex:speyer:6c}
\gll Die Therapie gab die beiden einander wieder. \\
       the therapy gave the.\textsc{acc} both.\textsc{acc} each=other.\textsc{dat} again \\
\glt ‘The therapy gave the two back to each other.’}
\ex[*]{\label{ex:speyer:6d}
\gll Die Therapie gab den beiden einander wieder. \\
       the therapy gave the.\textsc{dat} both.\textsc{dat} each=other.\textsc{acc} again \\
\glt ‘The therapy gave the two back to each other.’}
\z
\z

This indicates that the accusative object is higher in the tree than the dative object. 

Second: If we assume that the “Rektion” (i.e. assignment of case or other formal features to the complement) of a verb works via c-command (and not m-command), and the arguments which are not c-commanded receive structural case, we notice that the argument structure of most trivalent verbs is ‘nominative – accusative – X’. where ‘X’ stands for a variable morphosyntactic shape of the argument \REF{ex:speyer:7}. This could be, for instance, a dative DP \REF{ex:speyer:7a}, a genitive DP \REF{ex:speyer:7b}, a PP object \REF{ex:speyer:7c}\footnote{The PP headed by \textit{nach} counts as prepositional object because its form, that is, the choice of the preposition, is determined by the verb.} or an argument in the form of an adverbial (i.e. whose exact form is not determined by the governing verb, 7d). 

\ea\label{ex:speyer:7}
\ea \label{ex:speyer:7a}
\gll Gestern hat Uller das Buch dem Lehrer geliehen. \\
       yesterday has Uller the.\textsc{acc} book.\textsc{acc} the.\textsc{dat} teacher.\textsc{dat} lent  \\
\glt ‘Yesterday, Uller lent the book to the teacher.’
\ex\label{ex:speyer:7b}
\gll Gestern hat Gwendolyn den Anwalt des Betrugs bezichtigt. \\
       yesterday has Gwendolyn the.\textsc{acc} attorney.\textsc{acc} the.\textsc{gen} fraud.\textsc{gen} accused \\
\glt ‘Yesterday, Gwendolyn accused the lawyer of fraud.’
\ex\label{ex:speyer:7c}
\gll Gestern hat das Finanzamt die Steuern nach dem Einkommen bemessen. \\
       yesterday has the revenue=office the taxes.\textsc{acc} after the income measured \\
\glt ‘Yesterday, the revenue office measured the taxes according to the income.’
\ex\label{ex:speyer:7d}
\gll Gestern hat Uller die Vase auf den Kaminsims gestellt. \\
       yesterday has Uller the vase.\textsc{acc} on the mantelpiece put \\
\glt ‘Yesterday, Uller put the vase on the mantelpiece.’
\z
\z

This is explained rather neatly under the assumption that the nominative and the accusative are structural cases, whereas the variable argument shape of the sister argument to the verb is an outcome of c-command by the verb.

Third: If we assume that passivization is subject to a locality constraint in the sense that the DP which is closest to TP moves to SpecTP (in active clauses this would be the subject as it is the specifier of the highest V-shell, in passive clauses it would be the highest available DP below the subject position, the agentive subject having been removed), we notice that the “normal” passive of trivalent verbs promotes the accusative object (8a, 9).\footnote{Note that the order of the dative object and the subject might be inverted due to the strong tendency to put animated referents first, but this does not touch the argument.} For promoting of the dative object another auxiliary has to be used, viz. \textit{bekommen} ‘to get’ \REF{ex:speyer:8b}. The fact that passivization promotes the accusative object although other objects are available again indicates that the accusative is in a higher position than the dative\slash genitive\slash prepositional object.   

\ea\label{ex:speyer:8}
\ea \label{ex:speyer:8a}
\gll (... dass die Polizei dem Raser den Führerschein entzogen hat) \\
      {}   that the.\textsc{nom} police.\textsc{nom} the.\textsc{dat} speeder.\textsc{dat} the.\textsc{acc} driving-license.\textsc{acc} withdrawn has\\
\glt ‘… that the police withdrew the speeder his driving license.’
\exi{→}{
\label{ex:speyer:8a'}
\gll passive: ... dass der Führerschein dem Raser entzogen wurde.\\
     {}       {}  that the.\textsc{nom} driving-license.\textsc{nom} the.\textsc{dat} speeder.\textsc{dat} withdrawn became\\
\glt ‘… that the driving license was withdrawn to the speeder.’}
\ex \label{ex:speyer:8b}
\gll ... dass der Raser den Führerschein entzogen bekam.\\
     {}  that the.\textsc{nom} speeder.\textsc{nom} the.\textsc{acc} driving-license.\textsc{acc} withdrawn got\\
\glt ‘… that the speeder got his driving license withdrawn.’
\z
\z


\begin{figure}
\caption{Passivization with accusative object higher than dative object}
\label{fig:fromex:speyer:9}
\begin{forest}
[TP
  [DP\textsubscript{3} [der Führerschein, roof]]
  [T'
    [V\textsuperscript{aux}P
      [DP]
      [V\textsuperscript{aux}’
        [vP
          [v'
            [VP
              [DP\textsubscript{acc, structural}\\t\textsubscript{3}]
              [V'
                [DP\textsubscript{dat, by c-cmd}
                  [dem Raser, roof]
                ]
                [V°\\t\textsubscript{1}]
              ]
            ]
            [v°\\entzogen\textsubscript{1}]
          ]
        ]
        [V\textsuperscript{aux}°\\t\textsubscript{2}]
      ]
    ]
    [T°\\wurde\textsubscript{2}]
  ]
]
\end{forest}
\end{figure}

So the order in the prefield filled by a remnant VP should be accusative before dative. It is a general problem that this generalization does not hold, i.e. we do find both orders in complex prefields (s. 10a,b). Ex. \REF{ex:speyer:10a} with dative before accusative is slightly adapted from \citet[324, ex. 56a]{Müller2005}; ex. \REF{ex:speyer:10b} is a version in which the object order is inverted. However, both versions are acceptable. It is not clear how to account for that phenomenon. For partial VP-movement, both orders are on a par, for deep structural processes like binding and case assignment, the accusative before dative order appears to be the more basic one. This is left for further research, but see \citet{VoigtmannSpeyer2024} for a tentative solution.

For other constituents we would expect the base-generated order as well, for instance or adverbials. Their position is notoriously hard to determine, however. Adverbials of time and place scope over the subject (see e.g. \citealt{FreyPittner1998}), but the observable normal position in the clause is after the subject (10c vs. 10d).\footnote{\citet{FreyPittner1998} explain this word order by saying that definite DPs scramble across the temporal adverbial. But then it would be a scrambled word order, and consequently we would not expect it in a multiply filled prefield.} One might explain this by assuming that they are adjoined within the TP, from which they scope over the VP and thus over the base-position of the subject.

\ea \label{ex:speyer:10}
\ea[]{\label{ex:speyer:10a}
\gll Seiner Freundin diesen Ring glaube ich nicht, dass er schenken wird.\\
     his.\textsc{dat} girl-friend.\textsc{dat} this.\textsc{acc} ring.\textsc{acc} believe I not that he present will\\
\glt ‘I don’t think that he will give this ring to his girl-friend as a present.’}
\ex[]{\label{ex:speyer:10b}
\gll Diesen Ring seiner Freundin glaube ich nicht, dass er schenken wird.\\
     his.\textsc{dat} girl-friend.\textsc{dat} this.\textsc{acc} ring.\textsc{acc} believe I not that he present will\\
\glt ‘I don’t think that he will give this ring to his girl-friend as a present.’}
\ex[]{\label{ex:speyer:10c}
\gll ... dass Uller gestern einen Brief nach Ulm geschickt hat.\\
     {}  that Uller.\textsc{nom} yesterday a.\textsc{acc} letter.\textsc{acc} to Ulm sent has\\
\glt ‘… that Uller sent a letter to Ulm yesterday.’}
\ex[\textsuperscript{?}]{\label{ex:speyer:10d}
\gll ... dass gestern Uller einen Brief nach Ulm geschickt hat.\\
     {}  that yesterday Uller.\textsc{nom} a.\textsc{acc} letter.\textsc{acc} to Ulm sent has\\
\glt ‘… that Uller sent a letter to Ulm yesterday.’}
\z
\z

So, we would regularly expect time and place adverbials after the subject (although, again, we would not expect the subject to appear in a complex prefield anyway).

\subsection{Predictions of the s-CP-analysis}
\label{sec:speyer:2.3}

The predictions of the s-CP-analysis depend on the hierarchical order of the CP cascades which one assumes (see \sectref{sec:speyer:1}). I will outline the expectations following the model proposed in \citet{Speyer2008Doppelte, Speyer2008German, Speyer2019}.

If the prefield is occupied by only one constituent, the prefield constituents fall mostly in four information structurally defined categories: phrasal discourse relation markers (= PDRM, \ref{ex:speyer:11a}, e.g. \citealt{FetzerSpeyer2019}; in \REF{ex:speyer:11a} examples are given for markers of the discourse relations Narration and Consequence), scene-setting elements (\ref{ex:speyer:11b}, e.g. \citealt{Jacobs2001}) which are termed also stage topics (\citealt{Erteschik-Shir1997,Erteschik-Shir2019}), contrastive elements which roughly correspond to list contrastive topics (\ref{ex:speyer:11c}, see e.g. \citealt{Büring2003}; \citealt{Lee2006}), and aboutness topics in the sense of \citet{Reinhart1981} (\ref{ex:speyer:11d}, cf. e.g. \citealt{Molnár1991}; \citealt{BaderPortele2019}). These aboutness topics are given information as a rule; topics representing new information are only allowed in the prefield if they are relevant or the macrostructure of the discourse (e.g. if they continue to be topics for a certain stretch of clauses, see \citealt{Speyer2009}).  

\ea\label{ex:speyer:11}
\ea\label{ex:speyer:11a}
\gll Dann/Folglich hat Uller das Buch dem Lehrer geliehen.\\
     then/consequently has Uller the.\textsc{acc} book.\textsc{acc} the.\textsc{dat} teacher.\textsc{dat} lent  \\
\glt ‘Then/consequently, Uller lent the book to the teacher.’
\ex\label{ex:speyer:11b}
\gll Gestern hat Uller das Buch dem Lehrer geliehen. \\
       yesterday has Uller the.\textsc{acc} book.\textsc{acc} the.\textsc{dat} teacher.\textsc{dat} lent \\
\glt ‘Yesterday, Uller lent the book to the teacher.’
\ex\label{ex:speyer:11c}
\gll Das Buch hat Uller dem Lehrer geliehen \\
the.\textsc{acc} book.\textsc{acc} has Uller the.\textsc{dat} teacher.\textsc{dat} lent\\
\gll (und den Taschenrechner seiner Freundin).\\
and the.\textsc{acc} pocket-calculator.\textsc{acc} his.\textsc{dat} girl-friend.\textsc{dat} \\
\glt ‘As for the book, Uller lent it to the teacher, and as for the pocket calculator, he lent it to his girl friend.’
\ex\label{ex:speyer:11d}
\gll (context: Uller ist echt nett.) Er hat das Buch dem Lehrer geliehen. \\
Uller is really nice he has the.\textsc{acc} book.\textsc{acc} the.\textsc{dat} teacher.\textsc{dat} lent \\
\glt ‘(Uller is really nice.) He lent the book to the teacher.’\\
\z
\z

If a clause contains no constituent that fulfills one of the above\hyp mentioned conditions, there is an elsewhere-condition, i.e. that the highest middle field element is moved to the prefield or an expletive is base\hyp generated there (\citealt{Fanselow2002}; \citealt{Frey2006}; \citealt{Speyer2009}). If a clause contains more than one constituent that fulfills one of the conditions outlined above, the prefield slot is filled according to the hierarchy in \REF{ex:speyer:12}. 

\ea\label{ex:speyer:12}
PDRM $>>$ scene-setting element $>>$ contrastive element $>>$ aboutness topic
\z

While this does not say much yet about the structure of the left periphery it already suggests that the final targets of prefield movement are specialized positions bearing information structural features (at least in the case of scene-setting elements, contrastive elements and aboutness topics) in the spirit of \citegen{Rizzi1997} cartographic approach. \citet{Grewendorf2002Minimalistische} adopts \citegen{Rizzi1997} cascade \REF{ex:speyer:13a} and assumes it to hold for German as well, with FinP as bottleneck. \citet{Speyer2008Doppelte} applies the Rizzi\slash Grewendorf hierarchy to the cases which he observed in his study on an ENHG corpus (\ref{ex:speyer:13b}, adapted).

\ea \label{ex:speyer:13}
\ea ForceP > Top*P > FocP > Top*P > FinP\label{ex:speyer:13a}
\ex ForceP > Top\textsuperscript{scene}P > ContrP > Top\textsuperscript{about}P > FinP\label{ex:speyer:13b}
\z\z

The hierarchical order mirrors the hierarchy of prefield elements. The selection of prefield elements follows this hierarchy. PDRMs stand in the specifier of the highest projection ForceP, since they establish a link to the context. Scene-setting elements first are moved to SpecFin and finally land in SpecTop\textsuperscript{scene}P, because they bear the apt feature for this movement; likewise, contrastive elements move to SpecContrP, aboutness topics end up in SpecTop\textsuperscript{about}P. Finally, elements subject to the elsewhere condition either are moved to SpecFinP (if they are the highest middle field constituent and thus undergo formal movement) and are not moved further (as they lack a suitable information structural feature) or are directly merged in SpecFinP (expletives) in order to saturate the EPP feature located there (\citealt{Catasso2021Large, chapters/4_Catasso}). More important, however, is that the order mirrors scope relations: PDRMs scope over the whole utterance including scene and proposition. Scene-setting elements scope over the proposition regardless of the role the proposition plays in discourse. Within the proposition, contrastive elements, being members of an evoked set, function as very local topics in whose scope a (less local) aboutness topic can linger on (like the classic English topicalization example \textit{Beans he likes, but peas he hates}: Here the referent of \textit{he} functions as aboutness topic for both clauses, but is in the scope of the two enumerated members of the set of edible vegetables).  

Turning to the original question, we would expect the constituents in a multiply\hyp filled prefield to linearize in accordance with the hierarchical structure in \REF{ex:speyer:13b}.  

\section{Multiple prefields in ENHG}
\label{sec:speyer:3}

In this section, the predictions outlined in Sections~\ref{sec:speyer:2.1} and~\ref{sec:speyer:2.2} will be checked on ENHG data. A question which immediately comes to mind is, why ENHG and not another historical stage of German? There are several answers to that question. First, ENHG is the earliest period in the German language history in which there is a large production of original prose. The textual testimony of earlier stages of German either is mostly in rhyme (Middle High German and the original texts of Old High German) or is more or less literally translated (most Old High German texts are translations) and therefore only to be used with caution for the question followed here (see e.g. \citealt{Catasso2021Theoretical}). Second, the constraint that the prefield host only one constituent is almost categorical, but not yet quite so as in ModG, so there is the chance of finding examples for multiple prefields. Third, for ENHG a parsed corpus is available (\textit{Referenzkorpus Frühneuhochdeutsch}\footnote{\url{https://www.linguistics.rub.de/ref/corpus}}), which simplifies the search considerably.  

The ANNIS-search in the ReF yielded in 86 multiply filled prefields, after manual elimination of false hits (such as left dislocations).\footnote{Many thanks to Nils Dörr for conducting the ANNIS-search.}  \tabref{tab:speyer:1} orders the hits such that those, which would conform to the expectations set out in \sectref{sec:speyer:2.3}, are in the fields above the non-applicatur-diagonal, those which do not conform to the expectations are in the fields below this diagonal.


\begin{table}
\begin{tabular}{l ccccc}
\lsptoprule
                              & \multicolumn{4}{c}{2\textsuperscript{nd} element}\\\cmidrule(lr){2-5}
                              & PDRM & Scene-  & Constr.  & Aboutness  & \\
1\textsuperscript{st} element &      & setting & element  & topic      & Sum\\\midrule
PDRM                          &    & 9 & 7 & 35 & 51 \\
Scene-setting                 & 3  &   & 2 & 17 & 19 \\
Contrastive                   & 3  &   &   & 1  & 1  \\
Aboutness topic               & 4  & 2 & 3 &    &    \\\addlinespace
Total (exp. order)            &    &   &   &    & 71\\\midrule
\multicolumn{6}{l}{Sum tokens (unexpected order)}\\
                              & 10 & 2 & 3 &    & 15\\
Total (unexp. order)          &    &   &   &    & 15\\                         
\lspbottomrule
\end{tabular}
\caption{Information structural properties of 1\textsuperscript{st} and 2\textsuperscript{nd} element in prefield.}
\label{tab:speyer:1}
\end{table}

We see that 71 tokens, or 83\%, conform to the hierarchical order posited in \REF{ex:speyer:13b} while 15 do not. Very robust are especially the relative orders ‘PDRM > aboutness topic’ \REF{ex:speyer:14a} and ‘scene-setting element > aboutness topic’ \REF{ex:speyer:14b}. Contrastive elements and aboutness topics do not show a clear picture in this search. In fact, the reverse ordering ‘topic > contrastive element’ is more common. However, some examples have been found in the manual corpus search on which \citet{Speyer2008Doppelte} is based, I present one example from here \REF{ex:speyer:14c}.\footnote{The examples presented in \citet[ex.33]{Speyer2008Doppelte} are controversial in that they might be verb-final clauses, as the text discussing these examples clearly concedes. \citet{Catasso2021Verb} refutes these examples as exponent of the ordering ‘contrastive element > topic’, but his wording suggests that \citet{Speyer2008Doppelte} overlooked the possibility that these could be verb-final clauses. This is clearly not the case. The example on which the order is illustrated (ex. 38 in \citealt{Speyer2008Doppelte}) is not taken into consideration in \citet{Catasso2021Verb}.}

\ea\label{ex:speyer:14}
\ea\label{ex:speyer:14a}
\gll deßhalb sein lob erhal gar weit.\\
     therefore his praise sound.\textsc{sbj} very far\\
\glt ‘Therefore, his praise shall sound very far.’ (Pfaffe vom Kalenberg, from ReF)
\ex \label{ex:speyer:14b}
\gll darunder etliche schoene Teppich waren auff gebrait.\\
     thereunder many beautiful carpets were on spread \\
\glt ‘Under that, many beautiful carpets were spread.’ (Der Artzney Doctorn vnd bestellten Medici zu Augspurg, from ReF)
\ex\label{ex:speyer:14c}
\gll anders ich moehte sin nút erlitten haben \\
     other.\textsc{acc} I might he.\textsc{gen} not suffered have\\
\glt ‘Another thing I would not have been able to suffer for him.’ (Merswin, Bonner Frühneuhochdeutschkorpus 231.16.17, from \citealt{Speyer2008Doppelte}:ex.38)
\z
\z

But the results are quite robust for PDRMs high up in the tree, on the next level scene-setting elements and then the others. 

We see that orders such as in \REF{ex:speyer:14c} do not conform to the base order in the VP. So they cannot be instances of remnant movement. Moreover, \REF{ex:speyer:14c} contains the subject of the clause. If we assume that the subject is moved out of the VP on independent grounds, we would not expect subjects in the VP remnant. With order such as in (14a-b), it is different. Although they contain subjects as well, they could be analyzed in three ways: as remnant VP movement (if the scene-setting element or the PDRM is adjoined above the subject), as independent movement of the two constituents, or, following \citet{Catasso2021Large}, as movement of the topic plus base generation of the PDRM\slash scene-setting element.

There are some unambiguous examples for remnant VP movement in the corpus (if we allow subjects to be part of the VP remnant). To wit, these are those that do not conform to the predicted serialization in \tabref{tab:speyer:1}. In \REF{ex:speyer:15a}, an example is presented in which the topic precedes the contrastive element. Note that the topic is the subject and thus the highest VP element. In \REF{ex:speyer:15b}, a topic subject precedes the scene\hyp setting adverbial, which under the hierarchical stacking in \REF{ex:speyer:13b} should be excluded, but which would be a regular serialization in the middle field. 

\ea\label{ex:speyer:15}
\ea\label{ex:speyer:15a}
\gll Kung Karl von allen dingen was nicht pegirdiger zu thuen. \\
       king Charles of all things was not avid.\textsc{comp} to do \\
\glt ‘King Charles was not more avid to do any of all things.’ (Füetrer, from ReF)

\ex\label{ex:speyer:15b}
\gll Dieselben menschen alle vor peichten all ir sünd. \\
       the-same humans all before confess all their sins \\
\glt ‘Beforehand, the same people, all of them, confess all their sins’ (Hartlieb, from ReF)
\z
\z

So, the ENHG grammar seems to have allowed for both, independent movement of more than one constituent into the C-architecture and movement of a VP remnant. The necessary bypassing of the bottleneck SpecFin is assumed to be costly and therefore this is employed only rarely. On the other side, also remnant\hyp VP\hyp movement can only be applied if there is an information structural trigger, such as a contrastive feature on the whole VP. 

\section{The way to ModG: Syntactic change?}
\label{sec:speyer:4}

As many cases of multiple prefield filling are ambiguous between remnant-VP-movement (the u-CP-analysis) and independent movement of <1 constituent (the s-CP-analysis), language learners might eventually have settled for one of the variant, the u-CP-analysis.\footnote{It might be noted that a similar situation arose already 1000 years earlier, in Old High German. \citet{Petrova2023} pursues the question whether main clauses with the verb in a position after the second position are genuine verb-third clauses or verb final clauses. She finds that most examples are ambiguous with respect to both analyses and concludes that they coexisted at that time, without being sharply distinguished in terms of functions.}  This is largely the picture in ModG (see e.g. \citealt{Müller2005}). However, there is still a residue of the “old” s-CP-analysis, namely multiple prefields with a scene-setting element or a PDRM followed by a topical element \REF{ex:speyer:16}. Although this construction is often associated with Kiezdeutsch (16a, a variety spoken by multiethnic urban teenagers), such clauses occur in other varieties (such as standard German) as well \REF{ex:speyer:16b}.\footnote{As a reviewer points out, it is of course unlikely that migrants pick up an Early New High German pattern. However, they do not so much pick up a specific pattern, but a clause structure which allows for this pattern. The difference of Kiezdeutsch to standard German is that the constraint to use only one prefield element is less strict in Kiezdeutsch (see \citealt{Speyer2008German} for an OT analysis of single prefields) so that deviations from this constraint occur more often.} Note that the source of \REF{ex:speyer:16b} is not a source that is associated with a very casual register.

\ea \label{ex:speyer:16}
\ea \label{ex:speyer:16a}
\gll Dann die sind zur Ubahn gerannt. \\
     then they are to=the subway run \\
\glt ‘Then they ran to the subway.’ (Kiezdeutsch, from \citealt[ex.11b]{WieseEtAl2017})

\ex \label{ex:speyer:16b}
\gll An den SWR1-Wetterstationen wir messen... \\
     at the SWR1-weather-stations we measure \\
\glt ‘At SWR1’s weather stations we measure…’ (SWR1 BW, 2/15/2016, 2:05pm)

\ex \label{ex:speyer:16c}
\gll Nach ihrer\textsubscript{i} Tat die\textsubscript{i} sind zur Ubahn gerannt. \\
     after their action they are to=the subway run \\
\glt ‘After their action, they ran to the subway.’ 
\z
\z

As the scene-setting and/or PDRM elements occupying the first position in the clauses in \REF{ex:speyer:16} is possible also in the middle field, there is no good reason to assume that they are base-generated in SpecTop\textsuperscript{scene}P (contra \citealt{Catasso2021Large,chapters/4_Catasso}). Reconstruction of possessive binding relations is also possible (16c; altered version of \REF{ex:speyer:16a}), indicating that the subject originally c-commanded the scene-setter at an earlier stage of the derivation, which again speaks in favor of movement and against base-generation of the scene-setting element.

So the question is whether a syntactic change has occurred between ENHG and ModG. For syntactic change stricto sensu, the answer to this question is negative: ENHG offered at least two structural configurations that led to a multiply filled prefield: independent movement of more than one phrase to the C\hyp architecture, and movement of a partially emptied VP to the C-architecture. Both structural configurations are still possible in ModG. But we see nevertheless a change in frequency: Cases which are not analyzable as remnant-VP-movement have become more rare and are basically confined to one residual construction in ModG, viz. scene-setting element > topic > finite verb. We saw in \sectref{sec:speyer:3} that some ENHG examples are ambiguous between an analysis as remnant-VP-movement and an analysis as independently moved constituents. It might well be that the u-CP-analysis (the analysis of the multiple prefield as movement of a partially emptied VP) is the result of a reanalysis of an original structure following the s-CP-analysis \REF{ex:speyer:17}.\footnote{The extraposition of the subject in the original sentence complicates the analysis somewhat; I offer the analysis with the subject still in SpecTP, thus deviating from the original sentence, to keep things at least approximately clear.}

\ea\label{ex:speyer:17}
\ea\label{ex:speyer:17a}
\gll in dem hohen schöwen jm werdent fur gehaben kunftige ding. \\
       in the high vision him.\textsc{dat} become.\textsc{3pl} for held future thing.\textsc{pl} \\
\glt ‘In the high vision, future things are presented to him.’ (Marquard von Lindau: De Nabuchonodosor, from ReF)

\ex\label{ex:speyer:17b}
[\textsubscript{ForceP} [\textsubscript{TopsceneP} in dem hohen schöwen\textsubscript{4} [\textsubscript{ContrP}  [\textsubscript{TopaboutP}  jm\textsubscript{3} [\textsubscript{FinP} t\textsubscript{3} werdent\textsubscript{1} [\textsubscript{TP} kunftige ding\textsubscript{2} [\textsubscript{VP} t\textsubscript{2} t\textsubscript{3} t\textsubscript{4} fur gehaben t\textsubscript{1}] t\textsubscript{1}]]]]]]

\ex{[\textsubscript{ForceP} [\textsubscript{TopsceneP} [\textsubscript{ContrP}  [\textsubscript{TopaboutP}  [\textsubscript{VP} t\textsubscript{2} in dem hohen schöwen jm]\textsubscript{3} [\textsubscript{FinP} t\textsubscript{3} werdent\textsubscript{1} [\textsubscript{TP} kunftige ding\textsubscript{2} [\textsubscript{VP} t\textsubscript{3} fur gehaben] t\textsubscript{1}]]]]]]\label{ex:speyer:17c}}
\z
\z

The clause in \REF{ex:speyer:17a} would be analyzed following the (older) s-CP-analysis by the parent generation \REF{ex:speyer:17b}. A u-CP-analysis would be possible, as the order conforms to the structure of the VP, apparent in the middle field serialization \REF{ex:speyer:17c}. It is reasonable to assume that the language learner might opt for the u-CP-analysis in \REF{ex:speyer:17c} because it contains less structure and because it does not display a violation of the bottleneck effect. Since the u-CP-analysis is more economic, it becomes the default option to analyze multiple prefields and enters the competence in that the original freedom to utilize movement of more than one phrase in the C-architecture by bypassing the bottleneck SpecFinP is gradually lost, with the exception of the specialized scene\hyp setter\slash PDRM plus topic cases demonstrated in \REF{ex:speyer:16}. In ModG, most, probably all, instances of double filled prefield are either analyzable by the u-CP-analysis or the s-CP\hyp analysis in its stripped down version with only SpecTop\textsuperscript{scene}P and SpecTop\textsuperscript{about}P available.\footnote{Strictly speaking, a change pertaining to the structure has occurred consequently, in that not all landing positions are available any more. SpecContrP seems to be unavailable in Modern German, as opposed to Spec Top\textsuperscript{scene}P and SpecTop\textsuperscript{about}P.}   This change contributes to the further establishment of the strict V2-character of German, in that former exceptions to V2 are analyzed as only spurious violations of V2.

\section{Conclusion}
\label{sec:speyer:5}

We saw that multiple prefield filling in ENHG can be the result of two processes: Independent movement of more than one constituent into the C-architecture, or movement of a VP remnant as the only phrase. Whereas in ENHG cases which are at least analyzable as multiple movement are most common, in ModG the picture is reversed: Here the majority of cases is analyzable as remnant-VP-movement, with the residual “real” multiply filled prefield of the type scene\hyp setting element > topic > finite verb. This suggests that in both periods of the language an articulated C-architecture in the style of \citet{Rizzi1997} is available, although it is not utilized as freely as in e.g. the Romance or the Ugric languages because of the bottleneck effect. The bottleneck nature of German V2 became stronger over time (see \citealt{Catasso2021Large, Catasso2021Theoretical, chapters/4_Catasso}), showing in a more restricted version of multiple prefield filling as opposed to ENHG and the fact that remnant-VP-movement has become the unmarked case.  

\printbibliography[heading=subbibliography,notkeyword=this]
\end{document}
