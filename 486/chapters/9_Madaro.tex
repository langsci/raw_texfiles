\documentclass[output=paper]{langscibook}
\ChapterDOI{10.5281/zenodo.17077198}
\author{Madaro Romano\orcid{}\affiliation{University of Verona} and Alessandra Tomaselli\orcid{}\affiliation{University of Verona} and Ermenegildo Bidese\orcid{}\affiliation{University of Trento}}
\title[CP-expansion in the German enclave varieties across Northeast Italy]{Deriving CP-expansion in the German enclave varieties across Northeast Italy: Cracks in the bottleneck?}
\abstract{This paper compares and analyzes V2-related phenomena in the German-speaking varieties in North-Eastern Italy, particularly the expansion of the C-domain (violation of linear-V2) and the progressive specialization of its projections. Even though accounting for a split-CP configuration à la \citet{Rizzi1997}, we argue that CP-expansion in these German(ic) varieties differs substantially from the split-CP configuration typical of Romance languages. For this reason, the bottleneck effect fails to systematize the variation phenomena within these varieties, while the Feature-Scattering Hypothesis (\citealt{GiorgiPianesi1997, Hsu2017}) works better in catching intra/inter-linguistic differences. Moreover, the different degrees of expansion of the C-domain highlight two lines of variation: the West-to-East line represents the most conservative one; the North-to-South one shows a higher degree of variation. With this categorization, also supported through the analysis of the diachronic development of Cimbrian, we provide further confirmation that the geographical distribution of these varieties (i.e. the distance from the German core as well as the higher immersion in the Romance one) drastically influences the variation phenomena linked to the structural-V2 and OV/VO typology.}
\IfFileExists{../localcommands.tex}{
  \addbibresource{../localbibliography.bib}
  % add all extra packages you need to load to this file

\usepackage{tabularx,multicol}
\usepackage{url}
\urlstyle{same}

\usepackage{listings}
\lstset{basicstyle=\ttfamily,tabsize=2,breaklines=true}

\usepackage{langsci-basic}
\usepackage{langsci-optional}
\usepackage{langsci-lgr}
\usepackage{langsci-osl}
% \usepackage{./langsci/styles/langsci-lgr}
% \usepackage{./langsci/styles/langsci-osl}
% \usepackage{langsci-gb4e}

\usepackage{tikz}
\usetikzlibrary{patterns,calc}
\pgfdeclarepatternformonly{south east lines}{\pgfqpoint{-0pt}{-0pt}}{\pgfqpoint{3pt}{3pt}}{\pgfqpoint{3pt}{3pt}}{
    \pgfsetlinewidth{0.6pt}
    \pgfpathmoveto{\pgfqpoint{0pt}{3pt}}
    \pgfpathlineto{\pgfqpoint{3pt}{0pt}}
    \pgfpathmoveto{\pgfqpoint{.2pt}{-.2pt}}
    \pgfpathlineto{\pgfqpoint{-.2pt}{.2pt}}
    \pgfpathmoveto{\pgfqpoint{3.2pt}{2.8pt}}
    \pgfpathlineto{\pgfqpoint{2.8pt}{3.2pt}}
    \pgfusepath{stroke}}
    
\usepackage{stmaryrd}
\usepackage{wasysym}
\usepackage{multirow}
\usepackage{caption}
\usepackage{subcaption}
\usepackage{mathrsfs}
\usepackage{qtree}

\usepackage{linguex}


  %pminos do not split footnotes
% \interfootnotelinepenalty=10000 %Footnote in Laporte chapters has to be split SN


%\DeclareIndexNameFormat{default}{%
%\nameparts{#1}%
%\usebibmacro{index:name}%
%{\index[names]}%
%{\namepartfamily}%
%{\namepartgiveni}%
% {}% L1
% {}% L2
%{\namepartprefix}% generates spurious space L3
%{\namepartsuffix}% generates spurious space L4
%}

%  {\DeclareIndexNameFormat{default}{%
%     \usebibmacro{index:name}{\index[names]}{#1}{#3}{#5}{#7}}}

%\DeclareIndexNameFormat{default}{%
%  \usebibmacro{index:name}{\sindex[nom]}{#1}{#3}{#5}{#7}}

%\DeclareIndexNameFormat{default}{%
%  \usebibmacro{index:name}{\sindex[person]}{#1}{#3}{#5}{#7}}
%\DeclareIndexNameFormat{default}{%
%\nameparts{#1} \usebibmacro{index:name}{\sindex[person]]}{\namepartfamily}{‌​\namepartgiven}{\nam‌​epartprefix}{\namepa‌​rtsuffix}}

%\newcommand{\smiley}{:)}

%\renewbibmacro*{index:name}[5]{%
%\usebibmacro{index:entry}{#1}%
%{\iffieldundef{usera}{}{\thefield{usera}\actualoperator}\mkbibindexname{#2}{#3}{#4}{#5}}}

% \newcommand{\noop}[1]{}

%remove for final
%\overfullrule=1mm

\newcommand{\tobi}[2]}}
\renewcommand{\S}[1]{\tobi{#1}{\textsc{*}}}

% this volume references
% puts: [this volume]
% already defined: \citetv
%\newcommand{\citepv}[1]{(\citeauthor{#1} \citeyear*{#1} [this volume])}
\newcommand{\citealtv}[1]{\citeauthor{#1} \citeyear*{#1} [this volume]}

%parentheses around example number
\newcommand{\pref}[1]{(\ref{#1})}

% in-text examples

\newcommand{\lnex}[1]{\textit{#1}} %target lang word
\newcommand{\lnlit}[1]{(lit.: `#1')} %literal reading
\newcommand{\lnlat}[1]{(#1)} % latinization
\newcommand{\lntrans}[1]{`#1'} %translation
\newcommand{\lnexl}[2]%
{\lnex{#1}{} \lnlat{#2}} % ex with latinization
\newcommand{\lnexlat}[3]{\lnex{#1}{} \lnlat{#2}{} \lntrans{#3}} % ex with latinization and tranl.

%ch01
\newcommand{\co}[1]{\mbox{\textbf{#1}}}

%ch09

\newcommand{\cyrbulg}[1]{\begin{otherlanguage*}{bulgarian}#1\end{otherlanguage*}}


%ch10
\newcommand{\nlp}{{\small NLP}}
\newcommand{\mwe}{{\small MWE}}
\newcommand{\rae}{{\small RAE}}
\newcommand{\lvc}{{\small LVC}}
\newcommand{\pos}{{\small P}o{\small S}}
%\newcommand{\todo}[1]{ \textcolor{red}{#1} }

%\renewcommand{\labelenumi}{\theenumi}
%\ainamefmt{{vv}{ll}{, ff}{, jj}} % fullname

\newcommand{\biberror}[1]{{\color{red}#1}}

\newcommand{\osenovaitem}{--~}
  %% hyphenation points for line breaks
%% Normally, automatic hyphenation in LaTeX is very good
%% If a word is mis-hyphenated, add it to this file
%%
%% add information to TeX file before \begin{document} with:
%% %% hyphenation points for line breaks
%% Normally, automatic hyphenation in LaTeX is very good
%% If a word is mis-hyphenated, add it to this file
%%
%% add information to TeX file before \begin{document} with:
%% %% hyphenation points for line breaks
%% Normally, automatic hyphenation in LaTeX is very good
%% If a word is mis-hyphenated, add it to this file
%%
%% add information to TeX file before \begin{document} with:
%% \include{localhyphenation}
\hyphenation{
    Beck-man
    Ngu-yen
    back-chan-nel
    back-chan-nels
    mo-not-o-nous
    ste-reo-typ-i-cal
}

\hyphenation{
    Beck-man
    Ngu-yen
    back-chan-nel
    back-chan-nels
    mo-not-o-nous
    ste-reo-typ-i-cal
}

\hyphenation{
    Beck-man
    Ngu-yen
    back-chan-nel
    back-chan-nels
    mo-not-o-nous
    ste-reo-typ-i-cal
}

  \togglepaper[1]%%chapternumber
}{}

\begin{document}
\maketitle

\section{Introduction}
\label{sec:madaro:1}
Since \posscitet{denBesten1983} seminal work on continental West Germanic, this specific set of languages (e.g. German, Dutch) has been characterized by the well-known V2 phenomenon, which corresponds to the mandatory movement of the finite verb (henceforth: Vf) from its base position to the Head of the Complementizer Phrase (C\textsuperscript{0}) in the main sentence (V-to-C movement). Such structural movement involves the V2 (or linear) restriction, which is clarified by the following examples for standard German:

\ea%1
    \label{ex:madaro:1}
    \ea[]{\label{ex:madaro:1a}
    \gll Ich \textbf{gehe} morgen einkaufen.\\
         \textsc{1sg.nom} go tomorrow shop.\textsc{inf}\\}
    \ex[]{\label{ex:madaro:1b}
    \gll Morgen \textbf{gehe} ich einkaufen.\\
         Tomorrow  go \textsc{1sg.nom} shop.\textsc{inf}\\}
    \ex[*]{\label{ex:madaro:1c}
    \gll Morgen ich gehe einkaufen.\\
         Tomorrow \textsc{1sg.nom} go shop.\textsc{inf}\\
    }
    \z
    \glt ‘I’m going shopping tomorrow.’
\z


% % % \begin{table}
% % % \begin{tabularx}{\textwidth}{llllllllll lllll}
% % % \lsptoprule
% % % & Ich & \multicolumn{2}{c}{\textbf{gehe}} & \multicolumn{3}{c}{morgen} & einkaufen\\
% % % & 1SG & \multicolumn{2}{c}{go} & \multicolumn{3}{c}{tomorrow} & shop.inf &
% % %
% % % \multicolumn{2}{c}{Morgen} & \multicolumn{3}{c}{\textbf{gehe}} & ich & einkaufen\\
% % % & \multicolumn{2}{c}{Tomorrow} & \multicolumn{3}{c}{go} & 1SG & shop.inf &
% % %
% % % \multicolumn{2}{c}{*Morgen} & \multicolumn{2}{c}{ich} & \multicolumn{2}{c}{\textbf{gehe}} & einkaufen\\
% % % & \multicolumn{2}{c}{Tomorrow} & \multicolumn{2}{c}{ } & \multicolumn{2}{c}{go} & shop.inf\\
% % % \lspbottomrule
% % % \end{tabularx}
% % % \glt
% % % \end{table}

As \REF{ex:madaro:1a} and \REF{ex:madaro:1b} clearly show, when Vf moves to C\textsuperscript{0}, only one element can be found in the left periphery (i.e. in SpecCP), whether the sentence is subject-initial or not. The poly-occupation of this slot violates the V2-restriction, resulting in an ill-formed sentence, as in \REF{ex:madaro:1c}.

Solid evidence for C° as the target position for finite V-to-C movement is the complementary distribution between V2 and the lexical complementizer: in (standard) German subordinate clause introduced by an overt complementizer (e.g. \textit{dass}{}-sentences), Vf appears clause-finally \REF{ex:madaro:2b}, while in complementizer-less sentences (= COMP-deletion) a verb-second order mandatorily appears \REF{ex:madaro:2a}. The co-occurrence of an overt complementizer with V2 is generally ruled out, as in \REF{ex:madaro:2c}:

\ea%2
    \label{ex:madaro:2} (from \citealt{Vikner1995}: 66)
\ea[]{\label{ex:madaro:2a}  \gll Er    sagt,  diesen Film  \textbf{haben}     die Kinder    gesehen.\\
          \textsc{3sg.m.nom} says  \textsc{def.m.acc} Film  have   \textsc{def.pl.nom} children   seen\\}
\ex[]{\label{ex:madaro:2b}  \gll Er   sagt,  dass  diesen Film  die Kinder  gesehen \textbf{haben}.\\
          \textsc{3sg.m.nom}   says  that  \textsc{def.m.acc} Film  \textsc{def.pl.nom} children  seen      have\\}
\ex[*]{\label{ex:madaro:2c} \gll Er   sagt,  dass  diesen Film   \textbf{haben}   die Kinder   gesehen;\\
            \textsc{3sg.m.nom}   says    that   \textsc{def.m.acc} Film  have  \textsc{def.pl.nom} children  seen\\
        \glt ‘He says that the children saw/have seen the movie.’}
\z
\z

V-to-C movement in root sentences plus only one element allowed to the left of Vf represents the traditional correlation of V2 (but see \citealt{Holmberg2015} for a detailed analysis).

This specific restriction seems to represent a strong challenge to the hypothesis of an expanded C-layer, as postulated within the cartographic approach. Since \citet{Rizzi1997}, a split-CP configuration, i.e. a Complementizer Domain fragmented into multiple independent functional projections, has been adopted for Romance Languages to explain semantic and information-oriented features, as represented in \REF{ex:madaro:3}:

\ea%3
    \label{ex:madaro:3}
         ForceP > TopicP > FocusP > TopicP* > FinP > IP/TP…\footnote{In the configuration from Benincà \& \citet{Poletto2000} TopicP is higher than FocusP, while the lower TopicP from the Rizzian model is assumed to be an extension of FocusP (Benincà \& \citealt{Poletto2000}) and the structure is:ForceP > TopicP > FocusP > FinP > IP/TP…}
\z

Interestingly, German left periphery also clearly interrelates with information structure, even if limited to one constituent. As a matter of fact, object-fronting is allowed independently from its status within the information structure (either topic or focus) and restrictions occur only for elements that cannot be topicalized, such as indeterminate objects (see 4c) (\citealt{Fanselow2004Münchhausen, Mohr2009, Jouitteau2010}):\footnote{Even if object fronting in German is more complex than the discussion around \REF{ex:madaro:4} suggests (see \citealt{WierzbaFanselow2020}), we take it as a first step in order to discuss the complex relation between fronting (i.e. syntax) and information structure in V2-languages (see  \citealt{FanselowLenertová2011}). We thank one of the anonymous reviewers for pointing out these aspects and the related references to us.}

\ea%4
    \label{ex:madaro:4}(from \citealt{Mohr2009})
\ea[]{
\gll  Einen MINISTER \textbf{hat}   die Presse  schon lange kritisiert,   aber nicht  den Kanzler\\
      \textsc{indef.acc} MINISTER has   the press  already long criticized   but   not  \textsc{def.acc} chancellor\\
\glt ‘The press has long criticized a MINISTER, not the chancellor’}
\ex[]{\gll  Diesen Minister  \textbf{hat}  die Presse  schon lange kritiziert\\
     \textsc{dem.acc} Minister  has  the press  already long criticized\\
\glt ‘The press has long criticized this minister’}
\ex[*]{ \gll Einen Minister  \textbf{hat}  die Presse  schon lange kritisiert\\
            \textsc{indef.acc} minister  has  the press  already long criticized\\}
    \z
    \z

To account for these specific patterns in V2 languages (Germanic as well as for Old Romance and other Romance varieties), a theory-internal device, i.e. the bottleneck-effect, has been proposed (\citealt{BenincaPoletto2004, Haegeman1996, Roberts1996, Roberts2004, CardinalettiRepetti2010, Wolfe2015Microvariation, Wolfe2019VerbSecond}), claiming that in V2 languages stricto sensu all fronting operations target SpecFinP, which bears an EPP feature that needs to be checked \citep{Roberts2004}. Once a single phrase has moved to this specific position, the bottleneck activates, disabling any other movement process after f-checking. A further development has been proposed by \citet{Poletto2002}, claiming that the left periphery can be divided into two sub-areas/domains:

\ea%5
    \label{ex:madaro:5}

         [\textsubscript{ForceP} (XP) … [\textsubscript{TopicP} (XP) … {\textbar}{\textbar} [\textsubscript{FocusP} (XP\textsubscript{k}) … [\textsubscript{FinP} (XP\textsubscript{i}) [\textsubscript{Fin}\textsuperscript{0} Vf [\textsubscript{IP/TP} … t\textsubscript{i} t\textsubscript{k} ]]]]]]
    \z

As the structure shows, the lower domain inside the LP results as the target position for movement operation, while the elements inside the higher domain are considered to be base-generated directly there and do not interfere with the feature checking and the bottleneck in FinP.

As \citet{Hsu2017} points out, even though able to generalize the major difference in the restrictions among V2 and non-V2 languages, this hypothesis posits some theoretical issues, such as the reason for constituents that bear either topic- or focus-meaning to target SpecFinP before moving in their appropriate position, thus violating the Criterial Freezing (see \citealt{Rizzi2010}). Moreover, it fails to catch the degree of variation cross-linguistically and intra-linguistically: for these reasons, Hsu accounts for a revised model of head-bundling \citep[17]{Hsu2017}, based on the feature scattering hypothesis by \citet{GiorgiPianesi1997}.

Using the concepts of bundled/separate heads, this model seems to catch quite some interlinguistic differences across languages, as displayed in Figures~\ref{fig:madaro:1}--\ref{fig:madaro:3}.


\begin{figure}
% \includegraphics[width=\textwidth]{figures/a09MTBCPExpansionreviewed-img001.png}
\caption{Non-expanded CP, one specifier (e.g. German)}
\label{fig:madaro:1}
\begin{forest} for tree={align=center}
[Force/Top/Foc/FinP
  [XP$_k$]
  [Force/Top/Foc/Fin$'$
    [{Force/Top/Foc/Fin\\
      {}[Force]\\
      {}[Topic]\\
      {}[Focus]\\
      {}[Finiteness]\\
    }]
    [InflP
      [{\dots\st{XP$_k$}\dots}, roof]
    ]
  ]
]
\end{forest}
\end{figure}


\begin{figure}
% \includegraphics[width=\textwidth]{figures/a09MTBCPExpansionreviewed-img002.png}
\caption{First CP expansion, two specifiers (e.g. Ladin Badiot)}
\label{fig:madaro:2}
\begin{forest}for tree={minimum width=2.5cm}
[Force/TopP
  [XP]
  [Force/Top$'$
    [{Force/Top\\
      {}[Force]\\
      {}[Topic, $u$Top, EPP]\\
    }]
    [Foc/FinP
      [XP]
      [Foc/Fin$'$
        [{Foc/Fin\\
          {}[Focus, $u$Wh, EPP]\\
          {}[Fin]}]
        [InflP
            [~~~{\dots}~~~, roof]
        ]
      ]
    ]
  ]
]
\end{forest}
\end{figure}



\begin{figure}
% \includegraphics[width=\textwidth]{figures/a09MTBCPExpansionreviewed-img003.png}
\fittable{%
\begin{forest}for tree={minimum width=2cm}
[ForceP
  [XP]
  [Force$'$
    [{Force\\
      {}[Force, $u$Frame, EPP]
      }]
    [TopP
      [XP]
      [Top$'$
        [{Topic\\
          {}[Topic, $u$Top, EPP]}
        ]
        [Foc/FinP
          [XP]
          [Foc/Fin$'$
            [{Foc/Fin\\
              {}[Focus, $u$Foc, EPP]\\
              {}[Fin]
            }]
            [InflP
              [{~~\dots~~}, roof]
            ]
          ]
        ]
      ]
    ]
  ]
]
\end{forest}}
\caption{Fully articulated CP, more than two specifiers (e.g. Old Italian)}
\label{fig:madaro:3}
\end{figure}

Hsu’s typology appears to be solid and finds further confirmation in the case of the Germanic V2 varieties in Northeastern Italy (Triveneto), such as Timavese and Cimbrian, on which we focus in the next two sections respectively. Interestingly, Timavese displays a dual configuration regulated by an ordering constraint similar to the one proposed in \citet{GiorgiPianesi1997} and in a similar way as what has been suggested by \citet{Hsu2017} for Badiot (2\textsuperscript{nd} configuration) as a result of an interplay between movement vs. base-generation processes. Conversely in Cimbrian a more expanded CP seems to apply, with a configuration that resembles the one proposed for Old Italian (3\textsuperscript{rd} configuration), \textit{modulo} the trigger of Vf movement and pronominal syntax. The variations among Germanic linguistic islands in the Alpine region show a tendency towards a CP-expansion due to the separation from the Germanic core, suggesting two possible lines of development, that is a) from West to East, in parallel with the Austrian border, which includes Plodarisch, Saurian, Timavese, and b) from North to South, which includes the Cimbrian varieties of Lusern, Giazza, and Roana; the Mòcheno varieties are to be located as a pivot point between the two lines. We take as starting points of our analysis Timavese (for the West-East line) and Cimbrian (for the North-South line) to show how geographical distance from the Germanic core plays an important role in shaping the different degrees of variation, as suggested in \citet{GuardianoEtAl2016} for the Greek varieties in Apulia, which even though sharing macroparametric settings in common with other Greek varieties, tend to cluster with Extreme Southern Italo-Romance varieties (e.g. Salentino).

The higher line (the W-E one) is more conservative not only for the reduced CP-expansion but also for the maintenance of both nominal Subject inversion and the OV-patterns (cf. \sectref{sec:madaro:2}). As for the second line, the North-to-South one (cf. \sectref{sec:madaro:3}), we observe a further degree of distance from the Germanic core: the C domain results as more expanded, Subj-inversion is restricted to pronominal elements only and VO patterns have been fixed. Further evidence in favor of this analysis is provided by the conspicuous presence of historical sources for the Cimbrian varieties, which allows us to trace a diachronic evolution, showing a potential reflection of the different stages of synchronic variations attested in the two lines.

This chapter is organized as follows: in \sectref{sec:madaro:2} we provide the description and the analysis of the W-E line focusing on Timavese; \sectref{sec:madaro:3} is devoted to the description and the analysis of the N-S line focusing on the Cimbrian variety of Luserna; in \sectref{sec:madaro:4} we take into consideration the diachronic evolution of Cimbrian since the beginning of the XVII century; the last section  (\sectref{sec:madaro:5}) is dedicated to some conclusive remarks.

\section{The West-to-East line}
\label{sec:madaro:2}
Plodarisch, Saurian, and Timavese represent three of the four minority languages of Germanic origin in the Friuli-Venezia Giulia region (henceforth: FVG), together with Kanaltalerisch, spoken around Tarvisio.\footnote{The three varieties, however, differ from Kanaltalerisch as the latter represents an effective extension of Carinthian within Friuli, while Plodarisch, Saurian and Timavese are isolated from the German-speaking area.} Even geographically speaking, the three communities are located much further North than the Cimbrian and Mòcheno localities, almost on the Austrian border, with the only exception of Sauris, which is more “internal” than Sappada and Timau. In terms of external exposure, the last two are also geographically more accessible, in particular, Timau is characterized as a \textit{Straßendorf} and develops along a historical trade route that has connected the Romance and Germanic areas since ancient times. In this section we provide some data relating to the syntax of the German varieties spread across the West-to-East line concerning the aforementioned phenomena (OV/VO alternation, Subj-Verb inversion, and CP-expansion). As a general pattern, the V-to-C movement in root context still represents a solid trait in all these varieties, thus being classified as structural V2 languages, though more than one constituent may simultaneously appear to the left of Vf, that is within the C-domain, as the following examples show:

\ea%6
    \label{ex:madaro:6}
\ea Timavese\\
\gll Haintn avn plozzn  \textbf{hoon}=i  in Jani   (nitt)  tseachn\\
     Today  in.the square  have=\textsc{1sg}  \textsc{def.m.acc} John  (not)  seen\\
\glt ‘I haven’t seen John today in the center’
\ex Plodarisch\\
\gll Geischter  me bolde  \textbf{hot}=ar          si    (nét) werlezt\\
      Yesterday  in.the forest  has=\textsc{3sg.m.nom}   \textsc{3sg.refl}  (not) wounded\\
\glt ‘He didn’t wound himself in the woods yesterday’
\ex Saurian\\
\gll Geischter im poudn    on=i     (neit) pageignt  de schuelleararin\\
     Yesterday in square   have=\textsc{1sg}   (not) meet     the teacher.\textsc{f}\\
\glt ‘I didn’t meet the teacher in the city center’
\z
\z


The structural movement of Vf in C° is easily traced if we look at the post-verbal subject clitics, which appear in enclisis to the finite verb, as well as the postverbal negation, which are traditionally considered as diagnostics for Vf movement out of vP into the C-domain, as clarified in the following structure:

\ea%7
    \label{ex:madaro:7}

         [CP XP1 XP2… [C\textsuperscript{°} \textbf{Vf} [WackernagelP [TP [NegP [VP [… V° … \textbf{t\textsubscript{Vf}}]]]]]]]
    \z

Paired with the structural V2, these varieties share a “split-CP” configuration with an articulated periphery which seems to host multiple independent projections; in particular, Subj-V inversion (linear restriction) appears to be mandatory only with weak pronominal subjects, while Subject-DPs are usually placed pre-verbally, displaying a [XP DPSubj Vf] order in the main sentence.\footnote{Even if V3-patterns may be explained considering a multiple specifiers approach (\citealt{Lahne2009,BacskaiAtkari2023}), as suggested by an anonymous reviewer, it is not clear to us how this approach will derive restrictions on the possible number of specifiers. The gradual expansion of the C-domain attested throughout the Germanic varieties on which we are focusing finds a more appropriate explanation in a Feature-Scattering approach (see later for Timavese, “no more than two”-Constraint vs. Cimbrian fully CP-expansion).} As we will see (\sectref{sec:madaro:3.2}), the conditions which favor the nominal Subj-inversion represent the first major difference across these varieties, as well as the combination of both number and syntactic functions of the fronted XPs.

In the next sections, we present a fine-grained analysis of the varieties in FVG, namely Plodarisch, Saurian, and particularly Timavese, starting with the OV/VO alternation (\sectref{sec:madaro:2.1}), then proceeding with the different degrees of maintenance of nominal Subj inversion (\sectref{sec:madaro:2.2}), and finally the different modalities of CP-expansion (\sectref{sec:madaro:2.3}).

\subsection{OV/VO}
\label{sec:madaro:2.1}

Despite the great exposure to contact with Italo-Romance varieties, which characterized all the Germanic language islands across the Italian Alps, Plodarisch and Timavese seem to maintain some structural traits typical of German in a much more solid way (\citealt{GrewendorfPoletto2005}, \citealt{PolettoTomaselli2019, MadaroBidese2022}, \citealt{Madaro2024}). In particular, if we consider the OV structure in the main and subordinate clauses, it is easy to observe that the German-like verbal bracket is maintained:

\ea%8
    \label{ex:madaro:8}
\ea Plodarisch\\
\gll Hainte   \textbf{hot}  de mome  de alan  me morkt  \textbf{gakaaft}\\
     Today   has  \textsc{def.f.nom} mother  \textsc{def.pl.acc}  eggs  in market  bought\\
\ex Timavese\\
\gll Haitn  \textbf{hott}  da muatar  da ailan  im morkt  \textbf{gackaaft}\\
     Today  has   \textsc{def.f.nom}  mother  \textsc{def.pl.acc}  eggs  in market  bought\\
\glt ‘Today the mother bought eggs at the market’
  \z
\z

In subordinate clauses, both varieties show instances of V(P)R-related phenomena coherently with an OV typology, even though in Timavese VO structures with Objects-DPs outside the verbal brackets are attested and widely accepted, suggesting much more freedom within the syntactic structure due to pragmatic/information structure traits\footnote{As claimed in \citet{MadaroBidese2022}, verbal clusters in Timavese (as well as Saurian and Cimbrian) are generally found in an [AUX V] order in subordinates introduced by a lexical complementizer, while the order [V AUX], typically found in Continental West-Germanic with OV base, is considered not to be acceptable among speakers. This specific trait has led to the hypothesis that these patterns should not be treated as instances of VR, but of a crystallized setting of verbal cluster, probably a first instance for switching from an OV order to a VO one.}. These types of occurrences are ruled out in Plodarisch:

\ea%9
    \label{ex:madaro:9}
\ea Timavese\\
\gll Dar Jani  hot=mar   sok   [as=ar   (aa ringl) \textbf{hott} \textbf{gackaaft} aa ring   vir sai baib]\\
     The John   has=\textsc{1sg.dat} said  [that=\textsc{3sg.m.nom}   (a ring)   has bought     a ring  for his wife]\\
\glt ‘John told me that he bought his wife a ring’
\ex Plodarisch\\
\gll *Sie hont=mer   gesouk [as  ana Kronkhet \textbf{hot}   \textbf{geteten}  de Kie]\\
     \textsc{3pl} have=\textsc{1sg.dat} said     [that   a disease   has   killed    the cows]\\
\glt ‘They told me that a disease killed the cows’
    \z
    \z

On the other hand, Saurian displays a higher degree of OV/VO alternation as far as interrogatives/infinitives are considered (see examples 10a,b and 10c,d). The possibility of having a preverbal negation in subordinate context (example 10e) highlights a certain degree of asymmetry, even though variation occurs, as \REF{ex:madaro:10f} clearly shows:

\ea%10
    \label{ex:madaro:10}

     Saurian
\ea  \label{ex:madaro:10a}
\gll Bein  \textbf{ot}  der khiokhmon  \textit{in}   \textit{vuks}  \textbf{gesehn}?\\
     when  has  \textsc{def.m.nom} hunter   \textsc{def.m.acc} fox  seen?\\
\ex  \label{ex:madaro:10b}
\gll Bein  \textbf{ot}  der khiokhmon  \textbf{gesehn}  \textit{in}     \textit{vuks}?\\
     when  has  \textsc{def.m.nom} hunter    seen   \textsc{def.acc} fox?\\
\glt ‘When did the hunter see the fox?’
\ex  \label{ex:madaro:10c}
\gll I  hon=en    geheart  \textbf{vleitn}    \textit{‘s} \textit{gaigele}.\\
     \textsc{1sg}  have=\textsc{3sg.acc} heard    play.\textsc{inf}  the fiddle\\
\ex  \label{ex:madaro:10d}
\gll I  hon=en    geheart  \textit{‘s} \textit{gaigele}  \textbf{vleitn}.\\
       \textsc{1sg}  have=\textsc{3sg.acc} heard    the fiddle  play.\textsc{inf}\\
\glt ‘I heard him playing the violin.’
\ex  \label{ex:madaro:10e}
\gll I  houfe  [as=ar   \textit{net}  \textbf{geat}  in bolt].\\
     \textsc{1sg}   hope  [that=\textsc{3sg.m.nom}  not  goes  in woods]\\
\glt ‘I hope that he doesn’t go into the woods.’
\ex  \label{ex:madaro:10f}
\gll I  houfe  [as    ‘s  \textbf{schneibet}  \textit{neit}  haite].\\
     \textsc{1sg}   hope  [that \textsc{3sg.impers}  snows    not  today]\\
\glt ‘I hope that it doesn’t snow today.’\footnote{As the two possible positions for negation show, it seems unclear whether Saurian may or may not be considered an asymmetric language. For further evidence see the discussion in \citet{Costantini2019Verb} and \citet{Bidese2019}, both from the same volume.}
    \z
    \z

Nevertheless, a contrastive analysis between Modern Saurian and some early nineteenth-century texts shows an increase in terms of the diffusion of VO-type orders diachronically \citep{Costantini2019Aspetti}, an aspect that would lead to considering the variety as VO-dominant with alternation (\citealt{MadaroBidese2022}), with a possible shift from an asymmetric to a symmetric word order (see position of object clitics and negation in subordinates).

\subsection{Subject-verb inversion}
\label{sec:madaro:2.2}
As said before, multiple elements within the LP in root clauses are widely attested and violations of the V2 restriction seem to apply quite frequently even in Plodarisch, which is considered the most conservative among the varieties we analyze in this section.

A first hint of comparison is represented by the degree of acceptability of Subj-V inversion in the root declarative context: as already mentioned, the pattern applies mandatorily with “weak” pronominal subjects in all the FVG varieties, while full Subjects are more frequent preverbal. In Saurian, for instance, [XP Vf DPSubj] order represents the least preferred choice, even though acceptable (examples and acceptability rates taken from \citealt{Costantini2019Aspetti}: 101):

\ea%11
    \label{ex:madaro:11}

         Saurian
\ea \label{ex:madaro:11a}
\gll Geister   \textbf{ot}  \textit{der} \textit{Peter} gelezn   a puech.  [0.4]\\
     Yesterday  has  \textsc{def.m.nom} Peter  read  a book\\
\ex  \label{ex:madaro:11b}
\gll Geister   \textit{der} \textit{Peter}  \textbf{ot}    gelezn   a puech.  [0.9]\\
     Yesterday \textsc{def.m.nom} Peter  has  read      a book\\
\glt ‘Peter read a book yesterday.’
    \z
    \z

This pattern is divergent in Timavese (and apparently also in Plodarisch: see \citealt{GrewendorfPoletto2005}),\footnote{As stated in \citet{Madaro2024}, both orders seem to be acceptable among native speakers (six informants: 4 female informants, 2 male informants) on a Likert scale from 1 (not acceptable) to 6 (totally acceptable), here reported on a 0--1 scale through proportion to be aligned with the data for Saurian:
\ea
\gll
Da vargonnana boucha {}  dar Gianni    is   af Paluc   mit  dar       maschine gongan. [0.95]\\
The last.\textsc{f}   week    {}  \textsc{def.m.nom} John      is  to Paluzza with \textsc{def.f.dat} auto     gone\\
\ex[]{
\gll
Da vargonnana boucha is  dar Gianni    {}   af Paluc   mit  dar       maschine gongan. [1.0]\\
The last.\textsc{f}   week    is  \textsc{def.m.nom} John      {}  to Paluzza with \textsc{def.f.dat} auto     gone\\}
\glt ‘Last week John went to Paluzza by car.’
\z} both orders are well-accepted, and the acceptability may be due to external factors, such as information structure (see \citealt{Madaro2024}).

However, some instances of post-verbal DPSubjects are generally attested and well-accepted (but not mandatory) also in Saurian, specifically if the first constituent is contrastively focalized:

\newpage
\ea%12
    \label{ex:madaro:12}
         Saurian\\
\gll   GEISTER   (der Peter)  \textbf{ot}  (der Peter)  gelezn  a puech net haite   [0.9]\\
       Yesterday  (\textsc{def.m.nom} Peter)  has  (\textsc{def.m.nom} Peter)  read  a book not today\\
\glt ‘Peter read a book yesterday, not today’
    \z

Root contexts introduced by a Wh-operator appear to be interesting as far as Subj-V inversion is concerned. As \citet{Cognola2013Syntactic, Cognola2013Mixed} already pointed out for Mòcheno, subject inversion in Wh-questions occurs only with pronominal subjects, while Subjects-DPs are to be placed outside the matrix clause, either as left- or right-dislocated elements, with a resumptive element (pronoun) positioned clause-internally. This phenomenon strongly contrasts with Standard German and related varieties (e.g. Bavarian/South-Tyrolean) where Wh-operators trigger Subj-V inversion. Interestingly, it is well attested in other varieties we are analysing. See the examples below for Timavese (13a-b) in contrast with (standard) German \REF{ex:madaro:13c}:

\ea%13
\label{ex:madaro:13}
\ea  Timavese\\\label{ex:madaro:13a}
\gll \textit{Dar} \textit{Jani}   bos  \textbf{hott}=ar=en    tsok?\\
     \textsc{def.m.nom} John  what  has=\textsc{3sg.m.nom}=\textsc{ptc}  said?\\
\ex  *Bos  \textbf{hot}=n  \textit{dar Jani}  tsok?\label{ex:madaro:13b}
\ex  German\\\label{ex:madaro:13c}
\gll Was  \textbf{hat}  \textit{Johann}  gesagt?\\
     What  has  John    said?\\
\glt ‘What has John said?’
    \z
    \z

In these varieties, there seems to be an incompatibility between pre-verbal DPSubjects and WH-elements; in particular, the Wh-Criterion seems somehow to interfere with the EPP-checking/Subject-agreement in SpecFinP: the finite Verb in Fin° either satisfies the WH-criterion (i.e. the Spec-head relation with the WH-element) or enters a Subject-agreement relation. The fact that the DPSubject cannot occur immediately on the right of the finite verb~-- in the lower [Spec, TP], see \REF{ex:madaro:13b}~-- but only in a peripheral position could be explained considering the role of C°/Fin° as nominative case assigner in a structural V2 language. More specifically, in C-dominant languages (such as German) C is responsible for NOM-assignment (and eventually pro-licensing) in the lower T domain, while T is in charge in T-dominant languages (such as Italian).\footnote{\citet{BidesePadovanTomaselli2020} propose a re-visitation of the distinction between COMP-dominant versus INFL-dominant language (which goes back to  \citealt{HulkKemenade1995}) adopting Ouhali’s mechanism of feature-inheritance: SHARE versus DONATE. Cf, \sectref{sec:madaro:3.2} for its extension to Cimbrian.} Crucially SpecTP represents the structural position in which NOM is assigned in both C- and T-dominant languages. In Timavese the DPSubject may occur in [Spec, TP] -on the right of the finite verb- only under specific conditions but its unmarked position corresponds to the preverbal one, i.e.: [Spec, FinP]. We speculate on the hypothesis that the satisfaction of the WH-criterion inhibits Nominative case assignment by Fin° to the DPSubject which could be rescued only by the intervention of the enclitic pronoun which absorbs case and forms a chain with the dislocated DP, resembling the configuration which characterizes the Northeastern Italo-Romance varieties.\footnote{In
     north-eastern Italo-Romance varieties such incompatibility is well expressed by the mandatory presence of a (resumptive) clitic which undergoes enclisis to Vf in [+Wh]-sentences (differently from declarative root sentences), thus showing Subj-inversion, as in the case of Friulian (GEMONA):

     \ea
       \gll Maria l=à        cumprâ el libri\\
            Maria \textsc{3sg.fr}=has  bought a book\\
       \glt `Maria bought a book'
     \ex
       \gll Maria,         ce     ch'  à=la   cumprâ?\\
            {As for Mary,}   what   that has=\textsc{3sg.f} bought?\\
       \glt `What did Mary buy?'
     \z
}

Without entering further details, it is interesting to anticipate the following observation: the ban for the DPSubject to occur in [Spec, TP] – which characterizes only the WH-root clause in Timavese – represents a generalized restriction in the Cimbrian varieties where it applies in both the interrogative and the declarative root clause (cf. \sectref{sec:madaro:3}).  As a matter of fact, the WH-root clause confirms its double role in the process, which leads to the dismantling of V2: in English, the WH-root clause represents the context where V2 is maintained (Subject-AUX inversion as residual V2), in the Germanic enclaves of the Italian Alpine region it represents the context where Subject-Vf inversion “à la German” is first lost.

If we take into consideration all the aspects touched upon so far in a preliminary manner, it seems possible to outline a sort of continuum that connects all the varieties of the West-to East line as shown in \tabref{tab:madaro:1}.


\begin{table}
\footnotesize
\tabcolsep=.5\tabcolsep
\begin{tabularx}{\textwidth}{l *7{c}}
\lsptoprule
 &  \multicolumn{2}{c}{V2}  & \multicolumn{2}{c}{inversion in decl. clause} & \multicolumn{2}{c}{inversion in WH-Qs} & \\
 \cmidrule(lr){2-3}\cmidrule(lr){4-5}\cmidrule(lr){6-7}
 & {Linear} & {Structural} & {DPSubj-Vf} & {Subj\textsubscript{pron}}{{}-Vf} & {DPSubj-Vf} & {Subj\textsubscript{pron}}{{}-Vf}{\textsubscript{}}  & {asymmetry}\\
 \midrule
(German)    & + & + & +   & + & +  & + & +\\
Plodarisch  & − & + & +   & + & +? & + & +\\
Timavese    & − & + & +   & + & −  & + & +\\
Saurian     & − & + & (+) & + & −  & + & \#\\
(Mòcheno)   & − & + & (+) & + & −  & + & +\\
\lspbottomrule
\end{tabularx}
% % % % \todo[inline]{suggested transposition of table}
% % % % \begin{tabularx}{\textwidth}{Qcccccc}
% % % % \lsptoprule
% % % %  & {(German)} & {Plodarisch} & {Timavese} & {Saurian} & {(Mòcheno)} \\
% % % % \midrule
% % % % {Linear V2} & + & {}- & {}- & {}- & {}- \\
% % % % \tablevspace
% % % % {Structural V2} & + & + & + & + & + \\
% % % % \tablevspace
% % % % {DPSubj-Vf inversion in decl. clause} & + & + & + & (+) & (+) \\
% % % % \tablevspace
% % % % {Subj\textsubscript{pron}-Vf inversion in decl. clause} & + & + & + & + & + \\
% % % % \tablevspace
% % % % {DPSubj-Vf inversion in Wh-Qs} & +? & + & {}- & {}- & {}- \\
% % % % \tablevspace
% % % % {Subj\textsubscript{pron}-Vf inversion in WH-Qs} & + & + & + & + & + \\
% % % % \tablevspace
% % % % {asymmetry} & + & + & + & \# & + \\
% % % % \lspbottomrule
% % % % \end{tabularx}
\caption{W-to-E line features settings}
\label{tab:madaro:1}
\end{table}

As we will see in \sectref{sec:madaro:3} (cf. \tabref{tab:madaro:3}, \sectref{sec:madaro:3.2}) the gradualness which characterized the loss of DPSubj inversion \textit{à la} German has gone further in the North-to-South line \textit{au-pair} with the gradual expansion of the CP. In the next section (\sectref{sec:madaro:2.3}) this hypothesis will be explored in depth through an analysis of V3 orders in Timavese, trying to identify possible restrictions even within a more articulated CP.

\subsection{CP-Expansion}
\label{sec:madaro:2.3}
\begin{sloppypar}
There is robust evidence in the literature of non-canonical V3 orders across V2 languages with specific types of elements, such as adverbial clauses and, more generally, frame setters which seem not to trigger Subj-V inversion in root clauses with pronominal Subjects:
\end{sloppypar}

\ea%14
    \label{ex:madaro:14}
\ea  West Flemish (\citealt{HaegemanGreco2018West})\\
\gll Als ‘t geijzeld is  ze   \textbf{risschiert}  heur       niet   buiten\\
     when it frosty is  \textsc{3sg.f}  risks    \textsc{3sg.refl} not outside\\
\glt ‘When it is frosty, she does not venture outside’

\ex  Kiezdeutsch \citep{Wiese2009}\\
\gll   Morgen  ich  geh  arbeitsamt\\
       Tomorrow  \textsc{1sg}  go  job.centre\\
\glt ‘Tomorrow I will go to the job center’

\ex  Norvegian Urban Vernacular \citep{FreywaldEtAl2015}\\
\gll Med limewire  det       \textbf{tar}       en to dager\\
     with Limewire \textsc{3sg.n}    takes  one two days\\
\glt ‘Using Limewire it takes 1 or 2 days’
    \z
    \z

Similar examples may also be found in (Eastern/Western) Yiddish (\citealt{teVelde2017}), Urban Vernacular Danish, and Urban Vernacular Swedish \citep{Walkden2017Language}, and diachronically, OHG/OE show similar “deviation-from-linear-V2” patterns, thus highlighting a [FrameP + V2] order. Such elements seem not to directly influence the internal structure of the clause, either being base-generated externally or being merged with the following subject pronouns as a single phonetic unit (see \citealt{teVelde2017}). These phenomena result in an apparent (superficial) violation of the V2 restriction and may be due to some post-syntactical operation. This pattern strongly differs from V3 in non-V2 languages, as in Italian, where the Vf is preceded by two nominal objects both in main and embedded clauses:

\ea%15
    \label{ex:madaro:15}
\ea
\gll A Gianni   il libro     glie =l         =ho dato ieri.\\
    to J.   the book \textsc{3sg.dat} =\textsc{3sg.acc} =\textbf{have} given   yesterday\\
\glt ‘I gave the book to John yesterday.’

\ex
\gll Ho   detto   [che  a Gianni   il libro    glie        =l \textbf{ho}   dato   ieri].\\
     have   said   [that   to J.     the book \textsc{3sg.dat} =\textsc{3sg.acc} =have   given  yesterday]\\
\glt ‘I said that I gave the book to John yesterday.’
    \z
\z

Back to our varieties, non-inverted V2 orders with frame-setters\slash adverbial clauses are also attested in Timavese, independently from their status related to the information structure:\footnote{Preferred version. As the examples (i) and (ii) show, the DPSubject [+new] may occupy two distinct positions, either right before Vf or in a lower vP-focus-position (below the adverb \textit{obla}), which is favored by the Subject itself being [+new information] (for a standard assumption about a lower Focus position in vP see among others \citealt{Belletti2004,Belletti2018}, but alto \citealt{PetrovaHinterhölzl2010} for OHG). As a confirmation, it is important to note that a SpecTP position higher than \textit{obla} is available only by a [-new] DPSubject. See the following sentences:
\begin{exe}[(vi)]
\exi{(v)}[*]{In lohn ckaft da nona [+new] obla is batzameal}
\exi{(vi)}[]{In lohn ckaft da nona [−new] obla is batzameal}
\end{exe}
}

\ea%16
\label{ex:madaro:16}
\ea        Q: “Who does always buy the flour at the grocery shop?”\\
           SUBJ [new] + FS [given]
\ea[]{
\gll In lohn  da nona  \textbf{ckaft}    obla    is batzameal.\\
  In market    \textsc{def.f} grandma   buys     always   \textsc{def.n.acc} flour\\}

\ex[]{
\gll In lohn  \textbf{ckaft}  obla    da nona     is batzameal.\\
    In market      buys   always  \textsc{def.f} grandma     \textsc{def.n.acc} flour\\}

\ex[*]{
\gll Da nona    in lohn        \textbf{ckaft}  obla    is batzameal.\\
     \textsc{def.f} grandma  in market    buys     always  \textsc{def.n.acc} flour\\}
\z
\glt ‘The grandmother always buys the flour at the grocery shop.’

\ex        Q: “When did mom buy the dress?”\\
           SUBJ [given] + FS [new]
\ea{
\gll Geister nochmittag   da mama  \textbf{hot}      in keittel        gackaft.\\
     yesterday afternoon  \textsc{def.f} mom  has       \textsc{def.m.acc} dress  bought\\}
\ex[*]{
\gll Geister nochmittag \textbf{hot}  da mama in keittel           gackaft.\\
     yesterday afternoon  has  \textsc{def.fr} mom  \textsc{def.m.acc} dress  bought\\}
\z
\glt ‘The mom bought the dress yesterday.’
\z
\z

Apparently, in cases of frame setter/adverbs-initial sentences, DPSubjects are usually attracted in a pre-verbal position, as evidenced by sentences in 16b; with pronominal subjects, conversely, the pattern [XPframe-setter Vf=pronominal subject] is perfectly grammatical: this aspect may suggest a certain degree of specialization within left-peripheral position, as for SpecFinP in Cimbrian (cf. \sectref{sec:madaro:3.2}).

On the other end, object fronting (as [+topic] or [+focus] constituent, with no resumptive element) satisfies the EPP feature in SpecCP, thus triggering Subj-V inversion:

\ea%17
    \label{ex:madaro:17}


\ea       Q: “Who has bought the car to Maria for her birthday?”

          SUBJ [new] + OBJ [given]

\ea[]{
\gll Dar Maria-n     \textbf{hott}   da nona  in     auto    gackaft.\\
\textsc{def.f.dat} Maria-\textsc{dat}   has   \textsc{def.f.nom} grandma  \textsc{def.m.acc}  car       bought\\ }
\ex[]{
\gll *Dar Maria-n    da nona     \textbf{hott}  in  auto     gackaft\\
\textsc{def.f.dat} Maria-\textsc{dat}  \textsc{def.f.nom} grandma   has  \textsc{def.m.acc} car     bought\\}

\z
\glt ‘The grandmother bought Maria the car’

\ex    Q: “Did the kids eat everything?”

SUBJ [given] + OBJ [new = FOCUS]\footnote{As the examples show, Object-fronting with no resumption is still possible in Timavese, differently from other varieties, such as Cimbrian, where object-fronting always occurs as left-dislocation with a resumptive object pronoun (cf. \sectref{sec:madaro:3.2}). Nonetheless, fronting is acceptable if DP-Object is [+given], or topic information; when [+new], they cannot undergo simple fronting unless contrastively focussed.}
\ea[]{
 \gll  is VLAISCH \textbf{hont}   da ckindar    geissn  da greisn            nitt.\\
\textsc{def.n} meat have   \textsc{def.pl} kids        eaten   \textsc{def.pl} veggies   not\\}

\ex[*]{
\gll  is VLAISCH da ckindar \textbf{hont}   geissn     da greisn           nitt.\\
\textsc{def.n} meat   \textsc{def.pl} kids     have   eaten   \textsc{def.pl} veggies not\\
       }
\z

\glt ‘The kids ate the meat, not the vegetables’
    \z
   \z

If the topicalized element (either a DPSubject or DPObject) is doubled by a clause-internal resumptive element, poly-occupation is allowed, suggesting that in sentences like \REF{ex:madaro:18}, they occupy a different structural position within the C-domain, that is within the higher sub-domain dedicated to base-generated non-fronted DPs:
\ea%19
    \label{ex:madaro:18}
    \ea[]{
      \gll Dar Maria-n\textsubscript{i}     da nona  \textbf{hott}=ar\textsubscript{i}  in auto  gackaft.\\
           \textsc{def.f.dat} Maria-\textsc{dat} \textsc{def.f.nom} granma  has=\textsc{3sf.f.dat}  \textsc{def.m.acc} car  bought\\}
      \glt ‘As for Maria, the grandmother bought her the car.’

    \ex[]{
      \gll Da ckindar\textsubscript{i}  is VLAISCH  \textbf{hont}=sa\textsubscript{i}  geissn  da greisn nitt.\\
           \textsc{def.pl} kids  \textsc{def.n} meat  have\textsc{=3pl}    eaten    \textsc{def.pl} vegies      not\\}
      \glt ‘As for the kids, they ate the meat, not the vegetables.’
      \z
      \z

Apparently, the assumption of two subdomains in the CP – as proposed by \citet{Poletto2002} – is quite solid also in Timavese. Fronting via movement (no resumption) suffices to check the EPP feature: once a DPObject has been moved to the LP, any other kind of fronting via movement is disabled. If an element has already moved in this lower domain, additional elements may appear only if they are adjuncts (frame-setters) or doubled by a resumptive element (via left-dislocation).

Even in the higher domain (the base-generated one) though, some restrictions seem to apply, as it is shown in \REF{ex:madaro:19}:

\ea%19
    \label{ex:madaro:19}
\ea[]{
Q: “Who has bought the stuff I asked?”\\
     OBJ: [+new] SUBJ: [given (+ RE)] + FS\\
    \gll *Geister   mai pruadar\textsubscript{i}    is VLAISCH    \textbf{hott}  =ar\textsubscript{i}        gackaft       nitt   in visch.\\
    Yesterday   my.\textsc{nom} brother   \textsc{def.n} meat has  \textsc{=}\textsc{3sg.m.nom}  bought        not  \textsc{def.m.acc} fish\\}

\glt ‘Yesterday my brother bought the meat, not the fish (that was bought by my sister).’

\ex[]{  Q: “Who has bought the newspaper today?”\\
    SUBJ: [+new] OBJ: [given (+ RE)] + FS\\
\gll  *Haintn da     zaitin\textsubscript{t}     dar Jan und da Mari \textbf{hont}=sa\textsubscript{t}            gackaft.\\
        Today  \textsc{def.f.nom} newspaper   \textsc{def.m}   Jan and  \textsc{def.f} Mary have=\textsc{3sg.f.acc}    bought\\}
\glt ‘As for the newspaper, John and Mary have bought it.’
    \z
   \z

Apparently, the sentences in 19a-b are agrammatical in Timavese. However, adverbial phrases\slash frame-setters and left\hyp dislocated elements should not trigger any restriction, as they are elements positioned in the highest domain (intended for base generation) of an articulated CP.\footnote{In fact, a structure with [adv -- XP (left-disl) Vf=RE ...] turns out to be perfectly acceptable:

\ea
\gll Haintn      da      zaitin\textsubscript{t} hont=sa\textsubscript{t}  dar Jan und da  Maria      gackaft\\
     Today         \textsc{def.f} newspaper               have=\textsc{3sg.f}             \textsc{def.m} Jan and   \textsc{def.f} Maria      bought\\
\z
} The equivalent of this structure in Italian (or other Italo-Romance varieties) or Cimbrian would be acceptable. This leads us to assume that even within the highest domain there are restrictions in terms of element placement.

See \tabref{tab:madaro:2} for a summary of the possible combinations examined so far.


\begin{table}
\begin{tabularx}{\textwidth}{llQQllcl}
\lsptoprule
\multicolumn{2}{c}{{Base-g} {domain}} & \multicolumn{2}{c}{{Move-domain}} & {C°} & {CliticP} & {IP/TP} & \\
\midrule
\multicolumn{2}{c}{ FrameS} & \multicolumn{2}{c}{ DPSubj} & Vf & {}- & … & {Ok}\\
\multicolumn{2}{c}{ FrameS} & \multicolumn{2}{c}{ DPObj} & Vf & {}- & … & {Ok}\\
\multicolumn{2}{c}{ DPSubj (new/given)} & \multicolumn{2}{c}{ DPObj} & Vf & RE(DPSubj) & … & {Ok}\\
\multicolumn{2}{c}{ DPObj (new/given)} & \multicolumn{2}{c}{ DPSubj} & Vf & RE(DPObj) & … & {Ok}\\
 FrameS & DPSubj & \multicolumn{2}{c}{} & Vf & RE(DPSubj) & … & {Ok}\\
 FrameS & DPObj & \multicolumn{2}{c}{} & Vf & RE(DPObj) & … & {Ok}\\
 FrameS & DPSubj & \multicolumn{2}{c}{ DPObj} & Vf & RE(DPSubj) & … & {*}\\
 FrameS & DPObj & \multicolumn{2}{c}{ DPSubj} & Vf & RE(DPObj) & … & {*}\\
\multicolumn{2}{c}{} & DPObj & DPSubj & Vf &  &  & {*}\\
\lspbottomrule
\end{tabularx}
\caption{LP occurrences in Timavese}
\label{tab:madaro:2}
\end{table}

\newpage
After examining all the possible combinations of elements within an articulated CP in Timavese, it is possible to hypothesize some generalizations:

\ea LP constraints in Timavese:

\begin{itemize}
\item 2 (or more) XPs never occur in the LP via movement.
\item 2 XPs may occur if at least one of them is base-generated directly in the LP.
\item If (b) occurs, no other XP can be moved to or base-generated in the LP.
\end{itemize}
\z

Now going back to the original model assumed in \citet{Hsu2017} for bundled heads, it is possible to sketch a dual configuration of Timavese CP.

\begin{figure}
% % % \includegraphics[width=\textwidth]{figures/a09MTBCPExpansionreviewed-img004.png}
\footnotesize%
\begin{subfigure}{.55\textwidth}
\centering
\begin{forest}
[FrameP,for tree={minimum width=1.5cm}
  [XP]
  [Frame$'$
    [{Frame\\{}[Frame]}]
    [TopicP/FocusP/FinP
      [XP]
      [Topic/Focus/Fin$'$
        [{Topic/Focus/Fin\\
          {}[Topic]\\
          {}[Focus, $u$Wh]\\
          {}[Fin, EPP]\\
        }]
        [IP]
      ]
    ]
  ]
]
\end{forest}
\caption{}
\label{fig:madaro:4a}
\end{subfigure}%
\begin{subfigure}{.45\textwidth}
\centering
\begin{forest}
[FrameP/TopicP
  [XP]
  [Frame$'$'/Topic$'$
    [{Frame/Topic\\
      {}[Frame]\\
      {}[Topic/EPP]
    }]
    [FocusP/FinP
      [XP]
      [Focus/Fin$'$
        [{Focus/Fin\\
          {}[Focus, $u$Wh]
          {}[Fin, EPP]\\
          {~}\\
          {~}
        }]
        [IP]
      ]
    ]
  ]
]
\end{forest}
\caption{}
\label{fig:madaro:4b}
\end{subfigure}%
\caption{Dual CP configuration in Timavese, (only) two Specs allowed}
\label{fig:madaro:4}
\end{figure}

These configurations help us to catch the observable phenomena in Timavese C-domain: it is possible to hypothesize a [Base-gen +V2] CP, meaning a “Spatial” restriction: no more than two constituents seem to be allowed in Timavese LP, either both being base-generated or one moved plus one base-generated. Moreover, the bottleneck effect alone is not sufficient to determine the preferred CP configuration: the occurrences can be captured if we hypothesize two different structures, in which the feature [Topic] may be bundled together either to FrameP or to an already bundled FocusP-FinP: the choice follows the Universal Ordering Constraint \citep{GiorgiPianesi1997}, and the bundled FocusP-FinP head is the first one to be checked. Contrary to the West-to-East line, the varieties across the North-to-South line have developed a fully expanded CP with a structural TopicP higher than FocP and a lower [Spec, Fin] which has specialized as structural Subject position.

\section{The North-to-South line}
\label{sec:madaro:3}
In this section we will provide some data relating to the syntax of the German varieties spread across the North-to-South line. We will focus our attention on the Cimbrian variety spoken in Luserna (TN) – the enclave where the majority of Cimbrian speakers live – and we will compare it occasionally with:

\begin{itemize}
\item Mocheno – which is spoken in the Fersina valley and still maintains partial intercomprehension with the German dialects of South Tyrol
\item the Cimbrian variety spoken in Giazza/Ljetzan (VR) –which nowadays counts on no more than a dozen native speakers.\footnote{As for the Cimbrian variety spoken in Roana/Robaan (VI) we do not have, at least for the moment, enough data either collected or discussed in previus literature in order to draw a comparable picture and we will leave this further extension to future research.}
\end{itemize}

In line with what has been already shown for the West-to-East line, all these varieties have lost the V2 linear restriction but maintain structural V2, i.e. Vf movement to the C-domain in the root declarative clause, with the only possible exception of the Cimbrian variety spoken in Giazza\slash Ljetazn. Let us consider the following data:\footnote{For Mòcheno we refer to \citet{Cognola2013Syntactic, Cognola2013Mixed}; for the Giazza variety to \citet{Bidese2023, BideseEtAl2012} and \citet{BidesePadovanTomaselli2020}; for the Cimbrian variety spoken in Luserna to \citet{BideseEtAl2012, Bidese2023} and  \citet{BidesePadovanTomaselli2020}.}

\ea%21
    \label{ex:madaro:21}
Mòcheno\\
\gll Hait  en de boteig  \textbf{hòn} =e =en    nèt tsechen.\\
     Today  in the shop  have =\textsc{1sg.nom} =\textsc{3sg.m.acc}  (not) seen\\
\glt ‘Today I have (not) seen him in the shop.’
\ex%22
    \label{ex:madaro:22}
   Cimbrian (Luserna variety)\\
\gll Haüt  in balt    hatt =ar    =en    nèt gesek.\\
     Today  in forest  haa =\textsc{3sg.m.nom} =\textsc{3sg.m.acc}  not seen\\
\glt ‘Today he has not seen him in the wood.’
\ex%23
    \label{ex:madaro:23}
    Cimbrian (Giazza variety: from \citealt[157]{Scardoni2000})\\
  \gll  Vour    tz’ essan   er   besce=ci     sempar de hente.\\
        Before to eat.\textsc{inf}  \textsc{3sg.m.nom}   wash=\textsc{3sg.refl} always \textsc{def.pl} hands\\
  \glt ‘Before eating he always washes his hands.’
\z

As the examples 21 and 22 show, Mocheno and the Luserna variety of Cimbrian are still characterized by a reduced verbal bracket which embraces both the enclitic pronoun(s) and sentence negation (\textit{nèt}) reproducing the well-known German structure but with an expanded CP (cf. \sectref{sec:madaro:2}, point 7):

\ea%24
    \label{ex:madaro:24}

         [CP XP1 XP2… [C\textsuperscript{°} \textbf{Vf} [WackernagelP [TP [NegP [VP [… V° … \textbf{t\textsubscript{Vf}}]]]]]]]
    \z

Only in the Cimbrian variety spoken in Giazza/Ljetzan (see example 23) the subject pronoun occurs on the left of the finite verb and the lexicalization of the Wackernagel position is restricted to reflexive pronouns and object clitics hinting to a potential restructuring of the system (loss of structural V2 and maintenance of a low verbal bracket), as we will discuss in the final remarks (cf. 5).

In the following paragraphs we will show that the verbal bracket may be expanded only in Mòcheno which maintains both OV (cf. 3.1) and nominal subject inversion (cf. \sectref{sec:madaro:3.2}) as residual word order possibilities displaying a situation similar to the one already discussed for Saurian. Rather on the contrary the Cimbrian varieties have definitely lost nominal subject inversion and display a stable VO typology in contrast to what has been discussed for the varieties of the West-to-East line. As we are going to show, a further relevant difference which characterizes the North-to-South line with respect to the West-to-East line concerns the structure of the expanded CP with particular reference to the modality of object fronting (cf. \sectref{sec:madaro:3.3}).

\subsection{OV/VO}
\label{sec:madaro:3.1}
The North-to-South line displays a linear adjustment to the Romance word order patterns not only with respect to the systematic violation of the V2 linear restriction but also with respect to the typological change from the OV towards the VO type. The gradual change from the German OV type, which is maintained in the dialects spoken in South Tyrol, toward the VO type is well attested along the flow of the river Adige, as already noted, among others in \citet{TomaselliPanzeri2015}.

Let us consider the following examples:

\ea%25
    \label{ex:madaro:25}
German (standard)\\
\gll Heute  \textbf{hat}  Hans  das Holz  im     Wald gehackt.\\
     Today  has   John   \textsc{def.n.acc} wood  in.the   forest  chopped\\
\glt ‘Today John has chopped some woods in the forest.’
\ex%26
    \label{ex:madaro:26}
South Tyrolean (Meran variety)\\
\gll Haint  \textbf{hot}  der Hons  es holz  in wold   khockt.\\
     Today  haz  \textsc{def.m.nom} John  \textsc{def.n.acc} wood  in forest  chopped\\
\glt ‘Today John has chopped some woods in the forest.’

\ex%27
    \label{ex:madaro:27}
 Mòcheno\footnote{Data from \citet{Cognola2013Syntactic} who underlines the prevalence of the VO type (85\%) with respect to the OV type (15\%).}\\
 \gll Haüte  der Hons \textbf{hot}     (a puach)  kaft           (a puach).\\
      Today  \textsc{def.m.nom} John  has    (a book)  bought  (a book)\\
\glt ‘Today John has bought a book.'
\ex%28
    \label{ex:madaro:28}
 Cimbrian (Luserna/Lusern)\footnote{Data presented and discussed in \citet{GrewendorfPoletto2005}.}\\
\ea \gll Haüt  di måmma  \textbf{hatt}  \textbf{gespüalt}  di pjattn.\\
         Today  \textsc{def.f} mother  has  washed  \textsc{def.pl} dishes\\
\glt ‘The mother has washed the dishes today.’
\ex  \gll I  \textbf{hån}  \textit{niamat}  \textbf{getrofft}\\
          \textsc{1sg}  have  noone    met.\\
\ex  \gll I  \textbf{hån}  \textbf{getrofft}  \textit{niamat}\\
          \textsc{1sg}  have  met    noone\\
\glt ‘I haven’t met anyone.’
    \z
    \z

As the previous data show, while Mòcheno still admits the OV pattern with the definite object (\textit{a puach}) -much like Saurian- the Cimbrian varieties have fixed the VO type and admit the OV type in very restricted contexts, i.e. with either a pronominal object or a negative polarity item like \textit{niamat}.

Without entering into further details, it is important to stress that all the German varieties which display a VO typology (i.e.: Cimbrian, Mòcheno, Saurian) do not behave like the Italian “high” VO type but rather like the Scandinavian “low” VO sub-type.\footnote{For the distinction between a Germanic low VO subtype and a Romance higher one cf. \citet{PolettoTomaselli2009, PolettoTomaselli2018, PolettoTomaselli2019}.} The past participle is realized by an invariant form which is never endowed by overt object-agreement morphology. Furthermore, the past participle always follows sentence negation (\textit{nèt}) and low adverbs like \textit{hèrta} ‘always’:

\ea%29
    \label{ex:madaro:29}
    Cimbrian (Luserna/Lusern)\\
\ea \gll Haüt  di måmma  \textbf{hatt}  nèt    gespüalt di pjattn.\\
         today  \textsc{def.f} mother  has   \textsc{neg}   washed \textsc{def.pl} dishes\\
\glt ‘Today the mother didn’t wash the dishes.’
\ex \gll Haüt  di måmma  \textbf{hatt}  \textit{hèrta}    \textbf{gespüalt} di pjattn.\\
         today  the mother  has   always   washed the dishes\\
\glt ‘Today the mother always washed the dishes.’
\ex \gll Haüt  di pjattn   \textbf{soin} (\textit{hèrta}) \textbf{khent} (\textit{hèrta)} \textbf{gespüalt} vodar måmma.\\
         today  the dishes  are (always)         been  (always) washed by.the mother\\
\glt ‘Today the dishes are always washed by the mother.’
    \z
    \z

Rather on the contrary, the Italian past participle shows object-agreement morphology (\textit{stat-e/i/a/o, lavat-e/i/a/o}), may host an object clitic (\textit{salutat-o=lo}), and may move out of the vP to a high functional head:\footnote{Cf. \citet{Belletti2004} for an analysis of reduced past participle clauses in terms of V to C movement.}

\ea%30
    \label{ex:madaro:30}
    Italian (standard)
\ea \gll Le stoviglie  sono  stat-e   (spesso) lavat-e     (spesso).\\
         The dishes  are   been-\textsc{3pl.f}    (often)  washed-\textsc{3pl.f} (often)\\
    \glt ‘The dishes have often been washed.’
\ex \gll Salutat-o=lo, lei  partì.\\
          Greeted-\textsc{3sg.m}\textsc{=3sg.m.acc}  \textsc{3sg.f.nom}  left\\
    \glt ‘Having greeted him, she left.’
    \z
\z

Hence, the adjustment to the Italian word order pattern which seems to have reached a higher degree of fulfillment in the North-to-South line implies linear convergence rather than a structural one.\footnote{For a theory of language contact in the generative perspective, cf., among all, \citet{Bidese2023} and the literature cited there.}

\subsection{Subject-verb inversion}
\label{sec:madaro:3.2}
As we already pointed out before (cf. \sectref{sec:madaro:2} ex. 6a, b, c and \sectref{sec:madaro:3} ex. 21, 22) subject pronouns invert with the finite verb in all the Germanic varieties attested in the North-East of Italy with the only possible exception of the Cimbrian variety spoken in Giazza/Ljetzen (ex. 23).\footnote{Pronominal
     subject inversion with the finite verb in the declarative clause survives in the Cimbrian variety spoken in Giazza/Ljetzan only with the impersonal subject -\textit{pa} (which corresponds to \textit{man} in German):

     \ea  Cimbrian (Giazza/Ljetzan: \citealt{Scardoni2000}: 150)\
      \ea
        \gll Pitar segase \textbf{haka/me}=\textit{pa} iz gras.\\
            with a.scythe   mow=\textsc{impers}   the grass\\
        \glt ‘With a scythe, you mow the grass’.
      \ex
        \gll  Ote bisan  \textbf{see=}\textit{pa}   in vocs.\\
              in.the meadows  see=\textsc{impers}  the fox\\
        \glt ‘In the meadows, you can see the fox’.
      \ex
        \gll Disar langaz \textbf{mussa}=\textit{pa}\textbf{/muss=ma} ghian tze vingan-au  iz holtz.\\
             this spring    must=\textsc{impers}                               go    to collect=\textsc{part}   the wood\\
      \glt ‘In the spring, you have to collect wood.’ (p.147)\\
      \z
     \z

%      \begin{table}
%      \begin{tabularx}{\textwidth}{llQQQ}
%      \lsptoprule
%      \multicolumn{5}{c}{{ (i) \textbf{Cimbrian} \textbf{(Giazza/Ljetzan:} \textbf{\citealt{Scardoni2000}: 150)}}}\\
%      { a.} & \multicolumn{2}{l}{{ Pitar segase}} & { \textbf{haka/me}=\textit{pa}} & { iz gras}\\
%      & \multicolumn{2}{l}{{ with a scythe}} & { mow=IMPERS} & { the grass}\\
%      \multicolumn{5}{l}{{ ‘With a scythe, you mow the grass’.}}\\
%      { b.} & \multicolumn{2}{l}{{ Ote bisan}} & { \textbf{see=}\textit{pa}} & { in vocs}\\
%      & \multicolumn{2}{l}{{ in the meadows}} & { see=IMPERS} & { the fox}\\
%      \multicolumn{5}{l}{{ ‘In the meadows, you can see the fox’.}}\\
%      \multicolumn{2}{l}{{ c.}} & { Disar langaz} & { \textbf{mussa}=\textit{pa}\textbf{/muss=ma} ghian} & { tze vingan-au                  iz holtz}\\
%      \multicolumn{2}{l}{} & { this spring} & { must=IMPERS go} & { to collect=PART          the wood}\\
%      \multicolumn{5}{l}{{ ‘In the spring, you have to collect wood.’ (p.147)}}\\
%      \lspbottomrule
%      \end{tabularx}
%      \end{table}

     \ea \gll Haute regat’z.\\
              Today rains\textsc{=3sg.n}\\
         \glt ‘Today it’s raining.’
     \ex
         \gll Gheistar ha=s=t   gashnaibat   aljar in tak\\
               Yesterday have=\textsc{3sg=3sg.n}   snowed   all the day\\
         \glt ‘Yesterday, it has been snowing the whole day.’
     \z
} Nominal (i.e. DP) subject inversion “\textit{à la} German” is attested in all the varieties of the West-to-East line even if the degree of acceptability of the pattern [XP … Vf DP\textsubscript{NOM.} \textit{nèt} (O)V(O)] varies from full acceptability in Plodarisch to “less acceptable” or better to say “acceptable under specific condition” in Timavese (cf. ex. 17) and Saurian where it counts as a residual word order pattern (cf. ex. 11, 12).

In the North-to-South line, DP-Subject inversion with Vfin is lost with the only exception of Mòcheno (see 31b). In the Cimbrian varieties, the DP-Subject generally lexicalizes the pre-verbal position. Post-verbal Subjects occur on the right of the whole verbal complex “\textit{à la} Italian”, modulo the intervention of the expletive particle –\textit{da/ta} in the Luserna-variety (see 32d) and the preverbal expletive \textit{iz} in the variety of Giazza/Ljetzan \REF{ex:madaro:33c}. Let us consider the following examples:

\ea%31
    \label{ex:madaro:31}

     Mòcheno (cf. \citealt[2]{Cognola2013Syntactic})
\ea  \gll Gester             der Mario  \textbf{hòt}  kaft    a puach\\
          Yesterday \textsc{def.m.nom} Mario  has  bought  a book\\
\ex  \gll Gester   \textbf{hòt}  der Mario  kaft    a puach\\
          Yesterday  has  \textsc{def.m.nom} Mario  bought  a book\\
\glt ‘Yesterday Mario bought a book’
    \z
    \z

\ea%32
    \label{ex:madaro:32}

 Cimbrian (Lusern variety)
\ea[]{ \gll Haüt  dar nono  \textbf{iz}  khent  atz Lusern\\
         Today  \textsc{def.m.nom} grandpa  is  come  to Lusern\\}
\ex[*]{\gll Haüt  \textbf{iz}  dar nono  khent  atz Lusern\\
          Today  is  \textsc{def.m.nom} grandpa  come  to Lusern\\}
\ex[*]{\gll  Haüt  \textbf{iz}  khent  dar nono  atz Lusern\\
         Today is  come  \textsc{def.m.nom} grandpa  to Lusern\\}
\ex[]{ \gll Haüt  \textbf{iz}=ta  khent  dar nono  atz Lusern\\
         Today  is\textsc{=da}  come  \textsc{def.m.nom} grandpa  to Lusern\\}
\glt ‘Today the grandpa has come to Lusern’
    \z
    \z

\ea%33
    \label{ex:madaro:33}

         Cimbrian (Giazza variety, from \citealt[154]{Scardoni2000})\footnote{The examples for Giazza variety are adapted from \citet{Scardoni2000} and recently tested via fieldwork interviews which are collected in personal archives.}
\ea[]{\label{ex:madaro:33a}  \gll Gheistar     i=me       balt   in vatar    \textbf{ha=ci} gahakat  an vingarn   arbaitjgne.\\
          Yesterday  in=\textsc{def.dat} forest   the father has=\textsc{3sg.refl} cut     a finger   working\\}
\ex[*]{\label{ex:madaro:33b} \gll Gheistar  i=me balt         \textbf{ha=ci}    in vatar gahakat  an vingarn   arbaitjgne.\\
           Yesterday   in=\textsc{def.dat} woods has=\textsc{3sg.refl}   the father cut     a finger     working\\
\glt ‘Yesterday the father cut his finger in the woods while working.’}
\ex[]{\label{ex:madaro:33c}  (from \citealt{Scardoni2000}: 171)\\
\gll Gheistar   iz    \textbf{ist}  kent  tz‘ essan   in Giani.\\
      Yesterday  \textsc{3sg.n.nom} is  come  to eat.\textsc{inf}  \textsc{def.m.nom} Gianni\\
\glt ‘Gianni came to eat yesterday.’}
    \z
    \z

The lack of nominal subject inversion \textit{à la} German in the Cimbrian varieties discussed so far has led to the following hypothesis already discussed in the literature:

\begin{itemize}
\item the Cimbrian variety spoken in Giazza/Ljetzan (the southernmost point from the geographic perspective) has lost structural V2 (cf. \citealt{Tomaselli2004} and \citealt{Bidese2008}) even if it maintains a low verbal bracket which includes both the clitic object and negation, cf. ex. 23);
\item the Cimbrian variety spoken in Luserna has undergone a restructuring of the left periphery: the lower Specifier position of the CP domain, i.e. [Spec, FinP] has been reanalyzed as structural subject position, as the position where Nominative case is assigned. Following \citet[591]{BidesePadovanTomaselli2020} we will assume that Nominative case is assigned by C/Fin (as for German) in a Spec-head configuration similar to Italian (\figref{fig:madaro:5}).
\end{itemize}



\begin{figure}
% \includegraphics[width=\textwidth]{figures/a09MTBCPExpansionreviewed-img005.png}
\begin{subfigure}{.333\textwidth}
\centering
\begin{forest}for tree={nice empty nodes}
[FinP
  []
  [
    [Fin
      [Fin]
      [\textit{-da}, tier=bottom]
    ]
    [TP
      [T, tier=bottom]
    ]
  ]
]
\end{forest}
\caption{Cimbrian}
\end{subfigure}%
\begin{subfigure}{.333\textwidth}
\centering
\begin{forest}for tree={nice empty nodes}
[CP
  []
  [
    [C]
    [TP
      [\textit{pro}]
      [
        [T]
        [vP]
      ]
    ]
  ]
]
\end{forest}
\caption{German}
\end{subfigure}%
\begin{subfigure}{.333\textwidth}
\centering
\begin{forest}for tree={nice empty nodes}
[CP
  []
  [TP
    [T
      [T]
      [\textit{-pro}]
    ]
    [vP]
  ]
]
\end{forest}
\caption{Italian}
\end{subfigure}
\caption{NOM-assignment from \citealt[591]{BidesePadovanTomaselli2020}}
\label{fig:madaro:5}
\end{figure}

This reanalysis of [Spec, FinP] as structural subject position implies relevant consequences for the modality of CP-expansion, in particular for object fronting that always implies pronominal resumption, as we will see in the next section.

\subsection{CP-expansion}
\label{sec:madaro:3.3}
A first common trait of the varieties that form the North-to-South line is the possibility to have even more than two XPs on the left of the finite verb, contrary to what has been observed in detail for Timavese (cf. \sectref{sec:madaro:2.3}). Let us consider the following examples:

\newpage
\ea%35
    \label{ex:madaro:35}
    Mòcheno (cf. \citealt[185]{Cognola2013Syntactic})\\
    \gll   Gester   en de     Maria  der Mario  \textbf{hòt}  a puach (nèt)   kaft.\\
           yesterday   to \textsc{def.f.dat}   Maria \textsc{def.m.nom} Mario   has   a book  (not)   bought\\
    \glt ‘Yesterday Mario bought Maria a book.’
\ex%36
    \label{ex:madaro:36}
   Cimbrian (Lusern)\\
   \gll   Haüt  i=me     balt    dar nono  \textbf{hat}=en (nèt)  gesek.\\
         today  in=\textsc{def.dat}   wood    \textsc{def.m.nom} grandpa has\textsc{=}\textsc{3sg.m.acc}  (not)  seen\\
    \glt ‘Today the grandpa hasn’t seen him in the forest.’
\ex %37
    \label{ex:madaro:37}
    Cimbrian (Giazza/Lusern)\\
    \gll   Haüt  i=me balt  in nono  \textbf{hat}=en (nèt)  gesek.\\
           today  in=\textsc{def.dat} forest  \textsc{def.m.nom} grandpa  has=\textsc{3sg.m.acc} (not)  seen\\
    \z

It is interesting to note that the possibility to have more than two XPs on the left of the finite verb usually implies the preverbal position of the DPSUBJ which counts as third constituent.

A second relevant difference between the Cimbrian varieties and all the other Germanic language islands (i.e.: both Mòcheno and the ones which form the West-to-East line) consists in the clitic resumption which obligatorily applies with object fronting:

\ea%38
    \label{ex:madaro:38}
     Cimbrian (Lusern)\\
     \gll In   vuks     \textbf{hatt}=ar=en        nèt  gesek.\\
          \textsc{def.m.acc} fox  has=\textsc{3sg.m.nom}=\textsc{3sg.m.acc}  not  seen\\
     \glt ‘As for the fox, he hasn’t seen it.’
    \z

Rather on the contrary, the clitic recovery of the DPSubj is always excluded unless the prosodic intonation or the distance between the dislocated Subject and the finite verb hints to a left-dislocation construction, with the subject base-generated in a higher functional projection, i.e. in the higher CP subdomain:

\ea%39
    \label{ex:madaro:39}
 Cimbrian (Lusern)
\ea \label{ex:madaro:39a} \gll In      vuks   dar nono  \textbf{hatt}=en    nèt  gesek.\\
          \textsc{def.n.acc} fox   \textsc{def.m.nom} grandpa  has=\textsc{3sg.m.acc}  not  seen\\
\ex \label{ex:madaro:39b} \gll Dar nono  in   vuks  \textbf{hatt}=en    nèt  gesek.\\
          \textsc{def.m.nom} grandpa \textsc{def.m.acc} fox  has=\textsc{3sg.m.acc}  not  seen\\
\ex \label{ex:madaro:39c} \gll Dar nono, {\textbar}{\textbar}  in      vuks     \textbf{hatt}=ar=en nèt  gesek.\\
          \textsc{def.m.nom} grandpa  {}                \textsc{def.m.acc} fox       has=\textsc{3sg.m.nom}=\textsc{3sg.m.acc}  not  seen\\
\glt ‘As for the fox, the grandpa hasn’t seen it.’
    \z
\z

The analysis of \REF{ex:madaro:39c} as a left-dislocated construction is confirmed by the fact that the DPSubj allows\slash undergoes right-dislocation as well:\footnote{Right dislocation does not coincide with “free subject inversion” with the whole verbal complex. In the first case clitic resumption implies a personal pronoun, in the second case the expletive particle  -\textit{da/ta} is involved as the comparison between the following examples clearly shows (cf. \citealt{BideseTomaselli2018} for further details):
\ea Haüt iz=ta khent dar nono atz Lusern
\ex Haüt izz=ar khent atz Lusern, dar nono
\z}

\ea%40
    \label{ex:madaro:40}
    Cimbrian (Lusern)\\
\gll In   vuks  \textbf{hatt} =ar =en      nèt  gesek  dar nono.\\
  \textsc{def.m.acc} fox  has =\textsc{3sg.m.nom} =\textsc{3sg.m.acc}  not  seen  \textsc{def.m.nom} granpa\\
\z

It is interesting to note that Mòcheno still allows Object\hyp fronting without clitic recovery along with DPSUBJ inversion à la German. In our perspective, the coexistence of the following word order patterns:

\ea%41
    \label{ex:madaro:41}

       Mòcheno
\ea \gll  (Gester)  S   puach  \textbf{hòt}=se  (gester) kaft.\\
          yesterday  \textsc{def.n.acc} book  has=\textsc{3sg.f.nom}   (yesterday)  bought\\
\ex \gll  (Gester)  S puach  \textbf{hòt}  di mama      (gester)  kaft.\\
          yesterday  \textsc{def.n.acc} book  has  \textsc{def.f.nom} mother   (yesterday) bought\\
\glt ‘(Yesterday) the mother/she bought a book.’
    \z
\z

could be taken as an argument in favor of a not-yet-specialized pre-verbal position, i.e. [Spec, FinP] is free to host either the Subject or the Object and has not yet specialized as structural subject position as in Cimbrian (cf. diagram in 34).

In Cimbrian the specialization of [Spec, FinP] as Subject position necessarily implies a further step in the CP-expansion process, namely the projection of TopP as possible landing site for XP fronting. Differently from Timavese, where the feature [Top] collapses with [Frame] in a higher projection than the unspecialized Foc/FinProjection (cf. \sectref{sec:madaro:2.3}, \figref{fig:madaro:4}), Cimbrian has scattered the feature [Top] and presents a fully articulated CP:

\ea%42
    \label{ex:madaro:42}

         [FrameP … [ForceP [TopP [FocusP [FinP DPNOM [Fin° Vf [WP [TP … ]]]]]]]]
    \z

It is important to stress that the expansion of CP which we assume for the Cimbrian varieties does not overlap with the analysis assumed by \citet{Hsu2017} for Old Italian (cf. \sectref{sec:madaro:1}, \figref{fig:madaro:3}) at least under two aspects:

\begin{itemize}
\item the specialization of [Spec, FinP] as structural subject position (for Nominative case assignment) and the maintenance of a Wackernagel position which implies enclisis to the element which lexicalizes Fin° (either the finite verb or the lexical complementizer).
\item the development of a double class of lexical complementizers: a low one which alternates with the finite verb in Fin and a higher one which lexicalizes a functional head in the higher C-subdomain.
\end{itemize}

As for (I), the specialization of [Spec, FinP] as structural subject position correlates with the absence of DP-Subject inversion “à la German” in both the declarative and the interrogative clause. As a matter of fact, the only position in which a DP-Subject may occur without clitic resumption is the position on the left of the finite verb in the root declarative clause. In all other cases clitic resumption (either pronominal or expletive) is necessary to save the structure:

\ea%43
    \label{ex:madaro:43}

\ea \gll  (Haüt)  dar nono   (haüt)    \textbf{hatt}   nèt gesek      in vuks.\\
          (today) \textsc{def.m.nom} grandpa (today) has   not seen   \textsc{def.m.acc} fox\\
\ex \gll Haüt   \textbf{hatt}=ta   nèt  gesek  in vuks  dar nono.\\
         today  has\textsc{=da}   not   seen   \textsc{def.m.acc} fox   \textsc{def.m.nom} grandpa\\
\ex \gll Haüt  \textbf{hatt}=ar  nèt  gesek  in vuks   dar nono\\
         today  has=\textsc{3SG.m}  not  seen \textsc{def.m.acc} fox   \textsc{def.m.nom} grandpa\\
    \z
\ex%44
    \label{ex:madaro:44}


\ea \gll Benn  \textbf{hatt}=ar    gesek  in vuks  dar nono?\\
         when  has=textsc{3sg.m.nom}  seen   \textsc{def.m.acc} fox   \textsc{def.m.nom} grandpa?\\
\ex \gll Dar nono,  benn  \textbf{hatt}=ar             gesek  in vuks?\\
         \textsc{def.m.nom} grandpa  when    has=\textsc{3sg.m.nom}  seen   \textsc{def.m.acc} fox?\\
\ex \gll Benn  \textbf{hatt}=ta  gesek   in vuks   dar nono?\\
         When  has=\textsc{da}  seen   \textsc{def.m.acc} fox  \textsc{def.m.nom} grandpa?\\

\glt ‘As for the grandpa, when has he seen the fox?’
\z
\z

If Subject agreement (i.e.: nominative case assignment) applies in FinP, rightwards agreement need the intervention of a rescue element either expletive (-\textit{ta}) or pronominal (-\textit{ar}) in order to allow nominative case absorption.\footnote{The alternation of -\textit{da/ta} with a non-raised DP-Subject with the pronominal clitic -\textit{ar} in dislocated construction reproduces the difference between expletive and argumental pro in a Null Subject language like Italian.} In a WH-root clause the finite verb cannot enter a double Spec-Head relation within the same domain (it cannot satisfy both the WH-criterion in FocP and Nominative case assignment in FinP), hence Subject raising in [Spec, FinP] is blocked and either left or right dislocation favored.

As for (II), it is important to note that the existence of a double class of lexical complementizers, which was already attested in \citet{Baragiola1907}~– cf. \sectref{sec:madaro:4.2}~--  builds a fully developed system in the Cimbrian variety spoken in Luserna.\footnote{Cf. \citet{BidesePadovanTomaselli2012}, \citet{GrewendorfPoletto2011}, \citet{BideseTomaselli2018}.} The following examples clearly show that the low class of complementizers lexicalizes Fin° in complementary distribution with Vf (cf. \ref{ex:madaro:45b}) while the high one lexicalizes a higher functional head in the higher C-subdomain (i.e.: Force) and requires a V2 word order pattern (cf. \ref{ex:madaro:45c}):

\ea%45
    \label{ex:madaro:45}

\ea \label{ex:madaro:45a}
    \gll In   balt  \textbf{hatt}=ar    nèt  gesek  in   vuks\\
         in forest has=\textsc{3sg.m.nom}  not  seen  \textsc{def.m.acc} fox\\
\ex \label{ex:madaro:45b}
    \gll [az=ar      nèt  \textbf{hatt}  gesek  in vuks\\
         [that=\textsc{3sg.m.nom}  not  has  seen  \textsc{def.m.acc} fox\\
\ex \label{ex:madaro:45c}
    \gll [kè  in balt    \textbf{hatt}=ar    nèt  gesek  in vuks\\
         [that  in forest  has=\textsc{3sg.m.nom}  not  seen  \textsc{def.m.acc} fox\\
\z
\z

As expected, Object left dislocation cannot be embedded under \textit{az} (see ex. \ref{ex:madaro:46b}): the complementizer \textit{az} lexicalizes Fin°, the lowest sub-projection in the C-domain, and there is no possible position for object\hyp fronting on the right of it (i.e. in the lower I/T-domain). On the contrary, it is fully compatible with a subordinate clause introduced by \textit{kè}, as its position is in the higher C-domain (Force°), allowing object-fronting in TopP (see ex. \ref{ex:madaro:46c}):

\ea%46
    \label{ex:madaro:46}
\ea[]{ \label{ex:madaro:46a}
     \gll In vuks  \textbf{hatt} =ar =en            nèt gesek in   balt.\\
          \textsc{def.m.acc} fox  has =\textsc{3sg.m.nom} =\textsc{3sg.m.acc} not seen in forest\\
          }
\ex[*]{ \label{ex:madaro:46b}
     \gll [az  in vuks  hatt =ar =en        nèt gesek.\\
           that  \textsc{def.m.acc} fox  has =\textsc{3sg.m.nom} =\textsc{3sg.m.acc}  not seen\\
           }
\ex[]{ \label{ex:madaro:46c}
    \gll [kè  in vuks hatt =ar =en         nèt gesek in   balt\\
         that  \textsc{def.m.acc} fox has =\textsc{3sg.m.nom} =\textsc{3sg.m.acc}   not seen in forest\\
         }
    \z
    \z

The maintenance of the low class of complementizers is a residual phenomenon in the Cimbrian variety of Giazza, where only the high class of complementizers is now attested, coherently with the hypothesis that this variety has already lost structural V2.

As a matter of fact, the Cimbrian variety spoken in Giazza\slash Ljetzan shows a further degree of development. Our speculation that this variety has lost structural V2 is based on the following observations:  (i) in the root declarative clause the pronominal subject always precedes the finite verb; (ii) enclisis on Vf is restricted to object pronouns; (iii) the expletive pronoun \textit{iz} always shows up on the left of the finite verb (cf. ex. 33c \textit{versus} 32d, \sectref{sec:madaro:3.2}); (iv) \textit{ke} has quite completely replaced \textit{az} (a low class of lexical complementizers is no more attested). In this perspective, it is interesting to note that the interrogative clause represents the only context which still exhibits pronominal subject inversion (displaying the already noted convergence with the pattern attested in the contact Italo-Romance varieties, cf. fn. 6):\footnote{As already discussed in \citet{BideseTomaselli2018, BideseTomaselli2019}, the linear convergence derives from a different feature characterization of the functional head T/INFL: weak in V2 Germanic varieties (where the subject clitic in Wackernagel position is attracted by the dominant head C which is endowed with the relevant phi-features), strong in Romance null subject varieties (where the phi-features characterize the higher layer of an expanded T/INFL domain and need to be lexicalized by either the subject clitic – as in Northern Italian dialects – or by finite verb movement – as in Standard Italian.}

\ea%47
    \label{ex:madaro:47}
\ea \gll  Begnje   \textbf{he}=sa     gashlafat?\\
          when     have=\textsc{3pl}   slept ?\\
\glt ‘When did they sleep?’
\ex (Romance variety of Verona)\\
   \gll  Quando   \textbf{a}=le     dormio?\\
          when     have\textsc{=3pl} slept?\\
\glt ‘When did they sleep?’
    \z
    \z

Rather on the contrary, as expected, DPSubject always requires either left or right dislocation (no agreement relation between the finite verb in Fin and the specifier of the lower functional projection, i.e. [Spec, TP]).

\ea%48
    \label{ex:madaro:48}

\ea \gll (Daine toustarn) begnje   \textbf{he}=sa     gashlafat (daine toustarn)?\\
         your daughters when   have=\textsc{3pl}   slept (your daughters)?\\
\glt ‘When did your daughter sleep?’
\ex  \gll (Le to neode) quando \textbf{a}=le   dormio   (le to neode)?\\
         \textsc{def.pl} your nieces    when      have=\textsc{3pl}  slept     (\textsc{def.pl} your nieces)?\\
\glt ‘When did your nieces sleep?’
\z
\z

The position of Mòcheno among the varieties of the North-to-South line resembles the situation already discussed so far for Saurian and, partly, for Timavese:

\begin{itemize}
\item it hasn’t developed a double series of complemetizers;
\item object fronting does not necessarily imply clitic resumption (cf. 17a);
\item DP-Subject inversion with the finite verb is still attested (cf. ex 12a);
\item expletive \textit{pro} is allowed (exactly as in German) in the same contexts where –\textit{da} is required in Cimbrian
\end{itemize}


\ea \gll Hait   is   [pro] ka Verona   kemmen   der nonno\\
         today  is   [pro] to Verona   come     \textsc{def.m.nom} grandpa.\\
\glt ‘The grandpa has come in Verona today’
\z

\begin{itemize}
\item in WH- root clauses the DP-Subject is always dislocated (on the left of the WH-Phrase or on the right of the whole verbal complex); Furthermore, differently from Timavese (and/as in Saurian), Mòcheno allows more than two constituents on the left of the finite verb.
\end{itemize}

Even if the picture is far from being exhaustive (/needs to be further elaborated), it is sufficient to show that the North-to-South line displays a further degree of CP expansion with respect to what has been noted for the West-to-East line. The loss of Object fronting in the root declarative clause in favor of a left-dislocation construction (with clitic resumption) correlates with the loss of DP-Subject inversion “à la German” (i.e. on the right of Vf) and will possibly involve the loss of the low class of complementizers which competes with the finite verb for the same position in the low CP domain (i.e. Fin), hence to the loss of structural V2.

The table proposed at the end of \sectref{sec:madaro:2} (cf. \tabref{tab:madaro:2}) could now be completed as \tabref{tab:madaro:completed}.


\begin{table}
\fittable{\begin{tabular}{l cc cc cc c}
\lsptoprule
 &          &             & \multicolumn{2}{c}{inversion in} & \multicolumn{2}{c}{inversion in} & \\
 & \multicolumn{2}{c}{V2} & \multicolumn{2}{c}{decl. clause} & \multicolumn{2}{c}{Wh-Qs} & \\\cmidrule(lr){2-3}\cmidrule(lr){4-5}\cmidrule(lr){6-7}
 &          & Struc- & DPSubj      & Subj\textsubscript{pron}      &   DPSubj   & Subj\textsubscript{pron}      & asym-\\
 & {Linear} & tural  & {-Vf}       & {-Vf} & {-Vf} & {-Vf} & metry\\
 \midrule
(German) & + & + & + & + & + & + & +\\
Plodarisch & − & + & + & + & +? & + & +\\
Timavese & − & + & + & + & − & + & +\\
Saurian & − & + & (+) & + & − & + & \#\\
Mòcheno & − & + & (+) & + & − & + & +\\
Cimbrian & \\
\quad Lusern & − & + & − & + & − & + & −\footnote{+ with low complem}\\
\quad Giazza & − & −\footnote{low verbal bracket} & − & − & − & + & −\\
\lspbottomrule
\end{tabular}}
% % % \todo[inline]{suggested transposition of table}
% % % \small
% % % \begin{tabularx}{\textwidth}{QlllllQQ}
% % % \lsptoprule
% % %  & (Ger.) & Plo. & Ti. & Sau. & Mò. & Cim. (L) & Cim. (G) \\
% % % \midrule
% % % {Linear V2} & + & -- & -- & -- & -- & -- & -- \\
% % % \tablevspace
% % % {Structural V2} & + & + & + & + & + & + & { --} (low verbal bracket) \\
% % % \tablevspace
% % % {DPSubj-Vf inversion in decl. clause} & + & + & + & (+) & (+) & -- & -- \\
% % % \tablevspace
% % % {Subj\textsubscript{pron}-Vf inversion in decl. clause} & + & + & + & + & + & + & -- \\
% % % \tablevspace
% % % {DPSubj-Vf inversion in Wh-Qs} & + & +? & -- & -- & -- & -- & -- \\
% % % \tablevspace
% % % {Subj\textsubscript{pron}-Vf inversion in WH-Qs} & + & + & + & + & + & + & + \\
% % % \tablevspace
% % % {asymmetry} & + & + & + & \# & + & { --} + with low complem & -- \\
% % % \lspbottomrule
% % % \end{tabularx}

\caption{settings distribution across W-to-E and N-to-S lines}
\label{tab:madaro:completed}
\label{tab:madaro:3}
\end{table}

Plodarish is characterized by the same values as Standard German with the only exception of the V2 linear restriction.

Timavese has lost DP-Subject inversion in the WH-root clause but still maintains it in the root declarative clause.

Saurian and Mòcheno display a similar situation with a stronger tendency to maintain the DP-Subject in a pre-verbal position (which gives rise to V4 in Mòcheno: XP-YP-DPnom-Vf …)

The Cimbrian varieties display the furthest degree of separation from Standard German. They have lost DP-Subject inversion with the finite verb (which corresponds to a fully articulated CP system) and they have weakened the root\slash embedded word order asymmetry (due to the development of a high class of lexical complementizers involving the possible replacement of the low one).

The Cimbrian variety spoken in Giazza/Ljetan represents the final stage in the process of losing structural V2:  pronominal subject inversion with the finite verb in the declarative clause is no more attested and the word order asymmetry lost due to the dismissing of the low class of lexical complementizers. The verbal bracket is maintained in a lower portion of the sentence structure (i.e. vP) and includes only object pronouns and (possibly) negation.

The gradual expansion of the C-domain which emerges from the synchronic comparison between the varieties of the West-to-East line with the varieties of the North-to-South line finds an interesting confirmation in the diachronic perspective, taking into account the historic evolution of the Cimbrian varieties in the last four centuries.

\section{Cimbrian diachronic evolution}
\label{sec:madaro:4}
For most of the German(ic) dialects spoken in Northern Italy, there are no written records before the early 19th century, making it challenging to reconstruct earlier diachronic phases of their syntax. One exception is Cimbrian, which has been documented since the early 17th century. Based on this historical evidence, the loss of V2 in the history of the Cimbrian syntax can be categorized into four phases:\footnote{For a more detailed discussion about the diachronic development see \citet{BideseTomaselli2007}, \citet{Bidese2008, Bidese2023}.}

\begin{description}
  \item [Stage I   {\normalfont(Cimbrian of the 17th century)}:] the Germanic linear V2 word order is broken in favor of a V3 word order pattern, especially in WH main clauses (XP WH Vf …), where the dislocated XP usually corresponds to a hanging topic.
  \item [Stage IIa {\normalfont(Cimbrian of the 19\textsuperscript{th} century)}:] first argument in favor of a TopP.
  \item [Stage IIb {\normalfont(Cimbrian of the early 20th century)}:] DPSubject inversion in the root declarative clause is lost, Object fronting requires clitic resumption, first attestation of a high class of lexical complementizers.
  \item [Stage III {\normalfont(present-day Cimbrian of Lusern)}:] the paradigm of lexical complementizers is characterized by a systematic articulation into a low versus a high class of complementizers; first signs of the dismantling of the root-embedded word order asymmetry.
  \item [Stage IV  {\normalfont(Cimbrian of Giazza)}:] loss of structural V2, the verbal bracket is maintained in a lower portion of sentence structure.
\end{description}

Stages III and IV have been already illustrated and discussed in the preceding section (cf. in particular \sectref{sec:madaro:3.3}), stages I and II will be analyzed in \sectref{sec:madaro:4.1} and Sections~\ref{sec:madaro:4.2.1}--\ref{sec:madaro:4.2.2}, respectively.

\subsection{The first stage: 17th century}
\label{sec:madaro:4.1}
The oldest existing Cimbrian record is a religious work: the Cimbrian Catechism of 1602, which belongs to the varieties of the so-called Seven Municipalities in the North of Vicenza. This is the Cimbrian translation of the Catechism of the Catholic Church composed in 1598 by Cardinal Roberto Bellarmino; it was intended to oppose the spread of Protestantism south of the Alps. In the preface of the text, Bishop Marco Corner of Padova, who commissioned Bellarmino's work to be translated into Cimbrian, stated the following purpose.

\begin{quote}
We also pray through the Heart of Jesus Christ all lay people who are accustomed to teaching \textit{to abandon the other doctrines and turn to the one we are presenting with the same zeal and love} (\citealt{Meid1985Erste}: r. 81--84, italics added: EB, RM, AT).
\end{quote}

The concerns regarding the orthodoxy of the true faith are evident. In the preface, Marco Corner states what follows:

\begin{quote}
we have resolved to have this doctrine translated \textit{word for word} by pious persons, who are understanding in their native German language (\citealt{Meid1985Erste}: r. 33--35, italics added: EB, RM, AT).
\end{quote}

At the end of the translation, the orthodoxy examiner, Niccolò Molini, provides the following remarks:

\begin{quote}
I […] reviewed the aforementioned translation of the Christian Doctrine from Italian into the corrupted variety of the German language used in the Seven Parishes of the Diocese of Padua, and I determined that \textit{it has been translated verbatim} (\citealt{Meid1985Erste}: r. 1436--1447, italics added: EB, RM, AT).
\end{quote}

The consequence of the literal translation from Italian into Cimbrian can be detected at the beginning of the Catechism text. Let us consider the first sentence of the Italian text:

\ea%49
    \label{ex:madaro:49} \citep[48]{Meid1985Erste}\\
\gll [Santamente]   \textit{Nostro} \textit{Signor} \textit{Papa} \textit{Clemente} \textit{Ottavo}, [e con somma prudenza]   \textbf{hà} fatto   stampare   un nuovo   Libretto   della     Dottrina Christiana.\\
     holily      Our    Lord     Pope  Clement  Eighth, and with great wisdom   has let   print     a new     book    of.the    Doctrine Christian\\
\glt ‘With a holy intention and great wisdom, our Lord Pope Clement the VIII. let print a new book of the Christian doctrine.’
    \z

The sentence's subject is the DP ‘\textit{Nostro Signor Papa Clemente Ottavo}’ (Our Lord Pope Clement VIII) followed by the adverbial adjunct ‘\textit{e con somma prudenza}’ (and with great wisdom). The adverb ‘\textit{santamente}’ (holily), which appears at the beginning of the sentence, is actually connected with the prepositional phrase ‘[santamente] \textit{e con somma prudenza}’ and has been moved to the beginning of the sentence to emphasize it, the subject resulting in a sandwich construction between the two conjuncts of the adverbial phrase. The Cimbrian translation repeats the original text word for word (cf. 50):

\ea%50
    \label{ex:madaro:50}
    (\citealt{Meid1985Erste}: r. 56--58)\\
\gll [Hailiklik]  \textit{unser} \textit{Herre} \textit{Babst} \textit{Clemente} \textit{VIII.} [unt mit gròser    prudentia] \textbf{hat} ghemàkt stamparn   an liberle   vòn   der   Christlike Dottrìna neugez.\\
     holily    our Lord Pope Clement Eighth and with great wisdom        has   let   print     a booklet    of  the  Christian Doctrine new\\
    \z

In most cases, the subject-DP is realized in front of the finite verb without any other constituent; however, in a few sentences (7.54\%) the subject is preceded by another constituent, usually an adverb, as in the following examples:

\ea%51
    \label{ex:madaro:51}
    (\citealt{Meid1985Erste}: r. 496--497)\\
\gll [un also] \textit{der} \textit{fine} \textit{von} \textit{allen} \textit{Ghepott} \textbf{ist}  dez ghepott der Liebe.\\
     and so    \textsc{def.m.nom} end of all commandments   is \textsc{def.n.nom} commandment    of love\\
\glt ‘And so, the end of all commandments is the commandment of love.’
\ex%52
    \label{ex:madaro:52}
    (\citealt{Meid1985Erste}: r. 487--488)\\
\gll [unt dernak] \textit{unzer} \textit{Herre} \textit{Christo} \textbf{hat} si confirmart in der nèugien.\\
     and then   our   Lord   Christ  has it confirmed  in \textsc{def.f.dat} new.\textsc{dat}\\
\glt ‘And then our Lord Christ confirmed it in the new law.’
    \z

In the text of the Catechism, there is only one sentence in which the subject-DP is inverted with the finite verb (cf. 53a). It is in the lyrical part of the text. The syntactic difference with the Italian original, which shows the inversion of the subject-DP with the whole verbal complex, was a possible option in the Cimbrian syntax at the time of the Catechism, although only a residual one:

\newpage
\ea%53
    \label{ex:madaro:53}
\ea (\citealt{Meid1985Erste}: rr. 1272--1273)\\
 \gll [Unt hia]   \textbf{saint}   \textit{iere}   \textit{paineghe}     alle   ghegoltet.\\
     and there     are               their  torments   all  rewarded\\
 \glt ‘And there all their torments are rewarded.’


\ex \citep[126]{Meid1985Erste}\\
  \gll Et   ivi   \textbf{sono}   premiati   \textit{i}   \textit{lor}   \textit{tormenti}.\\
        and there  are  rewarded  \textsc{def.pl}   their   torments\\
    \z
    \z

Not surprisingly, the pronominal subjects regularly invert with the finite verb, as in the following examples (54a, 55a, 56a) that we compare with the original sentences in Italian (54b, 55b, 56b) where the subject is not overtly realized. This suggests that verb-subject inversion à la German was clearly not conditioned by the Italian model:

\ea%54
    \label{ex:madaro:54}
\ea (\citealt{Meid1985Erste}: r. 247)\\
\gll [unt daròme] \textbf{saint}=\textit{sa}   zbelf Artichèln\\
     and hence      are=\textsc{3pl.nom}  twelve articles\\
\glt ‘And hence, there are twelve articles.’

\ex \citep[60]{Meid1985Erste}\\
\gll e   però  \textbf{sono}     dodici articoli\\
    and hence  are.\textsc{3pl}  twelve   articles\\
    \z
\ex%55
    \label{ex:madaro:55}
\ea (\citealt{Meid1985Erste}: r. 702)\\
\gll [unt kan deme]  \textbf{indrizzat}=\textit{ar}  alle dink\\
   and towards it  directs=\textsc{3sg.m.nom}  all things\\
\glt ‘And towards it he directs all things.’

\ex \citep[86]{Meid1985Erste}\\
    \gll ed   à quello   indrizza  ogni cosa.\\
         and towards it  directs      everything\\
    \z
\ex%56
    \label{ex:madaro:56}
\ea (\citealt{Meid1985Erste}: r. 1302)\\\label{ex:madaro:56a}
\gll [Dez hantik] \textbf{bil}  \textit{ik}  trinken.\\
     the vinegar  want   \textsc{1sg.nom}  drink\\
\glt ‘I desire to drink the vinegar.’


\ex \citep[128]{Meid1985Erste}\\\label{ex:madaro:56b}
\gll Vò     ber    l’ aceto\\
     want.\textsc{1sg}  drink    the vinegar\\
    \z
\z

For our discussion, it is also interesting to note that the topicalization of the object in \REF{ex:madaro:56a} does not require a resumptive clitic pronoun, as it will be typical for the later stages of the Cimbrian syntax (cf. also \sectref{sec:madaro:3.3} above).

It is also possible to find more than one constituent in front of the finite verb which regularly attracts the subject pronoun on its right (either full or enclitic), as in the following examples (see the position of \textit{bier/ber}):

\ea%57
    \label{ex:madaro:57}
    (\citealt{Meid1985Erste}: r. 56--58)\\
\gll [Also palle, das unz      ist zoakemt   dazselbe liberle]      [mit reverentia, unt fròbde groazza] \textbf{habe} \textit{bier}=z   enffanghet.\\
      As   soon  as \textsc{1pl.obl}  is  come        this          booklet with reverence  and joy      great       have  \textsc{1pl.nom}=\textsc{3sg.n}  received\\
\glt ‘When this little book came to us, we received it with reverence and great joy.’
\ex%58
    \label{ex:madaro:58}
    (\citealt{Meid1985Erste}: r. 78--79)\\
\gll [Die selben medemen] [mit alem herzzen] \textbf{essortar}=\textit{ber} zo pensaran  offte.\\
   \textsc{def.pl.nom}   same ones     with whole heart     urge=\textsc{1pl.nom} to think  often\\
\glt ‘We urge the same ones wholeheartedly to think often.’
    \z

Another syntactic feature of the Catechism concerns the syntax of dislocation, i.e. the constructions of hanging topic and left dislocation (see \citealt{Bidese2008}). As for the former, we find a prototypical example of a hanging topic in the following declarative sentence (cf. 59):

\ea%59
    \label{ex:madaro:59}(\citealt{Meid1985Erste}: r. 380--381)\\
    \gll [\textit{Quanto} \textit{von} \textit{der} \textit{ubel}] iz  \textbf{vorset} sik,   daz …\\
        As for \textsc{def.f.dat} evil   \textsc{3sg.n}  asks  \textsc{refl}  that\\
    \glt ‘As for evil, it asks that ...’
\z

Furthermore, several direct interrogative clauses with a WH element are preceded by a dislocated XP, usually the DP subject, but in some cases also the object, which is resumed in the sentence by a resumptive pronoun. In all examples, the Cimbrian translation follows the original Italian version (cf. X-Y):

\ea%60
    \label{ex:madaro:60}
    (\citealt{Meid1985Erste}: r. 654)\\
\gll [\textit{De} \textit{Fede}]   nun,  \textbf{bia}   stet=se   zua Gott?\\
     \textsc{def.f.nom} faith now,   how   applies=\textsc{3sg.f.nom}   to God?\\
\glt ‘How is faith also related to God?’

\newpage
\ex%61
    \label{ex:madaro:61}
    (\citealt{Meid1985Erste}: r. 733)\\
\gll [\textit{Disa}] \textbf{bia}   kimet   se   unz   abeghereschet?\\
     \textsc{dem.f.nom},   how   comes \textsc{3sg.f.nom}   \textsc{1sg.obl}   abraded\\
\glt ‘How do we get this [= the original sin] erased from us?’
\ex%62
    \label{ex:madaro:62}
    (\citealt{Meid1985Erste}: r. 317)\\
\gll [\textit{Disa} \textit{Kirka}], \textbf{barume} cheu=sik     Hailega, unt Catholica?\\
      \textsc{dem.f.nom} church  why      calls=\textsc{refl} holy      and catholic\\
\glt ‘Why is this church called holy and catholic?’
    \z

Although hanging topic constructions are also possible in German, the thematization of the DP subject is considered marked in German, as already observed in \citet{BideseTomaselli2007}. In contrast, the thematization of the nominal subject in Italian is considered unmarked since it reproduces the SVO order of the language. Moreover, subject inversion in Italian regularly applies with the whole verbal complex. This means that subject fronting is supported by the syntax of the DPSubject in Italian, which usually requires a dislocated position on either the left or the right. Furthermore, thematized demonstrative pronouns (as exemplified by 61 and 62) are particularly difficult to find in German.

Nevertheless, if we take into consideration the German dislocation pattern (\textit{Linksversetzung}) which is still attested in the Cimbrian text, it is easy to recognize a coherent Germanic V2 structure: it shows a D-pronoun as resumptive element on the left of the finite verb, which agrees in case, number and gender with the initial segment (cf. 63--65):\footnote{A kind of topicalization which is defined as \textit{Contrastive Left-Dislocation} in \citealt{Anagnostopoulou1997}.}

\ea%63
    \label{ex:madaro:63}
    (\citealt{Meid1985Erste}: r. 75)\\
\gll [\textit{die} \textit{andere} \textit{sibna}]\textsuperscript{i} [\textbf{die}]\textsuperscript{I} lernt  unz zo tunan bol in prossimen.\\
      \textsc{def.pl.nom} other seven, \textsc{3pl.nom} teach  \textsc{1pl.obl}  to  do good  \textsc{def.pl.dat}  neighbour\\
\glt ‘The other seven rules teach us to do good to our neighbours’.
\ex%64
    \label{ex:madaro:64}
    \gll [\textit{Dain} \textit{bil}]\textsuperscript{i} [\textbf{der}]\textsuperscript{i} ghesceghe also   bia im Himele.\\
          your will,    \textsc{3sg.m.nom}     be.done     as   how in heaven\\
    \glt ‘Your will be done as in heaven.’

\newpage
\ex%65
    \label{ex:madaro:65}
    (Kat.1602: r. 213)\\
    \gll Baròm [\textit{das} \textit{bort} \textit{(in} \textit{namen)}]\textsuperscript{i} [\textbf{das}]\textsuperscript{i} deutet die onikot.\\
         because \textsc{def.n.nom} word in name,      \textsc{3sg.n.nom}    means \textsc{def.f.nom} unity\\
\glt ‘Because the word “in name” means the unity.’
    \z

This fact is properly confirmed in the following example, in which an instance of right dislocation is realized, again with a demonstrative pronoun as resumptive element:

\ea%66
    \label{ex:madaro:66}
    (Kat.1602: r. 371)\\
    \gll [\textbf{der}]\textsuperscript{i}  hat=z   ghemachet  [\textit{Christo} \textit{unzer} \textit{Here}]\textsuperscript{i}.\\
          \textsc{3sg.m.nom}  has=\textsc{3sg.n}  done    Christ our Lord\\
    \glt ‘Christ our Lord has done it.’
    \z

Summing up, the syntax of the First Catechism reproduces a situation that is similar to what we observed for Timavese, even if more conservative, with some peculiar aspects of German syntax, i.e.: a) subject-finite verb inversion, with both nominal and pronominal subject; b) topicalization of the DP-object without resumption c)  resumption of the dislocated element by a demonstrative pronoun on the left of the finite verb. On the other hand: i) there is already a first break in the linear order, in that the finite verb can be preceded by more than one constituent and ii) the inversion of the nominal subject is to be considered as residual (the less frequent option). Furthermore, in this diachronic stage of Cimbrian there is no trace of a double series of complementizers and the particle \textit{{}-da} has only a lexical value (in contrast with \textit{der} in ex. 66 above), as in the following example:

\ea%67
\label{ex:madaro:67} (Kat.1602: r. 577--578)\\
\gll \textbf{da}  saint  drei rette  ghebet von unzer Heren Giesu Christ\\
     there  are  three counsels given by our Lord Jesus Christ\\
\glt ‘There are three counsels that have been given by our Lord Jesus Christ’.
\z

The C-System of \REF{ex:madaro:67} can be then schematized in the following way, much similar to the one represented in \figref{fig:madaro:4a}:

\ea%68
\label{ex:madaro:68}\relax
  [\textsubscript{Frame} \textbf{\textit{XP}} [\textsubscript{CP} unspecialized \textbf{Spec}  [\textsubscript{C} \textbf{Vf}/Complementizer [\textsubscript{TP}…]]]]
\z

\subsection{The second stage: 19th and early 20th centuries}
\label{sec:madaro:4.2}
The second phase in the evolution of Cimbrian is evidenced by two written documents, which were published at the start of the 19th and 20th centuries respectively. The first record is a catechism from 1813 (refer to \citealt{Meid1985Zweite} for the edited text), and the second is an account of the building of a bridge in Roana village in the Seven Communities (refer to \citealt{Baragiola1907}).

\subsubsection{The second catechism}
\label{sec:madaro:4.2.1}
The initial significant change displayed by both texts is the complete disappearance of DP-subject inversion, which was already residual in the first Catechism (see above). Consequently, subject inversion is now definitely restricted to the pronominal subject, which appears enclitic to the finite verb (see 69--70):

\ea%69
    \label{ex:madaro:69}
    (Kat.1813: r. 65)\\
    \gll [Nia]   \textbf{bel}=\textit{bar} rìven zo vorkünnen eür linnez herze.\\
         never   want-\textsc{1pl.nom}  give.up   to praise  your tender heart\\
    \glt ‘We never want to give up praising your tender heart.’
\ex%70
    \label{ex:madaro:70}
    (Kat.1813: r. 349--350)\\
    \gll [von da] \textbf{hat}=\textit{ar} zo kemman zo judicáran.\\
          From there has=\textsc{3sg.m.nom} to come   to judge\\
    \glt ‘From there, he will come to judge.’
\z

In the interrogative context, the Subject-DP is always doubled by a subject pronoun enclitisized onto the finite verb (cf. 71 and 72) independently of its position:

\ea%71
    \label{ex:madaro:71}
    (Kat.1813: r. 137)\\
    \gll Unbrúmme   saint=\textit{sa}\textsuperscript{i} [\textit{dise} \textit{drai} \textit{persúun}]\textsuperscript{i} an      Gott anlóan?\\
         why     are=\textsc{3pl.nom} \textsc{dem.pl.nom} three persons one God alone?\\
    \glt ‘Why are these three persons one God?’
\ex%72
    \label{ex:madaro:72}
    (Kat.1813: r. 238)\\
    \gll Bibel   sáint=\textit{sa}\textsuperscript{i}   [{\textit{de} \textit{Comandaménten}} \textit{von} \textit{der} \textit{Kerchen}]\textsuperscript{i}?\\
         how.many are=\textsc{3pl.nom}  commandments of \textsc{def.f.dat} church\\
    \glt ‘How many commandments are there in the Church?’
\z

As for the syntax of dislocation, the 1813 Catechism shows an increase in dislocated structures. However, they are all in interrogative contexts and with a clitic subject as resumptive pronoun. This suggests that the typical Germanic construction of the left dislocation with a demonstrative pronoun as in the precedent Catechism is no longer part of the Cimbrian syntax.

With respect to the construction of the hanging topic, we find many examples like in the first Catechism, although only in direct interrogative structures, where the V3 linear sequence in WH main clauses always results from the occurrence of the subject-DP fronted to the left of the nexus WH-Vf (cf. X):

\ea%73
    \label{ex:madaro:73}
    (Kat.1813:r. 206)\\
    \gll   [\textit{De} \textit{Kercha}]\textsuperscript{i}  \textbf{ba} \textbf{langhe} hát =\textit{se}\textsuperscript{i} =da zo sainan\\
           \textsc{def.f.nom} church,   how long has =\textsc{3sg.f.nom} =here to be?\\
    \glt ‘How long will the church exist?’
    \z

However, there are also examples of WH interrogatives in which the Subject-DP is shifted to the right of the clause, in contrast to what we find in the Catechism of 1602 and in Standard German:

\ea%74
    \label{ex:madaro:74}
    (Kat.1813: r. 99--100)\\
    \gll \textbf{Brumme} hát =\textit{ar}\textsuperscript{i} =üz gaschàft un galèt af de belt [\textit{Gott} \textit{dar} \textit{Herre}]\textsuperscript{i}?\\
          why          has =\textsc{3sg.m.nom} =\textsc{1pl.obl} created and placed on \textsc{def.f} world, God \textsc{def.m.nom} Lord\\
    \glt ‘Why did God the Lord create and place us in the world?’
    \z

In general, a clear tendency to a more Romance type of hanging topic left dislocation/topicalization can be observed. Prepositional phrases, which either belong to the core of the sentence (cf. 75) or are used to connect the sentences to each other, as in the case of \REF{ex:madaro:76}, can be dislocated in front of the sentence without any resumption:

\ea%75
    (Kat.1813: r. 194)\\
    \label{ex:madaro:75}
    \gll [\textit{Un} \textit{nach} \textit{den} \textit{viarzk} \textit{taghen}] [\textbf{baz}] hat=ar gatànt?\\
          and after \textsc{def.pl.dat} forty days, what has-\textsc{3sg.m.nom} made?\\
    \glt ‘What did he make after forty days?’
\ex%76
    \label{ex:madaro:76}
    (Kat.1813: r. 132--133)\\
    \gll [\textit{In} \textit{minschen} \textit{boart}], [\textbf{baz}]   hat=sich   zo tünan zo volghen  allen disen Comandaménten?\\
    in {a few} words,    what  has=\textsc{refl}  to   do         to   follow all \textsc{dem.pl} precepts\\
    \glt ‘What is to do, in a few words, for following all these precepts?’
\z

If the PP in \REF{ex:madaro:76} can be actually interpreted as a FrameP, the one in \REF{ex:madaro:75} is clearly part of the sentence and must be understood as a Topic. In fact, it repeats what has been said in the sentence before (Kat.1813: r. 192--193), namely “Ear hatsich gamàcht seghen vor viarzk taghe sain schularn” (He let his disciples see him for forty days). In this respect, also the direct Yes/No questions show the same tendency to fronting the subject-DP and/or other lexical material (cf. 77--79):

\ea%77
    \label{ex:madaro:77}
    (Kat.1813: r. 114)\\
    \gll [\textit{Gott} \textit{der} \textit{Herre}]\textsuperscript{i}  \textbf{síghet}=\textit{ar}\textsuperscript{i} nun allez?\\
          God \textsc{def.m.nom} Lord,   sees=\textsc{3sg.m.nom} now  all?\\
    \glt ‘Does God, the Lord, know everything?’
\ex%78
    \label{ex:madaro:78}(Kat.1813: r. 157--158)\\
    \gll [\textit{Dar} \textit{Sun} \textit{von} \textit{Gotte} \textit{me} \textit{Herr-en}]\textsuperscript{i}   [\textit{máchent-en=sich}   {man}]   \textbf{hat}=ar\textsuperscript{i} galàzt zo sáinan   Gott?\\
    \textsc{def.m.nom} son of God \textsc{def.m.dat} Lord-\textsc{dat}     making-\textsc{dat}=\textsc{refl}     man has=\textsc{3sg.m.nom} given.up to be   God\\
    \glt ‘Did the son of God, the Lord, becoming man, give up being God?’
\ex%79
    \label{ex:madaro:79}(Kat.1813: r. 215)\\
    \gll [\textit{Zo} \textit{ghenan} \textit{in} \textit{Hümmel}] \textbf{tüü}=z        sain gatófet, un cloben?\\
          to go          in heaven     does=\textsc{3sg.n.nom}  be baptized and believe\\
    \glt ‘Is it enough to be baptized and believe to go to heaven?’
\z

To summarize: A further degree of expansion within the CP system finds a robust argument in the simple fact that the translations of the Cimbrian sentences with a dislocated element into standard German is no longer possible. We hypothesize that they imply a different structure with respect to the first Catechism. The relevant difference in comparison with the FrameP of the 1602 Catechism is the scattering of the feature [Top] above the projection which hosts the WH element, possibly as head of a TopP. It is still unclear at this point of development whether the WH element occurs in a dedicated FocP, different from FinP, or still in an unspecialized Spec on the left of the finite verb. The evidence is only indirect since DP-subject inversion is no more attested. In summary, the C system seems to have been split into a number of projections with at least a topic position inside, similar to \figref{fig:madaro:4b} (page~\pageref{fig:madaro:4b}):

\ea\label{ex:madaro:80}
  [\textsubscript{Frame} XP [\textsubscript{Topic} \textbf{\textit{Topic}} [\textsubscript{CP} unspecialized Spec [\textsubscript{C} Vf/Complementizer [TP…]]]]
\z

\subsubsection{\citet{Baragiola1907}}
\label{sec:madaro:4.2.2}
The text that we consider relevant for the assumption of a TopP in the C-domain is the story \textit{Dez dink vo' der prucka} (The affair of the bridge), written by Domenico Zotti from Camporovere in the Seven Communities and published by \citet{Baragiola1907}. This text completes the lines of development already observed in the 1813 Catechism. SpecFinP is now clearly specialized as (structural) DP-subject position (cf. 81, 82, 83), distinct from both FocP and a higher TopP. The assumption of TopP higher than FocP is since topicalized objects are now obligatorily resumed by a clitic pronoun on the right of the finite verb (cf. 84):

\ea%81
    \label{ex:madaro:81}
    \citep[8]{Baragiola1907}\\
    \gll [Un in doi Zait] [\textit{dear} \textit{erste} \textit{Deputato}] hat kött   ’me   Loite.\\
          and in that time \textsc{def.m.nom} first deputy has said  \textsc{def.n.dat} people\\
    \glt ‘And at that time, the first deputy said to the people.’
\ex%82
    \label{ex:madaro:82}
    \citep[11]{Baragiola1907}\\
    \gll [Benne dear Consilio hat gat garivet de andarn Materien ’me Komáune zo riven] [\textit{dear} \textit{Presidente}] as=sich gahevet af de Vüze.\\
          when   \textsc{def.m.nom} council has had  finished \textsc{def.pl.acc} other matters \textsc{def.n.dat} township to finish \textsc{def.m.nom} chairman has=\textsc{refl} stood on \textsc{def.pl.acc} feet\\
    \glt ‘When the council concluded the affairs of the township that were to be concluded, the chairman stood up.’
\ex%83
    \label{ex:madaro:83}
    \citep[20]{Baragiola1907}\\
    \gll [Af de noin Oarn]  [\textit{de} \textit{Klocka}]   \textbf{hat}   get   Avviso.\\
         at \textsc{def.pl.acc} nine hours    \textsc{def.f.nom} bell   has  given   alarm\\
   \glt ‘The bell gave alarm at 9 o’ clock.’
\ex%84
    \label{ex:madaro:84}
    \citep[23]{Baragiola1907}\\
    \gll [\textit{miar}]\textsuperscript{i} importar=z=\textit{mar}\textsuperscript{i}   nicht     zo sterben.\\
          \textsc{1sg.dat}  worries=\textsc{3sg.n.nom}=\textsc{1sg.dat}  nothing   to die\\
    \glt ‘I don’t care about dying.’
    \z

When the DP-subject is separated from the finite verb by an adverbial, i.e. temporal clause, the subject is usually resumed by a clitic pronoun at the finite verb, as in the following examples:

\newpage
\ea%85
    \label{ex:madaro:85}
    \citep[9]{Baragiola1907}\\
    \gll [\textit{Di} \textit{Robáner}]\textsuperscript{i},   [hörtenten   prechten   asó garecht disen ersten  Deputaten]   \textbf{haben}=\textit{sa}\textsuperscript{i}     kött   alle.\\
          \textsc{def.pl.nom} people.of.Roana   hearing   speak     so well \textsc{dem.m.acc}   first   deputy   have=\textsc{3pl.nom}    said  all\\
    \glt ‘When the people of Roana heard the first deputy speak so well, they all said.’
\ex%86
    \label{ex:madaro:86}
    \gll [\textit{De} \textit{Loite} ..., ba saint da gabest,]\textsuperscript{i}   [segenten asó], \textbf{haben}=\textit{se}\textsuperscript{i}=sich   ritirart   alle.\\
          the people ..., who were there present,     seeing      this, have=\textsc{3pl.nom}=\textsc{refl}  retired   all\\
    \glt ‘All the people ..., who were present, seeing this, retired.’     \citep[26]{Baragiola1907}
    \z

A final very significant syntactic feature of this text is the first occurrence of the high declarative complementizer \textit{che} `that', borrowed from the same Italian conjunction (cf. x-Y):

\ea%87
    \label{ex:madaro:87}
    \citep[7]{Baragiola1907}\\
    \gll … segenten \textbf{che} de Spesa  ist   zovil groaz.\\
         … seeing   that \textsc{def.f.nom} cost  is  too  great\\
    \glt ‘… when he realized that the costs will be too high.’
\ex%88
    \label{ex:madaro:88}
    \citep[17]{Baragiola1907}\\
    \gll … habent    gahoart \textbf{che}   se      machent   an Consilien.\\
         … have  heard    that  \textsc{3pl.nom}  do    \textsc{indef} council\\
    \glt ‘… heard that they will be holding a council meeting.’
    \z

Usually, the complementizer \textit{che} is introduced by perception verbs like ‘to see’ or ‘to hear’ or, in one case, by the factive verb ‘to make understand’. In the other cases, the autochthonous complementizer \textit{as} ‘that’ is the preferred choice. Nevertheless, an important difference (see \citealt{Bidese2004}) can be observed between the use of \textit{che} and that of \textit{as}: whereas \textit{as} realizes an asymmetric structure with respect to the position of the negation and the reflexive pronoun (cf. 89a and 90a with 89b and 90b), \textit{che} shows only a symmetric one (cf. 91):

\ea%89
    \label{ex:madaro:89}
\ea \citep[23]{Baragiola1907}\\
\gll … ’\textbf{az}  se  \textit{nette}   ghenan vüar.\\
      … that  \textsc{3pl.nom}  \textsc{neg}    go on\\
\glt ‘… that they do not go on.’

\ex  \citep[10]{Baragiola1907}\\
\gll disa Botta  \textbf{zeder}=bar   \textit{net}.\\
     \textsc{dem.f.nom} time  give.in=\textsc{1pl.nom}  \textsc{neg}\\
\glt ‘this time we don’t give in.’
\z

\ex%90
    \label{ex:madaro:90}
\ea \citep[28]{Baragiola1907}\\
\gll … ’\textbf{az}  se  \textit{sich}   legen   in Kiete.\\
    … that  \textsc{3pl.nom}  \textsc{refl}  put  in peace\\
\glt ‘… that they get calmed down.’


\ex \citep[19]{Baragiola1907}\\
\gll noch in de erste Lichte   von deme Tage \textbf{hevan}=se=\textit{sich}   alle.\\
     already at the first light    of \textsc{dem.m.dat} day       get.up=\textsc{3pl.nom}=\textsc{refl} all\\
\glt ‘Already with the first light of day, they all get up.’
\z

\ex%91
    \label{ex:madaro:91}\citep[21]{Baragiola1907}\\
\gll … zo machen segen    \textbf{che}   [dessen von-me Róane]  \textbf{degnarn}=se=\textit{sich}  gnanca segen=sen.\\
…    to let      see    that  \textsc{def.n.gen} of-\textsc{def.n.dat} Róane deign=\textsc{3pl.nom}=\textsc{refl}  even  look=\textsc{3sg.obj}\\
\glt ‘… to show that of the one in Roana, they didn't even deign to look at it.’
\z

In summary: at the beginning of the 20th century, Cimbrian syntax was already characterized by a fully articulated CP domain which maintains structural V2 (mandatory V to Fin in the declarative root clause). [Spec, FinP] has specialized as structural subject position and inversion with the finite verb is limited to the pronominal subject. Object fronting on the left of the finite verb always implies clitic resumption, i.e. the activation of [Spec, TopP] \textit{à la} Italian (but note that the resumptive clitic always occurs enclitic to the finite verb in Fin). Furthermore, the first use of the complementizer \textit{che} can be detected, although still limited to perceptive verbs. It is worth noting that the complementizer \textit{che} lexicalizes a functional head in the higher C-subdomain and hence triggers a symmetrical word order (requires V to Fin movement) exactly as in the Cimbrian variety of Lusern today. \REF{ex:madaro:92} schematizes the CP-system of Cimbrian at the end of the second stage, reproducing the hypothesis put forward for Lusern Cimbrian (cf. above 42):

\ea%92
    \label{ex:madaro:92}

          [\textsubscript{Frame} XP [\textsubscript{Topic} Topic [\textsubscript{Focus} Focus [\textsubscript{FIN} \textbf{Subject-DP} [\textsubscript{FIN} \textbf{Vf}/Complementizer [\textsubscript{TP}…]]]]]

    \z
\section{Final remarks}
\label{sec:madaro:5}
The diachronic development of Cimbrian syntax shows the gradual expansion of the C-domain which ultimately leads to the loss of structural V2. This process which covers more than four centuries could be summarized in the following four stages:

\begin{itemize}
\item[(I)]  [\textbf{Frame} \textbf{XP} [unspecialized Spec [\textsubscript{C} Vf/Complementizer [\textsubscript{TP}…]]]]

\item[(IIA)]  [Frame XP [\textbf{Topic} \textbf{XP} [\textsubscript{CP} (un)specialized Spec [\textsubscript{C} Vf/Complementizer [\textsubscript{TP}…]]]]

\item[(IIB)]  [Frame XP [\textsubscript{ForceP} [\textsubscript{FORCE} high Complementizer) [\textsubscript{TopicP} Topic XP [\textsubscript{FocusP} \textbf{WH-} [\textsubscript{FinP} \textbf{Subject-DP} [\textsubscript{FIN} Vf/low Complementizer [\textsubscript{TP}…]]]]

\item[(III)] Consolidation of phase IIB (and gradual reduction of the low class of complementizers)

\item[(IV)] Loss of structural V2 (no finite Verb movement to Fin in the root declarative clause): no subject inversion (either nominal or pronominal) in the root declarative clause, loss of the low class of complementizers, loss of the root-embedded word order asymmetry.
\end{itemize}

Stage I is attested by the first Cimbrian cathechism (1602) and  roughly corresponds to the first structure assumed for Timavese in \figref{fig:madaro:4a} (\sectref{sec:madaro:2.3}); stage IIA is attested by the second translation of the cathechism (1813) and finds a correspondence with the second structure assumed for Timavese (cf. \figref{fig:madaro:4b}, \sectref{sec:madaro:2.3}); stage IIB is attested by \citet{Baragiola1907} and represents a clear point of innovation with the specialization of [Spec, FinP] as structural subject position and the development of a fully articulated CP which hosts a higher class of lexical complementizers compatible with V to Fin movement; stage III corresponds to the Cimbrian variety spoken in Luserna (cf. \sectref{sec:madaro:3}, in particular \sectref{sec:madaro:3.3}) and shows a consolidation of stage IIB (combined with the first signs of a gradual dismantling of the root-embedded word order asymmetry)\footnote{Cf. \citet{BideseTomaselli2021} for a detailed discussion.}; stage IV corresponds to the Cimbrian variety spoken in Ljetzan\slash Giazza – the southernmost Germanic enclave attested in northern Italy – and already shows the loss of structural V2 (cf. table 5, \sectref{sec:madaro:3.3}).

The early stages of the diachronic evolution of Cimbrian (stages I and IIA) show an interesting overlapping with what has been observed for Timavese from a synchronic point of view and confirms the more conservative aspects of the West-to-East line with respect to the North-to-South line. The limited expansion of the C-domain which correlates with the violation of the linear V2 restriction (a common trait of all the varieties taken into consideration) finds a further development only in the North-to-South line due to the different modalities of topicalization. Since at least \citet{Baragiola1907} topicalization seems to resemble the Romance model of Clitic Left Dislocation: Object fronting implies the activation of TopP and requires clitic resumption (which coherently with structural V2 shows up as enclisis to the finite verb in Fin). This change allows (i.e. goes along with) the specialization of [Spec, FinP] as a structural subject position and opens the way to the borrowing of the Italian high complementizer \textit{ke}. The maintenance of structural V2 is linked to the maintenance of the low class of complementizers, i.e. the one which competes with the finite verb with respect of the same functional head in the low C-subdomain. As a matter of fact, the Cimbrian variety of Giazza\slash Ljetzan has already lost both pronominal subject inversion in the root declarative clause and the root-embedded asymmetry -- i.e.: the lexical complementizers of the low class -- and hence fails the diagnostic tests for the assumption of structural V2.\footnote{As suggested by one of the anonymous reviewers, the question arises about the status of FinP in the Cimbrian variety of Giazza. As a matter of fact, it is never lexically realized, by neither the low lexical complementizer nor the finite verb, as in a V2 German-like system. At the same time, it does not host the infinitival marker as it is assumed for \textit{di} in Italian \citep{Rizzi1997}; on the contrary, the infinitival particle \textit{zo} coherently behaves as its German cognate \textit{zu}, showing adherence to the Germanic model. For a more detailed cross-linguistic analysis regarding the structure of infinitival clauses in V2-varieties, see \citet{TomaselliBideseMadaro2024}.}

The possibility that the varieties of the West-to-East line follow the same path remains for the moment pure speculation and requires further research and data collection.

\section*{Acknowledgments}
This work is the result of the collaboration of the authors in all respects. For the Italian academic system, Romano Madaro takes responsibility for Sections 1 and 2; Alessandra Tomaselli takes responsibility for Sections 3 and 5, whereas Ermenegildo Bidese takes responsibility for \sectref{sec:madaro:4}. We are strongly in debt with Pierre-Yves Modicom and Sarah Harchaoui for the organization of the very fruitful Workshop “Cracks in the Bottleneck” and for their editorial work. Special thanks are due to the anonymous reviewers whose suggestions significantly contributed to improving our paper.

\sloppy\printbibliography[heading=subbibliography,notkeyword=this]
\end{document}
