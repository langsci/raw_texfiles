\documentclass[output=paper,colorlinks,citecolor=brown]{langscibook}
\ChapterDOI{10.5281/zenodo.17077192}
\author{Christine Meklenborg Nilsen\orcid{0000-0003-2389-6308}\affiliation{Universitetet i Oslo}}
\title{Tracing the origins of resumption in Swedish}
\abstract{This paper examines the changes in the resumption system that take place in Swedish from the 13th to the 18th centuries. The basis of the analysis is a data set of 6,052 clauses drawn from different kinds of texts. The analysis shows that while \textsc{då} was the dominating resumptive in the earliest texts, \textsc{så} takes over as the generalized resumptive in the 16th century. The paper argues that \textsc{då} resumption undergoes semantic restriction while \textsc{så} resumption shows an instance of semantic expansion. This is a reflection of a deeper underlying process, where the antecedent gets fully integrated into the V2 clause.}

\IfFileExists{../localcommands.tex}{
   \addbibresource{../localbibliography.bib}
   % add all extra packages you need to load to this file

\usepackage{tabularx,multicol}
\usepackage{url}
\urlstyle{same}

\usepackage{listings}
\lstset{basicstyle=\ttfamily,tabsize=2,breaklines=true}

\usepackage{langsci-basic}
\usepackage{langsci-optional}
\usepackage{langsci-lgr}
\usepackage{langsci-osl}
% \usepackage{./langsci/styles/langsci-lgr}
% \usepackage{./langsci/styles/langsci-osl}
% \usepackage{langsci-gb4e}

\usepackage{tikz}
\usetikzlibrary{patterns,calc}
\pgfdeclarepatternformonly{south east lines}{\pgfqpoint{-0pt}{-0pt}}{\pgfqpoint{3pt}{3pt}}{\pgfqpoint{3pt}{3pt}}{
    \pgfsetlinewidth{0.6pt}
    \pgfpathmoveto{\pgfqpoint{0pt}{3pt}}
    \pgfpathlineto{\pgfqpoint{3pt}{0pt}}
    \pgfpathmoveto{\pgfqpoint{.2pt}{-.2pt}}
    \pgfpathlineto{\pgfqpoint{-.2pt}{.2pt}}
    \pgfpathmoveto{\pgfqpoint{3.2pt}{2.8pt}}
    \pgfpathlineto{\pgfqpoint{2.8pt}{3.2pt}}
    \pgfusepath{stroke}}
    
\usepackage{stmaryrd}
\usepackage{wasysym}
\usepackage{multirow}
\usepackage{caption}
\usepackage{subcaption}
\usepackage{mathrsfs}
\usepackage{qtree}

\usepackage{linguex}


   %pminos do not split footnotes
% \interfootnotelinepenalty=10000 %Footnote in Laporte chapters has to be split SN


%\DeclareIndexNameFormat{default}{%
%\nameparts{#1}%
%\usebibmacro{index:name}%
%{\index[names]}%
%{\namepartfamily}%
%{\namepartgiveni}%
% {}% L1
% {}% L2
%{\namepartprefix}% generates spurious space L3
%{\namepartsuffix}% generates spurious space L4
%}

%  {\DeclareIndexNameFormat{default}{%
%     \usebibmacro{index:name}{\index[names]}{#1}{#3}{#5}{#7}}}

%\DeclareIndexNameFormat{default}{%
%  \usebibmacro{index:name}{\sindex[nom]}{#1}{#3}{#5}{#7}}

%\DeclareIndexNameFormat{default}{%
%  \usebibmacro{index:name}{\sindex[person]}{#1}{#3}{#5}{#7}}
%\DeclareIndexNameFormat{default}{%
%\nameparts{#1} \usebibmacro{index:name}{\sindex[person]]}{\namepartfamily}{‌​\namepartgiven}{\nam‌​epartprefix}{\namepa‌​rtsuffix}}

%\newcommand{\smiley}{:)}

%\renewbibmacro*{index:name}[5]{%
%\usebibmacro{index:entry}{#1}%
%{\iffieldundef{usera}{}{\thefield{usera}\actualoperator}\mkbibindexname{#2}{#3}{#4}{#5}}}

% \newcommand{\noop}[1]{}

%remove for final
%\overfullrule=1mm

\newcommand{\tobi}[2]}}
\renewcommand{\S}[1]{\tobi{#1}{\textsc{*}}}

% this volume references
% puts: [this volume]
% already defined: \citetv
%\newcommand{\citepv}[1]{(\citeauthor{#1} \citeyear*{#1} [this volume])}
\newcommand{\citealtv}[1]{\citeauthor{#1} \citeyear*{#1} [this volume]}

%parentheses around example number
\newcommand{\pref}[1]{(\ref{#1})}

% in-text examples

\newcommand{\lnex}[1]{\textit{#1}} %target lang word
\newcommand{\lnlit}[1]{(lit.: `#1')} %literal reading
\newcommand{\lnlat}[1]{(#1)} % latinization
\newcommand{\lntrans}[1]{`#1'} %translation
\newcommand{\lnexl}[2]%
{\lnex{#1}{} \lnlat{#2}} % ex with latinization
\newcommand{\lnexlat}[3]{\lnex{#1}{} \lnlat{#2}{} \lntrans{#3}} % ex with latinization and tranl.

%ch01
\newcommand{\co}[1]{\mbox{\textbf{#1}}}

%ch09

\newcommand{\cyrbulg}[1]{\begin{otherlanguage*}{bulgarian}#1\end{otherlanguage*}}


%ch10
\newcommand{\nlp}{{\small NLP}}
\newcommand{\mwe}{{\small MWE}}
\newcommand{\rae}{{\small RAE}}
\newcommand{\lvc}{{\small LVC}}
\newcommand{\pos}{{\small P}o{\small S}}
%\newcommand{\todo}[1]{ \textcolor{red}{#1} }

%\renewcommand{\labelenumi}{\theenumi}
%\ainamefmt{{vv}{ll}{, ff}{, jj}} % fullname

\newcommand{\biberror}[1]{{\color{red}#1}}

\newcommand{\osenovaitem}{--~}
   %% hyphenation points for line breaks
%% Normally, automatic hyphenation in LaTeX is very good
%% If a word is mis-hyphenated, add it to this file
%%
%% add information to TeX file before \begin{document} with:
%% %% hyphenation points for line breaks
%% Normally, automatic hyphenation in LaTeX is very good
%% If a word is mis-hyphenated, add it to this file
%%
%% add information to TeX file before \begin{document} with:
%% %% hyphenation points for line breaks
%% Normally, automatic hyphenation in LaTeX is very good
%% If a word is mis-hyphenated, add it to this file
%%
%% add information to TeX file before \begin{document} with:
%% \include{localhyphenation}
\hyphenation{
    Beck-man
    Ngu-yen
    back-chan-nel
    back-chan-nels
    mo-not-o-nous
    ste-reo-typ-i-cal
}

\hyphenation{
    Beck-man
    Ngu-yen
    back-chan-nel
    back-chan-nels
    mo-not-o-nous
    ste-reo-typ-i-cal
}

\hyphenation{
    Beck-man
    Ngu-yen
    back-chan-nel
    back-chan-nels
    mo-not-o-nous
    ste-reo-typ-i-cal
}

   \boolfalse{bookcompile}
   \togglepaper[23]%%chapternumber
}{}

\begin{document}
\maketitle


\section{Introduction}
Swedish is a Verb-Second language (henceforth V2), which is shorthand for the observation that the finite verb comes in second position in main declarative clauses regardless of what element precedes it. If any other XP than the subject is preverbal, the subject occurs after the finite verb. In \REF{ex:nilsen:1}, the PP \emph{på måndag} `on Monday' precedes the finite verb \emph{ska} `shall', and the subject \emph{vi} `we' is post-verbal.

\begin{exe}

\ex\label{ex:nilsen:1}
\gll På måndag  ska vi äta soppa. \\
on Monday shall we eat soup\\
\glt `On Monday, we will have soup.' \hfill (Swedish)


\end{exe}

This V2 word order is often challenged by the insertion of a so-called resumptive element that triggers a surface Verb-Third (V3) word order. In \REF{ex:nilsen:2} the resumptive \textsc{så} occurs between \emph{på måndag} `on Monday' and the finite verb.

\begin{exe}

\ex\label{ex:nilsen:2}
\gll På måndag  \textbf{så} ska vi äta soppa. \\
on Monday \textsc{så}  shall we eat soup\\
\glt `On Monday, we will have soup.'\\



\end{exe}



These resumptive structures are extremely common, especially in the spoken language \citep{Elmquist, Ekerot, Nordstrom}. However, this construction is by no means an innovation of the Modern Swedish language. In the earliest sources, from the 13th century, there are numerous constructions with a resumptive element \REF{ex:nilsen:3}.


\begin{exe}
\ex\label{ex:nilsen:3} \gll Nw æn man slar . præst . ællir vighdan . \textbf{tha} bøte biscope nyo marker\\
\textsc{nu} a man kills . priest . or consecrated .  \textsc{då} pay bishop.\textsc{dat}  nine marks \\ \label{prest}
\glt `If a man kills a priest or another consecrated person, then he should pay the bishop marks.' \hfill(EOS, ÖGL, 1350--1400)\footnote{References should be read in the following way: EOS refers to Early Old Swedish, ÖGL to Östgötalagen, and the numbers refers to the date of the text in question. The different periods in the Swedish language are described in \tabref{relative-dimen}. For a full list of sources and their abbreviations, see page \pageref{abbr}.}
      
 
\end{exe}




There is, nonetheless, a major difference between resumption in the old and the modern languages. As we can see in examples \REF{ex:nilsen:2} and \REF{ex:nilsen:3}, the resumptive element is not the same. In the modern example in \REF{ex:nilsen:2}, the resumptive is \textsc{så}, while in \REF{ex:nilsen:3} it is \textsc{tha}. Resumption with \textsc{då}, the modern descendent of \textsc{tha} is still possible, but its use is quite restricted. In the example we have already seen, we may, however, substitute \textsc{så} by \textsc{då} \REF{ex:nilsen:4}.

\begin{exe}
\ex\label{ex:nilsen:4}
\gll På måndag  \textbf{då} ska vi äta soppa. \\
on Monday \textsc{då}  shall we eat soup\\
\glt `On Monday, we will have soup.'
\end{exe}


Similarly, resumption with \textsc{så} is attested in the oldest sources, but its use is marginal. We may thus note that there has been a significant diachronic development, in which the use of the two resumptives has changed radically. 


There have been numerous studies on the modern \textsc{så} construction \citep{Elmquist, Ekerot, Nordstrom, Holmberg-saa, Holmberg-2023}, but fewer  have worked on the historical resumptive structure. To my knowledge, the only one addressing \textsc{då} resumption in the old language is \citet{Bergqvist}. In this paper, I will examine the changes that take place in Swedish before 1750 by tracing the evolution of the two resumptives \textsc{tha $\longrightarrow$ då} and \textsc{swa $\longrightarrow$ så}. 
%We will, however, first introduce the resumptive construction in Modern Swedish and examine the difference between \emph{så} and \emph{då}.

\subsection{Clines}\label{rescli}
Central to this paper is the notion of clines. In a diachronic perspective, \citet[115]{Meklenborg-saa} sees the evolution of resumptive structures as some kind of grammaticalization process. She distinguishes between two clines, one for adverbial elements and one for thematic constituents. Both are implicational hierarchies that describe the temporal evolution of the resumptive structure.

Meklenborg suggests that in the earliest stages, any nominal element containing a relative clause may be followed by a resumptive element. In the later stages, the presence of a relative clause is not necessary, and bare thematic DPs may be followed by a resumptive element. The cline will thus be as in \REF{ex:nilsen:5}.

\begin{exe}


\ex\label{ex:nilsen:5} DP\textsubscript{\textsc{rel}}  > DP\textsubscript{\textsc{θ}} \label{thetcline}

\end{exe}

With respect to the adverbial cline, Meklenborg argues that resumption first occurs after fronted adverbial clauses, and that the structure later expands to fronted PPs and AdvPs. With respect to AdvP, first temporal adverbials may serve as antecedents, then general adverbials, and finally so-called \emph{nynégocentrique} adverbials, e.g. speaker-oriented deictic adverbials expressing the `me--here--now'. This gives us the cline in \REF{ex:nilsen:6}.

\begin{exe}
\ex\label{ex:nilsen:6} \label{cline}
Adverbial use: CP\textsubscript{adv}  > PP  > AdvP\textsubscript{temp.} > AdvP\textsubscript{gen.}  > AdvP\textsubscript{nyn.}
\end{exe}


It should be noted that it is only generalized resumptives that may follow generalized or nynegocentric adverbials. Even in the case of the temporal \emph{nu} `now', the temporal resumptive \textsc{då} is not accessible.


\begin{exe}


\ex\label{ex:nilsen:7}

\gll Nu \textbf{så / ?då } säger han något annat.\\
now \textsc{så / då}  says he something else\\
\glt `Now he says something else.'




\end{exe}


However appealing, Meklenborg's adverbial cline is unfortunate in the sense that it lumps together semantic and syntactic properties. I would therefore suggest that the cline be divided in two different clines, that interact. One is a \emph{syntactic cline}, observing the different syntactic properties of the antecedent. The other is the \emph{semantic cline}, where only the semantics of the antecedent are taken into consideration. Thus, Meklenborg's cline corresponds to the two following clines, \REF{ex:nilsen:8} and \REF{ex:nilsen:9}.



\begin{exe}
\ex\label{ex:nilsen:8}  \textsc{Syntactic cline}: CP  > PP  > AdvP \label{syncline}
\ex\label{ex:nilsen:9} \textsc{Semantic cline}: specific semantics > general semantics > complete semantic bleaching (= nynegocentric use) \label{semcline}
\end{exe}




\subsection{Outline}


The paper is organized as follows: \sectref{method} presents the methodological choices that I have made as well as the data set. \sectref{teori} gives an overview of the theoretical backdrop and also provides data about the resumptive construction in Modern Swedish. In \sectref{morph}, I will present the syntactic status of the antecedent, while I will examine its semantics in \sectref{sem}. \sectref{clines} discusses the implications the data have for the notion of clines, while \sectref{structural} suggests a link between the observed changes and underlying structural changes. \sectref{konklusjon} concludes the paper.

\section{Method}\label{method}
This paper is devoted to exploring the origins of the resumptive elements \textsc{då} and \textsc{så} in the history of Swedish. \textsc{Då} is the descendant of the temporal adverb and conjunction \emph{þā}, which evolves into \emph{tha} before it becomes \emph{då} in the early 16th century. \emph{Þā} in its turn goes back to Proto-Germanic \emph{*þan}. As such, \textsc{då} is the sibling of Old High German \emph{tho} and Old English \emph{þa} and \emph{þonne}, while it is the aunt of Modern German \emph{da} and \emph{dann} and Modern English \emph{than}.


\emph{Så} goes back to Proto-Germanic \emph{*swā}, and is essentially an manner adverb denoting comparison (`so', `this way'). The word is the origin of a series of quite similar words in the Modern Germanic languages: \emph{so} (German, English), \emph{så} (Swedish, Norwegian, Danish), \emph{svo} (Icelandic). In Old Swedish, the form was \emph{swa} or \emph{sva}.

The descendents of both \emph{*þan} and \emph{*swō} are prone to appearing in resumptive structures in the Germanic languages (see among numerous others \cite{Links_1, Links_thesis, Links_2, Haeb-Pintz, Thim-Mabrey, Catasso, Donhauser-Petrova, Axel, Axel-Tober2023}).

In the following, I will use the forms \textsc{då} and \textsc{så} regardless of the form used in the examples. 

\subsection{Selecting the data}
In order to establish the corpus, I have searched the Swedish language bank for all historically attested varieties of \textsc{så}  and  \textsc{då}, according to the online dictionary of the Swedish Academy (\emph{Svenska Akademiens ordbok} -- SAOB\footnote{\url{http://g3.spraakdata.gu.se/saob/}}).\footnote{ For  \emph{så}  I searched for the varieties \emph{sva, svo, saa, so, soo, swa, swaa, swå}, and \emph{så}; for \emph{då} I searched for \emph{þa, tha, thaa, thå, tho, taa, tå, dhå, daa, då}, and \emph{dåå}.} Both words create a lot of noise in the material as they may have several different functions; \emph{så} may for instance be a adverb of degree, and \emph{tha} may be the definite determinative. These uses have been eliminated from my corpus. 

 In the database I chose a subset of the available texts in the historical corpus. The documents from Early Old Swedish consist of more than 1,000,000 words, where the subset of legal texts contain the oldest data. I therefore included those at the expense of  religious prose. For Late Old Swedish, I looked at both laws and other records. The legal text from this period is \emph{Kristoffers Landslag} (1442), which contains sections that have been copied from \emph{Magnus Erikssons Landslag} (ca 1350). I have manually deleted the passages that have been copied from the earlier text as these cannot be seen as  representatives of Late Old Swedish. The database also includes the legal texts  \emph{Skånelagen} and \emph{Gutalagen}, which I have removed as they were written in Old Danish and Old Gutnish, respectively.\footnote{For a presentation of the Old Swedish texts, the reader is referred to Fornsvenska Textbanken, \url{https://www.nordlund.lu.se}.}


 For the two periods of Modern Swedish, I chose two texts, both the first text in their respective periods. For Early Modern Swedish, this is Gustav Vasa's  Bible translation (The New Testament) from 1526, a translation of Luther's Bible. For Late Modern Swedish, I chose Dalin's \emph{Then Swenska Argus} (1732--1743), a popular newspaper, which is considered to reflect the spoken language of the time. There is a 200-year window between the two texts, which means that there is a risk that gradual changes in the language will appear abrupt.  Therefore, I have also consulted the database \emph{Ekeblads brev}, which contains letters written by Johan Ekeblad to his brother in the period 1639--1655, but these texts have not been systematically examined and do not appear in the  statistics. Since the two earliest periods contain legal texts, I have also searched the law \emph{Ridderskapets och adelns privilegier}\footnote{\url{https://lagen.nu/1723:1016\_1}} from 1723. The intention was to compare the legal language of the 18th century to the language of other prose texts in order to see if changes we explore could be seen as the reflection of the differences between genres.




In addition to these data taken from the portal \textsc{Korp}, I have used the \emph{Fornsvensk lexikalisk databas}\footnote{\url{https://spraakbanken.gu.se/fsvldb/}}, which is the digitalized version of Söderwall's dictionaries of Medieval Swedish (tomes I--V) and Schlyter's dictionary of Swedish legal texts.\footnote{Söderwall, K.F. \emph{Ordbok Öfver svenska medeltids-språket} Vol I-III. (Lund 1884--1918).\\
Söderwall, K.F. \emph{Ordbok Öfver svenska medeltids-språket} Supplement. Vol IV—V. (Lund 1953--1973) \\
Schlyter, C.J. \emph{Ordbok till Samlingen af Sweriges Gamla Lagar}. (Saml. af Sweriges Gamla Lagar 13.) Lund 1877.} Further, I have used \citet{norse}, \citet{Haugen}, and \citet{Noreen1904} as reference tools. The first two deal with Old Norse, the sibling of Early Old Swedish \citep{Berg-16}, whereas the last one is a detailed presentation of  the old Swedish language. Further, I have used the \textit{Karp} base \citep{Karp}.


\subsection{The data}\label{dataintro}


Using the method described above, I established a corpus  consisting of  6,052 clauses spanning six centuries, from the 13th to the 18th (\tabref{relative-dimen}). As two texts have been removed from the Early Old Swedish period, I have not been able to calculate the relative distribution of resumption in this period. It should however be noted that the dimension of this section before the removal of the two texts is 532,192 words and 29,042 sentences.


\begin{table}
\caption{The relative distribution of resumption}
\label{relative-dimen}
 \begin{tabular}{lrrrrr}
  \lsptoprule
Period					&	Words 		&	Sentences	&	Examples		&	 ex/stc &  ex/wrd\\ 
\midrule
EOS (1225--1375)&	?	  &	?& 4,149	&	?& ?\\
LOS (1375--1525)&	278,604	 & 	11,648& 1,137	&	9.0\%& 0.41\% \\
EMS (1526--1731)&   469,706  &	43,323 & 202 & 0.5\% & 0.04\%\\
LMS (1731--1900)&   260,093  & 10,747 & 564 & 5.3\% & 0.33\%\\
  \lspbottomrule
 \end{tabular}
\end{table}

A first glance at the data reveals  that the resumptive structure is more common in Early Old Swedish, and that its frequency goes down over the next two periods. From Late Modern Swedish, however, the use of the structure increases. In the paper we will investigate  these changes more in detail.


 


\section{Theoretical backdrop}\label{teori}

When discussing resumption, certain definitions need to be in place. In this paper, I will use the term \emph{antecedent} or \emph{fronted constituent} for the initial element, in other words the XP preceding the resumptive. The resumptive, thus, is the element intervening between the antecedent and the finite verb.

\begin{exe}
\ex\label{ex:nilsen:10} Terminology\medskip\\
    \begin{tabular}{@{}cccc@{}}
    på måndag           & så & ska & vi äta soppa\\
    \downarrow          &\downarrow  &\downarrow  &\\
    \textsc{antecedent} & \textsc{resumptive}&\textsc{finite verb}
    \end{tabular}
\end{exe}

In the following I will make a distinction between \emph{generalized} and \emph{specialized} resumptives (\sectref{genspec}) and I will  briefly describe the derivation of V2 clauses (\sectref{V2}).



\subsection{Generalized and specialized resumptives}\label{genspec}

\citet{Meklenborg-saa} makes a distinction between \emph{generalized} and \emph{specialized} resumptives. While a generalized resumptive is semantically bleached, the specialized resumptive has retained its original semantic properties. It follows from this that a  generalized resumptive can be used in all cases, while the specialized resumptive must match the semantics of its antecedent. 

The following examples  illustrate this \REF{ex:nilsen:11}.


\begin{exe}


\ex\label{ex:nilsen:11}
\begin{xlist}


\ex\label{ex:nilsen:11a}
\gll När du har gjort din läxa, \textbf{så} / \textbf{*då }  kan du få se på TV.\\
when you have done your homework  {\textsc{så}} {} {\textsc{då}}  can you get watch on TV\\
\glt `When you have done your homework, you may watch TV.' \label{lekse} \hfill [+\textsc{temporal}]

\ex\label{ex:nilsen:11b}
\gll {Igår} \textbf{så} / \textbf{då } gick vi på teater. \\
yesterday  {\textsc{så}} {} {\textsc{då}} went we to theater\\
\glt `Yesterday, we went to the theater.'  \label{koben} \hfill [+\textsc{temporal}]

\ex\label{ex:nilsen:11c}
\gll I Paris, \textbf{så} / \textbf{*då } mötte vi våra tidigare grannar.\\
I Paris,   {\textsc{så}} {} {\textsc{då}}  met we our former neighbors\\
\glt `In Paris, we met our former neighbors.'  \label{hund}\hfill [+\textsc{locative}]

\ex\label{ex:nilsen:11d}
\gll Lyckligtvis \textbf{så} / \textbf{*då } kommer vi i tid.\\
fortunately  {\textsc{så}} {} {\textsc{då}}  come we in time.\\
\glt `Fortunately, we will be there in time.' \label{bjeffer}\hfill [+\textsc{general}]

    \end{xlist}
    
\end{exe}


In order to use the resumptive \textsc{då}, the antecedent must be $[$+\textsc{temporal}$]$. Thus, it is acceptable after \emph{igår} `yesterday', but is impossible after the locative \emph{i Paris} `in Paris'\footnote{If \emph{in Paris} is interpreted as `when we were in Paris', \textsc{då} would be acceptable.}\footnote{Swedish also has a [+\textsc{locative}] resumptive, \emph{där} `there', which we will not discuss in this paper.} and the sentence adverbial \emph{lyckligtvis} `fortunately'. As is clear from these examples, \textsc{så} in Modern Swedish must be considered a generalized resumptive, while \textsc{då} is specialized. 

The examples in \REF{ex:nilsen:11} illustrate the importance of the semantics of the an\-te\-ced\-ent in the choice of resumptive. The syntax of the antecedent is also subject to variation. In Modern Swedish, the antecedent can be  a CP (\ref{lekse}), a PP (\ref{koben}, \ref{hund}) or an AdvP (\ref{bjeffer}).

\subsection{The V2 property}\label{V2}

From a typological point of view, V2 is  very rare, since it occurs first and foremost in the Germanic languages with the noteworthy exception of Modern English, and possibly Cimbrian\footnote{The status of Modern Cimbrian is debated. See \citet{Bidese2008} and \citet{GrewendorfPoletto2011} for different points of view.}. In addition, V2 is observed in Estonian, Breton, Sorbian, Kashmiri, and Himachali \citep{Holmberg-V2}. Historically, V2 was also attested in the Old Romance languages (see among others \cite{VRB, salvi,  Ben-2006,  Ledgeway-si, Ledgeway-2012, Radwan, Wolfe-2014, Wolfe-thesis, Wolfe-2019, Salvesen-2013-top, Poletto2014, Espen-thesis, Egerland-Falk}).


A V2 language is, as already mentioned, characterized by the fact that the subject may occur pre-verbally and post-verbally, while the finite verb always occupies the second position. I will follow Holmberg's definition, which splits up the V2 phenomenon into two different operations \REF{ex:nilsen:12} \citep[375]{Holmberg-V2}.


\begin{exe}
\ex\label{ex:nilsen:12}  The V2 property: \label{V2prop}

\begin{xlist}
\ex  A functional head in the left-periphery attracts the verbal head;
\ex  This functional head wants a constituent moved to its specifier position.

\end{xlist}
\end{exe}

Further, I will assume that there is a visibility requirement that says that the specifier position must contain overt phonetic material \citep{Mek-2020}.


It follows from this definition that a  resumptive that occurs between XP and V must be located in the left periphery. While this is not possible in a traditional  model, I will use a cartographic framework in which the CP is split into a number of projections following Rizzi's approach \citet{Rizzi-97}. More specifically, my starting point will be the elaborated structure  proposed by \citet{Ben-Pol} \REF{ex:nilsen:13}.

\begin{exe}
    \ex\label{ex:nilsen:13} Hanging Topics ... Scene Setting ... Force ... Topics ... Focus ... Finiteness ... \label{BenPol}
\end{exe}

I will come back to a more detailed description of the left periphery in \sectref{cartmod}.

\section{Overall tendencies}

As we saw in \sectref{dataintro}, the relative frequency of the resumptive structure drops from Early Old Swedish to Early Modern Swedish before increasing in Late Modern Swedish.  This is quite unexpected given Meklenborg's clines. We would have expected resumption to be on a steady climb in the different environments, not a rollercoaster path.

However, we must take into account that there are two different resumptive elements in Swedish: \textsc{då}  and \textsc{så}. When the data are sorted with respect to the two different resumptive elements, we find a different pattern. Most importantly, we see that there is a significant shift from Old to Modern Swedish: While \textsc{då} resumption is by far the most common  in Old Swedish, this role is completely taken over by \textsc{så} after 1526. The shift is remarkable: While \textsc{då} resumption accounts for 98.3\% of all resumption in Late Old Swedish, it drops to 13.9\% in Early Modern Swedish. The data are presented in \tabref{swedish-1}.



\begin{table}
\caption{Resumption in the history of Swedish}
\label{swedish-1}
 \begin{tabular}{l *4{rr}}
  \lsptoprule
&\multicolumn{2}{c}{{EOS}}&\multicolumn{2}{c}{{LOS}}&\multicolumn{2}{c}{{EMS}}&\multicolumn{2}{c}{{LMS}}\\ 
\cmidrule(lr){2-3}
\cmidrule(lr){4-5}
\cmidrule(lr){6-7}
\cmidrule(lr){8-9}
\textsc{då} & 4,142 & 99.8\% & 1,118 &98.3\% & 28&13.9\%&37 & 6.6\%\\
\textsc{så}	& 7&0.2\% &19&1.7\% &174 &86.1\%&527& 93.4\%\\ 
\midrule
Total	&	4,149& 100.0\%	&1,137 &100.0\% 			&	202&	100.0\%&	564& 100.0\%\\
\lspbottomrule
\end{tabular}
\end{table}



When we examine the data further, we see that that there is a clear decrease of the use of resumption with \textsc{då} from Early Old Swedish, while there is the opposite tendency with resumption with \textsc{så} (\tabref{thaswa}).

\begin{table}
\caption{Resumption with  \textsc{då} and \textsc{så}}
\label{thaswa}
\begin{tabular}{lrrrr rrrr}
\lsptoprule
&\multicolumn{4}{c}{{\textsc{då}}}
&\multicolumn{4}{c}{{\textsc{så}}}\\\cmidrule(lr){2-5}\cmidrule(lr){6-9}
 	&	 	EOS&  LOS &	EMS & 	LMS  	& 	EOS  & 	LOS&  EMS	&  LMS\\ \midrule
CP	&	 4,070	&  1,051 &  28  & 37	&  7&		 19  	&		 171&  510\\
PP	&	 45	& 46& &	 			&	& &1	&	  5\\
ADVP	&	4	&	1&&	 				&	&	&	& 9\\ 
QP	&	18	&	2 &&	 				&	&	&	& \\ 
DP&	5	&	13&&	 				&	&&	2&	3 \\ 
\midrule
Total	&	4,142& 1,118	&28&37 			&7	&19	&174	&527 \\	
\lspbottomrule
\end{tabular}
\end{table}


In the following sections I will first give a description of the syntax of the structure (\sectref{morph}) before turning to its semantics (\sectref{sem}). 





\section{The syntax of the antecedent}\label{morph}

In this section, we will examine the syntactic environment in which resumption takes place, in other words the \emph{form} of the antecedent. I will show that the most common antecedent for a resumptive element is a fronted subordinate clause (CP), and that adverbs (AdvP) rarely function as antecedents.




\subsection{Fronted CPs}\label{frontCP}

Fronted adverbial clauses account for more than 96\% of the cases in all periods. These can be both adverbial clauses (\sectref{advcp}) or free relatives (\sectref{free}). 

\subsubsection{Adverbial CPs} \label{advcp}

Fronted adverbial CPs can be both V1 conditionals (\ref{syn}) and headed adverbial clauses (\ref{um}).

\begin{exe}
\ex\label{ex:nilsen:14} Fronted adverbial clauses

    \begin{xlist}
        

\ex\label{ex:nilsen:14a} \label{godr}
\gll  Værþær han goþær kunungær	. \textbf{þa} lati	guþ han længi liwæ.\\
becomes he good king {} {\textsc{då}} let God he long live\\\label{syn}
\glt `If he becomes a good king, God will grant him a long life.' \begin{flushright} (EOS, ULL, 1300--1399) \end{flushright}


\ex\label{ex:nilsen:14b}
\gll Um nokor talar a konungen eller a konungx radh thet a thera hedher ganger, ok æro siex godhe men til witne, \textbf{tha} skal han i stadz gömo sætias ok hals huggen wardha.\\ 
if someone  speaks on king.\textsc{def} or on king.\textsc{gen} council that on their honor goes and are six good men to witness, \textsc{då} shall he in city watch put  and neck chopped become\\ \label{um}
\glt `If somebody slanders the king or his council in the witness of six good men, then he should be imprisoned in the city and beheaded.' \begin{flushright} (EOS, MES, 1350--1450)\end{flushright}

  
  \end{xlist}
\end{exe}




 
 
\subsubsection{Free relatives} \label{free}
While the majority of the fronted clauses are adverbial, it is also possible to find fronted \textsc{wh} clauses that function as free relative clauses \REF{ex:nilsen:15}.
These predominently correspond to a thematic constituent in the clause (\ref{hua}, \ref{ey}), but it is also possible to find fronted free relatives with an adverbial function (\ref{eld}). 




\begin{exe}
\ex\label{ex:nilsen:15} Free relatives

\begin{xlist}
    
\ex
\gll Ok hua þaghær houothtiunda gör först \textbf{tha} skall	 them wt göra. \\
and who immediately main.tithe make first {\textsc{då}} shall they out make\\
\glt `And whoever first immediately pays the tithe shall pay them.' \begin{flushright} (EOS, YVGL, 1280--1399)\end{flushright} \label{hua}

\ex
\gll Huilken thera ey wil ath thy litha, \textbf{tha} haui wald ath wædia som saght ær \\
who in-this \textsc{neg} will on  this  comply \textsc{då} have power to appeal \textsc{rel} said is  \\
\glt `Who will not comply to this in this matter has the possibility to appeal what has been said.' \begin{flushright} (LOS, KL, 1442)\end{flushright}\label{ey}


\ex
\gll Och hwar någhor wille göra them skadha,	\textbf{så} gåår	eelden vthaff theras munn \\
and where somebody wanted do them harm {\textsc{så}} goes fire out.of their mouth\\
\glt  `And if any man would do them damage, fire comes out of their mouth.' \hfill \label{eld}(EMS, Vasa, Revelation 11:5, 1526)


\end{xlist}
\end{exe}


This kind of fronted free thematic relative introduced by the relative \textsc{hv} word disappears at the turn of the 18th century. This is the consequence not only of changes in resumption but in the relative system. The \textsc{hv}-relative is replaced by a D-pronoun (\emph{huilken} \lra \emph{den}). 



In Late Modern Swedish, there are still fronted CPs introduced by a \textsc{hv}-word, but these are adverbial clauses  and  not free relatives. Thus, clauses like the one in \REF{ex:nilsen:16} have been counted as ordinary adverbial CPs.

\begin{exe}


    \ex\label{ex:nilsen:16}
    \gll Men ehuru det är,	\textbf{så} finnas	de.\\
    but how it is \textsc{så} exist they\\
    \glt `Nevertheless, they exist.' \hfill (LMS, Argus, 1732--1734)




\end{exe}



\subsection{Fronted PPs}\label{fropp}
\textsc{Då} resumption after PP antecedents increases from 1.1\% in Early Old Swedish  to 4.1\% in Late Old Swedish. The difference is statistically significant ($\chi^{2}\colon p = 0$).\footnote{Based on a two way table comparing 45/4142 and  46/1072, using \url{http://www.quantitativeskills.com/ sisa/statistics/twoby2.htm}} In the following period, \textsc{då} resumption with fronted PPs disappears from the corpus. There is one occurrence with \textsc{så} in Early Modern Swedish, and a slight increase into Late Modern Swedish.   

Fronted PPs may be bare PPs \REF{ex:nilsen:17} or PPs that contain a relative clause \REF{ex:nilsen:18}.


\begin{exe}

    \ex\label{ex:nilsen:17} Bare PPs

    \begin{xlist}

\ex
\gll Millan	þingariþi \textbf{þa} maghu	æi husaby men þing suman kalla \\
between thing.times {\textsc{då}} may not King's men thing together call\\
\glt `Between the determined times of the thing, the King's men may not call another thing.'  \hfill (EOS, DL, 1318--1335)\label{thing-1}

\ex
\gll Oc om	wintrin \textbf{tha} skal	middax maltidhen wara mindre än wm quälin\\
and about winter {\textsc{då}} shall midday's meal be less than about evening\\
\glt `And in winter the  meal at midday should be lesser than the evening meal.' \hfill (LOS, LB3, 1400--1500)
\end{xlist}

\end{exe}




\subsection{Fronted AdvPs}
There are very few examples of \textsc{då} resumption with fronted AdvPs in the earliest periods. In Early Modern Swedish, there are no examples of the structure, but it resurfaces with \textsc{så} in Late Modern Swedish.


\begin{exe}

\ex\label{ex:nilsen:18}

    \begin{xlist}
     
\ex
\gll  Ok þær	æptir \textbf{þa} fylli	han fore domarænom meþ sinum eþe hwat han horþe.\\
and there after {\textsc{då}} fill he before judges with his oath what he heard\\
\glt `And thereafter, under oath, he will tell the judges what he heard.' \begin{flushright}
(EOS, BÖR 1300--1350)
\end{flushright} 


\ex
\gll Å dermed \textbf{så} reste han öfwer hela Werlden.\\
and thereby {\textsc{så}} travelled he over whole world\\
\glt `And so he travelled all over the world.' \hfill (LMS, Argus, 1732--1734)

    \end{xlist}
\end{exe}



\subsection{Fronted QPs}
 
In the Old Swedish language, it is possible to have a fronted quantifier phrase followed by \textsc{då} resumption. In EOS, all the antecedents of this category contain a quantified DP followed by a relative clause.

The frequency of fronted quantifier phrases followed by the resumptive is 0.5\% in Early Old Swedish, and it goes down to 0.2\% in Late Old Swedish. This is statistically significant (Fisher Exact test: $p = 0.116134$).\footnote{Based on 2/1116 and 18/4124; using \url{https://www.quantitativeskills.com/sisa/statistics/fisher.htm}}  Resumption after fronted QPs is not attested in the following centuries.\footnote{It should, however, be noted that QP + resumptive is attested as late as in 20th century Swedish \citep{Elmquist}. In a later study, \citet{Nordstrom} concludes that the acceptability rate for this is low in contemporary Swedish.}


\begin{exe}

\ex\label{ex:nilsen:19}
\gll Ok alt þæt sum þær bötis	firi þa höre	hænne halft til.\\
and all that \textsc{rel} there pays for {\textsc{då}} belongs her half to\\
\glt `And of everything that is paid for there, half of it belongs to her.' \begin{flushright}
    (EOS, ÖGL, 1300--1399)
\end{flushright} 

\end{exe}



\subsection{Fronted DPs}
\largerpage
Fronted DPs with \textsc{då} resumption is a marginal construction in Old Swedish. We find 5 such examples in Early Old Swedish, where 2 follow a bare DP, while 3 follow a DP that contains a relative clause \REF{ex:nilsen:20}.  In Late Old Swedish, 8 of 19 contain a relative clause, while the majority are bare DPs \REF{ex:nilsen:21}.


\begin{exe}

\ex\label{ex:nilsen:20} DP + relative as antecedent
\begin{xlist}

\ex

\gll Then tiidh stadzens swen biwdher nokrom til radzstuffw koma vm skut, \textbf{tha} skall	tweggia manna witne widher wara \\
the time city's man ask someone to council.house come about taxes {\textsc{då}} shall two men witness by be\\
\glt `At the time the city's mayor asks someone to come to the council, two men should witness it.' \hfill (EOS, MES, 1350--1450)



\ex
\gll Galienus mestare sighir at kona the som ätir fröö aff merke i	barnbyrdh \textbf{tha} faar	thz barn bulda wtan ieff oc fulan saröka \\
Galenius master says that wife \textsc{def} that eats seeds of  celery in child.birth {\textsc{då}} gets the child bules without even and bad wounds\\
\glt `Master Galenius  says that if the woman  eats celery seeds in childbirth, the child will get bules and bad wounds.'\\~ \hfill  (LOS, LB3, 1400--1500)
    
\end{xlist}


\ex\label{ex:nilsen:21} Bare DP as antecedent
\begin{xlist}
    


\ex
\gll Ok þæn	kostin \textbf{þa} förin	böndær a kirkiu bol mote hanum.\\
and that food {\textsc{då}} bring peasants to church farm meet him\\
\glt `And with respect to the food, the peasants should bring it to him at the church's farm.' \begin{flushright}
(EOS, ÖGL, 1300--1399)
 \end{flushright}  

\ex
\gll Oc annat	arit \textbf{tha} wardha	the stor.  \\
and second year {\textsc{då}} becomes it big \\
\glt `And the second year, it [= the juniper] gets big.' 
\begin{flushright}
(LOS, LB4, 1400--1500) \end{flushright} 
    

\ex
\gll Oc then tidh wätzskan	brynder \textbf{tha}  ludhir	blodit samman. \\
and that time liquid burns {\textsc{då}} sticks blood together\\
\glt `And during the time that the liquid is burning, the blood will coagulate.' \hfill 
(LOS, LB5\&6, 1400--1500) 
\end{xlist}

\end{exe}







\section{The semantics of the antecedent}\label{sem}
Having examined the syntax of the antecedent, it is time to turn to its semantics.






\subsection{Fronted CP}
As we saw in \sectref{frontCP}, there are two different types of CP antecedent: adverbial and thematic. In the following, we will first look at the adverbial CPs (\sectref{adcp}), then at the free relatives (\sectref{frr}). 


\subsubsection{Adverbial CPs} \label{adcp}

\begin{table}[t]
\caption{Resumption after fronted adverbial CPs}
\label{CP}
\begin{subtable}{\textwidth}
\centering
\caption{\textsc{då}}
 \begin{tabularx}{.9\textwidth}{X *4{r@{~}r} }
  \lsptoprule
&\multicolumn{2}{c}{{EOS}}&\multicolumn{2}{c}{{LOS}}&\multicolumn{2}{c}{{EMS}}&\multicolumn{2}{c}{{LMS}}\\ \midrule
condition   &3838 &96.4\%& 930 & 88.5\%&15&53.6\%&19&51.4\%\\ 
time    &125&3.1\%&115&10.9\%&13&46.4\%&17&45.9\%\\ 
concession  &4&0.1\%&5&0.5\%&& && \\ 
comparison  &3&0.1\%&&&& && \\ 
exception   &2&0.1\%&&&& && \\ 
cause   &6&0.2\%&1&0.1\%&& &1&2.7\%\\ 
consequence & &&&&& && \\
locative & 1 & 0.1\% &&&\\
intention & 2 & 0.2\% \\\addlinespace
Total   &3981& 100.2\%&1051&100.0\%&28& 100.0\%&37&100.0\% \\
\lspbottomrule
\end{tabularx}
\end{subtable}\medskip\\
\begin{subtable}{\textwidth}
\centering
\caption{\textsc{så}}
 \begin{tabularx}{.9\textwidth}{X *4{r@{~}r} }
  \lsptoprule
&\multicolumn{2}{c}{{EOS}}&\multicolumn{2}{c}{{LOS}}&\multicolumn{2}{c}{{EMS}}&\multicolumn{2}{c}{{LMS}}\\ \midrule
condition   &4&57.1\%&9&47.4\%&119&69.6\%&327&64.3\%\\ 
comparison  &3&42.9\%&8&42.1\%&20&11.7\%&25&4.9\%\\ 
time    && &2&10.5\%&26&15.2\%&107&21.0\%\\ 
concession  && && &5&2.9\%&35&6.9\%\\ 
intention   && && && &1&0.2\%\\ 
cause   && && && &14&2.7\%\\
locative && && & 1&0.6\%&  \\\addlinespace
Total   &7&100.0\% &19&100.0\% &171&100.0\% &510&100.0\% \\ 
\lspbottomrule
\end{tabularx}
\end{subtable}
\end{table}



In Old Swedish, a fronted adverbial CP followed by the resumptive \textsc{då} is predominantly conditional \REF{ex:nilsen:22}. The second most common environment is when \textsc{då} follows a temporal antecedent \REF{ex:nilsen:23}.


\begin{exe}
 \ex\label{ex:nilsen:22}
\gll  [...]  will man giua sinu frillu barne nakuat. \textbf{þa} a	han fara til lionga þingxs ælla firi kunung.\\
{}  will man give his concubine children something  {} \textsc{då} {} always he go to  Linköping thing or for king\\
\glt `[...]  if a man wants to give something to his illegitimate children, then he must go to the thing at Linköping or to the king.' \hfill (EOS, ÖGL, 1300--1399)

\ex\label{ex:nilsen:23}
\gll Tha biskopir wil næmpd seea. \textbf{tha} skal han manadha budh fore sik faa. \\
when bishop will council sit \textsc{då} shall he months announcement for \textsc{refl} get\\
\glt `When the bishop wants to see the council, he shall send out a message one month in advance.'  \hfill (EOS, ÖGL, 1300--1399) 

\end{exe}

We can, however, observe a clear shift as the language evolves into the modern period. We have already seen that the amount of \textsc{då} resumption decreases. At the same time, the semantic environment of \textsc{då} resumption changes. While \textsc{då} resumption could occur after all kinds of antecedents in the earliest sources, in LOS, it almost exclusively occurs after an antecedent expressing condition or time. In both periods of Modern Swedish, there is almost an equal split between conditional \REF{ex:nilsen:24} and temporal \REF{ex:nilsen:25}  antecedents when \textsc{då} is the resumptive. The semantics of the subordinate temporal clause is often very close to that of a conditional clause, as is clear from the example in \REF{ex:nilsen:25b}. The adverbial clause is almost always introduced by \emph{när} in the sense `when' or `if'.


\begin{exe}
\ex\label{ex:nilsen:24} Conditional CP\\
\gll Ah, sade hon, är det ingen annan, \textbf{då} kan jag berätta hwad som uti mig har hendt.\\
ah said she is there no other {\textsc{då}} can I tell what that outof me has happened\\
\glt `Ah, she said, if there is nobody else, I can tell you what has happened to me.' \hfill (LMS, Argus, 1732--1734)

\ex\label{ex:nilsen:25} Temporal CP\\
   \begin{xlist} 
   \ex\label{ex:nilsen:25a}
    \gll Men när Herodes war dödh, sij, tå vppenbaradhes Herrans Ängel Joseph j sömpnen j Egyptj land \\
    but when Herodes was dead see \textsc{då} appeared Lord's angel Jospeh in sleep in Egypt land\\
    \glt `But when Herod was dead, an angel of the Lord came in a dream to Joseph in Egypt.' \hfill (EMS, Vasa, 1526)\footnote{Matt 2:19, translation from \url{https://www.o-bible.com/cgibin/ob.cgi?version=bbe\&book=mat\&chapter=2}}
  \ex\label{ex:nilsen:25b}
    \gll  När twänne redelige Män komma i Krig med hwarandra, \textbf{då} ër det en swår feigd at se uppå.\\
    when two honorable men come in war with each.other {\textsc{då}} is it a difficult feud to look upon\\
    \glt `When two honorable men go to war against each other, it is a difficult feud to watch.' \hfill (LMS, Argus, 1732--1734)
\end{xlist}
\end{exe}





Further, we see that there is a restriction in the possible semantics of the antecedent with \textsc{då} resumption. In Early Old Swedish, we find scattered examples of \textsc{då} resumption with  antecedents other than temporal and conditional ones. These are arguably few, but they cannot be discarded. \REF{ex:nilsen:26} is an example of a fronted causal clause, while \REF{ex:nilsen:27} is a fronted comparative clause.



\begin{exe}

        
   
\ex\label{ex:nilsen:26}
\gll  Thy at forne lagha rætte seen uiþninga uærþe. \textbf{tha} comber stundum sua til at umskiptis thæn lagha staþge sum skipaþer ær til at ræta  kranca manna siþi. \\
because that former laws court later confirmed become \textsc{då} come occasionally so to that change that law determined \textsc{rel}  proclaimed is sent to that rectify weakness man's side   \\
\glt `Because a former law later becomes confirmed, sometimes it is so that the laws that have been established are changed so that they take the side of the weaker man.'  \hfill (EOS, Con-SL, 1330–1400)


\ex\label{ex:nilsen:27}
\gll sua manga arftaka sum han	hauær \textbf{þa} ma	han giua slikan lut sum en þerra takær\\
so many heirs that he has {\textsc{då}} must he give such dowry that one theirs take \\
\glt `He should give as many appropriate dowries as he has heirs.' \begin{flushright}
    (EOS, ÖGL, 1300--1399)
\end{flushright}

        

\end{exe}



 When it comes to \textsc{så} resumption, this is strongly restricted in the earliest sources and is only found after conditional \REF{ex:nilsen:28} and comparative antecedents \REF{ex:nilsen:29}.

\begin{exe}

     \ex\label{ex:nilsen:28}
\gll Hugger maþer oc vp scogh tiældræ mællum böte oc swa. \textbf{swa} scal	bötæ baþe firi aldin viþu oc vndir viþu sum sakt ær.\\
chops man also up forest mark between pay and so {\textsc{så}} shall pay both for nuts by and under width as said is\\
\glt `If a man cuts down [trees in] a [common] forest between the marks and then pays for it, he should pay for the nuts that lay on the ground where he has cut down trees as well.’ \hfill (EOS, YVGL, 1280--1399)


 \ex\label{ex:nilsen:29}
\gll Swa sum han war mykin for sic till alþræ raþhæ. \textbf{swa} war 	oc han goþþær drængær till swærþh\\
so \textsc{rel} he was much for \textsc{refl} to all things  {\textsc{så}} was also he good boy to sword\\
\glt `Just like he was a talented squire in all things, he was also talented with the sword.' \hfill (EOS, Vid-an, 1280--1399)   




\end{exe}




\subsubsection{Fronted free relatives} \label{frr}

Fronted free relatives followed by a resumptive that corresponds to a thematic constituent are not found in the modern language. In the examples below in (\ref{froto}) the relative clauses correspond to the subject of the main clause. In (\ref{hva}) there is a null subject, but in (\ref{kill}) and (\ref{quinna}) the subject \emph{han} `he' is expressed and post-verbal. These clauses are interesting as it would seem that the fronted clause is picked up twice, first by the resumptive and then by the anaphor. An analysis of this construction needs to account for this apparent contradiction, which I will get back to in \sectref{structural}.

It should be noted that these are not instances of topic left dislocation, as this construction in Germanic requires that the left dislocated element be immediately followed by its pronominal anaphor.

The fronted free relative clauses often come with a nuance of concession: `whoever would do this'. In the corpus from Early Old Swedish, 61 of 86 fronted free relatives are of this kind. 

\begin{exe}

\ex\label{ex:nilsen:30} Fronted thematic clauses\label{froto}
    \begin{xlist}
   \ex
   \gll Ok hua þaghær houothtiunda gör	först \textbf{tha} skall	them wt göra. \\
and who immediately main.tithe make first \textsc{då} shall they out make\\
\glt `And whoever first immediately pays the tithe shall pay them' \begin{flushright} (EOS, YVGL, 1280--1399)\end{flushright} \label{hva}


\ex
\gll Æ hwar þen dræper han ærwæ skulli. \textbf{þa}  wari	han skilder wiþ arf.\\
and who then kills he inherita should {\textsc{då}} become he separated with inheritance\\
\glt `And whoever kills the man he was supposed to inherit from will not get the inheritance.' \hfill (EOS, SöL, 1330--1400)\label{kill}

\ex
\gll Hwar som äther mykyt aff henne, \textbf{thaa} faar	han quinna lwstha.\\
who \textsc{rel} eat much of her \textsc{då} get  he women urges\\
\glt `Whoever eats  much of it [parsnip] will get a woman's urges.' \begin{flushright}  (LOS, LB7, 1500--1600)\end{flushright}\label{quinna}


    \end{xlist}
    
\end{exe}




A free relative may also have an adverbial meaning. As these free relative clauses are introduced by \emph{hvar} `where'), they often depart from a locative reading. However, some of these clauses take on different adverbial interpretations. Thus, (\ref{skiptir} and \ref{kiere}) places the action in an abstract time--space location, while the sense of (\ref{ezore}) approaches that of a fronted conditional clause.

\newpage
\begin{exe}
\ex\label{ex:nilsen:31} Fronted adverbials
    \begin{xlist}


\ex
 \gll Æ hwar man skiptir iorþ. \textbf{þa}  wæri	þæt byrþ hans.\\
 always where man shifts earth {\textsc{då}} becomes that birthright his\\
 \glt `Anytime a man exchanges soil, then that soil becomes his and can be inherited.' \hfill (EOS, ULL, 1300--1399)\label{skiptir}



        \ex
        \gll E hwar bondæ faar ay ræt	. \textbf{þa} kiæræ	sik fore konungæ sinom.\\
        always where paysant gets not right {} {\textsc{då}} complain \textsc{rfl} before king his\\
        \glt `And when a peasant does not get his way, then he appeals before his king.' \hfill (EOS, HäL, 1350--1400) \label{kiere}



  \ex
        \gll  Hwar sum mæn hittæs satte i garþe oc skiliæs osatter æn þo at gierninger comin þeræ mællum. \textbf{þa}  ær	ei ezore.\\
        where that men find agreement in farms and separates enemies before  that that actions come him between {\textsc{då}} is not oath.breach\\
        \glt `Where men meet as friends and become enemies without anything having happened between them, the King's oath has not been broken.' \label{ezore}
        \hfill  (EOS, SöL, 1330--1400)
    


    \end{xlist}
\end{exe}







\subsubsection{Summing up}
We have seen \textsc{då} is by far the most common type of resumptive following fronted CPs in Old Swedish (pre 1526). Come Modern Swedish, the situation is turned upside down, and \textsc{så} is the most common resumptive. 

In Early Old Swedish, \textsc{då} accepts a wider range of semantics in its antecedent than its modern descendant \textsc{då}. \textsc{Tha} has in other words undergone semantic retraction implying a shift from a generalized resumptive in Old Swedish to a specialized resumptive in Modern Swedish.

\textsc{Så} is hardly attested as a resumptive in the oldest sources, and it is limited to antecedents expressing condition or comparison. We see that the semantic scope of the antecedent of \textsc{så} expands gradually. This way,  \textsc{så}  becomes a generalized resumptive.

With respect to the semantic type of the  CP antecedent, fronted conditional clauses are dominant, followed by temporal clauses. The elevated number of fronted conditionals in Early Old Swedish  can be explained by the fact that these are legal documents. Therefore, one might be tempted to ask if the distribution reflects the selection of texts rather than the language itself. While the corpus from Early Old Swedish consists of laws, there are no laws among the examined texts from Modern Swedish. In the case of Late Modern Swedish, the only accessible legal text is \emph{Kristoffers landslag}. In order to check the status of \textsc{då} and \textsc{så} resumption, I therefore examined the oldest law accessible through Lagen.nu, namely the \emph{Ridderskapets och adelns privilegier} from 1723. This text provides 18 examples of resumptive structures, all following a fronted clause. 15 of these have \textsc{så} as their resumptive, while 3 have \textsc{då}. 2 of the 3 examples of \textsc{då} follow a conditional subordinate clause, while the last follows a free relative introduced by \emph{hwar}. The subordinate clauses preceding \textsc{så} have different semantics: temporal, locative, conditional, and more. As we will see, the pattern we find in this law is  quite similar to the one we find in other texts from the same period. Therefore, I take the tendencies expressed in \tabref{thaswa} to be representative for the evolution of the resumptive structure in Swedish.

The strong link between time and condition is not surprising as the condition is temporal in nature in that it  will always  precede the  consequence \REF{ex:nilsen:32}. (See \citealt[10]{Bergqvist} for a similar observation.)

\ea\label{ex:nilsen:32}  $[ \alpha \Rightarrow \beta ] \rightarrow [ \alpha \prec_{t} \beta ]$\z

The fact that \textsc{då} is linked to temporal contexts  is embedded in its etymology: when not used as a resumptive, \textsc{då} is a temporal conjunction and an adverbial.




\subsection{Fronted PPs}

We have already seen in \sectref{fropp} that resumption after a PP  is  much more common in Old Swedish than in Modern Swedish. Further, it is more frequent in Late Old Swedish than in Early Old Swedish (1.1\% vs. 2.4\%). 

Again, we can observe the shift from \textsc{då} resumption to \textsc{så} resumption: There are no instances of \textsc{så} resumption with PP antecedents in Old Swedish, while there are no such examples with \textsc{då} in the modern language. Further, we see that resumption after PPs is a marginal phenomenon in all periods. 



\begin{table}
\caption{Resumption after fronted  PPs}
\label{PP}
\begin{subtable}{.6\textwidth}
\centering
\caption{\textsc{då}}
\begin{tabular}{l *2{r@{~}r}}
\lsptoprule
& \multicolumn{2}{c}{{EOS}} 
& \multicolumn{2}{c}{{LOS}}\\\midrule
time         &7  &15.6\%&   23& 50.0\%\\
place        &5  &11.1\%&   10& 21.7\%\\
instrument   &33 &73.3\%&    7& 15.2\%\\ 
cause        &   &      &    2&  4.3\%\\ 
other        &   &      &    4&  8.8\%\\\addlinespace
Total        &45 &100.0\%&   46& 100.0\%\\ \lspbottomrule
\end{tabular}
\end{subtable}\begin{subtable}{.4\textwidth}
\centering
\caption{\textsc{så}}
\begin{tabular}{l *2{c}}
\lsptoprule
           & EMS & LMS\\\midrule
time       &     &    \\
place      & 1   &    \\
instrument &     & 5  \\ 
\addlinespace
Total      & 1   & 5\\\lspbottomrule
\end{tabular}
\end{subtable}%
\end{table}

 
In the table, I have grouped a rather heterogeneous group under the label `instrument'. These are constituents with no specific semantic content, but that either express the instrument that affects the action in the clause \REF{ex:nilsen:33} or that introduces an element of the clause that follows \REF{ex:nilsen:34}. I have used a similar broad definition for the group `place' and included abstract locations such as \emph{i siukdom} `in sickness' and similar.



  \begin{exe}
  
\ex\label{ex:nilsen:33}

\gll Ok aff sinne	makt \textbf{tha} driffuir	hon bort the kaallosot som kallas febris acuta.\\
and of her power {\textsc{då}} drive she away the illness that called febris acuta\\
\glt `And by the means of her powers, she drives away the illness that is called \emph{febris acuta}.' \hfill (LOS, LB3, 1400--1500)

\ex\label{ex:nilsen:34}

\gll Jag finner, min Läsare, at för en stackars Dräng, \textbf{så} talade Jerker intet så galit.\\
I find my reader that for a poor boy \textsc{så} spoke Jerker not so badly\\
\glt `Well, my reader, I think that for a poor boy, Jerker did not speak so badly.'  \begin{flushright}
(LMS, Argus, 1732--1734)
\end{flushright}

\end{exe}
Note that in the example in \REF{ex:nilsen:34}, the PP is in initial position in a subordinate clause embedded under \emph{finner} `finds'.

We may argue that in clauses like \REF{ex:nilsen:33}, the instrument is a prerequisite of the action described in the main clause, and that a temporal reading is inherent, thus justifying the use of \textsc{då}. This is more difficult to sustain in the case of \REF{ex:nilsen:34}.


Even though resumption with \textsc{då} often follows instrumental antecedents in Early Old Swedish, \textsc{då} also often follows a PP antecedent that is \temporal {} \REF{ex:nilsen:35}.


  \begin{exe}

\ex\label{ex:nilsen:35}
\gll Millan	þingariþi \textbf{þa} maghu	æi husaby men þing suman kalla. \\
between thing.times {\textsc{då}} may not King's men thing together call\\
\glt `Between the determined times of the thing, the King's men may not call another thing.'  \hfill (EOS, DL, 1318--1335)\label{thing-2}

\end{exe}


In Late Old Swedish, the relative number of temporal contexts increases dramatically, and almost half of the resumptives after PPs follow a temporal antecedent. 


  \begin{exe}
 
\ex\label{ex:nilsen:36}

\gll Än äpter	maltidh \textbf{tha} skolin	j saktelika ganga, oc swa hwilas.\\
and after meal \textsc{tha} shall you slowly walk and then rest\\
\glt `And after a meal, you should walk slowly and rest.' \hfill   (LOS, LB3, 1400--1500)
\end{exe}


In my corpus from Modern Swedish, there are no instances of resumption after temporal PPs. In Late Modern Swedish, the structure is exclusively attested with instrumental PPs.

  

The data presented here point towards  a process of specialization, similar to the one we have already seen with CP antecedents. Resumption with \textsc{då} starts out as fairly generally accessible, but already in Late Old Swedish, there is a stronger tendency to use \textsc{då} with temporal antecedents. There is in other words a tendency towards a semantic restriction with respect to the antecedent.





\subsection{Fronted AdvP}
Resumption after a fronted AdvP remains fairly marginal throughout the first centuries of Swedish. Also, there are some examples in Old Swedish, but none in Vasa's Bible translation. However, the structure resurfaces at the beginning of the 18th century, but with \textsc{så} as its resumptive. 


\begin{table}
\caption{Resumption with ADVP antecedents}
\label{AdvP}
\begin{subtable}{.5\textwidth}
\centering
\caption{\textsc{då}}
\begin{tabular}{lcc}
\lsptoprule
           & EOS & LOS \\\midrule
time       & 2 & \\
place      &2  & \\
comparison &   &1 \\
\addlinespace
Total      &4  &1\\
\lspbottomrule
\end{tabular}
\end{subtable}%
\begin{subtable}{.5\textwidth}
\centering
\caption{\textsc{så}}
\begin{tabular}{lc}
\lsptoprule
     & LMS \\\midrule
time & 2  \\
condition & 2  \\
comparison & 1  \\
aboutness & 1 \\
sentence adverbial & 3 \\ \addlinespace
Total& 9\\
\lspbottomrule
\end{tabular}
\end{subtable}
\end{table}


There are no apparent semantic restrictions with respect to the fronted AdvP antecedent. We find relatively more resumption in this context in Late Modern Swedish than in the old language. The antecedent in \REF{ex:nilsen:37} expresses time, while the one in \REF{ex:nilsen:38} is a comparison.


 \begin{exe}
 
\ex\label{ex:nilsen:37}
\gll  ok þær	æptir \textbf{þa} fylli	han fore domarænom meþ sinum eþe hwat han horþe .\\
 and there after \textsc{då} fill he before judges with his oath what he heard\\
\glt `And then he tells the judges under oath what he heard.' 
\begin{flushright}
    (EOS, BÖR, 1300--1350)
\end{flushright} 


\ex\label{ex:nilsen:38}

\gll  thes mera han drikkir thes mera \textbf{tha} wanskas han \\
the more he drinks the more {\textsc{då}} troubles he\\
\glt `the more he drinks, the more troubles he has.' \hfill (LOS, LB3 1400--1500)
  


\end{exe}


Interestingly, we find resumption with sentence adverbials and nynegocentric adverbials as early as the 18th century. 

 
  \begin{exe}
\ex\label{ex:nilsen:39}
\gll Åtminstone \textbf{så} indraga hans Segel, at han intet seglar fram för oss\\
 {at least} \textsc{så} withdraw his sail that he \textsc{neg} sail forth for us\\
 \glt `At least take his sail from him so that he will not sail in front of us.' \begin{flushright}
(LMS, Argus 1732--1734)
\end{flushright} 
 
 
\ex\label{ex:nilsen:40}
\gll  Nu  \textbf{så} skall jag be ehr min lilla wackra Argus, at ... \\
 now \textsc{så}  shall I ask you my little beautiful Argus that\\
\glt  `And now, my little, beautiful Argus, I will ask you that ...' \begin{flushright}
(LMS, Argus 1732--1734)
\end{flushright} 

\end{exe}

 \subsection{Fronted QPs}
The fronted QPs followed by a resumptive element may in most cases be translated by `with respect to', as in \REF{ex:nilsen:41} and \REF{ex:nilsen:42}. As such, they may be described as some sort of hanging topic.


\begin{exe}

    \ex\label{ex:nilsen:41}
\gll Alla aflinga iorþ sum ei ær af byrþ köpt. \textbf{þa} hawi	þen wald sum aflat hawer. giwa oc giællæ oc sæliæ hwem han will.\\
all crops earth that not are of birthright bought {\textsc{då}} has the.one power as grown has give and pawn and sell whom he will
\\
\glt `With respect to all crops of land that has not been inherited, the one who grows the crops has the right to give and pawn and sell to whoever he wants.' 
\hfill (EOS, SöL, 1330--1400)

\ex\label{ex:nilsen:42}
\gll  Ok alt þæt konan	uærkar \textbf{þa} skal	hænna gipta man suara firi hana.\\
and all that woman do {\textsc{då}} shall her married man answer for her\\
\glt `And with respect to everything a married woman does, her husband will answer to it.' \hfill (EOS, ÖGL, 1300--1399)


\end{exe}



One of the clauses in the corpus from Early Old Swedish functions as a temporal adverbial, but again the closest translation is  `with respect to' \REF{ex:nilsen:43}.

\begin{exe}

    \ex\label{ex:nilsen:43}
\gll Alla dagha som marknadha skulu haldas i thæssom fornempdom köpstadhum, som i wæstraros, vpsalum eller hwilkum stadh thet ær, \textbf{tha} skal	ængin man nokot köp göra vm sunnodagh \\
all days that marked shall hold in these aforementioned market.towns as in Westerås, Uppsala or what city it is {\textsc{då}} shall no man anything purchase make on Sunday\\
\glt `With respect to all the days there is a market in the aforementioned market towns like Westerås, Uppsala or whatever city it might be, no man should buy anything on a Sunday.' \hfill (EOS, MES, 1350--1450)
\end{exe}

In Late Old Swedish, there are only two examples of a QP followed by a resumptive. These are both locative.

\begin{exe}
\ex\label{ex:nilsen:44}
\gll Alla stadi man find hitter, \textbf{tha} scal	han sik wrtiufua göra först meth lysninga witnom.\\
all places man finding find {\textsc{då}} shall he \textsc{refl} un.thief make first with calling witnesses\\
\glt `Every place a man finds something, he should make himself its rightful owner first by calling it in the presence of witnesses.' \hfill (LOS, KL, 1442)
\end{exe}



\subsection{Fronted DPs}\label{fdps}
We have seen that the number of bare DPs functioning as the antecedent of \textsc{då} increases in Late Old Swedish. The fronted DPs are either thematic or expressions of time. The two examples with DP + resumption in Late Modern Swedish are both topical.

\begin{table}
\caption{Resumption with DP antecedents}
\label{DP}

\begin{subtable}{.5\textwidth}
\centering
\caption{\textsc{då}}
\begin{tabular}{lrr}
\lsptoprule
        & EOS & LOS \\
\midrule
thematic & 2 & 6 \\
time     & 3 & 13\\\midrule
Total    & 5 & 19\\
\lspbottomrule
\end{tabular}

\end{subtable}%
\begin{subtable}{.5\textwidth}
\centering
\caption{\textsc{så}}
\begin{tabular}{lrr}
\lsptoprule
         & EMS & LMS \\\midrule
thematic &     & 2\\
time     &     & \\\midrule
Total    &     & 2\\ 
\lspbottomrule
\end{tabular}
\end{subtable}
\end{table}

\newpage
\begin{exe}
\ex\label{ex:nilsen:45} Thematic elements
\begin{xlist}
    
\ex
\gll Nu 	köpmalum \textbf{þa} koma	 i manga sældir.\\
now transactions {\textsc{då}} come in many sales\\
\glt `There may be many sales in the transactions.'\hfill (EOS, ÖGL, 1300--1399)

\ex
\gll Jtem then som wardir freneticus thz är galyn oc spyr aff offmiklom hoffwdwerk \textbf{tha} skal han taka synap oc honag oc thz gula aff äggit oc fröt aff grabo som hetir artimeisa\\
\textsc{item} the.one who becomes frenetic that is crazy and vomits of big headache {\textsc{då}} shall he take mustard and honey and the yellow of egg and grain of mugwort that called artimeisa\\
\glt `As for the one who gets frenetic, that means crazy, and vomits from big headaches, he shall take mustard and honey and egg yolks and the seeds of mugwort that is called artimeisa.' \hfill (LOS, LB3, 1400--1500)\label{gal}

\ex
\gll Det endaste den Wåpan duger til, \textbf{så} är det til Frökens Hemligheter, efter hon tror den Wåpan så wäl.\\
the only that weapon uses to {\textsc{så}} is it to missy's secrets after she thinks the weapon such well\\
\glt `The only thing that weapon is good for is for the missy's secrets, given that she has such a faith in that weapon.' \begin{flushright}
     (LMS, Argus, 1732--1734)
\end{flushright}




\end{xlist}

\end{exe}


It should be noted that in \REF{gal}, the fronted DP is resumed both by the \textsc{då} resumptive and the personal anaphor \emph{han} `he'. This double resumptive strategy is very different from what we have seen with respect to fronted adverbial elements. However, it strongly resembles the cases we saw with fronted free relatives in \sectref{frr}.

\newpage
Already in Early Old Swedish, there is a tendency that the fronted DP adjunct expresses time. This is even more so in Late Old Swedish, where 13 of 19 clauses are $[$+ \textsc{temporal}$]$ \REF{ex:nilsen:46}. However, this structure is not attested in Modern Swedish.

\begin{exe}

    \ex\label{ex:nilsen:46} DP expressing time

    \begin{xlist}
\ex
\gll Then tydh aarsens borghamestara ok radhmen skulu wælias, \textbf{tha} skulu	ok alle embetzmen wælias.\\
the time year mayor and councilors should elected {\textsc{då}} should also all civil.servants elect\\
\glt `At the  time of the year that the mayor and the councilors are elected, also the civil servants should be elected.'
\hfill (EOS, MSL, 1350--1450)


\ex
\gll Then siette manadh i	arit \textbf{tha} skal	man drycka kella watn fore lwngom.\\
the sixth month in year {\textsc{då}} shall one drink well water for lungs\\
\glt `In the sixth month of the year, one shall drink water from the well for [the benefit of] the lungs.' \hfill (LOS, LB10, 1400--1500)
\end{xlist}

\end{exe}

\subsection{Summing up}

We have seen  that  \textsc{då} is the dominating resumptive element in Old Swedish  but that it is more or less replaced by \textsc{så}   in Modern Swedish. Further, we have seen that resumption after \textsc{då} becomes specialized after CP and PP antecedents denoting time, while resumption after \textsc{så} is generalized.


\section{The clines}\label{clines}

In \sectref{rescli}, we discussed Meklenborg's clines, and we suggested that the adverbial cline be divided into a syntactic and a semantic cline. The clines are repeated in \REF{ex:nilsen:47}.

\begin{exe}
\ex\label{ex:nilsen:47} The clines
\begin{xlist}
\ex  \textsc{Thematic cline}: DP\textsubscript{\textsc{rel}}  > DP\textsubscript{\textsc{θ}} \label{thetc}
\ex   \textsc{Syntactic cline}: CP  > PP  > AdvP \label{sync}
\ex \textsc{Semantic cline}: specific semantics > general semantics > complete semantic bleaching (= nynegocentric use) \label{semc}
\end{xlist}
\end{exe}


\subsection{The thematic cline}

The implication of the thematic cline is that resumption first occurs with fronted DPs that contain a relative clause before it extends to bare adjunct DPs that correspond to a thematic element in the clause. While our data show a relatively elevated number of fronted free relative clauses resumed by a resumptive, we do not find bare fronted thematic DPs with the resumptive element.


However, we see that the number of bare DPs followed by the resumptive increases from Early to Late Old Swedish. However, these are all adverbial in nature and express time. 

If we examine the semantics of  all free relative clauses (CP-rel), DPs followed by a relative clause (DP+CP) and bare DPs, we get the distribution in \tabref{them-adv}.




\begin{table}
\caption{Thematic and adverbial antecedents}
\label{them-adv}
\begin{tabular}{l rr rr rr}
\lsptoprule
&	\multicolumn{2}{c}{CP-rel} & \multicolumn{2}{c}{DP+CP} & \multicolumn{2}{c}{DP}\\\cmidrule(lr){2-3}\cmidrule(lr){4-5}\cmidrule(lr){6-7}
&	thematic	&	adverbial	&	thematic	&	adverbial	&	thematic	&	adverbial\\\midrule
\multicolumn{7}{l}{\textsc{då}}\\
EOS	&	61	&	28	&		&	3	&	2\\
LOS	&	11	&	1	&	2	&	5	&	1	&	10\\
\addlinespace
\multicolumn{7}{l}{\textsc{så}}\\
EMS	&	2	&	6	&		&		&	\\
LMS	&	1	&	20	&		&		&		&	3\\
\lspbottomrule
\end{tabular}
\end{table}

We see in \tabref{them-adv} that the very few DPs modified by a relative clause are  adverbial in Early Old Swedish, while they may also be  thematic in Late Old Swedish. However,  the numbers are so small we may not draw any conclusions. With respect to bare DPs as antecedents, these are typically adverbial in Late Old Swedish.

When it comes to \textsc{så} resumption, we see that there is an increase of thematic free relatives followed by a resumptive in Late Modern Swedish. These are, however, almost exclusively adverbial in nature. There is only one example where the antecedent may be interpreted as corresponding to a thematic argument \REF{ex:nilsen:48}. The interpretation is not uncontroversial -- it might be argued that this is in fact a concessive adverbial clause introduced by \emph{hwad}.


\begin{exe}

    \ex\label{ex:nilsen:48}
    
    \gll Men hwad han icke hade i desse, så hade han i en del, som effterfölgde \\
but what he not had in these {\textsc{så}} had he in a part that followed\\
\glt A: `But that which he did not have in those [= compassion], he had in the following group' \hfill [ \textsc{thematic}]\\
B: `But regardless of what he did not have in those, ...' \hfill [\textsc{adverbial}]
\begin{flushright}
     (LMS, Argus, 1732--1734)
\end{flushright}
\end{exe}


We thus see that resumption after a constituent corresponding to a thematic element is a feature of the old language and restricted to \textsc{då} resumption. In the modern language, where \textsc{då} resumption is limited to [+\textsc{temporal}] antecedents, resumption after fronted thematic elements is ruled out. 

These observations lead us to reject Meklenborg's claim that there is a thematic cline in the case of Swedish. However, the lack of a clear-cut cline might be the consequence of the semantic strengthening of \textsc{då}. In Late Old Swedish, this is particularly clear after bare DP antecedents, where all but one are temporal adverbials. It would seem, again, that the impact of semantics overrule the possibility of having a syntactic cline. 





\subsection{The syntactic cline}

Meklenborg's cline is an implicational hierarchy, and it suggests that there is a unidirectional development, where the domain of the resumptive gets expanded. While structures with fronted CPs are most common at first, the structure extends to PPs, then to AdvPs. This, is not, however, what we find in the data presented thus far. While resumption after fronted CPs is predominant in all periods, resumption after fronted PPs increases from Early Old Swedish into Late Old Swedish, but then it drops, and by the time of Late Modern Swedish, the frequency of resumption in these contexts is lower than in the earliest period. 

    

When it comes to resumption after PPs,  there is a syntactic expansion from Early Old Swedish to Late Old Swedish with respect to the resumptive \textsc{då}. However, this expansion stops abruptly, and resumption after PPs with \textsc{då} disappears. It is resumed by \textsc{så} in Early Modern Swedish, and then it becomes more common in Late Modern Swedish.

The number of resumptive structures with a fronted AdvP in Old Swedish is, however, so small that it is not possible to draw any conclusions. We observe the same tendency as with resumption with PPs: resumption with \textsc{då} disappears completely with the ascent of Modern Swedish. It is not until Late Modern Swedish that we again find resumption with AdvPs, but then with the resumptive \textsc{så}. 


\subsection{The semantic cline}

As we have seen, \textsc{då} undergoes semantic strengthening from Early Old Swedish to Late Old Swedish. Both in the case of fronted CPs and fronted PPs, we see that the amount of {\temporal} antecedents increases. 

In the case of \textsc{så} resumption, the picture is the opposite. Resumption with \textsc{så} starts out after fronted CPs expressing condition or comparison and gradually takes on more contexts. We find the following evolution \REF{ex:nilsen:49}.



\begin{exe}
\ex\label{ex:nilsen:49} condition/comparison > time > concession > cause
\end{exe}

The examples following PPs are so few it is impossible to draw an conclusion, and in the case of adverbial antecedents, they do not appear until Late Modern Swedish, and then it is possible to find all kinds, both specific, general and nynegocentric ones. 

A brief sideglance to Ekblad's letters does, however, point to a certain development. Resumption occurs after fronted PPs and AdvPs in these texts, and the antecedents typically express some sort of aboutness topic `with respect to'. There are no instances of sentence adverbials or nynegocentric antecedents in these letters.

\begin{exe}

    \ex\label{ex:nilsen:50}
    
    \gll Anbelangandes iagten min kiäre bror haar warit i hosz hården med de små hundarna	\textbf{så} haar	iagh här sett en Copie aff en sådan iagdt. \\
    with.respect.to hunt my dear brother has been in by hoard with the small dogs \textsc{så} have I here seen a copy of a such hunt\\
    \glt `With respect to the hunt my dear brother has taken part in, with the hoard with the small dogs, I have seen a copy of such a hunt here.' \begin{flushright}
        (EMS, Ekblad 06. Oct. 1652)
    \end{flushright}

\end{exe}




We may therefore conclude that Meklenborg's cline holds in that there is a clear semantic development in the CP antecedent. With respect to adverbial antecedents, it is necessary to conduct a thorough analysis of the Early Modern Swedish period in order to get a more precise picture. 




\section{Structural changes}\label{structural}

In his quite remarkable thesis, Bergqvist observes some of the changes we have discussed so far \citep{Bergqvist}. He suggests that the starting point for the fronted conditional subordinate clause was parataxis. When hypotaxis evolved, it happened by means of \textsc{då} inserted in the initial position of the main clause. He identifies four stages in the evolution towards the modern system. In the first stage, the main clause has the form of an ordinary clause, typically S--V. In the second stage, the main clause is introduced by \textsc{då} with subject inversion. In the third stage, both the embedded and the main clause are introduced by \textsc{tha}, and in the final stage only the embedded clause retains \textsc{då} \citep[24]{Bergqvist}. This aligns neatly with what we have seen so far.


\citet{Kiparsky-95} has made a similar observation for subordinate clauses in  Proto-Indo-European. He suggests these were \emph{adjoined}, not \emph{syntactically} embedded. Syntactic embedding only evolved later in the  daughter languages. Kiparsky proposes a model where there are two positions to the left of \textsc{comp}, namely a focus and a topic position. Movement into the focus position would generate V2 word order, while elements in the topic position would give V3. He suggests that in the old languages, the adjoined, fronted adverbial clauses occupied the initial slot. The model is presented in Figure \ref{Kip1}.

\begin{figure}
\begin{forest}
[CP
  [TOPIC [XP\indi]]
  [CP
      [SPEC [XP\indj]]
      [C' 
          [COMP [V\indk]]
          [S [\dots{} \emph{t\indj} \dots{} (pro\indi) \dots{} \emph{t\indk},roof]]
      ]
  ]
]
\end{forest}
% \Tree 
% [.CP
% [.{TOPIC} {XP\indi}  ]
% [.CP
% [.{SPEC} {XP\indj} ]
% [.C' 
% [.{COMP} {V\indk} ]
% \qroof{\dots \emph{t\indj} \dots (pro\indi) \dots \emph{t\indk} \dots}.S ] ] ] 
\caption{Kiparsky's model}
\label{Kip1}
\end{figure}


The implication of Kiparsky's analysis is that the fronted CP is a topic that cannot count as the initial element in a V2 clause. In this case, some other element must occupy the first slot. This approach readily explains why \textsc{då} comes to become a resumptive element. Originally a $[$+ \textsc{temporal}$]$ adverb, it is versatile and lends itself naturally to this placeholder function in the sentence. We use the clause in (\ref{syn}), repeated as \REF{ex:nilsen:51} to exemplify the structure. The formal structure is presented in Figure \ref{Kip}.

\ea\label{ex:nilsen:51}  Værþær han goþær kunungær. þa lati	guþ han længi liwæ
\ex\label{ex:nilsen:52} $[$\textsubscript{CP} Værþær han goþær kunungær]. [\textsubscript{V2} \textbf{þa} lati	guþ han længi liwæ]
\z

\begin{figure}
\begin{forest}
[CP
  [CP [Værþær han goþær kunungær, roof]]
  [CP
    [\textsc{då}]
    [C'
      [COMP [lati\indi]]
      [S [guþ \emph{t}\indi han længi liwæ,roof]]
    ]
  ]
]
\end{forest}
% % % \Tree 
% % % [.CP
% % % [\qroof{Værþær han goþær kunungær}.CP  ]
% % % [.CP
% % % [.{\textsc{då}}  ]
% % % [.C' 
% % % [.{COMP} {lati\indi} ]
% % % \qroof{guþ \emph{t}\indi han længi liwæ}.S ] ] ] 
\caption{Kiparsky's model exemplified}
\label{Kip}
\end{figure}


If Kiparsky's model does indeed extend to Old Swedish,  we may make two predictions: \REF{ex:nilsen:1} that any XP may follow the initial CP: the resumptive, the subject or any other XP, generating an ordinary V2 structure; and \REF{ex:nilsen:2} that there are very few, if any, occurrences of V1 clauses following the initial CP. In other words, the CP may not count for V2 and as such it occurs outside the clausal core. These predictions are only partially borne out.


We do find  examples where the subject intervenes between a fronted CP and the finite verb, such as in \REF{ex:nilsen:53}. This corresponds to what Bergqvist describes as the initial phase of the development in Swedish \citep[11--12]{Bergqvist}. These constitute clear indications that in the earliest sources, the CP may still occupy a left peripheral position.\footnote{Interestingly, this structure is not attested in Old Norse \citep{Swan}, pointing to a certain divergence between Old Norse and Old Swedish.} The structure is paratactic in the sense that it semantically speaking is a subordinate clause, but it is not fully integrated into the clause.



 \begin{exe}
 
 \ex\label{ex:nilsen:53}
\gll  Taker man annæn oc leþir til stok. hogger af hænder eller föter. þet ær kununx ezöre.\\
 take man other and lead to trunk chop off hands or feet that is king's oath.swearing\\
 \glt `If a man takes another and leads him to a tree trunk and chops off his hands or feet, then it is a breach of the King's peace.'\label{king} \begin{flushright}
   (EOS, SöL, 1330--1400)  
 \end{flushright}
 
  \begin{xlist}
     
\ex
$[$\textsubscript{CP} Taker man annæn oc leþir til stok. hogger af hænder eller föter$]$ [\textsubscript{V2} þet ær kununx ezöre]





 \end{xlist}
 
\end{exe}

\begin{figure}
\begin{forest}
[CP
  [CP
    [TAker man annæn oc leþir til stok. \\ hogger af hænder eller föter, roof,tier=word]
  ]
  [CP
    [þet\indi]
    [C'
      [COMP [ær\indj,tier=word]]
      [S
        [\emph{t}\indi \emph{t}\indj kunux ezöre,roof,tier=word]
      ]
    ]
  ]
]
\end{forest}
% % % \Tree 
% % % [.CP
% % % [\qroof{TAker man annæn oc leþir til stok. \\ hogger af hænder eller föter}.CP  ]
% % % [.CP
% % % [.{þet\indi}  ]
% % % [.C' 
% % % [.{COMP} {ær\indj} ]
% % % \qroof{\emph{t}\indi \emph{t}\indj kunux ezöre}.S ] ] ] 
\caption{Fronted CP preceding a S--V structure}
\label{hogger}
\end{figure}




This way, we may claim that the first prediction is borne out. As for the other one, it is less clear. It is fairly common to find structures where the fronted CP is followed by the finite verb, as in example \REF{ex:nilsen:54}. If the CP is not subordinated and integrated into the central clause (\ref{drepe-1}), it is necessary to assume that the main clause has a V1 structure.




 \begin{exe}
 
\ex\label{ex:nilsen:54}
\gll Dræpæ flere mæn en man vm suknudagh böten alle .III. marker biscupi. \\
 kill more men one man on {week.day} pay all three marks bishop\\
\glt  `If several men kill one man during the week, all must pay three marks to the bishop.' \hfill 
   (EOS, SöL, 1330--1400)  


 \begin{xlist}
     

\ex  \label{drepe-1} $[$\textsubscript{V2} $[$\textsubscript{XP} Dræpæ flere mæn en man (...) $[$\textsubscript{V} böten 	$[$\textsubscript{S} alle  $[$.III. marker$]$$]$$]$




 \end{xlist}
 
 \end{exe}


\begin{figure}
\begin{forest}
[CP
  [CP [Dræpæ flere mæn en \\ man vm suknudagh, roof, tier=word]]
  [C' 
      [C0 [böten\indj,tier=word]]
      [IP [alle\indi, tier=word]
          [I' [\emph{t\indi}  \emph{t\indj} .III. marker,roof]]
      ]
  ]
]
\end{forest}
% % % \Tree 
% % % [.CP
% % % [. \qroof{Dræpæ flere mæn en \\ man vm suknudagh}.CP  ]
% % % [.C'
% % % [.{C0} {böten\indj} ]
% % % [.IP 
% % % [.{alle\indi}  ]
% % % \qroof{\emph{t\indi}  \emph{t\indj} .III. marker}.I' ] ] ] 
\caption{Integrated fronted CP}
    \label{ifc}
\end{figure}







Corpus searches for fronted adverbials or PPs followed by the subject have not generated any tokens. I take this as an indication that these are integrated into the clause. This way, we may assume that in Early Old Swedish,  fronted CPs may be clause-external and clause-internal, while fronted PPs or ADVPs are clause-internal only. 


\subsection{Resumption in a cartographic model}\label{cartmod}


Transposing Kiparsky's observations to the framework developed by \citet{Ben-Pol} (see \sectref{V2}), we would say that the fronted CP would occupy either a Hanging Topic or a Scene Setting position, located to the left of the projection associated with V2. 



\begin{figure}
\begin{floatrow}
\ffigbox{\begin{forest} for tree = {nice empty nodes}
[HTP
    [] [SSP
        [] [ForceP
            [] [TopicP
                [] [FocusP
                    [] [FinP
                        [XP (\indi)]
                        [Fin'
                           [Fin\zero [V]]
                           [IP
                               [\dots{} (\emph{t}\indi) \dots,roof]
                           ]
                        ]
                    ]
                ]
            ]
        ]
    ]
]
\end{forest}}
{\caption{The cartographic model}\label{cartog}}
% % \Tree 
% % [.{HTP}
% % [.{}   ]
% % [.{SSP}
% % [.{}   ]
% % [.ForceP
% % [.{}  ]
% % [.TopicP 
% % [.{}  ]
% % [.FocusP
% % [.{}  ]
% % [.FinP
% % [.{XP(\indi)}  ]
% % [.Fin'
% % [.{Fin\zero} {V} ]
% % [\qroof{\dots (\emph{t}\indi) \dots}.IP  ]
% % ] ]
% % ] ] ] ] ]

\ffigbox{%
\begin{forest} for tree = {nice empty nodes}
[FrameP
  [] [V2P
        [XP\indi]
        [V2'
            [V2\zero [V]]
            [IP [\dots{} \emph{t}\indi \dots, roof]]
        ]
  ]
]
\end{forest}}
{\caption{A simplified  model}\label{carto}}
% % \Tree 
% % [.{FrameP}
% % [.{}   ]
% % [.V2P
% % [.{XP\indi}  ]
% % [.V2'
% % [.{V2\zero} {V} ]
% % [\qroof{\dots \emph{t}\indi \dots}.IP  ]
% % ] ]
% % ] ] 
\end{floatrow}
\end{figure}

In Figure  \ref{cartog}, I suggest that the locus for V2 is FinP. Recent literature suggests that V2 may take place either in FinP or in ForceP (see among others \cite{Walkden2014, Wolfe-2019} and subsequent literature). For our purpose, the exact projection is immaterial. Thus, we may simplify the model above into the following. We lump Hanging Topics and Scene Setters together under the label Frame, and we eliminate the topic and focus projections under the assumption that V2 takes place in the specifier of the V2 projection, which serves as a bottleneck \citep{Haegeman-96} and that further movement into Topic (and possibly Focus) is strongly limited (see \cite{Grohmann-2000} and subsequent work for more details). 

Interestingly, there are examples where a topicalized DP precedes the fronted adverbial clause. I will assume that these occupy a Hanging Topic projection. 


 \begin{exe}

\ex\label{ex:nilsen:55} Hanging topics
 \begin{xlist}

    \ex
\gll Kirwil warder han stampad mz	olie \textbf{tha} duger	han för krancdom.\\
chervil becomes he crushed with oil {\textsc{då}} works he for illness\\
\glt `As for chervil, if it is ground with oil, it cures illness.' \\
$[$\textsubscript{\textsc{xp1}} kirwil] [\textsubscript{\textsc{xp2}} warder han stampad mz	olie] [\textsubscript{\textsc{res}} tha][\textsubscript{\textsc{Vfin}} duger ...] 
\begin{flushright}
    (LOS, LB8, 1500--1600)
\end{flushright}


\ex
\gll Men then som twiffuelachtigh är, äter han,	\textbf{så} är	han fördömd.\\
but the.one that doubtful is eats he {\textsc{så}} is he doomed\\
\glt `But whoever has doubts is condemned if he eats.'\\
$[$\textsubscript{\textsc{xp1}} then som twiffuelachtigh är] [\textsubscript{\textsc{xp2}} äter han ] [\textsubscript{\textsc{res}} så][\textsubscript{\textsc{Vfin}} är...] 
\begin{flushright}
(EMS, Bible, Rom 14:23, 1526)\footnote{Translation based on  the new international version of the Bible, accessed through \url{https://biblehub.com/parallel/romans/14-23.htm}}
\end{flushright}
     
 \end{xlist}
 
\end{exe}




This analysis also accounts for the fronted free relatives that we saw in Sections~\ref{frr} and \ref{fdps}, where the fronted thematic constituent is both resumed by \textsc{då} and by the pronominal anaphor. The example in (\ref{quinna}), repeated as \REF{ex:nilsen:56}, is an example of this.


  \begin{exe}
  
\ex\label{ex:nilsen:56}
\gll   Hwar som äther mykyt aff henne, \textbf{thaa} faar	han quinna lwstha.\\
who \textsc{rel} eat much of her \textsc{då} get  he women urges\\
\glt `Whoever eats  much of it [parsnip] will get a woman's urges.' \begin{flushright}  (LOS, LB7, 1500--1600)\end{flushright}\label{quinna2}


    \end{exe}
    
The structure is easily accounted for within the  model. The fronted free relative is a Hanging Topic, located in SpecFrameP. It is co-referential with the pronominal anaphor \emph{han} `he', which is in SpecTP. \textsc{Då} occupies the initial position of the V2 phrase, and the finite verb \emph{faar} `gets' has moved into V2\textsuperscript{0}.

The data strongly indicates that what we have called a resumptive so far is not what is understood by a resumptive in the modern languages. Rather than picking up the initial element of a V2 clause, the \textsc{då} resumptive in the oldest stages is an independent element that in itself satisfies the V2 constraint by occupying SpecV2P. As such, \textsc{då} is clearly a phrase. At this stage, the function of \textsc{då} resembles that of  an expletive. The analysis is presented in Figure \ref{kvinnelyst}.

\begin{figure}
\begin{forest}
[FrameP
  [CP-rel
    [Hwar som äther mykyt aff henne, roof]
  ]
  [V2P
    [thaa \indi]
    [V2'
      [V2\zero [faar\indj]]
      [IP
        [han \st{faar}\indj quinna lysta \st{thaa}\indi, roof]
      ]
    ]
  ]
]
\end{forest}
% % % \Tree 
% % % [.{FrameP}
% % % [\qroof{Hwar som äther mykyt aff henne}.CP-rel   ]
% % % [.V2P
% % % [.{thaa \indi}  ]
% % % [.V2'
% % % [.{V2\zero} {faar\indj} ]
% % % [\qroof{han \st{faar}\indj quinna lysta \st{thaa}\indi}.IP  ]
% % % ] ] ]   
\caption{Fronted free relative clause + \textsc{då} + pronominal anaphor}
    \label{kvinnelyst}
    
\end{figure}

We find clauses well into Late Old Swedish that have this structure, and we take this as evidence that the fronted subordinated clause could still be syntactically independent, e.g. that parataxis was still available. However, as we shall see in the next section, the integration of the initial clause leads to a reanalysis of the resumptive.

A natural question is why this happens. \textsc{Då} is not a general expletive, and it is an open question why the subject does not simply move to meet the V2 requirement. What in the derivation causes the subject to stop in SpecIP rather than moving to SpecV2P? If \textsc{då} were a true expletive, we would expect it to be merged in SpecV2P as a last-resort operation -- which might be the case if there were no accessible subject or any other constituent that could move. As this is clearly not the case here, it is difficult to argue that \textsc{då} is a full-blown expletive. Rather, I would suggest that at this stage, \textsc{då} is moved into the initial position of the clause, checking the relevant feature on the V2\zero head.


\subsection{Integration leading to syntactic retraction}



Both Hanging Topics and Scene Setters are merged in the high left peripheral position, and they are not accessible to the lower part of the clause, as we have seen. This explains why {\textsc{då}} assumes the role of a generalized resumptive in Early Old Swedish: it moves into initial  position of the V2 clause and is only loosely connected to the peripheral elements. When the antecedent gets integrated into the clause and acts as the initial element of the V2 clause, the situation changes radically and  resumption with \textsc{då} plummets. 


In order to understand what happens, it is necessary to take a look at a related construction, namely Left Dislocation. The link between the two structures is striking, as has been observed by several others  (see among others \cite{Altmann-81, Holmberg-1986, Holmberg-saa, Jorgensen}.) Axel labels this structure German Left Dislocation  \citep[204]{Axel}. In the Germanic V2 languages, a left-dislocated element is immediately followed by its anaphor, which in turns precedes the finite verb. An example is provided in \REF{ex:nilsen:57} where \emph{absintium} `absinth' is followed by an anaphoric pronoun \emph{thz}.

\begin{exe}
\ex\label{ex:nilsen:57} \gll Absintium thz är malyrt \\
absinth that is wormwood\\
\glt `Absinth is wormwood.' \hfill (LOS, LB3, 1400--1500)
\end{exe}


I will follow \citet{Grohmann-2000, Boeckx-Grohmann} in assuming that the anaphor is the spell-out of the trace of the topicalized DP. The V2 constraint requires overt phonetic material in the specifier of the V2P, and the spell-out of the anaphor is as such a last-resort option.\footnote{See also \citet{Grewendorf2002Left} for a slightly different analysis, where what moves through the bottleneck is a big XP, that is subsequently split into two components.} This operation assures that the bottleneck is not cracked and the effect is that there is co-reference between the two initial elements. The structure is presented in Figure \ref{kongen}.




\begin{figure}
% % %     \centering
\begin{forest} 
[Frame
  [, nice empty nodes] [ForceP
      [, nice empty nodes] [TopicP
        [absintium] [FinP
            [\st{absintium} \lra thz] [Fin'
                [Fin\zero [är]]
                [IP
                  [\st{absintium} \st{är} malyrt, roof]
                ]
            ]
        ]
      ]
  ]
]
\end{forest}
% % % \Tree 
% % % [.{Frame}
% % % [.{}   ]
% % % [.ForceP
% % % [.{}  ]
% % % [.TopicP 
% % % [.{absintium} ]
% % % [.FinP
% % % [.{\st{absintium} \lra thz}  ]
% % % [.Fin'
% % % [.{Fin\zero} {är} ]
% % % [\qroof{\st{absintium} \st{är} malyrt}.IP  ]
% % % ] ]
% % % ] ] ] 
\caption{Left-dislocated thematic element}
\label{kongen}
\end{figure}



In the case of fronting of integrated adverbial subclauses, I suggest that these are derived the same way as thematic topics. The clause moves through the SpecV2P and is then attracted to SpecTopP. As the V2 requirement demands phonetic material in its specifier position,  the trace of the clause is spelled out by \textsc{då}. As \textsc{då} is not semantically bleached, this explains the tight semantic relationship between the anaphor and \textsc{då}. In order for \textsc{då} to serve as an anaphor, it must carry the same semantic features as the antecedent. The basic semantic value of \textsc{då} is \temporal, and as such only antecedents that may be interpreted as \temporal {} can be followed by \textsc{då}. The derivation is presented in Figure \ref{anafor}. The analysis implies that \textsc{då} is a phrase. 


\begin{exe}
\ex\label{ex:nilsen:58} \gll Ok tha hon blomstras om sommaren, \textbf{tha} skal	hon wp takas ok törkas i skugganom.\\
and when she flowers in summer {\textsc{då}} shall she up take and dry in shadows\\
\glt `And when she is in bloom in summer, she should be picked and dried in the shadows.' \hfill (LOS, L3, 1400--1500)
\end{exe}

\begin{figure}
\fittable{
\begin{forest}
[Frame
  [, nice empty nodes] [ForceP
      [, nice empty nodes] [TopicP
        [tha hon blomstras\\ om sommaren]
        [FinP
          [\st{tha hon blomstras} \\\st{om sommaren} \lra \textsc{då}]
          [Fin'
            [Fin\zero [skal\indi]]
            [IP
              [hon \st{skal}\indi \dots  \st{tha hon } \\\st{blomstras om sommaren},roof]
            ]
          ]
        ]
      ]
  ]
]
\end{forest}
}
% \Tree 
% [.{Frame}
% [.{}   ]
% [.ForceP
% [.{}  ]
% [.TopicP 
% [.{tha hon blomstras\\ om sommaren} ]
% [.FinP
% [.{\st{tha hon blomstras} \\\st{om sommaren} \lra \textsc{då}}  ]
% [.Fin'
% [.{Fin\zero} {skal\indi} ]
% [\qroof{hon \st{skal}\indi \dots  \st{tha hon } \\\st{blomstras om sommaren}}.IP  ]
% ] ]
% ] ] ] 
\caption{Fronted integrated adverbial clause with resumption}
\label{anafor}
\end{figure}

We have seen that in the case of fronted PPs in Late Old Swedish, 17 of the 37 cases are [+\textsc{instrumental}]  rather than \temporal {} (see Figure \ref{PP}). This would be an argument against the analysis above. However, as we observed above in  \sectref{fropp}, these [+\textsc{instrumental}]  PPs have in common that the instrument is an action that precedes the action of the main clause and as such may be interpreted as some sort of condition. Thus, these may not be taken to constitute counter-examples to the analysis presented above.

The analysis above strongly resembles the one that \citet[186--187]{Axel-Tober2023} proposes for High German. While resumption with \emph{so} was ubiquitous in German in the 15th and 16th centuries, its use suddenly dropped. Axel suggests that this has to do with the incorporation of the fronted adverbial clause. Incorporation rules out \emph{so} resumption as \emph{so} is not semantically bleached. We see that it is exactly the same thing that happens in Swedish: once the fronted adverbial constitutent gets incorporated into the clause, a non-bleached resumptive element may only follow antecedents that match its semantics. However, in Swedish, unlike German, \textsc{så} is semantically bleached.


If we are on the right track and the integration of the initial element is what triggers semantic restriction of the antecedent of {\textsc{tha}}, one might ask why resumption is not lost altogether or, at the very least, gets restricted to temporal or conditional antecedents. Instead, what we find is that \textsc{så} resumption rises and quickly becomes the dominating resumptive element. While \textsc{då} remains a specialized resumptive, \textsc{så} becomes a fully generalized one.

One obvious reason for the rise of \textsc{så} is that the language already has access to the resumptive structure as such. When \textsc{då} no longer is accessible due to its semantics, the language swaps it for \textsc{så}. This, however, could only happen if \textsc{så} was semantically bleached. We saw in \tabref{CP} that when fronted adverbial clauses get integrated into the clause, in Late Old Swedish, almost 90\% of cases of \textsc{så} resumption  took place after fronted conditional and comparative clauses, which were the only contexts \textsc{så} resumption was found in Early Old Swedish. Only 2 clauses do not fall into these two categories, and these are temporal. Based on this, we may draw two conclusions: \REF{ex:nilsen:1} Fronted adverbial clauses get integrated into the main clause in Late Old Swedish; and \REF{ex:nilsen:2} The resumptive element is not semantically bleached. This also explains why the use of resumptive elements drops in the Old Swedish period.

When we enter the period of Modern Swedish, this changes. \textsc{Så} follows a variety of different semantic antecedents, and as such it appears to be semantically bleached. This is, in other words, a process that takes place no later than the late 15th or early 16th century. We can postulate that the bleaching of \textsc{så} took place as an internal change in the language, but it is tempting to point to two external factors that may have contributed to the bleaching and subsequent rise of \textsc{så} as a resumptive: phonetic changes, and the influence from German. In the following, I will sketch out some hypotheses as to why this change could take place.

\subsection{Phonetic collapse}
In Late Old Swedish, the siblants /$\theta$/ and /\dh/  are reduced to /t/ and /d/. However, building on medieval letters, \citet{th} finds that the change first is reflected in the written language in the second half of the 14th century, but that it did not become prominent until the 15th century. The findings support \citegen[202--203]{Noreen1904} claim that the process was over by 1450.

The data examined by \citet{th} further indicate that there was substantial confusion with respect to the etymology and thus the correct orthography of certain words. In the late Middle Ages, words that had never contained a /$\theta$/ would be written \emph{th} as if to indicate a lost sibilant~-- and the other way around. 

It is reasonable to presume that in the midst of this process, there was a certain confusion with respect to the different sounds, and the articulation point of /$\theta$/ and /\dh/ are close to that of /s/. This means that the language users may not always have felt the difference between \textsc{då} and \textsc{så} during this transition period. This would ease the transition from \textsc{då} to \textsc{så} resumption, but it would not explain the semantic bleaching.


\subsection{Influence from German?}

During the late Middle Ages, the influence from the Germanic Hansa was important all over Scandinavia, and their presence also came to change aspects of the language (see, among others, \citealt{torp-nordisk}). 

Therefore, it is not unlikely that  the language also borrowed \textsc{så} as its go-to resumptive.  As already mentioned, generalized resumption with \textsc{so} was universally accessible in both High and Low German, and the structure is attested as early as the 9th century. 


\begin{exe}
  
\ex\label{ex:nilsen:59} Resumption with \textsc{so} in Older German

\begin{xlist}

    \ex
\gll Áfter  mír  \textbf{so}  químit  er.\\
after me \textsc{so} comes he\\
\glt `He comes after me.' \hfill (Old High German, Otfr.ev. 1.27) 

\ex
\gll Morgen zu mittag so  solt ir mit mir essen in Jeronimus Roberti hauß.\\
 morning to dinner \textsc{so} shall you with me eat in Jeronimus Robert's house\\
 \glt `Tomorrow you will dine with me in Jeronimus Robert's house.' 
\begin{flushright}
(from  \cite[203]{Thim-Mabrey})
\end{flushright}

\end{xlist}

\end{exe}



Vasa's translation of Luther's Bible is the first attestation of \textsc{så} as a generalized resumptive. While the readings from the Bible in the Catholic mass were held in Latin, the new Protestant readings were in the vernacular language, in other words in Swedish. Therefore, linguistic use in the Bible would possibly have an important impact. Even if \textsc{så} resumption was possible, but marginal, by the time the translation was conducted, the use of \textsc{så} in the Bible would lead to the spread of this construction in the spoken language. It is thus interesting to compare the quotes below. The translations are from the new international version of the Bible, accessed through \url{biblehub.com}.


 \begin{exe}
 

\ex\label{ex:nilsen:60}
\begin{xlist}
\ex So ich aber mit Zungen bete, \textsc{so} betet mein Geist. \hfill (Luther)
  
\ex  När iagh nu bedher medh tungonne, {\textsc{så}} bedher min ande.\hfill Vasa\\
`For if I pray in a tongue, my spirit prays.' \hfill (1 Corinthians 14:14)
\end{xlist}

  
 \ex\label{ex:nilsen:61}
\begin{xlist}
\ex      Bistu aber los vom Weibe, \textsc{so}  suche kein weib.
  \hfill (Luther)
 \ex  Ästu ock vthan hustru, {\textsc{så}} begära icke hustru.\hfill (Vasa)\\
 `Are you pledged to a woman? Do not seek to be released.' \begin{flushright}
(1 Corinthians 7:27)
\end{flushright}
\end{xlist}

\end{exe}

The interesting question is of course if the translation mimics the German original also in places where the spoken language at the time would not use these structures. If this is the case, the text is leading the way towards the  change. I leave this question for future research.

However, it is unlikely that the influence from Vasa's Bible translation would  have led to the introduction of \textsc{så} as a resumptive unless the language already had access to the structure. It could  have encouraged the changes that were taking place, as such lending a hand to \textsc{så} on its way towards its status as a completely bleached resumptive element.





\section{Conclusion} \label{konklusjon}

This paper has examined the evolution of the resumptive structure in Swedish, from the 13th century to the first half of the 18th. We have seen that there have been important changes in the resumptive structures, where \textsc{då} resumption is dominant in the first period, before \textsc{så} resumption takes over from the 16th century. 

Two different mechanisms account for the changes we observe. First, in the earliest sources, the fronted subordinate clause was not integrated into the main clause, and the resumptive  \textsc{då} functions as an expletive in order to ensure the V2 word order in the main clause. At this stage \textsc{då} is a full phrase that is moved into initial position of the V2 clause. In Late Old Swedish, the fronted adverbial clause gets integrated into the clause, and there is a reanalysis in which the resumptive is the spell-out of the trace of the fronted constituent. This way, the resumptive must map the semantics of the antecedent; in other words there must be semantic identity between the two constituents preceding the finite verb. Therefore, the use of a specialized resumptive such as \textsc{då} gets heavily restricted. There is no indication that \textsc{så} was a specialized resumptive in Late Old Swedish, but data from Early Modern Swedish suggest that \textsc{så} had undergone bleaching. This in turn led to syntactic and semantic expansion, where \textsc{så} was admitted into all contexts as a resumptive.

In this paper, we set out to discuss Meklenborg's clines. Given what we have seen, we may not observe any cline with respect to \textsc{då} resumption. \textsc{då} is a specialized resumptive from the start, and it never undergoes semantic expansion. Once its antecedent gets integrated into the clause, its usage gets strongly limited.

With respect to \textsc{så}, there is indeed an evolution that seems to support Meklenborg's clines. However, we have seen that the evolution of the resumptive is closely linked to underlying changes in derivation of the clausal structure in Swedish. 


\section*{Abbreviations}\label{abbr}
The references to source texts are the following:

\begin{tabbing}
MMMM \= Bjärköarätten\kill
BÖR \> Bjärköarätten 1300--1350\\
Con-SL \> Confirmation of the Södermannalag, 1330–1400\\
DL \> Dalalagen, 1318--1335 \\
EMS \> Early Modern Swedish\\
EOS \> Early Old Swedish\\
HäL \> Hälsingelagen, 1350--1400\\
KL \> Kristoffers Landslag, 1442\\
LB \> Läkebok (the following number indicates the volume)\\
LMS \> Late Modern Swedish\\
LOS \> Late Old Swedish\\
MES \> Magnus Erikssons Stadslag, 1350--1450\\
ÖGL \> Östgötalagen, 1350--1400\\
SL \> Skånelagen, 1203--1212\\
SöL \> Södermannalagen, 1330--1400\\
ULL \> Upplandslagen, 1300--1399\\
Vid-an \> Vidhemsprästens anteckningar, 1280–1289\\
YVGL \> Yngre Västgötalagen, 1280--1399\\
\end{tabbing}

\section*{Acknowledgements}
I would like to thank the audience at the Cracks in the Bottleneck workshop in Paris 16--17 February 2023. I am most grateful for the comments from my anonymous reviewers, which have helped develop the analysis and the paper. Ivar Berg and Elise Kleivane have helped me with particularly tricky examples, for which I am thankful. Finally, I would like to thank the editors for the efforts they have put into this publication. 

\printbibliography[heading=subbibliography,notkeyword=this]
\end{document}
