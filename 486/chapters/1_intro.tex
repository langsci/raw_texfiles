\documentclass[output=paper,colorlinks,citecolor=brown]{langscibook}
\ChapterDOI{10.5281/zenodo.17077182}
\author{Sarah Harchaoui\orcid{}\affiliation{Sorbonne Université} and Pierre-Yves Modicom\orcid{}\affiliation{Université Jean Moulin Lyon 3}}
\title{Introduction: Verb-third phenomena in Germanic verb-second languages}
\abstract{\noabstract}

%move the following commands to the "local..." files of the master project when integrating this chapter\usepackage{tabularx}

\IfFileExists{../localcommands.tex}{
   \addbibresource{../localbibliography.bib}
   % add all extra packages you need to load to this file

\usepackage{tabularx,multicol}
\usepackage{url}
\urlstyle{same}

\usepackage{listings}
\lstset{basicstyle=\ttfamily,tabsize=2,breaklines=true}

\usepackage{langsci-basic}
\usepackage{langsci-optional}
\usepackage{langsci-lgr}
\usepackage{langsci-osl}
% \usepackage{./langsci/styles/langsci-lgr}
% \usepackage{./langsci/styles/langsci-osl}
% \usepackage{langsci-gb4e}

\usepackage{tikz}
\usetikzlibrary{patterns,calc}
\pgfdeclarepatternformonly{south east lines}{\pgfqpoint{-0pt}{-0pt}}{\pgfqpoint{3pt}{3pt}}{\pgfqpoint{3pt}{3pt}}{
    \pgfsetlinewidth{0.6pt}
    \pgfpathmoveto{\pgfqpoint{0pt}{3pt}}
    \pgfpathlineto{\pgfqpoint{3pt}{0pt}}
    \pgfpathmoveto{\pgfqpoint{.2pt}{-.2pt}}
    \pgfpathlineto{\pgfqpoint{-.2pt}{.2pt}}
    \pgfpathmoveto{\pgfqpoint{3.2pt}{2.8pt}}
    \pgfpathlineto{\pgfqpoint{2.8pt}{3.2pt}}
    \pgfusepath{stroke}}
    
\usepackage{stmaryrd}
\usepackage{wasysym}
\usepackage{multirow}
\usepackage{caption}
\usepackage{subcaption}
\usepackage{mathrsfs}
\usepackage{qtree}

\usepackage{linguex}


   %pminos do not split footnotes
% \interfootnotelinepenalty=10000 %Footnote in Laporte chapters has to be split SN


%\DeclareIndexNameFormat{default}{%
%\nameparts{#1}%
%\usebibmacro{index:name}%
%{\index[names]}%
%{\namepartfamily}%
%{\namepartgiveni}%
% {}% L1
% {}% L2
%{\namepartprefix}% generates spurious space L3
%{\namepartsuffix}% generates spurious space L4
%}

%  {\DeclareIndexNameFormat{default}{%
%     \usebibmacro{index:name}{\index[names]}{#1}{#3}{#5}{#7}}}

%\DeclareIndexNameFormat{default}{%
%  \usebibmacro{index:name}{\sindex[nom]}{#1}{#3}{#5}{#7}}

%\DeclareIndexNameFormat{default}{%
%  \usebibmacro{index:name}{\sindex[person]}{#1}{#3}{#5}{#7}}
%\DeclareIndexNameFormat{default}{%
%\nameparts{#1} \usebibmacro{index:name}{\sindex[person]]}{\namepartfamily}{‌​\namepartgiven}{\nam‌​epartprefix}{\namepa‌​rtsuffix}}

%\newcommand{\smiley}{:)}

%\renewbibmacro*{index:name}[5]{%
%\usebibmacro{index:entry}{#1}%
%{\iffieldundef{usera}{}{\thefield{usera}\actualoperator}\mkbibindexname{#2}{#3}{#4}{#5}}}

% \newcommand{\noop}[1]{}

%remove for final
%\overfullrule=1mm

\newcommand{\tobi}[2]}}
\renewcommand{\S}[1]{\tobi{#1}{\textsc{*}}}

% this volume references
% puts: [this volume]
% already defined: \citetv
%\newcommand{\citepv}[1]{(\citeauthor{#1} \citeyear*{#1} [this volume])}
\newcommand{\citealtv}[1]{\citeauthor{#1} \citeyear*{#1} [this volume]}

%parentheses around example number
\newcommand{\pref}[1]{(\ref{#1})}

% in-text examples

\newcommand{\lnex}[1]{\textit{#1}} %target lang word
\newcommand{\lnlit}[1]{(lit.: `#1')} %literal reading
\newcommand{\lnlat}[1]{(#1)} % latinization
\newcommand{\lntrans}[1]{`#1'} %translation
\newcommand{\lnexl}[2]%
{\lnex{#1}{} \lnlat{#2}} % ex with latinization
\newcommand{\lnexlat}[3]{\lnex{#1}{} \lnlat{#2}{} \lntrans{#3}} % ex with latinization and tranl.

%ch01
\newcommand{\co}[1]{\mbox{\textbf{#1}}}

%ch09

\newcommand{\cyrbulg}[1]{\begin{otherlanguage*}{bulgarian}#1\end{otherlanguage*}}


%ch10
\newcommand{\nlp}{{\small NLP}}
\newcommand{\mwe}{{\small MWE}}
\newcommand{\rae}{{\small RAE}}
\newcommand{\lvc}{{\small LVC}}
\newcommand{\pos}{{\small P}o{\small S}}
%\newcommand{\todo}[1]{ \textcolor{red}{#1} }

%\renewcommand{\labelenumi}{\theenumi}
%\ainamefmt{{vv}{ll}{, ff}{, jj}} % fullname

\newcommand{\biberror}[1]{{\color{red}#1}}

\newcommand{\osenovaitem}{--~}
   %% hyphenation points for line breaks
%% Normally, automatic hyphenation in LaTeX is very good
%% If a word is mis-hyphenated, add it to this file
%%
%% add information to TeX file before \begin{document} with:
%% %% hyphenation points for line breaks
%% Normally, automatic hyphenation in LaTeX is very good
%% If a word is mis-hyphenated, add it to this file
%%
%% add information to TeX file before \begin{document} with:
%% %% hyphenation points for line breaks
%% Normally, automatic hyphenation in LaTeX is very good
%% If a word is mis-hyphenated, add it to this file
%%
%% add information to TeX file before \begin{document} with:
%% \include{localhyphenation}
\hyphenation{
    Beck-man
    Ngu-yen
    back-chan-nel
    back-chan-nels
    mo-not-o-nous
    ste-reo-typ-i-cal
}

\hyphenation{
    Beck-man
    Ngu-yen
    back-chan-nel
    back-chan-nels
    mo-not-o-nous
    ste-reo-typ-i-cal
}

\hyphenation{
    Beck-man
    Ngu-yen
    back-chan-nel
    back-chan-nels
    mo-not-o-nous
    ste-reo-typ-i-cal
}

   \boolfalse{bookcompile}
   \togglepaper[23]%%chapternumber
}{}


\begin{document}
\maketitle

\section{Verb-third: A regular feature of Germanic V2 languages}

All Present-Day Germanic languages except English and its varieties display some kind of verb-second (V2) rule. This rule states that the finite verbal form has to be placed in the second position of the clause, at least in declarative utterances and \textit{wh}{}-questions. This means that a declarative utterance can begin with the subject, as in example \REF{ex:intro:1} from Danish, or by an adverbial, as in \REF{ex:intro:2} and \REF{ex:intro:3}. In \REF{ex:intro:2}, we find a temporal subordinate clause and in \REF{ex:intro:3}, an adverb with a discourse function.

\begin{exe}
\ex Danish (Danmarks Radio, \textit{Tirsdag med sommer og sol}, Aug. 6th, 2024)
\ea \label{ex:intro:1}En rest af regnvejret vil dog ligge omkring Bornholm 
\glt `However, a remnant of the rainy weather will lie around Bornholm.'

\ex \label{ex:intro:2}Når regnen har passeret, kommer der en opklaring.
\glt `When the rain has passed, there will be a clearing.'

\ex \label{ex:intro:3}Generelt har det været vådt denne sommer.
\glt `In general, the weather has been wet this summer.'
\z
\end{exe}

All three examples were taken from the same weather report, where we can also find another phenomenon typical for V2 languages: the use of a placeholder in the pre-finite position, in order to maintain an indefinite subject as low as possible in the clause:

\ea Danish (Danmarks Radio, \textit{Tirsdag med sommer og sol}, Aug. 6th, 2024)\\
\gll Der ligger nemlig skybrud og lurer.\\
     There lies namely cloudburst and lurk\\
\glt `There are cloudburst lurking.'
\z

It is slightly rarer to find arguments other than the subject in the pre-finite position, at least in Scandinavian. But that is still possible, especially when the fronted argument is a pronoun:

\ea Danish (Danmarks Radio, \textit{USA's udenrigsminister: Våbenhvileforhandlinger i sidste fase}, Aug. 7th, 2024)\\
\gll Det fortæller USA's udenrigsminister Anthony Blinken på et pressemøde natten til onsdag.\\
It claims USA-\textsc{gen} foreign-minister Anthony Blinken at a press-meeting night-\textsc{def} to Wednesday\\
\glt `That's what the US Secretary of State Anthony Blinken said during a press briefing on Tuesday night.'
\z

The present volume goes back to a conference held in Paris in February 2023 and devoted to violations of the V2 rules across Germanic languages, as exemplified by the following examples.

\begin{itemize}
    \item Resumption of an adverbial clause, with both the clause and the resumptive placed before the finite verb. For instance, in following sentence from a Danish weather report, the conditional clause in \textit{som om} `as if' is resumed by the adverbial pronoun \textit{så} in the pre-finite slot:

\begin{exe}
   \ex Danish (Danmarks Radio, weather report for Sept. 9th, 2024)\\
   Og som om kraftig regn, skybrud og torden ikke var nok, \textit{så} \textit{bliver} det også noget blæsende i dag.
   \glt `And as if heavy rain, cloudbursts and thunder were not enough, it will also be somewhat windy today.'
\end{exe}

\item Resumption of a non-clausal adverbial. For example, in Swedish, \textit{så} can also resume a prepositional phrase introduced by \textit{trots} `despite':

\begin{exe}
    \ex Swedish (Sveriges Radio, \textit{Så gör en miljon tyskar sin egen el med balkongkraftverk}, July 16th, 2024)\\
    Trots deras framgång i Tyskland \textit{så är} det en helt annan historia i Sverige, där balkongkraftverk belagts med säljförbud av ansvarig myndighet.
    \glt `In spite of their success in Germany, it is a totally different story in Sweden, where balcony power plants are banned from sale by the responsible authority.'
\end{exe}

    \item Stacking of arguments and verbal modifiers. In some contexts, several arguments and adjuncts depending from the same verb can be stacked in the pre-finite slot. This phenomenon has been identified in Standard German, where it is anything but rare. It seems, however, that the order of these dependents has to be what it would also have been in a subordinate clause, suggesting that this construction involves the movement of one and only one syntactic node, albeit of a higher level than the usual type of single-constituent fronting. 

    \ea German (ORF, \textit{Neun Parteien treten österreichweit an}, Aug. 2nd, 2024)\\
    Bundesweit auf dem Stimmzettel \textit{stehen} die fünf Parlamentsparteien ÖVP, SPÖ, FPÖ, Grüne und NEOS, sowie KPÖ, Bierpartei, Liste Petrovic und die Partei “Wandel”, die als “KEINE” in das Rennen geht.
    \glt `The five parliamentary parties ÖVP, SPÖ, FPÖ, Greens and NEOS, as well as KPÖ, Beer Party, the Petrovic list and the Party “Change”, running this time as “NONE”, have national ballot access.'
    \z

In that example, both \textit{bundesweit} `confederacy-wide, nationwide' and \textit{auf dem Stimmzettel} `on the ballot paper' are dependents of \textit{stehen} `stand' and could fill the pre-finite slot alone. However, \textit{bundesweit auf dem Stimmzettel stehen} can be regarded as a partially lexicalized expression (`to have nationwide ballot access'), so that the apparent multiple fronting can be explained via the movement of the whole complex, in order to leave the highly rhematic subject alone in post-verbal position.

\item Placement of a framesetting adverbial before the subject, with both phrases standing before the finite verb. While this pattern is well-described in many V2 varieties with pronominal subjects, it is also attested, at least sporadically, with full NP subjects, as in the following example, where \textit{im Wagen} `in the cab' is followed by \textit{die Kinder} `the children' and then by the finite verb.

\begin{exe}
    \ex German (Ernst Toller, \textit{Eine Jugend in Deutschland}, p. 25)\\
    Ich sitze neben dem Kutscher, im Wagen die anderen Kinder \textit{singen} und sind fröhlich, ich will nicht wie sonst die Zügel der Pferde führen.
    \glt `I'm sitting next to the driver, in the cab the other children are singing and having fun, I don't want to hold the reins as usual.'
\end{exe}
\end{itemize}

Even though the general picture of V2 holds for all Germanic languages except English, the exact contours of the V2 rule vary strongly from one Germanic language to the other, and the different possibilities of verb-third bare witness of this variation. 

The “bottleneck” (originally a theory-internal label, going back to \citealt{Haegeman1996}) demanding that one and only one constituent be placed before the finite verb is not equally respected in all Germanic varieties. This casts doubts on whether there is really such a thing as “the V2 rule” in Germanic: previous research on Medieval Romance languages already showed that at least two types of V2 rules should be distinguished \citep{Wolfe2020}. Previous cross-Germanic insights into these V3 phenomena can be found, among others, in \citet{FreywaldEtAl2015}, \citet{Walkden2017Language}, \citet{AlexiadouLohndal2018}, or in some of the studies collected in \citet{MeklemborgWolfe2021}. Within the framework of (cartographically-oriented) Generative Grammar, these two types have been modelled in terms of “phase edge”, with “FinV2” being opposed to “ForceV2”. FinV2 can be shown to license more “verb-third” phenomena than “ForceV2”. Independently of theory-internal conceptualizations, it seems that one type of V2 is more closely associated with illocutionary force than the other, and allows less exceptions. 

\subsection{What should count as verb-third?}

Further phenomena are regarded as examples of verb-third by some scholars but not by the majority and will be left out of consideration in the main body of this volume. For the sake of clarification, the most two important ones are presented here.

\subsubsection{Fronting in embedded contexts}

Germanic languages display V2 only in matrix clauses or as a “matrix clause phenomenon” in illocutionary autonomous utterances (e.g. High German or Dutch). However, Mainland Scandinavian languages or Yiddish also follow a superficial V2 rule in subordinate clauses, albeit of a very different kind from what can be observed in main clauses: non-matrix V2 suffers more exceptions than what can be seen in matrix clauses in the same language. The issue of embedded V2 is a much debated one in Nordic languages, which display various possibilities of verb placement in subordinate clauses (see \citealt{Julien2007}, \citealt{Petersson2009}, \citealt{WiklundEtAl2007}, \citealt{Vikner2020}, among many others). 
 
In Mainland Scandinavian languages, we expect subordinate clauses to exhibit an additional slot for adverbs between the subject and the verb. We can see this in the next example from Swedish, where the complementizer \textit{at} introduces a subordinate clause whose first constituent is an expletive (\textit{der}), followed by an adverbial (\textit{for alvor} `for real') and then the finite verb \textit{er} `is':

\begin{exe}
    \ex Danish (DR, \textit{Maersk har igen sat damp under kedlerne med milliardoverskud}, Aug. 7th, 2024)\\
    Det er særligt i andet kvartal, at der for alvor er kommet damp under kedlerne igen, efter at Maersk havde underskud i de sidste tre måneder af 2023.
    \glt `It is especially in the second quarter that steam has really come under the boilers again, after Maersk had a loss in the last three months of 2023.'
\end{exe}

Even though it is sometimes claimed that some adverbs, most prominently the clausal negation, have to be placed in that intermediate slot, it is quite easy to find counterexamples, suggesting that the S~-- NEG~-- V order in Scandinavian subordinate clauses is a preference, but not a grammatical rule:

\begin{exe}
\ex Norwegian (\citealt{Julien2007})\\
    Madsvåg mener at framtiden kan ikke alle drive med alt.
    \glt `Madsvåg thinks that in the future, not everyone will be able to cope with everything.'
\end{exe}


At least in some Scandinavian languages, it seems that the pre-finite slot of the subordinate clause needs not be occupied by the subject, so that a unified analysis or embedded and non-embedded verb-fronting is possible (in other words: verb-fronting in embedded clauses is V2, not SV). Even in languages allowing this phenomenon, stylistic fronting is not possible in all subordinate clauses and should rather be regarded as a main-clause phenomenon licensed in illocutionarily autonomous subordinate clauses.

\begin{exe}
\ex Danish \citep[377]{Vikner2020}
\ea[*]{
  At \textit{tomater} spiser Johan ofte overrasker de fleste.
  \glt (intended: `That John often eats tomatoes surprises most people.')
}

\ex[]{ Hun siger at \textit{tomater} spiser Johan ofte.
\glt `She says that John often eats tomatoes.'}

\ex[]{ Hun siger at Johan ofte \textit{spiser} tomater.
\glt `She says that John often eats tomatoes.'}
\z
\end{exe} 

This distribution is reminiscent of classical claims about V2 in non-embedded contexts, according to which the filling of the pre-finite slot is driven by functional pragmatic features.

It is beyond any doubt that the type of verb fronting observed in Scandinavian subordinate clauses is quite different from what can be found in main clauses in the same languages. Still, this type of alternation between SV and SXV may be motivated by information structure, as argued, among others, by \citet{dalberg2023}, who claims that the left half of Danish subordinate clauses mirror the structure of the middle field (the traditional \textit{neksusfelt}) of main clauses, devoted to topical and thematic material, whereas the right part of subordinate clauses mirror the structure of the “content field”, that is the part of the main clause that follows a VO principle and hosts rhematic material. In other words: the superficial violation of an apparent V2-constraint bears witness of fine-grained differences between verb-placement rules and has consequences for the whole topological model of the clause. This is a common experience taught by the study of V2 main clauses and of fronting in embedded clauses. But because the mechanisms behind embedded fronting are not the same as those behind matrix V2 in the languages which display both phenomena, at this stage of the scientific discussion, they cannot be amalgamated. For that reason, these matters will be let out of consideration in the rest of the present volume.

\subsubsection{Focus-sensitive particles}

Another typical example of controversial constituent order is the behaviour of focus-sensitive adverbials and particles taking another constituent in their scope and coexisting with that constituent in the pre-finite slot: 

\begin{exe}
\ex Dutch \citep[51]{FoolenEtAl2009}\\
    Zelfs Jan \textit{was} gisteren aanwezig.
\glt `Even Jan was there yesterday.'
\end{exe}

While many scholars have been keen on insisting that the two constituents should be counted as one phrase, others, such as \citet{BueringHartmann2001}, have claimed the focus particle is an adverb of its own, not merged with its partner and thus triggering V3.

\subsection{Approaches to V3}

V3 phenomena have been mentioned in syntactic descriptions of urban vernaculars, but also in diachronic studies, suggesting that multilingual settings are not responsible for their emergence: much rather, it seems that Germanic verb-second has always included possibilities of violations and/or of ambiguous constructions or “apparent polyoccupation”. For instance, the stacking of VP modifiers forming a consistent complex as often been called “apparent” or “superficial” verb-third. 

The reason why present-day examples are mostly identified in urban vernaculars and, to a lesser extent, other vernacular dialects such as West Flemish, should probably be found in the fact that these vernacular varieties are less submitted to the normative constraints prohibiting any apparent violation of V2.

The studies brought together in this book are concerned with various kinds of violations of this “bottleneck” in any variety of Germanic, present and past (including former V2 stages of English). At least three verb-third constructions are attested both in North and West Germanic languages and deserve a special place both in the introduction and in the volume.

\section{Recurring V3 patterns across Germanic languages}
\subsection{Adverbial – subject – finite verb}

In the first type of construction, an adverb or a phrase with an adverbial function stands at the beginning of the utterance and is followed by the subject, then the finite verb, whereas the standard V2 rule would state that only one of the two constituents can stand in the pre-finite slot at the same time. This phenomenon is particularly well described in urban vernaculars and typically involves an indexical adverb of time or place and a pronominal subject:

\begin{exe}
\ex Norwegian\\
    Nå de \textit{får} betale.
    \glt `Now they have to pay.'
\ex Swedish\\
    Då alla \textit{började} hata henne.
    \glt `Then everybody started to hate her.'
\ex High German \citep[82]{FreywaldEtAl2015}\\
     Dann \textit{wir} sind gegangen. 
    \glt `Then we left.'
\ex Dutch (\citealt{Appel1984}, quoted from \citealt[82]{FreywaldEtAl2015})\\ 
    En dan hij \textit{gaat} weg.
\glt `And then he’s going away.'
\end{exe}


Data from \citet{FreywaldEtAl2015} suggest that this pattern is much more restricted in Dutch urban vernaculars than in the corresponding Norwegian, Swedish and German varieties. \citet{WieseMueller2018} also show that this pattern, while it is widely attested in vernaculars from multilingual urban settings, is not an innovation caused by language contact. The same pattern is well-attested in the West Flemish variety of Dutch (see recent discussion in \citealt{HaegemanGreco2018West}), independently of the sociolinguistic context of urban vernaculars. The same seems to hold in colloquial Danish \citep{Joergensen2016}. Even though research on the acquisition of V2 suggests that language contact may facilitate the use of the Adv – S – V pattern \citep{Johansen2008}, just like code-switching \citep{Kahan-Newman2015}, the role of language contact in the rise of V3 is a matter of debate (compare \citealt{Walkden2017Language}, \citealt{WieseMueller2018}). Other factors or hypotheses also need to be addressed. For instance, if there is a privileged access of subject pronouns to this pattern, as opposed to heavier NP/DP subjects, one may wonder if this verb-third construction is not due to the pronoun partly losing its phrasal status, i.e. a phenomenon of pronominal deficiency \citep{CardinalettiStarke1999}.\footnote{This raises the question whether resumptive adverbials and personal pronouns stacked immediately before the finite verb should be treated separately, even when they are not clitics (see \citealt{Joergensen2016}).} In that case, one could wonder if the deficient pronoun still “counts” as a filler for the initial slot. Another possible interpretation, more significant consequences, would be to consider that this subject pronoun builds an inflectional node with the verb and that the adverb, on the other hand, is a framesetter endowed with a high illocutionary position and a certain degree of syntactic autonomy with respect to the inflectional layer. This would bring us in the vicinity of the FinV2\slash ForceV2 debate in Cartographic analyses of (historical) Romance V2, with the Adv~-- S~-- Fin construction being a sign of remnant (or incipient) FinV2.

\subsection{Topic – adverb – finite verb}

In this construction, an adverb fulfilling a text-structuring function can be inserted between the initial constituent (a topic) and the finite verb. Two subtypes of Top~-- Adv~-- VFin constructions must be distinguished:

\begin{itemize}
\item Adverbial resumption: the fronting of a non-clausal adverbial phrase, followed by an adverbial resumptive pronoun, and then by the finite verb. 
\item Topic change constructions: constructions involving an adverb (usually with a discourse\hyp organizational meaning) after a new or a resumed topic, without requesting that this discourse-connecting adverb resume the topic on its left.
\end{itemize}

\subsubsection{Framesetter resumption}

In Scandinavian languages, the adverbial resumptive type of V3 is particularly well-attested for the pronoun \textit{så} (\citealt{SollidEide2007}, \citealt{Nordström2010}). It is important to note that this pattern is not restricted to the resumption of clausal adjuncts: \textit{så} can also resume an adverb or a prepositional phrase.

\begin{exe}
\ex Swedish \citep[364]{Egerland2021}\\
    Jag anlände till Rom och sedan så \textit{tog} jag in på hotell.
\glt `I landed in Rome, then I checked into a hotel.'
\end{exe}

This set of constructions is subjected to many parameters which have to be taken into account in the study of (micro-)variation: Does the resumptive have to be placed in the pre-finite slot or whether it can be placed elsewhere? Can the resumptive be modified or focalized?  Is it obligatory or optional? Can it resume any kind of adjunct or is it licensed only after subordinate clauses? Is it even really restricted to adjuncts? Probably the most important distinction, however, is the one between specialized and generalized resumptives. Another important aspect is the status of the initial (resumed) constituent, especially on information-structural terms: Does it have to be a topic or a framesetter? Which kinds of topics are licit in this construction? Can this pattern be reduced to a subtype of left-dislocation? Is the initial constituent clause-external? Most recently, the studies collected in \citet{DeClercqEtAl2023} suggest that many Germanic and Romance varieties show a pattern where peripheral adjuncts (in the sense of Haegeman's peripheral adverbial clauses, as opposed to central ones) with a framesetting function can be resumed by an adverbial pronoun in pre-finite position in a dedicated V3 construction; in some of these languages, a generalized adverbial resumptive, which can resume adjuncts of various semantic types, has gradually grammaticalized, giving rise to a specific V3 pattern where the second constituent, the resumptive, is actually a grammatical head cliticized onto the finite verb. Scandinavian \textit{så}, in the construction mentioned above, would be an example of such construction. Even though these patterns often share striking similarities with left dislocation, a specific syntactic construction is likely to have played a role in the birth and rise of adverbial resumption, at least in Germanic languages: correlative constructions, most prominently those where the subordinator and the resumptive are formally identical (as opposed to \textit{wh}- ... \textit{th}-... correlative constructions).


\subsubsection{Topic change constructions}

A further case of real or apparent polyoccupation is attested when the first constituent of the utterance is a topic (aboutness topic or framesetting topic) with a strong discourse-structuring function, e.g. a contrastive topic:

\begin{exe}
\ex German (DEREKO, BRZ05/SEP.05028)\\
    Woran das liegt, das weiß niemand so recht. Lange hat sich Coach Falk von der Osten darüber den Kopf zerbrochen. Jetzt \textit{allerdings} sind solche Fragen unwichtig.
\glt `Nobody really knows why it is so. Coach Falk von der Osten has been thinking about it for a long time, in vain. Now, however, such questions are unimportant.'
\end{exe}

Both \textit{jetzt} and \textit{allerdings} are full lexical adverbs and as such, they can frequently be observed alone in the pre-finite slot of declarative sentences. However, it has already been observed in the literature that these two constituents are by no means equal. One of the first analyses of this construction can be found in \citet{altmann79}, who calls it a case of “idiosyncratic focus particle usage”, drawing on three observations which, to this date, are fully unchallenged:\largerpage

\begin{itemize}
    \item The initial constituent (here: \textit{jetzt}) bears a contrastive stress; it can be an adverb, a prepositional phrase, a DP, and have different functions in the VP, suggesting that the only relevant selection parameter is information-structural.
    \item The second constituent has to be an adverb or a coordinating conjonction; only a limited set of items are eligible to this slot.
    \item The relative order of the two constituents in the construction is fixed: the adverb or conjunction has to follow the constituent bearing the contrastive stress.
\end{itemize}

In High German, this pattern is attested in contexts of topic change; in Swedish, it also selects non-contrastive framesetting topics:

\begin{exe}
\ex German (contrastive topic) \citep[35]{Breindl2008}\\
    Der Juniaufstand ist Ausdruck spontaner Empörung; ihm fehlt Führung und Organisation. Der Gegenschlag indessen \textit{ist} ebenso kühl kalkuliert wie brutal.
\glt `The June Uprising is an expression of spontaneous indignation; it is deprived of leadership and organisation. The retaliation, on the other hand, is both well-calculated and brutal.'
\ex Swedish (non-contrastive topic) \citep[1]{Egerland1998}\\
    Hon kanske \textit{kommer} i kväll.
\glt `Maybe she’s coming tonight.' 
\end{exe}

However, even in German, it is not hard to find cases showing that at least one of the discourse\hyp organizational items licensed in that construction, \textit{aber}, can occur after non-topics. For instance, in the next example, \textit{aber} occurs after the adverb \textit{trotzdem} `in spite of that':

\begin{exe}
\ex German (Tushoisdorf.de, 15.12.2018)\\
    Wir haben leider gegen Timmerhorn und auch gestern gegen Zarpen jeweils 2 Punkte liegen gelassen. Das ist schade. \textit{Trotzdem} \textit{aber} ist der Trend in der Liga auch hier eindeutig positiv.
\glt `Unfortunately, we lost 2 points against Timmerhorn and also yesterday against Zarpen. That is a pity. Nevertheless, the trend in the league is also clearly positive here.'
\end{exe}

Furthermore, the alleged restriction to contrastive contexts or to new topics may well have to do with the fact that in German, the vast majority of the discourse organizers used in that syntactic slot are contrastive, additive or adversative \citep{Breindl2011}. As we can see above, this is not the case in Scandinavian languages: \textit{kanske} `maybe' is a modal adverb, and does not carry any adversative or additive meaning. Historically, German also allowed this construction with one non-adversative, non-additive marker: \textit{so} (maybe thanks to its otherwise well-attested uses as an adverbial resumptive). As \citet{chapters/7_Bloom} shows, the restriction to topical discontinuity generally observed in Present-Day German does not hold for the Early New High German construction with \textit{so}.

\subsection{Constituent stacking in the pre-finite slot}

\subsubsection{Apparent polyoccupation by circumstantial adjuncts}

The possibility to stack arguments and adjuncts in the pre-verbal slot is attested across Germanic V2 languages. The case of German has already been mentioned above, as these patterns have been extensively documented for that language since \citet{Mueller2003}. However, it is important to bear in mind that stacking is not restricted to arguments. The next example, quoted from \citet{Mueller2003}, includes two circumstancial adjuncts in the pre-finite slot, \textit{vor wenigen Wochen} `a few weeks ago' and \textit{im Deutschen Theater} `at the German Theater':

\begin{exe}
\ex German \citep[38]{Mueller2003}\\
    Vor wenigen Wochen im Deutschen Theater \textit{sagte} ich: „Unser Ziel muß sein, daß die Polizei friedliche Demonstranten schützt und damit den Namen Volkspolizei rechtfertigt.“
\glt `A few weeks ago, at the German Theater, I said: “Our goal must be that the police protect peaceful protesters, and thus deserve the name of People’s Police.”'
\end{exe}

The same type of adjunct stacking can be observed in Mainland Scandinavian, e.g. in Swedish. In the following sentence, the pre-finite slot is jointly occupied by \textit{i går} `yesterday' and by \textit{på jobbet} `at work':

\begin{exe}
\ex Swedish (TenTen2014 corpus, retrieved in February 2022 from SketchEngine, item number 1652707550) \\
    I går på jobbet \textit{var} det olidligt hur varmt det var.
\glt `Yesterday at work, it was unbearably hot.'
\end{exe}
This kind of polyoccupation is commonly considered as merely apparent, since all constituents fulfill the same framesetting function. However, they need not belong to the same semantic domain (e.g. indications of place). For instance, in the examples above, an indication of time is followed by an indication of place. This suggests that bluntly unifying these constituents into one and the same syntactic label is not trivial. The shared functional label under which they are subsumed in order to occupy the pre-finite slot together must be restricted to information-structurally marked constructions.

\subsubsection{Argument stacking and VP-fronting}
The same questions may be raised for other cases of “apparent polyoccupation” (\citealt{Mueller2003} and subsequent) involving the fronting of manner adverbs and object arguments:

\ea High German \citep[41]{Mueller2003}\\
    Eindeutig für die Existenz von Doppeltopikalisierungen \textit{spricht} das Beispiel 79a.
\glt `Example 79a speaks unequivocally for the existence of double topicalizations.'
\z

Depending on the theoretical framework, these phenomena can be interpreted as “remnant VP-movement” (within Generative Grammar) or any equivalent terminology: it can be admitted that this “polyoccupation” is actually no breach of the standard V2 configuration; rather, the unit filling the pre-finite slot is a syntactic complexn that could have stood at the end of the clause but was placed in the initial position for various reasons, mostly information-structural considerations. This type of apparent polyoccupation is presented in detail in the next chapter by Augustin Speyer, using data from the history of High German.


\section{Overview of the volume}

This introduction is followed by a review and theoretical discussion of verb-third phenomena in the history of (High) German (\textit{Verb-third in German: A diachronic perspective}). German is known for following the verb-second constraint rather strictly (hence its status as a paradigmatic example of “strong V2” language). Drawing on recent research on the fine structure of the left periphery, Augustin Speyer argues that in German, the tight “bottleneck” coexists with a rich left-peripheral hierarchy, hence two types of V2 violations: remnant VP movement, which is actually a merely apparent polyoccupation, and the independent movement of more than one phrase to the pre-finite slot. Diachronic data show that this second sort may have been historically dominant in Old High German, whereas present-day German rather displays phenomena of remnant VP movement, in line with the evolution of the V2 rule, becoming stricter over the centuries.

The introduction and Speyer's chapter show that verb-third can be regarded as a surface signal giving insights into the fine structure of the left periphery and into the very nature of the “bottleneck” of V2, which undergoes important variation from one language to the other. These theoretical concerns build the core of Part~\ref{part:2}, \textit{Verb-third as a theoretical challenge: Generative perspectives}. The first contribution to this part, by Espen Klævik-Petersen, takes a provocative stance against the use of the concept of bottleneck in the original sense of the generativist literature (\textit{On the probable inexistence of “bottlenecks” in V2-languages}). The author takes aim at the idea that in V2 languages, the finite verb moves to a low position of the left periphery, usually Fin, which is endowed with a feature (EPP-feature) blocking movement to the left except for one slot in the Spec position of the head (SpecFin or SpecForce in those languages where V moves to Force after going through Fin). He claims that this view fails to make significant predictions about V2 and V3 in relaxed verb-second languages, i.e. those where V stays in Fin (“FinV2 languages” in the terminology proposed by \citealt{Wolfe2018}), and makes too strong claims about “strict” or “strong” V2 languages, those where V moves to Fin and then to Force (“ForceV2 languages”). In the case of strict V2 languages, the crucial empirical issue is with clitic left dislocation and apparented constructions. The chapter also discusses two alternative accounts: a phase-based theory and a theory where the C° node of V2 languages would be regarded as compressing or bundling several features into one single projection.

In the following chapter, \textit{No cracks in the bottleneck! A cartographic model for the derivation of non-V2 in German}, Nicholas Catasso takes the opposite viewpoint. His chapter is a plea for a strictly cartographic approach to V2 and its apparent violations, drawing on the assumption that in German, V2 is a structural rule linked to the movement of V to C, and to an EPP feature of the same kind as the one discussed in the previous chapter. Catasso tackles the apparent counter\hyp examples to the predictions relying on the “bottleneck” hypothesis. Documenting a series of superficial violations, including V4 or V5, he derives these phenomena from the core structure of V2: V3 or V4 are variants of V2, and not proper violations of it. In his words, “the linear component of V2 can be violated, but its structural core is inescapable.” However, the fine structure of the left periphery revealed by these variants suggest that in German declarative sentences, V may actually be moved to the Fin position and not to the Force position.

The third part of the volume, \textit{Historical perspectives on verb-third with adverbial pronouns}, is devoted to the margins and appearances of adverbial resumption. Each of the three chapters investigates one phenomenon of apparent verb-third involving adverbial pronouns: Old English \textit{þa}, Old Saxon and Old High German \textit{thô}, Old and Modern Swedish \textit{då} and \textit{så} as well as Early New High German \textit{so}, all of which are “usual suspects” of adverbial resumption and/or of correlative constructions. However, it soon appears that not all constructions involving one of these items and another constituent coexisting in pre-finite position should be regarded as verb-third, and even less of adverbial resumption.

The first chapter of this part, \textit{Verb-third with initial \textit{þa/thô} in Old English, Old Saxon and Old High German verse}, by Élise Louviot and Thérèse Robin, addresses the question of what should really count as verb-third in the sense of a deviation from verb-second, by tackling a frequent assumption about Old English, Old Saxon and Old High German \textit{þa}\slash\textit{thô}: it is usually assumed that these forms either behave as subordinators triggering a low position of the verb (“V-late”, which needs not be V-final), or as adverbial pronouns with a resumptive, a connective or an expletive function: in that case, \textit{þa} or \textit{thô} tends to be immediately followed by the finite verb, leading to the assumption that the adverbial pronoun fills the pre-finite slot of the otherwise attested V2 structure. There are, however, numerous examples where the finite verb is not adjacent to \textit{þa} or \textit{thô}, especially in Old English. The aim of the chapter is determine whether this should be regarded as a verb-third pattern of any kind. The authors show that in the verse corpora that they use, which otherwise seem to follow the standard assumptions of verbal syntax for the languages under consideration, verb-first and verb-late are well attested in independent clauses, so that apparent verb-third after \textit{þa} or \textit{thô} should not automatically be understood as a deviation from V2. The authors argue that what looks like verb-third should probably rather be interpreted as a specific form of V-late.

In the next chapter, \textit{Tracing the origins of resumption in Swedish}, Christine Meklenborg examines the evolution of adverbial resumption in Swedish, from the 13th century to the first half of the 18th. Until the first half of that period, the dominant resumptive is \textit{då}. Later on, the resumptive \textit{så} takes over and becomes the standard resumptive, to this day. Christine Meklenborg shows that this change is tightly linked to the question of whether the fronted adverbial clause is syntactically integrated to the V2 clause or not. Initially, it was not. According to Meklenborg, the very characterization of \textit{då} as a resumptive is dubious: that adverbial pronoun should rather be considered an expletive. However, with the adverbial clause gradually becoming syntactically integrated into the V2 clause, the expletive undergoes reanalysis as a resumptive, but because of the V2 rule, the placement of the resumptive between a syntactically integrated fronted adverbial clause and the finite verb is possible only under strict conditions of semantic congruency. This narrows the field of contexts eligible for \textit{då}-resumptive and paves the way to the subsequent evolution of \textit{så} into a generalized resumptive as we know it today.

The last chapter of Part~\ref{part:3}, \textit{Cohesion and/or disruption? The discourse functions of Early New High German V3 with \textit{so}} by Barthe Bloom, is concerned with Early New High German constructions involving an adverbial followed by the adverbial pronoun \textit{so} and then by the finite verb. The main question of the chapter is whether this construction has to involve some degree of disruption in discourse, as would be the case if the construction were restricted to contrastive topics and other kinds of topic change. Using a corpus of narrative prose, Bloom investigates whether the preposed adverbials tend to be framesetters, whether the subject of the finite verb tends to be a pronoun or a full nominal phrase, and whether it is co-referential with the subject of the previous verb. She shows that the construction is compatible with disruptive as well as with cohesive contexts. But the study also reveals there is a significant difference between clausal and non-clausal adverbials before \textit{so}: clausal adverbials, especially temporal clauses, often have a framesetting value, whereas non-clausal adverbials, especially adverbs, tend to be on the cohesive side, and to link the utterance to previous discourse.

The final part of the volume is made up of two contributions devoted to \textit{Violations of verb-second and areal micro-variation in Northern Italy}. It is composed of two chapters. The first one, by Luca Riccardelli, is actually concerned with Romance: Riccardelli investigates \textit{V2 and V3 in Rhaeto-Romance Varieties} with the aim of “rethinking language contact”. Rhaeto-Romance varieties display the most robust V2 features in present-day Romance languages, and accordingly, they also display numerous verb-third phenomena. Because they are also characterized by long, extensive contact with Germanic languages, a plausible reason for their exceptional status may be a Germanic influence. Riccardelli argues against unilateral explanations resorting to contact as a major trigger for the Rhaeto-Romance typology of V2. He shows that the types of V2 and V3 observed in Friulian, Romansh and Ladin does not correspond to the observations made in German and its regional varieties: V2 in Rhaeto-Romance varieties is not a cohesive
phenomenon, and contact with Germanic probably just reinforced what Riccardelli calls “a general tendency of Rhaeto-Romance to develop a V2 system”.

The last chapter of the volume is \textit{Deriving CP-expansion in the German enclave varieties across Northeast Italy: Cracks in the bottleneck?}, by Romano Madaro, Alessandra Tomaselli and Ermenegildo Bidese. It is devoted to the range of V2 violations and to the fate of V2 itself in the German-speaking varieties in North-Eastern Italy. The authors are concerned with the progressive of the C-domain in these varieties, and the specialization of the various projections in the C-domain. The authors use data from Plodarisch, Saurian, Timavese and several varieties of Cimbrian, and compare them to other German(ic) languages. Although these varieties originally had (and, for some of them, still have) a structural V2 rule similar to what is known from other varieties of German, the bottleneck effect does not materialize in the same way, prompting the authors to resort an alternative model, the Feature-Scattering Hypothesis. They also investigate the variation along geographical lines (West-to-East and North-to-South). The North-to-South line documents the expansion in the left periphery across several varieties of Cimbrian, leading to the ultimate loss of V2. These findings are corroborated by diachronic data on Cimbrian. 

\printbibliography[heading=subbibliography,notkeyword=this]
\end{document}
