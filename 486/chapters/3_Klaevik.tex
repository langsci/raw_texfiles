\documentclass[output=paper,colorlinks,citecolor=brown]{langscibook}
\ChapterDOI{10.5281/zenodo.17077186}

\IfFileExists{../localcommands.tex}{
% add all extra packages you need to load to this file

\usepackage{tabularx,multicol}
\usepackage{url}
\urlstyle{same}

\usepackage{listings}
\lstset{basicstyle=\ttfamily,tabsize=2,breaklines=true}

\usepackage{langsci-basic}
\usepackage{langsci-optional}
\usepackage{langsci-lgr}
\usepackage{langsci-osl}
% \usepackage{./langsci/styles/langsci-lgr}
% \usepackage{./langsci/styles/langsci-osl}
% \usepackage{langsci-gb4e}

\usepackage{tikz}
\usetikzlibrary{patterns,calc}
\pgfdeclarepatternformonly{south east lines}{\pgfqpoint{-0pt}{-0pt}}{\pgfqpoint{3pt}{3pt}}{\pgfqpoint{3pt}{3pt}}{
    \pgfsetlinewidth{0.6pt}
    \pgfpathmoveto{\pgfqpoint{0pt}{3pt}}
    \pgfpathlineto{\pgfqpoint{3pt}{0pt}}
    \pgfpathmoveto{\pgfqpoint{.2pt}{-.2pt}}
    \pgfpathlineto{\pgfqpoint{-.2pt}{.2pt}}
    \pgfpathmoveto{\pgfqpoint{3.2pt}{2.8pt}}
    \pgfpathlineto{\pgfqpoint{2.8pt}{3.2pt}}
    \pgfusepath{stroke}}
    
\usepackage{stmaryrd}
\usepackage{wasysym}
\usepackage{multirow}
\usepackage{caption}
\usepackage{subcaption}
\usepackage{mathrsfs}
\usepackage{qtree}

\usepackage{linguex}


%pminos do not split footnotes
% \interfootnotelinepenalty=10000 %Footnote in Laporte chapters has to be split SN


%\DeclareIndexNameFormat{default}{%
%\nameparts{#1}%
%\usebibmacro{index:name}%
%{\index[names]}%
%{\namepartfamily}%
%{\namepartgiveni}%
% {}% L1
% {}% L2
%{\namepartprefix}% generates spurious space L3
%{\namepartsuffix}% generates spurious space L4
%}

%  {\DeclareIndexNameFormat{default}{%
%     \usebibmacro{index:name}{\index[names]}{#1}{#3}{#5}{#7}}}

%\DeclareIndexNameFormat{default}{%
%  \usebibmacro{index:name}{\sindex[nom]}{#1}{#3}{#5}{#7}}

%\DeclareIndexNameFormat{default}{%
%  \usebibmacro{index:name}{\sindex[person]}{#1}{#3}{#5}{#7}}
%\DeclareIndexNameFormat{default}{%
%\nameparts{#1} \usebibmacro{index:name}{\sindex[person]]}{\namepartfamily}{‌​\namepartgiven}{\nam‌​epartprefix}{\namepa‌​rtsuffix}}

%\newcommand{\smiley}{:)}

%\renewbibmacro*{index:name}[5]{%
%\usebibmacro{index:entry}{#1}%
%{\iffieldundef{usera}{}{\thefield{usera}\actualoperator}\mkbibindexname{#2}{#3}{#4}{#5}}}

% \newcommand{\noop}[1]{}

%remove for final
%\overfullrule=1mm

\newcommand{\tobi}[2]}}
\renewcommand{\S}[1]{\tobi{#1}{\textsc{*}}}

% this volume references
% puts: [this volume]
% already defined: \citetv
%\newcommand{\citepv}[1]{(\citeauthor{#1} \citeyear*{#1} [this volume])}
\newcommand{\citealtv}[1]{\citeauthor{#1} \citeyear*{#1} [this volume]}

%parentheses around example number
\newcommand{\pref}[1]{(\ref{#1})}

% in-text examples

\newcommand{\lnex}[1]{\textit{#1}} %target lang word
\newcommand{\lnlit}[1]{(lit.: `#1')} %literal reading
\newcommand{\lnlat}[1]{(#1)} % latinization
\newcommand{\lntrans}[1]{`#1'} %translation
\newcommand{\lnexl}[2]%
{\lnex{#1}{} \lnlat{#2}} % ex with latinization
\newcommand{\lnexlat}[3]{\lnex{#1}{} \lnlat{#2}{} \lntrans{#3}} % ex with latinization and tranl.

%ch01
\newcommand{\co}[1]{\mbox{\textbf{#1}}}

%ch09

\newcommand{\cyrbulg}[1]{\begin{otherlanguage*}{bulgarian}#1\end{otherlanguage*}}


%ch10
\newcommand{\nlp}{{\small NLP}}
\newcommand{\mwe}{{\small MWE}}
\newcommand{\rae}{{\small RAE}}
\newcommand{\lvc}{{\small LVC}}
\newcommand{\pos}{{\small P}o{\small S}}
%\newcommand{\todo}[1]{ \textcolor{red}{#1} }

%\renewcommand{\labelenumi}{\theenumi}
%\ainamefmt{{vv}{ll}{, ff}{, jj}} % fullname

\newcommand{\biberror}[1]{{\color{red}#1}}

\newcommand{\osenovaitem}{--~}
%% hyphenation points for line breaks
%% Normally, automatic hyphenation in LaTeX is very good
%% If a word is mis-hyphenated, add it to this file
%%
%% add information to TeX file before \begin{document} with:
%% %% hyphenation points for line breaks
%% Normally, automatic hyphenation in LaTeX is very good
%% If a word is mis-hyphenated, add it to this file
%%
%% add information to TeX file before \begin{document} with:
%% %% hyphenation points for line breaks
%% Normally, automatic hyphenation in LaTeX is very good
%% If a word is mis-hyphenated, add it to this file
%%
%% add information to TeX file before \begin{document} with:
%% \include{localhyphenation}
\hyphenation{
    Beck-man
    Ngu-yen
    back-chan-nel
    back-chan-nels
    mo-not-o-nous
    ste-reo-typ-i-cal
}

\hyphenation{
    Beck-man
    Ngu-yen
    back-chan-nel
    back-chan-nels
    mo-not-o-nous
    ste-reo-typ-i-cal
}

\hyphenation{
    Beck-man
    Ngu-yen
    back-chan-nel
    back-chan-nels
    mo-not-o-nous
    ste-reo-typ-i-cal
}

\boolfalse{bookcompile}
\togglepaper[23]%%chapternumber
}{}

\author{Espen Johan Klævik-Pettersen\affiliation{University of Agder}}
\title{On the probable non-existence of “bottlenecks” in V2-languages}
\abstract{This paper addresses the question of how to account for the restricted left periphery of verb-second (V2) languages. The locality constraint known in the literature as the “bottleneck” is critically reviewed, and theoretical and empirical arguments against this hypothesis are presented. Reviewing the basic empirical generalisations regarding so-called “relaxed” (Old Romance) and “strict” (Modern Germanic) V2 languages, it is argued that the bottleneck hypothesis does not make any predictions for the former, and that there is direct counterevidence against the latter in the form of the Contrastive Left-Dislocation construction. The paper also discusses two different approaches to verb-second that do not involve “bottlenecks”.\medskip\\
\textbf{Keywords}: Syntax, verb-second, left periphery, cartography, Contrastive Left\hyp Dislocation, phases, Germanic, Old Romance}

\begin{document}
\maketitle


\section{Introduction}

The verb-second (or V2) construction, which exists in all Modern Germanic varieties with the exception of English, generally requires the finite verb to be the linearly second constituent of the clause. In these languages, V2 is found in all declarative clauses and wh-questions as well as a narrowly definable subset of embedded clauses (\cite{EKP2019}), regardless of the categorial status or grammatical function of the initial constituent, as the following examples from Modern Norwegian illustrate:

\begin{exe} 
\ex 
\begin{xlist} 
\ex (Modern Norwegian)\\
    \gll [Jeg] \textbf{solgte} bilen i forrige uke.\\ 
         I sold car.\textsc{def} in last week\\ 
    \glt `I sold the car last week.'\\
\ex {[Bilen] \textbf{solgte} jeg i forrige uke}.
\ex {[I forrige uke] \textbf{solgte} jeg bilen}.
\end{xlist}
\end{exe}

\begin{sloppypar}
Cross-linguistically, V2 is a rare phenomenon, reportedly attested in quite diverse and geographically scattered varieties like Breton, Sorbian, Estonian, Kashmiri, Karitiana, and some dialects of Himachali (see \cite{Holmberg2015} and references therein). Within the Germanic family, however, the construction has shown remarkable diachronic and diastratic stability, being attested for many languages in written corpora that span a millennium back (\cite{Tholli1995, Walkden2015}) and found in all traditional dialects of the Germanic \textit{sprachbund} (again except English) without distinction (though see \sectref{semi-strict} for the emergence of new variteties with innovation in the V2-grammar).
\end{sloppypar}

\subsection{The “traditional analysis” of V2}\label{traditional}

Over the 1980's and early 1990's, an analysis of V2 was developed within the transformational generative framework (\cites{denBesten1983, Travis1984, Holmberg1986, Chomsky1986, IatridouKroch1992, Roberts1993, Vikner1995, HolmbergPlatzack1995}), which was generally accepted and enjoyed widespread consensus. According to this analysis, the finite verb in a V2 clause is attracted to the head of the maximal projection of the clause, the CP, while a single constituent is attracted to the specifier of CP. This analysis, which I will dub the “traditional analysis”, had the merit of explaining the linear second position of the verb and the restriction to a single constituent in the prefield, while also giving a plausible account of the (general) unavailability of verb-second in embedded clauses, since the head of CP in such would be lexicalised by a complementiser, thereby bleeding verb movement. All of these facts were given a unified account, and the verb-second construction was effectively reduced to an epiphenomenon of universal phrase structure coupled with V-to-C movement.\footnote{In (\ref{der}), Head-to-Head movement is for simplicity shown as replacement rather than adjunction. Note also that the T$^0$ head is not assumed to be an intermediate landing site of the verb, in line with the traditional assumption of only two verb position in Mainland Scandinavian (\cite{HolmbergPlatzack1995, Vikner1995}), namely C$^0$ (main clause V2) and in-situ (embedded clause non-V2). As a reviewer correctly points out, a direct movement of the verb from V-to-C is at odds with the Head Movement Constraint (\cite{Travis1984}). For a discussion of this issue, see \citet{Platzack1986}, \citet{HolmbergPlatzack1995} and \citet{Vikner1995}.}

\begin{figure} 
\caption{The traditional single CP-model of V2}
\label{der} 
% old xyling:
% % % \Treek{1} { & \CP \\ \K{Bilen} && \Cbar \\ & \K{C$^0$+v$^0$}\Below{\textbf{solgte}} && \K{TP}\B{dl}\B{dr} \\ && \K{jeg} && \K{T'}\B{dl}\B{dr} \\ &&& \K{T$^0$} && \K{vP}\TRi \\ &&&&& \K{\sout{jeg} \sout{\textbf{solgte}} \sout{bilen} i forrige uke}}
% new forest:
\begin{forest}
[CP [DP [Bilen, roof]] [C' [C$^0$+v$^0$\\\textbf{solgte},align=center] [TP [jeg] [T' [T$^0$] [vp [jeg \sout{\textbf{solgte}} \sout{bilen} i forrige uke, roof]]]]]]
\end{forest}
\end{figure}

In spite of its general success, the classical analysis encountered some empirical problems from the start. It was motivated by the facts of verb-second in Continental Germanic (Dutch and German), and even within other branches of the family like the Scandinavian languages, there is in fact no complementarity of verb-second and overt complementisers, counter to the predictions of the theory (\cite{Vikner1995}). Moreover, all Germanic V2-languages allow certain linear V3 orders, most notably in the so-called “Contrastive Left\hyp Dislocation” (CLD) construction (\ref{CLD}), forcing an exceptional CP-adjunction analysis (\figref{adj}).\footnote{It is important to note that linear non-V2 position in Modern Germanic V2 languages is not limited to the CLD construction. For instance, biscuit conditionals (\cite{Krifka2017}) and other “central” adverbial clauses can exhibit V3 in Continental Germanic (\cite{Csipak2019, Breitbarth2023}), and certain adverbs induce V3 in Northern Germanic (\cite{Thrainsson2007}). For discussion, see \citet[ch. 2]{EKP2019}.} 

\begin{exe} 
\ex\label{CLD} 
\gll Bilen, den \textbf{solgte} jeg i forrige uke.\\ 
     car.the that sold I in last week\\ 
\glt `The car, I sold it last week.'
\end{exe}

\begin{figure}
\caption{CP-recursion in the traditional model}
\label{adj} 
% old xyling:
% % % \Treek{1} {& \CP \\ \K{Bilen} && \CP \\ & \K{den} && \Cbar \\ && \K{C$^0$+v$^0$}\Below{\textbf{solgte}} && \K{TP}}
% new forest:
\begin{forest}
[CP [DP [Bilen, roof]] [CP [DP [den, roof]] [C' [C$^0$+v$^0$\\\textbf{solgte},align=center] [TP [..., roof]]]]]
\end{forest}
\end{figure}

\subsection{“Relaxed” V2 languages and cartography}\label{relcart}

Work on historical Romance varieties (\cite{Beninca1983, Fontana1993, Vance1997, Vanelli1998}) revealed that many Old Romance languages featured verb-second type inversion and quite regularly featured the verb in linear second position, while at the same time not being nearly as consequent as Modern Germanic with respect to this linearization pattern, since V3 or even $\text{V}\geq4$-orders are also found. At the time, this observational fact earned them the descriptive label “relaxed V2 languages”, as opposed to the “strict” V2 languages of the Modern Germanic kind. %Furthermore, many Romance varietes that today display non-V2 SVO order are purported to have evolved through a V2 phase in the Medieval period (\cite{Wolfe2018}).


%\begin{exe} \ex\label{Florentine} \gll [Ad ogni matto] [i savi] \textbf{paiono} matti\ldots\\ To each mad.person the wise.persons seem.3PL mad \\ `To every crazy person, the wise seem crazy\ldots'\sn (Old Florentine, \textit{Novellino} (40). Adapted from \cite[276]{Beninca2004}.)\ex\label{Sicilian} \gll [tamen poy di la morti loru], [li ossa loru] [pir virtuti divina] \textbf{operannu} miraculi. \\ Then after of the death their the bones their by virtue divine perform.3PL miracles \\ `Then after their death, their bones perform miracles through divine virtue'\sn (Old Sicilian, \textit{Sanctu Gregoriu} (262). Taken from \cite[26]{Wolfe2015Microvariation})\end{exe}


Alongside this emerging empirical landscape of variation saw new theoretical developments, noticeably the advent of cartography (\cite{Rizzi1997} et seq.), which has demonstrated that many languages can combine several constituents in the left periphery of the clause. Clearly, a single CP cannot be a universal phrase structural reality, and therefore the linear position of the verb in Germanic V2 languages is no longer an epiphenomenon.

To use a forensic allegory, verb-second today appears like a case that was filed and closed, but which had to be reopened due to the discovery of new evidence. In this new, “cartographic” phase of investigation, new hypotheses were proposed to explain the V2-linearization and its exceptions. Currently, a hypothesis in favour by many syntacticians working on V2 in cartographic model holds that the CP/left-periphery is strongly articulated with many projections, but that at some point of the derivation, a “bottleneck” emerges which allows only one constituent to pass by movement.

\subsection{The outline of the article}

In this paper, I take issue with the “bottleneck” hypothesis on both theoretical and empirical grounds. I argue that the “bottleneck” is not indispensable for any variety of V2-languages, whether “strict” or “relaxed”, and that for the latter, it is even vacuous and without effect. In sum, I question the existence of bottlenecks. I will also discuss other ways of accounting for the data, including the notoriously complicated question of explaining the restricted prefield of the Modern Germanic V2 languages.

The structure of the paper is as follows: In \sectref{bottle}, I present the bottleneck hypothesis and briefly discuss some conceptual problems and theory\hyp internal complications as well as empirical counterevidence. In \sectref{typ}, I consider the relevance of the bottleneck hypothesis for two different groups of V2-languages that have been proposed in the literature, the so-called “relaxed” (\sectref{relaxed}) and “strict” (\sectref{strict}) V2 languages, as well as a subgroup of the latter that I refer to as “semi-strict” (\sectref{semi-strict}) V2 languages. I conclude that the hypothesis is either unhelpful by making no relevant predictions, or that there is indeed direct counterevidence. Finally, \sectref{true} discusses two alternative ways to account for the linearization of the verb and the restricted nature of the prefield in the Modern Germanic V2 languages.



\section{The “bottleneck”-hypothesis}\label{bottle}

The evidence that has been summarily reviewed in the previous section should make it clear that the “traditional” analysis of verb-second (see \sectref{traditional}) does not account for all the facts. In fact, it does not even achieve full descriptive adequacy for the Modern Germanic V2 languages (“strict V2”) for which it was originally developed, since linear V3 is permitted across the board in these languages in the CLD construction (\ref{CLD}). With the benefit of hindsight, attempts to integrate these V3 orders in the form of “exceptional adjunction” (\cite{IatridouKroch1992}) to an otherwise unitary CP-projection (see (\ref{adj}) above) seem today little more than patchwork. When we consider the more relaxed varieties, the problem grows more acute, since recursion or adjunction at the CP-layer would have to be iterative without any clear principle. The conclusion seems clear: the underlying assumption, that a main clause is universally a single CP-projection, is faulty, which the cartographic research program has demonstrated beyond reasonable doubt.

If, on the other hand, we pursue a cartographic logic to its ultimate conclusion by assuming that all languages universally exhibit the same wide array of dedicated projections in the left periphery, we are in essence faced with the opposite problem, for now the conundrum is no longer to fit the recalcitrant relaxed varieties with their articulated CPs into the theory, but rather to explain why the “strict” V2 languages (generally) refuse to make use of more than a single projection. In other words, the V2-linearization of Modern Germanic V2 languages is again a fact in need of an explanation, and this is where the bottleneck hypothesis enters the stage.

\subsection{A questionable locality mechanism}\label{questionable}

The bottleneck hypothesis is essentially a proposed mechanism of \textit{locality}, and to be more precise, of \textit{Relativised Minimality} \parencite{Rizzi1990Relativized, Rizzi2004Structures, Starke2001, Abels2012}, which essentially states that a “specific” XP, commonly interpreted as an XP with a relatively rich feature make-up, can block crossing movement of a less specific XP. Originally proposed in \citet{Haegeman1996}, the perhaps most explicit formulation in the context of verb-second is provided by  \citet{Roberts2004}. Adopting the conventional view that a head in the C-domain of V2 languages is equipped with an EPP-feature, Roberts goes on to state:

\begin{quotation}
XP movement to Spec-Fin in full V2 clauses is movement caused only by Fin's EPP feature\ldots. The moved XP is thus of no particular type in terms of the typology of potential interveners, and so is able to block any type of movement. (\cite[316]{Roberts2004})
\end{quotation}

This line of reasoning is not only stretching the logic of Relativised Minimality, it is “turning it upside-down”, to borrow an expression from \citet{Abels2017}. Rather, the final clause should read: “\ldots is thus unable to block any type of movement”, since a blunt, formal EPP-feature should not be able to block the movement of topics or foci to the left periphery. Indeed, the EPP-feature that all textbook introductions to English syntax assume to be responsible for the merger of expletive subjects in Spec-TP does not have any blocking effect on A'-movement (\ref{EPP}), since a focalised constituent is allowed to precede it. It is unclear why an EPP-feature on a head in the C-domain should work any differently:

\begin{exe}
\ex\label{EPP} 
\begin{xlist}
\exi{A:} On Friday, it seemed like he was recovering.
\exi{B:} No, on \emph{Thursday} it seemed like he was recovering.
\end{xlist}
\end{exe}

In spite of this, the bottleneck\hyp hypothesis has been employed by many researchers working on V2 as an explanation for the restricted accessibility of the left periphery (\cite{Poletto2002, BenincaPoletto2004, Cardinaletti2009, Wolfe2018, Salvesen2019, Catasso2022, chapters/4_Catasso}). The hypothesis builds on the generally accepted idea that the lowest head in the left periphery, Fin$^0$, is equipped with an EPP-feature, since V2 languages are not allowed to leave the prefield empty. This feature is either checked by the merger of an expletive subject directly in Spec-FinP or by “formal movement” of the subject to Spec-FinP (\cite{Frey2004}).

Proponents of the “bottleneck” assume that the EPP-feature is also involved when a topic or a focus is moved to the left periphery. In the latter case, Spec-FinP acts as an intermediate landing site caused by the EPP-feature, before the topic or focus moves on to Spec-TopP or Spec-FocP, respectively. This means the initial constituent in a V2-clause can be a topic or a focus (\ref{yolanda}), but the copy of the topic/focus in Spec-FinP blocks any subsequent movement operations, ruling out the combination of a topic \emph{and} a focus (\ref{combo}). This gives a derivation as in \figref{fla} (omitting non-lexicalised heads):

\begin{exe} 
\ex[]{\label{yolanda} 
\gll [Meine \emph{Mutter}] \textbf{hat} das Buch gelesen, (nicht mein Vater) \\ 
     my.\textsc{nom} mother has the.\textsc{acc} book read not my.\textsc{nom} father \\ 
\glt `My \emph{mother} has read the book, (not my father)'}

\ex[*]{\label{combo}{[Das Buch] [Meine \emph{Mutter}] \textbf{hat} gelesen, (nicht mein Vater)}}
\end{exe}


\begin{figure}
\caption{The bottleneck analysis}
\label{fla} 
% old xyling:
% % % \Treek{2} { & \K{TopP}\B{dl}\B{dr} \\ \K{} && \K{FocP}\B{dl}\B{dr} \\ & \K{meine MUTTER}\Below{\sout{[iFoc]}} && \K{FinP}\B{dl}\B{dr} \\  && \K{\sout{Meine MUTTER}[iFoc]} && \K{Fin'}\B{dl}\B{dr}\B{dl}\B{dr} \\ \K{$\scriptstyle `The$}\Below{$\scriptstyle “bottleneck” $}\GBkk{2,6}{2,6}{rr}{->} &&& \K{Fin$^0$ + \textbf{hat}}\Below{[\sout{$\phi$}], [\sout{EPP}]} && \K{TP}\TRi[4] \\   &&&&& \K{\sout{meine MUTTER} \sout{\textbf{hat}} das Buch gelesen}\Below{(nicht mein Vater)} }
% new forest:
\resizebox{\textwidth}{!}{\begin{forest}
[FocP [DP [meine MUTTER\\{[iFoc]}, roof]] [FinP [DP [\sout{meine MUTTER}\\{[iFoc]}, name=mutti, roof]] [Fin' [Fin$^0$+v$^0$ \\ {\sout{[$\phi$}], \sout{[EPP}]} \\ \textbf{hat}] [TP [\sout{Meine MUTTER} \sout{\textbf{hat}} \\ das Buch gelesen{,} (nicht mein Vater), roof ]   ]]]]
\node[xshift=-2.5cm, yshift=-1.8cm,align=center, overlay] at (mutti) (bn) {“The bottleneck”};
\draw[->] (bn) -- (mutti);
\end{forest}}
\end{figure}

However, the bottleneck only constrains \textit{movement}, meaning it should still be possible to first-merge an XP directly in the left periphery.

\section{The role of the bottleneck in the derivation of V2-languages}\label{typ}

Having presented the relevant theory, I will next consider to what extent the “bottleneck”-hypothesis can help us account for the different linearization patterns of the verb in the typology of verb-second languages proposed in the literature. It has become quite common to talk about “relaxed V2 languages” (\sectref{relaxed}) and “strict V2 languages” (\sectref{strict}), but the empirical landscape does not conform to this neat dichotomy, which is the reason why I have chosen to add a section (\sectref{semi-strict}) on some varieties that I refer to as “semi-strict V2 languages.”



\subsection{“Relaxed” V2-languages: No evidence for a bottleneck}\label{relaxed}

The term “relaxed V2” languages has been used predominantly to characterize various Old Romance varieties. According to many researchers, these languages seem to feature V-to-C movement of the verb in declarative clauses, since they display subject-verb inversion with all types of predicates, but at the same time, deviations from the linear V2 pattern in the form or V3 or even V4 (or more) are not infrequently attested. It is well-known that the different branches and varieties of Old Romance do not behave in a uniform way with respect to the linearization of the verb. Old French (\cite{Vance1997}), Old Northern Italian (\cite{Poletto2014}) and possibly Old Spanish (\cite{Wolfe2018}) only feature a quantitatively as well as qualitatively quite limited set of linear $\text{V}\geq3$ orders (see \sectref{semi-strict}), while other varieties like Old Neapolitan (\cite{Ledgeway2009}), Sicilian, Sardinian or Occitan (\cite{Wolfe2018}) are more permissive of these verb-later orders. In this article, I will not be concerned with the proper classification of any individual variety as “relaxed” or “strict”, but rather with the relevance of the bottleneck\hyp hypothesis for the derivation of the different types of verb-second.\footnote{It is also worth pointing out that the quite frequent linear $\text{V}\geq3$-orders of Old Romance have led some researchers to reject the label “verb-second” for these languages altogether (\cite{Kaiser2002, RinkeMeisel2009, Sitaridou2012, Martins2019}). This debate is partially a matter of definition, since it raises the question whether V-to-C movement is the sole criterion for calling a language “verb-second” or whether the linear position of the verb should also matter, and partially a matter of analysis, since it raises the question whether all Romance languages did indeed feature consistent V-to-C movement at all. For discussion, see \citet{EKP2019}.}

The decisive empirical fact to explain is how some varieties generate linear sequences with the verb in 3rd-or-later position. An example like (\ref{Florentine}) from Old Florentine seem to feature two arguments of the verb in preverbal position, while a V4 order like (\ref{Sicilian}) potentially feature an initial scene-setting adverbial followed by a topic corresponding to the subject of the clause and then a focalised constituent (`by divine virtue'):

\begin{exe} 
\ex\label{Florentine}(Old Florentine, \textit{Novellino} (40). Adapted from \cite[276]{Beninca2004}.)\\
\gll [Ad ogni matto] [i savi] \textbf{paiono} matti\ldots\\ 
     To each mad.person the wise.persons seem.3\textsc{pl} mad \\ 
\glt `To every crazy person, the wise seem crazy\ldots'
\ex\label{Sicilian}(Old Sicilian, \textit{Sanctu Gregoriu} (262). Taken from \cite[26]{Wolfe2015Microvariation})\\
\gll [tamen poy di la morti loru], [li ossa loru] [pir virtuti divina] \textbf{operannu} miraculi.\\ 
     Then after of the death their the bones their by virtue divine perform.\textsc{3pl} miracles\\ 
\glt `Then after their death, their bones perform miracles through divine virtue.'
\end{exe}

An articulated left-periphery with distinct positions for topics and foci provides a better tool for describing such sequences than a unitary CP, but what role does the “bottleneck” play in the derivation of these clauses? If only one constituent can be moved from the clause into the C-layer before the copy in Spec-Fin blocks all subsequent fronting operations, then it must necessarily be the constituent to the immediate left of the finite verb, in other words \textit{i savi} in (\ref{Florentine}) and \textit{pir virtuta divina} in (\ref{Sicilian}). This would in turn mean that the preceding constituents in both cases have been first-merged in the left-periphery. This is a theoretically unattractive assumption, since \textit{ad ogni matto} and \textit{li ossa loru}, while potentially interpretable as topics, are both also arguments of the verb and as such equipped with features that need to be checked inside the core clause; as a minimum, theta-roles that should be checked in the vP/VP, and possibly also case features. If the arguments are merged directly in the left-periphery, we would have to countenance the possibility of “Upward Agree” from the relevant heads inside the clause, and presumably across several phase heads (\cite{Gallego2012}), an \textit{ad-hoc} solution that should be avoided on grounds of principle. But more importantly, even if one accepts this solution, it is highly unclear what predictive power the “bottleneck” has in these varieties. The reason is that there is no principled way of telling whether the constituents to the left of the immediately preverbal constituent have been moved~-- which would mean there is no “bottleneck”~-- or whether they have been first-merged. As a proposed principle of locality, the bottleneck hypothesis is supposed to predict what is \textit{not} grammatical or what is \textit{not} encountered in corpora, but in the absence of principled constraints on what can be first-merged directly in the left-periphery, no empirical predictions arise.

\subsection{“Strict” V2-languages: Evidence against the bottleneck}\label{strict}

The strict V2-languages are the varieties that allow only a highly limited set of linear $\text{V}\geq3$-positions of the verb, most notably in the CLD-construction. These are the Modern Germanic languages (except for English) which gave the phenomenon its name and for which the “classical analysis” was developed in the first place. In the previous section it was argued that the bottleneck\hyp hypothesis does not really achieve anything for the “relaxed” V2 varieties, but the empirical situation is different in the Modern Germanic V2 languages. These do not allow topic-focus combinations in the prefield of the verb, or for that matter most imaginable combinations of left peripheral elements, and the idea of a bottleneck accordingly has a lot stronger intuitive appeal here. They do, however, allow two different dislocation structures, namely the CLD and the closely related “Hanging Topic left-dislocation (HTLD)” \parencite{Catasso2022}. The surface structures of these constructions are very similar, making it hard to tease them apart. Both CLD and HTLD consist of a dislocated constituent in the initial position of the clause and a co-referential resumptive inside the clause. At least for Continental Germanic, research has revealed a battery of diagnostics that together help identify the construction. Unfortunately, many of these diagnostics are not exclusive to the one or the other construction, but rather preferentially associated with one of them. For example, in CLD, the form of the resumptive is “typically a D-pronoun, prerably (but not necessarily) placed in the left periphery”, while in HTLD “it is always a personal pronoun, often (but again not invariably) placed in the middle field” (\cite[544]{denDikkenSuranyi2017}). An very important diagnostic is that the dislocated topic and the resumptive always match for case in CLD, while in HTLD, the dislocated topic may carry default case, in other words nominative. Additionally, it is often possible to distinguish the HTLD and CLD on the basis of prosody (\cite{ShaerFrey2004}).\footnote{I thank an anonymous reviewer for calling this publication to my attention.}

If we consider an example like (\ref{CLD2}), the form and placement of the resumptive in the left-periphery as well as case agreement between the dislocated topic and the resumptive all point towards the CLD-construction:

\begin{exe} \ex\label{CLD2} 
\gll [Den Wagen$_i$], [den$_i$] \textbf{habe} ich letzte Woche verkauft.\\ 
     the.\textsc{acc} car that.\textsc{acc} have I last week sold\\ 
\glt `The car, I sold it last week.'
\end{exe}

According to the “bottleneck”-hypothesis, only one constituent can be moved from the clause into the left periphery, since a copy of this constituent in Spec-FinP will block subsequent movement. After this derivational move, the left periphery is only accessible through direct external merge. The question is thus if \textit{Den Wagen} in (\ref{CLD2}) can plausibly be assumed to be externally merged directly in its initial position in the left periphery. This is the same question that was raised and answered in the negative for some of the left-peripheral constituents attested in the “relaxed varieties” of Old Romance (see \sectref{relaxed}). Unlike the latter, however, example (\ref{CLD2}) features a resumptive pronoun inside the clause which could have checked the $\theta$-features of the verb and saturated its argument structure before moving to the left periphery and sealing off the bottleneck. The initial topic \textit{Den Wagen} could therefore potentially be merged just to check a topic feature. However, the fact remains that the left-dislocated topic carries overt case morphology, a strong indicator of a clause\hyp internal first\hyp merge position. Moreover, topics in CLD also show other hallmarks of connectivity into the clause, such as the binding of anaphora, which the HTLD construction fails to exhibit (\cite[545]{denDikkenSuranyi2017}). While HTLD, due to lack of case agreement and other signs of connectivity, seems like a very plausible candidate for an analysis based on late external merge, the CLD is a priori to be analysed as a movement dependency.

At first sight, this seems like counterevidence to the bottleneck hypothesis, since both the topic and the resumptive would have moved to the C-domain. At least two different movement\hyp analyses have been proposed in the literature. According to the first, CLDs are moved to the specifier of a low left peripheral projection before eventually moving on to a higher left peripheral projection. The lower copy is spelled out as a resumptive. This is the analysis in \citet{Grohmann2003}, which could potentially be incorporated into the “bottleneck hypothesis” by assuming that the lower C-projection in question is FinP, while the higher is a dedicated Topic\hyp projection (\cite{Catasso2021Verbspäterstellungen}). On this analysis, the CLD-construction does not involve two individual movement operations to the left periphery, but a single movement chain with spell\hyp out of the head (the topic) and the intermediate copy (the resumptive).

The other movement-analysis of CLD is the one proposed by \citet{Ott2014}, who argues for an asyndetic, bi-clausal coordination structure with sluicing\hyp like ellipsis of the entire TP of the first conjunct under conditions of identity. The ellipsis\hyp analysis of Ott only superficially involves a linear V3 order, since both the dislocated topic and the resumptive are in fact underlyingly the initial element of their own, perfectly well-behaved V2 clause. On this analysis, the need for additional external merge disappears completely for the CLD-construction.

\begin{exe}
\exr{CLD2} 
\gll [Den Wagen$_i$], [den$_i$] \textbf{habe} ich letzte Woche verkauft.\\ 
     the.\textsc{acc} car that.\textsc{acc} have I last week sold \\
\glt `The car, I sold it last week.'
\ex 
\gll [CP1 Den Wagen$_i$ [C$^0$ \sout{\textbf{habe} ich letze Woche \textit{t$_i$} verkauft}]], [CP den$_i$ [C$^0$ \textbf{habe} ich \textit{t$_i$} letzte Woche verkauft]]\\ 
     CP1 the.\textsc{acc} car C$^0$ have I last week (t) sold CP the.\textsc{acc} have I last week sold\\ 
\glt `The car, I sold it last week.'
\end{exe}


Discussing at length the two competing analyses, \citet{denDikkenSuranyi2017} highlight several theoretical and empirical problems for Ott's (\citeyear{Ott2014}) analysis. Theoretically, they argue that the sluicing\hyp like operation is \textit{sui generis} and does not assimilate to otherwise attested cases of ellipsis. On the empirical side, the analysis fails to account for cases of P-stranding and “floated” arguments. Furthermore, it does not extend to cover the HTLD construction, which cannot be derived through clausal ellipsis and which is therefore demonstrably a bona fide linear V3-construction. In fact, the authors argue that the most sensible way to derive the differences between the CLD and HTLD is precisely to assume that the former is a movement dependency while that latter involves base-generation of the topic directly in the left periphery.  I refer the reader to den Dikkens and Surányis paper for the details; I will hereafter assume that a movement analysis of CLD is correct.\footnote{It is worth pointing out that, even if Ott's (\citeyear{Ott2014}) analysis of CLD turned out to be correct (and if it turned out to extend to HTLD), it would be unclear what this would involve, if anything, for the “bottleneck hypothesis” of V2. On Ott's analysis, there is no need at all for external merge “on top” of the V2 construction, since CLD just conceals two normal V2 constructions that are juxtaposed. Linear V2 would be without exception, which suggests that if there is a bottleneck at play, it does not only block movement, but even external merge. Such a principle of locality is to my knowledge unattested.}

As was already mentioned, “vanilla”-style CLD with a resumptive in the left-periphery (\textit{Den Wagen, den...}) is not incompatible with the bottleneck\hyp hypothesis, since we could in fact potentially be dealing with a single movement chain. However, there is more empirical variation within the CLD-construction than this standard scenario. As den Dikken and Surányi point out, the resumptive can sometimes be located in the middle field rather then the prefield, and gives (\ref{dik}) as an example from Dutch:\footnote{A reviewer asks why example (\ref{dik}) cannot be analysed as a HTLD-construction rather than a CLD. According to den Dikken and Surányi, the resumptive in HTLD is “always as a personal pronoun” in Dutch. In other words, the presence of the resumptive D-prounoun in (\ref{dik}) is a reliable diagnostic of CLD.}

\begin{exe} 
\ex\label{dik}(\cite[554]{denDikkenSuranyi2017})\\
\gll  [Die man], [ik] \textbf{ken} die niet. \\ 
      the man I know \textit{d-pronoun} not \\
\glt  `That man, I don't know him.' 
\end{exe}

The authors argue that such examples do not compromise Grohmann's (\citeyear{Grohmann2003}) analysis of CLD, since the sequence could involve several left peripheral positions, with the topic and its resumptive still being part of the same movement chain. Be that as it may, this derivation clearly violates the bottleneck\hyp hypothesis, since the intervening subject \textit{ik} simply must have been moved to the left-periphery from the clause, meaning (\ref{dik}) involves at least two movement operations to the left periphery.
The only possible solution out of the problem would be to suggest that since there is no subject\hyp verb inversion, such examples exceptionally do not involve V-to-C movement of the verb at all, and that only the initial topic is in the left periphery, giving \figref{fig:fromex:tree1}.

\begin{figure} %\ex {[TopP Die man [FinP \sout{die man} [Fin$^0$ \sout{EPP} [TP ich \textbf{ken} niet \sout{die man}]]]] }
% old xyling:
% % % \Treek{1} { & \K{TopP}\B{dl}\B{dr} \\ \K{Die man} && \K{FinP}\B{dl}\B{dr} \\ & \K{\sout{die man}} && \K{Fin'}\B{dl}\B{dr} \\ && \K{Fin$^0$}\Below{\sout{EPP}} && \K{TP}\TRi \\ &&&& \K{ich ken niet \sout{die man}}}
% new forest:
\begin{forest}
[TopP [DP [Die man, roof]] [FinP [DP [\sout{die man}, roof]] [Fin' [Fin$^0$ \\{\sout{[EPP]}}] [TP [ick {\textbf{ken}} (die) niet \sout{die man}, roof] ]   ]]]
\end{forest}
\caption{Multiple left-peripheral constituents in Dutch}
\label{fig:fromex:tree1}
\end{figure}

This would salvage the bottleneck hypothesis, since there is only one movement to the left-periphery. However, it would be highly questionable for Continental Germanic, where the standard assumption is that the finite verb always moves to a C-head, and by corollary that all constituents in the prefield are always left-peripheral. Furthermore, it is possible to come across examples where a CLD-topic is followed by: an inversion structure with the resumptive in the middle field (\ref{inv}), a constituent marked prosodically and with a particle as a focus (\ref{foc}), or even a wh-phrase (\ref{wh}):

\begin{exe} 
\ex\label{inv}(\cite[551]{denDikkenSuranyi2017})\\
\gll [Die man], [aan Marie] \textbf{zou} ik \{die/hem\} nooit voorstellen.\\ 
     the man to Marie should I \textit{d-pronoun}/him never introduce\\ 
\glt `That man, I would never introduce to Marie.'
\ex\label{foc} 
\gll [Den Wagen], [selbst \emph{Hans}] \textbf{würde} den nicht verkaufen. \\ 
     the.\textsc{acc} car even Hans would that.\textsc{acc} not sell \\ 
\glt `The car, even \emph{Hans} wouldn't sell it.'
\ex\label{wh}(\cite[1212]{Abels2019})\\
\gll [Den Jungen], [wer] \textbf{hat} den zuletzt gesehen?\\ 
     the.\textsc{acc} boy who has him.\textsc{acc} last seen\\ 
\glt `The boy, who saw him last?'
\end{exe}

It does not seem possible to escape the conclusion that some cases of CLD involve multiple movement operations to the left periphery, which in turn entails that the “bottleneck hypothesis” of verb-second, at least in its current form, cannot be maintained. Considering that the analysis suffers from a dubious interpretation of Relativized Minimality (see \sectref{questionable}) and that it produces no relevant predictions for so-called “relaxed” V2 varieties (see \sectref{relaxed}), other explanations for the surface linearization patterns involved in verb-second should be sought. Before turning to this matter, however, I will briefly consider a sub-group of V2 languages that merit to be kept apart from both “relaxed” and “strict V2” languages.

\subsection{“Semi-strict” V2 languages}\label{semi-strict}

It has sometimes been claimed (see for instance \cite{Wolfe2018}) that certain Old Romance varieties were also “strict”, like Old French, Old Northern Italian and possibly Old Spanish. While it is true that these languages do not tend to accept linear V4 orders (see \ref{Sicilian}) or topic\hyp focus combinations in the prefield, they do not strictly speaking pattern together with the Modern Germanic V2 languages empirically (as pointed out in \citealt{Kaiser2002} and many others since), since they allow the sequence “adjunct\hyp subject\hyp verb”:

\begin{exe} 
\ex\label{semi}(\textit{La Vie d'Eustace}, VI.3--4; \cite[8]{Murray1929})\\
\gll [Qant il li ot tot conté], [sa feme] s’ \textbf{escria} e li dist\ldots\\
     when he her.\textsc{cl} had all told his wife \textsc{refl} cried and him said\ldots\\ 
\glt `When he had told her everything, his wife cried out and said to him\ldots'
\end{exe}

Interestingly, this pattern mirrors with astonishing precision the V3 orders found in certain Modern Germanic dialects that could be described as “contact varieties”, like Flemish spoken in bilingual communities on both sides of the Belgian-French border (\cite{HaegemanGreco2018West}) or in the “Germanic Urban Vernaculars” that are characteristic of immigrant communities in urban areas and that have been attested widely across the Germanic family (\cite{Kotsinas1998, NistovOpsahl2014, Quist2008, Walkden2017Language, CournaneEKP2023}). Empirically speaking, then, we have a group of languages that could be described as “semi-strict” since they are stricter than the “relaxed variteties” in not allowing topic-focus combinations in the prefield, yet not as strict as the Modern Germanic V2 languages in allowing (\ref{semi}--\ref{svensk}):

\begin{exe}
\ex\label{ba}(Oostkerke West Flemish dialect; from \citealt{HaegemanGreco2018West})\\
\gll [Vroeger] [de Siphon] \textit{was} een gekende restaurant.\\ 
     previously the Siphon was a well-known restaurant\\
\glt ‘In the old days, the Siphon used to be a famous restaurant.’
\ex\label{svensk}(Swedish Urban Vernacular; From \citealt[53]{Ganuza2008})\\
\gll [då] [alle]            \textit{började} hata  henne.\\
     then everyone started       hate.\tiny{INF}   her\\
\glt `Then everyone started hating her.’ \\
\end{exe}

The important thing to observe about these varieties is that they pattern with the “strict” V2 varieties in showing the same linear V2 order of the verb and its exceptions, in other words: the CLD construction. They are accordingly to be considered a sub-group of the strict languages, from which they indeed evolve, and the same arguments that were adduced against the bottleneck hypothesis in the previous section apply with equal force; nothing new needs to be added. Nonetheless, the question of the proper analysis of the “adjunct-subject-verb” sequence is not yet agreed upon. The fundamental question is whether these sequences feature V-to-C movement at all. For an analysis based on V-to-C, see \citet{Walkden2017Actuation}, and for an analysis with only V-to-T, see \citet{CournaneEKP2023}.\footnote{A reviewer asks what the arguments are in favour of a V-to-C or V-to-T analysis of such strings. In favour of V-to-C, \citet{Walkden2017Actuation} points out that only V-to-C is consistent with the global word order facts of Continental Germanic, since it accounts for main-embedded asymmetries naturally (V2 vs. verb-final), assuming competition between the verb and the complementiser for the C-head. There is no independent evidence for V-to-T, and the V-to-C parse is needed anyway for inverted V2, which is found in the Germanic Urban Vernaculars, too. \citet{CournaneEKP2023} argue in favour of V-to-T on the basis of the empirical observation that the constituent that follows the initial adjunct is virtually without exception the subject. These V3 strings are thus always non-inverted, and this is completely surprising on a V-to-C parse, since we know that any constituent can read Spec-CP in a string featuring V-to-C. On a V-to-T parse, it follows naturally, since Spec-TP is a canonical subject position. The authors also invoke derivational economy and present arguments from bilingual L1 acquisition.}

\section{Two other approaches to (“strict”) verb-second}\label{true}

As was pointed out in \sectref{relcart} of the Introduction, the advent of cartography produced paradoxical effects for research on the verb-second phenomenon. The “relaxed V2 varieties” of Old Romance, which could not be made to fit the “classical” analysis of V2 and its unitary CP projection, were suddenly amenable to description and explanation by the new tool of CP-cartography (\cite{Beninca2006, Ledgeway2009, Poletto2014, Wolfe2018}). At the same time, the Germanic V2 languages with their highly restricted left-peripheries again appeared like a puzzle in need of fresh answers. One prominent answer, the locality\hyp principle known as the “bottleneck” (\cite{Haegeman1996, Roberts2004}) has been critically reviewed and rejected in the preceding sections. In this section, I will briefly review two further approaches in recent literature that propose different ways of capturing the facts of V2, although I must refer the reader to the relevant papers for full coverage of the relevant theories.

\subsection{A phase-based locality account of V2}

The first approach, presented in \citet{EKP2022}, is compatible with an articulated, cartographic C-domain which at least contains distinct positions for topics and foci. The idea is that locality restricts lexicalisation of C-projections, not in terms of a bottleneck in Spec-FinP that limits access to the left-periphery, but rather through the way topics, foci and phase heads (\cite{Gallego2012}) interact. In a nutshell, the hypothesis is that locality constraints are checked at the interfaces. A constituent that has been moved to the left periphery blocks fronting of another constituent \textit{within the same phase domain}. The relevant phase heads in the left periphery are assumed to be Fin$^0$ and Force$^0$, in line with Rizzi's (\citeyear{Rizzi1997}) original idea that the left periphery is essentially an interface between the propositional content expressed by the TP and the articulation of discourse. Suppose that the roles of Fin$^0$ and Force$^0$ are to introduce and terminate, respectively, this final phase of derivation. The intermediate projections host focal and topical information, the latter possibly recursive or consisting of a field of different topic projections (\cite{BenincaPoletto2004, Grewendorf2009}).

The question is how this system could make any distinction between non-V2 languages with multiply accessible left peripheries like, say, Modern Italian, and the Modern Germanic V2 languages. The obvious difference being precisely V2 itself, the hypothesis developed by \citet{EKP2022} is that the movement of the verb to Fin$^0$ has the effect of \textit{deactivating} this phase head. This builds on the cue that phases are driven by the presence of uninterpretable features on the phase heads (\cite{Chomsky2000}), since the AGR/Fin/$\phi$-feature assumed to attract the verb to Fin$^0$ is precisely of this kind. In a more general vein, the verb is the axis of the extended projection of the clause, so if the verb moves higher, the core clause extends, too.

For concreteness, the ungrammaticality of a topic-focus-verb sequence (\ref{topfoc}) in Modern Germanic V2 is thus explained by the obligatory movement of the verb to Fin$^0$ and the immediate deactivation of this phase head (\figref{FinP}). This in turn means that when the final phase head Force$^0$ is merged and its complement is sent to the interfaces, the movement of the topic across the focus is detected and ruled out:

\begin{exe} 
\ex[*]{\label{topfoc} 
\gll Den Wagen meinem VATER habe ich verkauft, (nicht meiner MUTTER).\\ 
	 The.\textsc{acc} car my.\textsc{dat} father have I sold not my.\textsc{dat} mother\\ 
\glt 'The car I sold to my FATHER (not to my mother).'
}
\end{exe}


\begin{figure}
\caption{A phase-based locality explanation of “the linear V2 constraint”}
\label{FinP} 
% old xyling:
% % % \Treek{1} { & \K{ForceP}\B{dl}\B{dr} && \K{$\scriptstyle transfer $}\GBkk{2,1}{2,1}{ddlll}{-}  \\ \K{Force$^0$}\OO && \K{TopP}\B{dl}\B{dr} \\ & \K{Den Wagen} && \K{FocP}\B{dl}\B{dr} \\  && \K{Meinem VATER} && \K{FinP}\B{dl}\B{dr} && \K{$\scriptstyle transfer$}\Below{$\scriptstyle deactivated$}\GBkk{2,1}{2,1}{ddlll}{.} \\ &&& \K{Fin$^0$}\OO \Below{\textbf{habe} [\sout{$\phi$}]} && \K{TP}\TRi[0] \\ &&&&& \K{ich \sout{\textbf{habe}} \sout{meinem VATER}}\Below{\sout{den Wagen} verkauft} }
% new forest:
\begin{forest}
[ForceP,name=fp [Force$^0$, name=f0] [TopP [DP [Den Wagen\\{[iTop]}, roof]] [FocP [DP [meinem VATER\\{[iFoc]} , roof]] [FinP, name=FinP [Fin$^0$ \\{\sout{$\phi$}} \\ \textbf{habe}, name=fin0] [TP [ich \sout{\textbf{habe}} \sout{meinem VATER} \\ den Wagen verkauft, roof] ]]]]]
\node[xshift=8mm] at (fp.south east) (tf) {\textit{transfer}};
\node[circle,minimum size=10mm,draw] at (f0) (f0circ) {};
\draw[-] (f0circ) -- (tf.south);
\node[xshift=8mm] at (FinP.east) (bn) {\textit{transfer}};
\node[xshift=8mm] at (FinP.south east) (bn) {\textit{deactivated}};
\node[circle,minimum size=10mm,draw] at (fin0) (fin0circ) {};
\draw[dotted] (fin0circ) -- (bn.south east);
\end{forest}
\end{figure}

In \citet{EKP2022}, this derivation is argued to also explain differences in topicalization between V2 languages and non-V2 languages with respect to phenomena like resumption and Weak Cross-Over effects. It derives the “linear V2-constraint” without appealing to a bottleneck induced by a blind EPP-feature, but rather by a more standard phrase-phrase locality effect which rises exceptionally in the V2 languages due to V-to-C movement. However, the model also has to assume (i) that focus has no dedicated position in the clause (in line with the conclusions in \cite{SamekLodovici2015} but against much work in cartography) and (ii) that it in fact does not even reach the C-domain at all in non-V2 languages. This is requested to explain why left peripheral topic-focus sequences are possible in the latter.\footnote{Another question is what happens in relaxed V2 languages (see \sectref{relaxed}), where topic-focus are also permitted in spite of the assumption that there is V-to-C movement in these languages. It must be emphasized, however, that the deactivation of the universal phase head Fin$^0$ does not happen automatically in a language that features V-to-C movement, but only when V-to-Fin movement actually takes place. Since the vast majority of topic-focus sequences that are reported in the literature are \textit{uninverted} (like example (\ref{Sicilian})), it is quite plausible that V-to-C movement has not taken place at all in these cases. In the absence of V-to-C in the derivation of any given clause, we expect the left periphery to behave as in the Modern Romance languages, since Fin$^0$ remains an active phase head.}

\subsection{The left periphery as a feature bundle}
\largerpage
\begin{sloppypar}
Another recent proposal on verb-second departs more radically from all approaches discussed previously in this paper by explicitly rejecting the existence of dedicated topic and focus projections in the left periphery of V2 languages (\cite{CournaneEKP2023}). This also amounts to dismissing cartographic claims to universality of the functional sequence of the C-domain. It does not, however, entail a repudiation of the findings of cartography nor a return to the “traditional” analysis of verb-second outlined in the Introduction. What is rejected is the particular cartographic hypothesis of “One-Feature-One-Head” (\cite{Kayne2005}) which takes every feature with interpretable effect to be associated with a dedicated head that projects a specifier. Building on ideas proposed by \citet{GiorgiPianesi1997} and developed in \citet{Hsu2017}, the authors suggest that children acquiring a language dissociate the task of detecting what features are active in the language from the structure building process, in other words the postulation of projections in syntax. The latter process is assumed to be guided by a conservative approach, where children try to make do with what is already known. This leads children to “bundle”, in other words to assign several feature to the same head and to exploit the corresponding specifier for multiple purposes. If confronted with sufficient evidence, in the form of co-occurrence of phrases, that different features are carried by different heads, children will eventually “unbundle” (“unravel” in \cite{Pannemann2007}) and assign the relevant features to distinct projections, thereby expanding the phrase marker. Concretely, the absence of topic-focus sequences (and several other V3 combinations) in verb-second languages means no “unbundling” and feature reassignment takes place, such that the syncretic C-domain matures into a stable, adult I-grammar (\figref{bundle}).
\end{sloppypar}

\begin{figure} 
\caption{The feature bundle-approach of \citealt{CournaneEKP2023}}
\label{bundle} 
% old xyling:
% % % \Treek{1} { & \K{FrameP}\B{dl}\B{dr} \\  \K{} && \K{CP}\B{dl}\B{dr} \\ & \K{} && \Cbar \\ && \K{C$^0$}\Below{[Top]/[Foc], [Fin]} && \K{TP}}
% new forest:
\begin{forest} 
[FrameP [Dislocated XP] [CP [XP] [C' [C$^0$\\ {{[Top], [Foc], [Fin]}}] [TP [..., roof]]]]]
\end{forest}
\end{figure}

\largerpage
In this phrase marker, the role of the "FrameP" is to host constituents like the initial topic in the CLD and the HTLD-constructions (see \sectref{strict}), the former through movement, the latter presumably through base\hyp generation. It could of course also be the case that this is precisely the projection that hosts the initial adjunct in the “semi\hyp strict” varieties reviewed in \sectref{semi-strict}. The syncretic position, for simplicity called CP rather than something cumbersome like "TopP/FocP/FinP", is the highly diverse prefield of V2 languages, always filled through movement from the core clause.\footnote{There is evidence that ForceP is also a distinct position in the left periphery of at least some Germanic V2 languages. The reason is that embedded V2 is perfectly possible in all V2 languages in a subset of complement clauses and certain adverbial clauses. While languages like German need to omit the complementiser in complement clauses for verb raising to be acceptable, the Mainland Scandinavian languages display verb\hyp second inversion even in the presence of a complementiser (\cite{Vikner1995,HolmbergPlatzack1995, EKP2019}). This clearly suggests that the complementiser in question is Force$^0$, since Fin$^0$ is lexicalised by the finite verb. In other words, ForceP dominates FinP, as is well-known from the cartographic literature, but there is in fact evidence that it also dominates FrameP, since it seems possible to embed the CLD-construction in a language like Norwegian. Furthermore, the initial adjunct in “adjunct-subject-verb” sequences (\sectref{semi-strict}) is clearly embeddable in Old French (see \cite{EKP2019}; chapter 3). Since this has no bearing on the bottleneck hypothesis or the question of the linearization of the verb, I leave it out of the representation in (\ref{bundle}).}


\section{Summary}

In this paper, I have presented arguments against the bottleneck hypothesis that is often invoked in the literature to account for the restricted left\hyp periphery of verb\hyp second languages. On the theoretical side, the bottleneck hypothesis does not follow naturally from the theory of locality on which it essentially hinges, Relativised Minimality. Purely formal, EPP-induced movement should not block more motivated movement like topicalization and focalization. 

On the empirical side, I have shown that the bottleneck hypothesis does not really yield interesting predictions for the so-called “relaxed V2” languages.  Furthermore, there exists direct counterevidence against the bottleneck in the “strict V2 languages” of the Modern Germanic kind, since at least some cases of the Contrastive Left Dislocation construction very plausibly involve more than one instance of internal merge from the core clause to the left-periphery.

The fact remains, obviously, that the grammar of the Modern Germanic V2 languages somehow imposes restrictions on the combinations of constituents in the prefield. In this paper, I have also drawn attention to other possible approaches to this problem. These include the hypothesis of a “feature bundle” in the C-domain that syncretises several features into a single projection, or a phase\hyp based derivation within an articulated left\hyp periphery, where V-to-C movement interferes with the phase head Fin$^0$, with concomitant effects on locality. More research is needed to see if these approaches, or others,  can account for the linearization patterns of verb-second languages. Given the important arguments against the bottleneck hypothesis, the endeavour is certainly worthwhile.

%The final section has reviewed approaches to the restricted prefield of Modern Germanic V2 languages that do not rely on “bottleneck”-locality. Given the theoretical and empirical problems encountered by the “bottleneck”,

\section*{Acknowledgements}
I would like to thank two anonymous reviewers for constructive feedback on a previous version of this paper. Thanks also go the audience at the workshop “Cracks in the Bottleneck”, held at the Sorbonne University in Paris, February 16--17 2023.

\printbibliography[heading=subbibliography,notkeyword=this]
\end{document} 
