\documentclass[output=paper]{langscibook}
\ChapterDOI{10.5281/zenodo.17077196}
\author{Luca Riccardelli\orcid{}\affiliation{Università La Sapienza di Roma}}
\title{V2 and V3 in Rhaeto-Romance varieties: Rethinking language contact}
\abstract{The present contribution aims at re-evaluating the role of language contact in the emergence of V2 phenomena in Rhaeto-Romance varieties. Through a survey on the patterns of variation between V2 and V3 structures, the paper attempts (1) to determine the position of Rhaeto-Romance varieties within the spectrum of V2 languages devised by \citet{CruschinaSailor2022}, while also contextualising their development from Old Romance in accordance with \citegen{Wolfe2019Redefining} diachronic typology, and (2) to verify whether it is possible to draw a direct correlation between higher degrees of language contact and stricter V2 constraints. If, from a general perspective, the varieties that are characterised by a closer and more prolonged contact with German (such as Swiss Romansh and the Ladin varieties of South Tyrol) generally appear to exhibit V2 more systematically, a closer scrutiny of the single varieties and their instances of micro-variation does not support the hypothesis of a direct proportionality between syntactic structures and language contact: other factors, such as information structure, appear to override the general predictions. Furthermore, as stated in \citet{CruschinaSailor2022}, the literature on the diachronic development of V2 has mostly been focused on the loss or preservation of V2, rather than its extension or emergence. The paper thus also takes into account examples of V2 structures that appear to be emerging among the youngest informants of the Rhaeto-Romance varieties lacking a significant exposure to German influence (namely Friulian, Fassan and Fodom). Cumulatively, these observations (1) suggest that the syntactic system of Rhaeto-Romance presents a general tendency to develop V2 (as argued by \citealt{KaiserHack2013}) and (2) confirm that language contact does not constitute a trigger for the emergence of V2 in Rhaeto-Romance. Rather, it can act as a catalyst for the retention or reinforcement of pre-existing features.}
\IfFileExists{../localcommands.tex}{
  \addbibresource{../localbibliography.bib}
  % add all extra packages you need to load to this file

\usepackage{tabularx,multicol}
\usepackage{url}
\urlstyle{same}

\usepackage{listings}
\lstset{basicstyle=\ttfamily,tabsize=2,breaklines=true}

\usepackage{langsci-basic}
\usepackage{langsci-optional}
\usepackage{langsci-lgr}
\usepackage{langsci-osl}
% \usepackage{./langsci/styles/langsci-lgr}
% \usepackage{./langsci/styles/langsci-osl}
% \usepackage{langsci-gb4e}

\usepackage{tikz}
\usetikzlibrary{patterns,calc}
\pgfdeclarepatternformonly{south east lines}{\pgfqpoint{-0pt}{-0pt}}{\pgfqpoint{3pt}{3pt}}{\pgfqpoint{3pt}{3pt}}{
    \pgfsetlinewidth{0.6pt}
    \pgfpathmoveto{\pgfqpoint{0pt}{3pt}}
    \pgfpathlineto{\pgfqpoint{3pt}{0pt}}
    \pgfpathmoveto{\pgfqpoint{.2pt}{-.2pt}}
    \pgfpathlineto{\pgfqpoint{-.2pt}{.2pt}}
    \pgfpathmoveto{\pgfqpoint{3.2pt}{2.8pt}}
    \pgfpathlineto{\pgfqpoint{2.8pt}{3.2pt}}
    \pgfusepath{stroke}}
    
\usepackage{stmaryrd}
\usepackage{wasysym}
\usepackage{multirow}
\usepackage{caption}
\usepackage{subcaption}
\usepackage{mathrsfs}
\usepackage{qtree}

\usepackage{linguex}


  %pminos do not split footnotes
% \interfootnotelinepenalty=10000 %Footnote in Laporte chapters has to be split SN


%\DeclareIndexNameFormat{default}{%
%\nameparts{#1}%
%\usebibmacro{index:name}%
%{\index[names]}%
%{\namepartfamily}%
%{\namepartgiveni}%
% {}% L1
% {}% L2
%{\namepartprefix}% generates spurious space L3
%{\namepartsuffix}% generates spurious space L4
%}

%  {\DeclareIndexNameFormat{default}{%
%     \usebibmacro{index:name}{\index[names]}{#1}{#3}{#5}{#7}}}

%\DeclareIndexNameFormat{default}{%
%  \usebibmacro{index:name}{\sindex[nom]}{#1}{#3}{#5}{#7}}

%\DeclareIndexNameFormat{default}{%
%  \usebibmacro{index:name}{\sindex[person]}{#1}{#3}{#5}{#7}}
%\DeclareIndexNameFormat{default}{%
%\nameparts{#1} \usebibmacro{index:name}{\sindex[person]]}{\namepartfamily}{‌​\namepartgiven}{\nam‌​epartprefix}{\namepa‌​rtsuffix}}

%\newcommand{\smiley}{:)}

%\renewbibmacro*{index:name}[5]{%
%\usebibmacro{index:entry}{#1}%
%{\iffieldundef{usera}{}{\thefield{usera}\actualoperator}\mkbibindexname{#2}{#3}{#4}{#5}}}

% \newcommand{\noop}[1]{}

%remove for final
%\overfullrule=1mm

\newcommand{\tobi}[2]}}
\renewcommand{\S}[1]{\tobi{#1}{\textsc{*}}}

% this volume references
% puts: [this volume]
% already defined: \citetv
%\newcommand{\citepv}[1]{(\citeauthor{#1} \citeyear*{#1} [this volume])}
\newcommand{\citealtv}[1]{\citeauthor{#1} \citeyear*{#1} [this volume]}

%parentheses around example number
\newcommand{\pref}[1]{(\ref{#1})}

% in-text examples

\newcommand{\lnex}[1]{\textit{#1}} %target lang word
\newcommand{\lnlit}[1]{(lit.: `#1')} %literal reading
\newcommand{\lnlat}[1]{(#1)} % latinization
\newcommand{\lntrans}[1]{`#1'} %translation
\newcommand{\lnexl}[2]%
{\lnex{#1}{} \lnlat{#2}} % ex with latinization
\newcommand{\lnexlat}[3]{\lnex{#1}{} \lnlat{#2}{} \lntrans{#3}} % ex with latinization and tranl.

%ch01
\newcommand{\co}[1]{\mbox{\textbf{#1}}}

%ch09

\newcommand{\cyrbulg}[1]{\begin{otherlanguage*}{bulgarian}#1\end{otherlanguage*}}


%ch10
\newcommand{\nlp}{{\small NLP}}
\newcommand{\mwe}{{\small MWE}}
\newcommand{\rae}{{\small RAE}}
\newcommand{\lvc}{{\small LVC}}
\newcommand{\pos}{{\small P}o{\small S}}
%\newcommand{\todo}[1]{ \textcolor{red}{#1} }

%\renewcommand{\labelenumi}{\theenumi}
%\ainamefmt{{vv}{ll}{, ff}{, jj}} % fullname

\newcommand{\biberror}[1]{{\color{red}#1}}

\newcommand{\osenovaitem}{--~} 
  %% hyphenation points for line breaks
%% Normally, automatic hyphenation in LaTeX is very good
%% If a word is mis-hyphenated, add it to this file
%%
%% add information to TeX file before \begin{document} with:
%% %% hyphenation points for line breaks
%% Normally, automatic hyphenation in LaTeX is very good
%% If a word is mis-hyphenated, add it to this file
%%
%% add information to TeX file before \begin{document} with:
%% %% hyphenation points for line breaks
%% Normally, automatic hyphenation in LaTeX is very good
%% If a word is mis-hyphenated, add it to this file
%%
%% add information to TeX file before \begin{document} with:
%% \include{localhyphenation}
\hyphenation{
    Beck-man
    Ngu-yen
    back-chan-nel
    back-chan-nels
    mo-not-o-nous
    ste-reo-typ-i-cal
}

\hyphenation{
    Beck-man
    Ngu-yen
    back-chan-nel
    back-chan-nels
    mo-not-o-nous
    ste-reo-typ-i-cal
}

\hyphenation{
    Beck-man
    Ngu-yen
    back-chan-nel
    back-chan-nels
    mo-not-o-nous
    ste-reo-typ-i-cal
}
 
  \togglepaper[1]%%chapternumber
}{}

\begin{document}
\maketitle 
%\shorttitlerunninghead{}%%use this for an abridged title in the page headers


\section{Introduction}
\label{sec:riccardelli:1}%1 /
\subsection{Aim of the study}
\label{sec:riccardelli:1.1}

Since the publication of the monograph by \citet{BenincaHaiman1992}, and later with the extensive work by \citet{Poletto2000, Poletto2002}, the research concerning peculiar features of Rhaeto-Romance – given also its sociolinguistic and geographic position between Romance and Germanic – has often proven itself to be greatly beneficial for the deeper understanding of syntactical phenomena both on a synchronic and a diachronic level of analysis.

The purpose of the present contribution is to explore the issue of V2 and V3 word orders in Rhaeto-Romance varieties from a historical sociolinguistic perspective, with a specific focus on their relationship with Germanic varieties, whether Standard German or German dialects spoken in the area. Crucially, this paper does not aim at proposing a novel theory for the derivation or the representation of V2 and V3 structures in Rhaeto-Romance languages; rather, it strives at recontextualising them in light of the most recent developments in the research concerning the so-called “relaxed” and “residual” V2 languages.

\subsection{Defining Rhaeto-Romance}
\label{sec:riccardelli:1.2}%1.2 /

Firstly, it is essential to understand what is meant by Rhaeto-Romance, which as a label has been varyingly applied to a number of languages and varieties throughout the history of the discipline.\footnote{For further discussion, see \citet{BenincaHaiman1992} and \citet{Salvi2016}. \citet{Varga2009} provides an interesting study from a syntactic perspective. For a recent publication opposing the hypothesis of a cohesive Rhaeto-Romance family, see \citet{DeCiaIubiniHampton2020}.} In this paper, Rhaeto-Romance is used to refer to the following languages and their respective diatopic varieties:

\begin{itemize}
\item Romansh

\begin{itemize}
\item Sursilvan (incl. Tuatschin)
\item Sutsilvan
\item Surmeiran
\item Engadinese (Jauer, Putèr, Vallader)
\end{itemize}
\item Dolomitic Ladin

\begin{itemize}
\item Gaderan (Badiot, Mareo)
\item Gardenese
\item Fassan
\item Fodom
\end{itemize}
\item Friulian

\begin{itemize}
\item Carnic
\item Western
\item Central-Eastern
\end{itemize}
\end{itemize}

While the genealogical connection between Rhaeto-Romance languages remains a matter of ongoing debate, these linguistic varieties collectively exhibit a set of~– mostly conservative~– features, effectively setting them apart from other Northern Italian and Gallo-Italic varieties. Conveniently, the issue is not particularly relevant for the purposes of the present study. Instead, the focal point resides in the intricate sociolinguistic and geographic context in which these varieties are spoken~-- regions characterised by substantial and prolonged language contact. It is also crucial to emphasise that, although official, these all constitute minority languages, with a rather limited number of speakers.

Indeed, Romansh is one of the four official languages of Switzerland (along German, French and Italian), counting approximately 60,000 speakers, mostly in the canton of Graubünden (rom. \textit{Grischun}, fr. \textit{Grisons}, it. \textit{Grigioni}). Virtually all speakers are at least bilingual, with a majority of the population having German as an L2, and some cases of Italian as an L3. It should however be noted that,  historically speaking, the language contact mainly involved the regional varieties of German and Italian, rather than the standard languages themselves. Contact with the Italo-Romance varieties is predictably more consistent in the southernmost part of the canton (\citealt[170]{Anderson2016}).

The situation of Dolomitic Ladin\footnote{Dolomitic Ladin (sometimes referred to as \textit{Sella dialects}) constitutes the best-preserved dialect subdivision of Ladin – which, depending on the theoretical framework of the different scholars, also includes the Agordo and Cadore dialects, as well as the Anaunico group – and the one that underwent deepest investigation. In earlier studies (such as \citealt{Belardi1991} and \citealt{BenincàHaiman1992}), the group also included Ampezan, which is now often considered to be a Cadore variety (see \citealt[155]{Salvi2016}).} is equally interesting: Mareo, Badiot and Gardenese~– collectively definable as northern varieties (\citealt[149]{Casalicchio2020})~– are spoken in South Tyrol, a region characterised by a high degree of bilingualism, with a slight preponderance of the German element. On the other hand, the southern varieties Fassan and Fodom (also known as Livinallese) are, respectively, spoken in the regions of Trentino and Veneto, thus in an Italian-speaking area. The approximate number of Ladin speakers is 25,000--30,000 (\citealt[156]{Salvi2016}).  

Friulian constitutes the language with the highest number of speakers (as many as 600,000\footnote{According to the Regional Agency for the Friulian Language’s (Agjenzie Regjonâl pe Lenghe Furlane) 2014 census.}). The linguistic landscape of the Friuli-Venezia Giulia region is particularly fascinating: along with Italian and German, Friulian shares a border with Slovene, which also resulted in a number of loanwords (\citealt{SpinozziMonai2015, Frau2015}). 

\section{V2 and contact in Rhaeto-Romance}
 \label{sec:riccardelli:2}%2 /
\subsection{Areal distribution}
\label{sec:riccardelli:2.1} %2.1 /

The question concerning the possibility of contact-related effects in V2 structures in Rhaeto-Romance thus arises from the observation of their distribution across the different varieties. Indeed, V2 appears to be more systematic in those dialects that are characterised by a more consistent contact with German. Consider the following examples from \citet[170--174]{BenincàHaiman1992}:

\ea Sursilvan\\\label{ex:riccardelli:1}
\gll Ed aschia   fa=el     il   patg  cul     nausch\\
     and so     makes=he   the   pact   with.the   devil\\
\glt `Thus, he makes the pact with the Devil.'
\ex Putèr\\\label{ex:riccardelli:2}
\gll Eir   in Grischun   vains nus industrias   chi prodüan auncha memma bger   tössi   per l' ajer\\
     even   in Grisons   have we industries   that produce also   too much poison for the air\\
\glt `Even in Grisons, we have industries that produce too much poison in the air.'
\z

The emergence of XVS order and the avoidance of V3 is therefore described as “general” across Swiss Romansh varieties. As observed later in the present paper, there are indeed certain contexts that license V3 in Romansh dialects and, to a certain extent, their areal distribution still appears to align with the varying degree of German contact (i.e., V3 is more frequent in Engadinese than elsewhere). Indeed, the same pattern is observed in Dolomitic Ladin:

\ea Badiot\\\label{ex:riccardelli:3}
\gll inshö e l alkol   ruve   a fa   pert da nosta alimentatsiun\\
     thus   is the alcohol arrived to make part of our diet\\
\glt `Thus alcohol has become a part of our diet.'
\ex Gardenese\\\label{ex:riccardelli:4}
\gll ilò   a=l   scumencià   a mene na   stleta vita\\
     there has=he begun   to lead a   bad   life\\
\glt `There he began to lead a dissolute life.'
\ex Fassan\\\label{ex:riccardelli:5}
\gll dapò da sera,   la vaces   ju    n stala   i   se   perla\\
     after at evening the cows   down in   stable they   \textsc{refl}  talk\\
\glt `Afterwards, in the evening, the cattle down in the stable talk among themselves.'
\ex Fodom\\\label{ex:riccardelli:6}
\gll Nte le Filippine   mpruma l a   mpare   la   linga   visaja\\
     in the Philippines   first   he has learned   the language   Visayan\\
\glt `It was in the Philippines that he first learned Visayan.'
\z



The correlation between German or Italian influence and the presence or absence of V2 in main declarative clauses is quite apparent. Quoting directly from \citet[171]{BenincàHaiman1992}:

\begin{quote}
The geographical distribution of this feature clearly suggests the importance of German influence: the closer to Bressanone/Brixen, the greater the influence of German; the closer to Trento, the less the influence of German.
\end{quote}

Consistently with this view, no subject-verb inversion is found in main declarative clauses in Friulian:

\ea Friulian\\\label{ex:riccardelli:7}
\gll Une sabide   matine   il Pari   Eterno   al   puarte a ciase un biel ciavret\\
     one Saturday morning the Father Eternal he   brings at home a fine kid\\
\glt `One Saturday morning, the Eternal Father brings home a fine kid.'
\z

Notably, all the Friulian examples provided by \citet[171]{BenincàHaiman1992} feature an initial Frame Setter – a quite pervasive V3 trigger, even in so-called “strict” V2 languages (see e.g. \citealt{GrecoHaegeman2020}) –, thus providing no information about the status of orders such as Adv-Subj-V. 

A swift query on the CLAAP (Center for Applied Linguistics) Friulian Corpus\footnote{\url{https://claap.org/corpus-furlan/}} allows us to observe a few instances of V3 in main declarative clauses introduced by an adverb:

\ea Friulian\\\label{ex:riccardelli:8}
\gll Inmò   nissun   sa   che   la   fantate e   je   in cjase   tô\\
     Still   nobody  knows that  the  girl  she  is  in house  your\\
\glt `No one knows yet that the girl is in your house.'
\z

On the other hand, compatibly with \citegen{BenincaHaiman1992} description, no examples of V2 are found for adverb-introduced main declarative clauses. 

The pattern is undoubtedly striking, initially leading to a quite widespread consensus concerning the relevance of German influence:

\begin{itemize}
\sloppy
\item The distribution of subject-verb inversion within Rhaeto\hyp Romance strongly supports the hypothesis that XVS word order (whether in itself an inherited feature, as argued in \citealt{Benincà1985} and \citealt{Vanelli1984}, or a later development, as suggested in \citealt{Kuen1957}, \citealt{Haiman1974}, and \citealt{Hilty1975}) is a result of German influence \citep[174]{BenincàHaiman1992}.

\item \textit{Unser Phänomen tritt nur in den Gebieten auf, die stark von deutschen Mundarten geprägt wurden.} [...] \textit{Aufgrund dieser Verhältnisse, insbesondere aufgrund der differenzierten Sachlage im Dolomitenladinischen, ist der Schluß zu ziehen, daß die Herausbildung dieses besonderen Systems der Inversion und der Inversionsformen im Rätoromanischen mit deutschem Einfluß in Verbindung zu bringen ist}.\footnote{“Our phenomenon appears only in those areas which have been strongly influenced by German dialects.[...]. Because of these circumstances, in particular because of the complex situation in Dolomitic Ladin one has to conclude that the evolution of this particular system of inversion and inversion forms in Rhaeto\hyp Romance should be connected to German influence.”} (\citealt[95]{Linder1987}).
\end{itemize}


\subsection{Against the contact shortcut}
\label{sec:riccardelli:2.2}
However, the nature of such influence, as well as the modalities in which it operates, constitute a matter of debate. Indeed, especially in the early history of Rhaeto\hyp Romance studies, many syntactic traits have been far too hastily described as being triggered by contact with German and its alpine varieties. This is for example the case of Verb+Adverb constructions attested in Ladin, which were earlier described  as a form of syntactic calque on German separable verbs \citep{Plangg1980}.


\ea%9
    Gardenese
    \label{ex:riccardelli:9}
      \ea \textit{pensé do} (lit. “think behind”) < Ger. \textit{nach-denken}, “to reflect”;
      \ex \textit{mëter pro} (lit. “put nearby”) < Ger. \textit{zu-setzen}, “to add”.
      \z
      \ex Badiot \label{ex:riccardelli:10}
      \ea \textit{tó sö} (lit. “take up”) < Ger. \textit{auf-nehmen}, “to collect, record”;
      \ex \textit{odëi ite} (lit. “see inside”) < Ger. \textit{ein-sehen}, “to understand”.
      \z
\z

While these constructions display an undeniable resemblance with their German counterparts and constitute a significantly productive phenomenon, they represent lexical calques rather than syntactic ones. Indeed, such verbal constructions are also commonly found in other Northern Italian dialects that are not characterised by significant contact with German and, therefore, could not borrow a syntactic structure from it. 

Furthermore, syntactic interference phenomena – especially in the form of borrowings and calques – are by themselves a highly debated subject, with prominent scholars within the generative framework (such as \citealt{Baker1996} and \citealt{Longobardi2001}) deeming them to be marginal at best and utterly unconceivable at worst. Less extreme views allow contact-induced syntactic changes provided a superficial correspondence of the interested syntactic structures (\citealt[191]{Bidese2023}) and/or a high degree of bilingualism (\citealt[98--99]{Benincà1994}; cfr. \citealt[318]{Trudgill2016}).\footnote{See \citet[121--150]{HarrisCampbell1995} for several examples of syntactic interference between typologically distant languages.} 

However, such bilingualism did not actually characterise the majority of the Ladin community until the latter half of the 19\textsuperscript{th} century (\cites[135--141]{Belardi1991}[99]{Benincà1994}). The hypothesis of the independent emergence of these forms is therefore confirmed by the observation of diachronic data from Old Italian and Old French (\citealt[84]{Cordin2011}) as well as Ladin itself, where “phrasal verbs” are already attested in documents from the 16\textsuperscript{th} and 17\textsuperscript{th} centuries (\citealt[283]{Belardi1991}). 

In essence, caution is highly required when trying to ascribe the emergence of certain features to the effect of language contact, which – while not necessarily subordinated to other sources of language change – can only be reliably considered the triggering factor when no adequate internal factor can be individuated.\footnote{For an in-depth discussion of the interplay between intra- and extra-linguistic factors see \citet{FarrarJones2002}.} However, contact can contribute to the preservation or the spread of such features, as later discussed in the present study.

The same principle applies to the opposite situation, as in the case of Mòcheno, a Germanic language spoken in the Trentino region. The sociolinguistic situation of Mòcheno shares multiple points of resemblance with that of several Rhaeto-Romance varieties: it constitutes a speech\hyp island and virtually all of its speakers (about 1,300) are bilingual. Therefore, many of its properties, such as so-called \textit{relaxed} V2 (see below), have been considered to be an effect of language contact with Standard Italian and the neighbouring North Italian varieties. However, \citet{Cognola2015} argues that V3 – as well as other syntactic traits such as Verb-Object order – cannot be attributed to “competing grammars”, i.e. the availability for the speakers to rely on different syntactic structures from different languages within their competence, as previously stated by \citet{Rowley2003}, among others.

Instead, taking syntactic contexts and informational structures into account, the V3 phenomenon found in Mòcheno differs consistently from both German and Italian. As explicitly stated in an earlier work \citep{Cognola2013}:

\begin{quote}
contrariamente all’ipotesi che la variazione sia il risultato della presenza di due grammatiche in competizione – una conservativa di tipo tedesco, ed una innovativa di tipo romanzo – la distribuzione dei due ordini sintattici OV/VO [è] sensibile in mocheno al tipo di frase.\medskip\\
`contrary to the hypothesis that variation is the result of the presence of two competing grammars – a conservative one of the German type and an innovative one of the Italian type – the distribution of the VO/OV syntactic orders in Mòcheno is sensitive to the sentence type.'
\end{quote}

Besides providing a valuable explanation in terms of internal grammar that can be equally applied to the issue of V2 and V3 phenomena in Rhaeto-Romance, the previous quote also offers a valuable insight from a diachronic perspective. 

Indeed, it can actually be argued that German syntax is not necessarily more conservative than the Romance one. In his typology of V2, \citet{Wolfe2019Redefining} provides a categorisation of V2 through various diachronic stages of both Germanic and Romance languages, observing that deviations from V2 are indeed more frequently attested in Early Germanic  and Romance languages than in their modern counterparts (\citealt[3]{Wolfe2019Redefining}). Indeed, V3 is attested as late as the 15\textsuperscript{th} century in Early New High German (\citealt[481]{Speyer2008Doppelte}):

\ea Early New High German, ca. 1450\label{ex:riccardelli:11}\\
\gll Dar nach die edel kunginn fuer enhalb Ofen [...] mit grossem kummer\\
     after that the noble queen went beyond Ofen [...] with great concern\\
\glt `Thereafter the noble queen went, with great concern, beyond Ofen'
\z

The construction displayed in the previous example is virtually identical to the one attested in the following Old Venetian sentence (\citealt[11]{Wolfe2019Redefining}):

\ea Old Venetian, 14\textsuperscript{th} century\label{ex:riccardelli:12}\\
\gll Unde   Brat   levà la   man\\
     thus  Brat   raised   the   hand\\
\glt ‘Brat then raised his hand’ 
\z

\subsection{A typology of V2} %2.3 /
\label{sec:riccardelli:2.3}
At this point, in order to avoid confusion, it is crucial to provide a brief terminological overview of V2. The most general description of the V2 property is probably the one proposed by \citet[375]{Holmberg2015}, involving two core components:

\begin{itemize}
\item A functional head in the left periphery attracts the finite verb.
\item This functional head requires that a constituent be moved to its specifier position.
\end{itemize}

Following \citet{Wolfe2019Redefining} I therefore assume that the left periphery is constituted by the following functional fields, as initially described by \citet[283]{Rizzi1997}: 

\begin{quote}
[Frame…. [Force…. [Topic…. [Focus…. [Fin…. [TP…. ]]]]] 
\end{quote}

The detail that concretely sets apart the so-called strict and relaxed V2 languages is the functional head that attracts the finite verb: this is Force in the case of strict V2 (virtually all the standard Germanic languages with the exception of English) and Fin in the case of relaxed V2 (such as Old Romance and Early Germanic).\footnote{However, \citet{Catasso2024} argues for the classification of Modern German as FinP V2. Within Early Germanic, Old Saxon already shows strong tendencies towards generalised V-to-Force\textsuperscript{0} movement as early as the 9\textsuperscript{th} century (\citealt[15--29]{Walkden2015}). An alternate account of strict vs. relaxed V2 based on the feature scattering hypothesis (\citealt{GiorgiPianesi1997}) is proposed by \citet{Hsu2017}.} 


\begin{figure}
% % % \includegraphics[width=\textwidth]{figures/a08RiccardelliV2V3updatedabstract-img001.png}
\fittable{
\begin{tikzpicture}
  \node(left)[rounded corners, draw, text width=3.2cm,align=center,text depth=5.3cm]{Fin-V2 system\bigskip

  Frame-Setter + Topic + Focus\bigskip

\textit{Early Medieval Romance,
  Later Old Occitan and Sicilian;
  Middle Low German,
  Early Old High German,
  Old English}
  };
  \node(center)[rounded corners, draw,right=of left, text width=3.3cm,align=center,text depth=5.3cm]{Force-V2 system (1)\bigskip

  Frame-Setter + Topic/Focus\bigskip

  \textit{Later Old French,
  Spanish, Venetian;
  Later Old, Middle, New High German;
  Surmeiran and Vallader Rhaeto-Romance}
  };
  \node(right)[rounded corners, draw,right=of center, text width=3.3cm,align=center,text depth=5.3cm]{Force-V2 system (2)\bigskip

  Frame-Setter\textsubscript{\textsc{ht/ld}} + Topic/Focus\bigskip

  \textit{Modern German and Dutch, San Leonardo Rhaeto-Romance}
  };
  \draw[-{Triangle[]},thick,dashed](left)--(center);
  \draw[-{Triangle[]},thick,dashed](center)--(right);
\end{tikzpicture}
}
\caption{Types of V2 systems (\citealt[14]{Wolfe2019Redefining}).}
\label{fig:riccardelli:1}
\end{figure}

As shown in \figref{fig:riccardelli:1}, according to \citegen{Wolfe2019Redefining} typology, Rhaeto-Romance languages are classified within two subsets of strict V2. Surmeiran and Vallader are listed as type 1 Force-V2 languages in which the verb, moved to Force, can be only preceded by a Frame Setter and a topicalised or focussed element:\footnote{Notice the similarity between these two utterances and the examples in \REF{ex:riccardelli:11} and \REF{ex:riccardelli:12}.} 

\ea Surmeiran\\
\gll La seira         anturn las nov Tina     sa    prepara per   sorteir\\
    the evening around the nine Tina   \textsc{refl}   prepares to   go.out\\
\glt `In the evening around nine, Tina gets ready to go out.'

\ex  Vallader\\
\gll in quel istess mumaint il trid uorsin as transmüdet in ün bel   prinz\\
     in this same moment the ugly bear \textsc{refl} transforms in a handsome prince\\
\glt `In this moment the ugly bear changed into a handsome prince.'
\z

On the other hand, the Badiot dialect of San Leonardo (extensively analysed by \citealt{Poletto2000,Poletto2002}) only licenses V3 when the first of the two preverbal constituents is expressed as a Hanging Topic (HT) or via Left Dislocation (LD):

\largerpage
\ea Badiot (S. Leonardo)\\
\gll L liber, a Giani ti   l'ai   bel     dé\\
     the book, to Giani it have=I already   given\\
\glt `I already gave the book to John.'
\z

The data from S. Leonardo is therefore in contradiction with the contact hypothesis, since Romansh varieties (or at least Surmeiran) would be expected to exhibit stricter V2 constraints. As shown later, more complex and reliable data will emerge through the closer observation of the singular varieties. 

For the purposes of this paper, a further difference is marked between \textit{partial} V2 and \textit{residual} V2, as defined in a fairly recent contribution by \citet{CruschinaSailor2022}:\footnote{Notice that this typology may differ greatly from previous accounts relying on the same or similar labels (e.g. \citealt{Rizzi1996} or \citealt{Klævik-Pettersen2022}).}

\begin{itemize}
\item A language displays a \textit{partial} V2 system if the only clause types in which V2 emerges are non-declarative.\footnote{By the authors’ own admission, the “non-declarative” label requires further refinement, since English also exhibits V2 in a limited set of declarative clauses (see \citealt[127]{Sailor2020}).} Although V2 is only exhibited in limited environments, its derivation still follows the rules described in \citet{Holmberg2015} and the language is considered to be a V2 language.
\item A language displays \textit{residual} V2 if some syntactic environments display superficial V2 order as a vestige of an earlier stage when the language exhibited Full V2 (i.e. either strict or relaxed V2, see figure below). However, in this case, the derivation of the structure does not involve V-to-C movement and the subsequent movement of a constituent to its specifier position. Thus, the language is not considered to be a V2 language.
\end{itemize}

\begin{sloppypar}
According to this framework, despite displaying a superficially analogous word order in \textit{wh}{}-questions, Modern English  qualifies as a partial V2 language (since it exhibits V2 in a limited set of environments), while the Modern Romance languages are considered as residual (thus non-) V2, as summarised in \figref{fig:riccardelli:2}.
\end{sloppypar}

  
\begin{figure}
% % % \includegraphics[width=\textwidth]{figures/a08RiccardelliV2V3updatedabstract-img002.png}
\includegraphics[width=\textwidth]{figures/a08RiccardelliV2V3updatedabstract-img002.pdf}
\caption{V2 Spectrum (Adapted from \citealt{CruschinaSailor2022}).}
\label{fig:riccardelli:2}
\end{figure}

\subsection{Partial and residual V2 in Rhaeto-Romance}  %2.4 /
\label{sec:riccardelli:2.4}
Once such categories have been defined, it is possible to place the different Rhaeto\hyp Romance varieties on this spectrum according to their behaviour and to verify if a relationship between such behaviour and the degree of contact with German varieties exists. Let us start from Fassan, Fodom and Friulian which, as previously observed, display no V-to-C movement in main declarative clauses, thus not qualifying as Full-V2 languages according to the proposed terminology. 

\citet[180--182]{Munaro2011} proves quite convincingly that the subject (clitic)-verb inversion attested in Central-Eastern Friulian in a rather wide set of non\hyp declarative clauses – mostly connoted by an attitude of uncertainty or disbelief regarding the content of the proposition (see also \citealt{Poletto2000}) – is expressed through the raising of the inflected verb to a functional head in the C-layer. Consider the following examples:

\ea Central Friulian \label{ex:riccardelli:16}
\ea \label{ex:riccardelli:16a}
    \gll ce   mi     toci=al   di   vjodi!\\
         what   me-obj.cl   touches=it   of   see\\
    \glt `What am I forced to see!'

\ex \label{ex:riccardelli:16b}
    \gll  ti     vess=jo   dit   la veretàt!\\
          you\textsc{{}-}obj.cl.   had=I     told   the truth\\
    \glt `Had I only told you the truth!'

\ex \label{ex:riccardelli:16c}
    \gll vinisi=al   tjo   pari, o podaresin   là\\
         came=he   your father, we could   go\\
    \glt `If your father came, we could go.'
\z
\z

If Munaro’s observation is correct, then Friulian (or at least its Central-Eastern variety) does indeed satisfy the first requirement for V2. However, the second requirement – i.e. the attraction of a constituent to the specifier position of this functional head – only appears to be overtly satisfied in (16.a). A V2 analysis for the latter examples is possible under the assumption – shared e.g. by \citet[98--99]{Haegeman1995} and \citet[254]{Grosz2012} and included in \citegen[20]{Wolfe2019Redefining} typology – of a covert operator serving as XP1 in non-declarative clauses such as optatives\slash exclamatives (16.b) and conditionals (16.c). Thus, superficial V1 orders are analysed as having an underlying V2 structure:

\begin{enumerate}[label=\alph*.]
\item[b.]\relax [OP]\textsubscript{XP1} \textit{ti vess=jo} \textbf{\textsubscript{V2}} \textit{dit la veretàt!}
\item[c.]\relax [OP]\textsubscript{XP1} \textit{vinisi=al} \textsubscript{V2} \textit{tjo pari} […]
\end{enumerate}

From this perspective, Friulian can therefore be considered a partial V2 language within \citegen{CruschinaSailor2022} framework.\footnote{Arguably, such requirements are also satisfied by several Venetan dialects, displaying verb raising in progressively smaller subsets of the environments (\citealt[187]{Munaro2011}). This once again suggests that researchers ought to be extraordinarily careful before ascribing syntactic phenomena to the effect of German influence.} 

Despite being based on the same stimuli as \REF{ex:riccardelli:16}, the currently available data for Fassan – retrieved from the AThEME Verona-Trento Corpus \citep{TomaselliEtAl2022} – are unfortunately elusive to a direct comparison. That is because Fassan features an incomplete subject clitic paradigm (\citealt[47]{Hack2012}) and subjects lacking a clitic form can be omitted, while 3\textsuperscript{rd} person clitics (both singular and plural) are only obligatory in the absence of an overt subject (\citealt[164]{Salvi2016}).\footnote{\citet[164]{Salvi2016} actually states that 3\textsuperscript{rd} person clitics are obligatory in the absence of preverbal subjects. However, in main declaratives with a fronted object, subject clitic omission is preferred by 2/3 informants in the corpus, as in \textit{el caffè lounch no lo tol so fra, ma so mare}.} Furthermore, the 1\textsuperscript{st} person plural enclitic subject \textit{{}-e} is phonetically identical to the verb ending in \REF{ex:riccardelli:17c}.

\ea Fassan\label{ex:riccardelli:17}
    \ea\label{ex:riccardelli:17a} 
        \gll Che   che   me     tocia       veder!\\
             What  \textsc{comp}  me{}-obj.cl  touches    see\\
       \glt `What am I forced to see!'

\ex \label{ex:riccardelli:17b}
    \gll Te     aesse     dit     la verità!\\
         you\textsc{{}-}obj.cl.   had\textsc{.1.sing}  told    the truth\\
    \glt `Had I only told you the truth!'

\ex \label{ex:riccardelli:17c}
    \gll Se   ruasse   to   pare   podassane   partir\\
         If   came.\textsc{3.sing}  your  father  could.1.\textsc{pl}.  go\\
    \glt `If your father came, we could go.'
    \z
\z

As anticipated in \sectref{sec:riccardelli:2.1}, the standard literature (\cites[171]{BenincàHaiman1992}[164]{Salvi2016}) does not account for XVS orders in Fassan and Fodom. Compatibly with these descriptions, no such orders are attested for sentences introduced by conditional (map 101) or concessive (map 301 and 511) clauses on the ALD-II (Dolomitic Ladin Atlas, \citealt{GoeblEtAl2012}) either. The wh-\textsc{comp} sequence displayed in \REF{ex:riccardelli:17a} is traditionally described as being limited to the Brach and Cazet varieties of Fassan, where it is also found in partial interrogatives as an alternative to the more conservative wh-V-S.cl. order. This feature, as well as other instances of SV orders in interrogative clauses, has been interpreted by \citet[183]{Hack2012} as evidence of the highly residual V2 status of Brach and Cazet. On the other hand, in the same contexts, Moenat~– the remaining Fassan variety~– displays obligatory Subject-Verb inversion and is thus considered to feature a stricter V2-propriety.\footnote{The fact that it is precisely the variety of Moena, that in closest contact with Italian and non-Ladin dialects, to have retained a more conservative word order in these contexts has been explained by \citet[234]{Hack2012} in terms of  \textit{Überentäußerung} (over-renunciation), i.e. an higher resistance to language interference on the very edges of the bordering areas.}

Despite the limited sample, an informal survey conducted on seven Fassan speakers produced interesting results. The task consisted in the translation of the same stimuli provided in Italian. Firstly, the two Moenat participants (both between 30 and 40 years of age) translated the sentence as “\textit{Che che me tocia veder!}”, thus producing the wh-\textsc{comp} sequence that was unattested in Moenat at the time of \citegen{Hack2012} account. It is thus possible that the phenomenon is currently spreading into Moenat as well, or has done so over the course of the last decade. Secondly – and more relevantly to the purposes of this paper –  two informants (60 and 44 years old, respectively from Vigo di Fassa and Campitello) produced V-S.cl. order, as in “\textit{Che me tochel veder!”}. Further investigations featuring a larger sample and including purposefully devised stimuli (in order to avoid ambiguous responses) will certainly provide valuable insight into the more recent developments of Fassan.

Fodom and Fassan display identical strategies for the formation of the (pseudointerrogative) exclamative \REF{ex:riccardelli:18a} and conditional \REF{ex:riccardelli:18b}. Similarly to Fassan, the inverted order in \REF{ex:riccardelli:18b} constitutes an unexpected result, which is also completely unattested on the ALD-II (informants born between 1919 and 1953). Subject clitic inversion was exclusively produced by the youngest of the two Fodom speakers in the AThEME Verona-Trento Corpus (respectively born in 1989 and 1964). The same speaker also reformulated the conditional in \REF{ex:riccardelli:18c} as a desiderative: “\textit{ruasse-lo pur to pere!”}, once again displaying V-S.cl. inversion.

\ea \label{ex:riccardelli:18} Fodom
\ea \label{ex:riccardelli:18a}
    \gll Ci   che   me     toca     vedei!\\
         What  \textsc{comp}  me-obj.cl  touches  see\\
    \glt `What am I forced to see!'
\ex \label{ex:riccardelli:18b}
    \gll T’    es-io     pur   dit   la   verité!\\
         you\textsc{{}-}obj.cl.   had-\textsc{1.sing}  even  told  the   truth\\
    \glt `Had I only told you the truth!'
\ex \label{ex:riccardelli:18c}
    \gll Se   vegnisse   to   pere   podonsa   partì\\
         If   came.\textsc{3.sing}  your  father  could.1.\textsc{pl}.  go\\
    \glt `If your father came, we could go.'
    \z
\z

Although the unquestionable scarcity of data hinders the possibility of making broader statements, the pattern is undoubtedly striking. The traditional accounts for Fassan and Fodom as residual V2 varieties apply to virtually all the speakers born until the late 50’s, while younger speakers manifest a stronger tendency towards V2. However, without a larger sample of speakers from both areas and without further data concerning non-declarative clauses (or other environments) featuring obligatory subjects in Fassan, it is currently impossible to determine whether Fassan and Fodom could represents a partial V2 variety as it has been proposed for Friulian. If further findings were to confirm these tendencies towards the extension of V2 orders, that would bear significant implications both for the contact hypothesis (since the areas are not characterised by contact with German) and the V2 typology of Rhaeto-Romance as a whole. 

\subsection{V3 in strict-V2 Rhaeto-Romance varieties} 
\label{sec:riccardelli:2.5}%2.5 /
\subsubsection{Romansh} %2.5.1 /
\label{sec:riccardelli:2.5.1}

Having discussed the status of the Rhaeto-Romance varieties on the left-end of the V2 spectrum, let us now consider the instances of V3 and their contexts in the varieties with a higher degree of contact with German. 

The data found in Romansh seems to, at least partially, counter the assumption of a proportional adherence to German syntax. V3 of the kind S-Adv-V appears to be only attested in Sursilvan \REF{ex:riccardelli:19} – the Romansh variety in closest contact with German – but ungrammatical in Surmeiran \REF{ex:riccardelli:20}: 

\ea Sursilvan (\citealt[10]{Samo2019})\label{ex:riccardelli:19}\\
\gll Jeu   bunamein   sedurmentel\\
     I   almost   fell.asleep\\
\glt `I almost fell asleep'

\ex Surmeiran (\citealt[14]{Anderson2016})\label{ex:riccardelli:20}\\
\gll el saleida   curtaschevlamaintg igl plevant\\
     he greets   courteously     the minister\\
\glt `He greets the minister courteously.'
\z

On the other hand, Surmeiran differs from the other varieties by lacking inversion of the element \textit{ins} (< lat. UNUS) used for impersonal constructions, even in interrogatives. Notice the absence of inversion with \textit{ins}, as opposed with the second \textit{igl} (“they”) in the following sentence (\citealt[23]{Anderson2016}):

\ea Surmeiran\\\label{ex:riccardelli:21}
\gll Ainten chell’ustareia   ins na magl igl betg schi bagn,   on   igl detg.\\
     Inside that=inn     one not eat it   not so   well,   have   they said\\
\glt `They said one does not eat too well in that inn.'
\z

As opposed to Sursilvan and Putèr (\textit{ün} is the impersonal pronoun in the latter):

\ea Sursilvan\\\label{ex:riccardelli:22}
\gll Nua   en ins cun   la lavur? Alla   fin en ins mai.\\
     Where is one with the work? At.the   end is one never\\
\glt `At what point are we with the work? We’re never done.'

\ex Putèr\\\label{ex:riccardelli:23}
    \gll Passand tres     il desert chatta ün qualchevoutas skelets.\\
         Passing across   the desert finds one sometimes   skeletons\\
     \glt `Walking across the desert, you may sometimes find skeletons.'
\z

\citet[19--20]{Anderson2006} thus argues that Surmeiran \textit{ins} does not occupy subject position, rather it behaves like a preverbal clitic, similarly to the reflexive impersonal constructions found in Italian and Spanish.\footnote{A peculiar aspect of the divergent behaviour of Surmeiran is that, while some speakers do accept inversion with \textit{ins}, the resulting sentence is perceived as “sounding like German" (\citealt[24]{Anderson2016}). This is further proof that in multilingual environments, even if a structure is featured (and potentially acceptable) in the speaker’s competence, the supposed presence of “competing” grammars does not necessarily have to result in the adherence with the structures of the \textit{Dachsprache}. On the contrary, the speakers may – more or less consciously – display forms of hyperdialectalism, favouring features that emphasise the differences between the two languages. The phenomenon is well attested and analysed in the field of Scandinavian dialectology, where it is known as \textit{naboopposisjon} (neighbour opposition). For an overview see \citet{Jahr2017}.} This statements bears considerable implications about the V2 status of Surmeiran as a whole, since “non-inverted” impersonal declaratives as in \REF{ex:riccardelli:24} would then license a V1 structure, which ultimately leads Anderson to  deny a V2 grammar for Surmeiran. 

\ea%24
    Surmeiran \label{ex:riccardelli:24}\\
    \gll Ins na     pò   betg eir quant spert tg’ins   vot sen las autostradas svizras\\
         \textit{one} \textsc{neg}   can.\textsc{3sg} not go   as  fast that=one wants  on the freeways   Swiss\\
    \glt `You can’t go however fast as you want on the Swiss freeways'
\z


To this respect, it is legitimate to object that verb-initial orders are well attested both in Old Romance \citep{Wolfe2015} and Early Germanic (\citealt{HinterhölzlPetrova2010}). Furthermore, V1 is attested in impersonal clauses even in  rather strict V2 languages such as Icelandic.

While not directly involving verb-subject orders, the next few examples allow for some relevant observations concerning the diachronic development of Romansh in relation to its contact with German.

Sursilvan differs from the rest of Rhaeto\hyp Romance by lacking clitic pronouns (\citealt[125]{BenincàHaiman1992}), that were otherwise present in Old Sursilvan at least until 1718: 

\ea%25
    \label{ex:riccardelli:25}
    Vallader (\textit{La Soncha Scrittüra}, 1953)\\
    \gll Tü   m’hast   clamà\\
         you   me=have   called\\
    \glt ‘You have called me.’
\ex Sursilvan (\textit{Vegl Testament}, 1967)\\\label{ex:riccardelli:26}
    \gll Ti   has     clamau   mei\\
         You   have     called     me\\
    \glt `You have called me.'
\ex Sursilvan (\textit{Bibla da Cuera}, 1718)\\\label{ex:riccardelli:27}
    \gll Ti   mi=has   clumau\\
         you   me=have   called\\
    \glt `You have called me.'
\z


The loss of clitics in Sursilvan has been mentioned as a suitable example of German influence (\citealt{Ascoli1883, Stimm1973}). However, cross-linguistic observation suggests that such development can also be triggered by language-internal factors: Brazilian Portuguese, in contrast to its European counterpart, is indeed undergoing loss of object clitics.\footnote{Second person objects are systematically expressed with non-clitic object pronouns. The shift also applies to third person objects in colloquial Brazilian Portuguese \citep{Duarte2000}.} The diverging developments of European and Brazilian Portuguese, which by themselves do not appear to be motivated by language contact in general, are clearly not motivated by contact with a Germanic language (\citealt[92--93]{KaiserHack2013}). 

The syntax of Tuatschin (a Sursilvan variety spoken by around 900 people) is extensively described in \citegen{Mauer-Cecchini2021} grammar. According to his account, the V2 system of Tuatschin appears to be rather strict and quite comparable to German. However, when presenting one single instance of V>2, the author states that if  “a subject which would normally be inverted is focalised, it may occur preverbally” (\citealt[225]{Mauer-Cecchini2021}):

\ea Tuatschin\\\label{ex:riccardelli:28}
\gll [A lur   scha’l   vès ussa pagljau in grép tga   vès pudju bétar  èl] [scha]quèls fussan grat schulaj gjù   ajl’aua\\
     [and then if=it   had now hit   a rock that   had could   throw it]  [then] these would immediately fall down into=water\\
\glt `And then if it [the load] hit a rock that could have thrown it down, these [the mules] would have immediately fallen into the water.'
\z

According to the available literature, Tuatschin quite uncontroversially qualifies as a strict V2 variety. \citet[328]{Kaiser2002} argues that such structures (also found in different diachronic stages of Sursilvan and Engadinese) are analogous to forms of \textit{wenn…dann} dislocation found in German, where a whole subordinate clause is displaced and the adverbial coreferent “\textit{scha}” appears as the first constituent in the main clause. V3 orders introduced by a resumptive element are well attested in contemporary colloquial German (\citealt{WieseMüller2018}).

Finally, instances of a Frame Setter being followed by the subject and the verb are found in both the main Engadinese varieties (\citealt[152]{Samo2019}). With temporal Frame Setters (29b, 30b), Vallader differs from Putèr by only allowing V3 via subject focalisation:

\ea Vallader (\citealt[157--159]{Oetzel1994})\label{ex:riccardelli:29}
\ea \label{ex:riccardelli:29a}
    \gll Prö   üna   chasa da vaschins   üna   merla     ha   fat seis   gnieu\\
         On   a   house of neighbours   a   blackbird   has made her   nest\\
     \glt `On one of the houses of the neighbours, a black bird made its nest.'

\ex \label{ex:riccardelli:29b}
    \gll Fin   a quel   di   ingün  nu'm                 vaiva mai  dit il cuntrari\\
         Until to that  day  nobody \textsc{neg.me=obj.cl} had   never said the opposite\\
   \glt ‘Until that day, nobody told me the opposite.’
   \z

\ex  Putèr (\citealt[161]{Oetzel1994})\label{ex:riccardelli:30}
\ea \label{ex:riccardelli:30a}
    \gll Nodvart   l’alp   Giuvannes   evra   {a la fin} la buocha\\
         This.side.of   {the=alps} Giuvannes   opened   finally the mouth\\
     \glt `Once on this side of Alps, Giuvannes finally started talking.'
\ex \label{ex:riccardelli:30b}
    \gll Uossa Ciglia {la clama}\\
         Now   Ciglia her=calls\\
     \glt `Now Ciglia calls her.'
     \z
\z

Furthermore, Putèr distinguishes itself from the other Romansh varieties by allowing a topicalised PP to precede  the subject (focused or otherwise) and the inflected verb:

\ea Putèr\label{ex:riccardelli:31}
\ea \label{ex:riccardelli:31a} 
    \gll a Ludwig ils ögls   haun cumanzo   a glüschir\\
         to Ludwig the eyes   have started     to glow\\
    \glt `Ludwig’s eyes started glowing.'\footnote{At first glance, one might interpret the initial PP as expressing a possessive relation within a complex DP ([\textit{a Ludwig} [\textit{ils ögls}]]). However, the PP denotes a contrastive topic, a more fitting translation might be “as for Ludwig, his eyes started to glow”.}
\ex \label{ex:riccardelli:31b}
    \gll ma   da   que üngün   nu s'ho   inachüert   ünguotta\\
         but   of   that nobody   \textsc{neg} \textsc{refl}=have noticed   nothing\\
    \glt `But no one noticed anything about that.'
    \z
\z

Through this overview of Romansh varieties, some preliminary conclusions can be drawn. While, on a macroscopic level, the different degrees of adherence to XVS point towards a correlation with the neighbouring languages, deeper investigations and the observation of microvariation prove that further factors that can disrupt such a linear analysis are at play. \tabref{tab:riccardelli:1} provides an overview of the contexts that deviate from V2 in the different Romash varieties. While it is still possible to place the different varieties on a spectrum according to their contact with German or Italian, some phenomena elude this interpretation.

\begin{table}[t]
\caption{Deviations from V2 in Romansh varieties.}
\label{tab:riccardelli:1}
\begin{tabular}{lcccc}
\lsptoprule
                  & {S-Adv-V} & {Impersonal V1} & {Frame-S-V}  & {PP-S-V}\\
\midrule
{Sursilvan} & + & -- & -- (\footnote{Except for constructions with a resumptive element and a focused subject in Tuatschin \REF{ex:riccardelli:27}.}) & --\\
{Sutsilvan} & -- & -- & -- & --\\
{Surmeiran} & -- & + & -- & --\\
{Vallader} & -- & -- & +/-\footnote{Only via subject topicalisation with a temporal Frame Setter.} & --\\
{Putèr} & -- & -- & + & +\\
\lspbottomrule
\end{tabular}
\end{table}

Firstly, independent developments within the singular varieties can bring to deviations even in those that appear to have a stricter V2 systems, as in the case of  S-Adv-V orders in Sursilvan \REF{ex:riccardelli:19} and verb-initial impersonal declaratives in Surmeiran \REF{ex:riccardelli:24}. Secondly, although V3 in Engadinese varieties seems to be proportional with their distance to Italian, its licensing is also strongly affected by factors as informational structure and the type of the fronted constituent. Thus the observed data hinders the complete acceptance of \posscitet[330]{Kaiser2002} claim that:

\begin{quote}
\sloppy
Die regelmäßigen Abweichungen von der Verb-Zweit-Stellung beschränken sich im Bündnerromanischen weitgehend auf Dislokationsstrukturen und auf Kontexte, in denen durch die Versform die syntaktischen Stel\-lungs\-rexg\-eln gelockert sind. Im Alt- und Mit\-tel\-fran\-zö\-sisch\-en hingegen sind Abweichungen von der Verb-Zweit-Stellung nicht auf solche Kontexte beschränkt.\medskip\\
`The regular deviations from the verb-second position in Romansh are largely limited to dislocation structures and to contexts in which the syntactic position rules are relaxed by the verse form. In Old and Middle French, on the other hand, deviations from the verb-second position are not limited to such contexts.'
\end{quote}

It should be mentioned that Kaiser’s study is solely based on biblical material, thus no reliable comparative analysis between verse and prose is provided. Indeed, many of the V3 structures analysed by Kaiser are found in the spoken language as well, such as examples \xxref{ex:riccardelli:29}{ex:riccardelli:31}. Some further critique of Kaiser’s method and claims will be presented in the conclusions.

Finally, the fact that loss of object clitics in Sursilvan can be determined by language-internal factors (as suggested by the analogous phenomenon in Portuguese), contrasts \citegen[169]{Oetzel1994} statement that without the assumption of contact phenomena “there is no other way to explain why in Sutsilvan and Sursilvan – two idioms that have less contact with Italian – the VS serialisation is observed more consistently than in Engadinese”.\footnote{“Anders ist nicht zu erklären, weshalb im S und Ss –zwei Idiomen, die weniger Kontakt mit dem Italienischen haben – die VS-Serialisierung konsequenter beachtet wird als im Engadinischen” Notice, however, that Oetzel is assuming a strict V2 as the starting point for Romansh in general and attributing the deviations from V2 found in Engadinese to Italian influence.}

\subsubsection{Ladin} %2.5.2 /
\label{sec:riccardelli:2.5.2}

After providing such an overview of the Romansh varieties, it is now possible to move forward with the analysis of variation within (northern) Dolomitic Ladin. The following examples are primarily excerpted from \citegen{CasalicchioCognola2018} survey on micro\hyp variation within Ladin varieties.

The first example of V3 is represented by a Frame setter followed by a focalised subject, as produced by a Gardenese speaker:

\ea Gardenese\\\label{ex:riccardelli:32}
\gll Te   butëiga mami à   for   cumprà la farina\\
     in   shop   mum   has   always bought the flour\\
\glt `It was always mum who bought the flour in the shop.'
\z

Although this construction is commonly accepted, V2 still constitutes the preferred alternative among Badiot speakers and the only acceptable one for the Mareo informant, contrarily to predictions based on the degree of contact with German:

\ea Badiot\\\label{ex:riccardelli:33}
\gll Te botëga   à   tres     la mama cumprè la farina\\
     in shop   has   always   the mum bought the flour\\
\glt `In the shop, mum has always bought the flour.'
\z

As observed for the Engadinese varieties, a significant role is played by the informational structure of the utterance. Indeed, V3 is the preferred option when the subject is not focalised:

\ea Badiot\\\label{ex:riccardelli:34}
\gll Inier     Maria à   cumprè i   soni\\
     yesterday   Maria has   bought the   potatoes\\
\glt `Yesterday Maria bought the potatoes.'
\z

The following construction -- consisting of a fronted object and a focused subject – results in V2 being accepted by the Mareo speaker and one of the Badiot informants interviewed in Casalicchio \& Cognola’s survey. However, although accepted, V2 is not spontaneously produced by any of the speakers.\footnote{Regrettably, the Mareo speaker did not complete this task and incorrectly translated the sentence contained in the following task. While their acceptance of V2 and the absence of a spontaneously produced alternative suggests their preference for V2 in a context featuring a fronted object and a focused subject, no comparable inference can be brought forward for an analogous context without focus on the subject.} Indeed, the two main structures produced either consist of a Left Dislocation (as also found in Italian) \REF{ex:riccardelli:36} or no inversion at all \REF{ex:riccardelli:35}.

\ea Badiot\\\label{ex:riccardelli:35}
\gll La lëtra   al  foliet     la uma     ti   à   scrit\\
     the letter to.the   newspaper   the mum    it.\textsc{obj.cl} has  written\\
\glt `It was mom who wrote the letter to the newspaper.'

\ex Gardenese\\\label{ex:riccardelli:36}
    \gll La lëtra tla   zaita     l'  à  scrita   la l’oma\\
         the letter to.the   newspaper   \textsc{obj.cl} has   written the mum\\
     \glt `It was mom who wrote the letter to the newspaper.'
\z

V2 was produced by one of the Badiot informants when presented a sentence with a fronted object and no focalised subject. Similarly to the previous context, the rest of the Badiot speakers tend to produce a structure without inversion, while the Gardenese speakers produce a LD resembling Italian:

\ea Badiot\\\label{ex:riccardelli:37}
\gll Le liber   la uma   l'à   cumprè inier\\
     the book the mum   it=has bought yesterday\\
\glt `Mom bought the book yesterday.'
\ex  Gardenese\\\label{ex:riccardelli:38}
\gll L liber     l à   cumprà la l'oma inier\\
     the book   it=has bought the mum yesterday\\
\glt `Mom bought the book yesterday.'
\z

The provided results are summarised in \tabref{tab:riccardelli:2}. 

%%please move \begin{table} just above \begin{tabular
\begin{table}
\caption{V3 licensing in the northern Ladin varieties}
\label{tab:riccardelli:2}

\small
\begin{tabularx}{\textwidth}{l@{}Cccc}
\lsptoprule
  & \mbox{{Frame-\textsubscript{Foc}}{Subj-V}} & {Frame-Subj-V} & {\textsubscript{Front}}{Obj-\textsubscript{Foc}}{Subj-V} & {\textsubscript{Front}}{Obj-Subj-V}\\
                \midrule
{Mareo} & -- & -- & ? & ?\\
{Badiotto} & ~~~~~+/--
        \newline  \mbox{(V2 preferred)} & + & + & +\\
{Gardenese} & + & + & + & +\\
\lspbottomrule
\end{tabularx}
\end{table}

\newpage
\section{Final remarks: Rethinking contact} %3 /
\label{sec:riccardelli:3}

As previously mentioned, while – on a macroscopic level – the distribution of V2 and V3 orders within Dolomitic Ladin seems to point to the effect of contact either with German or Italian, the closer observation of instances of variation among speakers of the same variety highlights how the alternation between V2 and V3 is also related to the informational structure of the utterance and to sociolinguistic factors. These observations, along with the description of \REF{ex:riccardelli:1} deviations from V2 in the Romansh varieties and the language-internal factors that drove their development and \REF{ex:riccardelli:2} previously unattested V2 orders in Fassan and Fodom, lead us to reconsider and redefine the role of language contact in the emergence or retention of V2 in Rhaeto-Romance.

As stated in \citet[18]{CruschinaSailor2022}, the scholarship on the diachronic development of V2, especially in the Romance area, has mostly been focused on the \textit{loss} or \textit{preservation} of V2, rather than its \textit{extension}. This perspective justifies \citegen[331]{Kaiser2002} claim that Romansh V2 cannot be interpreted as the preservation of an earlier Old Romance stage (represented by Old French).\footnote{“Dies führt zu der Schlussfolgerung, dass die im Bünderromanischen existierende Verb-Zweit-Stellung nicht als ein “Überbleibsel” einer erb-Zweit-Stellungseigenschaft angesehen werden kann, die ursprünglich in den romanischen Sprachen allgemeine Gültigkeit gehabt haben soll.”} While this point is perfectly solid, there are some methodological aspects that might benefit from reconsideration. 

\citet{Kaiser2002} proposes a comparison between the changes in frequency of XVS orders in the different diachronic stages of French and Romansh (in particular Sursilvan), observing a considerably higher frequency both in Modern and Old Romansh when compared to Old French, thus excluding the hypothesis of V2 preservation. However, it is crucial to highlight that – given that the first written attestations of Romansh are rather recent – the label “Old” Sursilvan is used to describe the language contained in the Bible translation of 1718, while the Old French translation consulted by Kaiser dates back to the 12\textsuperscript{th} century. The frequency of XVS in Old Sursilvan is actually quite comparable to the 1494 Middle French translation (22.8\% vs. 26.8\%). 

Applying a diachronic-typological perspective such as that proposed by \citet{Wolfe2019Redefining}, the assumption that both French and Romansh syntax developed from an Old Romance type, characterised by a relaxed V2 system, appears perfectly plausible. Such a system would then evolve in different directions: one characterised by the progressive loss of V2 (as in most Romance languages), the other by its expansion as in the case of Romansh and other Rhaeto-Romance varieties. Crucially, French had initially undergone a similar expansion of V2 (Later Old French is in fact mentioned as a type1 Force-V2 system in Wolfe’s typology, cfr. fig.1). Unlike the rest of Romance, a more precise chronological subdivision of the process in Rhaeto-Romance is unfortunately hindered by the late date of the first written attestations.\footnote{It is however interesting to notice that, according to \citegen[325]{Kaiser2002} investigation, the strengthening of V2 in Sursilvan appears to be a rather recent development, since the percentage of XVS orders almost doubled between 1718 and 1967. While, by itself, this detail provides no sufficient chronological evidence, it does not clash with other observations concerning the timeline of contact with German and its supposed effect, as later discussed.} 

The contact with German and its regional varieties, which were undergoing (or had already undergone) a similar development of V2 strengthening (as observed in \citet{Speyer2008Doppelte}), therefore contributed to its expansion in Rhaeto-Romance varieties, rather than triggering it. Thus, rather than Germanic influence, it would be appropriate to define the process as \textit{contact-induced grammatical reinforcement}, i.e. the further spread of shared syntactic traits that were already independently developing in both language groups because of their similar starting syntactical proprieties. 

An analogous process is described by \citet[84--85]{Cordin2011} regarding the aforementioned topic of Verb+Adverb constructions. Contact with German did not cause the emergence of the construction in Romance varieties, rather it “attracted the phenomenon to the other end of the [grammaticalisation] spectrum”. Furthermore, contact-induced grammatical reinforcement is not limited to the domain of syntax but has been observed in other levels of analysis, such as morphology. \citet[322--323]{Trudgill2016} shows that the Old English two-copula system (\textit{wesan}{}-\textit{beon}) reflected a semantic nuance – distinguishing  habitual vs. non-habitual “to be” – identical to that found in Celtic. The contact between West Germanic and Celtic languages, coupled with the presence of this shared lexical and semantic trait, contributed to the preservation of the two copula stems, ultimately resulting in the suppletive inflection of the verb "to be" in present-day West Germanic. 

In conclusion, V2 in Rhaeto-Romance varieties is far from a cohesive phenomenon, and not one that can be swiftly accounted for as a form of syntactic Germanism. While the presented data is compatible with an hypothesis of contact-induced grammatical reinforcement, resulting in the development of a strict V2 system in areas characterised by a high degree of Romance-German bilingualism (especially in the last three centuries), a more insightful analysis of the diatopic, sociolinguistic, discourse-related and speaker-internal variation that results in deviations from V2 confirms that the phenomenon arose independently in the different languages (possibly as a common development from a SOV order attested both in Latin and Early Germanic). Additionally, the new attestations of seemingly V2 orders in Rhaeto-Romance varieties  that are not characterised by an equally strong and lasting contact with German varieties (such as Fassan and Fodom) could serve as a further proof of a general tendency of Rhaeto-Romance to develop a V2 system, as assumed in \citet{KaiserHack2013}:

\begin{quote}
strong contact with German has caused the emergence of a strong V2 order in Swiss Romansh in a situation where a tendency to develop such a word order already existed before.
\end{quote}

Thus, further research on micro-variation within Rhaeto-Romance will undoubtedly provide valuable insight on the validity of contact-related hypotheses and the syntactic environments in which they can operate.

\printbibliography[heading=subbibliography,notkeyword=this]
\end{document}
