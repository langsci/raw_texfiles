\title{Verb-third phenomena in Germanic verb-second languages}
\subtitle{Historical and variational perspectives}
\BackBody{With the exception of English and its varieties, all Present-Day Germanic languages display some kind of verb-second (V2) rule, according to which the finite verbal form has to be put in the second position of the clause in declarative utterances. But even within the Germanic domain, the exact contours of the V2 rule vary strongly in time and space. Above all, the so-called bottleneck demanding that one and only one constituent be placed before the finite verb is not equally respected in all Germanic varieties. The typology of V2 violations, apparent or real, is now regarded as a core question for the typology of V2 itself. The present volume is concerned with all kinds of alleged “cracks in the bottleneck”, involving argument stacking, remnant movement, or adverbial resumption. A general introduction by Modicom and Harchaoui discusses the current state of linguistic research on verb-third phenomena in Germanic languages, both in synchrony and diachrony. The introduction is followed by a diachronic panorama of V3 phenomena in the history of High German, by A. Speyer, who shows that behind the apparent stability of V2, the syntactic typology of apparent V3 in German has undergone significant changes over the last centuries. The other contributions to the volume follow this variational and historical thread: E. Klaevik-Pettersen and N. Catasso discuss the validity of the bottleneck hypothesis in present and ancient V2 varieties. E. Louviot, Th. Robin, Chr. Nilsen and B. Bloom focus on verb-third phenomena involving resumptive items in the history of English, High German, Low German and Swedish. In their paper on Old West Germanic verse corpora, Louviot and Robin concentrate on clause-initial \textit{tha}/\textit{tho}, investigating which factors determine its capacity to either be followed by the finite verb (V2) or by another constituent before the finite verb (V3). Nilsen is concerned with the semantic evolution of verb-third adverbial resumption involving \textit{da} and \textit{så} in Swedish. Bloom focuses on the V3 use of one resumptive, so, in Early New High German during the 16th century, tackling the discourse\hyp organizational factors behind adverbial resumption. Finally, the chapters by L. Riccardelli, R. Madaro, A. Tomaselli and E. Bidese investigate how contact between Germanic and Romance may have interacted with language-internal dynamics in the history of several varieties of Rhaeto-Romance (Riccardelli) and Upper German (Madaro, Tomaselli and Bidese).}
\author{Sarah Harchaoui and Pierre-Yves Modicom} 
 

\renewcommand{\lsISBNdigital}{978-3-96110-535-9}
\renewcommand{\lsISBNhardcover}{978-3-98554-155-3}
\BookDOI{10.5281/zenodo.16309627}
\typesetter{Sebastian Nordhoff}
\proofreader{Alexandra Fosså,
Amir Ghorbanpour,
Elliott Pearl,
Hella Olbertz,
Jean Nitzke,
Katja Politt,
Ludger Paschen,
Mary Ann Walter,
Mengyi Zhu,
Nicoletta Romeo,
Sebastian Nordhoff
}
\lsCoverTitleSizes{37pt}{13.5mm}% Font setting for the title page


\renewcommand{\lsSeries}{ogl}
\renewcommand{\lsSeriesNumber}{15}
\renewcommand{\lsID}{486}
