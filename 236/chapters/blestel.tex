\documentclass[output=paper]{langscibook}
\ChapterDOI{10.5281/zenodo.5643283}
\author{Élodie Blestel\orcid{0000-0001-7257-6859}\affiliation{Université Sorbonne Nouvelle}}
\title{Entramados lingüísticos e ideológicos a prueba de las prácticas: Español y guaraní en Paraguay}
\abstract{El objeto de este estudio es mostrar que la conceptualización de dos lenguas históricas y discretas en Paraguay -- español y guaraní -- hace que las gramáticas -- tan diversas como heterogéneas -- de las prácticas individuales de los hablantes se vuelvan opacas para los estudiosos que abordan esta área de contacto lingüístico. Comenzamos mostrando cómo la cuestión de lo que se entiende por “guaraní” debe verse como una de las manifestaciones del nacionalismo paraguayo que echa sus raíces en el período colonial. Luego enfrentamos estas reflexiones con los resultados de varios estudios que dan cuenta del hecho de que esta concepción diádica también se refleja en la esfera popular. Finalmente, mostramos que las herramientas de los lingüistas, que muchas veces también reflejan (y alimentan) esta concepción, introducen tres sesgos metodológicos (ideológico, diacrónico y perceptual) en los estudios del contacto español-guaraní. Para tratar de sortear estos sesgos, proponemos adoptar una perspectiva que sitúe al hablante en el centro de la investigación, al concebir sus prácticas lingüísticas como una serie de acciones, comportamientos y coordinaciones intersubjetivamente compartidos que implican un re-análisis continuo de las unidades que conforman sus repertorios individuales (\citealt{BlommaertBackus2011}). }
\IfFileExists{../localcommands.tex}{
  \addbibresource{localbibliography.bib}
  \usepackage{langsci-optional}
\usepackage{langsci-gb4e}
\usepackage{langsci-lgr}

\usepackage{listings}
\lstset{basicstyle=\ttfamily,tabsize=2,breaklines=true}

%added by author
% \usepackage{tipa}
\usepackage{multirow}
\graphicspath{{figures/}}
\usepackage{langsci-branding}

  
\newcommand{\sent}{\enumsentence}
\newcommand{\sents}{\eenumsentence}
\let\citeasnoun\citet

\renewcommand{\lsCoverTitleFont}[1]{\sffamily\addfontfeatures{Scale=MatchUppercase}\fontsize{44pt}{16mm}\selectfont #1}
   
  %% hyphenation points for line breaks
%% Normally, automatic hyphenation in LaTeX is very good
%% If a word is mis-hyphenated, add it to this file
%%
%% add information to TeX file before \begin{document} with:
%% %% hyphenation points for line breaks
%% Normally, automatic hyphenation in LaTeX is very good
%% If a word is mis-hyphenated, add it to this file
%%
%% add information to TeX file before \begin{document} with:
%% %% hyphenation points for line breaks
%% Normally, automatic hyphenation in LaTeX is very good
%% If a word is mis-hyphenated, add it to this file
%%
%% add information to TeX file before \begin{document} with:
%% \include{localhyphenation}
\hyphenation{
affri-ca-te
affri-ca-tes
an-no-tated
com-ple-ments
com-po-si-tio-na-li-ty
non-com-po-si-tio-na-li-ty
Gon-zá-lez
out-side
Ri-chárd
se-man-tics
STREU-SLE
Tie-de-mann
}
\hyphenation{
affri-ca-te
affri-ca-tes
an-no-tated
com-ple-ments
com-po-si-tio-na-li-ty
non-com-po-si-tio-na-li-ty
Gon-zá-lez
out-side
Ri-chárd
se-man-tics
STREU-SLE
Tie-de-mann
}
\hyphenation{
affri-ca-te
affri-ca-tes
an-no-tated
com-ple-ments
com-po-si-tio-na-li-ty
non-com-po-si-tio-na-li-ty
Gon-zá-lez
out-side
Ri-chárd
se-man-tics
STREU-SLE
Tie-de-mann
} 
  \togglepaper[1]%%chapternumber
}{}

\shorttitlerunninghead{Entramados lingüísticos e ideológicos a prueba de las prácticas}%%use this for an abridged title in the page headers
\begin{document}
\maketitle 
\shorttitlerunninghead{Entramados lingüísticos e ideológicos a prueba de las prácticas}%%use this for an abridged title in the page headers



\section{Introducción}


Desde la década de los sesenta, Paraguay ha recibido muchos calificativos que resaltan su diversidad lingüística: “el país más bilingüe del mundo”, “nación mestiza y bilingüe”, “país bilingüe y bicultural”, etc. \citep{Penner2014}. Sin embargo, este reconocimiento, sea institucional o académico, ha tenido como efecto colateral el surgimiento del concepto \emph {jopara}. \textit{Jopara} es un metatérmino derivado del idioma guaraní del morfema guaraní \textit{jo}{}- ‘reciprocidad’ y -\textit{para} ‘mezcla’, que abarca diversas formas de mezclar estos dos idiomas principales del país, español y guaraní. Asimismo, aparece este metatérmino como un correlato inseparable de los discursos ideológicos asociados con el idioma guaraní, pero también como una metáfora de la forma en que los hablantes perciben su habla.

En este artículo, queremos mostrar que la conceptualización de dos lenguas históricas y discretas en Paraguay -- español y guaraní -- hace que las gramáticas -- tan diversas como heterogéneas -- de las prácticas individuales de los hablantes se vuelvan opacas para los estudiosos que abordan esta área lingüística de contacto lingüístico.

Comenzaremos mostrando cómo la cuestión de lo que se entiende por “guaraní” o “jopara” debe verse como una de las manifestaciones del nacionalismo paraguayo: la comunidad nacional es imaginada como mestiza y bilingüe guaraní-español, excluyendo cualquier otra identidad étnica. Para comprender la especificidad de estas políticas, hace falta ubicarlas en continuidad con una gestión lingüística en Paraguay que echa sus raíces ya en el período colonial.

Como segundo paso, enfrentaremos estas reflexiones con los resultados de varios estudios que dan cuenta del hecho de que esta concepción diádica se refleja, como es de esperar, en la esfera popular.

Finalmente, mostraremos que las herramientas de los lingüistas, que muchas veces también reflejan (y alimentan) esta concepción, introducen tres sesgos metodológicos (ideológico, diacrónico y perceptual) en los estudios del contacto español-guaraní. Para tratar de sortear estos sesgos, proponemos adoptar una perspectiva que sitúe al hablante en el centro de la investigación, al concebir sus prácticas lingüísticas como una serie de acciones, comportamientos y coordinaciones intersubjetivamente compartidos que implican  un re-análisis continuo de las unidades que conforman sus repertorios individuales (\citealt{BlommaertBackus2011}).

\section{¿Una nación mestiza y bilingüe?}


Los signos de interrogación podrían sorprender en vista de lo que afirmamos en la introducción: ¿cómo, a este país -- el único en América que haya reconocido una lengua amerindia como cooficial de la nación --, negarle la singularidad lingüística y cultural que consiste en representárselo como mestizo y bilingüe? ¿No es esta una posición algo provocativa? La Constitución Nacional, sancionada el 20 de junio 1992, lo declara en su artículo 140: “El Paraguay es un país pluricultural y bilingüe. Son idiomas oficiales el castellano y el guaraní” (\citealt{RepúblicadelParaguay1992}). Y las recientes cifras proporcionadas por la Dirección General de Estadística, Encuestas y Censos (DGEEC) apuntan hacia una representación similar al registrar que: 

\begin{quote}
los idiomas hablados en el hogar la mayor parte del tiempo por la población de 5 años y más en Paraguay son: el guaraní (40\%), el castellano (26,5\%), guaraní y castellano (30\%), y otros idiomas entre los que se incluyen al alemán, árabe, coreano, francés, inglés, portugués, italiano, japonés y ucraniano (3\%) (\citealt{RepúblicadelParaguay2018a}). 
\end{quote}


Dicho de otra forma, la ley reconoce como cooficiales las dos lenguas cuyo uso es claramente mayoritario, pues, según estas cifras, un 70\% de los hogares utiliza el guaraní, y el 56,5\% el español, sea de manera exclusiva o no. 

Ahora bien, al ser estos cuestionarios declarativos -- los entrevistados son los que testifican que hablan una u otra lengua --, podríamos matizar estos resultados introduciendo desde ya dos comentarios. El primero tiene que ver con la formulación de la encuesta: preguntar a los hablantes qué idioma(s) hablan en sus hogares es tomar por sentada la existencia de códigos lingüísticos discretos que eventualmente se excluyen mutuamente en la práctica individual y sincrónica. Es decir, podríamos preguntarnos a qué clase de “español” (o castellano) y “guaraní” se refieren los entrevistados: es un verdadero problema que difícilmente se resuelve con una afirmación tajante, sobre todo en una zona donde el plurilingüismo remonta a varios siglos.\footnote{Sobre la complejidad del ejercicio de documentación lingüística que debe realizar cada censo, en particular en zonas de plurilingüismo, ver \citet{HumbertEtAl2018}.} El segundo comentario es de índole socio-política: la visión del país como bilingüe va de la mano con la construcción histórica de la identidad nacional; esta visión fluctuó en la historia del país y sigue evolucionando hoy en día hacia un mayor reconocimiento de la lengua guaraní, pero también hacia una creciente “des-indianización” \citep{Boidin2014a} de la misma.

\subsection{El reto del bilingüismo coordinado}


Ya se ha mostrado\footnote{Ver en particular \citet{Penner2010} y  \citet{ZuccolilloFrench2002}.}  cuán paradójica resulta la formulación constitucional del artículo 140 (\citealt{RepúblicadelParaguay1992}): “El Paraguay es un país \textit{pluri}cultural y \textit{bi}lingüe”\footnote{Las cursivas de los prefijos son nuestras.} ya que, no solamente las lenguas indígenas\footnote{Según los datos del  Censo Nacional de Pueblos Indígenas (2012), la población indígena en el Paraguay representa aproximadamente el 2\% de la población total del país y pertenece a cinco familias lingüísticas: guaraní (aché, avá guaraní, mbya guaraní, pa\~\i tavyterã, guaraní ñandeva, guaraní occidental), maskoy (toba maskoy, enlhet norte, enxet sur, sanapaná, angaité, guaná), mataco mataguayo (nivaclé, maká, manjui), zambuco (ayoreo, yvytoso, tomáraho) y guaicurú (qom) (\citealt{RepúblicadelParaguay2016}).}, así como las de las otras minorías “solo” -- por decirlo así --, “forman parte del patrimonio cultural de la Nación” (\textit{ibid.}), sino que además las disposiciones legales con relación a la obligatoriedad de la enseñanza en lengua materna sugieren que esta última solo pueda darse en una de las dos lenguas oficiales. Así, la ley dispone que:

\begin{quote}
La enseñanza en los comienzos del proceso escolar se realizará en la \textit{lengua oficial materna}\footnote{Las cursivas de los prefijos son nuestras.} del educando. Se instruirá, asimismo, en el conocimiento y en el empleo de ambos idiomas oficiales de la República. En el caso de las minorías étnicas cuya lengua materna no sea el guaraní, se podrá elegir uno de los dos idiomas oficiales (\citealt{RepúblicadelParaguay1992}: art.~77).
\end{quote}



Incluso si recientes disposiciones legales tienden a fortalecer la educación indígena y la promoción de la interculturalidad (\citealt{GómezBote2019}; \citealt{Tauli-Corpuz2015}), en particular mediante la Ley 3231/07, que crea la Dirección General de Educación Escolar Indígena (DGEEI),\footnote{Ver \citet{RepúblicadelParaguay2007}.} podemos constatar que lo que predomina en la Constitución es la visión de una nación bilingüe (español/guaraní) a la que los educandos tienen que conformarse, sea cual sea su bagaje lingüístico inicial.\footnote{La ley Nº 4251/2010 “Ley de Lenguas” (\citealt{RepúblicadelParaguay2010}) establece las modalidades que permiten salvaguardar el carácter “pluricultural y bilingüe” del Estado Paraguayo, “velando por la promoción y el desarrollo de las dos lenguas oficiales y la preservación y promoción de las lenguas y culturas indígenas” (art. 2), lo cual significa que solo “[l]as lenguas oficiales de la República tendrán vigencia y uso en los tres Poderes del Estado y en todas las instituciones públicas” (art. 3). En el ámbito educativo esto significa que “[e]l niño y la niña que habiten el territorio nacional tienen derecho a recibir educación inicial en su lengua materna, siempre que la misma sea una de las lenguas oficiales del Estado.~ Los pueblos indígenas utilizarán en la etapa inicial de la educación escolarizada sus respectivas lenguas. Las demás comunidades culturales optarán por una de las lenguas oficiales” (art. 26).} Y es ahí donde radica el problema: proclamar constitucionalmente que un pueblo o una nación es bilingüe no significa que todos y cada uno de sus integrantes lo sean.\footnote{Este es el motivo por el cual \citet[70]{Fasoli-Wörmann2002} distingue dos tipos de bilingüismo: el bilingüismo social (“gesellschaftliche Zweisprachigkeit”) \textit{vs} el de cada hablante (“Bilinguismus”) y muestra que el segundo se da muy poco en Paraguay.} En 1994, la difícil implementación de la ambiciosa reforma educativa que consistía en hacer del guaraní un idioma de enseñanza al igual que el castellano, lo atestiguó: la falta de docentes en lengua primera guaraní, el anhelo de algunos actores por enseñar y promover un guaraní libre de influencia hispana y, sobre todo, la inexistencia de estudios de campo que describieran lo que es, hoy en día, el habla guaraní, constituyeron serios escollos que todavía no se han superado.\footnote{Sobre el Plan Nacional de Educación Bilingüe implementado a partir de 1994, ver en particular \citealt{Boidin2014a};    \citealt{BoyerNatali2006}; \citealt{Hauck2014}; \citealt{Penner2014}; \citealt{Zimmermann2002}.}

Es interesante notar al respecto que, aunque el Ministerio de Educación y Ciencias tiene que conformarse a este marco legal, el término “bilingüismo” ha dejado paso al de “multilingüismo” en la Agenda educativa 2013 -- 2018:

\begin{quote} 
Multilingüismo e Interculturalidad: La distribución del uso de las lenguas en Paraguay obliga a un abordaje particular de su política educativa, pues la planificación, desarrollo y evaluación de los procesos pedagógicos debe responder a esa realidad. El sistema educativo nacional ha concebido una educación bilingüe castellano–guaraní trabajada desde dos dimensiones: la enseñanza de las dos lenguas y la enseñanza en las dos lenguas en todos los niveles/modalidades (\citealt{RepúblicadelParaguay2013}).\footnote{El Ministerio prosigue así, lo cual podría sorprender si el bilingüismo fuera evidente: “Una vasta justificación pedagógica, psicolingüística, sociolingüística, antropológica y legal, argumenta la pertinencia y relevancia de la educación bilingüe castellano--guaraní en Paraguay, proceso que se realiza de manera concomitante con la educación en lenguas indígenas dirigidas a las y los estudiantes de pueblos indígenas del país, y con la enseñanza de lenguas extranjeras” (\textit{ibid}.).} \footnote{Esta formulación, “multilingüe y pluricultural con dos lenguas oficiales”  se retoma en el Plan de acción educativa 2018--2023 (\citealt{RepúblicadelParaguay2018b}).}
\end{quote}


Sea como fuere, con este aparato legislativo y educativo queda patente la voluntad de elevar la lengua guaraní a un estatus similar al del español. Sea este bilingüismo anhelado fruto de una “ideología” (\citealt{CouchonnalCancio2018}) o de un “mito” (\citealt{Makaran2014}; \citealt{Pic-Gillard2008}), no deja de ser un horizonte político del que no podemos cuestionar que ha tenido algunos avances notables, aunque solo sea desde un punto de vista actitudinal (\citealt{BoyerNatali2006}; \citealt{Pic-Gillard2004}; \citealt{Zajícová2009}).

Para entender por qué el Paraguay se ha orientado hacia estas opciones políticas y educativas, -- y por qué estas últimas plantean tantos problemas de implementación -- hay que ubicarlas en continuidad con una gestión lingüística en Paraguay que echa sus raíces en el período colonial.

\subsection{De la lengua general al “guaraní paraguayo”}\label{sec:blestel:2.2}


Para entender la trayectoria del guaraní en el Paraguay, hay que recordar que en el siglo XVI se trataba del habla de los grupos de mayor expansión demográfica y geográfica en ese territorio, por lo que servía de lengua vehicular para muchos otros grupos de la región. Ahora bien, esto no significa que en esa época el guaraní fuera homogéneo: se trataba más bien de un~\textit{continuum}~dialectal \citep{CandelaMelià2015}, que la administración colonial contribuyó a homogeneizar al convertirlo paulatinamente en “lengua general” (\citealt{Estenssoro2015};    \citealt{EstenssoroItier2015}).\footnote{Estenssoro e Itier muestran que las “lenguas generales” tenían características que las distinguían de otras categorías sociolingüísticas como las lenguas vehiculares, \textit{linguae francae} o \textit{koinés}, pues comprendían varios registros y eran las lenguas de un grupo predominante política o demográficamente. Su difusión durante la era colonial tuvo como objetivo imponer el monolingüismo en espacios cuyo perfil lingüístico, étnico y sociológico era particularmente heterogéneo. Permitieron así el establecimiento de instituciones de gobierno comunes y fueron, de hecho, instrumentos clave de dominación colonial en las Américas.} Es a esta última -- que conocemos hoy como guaraní “jesuítico” o “misionero” -- a la que tenemos acceso en las fuentes documentales que emanaban del clero regular (franciscanos y jesuitas, en particular) pero también de indígenas que pertenecían a cierta élite letrada. Se suele considerar que esta variedad desapareció con la expulsión de los miembros de la Compañía de Jesús en 1767, pero sabemos que el guaraní escrito se siguió usando hasta principios del siglo XIX e incluso durante la dictadura del doctor Francia (1813--1840).\footnote{Nora  Esperanza \citet{Bouvet2009} citada por \citet{Boidin2014b}, encuentra “producciones en guaraní en declaraciones sumarias e informes de los pueblos de indios que no se traducen”  (2009 pp.~99-100).}  Sin embargo, con la Independencia, esta variedad de guaraní -- variedad, repitámoslo, que la administración colonial había contribuido a forjar -- parece haber perdido su estatus de lengua escrita, pues son muy escasos los textos que hallamos en el transcurso del siglo XIX. Es más, cuando el “guaraní” escrito aparece de nuevo en los diarios de guerra de la Triple Alianza (1864--1870), se trata de una variedad rotundamente distinta de la que se practicaba en las Misiones y mucho más cercana a la que se usa hoy en día y se conoce como “guaraní paraguayo” o \textit{jopara} (\citealt{CaballeroCamposFerreiraSegovia2006}). El guaraní de las fuentes de finales del siglo XIX se presenta así como una variedad muy marcada por su contacto con el español, y distinta a la vez de lo que hoy se designa como “guaraní étnico”, a saber, las variedades de los grupos de la familia guaraní que permanecieron relativamente
\largerpage[-1]
aislados hasta el siglo XIX: mbyá, kaiwá (pa\~\i tavyterã), ñandeva (avá-Guarani), chiriguano y tapiete (\citealt{Dietrich2010}; \citealt{Cerno2017}).\footnote{Sin embargo, estos grupos mantienen también muchos contactos con hablantes del “guaraní paraguayo”, por lo que habría que estudiar detenidamente el alcance de esta fragmentación dialectal.} ¿Cómo explicar tantas diferencias entre ese “guaraní paraguayo” de finales del XIX y la lengua general de la que tenemos constancia en las fuentes históricas? Las fuentes que tenemos en este momento no nos permiten evaluar cuál pudo haber sido el impacto lingüístico de la dispersión de los habitantes de las Misiones tras la expulsión de los jesuitas. Sin embargo, sabemos que esa población era mucho más numerosa que la del resto del territorio y es poco probable que su habla no tuviera influencia en la de los habitantes que vivían fuera del ámbito misionero. Ahora, los textos misioneros y de la Independencia tampoco son una fuente de fiar en cuanto a las variedades que se hablaban de manera efectiva: los textos que circulaban eran escritos por el clero, por los indios alfabetizados o por la élite letrada y por ende es muy probable que disten de las producciones orales de la época, que podrían haber sido mucho más impregnadas del español de lo que sugieren los textos oficiales de las Misiones.\footnote{De hecho, sabemos que desde el inicio del periodo misionero los miembros del clero, como el Padre Cardiel, condenaban la “jerigonza” que consistía en mezclar guaraní y español: “El lenguaje o jerigonza que a los principios sabían no es otra cosa que un agregado de solecismos y de barbarismos de la lengua guaran\'{í} y castellano [...]” (\citealt{Cardiel1900}: 392).}

\largerpage[-2]
En cambio, de lo que no tenemos constancia en ninguna fuente es que la población de la época fuera bilingüe: hasta finales del siglo XIX, la gran mayoría de la población hablaba guaraní \citep{Melià1992} y el español no era la lengua sino de un segmento minoritario de la población que gozaba sin embargo de mucho prestigio social, económico y político.\footnote{Ver también \citet[27 \textit{et seq.}]{Zajícová2009}.} La influencia de esta élite tuvo como resultado la expansión del castellano a través de las instituciones políticas educativas y los medios de comunicación, sin que los regímenes autoritarios que se sucedieron mostraran interés en elevar el estatus de la lengua indígena. Fue solamente durante la guerra de la Triple Alianza cuando el guaraní apareció como símbolo de cohesión e integridad nacional.\footnote{Sin embargo, ese estatus simbólico no repercutió en el ámbito educativo de la época.} Este movimiento se agudizó en los años 1920 -- cuando la nación paraguaya empezó a definirse como una raza mestiza, unificada por “la lengua guaraní” -- , hasta consolidarse, durante la dictadura del general Alfredo Stroessner (1954--1989), con un reconocimiento explícito dos idiomas nacionales, guaraní y español, en la Constitución de 1967 \citep{Boidin2014a}. Mientras tanto, la escolarización más eficiente, los nuevos medios de transporte y el impacto de los medios de comunicación beneficiaron al castellano que siguió expandiéndose durante todo el siglo XX. En 1992, la Constitución dio un paso más con la afirmación de una nación bilingüe: por supuesto, bilingüe español/guaraní paraguayo, excluyendo de paso las lenguas indígenas (pertenecientes a la familia guaraní o no) y obviando también, lo cual fue muy problemático, establecer una variedad estándar que se apoyara en el habla de la mayoría.\footnote{La Academia de la Lengua Guaraní fue creada en 2012 (con la Resolución n°80/2012), veinte años después de la promulgación de la Constitución, a raíz de que la Ley de lenguas (2010) permitiera la creación de la Secretaría Nacional de Políticas Lingüísticas. La primera gramática de la lengua guaraní editada por esta academia salió en 2017. Sin embargo, siguen faltando trabajos de campo que recolecten la eventual variación del guaraní a lo largo del país (ver \textit{infra}).}

Estas opciones políticas del final del siglo XX, hacen eco al giro multicultural que en los años 90 se dio en muchos países de América. No obstante, el guaraní no se percibía en absoluto como lengua indígena que había que defender ya que se veía como segundo pilar, junto con el español, de una nación concebida como mestiza. En este sentido, se \textit{des-indianizó} \citep{Boidin2014a} el guaraní paraguayo \-—cosa que no se dio con los demás dialectos del \textit{continuum} dialectal guaraní— y se impuso, en las escuelas y en las mentes, la idea de un guaraní paraguayo puro\footnote{Con el adjetivo “puro”, aludimos a la idea de un guaraní libre de hispanismos. Se suele referir a ello con la palabra \textit{guaraniete} (‘guaraní verdadero’). Sabemos que ninguna lengua es pura, pero la historia del guaraní paraguayo muestra cómo la variedad que se fue imponiendo emergió de la mano de la administración colonial \- -- dejando de lado otros dialectos de la familia lingüística guaraní y de otras familias indígenas -- lo cual implicó, de por sí y desde el inicio, cierto contacto más o menos intenso con la lengua española. Las circunstancias históricas, entre las cuales figura el aislamiento del país tras una guerra devastadora, permiten entender por qué el Paraguay necesitó del guaraní como base sobre la cual se construyera esta concepción subjetiva de la nación.} que, por las razones que hemos venido exponiendo, nunca existió \textit{de facto}.

\section{El problema “jopara”}


Que la identidad nacional sea un constructo ideológico en el cual la determinación de la unidad lingüística desempeña un papel clave no es extraño, y en este sentido la experiencia paraguaya no es un caso aislado. Lo que sí llama la atención es la manera con la cual los mismos hablantes también oponen dos polos lingüísticos bien diferenciados, vehiculando así los ingredientes del rechazo \- -- o, al menos, de la ambivalencia -- ante lo propio, es decir la variedad “jopara”\footnote{En guaraní, el término \textit{jopara} [ʤopaˈɾa] ‘mezcla’, formado a partir del morfema de reciprocidad \textit{jo-} y la raíz -\textit{para} y que podemos traducir por ‘mezcla heterogénea’, remite a una mezcla de la cual podemos distinguir al menos dos de los constituyentes a simple vista. En cambio, el concepto de ‘mezcla homogénea’ se expresa mediante el término \textit{jeheˈa} [ʤeheˈʔa]. Muchas veces, el “jopara” o “guaraní jopara” se solapa con el concepto de “guaraní coloquial” (\citealt{Zajícová2009}: 93).} que muchas veces lamentan usar (\citealt{Hauck2014}; \citealt{Stewart2017}).

\subsection{\textit{Jopara}: un glotónimo significativo}


Es interesante notar al respecto que este calificativo, que en la época colonial solía aplicarse a toda clase de conceptos vinculados a la idea de mezcla heterogénea (\textit{ao para} ‘vestido de colores’, \textit{ava para} ‘hombre pintado’, \textit{teko jopara} ‘varios modos de vivir’, etc.) no se atribuía a la lengua hasta que, a mediados del siglo XX, el padre Antonio \citet{Guasch1948} empezó a usarlo para remitir a una mezcla de idiomas que cabía evitar. Esta acepción “lingüística” del calificativo \textit{jopara} apareció entonces precisamente cuando la idea de “bilinguismo” se fue imponiendo en el Paraguay \citep{Penner2014}, un bilinguismo que vino acompañado de una concepción purista de cada una de las lenguas que lo componían. Fue así como la necesidad de promover la idea de un habla desprovista de influencia hispana tuvo como efecto desterrar duraderamente el guaraní espontáneo (\textit{jopara}) de las esferas de prestigio\footnote{Según \citet{Stewart2017}, esta polarización del \textit{continuum} guaraní-español, es el fruto de un “esencialismo estratégico” que ha consistido en eclipsar \textit{el jopara} de las variedad promovida por el Ministerio de Educación paraguayo (MEC) pero también por el Ateneo de Lengua y Cultura Guaraní, la Academia e incluso algunos escritores que representan a la élite cultural.}, en pos de un guaraní genuino que quedaba por definir, tanto más cuanto que el guaraní escrito de los registros administrativos, religiosos y jurídicos había desaparecido desde los principios del siglo XIX.\footnote{Ver \textit{supra,} \sectref{sec:blestel:2.2}.}

El éxito del término \textit{jopara} (y no de \textit{jeheˈa} ‘mezcla homogénea’) para calificar las prácticas lingüísticas del pueblo es en sí significativo en la medida en que la palabra guaraní remite a una idea de mezcla de yuxtaposición, en la cual los elementos que se unen conservan su identidad (\citealt{Boidin2005}: 326). Esta denominación arroja luz sobre el hecho de que los propios hablantes cultivan la idea de la existencia dos lenguas discretas e ideales, lenguas que de hecho muy pocos afirman dominar como tales en sus propias prácticas. El estudio de Penner (\citeyear{Penner2003}; \citeyear{Penner2014}, cap. 4: 71--91) es muy esclarecedor al respecto ya que muestra que entre unos treinta adjetivos glotonímicos utilizados por los hablantes para calificar el guaraní, solo seis denominaciones tienen una dimensión endocéntrica, en el sentido de que los usuarios las asocian con sus propias prácticas de lengua, y siempre con una evaluación negativa (guaraní “mezclado”, “jopara”, “incorrecto”, etc.). Los demás glotónimos reciben una definición que la autora califica de “exocéntrica”, como referencia a un guaraní siempre asociado con la alteridad, ya sea porque está vinculado con el pasado (guaraní “de los antepasados” o “de los abuelos”), ya sea porque se relaciona con el ámbito académico y formal (guaraní “de los profesores” o “del Ministerio de Educación”).

Al fin y al cabo, el discurso popular sobre las dos lenguas oficiales -- a saber dos lenguas concebidas como ideales, puras y, sobre todo, discontinuas -- contribuye a alimentar el sentimiento de inseguridad lingüística, sobre todo en quienes hablan principalmente el guaraní (\textit{jopara}). 

 \subsection{Una percepción selectiva del \textit{jopara}}


Otro argumento a favor de la idea de que los propios hablantes vehiculan la representación de dos ideales lingüísticos discontinuos es el que nos proporcionan los escasos estudios que se han dedicado a la percepción del \textit{jopara}. Así, comentando dos estudios que trataron de percibir lo que representaba el \textit{jopara} para sus hablantes\footnote{\citet{RepúblicadelParaguay2001} y \citet{Gynan2003}.}, Zajícová (\citeyear{Zajícová2014}:80) observa que mientras que las interferencias en la estructura no son clasificadas por los hablantes como mezcla \- -- porque, según la autora, tal vez ni siquiera sean percibidas -- \-, palabras funcionales castellanas que expresan relaciones sintácticas son identificadas como ajenas, e incluso más rápidamente que los hispanismos de otras clases, por lo que la autora infiere que algunos de estos últimos ya son admitidos por los hablantes como parte integrante del guaraní.

Lo que se hace patente con estos estudios es que no solamente la palabra \textit{jopara} remite a “resultados lingüísticos” muy variados como bien lo observa Zajícová, sino que, sobre todo -- y es adonde queremos llegar --, no toda “mezcla” se percibe como tal, pues con este tipo de estudios se hace manifiesto que hablantes y lingüistas no posicionan el cursor en el mismo lugar cuando se trata de percibir o conceptualizar el efecto de una lengua sobre otra. 

Hemos mostrado que una ideología común entre los hablantes paraguayos es la de conceptualizar español y guaraní como dos lenguas puras. Este ideal no lo comparten los lingüistas ya que es generalmente admitido que la idea de “lengua pura” es un disparate desde un punto de vista lingüístico. Lo que sí en cambio constituye una de las ideologías más comunes entre nosotros los lingüistas es el presupuesto de que en el Paraguay se da una situación prototípica de “contacto de lenguas”, lo cual implica, a nivel societal, el encuentro de dos comunidades de lengua diferentes (español y guaraní) y, a nivel individual, la existencia de bilingües más o menos coordinados. Sin embargo, hemos visto que, por razones históricas, no se dan (siempre) estas dos condiciones.\footnote{Obviamente, es más fácil imaginar un español libre de guaranismos dado que se habla en países donde no se habla guaraní. Además, el proceso de normalización de la lengua española remonta a varios siglos, lo cual ha contribuido a asentar en las mentes la idea de que existe algo que se llama “español” y que trasciende las fronteras, a pesar de la variación dialectal: obviamente, también se trata de una construcción socio-histórica como es el caso de cualquier “lengua” (supra)nacional.}

 \section{La prueba de las prácticas}


Hasta ahora hemos tratado de mostrar que la concepción de dos lenguas discretas y diferenciadas era más una cuestión de constructo ideológico -- constructo respaldado por las instituciones políticas y el discurso popular --, que una realidad efectiva. Los lingüistas tampoco escapan de esta trampa, pues ciertas herramientas de análisis también suponen la existencia de lenguas concebidas como discretas. Ahora, para que esta trampa no invisibilice las gramáticas individuales y unitarias de los hablantes, haría falta tomar conciencia de ello y tratar de enfocar los estudios en los hablantes y sus prácticas.

 \subsection{De los ideales lingüísticos a las herramientas de estudio}


El presupuesto según el cual el Paraguay es bilingüe -- si no siempre a nivel individual, por lo menos a nivel nacional --, lleva a que hagamos uso de herramientas que contribuyen a alimentar esta ideología, ya que, si nuestro objetivo es estudiar el habla sincrónica de los hablantes paraguayos, tendríamos que prestar especial atención a que no estorben los siguientes sesgos:

\begin{itemize}
\item
Sesgo ideológico: que consiste en clasificar un “resultado lingüístico” sincrónico e individual, o bien como “español”, o bien como “guaraní (paraguayo)” sin cuestionar el alcance del repertorio\footnote{Entendemos “repertorio” en la óptica de  \citet{BlommaertBackus2011} según los cuales los sujetos se integran en una gran variedad de grupos, redes y comunidades, y en consecuencia aprenden sus recursos lingüísticos a través de una amplia variedad de trayectorias, tácticas y tecnologías, que van desde el aprendizaje formal de lenguas a “encuentros” completamente informales con las lenguas. Estos modos de aprendizaje diferentes llevan a distintos grados de conocimiento del lenguaje, desde un conocimiento estructural y pragmático muy elaborado hasta el “reconocimiento” elemental de las lenguas.} del o de la hablante. Como hemos tratado de mostrar \textit{supra}, lo que denominamos con estas etiquetas “ideológicas” es variable y fruto de negociaciones sociales y políticas independientes de la experiencia lingüística individual de cada uno.\footnote{A modo de ejemplo, podríamos tomar el estudio de   \citet{PintaSmith2017}, que ofrecen un análisis fonológico de las diferentes estrategias para adaptar los préstamos del español al idioma guaraní y proponen que el léxico de este último se estratifica sincrónicamente en una estructura de núcleo-periferia según el grado de adaptación de cada uno de los elementos. Sin embargo, el sistema fonológico estándar en el que se basan las comparaciones no tiene en cuenta las realizaciones específicas del castellano paraguayo.}

\item
Sesgo diacrónico: que consiste en analizar un “resultado lingüístico” sincrónico a la luz de conceptos que implican una dimensión diacrónica. Por ejemplo, oponer \textit{code mixing} o préstamos integrados \textit{vs} \textit{code switching} supone, para los primeros, tomar en consideración una historicidad que no supone el segundo concepto. Podemos cuestionar la validez de esta clasificación a la luz de lo experimentado por el hablante.

\item
Sesgo perceptual: es muy frecuente que los fenómenos que llaman la atención de los lingüistas sean justamente aquellos que no corresponden a la idea que tenemos de la “lengua” normativa. Dicho de otra forma, los lingüistas tendemos a seleccionar como objetos de estudio aquellos fenómenos que nos sorprenden porque justamente no corresponden a la idea que tenemos del “español” o del “guaraní” y en general formulamos la hipótesis que lo sorprendente se debe a la influencia de la otra lengua. Tenemos que tener conciencia de que, haciendo esto, dejamos de lado todos aquellos fenómenos, -- que también pueden deberse al manejo de repertorios plurilingües\- -- que no llegan a ser percibidos, por parecer totalmente “normativos”. En este aspecto también hay que tener en mente que lo que sorprende al lingüista \-\-\-se debe a sus propios conocimientos, expectativas e ideologías, pero forma parte de la gramática individual del hablante, de la misma manera que los fenómenos que como lingüistas no percibimos.

\end{itemize}

A ello hay que añadir que, en el caso paraguayo, la ausencia de estudios de campo que den cuenta de la variación en las prácticas (guaraní y/o español) contribuye a alimentar la falacia de que existe un guaraní homogéneo. Como lo muestran \citet{Penner2014} y  \citet{FernándezBarrera2015}, la variedad de los perfiles de bilingüismo, en parte correlacionados con la ubicación geográfica y el estado socioeconómico, deja en claro que una descripción adecuada del guaraní y del español paraguayo depende de la identificación precisa de esos perfiles.

 \subsection{Enfocar al hablante y sus prácticas}


No siempre es posible evitar los sesgos metodológicos que hemos identificado, a lo sumo podemos tomarlos en cuenta, y considerar que lo que se entiende por “contacto de lenguas” supone concebir entidades pre-existentes (las que entran en contacto) que lo actualizan \citep{Nicolaï2017}. Ahora, hemos querido mostrar que estas entidades pre-existentes no son homogéneas y somos nosotros los que participamos en su delimitación. Además, y como lo comentábamos arriba, lingüistas y hablantes no perciben lo mismo a la hora de describir el contacto. Así lo muestran los imprescindibles estudios de percepción que hemos mencionado \textit{supra}: los hablantes reconocen por ejemplo la presencia del artículo de origen castellano \textit{la} en la frase \textit{Upéinte ou} \textbf{\textit{la}} \textit{iména} ‘después vino su marido’ como manifestación de jopara (\citealt{Zajícová2009}: 78) pero el calco de una frase como \textit{Voy a ir a comprar para mi ropa} con el sentido de ‘lo que será mi ropa en el futuro’ es considerado como “castellano”, y no como una manifestación de mezcla. Sin embargo, este último calco será seleccionado como giro de interés por los lingüistas, simplemente porque no corresponde al español normativo. Pero, si observáramos cómo ese mismo hablante usa la preposición \textit{para} en otros contextos, encontraríamos también empleos no salientes, es decir, no sorprendentes desde el punto de vista de la norma, y perfectamente compatibles -- desde un punto de vista cognitivo-estructural -- con la preposición del primer ejemplo, dentro de su repertorio.\footnote{Lo mismo podría decirse del uso de la preposición \textit{por}, como lo muestran   \citet{BlestelFontanier2017}.}



Entonces, podemos formular la hipótesis de que los hablantes recrean gramáticas propias con los elementos a disposición, analizando y actualizando de forma permanente las formas semiológicas para crear y negociar su significación de forma dialógica, como en cualquier práctica de \textit{languaging} \citep{Maturana1970}.



En este sentido, es de gran auxilio la propuesta de \citet{OtheguyEtAl2015,OtheguyEtAl2018}, quienes abordan el contacto desde el enfoque del \textit{translanguaging}, que supone considerar que el hablante \- -- sea monolingüe, bi- o pluri bilingüe, ya no es lo importante\-\- -- posee solamente un único repertorio \textit{indiferenciado} en su sincronía.\footnote{Obviamente, aquellos hablantes que pueden adaptar su habla a un contexto más guaranihablante o más hispanohablante son aquellos que poseen un repertorio suficientemente amplio como para hacerlo. De la misma manera, no todos los hablantes son igualmente competentes para adaptar su habla a todos los registros.} La adopción de tal enfoque implica entonces centrarse ya no en el devenir de los sistemas lingüísticos idealizados, sino en la forma con la que los hablantes crean y re-crean significación con los elementos que tienen a su disposición, sin que importe  -- por lo menos en una primera aproximación --  el origen etimológico o el legado socio-político de dichos elementos.



De forma más general, en nuestro enfoque heurístico de las prácticas lingüísticas en Paraguay, esto supone dejar temporalmente de lado la cuestión de saber si los hablantes son monolingües o bilingües, pero también la cuestión de la asignación de las unidades resultantes de este contacto a uno u otro “código” o lengua institucional. Al contrario, tendríamos que considerar que gran parte de los paraguayos, nacidos en Asunción o en una comunidad rural en el interior del país, experimentan muy temprano comportamientos y acciones asociados con ambos idiomas (o más) desde sus primeras experiencias dialógicas, las cuales acaban conformando un solo repertorio; nuestro objetivo es pues tratar de entender cómo estos comportamientos lingüísticos, sin importar su origen y etiqueta lingüística, tienen sentido en la experiencia dialógica y dan cuenta de la elaboración de actos de conceptualización inéditos en otros repertorios. 


 \section{Conclusión}


En este trabajo, hemos querido mostrar cómo se entretejen las distintas conceptualizaciones de las lenguas nacionales oficiales del Paraguay: español y guaraní. Hemos mostrado primero que el bilingüismo institucional es fruto de una construcción socio-política e incluso ideológica: por un lado, hubo que esperar el siglo XX para que se expandiera el español y por otro lado la variante de guaraní que se ha oficializado todavía está en proceso de descripción y normalización. Se trata pues de una situación en devenir, que responde a lógicas institucionales y políticas muchas veces (si no siempre) independientes de las del habla espontánea de la gente. Mostramos luego que el discurso popular también alimenta la visión de una nación mestiza y bilingüe: vimos que la propia palabra que designa el tipo de mezcla, \textit{jopara}, remite en guaraní a un tipo de mezcla heterogénea, dejando en claro que los hablantes también cultivan un ideal de dos lenguas bien diferenciadas, que no habría que mezclar, lo cual confirma el estudio sobre los glotónimos al que hicimos referencia (\citealt{Penner2003}; \citeyear{Penner2014}, cap. 4: 71--91). Finalmente, mostramos que, si bien los lingüistas consideramos las “mezclas” de forma diferente, algunas de nuestras herramientas también implican y alimentan la concepción de dos lenguas discretas que entran en contacto, lo cual nos condujo a identificar tres sesgos metodológicos en los estudios: ideológico, diacrónico y perceptual. Finalmente, propusimos que una forma de sortear estos sesgos tal vez consista en centrar el estudio en el hablante, y la conformación de repertorios individuales e inéditos, en los que los sujetos van conformando nuevos sistemas gramaticales con los elementos que tienen a su disposición. Echar luz sobre estos mecanismos dinámicos y sus implicancias semióticas y sociales tal vez nos permita escapar de nuestros propios prejuicios.

\sloppy\printbibliography[heading=subbibliography,notkeyword=this]
\end{document}
