\documentclass[output=paper]{../langscibook}
\ChapterDOI{10.5281/zenodo.5643291}
\author{Azucena Palacios Alcaine\orcid{0000-0002-9122-4362}\affiliation{Universidad Autónoma de Madrid}}
\title{Dinamismo y cambio lingüístico: una perspectiva pandialectal de los valores evidenciales de los tiempos de pasado en el área andina}
\abstract{Esta contribución analiza los valores evidenciales y modalizadores de los tiempos de pasado (pretérito simple, compuesto y pluscuamperfecto) en las distintas variedades de español andino (colombiano, ecuatoriano, peruano y boliviano) y el proceso de gramaticalización en el que estos están inmersos. Como es bien sabido, a estos tiempos de pasado se les atribuye valores evidenciales (y también (ad)mirativos), no temporales, producto del contacto intenso con el quechua. Ahora bien, si se trata de áreas de contacto intenso donde conviven las mismas lenguas (español y quechua y/o aimara), cabe preguntarse por qué han emergido entonces soluciones diferentes para los mismos fenómenos en Colombia, Ecuador, Perú y Bolivia; esto es, si los pretéritos han tomado valores evidenciales a causa del contacto histórico con las mismas lenguas ancestrales, ¿por qué no se dan las mismas soluciones en las distintas áreas andinas? \par Más allá de las soluciones adoptadas localmente, lo que sí parece común a todas las áreas andinas es que el hablante hace un uso subjetivo de los tiempos verbales de pasado. El contacto intenso prolongado del español y del quechua y/o el aimara en el área andina posibilita así la emergencia de nuevos valores evidenciales, estrategias comunicativas construidas a partir del uso subjetivo y dinámico de los tiempos verbales que hace el hablante en función de la evaluación de los hechos y el distanciamiento/implicación que desea mantener con respecto a los mismos. 
% \keywords{Evidencialidad, tiempos de pasado, contacto de lenguas, español andino, gramaticalización}
}
\IfFileExists{../localcommands.tex}{
  \addbibresource{localbibliography.bib}
  \usepackage{langsci-optional}
\usepackage{langsci-gb4e}
\usepackage{langsci-lgr}

\usepackage{listings}
\lstset{basicstyle=\ttfamily,tabsize=2,breaklines=true}

%added by author
% \usepackage{tipa}
\usepackage{multirow}
\graphicspath{{figures/}}
\usepackage{langsci-branding}

  
\newcommand{\sent}{\enumsentence}
\newcommand{\sents}{\eenumsentence}
\let\citeasnoun\citet

\renewcommand{\lsCoverTitleFont}[1]{\sffamily\addfontfeatures{Scale=MatchUppercase}\fontsize{44pt}{16mm}\selectfont #1}
  
  %% hyphenation points for line breaks
%% Normally, automatic hyphenation in LaTeX is very good
%% If a word is mis-hyphenated, add it to this file
%%
%% add information to TeX file before \begin{document} with:
%% %% hyphenation points for line breaks
%% Normally, automatic hyphenation in LaTeX is very good
%% If a word is mis-hyphenated, add it to this file
%%
%% add information to TeX file before \begin{document} with:
%% %% hyphenation points for line breaks
%% Normally, automatic hyphenation in LaTeX is very good
%% If a word is mis-hyphenated, add it to this file
%%
%% add information to TeX file before \begin{document} with:
%% \include{localhyphenation}
\hyphenation{
affri-ca-te
affri-ca-tes
an-no-tated
com-ple-ments
com-po-si-tio-na-li-ty
non-com-po-si-tio-na-li-ty
Gon-zá-lez
out-side
Ri-chárd
se-man-tics
STREU-SLE
Tie-de-mann
}
\hyphenation{
affri-ca-te
affri-ca-tes
an-no-tated
com-ple-ments
com-po-si-tio-na-li-ty
non-com-po-si-tio-na-li-ty
Gon-zá-lez
out-side
Ri-chárd
se-man-tics
STREU-SLE
Tie-de-mann
}
\hyphenation{
affri-ca-te
affri-ca-tes
an-no-tated
com-ple-ments
com-po-si-tio-na-li-ty
non-com-po-si-tio-na-li-ty
Gon-zá-lez
out-side
Ri-chárd
se-man-tics
STREU-SLE
Tie-de-mann
}
  \togglepaper[1]%%chapternumber
}{}
\shorttitlerunninghead{Dinamismo y cambio lingüístico}
\begin{document}
\shorttitlerunninghead{Dinamismo y cambio lingüístico}
\maketitle

\section{Introducción}\label{sec:palacios:1}

La complejidad y heterogeneidad de las situaciones de contacto lingüístico exigen que atendamos a distintos aspectos que, en ocasiones, la bibliografía ha transmitido como supuestos más o menos aceptados. Centrándonos en el ámbito hispanoamericano, la carencia de investigaciones sobre amplias zonas de contacto agudiza la sensación de desconocimiento y, por tanto, se asumen más fácilmente los supuestos ya establecidos, a pesar de que, desde hace ya unas décadas, se han publicado numerosos estudios de caso que abordan fenómenos lingüísticos sujetos a variación cuyo análisis había sido parcial e incluso inexistente, contribuyendo a entender mejor la complejidad del contacto y perfilando así poco a poco una teoría más dinámica del contacto lingüístico. 

Una perspectiva de estas situaciones complejas como mero contacto de sistemas lingüísticos, centrada en los efectos del contacto y no en los hablantes que lo hacen posible, no puede explicar algunos de estos supuestos aceptados, como que un cambio inducido por contacto tenga que desarrollar soluciones lingüísticas similares en todas las áreas cuando ocurre entre las dos mismas lenguas. Sin embargo, el caso de los tiempos de pasado del español andino, zona de bilingüismo histórico intenso de español y quechua (y/o aimara), contradice esta afirmación. Este es el supuesto que quiero tratar en estas páginas, si bien en esta investigación solo me ocuparé de los valores evidenciales de los tiempos de pasado.\footnote{Al respecto,   \citet{PalaciosAlcainePfänder2018} realizan un estudio de los valores (ad)mirativos del pretérito pluscuamperfecto en diversas variedades americanas.}

Antes de abordar el citado fenómeno, contextualizo brevemente el español andino (\sectref{sec:palacios:2}). En \sectref{sec:palacios:3} describo los valores pragmáticos y discursivos que se han postulado para los tiempos verbales en estudio en variedades de español sin bilingüismo histórico y en \sectref{sec:palacios:4} describo los valores emergentes de los tiempos de pasado en las distintas variedades de español andino. En \sectref{sec:palacios:5} me ocupo de los procesos de gramaticalización de las formas de pasado en el área andina y dedico \sectref{sec:palacios:6} a explicar cómo surgen estos valores evidenciales en el área andina. A partir de la descripción de la codificación de la evidencialidad en quechua y aimara (\sectref{sec:palacios:6.1}), muestro cómo estas soluciones novedosas se explican como cambios indirectos inducidos por contacto (\sectref{sec:palacios:6.2}). En \sectref{sec:palacios:7} concluyo la investigación con algunas reflexiones finales.

\section{El español andino: un poco de historia}\label{sec:palacios:2}


Una de las áreas de bilingüismo histórico que más atención ha recibido por parte de la crítica es la del español andino, una vasta zona de influencia quechua y/o aimara que comprende el sur de los Andes colombianos, las sierras ecuatorianas y peruanas, gran parte de Bolivia, noroeste de Argentina y norte de Chile, si bien esa atención se ha centrado casi de manera exclusiva en Perú y Ecuador por el momento. Esta área lingüística muestra ciertos rasgos comunes que permiten abogar por considerarla como tal, como la tendencia a la reducción del sistema vocálico, las concordancias alternativas de género y número, la reestructuración del sistema pronominal, las construcciones con gerundio, la adopción de valores evidenciales y (ad)mirativos en los tiempos verbales de pasado, la emergencia de valores modales y evidenciales en ciertos marcadores discursivos, las construcciones causativas, las formas de atenuación, las alteraciones de orden de constituyentes, el cambio del régimen preposicional, entre otros.

Es interesante constatar cómo algunos de estos rasgos aparecían ya en documentos históricos de autores bilingües de los siglos XVI y XVII (\citealt{Cerrón-Palomino1992,Cerrón-Palomino1995,Cerrón-Palomino2004,Cerrón-Palomino2010,NavarroGala2007,NavarroGala2015,PalaciosAlcaine1996,PalaciosAlcaine1998,PalaciosAlcaine2000,PalaciosAlcaine2002,RivarolaRubio1989,RivarolaRubio1995,RivarolaRubio1990a,RivarolaRubio1990b}, entre otros), por lo que podemos concluir que esta es una modalidad lingüística con continuidad histórica y no meramente el habla de personas bilingües que han aprendido deficientemente una lengua. Al contrario, esas “desviaciones” de la norma estándar del español se han transmitido de generación en generación hasta llegar a nuestros días en lo que puede considerarse una variedad estable (hay que matizar que no todos los rasgos aparecen en todos los hablantes y que esto dependerá, entre otras cosas, de la naturaleza bilingüe o monolingüe, del nivel sociocultural de los hablantes y de sus redes sociales, entre otros factores). En este sentido, los trabajos de  \citet{Calvo-Pérez2008,Delforge2012,Escobar2000,Escobar2011,GarateaGrau2004,GarateaGrau2009,GarcíaTesoro2013,DeGranda2001a,Haboud1998,PalaciosAlcaine2007idayvuelta,PalaciosAlcaine2011,PalaciosAlcaine2013,PalaciosAlcaine2015,PalaciosAlcaine2017book,Pfänder2009,PfänderPalaciosAlcaine2013} o \citet{Zavala1999}, entre otros, aducen pruebas documentales que corroboran esta hipótesis y aportan datos que deben ser presentados y, en algunos casos, reinterpretados en el marco de la lingüística del contacto moderna.

Estamos, pues, ante una variedad oral bien definida que, desde finales del siglo XVI se ha consolidado paulatinamente en una variedad estable con características propias en un proceso lento y complejo, pero constante. Ciertos patrones de las lenguas nativas se fueron transfiriendo al español local de manera continuada, produciendo variaciones lingüísticas significativas, primero en un proceso individualizado que, posteriormente, se difundiría socialmente. Dicho de otro modo, la variedad de español se fue apropiando de esquemas y contenidos de las lenguas originarias y conformando, así, su especificidad.

En el caso del español andino ecuatoriano, es preciso destacar que muchos de los rasgos que lo caracterizan se encuentran en el registro oral de los hablantes monolingües de español, indistintamente del sociolecto al que pertenezcan. Otros, por el contrario, los más fuertemente indexados con el imaginario indígena, solo aparecen en las clases con menor nivel de instrucción, pues la escolarización, como es bien sabido, supone una nivelación que tiende a diluir los rasgos que se apartan en mayor medida de la norma lingüística estándar. Sin duda en este punto, la conciencia lingüística y las actitudes hacia ciertas variedades juegan un papel importante también. 

Es preciso insistir en la complejidad de las situaciones de contacto, aunque se trate de una misma área. El grado de conocimiento de la segunda lengua por parte de los hablantes y los factores socioculturales que rodean históricamente cada comunidad desempeñan un papel fundamental en cómo se produce el contacto lingüístico y sus efectos. En cualquier caso, no es posible hablar únicamente de errores de aprendizaje en la segunda lengua, el español, para explicar la (re)creación de esta variedad de español. 

\section{Valores pragmáticos y discursivos de los tiempos de pasado en zonas sin bilingüismo histórico}\label{sec:palacios:3}


Como ya se ha descrito con profusión (\citealt{BentivoglioSedano1992,DeJong1999,Harris1982,Hernández2006,Hernández2013,HoweSchwenter2003,JaraYupanqui2009,Martínez-Atienza2008}, entre otros), el pretérito perfecto está inmerso en un proceso de gramaticalización desde su creación que aún no ha concluido. Desde el español medieval hay una tendencia de evolución en las formas de pasado que origina que la forma compuesta se apodere de usos originarios de la forma simple. En esta evolución, la distancia temporal del momento en el que se produjo el evento con respecto al momento de la enunciación es un factor predominante mediante el que la forma asume un valor aspectual perfecto, si bien también puede asumir valor aspectual aoristo o perfectivo en determinadas variedades. Se ha demostrado, además, que la evolución de esta forma ha tenido lugar a distintas velocidades en las diferentes variedades de español, por lo que los resultados actuales de este cambio que se constatan en las mismas se corresponden con diferentes etapas del proceso de gramaticalización.

Diversos autores (\citealt{BentivoglioSedano1992,CriadodeVal1948,DeJong1999,GarcíaNegroni1999,GutiérrezAraus2001,Hernández2006,Hernández2013,Howe2018,LopeBlanch1972,JaraYupanqui2009,Martínez-Atienza2008,TorresCacoullosHernández1999,SerranoMontesinos1994}, entre otros), han analizado el avance de los usos del pretérito compuesto en detrimento del simple en distintas variedades del español de España y de Hispanoamérica, y han propuesto valores que tienen que ver con la necesidad del hablante de marcar eventos que tienen especial relevancia expresiva, como en el ya famoso ejemplo de \citet[790--791]{BentivoglioSedano1992} del español venezolano: “y de repente vino una persona, vino una mano, y le \textit{ha dado} un golpe tan duro en la espalda que le quedó la marca de la mano…”.  En la misma línea, \citet{Soto2014} analiza el uso del pretérito compuesto en nueve ciudades del mundo hispanohablante. Constata la relación estrecha que se da entre esta forma y la evaluación subjetiva del hablante, y cómo esta forma se emplea en secuencias narrativas propias del pretérito simple en algunas de estas ciudades. Establece tres etapas de evolución dialectal de los valores del pretérito compuesto \citep[144]{Soto2014}:

\begin{quote}
La de la mayor parte de las capitales americanas, en que el pretérito compuesto se restringe, en términos generales, a ser un perfecto resultativo, existencial o continuativo, aunque posiblemente con una valoración subjetiva asociada; la de Madrid, en que, junto a los casos de perfecto, se dan hodiernales y esporádicamente aoristos, y la de La Paz, en que, además de perfectos, ocurren frecuentemente aoristos de relieve. Lima, a este respecto, tiende a comportarse, \textit{grosso modo}, como la capital boliviana. 
\end{quote}

Por su parte, \citet[276--277]{Hernández2013} muestra que el pretérito compuesto, desde antiguo, se usa en el discurso narrativo como recurso estilístico para expresar evaluación (juicios, opiniones, punto de vista del narrador) y potenciar la implicación del hablante en el discurso. Aduce ejemplos del siglo XVII, como el de \REF{ex:palacios:1}, para mostrar que este recurso tiene una larga trayectoria y que el proceso de gramaticalización de esta forma verbal continúa en la actualidad, para lo cual aporta usos similares vigentes en el español salvadoreño actual, como en \REF{ex:palacios:1}:

\ea\label{ex:palacios:1}
 Y le pareçe aver sido esta narraçion a la puerta de su casa désta, porque se acuerda bien que, cabadas de volver de la iglesia, \textit{estuvieron} alli un ratillo en pie antes de despedirse, diciendo ésta: “buen confesor es el padre Rengel”, \textit{respondio} la dicha mulata: “no lo es porque en la confesión \textbf{me} \textbf{ha} \textbf{requebrado} \textbf{me} \textbf{ha} \textbf{requebrado}” Y la Joanna \textit{dixo}: “tambien a ti te \textit{requebro}?” [DLNE 1621, 86.256].
\z

\ea\label{ex:palacios:2}
\textit{En el propio tiempo de la guerra} ‘taba el toque de queda, que era el toque de queda a las ocho. Había un señor que se llamaba (...), era mayor del ejército, ya estaba jubilado el señor. Y yo tenía un cuñado que era teniente, vivía con una mi hermana, pues. \textit{Y ese día}, pues, extorcimiento, ya pues, ya, ya me convenía, pues. Hoy digo yo, sé que, que ya me convenía pues. Me \textit{robó} quinientos colones, le dije yo, que esos no eran ni míos, le \textit{dije} yo, eran de mi hermana, le dije yo, me robó. Me rompieron los documentos, le dije yo, todos los documentos, le dije yo, una constancia de baja, le dije yo, que yo había obtenido, le dije yo, ellos me la rompieron, le dije yo. Así es que todo, no tengo ningún con que presentarme le dije yo, algo le dije yo, cuando me pare algún soldado o un guardia, lo que sea, le dije yo, con qué documentos me voy a identificar, le dije yo. \textbf{me} \textbf{ha} \textbf{dejado} \textbf{fregado,} \textbf{le} \textbf{dije} \textbf{yo} \textbf{y} \textbf{(me} \textbf{ha} \textbf{dejado)} \textbf{todo} \textbf{morado}, le dije yo... (SSC 2000, M, 38).
\z

Se trata de usos innovadores para el español, congruentes con lo sucedido en otras lenguas en las que el pretérito compuesto ha experimentado un proceso de gramaticalización más evolucionado. \citet[17]{Howe2018} afirma al respecto que el compuesto es una forma que puede expresar relevancia actual en el nivel del discurso general o de eventos individuales que el hablante quiere destacar subjetivamente. Esta idea de relevancia expresiva es similar a la que describen  \citet{JaraYupanqui2011} o    \citet{CaravedoKlee2012} en el español limeño. Consideran las autoras que el compuesto codifica en esta variedad la implicación emocional del hablante con respecto a la acción que transmite. Permite al narrador evaluar un hecho que está narrando, para subjetivarlo, y es usado así para expresar su perspectiva y marcar subjetividad, ya sea para evaluar, resumir, comentar, reproducir voces del discurso o marcar la fuente de la información. Este significado evaluativo tiene valor pragmático y discursivo.

\section{Los valores evidenciales en los tiempos de pasado en el español andino}\label{sec:palacios:4}


Como es bien sabido, los tiempos de pasado pretérito simple, compuesto y pluscuamperfecto muestran cierta variación dialectal en el español, variación que no solo afecta a los valores temporales y aspectuales de estos tiempos, sino también a valores relacionados con la evaluación que hace el hablante de la información transmitida (\citealt{BentivoglioSedano1992,DeJong1999,Harris1982,Hernández2006,Hernández2013, Howe2006,Howe2009,HoweSchwenter2003,JaraYupanqui2009,JaraYupanqui2011}, entre otros). En el caso del español andino, esta variación de los valores pragmáticos se ha enriquecido con usos evidenciales y (ad)mirativos (\citealt{Blestel2015rioplatense,Bustamante1991,Calvo-Pérez2008,Escobar1997,Escobar2000,GarcíaTesoro2013,GarcíaTesoro2015,GarcíaTesoro2017,GarcíaTesoro2018,Godenzzi1986,DeGranda2001a,DeGranda2001b,Haboud1998,KleeOcampo1995,PalaciosAlcaine2007idayvuelta,Pfänder2009,PfänderPalaciosAlcaine2013,Sánchez2004,Stratford1991}), entre otros. La bibliografía ha mostrado que estos valores evidenciales y (ad)mirativo\footnote{En esta investigación solo me ocuparé de los valores evidenciales de los tiempos de pasado. Para una explicación de los usos (ad)mirativos véase    \citet{PalaciosAlcainePfänder2018}.} se deben a la singular situación de bilingüismo histórico intenso de la zona, donde la influencia de las lenguas de contacto (quechua y aimara) juega un papel esencial como desencadenante de estos valores emergentes. Ahora bien, la bibliografía menciona que estos valores emergentes no se codifican a través de las mismas formas verbales de pasado en toda el área andina. Al respecto, \citet{DeGranda2001b}, a partir de datos descritos por otros autores, reordenaba esta diversidad en dos subáreas de español andino:\footnote{Esta clasificación de las zonas de uso de los pretéritos andinos en español no se corresponde exactamente con las diferentes variedades de quechua (I y II) descritas por \citet{Torero1964}.} un área 1, que comprendería el español andino ecuatoriano y la serranía norte y central peruana, donde se mantendrían las formas verbales de pretérito simple, compuesto y pluscuamperfecto; y un área 2, integrada por el sur andino peruano, Bolivia y el noroeste argentino\footnote{Afirma \citet{DeGranda2001b} que en el noroeste argentino la oposición con significados evidenciales o epistémicos del pretérito compuesto \textit{vs}. el pluscuamperfecto sería residual en la actualidad y circunscrita a las áreas rurales; por el contrario, se mantendría el valor (ad)mirativo del pluscuamperfecto.}, en la que el pretérito simple habría desparecido casi totalmente en la variedad oral coloquial.

Es evidente que la asimetría del paradigma de pasado que se describe en ambas zonas se corresponde con soluciones diferentes para codificar los valores evidenciales; por ejemplo, en el área 1 los valores de evidencia indirecta corresponderían al pretérito compuesto, mientras que en el área 2 los absorbería el pluscuamperfecto, según \citet{DeGranda2001b}. Lo realmente interesante es que las soluciones que muestran las distintas variedades andinas que mantienen las tres formas de pretérito (zona 1), no son las mismas en Ecuador que en Perú, por tanto, es preciso comprender cómo funcionan los paradigmas de pasado en las distintas variedades andinas (más allá de si se mantienen o no las dos zonas recogidas por de Granda) y qué valores pragmáticos involucran cada una de estas formas.

Dado este panorama, la pregunta que surge de inmediato es ¿por qué se dan diferentes soluciones en las formas de pasado si en el contacto lingüístico están involucradas las mismas lenguas? Cabe preguntarse, entonces, cuáles son los factores que influyen en la variación de esas soluciones y si es posible pensar en reorganizaciones del paradigma de pasado en cada una de estas áreas en vez de considerar la evolución individual de cada uno de los tiempos.

Para avanzar en estas cuestiones es preciso, primero, hacer algunas reflexiones sobre la naturaleza de los cambios inducidos por contacto en cada una de estas áreas. 

En cuanto a la sierra andina ecuatoriana,  \citet{PfänderPalaciosAlcaine2013} consideran que la oposición pretérito simple/compuesto/pluscuamperfecto está vertebrada por la dimensión evidencial entendida en términos no solo de codificación de la fuente de información sino también de compromiso y distanciamiento del hablante con respecto a la veracidad de la información que transmite. Así, el pretérito simple, en términos canónicos, alude a un evento que finalizó en el pasado perfectivo sin relevancia en el presente, pero, en lo que constituye el paradigma verbal evidencial, puede expresar también valores que indican la fuente de información (experiencia directa) y certeza sobre la información transmitida (aserción), que implica el compromiso del hablante con la veracidad de la aserción. El pretérito compuesto, además de los valores canónicos (perfecto de resultado, de experiencia o de situación persistente), puede codificar valores de evidencia indirecta, esto es, que la información que se transmite es de una fuente reportada y, por tanto, el hablante no se compromete con su veracidad.\footnote{En términos pragmáticos, el hablante atenúa la aserción para mitigar su compromiso con la información transmitida y distanciarse, por tanto, de la veracidad de la misma, eludiendo así su responsabilidad.} Finalmente, el pluscuamperfecto, además de los valores canónicos, expresa valores evidenciales de información indirecta con poca credibilidad para el hablante, información supuesta o conjetura.

Los ejemplos de \REF{ex:palacios:3} tomados de  \citet{PfänderPalaciosAlcaine2013} a partir del trabajo de campo realizado en 2012 en Quito con hablantes monolingües de español ilustran este paradigma, donde el pretérito simple expresa eventos pasados experimentados por el hablante o que conocen bien y transmiten como fiables; en los casos de \REF{ex:palacios:4}, procedentes de hablantes monolingües tomados del trabajo de campo realizado en 2012 en Quito, el pretérito compuesto codifica evidencialidad indirecta, esto es, una información reportada por otra persona que la experimentó (los verbos de habla aparecen en cursiva), pero de la que se distancia el hablante al no hacerse responsable de su veracidad; en \REF{ex:palacios:5}, ejemplos de hablantes monolingües y bilingües tomados del corpus COREC\footnote{\textit{Corpus oral de referencia de español en contacto} (\textit{COREC}), coordinado por Azucena Palacios Alcaine (\url{http://espanolcontacto.fe.uam.es/wordpress/index.php/corpus-oral-de-referencia}). Los trabajos de campo se realizaron en 2015, 2016 y 2017 por diferentes encuestadores ecuatorianos con colaboradores bilingües y monolingües.}, el pluscuamperfecto funciona como una forma narrativa de pasado lejano: el hablante se distancia de la información si son hechos pasados antiguos reportados por terceros o si son una suposición:

\largerpage
\ea\label{ex:palacios:3}
\ea El único castigo que me \textbf{dieron} allí \textbf{fue} un reporte fuerte, bueno el primero el primero me \textbf{llevaron} a un depósito de una almacenería (Quito, 2012, hablante HM4, hombre, monolingüe).
\newpage
\ex De de cinco hijas, no \textbf{tuvo} nunca varoncitos, a las cinco hijas, la mayor, XXX, ella se \textbf{quedó} aquí, de profesora, y otra que se llama XXX también \textbf{quedó} aquí y a los diecisiete años se \textbf{casó} y \textbf{tuvo} hijos (Quito, 2012, hablante HM2, hombre, monolingüe).
\z
\z

\ea\label{ex:palacios:4}
\ea Eso es lo que él \textit{cuenta} lo más que \textbf{ha} \textbf{pasado} bien [en España], pero lo que \textbf{ha} \textbf{pasado} mal, nunca \textit{cuenta}. Sólo lo \textit{cuenta} lo que… se \textbf{ha} \textbf{ido} a los parques, se \textbf{han} \textbf{encontrado} con los primos, eh/ \textbf{han} \textbf{estado} bebiendo, se \textbf{han} \textbf{ido} a ver un concierto, \textbf{han} \textbf{estado}… o sea, (…), eso es lo que \textbf{ha} \textbf{conversado}. (…). A ver, se fueron los tres. Se fueron los tres y regresaron un hijo más (Quito, 2012, hablante HM2, hombre, monolingüe).
\ex  Eso encuentran allá, será la… no sé, ahí no había un baúl, pero \textit{dice} que \textbf{ha} \textbf{estado} \textbf{tocando} al baúl [...] que \textbf{ha} \textbf{estado} \textbf{cavando}, pero \textit{dicen} que cuando una mujer asoma se... va huyendo esa caja… (Quito, 2012, hablante HM5, hombre, monolingüe).
\z
\z

\ea\label{ex:palacios:5}
\ea No había la escuela; esto se hacía después, después... ya llegaron a / En mi pueblo, sí, porque mi abuelito le \textbf{había} \textbf{donado} la tierra y \textbf{había} \textbf{hecho} una escuela y una iglesia, porque mi abuelito \textbf{había} \textbf{tenido} bastante tierra y le \textit{donó}\footnote{Esta donación es un hecho contrastable y, por tanto, aparece narrado en pretérito simple.} a todo el barrio, y entonces ahí ya desde pequeños estudiábamos, y no/ para ir a aprender a leer (Quito, 2015, hablante COREC-MM12, mujer, monolingüe).
\ex Las historias: que es de nuestra comunidad de Pijal... / eh: tamién [también] \textbf{habían} \textbf{sabido} ser de ellos también / no solamente \textit{han sido} nuestras // (Quito, 2015, hablante COREC-MB18, mujer, bilingüe, procedente del cantón Otavalo).
\z
\z

Los valores evidenciales que hemos documentado en Ecuador se dan tanto entre hablantes bilingües como monolingües, si bien son los hablantes con menor nivel de instrucción los que tienden a utilizarlos más frecuentemente.

El español andino del sur de Colombia ha sido insuficientemente investigado, no obstante,  \citet[172]{PortillaMelo2010} describe para los tiempos compuestos de pasado valores similares a los descritos para el español andino ecuatoriano. Así, el pretérito compuesto, además de los usos canónicos, expresaría valores de evidencia indirecta reportada ––\textit{Que le ha rogado pa’trabajar, para sacar unos postas de allá bajo de una peña y quez- que [que dice que]} \textit{lo ha derrumbado}––, donde el hablante se distancia de la información para achacar la responsabilidad sobre la misma a la fuente de donde procede. El pluscuamperfecto codifica un distanciamiento mayor del hablante sobre la información transmitida, hechos ocurridos en un tiempo lejano y que él no presenció, por lo que la responsabilidad sobre la información transmitida se atenúa ostensiblemente –– \textit{Contaban mis tíos, porque ellos quedaron viviendo, que el papá, don Salvador, habían sido tres hermanos que habían nacido en Santa Rosa del Quilichao, en el norte, propiamente que ellos habían vivido en Popayán, contaban mis tíos.}

Por lo que respecta al español andino peruano (\citealt{Escobar1997,Escobar2000,Godenzzi1986,GarcíaTesoro2015,GarcíaTesoro2017,GarcíaTesoro2018,Howe2018,PalaciosAlcaineEtAl2018}), parece que el perfecto simple es la forma no marcada para expresar pasado sin relevancia en el presente, a diferencia de la forma compuesta que puede expresar, además de los canónicos, valores pragmáticos de relevancia discursiva, donde el hablante destaca y actualiza en el discurso determinados eventos de su narración, para llamar la atención del oyente o mostrar su actitud hacia los mismos. En este sentido, \citet{Escobar2000}, a partir del análisis de un corpus de 30 horas de grabaciones con hablantes bilingües de zonas rurales y urbanas, concluía que había documentado las tres formas de pasado y que la noción de relevancia temporal, espacial y experiencial es la que marcaba los valores pragmáticos del pretérito compuesto, indistintamente de la procedencia del hablante bilingüe y de la variedad de quechua que hablara. Los ejemplos de \REF{ex:palacios:6}, tomados de \citet[242––245]{Escobar2000}, recrean ese recurso verbal del que dispone el hablante para enfatizar el evento pasado experimentado:

\ea\label{ex:palacios:6}
\ea maíz abundanza ese tiempo / maíz eran grandes señorita / ese tiempo no había ni carretera / cuando \textit{llegaron} carretera creo que es 40 41 por allí creo \textit{llegaron} / cuando carretera \textbf{ha} \textbf{llegado} / entonces \textbf{ha} \textbf{venido} carros y carros venden con gasolina petrolero / entonces que la sembría se malograba / ya no se cosechaba como antes /se poquito chiquito nomás
\ex \textbf{ha} \textbf{sido} rateros / entonces \textbf{han} \textbf{entrado} dice dos ratero pe \textbf{han} \textbf{estado} \textbf{saliendo} con televisor cargado ya / entonces chiquitos toitos \textbf{han} \textbf{despertados} \textbf{han} \textbf{agarrado} por aquí \textbf{han} \textbf{gritao} / toitos se \textbf{han} \textbf{salido} pe / y bien amarrado ya todos los vecinos se \textbf{han} \textbf{salido} con soga todo / acá la población \textbf{hemos} \textbf{hecho} andar toa la noche /… el hombre tenía cuchillo pues / cuchillo grande tenía / y el hombre pe decía no no \textbf{he} \textbf{sido} ratero el otro \textbf{ha} \textbf{sido} / … [la población] querían amarrar acá al poste del local comunal / así para ahorcarlo / … entonces de allí ya ahí \textit{dijeron} no mejor mejor no hay que ahorcar sino hay que mandarlo al hueco allá al fondo / y ahí estaban ya llevando estamos llevando ya / y no sé quién habrá ido a la comisaría pue / a la PIP / y \textit{vino} pe la comisaría \textit{vino} este patrullero \textit{vino} / cuando tamos justo ya cerca al hueco ya / \textit{llegó} el guardián toito y nos \textit{quitó} al ratero y se \textit{llevó} / se \textit{llevó}  / y todavía encimo se \textbf{ha} \textbf{llevado} a los vecinos que tenían agarrado a él / l’\textbf{han} \textbf{calumniado} /…/ o sea falso no todavía \textbf{han} \textbf{quedado} detenido / creo tres días cuatro días
\z
\z

\citet{GarcíaTesoro2015,GarcíaTesoro2017,GarcíaTesoro2018} y  \citet{GarcíaTesoroJang2018}, con datos de hablantes bilingües y monolingües del departamento de Cuzco realizados con la misma metodología (historias de vida en contextos informales), abundaban en la misma línea y mostraban tres funciones pragmáticas para el pretérito compuesto relacionadas con el ámbito de la modalidad que expresan valores subjetivos sobre la actitud del narrador frente a lo narrado: a) codificar en el discurso la relevancia que el hablante otorga a determinados eventos para hacerlos más vívidos y actualizarlos, b) codificar la relevancia de eventos experimentados y el compromiso del hablante con los mismos, c) codificar eventos en la narración que discursivamente pueden ser relevantes (marcar un punto culminante, comentar, evaluar o resumir la narración, etc.).

En el trabajo de  \citet[112]{GarcíaTesoroJang2018} se pone de manifiesto que el uso del pretérito compuesto con estos valores pragmáticos se da con una alta frecuencia entre los hablantes con estudios primarios (63.6\%); frecuencia que disminuye entre los colaboradores con estudios secundarios (37.4\%) y que apenas tiene presencia entre los que poseen estudios universitarios (13.2\%), por lo que concluyen que se trata de usos marcados, indexados a evaluaciones negativas que suelen ser evitados por los hablantes con mayor conocimiento de la norma. Para corroborarlo, cuantifican los usos de pretérito compuesto y simple entre hablantes rurales (Chinchero) y urbanos (Cuzco). Los datos confirman que, en efecto, el abandono de la norma andina (usos pragmáticos del pretérito compuesto) se extiende entre los colaboradores residentes en la ciudad (29.1\%) \textit{versus} los de la zona rural (58.7). Estos resultados están en línea con lo que \citet[323]{Godenzzi1986} afirma sobre el uso de las formas de pretérito simple y compuesto en Puno, puesto que documenta que los sociolectos “urbanos con mayor capital económico-escolar recurren de un modo predominante a las formas simples del pasado, en tanto que los grupos más desfavorecidos y de tradición rural quechua-aimara son los que hacen mayor uso de las formas compuestas”.

Los casos de pretérito compuesto que aparecen en los ejemplos de \REF{ex:palacios:7}, tomados de García Tesoro (\citeyear{GarcíaTesoro2017}: 86––88; \citeyear{GarcíaTesoro2018}: 125), aluden a un pasado lejano e ilustran la explotación discursiva que el hablante hace de ciertos eventos experienciales en su discurso para situarlos en un primer plano de la escena. En cuanto a las formas de pluscuamperfecto, parecen tener valores evidenciales de distanciamiento del hablante con respecto a una información no experimentada sino reportada por otra persona, como se muestra en \REF{ex:palacios:8}, tomado de \citet[62––63]{GarcíaTesoro2015}):

\ea\label{ex:palacios:7}
\ea P: ¿allí fue fuerte el terrorismo (años 80 y 90), más que en Cuzco?\\
J: Fuerte, uff, fuerte, fuerte… Incluso a mí me \textbf{han} \textbf{agarrado} pero yo no \textbf{he} \textbf{hecho} daño a nadie, no me \textit{hicieron} naaada…
\ex A: Es que también lo \textit{mataron} a él antes de su gobierno loo… \textit{murió}, entonces \textit{quedó} ahí. \textit{Entraron} otros gobiernos ahora, eh... Fujimori \textit{implantó} pero todo era así. Ahora ya no quieren. Fujimori \textbf{ha} \textbf{dado} que las mujeres se hagan el \textbf{\textit{ligado}}. Después que los- las mujeres sean atendidas gratuitamente en todo. Después el vaso de leche él \textbf{ha} \textbf{creado}. Con... Alán García igualito, \textbf{ha} \textbf{repartido} dineros a todos y campesinos después los \textbf{ha} \textbf{condonado,} \textbf{no} \textbf{han} \textbf{pagado}, ¿pues qué más quieren?
\ex G: Sí, \textit{nací} aquí y mi pa- mi mamá se \textbf{ha} \textbf{fallecido} dejando tres mesecitos vivita, sí, vivita no más me \textbf{ha} \textbf{dejado}. Mi papá se \textbf{ha} \textbf{fallecido} cuando tenía tres añitos, también se \textbf{ha} \textbf{fallecido}, a mí me \textit{tenieron} (tuvieron) mi, mis tíos, tía tío no más me \textit{tenieron} (tuvieron) y después me \textit{llevaron} grandecita, unos doce años ya a la escuela.
\z
\z

\ea\label{ex:palacios:8}
I: A un matrimonio, un viejito creo que \textbf{había} \textbf{entrado}... (\textit{perro ladrando}) a saludar, y esa señora, los que se han casado... a una casa que se han matrimoniado \textbf{había} \textbf{entrado} un viejito, y le \textbf{habían...} \textbf{habían} \textbf{tenido} asco porque estaba el abuelito con su moco así todo cochinito, entonce \textbf{habían} \textbf{tenido} asco, entonces de ahí dice una señora lo \textbf{había} \textbf{limpiado}, ¿no?, al abuelito. Entonces a esa señora le \textbf{había} \textbf{dicho}: “Anda, hijita al cerro, anda, pero no te vas a mirarte a tu detrás porque... no te vas a mirar, sin mirar te vas irte”. Y la señora dice que se fue al cerro ahí, dice cuando del cerro volteó miró atrás entonce la laguna estaba creciendo, a las casas así estaba tragando, entonces así en ahí la señora dice que se ha encantado, piedra había, ahí había.


E: ¿Ah sí?

I: Sí, hay una señora con... con su pushka así está, está hilando, ahí se \textbf{había} \textbf{encantado} y así.
\z

En definitiva, el pretérito compuesto permite al hablante codificar la relevancia epistémica de un evento que considera sobresaliente. En cuanto al pluscuamperfecto, expresa o codifica el distanciamiento (temporal y/o metafórico) del hablante con respecto a la información transmitida. 

En el caso de Bolivia (Cochabamba), parece que el pretérito simple es una forma poco frecuente en la variedad oral coloquial y está indexada con un registro formal y con sociolectos altos; el compuesto es la forma no marcada para expresar eventos pasados y el pluscuamperfecto permite expresar valores de evidencia indirecta\footnote{Como es bien sabido, también expresa (ad)miratividad (\citealt{PalaciosAlcainePfänder2018}), pero ese valor no se trata en esta investigación. Los ejemplos (9--11) han sido tomados del corpus \textit{Rosa Alcira Cuenca Quirós. Más que una heroína}, editado por  \citet{CoelloVilaPfänder2012}.}, esto es, información reportada de la que el hablante no está seguro o no tiene evidencia directa, como en \REF{ex:palacios:9}. Puede expresar igualmente valores epistémicos a partir de la evaluación que el hablante hace de la información transmitida, como su distanciamiento cuando la categoriza como suposición, rumor o conjetura (no certeza) y no solo como fuente de información indirecta, como en \REF{ex:palacios:10}. Esto es, el pluscuamperfecto permite codificar la evaluación que el hablante hace sobre la información que transmite y su falta de compromiso con la misma para establecer su valor de verdad.

\ea\label{ex:palacios:9}

—¿Y cómo es el calvario?\\
—Se sube arriba al cerro, se saca piedra, \textit{dice} que había unos borrachitos, que \textbf{habían} \textbf{hecho} su virgencita de piedra\\
—Ah sí, ¿cómo?\\
—\textit{Dice} que una virgencita de piedra \textbf{habían} \textbf{tallado} y allí ovejitas \textit{dice} \textbf{habían} \textbf{hecho.}


—Sí, [...] pero como han pasado todo el día pues, todo el día han estado ¿no vé?\\
—Se han ido ellos, nosotros estábamos jugando, helados nos hemos comprado, refrescos, al baño hemos ido, la Neli se ha trancado en el baño, no podía salir, “ay no se puede”, yo igual al mismo baño he entrado, “ay no se puede”, chistoso hemos salido [CC.1,301]. 
\z

\ea\label{ex:palacios:10}
Po:  \textbf{habiá} \textbf{sido} su:: de mi padrino su comadre \textbf{habiá} \textbf{sido} pues


To:  ah ¿sí?

Po:  no, no, de su compadre: su hermana \textbf{habiá} \textbf{sido}

To:  ¡ah:::!

Po:  una vez así, ese no, pero a esa señora le conozco, me \textit{dice} pues, ah:::: ¿quien es ésa? ella es pues del trifón su del trifón su hermana \textbf{habiá} \textbf{sido} ¿no? y ahí ps su marido: \textbf{habiá} \textbf{sido} ahí, mareados \textbf{habián} \textbf{sido} ps…
\z

Tras revisar la bibliografía, podemos concluir que el pluscuamperfecto es la única forma que comparte los mismos valores en todas las áreas andinas, se convierte en una estrategia de distanciamiento del hablante con respecto a la información transmitida, ya sea debido a que se trata de una información reportada, no contrastada o inferida de la que no se tiene seguridad. En cuanto al pretérito compuesto, el español andino peruano avanza en los usos pragmáticos y discursivos ya descritos para esta forma en variedades de español sin contacto (\textit{vid}. \sectref{sec:palacios:3}). Los casos andinos ecuatoriano y colombiano, sin embargo, se apartan de esta tendencia, ya que el pretérito compuesto adopta en estas áreas usos más innovadores relacionados con el dominio evidencial (evidencia indirecta). Finalmente, en el español andino boliviano, el pretérito compuesto ha pasado a ser la forma canónica, no marcada, para codificar eventos pasados al haber desaparecido casi por completo el pretérito simple en el registro oral coloquial.

Es preciso destacar que se trata de valores pragmáticos, por lo que el hablante tiene agencia para activar o no los valores evidenciales y epistémicos. Parece que los hablantes con nivel de instrucción más alto son los que menos los activan (o no los activan), utilizando preferentemente recursos como “dice”. Estos valores evidenciales, marcados, se dan con mayor o menor frecuencia tanto en bilingües como en monolingües. Son estrategias discursivas que el hablante activa o no en función de sus intereses comunicativos.

El cuadro siguiente recoge los usos de estas formas verbales en las áreas andinas descritas:

\begin{table}
\caption{\label{tab:palacios:1}Valores pragmáticos de los tiempos verbales en el español andino}

\begin{tabularx}{\textwidth}{lQQQ}
\lsptoprule
& Ecuador/ Colombia & Perú & Bolivia\\
\midrule
P. Simple & Evidencia directa & {}- Relevancia informativa & ---\\
Compuesto & Evidencia indirecta, reportativo & + Relevancia informativa & {}- Distanciamiento\\
Pluscuamp. & + Distanciamiento & + Distanciamiento & + Distanciamiento\\
\lspbottomrule
\end{tabularx}
\end{table}

\section{Procesos de gramaticalización del pretérito compuesto y del pluscuamperfecto en el área andina}\label{sec:palacios:5}

Como hemos visto en \sectref{sec:palacios:4}, los procesos de gramaticalización de las formas de pretérito compuesto desencadenados en las zonas andinas no parecen seguir la misma evolución; al contrario, parece que existen mecanismos divergentes que desembocan en soluciones específicas para cada una de las áreas. Otra cosa es la evolución del pluscuamperfecto, que parece desarrollar mecanismos paralelos de gramaticalización en todas las áreas andinas.

En el caso de Bolivia (Cochabamba), la forma de pretérito perfecto no parece haber desarrollado valores pragmáticos dado que es empleada en contextos canónicos, al menos en el oral coloquial, pues es la forma no marcada para codificar eventos pasados.

El pretérito perfecto en el español andino peruano, por el contrario, ha desarrollado usos pragmáticos y discursivos que parecen seguir la evolución iniciada en otras variedades del español sin contacto, en lo que podríamos denominar una evolución “canónica”, en el sentido de que sigue las tendencias interlingüísticas vistas no solo para el español sino también para otras lenguas. Como ya han mostrado   \citet{BybeeEtAl1994}, existen ciertas tendencias interlingüísticas que apuntan a que los tiempos verbales que expresan pasado gramaticalizan hacia elementos que indiquen aspecto perfectivo para asumir después interpretaciones de tiempo pasado. La ampliación de los valores de la forma compuesta a costa de significados que anteriormente expresaba la forma simple supone así un cambio lingüístico, un proceso de gramaticalización condicionado por factores temporales en los que interviene la relación de las referencias del proceso pasado y del momento de la enunciación. Y este es un cambio en progreso, ya que el pasado compuesto está inmerso en una situación de cambio lingüístico que no termina en estas variedades. Ahora bien, la gramaticalización del pretérito compuesto hacia formas más subjetivas y evaluativas en el español andino peruano parece acentuarse con una evolución más rápida, alcanzando frecuencias y contextos más amplios que las experimentadas en otras zonas no andinas, esto es, seguiría la evolución ‘natural’ en el proceso de gramaticalización iniciado por esta forma verbal en español en lo que podría ser un estadio más avanzado de gramaticalización.

  Finalmente, el español andino ecuatoriano y colombiano han tenido una evolución más innovadora, dado que el pretérito compuesto ha desarrollado valores de evidencia indirecta (reportativo) en un proceso de gramaticalización que promueve soluciones divergentes con respecto al resto de variedades. En el \tabref{tab:palacios:2} se muestran estas evoluciones:

\begin{table}
\caption{\label{tab:palacios:2} Gramaticalización del pretérito compuesto en el español andino}

\begin{tabularx}{\textwidth}{lQ}
\lsptoprule
Variedades no andinas & Valores canónicos\footnotemark{} → valores pragmáticos y discursivos (relativamente poco habituales, en contextos marcados)\\
\midrule
Perú & Valores canónicos → valores pragmáticos y discursivos (muy frecuentes)\\
Ecuador & Valores canónicos→ valores de evidencia indirecta (reportativo)\\
Bolivia & Valores canónicos \\
\lspbottomrule
\end{tabularx}
\end{table}
\footnotetext{Entiendo por valores canónicos los que tienen significados temporales/aspectuales básicos y no han adoptado valores pragmáticos como los estudiados aquí.}

En cuanto al pluscuamperfecto, hay cierto consenso en describirlo como un tiempo relativo de pasado que remite a un estado anterior y codifica dos eventos realizados en un pasado inactualizado que se relacionan mediante  un cierto distanciamiento temporal, pero también cognitivo (\citealt{Bermúdez2011,Pfänder2009,SotoHasler2013,SotoOlguín2010,Söhrman2013}). Ese distanciamiento del hablante con la conceptualización de un proceso o de una situación concreta será esencial para explicar qué ocurre con los valores evidenciales de esta forma verbal en el español andino, donde se acentúa ese distanciamiento no solo temporal sino metafórico, codificando así información indirecta, rumor, información supuesta, no contrastada, conjetura, etc.

En esta línea, \citet[230]{Pfänder2009} considera que “La oposición entre perfecto y pluscuamperfecto en castellano estándar es de carácter temporal: el pluscuamperfecto está situado más lejos del \textit{origo} del hablante –– en el sentido de Bühler –– que el perfecto. De acuerdo con los desarrollos metafóricos-metonímicos de la teoría de la gramaticalización, un ‘mantenimiento’ de la distancia (figurada) podría haber facilitado la evolución. El perfecto y el pluscuamperfecto son, así, semánticamente reelaborados [en el español andino] a través de una oposición conocida ya por el quechua, en el que es obligatoria”.

Como hemos visto, en todas las áreas andinas el pluscuamperfecto ha adquirido valores evidenciales y de evaluación, codificando el distanciamiento del hablante con respecto a la información transmitida. En el caso de Perú y Bolivia, esta forma verbal adopta valores de evidencia indirecta ya sea información reportada, ya sea información inferida o no confiable. En el caso de Ecuador, expresa valores de evidencia indirecta inferida o no atribuible a un tercero concreto que se responsabilice de la misma, ya que los valores reportativos se codifican con el pretérito compuesto.  En el cuadro 3 se muestran las tres soluciones adoptadas.

\begin{table}
\caption{\label{tab:palacios:3} Gramaticalización del pluscuamperfecto en el español andino}
\begin{tabularx}{\textwidth}{lQ}
\lsptoprule
Perú & Valores canónicos → valores de evidencia indirecta reportada, inferida, dudosa\\
Bolivia & Valores canónicos → valores de evidencia indirecta reportada, inferida, dudosa\\
Ecuador & Valores canónicos canónicos → valores de evidencia indirecta inferida, no contrastable\\
\lspbottomrule
\end{tabularx}
\end{table}

\section{¿Cómo surgen los valores evidenciales en el área andina?}\label{sec:palacios:6}

 \subsection{La evidencialidad en quechua y en aimara}\label{sec:palacios:6.1}

Se ha descrito con profusión que el dominio semántico de la evidencialidad es, comunicativamente, muy importante en el quechua y el aimara hasta el punto de que esta categoría se codifica en los tiempos verbales y mediante un sistema de afijos evidenciales –– el quechua mantiene ambos sistemas separados, mientras que el aimara ha incorporado parcialmente estos morfemas evidenciales en el sistema verbal \citep[6]{Adelaar1997} ––.

Si centramos la atención en el quechua, en general se describen tres tipos de sufijos evidenciales y epistémicos que aluden a la fuente de la información y al compromiso del hablante con la información transmitida, esto es, si es responsable en alguna medida de la veracidad de esa información o si hay un distanciamiento respecto de la misma. En cuanto a la evidencia directa, -\textit{mi} expresa testimonio y experiencia; -\textit{shi} evidencia indirecta o reportativa, por lo que el hablante no se siente implicado con el valor de verdad de la información que transmite, lo que supone una atenuación de la responsabilidad del hablante que solo reporta o refiere lo que otro ha dicho; y el conjetural -\textit{chi} supone un distanciamiento mayor de la información, ya que no implica a un tercero a quien responsabilizar de esa información (\citealt{Adelaar1997}; \citealt{Weber1989}).

En cuanto a los tiempos verbales, \citet{Cusihuamán1976} describe dos formas de pasado que codifican valores evidenciales: a) {}-\textit{ra/rqa}, que expresa una acción concreta y terminada en el pasado, sobre la que el hablante ejerce algún control o tiene experiencia\footnote{El subrayado es mío.}; b) \textit{-ska/shka}, que expresa una acción realizada y terminada en el pasado, pero en la que el hablante no ha participado, solo ha oído narrar esa acción. En opinión de \citet{Mannheim1987}, esta forma verbal permite expresar además situaciones nuevas, inesperadas, que el hablante acaba de descubrir como sucesos soñados, imaginados, delirados o míticos.

Lo interesante es que la evidencialidad en estas lenguas parece que no codifica solo la fuente de información sino que además entraña valores epistémicos, evaluativos relacionados con el compromiso del hablante hacia la información transmitida. En esta línea \citet{Adelaar1997}, \citet{Faller2007}, \citet{Gipper2014}, \citet{Howard-Malverde1988}, entre otros, afirman que el hablante codifica la evidencialidad de manera creativa y dinámica y la usa como una estrategia para conseguir réditos comunicativos, agregando un elemento subjetivo a la noción de la fuente de datos. En este sentido,  \citet[8––10]{Adelaar1997} considera que “[e]l hablante explota las posibilidades de la lengua haciendo un uso subjetivo de estos marcadores [evidenciales] en función de la \textit{distancia} con los hechos que quiere transmitir”.\footnote{El destacado en cursiva es mío.} Por lo que respecta a los tiempos verbales, \citet[22]{Howard-Malverde1988} o    \citet{MannheimVanVleet2000} aluden igualmente a su empleo en un juego dinámico donde no solo se codifican las características temporales de un evento narrado sino también el grado de personalización e implicación del hablante en el relato.

En definitiva, las distintas formas de marcar la evidencialidad permiten desarrollar diversas estrategias pragmáticas para reorganizar de manera creativa el papel del narrador y su relación con los hechos narrados (distanciamiento o compromiso) en función de su evaluación, marcar la relevancia de determinados hechos subjetivos o destacar la relación de los participantes en el discurso. 

 \subsection{Cambios indirectos inducidos por contacto}\label{sec:palacios:6.2}


Considero que las lenguas quechua y aimara juegan un papel esencial en la expresión de valores evidenciales y evaluativos en los tiempos de pasado del español andino, pero no se trata de calcos semánticos, sino de cambios indirectos inducidos por contacto. Siguiendo a \citet{PalaciosAlcaine2007idayvuelta,PalaciosAlcaine2011}, estos cambios tienen lugar mediante la influencia indirecta de una lengua en contacto A sobre una lengua B, donde surgen variaciones gramaticales muy significativas. Esas variaciones aprovechan la evolución interna de la lengua B para crear soluciones cuya funcionalidad comunicativa obedece a procesos cognitivos de la lengua A de contacto. Estos cambios pueden conllevar la aceleración de un cambio lingüístico en proceso y la eliminación de las restricciones lingüísticas que impidan su expansión, la reorganización de un sistema completo, la reasignación de nuevos valores a estructuras existentes en la lengua, entre otros efectos, que se materializan en las prácticas lingüísticas de la comunidad. Las pautas del cambio, a diferencia de lo que ocurre con los cambios directos -- donde hay un trasvase de patrones o elementos de la otra lengua de contacto --, siguen procedimientos generales y sistemáticos, cuyos efectos pueden, en cierta medida, preverse en función de las características estructurales, y quizá también cognitivas, de las lenguas implicadas.

En el caso que nos ocupa, el quechua y el aimara poseen valores evidenciales y epistémicos gramaticalizados en el sistema verbal y reforzados en un sistema de afijos propio; la codificación de estos valores en los tiempos de pasado del español andino puede explicarse mediante el acercamiento cognitivo de la variedad castellana a estas lenguas amerindias. La confluencia de factores externos (el contacto con estas lenguas) e internos (las tendencias evolutivas del sistema verbal) actúan conjuntamente para hacer emerger soluciones novedosas (español andino ecuatoriano y colombiano) o recrear valores pragmáticos ya vislumbrados tímidamente en el español (español andino peruano).

La reelaboración de estructuras ya existentes en español, la preferencia por una forma alternativa frente a otra porque tiene elementos significativos o cognitivos comunes con formas similares en la lengua materna de los hablantes bilingües o la adopción de nuevos significados pragmáticos son estrategias lingüísticas en las que subyace el mecanismo de la convergencia lingüística. Este mecanismo se activa cuando los hablantes bilingües perciben similitudes entre los significados de los tiempos verbales en español y los valores del quechua y del aimara, que objetivamente existen. Los hablantes bilingües que hicieron emerger esas soluciones novedosas\footnote{Considero que la creación de las soluciones novedosas tuvo que darse en el pasado entre hablantes bilingües; posteriormente, esas soluciones fueron formando parte de la variedad estable que es el español andino y que se ha trasmitido de generación en generación llegando, incluso, a la población monolingüe, como se constata en la actualidad.} asumirían que esas similitudes son equivalentes, aunque objetivamente no lo sean del todo. No se trata por tanto de un calco sintáctico, de una mera copia, sino de una aproximación cognitiva que se plasma en una convergencia lingüística entre ambas lenguas.

En el caso del español andino ecuatoriano y colombiano, los más innovadores de las variedades analizadas, el hablante selecciona el perfecto simple para codificar experiencia directa o información sobre la que tiene cierto control, una información no evaluable. Este tiempo en español expresa un pasado cerrado, objetivo; y esa es la forma que el hablante elige del paradigma verbal pretérito simple/compuesto/pluscuamperfecto para codificar evidencia directa, control sobre la información transmitida. Cuando el hablante no está seguro de la información, porque es una información reportada con la que no puede comprometerse y quiere codificar un valor de evidencia indirecta, recurre al pretérito compuesto, un tiempo que en español ya expresaba evaluación, punto de vista del narrador o introducción de voces en el discurso; en definitiva, subjetividad. Así, a partir de las propias posibilidades que el español ofrece (precisión \textit{vs}. subjetividad), se produce un cambio lingüístico inducido por el contacto con el quechua en el que los valores de precisión o subjetividad de las formas de pasado evolucionan a valores evidenciales de experiencia, conocimiento o certeza de la información transmitida \textit{vs}. no experiencia vivida o conocimiento dudoso de la información transmitida por ser reportada. El pluscuamperfecto, que es un tiempo que expresa en español cierto distanciamiento temporal, adquiere valores de distanciamiento cognitivo codificando información dudosa, conjetural, etc.). Así, el hablante acerca su variedad de castellano al sistema de evidencialidad/validación del quechua aprovechando las estructuras del castellano y del quechua para introducir diferencias cognitivas que no tiene el castellano, pero sí el quechua. 

En el caso de Perú, el hablante bilingüe explotaría más los valores pragmáticos de relevancia discursiva que ya tenía el pretérito compuesto en otras variedades no andinas, abundando en los valores de relevancia expresiva que hoy conocemos. El pluscuamperfecto codificaría valores de evidencia indirecta y conjetura aprovechando el cierto distanciamiento temporal y cognitivo que tiene ya la forma verbal en español.

En cuanto al caso boliviano, dado que el pretérito simple está en desuso en el oral coloquial de los hablantes del corpus estudiado, el pluscuamperfecto es la forma verbal en la que recae la codificación de la evidencialidad indirecta, ya sea reportada, inferida o conjeturada; información de la que el hablante no está seguro en definitiva y de la que no puede responsabilizarse.

En lo que respecta a las vías de gramaticalización, ya se ha descrito que el pretérito perfecto, debido a que permite conectar dos eventos, implica una relación epistémica evaluativa, subjetiva, que puede derivar en una interpretación evidencial (\citealt{Aikhenvald2004,BybeeEtAl1994,Dik1997}), como muestra \figref{fig:palacios:1} adaptada de \citet[105]{BybeeEtAl1994}:


\begin{figure}
\begin{tikzpicture}
  \node(resultativo)[rounded corners, draw]{Resultativo};
  \node(inferencia)[rounded corners, draw,right=of resultativo,yshift=1cm,text width=4.5cm]{Inferencia desde resultados};
  \node(anterior)  [rounded corners, draw,right=of resultativo,yshift=-1cm,text width=4.5cm]{Anterior};
  \node(evidencia) [rounded corners, draw,right=of inferencia]{Evidencia indirecta};
  \node(perfectivo)[rounded corners, draw,right=of anterior]{Perfectivo};
  \draw[-latex](resultativo)--(inferencia);
  \draw[-latex](inferencia)--(evidencia);
  \draw[-latex](resultativo)--(anterior);
  \draw[-latex](anterior)--(perfectivo);
\end{tikzpicture}

\caption{\label{fig:palacios:1}Evolución del pretérito compuesto hacia valores evidenciales}
\end{figure}

Este sería el estadio de gramaticalización más avanzado del pretérito compuesto, que es el del español andino ecuatoriano y colombiano (etapa 2), a diferencia del peruano, cuya evolución menos avanzada seguiría en la etapa de relevancia subjetiva del resto de variedades del español (etapa 1), si bien en esta variedad andina se habrían disminuido las restricciones contextuales para que esta forma verbal sea empleada tanto en contextos no narrativos como narrativos, lo que conllevaría un incremento de la subjetividad que conducirá a mayores posibilidades de gestionar más satisfactoriamente la relevancia expresiva de ciertos eventos, como se muestra en la \figref{fig:palacios:2}:

\begin{figure}
\small
\begin{tabular}{ccccc}
Perfecto & → &    relevancia expresiva & → &   evidencialidad indirecta\\

 & &                      Etapa 1       & &                        Etapa 2\\
\end{tabular}
\caption{\label{fig:palacios:2} Evolución del perfecto en español}
\end{figure}

En cuanto al pluscuamperfecto, seguiría un camino de gramaticalización similar, ya que este tiempo verbal también relaciona dos eventualidades en el pasado, lo que conllevaría cierto distanciamiento del hablante en una relación epistémica semejante a la que hemos visto para el pretérito perfecto. A partir de esta interpretación epistémica, la forma verbal adoptaría interpretaciones de mayor distanciamiento que resultarían en lecturas de evidencialidad indirecta, como se muestra en la \figref{fig:palacios:3}:


\begin{figure}
\small
\scalebox{.9125}{
\begin{tabular}{ccccc}
Pluscuamperfecto & → &  distancia temporal y cognitiva & → &  evidencialidad indirecta\\
&&                                  Etapa 1          &&                                Etapa 2\\
\end{tabular}
\caption{\label{fig:palacios:3} Evolución del pluscuamperfecto en español}
}
\end{figure}

Estos cambios podrían explicarse a partir del concepto de subjetivación (\citealt{Traugott1982}; \citealt{TraugottDasher2002}) como itinerarios de gramaticalización secuenciales hacia funciones pragmático-discursivas. Se trata de aproximaciones congruentes del hablante bilingüe en origen entre las dos lenguas que maneja, debido a las similitudes que percibe y a las necesidades comunicativas (y quizá cognitivas si la evidencialidad se entiende como dominio cognitivo) que ambas tienen, una ampliación de las potencialidades que ya ofrece el castellano, un paso más en la evolución de los tiempos de pasado hacia valores más pragmáticos. Debido a la situación de contacto histórico intenso y continuado de estas áreas andinas, estos cambios son adoptados incluso por los hablantes monolingües, como ya he comentado.

\section{Para concluir}\label{sec:palacios:7}
\largerpage
En esta contribución se revisa el uso de los tiempos de pasado simple, compuesto y pluscuamperfecto en el español andino ecuatoriano y colombiano, peruano central y boliviano a partir de las descripciones realizadas por la bibliografía, y se pone de manifiesto que la situación de contacto histórico intenso en estas zonas ha dado lugar a soluciones diferentes. Dado este panorama, surgió la pregunta central de esta investigación: ¿cómo es posible que para las mismas formas de pasado emerjan soluciones diferentes en estas zonas andinas de bilingüismo histórico donde están implicadas las mismas lenguas de contacto? En este sentido, se ha intentado dar cuenta de la diversidad de soluciones documentadas en las diferentes áreas andinas de contacto lingüístico intenso teniendo en cuenta una dimensión cognitiva, esto es, a partir de las similitudes que el hablante bilingüe cree percibir entre las lenguas implicadas, dado que necesita incorporar ciertas distinciones importantes en su lengua materna, que se materializan en nuevas formas lingüísticas, para lograr una comunicación más exitosa. 

El dominio de la evidencialidad, como hemos visto en estas páginas, es clave para explicar el acercamiento pragmático que haría el hablante bilingüe cuando (re)crea soluciones convergentes para usar los pasados en su variedad andina de español. Hemos mostrado, igualmente, cómo la codificación creativa y dinámica de la evidencialidad en la lengua originaria trasciende el automatismo de las referencias a la fuente de información y permite diferentes estrategias pragmáticas para recrear el papel del narrador y negociar su relación con los hechos narrados (distancia/cercanía), marcar la relevancia de los hechos subjetivos o las voces en el discurso. En este juego creativo la percepción de similitudes con los valores de los tiempos de pasado del español posibilitará la convergencia de significados (re)creados. Se propone, por ello, que se trata de cambios indirectos inducidos por contacto donde la lengua originaria desempeña un papel esencial; cambios inmersos en un proceso de gramaticalización inducido por contacto, donde las soluciones creadas para las formas de pretérito perfecto compuesto en las distintas áreas corresponden a diferentes etapas evolutivas del mismo proceso de gramaticalización. Así, las variedades más innovadoras (español andino ecuatoriano y del sur colombiano) desarrollan soluciones más novedosas que parecen haber incorporado valores evidenciales en estas formas de pretérito compuesto (etapa 2, \figref{fig:palacios:2}), a diferencia de otras más conservadoras (andinas peruanas) que explotan mucho más los valores pragmáticos ya existentes en los tiempos de pasado del español (etapa 1, \figref{fig:palacios:2}). En el caso del pluscuamperfecto, el proceso de gramaticalización inducido por contacto tiene como consecuencia que la forma verbal adopte valores de mayor distanciamiento que evolucionan hacia lecturas de evidencialidad indirecta en las diferentes zonas de estudio.

Constatamos, así, que se puede ofrecer una explicación más satisfactoria si consideramos la evolución y reorganización del paradigma de las formas verbales en cada área estudiada y no solo focalizamos la explicación en cada tiempo tomado individualmente. 

Es interesante, también, comprobar cómo en estas áreas el hablante puede activar o no esas formas emergentes en función de sus necesidades comunicativas y cómo el nivel de instrucción también parece ser un poderoso disparador de uso de estas formas, ya que, como se ha visto, son los hablantes con menor nivel de instrucción los que parecen explotar más estos nuevos significados. En definitiva, la agencia y la creatividad de los hablantes en explotar las potencialidades de los sistemas implicados en el contacto son claves para explicar estos cambios lingüísticos inducidos por contacto, que se materializan en prácticas lingüísticas heterogéneas que sitúan al hablante en el centro de la investigación.  

\section*{Agradecimientos}
Esta investigación se desarrolla en el marco del proyecto \textit{COREC}: \textit{Corpus oral de referencia del español en contacto.} \textit{Fase I: lenguas minoritarias.} Referencia: PID 2019-105865GB-I00, financiado por el Ministerio de Economía y Competitividad, y coordinado por Azucena Palacios Alcaine y Sara Gómez Seibane.


\sloppy\printbibliography[heading=subbibliography,notkeyword=this]
\end{document}
