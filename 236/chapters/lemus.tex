\documentclass[output=paper]{langscibook}
\ChapterDOI{10.5281/zenodo.5643289}
\author{Aura Lemus Sarmiento\orcid{0000-0002-2912-7078}\affiliation{Université Paris IV} and Magdalena Lemus Serrano\orcid{0000-0001-8312-6466}\affiliation{Aix-Marseille Université}}
\title{Las prácticas lingüísticas de los hablantes de yukuna en Leticia (Amazonas, Colombia)}
\abstract{En este artículo estudiamos las estrategias discursivas de un grupo de hablantes yukuna del suroccidente colombiano cuya lengua principal es el yukuna y que usan el español en los intercambios sociales cotidianos. Siguiendo a \citet{Milroy1992} y a \citet{PalaciosAlcaine2019} nuestro estudio adopta como marco teórico el postulado según el cual los hablantes encuentran formas expresar sus necesidades comunicativas en su repertorio plurilingüístico, combinando los recursos de las lenguas que lo componen. Así, a través de un corpus oral en español y en yukuna, observaremos desde una perspectiva sincrónica las actitudes de los hablantes hacia ambas lenguas y las variaciones lexicosintácticas que resultan de la situación de contacto.
}
%

    \IfFileExists{../localcommands.tex}{
    \addbibresource{localbibliography.bib}
    \usepackage{langsci-optional}
\usepackage{langsci-gb4e}
\usepackage{langsci-lgr}

\usepackage{listings}
\lstset{basicstyle=\ttfamily,tabsize=2,breaklines=true}

%added by author
% \usepackage{tipa}
\usepackage{multirow}
\graphicspath{{figures/}}
\usepackage{langsci-branding}

    
\newcommand{\sent}{\enumsentence}
\newcommand{\sents}{\eenumsentence}
\let\citeasnoun\citet

\renewcommand{\lsCoverTitleFont}[1]{\sffamily\addfontfeatures{Scale=MatchUppercase}\fontsize{44pt}{16mm}\selectfont #1}
  
    %% hyphenation points for line breaks
%% Normally, automatic hyphenation in LaTeX is very good
%% If a word is mis-hyphenated, add it to this file
%%
%% add information to TeX file before \begin{document} with:
%% %% hyphenation points for line breaks
%% Normally, automatic hyphenation in LaTeX is very good
%% If a word is mis-hyphenated, add it to this file
%%
%% add information to TeX file before \begin{document} with:
%% %% hyphenation points for line breaks
%% Normally, automatic hyphenation in LaTeX is very good
%% If a word is mis-hyphenated, add it to this file
%%
%% add information to TeX file before \begin{document} with:
%% \include{localhyphenation}
\hyphenation{
affri-ca-te
affri-ca-tes
an-no-tated
com-ple-ments
com-po-si-tio-na-li-ty
non-com-po-si-tio-na-li-ty
Gon-zá-lez
out-side
Ri-chárd
se-man-tics
STREU-SLE
Tie-de-mann
}
\hyphenation{
affri-ca-te
affri-ca-tes
an-no-tated
com-ple-ments
com-po-si-tio-na-li-ty
non-com-po-si-tio-na-li-ty
Gon-zá-lez
out-side
Ri-chárd
se-man-tics
STREU-SLE
Tie-de-mann
}
\hyphenation{
affri-ca-te
affri-ca-tes
an-no-tated
com-ple-ments
com-po-si-tio-na-li-ty
non-com-po-si-tio-na-li-ty
Gon-zá-lez
out-side
Ri-chárd
se-man-tics
STREU-SLE
Tie-de-mann
}
    \togglepaper[1]%%chapternumber
    }{}


    \shorttitlerunninghead{Las prácticas lingüísticas de los hablantes de yukuna en Leticia}%%use this for an abridged title in the page headers

    \begin{document}
    \maketitle
\shorttitlerunninghead{Las prácticas lingüísticas de los hablantes de yukuna en Leticia}%%use this for an abridged title in the page headers

\section{Introducción}

A pesar del carácter impenetrable que reviste la Amazonía colombiana en el imaginario colectivo, se trata de una zona de intercambios y encuentros en múltiples aspectos. El habla del pueblo yukuna, su historia y su porvenir, dan fe de esta característica esencial del territorio Amazónico y de los grupos que lo habitan. 

El yukuna (ISO 693-3:ycn, Glottocode: yucu1253) es el idioma principal de los grupos étnicos yukuna y matapí, cuyo territorio tradicional abarca una docena de aldeas ribereñas a lo largo del río Mirití-Paraná en el sureste colombiano \citep{Fontaine2001}. Las entnias de habla yukuna conviven en este territorio con personas pertenecientes a otras comunidades como los miraña, muinane, witoto del bajo Caquetá, los tanimuka y letuama del alto Apaporis y, en menor proporción, los cabiyarí, cubeo, yaúna, makuna y barasano, provenientes del río Vaupés y desplazados a esta zona durante la época de las caucherías (finales del siglo XIX, principios del siglo XX) \citep{Eraso2015}. Esta diversidad étnica del Mirití-Paraná implica además una gran diversidad lingüística, ya que cada etnia tiene un idioma principal distinto.

La mayoría de los grupos étnicos de esta región en el Amazonas comparten una estructura social en que la identidad está fuertemente vinculada al idioma propio de cada etnia, de modo que, al nacer, a cada individuo se le asigna una identidad étnica y debe hablar su idioma correspondiente, que llamaremos de ahora en adelante idioma identitario. Otro aspecto fundamental de la estructura social de los grupos étnicos de esta región es su sistema de alianzas matrimoniales exogámicas, en donde cada cónyuge debe pertenecer a un grupo étnico distinto. En este sistema, las familias son siempre multiétnicas, en muchos casos multilingües, y la identidad etnolingüística de cada individuo es patrilineal, heredada por el padre. Sin embargo, aunque las personas tienen una lengua identitaria, en general dominan completamente la lengua identitaria de la madre y, en grados diferentes, otras lenguas indígenas de su entorno, además del español. Este sistema de alianzas matrimoniales da\label{lemus:quoteforintro} como resultado comunidades multilingües que comunican entre sí en una o varias de las lenguas que tienen en común.

Aunque es difícil saber con precisión cuántas personas de etnia yukuna y cuántos hablantes de yukuna hay actualmente, según Ethnologue (\citealt{EberhardEtAl2020}), la población yukuna-hablante sería de aproximadamente 770 personas \citep{Crevels2012}. Este reducido número clasifica por defecto al yukuna como una lengua en peligro de extinción; no obstante, en este territorio, el yukuna es la lengua más hablada \citep{Eraso2015} y aún hoy se sigue transmitiendo de generación en generación.

El yukuna, como las lenguas amazónicas en general, es una lengua de tradición oral. En las escuelas comunitarias, a pesar de que la comunicación en el aula de clase es en una lengua local, a los niños se les enseña a escribir en español, usando material educativo en español.\footnote{Aunque existe un alfabeto yukuna, propuesto por los misionarios del Instituto Lingüístico de Verano Junia y Stanley Schauer en los años 70, la población en general no conoce ni usa este alfabeto, ya que la escritura y la lectura no constituyen practicas comunes en la zona.} Por otra parte, a partir de tercero de primaria, la única opción de escolarización, exclusivamente en español, son los internados: instituciones educativas gestionadas por la iglesia católica y auspiciadas por el gobierno.\footnote{En 1935 y 1949 fueron fundados dos internados administrados por la iglesia católica en la región del Mirití-Paraná. En esta época comenzó una labor de desestructuración social de los pueblos indígenas, por la cual los niños eran sacados a la fuerza de sus hogares para ser escolarizados en español \citep{Eraso2015}; esto impidió la transmisión intergeneracional de los saberes tradicionales y obligó a muchas familias a abandonar sus comunidades para mudarse cerca de los internados y poder ver a sus hijos \citep{Fontaine2001}. Hoy en día, si el secuestro ya no forma parte de las prácticas educativas de los internados, estas instituciones siguen siendo la única opción de escolarización, símbolo de esperanza de un mejor nivel de vida, y la causa del desplazamiento de numerosas familias.} Estos factores reducen considerablemente el índice de vitalidad del idioma, siguiendo los criterios postulados por la \citet{UNESCO2003}.

A pesar de estos factores negativos y de la omnipresencia de la lengua nacional cada vez más indispensable, en las comunidades tradicionales la lengua se mantiene activa, hablada por niños, adultos y ancianos. Esta situación cambia con la emigración de personas pertenecientes a estos grupos étnicos hacia pueblos y zonas urbanas cercanas. Esta migración implica para los yukuna-hablantes provenientes del Mirití-Paraná un cambio drástico; la mayoría de la población está compuesta por mestizos monolingües en español que viven entre indígenas pertenecientes a otras etnias que no son necesariamente aquellas con las que convivían en su territorio tradicional. Todos estos factores conllevan a que el español se esté convirtiendo en la lengua de comunicación principal por fuera de la familia. 

En efecto, en las últimas décadas, muchos indígenas del Mirití-Paraná han abandonado su territorio tradicional por razones económicas. Así, hoy en día encontramos a personas de etnia yukuna y matapí en el corregimiento de La Pedrera y los resguardos de sus alrededores (Comeyafú y Camaritagua), en el municipio de Taraira en el alto Apaporis, así como en la ciudad de Leticia, capital del departamento del Amazonas colombiano. Es este último contexto el que nos sirve de marco geográfico a nuestro estudio.  

En el presente estudio presentaremos el marco teórico, el contexto sociolingüístico de nuestro análisis y la metodología de recolección del corpus, para luego focalizarnos en las prácticas lingüísticas de los hablantes de yukuna en Leticia y en los cambios directos e indirectos inducidos por la situación de contacto (\citealt{PalaciosAlcaine2007compatibles}).

\section{Marco teórico}


El análisis que proponemos a continuación tendrá como marco teórico el modelo de cambio lingüístico inducido por contacto propuesto por \citet{PalaciosAlcaine2007compatibles,PalaciosAlcaine2010,PalaciosAlcaine2019}.

Se trata de un marco teórico centrado en el hablante, puesto que considera, siguiendo a \citet{Milroy1992}, que “no es la lengua la que cambia, sino que son los hablantes con sus recursos en permanente emergencia quienes […] encuentran formas diferentes de expresar sus necesidades comunicativas en una u otra lengua, o combinando los recursos lingüísticos de ambas” (\citealt{PalaciosAlcaine2019}: 236). Este modelo de cambio propone considerar la situación de contacto como un continuum lingüístico en el que se sitúan los hablantes con distinto grado de dominio de ambas lenguas; bilingües simétricos o consecutivos, cuya lengua dominante será la lengua indígena o el español.  \citet[262]{PalaciosAlcaine2007compatibles} propone una visión dinámica del cambio inducido por contacto que plantea como bidireccional, en la medida en que las variaciones se producen en ambas lenguas.

Según este marco teórico, Palacios Alcaine distingue dos tipos de cambios. Por una parte, los cambios directos inducidos por contacto que consisten en la incorporación de algún tipo de material léxico o estructural de una lengua a otra: “la incorporación de elementos de la lengua fuente en la lengua objeto”\citet[262]{PalaciosAlcaine2007compatibles}. Estos cambios permiten resolver carencias comunicativas en la lengua de contacto y reflejan procesos cognitivos en su lengua materna o en su lengua fuente. Por otra parte, los cambios indirectos, que no implican importación de material ajeno, sino que “aprovechan la propia evolución interna de la lengua B para hacer aflorar estrategias gramaticales cuya funcionalidad comunicativa obedece a procesos cognitivos de la lengua A” a través de un mecanismo de convergencia lingüística. Este concepto, postulado en un primer momento por  \citet{GumperzWilson1971} abarca, según \citet[263]{PalaciosAlcaine2007compatibles}, los “procesos paralelos que desembocarán en el desarrollo de estructuras gramaticales comunes en las lenguas en contacto”.

Antes de analizar cómo los hablantes recurren a estructuras pertenecientes a las lenguas que forman parte de su repertorio lingüístico alternándolas, mezclándolas e ilustrando su riqueza lingüística, presentamos el corpus de trabajo y el perfil sociolingüístico de los participantes que contribuyeron a este estudio.

\section{Corpus y hablantes}

El presente estudio se basa en diferentes tipos de datos de primera mano recopilados durante varios trabajos de campo. Los datos en yukuna fueron grabados en diversos sitios, pero las entrevistas en español fueron realizadas en Leticia, una ciudad hispanohablante con una gran proporción de población indígena. Para contextualizar el corpus, presentamos a continuación un panorama sociolingüístico de Leticia y luego describimos el proceso de recolección de datos.

\subsection{El yukuna en contexto urbano}

Leticia es un municipio localizado en el extremo sur del país y es la capital del departamento del Amazonas colombiano. El municipio de Leticia, situado junto a Tabatinga (Brasil) y Santa Rosa de Yavarí (Perú), está situado en una zona conocida como “Tres Fronteras”, donde, irónicamente, las fronteras políticas resultan casi imaginarias y no impiden la libre circulación de las personas entre los tres países.\footnote{Sobre el contexto sociolingüístico en el municipio de Leticia, ver  \citet{FaguaRincón2000},  \citet{DeCamposBarbosa2008} y  \citet{SánchezMoreano2019}.}  En efecto, Leticia constituye una zona de increíble diversidad étnica y lingüística, como ya lo reconocía Manuel \citet[32]{Alvar1977} el siglo pasado en sus estudios lingüísticos sobre  Leticia:

\begin{quote}
Leticia es un mundo abigarrado en el que convergen gentes de muy variopinta condición: a los colombianos de variados departamentos, se unen hombres que aquí estaban antes de la cesión o venidos tardíamente; los brasileños que comercian por las aguas del Amazonas o que, asentados, constituyen un fermento imprescindible para saber cómo se fraguó la realidad que hoy vivimos; indígenas trasplantados en épocas de muy dura explotación.
\end{quote}

Según el Departamento Administrativo Nacional de Estadística (DANE) (\citeyear{DANE2005}), el total de la población del municipio de Leticia sumaba 37 832 habitantes, entre los cuales el 25,8\% se autorreconoce como indígena. Aunque desconocemos la distribución por grupo étnico de la población indígena del municipio, sabemos que los grupos más numerosos son los tikuna y los witoto.  Tampoco tenemos datos recientes sobre el número de personas de habla yukuna en este municipio, pero según el censo de 1988, había alrededor de 60 individuos yukuna y matapí.

Al llegar a Leticia, los indígenas del Mirití-Paraná se ven obligados a abandonar su modo de vida tradicional por completo. Las familias ya no viven en malocas comunitarias, sino que arriendan pequeñas habitaciones en la ciudad. Los niños son inscritos en colegios colombianos donde, si bien hay muchos niños indígenas, la lengua común a todos es el español. En este contexto urbano, el yukuna pierde casi por completo la vitalidad que lo caracteriza en su territorio tradicional y pasa a un segundo plano como lengua relegada al contexto familiar. Como lo decía \citet[32]{Alvar1977} “[En Leticia]~Los indígenas están muy dispersos y dentro de la ciudad, se integran como colombianos”, señalando así una asimilación lingüística y cultural fuerte.

Así, en los últimos años, ha aparecido la primera generación de niños yukuna y matapí que no son hablantes activos de la lengua, dando lugar a familias cuyos miembros tienen perfiles sociolingüísticos diferentes; la generación de los abuelos, en general con poco o ningún dominio del español, la generación de los padres, con un mayor dominio del español, pero cuya lengua principal sigue siendo su lengua identitaria y, finalmente, los niños nacidos en Leticia cuya lengua principal es el español. 

La migración desde el Mirití-Paraná hacia Leticia no es colectiva, sino que se limita en general a un núcleo familiar y, por ende, no se puede hablar de una comunidad yukuna-hablante homogénea en la zona urbana.  En estas condiciones es altamente probable que la generación afectada por el proceso de substitución lingüística, cuya legua principal es el español, participe cada vez menos en la vida cultural tradicional hasta el punto en que ya no pueda transmitirle a la generación siguiente ni su lengua ni su cultura. 

\subsection{Recolección y anotación de datos}

Nuestro corpus contiene en su mayoría grabaciones en yukuna y, en menor proporción, grabaciones en español. El corpus de grabaciones en yukuna contiene narraciones de la literatura oral yukuna y matapí, relatos autobiográficos, recetas de cocina, cantos y conversaciones. Todas las grabaciones fueron hechas en presencia de la investigadora. El protocolo adoptado consistía en escoger una temática en particular con la que el hablante se sintiera a gusto y luego darles la palabra para que estos se expresaran. Puesto que las temáticas son previamente escogidas y las grabaciones hechas en presencia de una investigadora externa a la comunidad que no domina el idioma, el contexto de recolección de datos es altamente artificial. La decisión de adoptar este protocolo de recolección de datos se debe principalmente a que el yukuna no cuenta con muchos estudios lingüísticos profundizados. Así, el objetivo de la investigación era recoger datos suficientes para describir los aspectos básicos de la morfosintaxis del idioma.  Una parte de estas grabaciones (aproximadamente 5 horas) ha sido transcrita alfabéticamente en yukuna, traducida al español en ELAN y luego exportada a Flex para la segmentación e interlinearización.

Además de las grabaciones en yukuna, el corpus contiene una gran cantidad de datos en español recogidos con el objetivo específico de hacer un estudio preliminar del uso del español en los yukuna-hablantes. Dentro de este marco, grabamos narraciones de historias tradicionales y entrevistas cortas con tres hablantes, con el objetivo de constatar el manejo de ciertas estructuras. Estos datos fueron, una vez más, parcialmente transcritos en ELAN. 

Una vez más, debido al contexto rígido de recolección de datos, se desprende cierto grado de artificialidad que restringe el alcance de nuestra investigación. En efecto, la falta de espontaneidad característica de las sesiones de elicitación se traduce en cierto esfuerzo de normatividad por parte de los hablantes. Cabe resaltar que la actitud normativa de los hablantes durante las sesiones de grabación no fue un impedimento para que estos se expresaran con fluidez durante largos lapsos de tiempo sin interrupción. Gracias a esta aptitud de los participantes pudimos recopilar suficientes datos para este estudio piloto.

\subsection{Perfil sociolingüístico de los informantes}


De manera general, los hablantes que participaron al estudio declaran haber aprendido el español en su casa o antes de los 10 años. Esta edad puede relacionarse con la edad de llegada al internado, el contexto principal que los hablantes asocian a su aprendizaje del español. Los hablantes identifican el yukuna como idioma principal de comunicación con miembros de la familia y de la comunidad yukuna-hablante. Sin embargo, al preguntarles sobre el uso del español, reconocen que el español también forma parte íntegra de las interacciones discursivas familiares y declaran recurrir frecuentemente a una alternancia códigos en una misma interacción comunicativa, estrategia que ellos mismos califican como “hablar mezclado”. Finalmente, identifican el español como la única lengua que utilizan en sus relaciones con hablantes extracomunitarios, ya sean indígenas o no.

El contexto de vida de los hablantes determina la relación que tienen con las lenguas que hablan y es por ende evolutivo; según el lugar de residencia y su configuración social, un hablante tendrá como lengua principal de comunicación el yukuna o el español en diferentes etapas de su vida. En el caso que nos incumbe, la ausencia de oportunidades educativas y laborales en los territorios tradicionales provoca una situación de emigración cíclica, en que los hablantes van y vienen entre Leticia, el Mirití-Paraná y pueblos vecinos en diferentes etapas de su vida. Por ejemplo, los hablantes con los que trabajamos por lo general pasaron sus primeros años de vida en su comunidad natal, hicieron su escolarización en el internado, se mudaron a la Pedrera, el pueblo más cercano, después a Leticia, la ciudad más grande de la región y, en algunos casos, regresaron a la Pedrera o al Mirití-Paraná. Cada uno de estos contextos se caracteriza por condiciones sociolingüísticas distintas que influyen en la manera en la que el hablante utiliza los recursos de su repertorio lingüístico. A continuación, nos enfocaremos en las producciones discursivas de algunos de estos hablantes.

\section{ Prácticas discursivas de hablantes bilingües yukuna-español}


A continuación analizaremos los datos recopilados desde diferentes niveles y perspectivas, con el fin de dar una visión detallada de la producción discursiva bilingüe de los hablantes. Presentaremos en un primer lugar la alternancia de códigos, para luego establecer los cambios directos e indirectos en el continuum bilingüe de los hablantes. Cabe recalcar que al describir los idiolectos de los participantes entrevistados no pretendemos hacer generalizaciones sobre la difusión y convencionalización de cambios a nivel de la comunidad, sino dar muestras de las prácticas comunicativas de los hablantes y sentar las bases para estudios posteriores que sí lo permitirían.

\subsection{Alternancia de códigos}


Entendemos por alternancia de códigos la estrategia inherente a los comportamientos comunicativos de los hablantes bilingües que consiste en alternar elementos pertenecientes a los repertorios lingüísticos de los hablantes en una misma interacción comunicativa.  En nuestro estudio, las conversaciones grabadas en español en presencia de la investigadora presentan una baja incidencia de casos de alternancia de código. Esta actitud normativa contrasta radicalmente con las estrategias discursivas empleadas por los yukuna-hablantes entre sí. El caso más común es el de las llamadas telefónicas que recibían los participantes mientras estaban siendo grabados durante una sesión de trabajo. Este tipo de datos es muy revelador, ya que constituyen un caso de conversación espontánea, no inducida por la investigadora ni influenciada por la presencia de esta. 

Con los hablantes extracomunitarios no indígenas, así como lo demuestran las grabaciones hechas únicamente en presencia de la investigadora, los hablantes producían un español desprovisto de cualquier forma yukuna. Entre yukuna-hablantes los hablantes utilizaban con mayor libertad sus repertorios bilingües, conservando sin embargo una mayoría de formas en yukuna. Ilustramos este escenario con un fragmento extraído de una conversación telefónica entre dos mujeres yukuna-hablantes en el ejemplo \REF{ex:lemus:1}:

\ea\label{ex:lemus:1}
{español/yukuna}\\
\ea
\gll Entonces ru=ímá…             ¿méké iná kema-ká ?…\\
\textsc{esp} \textsc{3sg.f=}decir cómo \textsc{gnr.pro}  {decir}\textsc{-nmlz}\\
\glt `Entonces ella me dijo... ¿cómo es que uno dice?'

\ex
\gll ru=le‘jé   cuñada tá     i‘má-yó ri=é,\\
\textsc{3sg.f=poss}   \textsc{esp} \textsc{emph}  \textsc{cop-f}   \textsc{3sg.nf={en}}\\
\glt ``la cuñada de ella estaba ahí''

\ex
\gll ella me dijo: “kája ru=jácho‘-cha=o”\\
\textsc{esp} \textsc{esp}  \textsc{esp} \textsc{ya} \textsc{3sg.f={salir}-pst=mid}\\
\glt `ella me dijo: “ya ella salió.”

\ex
\gll entonces nu=ímá ru=jló: “¿é maáré kélé pajlúwája iphí-cha-ri aquí ?”\\
\textsc{esp} \textsc{1sg}={decir} \textsc{3sg.f}=para \textsc{q} aquí \textsc{dem} uno llegar-\textsc{pst}-\textsc{nf}   \textsc{esp}\\
\glt `Entonces yo le dije: ``¿aquí está uno que llegó aquí''{}' [ycn0504,1]
\z
\z

Inversamente, con los hablantes pasivos de yukuna, las proporciones cambian y los hablantes pasan a una producción con mayoría de formas en español, pero igualmente caracterizada por la alternancia con el yukuna como en el ejemplo \REF{ex:lemus:2}:

\ea\label{ex:lemus:2}
{español/yukuna}\\
\gll Yo no sé si Bebé comió nu=warúwi’-chaje kujnú=ñáni...\\
     \textsc{esp} \textsc{esp} \textsc{esp} \textsc{esp}  \textsc{esp} \textsc{esp} \textsc{1sg=}comprar\textsc{-nmlz} casabe\textsc{=dim}\\
\glt `Yo no sé si Bebé comió el casabito que yo le compré…' [ycn0560]
\z

Los casos similares al ejemplo anterior son escasos en todo nuestro corpus, pero nuestra experiencia de campo corrobora que esta configuración es característica de los intercambios entre hablantes activos y hablantes pasivos –– en particular entre padres nacidos en el Miriti-Paraná y cuyo idioma principal es el yukuna, e hijos que nacieron o han vivido la mayor parte de sus vidas por fuera del Miriti-Paraná y cuyo idioma principal es el español.

Estos ejemplos demuestran cómo entre hablantes bilingües pertenecientes a la comunidad se alternan estructuras de las lenguas que conforman sus repertorios lingüísticos.  Ahora bien, la alternancia de códigos constituye una estrategia comunicativa identitaria y cultural de los hablantes bilingües y no evidencia cambios en los sistemas de los hablantes. Los siguientes ejemplos sí lo hacen; ilustran cómo la interinfluencia lingüística entre el español y el yukuna se manifiesta a través de cambios directos e indirectos en ambas lenguas.  

\largerpage
\subsection{Cambios lingüísticos directos}


Cabe recordar que \citet[22]{PalaciosAlcaine2011} entiende los cambios directos como “aquellos en los que existe importación de material ajeno a la lengua, ya sea este léxico o funcional, ya sea importación de patrones o estructuras”.

El léxico, considerado como la categoría más susceptible de ser importada en situación de contacto, es la que muestra los primeros síntomas de emancipación con respecto a la norma.\footnote{Hacemos aquí referencia a la escala de préstamos de    \citet{ThomasonKaufman1988}.} Así, al comenzar este trabajo esperábamos encontrar una gran cantidad de importaciones léxicas en el habla de los hablantes entrevistados.  No fue el caso. En nuestro corpus, el español de los hablantes está casi desprovisto de palabras del repertorio léxico yukuna. Esta situación podría explicarse por una actitud normativista consciente o inconsciente por parte de los hablantes, provocada tanto por el tipo de datos recopilados (narración de relatos míticos) como por la situación comunicativa formal en presencia de un participante ajeno a la comunidad. Ahora bien, esta ausencia de palabras del yukuna en nuestro corpus parece confirmar la teoría avanzada por \citet[937]{EppsMichael2017} sobre la actitud normativista de los grupos étnicos del Amazonas. En efecto, se trata aquí de una situación característica del Amazonas, donde la lengua reviste el papel de símbolo identitario, lo cual tiende a limitar el uso de préstamos léxicos incluso en situaciones de intenso plurilingüismo, con el fin de marcar la pertenencia a una comunidad:

\begin{quote}
Language plays a recurrent role as an emblem of difference in these zones, and local ideologies of language tend to strictly constrain the mixing of codes, even where frequent interaction among groups fosters intensive multilingualism \citep{Hill1996}. We widely encounter long-term language maintenance, limited code-switching, and low levels of lexical borrowing, often buttressed by explicit articulations of the quality and importance of linguistic difference.
\citet[937]{EppsMichael2017}\footnote{El lenguaje juega un rol recurrente como emblema de la diferencia en estas zonas, y las ideologías lingüísticas locales tienden a restringir la mezcla de códigos, aun cuando la interacción frecuente entre grupos conlleva a casos de multilingüísmo intensivo \citep{Hill1996}. Encontramos comúnmente casos de mantenimiento de lenguas en contacto, una presencia limitada de code-switching, bajos niveles de prestamos léxicos, muy  a menudo acompañados de afirmaciones explícitas sobre la calidad e importancia de la diferencia lingüística.}\end{quote}

Esta dinámica ilustra una gran capacidad de control cognitivo por parte de los hablantes para separar los repertorios léxicos de cada sistema, una capacidad ligada al contexto de plurilingüismo en que viven los hablantes desde su nacimiento y al papel de diferenciación identitaria que reviste la lengua.  Sin embargo, la situación descrita por Epps y Michael en el Vaupés no es idéntica a la del Mirití Paraná. Para los yukuna la lengua no es el único marcador identitario, puesto que hay etnias yukuna-hablantes que conservan su identidad étnica, como los matapí.  Sin embargo, los yukuna confieren a la lengua un gran valor de pertenencia cultural, afirmando a menudo que la lengua le pertenece al clan originario kamejeya; de hecho, en yukuna, la lengua se llama “lengua de los kamejeyas”. Así, puesto que la lengua reviste un gran valor de pertenencia cultural, existiría en los yukuna una práctica normativista integrada desde la infancia. Esta actitud hacia su lengua y hacia las lenguas habladas en la comunidad explicaría un esfuerzo normativista por parte de los hablantes. En efecto, aunque no haya una tradición escrita o estandardización del yukuna, los hablantes tienen una fuerte ideología lingüística normativa, como pudimos confirmar durante las sesiones de traducción y elicitación.\footnote{Los hablantes que participaron en las sesiones de elicitación y de traducción no dudan en afirmar que una palabra está usada de manera incorrecta o que pertenece a otra lengua.}

Sin embargo, cabe matizar esta generalización señalando que, en contextos comunicativos menos normativos tanto en español como en yukuna, los hablantes sí recurren de manera natural a estructuras lingüísticas asociadas a otras lenguas, haciendo muestra de su competencia plurilingüe. 

Así, aunque los hablantes evitan el uso de palabras yukuna en español, sí incluyen elementos lexicales provenientes de otras lenguas indígenas: 

\ea
{español}\\
De ahí pues bajamos hasta ahí para que él toma con \textbf{caguana.} [ycn0200,15]
\z

\ea
{español}\\
Bueno, como ella se fue a la \textbf{chagra} y llegó y de ahí, él fue y él dijo de aquí yo miré que ella sacó ese totumo de, totumada de \textbf{manicuera.} {[ycn0190, 112]}
\z

\ea
{español}\\
 Como yo le dije, como nosotros los indígenas vivimos de \textbf{casabe}, entonces ya entonces hacer \textbf{chagra} para comer \textbf{casabe} [ycn0200,10]
\z

\ea\label{ex:lemus:6}
{español}\\
 Y pues para el otro año pensamos hacer una \textbf{maloquita} ahí {[ycn0200, 16]}.
\z

La presencia de estas palabras en el español de los hablantes que participaron en nuestro estudio ilustra el carácter convergente de la región Amazónica como zona de encuentro y de diálogo entre diferentes etnias. En efecto, estas palabras tienen su origen en otras lenguas indígenas como el taíno (\textit{casabe} 'tortilla de yuca'), el quechua (\textit{chagra} 'terreno de cultivo') o el tupí (\textit{manicuera}'bebida de yuca brava'), mientras que el \textit{casabe} y la \textit{maloka} ('casa comunitaria') son palabras propias del universo cultural amazónico. Ahora bien, es posible que los hablantes no sientan estas palabras como préstamos de lenguas indígenas. La palabra \textit{maloka} es de amplio uso en las variedades diatópicas colombianas y la extensión dialéctica de la palabra \textit{chagra/chacra} va desde los territorios andinos hasta el Cono Sur.

Este repertorio léxico de los hablantes refleja realidades sociológicas específicas de la región, en el que las mujeres yukuna se encargan del cultivo de la yuca en la \textit{chagra}, con la cual preparan el \textit{casabe}, la \textit{manicuera} y el \textit{caguana} ('colada de yuca brava') en la \textit{maloka}. Así, estos préstamos remiten a realidades propias de los hablantes, revisten características identitarias y suplen carencias léxicas en la lengua española. Esta evidencia, aunque escasa, apuntaría, por su carácter plurilingüe, a la existencia de un español amazónico aún por explorar en los estudios dialectológicos. 

Por otra parte, no es raro encontrar inserciones léxicas del español en conversaciones en yukuna, entre yukuna-hablantes. Estos utilizan libremente no sólo palabras provenientes del español o del portugués, adaptadas a los parámetros morfológicos de la lengua como \textit{kamichá} (camisa) y \textit{paratú} (plato, pt. \textit{prato}), sino también préstamos más recientes. Entre los préstamos del español más comunes en yukuna se encuentran sobre todo palabras vinculadas al mundo occidental no-indígena, como los números, los meses del año y los días de la semana, así como actividades vinculadas a la vida en la ciudad. Los hablantes usan casi sistemáticamente los números en español a partir de cuatro sin importar el tipo de contexto (normativo o espontáneo) como se ve en el ejemplo \REF{ex:lemus:7}, extraído de un relato autobiográfico. El uso de palabras en español para referirse a actividades urbanas es claramente evitado en contextos normativos, pero se observa claramente en otros contextos como vemos en el ejemplo \REF{ex:lemus:8}:

\ea\label{ex:lemus:7}
{yukuna/español}\\
\gll É      nu=yurí=o                       Leticia é jápá-kana              nakú quince jarechí kétána.\\
     luego \textsc{1sg}=quedar=\textsc{mid} Leticia en trabajar-\textsc{nmlz} sobre \textsc{esp} año durante\\
\glt `Luego me quedé en Leticia trabajando quince años.' [ycn0018,8]
\z

\newpage
\ea\label{ex:lemus:8}
{español/yukuna}\\
\gll Porque   las dos ri=la’-jé tâ viajá Bogotá éj\'{õ}.\\
\textsc{esp}  \textsc{esp} \textsc{esp}  \textsc{3sg.nf=}hacer-\textsc{fut} \textsc{emph}    \textsc{esp}  Bogotá   hacia\\
\glt `Porque a las dos él va a viajar a Bogotá.' [ycn0504,9]
\z

Este ejemplo, sacado de una conversación telefónica entre dos mujeres cuya lengua principal es el yukuna, ilustra la inserción de la palabra “viajar” en la cláusula yukuna como forma dependiente del verbo \textit{la’kajé} 'hacer'. Ahora bien, este verbo en yukuna nunca tiene complementos verbales (verbos nominalizados o verbos no finitos).  Estaríamos ante la incorporación de una estructura ajena al yukuna, que refleja el transvase de una nueva conceptualización del verbo \textit{la’kajé} y que abre nuevas posibilidades sintácticas en esta lengua.  Cabe además recalcar que en el ejemplo precedente el hablante usa una conjunción en español, estrategia que los hablantes producen con espontaneidad en el corpus yukuna (\ref{ex:lemus:9}--\ref{ex:lemus:11}):

\ea\label{ex:lemus:9}
{español/yukuna}\\
\gll \textbf{Entonces} ru=ímá: “Unká, Valeria   unká   i’ma-lá”.\\
        \textsc{esp} \textsc{3sg.f=}decir \textsc{neg}  Valeria \textsc{neg} \textsc{cop-neg}\\
\glt `Entonces ella me dijo: “No, Valeria no está.”' [ycn0504,1]
\z

\ea\label{ex:lemus:10}
{yukuna/español}\\
\gll unká méla‘jé ná ata‘-lá-cha ri=nakiyá \textbf{porque}\\
     \textsc{neg} \textsc{indf.manner} \textsc{indf.pro}  robar-\textsc{neg-pst} 3\textsc{sg.nf=}de \textsc{esp}\\
\glt `Nadie podía robarle a él porque'
\z

\ea
\gll kája ri=we’pí   mékelé-ka   ri=yuí-ka\\
     ya \textsc{3sg.nf}=saber cuanto-\textsc{conj} \textsc{3sg.nf=}dejar-\textsc{nmlz}\\
\glt `ya él sabía cuántas había dejado.' [ycn0108,136]
\z

\ea\label{ex:lemus:11}
{yukuna/español}\\
\gll Ré ri=i’mi-chá   maáré, \textbf{pero}   unká   ri=kamáta-la-je=o maáré.\\
entonces \textsc{3sg.nf=cop-pst} aquí \textsc{esp} \textsc{neg} \textsc{3sg.nf=}dormir-\textsc{neg-fut=mid} aquí\\
\glt `Él estuvo aquí pero él no va a dormir aquí.' [ycn0504,3]
\z

Como podemos ver, estos cambios directos son bidireccionales e intervienen tanto en las producciones en yukuna como en español. También ilustran la capacidad de los hablantes para lograr una mayor eficiencia comunicativa y suplir carencias lexico-pragmáticas, recurriendo a la riqueza de los repertorios lingüísticos que poseen. Ahora bien, si los casos que acabamos de estudiar evidencian la importación de material de una lengua en otra, el intenso contacto lingüístico entre el español y el yukuna también ha implicado la reactivación de estructuras preexistentes en español por influencia de una estructura paralela en yukuna. Estos cambios reflejan así su influencia indirecta.  



\subsection{Cambios lingüísticos indirectos}



Hablaremos a continuación de varias situaciones que evidencian un proceso de convergencia implicando el desarrollo de estructuras gramaticales comunes en ambas lenguas.


\subsubsection{(de/hasta) ahí}


Se destaca el frecuente uso del adverbio \textit{ahí} para describir secuencias de acciones, que recuerda a la coordinación yukuna con \textit{é} (luego). Veamos los siguientes ejemplos:

\ea\label{ex:lemus:12}
{\label{bkm:Ref484433697}español}\\
Como recién desarrollada que ellos dicen, que uno es el primer periodo que le baja, \textbf{ahí} ya le rezan agua, le rezan casabe, todo lo que es de comer, para tocar candela, para hacer yuca, a eso todo le hizo a ella, en cambio a mí no porque yo me desarrollé acá en Nazareth. {[ycn0190, 184}]\\
\z

\ea\label{ex:lemus:13}
{español}\\
\label{bkm:Ref484433706}Tuve como dos años con él, demoré dos años con él, \textbf{ahí} yo tuve a Aristides. {[ycn0190, 194}]\\
\z
La narración de acciones secuenciales en yukuna se realiza por medio de la conjunción de coordinación \textit{é} 'luego', proveniente de la posposición \textit{é} prefijada opcionalmente por la marca de posesión \textit{ri=} 3\textsc{sg.nf}, usada comúnmente ya sea en posición inicial como conjunción \REF{ex:lemus:14} o como adverbio demostrativo espacio-temporal equivalente a \textit{ahí} \REF{ex:lemus:15}.

\newpage
\ea\label{ex:lemus:14}
{yukuna/español}\\
\gll É  nu=yurí=o Leticia é   jápá-kaje   nakú   quince jarechí  kétána \\
luego 1\textsc{sg}=quedarse=\textsc{mid} \textsc{esp} en trabajar-\textsc{nmlz} {en} \textsc{esp} año    durante\\
\glt 'Ahí me quedé en Leticia trabajando 15 años.' [ycn0018,8]
\z

\ea\label{ex:lemus:15}
{yukuna}\\
\gll É na=i’mi-chá ri=é\\
     luego \textsc{3pl}-vivir-\textsc{pst}  \textsc{3sg.nf}=en\\
\glt 'Luego ellos vivieron ahí (lit. En eso).' [ycn0058,33]
\z

Se trata aquí de un uso de alta frecuencia en nuestro corpus que, a pesar de la clara influencia del yukuna, no podemos atribuir completamente a una supuesta transferencia directa. La Real Academia Española reconoce la existencia de estructuras parecidas en el habla americana.  Para los académicos el adverbio \textit{ahí} estaría desemantizado en algunos usos lexicalizados propios de muchos países americanos. Se trata, según ellos, de un uso expletivo o cuasiexpletivo en~“\textbf{ahí} nos vemos”, “\textbf{ahí} me llamas cuando llegues”, “\textbf{ahí} te das cuenta de lo que pasa” (\citealt{RAE2010}, 340).  Nuestros ejemplos son asimilables a estos últimos citados por la RAE.  Sin embargo, analizando discursivamente tanto los ejemplos de la RAE como los de nuestro corpus, el adverbio \textit{ahí,} lejos de estar desprovisto de contenido semántico, parece marcar la continuidad de una acción con respecto a otra (explícita o implícita): \textit{llegar $\to$ llamar}; \textit{primera menstruación}$\to$ \textit{ceremonia de rezos} (\ref{bkm:Ref484433697}) ; \textit{2 años de matrimonio} $\to$ \textit{hijo} \REF{ex:lemus:13}.  Se trata aquí de una relación de continuidad y no de consecuencia.

Ahora bien, el uso del adverbio demostrativo \textit{ahí} con un valor deíctico temporal no remplaza su uso con un valor espacial, como lo atestan los siguientes ejemplos (\ref{ex:lemus:16} y \ref{ex:lemus:17}):

\ea\label{ex:lemus:16}
{español}\\
Pisó la leña, pues \textbf{ahí} cayó y \textbf{ahí} estaba el muerto. {[ycn0190, 138]}\\
\z

\ea\label{ex:lemus:17}
{español}\\
\textbf{Ahí} fui, hice quinto, porque ese tiempo no había quinto \textbf{ahí} en el, en Mirití. {[ycn0190, 175]}\\
\z

Por otra parte, en nuestro corpus constatamos a menudo el uso del adverbio \textit{ahí} acompañado por el adverbio \textit{ya}, usado a menudo para secuencias de acontecimientos pasados: 

\ea\label{ex:lemus:18}
{español}\\
Y se se fue y \textbf{ya} se metió al monte y \textbf{ya} se quedó, \textbf{ya} se transformó en venado \textbf{ahí.} {[ycn0190, 064]}\\
\z

\ea\label{ex:lemus:19}
{español}\\
 \textbf{Ahí} \textbf{ya} no, \textbf{ahí} fue que \textbf{ya} peor ella, \textbf{ya} le cogió rabia a los hijos, pues. {[ycn0190, 144]}\\
 \z

Se trataría aquí de un cambio indirecto inducido por el contacto con el yukuna, en el cual narración de secuencias se hace por medio de \textit{é}, usado como conjunción de coordinación seguida frecuentemente del adverbio \textit{kája}~'ya', en lo que podríamos asimilar a una locución conjuntiva:

\ea\label{ex:lemus:20}
{yukuna}\\
\gll Ré \textbf{kája},   ri=é i‘ma-ká-ño   jló   iná   kémá:\\
     entonces   ya \textsc{3sg.nf=}en \textsc{cop-nmlz-pl} para  \textsc{gnr}.\textsc{pro} decir\\
\glt `Entonces ya a los que están ahí uno dice:'
\z

\ea
\gll “i=ajñá   waláko jakú.”\\
     2\textsc{pl}=comer tucupi dentro\\
\glt `“coman tucupí”'. [ycn0059,6]
\z

Esta estructura también aparece reflejada en nuestro corpus español acompañada por la preposición \textit{de}:

\ea
{español}\\
Y \textbf{de} \textbf{ahí} \textbf{ya}, ella \textbf{ya} quedó embarazada de ese, ese niño que se convirtió en venado. {[ycn0190, 145]}\\
\z

\ea
{español}\\
Ya mi hermano limpió un pedazo, \textbf{de} \textbf{ahí} falta cortar más, rozar más, por ahí una hectárea eso ya lo van a tumbar, \textbf{de} \textbf{ahí} \textbf{ya} cuando va a haber verano eso se le va a quemar \textbf{de} \textbf{ahí} \textbf{ya}, cuando \textbf{ya} está bien quemado \textbf{ya}, \textbf{ahí} \textbf{ya} van a limpiar, donde que va a ser la maloca, \textbf{ahí} van destroncar todos los palos que están ahí, \textbf{de} \textbf{ahí} pues \textbf{ya} Fermín va a hacer su maloquita. {[ycn0201, 023]}\\
\z

Se trata aquí de estructuras que, a pesar de cierta impresión de extrañeza que puedan provocar en un interlocutor acostumbrado a contextos más normativos, no son ajenas a otras variedades del español, sobre todo en registros orales:

\ea
{español}\\
“Sí, y eso me levantó mucho la moral y al día siguiente amanecí jugando bien y \textbf{de} \textbf{ahí} \textbf{ya} me tomaban en cuenta para las categorías”. {\citep{RAE2020}} {[CREA]}\\
\z

El uso de esta estructura en el habla de los hablantes no responde entonces a una importación directa del yukuna.

Otro recurso de los hablantes yukunas en la narración de secuencias, consiste en la utilización de la posposición~\textit{ejená} 'hasta' para terminar un relato: 

\ea
{yukuna}\\
\ea
\gll \textbf{Ejená=ja},   ñakaré-ji   yukúná   tajná=o, \\
      hasta=\textsc{emph}  casa-\textsc{unposs}  historia  terminar=\textsc{mid}\\
\glt ‘Hasta ahí, la historia de la casa se termina'
\ex
\gll iná     la‘-karé,   hasta     ahí.\\
      \textsc{gnr.pro}  hacer\textsc{-nmlz}   \textsc{esp}         \textsc{esp}\\
\glt 'lo que uno hizo, hasta ahí.’[ycn0119,44]
\z
\z

Una estructura que encontramos reflejada en nuestro corpus español, como un marcador discursivo finalizador:

\ea
{español}\\
El morrocoy le dijo, paisano, sólo eso, \textbf{hasta} \textbf{ahí} no más es usted, le dijo. {[ycn0187, 018]}\\
\z

\ea
{español}\\
Se murió del cansancio el venado, \textbf{hasta} \textbf{ahí} aguantó el venado. {[ycn0187,014]}\\
\z

Podríamos hablar para estos tres casos \textit{ahí, de ahí, hasta ahí}, del uso de nuevos marcadores discursivos, continuativos para los dos primeros, finalizador, para el último, que explotan las capacidades referenciales de sus componentes léxicos. Este cambio indirecto implicaría el incremento en la frecuencia de uso de estructuras del español para lograr un mayor paralelismo entre las lenguas de los hablantes bilingües.  Ahora bien, para corroborar esta hipótesis y afirmar una eliminación de restricciones de uso, cabría realizar un análisis cuantitativo de la frecuencia de estas estructuras en el español de la región. 


\subsubsection{Donde que}


Notamos también en nuestro corpus frases relativas con el adverbio relativo de lugar \textit{donde}, seguido por el pronombre \textit{que}: 

\ea
{español}\\
Ahí mismo, más al ladito \textbf{donde} \textbf{que} estamos. {[ycn0200, 012]}\\
\z

\ea
{español}\\
Y eso ya lo, ya lo tumbaron todo, está vacío \textbf{donde} \textbf{que} \textbf{estaba} esa maloca. {[ycn0201, 010]}\\
\z

\ea
{español}\\
Pues él va a estar un mes, pues no sé él dice que él va a pasar vacaciones allá donde la tía en Bogotá, \textbf{donde} \textbf{que} \textbf{estaba} Ferley, \textbf{donde} \textbf{mi} \textbf{hermana.} {[ycn0201, 017]}\\
\z

\ea\label{ex:lemus:32}
{español}\\
Cuando ya está bien quemado ya, ahí ya van a limpiar, \textbf{donde} \textbf{que} \textbf{va} \textbf{a} \textbf{ser} la maloka, ahí van destroncar todos los palos que están ahí, de ahí pues ya Fermín va a hacer su maloquita. {[ycn0201, 023]}\\
\z

Esta variación puede analizarse como una transferencia indirecta del yukuna en el que las relativas de lugar se construyen con el adverbio interrogativo \textit{méño’jó} 'dónde' y con el sufijo \textit{-ka} que hace del interrogativo una conjunción situada al inicio de la subordinada:

\ea\label{ex:lemus:33}
{yukuna}\\
\gll Wa=i‘jná jana-jé ya‘jnáje, méño‘jó-ka iná  jña‘-ká  jíña-na\\
     1\textsc{pl}-ir pescar\textsc{-purp.mot} lejos dónde-\textsc{conj} \textsc{gnr.pro} tomar-\textsc{nmlz} pez-\textsc{pl}\\
\glt `Vamos a pescar lejos \textit{donde que} se coge pescao.' [ycn0042,25]
\z

Cabe resaltar la existencia de un uso del español medieval y del Siglo de Oro que asocia ambos elementos 
(\ref{ex:lemus:34}) -- (\ref{ex:lemus:35}), así como la existencia de un uso marginal del español mexicano~con cierto valor enfático   \REF{ex:lemus:36prim} y en el español de Huesca en España  \REF{ex:lemus:37prim}:

\ea\label{ex:lemus:34}
{español}\\
Señor, \textbf{onde} \textbf{que} sea, embíanos pastor,  /que ponga esta casa en estado mejor;  / mal nos face la mengua, la bergüença peor, / esto por qué abiene tú eres sabidor.  {\citep{RAE2020}} {[CORDE].}\\
\z

\ea\label{ex:lemus:35}
{español}\\
En el Génesis podrás  / (si quieres) verlo abonado,  / y de su nombre firmado,  / \textbf{donde} \textbf{que} dice hallarás, / cuando de hermosuras llenas / su bondad se satisfizo, / vio Dios cuantas cosas hizo / y eran por extremo buenas. {\citep{RAE2020}} {[CORDE].}\\
\z

\ea\label{ex:lemus:36prim}
{español}\\
Yo no voy a hacer como mi prima Loretito que para comer carne en vigilia anda pidiendo permiso al señor cura y se pasa la vida pagando dispensas,{~}\textbf{donde} \textbf{que}{~}están tan caras las dichosas dispensas. {\citep{RAE2020}} {[CREA].}\\
\z

\ea\label{ex:lemus:37prim}
{español}\\
Allá \textbf{donde} \textbf{que} eso, allí estaban en esa casica que hemos hecho allí {\citep{Fernández-Ordóñez2005}}, {[Oliván (Biescas) (COSER-2222\_01)]}  \\
\z

Así, el uso de \textbf{donde que} en nuestro corpus de yukuna-hablantes revela un mecanismo de convergencia que reactivaría el uso de una estructura existente en otras variedades diacrónicas y diatópicas del español. Ahora bien, la aparición de este uso en nuestro corpus sería influenciada por la situación de contacto con el yukuna que presenta estructuras lexico-pragmáticas similares. 

\subsubsection{Discurso directo}


Otra variación sintáctica concierne el discurso directo omnipresente en las narraciones de nuestros informantes. En efecto, las diferentes narraciones recogidas conciernen relatos de historias tradicionales con varios personajes y la restitución de las conversaciones entre ellos. Estos diálogos entre los personajes, son relatados por nuestros hablantes casi exclusivamente utilizando el discurso directo \xxref{ex:lemus:36bis}{ex:lemus:38}:

\ea\label{ex:lemus:36bis}
{español}\\
\label{bkm:Ref484459607}Luego no hay gente–, \textbf{dijo} –, por qué no me contesta, – \textbf{dijo} –, sí, yo soy, yo soy –\textbf{dijo} \textbf{él} –, ¿usted quién es ?– \textbf{dijo} \textbf{él} –, yo, – \textbf{dijo} –, yo tío, – \textbf{dijo} –, ahí sí – \textbf{dijo} \textbf{él}–, tío. {[ycn0184, 018 – EYM}]\\
\z

\ea\label{ex:lemus:37bis}
{español}\\
\label{bkm:Ref484461550}Entonces la mujer \textbf{dijo}: ¡ah! – \textbf{ella} \textbf{dijo}  –, los niños están llorando de hambre, – \textbf{ella} \textbf{dijo} –, al menos sálgase a pescar un rato por ahí. Entonces \textbf{él} \textbf{dijo}: no es que mi hermano me dijo… ¿Acaso su hermano…? , cómo es, cómo fue que ella dijo, ajá, ¿acaso tu hermano te tiene en la vista? – \textbf{ella} \textbf{le} \textbf{dijo} así. {[ycn0190, 223 – VMY}]\\
\z

\ea\label{ex:lemus:38}
{español}\\
\label{bkm:Ref484461552}Vea todo el bagazo de ají que tengo en mi brazo, porque yo soy muy hombre, – él \textbf{dijo}. \label{bkm:Ref484459614}Y él le \textbf{dijo}: pero eso tiene su misterio, – le \textbf{dijo}  –, si yo lo tiro en el suelo así, no, eso cae sin pepa, pero si usted se pone boca arriba y le tiro en todo el pecho, – le \textbf{dijo} –, sale con pepa, – le \textbf{dijo.} {[ycn0187, 008 – JK}] \\
\z

Un solo informante, que tiene un grado elevado de uso del español, se aventuró en un momento a utilizar un discurso indirecto:

\ea
{español}\\
Lo saludó y \textbf{le} \textbf{dijo}, le pidió el favor y \textbf{le} \textbf{dijo} \textbf{que} \textbf{le} \textbf{hiciera} \textbf{un} \textbf{favor} pa’ que le hiciera una cueva {[ycn0187, 012 – JK].} \\
\z

Esta baja incidencia del discurso indirecto en los hablantes con diferentes grados de exposición a estructuras normativas del español se debe, por una parte, a la inexistencia en yukuna de una estructura equivalente al discurso indirecto, ilustrando así una búsqueda de paralelismos sintácticos por la cual los hablantes favorecen el uso de estructuras equivalentes en la lengua de contacto. 

Por otra parte, podríamos postular que, puesto que el discurso indirecto requiere en español normativo la concordancia temporal con un verbo en subjuntivo imperfecto, su uso implica para los hablantes más carga cognitiva y habría cierta tendencia a eludirlo.   

En cuanto a la sintaxis, cabe resaltar en los ejemplos anteriores la repetición del verbo introductor \textit{decir}, pospuesto a cada oración del emisor \REF{ex:lemus:38}--\REF{ex:lemus:39} o antepuesto y pospuesto a cada oración del emisor \REF{ex:lemus:40}, un uso influenciado por la sintaxis del yukuna, en el cual la anteposición, posposición (con la posposición \textit{ké} ‘como’) y el redoblamiento del verbo introductor son opciones sintácticas posibles:

\newpage
\ea\label{ex:lemus:39}
{yukuna}\\
\gll \textbf{Ri=ímí-cha}: “nu=jmerémi pi=apú náke,\\
     \textsc{3sg.nf=}decir-\textsc{pst} \textsc{1sg}=hermano\_menor \textsc{2sg=}despertar eh\\
\glt `Y él dijo: “levántate hermanito'

\gll pi=apho‘-tá     wa-jló,   ipe‘ní wáni”\\
     \textsc{2sg}=encender\nobreakdash-\textsc{caus}  \textsc{1pl=}para  frío \textsc{emph}\\
\glt `prende fogón para nosotros, hace mucho frío”,  [ycn0108,47]
\z

\ea\label{ex:lemus:40}
{yukuna}\\
\gll É \textbf{ru=ímí-cha} ru=o‘we-ló  jló: „yu‘wí“ \textbf{ké} \textbf{ru=ímí-cha} \\
     luego \textsc{3sg.f}=decir-\textsc{pst}  \textsc{3sg.f}=hermano-\textsc{f} para hermano\_menor como \textsc{3sg.f}=decir-\textsc{pst}\\
\glt `Ella le dijo a la hermana: “hermanita”, dijo.’ [ycn0068,53]
\z

Las variaciones sintácticas desarrolladas en esta parte pueden ser analizadas dentro del marco de una influencia indirecta del yukuna en el español de los hablantes. Estos adaptan su español favoreciendo el uso de aquellas estructuras que tienen un equivalente en el yukuna a través de un proceso de convergencia.  Ahora bien, este proceso también puede resultar en una reestructuración de paradigmas, --es el caso del sistema pronominal de esta variedad, como  veremos a continuación.

 \subsubsection{Pronombres demostrativos}


El análisis de nuestro corpus revela un uso de pronombres demostrativos que no concuerdan en género y número con el sustantivo que actualizan: 

\ea
{español}\\
Y él dijo cómo usted va a cargar mucho, acaso usted tiene cuerpo de tres, dijo, pa’ que usted cargue \textbf{ese} \textbf{mucho} \textbf{catarijana}, haga pequeñito, dijo, yo voy empacar. {[ycn0184, 032]}\\
\z

\ea
{español}\\
Se puso cortar otra vez pa’ hacer \textbf{otro} \textbf{catarijana} ahí llegó él, atrás de él, tío, dijo, yo le dije que no soltara \textbf{ese}, dijo. {[ycn0184, 035]}\\
\z

\ea
{español}\\
\textbf{Esa} \textbf{historia} es de nosotros, \textbf{ese} si fue real. {[ycn0190, 198]}\\
\z

\ea
{español}\\
Entonces ahí se fueron, por ahí en el monte, sacaron \textbf{ese} \textbf{cáscara} \textbf{de} \textbf{palo} {[ycn0190, 009]}\\
\z

\ea
{español}\\
No, de tanto revolcarse en \textbf{ese} \textbf{ceniza} que ellos quemaban cáscara de palo. {[ycn0190, 015]}\\
\z

\ea
{español}\\
Entonces él dijo, yo veo \textbf{ese} como \textbf{escamas} de culebra, dijo. {[ycn0190, 074]}\\
\z

Cabe resaltar en este caso una influencia de la morfosintaxis de los demostrativos en yukuna. En efecto, en yukuna existen los demostrativos~\textit{kélé} ('ese'), \textit{marí} y \textit{kháãjí} ('este') cuya concordancia en género y número con el sustantivo solo se da con sustantivos animados y aun así no es obligatoria.

Como en otras partes de la gramática del yukuna, existe en los modificadores del sustantivo una tendencia a la neutralización del género y número a través del uso de la forma singular masculina para todos los casos:

\ea
{yukuna}\\
\gll Kajú  \textbf{kélé}     pátoto-na  pui’-chá-ka-o.\\
     mucho  \textsc{med.dem}  rana-\textsc{pl}  hablar-\textsc{pst-nz-mid}\\
\glt `\textbf{Esas} ranas cantan mucho' (lit. ‘mucho es que ese ranas cantan’) [ycn0151,76]
\z

\ea
{yukuna}\\
\gll \textbf{Kélé}  yuwa-ló   kémí-cha-ka\\
     \textbf{\textsc{med.dem}}  niño-\textsc{f}    decir-\textsc{pst-nz}\\
\glt `\textbf{Esa} muchacha dijo' (lit. que ese muchacha dijo) [ycn0151, 084]
\z

Esto explicaría que, en nuestro corpus, la forma masculina \textit{ese} aparezca en 44 frases, mientras que su equivalente femenino aparece en sólo 11 ocurrencias. Una tendencia que se repite en el caso de \textit{este/a}. (10/4). 

En el presente caso, la convergencia con el yukuna ha resultado en un proceso de simplificación del paradigma de demostrativos a través de la neutralización del género y número.

 \subsubsection{Pronombres de COI/COD}


Los pronombres de objeto directo e indirecto presentan también variaciones en el español de estos hablantes. Si un análisis general del corpus permite establecer un uso mayoritariamente normativo de pronombres de objeto directo e indirecto, cabe de todas formas analizar algunas variaciones. 

El estudio del corpus revela principalmente una tendencia leísta en algunos casos de complemento de objeto directo con referente animado masculino y femenino en singular: 

\ea
{español}\\
Ella \textbf{le}[M] regañó que no se qué, ay que no sé qué, ay, salieron corriendo ahí. {[ycn0190, 140]}\\
\z

\ea
{español}\\
Entonces me toca ir por ahí las doce y veinte será, al aeropuerto, a esperar\textbf{le}[M] {[ycn0201, 006]}\\
\z

\ea
{español}\\
No, no él viene, ya pero ya muy tarde, ya \textbf{le}[M] traen, ya viene de todo, no sé para donde \textbf{le} van a trasladar a él. {[ycn0201, 014]}\\
\z

\ea
{español}\\
O, sea él curó ese flecha, como a él nadie lo podía matar entonces el mismo rezó esa flecha, con esa flecha mismo \textbf{le}[M] mataron a él. {[ycn0190, 209]}\\
\z

\ea
{español}\\
Y a bañar a la niña, alistarle todo y después pues \textbf{llevarle}[F] al colegio, de ahí me vengo, hago mi desayuno, ahí de las ocho pues ya yo estoy aca, de ahí salgo a las doce. {[ycn0201, 007]}\\
\z

Esta tendencia incipiente a la restructuración del sistema pronominal de objeto directo e indirecto en el español en contacto con el yukuna es una evidencia más de la inestabilidad de esta categoría gramatical del español. En efecto, tanto en sincronía y diacronía, a lo largo y ancho de los territorios hispanohablantes, incluso en variedades sin situación de contacto, el español presenta casos de reestructuración de su sistema pronominal.\footnote{Es el caso del español en contacto con lenguas náhuatl y maya (\citealt{FloresFarfán1999}), quechua (\citealt{Godenzzi1986}; \citealt{Klee1990}; \citealt{DeGranda2001};  \citealt{PalaciosAlcaine2005}), guaraní \citep{DeGranda1996}, así como de variedades peninsulares (\citealt{Landa1995}; \citealt{Fernández-Ordóñez1999};  \citealt{KleinAndreu2000}).} Así mismo, como apunta \citet[267]{PalaciosAlcaine2019}, “todas las variedades de español en contacto con lenguas amerindias han experimentado reestructuraciones parciales o totales de su sistema pronominal”.

Por ende, estas variaciones, más allá de la influencia indirecta del yukuna, tendrán que analizarse dentro del marco de tendencias internas al español aceleradas por su situación de segunda lengua de los hablantes de yukuna. 

\section{Conclusión}


En este artículo hemos analizado las prácticas comunicativas de los hablantes bilingües yukuna/español cuyo contexto socio-lingüístico está marcado por un intenso contacto histórico con el español. 

Si el yukuna aparece en expansión en su territorio tradicional y su transmisión intergeneracional está temporalmente asegurada por el fuerte sentimiento de pertenencia cultural que le confieren los hablantes, esta situación contrasta con la imposición del español fuera del territorio tradicional.  En la ciudad, el peso del español como lengua vehicular se hace sentir hasta en los ámbitos familiares. Así, la segunda generación de niños yukuna o matapí que viven en Leticia tiene un conocimiento pasivo de la lengua y es muy probable que la tercera generación sea únicamente hispanohablante. Aun así, es posible que esta generación haga uso de las mismas variaciones aquí descritas.

En cuanto a los hablantes cuya lengua principal es el yukuna, estos reivindican un discurso y una actitud normativa en situaciones discursivas con hablantes extracomunitarios, actitud que contrasta radicalmente con interacciones entre miembros de la misma comunidad y que se caracterizan por el uso de estructuras y recursos lingüísticos pertenecientes al español y al yukuna de manera indiferenciada, marcada en algunos casos, anodina en muchos otros. Olvidando el discurso normativo, los hablantes se sienten libres de recurrir a la totalidad de sus recursos discursivos, sin barreras ni fronteras lingüísticas. 

El caso de variaciones léxico-sintácticas ilustra también este fenómeno.  Los hablantes, al reproducir estructuras del yukuna con material léxico español, buscan paralelismos estructurales entre ambas lenguas y explotan las potencialidades discursivas del español. Es el caso de las estructuras con \textit{ahí} y \textit{donde que}, constatadas en versiones anteriores del español o en otras variedades diatópicas del español. 

Más allá de estos fenómenos, la observación del corpus nos ha permitido constatar una variación en el paradigma de los clíticos pronominales, una reestructuración incipiente que puede ser analizada dentro del marco de la influencia indirecta del yukuna, que activa un proceso de variación interna del español. En efecto, en otras variedades del español con o sin contacto se ha constatado una tendencia a la reestructuración de los sistemas pronominales.  Esta constatación plantea pistas que tendrán que ser analizadas en estudios posteriores desde una perspectiva comparativa con procesos similares en otras variedades americanas del español en contacto con lenguas indígenas.  

A través de nuestro estudio hemos querido analizar las tendencias variacionales de unos pocos hablantes de español/yukuna. Los cambios analizados en el presente trabajo dentro de una perspectiva bidireccional revelan la conjunción de factores externos e internos de la situación de contacto. Ahora bien, como lo indicamos al principio de este trabajo, una de las particularidades de estas poblaciones nativas del Amazonas es el carácter plurilingüístico, reforzado por sus asociaciones matrimoniales exogámicas entre los diferentes grupos étnicos que conviven en las mismas comunidades. Un enfoque multidireccional revelaría, sin duda alguna, todo el entramado de cambios directos e indirectos de esta variedad del español y toda la riqueza lingüística de sus hablantes.  

\section*{Agradecimientos}
Agradecemos la ayuda invaluable de los hablantes de yukuna que participaron en este estudio. Un millón de gracias por su hospitalidad y por su paciencia al enseñarnos su lengua y compartir con nosotros sus tradiciones.

\section*{Abreviaciones}

\begin{tabular}{ll}
\textsc{caus} & causativo\\
\textsc{dem} & demostrativo\\
\textsc{dim} & diminutivo\\
\textsc{emph} & enfático\\
\textsc{esp} & palabra en español\\
\textsc{gnr} & pronombre genérico\\
\textsc{med}&  distancia media\\
\end{tabular}
\begin{tabular}{ll}
\textsc{mid} & voz media\\
\textsc{nf} & género no-femenino\\
\textsc{poss} &  posesión\\
\textsc{pro} & pronombre\\
\textsc{purp.mot} & propósito de movimiento\\
\textsc{unposs} & sustantivo  no poseído\\
\\
\end{tabular}



\sloppy\printbibliography[heading=subbibliography,notkeyword=this]
\end{document}
