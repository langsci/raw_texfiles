\documentclass[output=paper]{langscibook} 
\ChapterDOI{10.5281/zenodo.5643277}
\author{Santiago Sánchez Moreano\orcid{0000-0002-8434-3726}\affiliation{The Open University} and Élodie Blestel\orcid{0000-0001-7257-6859}\affiliation{Université Sorbonne Nouvelle}}
\title{Español en contacto con lenguas amerindias: nuevas perspectivas}
\abstract{En esta introducción se brinda un panorama de la obra, ofreciendo un cambio de perspectiva en la descripción de los fenómenos de contacto lingüístico en el campo del español en contacto con lenguas amerindias a partir de marcos metodológicos y teóricos de la sociolingüística del multilingüismo. Este cambio de perspectiva implica pasar del análisis de “sistemas” y de “códigos” en contacto, y de los resultados que se desligan, a la descripción y el análisis de prácticas sociales de lenguaje, centrados en la utilización de recursos semióticos y lingüísticos por parte de los hablantes para expresar mensajes, transmitir conocimientos o negociar posturas sociales, es decir, para construir significados en la interacción social.}
\IfFileExists{../localcommands.tex}{
  \addbibresource{localbibliography.bib}
  \usepackage{langsci-optional}
\usepackage{langsci-gb4e}
\usepackage{langsci-lgr}

\usepackage{listings}
\lstset{basicstyle=\ttfamily,tabsize=2,breaklines=true}

%added by author
% \usepackage{tipa}
\usepackage{multirow}
\graphicspath{{figures/}}
\usepackage{langsci-branding}

  
\newcommand{\sent}{\enumsentence}
\newcommand{\sents}{\eenumsentence}
\let\citeasnoun\citet

\renewcommand{\lsCoverTitleFont}[1]{\sffamily\addfontfeatures{Scale=MatchUppercase}\fontsize{44pt}{16mm}\selectfont #1}
   
  %% hyphenation points for line breaks
%% Normally, automatic hyphenation in LaTeX is very good
%% If a word is mis-hyphenated, add it to this file
%%
%% add information to TeX file before \begin{document} with:
%% %% hyphenation points for line breaks
%% Normally, automatic hyphenation in LaTeX is very good
%% If a word is mis-hyphenated, add it to this file
%%
%% add information to TeX file before \begin{document} with:
%% %% hyphenation points for line breaks
%% Normally, automatic hyphenation in LaTeX is very good
%% If a word is mis-hyphenated, add it to this file
%%
%% add information to TeX file before \begin{document} with:
%% \include{localhyphenation}
\hyphenation{
affri-ca-te
affri-ca-tes
an-no-tated
com-ple-ments
com-po-si-tio-na-li-ty
non-com-po-si-tio-na-li-ty
Gon-zá-lez
out-side
Ri-chárd
se-man-tics
STREU-SLE
Tie-de-mann
}
\hyphenation{
affri-ca-te
affri-ca-tes
an-no-tated
com-ple-ments
com-po-si-tio-na-li-ty
non-com-po-si-tio-na-li-ty
Gon-zá-lez
out-side
Ri-chárd
se-man-tics
STREU-SLE
Tie-de-mann
}
\hyphenation{
affri-ca-te
affri-ca-tes
an-no-tated
com-ple-ments
com-po-si-tio-na-li-ty
non-com-po-si-tio-na-li-ty
Gon-zá-lez
out-side
Ri-chárd
se-man-tics
STREU-SLE
Tie-de-mann
} 
  \togglepaper[1]%%chapternumber
}{}

\begin{document}
\maketitle 
%\shorttitlerunninghead{}%%use this for an abridged title in the page headers

 
 
\section{El aporte de la sociolingüística del multilingüismo a los estudios sobre el contacto}

En las últimas décadas, han surgido nuevas formas de conceptualizar la relación entre el lenguaje, la sociedad y diversidad multilingüe de nuestras sociedades contemporáneas. Estas nuevas conceptualizaciones reposan en giros epistemológicos que han operado en el campo de los estudios sobre el multilingüismo en los últimos años. Uno de esos giros ha sido el de utilizar enfoques críticos, etnográficos y post-estructuralistas en el estudio de los cambios lingüísticos, sociales y culturales. Otro ha sido el de empezar a tener en cuenta la intensificación de los flujos de poblaciones, el incremento del uso de nuevas tecnologías y los cambios sociales, políticos y económicos a los que como individuos hemos venido asistiendo (\citealt{Martin-JonesMartin2017}: 1).

Estas nuevas conceptualizaciones se enmarcan en una \textit{sociolingüística del multilingüismo} que tiene en cuenta el nuevo orden comunicativo y las condiciones culturales particulares de nuestro tiempo, manteniendo una preocupación central por los procesos sociales e institucionales implicados en la construcción de la diferencia social y la desigualdad social (Ibid). Esta sociolingüística del multilingüismo toma sus bases en los años setenta con los trabajos pioneros de John Gumperz y Dell Hymes (\citeyear{GumperzHymes1972}) y \citet{Hymes1974} en etnografía de la comunicación y en sociolingüística de la interacción, cuyo interés común, ya en ese entonces, era el de entender cómo el mundo social y las identidades de los individuos se construyen a través de la interacción social en contextos multilingües. Estas perspectivas estaban más influenciadas por los enfoques etnográficos propios de la antropología lingüística, que por las visiones netamente lingüísticas de la sociolingüística. Estas últimas dieron paso en Estados Unidos a la sociolingüística variacionista, la cual tomó distancia de la dimensión multilingüe y se centró principalmente en el estudio de los aspectos sociales como el sexo, la edad y la clase social, que entran en juego en los fenómenos de variación y cambio lingüístico, sobre todo en las variedades monolingües. Por ejemplo, los trabajos pioneros de \citet{Labov1966,Labov1972transformation} sobre la estratificación social en Nueva York y de \citet{Trudgill1974} sobre la diferenciación social en el inglés de Norwich se focalizaban en poblaciones monolingües, aun cuando las comunidades estudiadas eran heterogéneas y sociolingüísticamente diversas (\citealt{LégliseChamoreau2013}: 2).

Asimismo, la sociolingüística del multilingüismo ha dado cabida a otras reflexiones fundamentales en el estudio de las relaciones entre el lenguaje y el mundo social. Por ejemplo, la noción misma de “contacto de lenguas” o de “sistemas lingüísticos”, así como la de “bi-lingüismo” y “multi-lingüismo” implican la existencia de \textit{lenguas} que podemos identificar y nombrar. Esta idea ha sido debatida en los últimos años desde la antropología lingüística, la sociolingüística del multilingüismo y la lingüística aplicada.\footnote{Sobre los problemas epistemológicos que acarrea la conceptualización del “contacto de lenguas”, ver \citet{Nicolaï2012}.} Autores como    \citet{MakoniPennycook2006}, \citet{Léglise2007} y \citet{Heller2007}, y más recientemente    \citet{OtheguyEtAl2015} y \citet{Léglise2018}, han criticado el concepto de \textit{lengua} como una entidad autónoma y separable, de invención sociohistórica y sociopolítica, que ha sido construido sobre la base de ideologías normativas. Esta concepción de lengua, como lo sugieren   \citet[138]{MakoniPennycook2005}, ha tenido efectos reales y materiales en la forma como las lenguas han sido entendidas, en cómo las políticas lingüísticas y la educación han sido concebidas, en cómo la gente ha llegado a identificarse con ellas, e incluso hasta en cómo los lingüistas hemos construido nuestro objeto de estudio. Para \citet[27]{JørgensenEtAl2011}, la idea de las lenguas como sistemas delimitados podría no ser suficiente para capturar y comprender el uso real que hacen los hablantes de sus recursos lingüísticos. En este sentido, en lugar de \textit{lenguas}, puede resultar más adecuado hablar de lenguaje, de prácticas de lenguaje o incluso de prácticas o actividades de lengua, en el sentido en que no se tiene como objetivo directo la descripción de lenguas y de sus estructuras, sino la descripción de la práctica que los actores sociales hacen de ellas y de las actitudes que tienen hacia ellas \citep[14]{Léglise2007}. Estas visiones permiten desplazar pertinentemente el foco hacia los recursos lingüísticos y semióticos en el marco de prácticas interaccionales socialmente situadas \citep{Mondada2002}.

Estas mismas percepciones son retomadas, en el ámbito hispánico, por autoras como \citet[273]{Nussbaum2012},  \citet[7]{PalaciosAlcaine2017intro} y   \citet[4]{BürkiPatzelt2020}, entre otros. Palacios Alcaine, por ejemplo, afirma que, en los últimos decenios, “las investigaciones se centran menos en ocuparse de cuáles son las restricciones lingüísticas que impiden el transvase de elementos y más en concebir estas situaciones desde un punto de vista dinámico, donde el hablante ocupa el lugar central y no las lenguas" (\citealt{PalaciosAlcaine2017intro}: 7).

Esta focalización en los hablantes y en sus recursos nos remite necesariamente a la noción de lenguaje como práctica social -- o lo que se conoce en la sociología del lenguaje y la sociolingüística francófonas como \textit{pratiques langagières}, cuya traducción al español, como se discute más adelante, resulta problemática. Las \textit{pratiques langagières} son las prácticas sociales en las que los hablantes utilizan dinámicamente sus recursos lingüísticos en función del contexto social en el que se desenvuelven, para construir significados sociales y, por ende, para construir el mundo social. Para \citet{Léglise2018}, siguiendo a   \citet{BoutetEtAl1976}, la idea de \textit{práctica} (\textit{pratique}) es fundamental en la comprensión de la imbricación del uso del lenguaje que hacen los hablantes en la construcción del mundo social. Una \textit{pratique langagière} es fundamentalmente una práctica social, ya que hace parte del conjunto de prácticas sociales y, como tal, contribuye a la construcción del mundo social.

\citet{Léglise2018} sugiere que existe una dificultad al intentar traducir el adjetivo \textit{langagier} a otras lenguas como el inglés o el español. Es decir que si lo traducimos como \textit{linguistic practices} o \textit{prácticas lingüísticas} estaríamos hablando de \textit{pratiques linguistiques} y no de \textit{pratiques langagières.} Al utilizar el adjetivo \textit{lingüísticas, linguistiques} o \textit{linguistic} estaríamos contribuyendo a la identificación de diferentes lenguas en un corpus plurilingüe, en una conversación o en un repertorio, por ejemplo. Algo similar pasa con los términos “multilingüe” y “plurilingüe” en los que los prefijos \textit{multi-} y \textit{pluri-} remiten a la adición y manejo de varias \textit{lenguas} por parte de los hablantes (\citealt{JørgensenEtAl2011,Léglise2018}). En este sentido, una competencia plurilingüe, como lo sugieren   \citet{CosteEtAl1997}, no es la adición del manejo de varias lenguas, sino la capacidad de comunicar \textit{langagièrement} y de interactuar \textit{culturellement}. Para estos autores, no hay una superposición o yuxtaposición de competencias distintas, sino una competencia plural y compleja, variopinta y heterogénea (\citealt{CosteEtAl1997}: 12), que se constituye a lo largo de la vida.

Estas nuevas perspectivas del contacto lingüístico, del bilingüismo y del multilingüismo emergen en las últimas décadas, en particular en el estudio del lenguaje y de su relación con el mundo social en situaciones de movilidad y migración (\citealt{Blommaert2010}; \citealt{Gugenberger2010,Gugenberger2013}), pero también de superdiversidad (\citealt{Vertovec2007,BlommaertRampton2011,JørgensenEtAl2011}). Bajo este concepto se han venido estudiando en los últimos años las complejas situaciones de contacto que se generan por diferentes factores como la migración, la movilidad social y espacial, las relaciones de poder, las estrategias de inserción económica, las relaciones entre los actores sociales y las instituciones, entre otros.  Los estudios muestran que todos estos factores determinan las prácticas sociales del lenguaje de los hablantes haciéndolas cada vez más complejas y heterogéneas, es decir, cada vez más plurilingües, pluridialectales, pluriestilísticas, pluriacentuales (\citealt{Léglise2013}; \citealt{LégliseAlby2013}). De estas perspectivas surgen o se popularizan conceptos como los de \textit{translanguaging} (\citealt{García2009}; \citealt{GarcíaWei2014}), prácticas translingües, prácticas transcódicas \citep{Lüdi1987}, translingualización \citep{Zimmermann2019}, \textit{polylanguaging} (Jørgensen et al. 2011), \textit{language crossing} (\citealt{Rampton1995,Rampton2005}),  surgidas en los últimos años\footnote{Léglise presenta una discusión sumamente detallada del surgimiento de estos conceptos \citep{Léglise2018} y una crítica del uso del término \textit{translanguaging} en los trabajos sobre multilingüismo en estudios realizados en el Norte Global \citep{Legliseenprensa}.}  a partir de la noción de repertorio verbal de \citet{Gumperz1964} y como respuesta a otras nociones como la de \textit{performance} y competencia lingüística.  Con trasfondo posestructuralista, estas se centran en los recursos lingüísticos y semióticos que utilizan los hablantes para expresar mensajes, transmitir saberes y expresar posturas sociales de (des)afiliación, es decir, para crear significaciones sociales.

Esta heterogeneidad del lenguaje como práctica social conduce a una multiplicidad de manifestaciones de formas convergentes, híbridas, e indexicales que obligan al investigador a replantearse el concepto de lengua como sistema \citep{MakoniPennycook2005}; o incluso, lo llevan a replantearse la noción de fronteras entre lenguas o sistemas de contacto (\citealt{Léglise2018}; \citealt{AuzanneauGreco2018}). En estos contextos de contacto, la construcción de significación -- en sus dimensiones cognitivas y sociales -- debe reflejar esta heterogeneidad. Tal cambio de enfoque -- ya no hacia sistemas lingüísticos abstractos e independientes, sino hacia prácticas contextuales y dinámicas -- no descarta la dimensión sistemática de los fenómenos, pero siempre implica que los investigadores delimiten estos contextos y cuestionen la concepción misma de lo que es el contacto de lenguas, tanto en sus dimensiones metodológicas como epistemológicas.

\section{Español en contacto con lenguas amerindias} 

De manera general, se dice que las lenguas están en “contacto” si se usan en una misma sociedad o en un sector de la población (\citealt{KleeLynch2009indigenas}: 1) o si conviven en un mismo espacio geográfico y son usadas por los mismos individuos (\citealt{Silva-CorvalánEnrique-Arias2001}: 269). Estas visiones del contacto de lenguas implican sus propias dinámicas que pueden ser “bilingües” o “multilingües” y no necesariamente son estáticas; sin embargo, podrían resultar problemáticas desde la sociolingüística del multilingüismo como se discute en el apartado anterior.

Si bien los primeros trabajos que se interesaron por el contacto de lenguas como posible explicación del cambio lingüístico se remontan al final del siglo XIX en el campo de la lingüística histórica\footnote{Ver por ejemplo los trabajos del lingüista americano William Dwight Whitney (1827--1894), del criollista Hugo Schuchardt (1842--1927) o del filólogo y folklorista alemán Rodolfo Lenz (1863--1938), este último conocido por su debatido trabajo sobre el español de Chile.} y a inicios del siglo XX con el advenimiento del estructuralismo \citep{Winford2003book}, su estudio contemporáneo empieza con los trabajos de Uriel \citet{Weinreich1953} y Einar \citet{Haugen1950a,Haugen1950b,Haugen1953} a quienes se reconoce como los precursores del estudio contemporáneo del contacto lingüístico.\footnote{En lo que atañe al campo de la lingüística hispánica, el estudio del contacto remontaría más bien a las refutaciones de \citet{Alonso1953} y \citet{Oroz1966} sobre el trabajo de \citet{Lenz1893}, quien sugería que el español chileno era español con sonidos araucanos.} Estos autores intentaron completar el análisis lingüístico con explicaciones sociales para dar cuenta del contacto de lenguas y de sus consecuencias de manera sistemática, siendo los primeros en poner de relieve la importancia de analizar los fenómenos de contacto no solo desde una perspectiva lingüística sino también desde una perspectiva social. Su mayor contribución fue sin duda alguna la formulación de un marco teórico extenso y completo para el contacto lingüístico en su contexto social \citep{Winford2003book}, para cuyo estudio no existía hasta entonces ninguna teoría sistemática \citep[456]{Clyne1987}. A raíz de ello, la mayoría de los trabajos realizados en el estudio contemporáneo del contacto de lenguas se han focalizado en lo lingüístico o en lo social, como metáforas de lo interno y lo externo. En el ámbito de la lingüística hispánica, estas dos perspectivas han constituido las dos líneas generales de investigación que se han seguido en las últimas décadas, a veces oponiéndose, a veces complementándose,\footnote{Un ejemplo de conjunción de factores externos e internos fue el del concepto de “causación múltiple” propuesto por Yakov \citet{Malkiel1967}, que fue retomado luego en los numerosos trabajos del filólogo y dialectólogo Germán de Granda, cuyos trabajos sobre el contacto del español con lenguas amerindias y africanas han marcado de forma sustancial y constante este campo de estudio (ver por ejemplo \citealt{DeGranda1991,DeGranda1994,DeGranda1999,DeGranda2001}).} pero siempre creando nuevos enfoques para el estudio de los fenómenos de contacto.

Durante muchos años,  en Estados Unidos, la gramática generativa y la sociolingüística variacionista dominaron la investigación en las ciencias del lenguaje en Estados Unidos, centrándose a menudo en los factores internos de variedades monolingües. Y si bien el contacto lingüístico se encontraba en el centro de muchas investigaciones, sobre todo en aquellas que no incluían las lenguas indoeuropeas, no formaba parte del \textit{mainstream} que constituían los estudios generativistas y variacionistas \citep[1]{Hickey2010}. En el caso del español, por ejemplo, antes de que los investigadores se interesaran por los efectos del contacto en la variación y en su evolución, la mayoría de los estudios se focalizaban en la descripción de fenómenos desde la perspectiva de la evolución interna de las lenguas. Así, los primeros estudios sobre la variación en el español eran exclusivamente descriptivos y se centraban en la variedad hablada en un solo país, dejando de lado la posibilidad de que un rasgo particular pudiera encontrarse también en otras variedades geográficas del español \citep{Lipski2007}. A pesar de ello, algunos empezaban a reconocer ya las contribuciones de las lenguas indígenas, sobre todo en términos de léxico, pero sin ir más allá de los efectos mismos del contacto (ibid).

De un lado y otro del Atlántico, uno de los mayores impulsos para la consolidación de la lingüística de contacto como disciplina fue la publicación de  \citet{ThomasonKaufman1988} “Language Contact, Creolization, and Genetic Linguistics” y \citet{Thomason2001} “Language Contact: An Introduction”. En ellos se propone una tipología de los resultados del contacto y un marco de referencia empírica y teórica para comprender los distintos escenarios y posibilidades de contacto a gran escala desde una perspectiva histórica. Hoy por hoy, la lingüística de contacto es una disciplina que se ha diversificado ampliamente y que, además de interesarse por la perspectiva histórica, también se ha interesado por fenómenos como la alternancia de códigos, la creación de lenguas de contacto como los pidgins, criollos, lenguas mixtas \citep{Winford2003book}, la descripción tipológica de lenguas (\citealt{Ross1996}; \citealt{Matras2009}) o los aspectos psicológicos del contacto (\citealt{VanCoetsem1988,VanCoetsem2000}).

Estos trabajos fundamentales aportan a la discusión histórica sobre el papel de los factores internos y externos en la variación y el cambio lingüístico inducido por contacto, un tema controversial para la sociolingüística hispánica (\citealt{KleeLynch2009indigenas}: 114). En términos generales, el contacto de lenguas ha proporcionado a la sociolingüística hispánica formas nuevas y complementarias de explicar la variación y el cambio lingüístico. Tanto es así que, en nuestros días, el español en contacto con lenguas amerindias constituye un campo de estudio bien definido que se ha nutrido durante muchos años de diversos métodos y teorías. Así, en palabras de John Lipski,

\begin{quote}
[u]n componente esencial de la investigación sociolingüística es el reconocimiento de los efectos del contacto con otras lenguas y dialectos sobre la diversificación del español. Podemos afirmar sin exagerar que, en la actualidad, aunque se siguen produciendo trabajos descriptivos, así como análisis formales (de sintaxis y fonología) sobre la variación regional y social del español de América, el estudio del contacto –de lenguas y dialectos– representa el área de investigación más fructífera.” (\citeyear{Lipski2007}: 309)
\end{quote}

Una mirada a las publicaciones en este campo en las dos últimas décadas revela el vigor de la literatura especializada. Así, encontramos decenas de artículos, monografías, libros editados, dosieres temáticos en revistas científicas, proyectos en curso, etc. que estudian fenómenos de contacto entre el español y las lenguas amerindias desde diversos enfoques.\footnote{Una de las últimas publicaciones en este campo la han dirigido  \citet{BlestelPalaciosAlcaine2021}.} \citet{KleeLynch2009indigenas} y \citet{Escobar2012} proponen sendas síntesis de las situaciones de contacto entre el español y lenguas indígenas en las zonas americanas más estudiadas como México, los Andes y Paraguay.

Tal vez una de las variedades de español en contacto más estudiadas hasta hoy es la denominada “español andino”, la cual ha sido analizada a través del contacto con variedades de quechua y aimara en la amplia zona de los Andes (del sur de Colombia hasta el norte de Argentina). Numerosos autores se han dedicado a su descripción como por ejemplo
\citet{KleeCaravedo2006,CaravedoKlee2012,PfänderPalaciosAlcaine2013,PalaciosAlcainePfänder2018,Muntendam2008,Muntendam2013,Muysken1984,Muysken2005,MartínezLópez2012,MartínezLópez2017,Olbertz2005,Olbertz2008,Haboud1998,Haboud2005,Cerrón-Palomino2003,DeGranda1992,MermaMolina2008,ArboledaToro2003,GarcíaTesoro2013,GarcíaTesoro2015,Babel2009,Babel2014stereotypes,GómezRendón2008} y \citet{SánchezMoreano2017,SánchezMoreano2019}; entre muchísimos otros más que no podríamos listar aquí.

Sin embargo, también han sido descritas muchas otras situaciones de contacto entre el español y otras lenguas amerindias, como es el caso del contacto entre el español con lenguas indígenas en México como
el otomí (\citealt{Zimmermann1986,GuerreroGalván2009othobui}),
el purepecha \citep{Chamoreau2007},
el tepehuano (\citealt{TorresSánchez2015}) y
el maya yucateco (\citealt{HernándezMéndez2017}). También encontramos en la literatura situaciones de contacto entre el español y el guaraní (\citealt{DeGranda1988,PennerEtAl2012,PalaciosAlcaine2019b,GómezRendón2007,Blestel2015habiasido,BlestelFontanier2017}),
el tz'utujil en Guatemala (\citealt{GarcíaTesoro2005}), y variedades como
el jopara (\citealt{Zajícová2014,Blestel2019}) en Paraguay,
el malecu en Costa Rica (\citealt{SánchezAvendaño2017}) y
el mapuzugun (\citealt{OlateVinet2017,OlateVinetWittigGonzález2019}) en Chile. Asimismo, en los últimos años, han surgido denominaciones generales de variedades de español en contacto como la de español amazónico \citep{Jara2017}, español amerindio o incluso la de español andinoamazónico (\citealt{RamírezCruz2009}) que, más allá de resultar problemáticas,\footnote{Ver \citet{Zimmermann2016} para una discusión de las denominaciones “español indígena” y “español amerindio”.} dan cuenta de lo concreto de los contactos entre variedades de español y lenguas amerindias.

A pesar de esta vitalidad y del hecho de que los estudios sobre el contacto entre el español y las lenguas amerindias se han multiplicado en los últimos años, la complejidad de las situaciones de contacto en el ámbito americano dista de ser bien conocida (\citealt{PalaciosAlcaine2019a}: 23). Por esta razón, esta obra colectiva de nueve capítulos de lectura independiente es una pequeña contribución al estudio de esta complejidad, que como bien señala Palacios Alcaine, debe ser estudiada “sin prejuicios ni ideas preconcebidas” (\citealt{PalaciosAlcaine2019a}: 28). Sobre todo, poniendo al hablante en el centro de nuestras preocupaciones como (socio)lingüistas especialistas del contacto y considerando lo que sucede en el contexto social donde se producen las interacciones.

El hecho de tener en cuenta los factores sociales y la complejidad de los contextos sociales nos conduce a abordar necesariamente el aporte de los enfoques sociales del lenguaje y de perspectivas interdisciplinarias a la lingüística de contacto. Así pues, las nueve contribuciones de esta obra ofrecen un cambio de perspectiva en el estudio del español en contacto con lenguas amerindias. Este cambio implica pasar del análisis de sistemas en contacto, y de sus consecuencias en términos de variación y cambio, a la descripción y análisis de prácticas de lenguaje heterogéneas o \textit{pratiques langagières hétérogènes} \citep{Léglise2013}. Esto conlleva necesariamente algunos desafíos metodológicos, como afirman  \citet{BürkiPatzelt2020}, que apelan a lo etnográfico, lo cualitativo, a la perspectiva del individuo, a lo interpretativo, lo ideográfico y a su complementación con lo cuantitativo (ver \citealtv{chapters/klee}). Estudiar tales prácticas sociales de lenguaje implica la necesidad de implementar enfoques y metodologías dinámicas y transdisciplinarias que pongan en primera línea lo heterogéneo de las situaciones de contacto. Esto es lo que propone mostrar esta obra colectiva a través de cada uno de los nueve capítulos. En efecto, la mayor parte de los estudios sobre el contacto entre el español y lenguas amerindias, tanto en diacronía como en sincronía, describen las consecuencias lingüísticas de la convergencia de dos sistemas tipológicamente distintos en el habla de personas bilingües en zonas de contacto prolongado. Los fenómenos tratados en estos estudios muestran claramente que el contacto entre dos sistemas produce transferencias lingüísticas, préstamos, calcos, alternancias de código y toda clase de variación lingüística a través de diferentes mecanismos de variación y de cambio lingüístico (\citealt{DeGranda1997,Haboud1998,Escobar2000,PalaciosAlcaine2005,Olbertz2008,MermaMolina2008,Pfänder2009,PalaciosAlcaine2011,Muysken2011,Muntendam2013}). Por el contrario, esta obra colectiva se posiciona en una perspectiva algo distinta. Aquí reunimos el trabajo de investigadores interesados, por un lado, por la forma en que podemos describir y analizar los fenómenos lingüísticos inducidos por contacto en situaciones sociales y lingüísticas heterogéneas y dinámicas; y, por otro lado, por la manera como la significación emerge en su dimensión cognitiva y en su dimensión social en estas situaciones de contacto. Estos cuestionamientos nos obligan a revisar nuestra concepción misma de lo que es el contacto de lenguas, en el plano metodológico y epistemológico. También, nos obligan a revisar la manera como construimos nuestros corpus en el sentido en que estos deberían reflejar el dinamismo y la heterogeneidad de las prácticas de lenguaje de los actores sociales (\citealt{LégliseSánchezMoreano2017} y \citealtv{chapters/leglise}).

\section{Nuevas perspectivas}

Las reflexiones que orientan el trabajo de recopilación de esta obra colectiva se originan en diferentes espacios de discusión e intercambio. Algunos de los capítulos fueron discutidos en jornadas de estudio\footnote{Jornadas de estudio internacionales “Variedades de español en contacto con lenguas amerindias: sistemas en contacto o prácticas lingüísticas heterogéneas”, 12 y 13 de junio de 2017, Universidad Sorbonne Nouvelle, Maison de la Rercherche, París (UMR 8202 SeDyL/LABEX-EFL y EA 7345 CLESTHIA). Organización: Santiago Sánchez Moreano y Élodie Blestel.}, mientras que otros fueron escogidos por su afinidad con la orientación de la obra. Los capítulos están organizados en dos partes. La primera parte, “Perspectivas teóricas y metodológicas”, incluye cuatro capítulos que abogan por un cambio de perspectiva en nuestra forma de concebir el contacto lingüístico en las prácticas del lenguaje que esté más centrada en los recursos heterogéneos y en la variación en lugar de las estructuras o sistemas en contacto. Esta perspectiva, aunque no es nueva para la antropología lingüística y la lingüística sociocultural, ha sido poco explorada en el ámbito de los estudios generales sobre el contacto (aunque existen notables excepciones). Y lo ha sido aún menos en los estudios sobre el español en contacto con lenguas amerindias. 

Así pues, los artículos de Carol Klee (University of Minnesota), Isabelle Léglise (CNRS), Élodie Blestel (Université Sorbonne Nouvelle), Nadiezdha Torres Sánchez (Universidad Nacional Autónoma de México) y Alonso Guerrero Galván (DL-INAH) proponen nuevas perspectivas teóricas y metodológicas aplicables al estudio del español en contacto con lenguas amerindias. Carol Klee, por ejemplo, propone un diálogo entre dos enfoques, en apariencia opuestos, pero que la autora presenta pertinentemente como complementarios en la comprensión de la complejidad del contacto de lenguas. El primer enfoque, de tipo variacionista, está basado en el estudio de \citet{KleeEtAl2011} sobre las funciones pragmáticas y las situaciones discursivas que motivan una frecuencia mayor del orden de palabras Objeto-Verbo (OV) en el habla de migrantes con quechua como primera lengua de socialización en Lima. El segundo enfoque está basado en los trabajos de \citet{Babel2014stereotypes} sobre el desarrollo de un marcador de pasado evocativo en el español andino de Bolivia y el de  \citet{SánchezMoreano2017}, que versa sobre el orden de palabras OV en el español de ecuatorianos hablantes de quechua que han emigrado a Cali, Colombia. Ambos estudios se fundamentan en las prácticas lingüísticas heterogéneas de los hablantes y son de corte cualitativo e idiográfico, es decir, se enfocan en cómo los hablantes hacen uso de sus repertorios lingüísticos para crear significados y alcanzar los objetivos de la comunicación.

Por su parte, Isabelle Léglise propone una sólida metodología para el análisis y anotación de corpus que muestra no solo lo heterogéneo de las prácticas lingüísticas, sino también cómo las lenguas o variedades de lengua pueden por momentos solaparse, haciendo que sea irrelevante asignar categorías arbitrarias y límites a los recursos lingüísticos de los hablantes. En este sentido, la variación, según Léglise, es un recurso lingüístico que tienen los hablantes a disposición en sus prácticas de lenguaje cotidianas. La autora afirma que este enfoque tiene un impacto en la manera en que los lingüistas consideramos la asignación de etiquetas a las formas lingüísticas. Léglise demuestra también cómo los hablantes pueden utilizar, por momentos, formas no marcadas o ambivalentes \citep{Woolard1998} y en otros, formas marcadas atribuibles a una lengua o variedad de lengua específicas. Esto es indicio de que los hablantes pueden por momentos transgredir las “fronteras” lingüísticas y/o marcarlas claramente para crear significaciones sociales particulares como la de reivindicar una identidad pangrupal o una manera de expresar urbanidad y masculinidad. Y, en otros momentos, pueden utilizar formas lingüísticas específicas para marcar fronteras dialectales a través de desalineamientos y posturas de diferenciación y des-afiliación.

Élodie Blestel muestra cómo la conceptualización de dos lenguas históricas en Paraguay -- español y guaraní -- hace que las gramáticas que subyacen las prácticas lingüísticas de los hablantes se vuelvan opacas para los estudiosos del contacto de lenguas. La autora afirma que las herramientas desarrolladas por los lingüistas, en este contexto (y que se pueden reproducir en muchos otros), muchas veces reflejan y alimentan la representación ideal de los fenómenos de contacto. Blestel sostiene que esto se debe a tres sesgos metodológicos (ideológico, diacrónico y perceptual) que existen y persisten en los estudios del contacto lingüístico. Una forma de sortear estos sesgos es adoptar una perspectiva que sitúe al hablante en el centro de la investigación. Dicha perspectiva le puede permitir al lingüista confrontarse con mejores herramientas a la descripción y comprensión de las complejas realidades sociolingüísticas y a las prácticas de lenguaje que tienen lugar en ellas, observando cómo las unidades que conforman los repertorios lingüísticos de los individuos sufren un continuo re-análisis. 

Finalmente, Nadiezdha Torres Sánchez y Alonso Guerrero Galván formulan una propuesta novedosa para describir y comparar las distintas situaciones sociolingüísticas en que se desenvuelven las comunidades. Presentan un trabajo, con una metodología comparativa, en el que se analizan, a través de la noción de mercado lingüístico y \textit{habitus} de Bourdieu, los espacios de usos del español y lenguas indígenas (otomí, chichimeca jonaz y tepehuano del sureste) en tres redes comunitarias. Los autores argumentan que, dentro de la compleja situación sociolingüística de México, las comunidades construyen una serie de disposiciones o esquemas generadores de conductas lingüísticas o \textit{habitus} que hacen que el uso de varias lenguas sea más o menos ventajoso para ciertas coyunturas situacionales y discursivas. El principal interés de su estudio es, por un lado, comparar los tres contextos sociolingüísticos en los que se desenvuelven las tres redes comunitarias y, por otro, proponer una metodología conjunta que permita hacer dicha comparación y plantear una tipología de las comunidades multilingües. 

La segunda parte, “Perspectivas aplicadas”, está compuesta por cinco contribuciones que proponen estudios puntuales y novedosos, centrados en los recursos de los hablantes y no en el contacto de estructuras o sistemas, o en todo caso, alejándose de dichas perspectivas. 

Por ejemplo, Ignacio Satti y Mario Soto (Albert-Ludwigs-Universität Freiburg) presentan un estudio comparativo interaccional y multimodal de la iniciación de interrupciones en la interacción por parte de los potenciales co-narradores en diferentes variedades del español y de quechua (Cochabamba, Bogotá, Buenos Aires, Friburgo). Los autores se focalizan en el uso de recursos verbales y no verbales, observando diferencias significativas en las frecuencias de contacto visual y del uso de la gestualidad entre Cochabamba y el resto de contextos sociolingüísticos. Una mirada cualitativa muestra que en Cochabamba los participantes suelen recurrir a recursos lingüísticos disponibles en los repertorios de los hablantes para realizar las prácticas de interrupción en donde se observan más bien estrategias modales e interaccionales. Su estudio pretende contribuir a la discusión sobre el contacto de lenguas mostrando la relevancia de incorporar un acercamiento multimodal a las prácticas de los hablantes. Además, el enfoque multimodal utilizado permite observar con más facilidad el dinamismo y la heterogeneidad de las prácticas de lenguaje de los hablantes, cuestionando la visión tradicional del contacto como solamente el contacto de dos sistemas lingüísticos. 

Por su parte, Aura Lemus Sarmiento (Université Paris-Sorbonne, RELIR) y Madgalena Lemus Serrano (Université Lumière Lyon 2, DDL) proponen un estudio sobre las prácticas de lenguaje de un grupo indígena del suroccidente colombiano, los yukuna, cuyos repertorios lingüísticos se constituyen al menos de la lengua yukuna y del español, pero también de otras lenguas gracias al sistema de alianzas matrimoniales, que como lo indican las autoras, “da como resultado comunidades multilingües que comunican entre sí en una o varias de las lenguas que tienen en común” (p. \pageref{lemus:quoteforintro}). Si bien las autoras se focalizan en los resultados del contacto entre el yukuna y el español, su estudio resulta relevante en el sentido en que se describe una situación de contacto extremadamente interesante de una población indígena de la Amazonía colombiana en situación de contacto urbano. Las autoras presentan una descripción del uso de “códigos” presentes en los repertorios lingüísticos, pero también de fenómenos de variación y cambio directos e indirectos, reconociendo siempre su carácter bidireccional. Asimismo, dado el carácter multilingüe del contexto estudiado dejan la puerta abierta a la aplicación de un enfoque multidireccional del contacto que revelaría el complejo entramado de los fenómenos de contacto. 

Más adelante, Azucena Palacios Alcaine (Universidad Autónoma de Madrid) se interroga sobre la variación de los valores evidenciales y modalizadores de los tiempos de pasado en el español andino ecuatoriano, peruano y boliviano en contacto con diferentes variedades de quechua. La autora argumenta que el contacto intenso y prolongado en el área andina posibilita la emergencia de nuevos valores evidenciales como estrategias comunicativas construidas a partir del uso subjetivo y dinámico de los tiempos verbales que hace el hablante. Así, los hablantes explotan las potencialidades que el español ya tiene y que el quechua y el aymara ofrecen. Para esto, la autora se apoya en una perspectiva dinámica del contacto en la que los hablantes y sus recursos son el eje central. Estos perciben similitudes entre los dos sistemas en contacto y las optimizan en función de sus necesidades comunicativas y cognitivas. En este sentido, los fenómenos de variación y cambio lingüístico no son meros calcos o préstamos, sino que más bien obedecen a una ampliación de las posibilidades de expresión para los hablantes, producto de lo heterogéneo que son las prácticas lingüísticas.

Por su parte, Aldo Olate Vinet (Universidad de La Frontera, Chile), también desde una perspectiva centrada en los hablantes, nos propone una visión del contacto que engloba no solo las dinámicas intra--, inter-- y extra--lingüísticas, sino también los propios aspectos socio-históricos de los hablantes, es decir, sus experiencias de vida, sus trayectorias sociolingüísticas, el uso de variedades lingüísticas, sus actitudes y representaciones hacia sus lenguas, etc. El autor propone considerar la complejidad y la dinamicidad de las variedades de contacto a través de un marco de referencia y una metodología que discutan nuestra concepción reduccionista del contacto lingüístico y que permitan una comprensión integrada de los fenómenos. Para esto, el autor muestra cómo, por ejemplo, las experiencias de vida de los hablantes de mapuzugun y castellano son elementos cruciales que aportan a la comprensión de los procesos de desplazamiento que viven las lenguas minorizadas en situación de contacto.

Finalmente, Carola Mick (Université Paris Descartes), propone un estudio sobre las dinámicas del contacto de los sistemas de pronombres clíticos átonos de 3a persona en el español de actores sociales andinos y limeños de la Consulta Previa con distintos perfiles de movilidad. Mick demuestra que hay un acercamiento de los sistemas de pronombres clíticos a medida en que se intensifica la movilidad de los hablantes hacia la capital. Asimismo, la autora muestra que mientras más los hablantes se reconocen como indígenas, una categoría social protegida por las instituciones, más hacen hincapié en sus recursos lingüísticos que los identifican como tales. Centrada en los hablantes y en el uso que hacen de sus recursos lingüísticos, Mick interpreta esto como una estrategia que es posible gracias a lo heterogéneo de sus recursos y de sus prácticas de lenguaje.  

Estas cinco contribuciones no solo proponen visiones dinámicas del contacto entre el español y lenguas amerindias, sino que tienen en común el hecho de que se focalizan en los hablantes, en sus recursos y en la complejidad de los contextos socioculturales en los que se desenvuelven. Junto con los cuatro capítulos de la primera sección, consideran que más allá del simple contacto, las lenguas, variedades de lengua, dialectos, registros y otras formas de hablar hacen parte de los repertorios lingüísticos de los hablantes, a través de los cuales construyen el mundo social en la interacción.  

\section*{Agradecimientos}

Agradecemos a los pares que han evaluado este trabajo por sus invaluables sugerencias y comentarios.

\sloppy\printbibliography[heading=subbibliography,notkeyword=this]
\end{document} 
