\documentclass[output=paper]{langscibook}
\ChapterDOI{10.5281/zenodo.5643281}
\author{Isabelle Léglise\orcid{0000-0002-2415-4173 }\affiliation{Centre National de la Recherche Scientifique, CNRS SeDyL}}
\title{Marcar o no marcar las fronteras: la variación como recurso lingüístico en las prácticas multilingües}
\abstract{La variación lingüística y los cambios lingüísticos inducidos por contacto han sido considerados por mucho tiempo como consecuencias del contacto entre ``comunidades'' y sistemas lingüísticos estables y bien delimitados. Al contrario de esta visión, considero la variación como un recurso lingüístico que tienen los hablantes a disposición en sus prácticas de lenguaje cotidianas, multilingües, que también pueden ser llamadas \textit {trans} o \textit {(poly)languaging}. Seguir tal perspectiva implica primeramente un cambio de enfoque de los sistemas lingüísticos a los hablantes y a sus repertorios. Segundo, esto necesita una sólida metodología que revele la heterogeneidad de las prácticas de lenguaje a través de la anotación de corpus. Este método, al mismo tiempo, muestra cómo las lenguas o variedades de lengua se pueden solapar haciendo que sea irrelevante asignar categorías arbitrarias y límites a los recursos lingüísticos de los hablantes. Esto tiene por supuesto un impacto en la manera en que nosotros, como lingüistas, consideramos la asignación de etiquetas a las formas lingüísticas. Veremos una tendencia entre hablantes bilingües (en contextos endolingües) a utilizar formas no marcadas o bivalentes \citep{Woolard1998} que podrían pertenecer a dos o más variedades de lengua. Usar formas bivalentes como recursos lingüísticos puede constituir una manera de desafiar las fronteras entre lenguas y mostrar lo borroso que éstas pueden llegar a ser. También, puede constituir una manera de reivindicar una identidad pangrupal (desafiando las fronteras entre variedades de lenguas) o una manera de expresar urbanidad y masculinidad (desafiando las fronteras entre lenguas). Finalmente, mostraré también que es posible utilizar formas lingüísticas específicas para marcar fronteras dialectales a través de desalineamientos, diferenciación y des-afiliación, lo cual produce significados sociales}

\IfFileExists{../localcommands.tex}{
  \addbibresource{localbibliography.bib}
  \usepackage{langsci-optional}
\usepackage{langsci-gb4e}
\usepackage{langsci-lgr}

\usepackage{listings}
\lstset{basicstyle=\ttfamily,tabsize=2,breaklines=true}

%added by author
% \usepackage{tipa}
\usepackage{multirow}
\graphicspath{{figures/}}
\usepackage{langsci-branding}

  
\newcommand{\sent}{\enumsentence}
\newcommand{\sents}{\eenumsentence}
\let\citeasnoun\citet

\renewcommand{\lsCoverTitleFont}[1]{\sffamily\addfontfeatures{Scale=MatchUppercase}\fontsize{44pt}{16mm}\selectfont #1}
  
  %% hyphenation points for line breaks
%% Normally, automatic hyphenation in LaTeX is very good
%% If a word is mis-hyphenated, add it to this file
%%
%% add information to TeX file before \begin{document} with:
%% %% hyphenation points for line breaks
%% Normally, automatic hyphenation in LaTeX is very good
%% If a word is mis-hyphenated, add it to this file
%%
%% add information to TeX file before \begin{document} with:
%% %% hyphenation points for line breaks
%% Normally, automatic hyphenation in LaTeX is very good
%% If a word is mis-hyphenated, add it to this file
%%
%% add information to TeX file before \begin{document} with:
%% \include{localhyphenation}
\hyphenation{
affri-ca-te
affri-ca-tes
an-no-tated
com-ple-ments
com-po-si-tio-na-li-ty
non-com-po-si-tio-na-li-ty
Gon-zá-lez
out-side
Ri-chárd
se-man-tics
STREU-SLE
Tie-de-mann
}
\hyphenation{
affri-ca-te
affri-ca-tes
an-no-tated
com-ple-ments
com-po-si-tio-na-li-ty
non-com-po-si-tio-na-li-ty
Gon-zá-lez
out-side
Ri-chárd
se-man-tics
STREU-SLE
Tie-de-mann
}
\hyphenation{
affri-ca-te
affri-ca-tes
an-no-tated
com-ple-ments
com-po-si-tio-na-li-ty
non-com-po-si-tio-na-li-ty
Gon-zá-lez
out-side
Ri-chárd
se-man-tics
STREU-SLE
Tie-de-mann
}
  \togglepaper[1]%%chapternumber
}{}

\shorttitlerunninghead{Marcar o no marcar las fronteras}%%use this for an abridged title in the page headers
\begin{document}

\maketitle 
\shorttitlerunninghead{Marcar o no marcar las fronteras}%%use this for an abridged title in the page headers

\section{Introducción}


Tomando como centro de interés la heterogeneidad y la hibridez, la lingüística de contacto (\citealt{Thomason2001,Winford2003intro}) ha surgido como un área importante de investigación, aportando  conocimientos sobre los tipos de fenómenos de contacto, los contextos y los factores condicionantes a través de numerosos estudios.  Sin embargo, estos estudios por lo general tratan fenómenos de contacto en diacronía y dan cuenta de cambios lingüísticos finalizados. Incluso si el multilingüismo y el bricolaje lingüístico están lejos de ser poco comunes, la heterogeneidad (entendida como diversidad lingüística en términos de recursos lingüísticos, incluyendo no solo las lenguas, sino también las características asociadas a dialectos, estilos y registros) no ha ganado en visibilidad en la lingüística de corpus ni en la lingüística de contacto. En efecto, las prácticas de lenguaje heterogéneas siguen siendo tratadas como excepciones.

En el campo de la lingüística de contacto se observan dos tendencias que asumen las lenguas como entidades delimitadas. Por un lado, la mayor parte de investigaciones sobre cambios inducidos por contacto se focalizan en una lengua y en los procesos de reorganización y los resultados del contacto con otra lengua (\citealt{Thomason2001,HeineKuteva2005}). Por otro lado, las investigaciones sobre la alternancia de códigos o \textit {codeswitching}, orientadas ya sea hacia su estructura lingüística (\citealt{Myers-Scotton1993,Poplack1981}) o hacia su significado social \citep{Auer1998}, describen en sincronía datos orales espontáneos en los que se identifican claramente los “códigos” en alternancia. Así, términos como “lengua matriz" (\textit {matrix language}) y “alternancia de códigos", comúnmente utilizados en estas dos tendencias, son buenos ejemplos del hecho de que en la lingüística de contacto se consideran las lenguas y los repertorios como entidades delimitadas (ver \citet{Léglise2018} para una discusión más amplia).

Por su lado, el estudio de la variación en sociolingüística se ha focalizado en poblaciones monolingües, aun cuando los variacionistas sabían que las comunidades de habla que estudiaban eran heterogéneas, es decir, social y lingüísticamente diversas. Los estudios pioneros sobre la estratificación social en Nueva York (\citealt{Labov1966,Labov1972book}) o sobre la diferenciación social del inglés en Norwich, Inglaterra \citep{Trudgill1974} excluían, por ejemplo, a los hablantes no nativos y se focalizaban en los cambios al interior de una sola variedad de lengua (\textit{intra-varietal change}) \citep[20]{Labov1994}. Aún son relativamente escasos los trabajos publicados sobre variación en contextos multilingües (ver sin embargo los trabajos de \citet{MeyerhoffNagy2008} o \citet{LégliseChamoreau2013} que traten de integrar los resultados de la lingüística de contacto (multilingüismo y alternancia de códigos) y el estudio de la variación. El sub-campo de dialectos en contacto (\citealt{Gumperz1958,Trudgill1986,Siegel1987,Mesthrie1993,AuerEtAl2005}, etc.) es una excepción, mientras que el estudio de dialectos del español en contacto constituye una línea específica de investigación (\citealt{Elizaincín1992,Butragueño2017,Penny2000,PesqueiraButragueño2012}, entre otros).

Para resumir, la variación lingüística y los cambios inducidos por contacto han sido considerados por mucho tiempo como consecuencias del contacto de "comunidades" y/o de sistemas lingüísticos delimitados y estables. En este artículo, propongo que la variación puede ser vista como un recurso que está a disposición de los hablantes que estos utilizan en sus prácticas multilingües cotidianas. Como demostraré en la primera parte, primero se necesita un cambio de perspectiva que vaya “de los sistemas lingüísticos a los hablantes […y a sus] repertorios constituidos en sus experiencias vividas, y que pueden alterar las presuntas conexiones entre lengua, comunidad y espacio” (\citealt{HallNilep2015}: 615).\footnote {Todas las traducciones de este capítulo fueron realizadas por Santiago Sánchez Moreano e Isabelle Léglise} Segundo, se necesita una sólida metodología que revele la heterogeneidad de las prácticas de lenguaje a través de la anotación de corpus. La segunda parte está dedicada al método de anotación al que recurro para revelar la heterogeneidad de las prácticas de lenguaje. También se mostrará cómo las lenguas o variedades de lengua se pueden solapar haciendo que sea irrelevante asignar categorías arbitrarias y límites entre los recursos lingüísticos de los hablantes. Esto tiene por supuesto un impacto en la manera en que nosotros, como lingüistas, consideramos la asignación de etiquetas a las formas lingüísticas. Veremos una tendencia entre hablantes bilingües (en contextos endolingües) a utilizar formas no marcadas o bivalentes \citep{Woolard1998} que podrían pertenecer a dos o más variedades de lengua. La tercera parte muestra cómo los actores sociales hacen uso de sus recursos lingüísticos, marcando o des-marcando las fronteras entre lenguas. Usar formas bivalentes como recursos lingüísticos puede constituir una manera de desafiar las fronteras entre lenguas y mostrar lo borroso que estas pueden llegar a ser. Constituye también una manera de reivindicar una identidad pangrupal (desafiando las fronteras entre variedades de lenguas) o una manera de expresar urbanidad y masculinidad (desafiando las fronteras entre lenguas). Finalmente, mostraré que es posible también utilizar formas lingüísticas específicas para marcar fronteras dialectales a través de desalineamientos, diferenciación y des-afiliación, lo cual produce significados sociales.
 \section{Prácticas de lenguaje heterogéneas}


Siguiendo la tradición francesa en sociología del lenguaje y en sociolingüística, considero que las prácticas de lenguaje son prácticas sociales. El término \textit{pratiques langagières} fue acuñado hace 40 años para insistir en el hecho de que las prácticas de lenguaje están determinadas y constreñidas por el orden social, y, al mismo tiempo, construyen significado social, producen efectos sociales y contribuyen a modificar dicho orden (\citealt{BoutetEtAl1976}). Las prácticas sociales, las formaciones sociales y el poder simbólico deben ser entendidos aquí a la luz de una teoría de la práctica como la de \citet{Bourdieu1977}. Como señala Boutet:

\begin{quote}
“El lenguaje forma parte del conjunto de prácticas sociales, sean estas prácticas de producción, de transformación o de reproducción. Hablar de “práctica” significa insistir en la dimensión praxeológica de esta actividad. Como toda práctica social, las prácticas de lenguaje están determinadas y constreñidas por lo social, y, al mismo tiempo, producen efectos en él, contribuyendo a su transformación. Bajo esta perspectiva, el lenguaje no es solamente un reflejo de las estructuras sociales, sino más bien un componente integral de dichas estructuras. […] Hablar no es únicamente una actividad de representación del mundo, es también un acto por el cual modificamos el orden de las cosas y transformamos las relaciones sociales.” \citep[459]{Boutet2002}.
\end{quote}

Uso el término \textit {“pratiques langagières”} o \textit {“prácticas de lenguaje”} puesto que de esta manera se insiste más en el hecho de que los hablantes poseemos prácticas hechas de lenguaje (como una actividad), mientras que si decimos \textit {“prácticas lingüísticas”} hacemos referencia más bien a sistemas lingüísticos bien delimitados. \textit {“Languaging”} puede constituir una buena traducción. Como señalan \citet[2]{JørgensenJuffermans2011}: “[una perspectiva de \textit {languaging}] enfatiza el hecho de que la gente fundamentalmente no utiliza ‘una lengua’ o ‘varias lenguas’, sino que utilizan el lenguaje como una serie de recursos lingüísticos.”

En un contexto monolingüe, existen diversas evaluaciones sociales de las prácticas de lenguaje (\citealt{Bakhtin1977,Boutet1982}). Es decir que existe una heterogeneidad lingüística (de la manera de hablar), vinculada a una evaluación social que puede ser valorizante o desvalorizante. En un ambiente multilingüe, estas prácticas de lenguaje son heterogéneas \citep{Léglise2013} en el sentido en que están constituidas por recursos lingüísticos atribuibles a varias fuentes. También son heterogéneas debido a la diversidad de los repertorios lingüísticos de los hablantes. En este sentido, las prácticas de lenguaje son inherentemente heterogéneas, puesto que pueden estar constituidas de toda clase de elementos descritos antes como \textit {code-switching}, \textit {code-mixing} o bricolajes lingüísticos (\citealt{Lüdi1994,Mondada2012}) a través de los cuales los actores sociales utilizan sus recursos lingüísticos para crear nuevos significados.

Las prácticas de lenguaje heterogéneas revelan la diversidad o superdiversidad (\citealt{Vertovec2007,BlommaertRampton2011}) de los contextos y al mismo tiempo son producto de ellos. Muchos otros términos han sido acuñados en los últimos años con el objetivo de darle voz a esta heterogeneidad. Así por ejemplo, encontramos las nociones de “prácticas transidiomáticas” \citep{Jacquemet2005}, “crossing” \citep{Rampton2005}, “translanguaging” (\citealt{García2009,CreeseBlackledge2010,GarcíaWei2014}), “polylanguaging” o “languaging” \citep{JørgensenEtAl2011}. Todas ellas hacen parte de lo que se conoce como el \textit{multilingual turn} \citep{May2014}.

Para Jørgensen, pasar de las lenguas al \textit{languaging} (lenguar) implica considerar

\begin{quote}
el lenguaje, en la práctica real, no como entidades contables y delimitadas que existen en el mundo natural, sino como un potencial dinámico y creativo que permite hablar. Eso enfatiza el hecho que la gente no utiliza fundamentalmente ‘una’ o ‘varias lenguas’, sino que utilizan el lenguaje, como una serie de recursos lingüísticos. Los hablantes bilingües no se consideran como personas que ‘hablan dos lenguas’, sino como ‘agentes de habla’ [\textit{languagers}] que hacen uso de recursos que son reconocidos como pertenecientes a dos conjuntos de recursos lingüísticos por otros hablantes. (\citealt{JørgensenJuffermans2011}: 2)
\end{quote}

En cuanto a mí, me focalizo en el análisis de prácticas de lenguaje heterogéneas como llave de acceso a la construcción de significados sociales, al papel de las prácticas de lenguaje y de los recursos lingüísticos en la (re)producción de estratificaciones sociales y en el refuerzo de desigualdades asociadas a relaciones sociales racializadas (en la escuela, o en hospitales) en contextos multilingües como la Guayana Francesa, Brasil o Surinam.

 \section{Anotación de datos plurilingües y bivalencia de elementos lingüísticos}

Consideremos ahora el campo de la lingüística de corpus y la anotación de datos orales en contextos multilingües. El multilingüismo aún no está bien establecido en esta disciplina. De hecho, los corpus multilingües incluyen por lo general textos en diferentes lenguas, pero cada texto sigue siendo monolingüe. A veces se constituyen corpus multilingües comparables (comparables, por ejemplo, en número y tipo de textos: con géneros de textos en cada lengua) o corpus paralelos comparables, y se realizan equivalencias de traducción entre los ítems que constituyen el texto. Este campo sigue una perspectiva de comparación de objetos estables y monolíticos llamados lenguas o géneros.

En contraste, utilizo el término \textit {“corpus plurilingües”} para designar los corpus heterogéneos que necesitamos para trabajar con prácticas de lenguaje heterogéneas. Estos corpus están constituidos por elementos lingüísticos provenientes de diferentes lenguas que han sido utilizados en una misma interacción. Además, no solo ilustran ejemplos de alternancia y mezcla de códigos (\textit {code-switching, code-mixing}), sino también de prácticas plurilingües o translingües a través de bricolajes lingüísticos. Estos son realizados por hablantes plurilingües con variadas habilidades y competencias, y variados repertorios (\citealt{BlommaertBackus2011}). Este tipo de corpus muestra también la variación lingüística y el uso de formas no estandarizadas, a menudo desatendidas en los corpus monolingües o controladas a través de parámetros generales (como los géneros de textos o géneros discursivos). Contienen formas lingüísticas difíciles de clasificar, las cuales plantean enormes problemas, no solo en términos de identificación, sino también en términos de transcripción y anotación. Los corpus plurilingües aún son minoritarios y poco disponibles para la comunidad de lingüistas, y cuentan con pocos instrumentos diseñados para su anotación.

En lingüística de corpus, el Text Encoding Initiative (TEI) propone un conjunto de normas para la anotación de corpus. El TEI recomienda identificar la lengua base de cada oración y anotar en comillas angulares < > los ítems introducidos pertenecientes a otras lenguas, por ejemplo <ítem foráneo perteneciente a la lengua x>. Un primer problema, en el caso de prácticas de lenguaje heterogéneas, es que la asignación sistemática de una lengua base para cada turno de habla no es posible (ver \citet{LégliseAlby2016} para una discusión sobre el tema). En la mayoría de los casos, nosotros observamos recursos lingüísticos de diferentes lenguas en el mismo turno de habla producido por un mismo hablante. En este sentido, el TEI tuvo que ser adaptado a los corpus plurilingües y fue así como propusimos nuevas normas. \citet{VaillantLéglise2014} presentan los detalles técnicos del sistema de anotación que establecimos. Una de las mayores adaptaciones que realizamos para la descripción de corpus heterogéneos concierne no solo la lengua del enunciado, el cual es “múltiple” en la mayoría de los casos, sino la atribución de características multilingües a varios segmentos (dentro del mismo enunciado). Por ejemplo, en \REF{ex:leglise:1}, un cliente se dirige a una empleada de la compañía nacional de electricidad en Cayena, Guayana Francesa. El fragmento parece comenzar en francés y terminar en criollo (\textit {créole}). Aquí, las negrillas corresponden al criollo guyanés, las redondas corresponden al francés y las cursivas junto con la negrilla a las inserciones de criollo antillano.

En muchos casos, los elementos lingüísticos son atribuibles a varias lenguas, variedades de lengua o estilos posibles. En ese caso, decidimos codificar la mayor parte de posibilidades volviéndolas visibles en la transcripción, como en el ejemplo siguiente: \textit{i} al inicio del enunciado puede ser interpretado, a la vez, como el pronombre de tercera persona del singular en francés hablado (pronunciación de \textit{i} en lugar de \textit{il}) o en un criollo de base francesa (criollo antillano o guayanés). Estas formas han sido ampliamente identificadas en la literatura sobre las alternancias de códigos o \textit{code-switching} como diamorfos homófonos (\textit {homophonous diamorphs}) \citep{Muysken1990} o formas ambivalentes (\textit {bivalent elements}) \citep{Woolard1998}, pero nunca han sido tratadas como tal en la lingüística de corpus. Seguimos aquí la propuesta de  \citet{Ledegen2012} quien adopta una transcripción doble o flotante para resaltar las varias posibles interpretaciones:


\ea\label{ex:leglise:1}
\gllll \textbf{\textit{i}}   ~            ~  ~      ~   ~                     ~               ~          problème\\
       \textbf{i}   ~            ~  ~      ~   ~                     \textbf{\textit{té}}     ~          \textbf{\textit{problem}}\\
       i(l)        m ’a  donné  […] comme    \textbf{té}    \textbf{\textit{ni}}  \textbf{problem}\\
       3\textsc{sg} 1\textsc{sg} ha dado   ~   como\_si.\textsc{conj} \textsc{te.pst} haber problema.\textsc{n}\\
\glt `El(la) me ha dado […]  como si hubiera un problema'
\z

Muchos términos han sido acuñados para referirse a estos asuntos desde las identidades interlinguales de \citet{Haugen1972} como por ejemplo \textit{code gliding} o deslizamiento de códigos \citep{Baggioni1992}, \textit{transition zones} o zonas de transición \citep{Franceschini1998}, \textit{zones flottantes} o zonas flotantes \citep{Ledegen2012}, \textit{homophonous diamorphs} o diamorfos homófonos \citep{Muysken1990} o bivalencia \citep{Woolard1998}.

Siguiendo a Woolard, en la comunicación bilingüe existen bivalencias o simultaneidades que muestran “la membrecía simultánea de un elemento en más de un sistema lingüístico” \citep[6]{Woolard1998}. 


\begin{quote}
“Cuando utilizo la palabra bivalencia me refiero al uso, por parte de un hablante bilingüe, de palabras o segmentos que pueden ‘pertenecer’ igualitaria, descriptiva e incluso prescriptivamente, a ambos códigos”. \citep[7]{Woolard1998}
\end{quote}

\begin{quote}
“[…]podemos discernir en enunciados aislados la presencia simultánea y equitativamente concreta (o equitativamente efímera) de más de un valor de un contraste paradigmático” (ibid: 5).
\end{quote}

\begin{quote}
“[…] una práctica bilingüe puede desmantelar (pero no simplemente neutralizar) distinciones binarias, en este caso, entre variedades de una lengua, a través del “desmontaje o incluso la preservación de toda oposición” la cual “mantiene viva una contradicción no resuelta” (Spivak in \citealt{Derrida1974}, xx 1”.  (ibid: 6).
\end{quote}

   Un buen ejemplo de esto es \REF{ex:leglise:2}, que muestra el inicio de un discurso en castellano y catalán en el que el sujeto gramatical es el catalán. El verbo puede ser castellano o catalán, mientras que el objeto es castellano:

\ea\label{ex:leglise:2}
 \gll  el          saben     aquel  \\
     Cat  Cat/Cas   Cas \\
  \glt `Do you know the one…' (\citealt{Woolard1998} :7)
\z

Siguiendo a Woolard, el hablante está “recurriendo a similitudes para inhibir la definición de la variedad que está utilizando. Esto no es, sin embargo, una supresión sino la explotación de la oposición entre el catalán y el castellano” (\citeyear{Woolard1998}: 12). En su perspectiva, la bivalencia es un recurso lingüístico “que es estratégicamente guiada y retóricamente manipulada por los hablantes” (ibid: 12), por lo que la bivalencia “no pasa desapercibida, sino más bien es politizada y controvertida” (ibid: 14).

Casos de ambivalencia ya han sido documentados en lenguas cercanas (\citet{Woolard1998} para el catalán y el castellano; \citet{ÁlvarezCáccamo1990} para el caso del gallego y el castellano; \citet{MiggeLéglise2013} para el caso de los criollos de base inglesa;  \citet{SánchezMoreano2015} para el caso de variedades de español en contacto en Cali) y en contextos en los que el criollo y su lengua lexificadora aún están en contacto. En contextos a menudo descritos como “contextos de descriollización” como el caso del criollo y el francés en la Reunión, Ledegen ha mostrado que los “elementos flotantes” representan el 15\% de las prácticas de lenguaje y el 16\% de todos los predicados (\citeyear{Ledegen2012}: 257). Mientras tanto, en contextos trilingües como en Senegal \citep{Nunez2015}, en el que tres lenguas genéticamente y tipológicamente no relacionadas como el criollo casamancés (de la región de Casamanza), el wolof y el francés están en contacto, solamente el 4\% de los elementos lingüísticos son ambivalentes.

La presencia de elementos ambivalentes es particularmente alta en contextos endolingües--bilingües (\citealt{DePietro1988}), es decir,  entre hablantes bilingües que comparten las mismas lenguas.\footnote{Retomo aquí la terminología propuesta por De Pietro en un sentido amplio: para mí, el contexto es endolingüe--bilingüe cuando los interlocutores comparten un trasfondo lingüístico que les permite comunicar fácilmente. Así, las mismas lenguas o variedades cercanas hacen parte de sus repertorios sea cual sea el grado de manejo de dichas lenguas y variedades. Diferente es una situación exolingüe en la que los interlocutores no comparten las mismas lenguas para comunicar.} En algunos corpus, la mayor parte de turnos de habla están compuestos por elementos ambivalentes o compartidos (atribuibles al menos a dos lenguas). Por ejemplo, en el fragmento \REF{ex:leglise:3}, de los 9 elementos, 6 son compartidos entre el sranan tongo (en subrayado) y un criollo cimarrón (Eastern Maroon Creole, EMC,\footnote{Término genérico para designar el criollo de base inglesa hablado por Cimarrones (Alukus, Ndyukas y Pamakas) en la Guayana Francesa y en Surinam (\citealt{MiggeLéglise2013}).} en cursiva). Un solo elemento está claramente asociado al sranan tongo y dos están asociados al EMC.

\let\eachwordtwo=\itshape
\ea\label{ex:leglise:3}
\glll \uline{fu}   \uline{a} \uline{no}  \uline{go}  ~    \uline{fu} ~  \uline{a} \uline{moni}\\
      fu           a         ~           go       leli   fu        puu a         moni\\
para.\textsc{comp} 3\textsc{sg.subj} \textsc{neg}   ir aprender para.\textsc{comp} halar \textsc{art.def.sg} dinero\\
\glt `para que ella no aprenda a robar dinero y a gastárselo todo'
\z
\let\eachwordtwo=\upshape

Definitivamente tenemos que deshacernos de la perspectiva de la alternancia de códigos. En dicha perspectiva, se trataría de buscar la lengua matriz como en los siguientes fragmentos de \citet[capítulo 8]{MiggeLéglise2013}, en el que se considera \REF{ex:leglise:4} como una conversación en criollo cimarrón con algunas inserciones de sranan tongo (en subrayado); o \REF{ex:leglise:5} como una conversación en sranan tongo con algunas inserciones del criollo cimarrón (en cursiva). La anotación de las partes de discurso y la traducción aparecen en \REF{ex:leglise:6} sin mención a las lenguas.

\ea\label{ex:leglise:4}
{ EMC + [inserciones en sranan]}\\
\textit{I} \uline{ne} \textit{e} \uline{ferstan} \textit{san}  \textit{mi} \textit{e} \textit{du} \textit{nou}. \textit{Mi} \uline{kan rei} \textit{tu} \uline{trip}, \textit{i} \uline{ferstan} \uline{tok}, \textit{mi} \uline{kan} \textit{lei tu} \uline{trip} / \textit{mi e lei den man fu mi} \\
\z

\ea\label{ex:leglise:5}
{ Sranan + [inserciones en EMC]}\\
\uline{I ne e ferstan san mi e du nou. Mi kan rei tu trip, i ferstan tok, mi kan} \textit{lei} \uline{tu trip / mi e} \textit{lei} \uline{den man fu mi}\\
\z

\ea\label{ex:leglise:6}
\ea
\gll i  ne  e  ferstan\\
     tu  \textsc{neg}  \textsc{impf}  entender\\
\glt tú no entiendes

\ex
\gll san  mi  e  du  nou  \\
     que  yo  \textsc{impf}  hacer  ahora\\
\glt lo que estoy haciendo ahora

\ex
\gll mi  kan    rei    tu  trip\\
     yo  poder     conducir  dos  viaje\\
\glt yo puedo conducir por dos viajes

\ex
\gll i  ferstan        tok,  mi  kan  lei           tu  trip\\
     tu   entender    ok,   yo  poder  conducir dos viaje\\
\glt entiendes, cierto, yo puedo hacer dos viajes.

\ex
\gll mi  e  lei    den  man  fu  mi\\
     yo  \textsc{impf}  conducir  \textsc{det}  chicos   \textsc{pos}  mi\\
\glt yo llevo a mis chicos
\z
\z

Por supuesto, definir la lengua (o la variedad de lengua) de un enunciado o una conversación es sumamente ideológico. Además, esta definición puede cambiar dependiendo de la gente implicada y de las circunstancias. En este caso, un lingüista podría argumentar que, como casi todos los elementos lingüísticos pueden ser sranan tongo, la lengua matriz del enunciado debe ser el sranan. El hablante podría argumentar que, por su etnicidad cimarrona, la lengua del enunciado es la variedad que él habla (es decir, el EMC o una variedad específica como el ndyuka). Pero, en otras circunstancias, el hablante podría afirmar que habla una mezcla de EMC y sranan tongo para parecer “cool” y “moderno”; o en otras circunstancias, podría afirmar que la lengua en sí no importa, sino la flexibilidad… Claramente vemos cómo las lenguas o variedades de lengua se superponen o solapan haciendo que se vuelva irrelevante trazar líneas y fronteras arbitrarias entre los recursos lingüísticos. Esto tiene por supuesto un impacto en la manera en que, como lingüistas, consideramos la atribución de etiquetas lingüísticas de tal o tal lengua o variedad de lengua a las formas lingüísticas. La decisión que tomamos aquí, al anotar este tipo de datos, es la de proveer sistemáticamente una doble trascripción, haciendo que sea posible varias interpretaciones simultáneas como en \REF{ex:leglise:7}.



\let\eachwordone=\itshape
\ea\label{ex:leglise:7}
\ea
\gll i           ~            e           ~ 	           san                  mi            e            du            nou,            mi   ~                    ~                tu     ~ \\
    \uline{i}  \uline{ne}  \uline{e}  \uline{ferstan}  \uline{san}  \uline{mi}   \uline{e}   \uline{du}  \uline{nou},   \uline{mi}  \uline{kan}  \uline{rei}    \uline{tu}   \uline{trip}\\
\ex
\gll i   ~ ~ mi   ~ ~    tu  ~  mi e    ~   den  man   fu mi\\
\uline{i}   \uline{ferstan} \uline{tok}, \uline{mi} \uline{kan}   \textit{lei}       \uline{tu} \uline{trip}  \uline{mi} \uline{e}      \textit{lei}      \uline{den} \uline{man} \uline{fu} \uline{mi}\\
\z
\z
\let\eachwordone=\upshape

 \section{Marcar o no marcar las fronteras dialectales}


Así, en lugar de adoptar la perspectiva de la alternancia de códigos, proponemos más bien adoptar una perspectiva de \textit{languaging} en la que el uso de elementos lingüísticos bivalentes por parte de sujetos plurilingües es visto como performance. En algunos casos, el uso de elementos bivalentes puede ayudar a realizar y representar la borrosidad y las fronteras opacas entre lenguas, como en los ejemplos \REF{ex:leglise:3} a \REF{ex:leglise:7}. La opacidad es realizada a través de elementos que podemos considerar como compartidos, bivalentes o no marcados. Esta opacidad es también, en nuestro caso, expresada por el apelativo \textit {takitaki}, algunas veces utilizado para referirse a este tipo de prácticas (\citealt{LégliseMigge2006}). \textit {Takitaki} ha sido utilizado como término desvalorizante en la Guayana Francesa (utilizado como “galimatías”), pero se revela muy útil para los hablantes: \textit {takitaki} es un término indeterminado, su propio significado es confuso, de modo que evita hacer referencia a identificaciones étnicas entre cimarrones (Pamaka, Ndyuka, Aluku) y entre nombres de lenguas y fronteras (como EMC o Sranan Tongo). \textit {Takitaki} es asociado a procesos de homogeneización cuando una identidad pan--cimarrona es reivindicada (\citealt{MiggeLéglise2013}) y algunas veces a procesos de diferenciación cuando se requiere (\textit {Takitaki} de los negros vs. \textit {Takitaki} de los indígenas -- “hablamos lo mismo, pero diferente”).

Lo que se observa en numerosos contextos endolingües, entre gente que comparte las mismas lenguas o variedades cercanas de una lengua, es que los hablantes plurilingües tienden a utilizar elementos comunes, compartidos, desmarcando claramente las variedades de lengua en fragmentos bastante largos. Aquí, citamos otro ejemplo en el que la mayor parte de los elementos lingüísticos son bivalentes (EMC y sranan tongo) y donde algunas inserciones en inglés son utilizadas (como ‘libres’). Es una manera de hablar muy común entre hablantes jóvenes cimarrones que viven en los centros urbanos (\figref{exfig:leglise:8}).

\begin{figure}
\caption{\label{exfig:leglise:8}El habla de jóvenes cimarrones en centros urbanos}
\includegraphics[width=\textwidth]{figures/Lglise-img001.png}
\end{figure}
 
\newpage
En algunos casos, se observa un amplio uso de formas lingüísticas atribuibles a varias lenguas. En el ejemplo \REF{ex:leglise:9}, los interlocutores están utilizando cinco lenguas o variedades de lengua diferentes: el EMC (en su variante ndyuka en rojo), el sranan tongo (en verde), elementos indeterminados (que pueden ser tanto el EMC como el sranan tongo, en normal), elementos indeterminados en francés o criollo guyanés (en azul) y el holandés (en naranja). A través del uso de todos esos diferentes recursos lingüísticos, es decir, a través de esas prácticas de lenguaje heterogéneas, los interlocutores están expresando el hecho de ser urbanos o modernos, afiliándose a todas las lenguas presentes en la esfera pública en centros urbanos como en Saint Laurent de Maroni en la Guayana Francesa o en Paramaribo en Surinam. Ambos interlocutores están haciendo \textit{languaging} en cuanto des-marcan (des-hacen) las fronteras dialectales entre el SMC y el sranan tongo, e incluso entre el francés y el criollo guyanés. En este sentido, podemos categorizar este fragmento como endolingüe y en el mejor de los casos como convergencia y similitud en la manera en que los actores sociales utilizan sus prácticas de lenguaje.

\newcommand{\emc}[1]{{\color{lsLightWine}\textit{#1}}}
\newcommand{\sra}[1]{{\color{lsDarkGreenOne}\uline{#1}}}
\newcommand{\fra}[1]{{\color{lsMidBlue}#1}}
\newcommand{\hol}[1]{{\color{lsDarkOrange}#1}}
\ea\label{ex:leglise:9}
(\emc{EMC} en rojo, sranan en verde, elementos compartidos en normal, francés / criollo en azul, {holandés} {en} {naranja)}\\\bigskip

% \ttfamily
B: Da \emc{i de anga} \fra{congé} nounou?\\
‘Entonces \emc{¿Estás de} \fra{vacaciones} ahora?’
\bigskip

E: Aii, mi de \sra{nanga} \fra{congé} nou te \sra{lek' tra} mun, bigin fu \sra{tra} mun. \fra{Le} \fra{sept}, da mi bigin baka.\\
‘Sí, estoy \sra{de} \fra{vacaciones} ahora hasta \sra{más o menos el próximo} mes, empezando el \sra{próximo} mes. \fra{El siete}, estaré empezando a trabajar otra vez.’
\bigskip

B: \emc{Soutu wooko} i e du?\\
‘\emc{¿Qué clase de trabajo} haces?’
\bigskip

E: \sra{Sortu wroko} mi e du? Mi e du wan \hol{sers}.\\
‘\sra{Qué clase de trabajo}  hago? Estoy trabajando como  \hol{guardia de seguridad}.’
\bigskip

B: Mh?
\bigskip

E: Wan \hol{sers} mi e du, wan \sra{tra wroko leki} \emc{soudati}, fu \fra{la mairie / lameri} gi \fra{la mairie / lameri}. Ma mi \sra{hoop} taki \sra{nanga} \hol{kontrakt}, den man \emc{ná} e gi wan langa \hol{kontrakt}, \sra{siksi} mun, \sra{ef'} i e \sra{wroko} bun, den man gi i \sra{siksi} mun baka, te \sra{nanga} tu \emc{yali}, a \emc{kaba}, den man \hol{stop} en.\\

‘Estoy trabajando como \hol{guardia de seguridad}, \sra{otro tipo de trabajo como el de} \emc{soldado}, del \fra{ayuntamiento}, para \fra{el ayuntamiento}.  Pero \sra{espero} que \sra{con} un \hol{contrato}, esa gente \emc{no} hace \hol{contratos} largos, \sra{seis} meses, \sra{si} tu \sra{trabajas} bien, la gente te da \sra{seis} meses otra vez, hasta \sra{más o menos} dos \emc{años}, \emc{se acaba}, lo \hol{paran}.'
\z

Sin embargo, este fragmento conversacional \REF{ex:leglise:9} es también interesante como un caso de no alineamiento con la lengua introducida por el interlocutor. Utilizo aquí los conceptos de convergencia y alineamiento en la secuencialidad de los turnos de habla \citep{Auer1995}. Lo que sucede en la línea 4 es que E no se alinea con la elección de lengua de B, más bien reformula en sranan tongo \sra{Sortu wroko} lo que B dice en EMC
(\emc{Soutu wooko} “tipo de trabajo”). La secuencialidad exacta se provee en la siguiente transcripción en la que se anotan las alternancias:

\ea
Línea 3: B \textit{Soutu wooko (EMC)} i e du (indeterminado) ?\\
= EMC + indeterminado\\
Línea 4: E \uline{Sortu wroko}  (sranan) mi e du (indeterminado) ? \\
= Sranan tongo + indeterminado
\z

La reformulación en la línea 4, aunque se realiza en una variedad de lengua muy cercana a la variedad utilizada en la línea anterior, es un caso de marcación de las fronteras dialectales. Es un caso claro de diferenciación en el que E está reivindicando una identidad masculina moderna con elementos lingüísticos del sranan tongo urbano, desalineándose de los elementos lingüísticos utilizados por B y que están asociados a una variedad de lengua cimarrona tradicional más cercana a la de las zonas rurales y de formas tradicionales de hablar -- la cual es apropiada para la forma de hablar de las mujeres, pero no para la reivindicación de una forma moderna de masculinidad. Podemos imaginar que esta forma de hablar está relacionada con los géneros de E y B, pero también con el tema de la conversación: E está hablando del tipo de trabajo de hombres que está haciendo (una suerte de soldado del ayuntamiento).

Este fragmento es un buen ejemplo de prácticas de lenguaje heterogéneas en las que los actores sociales pueden converger en la manera como mezclan elementos lingüísticos y "des-hacen" las fronteras entre lenguas, expresando una identidad moderna común. Pero, al mismo tiempo, pueden expresar a cada instante la desafiliación con respecto a la formulación del interlocutor y expresar otro tipo de identidad (como la masculinidad, por ejemplo) marcando claramente las fronteras dialectales, y excluyendo otras identidades como las “femeninas y tradicionales”. 

\section{Conclusión}


La variación lingüística y el cambio inducido por contacto han sido vistos por mucho tiempo como consecuencias del contacto entre “comunidades” estables y sus presuntas lenguas como sistemas lingüísticos delimitados. Por el contrario, nosotros vemos aquí la variación, desde el punto de vista de los hablantes, como un recurso lingüístico en las prácticas de lenguaje multilingües y heterogéneas. Esta visión necesita un cambio de enfoque, como subrayan   \citet[615]{HallNilep2015}, que vaya de los sistemas lingüísticos hacia los usuarios de las lenguas, ya que sus experiencias vividas pueden “alterar las presuntas conexiones entre lengua, comunidad y espacio”.

El método de anotación que proponemos es particularmente útil para describir prácticas de lenguaje heterogéneas. Este método revela la heterogeneidad del lenguaje y al mismo tiempo muestra cómo las lenguas o variedades de lengua se pueden solapar haciendo que se vuelva irrelevante delimitar arbitrariamente las fronteras entre recursos lingüísticos. Esto tiene un impacto en la manera en que, como lingüistas, consideramos la atribución de etiquetas a las formas lingüísticas. La mejor manera, como demostramos, es abrirse a la multiplicidad y no depender de lo unívoco de la categorización: las categorizaciones pueden ser múltiples.

El uso de formas lingüísticas por parte de los actores sociales, como lo vimos, es socialmente significativo. Existe una tendencia por parte de los hablantes plurilingües, en contextos endolingües, a utilizar formas no marcadas o bivalentes que pueden pertenecer a dos o más lenguas (o variedades de lengua).  Utilizar elementos bivalentes como recursos lingüísticos puede interpretarse como una manera de mostrar la opacidad o lo borroso de las fronteras entre lenguas y desafiar dichas fronteras, como hemos visto. Puede interpretarse como una manera de reivindicar una identidad pan-cimarrona (desafiando las fronteras de variedades entre lenguas cimarronas) o una manera de expresar la urbanidad y la masculinidad (desafiando las fronteras lingüísticas entre EMC y sranan). Pero, al mismo tiempo, para los hablantes, siempre es posible utilizar formas lingüísticas específicas para marcar fronteras lingüísticas o dialectales en un movimiento de desalineamiento, para diferenciarse, desafiliarse y crear significados sociales.

Necesitamos entonces una concepción fluida del lenguaje y de las fronteras entre lenguas, no establecida con anticipación, sino fluctuante a través la interacción, en la que los hablantes puedan marcar o des-marcar dichas fronteras a lo largo de la conversación. El uso específico de formas lingüísticas, a través de procesos de convergencia y divergencia con la formulación del interlocutor, puede, a su vez, conducirnos hacia procesos más amplios de diferenciación y de homogeneización.

\section*{Agradecimientos}
Agradezco a los editores y revisores anónimos por sus comentarios que han mejorado esta contribución. 
This is a Spanish translation provided by Santiago Sánchez Moreano of two papers I gave as an invited speaker in Paris in June 2017 at their international workshop on Prácticas Lingüísticas Heterogéneas and in Utrecht in April 2019 at the Sociolinguistics Circle.

\sloppy\printbibliography[heading=subbibliography,notkeyword=this]
\end{document}
