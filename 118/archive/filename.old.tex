

\setcounter{page}{1}\begin{styleContentsi}
A Grammar of Moloko 
\end{styleContentsi}

\begin{styleContentsi}
with a short Moloko – English lexicon
\end{styleContentsi}

Dianne Friesen

with

Mana Isaac, Ali Gaston, and Mana Samuel

\MakeUppercase{Forward}

\MakeUppercase{Acknowledgements}

Many thanks

To the Moloko men and women who shared their stories and fables with us.  These are the people whose stories we have used for this analysis: Abelden, Ali Gaston, Baba Abba, Dungaya, Dungaya Daniel, Dugujé, Kama Joseph, Majay Moïze, Mala, Malatina Moïze, Mana Samuel, Njida, Sali Anouldéo Justin, Tajay Suzanne, Tajike, and Tsokom.

To the Moloko men who transcribed and translated the texts, entered them into the computer, and helped us understand the Moloko language: Ali Gaston, Holmaka Marcel, Mana Isaac, Oumar Abraham, Sali Anouldéo Justin, and Sambo Joël.

To colleagues who also worked among the Moloko people:  Megan Mamalis, Alan and DeEtte Starr, Ginger Boyd, and Catherine Bow.  

To Jenni Beadle, for smoothly taking the verb files from shoebox to the chart in the appendix.

To Dr. Aaron Shryock, Rhonda Thwing, and Richard Gravina, for tireless interest in the intricacies of Moloko, and miles and miles of red ink in the early drafts.

To Sean Allison, for gracious, detailed comments and challenges on one of the last drafts.

To Dr. Doris Payne, for incredible insights, encouragement, and perseverance. 

To the Moloko people who have welcomed us to their land and into their homes, and for whose sakes we strive to understand this language.  

\textit{Malan manjan ana Hərmbəlom!}

\setcounter{tocdepth}{3}
\renewcommand\contentsname{}
\tableofcontents

\textbf{Abbreviations used:}

\begin{tabular}{ll}
\lsptoprule
/ verb stem / & underlying form\\
1 & 1st person\\
2 & 2\textsuperscript{nd} person\\
3 & 3\textsuperscript{rd} person \\
ADJ & adjectiviser\\
\textstylePartofspeech{\textup{adp.}} & adposition \\
ADV & adverbiser\\
\textstylePartofspeech{\textup{adv.}} & \textstylePartofspeech{\textup{adverb}}\\
CL & verb class (/-j/ suffix)\\
conj. & conjunction\\
DAT & dative preposition\\
\textstylePartofspeech{\textup{dem.}} & \textstylePartofspeech{\textup{demonstrative}}\\
DEP & dependent form of verb\\
disc. & discourse marker \\
DO & direct object pronominal\\
EX & exclusive (first person plural)\\
\textsc{EXT} & existential\\
\textsc{GEN} & Genitive particle\\
\textsc{HOR} & Hortative mood \\
\textsc{ID} & Ideophone\\
\textsc{IFV} & Imperfective aspect\\
\textsc{IMP} & Imperative\\
IN & inclusive (first person plural)\\
interj. & interjection \\
IO & indirect object pronominal \\
\textsc{ITR} & habitual iterative aspect\\
n. & noun\\
nclitic & noun clitic\\
\textsc{NEG} & negative \\
n.pr. & proper noun\\
nsfx & noun suffix\\
\textsc{NOM} & nominalised form of verb\\
num. & numeral\\
P & Plural\\
\textsc{PBL} & Possible mood\\
\textsc{PFV} & Perfective aspect\\
Pl & plural noun clitic\\
\textsc{PLU} & pluractional\\
pn. & pronoun\\
\textsc{POSS} & possessive pronoun\\
\textsc{POT} & Potential mood\\
\textsc{PRF} & Perfect\\
PSP & presupposition marker\\
quant. & quantifier \\
R & realis mood\\
S & singular\\
S. \# & sentence number from text\\
spp. & species\\
v. & verb\\
vclitic & verb clitic\\
vpfx. & verb prefix\\
vsfx. & verb suffix\\
\lspbottomrule
\end{tabular}
\setcounter{page}{1}\chapter[Introduction]{Introduction}
\hypertarget{RefHeading1210241525720847}{}\hypertarget{Toc450584435}{}
\begin{styleFooter}
{\thepage{}}
\end{styleFooter}

Although this grammar is currently more than 100,000 words long, it truly only scratches the surface of this beautiful language. Moloko grammar is interesting and complex, and much further study can be done to study some of its genius in more detail. 

The notable features of the language include:

\begin{itemize}
\item the consonantal skeleton of words (see \sectref{sec:6.2}), 
\item the simplicity of the vowel system (there is only one underlying phoneme with ten phonetic representations and 4 graphemes, see \sectref{sec:2.3}), 
\item the absence of adjectives as a basic word class (all adjectives are derived from nouns, Chapter 5.3), 
\item the complexity of the verb word (Chapter 7), with information in the verb word indicating in addition to the verbal idea, subject, direct object (semantic Theme), indirect object (recipient or beneficiary), direction, location, aspect (Imperfective and Perfective), mood (indicative, irrealis, iterative), and Perfect aspect, 
\item Moloko is a somewhat agglutinative language, since easily separable morphemes can be added to noun and verb stems,
\item cliticisation is very productive within the language. Clitics are both inflectional and derivational, and in nouns and verbs, always follow the lexical root they modify. Cliticisation in verbs allows several layers of clitics to be added. Verbal clitics are called \textit{extensions} in this paper, following Chadic linguistic terminology.\footnote{Paul \citet{Newman1973} noted that the term ‘verbal extension’ was widely used in Chadic languages to describe “optional additions that serve to expand or modify the meaning of the basic verb (p. 334). Note that verbal extensions also exist in languages from the Niger-Congo, Nilo-Saharan, and Khoisan families, where they can have derivational or inflectional functions (Hyman, 2007). Note that the term ‘extension’ for Chadic languages has a different use than for Bantu languages.} In Chadic languages, ‘extension’ refers to particles or clitics in the verb word or verb phrase.
\item the fact that verbs are not inherently transitive or intransitive, but rather the semantics is tied to the number and type of core grammatical relations in a clause (Chapter 9), 
\item ideophones (Chapter 3.6), found in many African languages, are lexical items giving a ‘picture’ or a ‘sound’ idea of the event they symbolise. They function in Moloko as adverbs, adjectives, and in particular contexts, as verbs, 
\item the presupposition construction (Chapter 12), which is the main organisational structure in Moloko discourse, 
\item interrogative formation (see \sectref{sec:11.3}), including re-arrangement of the clause so that the interrogative particle occurs clause final,
\item reduplication that occurs in verbs (see \sectref{sec:54}) and nouns (see \sectref{sec:26}), and can be inflectional or derivational.  
\end{itemize}

Linguistic classification, language use, and previous research are outlined in Sections 1.1 to 1.3. The four texts that follow in Sections 1.4{}-1.7 are among many that were collected while the first author lived in the Moloko region from 1999 to 2008. Many of the examples from the grammar section are taken from these stories. The sentence numbers are given in the examples so that the reader can refer to the complete texts in these chapters and locate the example in its context. The first line in each sentence is the orthographic form. The second is the phonetic form (slow speech) with morpheme breaks. The third line is the gloss and the fourth is the translation.

\section{Linguistic classification}
\hypertarget{RefHeading1210261525720847}{}
Moloko (or Melokwo, Molkore,\footnote{Molkore is the Fulfulde name for Moloko.} Məloko\footnote{Məloko is the spelling for this name using the Moloko orthography. The orthography is being used by the Moloko people (some ten titles, see reference Section). It is described in \citet{Friesen2001}.}) is classified by Gordon (2005\footnote{Dieu and \citet{Renaud1983} classify it as [154] Chadic family, Biu Mandara branch, center-west sub-branch, Wandala-Mafa group, Mafa-south sub-group (A5).}) as Central Chadic Biu-Mandara A5, as seen below:\footnote{A more detailed discussion of the classification of Moloko is found in \citet{Bow1997a}.}  

\textbf{Afro-asiatic}Berber

    Cushitic

Egyptian

Omotic

Semitic

    \textbf{Chadic}  East

      Masa

      West

\textbf{Biu-Mandara}  \textbf{A}  A1

            A2

            A3

            A4

            \textbf{A5}  Baldemu

              Cuvok

              Dugwor

              Giziga, North

              Giziga, South

              Zulgo-Gemzek

              Mafa

              Merey

              Matal

              Mefele

              Mofu, North

              Mofu-Gudur

              Vame

              \textbf{Moloko}

              Mbuko

              Muyang

              Mada

              Wuzlam

            A6

            A7

            A8

          B

          C


\begin{itemize}
\item \begin{styleFiguretitle}
Classification of Moloko
\end{styleFiguretitle}\end{itemize}

The Ethnologue (Lewis, 2009) reports 8,500 speakers of Moloko in 1992. A survey by Starr in 1997 estimated 10,000-12,000 speakers. Most live near Moloko mountain, 30km north of Maroua in the district of Tokombere, department of Mayo-Sava in the Far North Province of the Republic of Cameroon. Local oral history indicates that the Moloko people actually are not a single people group historically, but that people from at least three ethnic groups sought refuge on Moloko mountain during the Fulani invasions of the 19\textsuperscript{th} century. Eventually they all came to speak the same language. 

Moloko mountain remains the center of Moloko culture. A few people still live in three villages on the summit itself. The villages immediately surrounding Moloko mountain are organised by clan, each village being the male descendants of a particular clan and their families. Since the 1960’s, some of the Moloko language group have emigrated to the plains between the mountain and Maroua, and have settled in Moloko or Giziga-Moloko villages. Others have moved further away and live in small communities in and around the cities of Maroua, Garoua, Toubouro, Kousseri, and Yaounde. Minor dialectal differences exist in pronunciation and vocabulary but all speakers can understand one another without difficulty.   

\section{Language use, language contact, and multilingualism}
\hypertarget{RefHeading1210281525720847}{}
A minority of Moloko speakers are monolingual.  Most people speak three to five other languages.  Men and most women have at least a market level knowledge of Fulfulde, the language of wider communication, and also speak at least one of the neighboring languages Giziga, Muyang, Gemjek, Mbuko, or Dugwor.  People with several years of education also speak French.  

Men often take wives from neighboring language groups, so homes can be multilingual, but the spoken language at home tends to be the language of the father.  Friends will often switch languages as they are conversing, perhaps when talking in different domains, but also simply to bond.  Dealings in the market can be done in the trade language, but people prefer to bargain in the language of the seller, if possible.  

Language viability for Moloko is only at risk in communities where Moloko is not the primary language, especially in cities like Maroua or Yaounde.  In the city, children grow up in neighborhoods where many different languages are spoken and so they tend to speak Fulfulde (as well as learn French at school).  In such places, Moloko may be lost in the next generation.  Otherwise, in areas where Moloko people are together, Moloko language use is strong among people of every age and in every domain of home life.

\section{Previous research}
\hypertarget{RefHeading1210301525720847}{}
\citet{Bradley1992} is a dialect survey of the Moloko region from Moloko mountain to Maroua. \citet{Bow1997c} is a phonological description which included some discussion on tone. \citet{Bow1999} is an M.A. thesis which further studied the vowel system.  These two documents, along with discoveries since their work form the basis of the phonology chapter and phonology sections in the verb and noun chapters. \citet{Starr2000} is a 1500 word lexicon, and \citet{Friesen2001} an orthography.  \citet{Boyd2002} analyses lexical tone in nouns.  \citet{Boyd2001}, Oumar and \citet{Boyd2002}, Holmaka and \citet{Boyd2002}, \citet{Holmaka2002}, and \citet{Friesen2003} present interlinearised texts. \citet{Friesen2003} also presents two Moloko fables with a cultural commentary concerning each.  

Friesen and \citet{Mamalis2008} describe the Moloko verb phrase, an analysis which is reflected in this work. Prior to Friesen and Mamalis, only a few documents touched on the syntax of Moloko. Bow’s phonology statement (1997c) explored the grammar of verbs in relation to tone, and a few comparative studies of several Chadic languages included Moloko data (Rossing, 1978, Blama, 1980, and de Colombel, 1982). Rossing described Moloko noun prefixes and suffixes, plural and adjective markers, and pronouns.  He also mentioned a nominalising prefix on the verb stem that formed the nominalised form. \citet{Boyd2003} is a draft of a grammar sketch; her findings are cited where they add to this present work. 

\section{The Snake story                    \textmd{  M. Mavar}}
\hypertarget{RefHeading1210321525720847}{}
This true story was recorded in Lalaway, Far North Province of Cameroon on February 2, 2007. It was transcribed and translated by Ali Gaston and analysed by Ali Gaston and Alan Starr. 


\textbf{Setting}



\textbf{\textit{1.    Ele  ndana  ege  na,  ne  a  K}}\textbf{\textit{ose}}\textbf{\textit{wa.}}



\textit{ɛlɛ     ndana  ɛ{}-g-ɛ    na    nɛ   a   K}\textit{\textsuperscript{w}}\textit{ɔ}\textit{ʃ}\textit{ɛwa}



thing    DEM    3S-do{}-CL  PSP     1S  at  Kossewa,



‘When this thing that happened, I was [living] at Kossewa.’



\textit{2.    }\textbf{\textit{Ne  m}}\textbf{\textit{ə}}\textbf{\textit{nd}}\textbf{\textit{əye  }}\textbf{\textit{ga  elé  }}\textbf{\textit{ə}}\textbf{\textit{wla.}}



\textit{nɛ  mɪ-nd-ijɛ    ga  ɛlɛ  =uwla} 



1S  \textsc{NOM}{}-lie down-CL   ADJ   eye   =1S.POSS  



‘I was lying down.’



\textit{3.}\textbf{\textit{    }}\textbf{\textit{Ne ɗəwer ga.}}\textbf{ }



\textit{nɛ   ɗuwɛr  ga} 



1S   sleep    ADJ 



‘I was sleeping.’



\textbf{E}\textbf{pisode}\textbf{ 1}



\textit{4.}\textbf{\textit{    Alala  na}}\textit{,  }\textbf{\textit{gogolvan  na}}\textit{,  }\textbf{\textit{olo  alay.}}



\textbf{\textit{a-l=la        na}}\textit{  }\textbf{\textit{g}}\textit{\textsuperscript{w}}\textbf{\textit{ɔg}}\textbf{\textit{\textsuperscript{w}}}\textbf{\textit{ɔlvaŋ   na}}\textit{  }\textbf{\textit{ɔ{}-lɔ    =alaj}}



3S-come       PSP  snake      PSP   3S.PFV-go   =away



‘Some time later, the snake went.’



\textbf{Inciting moment}



\textit{5.   }\textbf{\textit{Acar  a  hay  kəre  ava  fo fo fo.}}



\textit{à-tsar  a  haj  kɪrɛ  ava  fɔ fɔ fɔ}



3S-climb       in   house   wood      in          \textsc{ID}sound of snake



‘It climbed into the roof of the house \textit{fofofo}.’



\textit{6.    }\textbf{\textit{Sen  ala  na}}\textit{,}\textbf{\textit{  okfom  adaɗ  ala   ɓav!}}



\textit{ʃɛŋ    =ala     na  ɔk}\textit{\textsuperscript{w}}\textit{fɔm   à-dəɗ        =ala   ɓav}



\textsc{ID}go   =to      PSP   mouse  3S.PFV-fall  =to     \textsc{ID}sound of falling 



‘And walking, a mouse fell \textit{bav}!’



\textit{7.    }\textbf{\textit{Ne  awəy,  “Alma  amədəv  ala  okfom  nehe  may?”}}



\textit{nɛ   awij    alma   amə-dəv=ala     ɔk}\textit{\textsuperscript{w}}\textit{fɔm   nɛhɛ    maj}



1S    said     what   DEP-fall=to     mouse    DEM  what 



‘[I woke up]  I said [to myself], “What made that mouse fall?\textit{”}’



\textbf{Peak episode}



\textit{8.    }\textbf{\textit{Mbaɗala  ehe  na,}}\textbf{  }\textbf{\textit{nabay  oko}}\textbf{,       }



\textit{mbaɗala   ɛhɛ   na}     \textit{nà-b-aj              ɔk}\textit{\textsuperscript{w}}\textit{ɔ}  



 then       here   PSP  1S.PFV-light{}-CL      fire 



‘Then, I turned on a light,’



\textit{9.    }\textbf{\textit{nazaɗ  ala  təystəlam  əwla,}}



\textit{nà-zaɗ        =ala   tijstəlam  =uwla}



1S.PFV-take  =to     torch   \textit{       }=1S.POSS 



‘I took my flashlight,’



\textit{10.   }\textbf{\textit{nabay  cəzlar.                }}\textit{  }



\textit{nà-b-aj            tsəɮar            }



1S.PFV-light{}-CL       \textsc{ID}shining the flashlight up



‘I shone it up \textit{tsilar}.’



\textit{11.}\textbf{\textit{   Nábay  na,      }}



\textit{ná-b-aj     na} 



1S-light{}-CL  PSP   



‘[As] I shone [it],’



\textbf{\textit{námənjar  na,  mbajak  mbajak  mbajak  gogolvan!}}



\textit{ná-mənzar     na     mbadzak   mbadzak    mbadzak  g}\textit{\textsuperscript{w}}\textit{ɔg}\textit{\textsuperscript{w}}\textit{ɔlvaŋ}



1S.IFV-see  PSP    \textsc{ID}something big and reflective   snake



‘I was seeing, something big and reflective, a snake!’



\textit{12.   }\textbf{\textit{Ne  awəy,    “A,  enen  baj  na,  memey  na!”}}



\textit{nɛ   awij    a          ɛnɛŋ       baj     na  mɛmɛj   na}



1S   say    EXCL   snake  \textsc{NEG}   PSP  how       PSP 



‘I said to myself, “Wah! It’s a snake!”’ (lit. a snake, if not, how)



\textit{13.   }\textbf{\textit{Ne  mbət  məmbete  oko  əwla  na,}}



\textit{nɛ   mbət           m}\textit{ɪ}\textit{{}-mbɛt-ɛ           ɔk}\textit{\textsuperscript{w}}\textit{ɔ     =uwla        na}



1S    \textsc{ID}turn off  \textsc{NOM}{}-turn\_off-CL  light    =1S.POSS  PSP 



\textbf{‘}I turned off my light,\textbf{’}



\textbf{\textit{kaləw  nazaɗ}}\textbf{  }\textbf{\textit{ala  ɛɮɛrɛ  =uwla.}}



\textit{kàluw               nà-zaɗ}        =\textit{ala  ɛɮɛrɛ  =uwla}



\textsc{ID}take quickly  1S.PFV-take=to spear     =1S.POSS 



‘[and] quickly took my spear.’



\textit{14.  }\textbf{\textit{Mək  ava  alay,   }}



\textit{mək         =ava   =alaj   }



\textsc{ID}positioning self for throwing   =in    =to 



‘[I] positioned [myself] \textit{muk}!’



\textit{15.   }\textbf{\textit{Mecesle  mbəraɓ! }}



\textit{mɛ-tʃɛɬ-ɛ            mbəraɓ}



\textsc{NOM}{}-penetrate-CL      \textsc{ID}penetrate 



‘It penetrated, \textit{mburab}!’



\textit{16.}\textbf{\textit{  Ele  a  Hərmbəlom  ele  ga  ajənaw  ete }}\textit{ }



\textit{ɛlɛ     a         Hʊrmbʊlɔm  ɛlɛ     ga      à-dzən    =aw    ɛtɛ  }



thing   GEN   God    thing  ADJ   3S.PFV-help  =1S.IO    also 



‘It was by chance and with God’s help’



\textbf{\textit{kəl kəl   kə ndahaŋ aka}}



\textit{kəl kəl   kə ndahaŋ aka}



exactly  on  3S        on 



‘[that the spear went] exactly on him.’



\textit{17.  }\textbf{\textit{Ádəɗala  }}\textbf{\textit{ⱱ}}\textbf{\textit{aɓ  a  wəyen  ava.}}



\textit{á-dəɗ           =ala     }\textit{ⱱ}\textit{aɓ         a     wijɛŋ      ava}



3S.IFV-fall  =to      \textsc{ID}falling    on    ground   on 



‘He fell on the ground \textit{vab}.’



\textit{18.  }\textbf{\textit{Ne dəy day  məkəɗe  na  aka}}



\textit{nɛ dij daj              mɪ-kɪɗ-ɛ      na      =aka} 



1S  approximately     \textsc{NOM}{}-kill-CL  3S.DO   =on



‘I clubbed it to death (approximately).’



\textbf{D}\textbf{énouement}



\textit{19.  }\textbf{\textit{hor  əwla   olo  alay  awəy  egege,      }}\textit{      }



\textit{h}\textit{\textsuperscript{w}}\textit{ɔr     =uwla      ɔ{}-lɔ   =alaj            awij ɛgɛgɛ}



woman =1S.POSS 3S-go =to           she said that 



‘My wife went and said,’



\textbf{\textit{“A a  nəngehe  na,  Hərmbəlom  aloko  ehe.    }}



\textit{a a     nɪŋgɛhɛ   na   Hʊrmbʊlɔm   =alɔk}\textit{\textsuperscript{w}}\textit{ɔ        ɛhɛ    }



EXC   DEM     PSP  God\textit{         =}1\textsc{Pin}.POSS   here  



‘“Wah! This one here, our God [is] really here [with us].’



\textbf{\textit{Bəyna  anzakay  nok}}\textit{\textsuperscript{  }}\textbf{\textit{ha  a  slam  məndəje  ango  ava,}}



\textit{bijna      à-nzak-aj          nɔk}\textit{\textsuperscript{w   }}\textit{ha      a    ɬam      mɪ-nd-ijɛ        aŋg}\textit{\textsuperscript{w}}\textit{ɔ  ava}



because   3S.PFV-find{}-CL  2S      until  to      place  \textsc{NOM}{}-sleep-CL  2S       in



‘Because it found you even in your bed.’ (lit. all the way at your place of lying)



\textbf{\textit{“Alala  Hərmbəlom  ajənok  na,  səwse   Hərmbəlom.  }}



\textbf{\textit{a-l  =ala   Hʊrmbʊlɔm    à-dzən    =ɔk}}\textit{\textsuperscript{w}}\textbf{\textit{     na   ʃuwʃɛ   Hʊrmbʊlɔm  }}



3S-go=to  God         3S.PFV-help  =2S.IO  PSP  thanks   God \textit{   }



“And then God helped you; thanks [be to] God!”’



\textit{20.  }\textbf{\textit{Hor  əwla  ahaw  kəygehe.}}



\textit{h}\textit{\textsuperscript{w}}\textit{ɔr   =uwla     à-h=aw                      kijgɛhɛ}



woman  =1S.POSS    3S.PFV-tell=1S.IO     like that



‘My wife said it like that.’



\textit{21.}\textbf{\textit{  Alala,  nəzlərav  na ala  gogolvan  na   a  amata  ava.}}



\textit{a-l-ala    nə-ɮərav         na       =ala  g}\textit{\textsuperscript{w}}\textit{ɔg}\textit{\textsuperscript{w}}\textit{ɔlvaŋ  na  a  amata  ava}



sometime later  1S.PFV-exit  3S.DO   =to  snake        PSP    in   outside  in



‘Sometime later I took the snake outside.’ (lit. it came, I caused the snake to exit outside)



\textit{22.}\textit{  }\textbf{\textit{Ko  dedew  babəza  əwla  ahay  aməzləravala  amata  na,            tawəy,  }}



\textit{k}\textit{\textsuperscript{w}}\textit{ɔ     dɛdɛw     babəza  =uwla       =ahaj  amə-ɮərav =ala  amata      na    tawij  }



early   morning  child      =1S.POSS   =Pl  \textsc{NOM}{}-exit  =to     outside   PSP    3P+say



‘Early the next morning, when my children came outside, said,’



\textbf{\textit{“Baba  áka}}\textbf{\textit{ɗ}}\textbf{\textit{  gogolvan,  baba   áka}}\textbf{\textit{ɗ}}\textbf{\textit{  gogolvan!”}}



\textit{baba     á-ka}\textit{ɗ}\textit{             g}\textit{\textsuperscript{w}}\textit{ɔg}\textit{\textsuperscript{w}}\textit{ɔlvaŋ  baba   á-ka}\textit{ɗ}\textit{             g}\textit{\textsuperscript{w}}\textit{ɔg}\textit{\textsuperscript{w}}\textit{ɔlvaŋ}



father   3S.IFV-kill      snake    father   3S.IFV-kill      snake



 “Papa killed a snake, Papa killed a snake!”



\textit{23}\textit{.}\textbf{\textit{  Tájaka  kəygehe.}}



\textit{tá-dz          =aka   kijgɛhɛ}



3P+I\={ }FV{}-say  =on  like that



‘They said it like that.’



\textbf{Conclusion}



\textit{24.}\textbf{\textit{  }}\textbf{\textit{Ka  nehe  ləbara  a  ma  ndana  }}\textbf{\textit{ɗ}}\textbf{\textit{əwge.     }}



\textit{ka  nɛhɛ  ləbara  a       ma      ndana  }\textit{ɗ}\textit{uwgɛ     }



like  DEM  news    GEN  word  DEM      actual



‘And so was that story.’


\section{The Disobedient Girl story                \textmd{  Sali Anouldéo Justin}}
\hypertarget{RefHeading1210341525720847}{}
This fable was recorded in Lalaway, Far North Province of Cameroon in 2002. It was transcribed and translated by Sambo Joël, and edited by the Moloko translation committee.


\textbf{Setting}



\textit{1.    }\textbf{\textit{Bamba  bamba  k}}\textbf{\textit{ə}}\textbf{\textit{lo  d}}\textbf{\textit{ə}}\textbf{\textit{rgoɗ !}}



\textit{bamba   bamba,   kʊlɔ     d}\textit{ʊ}\textit{rg}\textit{\textsuperscript{w}}\textit{ɔɗ}



story        story     under    silo



‘Once upon a time…’ (lit. there’s a story under the silo)



2.    \textbf{\textit{Taw}}\textbf{\textit{əy}}\textbf{\textit{  ab}}\textbf{\textit{əy}}\textbf{\textit{a,  ma  bamba  a  war  dalay  cezlere  ga.}}



\textit{tawij  abija,    ma   bamba   a       war     dalaj      t}\textit{ʃ}\textit{ɛɮɛrɛ        ga}



3P+say  saying   word   story   GEN   child   female  disobedience   ADJ



       ‘They say,~the story of the disobedient girl [goes like this:]’



3.    \textbf{\textit{Zlezle  na,  }}\textbf{\textit{M}}\textbf{\textit{ə}}\textbf{\textit{loko  ahay  na,  }}\textbf{\textit{H}}\textbf{\textit{ə}}\textbf{\textit{rmb}}\textbf{\textit{ə}}\textbf{\textit{lom  ávəlata  barka  va.}}



\textit{ɮ}\textit{ɛ}\textit{ɮ}\textit{ɛ}\textit{        na  }\textit{Mʊlɔk}\textit{\textsuperscript{w}}\textit{ɔ  =ahaj     na  }\textit{Hʊrmbʊlɔm  á-vəl=ata     barka    =va}



long ago         PSP  Moloko   =Pl          PSP  God       3S.IFV-send=3S.IO   blessing  =\textsc{PRF}  



‘Long ago, to the Moloko people, God had given his blessing.’



\textit{4.    }\textbf{\textit{K}}\textbf{\textit{ə}}\textbf{\textit{waya  asa  təwas  va  neken  k}}\textbf{\textit{əy}}\textbf{\textit{gehe  ɗ}}\textbf{\textit{ə}}\textbf{\textit{w,}}



\textit{kuwaja   asa    tə-was                =va  nɛk}\textit{\textsuperscript{w}}\textit{ɛŋ    kijgɛhɛ     ɗuw,}



that is   if     3P.PFV-cultivate  =\textsc{PRF}  little       like this    also



       ‘‘That is, even if they had only sowed a little [millet] like this,’



\textbf{\textit{ávata  m}}\textbf{\textit{ə}}\textbf{\textit{v}}\textbf{\textit{əye}}\textbf{\textit{  haɗa.  }}



\textit{á-v    =ata  mɪ-v-ijɛ         haɗa  }



3S.IFV-pass  =3P.IO   \textsc{NOM}{}-pass(year) -CL  a lot



‘it would last them enough for the whole year.’  



5.    \textbf{\textit{Aməhaya  kə  ver  aka  na,  tázaɗ  war  elé  hay  bəlen,}}



\textit{amə-h  =aja   kə   vɛr   aka  na  tá-zaɗ       war   ɛlɛ  haj  bɪlɛŋ}



DEP-grind  =\textsc{PLU}  on   stone   on    PSP  3S.IFV-grind  child   eye   millet    one



‘For grinding on the grinding stone, they would take one grain of millet.’



6.    \textbf{\textit{Nde, }}\textbf{ }\textbf{\textit{asa}}\textbf{\textit{  t}}\textit{ə}\textbf{\textit{nday  táhaya  na  na}}\textit{,}\textbf{\textit{    }}



\textit{ndɛ}  \textit{asa}\textit{  t}\textit{ə}\textit{{}-ndaj    tá-h  =aja      na  na}



so  if     3P.IFV-PRG  3P.IFV-grind=\textsc{PLU}  3S.DO  PSP



       ‘So, whenever they were grinding it,’



\textbf{\textit{həmbo  na}}\textit{,  }\textbf{\textit{ásak  nə  məsəke.}}



\textit{hʊmbɔ   }\textit{na}  \textit{á-sak             nə        mɪ-}\textit{ʃ}\textit{ɪk-ɛ}



flour     PSP  3S.IFV-multiply  with  \textsc{NOM}{}-multiply-CL



‘the flour, it multiplied with multiplying.’ 



7.    \textbf{\textit{War  elé  hay  bəlen  fan  na}}\textit{,}



\textit{war     ɛlɛ   haj   bɪlɛŋ   faŋ   na}



child   eye   millet   one    yet   PSP



‘Just one grain of millet,’



\textbf{\textit{ájata   pɛw  ha  ámbaɗ  ɛʃɛ.}}



\textit{á-nz=ata                    pɛw   ha  á-mbaɗ    ɛʃɛ}



3S.IFV-suffice=3P.IO       enough   until     3S.IFV-remain   again



‘it sufficed for them, and there were leftovers.’



8.    \textbf{\textit{Waya  a  məhaya  ahan  ava  na}}\textit{,}



      \textit{waja   a  mə-h    =aja       =ahaŋ      ava    na}



       because  in   \textsc{NOM}{}-grind  =\textsc{PLU}   =3S.POSS   in   PSP



      ‘Because, during its grinding,’



     \textbf{\textit{á}}\textbf{\textit{sak  kə  ver  aka  nə  məsəke.}}



     \textit{á}\textit{{}-sak                    kə   vɛr               aka   nə   mɪ-ʃɪk-ɛ}



     3S.IFV-multiply   on     grinding stone     on    with   \textsc{NOM}{}-multiply-CL



     ‘it would multiply on the grinding stone with multiplying.’  



\textbf{Episode 1}



9.    \textbf{\textit{Nde  ehe  na,    albaya  ava  aba.        }}



\textit{ndɛ    ɛhɛ  }\textit{na}\textit{  albaja       ava  aba        }



so       here    PSP  young.man  \textsc{EXT}{}-in   \textsc{EXT}                                                            



‘And so, there once was a young  man.’



10.    \textbf{\textit{Olo}}\textbf{\textit{  azala  dalay}}



         \textit{à-lɔ            à-z    =ala  dalaj}



         3S.PFV-go    3S.PFV-take  =to  girl



  ‘He went and took a wife.’



11.   \textbf{\textit{Azla}}\textbf{ }\textbf{\textit{ na,}}\textbf{  }\textbf{\textit{war  dalay  ndana,  }}\textbf{\textit{cezlere  ga.}}



\textit{aɮa}        na  \textit{war   dalaj   ndana}   \textit{t}\textit{ʃ}\textit{ɛɮɛrɛ          ga}



now      PSP  child      female    DEM  disobedience ADJ



‘Now, that young girl was disobedient.’



12.   \textbf{\textit{Sen  }}\textbf{\textit{ala  na}}\textit{,  }\textbf{\textit{zar  ahan  na}}\textit{,}



\textit{ʃ}\textit{ɛŋ     =ala   }\textit{na}\textit{  zar  =ahaŋ    }\textit{na}



\textsc{ID}go   =to          PSP  man    =3S.POSS    PSP



‘Then her husband’



\textbf{\textit{dək  medakan  na,  mənjəye  ata.}}



\textit{dək    mɛ-dak=aŋ            na        mɪ-nʒ-ijɛ     =ata}



\textsc{ID}show   \textsc{NOM}show=3S.IO   3S.DO   \textsc{NOM}{}-sit-CL   =3P.POSS



‘instructed her in their habits.’(lit. instructing their ways)



13.  \textbf{\textit{Awəy}}\textbf{ , “}\textbf{\textit{Hor}}\textbf{\textit{  }}\textbf{\textit{golo,  afa  ləme  na}}\textit{,}



\textit{awij}    \textit{h}\textit{\textsuperscript{w}}\textit{ɔr}\textit{   }\textit{g}\textit{\textsuperscript{w}}\textit{ɔlɔ,}\textit{   }\textit{afa      lɪmɛ   }\textit{na}



he said   woman  VOC  at place of    1\textsc{Pex}     PSP



‘He said, “My dear wife, here at our (not your) place,



\textbf{\textit{m}}\textbf{\textit{ə}}\textbf{\textit{n}}\textbf{\textit{j}}\textbf{\textit{əye}}\textbf{\textit{  al}}\textbf{\textit{ə}}\textbf{\textit{me  na}}\textit{,}\textbf{\textit{  }}\textbf{\textit{k}}\textbf{\textit{əy}}\textbf{\textit{gehe.}}



\textit{mɪ-n}\textit{ʒ}\textit{{}-ijɛ    =alɪmɛ     na  }\textit{kijgɛhɛ}



\textsc{NOM}{}-sit-CL   =2\textsc{Pex}.POSS    PSP  like this



‘it is like this.’



\textbf{\textit{Asa  asok  aməhaya  na,}}



\textit{asa   à-s=ɔk}\textit{\textsuperscript{w}}\textit{      amə-h    =aja     }\textit{na}\textit{ }



if     3S.PFV-please=2S.IO  DEP-grind  =\textsc{PLU}   PSP



‘If you want to grind,’



\textbf{\textit{k}}\textbf{\textit{á}}\textbf{\textit{zaɗ  war  elé  háy  bəlen.}}



\textit{k}\textit{á}\textit{{}-zaɗ  war     ɛlɛ      haj       bɪlɛŋ}



2S.IFV-take  child  eye  millet  one



‘you take only one grain.’



\textbf{\textit{War  elé  háy  bəlen  ga  nəndəye  nok  amezəɗe  na}}\textit{,}



\textit{war    ɛlɛ      haj     bɪlɛŋ  ga   nɪndijɛ  nɔk}\textit{\textsuperscript{w}}\textit{   amɛ-}\textit{ʒ}\textit{ɪɗ{}-ɛ   }\textit{na}



child     eye  millet  one      ADJ  DEM    2S    DEP-take-CL      PSP



‘That one grain that you have taken,’



\textbf{\textit{k}}\textbf{\textit{á}}\textbf{\textit{haya  na  kə  ver  aka.}}



\textit{k}\textit{á}\textit{{}-h    =aja   na       kə  vɛr        aka}



2S.IFV-grind  =\textsc{PLU}  3S.DO  on   grinding stone  on



‘grind it on the grinding stone,’



\textbf{\textit{Á}}\textbf{\textit{njaloko  de  pew. }}



\textit{á}\textit{{}-nz      =alɔk}\textit{\textsuperscript{w}}\textit{ɔ             dɛ     pɛw }



3S.IFV-suffice  =1\textsc{Pin}.IO    enough  finished



‘It will suffice for all of us just enough.’ 



\textbf{\textit{Ádaloko  ha  ámb}}\textbf{\textit{a}}\textbf{\textit{ɗ  ese.}}



\textit{á-d      =alɔk}\textit{\textsuperscript{w}}\textit{ɔ               ha      á-mbəɗ            ɛ}\textit{ʃ}\textit{ɛ}



3S.IFV-prepare  =1\textsc{Pin}.IO  until   3S.IFV-left over  again



‘It will make food for all of us, until there is some left over.’



\textbf{\textit{Waya}}\textbf{\textit{  a  mə-haya  ahan  ava  na,}}



\textit{waja   a    mə-h=aja                =ahaŋ     ava    }\textit{na}



because  at   \textsc{NOM}{}-grind=\textsc{PLU}     =3S.POSS   in   PSP



‘because, during the grinding\textit{,}’



\textbf{\textit{H}}\textbf{\textit{ə}}\textbf{\textit{rmb}}\textbf{\textit{ə}}\textbf{\textit{lom  anday  ásakal}}\textbf{\textit{ə}}\textbf{\textit{me  na  aka.}}



\textit{Hʊrmbʊlɔm   a-ndaj    á-sak    =alɪmɛ     na  aka}



God                3S-PRG  3S.IFV-multiply  =2\textsc{Pex}.IO  3S.DO  on 



‘God is multiplying it for us.”’



14.   \textbf{\textit{Hor  na,  }}\textbf{\textit{ambəɗan  aka,   }}



\textit{h}\textit{\textsuperscript{w}}\textit{ɔr           na,  }\textit{a-mbəɗ=aŋ        =aka   }



woman            PSP  3S-change-3S.IO    =on



‘She replied,’



15.  \textbf{\textit{Aw}}\textbf{\textit{əy,}}\textbf{\textit{  “A}}\textbf{\textit{yo}}\textbf{\textit{kon  zar  golo.”}}



\textit{awij    }\textit{aj}\textit{ɔk}\textit{\textsuperscript{w}}\textit{ɔŋ   zar  g}\textit{\textsuperscript{w}}\textit{ɔlɔ} 



3S+say   agreed\textit{   }man  VOC



‘saying, “Yes, my dear husband.”’~



\textbf{Episode 2}



16.  \textbf{\textit{Ndahan  amandava  ɓəl  na}}\textit{,  }\textbf{\textit{zar  ahan  olo  }}



\textit{ndahaŋ  ama-nd    =ava   ɓəl   }\textit{na}\textit{  zar  =ahaŋ      ɔ{}-lɔ     }



3S         DEP-sleep  =in    \textsc{ID}some  PSP  man  =3S.POSS  3S.PFV-go  



‘She, sleeping there for some time, her husband went away’



\textbf{\textit{ametele  kə  dəlmete  ahan  aka  a  slam  enen.}}



\textit{amɛ-tɛl-ɛ     kə  dɪlmɛtɛ   =ahaŋ      aka   a  ɬam  ɛnɛŋ}



DEP-walk-CL  on   neighbor  =3S.POSS  on  at place another



‘to walk~in the neighborhood to some place.’ 



17.  \textbf{\textit{Azla  na,  hor  na,  asərkala  afa  təta  va  na,}}



\textit{aɮa      na  h}\textit{\textsuperscript{w}}\textit{ɔr   }\textit{na}\textit{  à-sərk            =ala     afa            təta  =va      na}



now      PSP  woman  PSP  3S.PFV-HAB  =to     at house of  3P    =\textsc{PRF}  PSP



‘Now, that woman, she was in the habit at their place’



\textbf{\textit{aməhaya  háj  na  gam.}}



\textit{amə-h  =aja      haj        na  gam}



DEP-grind  =\textsc{PLU}  millet  PSP    a lot



‘of grinding a lot of millet.’  



18.  \textbf{\textit{Ndahan  jo  madala  háy  na}}\textit{,}\textbf{\textit{           }}



\textit{ndahaŋ    dzɔ      ma-d    =ala      háj  na          }



3S         \textsc{ID}take   \textsc{NOM}{}-prepare  =to    millet   PSP



‘After having gotten ready to grind (she taking millet),’



\textbf{\textit{Ɗen  bəlen  tə  kə  ver  aka,}}



\textit{ɗɛŋ        bɪlɛŋ  tə               kə  vɛr      aka,}



\textsc{ID}put   one      \textsc{ID}put one  on      stone  on



‘[she put] one grain on the grinding stone.’



19.  \textbf{\textit{Awəy  }}\textbf{\textit{g}}\textbf{\textit{ə}}\textbf{\textit{lo  ahay  }}\textbf{\textit{nehe  azla  na}}\textit{,}\textbf{\textit{  malmay  nəngehe  na  may?  }}



\textit{awij}   \textit{gʊlɔ  =ahaj}\textit{  nɛhɛ     aɮa     na  malmaj  nɪŋgɛhɛ  na  maj  }



3S+say  fellow  =Pl   DEM    now   PSP    what      DEM      PSP    what



‘She said, “Friends, here, what is this?’



\textbf{\textit{Háy  bəlen  azla  na}}\textit{,  }\textbf{\textit{náambəzaka  məhaya  əwla  na}}\textit{,}



\textit{haj    bɪlɛŋ   aɮa   na,  náá-mbəz  =aka   mə-h      =aja    =uwla   }\textit{na}



millet   one        now   PSP  1S.POT-ruin  =on  \textsc{NOM}{}-grind =\textsc{PLU} =1S.POSS  PSP



‘One grain, [with it] I know I will ruin my grinding.’



\textbf{\textit{Meme  ege  mey?”}}



\textit{mɛmɛ   ɛ{}-g-ɛ      mɛj}



how       3S-do-CL  how



‘What is happening?’ (What should I do?)



\textbf{\textit{Nehe  na}}\textit{,}\textbf{\textit{  məseɓete  hərav  əwla  ɗaw?}}



\textit{nɛhɛ  }\textit{na}\textit{  mɪ-}\textit{ʃ}\textit{ɛɓɛt-ɛ    hərav  =uwla    ɗaw}



DEM             PSP  \textsc{NOM}{}-deceive-CL  body  =1S.POSS  QUEST



‘This, am I deceiving my body?’ 



\textbf{\textit{B}}\textbf{\textit{ə}}\textbf{\textit{y  na}}\textit{,}\textbf{\textit{  }}\textbf{\textit{malmay?}}\textbf{\textit{ }}



\textit{bij      na  }\textit{malmaj}



\textsc{NEG}        PSP  what~



‘If not, what is it then?’



\textbf{\textit{Aya  jen  }}\textbf{\textit{ele  ahay  nend}}\textbf{\textit{əye}}\textbf{\textit{  na}}\textit{,  }\textbf{\textit{nagala  k}}\textbf{\textit{əy}}\textbf{\textit{ga  bay.”    }}



\textit{aja   dʒ}\textit{ɛ}\textit{ŋ  }\textit{ɛlɛ  =ahaj  nɛndijɛ  }\textit{na}\textit{  }\textit{nà-g    =ala  kijga  baj    }



so  chance  thing   =Pl       DEM     PSP  1S.PFV-do  =to  like this  \textsc{NEG}           



‘Above all, these things, I have never done like this.”’



\textbf{Peak Episode}



20.  \textbf{\textit{Jo  madala  }}\textbf{\textit{hay  na}}\textit{,}\textbf{\textit{  gam.  }}



\textit{dzɔ    ma-d    =ala  haj        }\textit{na}\textit{  gam}



\textsc{ID} take  \textsc{NOM}prepare  =to  millet   PSP    a lot



‘Millet was prepared, lots.’



21.  \textbf{\textit{Ndahan  bah  məbehe  háy  ahan  }}



\textit{ndahaŋ  bax        mɪ-bɛh-ɛ       haj       =ahaŋ         }



3S          \textsc{ID}pour  \textsc{NOM}{}-pour-CL  millet  =3S.POSS  



‘She poured her millet.’



\textbf{\textit{amadala  na  kə  ver  aka  azla.}}



\textit{ama-d  =ala     na        kə  vɛr     aka  aɮa}



DEP-prepare  =to   3S.DO  on   stone  on   now



‘to prepare it on the grinding stone.’



22.  \textbf{\textit{Njəw  njəw  njəw  aməhaya  azla.}}



\textit{nzuw  nzuw  nzuw  amə-h    =aja  aɮa}



\textsc{ID}grind                        DEP-grind  =\textsc{PLU}   now



‘\textit{Nzu}, \textit{nzu}, \textit{nzu} [she] ground [the millet] now.’   



23.  \textbf{\textit{H}}\textbf{\textit{ə}}\textbf{\textit{mbo  na  ɗ}}\textbf{\textit{ə}}\textbf{\textit{w,  }}\textbf{\textit{anday  ásak   ásak  ásak.}}



\textit{hʊmbɔ  na   ɗuw,}  \textit{à-ndaj            á-sak     á-sak     á-sak}



flour  PSP  also   3S.PFV-PRG   3S.IFV-multiply  3S.IFV-multiply 3S.IFV-multiply 



‘The flour, it is multiplying [and] multiplying [and] multiplying.’



24.  \textbf{\textit{Ndahan  na,  ndahan  aka  njəw  njəw  njəw.      }}



\textit{ndahaŋ   }\textit{na}\textit{      ndahaŋ  aka       nzuw  nzuw  nzuw      }



3S                PSP  3S       \textsc{EXT}  \textsc{ID}{}-grind



‘And she/it, she/it is grinding some more \textit{nzu}, \textit{nzu}, \textit{nzu}.’



25.  \textbf{\textit{Anday  ahaya  nə  məzere  ləmes  ga.}}



\textit{à-ndaj             à-h    =aja   nə  mɪ-}\textit{ʒ}\textit{ɛr-ɛ         lɪmɛʃ  ga}



3S.PFV-PRG  3S.IFV-grind  =\textsc{PLU}  with  \textsc{NOM}{}-do\_well-CL  song     ADJ



‘She/it is grinding while singing well.’ 



26.  \textbf{\textit{Alala  na}}\textit{,  }\textbf{\textit{ver  na}}\textit{,  }\textbf{\textit{árəh  mbaf,  nə  həmbo  na}}\textit{,}



\textit{a-l=ala      }\textit{na}\textit{  vɛr     }\textit{na}\textit{  á-rəx       mbaf,    nə   hʊmbɔ  }\textit{na}



3S-go=to   PSP   room     PSP  3S.IFV-fill      up to the roof     with    flour  PSP



‘After a while, the room, it filled up to the roof with the flour,’



\textbf{\textit{ɗək  mə-ɗəkaka  alay  ana  hor  na}}\textit{,}



\textit{ɗək         mə-ɗək    =aka    =alaj   ana      h}\textit{\textsuperscript{w}}\textit{ɔr     }\textit{na}



\textsc{ID}{}-stuff   \textsc{NOM}{}-plug  =on    =away   DAT   woman   PSP



‘[The flour] plugged the room for the woman [so there was no place for her to even breathe],’



\textbf{\textit{nata  }}\textbf{\textit{ndahan  d}}\textbf{\textit{ə}}\textbf{\textit{ɓ}}\textbf{\textit{ə}}\textbf{\textit{sol}}\textbf{\textit{ə}}\textbf{\textit{k  məmətava  alay  }}



\textit{nata   }\textit{ndahaŋ  dʊɓʊsɔlʊk}\textit{\textsuperscript{w}}\textit{      mə-mət  =ava  =alaj  }



and then  3S           \textsc{ID}collapse/die    \textsc{NOM}{}-die=in   =away   



‘and she collapsed \textit{dubusoluk}, dying’



\textbf{\textit{a  hoɗ  a  haj  na  ava.}}



\textit{a    h}\textit{\textsuperscript{w}}\textit{ɔɗ  a  haj  na  ava}



in    stomach  GEN   house   PSP    in



‘inside the house.’



\textbf{Dénouement}



27.\textbf{  }\textbf{\textit{Embesen  cacapa  na,  zar  ahan  angala.}}



\textit{ɛ{}-mbɛʃɛŋ    tsatsapa    }\textit{na}\textit{   zar   =ahaŋ         à-ŋgala}



3S-rest       some time       PSP  man    =3S.POSS  3S.PFV-return



‘After a while, her husband came back.’



28.  \textbf{\textit{Pok  mapalay  mahay  na,}}



\textit{pɔk}\textit{\textsuperscript{w}}\textit{        ma-p    =alaj  mahaj   }\textit{na}



\textsc{ID}open  \textsc{NOM}{}-open  =away  door   PSP



‘When he opened the door, (lit. opening the door \textit{pok})



\textbf{\textit{həmbo  árah  na  a  hoɗ  a  hay  ava.}}



\textit{hʊmbɔ  á-rax           na      a     h}\textit{\textsuperscript{w}}\textit{ɔɗ       a        haj     ava}



flour     3S.IFV-fill   3S.DO   at   stomach  GEN   house     in



‘the flour filled the stomach (the interior) of the house.’



29.  \textbf{\textit{Ndahan  aməmən}}\textbf{\textit{jere}}\textbf{\textit{   ele  nendəye  na,  }}\textbf{\textit{aw}}\textbf{\textit{əy,}}



\textit{ndahaŋ  amɪ-}\textit{mɪnʒɛr}\textit{{}-ɛ   ɛlɛ    nɛndijɛ  }\textit{na}\textit{  }\textit{awij}



3S           DEP-see-CL        thing     DEM   PSP  3S-saying



‘He, seeing the things, he said,’



\textbf{\textit{“Aw  aw  aw,  hor  ngehe  na,  }}\textbf{\textit{acaw  aka  va  }}



\textit{aw aw aw  h}\textit{\textsuperscript{w}}\textit{ɔr      ŋgɛhɛ   }\textit{na  à-ts      =aw   =aka =va}



cry of death  woman  DEM      PSP  3S.PFV-understand  =1S.IO  =on    =\textsc{PRF}



‘“Ah, this woman, today, she didn’t listen’



\textbf{\textit{ma  }}\textbf{\textit{ə}}\textbf{\textit{wla  amahan  na  bay  es}}\textbf{\textit{ə}}\textbf{\textit{mey?}}



\textit{ma    =uwla      ama-h    =aŋ  na  baj    ɛʃɪmɛj}



word  =1S.POSS  DEP-speak  =3S.IO   3S.DO   \textsc{NEG}   not so



‘to my instructions, did she?’ 



\textbf{\textit{Agə  na  va  ele  ne  amahan  aməjəye  }}



\textit{à-gə              }\textit{na         =va       ɛlɛ      nɛ    ama-h  =aŋ  amɪ-dʒ-ijɛ  }



3S.PFV-do  3S.DO  =\textsc{PRF}  thing  1S  DEP-say  =3S.IO  DEP-say-CL      



‘She has done the thing that I told her’



\textbf{\textit{mege  bay  na  esəmey?}}



\textit{mɛ-g-ɛ            baj  na     ɛʃɪmɛj}



3S.HOR-do-CL  \textsc{NEG}   PSP  not so



‘she should not do, not so?’



\textbf{\textit{Nde  }}\textbf{\textit{nége  ehe  na}}\textit{,}\textbf{\textit{  memey  g}}\textbf{\textit{ə}}\textbf{\textit{lo  ahay?” }}



\textit{ndɛ   }\textit{nɛ-g-ɛ    ɛhɛ     na     mɛmɛj  gʊlɔ  =ahaj }



so  1S.IFV-do{}-CL  here   PSP  how     friend   =Pl



‘So, what can I do here, my friends?”’



30.  \textbf{\textit{Kəlen  }}\textbf{\textit{tazlərav   na  ala.    }}



\textit{kɪlɛŋ  }\textit{tà-ɮərav          na       =ala    }



then    3P.PFV-exit  3S.DO  =to 



‘Then, they took her out of the house.’



31.  \textbf{\textit{Babək  mələye  na.}}



\textit{babək   mɪ-l-ijɛ        na}



\textsc{ID}bury  \textsc{NOM}bury-CL  3S.DO



‘She was buried.’ 



\textbf{Conclusion}



32.  \textbf{\textit{Nde  ko  ala  a  ɗəma  ndana  ava  pew!}}



\textit{ndɛ    k}\textit{\textsuperscript{w}}\textit{ɔ       =ala  a    ɗəma   ndana    ava      pɛw}



so    until  =to    at    time      DEM    in    enough



‘So, ever since that time, it’s done!’



33.  \textbf{\textit{M}}\textbf{\textit{ə}}\textbf{\textit{loko  ahay  t}}\textbf{\textit{a}}\textbf{\textit{w}}\textbf{\textit{əy,  Hərmbəlom  }}\textbf{\textit{ága  ɓərav  va  }}



\textit{Mʊlɔk}\textit{\textsuperscript{w}}\textit{ɔ  =ahaj  t}\textit{a}\textit{wij  }\textit{Hʊrmbʊlɔm     }\textit{á-g-a      ɓərav   =va  }



Moloko     =Pl       3P+say  God        3S.IFV-do   heart    =\textsc{PRF}     



‘The Molokos say, God got angry’ (lit. God did heart)



\textbf{\textit{kəwaya  war  dalay  na,  amecen  sləmay  bay  ngəndəye.}}



\textit{kuwaja        war    dalaj     na}   \textit{amɛ-tʃɛŋ      ɬəmaj  baj     ŋgɪndijɛ}



because of  child    girl    PSP  DEP-hear   ear      \textsc{NEG}  DEM



‘because of that girl, that one that was disobedient.’



34.  \textbf{\textit{Waya  ndana  Hərmbəlom  ázata  aka  barka  ahan  va.}}



\textit{waja   ndana  Hʊrmbʊlɔm   á-z    =ata      =aka   barka     =ahaŋ    =va }



because   DEM   God             3S.IFV-take  =3P.IO  =on   blessing  =3S.POSS  =\textsc{PRF}



‘Because of that, God had taken back his blessing from them.’



35.\textbf{\textit{  Cəcəngehe  na,  war  elé  háy  bəlen  na,  ásak  asabaj.}}



\textit{tʃɪtʃɪŋgɛhɛ  na,  war  ɛlɛ  haj  bɪlɛŋ  na  á-sak                     asa-baj}



now               PSP  child   eye   millet   one   PSP  3S.IFV-multiply    again-\textsc{NEG}



‘And now, one grain of millet, it doesn’t multiply anymore.’



36.  \textbf{\textit{Talay  war  elé  háy  bəlen  kə  ver  aka  na,  ásak  asabay.}}



\textit{talaj     war  ɛlɛ  haj  bɪlɛŋ  kə  vɛr  aka  na  á-sak       asa-baj}



\textsc{ID}put  child   eye   millet    one    on    stone    on    PSP  3S.IFV-multiply  again-\textsc{NEG}



‘[If] one puts one grain of millet on the grinding stone, it doesn’t multiply anymore.’



37.  \textbf{\textit{S}}\textbf{\textit{əy  kádəya  gobay.}}



\textit{sij}\textit{     ká-d    =ija  g}\textit{\textsuperscript{w}}\textit{ɔbaj}



only    2S.IFV-prepare  =\textsc{PLU}   a lot



‘You must put on a lot.’



38.  \textbf{\textit{Ka  nehe  tawəy,  metesle  anga  war  dalay  ngəndəye  }}



\textit{ka  nɛhɛ  tawij  mɛ-tɛɬ-ɛ      aŋga  war    dalaj  ŋgɪndijɛ  }



like  DEM   3P+say  \textsc{NOM}{}-curse-CL   \textsc{POSS}   child  girl       DEM      



‘It is like this they say, “The curse [is] belonging to that young woman’



\textbf{\textit{amazata  aka  ala  avəya  nengehe  ana  məze  ahay  na.}}



\textit{ama-z  =ata      =aka  =ala      avija    nɛŋgɛhɛ  ana    mɪʒɛ  =ahaj   na}



DEP-take  =3P.IO   =on     =to  suffering  DEM      DAT  person    =Pl  PSP



‘that brought this suffering onto the people.”’  



39.  \textbf{\textit{Ka  nehe  ma  bamba  ga  andavalaj.    }}



\textit{ka  nɛhɛ  ma  bamba   ga  à-ndava    =alaj    }



like  DEM  word   story     ADJ     3S.PFV-finish  =away 



‘It is like this the story ends.’  


\section{The Cicada story                \textmd{Suzanne Tajika}}
\hypertarget{RefHeading1210361525720847}{}
This fable was recorded in Maroua, Far North Province of Cameroon in 2001. It was transcribed and translated by Sambo Joël. 


\textbf{Setting}


\textit{1.    }\textbf{\textit{Bamba  bamba!~}}

\textit{       bamba   bamba~}

       story         story   

      ‘Once upon a time…’ (lit. story, story)

\textit{2.     }\textbf{\textit{Taw}}\textbf{\textit{əy:}}\textbf{    }

        \textit{tawij} 

        3P+say    

       ‘They say:’

\textit{3.     }\textbf{\textit{Albaya  ahay  aba.}}

        \textit{albaja  =ahaj              aba}

\textbf{\textit{       }}youth      =Pl    \textsc{EXT}

       ‘There were some young men.’

\textit{4.     }\textbf{\textit{T}}\textbf{\textit{á}}\textbf{\textit{nday  t}}\textit{ə}\textbf{\textit{talay  a  ləhe.  }}

\textbf{         }\textbf{\textit{t}}\textbf{\textit{á}}\textbf{\textit{{}-ndaj         t}}\textbf{\textit{ə}}\textbf{\textit{{}-tal-aj           a  lɪhɛ}}

         3P.IFV-PRG   3P.IFV-walk{}-CL   at    bush

         ‘They were walking in the bush.’

\textbf{Episode 1}

5.      \textbf{\textit{T}}\textbf{\textit{á}}\textbf{\textit{nday  t}}\textit{ə}\textbf{\textit{talay  a  l}}\textbf{\textit{ə}}\textbf{\textit{he  na}}\textit{,}

         \textit{t}\textit{á}\textit{{}-ndaj        t}\textit{ə}\textit{{}-tal-aj    a   lɪhɛ     na}

         3P.IFV-PRG   3P-walk{}-CL   at  bush   PSP

         \textbf{\textit{tolo  tənjakay  ngəvəray  malan  ga  a  ləhe.  }}

         \textit{tə-lɔ           tə-nzak-aj        ŋgəvəraj   malaŋ  ga      a    lɪhɛ}

         3P.PFV-go   3P.PFV-find{}-CL   spp. of tree     large   ADJ  at   bush

        ‘\textstyleExampleglossChar{[}As]they were walking in the bush, they found a large tree (a particular species) in the bush.’

\textbf{Episode 2}

\textit{6.      }\textbf{\textit{Albaya  ahay  ndana  kəlen  təngalala  ma  ana  bahay.  }}

          \textit{albaja  =ahaj  ndana  kɪlɛŋ  tə-ŋgala        =ala     ma    ana      bahaj.  }

          youth    =Pl       DEM  then\textit{   }3P.PFV-return  =to  word    DAT  chief

        ‘Those young men then took the word (response) to the chief.’  

7.     \textbf{\textit{Taw}}\textbf{\textit{əy}}\textbf{\textit{  bahay,  }}\textbf{\textit{mama  ngəvəray  ava  a  ləhe  na}}\textit{,}\textbf{\textit{  }}\textbf{  }

        \textit{tawij~}  \textit{bahaj,  }\textit{mama   ŋgəvəraj       ava  a   lɪhɛ   }\textit{na}\textit{  }  \textit{malaŋ}\textit{  }\textit{ga  }\textit{  na}

\textbf{       } 3P+say  chief,   mother   spp. of tree  \textsc{EXT}   at   bush    PSP  large     ADJ  PSP

‘They said, “Chief, there is a mother-tree in the bush, a big one,’


\textbf{\textit{agasaka  na  ka  mahay  ango  aka  }}\textbf{\textit{am}}\textbf{\textit{ə}}\textbf{\textit{mbese}}\textbf{\textit{.~}}



\textit{à-gas    =aka   na  ka    mahaj   aŋg}\textit{\textsuperscript{w}}\textit{ɔ           aka  }\textit{amɪ-mbɛʃ-ɛ}\textit{~}



3S.PFV-get  =on  PSP   on     door        =2S.POSS  on  DEP-rest-CL



‘ [and] it would please you to have that tree at your door, so that you could rest under it.”’


8.     \textbf{\textit{K}}\textbf{\textit{ə}}\textbf{\textit{len  }}\textbf{\textit{albaya  ahay  ndana  tolo.  }}


\textit{kɪlɛŋ  }\textit{albaja   =ahaj  ndana  tə-lɔ}


then   youth     =Pl     DEM   3P.PFV- go

‘Then, those young men went.’

9.     \textbf{\textit{Nde, }}\textbf{\textit{bahay awəy  “Nde na,  səy}}\textbf{\textit{  }}\textbf{\textit{slərom  alay  war.}}


\textit{ndɛ}\textit{     bahaj   awij     nd}\textit{ɛ  }\textit{na~  s}\textit{ij}\textit{   ɬər-ɔm           =alaj         war}


so    chief  3S –say  so  PSP  only  send \textsc{IMP}{}-2P  =away   child

‘And so the chief said, “So, you must send a child.’

\textbf{\textit{K}}\textbf{\textit{áa}}\textbf{\textit{zəɗom  anaw  ala  ngəvəray  ndana  ka  mahay  əwla  aka.}}

\textit{k}\textit{áá}\textit{{}-z}\textit{ʊ}\textit{ɗ{}-ɔm    an    =aw    =ala    ŋgəvəraj    ndana    ka   mahaj    =uwla      aka}

2P.POT-take-2P      DAT =1S.IO  =to       spp.of tree   DEM   on   door     =1S.POSS  on

‘I want you to bring that tree to my door for me.’

\textbf{\textit{K}}\textbf{\textit{áa}}\textbf{\textit{fəɗom  anaw  ka  mahay  əwla  aka.”}}

\textit{k}\textit{áá}\textit{{}-f}\textit{ʊ}\textit{ɗ{}-ɔm          an  =aw      ka   mahaj   =uwla     aka}

2P.POT-put-2P    DAT  =1S.IO   on  door     =1S.POSS   on

‘I want you to put it by my door.”’ 

\textit{10.   }\textbf{\textit{Bahay  kəlen  ede  gəzom.  }}

\textit{bahaj  kɪlɛŋ  à-d-ɛ                 gʊzɔm  }

chief  then  3S.PFV-prepare{}-CL  wine

‘The king then made wine.’

11.   \textbf{\textit{Aslar  məze  ahay.  }}

\textit{à-ɬar    mɪʒɛ  =ahaj}

3S.PFV-send    person  =Pl

‘He sent out the people.’

12.   \textbf{\textit{Tolo  tamənjar  na  ala  mama  ngəvəray  nəndəye.}}

\textit{tə-lɔ             tà-mənzar            na      =ala  mama   ŋgəvɛraj     nɪndijɛ}

3P .PFV-go  3P.HOR-see  3S.DO  =to     mother  spp. of tree  DEM

‘They went to see mother-tree there.’

13.   \textbf{\textit{Məze  ahay  tangala  ma  ana  bahay.}}

\textit{mɪʒɛ  =ahaj   tà-ŋg          =ala  ma      ana  bahaj}

person   =Pl    3P.PFV-return  =to   word  DAT   chief

‘The people brought back word to the chief.’

14.   \textbf{\textit{Tawəy,  “}}\textbf{\textit{Ɗ}}\textbf{\textit{eɗen  bahay,  }}\textbf{\textit{ngəvəray  ngəndəye  }}\textbf{\textit{á}}\textbf{\textit{gasaka  ka  mahay  ango  aka,}}

\textit{tawij     }\textit{ɗ}\textit{ɛɗɛŋ   bahaj    }\textit{ŋgəvəraj  ŋgɪndijɛ   }\textit{á}\textit{{}-gas       =aka   ka    mahaj    aŋg}\textit{\textsuperscript{w}}\textit{ɔ         aka}

3P+say  truth     chief     spp. of tree    DEM     3S.IFV-get =on    at   door    =2S.POSS on

‘They said, “It is true, chief. It would be pleasing if that particular tree would be by your door,’

\textbf{\textit{b}}\textbf{\textit{əy}}\textbf{\textit{na  ngəvəray  ga  s}}\textbf{\textit{ə}}\textbf{\textit{lom  ga;  }}\textbf{\textit{aɓəsay  ava  bay.”}}

\textit{bijna      ŋgəvəraj  ga    sʊlɔmga   }\textit{aɓəsaj  ava     baj}

because   sp.of.tree  ADJ   good   ADJ   blemish   \textsc{EXT}  \textsc{NEG}

‘because this tree is good;  it has no faults.”’

\textbf{Episode 3}

15.   \textbf{\textit{Bahay  alala  a  həlan  na,  ndahan  gədok  mədəye  gəzom.}}

\textit{bahaj  à-l=ala    a   həlaŋ   na  ndahaŋ  gʊdɔk}\textit{\textsuperscript{w}}\textit{     m}\textit{ɪ}\textit{{}-d-ijɛ      gʊzɔm}

chief   3S.PFV-go=to   at   back   PSP  3S        \textsc{ID}prepare wine  \textsc{NOM}{}-prepare{}-CL wine


‘The chief then came behind [and] he made wine.’


16.  \textbf{\textit{Kəlen  albaya  ahay  tolo  amazala  ngəvəray  na,}}

\textit{kɪlɛŋ}  \textit{albaja    =ahaj    tɔ-lɔ      ama-z     =ala   ŋgəvəraj  na}

then   youth   =Pl      3P.PFV-go   DEP-take  =to  spp.of tree  PSP

‘And then, the young men left to bring back the tree,’

\textbf{\textit{tààzala təta}}\textbf{\textit{  }}\textbf{\textit{bay.}}

\textit{tàà-z     =ala  təta}\textit{    }\textit{baj}

3P.HOR-take  =to      ABILITY  \textsc{NEG}


‘[but] they were not able to bring [it].’  


17.   \textbf{\textit{M}}\textit{ə}\textbf{\textit{d}}\textit{əye}\textbf{\textit{  g}}\textit{ə}\textbf{\textit{zom  makar.  }}

\textit{m}\textit{ɪ}\textit{{}-d-ijɛ      gʊzɔm     makar.  }

\textsc{NOM}{}-prepare{}-CL       wine  three

‘He made wine for the third time.’

18.   \textbf{\textit{Bahay  alala  a  həlan  na,  awəy,~ }}

\textit{bahaj    à-l    =ala      a   həlaŋ   }\textit{na}\textit{    }\textit{awij~}

chief   3S.PFV-go  =to   at   back   PSP  3S+say    


‘[And then], the chief came behind, saying,’


\textbf{\textit{“N}}\textbf{\textit{áa}}\textbf{\textit{njakay  na  wa  amazaw  ala  ngəvəray  ana  ne  na  way?}}

\textit{n}\textit{áá}\textit{{}-nzak-aj        na       wa  ama-z  =aw =ala      ŋgəvəraj    ana     nɛ   na  waj}

1S.\textsc{POT}{}-find{}-CL   PSP   who   DEP-take=1S.IO  =to    spp. of tree     DAT     1S    PSP  who

‘“Who can I find to bring to me this tree for me?’

\textbf{\textit{Kə  mahay  aka  na  }}\textbf{\textit{n}}\textbf{\textit{áa}}\textbf{\textit{mbasaka  na,      ~}}

\textit{kə   mahaj    aka  na    }\textit{n}\textit{áá}\textit{{}-mbas     =aka   na      ~}

on    door       on      PSP   1S.POT-rest    =on      PSP

‘By my door I want to rest.’

\textbf{\textit{Mama  ngəvəray  səlom  ga  lala.}}

\textit{mama    ŋgəvəraj       sʊlɔm   ga     lala}

mother   spp. of tree     good     ADJ   well

‘The mother tree is very good.”’

\textbf{Prepeak}

19.   \textbf{\textit{K}}\textbf{\textit{ə}}\textbf{\textit{len  }}\textbf{\textit{bahay  na,  olo  kə  mətəɗe  aka.  }}

\textit{kɪlɛŋ}  \textit{bahaj  na  ɔ{}-lɔ        kə  mɪtɪɗɛ  aka }

then    chief  PSP  3S.PFV-go    on  cicada  on

‘Then, the chief went to the cicada.’  

20.  \textbf{\textit{Mətəɗe  awəy, “B}}\textbf{\textit{ahay}}\textbf{,  }\textbf{\textit{toko!  ~}}

\textit{mɪtɪɗɛ   awij   }\textit{bahaj}  \textit{tɔk}\textit{\textsuperscript{w}}\textit{ɔ  ~}

cicada  3S-say  chief   go \textsc{IMP}{}-1\textsc{Pin}    

‘The cicada said, “Chief, let’s go!’  

\textbf{\textit{N}}\textbf{\textit{áa}}\textbf{\textit{mənjar  na  alay  memele  ga  ndana  əwɗe.~}}

\textbf{\textit{n}}\textbf{\textit{áá}}\textbf{\textit{{}-mənzar   na  =alaj   mɛmɛlɛga   ndana  uwɗɛ~}}

1S.POT-see   3S.DO  =away   tree   ADJ  DEM   first

“First I want to see the tree that you spoke of.”’

21.  \textbf{\textit{Məze  ahay  tawəy,  }}\textbf{\textit{“}}\textbf{\textit{Aa}}\textbf{\textit{  }}\textbf{\textit{məze  ahay  səlom  ahay  ga  na,  }}

\textit{mɪʒɛ  =ahaj  tawij   }\textit{aa}\textit{  mɪʒɛ  =ahaj   sʊlɔm   =ahaj   ga   na}

person  =Pl  3P+say  ah  person   =Pl     good    =Pl   ADJ   PSP

‘The people said, “O, even good people,’

\textbf{\textit{t}}\textbf{\textit{á}}\textbf{\textit{zala  təta}}\textbf{\textit{  }}\textbf{\textit{bay  na,  }}

\textit{t}\textit{á}\textit{{}-z     =ala     təta}\textit{    }\textit{baj         na  }

3P.IFV-take  =to  ABILITY  \textsc{NEG}   PSP


‘[if] they can’t bring it,’  


\textbf{\textit{azl}}\textbf{\textit{ə}}\textbf{\textit{na  }}\textbf{\textit{mətəɗe  azla,  }}\textbf{\textit{engeren  azla   }}\textbf{\textit{k}}\textbf{\textit{áa}}\textbf{\textit{zala  }}\textbf{\textit{təta  na,}}\textbf{\textit{          }}

\textit{aɮana  }\textit{mɪtɪɗɛ    aɮa    }\textit{ɛŋgɛrɛŋ   aɮa  }\textit{k}\textit{áá}\textit{{}-z    =ala      }\textit{təta          }\textit{na          }

but    cicada    now      insect     now   2S.POT-take  =to    ABILITY   PSP    

‘but you, cicada, an insect, you think you can bring it,

\textbf{\textit{k}}\textbf{\textit{áa}}\textbf{\textit{zala  na}}\textbf{\textit{,  malma  ango  may?”~}}

\textit{k}\textit{áá}\textit{{}-z  =ala     na}\textit{  malma  aŋg}\textit{\textsuperscript{w}}\textit{ɔ    maj~}

2S.POT-take=to  PSP   what  =2S.POSS  what


 ‘[if] you do bring it,  [then] what is with you?”’


22.  \textbf{\textit{M}}\textbf{\textit{ə}}\textbf{\textit{t}}\textbf{\textit{ə}}\textbf{\textit{ɗe  aw}}\textbf{\textit{əy,  }}\textbf{\textit{“N}}\textbf{\textit{áa}}\textbf{\textit{zala!”                   }}

\textit{mɪtɪɗɛ  awij~    }\textit{n}\textit{áá}\textit{{}-z=ala,                   }

cicada          3S+say    1S.POT-take =to      


‘The cicada said, “I will bring [it].”’


23.  “\textbf{\textit{Káa}}\textbf{\textit{zala   təta   bay!”}}

\textit{káá}\textit{{}-z  =ala     təta     baj}

2S .POT-take=to  ABILITY  \textsc{NEG}


‘“You can’t bring [it].”’


24.  \textit{“N}\textbf{\textit{áa}}\textbf{\textit{zala!  }}\textbf{\textit{Nde  }}\textbf{\textit{toko  əwɗe!{\textquotedbl}~                }}

\textit{n}\textit{áá}\textit{{}-z  =ala,  }\textit{ndɛ  }\textit{tɔkɔ              uwɗɛ~      }

1S.POT-take=to  so,  go(\textsc{IMP})-1\textsc{Pin}    first

‘“I will bring it, but first, let’s go!”’

\textbf{Peak}

25.  \textbf{\textit{Nata  }}\textbf{\textit{olo.}}

\textit{nata      }\textit{ɔ{}-lɔ}

and then     3S.PFV-go  

‘And then, he went.’

26.  \textbf{\textit{Albaya  ahay  tolo  sen  na,    }}

\textit{albaja    =ahaj   tɔ-lɔ    ʃɛŋ      na  }

youth     =Pl      3P-go   \textsc{ID}go   PSP  

‘The young men went,’

\textbf{\textit{albaya  ahay  weley  təh  anan  dəray  na}}\textbf{ , }\textbf{\textit{abay.}}

\textit{albaja   =ahaj   wɛlɛj  təx     an=aŋ         dəraj   na}  \textit{abaj}

youth    =Pl    which   \textsc{ID}put   DAT=3S.IO   head   PSP   \textsc{EXT} \textsc{NEG}


‘[and] no one could lift it.’ (lit. whichever young man put his head [to the tree in order to lift it], there was none)


27.  \textbf{\textit{Nata  mətəɗe  təh  anan  dəray  ana  ngəvəray  ngəndəye.}}

\textit{nata  mɪtɪɗɛ  təx  an=aŋ    dəraj  ana  ŋgəvəraj  ŋgɪndijɛ}

and then   cicada   \textsc{ID}put    DAT=3S.IO   head   DAT   spp. of tree    DEM

‘And then,  the cicada put his head to that tree.’

28.  \textbf{\textit{K}}\textbf{\textit{ə}}\textbf{\textit{wna!}}\textbf{   }

\textit{kuwna}

\textsc{ID}getting  

‘He got it.’ 

29.  \textbf{\textit{Dergwecek!}}\textbf{~}

\textit{dɛrg}\textit{\textsuperscript{w}}\textit{ɛtʃɛk}~

\textsc{ID}lifting onto head

‘He lifted it onto his head.’

\textbf{Dénouement}

30.  \textbf{\textit{A}}\textbf{\textit{magala  l}}\textbf{\textit{ə}}\textbf{\textit{mes,  “}}\textbf{\textit{Te  te  te  te ver na tepəɗek təvəw na tambəɗek…”}}

\textit{a}\textit{ma-g=ala    lɪmɛʃ  }\textit{Tɛ  tɛ  tɛ  tɛ vɛr na tɛp}\textit{ɪ}\textit{ɗɛk təvuw na tamb}\textit{ɪ}\textit{ɗɛk…}


DEP-do  DIR  song  [words of the song]



‘He was singing (song is given), [on his way] to [the chief’s house].’


31.  \textbf{\textit{Sen  ala.  }}

\textit{ʃɛŋ      =ala}

\textsc{ID}go        =to


‘Going, [he came to the chief’s house].’  


32.  \textbf{\textit{Tahan  na.}}

\textit{tà-h    =aŋ      na}

3P.PFV-greet=3S.IO        PSP

‘They greeted him.’

\textit{33.  }\textbf{\textit{Mama  ngəvəray  na,  ka  mahay  aka  afa  bahay  gəɗəgəzl!         }}

\textit{mama  ŋgəvəraj     na  ka   mahaj   aka   afa       bahaj  gəɗəgəɮ~       }

mother  spp. of tree   PSP  on     door        on      at place of     chief   \textsc{ID}put down


‘The mother tree,  at the door of the chief’s house, [he] put [it] down.’  


34.  \textbf{\textit{Bahay  na  }}\textbf{\textit{membese}}\textbf{.  “}\textbf{\textit{Səwse,   mətəɗe   səwse, səwse, səwse}}\textbf{!}

\textit{bahaj      na  }\textit{mɛ-mbɛʃ-ɛ  }\textit{ʃuwʃɛ   mɪtɪɗɛ   ʃuwʃɛ  ʃuwʃɛ  ʃuwʃɛ}

chief  PSP  \textsc{NOM-}smile{}-CL  thanks   cicada   thanks   thanks    thanks


‘The chief smiled, [saying] “Thank you, thank you, thank you cicada!’


35.  \textbf{\textit{Mama  ngəvəray  na,  kə  mahay  anga  bahay  aka.~~~}}

\textit{mama  ŋgəvəraj   na  kə  mahaj  aŋga  bahaj  aka~~~}

mother  spp. of tree       PSP  on     door  \textsc{POSS}  chief  on


‘The mother tree [is] by the door that belongs to the chief.’


36.  \textbf{\textit{A}}\textbf{\textit{ndavalay.}}

\textit{a-ndava  = alaj}

3S-finish    =away


‘It is finished.’


\section{Values exhortation                \textmd{Dungaya Daniel}}
\hypertarget{RefHeading1210381525720847}{}
This exhortation was given in Lalaway, Far North Province of Cameroon in 2002. It was transcribed and translated by Sambo Joël. 


\textbf{Setting}



\textit{1.    }\textbf{\textit{Səlom  ga  yawa  təde  kəyga!}}


\textit{sʊlɔm  ga  jawa   tɪdɛ   kijga}


goodness  ADJ  well  good  like this



‘Good, well, good, [it is] like this:’



\textit{2.    }\textbf{\textit{Ehe  na,  wəyen  amba}}\textbf{\textit{ɗ}}\textbf{\textit{ala  a  jere  azla.}}


\textit{ɛhɛ     }\textit{na}\textit{  wijɛŋ   à-mba}\textit{ɗ    =}\textit{ala   a   dʒɛrɛ   aɮa}


here    PSP  earth  3S.PFV-change  =to  at  truth  now



‘Here, the earth has changed to truth now (sarcastic).’



\textit{3.    }\textbf{\textit{S}}\textbf{\textit{ə}}\textbf{\textit{wat  na,   təta  a  m}}\textbf{\textit{ə}}\textbf{\textit{s}}\textbf{\textit{yon}}\textbf{\textit{  na  ava  n}}\textbf{\textit{ə}}\textbf{\textit{nd}}\textbf{\textit{əye  }}\textbf{\textit{na,  pester  áhata,  }}\textbf{\textit{“Ey!  Ele  nehe  na,  kógom  bay!”}}\textbf{\textit{  }}


\textit{suwat  }\textit{na}\textit{   təta  a  mɪsijɔŋ   na   ava  nɪndijɛ  na     }


\textsc{ID}disperse  PSP  3P      in   mission  PSP  in  DEM  PSP  



‘As the people go home from church, the pastor tells them,’ (lit. disperse, they in the mission there),’ 



\textit{pɛʃtɛr  á-h    =ata   ɛj     ɛlɛ      nɛhɛ   na   kɔ-g}\textit{\textsuperscript{w}}\textit{{}-ɔm           baj}



pastor  3S.IFV-tell  =3P.IO  hey  thing  DEM  PSP  2.IFV-do-2P  \textsc{NEG}



‘Pastor told them, “Hey! These things, don’t do them!”’



\textit{4.    }\textbf{\textit{Yawa,  war  dalay  ga ándaway  mama  ahan.}}


\textit{jawa   war   dalaj  ga  á-ndaw-aj   mama   =ahaŋ}


well    child  female  ADJ  3S.IFV-insult{}-CL  mother  =3S.POSS  



‘Well, the girls insult their mothers.’ 



\textit{5.    }\textbf{\textit{War  zar  ga  ándaway  baba  ahan.}}


\textit{war     zar  ga  á-ndaw-aj   baba   =ahaŋ}


child  male  ADJ  3S.IFV-insult{}-CL  father  =3S.POSS



‘[And] the boys insult their fathers.’ 



\textit{6.     }\textbf{\textit{Yo  ele  ahay  aməgəye  bay  nəngehe    pat,   }}


\textit{jɔ      ɛlɛ   =ahaj  amə-g-ijɛ   baj      nɪŋg}\textit{ɛ}\textit{hɛ    pat   }


well   thing  =Pl      DEP-go-CL    \textsc{NEG}    DEM     all  



‘Well, all these particular things that we are not supposed to do,’



\textbf{\textit{tahata  na  va  kə  dəftere  aka.}}



\textit{tà-h    =ata     na        =va   kə   dɪftɛrɛ  aka}



3P.PFV-tell  =3P.IO   3S.DO    =\textsc{PRF}  on  book  on



‘they have already told them in the Bible.’



\textit{7.     }\textbf{\textit{Hərmbəlom  awacala  kə  okor  aka.}}


\textit{Hʊrmbʊlɔm   à-wats    =ala   kə   ɔk}\textit{\textsuperscript{w}}\textit{ɔr   aka}


God    3S.PFV-write  =to  on  stone  on



‘God wrote them on the stone [tablet].’



\textit{8.     }\textbf{\textit{Alala,  asara  agas.     }}



\textit{á-l     =ala    asara     à-gas     }



3S.IFV-come=to    white man  3S.PFV-catch    



‘Later, the white man accepted (lit. caught) [it].’



\textit{9.   }\textbf{\textit{Ege  dəftere  ahan  kə  dəwnəya  aka.}}



\textit{ɛ{}-g-ɛ     dɪftɛrɛ   =ahaŋ          kə   duwnija    aka} 



3S.PFV-do-CL  book  =3S.POSS       on  earth    on



‘He made his book on the earth.’ 



\textit{10.  }\textbf{\textit{Ahata  na  va,  “Ele nehe na, awasl,        }}



\textit{à-h    =ata   na        =va      ɛlɛ     nɛhɛ      na    à-wa}\textit{ɬ}\textit{              }



3S.PFV-tell  =3P.IO   3S.DO  =\textsc{PRF}   thing  DEM  PSP  3S.PFV-forbid  



‘He has told them already, “This thing is forbidden,’



\textbf{\textit{ele nehe na, awasl, ele nehe na, awasl,}}



\textit{ɛlɛ       nɛhɛ     na      à-wa}\textit{ɬ    }\textit{ɛlɛ       nɛhɛ       na      à-wa}\textit{ɬ    }



thing    DEM     PSP   3S.PFV-forbid  thing    DEM     PSP   3S.PFV-forbid    



‘this thing is forbidden, this thing is forbidden,’



\textbf{\textit{ele nehe na, awasl, kəro!”}}



\textit{ɛlɛ       nɛhɛ     na     à-wa}\textit{ɬ    }\textit{kʊrɔ}



thing    DEM     PSP   3S.PFV-forbid    ten



 ‘this thing is forbidden – ten [commandments]”’



\textit{11.}\textbf{\textit{   Ahata  na  cece.}}


\textit{à-h    =ata   na   tʃɛtʃɛ.}


3S.PFV-tell  =3P.IO  3S.DO  all



‘He told all of them.’ 



\textit{12.   }\textbf{\textit{Yawa  nde  ele  nehe  }}\textbf{\textit{ɗ}}\textbf{\textit{əw,  kóogəsok}}\textbf{\textit{\textsuperscript{  }}}\textbf{\textit{ma  Hərmbəlɔm.  }}



\textit{jawa   ndɛ   ɛlɛ   nɛhɛ   }\textit{ɗ}\textit{uw  kɔɔ-g}\textit{\textsuperscript{w}}\textit{ʊs-ɔk}\textit{\textsuperscript{w}}\textit{     ma   Hʊrmbʊlɔm  }



well    so  thing  DEM  also  2S.POT-catch-2P    word  God



‘So, this thing here, you should accept the word of God.’ 



\textit{13.   }\textbf{\textit{A  məsyon  ava  na,  ele  ahay  aməwəsle  na,  tége  bay.}}



\textit{a  mɪsijɔŋ    ava   na  ɛlɛ   =ahaj  amu-wu}\textit{ɬ{}-}\textit{ɛ     na   tɛ-g-ɛ     baj}



to  mission    in  PSP  thing  =Pl        DEP-forbid-CL   PSP  3P.IFV-do-CL  \textsc{NEG}



‘In the church, these things that they have forbidden, they don’t do.’



\textit{14.  }\textbf{\textit{Yo, asara  ahata  na  va.    }}



\textit{jɔ     asara     à-h    =ata  na   =va    }



well    white man  3S.PFV-tell  =3P.IO  3S.DO  =\textsc{PRF}  



‘Well, the white man told it to them already.’ 



\textit{15.  }\textbf{\textit{Pester  ahata  na  va.  }}



\textit{pɛʃtɛr   à-h    =ata   na   =va  }



pastor  3S.PFV-tell  =3P.IO  3S.DO  =\textsc{PRF}



‘The pastor told it to them already.’ 



\textit{16.  }\textbf{\textit{Təlala a  həlan  ga  ava  ese,   }}



\textit{tə-l    =ala   a   həlaŋ  ga   ava   ɛ}\textit{ʃ}\textit{ɛ   }



3P.IFV-go  =to  in  back  ADJ  in  again  



‘After [church] again, (lit. they come at the back of again)



\textbf{\textit{təwə}}\textbf{\textit{ɗ}}\textbf{\textit{akala  har  a  məsyon  ava.}}



\textit{tə-wə}\textit{ɗ}\textit{ak    =ala   har   a   mɪsijɔŋ   ava} 



3P.IFV-divide  =to  body  in  mission  in



‘they go home after church.’ (lit. they divide body in mission)



\textit{17.  }\textbf{\textit{Álaway  war  ahan.}}



\textit{á-law-aj      war   =ahaŋ} 



3S.PFV-mate{}-CL  child  =3S.POSS    



‘[One] sexually abuses his child.’



\textit{18.  }\textbf{\textit{Ólo  á}}\textbf{\textit{ɓ}}\textbf{\textit{an  ana  baba  ahan. }}



\textit{ɔ{}-lɔ     á-}\textit{ɓ    }\textit{=aŋ   ana  baba  =ahaŋ} 



3S.IFV-go  3S.IFV-hit  =3S.IO  DAT        father    =3S.POSS



‘[Another] goes and hits his father.’ 



\textit{19.  }\textbf{\textit{Ólo  ápa}}\textbf{\textit{ɗ}}\textbf{\textit{ay  məze  nə  madan. }}



\textit{ɔ{}-lɔ     á-pa}\textit{ɗ{}-}\textit{aj       m}\textit{ɪʒ}\textit{ɛ   nə   madaŋ} 



3S.IFV-go  3S.PFV-crunch{}-CL  person  with  magic  



‘[Another] goes and kills someone with sorcery.’ (lit. he goes he eats a person with magic)



\textit{20.  }\textbf{\textit{Olo  aka  akar.}}



\textit{à-lɔ         aka   akar} 



3S.PFV-go   on  theft  



‘[Another] goes and steals.’(lit. he went on theft)



\textit{21.  }\textbf{\textit{Ege  adama.}}



\textit{à-g-ɛ     adama} 



3S.PFV-do-CL  adultery



‘[Another] commits adultery.’ 



\textit{22.  }\textbf{\textit{Təta  dəl  na  ma  H}}\textbf{\textit{ə}}\textbf{\textit{rmb}}\textbf{\textit{ə}}\textbf{\textit{lom  nend}}\textbf{\textit{əye.}}



\textit{təta   dəl     na   ma   Hʊrmbʊlɔm   nɛndijɛ} 



3P    \textsc{ID}insult    3S.DO  word  God    DEM



‘They insult the word of God!’



\textit{23.  }\textbf{\textit{Nde na  cəve}}\textbf{\textit{ɗ}}\textbf{\textit{  ahan  na,  memey?}}



\textit{ndɛ  na   tʃɪvɛ}\textit{ɗ}\textit{   =ahaŋ     na   mɛmɛj} 



so  PSP  path  =3S.POSS  PSP  how



‘So, what can he do?’ (lit. how [is] his pathway)



\textit{24.  }\textbf{\textit{Táágas  na  anga  way?}}



\textit{táá-gas     na   aŋga   waj} 



3P.POT-catch  PSP  \textsc{POSS}  who



‘They will accept whose word?’ (lit. they will catch it, [something] that belongs to whom?)



\textit{25.}\textbf{\textit{  Ma   a  baba  ango  kagas  asabay.}}



\textit{ma     a  baba   =aŋg}\textit{\textsuperscript{w}}\textit{ɔ     kà-gas     asa-baj} 



word  GEN  father  =3S.POSS  2S.PFV-catch  again-\textsc{NEG}  



‘Your father’s word you no longer accept.’



\textit{26.}\textbf{\textit{  Ma   a  mama  ango  kagas  asabay.}}



\textit{ma     a  mama   =aŋg}\textit{\textsuperscript{w}}\textit{ɔ    kà-gas     asa-baj} 



word  GEN  mother  =3S.POSS  2S.PFV-catch  again-\textsc{NEG}



‘Your mother’s word you no longer accept.’



\textit{27.  }\textbf{\textit{Nde na  káagas  anga  way?}}



\textit{ndɛ  na   káá-gas       aŋga   waj} 



so  PSP  2S.POT-catch    \textsc{POSS}  who    



‘So, you don’t accept anyone’s word!’ (lit. you will catch [that which] belongs to whom?)



\textit{28.  }\textbf{\textit{Anga  Hərmbəlom  ga  kagas  asabay. }}



\textit{aŋga   Hʊrmbʊlɔm   ga     kà-gas     asa-baj} 



\textsc{POSS}  God    ADJ    2S.PFV-catch  again-\textsc{NEG}



‘The very [word] of God himself you no longer accept.’ (lit. the quality of belonging to God himself)



\textit{29.  }\textbf{\textit{Hərmbəlom  na,  ama}}\textbf{\textit{ɗ}}\textbf{\textit{a}}\textbf{\textit{sl}}\textbf{\textit{ava  ala  məze   na,}}



\textit{Hʊrmbʊlɔm  na  ama-}\textit{ɗ}\textit{a}\textit{ɬ    =}\textit{ava  =ala  mɪʒɛ  na}



God    PSP    DEP-multiply  =in  =to   person   PSP     



‘God, who multiplied the people,’



\textbf{\textit{ndahan  ese  na,  kagas  ma  Hərmbəlom  na,  asabay  na,  }}



\textit{ndahaŋ  ɛʃɛ  na  ka-gas      ma  Hʊrmbʊlɔm  na       asa-baj         na}



3S     again    PSP    2S-catch    word     God          PSP  again-\textsc{NEG}  PSP 



‘if you will never accept the word of God,’ (lit. him again, you never catch the word of God)



\textbf{\textit{káágas  na  anga  way?}}



\textit{káá-gas            na      aŋga     waj}



2S.POT-catch    PSP   \textsc{POSS}  who



‘whose word \textit{will} you accept then?’ (lit. you will catch it,that which belongs to who)



\textit{30.  }\textbf{\textit{Səlom  ga.  }}



\textit{sʊlɔm  ga  }



goodness  ADJ



‘Good!’ [narrator to himself].



\textit{31.  }\textbf{\textit{Asara  anday  á}}\textbf{\textit{ɗ}}\textbf{\textit{akaləme  ma  a  dəwnəya.}}



\textit{asara   à-ndaj        á-}\textit{ɗ}\textit{ak    =alɪmɛ               ma     a   duwnija} 



white man     3S.PFV-PROG  3S.IFV-show  =1\textsc{Pex}.IO    word   GEN    earth    



‘The white man is showing us how the world is (lit. the word of the earth).’ 



\textit{32.  }\textbf{\textit{Anday  á}}\textbf{\textit{ɗ}}\textbf{\textit{akal}}\textbf{\textit{ə}}\textbf{\textit{me  ende}}\textbf{\textit{ɓ.}}\textbf{\textit{  }}



\textit{à-ndaj           á-}\textit{ɗ}\textit{ak    =alɪmɛ     ɛndɛ}\textit{ɓ}\textit{  }



3S.PFV-PROG    3S.IFV-show  =1\textsc{Pex}.IO  wisdom  



‘He is showing us wisdom.’ 



\textit{33}\textbf{\textit{.  Tágas  bay.}}



\textit{tá-gas     baj} 



3P.IFV-catch  \textsc{NEG}



‘They aren’t the accepting kind.’ (lit. they don’t accept) 



\textit{34.  }\textbf{\textit{Ehe  na,  təta  na,  kəw  na,  bəw}}\textbf{\textit{ɗ}}\textbf{\textit{ere!}}



\textit{ɛhɛ    na      təta   na       kuw      na  buw}\textit{ɗ}\textit{ɛrɛ} 



here   PSP    3P    PSP    \textsc{ID}take   PSP  \textsc{ID}foolishness



‘Here, what they are taking is foolishness!’ (lit. here, they, taking, foolishness)



\textit{35.  }\textbf{\textit{Epəlepəle  na,  wəyen  amba}}\textbf{\textit{ɗ}}\textbf{\textit{ala  }}\textbf{\textit{sl}}\textbf{\textit{am  a  yam  av}}\textbf{\textit{ə}}\textbf{\textit{lo.}}



\textit{ɛpɪlɛ-pɪlɛ   na   wijɛŋ   à-mba}\textit{ɗ    =}\textit{ala   }\textit{ɬ}\textit{am   a   jam   avʊlɔ} 



in the future  PSP  earth  3S.PFV-change  =to    place  GEN  water  above



‘Someday, the earth will change into heaven (the place of water above).’ 



\textit{36.  }\textbf{\textit{Nde  na,  oko  ndana  anga  way?}}



\textit{ndɛ  na   ɔk}\textit{\textsuperscript{w}}\textit{ɔ   ndana   aŋga   waj} 



so  PSP  fire  DEM  \textsc{POSS}  who



‘So who are the fires [of hell] going to strike?’ (lit. so, that fire, belonging to who)



\textit{37.  }\textbf{\textit{C}}\textbf{\textit{əcəngehe  }}\textbf{\textit{na,  asa  tágalay  janga  ana  ende}}\textbf{\textit{ɓ}}\textbf{\textit{  ango,      }}



\textit{tʃɪtʃɪŋgɛhɛ    na  }\textit{asa  tá-g    =alaj      dzaŋga   ana    ɛndɛ}\textit{ɓ}\textit{     =aŋg}\textit{\textsuperscript{w}}\textit{ɔ}



now            PSP    if     3P.IFV-do  =away  reading  DAT    wisdom  =2S.POSS  



‘These days, if they look at your life,’ (lit. now if they do a reading to your wisdom)



\textbf{\textit{nafta  w}}\textbf{\textit{əyen  }}\textbf{\textit{am}}\textbf{\textit{ə}}\textbf{\textit{ndeve  na,  H}}\textbf{\textit{ə}}\textbf{\textit{rmb}}\textbf{\textit{ə}}\textbf{\textit{lom  ágok  sər}}\textbf{\textit{əy}}\textbf{\textit{a  na,}}



\textit{nafta   wijɛŋ  amɪ-ndɛv-ɛ  na  Hʊrmbʊlɔm  á-g     =ɔk}\textit{\textsuperscript{w}}\textit{    sərija         na}



day     earth    DEP-finish-CL   PSP    God           3S.IFV-do =2S.IO   judgement  PSP    



‘on the day that the earth ends, [and] God judges you [and you fail of course],’



\textbf{\textit{kéege  na,  memey?}}



\textit{kɛɛ-g-ɛ    na  mɛmɛj}



2S.POT-do-CL  PSP  how



‘what will you do [as you burn]?’



\textit{38.  }\textbf{\textit{Nde  ehe  kəyga.}}



\textit{ndɛ  ɛhɛ   kijga} 



so  here  like this



‘So, it is like this here.’ 



\textit{39.  }\textbf{\textit{Pepenna  na  taka}}\textbf{\textit{ɗ}}\textbf{\textit{  }}\textbf{\textit{sl}}\textbf{\textit{a.}}



\textit{pɛpɛŋ  =ŋa   na  tà-ka}\textit{ɗ}\textit{     }\textit{ɬ}\textit{a} 



long ago  =ADV  PSP  3P.PFV-kill  cow  



‘Long ago, they killed cows.’



\textit{40.  }\textbf{\textit{Tége  almay?}}



\textit{tɛ-g-ɛ     almaj} 



3P.\textsc{IFV}{}-do-CL  what



‘What were they doing?’



\textit{41.  }\textbf{\textit{M}}\textbf{\textit{əze  }}\textbf{\textit{ákosaka  j}}\textbf{\textit{əy}}\textbf{\textit{ga  d}}\textbf{\textit{ə}}\textbf{\textit{res. }}



\textit{mɪʒɛ   á-k}\textit{\textsuperscript{w}}\textit{as    =aka   d}\textit{ʒ}\textit{ijga   dɪrɛʃ }



person  3S.IFV-unite  =on  all  \textsc{ID}many



‘The people were all united together.’



\textit{42.}\textbf{\textit{  Tápa}}\textbf{\textit{ɗay. }}



\textit{tá-pa}\textit{ɗ{}-}\textit{aj} 



3P.IFV-crunch{}-CL



‘They ate [the meat].’



\textit{43.  }\textbf{\textit{Tágaka  h}}\textbf{\textit{ə}}\textbf{\textit{rnje  bay.}}



\textit{tá-g    =aka   hɪrnʒɛ   baj} 



3P.IFV-do  =on  hate  \textsc{NEG}



‘On top of that, they divided it without hate.’ (lit. they did no hate)



\textit{44.  }\textbf{\textit{Nde  ehe  na,   }}



\textit{ndɛ   ɛhɛ      na   }



so  here   PSP     



‘So, here,’



\textbf{\textit{c}}\textbf{\textit{əcəngehe  na  }}\textbf{\textit{m}}\textbf{\textit{əze  }}\textbf{\textit{ahay  tanda}}\textbf{\textit{ɗ}}\textbf{\textit{ay  m}}\textbf{\textit{əze  }}\textbf{\textit{asabay  pat.}}



\textit{tʃɪtʃɪŋgɛhɛ  na   mɪʒɛ  =ahaj   ta-nda}\textit{ɗ{}-}\textit{aj     mɪʒɛ  asa-baj     pat}



now           PSP  person  =Pl  3P-like-CL    person  again-\textsc{NEG}  all



‘[and] now, people don’t like each other at all any more.’



\textit{45.  }\textbf{\textit{Se  məze  aməde}}\textbf{\textit{ɗe  }}\textbf{\textit{məze  ehe  na,   }}



\textit{ʃɛ    mɪʒɛ     amɪ-dɛ}\textit{ɗ{}-}\textit{ɛ  mɪʒɛ  ɛhɛ     na   }



only    person  DEP-like-CL  person  here  PSP  



\textbf{\textit{c}}\textbf{\textit{əcəngehe  }}\textbf{\textit{na,  se  ngomna.   }}



\textit{tʃɪtʃɪŋgɛhɛ  na   ʃɛ   ŋg}\textit{\textsuperscript{w}}\textit{ɔmna}



now    PSP  only  government



‘The only person that likes people now is the government.’ (sarcastic)



\textit{46.  }\textbf{\textit{Ng}}\textit{o}\textbf{\textit{mna  na,    ele  aga  kə  wəyen  aka  na,}}



\textit{ŋg}\textit{\textsuperscript{w}}\textit{ɔmna   na  ɛlɛ   à-g-a     kə  wijɛŋ   aka  na}



government  PSP  thing  3S.PFV-do  on    earth  on      PSP  



‘The government, [if]  there is a problem (lit. a thing does) on the earth,’



\textbf{\textit{ndahan  na  ágas  na  təta.}}



\textit{ndahaŋ   na  á-gas        na   təta}



3S    PSP  3S.IFV-catch  3S.DO  ABILITY



‘it (the government) will be able to take care of it.’  (lit. he, he can catch it)



\textit{47.  }\textbf{\textit{Waya  ləme  Məloko  ahay  na,  nəmbə}}\textbf{\textit{ɗo}}\textbf{\textit{m  a  dəray  ava  na,     }}



\textit{waja  lɪmɛ  Mʊlɔk}\textit{\textsuperscript{w}}\textit{ɔ  =ahaj    na   nə-mbʊ}\textit{ɗ{}-}\textit{ɔm      a  dəraj  ava  na   }



because    1\textsc{Pex}     Moloko    =Pl  PSP  1.PFV-change-1Pex  in   head  in    PSP



‘Because we the Moloko, have changed in our head,’  



\textbf{\textit{ka  kərka}}\textbf{\textit{ɗ}}\textbf{\textit{aw  ahay  nə  hərgov  ahay  ga  a  }}\textbf{\textit{ɓ}}\textbf{\textit{ərzlan  ava  na,}}



\textit{ka    kərka}\textit{ɗ}\textit{aw    =ahaj  nə   hʊrg}\textit{\textsuperscript{w}}\textit{ɔv  =ahaj    ga     a      }\textit{ɓ}\textit{ərɮaŋ     ava    na}



like  monkey    =Pl        with  baboon    =Pl  ADJ  in  mountain    in    PSP



‘like monkeys and baboons on the mountains,’



\textbf{\textit{ka ala  kəra  na,  nəsərom  dəray  bay  pat.     }}



\textit{ka   =ala   kəra      na  nə-sʊr-ɔm     dəraj   baj   pat     }



like  =to  dog   PSP  1.PFV-know-1Pex  head  \textsc{NEG}  all



‘[and] like dogs, we don’t understand anything!’ 



\textit{48.  }\textbf{\textit{Kə  wəyen  aka  ehe  tezl tezlezl.}}


\textit{kə   wijɛŋ   aka   ɛhɛ  tɛɮ tɛɮɛɮ}


on  earth  on  here  \textsc{ID}hollow



‘[Among the people] on this earth here, [we are like] the sound of a hollow cup bouncing on the ground.’ (lit. on the earth here, hollow)



\textit{49.  }\textbf{\textit{Nde  məze  ahay  gogor  ahay  ga  na,  ngama.}}


\textit{ndɛ  mɪʒɛ  =ahaj  g}\textit{\textsuperscript{w}}\textit{ɔg}\textit{\textsuperscript{w}}\textit{ɔr   =ahaj ga  na   ŋgama}


so    person     Pl  elder   =Pl   ADJ   PSP  better



So, our elders [had it] better. 



\textit{50.  }\textbf{\textit{Epele epele  na  me,  Hərmbəlom  anday  agas  ta   }}



\textit{ɛpɛlɛ ɛpɛlɛ   na     mɛ        Hʊrmbʊlɔm  à-ndaj       à-gas          ta   }



in the future  PSP  opinion  God           3S.PFV-PROG  3S.PFV-catch  3P.DO  



‘And so in the future in my opinion, God is going to accept them [the elders]’



\textbf{\textit{a  ahar  ava  re.}}



\textit{a    ahar  ava   rɛ}



to   hand  in      sure



‘in his hands, in spite [of what the church says].’ 



\textit{51.}\textbf{\textit{  Ádal  hwəsese  ga.}}


\textit{á-dal    h}\textit{\textsuperscript{w}}\textit{ʊʃɛʃɛ  ga}


3S.IFV-surpass  small  ADJ



‘It is greater than the small ones.’



\textit{52.   }\textbf{\textit{Nde  na,  kəygehe.}}


\textit{ndɛ    na   kijgɛhɛ}


so    PSP  like this    



‘So [it is] like this.’


\chapter[Phonology]{Phonology}
\hypertarget{RefHeading1210401525720847}{}
The phonology of Moloko has been fully discussed by \citet{Bow2011}. The following is a summary of the aspects that are necessary to understand the grammar, with focus on the new work that has been done since her manuscript was initially published. 

Bow based her original phonological outline of \citet{Moloko1997c} on a database she compiled consisting of around 1500 words, including some 400 verbs and 1000 nouns.  This phonology was later published as an online publication in 2011.  Bow’s database was modified and extended by \citet{Boyd2002} with a focus on nouns. Later, Mamalis built on their work to describe the tone on verbs, and Friesen discussed phonological word structure of the verb word in Friesen and \citet{Mamalis2008}. 

Three inter-related phonological factors must be touched on before a discussion of any of them can be fully understood. The first is that Moloko words are built on a consonantal skeleton with only one underlying vowel /a/ (phonetically expressed as the ‘full vowels’ [a, o, œ, ᴂ, e], see \sectref{sec:2.3}) that occurs between only some of the consonants.\footnote{\citet{Bow1997a} used the distinction +/- Low, which focused on one phonetic feature, however we have found that the most salient issue in discussing the vowel patterns of this language is the concept of ‘full’ vs. epenthetic vowels.  For clarity, therefore, this work will use the terms ‘full’ and ‘epenthetic’ to distinguish between the two sets of vowel phones, with ‘full’ referring to /a/ and its prosodically conditioned allophones, and ‘epenthetic’ referring to schwa and its allophones.} Some consonant clusters (caused by the absence of an underlying vowel between them) are broken up by epenthetic schwa insertion when they are pronounced (and phonetically expressed as [ə, ʊ, u, ø, ɪ, i]).\footnote{Likewise in Muyang, another Central Chadic language closely related to Moloko, it can be shown that syllables are built postlexically from the consonant skeleton by regular rules. Only the low vowel /a/ is phonemic, and all high vowels can be accounted for by means of epenthesis (Smith, 1999). } Although syllable structure will be mentioned in this work, attention will be focussed on the underlying consonantal skeleton. \citet[15]{Roberts2001} notes for Central Chadic languages, 

\begin{quote}
{}[because] “the consonant skeleton is all-important to the phonological structure, the traditional unit of the syllable is much less useful in the description of Central Chadic languages since at the core of every syllable must be a vowel (or some syllabic segment, at least). And in fact, it can be shown for most of these languages that the syllable is a very superficial phenomenon.”
\end{quote}

(And further on p. 16) 

\begin{quote}
“We conclude then that the syllable is not a unit that can be exploited as it is in other languages to elucidate the phonological  structure. It is a surface structure phenomenon whose character is completely predictable from other phonological aspects of lexemes. On the other hand, an underlying structure that is more worthy of study in Central Chadic languages is that of the consonant skeleton that can take up lexical roots; to this core are added other peripheral phonological elements such as vowels, prosodies, and tones.”
\end{quote}

The second basic phonological factor for Moloko is that all of the vowels (both ‘full’ and epenthetic) and some of the consonants are affected by word-level labialisation or palatalisation prosodies\footnote{Prosodies in Chadic languages are word-level suprasegmental processes that labialise or palatalise the entire word and affect all vowels and some consonants. See \citet{Roberts2001} for a fuller discussion. } (see \sectref{sec:2.1}). These prosodies account for most of the vowel and consonant allophones in the language. Palatalisation can be part of certain morphemes, but Moloko is unlike other Chadic languages where palatalisation and labialisation alone can have morphemic status (for example in Muyang where the application of the palatisation prosody on a noun produces a diminutive, and application of the labialisation prosody produces an augmentative, Smith, personal communication).

The third basic factor is that the final syllable before a pause is stressed in pronunciation. The stressed syllable necessitates a ‘full’ vowel, meaning that any epenthetic vowel in that syllable will be changed to its full counterpart.  The following two example pairs each show the same word in non-stressed and stressed environments. Compare [zij] (non-stressed with epenthetic vowel) with [zaj] (stressed with full vowel) in ex. 1 and 2, and [nɔ-zʊm] (non-stressed with epenthetic vowel in final syllable) with [nɔ-zɔm] (stressed with full vowel) in ex. 3 and 4. 


\ea
\textup{[zij  ɗaw ]}\textup{   }
\z

peace  QUEST



‘Is there peace?’ 


\ea
\textup{[zaj ]}
\z

peace



‘There is peace.’ 


\ea
\label{bkm:nozomdaf}\textup{[nɔ-zʊm   ɗaf}\textup{ ]}
\z

1S.PFV-eat  boule



‘I ate boule.’


\ea
\textup{[nɔ-zɔm}\textup{ ]}
\z

1S.PFV-eat



‘I ate.’


Due to these interrelated factors, much of the phonological discussion will require representation of both the underlying and surface forms of lexemes. The underlying form  consists of the consonant and vowel phonemes (written between slashes) and the word prosody (written as a superscripted \textsuperscript{e} for palatalisation, \textsuperscript{o} for labialisation at the right of the morphemes). A neutral prosody has no superscript. The following examples illustrate the phonetic forms (between square brackets) and underlying forms (between slashes) of nouns that are palatalised (ex. 5), labialised (ex. 6), and neutral with respect to prosody (ex. 7). All of the examples in this work will be presented in the phonetic form unless otherwise indicated. 

\ea
\textup{[mɪdɪgɛr ]  / m d g r }\textup{\textsuperscript{e}}\textup{ /}
\z

‘hoe’


\ea
\textup{[lʊhɔ ]   / l ha }\textup{\textsuperscript{o}}\textup{ /}
\z

‘late afternoon’


\ea
\textup{[ɗaf ]  /ɗ f /}
\z

‘millet boule’  


The phonetic forms of the examples given in this paper are all in citation form (the form of the word when it is pronounced in isolation), and therefore show each word with a stressed final syllable.  In each case, the final syllable (whether open or closed) always contains a full vowel, regardless of whether the underlying form has a full vowel or not.  

The phonology section of the present work begins with a description of the prosodies of labialisation and palatalisation and their effects (\sectref{sec:2.1}), which leads to a description of the consonant and vowel systems (Sections 2.2 and 2.3, respectively). An examination of the tone system follows (\sectref{sec:2.4}). Finally, notes on the syllable and word breaks are discussed (see Sections 2.52.5 and 2.6, respectively). The appendix (\sectref{sec:14.1}) includes a list of verbs used in this analysis, showing their Imperative form, underlying form, and underlying tone.

\section{Labialisation and palatalisation prosodies}
\hypertarget{RefHeading1210421525720847}{}
One of the most basic phonological processes in Moloko is prosody.  Chadic linguists refer to prosody as a suprasegmental process where a labialisation or palatalisation feature is applied to a phonological word. \citet{Wolff1981} refers to prosodies as suprasegmental \textstyleQuoteChar{\textup{sources of palatalisation and labiovelarisation.}}

\citet{Bow1997c} has discovered that labialisation and palatalisation work at the morpheme level in Moloko. Both prosodies are attached to a particular morpheme and spread leftward over the entire phonological word. Labialisation affects the back consonants (k, g, ŋɡ, and h) and vowels; palatalisation affects alveolar fricatives (s and z), affricates (ts and dz), and vowels (see Sections 2.2 and 2.3).  All Moloko words are either labialised, palatalised, or are neutral with respect to prosody. Recent work demonstrates that some syllables can be affected by both labialisation and palatalisation (see \sectref{sec:4} and 5).  

As stated above, in this work prosody is indicated in the underlying form using superscript symbols included at the right edge of the word: / \textsuperscript{o}/ to represent labialisation and / \textsuperscript{e}/ to represent palatalisation.  In the phonetic form, the prosody is indicated by the quality of the full vowel in the word ([ɔ] for labialisation, [ɛ] for palatalisation, and [a] for no  prosody (see \sectref{sec:2.3}). The following examples from \citet{Bow1997c} give evidence of contrast between the prosodies in a minimal triplet:


\ea
\textup{/k ra/      [kəra]}\textup{    }
\z

‘dog’


\ea
\textup{/k ra }\textup{\textsuperscript{o}}\textup{/      [k}\textup{\textsuperscript{w}}\textup{ʊrɔ]    }
\z

‘ten'    


\ea
\textup{ }\textup{/k ra }\textup{\textsuperscript{e}}\textup{/      [kɪrɛ]}\textup{    }
\z

‘stake/post’


The effects of both prosodies on a single underlying form can be seen in the paradigm for the verb /mnzar/ ‘see’ shown in \tabref{tab:1}. (adapted from Bow, 1997c). The verb stem is bolded in the table. The 2S imperative is neutral with respect to prosody, while the 2P imperative form involves a labialisation prosody and the addition of a suffix /-am \textsuperscript{o}/ (see \sectref{sec:49}). The nominalised form carries a palatalisation prosody, and involves the addition of both a prefix /m{}-/ and suffix /-a \textsuperscript{e}/. Note that vowels and some consonants are affected by the prosodies. The vowel /a/ is realised as [ɔ] in labialised forms, and [ɛ] in palatalised forms, while [ə] is realised as [ʊ] in labialised forms and [ɪ] in palatalised forms (see \sectref{sec:5}). The consonant /nz/ is realised as [nʒ] in palatalised forms (see \sectref{sec:2}). 

\begin{tabular}{llll} & \textbf{Underlying form} & \textbf{Phonetic form} & \textbf{Gloss}\\
\lsptoprule
\textbf{2S Imperative form} & \textsc{/}\textbf{m n}\textbf{za r}/ & [\textbf{mənzar}] & ‘see! (2S)’\\
\textbf{2P Imperative form} & \textsc{/}\textbf{m n}\textbf{za r}{}-am\textsuperscript{o}/ & \textsc{[}\textbf{m}\textbf{ʊ}\textbf{n}\textbf{zɔr}ɔm\textsc{]} & ‘see! (2P)’\\
\textbf{Nominalised form}\textbf{\textsc{:}} & \textsc{/}m-\textbf{m n}\textbf{za r}{}-a\textsuperscript{e}/ & \textsc{[}mɪ\textbf{mɪnʒɛr}ɛ\textsc{]} & ‘seeing’\\
\lspbottomrule
\end{tabular}

\begin{itemize}
\item \begin{styleTabletitle}
Paradigm for /mnzar/
\end{styleTabletitle}\end{itemize}

Labialisation and palatalisation prosodies are lexical features that are applied to a morpheme, and can spread over an entire word. A prosody in the root will spread to a prefix. Compare the prosody in the subject prefixes of the following verbs. In ex. 11, the root is labialised, in ex. 12, the root is palatalised, and in ex. 13, the root is neutral. The underlying forms are given in the examples. 


\ea
\textup{[nɔ-zɔm}\textup{]      /na-  z m}\textup{\textsuperscript{o}}\textup{/}
\z

1S-eat



‘I eat.’


\ea
\textup{[nɛ-ʃ{}-ɛ}\textup{]       /na-  s-aj}\textup{\textsuperscript{o}}\textup{/}
\z

1S-drink{}-CL



‘I drink.’


\ea
\textup{[na-zaɗ}\textup{ ]       /na-  z ɗ}\textup{\textsuperscript{ }}\textup{/}
\z

1S-take



‘I take.’


When initiated by a suffix carrying a prosody, the prosody spreads leftwards, affecting all morphemes within the word including prefixes.\footnote{When the prosody is neutral, the prosody on the root is neutralised (see examples 16 and 17).} The following examples contrast the vowels and consonants in a verb root with no underlying prosody ([kaɬ] ‘wait’) in the second person singular verb form (ex. 14, also prosody neutral) and the second person plural which contains the labialised suffix /-ak \textsuperscript{o}/ (ex. 15). The underlying forms are given in each example. Note that the prosody does not spread to the right across word boundaries since \textit{na}, a separate word, is not affected by the prosody of the verb stem (nor does it neutralise the prosody on the verb).

\ea
\textup{[ka-kaɬ}\textup{  na]      /ka-  ka ɬ  na/}
\z

2S.PFV-wait    3S.DO



You waited [for] it.


\ea
\textup{[mɔ-kɔɬ-ɔkʷ}\textup{      na]     /ma-  ka ɬ  {}-ak}\textup{\textsuperscript{o}}\textup{  na/}
\z

1\textsc{Pin}.PFV-cultivate-1\textsc{Pin}  3S.DO



We waited [for] it. 


Palatalised verbs almost always have a palatalised suffix [{}-ɛ ] (see \sectref{sec:6.6}).\footnote{With the exception of verb stems whose final consonant is /n/, eg,̣ [\textit{t}\textit{ʃ}\textit{ɛŋ}], /tsan\textsuperscript{e }/, ‘know’.}  Whenever there is another suffix or enclitic attached to the verb stem, the [-ɛ] is deleted, taking with it the palatalisation prosody (see \sectref{sec:6.3}). The verb becomes neutral with respect to prosody, as is shown by ex. 16 - 17. In ex. 16, the verb ends with [{}-ɛ] and the entire verb form is palatalised. In ex. 17, the enclitic [=va] has replaced the [{}-ɛ] and the entire verb form is neutral in prosody.

\ea
nɛ{}-tʃɪk-ɛ    \textup{/n-   ts k }\textup{\textsuperscript{e}}\textup{/}\textup{    }
\z

1S-move-CL



‘I move.’      


\ea
nə-tʃək-ə=va    \textup{/n-   ts k }\textup{\textsuperscript{e}}\textup{   =va/}\textup{    }
\z

1S-move-CL  =\textsc{PRF}



‘I moved already.’


\citet{Bow1997c} found that prosodies seem to have the least effect on word-initial V syllables. She notes that in palatalised words, the first syllable of nouns that begin with /a/ will sometimes be completely palatalised and pronounced  [ɛ]. However, often it will have an incomplete palatalisation and be pronounced [æ] or even [a]. See the alternate pronunciations that Bow has found for the words /a- la la\textsuperscript{e}/ (ex. 18) and /a- nd ɓ\textsuperscript{e}/ (ex. 19). Palatalisation is a stronger process than labialisation. In labialised words, the first syllable in words that begin with /a/ will often\footnote{Bow found these first syllables always unaffected by labialisation; Friesen has found that some speakers do pronounce vowel-initial syllables with labialisation [ɔ].} be unaffected by the labialisation and be pronounced [a] (see the alternate pronunciations for the words /a- la ka\textsuperscript{o}/ in ex. 20 and /a- g ra\textsuperscript{o}/ ex. 21). 

\ea
\textup{[alɛlɛ] {\textasciitilde} [ælɛlɛ] {\textasciitilde} [ɛlɛlɛ]  }
\z

‘leaf sauce’        

\ea
\textup{[andɛɓ] {\textasciitilde} [ændɛɓ] {\textasciitilde} [ɛndɛɓ]}
\z

‘brain’

\ea
 \textup{[alɔkʷɔ] {\textasciitilde} [ɔlɔkʷɔ]      }
\z

 ‘fire’        


\ea
\textup{[aɡʊrɔ] {\textasciitilde}[ɔɡ}\textup{\textsuperscript{w}}\textup{ʊrɔ]}
\z

‘gold’


\section{Consonants}
\hypertarget{RefHeading1210441525720847}{}
Bow reported 31 consonant phonemes{.}\footnote{\citet{Bow1997c} described 30 consonant phonemes although her chart of consonant phonemes included ŋɡʷ, making the total 31. } Since her work, the labiodental flap /ⱱ/ in Moloko has been noted, making the total 32 consonantal phonemes.  

The labio-dental flap /ⱱ / was first described by Olson and \citet{Hajek2004} and is typical of many of the Chadic languages in the Far North Province of Cameroon. In Moloko it is found in ideophones (ex. 22, 23, Chapter 3.6). 


\ea
\textup{[}\textup{ⱱ}\textup{aɓ]}\textup{ }
\z

‘snake falling’


\ea
\textup{[}\textup{ɓ}\textup{a}\textup{ⱱ}\textup{aw]}\textup{ }
\z

‘men running’ 


Moloko has three sets of sequences which Bow has interpreted as single units (C) rather than sequences of two consonants (CC).  These are pre-nasalised consonants [mb, nd, ŋɡ, nz], affricates [ts, dz], and labialised consonants [kʷ, ɡʷ, ŋɡʷ, hʷ]. In the case of pre-nasalised consonants, the nasal is always homo-organic with the following consonant.\footnote{\textstyleExampleglossChar{Note that the phoneme /n/ assimilates to the point of articulation of a following consonant throughout the langauge.}} Only voiced consonants are pre-nasalised. 

Allophonic variation for consonants occurs in Moloko due to prosodic conditioning (\sectref{sec:2}) and word final variations (\sectref{sec:3}). There is a relationship between consonants and tone which will be considered in \sectref{sec:2.6.} 

%%please move \begin{table} just above \begin{tabular
\begin{table}
\caption{(adapted from Bow, 1997c) shows place and manner of articulation of all phonetic realisations of consonants in Moloko.  Allophones are shown in parentheses. The individual phonemes and their allophones will be considered in Sections 1 - 3.}
\label{tab:2}
\end{table}

\begin{tabular}{llllll} &  & \textbf{Labial} & \textbf{Alveolar} & \textbf{Velar / Glottal} & \textbf{Labio-Velar}\\
\lsptoprule
\textbf{Stops} & {}-voice & p & t & k & kʷ\\
& +voice & b & d & ɡ & ɡʷ\\
& nasal & m & n   (ŋ) &  & \\
& pre-nasal & mb & nd & ŋɡ & ŋɡʷ  \\
& implosive & ɓ & ɗ &  & \\
\textbf{Affricates} & {}-voice &  & ts   (tʃ) &  & \\
& +voice &  & dz  (dʒ) &  & \\
& pre-nasal &  & nz  (nʒ) &  & \\
\textbf{Fricatives} & {}-voice & f & s    (ʃ) & h    (x) & hʷ\\
& +voice & v & z    (ʒ) &  & \\
\textbf{Lateral }

\textbf{fricatives} & {}-voice &  & ɬ &  & \\
\hhline{-~~~~~} & +voice &  & ɮ &  & \\
\hhline{~-----}
\textbf{Lateral approximants} &  &  & l &  & \\
\textbf{Approximants} &  &  & j &  & w\\
\textbf{Flaps} &  & ⱱ & r &  & \\
\lspbottomrule
\end{tabular}

\begin{itemize}
\item \begin{styleTabletitle}
 Consonant phonemes
\end{styleTabletitle}\end{itemize}
\subsection{Phonetic description}
\hypertarget{RefHeading1210461525720847}{}
This list of phonemes and allophones with phonetic description is adapted from \citet{Bow1997c} with additions from our work done since then. The phoneme (inside slashes), the phonetic form (in square brackets), and the orthographic form (non-bracketed) are shown for each consonant phone. All sounds are made with egressive lung air except where otherwise stated (i.e. implosives are made with ingressive pharynx air). The orthography is discussed in \citet{Friesen2001}. The orthography conforms to the General Alphabet for Cameroonian Languages. All examples in the paper are written using both the orthography (top line) and phonetic transcription so that both speakers of Moloko and outside linguists can appreciate them. 

/p/p  [p]    voiceless bilabial unaspirated stop

/b/   b  [b]    voiced biliabial stop

/mb/  mb  [mb]    pre-nasalised voiced bilabial stop        

/m/  m  [m]    voiced bilabial nasal

/ɓ/  ɓ  [ɓ]    voiced bilabial stop with ingressive pharynx air (implosive)

/f/  f  [f]    voiceless labio-dental fricative

/v/  v  [v]    voiced labio-dental fricative

/t/  t  [t]    voiceless alveolar unaspirated stop

/d/  d  [d]    voiced alveolar stop

/n/  n  [n]    voiced alveolar nasal

{}[ŋ]    voiced velar nasal occurring word-finally 

/nd/  nd  [nd]    pre-nasalised voiced alveolar stop        

/ɗ/  ɗ  [ɗ]    voiced alveolar stop with ingressive pharynx air (implosive)

/ts/  c  [ts]    voiceless alveolar affricate occurring in unpalatalised syllables

{}[tʃ]    voiceless alveopalatal affricate occurring in palatalised syllables

/dz/  j  [dz]    voiced alveolar affricate occurring in unpalatalised syllables

  [dʒ]    voiced alveopalatal affricate occurring in palatalised syllables

/s/  s  [s]    voiceless alveolar fricative occurring in unpalatalised syllables

  [ʃ]    voiceless alveopalatal fricative occurring in palatalised syllables

/z/  z  [z]    voiced alveolar fricative occurring in unpalatalised syllables

  [ʒ]    voiced alveopalatal fricative occurring in palatalised syllables

/nz/  nj  [nz]    pre-nasalised voiced alveolar nasal occurring in unpalatalised syllables

    [nʒ]    pre-nasalised voiced alveopalatal nasal occurring in palatalised syllables

/ɬ/  sl  [ɬ]    voiceless alveolar lateral fricative

/ɮ/  zl  [ɮ]    voiced alveolar lateral fricative   

/l/  l  [l]    voiced alveolar lateral approximant

/r/  r  [r]    voiced alveolar flap

/ⱱ/  vb  [ⱱ]    voiced labiodentals flap

/j/  y  [j]    voiced palatal semi-vowel

/k/  k  [k]    voiceless velar unaspirated stop occurring in unlabialised syllables

  [kʷ]    voiceless labialised velar stop occurring in labialised words

/kʷ/  kw/wk\footnote{Orthographically, ‘kw’ is word-initial and word medial, ‘wk’ is word-final.}[kʷ]    voiceless labialised velar stop          

/ɡ/  g  [ɡ]    voiced velar stop occurring in unlabialised syllables

  [ɡʷ]    voiced labialised velar stop occurring in labialised syllables

/ɡʷ/  gw  [ɡʷ]    voiced labialised velar stop    

/ŋɡ/  ng  [ŋɡ]    pre-nasalised voiced velar stop occurring in unlabialised syllables

  [ŋɡʷ]    voiced pre-nasalised labialised velar stop occurring in labialised syllables

/ŋɡʷ/  ngw  [ŋɡʷ]    voiced pre-nasalised labialised velar stop

/h/  h  [h]    voiceless glottal fricative occurring word-medially

  [x]    voiceless velar fricative occurring word-finally  

/hʷ/  hw  [hʷ]    voiceless labialised glottal fricative        

/w/  w  [w]    voiced labio-velar semi-vowel

\subsection{Underlyingly labialised consonants}
\hypertarget{RefHeading1210481525720847}{}
Bow posited the existence of a set of underlyingly labialised consonant phonemes [k\textsuperscript{w}, ɡ\textsuperscript{w}, ŋɡ\textsuperscript{w}, h\textsuperscript{w}]. She showed them to be phonemes even though each of these consonants is also the realisation in labialised words of their non-labialised counterpart (see \sectref{sec:2}). At the surface phonetic level, Bow showed that a labialised velar\textsuperscript{ }can have two possible sources, either a labialisation prosody across the whole word (ex. 24), or the presence of an underlyingly labialised consonant (ex. 25). Ex. 24 shows consistently labialised vowels indicating labialisation across whole word; while the palatalised vowels in ex. 25 indicate that there is a palatalisation prosody across the whole word; with the presence of an underlyingly labialised velar consonant.x17


\ea
\textup{/dz ɡ r }\textup{\textsuperscript{o}}\textup{/  [dzʊɡʷɔr]}\textup{   }
\z

‘stake’ 



x43


\ea
\textup{/}\textup{\textsuperscript{ }}\textup{dza ɡ}\textup{\textsuperscript{w }}\textup{r }\textup{\textsuperscript{e}}\textup{/  [dʒœɡʷɛr]}\textup{   }
\z

‘limpness’ 


Bow found underlyingly labialised consonants in words which do not have a labialisation prosody across the whole word.  She concluded that the labialisation feature was attached only to these velar consonants within a word since the prosody only affected those particular consonants and the vowels immediately adjacent to them, while other consonants and vowels within the word were unaffected by the labialisation prosody.\footnote{Another interpretive option could be positing that the labialisation prosody touches down on the velar consonant but something prevents it from spreading to the rest of the word (Smith, personal communication). For the purposes of this work, we will consider the labialised velar to be a separate phoneme rather than a supra-segmental phenomenon.}  

%%please move \begin{table} just above \begin{tabular
\begin{table}
\caption{(adapted from Bow, 1997c) shows two pairs of words that are distinguished by the contrast between the underlyingly labialised and non-labialised velars.}
\label{tab:3}
\end{table}

\begin{tabular}{llllll}
\lsptoprule

\multicolumn{3}{l}{\textbf{Word-level prosody}} & \multicolumn{3}{l}{\textbf{Labialised consonant}}\\
\textbf{Underlying form} & \textbf{Phonetic form} & \textbf{Gloss} & \textbf{Underlying form} & \textbf{Phonetic form} & \textbf{Gloss}\\
/s l k\textsuperscript{ e}/ & [ʃɪlɛk] & ‘jealousy’ & /s l kʷ  \textsuperscript{e}/ & [ʃɪlœkʷ] & ‘broom’\\
/ɡ la \textsuperscript{o}/ & [ɡʷʊlɔ] & ‘left’ & /ɡʷ la/ & [ɡʷʊla] & ‘son’\\
/ha ɗa \textsuperscript{o}/ & [hʷɔɗɔ] & ‘wall’ & /hʷa ɗa/ & [hʷɔɗa] & ‘dregs’\\
\lspbottomrule
\end{tabular}

\begin{itemize}
\item \begin{styleTabletitle}
Minimal pairs for word-level labialised prosody vs. labialised consonant
\end{styleTabletitle}\end{itemize}
%%please move \begin{table} just above \begin{tabular
\begin{table}
\caption{illustrates words containing each of the labialised velar phonemes. The labialised velars may occur as the word-initial consonant, medial consonant in palatalised words or words of neutral prosody. Only voiceless labialised velars can occur in word-final position (see \sectref{sec:3}). It is interesting that there are no words of neutral prosody which can have a labialised velar in word final position. Note that only the vowels that immediately surround a labialised velar consonant are affected by the prosody of the velar consonant (see \sectref{sec:5} and 6).}
\label{tab:4}
\end{table}

\begin{tabular}{llll} & \textbf{Word initial} & \textbf{Word medial} & \textbf{Word final}\\
\lsptoprule
\textbf{Neutral prosody} & [kʷʊsaj]     ‘haze’ & [tʊkʷʊrak]     ‘partridge’

{}[aɡʷɔɮak]      ‘rooster’ & \\
\textbf{Palatalisation} & [kʷʊtʃɛɬ]     ‘viper’

{}[ɡʷʊdɛɗɛk] ‘frog’ & [mɛtʃœkʷɛɗ]  ‘maggot’

{}[mɛdɛlœŋɡʷɛʒ] ‘leopard’

{}[ah\textsuperscript{w}œɗɛ]       ‘fingernail’ & [pɛɗœkʷ]       ‘blade’\\
\lspbottomrule
\end{tabular}
\begin{itemize}
\item \begin{styleTabletitle}
Distribution of labialised velar phonemes
\end{styleTabletitle}\end{itemize}

Bow found there are several cases in the data where it was impossible to tell whether the consonant is underlyingly labialised or there is a labialisation prosody across the word, as in the ex. 26 and 27 (from Bow, 1997c).


\ea
\textup{/s kʷ m/ {\textasciitilde} /s k m }\textup{\textsuperscript{o}}\textup{/       [sʊkʷɔm]}\textup{  }
\z

‘buy/sell’      


\ea
\textup{ }\textup{/ma ɡʷ m/ {\textasciitilde}  /ma ɡ m }\textup{\textsuperscript{o}}\textup{/    [mɔɡʷɔm]}
\z

‘home’


Further work on verb conjugations clarified that ex. 26 actually contains a labialised velar (i.e., the underlying form is /s kʷ m/). Ex. 28 shows the nominalised form of the verb which is palatalised, yet the labialised velar is still present. If there was no underlyingly labialised velar, the nominalised form would have been *[mɪsɪkɪmɛ]

\ea
mɪ-sɪkʷøm-ɛ
\z

\textsc{NOM}{}-buy-CL



‘buying’


\subsection{Prosodic conditioning of consonant allophones}
\hypertarget{RefHeading1210501525720847}{}%%please move \begin{table} just above \begin{tabular
\begin{table}
\caption{(adapted from Bow, 1997c) shows the effect of prosodic conditioning on each consonant phone.  Each consonant phone (reading down the table) is shown in three environments, one without any prosody, one with a labialisation and one with a palatalisation prosody.{ }The table illustrates that prosody has an effect on fricatives, affricates, and back consonants (velar and glottal).}
\label{tab:5}
\end{table}

The fricatives [s, z, nz] and affricates [ts, dz] are in complementary distribution with [ʃ, ʒ, nʒ] and [tʃ, dʒ], respectively, with the second group only appearing in palatalised words.  

Labialisation affects the back consonants such that [k, ɡ, ŋɡ, h] are in complementary distribution with [k\textsuperscript{w}, ɡ\textsuperscript{w}, ŋɡ\textsuperscript{w}, h\textsuperscript{w}], respectively, with the second group only appearing in labialised words. Note however that there is a set of underlyingly labialised back consonant phonemes (see Section  1).

Note also that the labiodental flap [ⱱ] is found only in ideophones (Chapter 3.6) which have a neutral prosody.

\begin{tabular}{llllllll} &  & \textbf{Neutral} & \textbf{Gloss} & \textbf{Labialised} & \textbf{Gloss} & \textbf{Palatalised} & \textbf{Gloss}\\
\lsptoprule
\textbf{Stops} & p & [paj] & ‘open’ & [apɔŋɡʷɔ] & ‘mushroom’ & [pɛmbɛʒ] & ‘blood’\\
& b & [baj] & ‘light’ & [abɔr] & ‘lust’ & [bɛkɛ] & ‘slave’\\
& ɓ & [ɓaj] & ‘hit’ & [aɓɔlɔ] & ‘yam’ & [ɓɛɮɛŋ] & ‘count’\\
& m & [maj] & ‘hunger’ & [mɔlɔ] & ‘twin’ & [amɛlɛk] & ‘bracelet’\\
& mb & [mbaj] & ‘follow’ & [ambɔlɔ] & ‘bag’ & [mbɛ] & ‘argue’\\
& t & [tar] & ‘call’ & [atɔs] & ‘hedgehog’ & [tɛʒɛh] & ‘boa’\\
& d & [dar] & ‘burn’ & [dɔkʷaj] & ‘arrive’ & [dɛ] & ‘cook’\\
& ɗ & [ɗas] & ‘weigh’ & [ɗɔɡʷɔm] & ‘nape’ & [ɗɛ] & ‘flourish’\\
& n & [nax] & ‘ripen’ & [sɔnɔ] & ‘joke’ & [ɛnɛŋ] & ‘snake’\\
& ŋ & [ɮaŋ] & ‘start’ & [tɔlɔlɔŋ] & ‘heart’ & [ɓɛɮɛŋ] & ‘count’\\
& nd & [ndar] & ‘weave’ & [ndɔɮaj] & ‘explode’ & [ndɛ] & ‘lie down’\\
& k & [kaɬ] & ‘wait’ &  &  & [bɛkɛ] & ‘slave’\\
& ɡ & [ɡar] & ‘grow’ &  &  & [ɡɛ] & ‘do’\\
& ŋɡ & [ŋɡaj] & ‘set’ &  &  & [fɛŋɡɛ] & ‘termite mound’\\
& kʷ & [kʷʊsaj] & ‘fog’ & [kʷɔndɔŋ] & ‘banana’ & [ajœkʷ] & ‘ground nut’\\
& ɡʷ & [aɡʷɔɮak] & ‘cockerel’ & [ɡʷɔrɔ] & ‘kola’ & [dʒœɡʷɛr] & ‘limpness’\\
& ŋɡʷ & [ŋɡʷʊdaɬaj] & ‘simmer’ & [aŋɡʷɔlɔ] & ‘return’ & [adɔnɡʷɛrɛɗ] & ‘type of tree’\\
\textbf{Fricatives}

\textbf{and Affricates} & f & [far] & ‘itch’ & [fɔkʷaj] & ‘whistle’ & [fɛ] & ‘play instrument’\\
& v & [vaj] & ‘winnow’ & [avɔlɔm] & ‘ladle’ & [vɛ] & ‘spend (time)’\\
& s & [sar] & ‘know’ & [sɔnɔ] & ‘joke’ &  & \\
& z & [zaj] & ‘peace’ & [zɔm] & ‘eat’ &  & \\
& ts & [tsar] & ‘climb’ & [tsɔkʷɔr] & ‘fish net’ &  & \\
& dz & [dzaj] & ‘speak’ & [dzɔɡʷɔ] & ‘hat’ &  & \\
& nz & [nzakaj] & ‘find’ & [nzɔm] & ‘sit down’ &  & \\
& h & [haj] & ‘millet’ &  &  & [mɛhɛr] & ‘forehead’\\
& x & [rax] & ‘satisfy’ &  &  & [tɛʒɛx] & ‘boa’\\
& hʷ & [hʷɔɗa] & ‘dregs’ & [h\textsuperscript{w}ɔr] & ‘woman’ & [ahʷœɗɛ] & ‘fingernail’\\
& ʃ &  &  &  &  & [ʃɛ] & ‘drink’\\
& ʒ &  &  &  &  & [ʒɛ] & ‘smell’\\
& tʃ &  &  &  &  & [tʃɛ] & ‘lack’\\
& dʒ &  &  &  &  & [dʒɛŋ] & ‘luck’\\
& nʒ &  &  &  &  & [nʒɛ] & ‘sit down’\\
\textbf{Laterals} & ɬ & [ɬaj] & ‘slit’ & [ɬɔkʷɔ] & ‘earring’ & [aɬɛɬɛɗ] & ‘egg’\\
& ɮ & [ɮaŋ] & ‘start’ & [bɔɮɔm] & ‘cheek’ & [aɮɛrɛ] & ‘lance’\\
& l & [laj] & ‘dig’ & [lɔ] & ‘go’ & [ləhɛ] & ‘bush’\\
\textbf{Flaps} & r & [rax] & ‘satisfy’ & [arɔx] & ‘pus’ & [tɛrɛ] & ‘other’\\
& ⱱ & [pəⱱaŋ] & ‘start of race’ &  &  &  & \\
\textbf{Semivowels} & j & [jam] & ‘water’ & [sʊkʷɔj] & ‘clan’ & [ajɛwɛɗ] & ‘whip’\\
& w & [war] & ‘child’ & [wuldɔj] & ‘devour’ & [wɛ] & ‘give birth’\\
\lspbottomrule
\end{tabular}

\begin{itemize}
\item \begin{styleTabletitle}
Prosodic conditioning of consonant phonemes
\end{styleTabletitle}\end{itemize}
\subsection{Non-prosodic conditioning of consonants}
\hypertarget{RefHeading1210521525720847}{}
Word-final position influences the distribution of certain phonemes as well as the production of allophones. The following phonemes do not occur in word-final position: voiced stops (including prenasalised stops but excluding /m/ and the implosives), voiced affricates, and the labiodental flap i.e., [b, mb, d, nd, ɡ, ɡʷ, ŋɡ, ŋɡʷ, dz, dʒ , nz, nʒ, ⱱ]. Also, [x] and [ŋ] are the word-final allophones of /h/ and /n/, respectively (\sectref{sec:1}). In some contexts, word-final /r/ can be realised as [l] (\sectref{sec:2}). \tabref{tab:6}. (adapted from Bow, 1997c) shows the distribution of each consonant phone (reading down) in different positions within the word (reading across).  

\begin{tabular}{llllllll} & [F050?] & \textbf{Initial} &  & \textbf{Medial} &  & \textbf{Final} & \\
\lsptoprule
Voiceless stops and affricates & p & [palaj] & ‘choose’ & [kapaj] & ‘roughcast’ & [dap] & ‘fake’ \\
& t & [talaj] & ‘walk’ & [fataj] & ‘descend’ & [mat] & ‘die’\\
& k & [kapaj] & ‘roughcast’ & [makaj] & ‘leave/let go’ & [sak] & ‘multiply’\\
& kʷ & [kʷʊsaj] & ‘fog’ & [tʊkʷasaj] & ‘cross/fold’ & [ajœkʷ] & ‘ground nut’\\
& ts & [tsahaj] & ‘ask’ & [watsaj] & ‘write’ & [harats] & ‘scorpion’ \\
& tʃ & [tʃɛ tʃɛ] & ‘all’ & [mɛtʃɛk\textsuperscript{w}ɛɗ] & ‘worm’ & [mɛkɛtʃ] & ‘knife’\\
Implosives & ɓ & [ɓalaj] & ‘build’ & [ndaɓaj] & ‘wet/whip’ & [haɓ] & ‘break’\\
& ɗ & [ɗakaj] & ‘indicate’ & [jaɗaj] & ‘tire’ & [zaɗ] & ‘take’ \\
Fricatives & f & [fataj] & ‘descend’ & [dafaj] & ‘bump’ & [taf ] & ‘spit’\\
& v & [vakaj] & ‘burn’ & [ɮavaj] & ‘swim’ & [dzav] & ‘plant’\\
& s & [sakaj] & ‘sift’ & [pasaj] & ‘detatch’ & [was] & ‘farm’\\
& ʃ & [ʃɛdɛ] & ‘witness’ & [ʃɛʃɛ] & ‘meat’ & [pɪlɛʃ] & horse\\
& z & [zaɗ] & ‘take’ & [wazaj] & ‘shake’ & [baz] & ‘reap’\\
& ʒ & [ʒɛ] & ‘smell’ & [F05B?]məʒɛ] & ‘person’ & [mɛdɪlɪŋg\textsuperscript{w}œʒ] & ‘leopard’\\
& h & [halaj] & ‘gather’ & [mbahaj] & ‘call’ &  & \\
& hʷ & [hʷʊlɛŋ] & ‘back’ & [tʃœhʷɛɬ] & ‘stalk’ &  & \\
& x &  &  &  &  & [ɓax] & ‘sew’\\
Laterals, approximants, flap, and semivowels & ɬ & [ɬaraj] & ‘slide’ & [tsaɬaj] & ‘pierce’ & [kaɬ] & ‘wait’\\
& ɮ & [ɮavaj] & ‘swim’ & [daɮaj] & ‘join/tie’ & [mbaɮ] & ‘demolish’ \\
& l & [laɡaj] & ‘accompany’ & [balaj] & ‘wash’ & [wal] & ‘attach’\\
& r & [rax] & ‘pluck’ & [ɡaraj] & ‘command’ & [sar] & ‘know’\\
& ⱱ & [ⱱəⱱəⱱə] & ‘rapidly’ & [ɓaⱱaw] & ‘man running’ &  & \\
& j & [jaɗaj] & ‘tire’ & [haja] & ‘grind’ & [balaj] & ‘wash’\\
& w & [watsaj] & ‘write’ & [ɮawaj] & ‘fear’ & [mahaw] & ‘snake’\\
Voiced stops

and affricates & m & [makaj] & ‘leave/let go’ & [lamaj] & ‘touch’ & [tam] & ‘save’\\
& b & [balaj] & ‘wash’ & [abaj] & ‘there is none’ &  & \\
& mb & [mbahaj] & ‘call’ & [hambar] & ‘skin’ &  & \\
& d & [daraj] & ‘snore’ & [hadak] & ‘thorn &  & \\
& nd & [ndavaj] & ‘finish’ & [dandaj] & ‘intestines’ &  & \\
& n & [nax] & ‘ripen’ & [zana] & ‘cloth’ &  & \\
& ɡ & [ɡaraj] & ‘command’ & [laɡaj] & ‘accompany’ &  & \\
& ɡʷ & [ɡʷʊlɛk] & ‘small axe’ & [aɡʷɔɮak] & ‘rooster’ &  & \\
& ŋɡ & [ŋɡaɮaj] & ‘introduce’ & [maŋɡaɬ] & ‘fiancée’ &  & \\
& ŋɡʷ & [ŋɡʷʊdaɬaj] & ‘simmer’ & [aŋɡʷʊrɮa] & ‘sparrow’ &  & \\
& dz & [dzakaj] & ‘lean’ & [dzadzaj] & ‘dawn/light’ &  & \\
& dʒ & [dʒɛŋ] & ‘luck’ & [tʃɪdʒɛ] & ‘illness’ &  & \\
& nz & [nzakaj] & ‘find’ & [manzaw] & ‘beignet’ &  & \\
& nʒ & [nʒɛ] & ‘sit’ & [hɪrnʒɛ] & ‘quarrel’ &  & \\
& ŋ &  &  &  &  & [hadzaŋ] & ‘tomorrow’\\
\lspbottomrule
\end{tabular}

\begin{itemize}
\item \begin{styleTabletitle}
Non-prosodic conditioning of consonant phonemes  
\end{styleTabletitle}\end{itemize}
\paragraph[Word final allophones of /n/ and /h/]{Word final allophones of /n/ and /h/}

Bow demonstrates that [n] and [ŋ] are allophones of /n/ with a distribution as shown in \figref{fig:2}.. 

n [F0AE?] ŋ / \_ \#


\begin{itemize}
\item \begin{styleFiguretitle}
Word final allophone of /n/
\end{styleFiguretitle}\end{itemize}
%%please move \begin{table} just above \begin{tabular
\begin{table}
\caption{(adapted from Bow, 1997c) illustrates [n]{ }and [ŋ] in complementary distribution (with [n]{ }word initial and word medial and [ŋ] word final).}
\label{tab:7}
\end{table}

\begin{tabular}{lllllll}
\lsptoprule

\textbf{Prosody} & \multicolumn{2}{l}{ x20  \textbf{Initial}} & \multicolumn{2}{l}{ \textbf{Medial}} & \multicolumn{2}{l}{ \textbf{Final}}\\
\textbf{Neutral} & [nax] & ‘ripen’ & [ɡənaw] & ‘animal’ & [=ahaŋ] & =3S.POSS\\
\textbf{Labialised} & [nɔkʷ] & ‘you’ & [ana] & ‘to’ (dative) & [ɡəlaŋ] & ‘threshing area’\\
\textbf{Palatalised} & [nɛ] & ‘me & [mɪtɛnɛŋ] & ‘bottom’ & [ɓərɮaŋ] & ‘mountain’\\
\lspbottomrule
\end{tabular}

\begin{itemize}
\item \begin{styleTabletitle}
Complementary distribution for /n/
\end{styleTabletitle}\end{itemize}

Likewise, Bow demonstrates that [h] and [x] are allophones of /h/ with a distribution as shown in \figref{fig:3}.. 

x  h [F0AE?] x / \_ \#


\begin{itemize}
\item \begin{styleFiguretitle}
Word final allophone of /h/
\end{styleFiguretitle}\end{itemize}
%%please move \begin{table} just above \begin{tabular
\begin{table}
\caption{shows [x] and [h] in complementary distribution (with [h]{ }word initial and word medial and [x] word final).}
\label{tab:8}
\end{table}

\begin{tabular}{lllllll}
\lsptoprule

\textbf{Prosody} & \multicolumn{2}{l}{\textbf{INITIAL}} & \multicolumn{2}{l}{\textbf{MEDIAL}} & \multicolumn{2}{l}{\textbf{FINAL}}\\
\textbf{Neutral} & [har] & ‘make’ & [ahar] & ‘hand’ & [rax] & ‘satisfy’\\
\textbf{Labialised} & [h\textsuperscript{w}ʊdɔ] & ‘wall’ & [tɔh\textsuperscript{w}ɔr] & ‘cheek’ & [h\textsuperscript{w}ɔmbɔx] & ‘pardon’\\
\textbf{Palatalised} & [hɛrɛɓ] & ‘heat’ & [mɛhɛr] & ‘forehead’ & [tɛʒɛx] & ‘boa’\\
\lspbottomrule
\end{tabular}

\begin{itemize}
\item \begin{styleTabletitle}
Complementary distribution for /h/
\end{styleTabletitle}\end{itemize}
\paragraph[Word final allophones of /r/]{Word final allophones of /r/}

Friesen and \citet{Mamalis2008} demonstrated that for some verb roots, final /r/ is realised as [l] in certain contexts.\footnote{This process does not appear to be free variation. } In ex. 29 and 30, which are consecutive lines from a narrative text, the final /r/ of the verb /v r /\textit{ }‘give’ is [r ] in  \textit{navar}\textit{ }‘I give’ (ex. 30) but is realised as [l] when the indirect object pronominal enclitic =\textit{aw} (see \sectref{sec:491}) is attached (ex. 29): 


\ea
\textup{[vəl=aw   kɪndɛw     =aŋg}\textup{\textsuperscript{w}}\textup{ɔ     na   ɛhɛ}\textup{]}
\z

give 2S.IMP=1S.IO  guitar    =2S.POSS  PSP  here



‘Give me your guitar, here!’


\ea
\textup{[}\textup{na-var   na   baj}\textup{]   }
\z

1S-give  3S.DO  not



‘I won’t give it.’  


Likewise, the verb /wal/ ‘hurt’ exhibits similar changes, where the word-final [r] in ex. 31 becomes /l/ when the indirect object pronominal enclitic attaches (ex. 32). 

\ea
həmaɗ   a-war   gam
\z

wind  3S-hurt  much



‘It’s very cold.’ (lit. wind hurts a lot)


\ea
həmaɗ   a-wal  =alɔk\textsuperscript{w}ɔ
\z

wind  3S-hurt  =1\textsc{Pin}.IO



‘We’re cold.’ (lit. wind hurts us)


\section{Vowels}
\hypertarget{RefHeading1210541525720847}{}
There are eleven surface phonetic vowels in Moloko (\tabref{tab:9}.) but the vowel system can be analysed as having one underlying vowel /a/.\footnote{An analysis by \citet{Bow1999} using Optimality Theory allowed both a single underlying vowel system (/a/) or a two underlying vowel system (/a/ and /ə/).  For the purposes of this paper, the schwa is considered as epenthetic since its presence is predictable, and /a/ is considered the only underlying vowel phoneme.}  This vowel may be either present or absent between any two consonants in the underlying form of a morpheme. \citet{Bow1997c} found that the absence of a vowel requires an epenthetic vowel to break up some consonant clusters in the surface form.\footnote{Certain consonants do not require epenthetic schwa insertion (\sectref{sec:119}).} Different environments acting on the underlying vowel and the epenthetic [ə] result in the ten allophones in Moloko (four from /a/: [a, ɛ, ɔ,\textit{ }œ]\footnote{Bow reported ten surface vowel forms including [æ] which she did not consider as a distinct allophone since not all speakers distinguish between [a] and [æ], leaving nine allophones. Friesen has added [ø].} and six from the epenthetic schwa: [ə, ɪ, ʊ, ø, i, u]).  Note the addition of the vowel [ø] not in Bow’s analysis. Bow noted “a phonetic gap left by the absence of a high vowel with both palatalisation and labialisation.” This work reports the presence of this vowel in environments affected by both prosodies (see \sectref{sec:6}). 

Bow distinguished the vowels in Moloko using four features: height, tense (or ATR), palatalisation, and labialisation. In this work, the conditioning environments that affect the phonetic expression of a full or epenthetic vowel include the labialisation and palatalisation prosodies (\sectref{sec:5})  and adjacency of the epenthetic vowel to particular consonants (\sectref{sec:6}). 

\subsection{Vowel phonemes and allophones}
\hypertarget{RefHeading1210561525720847}{}%%please move \begin{table} just above \begin{tabular
Table 9 is a summary table showing the sources of allophonic variation and the resulting phonetic realisations. In the table, the orthographic representation of each of these phonetic vowels is bolded and follows each vowel or example in the table.\footnote{The orthographic representation is not employed elsewhere in the paper, since it is important that the reader appreciate the phonetic expression. } For each source of allophonic variation, an example is also given. In a word which is neutral with respect to prosody (line 1), the underlying vowel is pronounced [a] and epenthetic schwa [ə]. In labialised words, (line 2), /a/ becomes [ɔ] and the epenthetic schwa becomes [ʊ].  In palatalised words (line 3), /a/ is pronounced [ɛ] and the epenthetic schwa is pronounced [ɪ]. The epenthetic vowel can also be assimilated to a neighbouring approximant: it is realised as [i] when it occurs beside [j] (line 4) and as [u] when it occurs beside a labialised velar [w, k\textsuperscript{w}, ɡ\textsuperscript{w}, ŋɡ\textsuperscript{w}, h\textsuperscript{w}] or consonant (line 5). Under the influence of labialised velars and an adjacent /j/, the /a/ becomes [œ] and the epenthetic schwa becomes [ø] (line 6).

\begin{tabular}{llllll} &  & \textbf{/a/} & \textbf{Example } & \textbf{Epenthetic}\textbf{ ə} & \textbf{Example}\\
\lsptoprule
\textbf{1} & \textbf{No word-level process} & [a]              \textbf{a} & [awak]             \textbf{awak}

‘goat’ & [ə]             \textbf{ə} & [ɡəɡəmaj]    \textbf{ɡəɡəmay}

‘cotton’\\
\textbf{2} & \textbf{Labialisation } & [ɔ]               \textbf{o} & [sɔnɔ]              \textbf{sono}

‘game’ & [ʊ]             \textbf{ə} & [mʊlɔkʷɔ]    \textbf{Məloko}

‘Moloko’\\
\textbf{3} & \textbf{Palatalisation} & [ɛ]               \textbf{e} & [ʃɛʃɛ]               \textbf{sese}

‘meat’ & [ɪ]              \textbf{ə} & [ʃɪlɛk]           \textbf{səlek}

‘jealousy’\\
\textbf{4} & \textbf{Adjacent to  [}\textbf{j]} & [a]               \textbf{a} & [haja]              \textbf{haya}

‘grind’ & [i]               \textbf{ə} & [kija]            \textbf{kəya}

‘moon’\\
\textbf{5} & \textbf{Adjacent to  [}\textbf{w}\textbf{]} & [a]               \textbf{a} & [mawar]        \textbf{mawar}

‘tamarind’ & [u]              \textbf{ə} & [ɗuwa]          \textbf{ɗəwa}

‘milk’\\
\textbf{6} & \textbf{Adjacent to an inherently labio-velar and /j/} & [œ]              \textbf{e} & [ʃɪlœkʷ]         \textbf{səlewk}

‘broom’ & [ø]              \textbf{ə} & [lʊkʷøjɛ]       \textbf{ləkwəye}

‘you’ (Pl)\\
\lspbottomrule
\end{tabular}

\begin{itemize}
\item \begin{styleTabletitle}
Sources of allophonic variation in vowels with orthographic representation
\end{styleTabletitle}\end{itemize}

The working orthography for Moloko (Friesen, 2001) indicates the word-level processes by the three full vowel graphemes in the word pronounced in isolation: ‘e’ in palatalised words, ‘o’ in labialised words, and ‘a’ in words with neutral prosody.\footnote{Even if the palatalisation or labialisation is incomplete in a word beginning with /a/, that first vowel will be written ‘e’ or ‘o.’ respectively in the orthography. }  Epenthetic vowels are written as ‘ə’ in the orthographic representation regardless of the word prosody, because their pronunciation is predictable from the word prosody (discernable from the full vowel in the word) and the surrounding consonants.  The result is four orthographic vowel symbols (a, e, o, ə).

\subsection{Prosodic conditioning of vowel allophones}
\hypertarget{RefHeading1210581525720847}{}
\citet{Bow1997c} reports that there is a clear prosodic pattern in Moloko where with very few exceptions,\footnote{Labialisation and palatalisation in words which begin with a vowel will sometimes be incomplete, leaving the first syllable as [a] for labialised words and [æ] for palatalised words (see \sectref{sec:2.1}). } all vowels in any word will have the same prosody, be it labialised, palatalised, or neutral.  \tabref{tab:10}. (adapted from Bow, 1997c) illustrates the three possible underlying prosody patterns in two and three syllable words.\footnote{Adjacency to certain consonants can also affect the quality of a particular vowel (\sectref{sec:116}).}

\begin{tabular}{lllllll} & \multicolumn{3}{l}{\textbf{Two syllable stems}} & \multicolumn{3}{l}{\textbf{Three syllable stems}}\\
\lsptoprule
\textbf{Neutral} & /ha r ts/ & [harats] & ‘scorpion’ & /ma ta b ɬ/ & [matabaɬ] & ‘cloud’\\
& /d r j/ & [dəraj] & ‘head’ & /ɡ ɡ m j/ & [ɡəɡəmaj] & ‘cotton’\\
\textbf{LAB} & /ba ɮ m \textsuperscript{o}/ & [bɔɮɔm] & ‘cheek’ & /ta la l n \textsuperscript{o}/ & [tɔlɔlɔŋ] & ‘chest’\\
& /s k j \textsuperscript{o}/ & [sʊkʷɔj] & ‘clan’ & /ɡa ɡ l v n \textsuperscript{o}/ & [ɡʷɔɡʷʊlvɔŋ] & ‘snake’\\
\textbf{PAL} & /ma h r \textsuperscript{e}/ & [mɛhɛr] & ‘forehead’ & /ma ba b k \textsuperscript{e}/ & [mɛbɛbɛk] & ‘bat’\\
& /ɮ ɡa \textsuperscript{e}/ & [ɮɪɡɛ] & ‘sow’ & /ts ka la \textsuperscript{e}/ & [tʃɪkɛlɛ] & ‘price’\\
\lspbottomrule
\end{tabular}

\begin{itemize}
\item \begin{styleTabletitle}
Underlying prosody patterns in two and three syllable words  
\end{styleTabletitle}\end{itemize}
\subsection{ Non-prosodic conditioning of vowel allophones}
\hypertarget{RefHeading1210601525720847}{}
Bow reported that besides the prosodies of labialisation and palatalisation, the epenthetic vowel allophones are conditioned by the phonemes /j/ and /w/ as well as the underlyingly labialised consonants.  The rules governing these two conditioning environments follow, along with examples of each. Bow found that the epenthetic vowel assimilates to the palatal and labial features of an adjacent semi-vowel even when there is a prosody on the root.  \figref{fig:4}. and \figref{fig:5}. illustrate the rules for the influence of /j/\footnote{There are no cases of *[ji].} and /w/ with examples of each. 

\begin{styleParagraph}
{}[ə] [F0AE?] [i] / \_ j
\end{styleParagraph}


\begin{itemize}
\item \begin{styleFiguretitle}
Influence of j on ə
\end{styleFiguretitle}\end{itemize}

\ea
\textup{/k ja/     [kija]    }
\z

‘moon’  


\ea
\textup{/m j k }\textup{\textsuperscript{e}}\textup{/    [mijɛk]   }
\z

‘deer’


\begin{styleParagraph}
 [ə] [F0AE?] [u] / \_ w
\end{styleParagraph}

{}[ə][F0AE?] [u] / w \_


\begin{itemize}
\item \begin{styleFiguretitle}
Influence of w on ə
\end{styleFiguretitle}\end{itemize}

\ea
\textup{/ɗ wa/    [ɗuwa]   }
\z

‘milk/breast’    


\ea
\textup{/ɗ w r }\textup{\textsuperscript{e}}\textup{ /      [ɗuwɛr]  }
\z

‘sleep’


\ea
\textup{/w ɗa k -j/     [wuɗakaj]  }
\z

‘separate/share’\textit{  }


Bow found that the vowel phoneme /a/ is not affected by semi-vowels, as demonstrated in ex. 38 and 39.

\ea
\textup{/ja ɗ -j/      [jaɗaj]    not *[jɛɗɛj]}\textup{  }
\z

‘tire’   


\ea
\textup{ }\textup{/ɡ n w/    [ɡənaw]     not *[ɡʊnɔw]}
\z

‘animal’    


Bow noted that the semi-vowels themselves do not cause morpheme-level palatalisation or labialisation to occur.  Ex. 40 - 44 illustrate that the presence of the labiovelar semi-vowel /w/ in any position within a word (including word-finally) does not effect a labialisation prosody across the word. In fact, the existing data lists no examples of words containing /w/ which have a word-level labialisation prosody.

\ea
\textup{/ma w r/     [mawar]}\textup{    }
\z

‘tamarind’   


\ea
\textup{/da da wa  }\textup{\textsuperscript{e}}\textup{/     [dɛdɛwɛ]}\textup{    }
\z

‘a species of bird’


Similarly with the palatal semi-vowel, Bow shows that the presence of [j] does not effect a palatalisation prosody across the word, although it may occur within a palatalised or labialised word.

\ea
\textup{/la j w/      [lajaw]}\textup{     }
\z

‘large squash’


\ea
\textup{/s k j }\textup{\textsuperscript{o}}\textup{/        [sʊkʷɔj]     }
\z

‘clan’    


\ea
\textup{/ha j w}\textup{\textsuperscript{ e}}\textup{/      [hɛjɛw]}\textup{    }
\z

‘cricket’  


This work also illustrates the rules governing the production of [œ] and the combined influence on the epenthetic vowel of adjacency to /j/ and either /w/ or /k\textsuperscript{w}/  to produce [ø]. An underlying /a/ is realised as [œ] when it occurs before the labialised velar /k\textsuperscript{w}/ in a palatalised word (\figref{fig:6}., ex. 45). When an epenthetic schwa occurs between /j/ and a labialised velar /k\textsuperscript{w} or w/,\footnote{We have not found the epenthetic vowel between /j/ and any other of the underlyingly labialised consonants (ɡ\textsuperscript{w}, ŋɡ\textsuperscript{w}, h\textsuperscript{w}, see \sectref{sec:111}). } it is realised as [ø] (\figref{fig:7}., ex. 46 - 47). It is important to note that the presence of an underlyingly labialised velar consonant also does not cause labialisation of the entire phonological word; in fact, the evidence for their existence stems from this fact (see \sectref{sec:1}). 

/a/[F0AE?] [œ] /  \textsuperscript{ }    \_\_ C\textsuperscript{w} \textsuperscript{e}/


\begin{itemize}
\item \begin{styleFiguretitle}
Influence of labialised velar on /a/
\end{styleFiguretitle}\end{itemize}

\ea
\textup{/azɛk}\textup{\textsuperscript{w  e}}\textup{/  [æʒœk}\textup{\textsuperscript{w}}\textup{]}
\z

‘sorry’


{}[ə][F0AE?] [ø] /   k\textsuperscript{w} \_ j


\begin{itemize}
\item \begin{styleFiguretitle}
Influence of labialised velar and j on ə
\end{styleFiguretitle}\end{itemize}

\ea
\textup{ }\textup{/l k}\textup{\textsuperscript{w }}\textup{ja }\textup{\textsuperscript{e}}\textup{/    [lʊk}\textup{\textsuperscript{w}}\textup{øjɛ]}\textup{  }
\z

 ‘you (plural)’\textit{  }


\ea
\textup{/w j n}\textup{\textsuperscript{ e}}\textup{ /  [wøjɛ}\textup{ŋ}\textup{]}
\z

‘land’


\section{Tone}
\hypertarget{RefHeading1210621525720847}{}
In addition to published manuscripts and a thesis, Bow produced a database and an extensive series of observations relating to lexical and grammatical tone in Moloko nouns and verbs.  This database was later expanded and modified, leading to an initial analysis of tone in noun phrases by \citet{Boyd2002} and later to tone in verbs by Mamalis (Friesen and Mamalis, 2008).  

\citet{Bow1997c} describes three phonetic tones (H, M, and L) but only two phonemic tones. In this work, lexical tone and grammatical tone are marked when relevant.\footnote{Some data was transcribed without tone.}  The phonetic tone patterns will be indicated on the words using accent marks for H (  \'{ }), M (  \={ }, when necessary), or L tone (  \`{ }).  

%%please move \begin{table} just above \begin{tabular
Table 11 adapted from Bow, 1997c with additional data) shows minimal pairs which illustrate the underlying two tone system in Moloko. Tone does not carry a high lexical load, and so there are only a limited number of lexical items distinguished by tone.\footnote{One of each in these minimal pairs are marked in the orthography with a diacritic so that the pairs can be distinguished. } The examples in \tabref{tab:11}. are divided into grammatical categories. Some of the minimal pairs are from different grammatical categories. 

\begin{tabular}{lllll} & \multicolumn{2}{l}{\textbf{H tone}} & \multicolumn{2}{l}{\textbf{L tone}}\\
\lsptoprule
\textbf{Nouns} & [háj] & ‘millet’ & [hàj] & ‘house/compound’\\
& [ánɛŋ] & ‘other’ & [ànɛŋ] & ‘snake’\\
& [ɡəláŋ] & ‘threshing floor’ & [ɡəl\={a}ŋ] & ‘kitchen/clan’\\
& [háhàr] & ‘bean’ & [h\={a}hár] & ‘straw granary’\\
& [mədár\={a}] & ‘fire’ & [mədərà] & ‘bicep’\\
& [mɔlɔ] & ‘twin’ & [mɔlɔ] & ‘vulture’\\
& [ɛlɛ] & ‘eye’ & [ɛlɛ] & ‘thing’\\
& [vɛr] & ‘grinding stone’ & [vɛr] & ‘room’\\
\textbf{Verbs} & [dár] & ‘burn’ & [dàr] & ‘withdraw/recoil’\\
\hhline{-~~~~} & [h\={a}r] & ‘pick up/transport’ & [hàr] & ‘build/make’\\
& [nʒɛ] & ‘suffice’ & [nʒɛ] & ‘sit’\\
& [tsáháj] & ‘ask’ & [ts\={a}háj] & ‘get water’\\
& [tsáwáj] & ‘cut off the head’ & [tsàw\={a}j] & ‘grow’\\
& [pəɗ\={a}káj] & ‘wake up’ & [pəɗàk\={a}j] & ‘melt’\\
\textbf{Different grammatical categories} & [ává] & ‘there is’ (\textsc{EXT}) & [àvà]\footnotemark{} & ‘arrow’ (noun)\\
\hhline{-~~~~} & [kʊrsáj] & ‘sweep’ (verb) & [kʊrs\={a}j] & ‘cucumber’ (noun)\\
& [l\={a}lá] & ‘come back’ (verb) & [l\={a}l\={a}] & ‘good’ (adverb)\\
& [ɛhɛ] & ‘no’ (interjection) & [ɛhɛ] & ‘here’ (adverb)\\
& [tətá] & 3P & [tət\={a}] & ‘is able to’\\
& [vá] & ‘Perfect extension’ & [và] & ‘body’ \\
& [ndán\={a}] & ‘therefore’ / 

‘you (S) must’ & [nd\={a}nà] & ‘previously mentioned’\\
& [\={a}háŋ] & 3P \textsc{POSS} & [àh\={a}ŋ] & ‘he said’\\
\hhline{~----}
\lspbottomrule
\end{tabular}
\footnotetext{ A third example ([áv\={a}]\textit{ }\textit{‘}under’) makes this line a minimal triplet for tone.}

\begin{itemize}
\item \begin{styleTabletitle}
 Minimal pairs for phonetic tone
\end{styleTabletitle}\end{itemize}

From an underlying two-tone system, with the influence of depressor consonants, certain melodies can be derived.  There are different melodies for nouns and verbs. These melodies will be discussed in the noun and verb sections (see Sections 4.1 and 6.7). Bow described three different categories of verbs, those with underlying high tone, those with underlying low tone, and those with no underlying tone at all (toneless). A list of verbs with underlying tone is in the appendix (see \sectref{sec:14.1}).

Lexical tone is not marked in the orthography since there are only a few minimal pairs. The minimal pairs are distinguished by a diacritic on one of the words in each pair. Grammatical tone indicating Imperfective and Perfective aspect is marked in Moloko on the subject pronominal verb prefix (see \sectref{sec:7.4}). Lexical tone will not be marked in examples in the morphosyntax part of this work.

\subsection{Depressor consonants}
\hypertarget{RefHeading1210641525720847}{}
There are certain consonants which affect tone in Moloko.  \citet{Bow1997c} discovered that the voiced obstruents [b, d, ɡ, mb, nd, ŋɡ, v, z, dz, nz, ɮ]\footnote{Bow notes that the phonemes /h, w, r, l/ can appear to function as depressors.} have the effect of lowering an underlying low tone of the syllable in which they occur.  Yip (2002: 133, 158) notes that: 

\begin{quote}
{\textquotedbl}The most frequent form of interaction between tone and laryngeal features in African languages is the presence of 'depressor' consonants. This term describes a subset of consonants, usually voiced, which lower the tone of neighbouring high tones, and may also block high spreading across them. This is a departure from the usual inertness of consonants in tonal systems…The set of depressor consonants may include all voiced consonants, or often only non-glottalized, non-implosive voiced obstruents. In some languages, such as Ewe, we find a three-way split, with voiced obstruents most active as depressors, voiceless obstruents as non-depressors, and voiced sonorants having some depressor effects, but fewer than the obstruents.{\textquotedbl}
\end{quote}

Depressor consonants do not affect high tone words in Moloko. The phonetic mid and low tones thus both represent an underlying low tone.The phonetic low tone is triggered by the presence of depressor consonants.  \tabref{tab:12}. demonstrates the effect of depressor consonants on the tone of the verb root in Moloko. The table shows minimal pairs of mid and low tone verb roots with and without depressor consonants. The roots with no depressor consonants have phonetically mid tone, whereas all of the roots with depressor consonants have phonetically low tone.

\begin{tabular}{llllll}
\lsptoprule

\multicolumn{3}{l}{\textbf{Root with no depressor consonants}} & \multicolumn{3}{l}{\textbf{Root with depressor consonants}}\\
\textbf{Phonetic tone on root} & \textbf{Verb in 2S imperative form} & \textbf{Gloss} & \textbf{Phonetic tone on root} & \textbf{Verb in 2S imperative form } & \textbf{Gloss}\\
M & \textit{f}\textit{ɛ} & ‘play an instrument’ & L & \textit{v}\textit{ɛ} & ‘spend time’\\
M & \textit{taf} & ‘spit’ & L & \textit{dav} & ‘plant’\\
M & \textit{taɬ-aj} & ‘curse’ & L & \textit{baɮ-aj} & ‘breathe’\\
\lspbottomrule
\end{tabular}

\begin{itemize}
\item \begin{styleTabletitle}
Effect of depressor consonants on tone of verb root
\end{styleTabletitle}\end{itemize}
\subsection{Tone spreading rules}
\hypertarget{RefHeading1210661525720847}{}
At the phrase level, Bow found that a surface mid tone can have two sources: either an underlying low tone with no depressor consonants (see \sectref{sec:7}), or a surface high tone lowered by a preceding low.  Bow found no LH melodies within words, and illustrated that a noun whose final syllable is low tone will lower a high tone on the first syllable of any word that follows. \tabref{tab:13}. (from Bow, 1997c) illustrates high tone lowering. Bow also describes a spreading rule which is optional across word boundaries where the mid or high final tone of a noun optionally spreads over a low tone on the first syllable of an adjective. 

\begin{tabular}{lllll} & \textbf{Words in isolation} & \textbf{Words in context} & \textbf{Tone change} & \textbf{Gloss}\\
\lsptoprule
\textbf{Across morpheme boundary} & [ɬàlà] +[aháj] & [ɬàlàh\={a}j] & LL+H [F0AE?] LLM & ‘villages’\\
\hhline{-~~~~} & [jàm]+    [áh\={a}ŋ] & [jàm\={a}h\={a}ŋ] & L+HM [F0AE?] LMM & ‘his/her water’\\
\textbf{Across word boundary} & [jàm]+  [ábá] & [jàm \={a}bá] & L+HH [F0AE?] LMH & ‘there is water’\\
\hhline{-~~~~} & [ázʊŋɡʷ ɔ]+ [ná] + [ɬ\={a}] & [ázʊŋɡʷɔ n\={a} ɬ\={a}] & HHL+H+M [F0AE?] HHLMM & ‘donkey and cow’\\
\hhline{~----}
\lspbottomrule
\end{tabular}

\begin{itemize}
\item \begin{styleTabletitle}
High tone lowering at morpheme boundaries
\end{styleTabletitle}\end{itemize}
\section{Notes on the syllable}
\hypertarget{RefHeading1210681525720847}{}
The syllable in Moloko is a somewhat fluid entity, since Moloko words have a consonantal skeleton with optional vowels, making the syllable a surface structure (see introduction to Chapter 2). \citet{Bow1997c} has discussed the syllable in Moloko in detail. This section deals with aspects of syllable structure that pertain to the grammar (\sectref{sec:9}) and syllable restructuring when words combine in speech (\sectref{sec:10}). 

\subsection{Syllable structure}
\hypertarget{RefHeading1210701525720847}{}
Bow notes that “The basic syllable in Moloko has a consonantal onset, a vocalic nucleus and an optional consonant coda: CV(C), and carries tone.”\footnote{Bow 1997c, p. 1.} She found three syllable types in Moloko.  CV, CVC, and initial V are the most common syllable types in the language.  Both CV and CVC syllables can appear anywhere within the word. V syllables occur only in word initial position and are most likely to have come from what was once a separate morpheme – the /\textit{a-}/ prefix in nouns (see \sectref{sec:4.1}), the third singular prefix in verbs (see \sectref{sec:49}), and an adposition (see Sections 45 and 42).

Bow notes no restrictions on consonantal onsets.\footnote{Friesen and \citet{Mamalis2008} discovered that although there are no restrictions on consonantal onsets for nouns, verb stems beginning with /n/ or /r/ are rare.  } Friesen and \citet{Mamalis2008} noted that although nouns ending in CV can have any prosody (see \sectref{sec:4.1}), almost all verb stems phonetically ending in CV are palatalised (ex. 48{}-49), where the V is the \textit{{}-ɛ} suffix discussed in \sectref{sec:2.1.2.}\footnote{The only non-palatalised verb stems ending in CV end with the pluractional clitic \textit{=aja} or \textit{=ija}, e.g., [h=aja] ‘grind.’ [s=ija] ‘cut.’ see \sectref{sec:1157.}  These verbs do not occur without the clitic so we do not know if they carry an underlying prosody or /-j/ suffix. } 


\ea
\textup{[g-ɛ ]}
\z

do[2S.IMP]-CL



‘Do!’


\ea
\textup{[d-ɛ]}
\z

prepare[2S.IMP]-CL



‘Prepare!’


The coda position carries more restrictions. Firstly, in word-medial position, consonants that are permitted as coda are restricted. Bow reported that liquids can function as the coda to a non-word-final syllable.\footnote{Bow also reports that liquids can function as the nucleus of a syllable and also as the second component of a consonantal onset.} Further research has also shown that a semivowel /w/, /j/ or nasal /m, n/ can also function as the coda of a non word-final closed syllable (ex. 50 -52).

\ea
duwlaj
\z

‘millet drink’


\ea
kijɡa
\z

‘like this’


\ea
amsɔk\textsuperscript{w}ɔ
\z

‘sorghum’


Secondly, consonants that can fill the coda position word-finally have other restrictions. Bow reported that the voiced plosives [b, d, dz, ɡ, ɡʷ] and pre-nasalised consonants [mb, nd, nz, ŋɡ, ŋɡʷ] do not appear in word-final position, and /n/ and /h/ have word-final allophones (see \sectref{sec:31}). In addition, Friesen and \citet{Mamalis2008} found that word final consonants in verb stems that do not take the /-j/ suffix exclude all of the above and also exclude the voiceless affricate /ts/ and the approximants /w, j/.  

Friesen and \citet{Mamalis2008} postulated that a function of the /-j/ suffix of verb stems (see \sectref{sec:6.3}) is to allow root-final consonants which can’t occur word-finally to occur. Verb roots that take the /-j/ suffix permit /b, g, ts/, and /w/ as final consonant (ex. 53{}-55), all consonants that are restricted in the coda position either in all Moloko words or in verb stems. The presence of the /-j/ suffix, another suffix, or an enclitic ensures that in context, the final consonants of /-j/ roots never occur word finally.  

\ea
\textup{[dab-aj]}
\z

follow[2S.IMP]{}-CL



‘Follow!’


\ea
\textup{[lag-aj]}
\z

accompany[2S.IMP]{}-CL



‘Accompany!’


\ea
\textup{[ndaw-aj]}
\z

swallow[2S.IMP]{}-CL



‘Swallow!’


In some contexts, two voiceless consonants do not permit a voiced epenthetic schwa between them. A voiceless syllable results. In some cases, speakers could assign tone to the syllable (ex. 56 - 59),\footnote{Bow’s data reports tone in every syllable for all of these words except \textit{m}\textit{ɔ}\textit{k}\textit{\textsuperscript{w}}\textit{t}\textit{ɔ}\textit{n}\textit{ɔ}\textit{k}\textit{\textsuperscript{w}} ‘toad,’ \textit{ɔ}\textit{k}\textit{\textsuperscript{w}}\textit{f}\textit{ɔ}\textit{m} ‘mouse,’ \textit{Ftak} ‘Ftak’ (a proper name) and \textit{dɛftɛrɛ} ‘book’ (a borrowed word from Fulfuldé).} and in other cases, they could not assign tone to the voiceless syllable (ex. 60 - 63). In the example, the syllables are separated by a period in the phonetic form. The voiceless syllable is underlined. 

\ea
\textup{[}\textup{s}\textup{ʊ.}\textup{k}\textup{\textsuperscript{w}}\textup{ɔ}\textup{m}\textup{]}
\z

‘buy/sell’


\ea
\textup{[}\textup{t}\textup{ə.}\textup{ka.raj}\textup{]}
\z

‘taste’


\ea
\textup{[mɪ.}\textup{tɪ.}\textup{fɛ}\textup{]}
\z

‘spitting’ (\textsc{NOM})


\ea
\textup{[mɪ.}\textup{t}\textup{ʃ}\textup{ɪ.}\textup{kɛ}\textup{]}
\z

‘standing’ (\textsc{NOM})


\ea
\textup{[}\textup{m}\textup{ɔ}\textup{k}\textup{\textsuperscript{w}}\textup{ʊ}\textup{.}\textup{t}\textup{ɔ.}\textup{n}\textup{ɔ}\textup{k}\textup{\textsuperscript{w}}\textup{ ]}
\z

‘toad’


\ea
\textup{[d}\textup{ɛ}\textup{fɪ.}\textup{t}\textup{ɛ}\textup{.r}\textup{ɛ}\textup{]}
\z

‘book’


\ea
\textup{[}\textup{f}\textup{ə.}\textup{tak}\textup{]}
\z

‘Ftak’


\ea
\textup{[}\textup{ɔ.}\textup{k}\textup{\textsuperscript{w}}\textup{ʊ}\textup{.}\textup{f}\textup{ɔ}\textup{m}\textup{]}
\z

‘mouse’


\subsection{Syllable restructuring}
\hypertarget{RefHeading1210721525720847}{}
In fast speech, changes may happen within words or at word boundaries affecting adjacent syllables. At word boundaries, certain word-final consonants are lost and there may be vowel elision and reduction of vowels. Within the word, the segments may be restructured into new syllables, vowels may be reduced or deleted, and certain consonants may be deleted.

\citet{Bow1997c} notes vowel elision and semivowel assimilation at morpheme boundaries. Other changes that we have noted are illustrated in \tabref{tab:14}.. When morphemes are added (lines 1 and 10) or words juxtaposed within a construction (lines 2-8), syllables within the morphemes are sometimes reorganised or deleted. Syllables in the table are separated by a period. Line 1 shows the resyllabification of /anzakr/ where [k] (originally the onset) becomes the coda, making [r] the onset of a syllable that includes the first vowel of the following word. Line 2 illustrates vowel elision and loss of prosody. Lines 3-5 illustrate that in fast speech, word final /-n/ is deleted. Note in line 5 that although /-n/ is deleted, the high tone of the suffix remains on the vowel and there is no vowel elision. Line 6 illustrates deletion of /h/.\footnote{This kind of deletion seems to be irregular and may relate to a language change, since in some neighbouring languages, ‘chief’ is [baj]. ‘Chief is [baj] in Cuvok (Ndokabai, 2006: 120), Gemzek (Gravina, 2005: 9), Muyang (Smith, personal communication), Vame (Kinnaird, 2006: 17), but [bahaj] in Mbuko (Gravina, 2001: 9).} Note that stress is phrase-final necessitating a full vowel in the final syllable of an utterance (introduction to Chapter 21).

\begin{tabular}{llll}
\lsptoprule

\textbf{Number} & \textbf{Underlying form} & \textbf{Phonetic pronunciation}

\textbf{in isolation} & \textbf{Phonetic pronunciation in fast speech}\\
1 & /anzakr     wla/

chicken      1S.POSS

‘my chicken’ & [a.nza.kar] [u.la] & [anzakrula]\\
2 & /a-  \textsuperscript{w}la   \textsuperscript{j}ala      ahan/

3S-   go -thing    =3S.POSS

‘he went away’ & [a.lɔ ] [ɛ.lɛ ] [a.haŋ ] & [alɔlahaŋ]\\
3 & /n-la\textsuperscript{w}           a   ɓ r ɮ n      ava/

1S.PFV-go to  mountain in

‘I went to the mountain’ & [nʊ.lɔ ] [a] [ɓər.ɮaŋ ] [a.va] & [nʊlɔɓərɮava]\\
4 & / ɡln                 =ahaj/

threshing area   =Pl

‘threshing areas’ & [ɡə.laŋ ] [a.haj] & [ɡəlahaj ]\\
5 & /a-mbɗ   =an   =aka/

3S-change=3S.IO  =on  

‘He/she replied’ (lit. he changed on him) & [a.mbə.ɗaŋ] [a.ka] & [àmbəɗááka]\\
6 & / bahj   ʷalaka /

chief    1PIN.POSS

‘our (in.) chief’ & [ba.haj ]  [a.lɔ.kʷɔ] & [bajalɔkʷɔ]{ }\\
\lspbottomrule
\end{tabular}

\begin{itemize}
\item \begin{styleTabletitle}
Changes due to syllable restructuring
\end{styleTabletitle}\end{itemize}
\section{Word boundaries}
\hypertarget{RefHeading1210741525720847}{}
\citet{Bow1997c} notes that “the phonological word in Moloko is made up of a root with the optional addition of affixes.” Further research has revealed that phonologically bound morphemes added to the root include affixes and several kinds of clitics. Specific phonological aspects of nouns and verbs will be discussed in their respective chapters (Chapters 4 and 6).

Word breaks are determined in this work by the phonological criteria below (as well as using the grammatical criteria discussed in \sectref{sec:12}). Using these criteria, affixes, clitics, and extensions\footnote{Note that the term ‘extension’ for Chadic languages has a different use than for Bantu languages. In Chadic languages, ‘extension’ refers to particles or clitics in the verbal complex (\sectref{sec:7.5}).} can be distinguished from separate words in Moloko. Phonological criteria are illustrated for both nouns and verbs, when applicable (\sectref{sec:11}). ‘Affix,’ ‘clitic,’ and ‘extension’ are categorised for Moloko in \sectref{sec:12.}

\subsection{Phonological criteria for word breaks}
\hypertarget{RefHeading1210761525720847}{}
Five phonological criteria are used in this work. The criteria are illustrated for both nouns and verbs. Examples are given in pairs showing word breaks in the first example and phonologically bound morphemes in the second example. 


\begin{itemize}
\item A word-final /h/ is realized as [x] (Bow, 1997c).
\end{itemize}

Ex. 64 shows the presence of the word-final allophone [x], indicating a word break between \textit{ɡəvax} and \textit{nɛhɛ}. The 3P possessive (=\textit{atəta}) is shown to be phonologically bound to the same noun (ex. 65) since this word-final change does not occur (Bow, 1997c, see \sectref{sec:14}).\footnote{Note that although /=atəta/ is not completely phonologically bound to \textit{ɡəvax} since the neutral prosody of /=atta/ does not neutralise the prosody of the noun (criterion c), it is a type of noun clitic since it fulfills the grammatical criteria for a clitic \citep{Section1112}. } 


\ea
\textup{[ɡəvax]  /ɡvah     naha }\textup{\textsuperscript{e}}\textup{/  [F0AE?]  [ɡəvax nɛhɛ]}
\z

‘field’     ‘field’    DEM\textsc{    } ‘this field’


\ea
\textup{[ɡəvax]    /ɡvah     =atəta/  [F0AE?]  [ɡəvahatəta]}\textup{   }
\z

‘field’     ‘field’  \textsc{3P.POSS    } ‘their field’    


Ex. 66 shows word-final changes for /h/for the verb stem /b h/. In contrast, the 1S indirect object pronominal clitic /=aw / (ex. 67, see \sectref{sec:491}) is phonologically bound to its stem since the /h/ does not undergo word-final changes.

\ea
\textup{[a-bax   jam}\textup{]}
\z

3S-pour  water



He poured water.


\ea
\textup{ }\textup{[ɓax]  /a-ɓh =aw/  [F0AE?]  [aɓahaw]}\textup{  }
\z

‘sew’\textsc{   3S}{}-sew=\textsc{1S.IO}     ‘he/she sews for me’



\begin{itemize}
\item A word-final /n/ is realised as [ŋ] (Bow, 1997c). 
\end{itemize}

Word-final changes in ex. 68 indicate a word break between the noun \textit{həlaŋ} ‘back’ and \textit{na}. Ex. 69 is more complicated. The initial consonant of the adverbiser [ŋa] (see \sectref{sec:26}) has assimilated to the final consonant of the noun, indicating that they are phonologically bound. However, the fact that the noun [dedeŋ] ‘truth’ exhibits  word-final changes indicates that [ŋa] cliticises after word final changes in the noun have occurred. 


\ea
\textup{[həlaŋ]    /a  hlan     na /  [F0AE?]   [ahəlaŋna]}\textup{    }
\z

‘back’    to  back     PSP      ‘behind’ 


\ea
\textup{[dedeŋ]    /dɛdɛn }\textup{\textsuperscript{e}}\textup{    =Ca /  [F0AE?]   [dɛdɛŋŋa]}
\z

‘truth’     ‘truth’        ADJ     ‘truly’


Word-final changes indicate a word break after the verb [ahaŋ] in ex. 70. In contrast, ex. 71 demonstrates no word-final allophones indicating that the indirect object pronominal enclitic [=aw] is phonologically bound to the verb stem /dz n –aj/\footnote{The verb stems /h-aj/ ‘greet’and /dz n -j/  ‘help’ both carry the /-j/ suffix. This suffix is deleted whenver an affix or extension is attached to the verb stem (\sectref{sec:6.3}).} (see \sectref{sec:491}).

\ea
\textup{[ahaj]  / a-h-aj    =an      ma /    [F0AE?]   [ahaŋma]}
\z

‘he/she speaks’  3S-say{}-CL  =3S.IO  mouth       ‘he/she greeted him/her’


\ea
\textup{[adzənaj]    / a-dz n-aj  =aw /    [F0AE?]   [ajənaw]}
\z

‘he/she helps’  3S-help{}-CL  =1S.IO    ‘he/she helped me’



\begin{itemize}
\item Prosodies spread over a word but do not cross word boundaries (Bow, 1997c). 
\end{itemize}

Ex. 72 - 74 illustrate nouns. Ex. 72 and 73 show that the possessive pronouns are phonologically separate from the nouns that they modify since the prosodies do not spread leftwards over the nouns (labialisation in ex. 72, palatalisation in ex. 73). In contrast, ex. 74 shows that the /a-/ prefix is part of the same phonological word as the noun root, since the prosody of the noun root spreads to the prefix.\footnote{Note that the labialisation prosody may not spread as far left as the prefix in some words (\sectref{sec:2.1}). The fact that it sometimes spreads indicates that the /a-/ is indeed phonologically bound. } 


\ea
\textup{/m ze }\textup{\textsuperscript{e}}\textup{       s l m }\textup{\textsuperscript{o}}\textup{/  [F0AE?]  [m}\textup{ɪʒɛ}\textup{s}\textup{ʊ}\textup{lɔm]}
\z

person    \textsc{     }peace      ‘person characterised by peace’


\ea
\textup{/war     ala }\textup{\textsuperscript{e }}\textup{/    [F0AE?]  [war}\textup{ɛ}\textup{lɛ] }
\z

child       eye        ‘grain’ (lit. child eye)


\ea
\textup{/a-tama }\textup{\textsuperscript{e}}\textup{/      [F0AE?]  [ɛtɛmɛ]}
\z

onion        ‘onion’


Ex. 75 - 79 illustrate verbs. The words [awij] and [nɛʃɛ] in ex. 75 are shown to be separate words since the palatalisation prosody of the verb [nɛʃɛ] does not spread to [awij]. In contrast, the subject pronominal prefixes (shown in ex. 76 and 78) and suffixes (shown in ex. 77 and 79) are phonologically bound to the verb stem since prosodies will spread leftwards from verb stem to prefix and suffix to verb stem. Note that the subject prefix in ex. 76 takes on the palatalisation prosody of the verb stem. The pronominal morphemes shown in ex. 77 and 79 are shown to be phonologically bound suffixes. Compare ex. 76 with 77 and ex. 78 with 79. In the second example of each pair, the labialisation prosody of the subject pronominal morphemes /{}-am \textsuperscript{o}/ (ex. 77) and /{}-ak \textsuperscript{o}/ (ex. 79) spreads over the verb stems, even overcoming the underlying palatalisation prosody on the verb stem in ex. 77. 

\ea
\textup{/awj      n-   s }\textup{\textsuperscript{e}}\textup{/  [F0AE?]  [awij}\textup{nɛ}\textup{ʃɛ]}
\z

3S\_say  1S-drink    ‘he/she says ‘I drink’’


\ea
\textup{/n-   s }\textup{\textsuperscript{e}}\textup{/    [F0AE?]  [}\textup{nɛ}\textup{ʃɛ]}
\z

1S-  drink      ‘I drink’


\ea
\textup{/n- s }\textup{\textsuperscript{e}}\textup{ -am }\textup{\textsuperscript{o}}\textup{/   [F0AE?]  [nɔsɔm]}
\z

1-drink-1\textsc{Pex}      ‘we drink’


\ea
\textup{/n- ɮar/     [F0AE?]  [naɮar]}\textup{  }
\z

1S-kick      ‘I kick’


\ea
\textup{/m-      ɮar   {}-ak }\textup{\textsuperscript{o}}\textup{/  [F0AE?]  [mɔɮʊrɔk}\textup{\textsuperscript{w}}\textup{]}
\z

1\textsc{Pex}{}-  kick   {}-1\textsc{Pex}    ‘we kicked’



\begin{itemize}
\item The -\textit{aj} suffix in verbs drops off when suffixes or extensions are attached to the verb. 
\end{itemize}

Ex. 80 and 81show the verb /p -j/ ‘open.’ In the 3S form, the verb carries the \textit{{}-}\textit{aj} suffix. The 3S direct object\textit{ na }is a separate word since the \textit{{}-}\textit{aj} suffix remains on the stem (ex. 81). The directional \textit{ala} is shown to be phonologically bound to the verb stem since when \textit{ala} is present (ex. 81) the \textit{{}-}\textit{aj} suffix drops off.


\ea
a-p-aj       na
\z

3S-open{}-CL  3S.DO



‘He/she opens it.’


\ea
a-p  =ala
\z

3S-open{}-CL  =towards



 ‘It opens towards.’



\begin{itemize}
\item Deletion of word-final /n/ (Bow, 1997c)
\end{itemize}

Deletion of word-final /n/ occurs before certain clitics (the possessive and plural in nouns, see Sections 14 and 33, respectively) and before some verbal extensions (see \sectref{sec:56}).\footnote{Word-final /n/ is not deleted in any other environment. } Ex. 82 shows that word-final /n/ is deleted before the plural marker [=ahaj ]. For comparison, ex. 83 shows word-final changes between [ɛŋgɛrɛŋ] and [aɮa], necessitating [ŋ] the word-final allophone of /n/). Syllables are separated by a period in the phonetic form.


\ea
\textup{/ɓərɮan =ahaj/[F0AE?]  [ɓər.ɮa.haj]}
\z

mountain =Pl  



‘mountains’


\ea
\textup{/ɛŋgɛrɛn    aɮa/  [F0AE?]  [ɛ.ŋgɛ.rɛ.ŋa.ɮa]}
\z

insect        now    ‘insect now’


A similar phenomenon occurs in the verb complex (ex. 84 - 85). The adpositional \textit{=aka} (see \sectref{sec:56}) causes the deletion of word-final /n/ in a verb stem (ex. 84).\footnote{The vowel is not deleted, resulting in a long vowel. } Ex. 85 shows the typical word final allophone [ŋ] for comparison. 

\ea
\textup{/a-mbəɗ   =aŋ   =aka/[F0AE?]  [a.mbə.ɗaa.ka]}
\z

3S-change =3S.IO =on    ‘he/she replied.’


\ea
\textup{/a-b=an        ana  m}\textup{ɪ}\textup{ʒ}\textup{ɛ}\textup{/  [F0AE?]  [a.ba.ŋa.na.m}\textup{ɪ.}\textup{ʒ}\textup{ɛ}\textup{]}
\z

3S-hit=3S.IO  to    person     ‘he/she hit someone.’


\subsection{Affix, clitic, and extension}
\hypertarget{RefHeading1210781525720847}{}
Five criteria are also used to categorise affixes, clitics, and extensions in Moloko. The first is whether the morpheme can occur in discourse without being bound to some other morpheme. Affixes, clitics, and extensions in Moloko are bound morphemes – they cannot occur alone in discourse. The second criterion is whether prosodies will spread freely between the stem and morpheme in question. Prosodies will always spread between affix and stem, and sometimes between clitic or extension and stem. The third criterion is whether word-final alternations are found in the final consonant of the stem when a morpheme is attached. Suffixes, clitics, and extensions will always block word-final changes in the stem. The fourth and fifth criteria are to distinguish clitics from affixes. Clitics can attach to words of different syntactic categories; whereas no separate word can be inserted between an affix and its stem. Finally, clitics function at the phrase or clause level with grammatical rather than lexical meaning.\footnote{Payne, 1997: 22.} In contrast, affixes may have grammatical meaning but their meaning is applied to the word they modify. 

What we have classified as an affix in Moloko is tightly bound to the stem. No morpheme known to be a separate word can occur between the affix and its stem. Prosodies spread freely between affix and stem. There are no word-final alternations in the final consonant of the stem when a suffix is attached. Examples of affixes in this section include the /a-/ prefix in nouns and subject pronominal prefixes and suffixes in verbs. 

A clitic carries some of the characteristics of an affix and some of an independent word, and different clitics in Moloko fulfil the above criteria differently. A clitic is similar to an affix in that it is phonologically bound to the stem to which it is attached. However the nature of that phonological bondedness is different than for an affix and its stem. Grammatically, a clitic is different from an affix because a known separate word can occur in between the relevant stem and the clitic, and the clitic will then attach itself phonologically to the inserted word.  

The verbal extensions are a special class of clitics which are something between a prototypical affix and a prototypical clitic. They form a close phonological unit with the verb stem. The phonological structure of the verb word will be more fully discussed with examples in \sectref{sec:7.1}, but a few summary statements are included here. When there is no suffix on the verb, extensions will cliticise to the verb stem. Prosodies on verb clitics always spread to the verb stem (see \sectref{sec:7.5}).  When there is a suffix on the verb, extensions form a separate phonological word and they cliticise to each other. In addition, the direct object pronominal extension is a separate word from the verb stem, but will be embedded amongst any other extensions that occur. In the presence of the direct object extension, the other extensions will cliticise to the direct object extension. The Perfect extension is a special enclitic in Moloko. It can occur at the end of the verb word or the end of the verb phrase (see \sectref{sec:58}).  The Perfect extension appears to have a stronger phonological connection with the verb stem than the other extensions because the neutral prosody of the extension will neutralise the prosody of the verb word even if the Perfect is phrase-final with many intervening words (see \sectref{sec:58}).  

The adverbiser /Ca/ (see \sectref{sec:26}) is an interesting clitic in the way it is phonologically bound to its noun. The noun displays word-final changes, which would normally indicate a word break. However, initial consonant of the adverbiser enclitic is a reduplication of the final consonant of the noun (ex. 69 in \sectref{sec:11}) which indicates that the reduplication occurs after phonological word-final alterations are made to the noun. 

We consider both the plural marker (see \sectref{sec:33}) and possessive (see \sectref{sec:14}) to be clitics even though neither the plural nor the possessive will affect the prosody of the stem (see \sectref{sec:11}). Both plural marker and possessive are phonologically bound to a stem yet modify a larger structure (a noun phrase). There are no word-final changes that indicate a word break on the stem when the plural or possessive is added. They are clitics and not affixes since they bind to elements of different grammatical classes (noun or noun phrase in the case of the possessive; noun, noun phrase, numeral, or pronoun in the case of the plural). 

\chapter[Grammatical classes]{Grammatical classes}
\hypertarget{RefHeading1210801525720847}{}
Moloko has the following grammatical classes, each described in successive sections or chapters below:\footnote{Note the absence of adjectives as a word class, since all adjectives in Moloko are derived from nouns (section 5.3).}

\begin{itemize}
\item nouns, which can be simple, compound, or derived from a verb (Chapter Error: Reference source not found)
\item verbs  (Chapter 6)
\item pronouns, both free and attached (as prefixes, suffixes, or clitics; \sectref{sec:3.1})
\item demonstratives and demonstrationals (\sectref{sec:3.2})
\item numerals and quantifiers (\sectref{sec:3.3})
\item existentials (\sectref{sec:3.4}), which are verb-like but pattern differently than verbs
\item adverbs (see \sectref{sec:25}), which can be simple or derived from nouns or verbs
\item ideophones (\sectref{sec:3.6}), which pattern as adverbs, adjectives, or in particular cases, as verbs
\item adpositions (\sectref{sec:5.6}), 
\item discourse markers, including the presupposition marker (\sectref{sec:12}), 
\item conjunctions and conjunctive adverbs (see Sections 13.3), 
\item interjections (see \sectref{sec:3.7})
\item the negative (\sectref{sec:11.2}), which can be simple or compounded with certain adverbs
\end{itemize}

In the following sections, a detailed treatment will be given for each of these word classes and the morphological structure of each class.  An operational definition will be given for each class, so that any word in the language can be readily classified.

The first line in the examples is written in the orthography. The second line is the phonetic form for slow speech with morpheme breaks. All consonantal and vowel allophones will be indicated for the sake of the non-speaker.  Palatisation and labialisation prosodies will be discernible from the quality of the vowels and the consonants. When an  underlying form (typically identified by / / brackets) is cited, only the consonants and the full vowels will be written (i.e. not the epenthetic schwas). The palatalisation or labialisation prosody on the form will be marked by a superscripted ‘e’ or ‘o,’ respectively, after the morpheme. 

\section{Pronouns }
\hypertarget{RefHeading1210821525720847}{}
Pronouns stand in the place of a noun phrase in a clause.  Pronouns are deictic elements – their reference changes according to the context of the utterance.  The role of the speaker furnishes the basic point of reference (first person). The addressee is defined with respect to the speaker (second person).  The third person pronouns refer to people or things being talked about by the first and second persons. There are definite and indefinite third person pronouns. Definite pronouns can be used anaphorically, and their reference is determined by linguistic or pragmatic elements in the textual or extratextual environment. Indefinite pronouns have a non-identified referent. Other types of pronouns are not discussed in this work. 

Moloko personal pronouns and proforms are illustrated in \tabref{tab:15}.. Moloko has one set of free personal pronouns (regular, see \sectref{sec:131}), one set of bound pronouns (possessive, see \sectref{sec:14}), and three sets of pronominals within the verb complex for subject, direct object, and indirect object (see \sectref{sec:7.3}). All personal pronouns and pronominals are shown in \tabref{tab:15}..{ }The regular free pronouns can refer to any of the subject or direct object or indirect object. An emphatic subset of free pronouns exists, formed by adding the adjectiviser \textit{ga} to the regular personal pronouns. Possessive pronouns always occur within a noun phrase or a relative clause.

Special vocative pronouns that attach to nouns are honorific (\sectref{sec:15}).  There are also interrogative pronouns (\sectref{sec:16}), unspecified pronouns (\sectref{sec:17}), and pro-clauses (Section Error: Reference source not found). 

In some of the pronoun sets, there is an inclusive/exclusive distinction in the first person plural. There are no dual nor gender-specific forms, nor are there logophoric pronouns.\footnote{\citet{Frajzyngier1985} describes the types of logophoric systems found in some Chadic languages. No logophoric pronouns are described for Biu-Mandara. }  

\begin{tabular}{lllllll} & \multicolumn{2}{l}{\textbf{Free pronouns}} & \textbf{Bound} & \multicolumn{3}{l}{\textbf{Pronominal affixes and extensions}\footnotemark{}}\\
\lsptoprule
\textbf{Person} & \textbf{Regular} & \textbf{Emphatic} & \textbf{Possessive suffix} & \textbf{Subject pronominal affixes}\footnotemark{} & \textbf{Dedicated direct object pronominals}\footnotemark{} & \textbf{Indirect object pronominal enclitic}\\
1S & \textit{nɛ} & \textit{nɛ ga} & \textit{=uwla} & \textit{n-} &  & \textit{=aw }\\
2S & \textit{nɔk}\textit{\textsuperscript{w}} & \textit{nɔk}\textit{\textsuperscript{w}}\textit{ ga} & \textit{=aŋg}\textit{\textsuperscript{w}}\textit{ɔ (k)}\footnotemark{} & \textit{k-} &  & \textit{=ɔk}\textit{\textsuperscript{w}}\\
3S & \textit{ndahaŋ} & \textit{ndahaŋ ga} & \textit{=ahaŋ} & \textit{a-} & \textit{na} & \textit{=aŋ }\\
1\textsc{Pin} & \textit{lɔk}\textit{\textsuperscript{w}}\textit{ɔ} & \textit{lɔk}\textit{\textsuperscript{w}}\textit{ɔ ga} & \textit{=alɔk}\textit{\textsuperscript{w}}\textit{ɔ} & \textit{m/k-…-ɔk} &  & \textit{=alɔk}\textit{\textsuperscript{w}}\textit{ɔ}\\
1\textsc{Pex} & \textit{lɪmɛ} & \textit{lɪmɛ ga} & \textit{=alɪmɛ} & \textit{n-  . . . -ɔm} &  & \textit{=alɪmɛ}\\
2P & \textit{lʊk}\textit{\textsuperscript{w}}\textit{øjɛ} & \textit{lʊk}\textit{\textsuperscript{w}}\textit{øjɛ ga} & \textit{=alʊk}\textit{\textsuperscript{w}}\textit{øjɛ} & \textit{k- . . .  {}-ɔm} &  & \textit{=alʊk}\textit{\textsuperscript{w}}\textit{øjɛ}\\
3P & \textit{təta} & \textit{təta ga} & \textit{=atəta} & \textit{ta- } & \textit{ta} & \textit{=ata}\\
\lspbottomrule
\end{tabular}
\addtocounter{footnote}{-4}
\stepcounter{footnote}\footnotetext{ Pronominals will be discussed in \sectref{sec:7.3.}}
\stepcounter{footnote}\footnotetext{ Note that the 1P and 2P bound pronominals consist of both a prefix and a suffix. They are further discussed in \sectref{sec:7.3.}}
\stepcounter{footnote}\footnotetext{ Note that although \textit{na} and \textit{ta} are free in that they are phonologically separate from the verb word, they are closely bound parts of the verb complex and so are called pronominal extensions, see \sectref{sec:1150.}}
\stepcounter{footnote}\footnotetext{ This pronoun is pronounced either \textit{aŋg}\textit{\textsuperscript{w}}\textit{ɔ} or \textit{aŋg}\textit{\textsuperscript{w}}\textit{ɔk}\textit{\textsuperscript{w }} by people from different regions.}

\begin{itemize}
\item \begin{styleTabletitle}
Moloko personal pronouns and pro-forms
\end{styleTabletitle}\end{itemize}
\subsection{Regular and emphatic free personal pronouns}
\hypertarget{RefHeading1210841525720847}{}
Free pronouns express subject, direct object, and indirect object. They are relatively rare in texts since participants are generally tracked by the bound verbal pronominals.  Free pronouns are found in cases of switch reference, at the peak of a story where the verbal pronominals disappear, or in cases of emphasis (see \sectref{sec:2}).  

\paragraph[Regular pronouns]{Regular pronouns}

When free subject, direct object, or indirect object pronouns do occur, they are in the same place within a clause or noun phrase where one would expect the full noun phrase to be (see Sections 69 and 5.1). 

The clause in ex. 86 has subject (\textit{Mala}), direct object (\textit{dalaj} ‘girl’), and indirect object (\textit{Arsakaj}, a proper name for a male).  Note that the subject is also indicated on the verb by the subject pronominal \textit{à-} and the indirect object is indicated on the verb by the indirect object pronominal enclitic \textit{=aŋ} (see \sectref{sec:7.3}). The noun phrase representing the indirect object is within a prepositional phrase (see \sectref{sec:45}). 


\ea
Mala  avəlan  dalay  ana  Arsakay.
\z

Mala   à-vəl=aŋ     dalaj   ana   Arsakaj



Mala    3S.PFV-give=3S.IO  girl  DAT  Arsakay



‘Mala gave the girl to Arsakay’


When the subject is replaced by a free pronoun (ex. 87), the pronoun must be marked as presupposed in the clause (see \sectref{sec:12.2}). Note that since subject is pronominalised in the verb word (see \sectref{sec:7.3}); and thus the presence of any noun phrase or free pronoun is for pragmatic purposes.

\ea
\textbf{Ndahan  }na,    avəlan  dalay  ana  Arsakay.
\z

\textbf{ndahaŋ}  na    à-vəl=aŋ     dalaj   ana   Arsakaj



3S    PSP  3S.PFV-give=3S.IO  girl  DAT  Arsakay



‘He (for his part), he gave the girl to Arsakay’


When the direct object is replaced by a free pronoun (compare ex. 86 and 88), the pronoun \textit{ndahaŋ} (replacing\textit{ }\textit{dalaj}) occurs in the normal direct object slot in the clause.\footnote{The dedicated direct object pronominal \textit{na} is can also replace a direct object noun phrase in the case of an inanimate object, \sectref{sec:1150.} }  

\ea
Mala  avəlan  \textbf{ndahan  }ana  Arsakay.
\z

Mala a-vəl=aŋ     \textbf{ndahaŋ}   ana   Arsakaj



Mala  3S-give=3S.IO    3S  DAT  Arsakay



‘Mala gave her to Arsakay’


When the indirect object is replaced by a free pronoun, the pronoun occurs in a prepositional phrase (ex. 89). The prepositional phrase is delimited by square brackets. Note that the indirect object pronominal enclitic =\textit{aŋ} co-occurs on the verb complex (see \sectref{sec:491}).

\ea
Mala  avəl\textbf{an  }dalay  [ana  \textbf{ndaha}\textbf{n.}]
\z

Mala   a-vəl\textbf{=aŋ}     dalaj   [ana   \textbf{ndahaŋ}]



Mala  3S-give=3S.IO    girl  DAT  3S



‘Mala gave the girl to him.’


The indirect object pronominal enclitic can entirely stand in the place of the prepositional phrase expressing indirect object with no loss in meaning (see 90, \sectref{sec:491}).

\ea
Mala  avəl\textbf{an  }dalay.
\z

Mala   a-vəl\textbf{=aŋ}     dalaj 



Mala  3S-give=3S.IO    girl



‘Mala gave the girl to him.’


\paragraph[Emphatic pronouns]{Emphatic pronouns}

Emphatic pronouns are formed by adding either the adjectiviser \textit{ga} (Chapter 5.3) or the third person singular possessive pronoun form \textit{=ahaŋ}\textbf{ }to the free pronoun (ex. 91 - 93). 


\ea
\textbf{Ne  ga}  nege.
\z

\textbf{nɛ  ga}     nɛ-g-ɛ



1S  ADJ    1S.PFV-do-CL



‘It was me, I did it.’ (lit. me, I did)


\ea
\textbf{Ne  ga}  aməgəye. 
\z

\textbf{nɛ  ga}     amɪ-g-ijɛ 



1S   ADJ    DEP-do-CL



‘It was me who did it.’ (lit. me, the one that did)


\ea
\textbf{Ne  ahan}  nege. 
\z

\textbf{nɛ   =ahaŋ}     nɛ-g-ɛ



1S  =3S.POSS  1S.PFV-do-CL



‘It was me, I did it.’ (lit. me, I did)


\subsection{Possessive pronouns }
\hypertarget{RefHeading1210861525720847}{}
Another set of Moloko pronouns occurs only within noun phrases and among its primary uses, indicates a possessive relationship, i.e. these pronouns relate the possessor referent to the person or thing that is possessed. Possessive pronouns immediately follow the noun or noun phrase they modify (ex. 94 - 96) and occur before the plural (ex. 97).{ }\footnote{\citet{Bow1997c} postulated that the set of possessive pronouns did not include the plural possessive pronouns. Rather, she proposed that the plural possessive was actually an associative noun phrase formed by the preposition /a/ and the free pronoun (\textit{a} \textit{lɔk}\textit{\textsuperscript{w}}\textit{ɔ, a lɪmɛ, a lʊk}\textit{\textsuperscript{w}}\textit{øjɛ, }and\textit{ a tta}). We found that possessives are viewed as a set in the minds of speakers, and that there is no difference in distribution between singular and plural possessives. Therefore we will treat the possessive pronouns as a set of in Moloko (\textit{alɔk}\textit{\textsuperscript{w}}\textit{ɔ, alɪmɛ, alʊk}\textit{\textsuperscript{w}}\textit{øjɛ, }and\textit{ ata}). }  


\ea
hor  \textbf{aha}\textbf{n}
\z

h\textsuperscript{w}ɔr   =\textbf{ahaŋ}



woman  =3S.POSS



‘his wife’


\ea
məgəye  \textbf{ang}\textbf{o}
\z

mɪ-g-ijɛ   =\textbf{aŋg}\textbf{\textsuperscript{w}}\textbf{ɔ}



\textsc{NOM}{}-do-CL  =2S.POSS



‘your doings’


\ea
war   dalay  \textbf{ahan}
\z

war     dalaj    =\textbf{ahaŋ}



child  girl  =3S.POSS



‘his daughter’


\ea
anjakar  \textbf{atəta}  ahay
\z

anzakar   =\textbf{atəta}     =ahaj



chicken  =3P.POSS  =Pl



their chickens’


We consider the possessive pronouns to be noun clitics.{ }They are phonologically bound to the noun. Even though prosodies on the possessive pronouns do not spread to the noun (ex. 94 and 95), \citet{Bow1997c} demonstrated that word-final changes indicating a word break do not occur (\tabref{tab:16}.). They are clitics, not affixes, since they bind to the right edge of the head of the noun phrase, binding to the final noun where the head is composed of more than one noun, yet modifying the entire structure (ex. 96, see \sectref{sec:43}).{ }

\begin{tabular}{lllll}
 & \textbf{Underlying form} & \multicolumn{2}{l}{\textbf{Surface forms of isolated words}}


 & \textbf{Gloss}\\
\lsptoprule
\textbf{Neutral } & /ɡ v h/ & [ɡəvax]   [uwla]   [F0AE?]

‘field’     =1S.POSS & [ɡəvəhuwla] & ‘my field’\\
\textbf{Labialised} & /hamb h \textsuperscript{o}/ & [hɔmbɔx] [uwla]   [F0AE?]

‘pardon’  =1S.POSS & [hɔmbʊhuwla] & ‘my pardon’\\
\textbf{Palatalised} & /ta z h \textsuperscript{e}/ & [tɛʒɛx]   [uwla]   [F0AE?]

‘snake’  =1S.POSS & [tɛʒɛhuwla] & ‘my snake’\\
\lspbottomrule
\end{tabular}

\begin{itemize}
\item \begin{styleTabletitle}
Possessive cliticising to nouns with word-final /h/ 
\end{styleTabletitle}\end{itemize}
\paragraph[Semantic range of possessive constructions]{Semantic range of possessive constructions}

The semantic relation between the possessor and possessed can be flexible and covers the same range of possibilities as the associative construction (see \sectref{sec:42}). These semantic categories include ownership (ex. 98{}-100),\footnote{Examples 98{}-100 show that alienable and inalienable is not a relevant disctinction for Moloko. } kinship relationships (ex. 101), part-whole (ex. 102) and other associations (ex. 103{}-104).  


\ea
awak  \textbf{əwla}
\z

awak =\textbf{uwla}



goat    =1S.POSS



‘my goat’ (i.e. the goat I own)


\ea
hay   \textbf{əwla}
\z

haj     =\textbf{uwla}



house  =1S.POSS



‘my house’ (i.e. the house I own/live in)


\ea
gəvah  \textbf{əwla}
\z

gəvax  =\textbf{uwla}



field  =1S.POSS



‘my field’ (i.e. the field I own)


\ea
baba  \textbf{əwla}
\z

baba   =\textbf{uwla}



father  =1S.POSS



‘my father’ (also, an older man in my father’s family)


\ea
asak  \textbf{əwla}
\z

asak   =\textbf{uwla}



foot    =1S.POSS



‘my foot’ 


\ea
məgəye  \textbf{əwla}
\z

mɪ-g-ijɛ   =\textbf{uwla}



\textsc{NOM}{}-do-CL  =1S.POSS



‘my doings’ (i.e. the things I do)


\ea
məzəme  \textbf{əwla}
\z

mɪ-ʒʊm-ɛ     =\textbf{uwla}



\textsc{NOM}{}-eat-CL  =1S.POSS



‘my food’ (i.e. the food I grew/ the food that I am eating)


\paragraph[Tone of possessive pronouns]{Tone of possessive pronouns}

\citet{Bow1997c} concluded that the underlying tone melody for possessive pronouns is HLH. \tabref{tab:17}. (from Bow, 1997c) shows the surface tonal melodies and underlying tone pattern for all the possessive pronouns with the noun \textit{ɗ\={a}f}  ‘loaf.’\footnote{In Moloko, \textit{ɗ}\textit{af} is the basic starch form consumed by the people, a millet porridge eaten with various sauces. The word can refer to one loaf of the porridge, and can also simply mean ‘food’.}. The singular forms with only two syllables drop the final high tone. All forms but the 2S have the HM(H) surface pattern; the 2S form contains the depressor consonant /ŋɡ/ and so the second syllable is low tone. 

\begin{tabular}{llll} & \textbf{Possessive pronoun in noun phrase} & \textbf{Surface tone of possessive pronoun} & \textbf{Underlying tone of possessive pronoun}\\
\lsptoprule
1S & \textit{ɗ\={a}f} \textit{ú}\textit{wl\={a}}

‘my loaf’ & HM & HL\\
2S & \textit{ɗ\={a}f ɔŋɡʷɔ}

‘your loaf’ & HL & HL\\
3S & \textit{ɗ\={a}f áh\={a}ŋ}

‘your loaf’ & HM & HL\\
1\textsc{Pin} & \textit{ɗ\={a}f álɔkʷɔ}

‘our (inclusive) loaf’ & HMH & HL\\
1\textsc{Pex} & \textit{ɗ\={a}f álɪmɛ}

‘our (exclusive) loaf’ & HMH & HLH\\
2P & \textit{ɗ\={a}f álʊk}\textit{[1FF?]}\textit{jɛ}

‘your (P) loaf’ & HMH & HLH\\
3P & \textit{ɗ\={a}f átətá}

‘their loaf’ & HMH & HLH\\
\lspbottomrule
\end{tabular}

\begin{itemize}
\item \begin{styleTabletitle}
Possessive pronoun paradigm with tone marked
\end{styleTabletitle}\end{itemize}
%%please move \begin{table} just above \begin{tabular
\begin{table}
\caption{(from Bow, 1997c) gives examples of nouns with each underlying tone melody combined with 2S, 3S and 1PEX possessive pronouns.  Some of the rules governing variations in the surface form were considered in \sectref{sec:8.} The possessive pronoun maintains its tonal melody in every environment. Note that the low surface tone of [dàndàj] ‘intestines’ (due to the depressor consonant) lowers the first high tone of the 3S and 1PEX possessive.}
\label{tab:18}
\end{table}

\begin{tabular}{llllll}
\lsptoprule

x121 & \textbf{Example} & \textbf{Gloss} & \textbf{2S} & \textbf{3S} & \textbf{1}\textbf{\textsc{Pex}}\\
H & [tsáf] & ‘shortcut’ & [tsəf ɔŋɡʷɔ] & [tsəf áh\={a}ŋ] & [tsəf álɪmɛ]\\
& [bɔɮɔm] & ‘cheek’ & [bɔɮʊm ɔŋɡʷɔ] & [bɔɮʊm áh\={a}ŋ] & [bɔɮʊm álɪmɛ]\\
L & [ɗ\={a}f] & ‘loaf’ & [ɗəf ɔŋɡʷɔ] & [ɗəf áh\={a}ŋ] & [ɗəf álɪmɛ]\\
& [dàndàj] & ‘intestines’ & [dànd\`{i}j ɔŋɡʷɔ] & [dànd\`{i}j \={a}h\={a}ŋ] & [dànd\`{i}j \={a}lɪmɛ]\\
HL & [mɛkɛtʃ] & ‘knife’ & [mɛkɪtʃ ɔŋɡʷɔ] & [mɛkɪtʃ áh\={a}ŋ] & [mɛkɪtʃ álɪmɛ]\\
& [mɔɡʷɔdɔkʷ] & ‘hawk’ & [mɔɡʷɔdʊkʷ ɔŋɡʷɔ] & [mɔɡʷɔdʊkʷ \={a}h\={a}ŋ] & [mɔɡʷɔdʊkʷ \={a}lɪmɛ]\\
LH & [ɬəmáj] & ‘ear’ & [ɬəm\'{i}j ɔŋɡʷɔ] & [ɬəm\'{i}j áh\={a}ŋ] & [ɬəm\'{i}j álɪmɛ]\\
& [bɔɡʷɔm] & ‘hoe’ & [bɔɡʊm ɔŋɡʷɔ] & [bɔɡʊm áh\={a}ŋ] & [bɔɡʊm álɪmɛ]\\
HLH & [ákʊfɔm] & ‘mouse’ & [ákʊfʊm ɔŋɡʷɔ] & [ákʊfʊm áh\={a}ŋ] & [ákʊfʊm álɪmɛ]\\
& [dɛdɪlɛŋ] & ‘black’ & [dɛdɪl ɔŋɡʷɔ] & [dɛdɪl \={a}h\={a}ŋ] & [dɛdɪl \={a}lɪmɛ]\\
LHL & [səsáj\={a}k] & ‘wart’ & [səsájəkʷ ɔŋɡʷɔ] & [səsájək áh\={a}ŋ] & [səsájək álɪmɛ]\\
& [məŋɡáhàk] & ‘crow’ & [məŋɡáhəkʷ ɔŋɡʷɔ] & [məŋɡáhək \={a}h\={a}ŋ] & [məŋɡáhək \={a}lɪmɛ]\\
\lspbottomrule
\end{tabular}
\begin{itemize}
\item \begin{styleTabletitle}
Tonal melodies in possessive constructions
\end{styleTabletitle}\end{itemize}
\subsection{Honorific possessive pronouns}
\hypertarget{RefHeading1210881525720847}{}
There are two special possessive pronouns used within vocative expressions to give honour to the person addressed.  The honorific pronouns are grammatically bound to the noun they follow.  They are used to honour people both within and outside the family. For men and women, whether married or not, to address one another with honour, \textit{g}\textit{\textsuperscript{w}}\textit{ɔlɔ} ‘dear/honourable’ follows the noun (ex. 105 and 106); for other relationships (mother, father, grandmother)  \textit{ja} ‘dear/honourable’ follows the noun (ex. 107 - 109). 


\ea
hor  \textbf{golo} 
\z

h\textsuperscript{w}ɔr \textbf{g}\textbf{\textsuperscript{w}}\textbf{ɔlɔ} 



woman  HONOUR



my dear wife 


\ea
zar  \textbf{g}\textbf{olo} 
\z
\ zar     \textbf{g}\textbf{\textsuperscript{w}}\textbf{ɔlɔ} 



man    HONOUR



my dear husband


\ea
baba  \textbf{ya} 
\z

baba  \textbf{ja} 



father  HONOUR



my dear father\textit{ }


\ea
dede  \textbf{ya} 
\z

grandmother  HONOUR



my dear grandmother


\ea
Mama  \textbf{ya}  asaw  ɗaf.
\z

mama  \textbf{ja}     a-s=aw      ɗaf



mother  HONOUR  3S-please=1S.IO    loaf



My dear mother, I want food! (lit. food is pleasing to me)


\subsection{Interrogative pronouns}
\hypertarget{RefHeading1210901525720847}{}
Interrogative pronouns request content information about an event, state, or participant (who, what, when, where, why, how). The basic interrogative words in Moloko are shown in \tabref{tab:19}..\footnote{Table adapted from Boyd, 2003.} 

\tablehead{
\textbf{Element questioned} & \textbf{Interrogative pronoun} & \textbf{Gloss} & \textbf{Example numbers}\\
}
\begin{tabular}{llll}
\lsptoprule
\textbf{Clause constituent} & \textit{waj} & ‘who’ (human) & 110 and 111\\
& \textit{almaj} & ‘what’ (non-human) & 112 and 113\\
& \textit{ɛpɛlɛj} & ‘when’ & 114\\
& \textit{amtamaj} & ‘where’ & 115\\
& \textit{kamaj} & ‘why’ & 116\\
& \textit{mɛmɛj} & ‘how/ explain’ & 117 and 118\\
& \textit{malmaj} & ‘what is this’ & 120 and 119\\
\textbf{Noun phrase constituent} & \textit{mɪtɪmɛj} & ‘how much’ & 121\\
& \textit{wɛlɛj} & ‘which one’ & 122\\
\hhline{~---}
\lspbottomrule
\end{tabular}

\begin{itemize}
\item \begin{styleTabletitle}
Interrogative pronouns
\end{styleTabletitle}\end{itemize}

The normal position for interrogative pronouns is clause or noun phrase final.\footnote{See question constructions in Moloko, \sectref{sec:11.3}}  Two of the interrogative pronouns (\textit{memej} ‘how,’ and \textit{malmaj} ‘what’) can question a clause in and of themselves (ex. 118 and 120). In each example, the interrogative pronoun is bolded. 


\ea
Aməvəlok  baskor  na  \textbf{wa}\textbf{y?}
\z

amə-vəl=ɔk\textsuperscript{w} bask\textsuperscript{w}ɔr   na   \textbf{waj}



DEP-give=2S.IO  bicycle  PSP  who



‘Who gave you the bicycle?’ (lit. the one that gave you the bicycle [is] who?)


\ea
Mana  amənjar  \textbf{wa}\textbf{y?}
\z

Mana   à-mənzar   \textbf{waj}



Mana  3S.PFV-see  who



‘Whom did Mana see?’


\ea
Kənjakay  \textbf{alma}\textbf{y?}
\z

kə-nzak-aj    \textbf{almaj}



2S.PFV-find-CL  what



‘What did you find?’


\ea
Kəzom  \textbf{alma}\textbf{y?}
\z

kə-zɔm  \textbf{almaj}



2S\_P\v{ }FV-eat  what



‘What did you eat?’


\ea
Kálala  \textbf{epel}\textbf{ey?}~
\z

ká-l=ala     \textbf{ɛpɛlɛj}~



2S.IFV-go=to   when



‘When are you going?’


\ea
Kólo  amtamay?~
\z

kɔ-lɔ     amtamaj



2S+IPV-go    where



‘Where are you going?’


\ea
Kólo  a  Lalaway  \textbf{kama}\textbf{y?}  
\z

kɔ-lɔ   a  Lalawaj    \textbf{kamaj}  



2S+I\textsc{PFV}{}-go  at  Lalaway    why  



‘Why are you going to Lalaway?’


\ea
Kəlala  na  \textbf{ memey}\textbf{?}
\z

kə-l=ala  na  \textbf{mɛmɛj}



2S.PFV  PSP  how



‘Why  did you come?’


\ea
Memey?
\z

mɛmɛj



how



‘Explain?’ (what do you mean?, lit. how?)


\ea
Nehe  na  \textbf{malma}\textbf{y?}
\z

nɛhɛ   na   \textbf{malmaj}



DEM  PSP  what



‘What is this here?’


\ea
Malmay?
\z

malmaj



what is it



‘What is it?’


\ea
Dala  \textbf{m}ə\textbf{t}ə\textbf{m}\textbf{e?}
\z

dala    \textbf{mɪtɪmɛ}



money  how much



‘How much money [is that]?’


\ea
Məlama  ango  na  \textbf{wel}\textbf{ey?}
\z

məlama     aŋg\textsuperscript{w}ɔ     na   \textbf{wɛlɛj}



brother    =2S.POSS  PSP  which



‘Which (one among these) is your brother?’ (lit. your brother [is] which one?)



Cicada S. 26


\ea
Albaya  ahay  \textbf{weley  }təh  anan  dəray  na  abay.
\z

albaja   =ahaj   \textbf{wɛlɛj}  təx     an=aŋ         dəraj   na  abaj



youth    =Pl    which   \textsc{ID}put   DAT=3S.IO   head   PSP   \textsc{EXT} \textsc{NEG}



‘No one could lift it.’ (lit. whichever young man put his head [to the tree], there was none)


In an emphatic question, a reduced interrogative pronoun both commences and finishes the clause (ex. 124{}-127). The interrogative pronouns \textit{waj} ‘who,’ \textit{malmaj} ‘what is this,’ \textit{memej} ‘why,’ and \textit{almaj} ‘what’ are reduced, (without a change in meaning), to \textit{wa}\textit{ }(ex. 124), \textit{malma}\textit{ }(ex. 125), \textit{meme}\textit{ }(ex. 126), and \textit{alma}\textit{ }(ex. 127), respectively. These reduced forms occur at the beginning of an emphatic question. At the end of the clause, some of these same pronouns are reduced in a different manner. The interrogative pronoun \textit{memej} becomes \textit{mej} (ex. 126) and \textit{almaj} becomes \textit{maj}\textit{ }(ex. 127).

\ea
\textbf{Wa}  andaɗay  \textbf{wa}\textbf{y?}
\z

\textbf{wa}    a-ndaɗ-aj   \textbf{waj}



who    3S-love-CL  who



‘No one loves him.’ (lit. who loves him?)


\ea
\textbf{Malma}  awəlok\textsuperscript{  }\textbf{ma}\textbf{y?}
\z

\textbf{malma}   a-wəl=ɔk\textsuperscript{w}   \textbf{maj}



what  3S-hurt=2S.IO  what



‘What is bothering (hurting) you?’


\ea
Meme  ege  mey?
\z

\textbf{mɛmɛ}   ɛ{}-g-ɛ     \textbf{mɛj}



how    3S-do-CL  how



‘What is going on here?’/ ‘What are you doing?’ (when something is wrong) (lit. how is it doing?)



Snake S. 7


\ea
\textbf{Alma}  amədəvala  okfom  na  \textbf{ma}\textbf{y?}
\z

\textbf{alma}  amə-dəv=ala    ɔk\textsuperscript{w}fɔm  na  \textbf{maj}



what  DEP-trip=to    mouse  PSP  what



‘What was it that made that mouse fall?’


\subsection{Unspecified pronouns}
\hypertarget{RefHeading1210921525720847}{}
A few pronouns refer to unspecified referents.  \textit{Me}\textit{ɬ}\textit{eneŋ} is a negative indefinite ‘no one’ (ex. 128) and must occur in a clause that is negated (see \sectref{sec:11.3}). \textit{Mana} is purposefully indefinite, referring to a person ‘who shall remain nameless’ (ex. 129).  \textit{ Eneŋ} ‘another’ (ex. 130) is an indefinite determiner, used to introduce new participants or things not previously mentioned.


\ea
Nəmənjar  \textbf{meslenen  }bay.
\z

nə-mənzar     \textbf{mɛɬɛnɛŋ}    baj



1S.PFV-see    no one    \textsc{NEG}



‘I didn’t see anyone.’


\ea
Anjaka  aməɓezlata  azla  \textbf{mana}  \textbf{mana}  \textbf{mana}\textbf{.}  
\z

a-nz     =aka  amə-ɓɛɮ    =ata   aɮa    \textbf{mana}    \textbf{mana}    \textbf{mana}  



3S-left=on  DEP-count  =3P.IO  emphasis  so and so  so and so  so and so



‘He started telling their names: so and so, and so and so, and so on.’  


\ea{}
{}[Nafat  \textbf{enen }]\textbf{  }aba
\z

{}[nafat  \textbf{ɛnɛŋ }]  aba



day    another  \textsc{EXT}



‘One day. . .’  (a usual way to start a story)


\section{Demonstratives and demonstrationals}
\hypertarget{RefHeading1210941525720847}{}
Moloko has three main types of demonstratives: nominal demonstratives (\sectref{sec:18}) which point to a person or object and modify a noun in a noun phrase, local adverbial demonstratives (\sectref{sec:19}) which point to a place and modify a noun in a noun phrase, and manner adverbal demonstratives (\sectref{sec:20}), which point to an action and modify a verb.\footnote{\citet{Dixon2003} describes three types of demonstratives: nominal, local adverbial, and verbal. Verbal demonstratives do not occur in Moloko. Dixon considers manner adverbial demonstratives to be a subtype of nominal demonstratives. } 

%%please move \begin{table} just above \begin{tabular
\begin{table}
\caption{shows all of the demonstratives in Moloko. All demonstratives have the same form for both singular and plural referents.  All are anaphoric in that the referent must be known from the preceding context. Place/time adverbs are also shown for comparison. The proximal demonstratives are morphologically similar to the locational adverb \textit{ehe} ‘here/now’ (shown for comparison in \tabref{tab:20}.).}
\label{tab:20}
\end{table}

It can be seen that the near speaker and distant from speaker demonstratives are morphologically derived from the corresponding adverbs. Note that there are no non-visible demonstratives or place/time adverbs.

\begin{tabular}{lllll} & \textbf{Nominal demonstratives } & \textbf{Local adverbial demonstratives } & \textbf{Manner adverbial demonstratives} & \textbf{Place/time adverbs}\\
\lsptoprule
\textbf{Proximal (near speaker)} & \textit{ŋgɛhɛ / }\textit{nɪŋgɛhɛ / nɛŋgɛhɛ }\footnotemark{}

‘this’ & \textit{nɛhɛ}

‘here’ & \textit{ka nehe }

‘like this’ 

\textit{kiygehe}

‘this way’ & \textit{ehe}

‘here’

\textit{tʃɪtʃɪŋgɛhɛ}

‘now’\\
\textbf{Distal (away from speaker)} & \textit{ŋgɪndijɛ / ŋgɪndɪgɛ}\footnotemark{}

‘that’ & \textit{nɪndijɛ / nɛndɪgɛ}\footnotemark{}\textit{ }

‘there’ &  & \\
\textbf{Distant from speaker} &  & \textit{tɔh}\textit{\textsuperscript{w}}\textit{ɔ}\footnotemark{}

‘over there’ &  & \textit{tɔh}\textit{\textsuperscript{w}}\textit{ɔ}

‘over there’\\
\textbf{Anaphoric} &  & \textit{ndana}

‘that previously mentioned’ & \textit{ka ndana }

‘like what was described’

\textit{kiyga}

‘like that’ & \\
\lspbottomrule
\end{tabular}
\addtocounter{footnote}{-4}
\stepcounter{footnote}\footnotetext{ The demonstrative \textit{ŋgɛhɛ} is a contraction of \textit{nɪŋgɛhɛ}.}
\stepcounter{footnote}\footnotetext{ This demonstrative is pronounced either \textit{nɪŋgɪndijɛ }or \textit{nɪŋgɪndɪgɛ} by people from different regions.}
\stepcounter{footnote}\footnotetext{ Likewise, dialect differences account for the different pronunciations. }
\stepcounter{footnote}\footnotetext{ In a genitive or possessive construction. }

\begin{itemize}
\item \begin{styleTabletitle}
Demonstratives in Moloko
\end{styleTabletitle}\end{itemize}
\subsection{Nominal demonstratives}
\hypertarget{RefHeading1210961525720847}{}
Nominal demonstratives have a referent that is a person or object. They modify a noun within a noun phrase to specify or point out the referent. Moloko has two nominal demonstratives: proximal (near the speaker) and distal (away from the speaker). There is no nominal demonstrative to indicate a referent that is far away from the speaker. 

In the examples below,\footnote{The first line in each example is the orthographic form. The second is the phonetic form (slow speech) with morpheme breaks.} the demonstrative is bolded and the noun phrase is marked by square brackets. In ex. 140 from section 191, the demonstrative is head of the noun phrase, suggesting that it can act as a demonstrative pronoun. 


\ea
Náskwom  [zana\textbf{  ngehe.}]
\z

ná-sk\textsuperscript{w}ɔm[zana  \textbf{ŋgɛhɛ} ]



1S.IFV-buy  cloth  DEM



‘I will buy this particular cloth here’ (pointing to or holding a particular cloth among others).


\ea
Asaw  [awak\textbf{  ngəndəye} .]
\z

a-s=aw    [awak  \textbf{ŋgɪndijɛ} ]



3S-please=1S.IO  goat  DEM



‘I want that particular goat there’ (pointing to a particular goat among others).


\ea
{}[Babəza  ahay  \textbf{n}\textbf{gəndəye }]  anga  əwla  ahay.
\z

{}[babəza  =ahaj   \textbf{ŋgɪndijɛ}   ]  aŋga  =uwla     =ahaj



children  =Pl  DEM     \textsc{POSS}  =1S.POSS  =Pl  



‘These particular children here [are] belonging to me.’ 


Besides their use to point out specific referents, the nominal demonstratives can also be used anaphorically in discourse.\footnote{Moloko has one specifically anaphoric demonstrative used in discourse (\textit{ndana}, \sectref{sec:11192}). Also, two other particles function in cohesion as discourse anaphoric referent markers. They are \textit{ga} (Chapter 5.3) and \textit{na} (Chapter 12).} The distal nominal demonstrative \textit{ŋgɪndijɛ} in line S. 14 of the Cicada story (ex. 134) identifies the tree as being that particular previously mentioned one that the men wanted the chief to have. 


S. 14


\ea
{}[Ngəvəray  \textbf{ngəndəye}\textbf{ }]  ágasaka  ka  mahay  ango  aka.
\z

{}[ŋgəvɛraj\textbf{ŋgɪndijɛ }]  á-gas=aka    ka    mahaj  =aŋg\textsuperscript{w}ɔ    aka



spp. of tree  DEM              3S.IFV-catch=on    at    door  =2S.POSS  on



‘That particular (previously mentioned) tree would be pleasing by your door.’ 


At the conclusion of the Millet story, nominal demonstratives are used anaphorically to mark two different referents – the suffering brought to the Moloko people and the young girl whose disobediance resulted in the suffering. Both are shown in ex. 135. The beginning of the millet story describes the blessing – that Moloko people could make an entire meal for a whole family from one grain of millet. The blessing occurred because the millet would multiply during its grinding. The story describes how a young, newly-married non-Moloko girl hears how to handle the millet yet disobeys the rules on how to handle it. As a result, the disobedient girl was killed by the millet. The story tells how God was offended by her act and withdrew his blessing from the Moloko people such that millet would not multiply any more and the Moloko had to work hard to even get enough food to feed their families. The suffering that the Moloko people experienced as a result of the withdrawal of God’s blessing is described in lines 33-37 but it is not named as such until line S. 38. In that line, the particular suffering of the Moloko people that was brought on by the girl is marked by the proximal nominal demonstrative \textit{avija nɛŋgɛhɛ} ‘this particular previously mentioned suffering. Also, the young woman who, by her disobedience, brought suffering to the entire Moloko population is marked in lines 33 and 38 by the distal nominal demonstrative. Line 33 contains \textit{war dalaj na} \textit{amɛtʃɛŋ ɬəmaj baj }\textit{ngəndəye} ‘the young woman, the previously mentioned disobedient one’ and line 38 contains \textit{war dalaj ŋgɪndijɛ} ‘that previously mentioned young woman.’


S. 33


\ea
Məloko  ahay  tawəy,  Hərmbəlom  ága  ɓərav  va
\z

\textit{Mʊlɔk}\textit{\textsuperscript{w}}\textit{ɔ  =ahaj  t}\textit{a}\textit{wij  Hʊrmbʊlɔm     á-g-a      ɓərav   =va  }



Moloko     =Pl       3P+say  God        3S.IFV-do   heart    =\textsc{PRF}     



‘The Molokos say, God got angry’ (lit. God did heart)



\textit{kəwaya  war  dalay  na,  amecen  sləmay  bay  }\textbf{\textit{ngəndəye}}\textit{.}



\textit{kuwaja        war    dalaj     na}   \textit{amɛ-tʃɛŋ      ɬəmaj  baj     }\textbf{\textit{ŋgɪndijɛ}}



because of  child    girl    PSP  DEP-hear   ear      \textsc{NEG}  DEM



‘because of that (particular previously mentioned) girl, that one that was disobedient.’



S. 34



\textit{Waya  ndana  Hərmbəlom  ázata  aka  barka  ahan  va.}



\textit{waja   ndana  Hʊrmbʊlɔm   á-z    =ata      =aka   barka     =ahaŋ    =va }



because   DEM   God             3S.IFV-take  =3P.IO  =on   blessing  =3S.POSS  =\textsc{PRF}



‘Because of that, God had taken back his blessing from them.’



S. 35



\textit{Cəcəngehe  na,  war  elé  háy  bəlen  na,  ásak  asabaj.}



\textit{tʃɪtʃɪŋgɛhɛ  na,  war  ɛlɛ  haj  bɪlɛŋ  na  á-sak                     asa-baj}



now               PSP  child   eye   millet   one   PSP  3S.IFV-multiply    again-\textsc{NEG}



‘And now, one grain of millet, it doesn’t multiply anymore.’



S. 36



\textit{Talay  war  elé  háy  bəlen  kə  ver  aka  na,  ásak  asabay.}



\textit{talaj     war  ɛlɛ  haj  bɪlɛŋ  kə  vɛr  aka  na  á-sak       asa-baj}



\textsc{ID}put  child   eye   millet    one    on    stone    on    PSP  3S.IFV-multiply  again-\textsc{NEG}



‘[If] one puts one grain of millet on the grinding stone, it doesn’t multiply anymore.’



S. 37



\textit{S}\textit{əy  kádəya  gobay.}



\textit{sij}\textit{     ká-d    =ija  g}\textit{\textsuperscript{w}}\textit{ɔbaj}



only    2S.IFV-prepare  =\textsc{PLU}   a lot



‘You must put on a lot.’



S. 38



\textit{Ka  nehe  tawəy,  metesle  anga  war  dalay  }\textbf{\textit{ngəndəye}}\textit{  }



\textit{ka  nɛhɛ  tawij  mɛ-tɛɬ-ɛ      aŋga  war    dalaj  }\textbf{\textit{ŋgɪndijɛ}}\textit{  }



like  DEM   3P+say  \textsc{NOM}{}-curse-CL   \textsc{POSS}   child  girl       DEM      



‘It is like this they say, “The curse [is] belonging to that particular (previously mentioned) young woman’



\textit{amazata  aka  ala   }[\textit{av}\textit{əy}\textit{a}\textit{  }\textbf{\textit{nengehe}} ]\textit{ }\textbf{\textit{ }}\textit{ana  məze  ahay  na.}



\textit{ama-z  =ata      =aka  =ala      avija    }\textbf{\textit{nɛŋgɛhɛ}}\textit{  ana    mɪʒɛ  =ahaj   na}



DEP-take  =3P.IO   =on     =to  suffering  DEM      DAT  person    =Pl  PSP



‘that brought this  (particular previously mentioned)  suffering onto the people.”’  


\subsection{Local adverbial demonstratives}
\hypertarget{RefHeading1210981525720847}{}
Local adverbial demonstratives point to a referent that is a place. They commonly occur with a noun but can also occur as the only element in a noun phrase. Moloko has three local adverbial demonstratives: proximal (near the speaker), distal (away from the speaker) (\sectref{sec:1}), and an anaphoric demonstrative used only in discourse (\sectref{sec:2}). There is no demonstrative to indicate a place far away from the speaker. However the adverb \textit{tɔh}\textit{\textsuperscript{w}}\textit{ɔ} ‘over there’ is used within noun phrases where such a place needs to be indicated. 

\paragraph[Proximal and distal local adverbial demonstratives]{Proximal and distal local adverbial demonstratives}

Proximal and distal local adverbial demonstratives refer to a physical place (here or there). In a noun phrase, the position for the local adverbial demonstrative is different than for a nominal demonstrative. The local adverbial demonstrative occurs as a separate final element (ex. 136 - 139).\footnote{Note that nominal demonstratives do not occur after the adjectiviser, \sectref{sec:5.1.}} In the examples below, the demonstrative is bolded and the noun phrase is marked by square brackets.


\ea
{}[Ɗaf  \textbf{nehe} ]  acar.
\z

{}[ɗaf  \textbf{nɛhɛ} ]  a-tsar



loaf    DEM  3S-taste good



‘This loaf here (in this place) tastes good.’


\ea
Nazalay  [awak  ahay  \textbf{nəndəye} ]  a  kosoko  ava.
\z

na-z-alaj    [awak  =ahaj  \textbf{nɪndijɛ} ]  \textbf{  }a  kɔsɔk\textsuperscript{w}ɔ  ava



1S-carry-away  goat  =Pl  DEM    in  market  in



‘I take the goats there (in that place) to the market.’


\ea
{}[War  elé  hay  bəlen  ga  \textbf{nəndəye} ] [nok\textsuperscript{  }amɛzəɗe  na, ]
\z

{}[war    ɛlɛ    haj  bɪlɛŋ  ga   \textbf{n}\textbf{ɪ}\textbf{ndij}\textbf{ɛ} ]  [nɔk\textsuperscript{w}   amɛ{}-zɪɗ{}-ɛ     na ]



child   eye      millet  one      ADJ   DEM    2S    DEP-take-CL   PSP



‘That one grain there (highlighted\footnote{See below for the discourse function of local adverbial demonstratives.}), the one that you have taken’



\textit{káhaya  na  kə  v}\textit{e}\textit{r  aka.}



\textit{ká-h  =aja     na       kə       v}\textit{ɛ}\textit{r        aka}



2S.IFV-grind=\textsc{PLU}  3S.DO  on   grinding stone  on



‘grind it on the grinding stone.’


\ea
Səwat  na, [təta  a  məsəyon  na  ava  \textbf{nəndəye}  na,]  pester  áhata,  “Ey!  Ele  nehe  na,  kógom  bay!”       
\z

\textit{suwat  na   }[\textit{təta  a  mɪsijɔŋ   na   ava  }\textbf{\textit{nɪndijɛ}}\textit{  na }]\textit{     }



\textsc{ID}disperse  PSP  3P      in   mission  PSP  in  DEM  PSP  



‘As the people go home from church, the Pastor tells them,’ (lit. disperse, they in the mission there),’ 


The local adverbial demonstrative can be the head of a noun phrase. In ex. 140 the demonstrative is modified by the plural. 

\ea
Nde  [\textbf{nehe  }ahay   na ]  sla  ango \textsuperscript{ }ahay  ɗaw?
\z

ndɛ     [\textbf{nɛhɛ}  =ahaj   na ]     ɬa         =aŋg\textsuperscript{w}ɔ     =ahaj  ɗaw



so    DEM  =Pl  PSP  cow  =2S.POSS  =Pl\textit{  }QUEST



‘So, these [cows] here (in this place), are they your cows?’


For locations far away from the speaker, the locational adverb \textit{tɔh}\textit{\textsuperscript{w}}\textit{ɔ }is used in a possessive or genitive construction with the noun it modifies, (\textit{aŋga tɔh}\textit{\textsuperscript{w}}\textit{ɔ}, ex. 141 see \sectref{sec:45}; or  \textit{a tɔh}\textit{\textsuperscript{w}}\textit{ɔ}, ex. 142, see \sectref{sec:42}).

\ea
{}[Hay  əwla  \textbf{anga}\textbf{  toh}\textbf{o  }na,]  eleməzləɓe  tanday  tozom  na.
\z

{}[haj    =uwla     \textbf{aŋga}\textbf{  tɔh}\textsuperscript{w}\textbf{ɔ}  na ]  ɛlɛmɪɮɪɓɛ  ta-ndaj    tɔ-zɔm  na



house  =1S.POSS  \textsc{POSS}  DEM  PSP  termites    3P-PROG  3P-eat  3S.DO



‘My house way over there (pointing to a particular house among others in the distance), termites are eating it.’ (lit. my house, the one that belongs to over there, termites are eating it)


\ea
{}[Awak  ahay  \textbf{a}  \textbf{toho} ]  anga  əwla.
\z

{}[awak  =ahaj  \textbf{a}  \textbf{tɔh}\textsuperscript{w}\textbf{ɔ} ]  aŋga  =uwla



goat    =Pl  GEN  DEM  \textsc{POSS}  =1S.POSS



‘The goats over there (in that place) belong to me.’(lit. the goats over there [are] belonging to me)


The function of local adverbial demonstratives to point out a place can be seen in the Cicada text (ex. 143 - 144, found in its entirety in \sectref{sec:1.6}). In the story, a beautiful tree is found in the bush and the chief decides that he wants to have it moved to his yard. The tree is first mentioned as being \textit{a ləhe} ‘in the bush’ in line S. 5 (ex. 143). The tree is mentioned again in line S. 12 marked by the local adverbial demonstrative \textit{nɪndijɛ}\textit{ }‘that one there’ (ex. 144). 


S. 5


\ea
Təlo  tənjakay  ngəvəray  malan  ga  a  ləhe.
\z

\textit{tə-lɔ            tə-nzak-aj           ŋgəvəraj    malaŋ     ga   a  lɪhɛ}



3P.PFV-go   3P.PFV-find-CL  spp. of tree  large    ADJ  at  bush



‘They went and found a large tree (of a particular species) in the bush.’



S. 12


\ea
Təlo  tamənjar  na  ala  [mama  ngəvəray  \textbf{nənd}\textbf{əye} ] 
\z

tə-lɔ               tà-mənzar       na   =ala     [mama  ŋgəvɛraj    \textbf{nɪndijɛ} ] 



3P .PFV- go    3P.HOR-see  3S.DO  =to  mother  spp. of tree  DEM



‘They went to see the mother tree there.’


Sometimes local adverbial demonstratives have a highlighting function for new information in a narrative, drawing attention to their referent.\footnote{\citet{Dixon2003} mentions that demonstratives can function to introduce new information. Note that in Moloko, all new information need not be marked with a demonstrative. } In the ‘Cows in the Field’ story (not illustrated in its entirety in this work), \textit{ɗ}\textit{ɛrijwɛl nɛndɪgɛ} ‘this paper here’ (ex. 145) was not with the speaker when he told the story; neither was it previously mentioned in the discourse. According to the discourse, the paper should have helped to bring justice to the men whose cotton was destroyed, but didn’t. Its marking with a demonstrative therefore has the function to highlight the paper at that moment of the eventline. 

\ea
Alala  na,  ta  anaw  [ɗerəywel  \textbf{nend}\textbf{ə}\textbf{g}\textbf{e.} ]
\z

a-l=ala  na      ta   an  =aw   [ɗɛrijwɛl   \textbf{nɛndɪgɛ} ]



3S-go=to  PSP   3P  DAT   =1S.IO  paper        DEM 



‘Later, they [gave] me this here paper.’ 


In the Values exhortation (ex. 146, shown in its entirety in \sectref{sec:1.7}) the local adverbial demonstrative \textit{nɛhɛ}  ‘this here’ is used to draw attention to new information. In the exhortation, the phrase \textit{ɛlɛ nɛhɛ}  ‘these things here’ introduces information not previously mentioned in the discourse.\footnote{Note that the local adverbial demonstrative \textit{nɪndijɛ} ‘here’ in the same example functions to simply point out a place in the phrase \textit{təta a məsjɔŋ na ava nɪndijɛ} ‘the ones in church there’. Also, compare the function of the proximal local adverbial demonstrative \textit{nɛhɛ}  with that of the proximal nominal demonstrative \textit{nɪŋgɛhɛ}  in the same example. The nominal demonstrative in the phrase\textit{ ɛlɛ =ahaj amɪgijɛ baj nɪŋgɛhɛ} ‘these particular things that one shouldn’t do’ points out particular things which are previously mentioned \citep{Section1118}. } This information – the things that people are not supposed to do – is the main topic of the entire discourse. The demonstrative marking functions to notify the reader of the importance of the new information. Note that the demonstrative is not functioning cataphorically here. It is the narrator who specifies the things that people are not supposed to do in the discourse which follows (S. 4-5 in ex. 146), not the pastor in his speech. 


S. 3


\ea
Səwat  na,  [təta  a  məsyon  na  ava  nəndəye  na,]  Pester  ahata,      
\z

suwat na   [təta   a   məsjɔŋ   na   ava \textbf{  }nɪndijɛ  na ]   Pɛstɛr    a-h  =ata      



\textsc{ID}disperse  PSP  3P  in  mission  PSP  in  DEM  PSP  pastor  3S-told  =3P.IO



‘As they disperse, the ones in church there, the Pastor said, 



\textit{“Ey,  }[\textit{ele  }\textbf{\textit{nehe  }}na]\textit{  kogom  bay!”}



\textit{ɛj    }[\textit{ɛlɛ}\textbf{\textit{   nɛhɛ}}\textit{   na }]\textit{   kɔ-g}\textit{\textsuperscript{w}}\textit{{}-ɔm   baj}



hey  thing  DEM  PSP  2-do-2P    \textsc{NEG}



‘“Hey! These here things, don’t do them!”’



S. 4



\textit{Yawa,  war  dalay  ga ándaway  mama  ahan.}


\textit{jawa   war   dalaj  ga  á-ndaw-aj   mama   =ahaŋ}


well    child  female  ADJ  3S.IFV-insult{}-CL  mother  =3S.POSS  



‘Well, the girls insult their mothers.’ 



S. 5



\textit{War  zar  ga  ándaway  baba  ahan.}


\textit{war     zar  ga  á-ndaw-aj   baba   =ahaŋ}


child  male  ADJ  3S.IFV-insult{}-CL  father  =3S.POSS



‘[And] the boys insult their fathers.’ 



S. 6



\textit{Yo,} [\textit{ele  ahay  aməgəye  bay  nəngehe}\textbf{\textit{  }}\textit{pat, }]\textit{  }\textit{ }


\textit{jɔ       [ɛlɛ  =ahaj  amɪ-g-ijɛ   baj  }\textit{nɪŋgɛhɛ}\textit{    pat ]   }


well    thing  =Pl  DEP-go-CL     \textsc{NEG}  DEM     all  



\textit{tahata  na  va  kə  }\textit{dəftere  }\textit{aka.  }



\textit{ta-h=ata        na  =va   kə   dɪftɛrɛ  aka.  }



3P-tell-3P.IO    3S.DO  =\textsc{PRF}   on  book  on



‘Well, all these particular things that we are not supposed to do, they have already told them in the Bible.’ 


The highlighting function of local adverbial demonstratives does not have to be associated with the introduction of new information. For example, in the Disobedient Girl story (ex. 147, shown in its entirety in \sectref{sec:1.5}), the one grain of millet is introduced in the first line of the husband’s speech to his wife (line S. 13 in ex. 147). The next mention of the one grain of millet is in the next line of his speech is where the grain is marked by the local adverbial demonstrative in \textit{war }\textit{ɛ}\textit{l}\textit{ɛ}\textit{ haj b}\textit{ɪ}\textit{l}\textit{ɛ}\textit{ŋ ga}\textit{ n}\textit{ɪ}\textit{ndij}\textit{ɛ} ‘that one grain there.’ In this case, \textit{n}\textit{ɪ}\textit{ndij}\textit{ɛ} ‘that there’ does not mark new information; the one grain of millet has already been mentioned in the previous sentence. However, the highlighting function of the demonstrative identifies the one grain of millet as being important in the developing story. It is the one grain of millet which becomes transformed and multiplied and suffocates the disobedient girl by the end of the story. 


S. 13


\ea
Asa  asok\textsuperscript{  }aməhaya  na,  kázaɗ   war  elé  hay  bəlen.
\z

asa  à-s    =ɔk\textsuperscript{w}  amə-h  =aja    na   ká-zaɗ         war     ɛlɛ      haj       bɪlɛŋ



if     3S.IFV-please  =2S.IO   DEP-grind=\textsc{PLU}   PSP  2S.IFV-take  child  eye  millet  one



‘If you want to grind, you take only one grain.’



{}[\textit{War  }\textit{e}\textit{l}\textit{é  }\textit{hay  b}\textit{ə}\textit{l}\textit{en  }\textit{ga  }\textbf{\textit{nəndəye}} ] [\textit{n}\textit{o}\textit{k}\textit{\textsuperscript{  }}\textit{am}\textit{ɛ}\textit{z}\textit{ə}\textit{ɗ}\textit{e  na, }]



{}[\textit{war    }\textit{ɛ}\textit{l}\textit{ɛ}\textit{    haj  b}\textit{ɪ}\textit{l}\textit{ɛ}\textit{ŋ  ga   }\textbf{\textit{n}}\textbf{\textit{ɪ}}\textbf{\textit{ndij}}\textbf{\textit{ɛ}}\textit{     n}\textit{ɔ}\textit{k}\textit{\textsuperscript{w}}\textit{   am}\textit{ɛ}\textit{{}-z}\textit{ɪɗ}\textit{{}-}\textit{ɛ}\textit{     na }]



child   eye      millet  one      ADJ   DEM    2S    DEP-take-CL   PSP



‘That (highlighted) one grain, the one that you have taken,’



\textit{káhaya  na  kə  v}\textit{e}\textit{r  aka.}



\textit{ká-h    =aja     na       kə       v}\textit{ɛ}\textit{r        aka}



2S.IFV-grind=\textsc{PLU}  3S.DO  on   grinding stone  on



‘grind it on the grinding stone.’


The distal non local demonstrative is employed in a common discourse idiom  – \textit{a ɬam nɛndijɛ ava}\textit{ }‘at that time.’ The idiom notifies the reader of an important pivotal moment in a story. Ex. 148 is from the ‘Cows in the Field’ story (not illustrated in its entirety in this work). The narrative concerns dealings with the owners of a herd of cows that had destroyed someone’s field of cotton. \textit{A ɬam nɛndijɛ ava} marks the transition point in the way that the speaker dealt with the cows. 

\ea
A  [slam  \textbf{nendəye} ]  ava  na,  nawəy,
\z

a   [ɬam   \textbf{nɛndijɛ} ]   ava   na    nawij



to  place      DEM      in   PSP   1S-saying  



\textit{“Sla  ahay  na,  m}\textit{ə}\textit{mokok  ta  bay,  golok}\textit{\textsuperscript{  }}\textit{ta  a  K}\textit{ə}\textit{ɗ}\textit{ə}\textit{mbor,   }



\textit{ɬa   =ahaj   na    m}\textit{ʊ}\textit{{}-mɔk}\textit{\textsuperscript{w}}\textit{{}-ɔk}\textit{\textsuperscript{w}}\textit{    ta  baj  g}\textsuperscript{w}\textit{ɔl-ɔk}\textit{\textsuperscript{w}}\textit{  ta  a  K}\textit{ʊ}\textit{ɗ}\textit{ʊ}\textit{mbɔr   }



cow   =Pl  PSP  1\textsc{Pin}.IFV-leave-2\textsc{Pin}   3P.DO   \textsc{NEG}   drive \textsc{IMP}{}-1\textsc{Pin}   at   Tokombere



\textit{ɗ}\textit{e}\textit{ɗ}\textit{en  bay  na  memey?”}



\textit{ɗ}\textit{ɛ}\textit{ɗ}\textit{ɛŋ   baj     na      mɛmɛj}



truth  \textsc{NEG}   PSP   how    



‘At that moment, I said, “These cows, let’s not leave them at all, let’s drive them to Tokembere, if it’s not true, then how?”’


\paragraph[Anaphoric demonstrative]{Anaphoric demonstrative}

The anaphoric demonstrative \textit{ndana} ‘that previously mentioned’ is used only in discourse for anaphoric marking of a participant that is important to the message of the discourse. In the Disobedient Girl story, \textit{war dalaj ndana}\textit{ ‘}that previously mentioned young woman’ occurs in the introduction of the major characters in the story (S.11, ex. 149). The three major characters in the story are the husband, the woman, and the grain of millet. The woman will, by her disobedience, bring a curse on the Moloko people. 


Disobedient Girl S. 11



\ea
Azla  na,  [war  dalay  \textbf{ndana }]  cezlere  ga.
\z

aɮa        na  [war   dalaj   \textbf{ndana }]  tʃɛɮɛrɛ         ga



now      PSP  child      female    DEM  disobedience   ADJ



‘Now, the above-mentioned young girl was disobedient.’


Likewise, in the Cicada story (ex. 150 - 152, found in its entirety in \sectref{sec:1.6}), the demonstrative \textit{ndana} ‘previously mentioned’ is used anaphorically to mark the young men and the tree, both of which are key elements in the story. In ex. 151 (from S. 6), \textit{albaja =ahaj ndana} ‘those previously mentioned young men’and ex. 152 (from S. 9)  \textit{ŋgəvɛraj ndana} ‘that tree just mentioned,’ \textit{ndana} is used to refer back to the man introduced in S3 and the tree introduced in S5. 


Cicada S. 3      S. 5


\ea
Albaya  ahay  aba.…  Təlo  tənjakay  ngəvəray  malan  ga  a  ləhe.
\z

albaja  =ahaj  aba.…  



young man   =Pl  \textsc{EXT}  



‘There were some young men.. . .



\textit{tə-lɔ            tə-nzak-aj           ŋgəvəraj    malaŋ     ga   a  lɪhɛ}



3P.PFV-go   3P.PFV-find-CL  spp. of tree  large    ADJ  at  bush



‘They went and found a large tree (of a particular species) in the bush.’



S. 6


\ea
{}[Albaya  ahay  \textbf{ndana }]  kəlen  təngalala  ma  ana  bahay.
\z

{}[albaja   =ahaj  \textbf{ndana} ]  kɪlɛŋ  tə-ŋgala      =ala   ma  ana   bahaj



young man    =Pl         DEM  then  3P.PFV-come back  =to  word  DAT  chief



‘Those above-mentioned young men then took the word (response) to the chief.’



S. 9


\ea
Káazaɗom  anaw  ala  [ngəvəray  \textbf{ndana }]  ka  mahay  əwla  aka.
\z

káá-zaɗ{}-ɔm    an   =aw   =ala  [ŋgəvəraj  \textbf{ndana }]  ka  mahaj  =uwla       aka



2P.POT-take-2P  DAT=1S.IO =to    spp. of tree  DEM  on       door         =1S.POSS  on



‘[the chief desired to have that tree transplanted at his gate. He commissioned his people to do it, saying:] You will bring the above-mentioned tree to my door for me.’


\textit{Ndana} ‘the above-mentioned’ can also replace an entire thought. Ex. 153 is from line S. 34 of the Millet story. In this sentence, \textit{ndana}  ‘the above-mentioned’ is head of the noun phrase and refers to the entire preceding story of the disobedience and death of the girl. 

\ea
Waya  \textbf{ndana}  Hərmbəlom  ázata  aka  barka  ahan  va.
\z

\textit{waja   ndana  Hʊrmbʊlɔm   á-z    =ata      =aka   barka     =ahaŋ    =va }



because   DEM   God             3S.IFV-take  =3P.IO  =on   blessing  =3S.POSS  =\textsc{PRF}



‘Because of the above-mentioned, God had taken back his blessing from them.’


\subsection{Manner adverbial demonstratives}
\hypertarget{RefHeading1211001525720847}{}
Manner adverbial demonstratives have been described by \citet{Dixon2003} to function as non-inflecting modifiers to verbs. There are two types in Moloko, depending on how they are derived.{ }\footnote{\citet{Dixon2003} notes that manner adverbial demonstratives are morphologically derived from nominal demonstratives. In Moloko they are derived from the nominal demonstrative, an adverb, or the adjectiviser. } 

The first type in Moloko is derived from the demonstrative by the addition of \textit{ka} ‘like.’ The adverbial demonstrative \textit{ka nehe} ‘like this’ (ex. 154) is used when the speaker indicates with hand or body movements how a particular action is carried out. It is derived from the proximal nominal demonstrative \textit{nehe} ‘this here’ (see \sectref{sec:191}).


\ea
Enje  ele  ahan  dəren  \textbf{ka  n}\textbf{ehe.}
\z

à-ndʒ-ɛ  ɛlɛ  =ahaŋ    dɪrɛŋ  \textbf{ka  nɛhɛ}



3S.PFV-leave-CL  thing  =3S.POSS  far  like   this



‘He went (lit. took his things away) far away like this.’  


The adverbial demonstrative \textit{ka ndana} ‘like what was just said’ (ex. 155) is derived from the anaphoric demonstrative \textit{ndana} ‘the above-mentioned’ (see \sectref{sec:192}). \textit{Ka ndana} can be negated; compare ex. 155 and 156.

\ea
Nəvəye  ngehe  na,  ngama  aməgəye  jerne  nə  eteme.      Nəɗəgalay  \textbf{ka  ndana}\textbf{.}
\z

nɪvijɛ  ŋgɛhɛ    na    ŋgama   amɪ-g-ijɛ  dʒɛrnɛ   nə    ɛtɛmɛ  nə-ɗəgal-aj  \textbf{ka  ndana}



season  DEM   PSP    better    DEP-do-CL        garden  with      onion  1S.IFV-think-CL  like  DEM



‘This season I think it is better to grow onions.’        ‘I think so too.’


\ea
Nəvəye  ngehe  na,  ngama  aməgəye  jerne  nə  eteme.        Nəɗəgalay  \textbf{ka  ndana}  bay.
\z

nɪvijɛ   ŋgɛhɛ  na   ŋgama    amɪ-g-ijɛ  dʒɛrnɛ  nə      ɛtɛmɛ  nə-ɗəgal-aj      \textbf{ka  ndana}   baj



season  DEM    PSP    better   DEP-do-CL  garden  with     onion  1S.IFV-think-CL  like DEM   \textsc{NEG}



‘This season I think it is better to grow onions.’        ‘I don’t think so.’


The second type of adverbial demonstrative in Moloko is derived from the adverb \textit{ehe} by the addition of the tag \textit{kijga} ‘like that’ (see \sectref{sec:76}). \textit{Kijgehe} ‘like this’ will be accompanied by gestures demonstrating the place where the action will occur (ex. 157 - 158). 

\ea
Adəkwalay  ana  Hərmbəlom  ton  \textbf{kəygehe.}    
\z

à-dʊk\textsuperscript{w}     =alaj  ana  Hʊrmbʊlɔm  tɔŋ    \textbf{kijgɛhɛ}    



3S.PFV-arrive   =away  DAT  God    \textsc{ID}touch     like.this    



‘It touched God like this [in the eye]. (lit. it arrived to God, touching [Him] like this)’


\ea
Lo  kəygehe.
\z

lo    kijgɛhɛ  



go[2S.IMP]  like.this



‘Go that way (along that pathway].’


\section{Numerals and quantifiers}
\hypertarget{RefHeading1211021525720847}{}
Three systems of numerals are found in Moloko:


\begin{itemize}
\item A base ten system for counting in isolation and for cardinal numbers (counting items excluding money, \sectref{sec:21}).
\item A base five system for counting money (\sectref{sec:22}).
\item A base ten system for ordinal numbers (ordering items with respect to one another, \sectref{sec:23}). 
\end{itemize}
\subsection{Cardinal numbers for items}
\hypertarget{RefHeading1211041525720847}{}
Cardinal numbers for counting items follow a base-ten system are shown in \tabref{tab:21}.. 

\begin{tabular}{llll}
\lsptoprule

1 & \textit{b}\textit{ɪ}\textit{l}\textit{ɛ}\textit{ŋ} & 21 & \textit{k}\textit{ɔ}\textit{k}\textit{\textsuperscript{w}}\textit{ʊ}\textit{r t}\textit{ʃ}\textit{ɛ}\textit{w h}\textit{ə}\textit{r b}\textit{ɪ}\textit{l}\textit{ɛ}\textit{ŋ}\\
2 & \textit{t}\textit{ʃ}\textit{ɛ}\textit{w} & 30 & \textit{k}\textit{ɔ}\textit{k}\textit{\textsuperscript{w}}\textit{ʊ}\textit{r m}\textit{á}\textit{k}\textit{\={a}}\textit{r}\\
3 & \textit{m}\textit{á}\textit{k}\textit{\={a}}\textit{r} & 100 & \textit{s}\textit{ə}\textit{k}\textit{\={a}}\textit{t}\\
4 & \textit{m}\textit{ə}\textit{f}\textit{\={a}}\textit{ɗ / uwf}\textit{\={a}}\textit{ɗ}\footnotemark{} & 101 & \textit{s}\textit{ə}\textit{k}\textit{\={a}}\textit{t n}\textit{ə}\textit{ b}\textit{ɪ}\textit{l}\textit{ɛ}\textit{ŋ}\\
5 & \textit{ɮ}\textit{ɔ}\textit{m} & 122 & \textit{s}\textit{ə}\textit{k}\textit{\={a}}\textit{t n}\textit{ə}\textit{ k}\textit{ɔ}\textit{k}\textit{\textsuperscript{w}}\textit{ʊ}\textit{r t}\textit{ʃ}\textit{ɛ}\textit{w h}\textit{ə}\textit{r t}\textit{ʃ}\textit{ɛ}\textit{w}\\
6 & \textit{m}\textit{ʊ}\textit{k}\textit{\textsuperscript{w}}\textit{ɔ} & 200 & \textit{s}\textit{ə}\textit{k}\textit{\={a}}\textit{t t}\textit{ʃ}\textit{ɛ}\textit{w}\\
7 & \textit{ʃ}\textit{ɪ}\textit{ʃ}\textit{ɪ}\textit{r}\textit{ɛ} & 300 & \textit{s}\textit{ə}\textit{k}\textit{\={a}}\textit{t m}\textit{á}\textit{k}\textit{\={a}}\textit{r}\\
8 & \textit{ɬ}\textit{\={a}}\textit{l}\textit{á}\textit{k}\textit{á}\textit{r} & 1,000 & \textit{d}\textit{ʊ}\textit{b}\textit{ɔ}\\
9 & \textit{h}\textit{ɔ}\textit{l}\textit{ɔ}\textit{mb}\textit{ɔ} & 1,001 & \textit{d}\textit{ʊ}\textit{b}\textit{ɔ}\textit{ n}\textit{ə}\textit{ b}\textit{ɪ}\textit{l}\textit{ɛ}\textit{ŋ}\\
10 & \textit{k}\textit{ʊ}\textit{r}\textit{ɔ} & 1,100 & \textit{d}\textit{ʊ}\textit{b}\textit{ɔ}\textit{ n}\textit{ə}\textit{ səkat}\\
11 & \textit{k}\textit{ʊ}\textit{r}\textit{ɔ}\textit{ h}\textit{ə}\textit{r b}\textit{ɪ}\textit{l}\textit{ɛ}\textit{ŋ} & 2,000 & \textit{d}\textit{ʊ}\textit{b}\textit{ɔ}\textit{ t}\textit{ʃ}\textit{ɛ}\textit{w}\\
12 & \textit{k}\textit{ʊ}\textit{r}\textit{ɔ}\textit{ h}\textit{ə}\textit{r t}\textit{ʃ}\textit{ɛ}\textit{w} & 3,000 & \textit{d}\textit{ʊ}\textit{b}\textit{ɔ}\textit{ m}\textit{á}\textit{k}\textit{\={a}}\textit{r}\\
13 & \textit{k}\textit{ʊ}\textit{r}\textit{ɔ}\textit{ h}\textit{ə}\textit{r m}\textit{á}\textit{k}\textit{\={a}}\textit{r} & 5,000 & \textit{d}\textit{ʊ}\textit{b}\textit{ɔ}\textit{ ɮ}\textit{ɔ}\textit{m}\\
14 & \textit{k}\textit{ʊ}\textit{r}\textit{ɔ}\textit{ h}\textit{ə}\textit{r m}\textit{ə}\textit{f}\textit{\={a}}\textit{ɗ} & 10,000 & \textit{d}\textit{ʊ}\textit{b}\textit{ɔ}\textit{ k}\textit{ʊ}\textit{r}\textit{ɔ}\\
15 & \textit{k}\textit{ʊ}\textit{r}\textit{ɔ}\textit{ h}\textit{ə}\textit{r ɮ}\textit{ɔ}\textit{m} & 10,001 & \textit{d}\textit{ʊ}\textit{b}\textit{ɔ}\textit{ k}\textit{ʊ}\textit{r}\textit{ɔ}\textit{ n}\textit{ə}\textit{ b}\textit{ɪ}\textit{l}\textit{ɛ}\textit{ŋ}\\
16 & \textit{k}\textit{ʊ}\textit{r}\textit{ɔ}\textit{ h}\textit{ə}\textit{r m}\textit{ʊ}\textit{k}\textit{\textsuperscript{w}}\textit{ɔ} & 100,000 & \textit{d}\textit{ʊ}\textit{b}\textit{ɔ}\textit{ d}\textit{ʊ}\textit{b}\textit{ɔ}\textit{ s}\textit{ə}\textit{k}\textit{\={a}}\textit{t}\\
17 & \textit{k}\textit{ʊ}\textit{r}\textit{ɔ}\textit{ h}\textit{ə}\textit{r }\textit{ʃ}\textit{ɪ}\textit{ʃ}\textit{ɪ}\textit{r}\textit{ɛ} & 100,001 & \textit{d}\textit{ʊ}\textit{b}\textit{ɔ}\textit{ d}\textit{ʊ}\textit{b}\textit{ɔ}\textit{ s}\textit{ə}\textit{k}\textit{\={a}}\textit{t n}\textit{ə}\textit{ b}\textit{ɪ}\textit{l}\textit{ɛ}\textit{ŋ}\\
18 & \textit{k}\textit{ʊ}\textit{r}\textit{ɔ}\textit{ h}\textit{ə}\textit{r ɬ}\textit{\={a}}\textit{l}\textit{á}\textit{k}\textit{á}\textit{r} & 1,000,000 & \textit{d}\textit{ʊ}\textit{b}\textit{ɔ}\textit{ d}\textit{ʊ}\textit{b}\textit{ɔ}\textit{ d}\textit{ʊ}\textit{b}\textit{ɔ}\\
19 & \textit{k}\textit{ʊ}\textit{r}\textit{ɔ}\textit{ h}\textit{ə}\textit{r h}\textit{ɔ}\textit{l}\textit{ɔ}\textit{mb}\textit{ɔ} & 1,000,001 & \textit{d}\textit{ʊ}\textit{b}\textit{ɔ}\textit{ d}\textit{ʊ}\textit{b}\textit{ɔ}\textit{ d}\textit{ʊ}\textit{b}\textit{ɔ}\textit{ n}\textit{ə}\textit{ b}\textit{ɪ}\textit{l}\textit{ɛ}\textit{ŋ}\\
20 & \textit{k}\textit{ɔ}\textit{k}\textit{\textsuperscript{w}}\textit{ʊ}\textit{r t}\textit{ʃ}\textit{ɛ}\textit{w} &  & \\
\lspbottomrule
\end{tabular}
\footnotetext{ This numeral is pronounced either \textit{məfaɗ} or \textit{uwfaɗ}  by people from different regions.}

\begin{itemize}
\item \begin{styleTabletitle}
 Cardinal numerals for counting items
\end{styleTabletitle}\end{itemize}

Numbers used for counting in isolation are identical to the system shown in \tabref{tab:21}.. When modifying a noun, the numerals follow the noun in a noun phrase (ex. 159{}-160). The consitiutent order of the noun phrase is discussed in \sectref{sec:5.1.}\footnote{The first line in each example is the orthographic form. The second is the phonetic form (slow speech) with morpheme breaks.}


\ea
Məze  ahay  dəbo  cew  tolo  aməmənjere  məkəɗe  balon.
\z

mɪʒɛ=ahaj  dʊbɔ   tʃɛw  tɔ-lɔ    amɪ-mɪnzɛr-ɛ  mɪ-kɪɗ-ɛ    balɔŋ



person  =Pl  1000  two  3P.PFV-go  DEP-see-CL  \textsc{NOM}{}-kill-CL  ball



‘Two thousand people went to see the football game.’


\ea
Nəmənjar  awak  ahay  kəro  a  kosoko  ava.
\z

nə-mənzar  awak  =ahaj  kʊrɔ  a  kɔsɔk\textsuperscript{w}ɔ  ava



1S.PFV-see  goat  =Pl  10  in  market  in



‘I saw ten goats at the market.’


The numerals can stand as head of a noun phrase in a clause (ex. 161{}-162) but the immediate context must give the referent. In Ex. 161, the response need only give the number. 

\ea
 Kənjakay  awak  mətəmey?  Nənjakay  bəlen.  
\z

kə-nzak-aj     awak   mɪtɪmɛj        nə-nzak-aj   bɪlɛŋ  



2S.PFV-find-CL  goat  how-many      1S-find-CL  one



‘How many goats did you find?’        ‘I found one.’


\ea
Babəza  əwla  ahay  na,  cew.
\z

babəza   =uwla    =ahaj  na  tʃɛw



children  =1S.POSS  =Pl  PSP  two



‘I have two children.’ (lit. my children, two)


%%please move \begin{table} just above \begin{tabular
\begin{table}
\caption{shows that the numbers one to ten are unique. The numbers eleven through nineteen are composites of ten plus one, ten plus two, etc. The word to indicate ‘plus’ is \textit{hər}, which has no other meaning in the language. Twenty is \textit{kɔk}\textit{\textsuperscript{w}}\textit{ʊr tʃɛw}, which is some kind of derivitave of \textit{kʊrɔ}  ‘ten.’ After 100, numbers are made of a coordinate noun phrase composed of  \textit{səkat }‘one hundred,’ the conjunction \textit{nə} ‘with,’ and a second number. One thousand is \textit{dʊbɔ}, and higher numbers are seen as multiples of \textit{dʊbɔ}.}
\label{tab:21}
\end{table}

There is a culturally governed exception to the use of cardinal numbers in Moloko.  To give the age of a one year old child, a Moloko speaker will say \textit{mɪvijɛ daz}\textit{ }(not *\textit{mɪvijɛ }\textit{bɪlɛŋ} ‘year one’). \textit{Mɪvijɛ daz} means that the child has lived through one Moloko New Year (celebrated in September). The word \textit{daz} does not have any other meaning apart from its use here. 

\subsection{Numbers for counting money}
\hypertarget{RefHeading1211061525720847}{}
Money is counted using two different systems which overlap (see \tabref{tab:22}.). A base-five system is used for amounts under about 250 Central African Francs (Fcfa). Many languages in Cameroon use a base five system for counting money. The reason for its use is probably based on the fact that the smallest coin was worth 5 Fcfa, and it became the basic unit for monetary transactions.\footnote{The generic term for money in Moloko is \textit{dala}, possibly a borrowed term from the American dollar.} Ten francs, being two of these coins, is \textit{dal tʃɛw }‘two coins,’ fifteen francs is \textit{dal makar}\textit{ }‘three coins,’ and so on (the values for the other coins that were available are indicated in the left column of \tabref{tab:22}.). The system becomes awkward for higher amounts (above 50 coins, or 250 Fcfa) because of the high numbers, and so a base ten system is superimposed (right column of \tabref{tab:22}.). Between 100 Fcfa and 250 Fcfa, both the base five and base ten are used, although within the Moloko mountain region, the base five system predominates. 

\begin{tabular}{lll}
\lsptoprule

\textbf{Amount of money} & \textbf{Base five system} & \textbf{‘Base ten’ system}\\
5 Fcfa (coin) & \textit{síj-sàj} & \\
10 Fcfa (coin) & \itshape d\={a}l\textup{ }tʃɛw & \\
15 Fcfa & \textit{d}\textit{\={a}}\textit{l m}\textit{á}\textit{k}\textit{\={a}}\textit{r} & \\
50 Fcfa (coin) & \textit{d}\textit{\={a}}\textit{l k}\textit{ʊ}\textit{r}\textit{ɔ} & \\
100 Fcfa (coin) & \textit{d}\textit{\={a}}\textit{l k}\textit{ɔ}\textit{k}\textit{\textsuperscript{w}}\textit{ʊ}\textit{r t}\textit{ʃ}\textit{ɛ}\textit{w} & \textit{(s}\textit{ʊ}\textit{l}\textit{ɔ}\textit{j) s}\textit{ə}\textit{k}\textit{\={a}}\textit{t}\\
150 Fcfa & \textit{d}\textit{\={a}}\textit{l k}\textit{ɔ}\textit{k}\textit{\textsuperscript{w}}\textit{ʊ}\textit{r m}\textit{á}\textit{k}\textit{\={a}}\textit{r} & \textit{s}\textit{ʊ}\textit{l}\textit{ɔ}\textit{j s}\textit{ə}\textit{k}\textit{\={a}}\textit{t n}\textit{ə}\textit{ d}\textit{\={a}}\textit{l k}\textit{ʊ}\textit{r}\textit{ɔ}\\
200 Fcfa & \textit{d}\textit{\={a}}\textit{l k}\textit{ɔ}\textit{k}\textit{\textsuperscript{w}}\textit{ʊ}\textit{r m}\textit{ə}\textit{f}\textit{\={a}}\textit{ɗ} & \textit{s}\textit{ə}\textit{k}\textit{\={a}}\textit{t t}\textit{ʃ}\textit{ɛ}\textit{w}\\
250 Fcfa & \textit{d}\textit{\={a}}\textit{l k}\textit{ɔ}\textit{k}\textit{\textsuperscript{w}}\textit{ʊ}\textit{r ɮ}\textit{ɔ}\textit{m} & \textit{s}\textit{ə}\textit{k}\textit{\={a}}\textit{t t}\textit{ʃ}\textit{ɛ}\textit{w d}\textit{\={a}}\textit{l k}\textit{ʊ}\textit{r}\textit{ɔ}\\
300 Fcfa &  & \textit{s}\textit{ə}\textit{k}\textit{\={a}}\textit{t m}\textit{á}\textit{k}\textit{\={a}}\textit{r}\\
500 Fcfa (coin) &  & \textit{s}\textit{ə}\textit{k}\textit{\={a}}\textit{t ɮ}\textit{ɔ}\textit{m}\\
1,000 Fcfa (bill) &  & \textit{ɔ}\textit{mb}\textit{ɔ}\textit{l}\textit{ɔ}\\
2,000 Fcfa (bill) &  & \textit{ɔ}\textit{mb}\textit{ɔ}\textit{l}\textit{ɔ }\textit{t}\textit{ʃ}\textit{ɛ}\textit{w}\\
3,250 Fcfa &  & \textit{ɔ}\textit{mb}\textit{ɔ}\textit{l}\textit{ɔ }\textit{m}\textit{á}\textit{k}\textit{\={a}}\textit{r n}\textit{ə}\textit{ s}\textit{ʊ}\textit{l}\textit{ɔ}\textit{j k}\textit{ɔ}\textit{k}\textit{\textsuperscript{w}}\textit{ʊ}\textit{r ɮ}\textit{ɔ}\textit{m}\\
5,000 Fcfa (bill) &  & \textit{ɔ}\textit{mb}\textit{ɔ}\textit{l}\textit{ɔ }\textit{ɮ}\textit{ɔ}\textit{m}\\
10,000 Fcfa (bill) &  & \textit{ɔ}\textit{mb}\textit{ɔ}\textit{l}\textit{ɔ }\textit{k}\textit{ʊ}\textit{r}\textit{ɔ}\\
50,000 Fcfa &  & \textit{ɔ}\textit{mb}\textit{ɔ}\textit{l}\textit{ɔ }\textit{k}\textit{ɔ}\textit{k}\textit{\textsuperscript{w}}\textit{ʊ}\textit{r ɮ}\textit{ɔ}\textit{m}\\
100,000 Fcfa &  & \textit{ɔmbɔlɔ s}\textit{ə}\textit{k}\textit{\={a}}\textit{t}\\
1,000,000 Fcfa &  & \textit{ɔmbɔlɔ s}\textit{ə}\textit{k}\textit{\={a}}\textit{t k}\textit{ʊ}\textit{r}\textit{ɔ}\\
\lspbottomrule
\end{tabular}

\begin{itemize}
\item \begin{styleTabletitle}
 Numbers for money
\end{styleTabletitle}\end{itemize}

The basic unit for the monitary base ten system is the 100 Fcfa coin (\textit{sʊlɔj} \textit{səkat} ‘coin 100’). This system uses the same number for one hundred as the system for counting items (\textit{səkat}). Ten of these coins make the 1000 Fcfa bill, so not unexpectedly, the term for the 1000 Fcfa bill is not the same as the number ‘1000’ for counting non-money items (\textit{dʊbɔ} see \tabref{tab:21}.), but rather is a term specific to money – \textit{ɔmbɔlɔ}. 

When larger amounts of money are counted, both base ten and base five systems are used. For example, 13,250 Fcfa is \textit{ɔmbɔlɔ kʊrɔ hər makar nə s}\textit{ʊ}\textit{l}\textit{ɔ}\textit{j kɔkʊr ɮɔm} ‘thirteen thousand Fcfa (base ten) and fifty 5 Fcfa coins (base five)’ (lit. 13 thousand with 50 5Fcfa coins).

It is interesting that recently, a one franc coin has been made available. The term for this coin wasn’t in the original counting system where the 5 Fcfa coin was the basic unit. It is now called [ɛlɛ bɪlɛŋ] literally ‘one eye.’ 

\subsection{Ordinal numbers}
\hypertarget{RefHeading1211081525720847}{}
Only the first ordinal number is a unique vocabulary word in Moloko: \textit{tʃɛkɛm} ‘first.’  The other ordinal expressions use a noun phrase construction using the cardinal counting numbers (cf. \tabref{tab:21}.):


\ea
cekem
\z

tʃɛkɛm



 ‘first’


\ea
anga  baya cew
\z

aŋga baja   tʃɛw



POSS   time     two 



‘second’


\ea
anga  baya  makar
\z

aŋga   baja   makar



\textsc{POSS}   time     three 



‘third’ 


\subsection{Non-numeral quantifiers}
\hypertarget{RefHeading1211101525720847}{}
Non-numeral quantifiers include \textit{gam} ‘much.’\footnote{Some of these quantifiers can also pattern as adverbs, e.g., \textit{gam} ‘much.’ ex. 192.}\textit{ }\textit{nɛk}\textit{\textsuperscript{w}}\textit{ɛŋ} ‘little,’ \textit{ʣijga} ‘all,’ \textit{dij}\textit{ }\textit{daj} ‘approximately,’ \textit{haɗa} ‘enough.’  When they occur, they are the final element of a noun phrase (ex. 166, the noun phrase is delimited by square brackets).  


\ea
{}[Məze  ahay  \textbf{gam} ]  təlala  afa  ne.
\z

{}[mɪʒɛ =ahaj  \textbf{gam} ]  tə-l=ala    afa    nɛ



people  =Pl  much  3P-go=to  at house of  1S



‘Many people came to my house.’


\ea
 Slərele  \textbf{gam!}
\z

ɬɪrɛlɛ   \textbf{gam}



work  much



{}[That is] a lot of work!



Disobedient Girl S. 4


\ea
Ávata  [məvəye  \textbf{haɗa} ]. 
\z

á-v=ata    [mɪ-v-ijɛ     \textbf{haɗa} ] 



3S.IFV-pass=3P.IO   \textsc{NOM}{}-pass(year)-CL     enough



‘It lasted them enough for the whole year.’  


\ea
Nok\textsuperscript{  }[\textbf{haɗa}  bay.]
\z

nɔk\textsuperscript{w}  [\textbf{haɗa}   baj ]



2S    enough  \textsc{NEG}



‘You [are] small’ (lit. not enough)


\section{Existentials}
\hypertarget{RefHeading1211121525720847}{}
Moloko has three positive existentials and one negative existential. The prototypical existential \textit{aba} ‘there exists’ carries the most basic idea of existence. Its negative is \textit{abaj} ‘there does not exist.’ The locational existential \textit{ava} ‘there exists in a particular place,’ and the possessive existential \textit{aka} ‘there exists associated with’ each carry the concept of existence along with their own specific meaning.The possessive existential must be accompanied by a indirect object pronominal.  

Existentials are verb-like and fill the verb slot in a clause, but are not conjugated for aspect or mood and do not take subject or direct object pronominals. Some of the existentials can carry verbal extensions or indirect object pronominals. The existential clause contains few elements – most commonly just a subject and the existential. The existential clause can be in a presupposition construction (Chapter 12) or interrogative construction (\sectref{sec:11.3}). 

The prototypical existential is \textit{aba} ‘there is’ (ex. 170{}-171) and its negative is \textit{abaj} ‘there is none’ (ex. 172{}-173).\footnote{This existential is perhaps a compound of the existential \textit{aba} and the negative \textit{baj}. } A clause with one of these existentials requires a subject but there are no other core participants or obliques. The existential is bolded in the examples.\footnote{The first line in each example is the orthographic form. The second is the phonetic form (slow speech) with morpheme breaks.} 


\begin{itemize}
\item 
\textit{Məze}  \textbf{\textit{aba.}}\textit{ }
\end{itemize}

mɪʒɛ \textbf{aba} 



person  \textsc{EXT}



‘There was a man . . .’ (a common beginning to a story)


\ea
Babəza  əwla  ahay  \textbf{aba.}
\z

babəza   =uwla    =ahaj  \textbf{aba}



children  =1S.POSS  =Pl  \textsc{EXT}



‘I have children.’ (lit. my children exist)


\ea
Babəza  əwla  ahay  \textbf{abay.}
\z

babəza  =uwla    =ahaj  \textbf{abaj}



children  =1S.POSS  =Pl  \textsc{EXT} \textsc{NEG}



‘I have no children.’  (lit. my children do not exist)


\ea
Dala  \textbf{abay.}
\z

dala    \textbf{abaj}



money  \textsc{EXT} \textsc{NEG}



‘I have no money.’ (lit. there is no money)


The existentials \textit{aba} and \textit{abaj} can also carry an extended sense to indicate the health of the person. Ex. 174 and 175 are greetings, which are questions that can occur with (ex. 174) or without (ex. 175) the word \textit{zaj} ‘peace.’ Ex. 176 is a possible reply to either of these questions. Likewise, ex. 177{}-178

show inquiries and possible replies as to the health of a third person.

\ea
nɔk\textsuperscript{w}  \textbf{aba}    zaj  ɗaw        nɛ  \textbf{aba}
\z

nɔk\textsuperscript{w}  \textbf{aba}    zaj  ɗaw        nɛ  \textbf{aba}



2S    \textsc{EXT}    peace  QUEST        1S  \textsc{EXT}



‘Are you well?’ (lit. ‘Do you exist [in] peace?’)        ‘I am well.’ (lit. I exist)


\ea
Nok\textsuperscript{  }\textbf{aba}  ɗaw?
\z

nɔk\textsuperscript{w}  \textbf{aba}    ɗaw



2S    \textsc{EXT}    QUEST



‘Are you well?’ (lit. ‘Do you exist?’)


\ea
Asak  əwla  \textbf{aba}\textbf{y.}
\z

asak    =uwla    \textbf{abaj}



foot    =1S.POSS  EXIST-\textsc{NEG}



‘My foot hurts.’ (lit. my foot doesn’t exist)


\begin{itemize}
\item 
\textit{Baba  ango  }\textbf{\textit{aba}}\textit{  ɗaw?   Ayaw,  ndahan  }\textbf{\textit{aba.}}
\end{itemize}

baba   =aŋg\textsuperscript{w}ɔ     \textbf{aba}   ɗaw        ajaw   ndahaŋ  \textbf{aba}



father  =2S.POSS  \textsc{EXT}  QUEST        yes  3S  \textsc{EXT}



‘Is your father well?’  (lit. does your father exist?)      ‘Yes, he is well.’


\begin{itemize}
\item 
\textit{Baba  əwla  na,  hərva  ahan  }\textbf{\textit{abay.}}
\end{itemize}

baba    =uwla      na  hərva  ahaŋ      \textbf{abaj}



father  =1S.POSS  PSP  body  =3S.POSS    \textsc{EXT}+\textsc{NEG}



‘My father is sick.’ (lit. my father, his body doesn’t exist)


The existential \textit{aba} is also used in presentational clauses to introduce participants into a story. \textit{Aba} is used in presentational clauses in a narrative to introduce some major participants in the setting. Ex. 179 is the introduction to the Cicada story. 


Cicada S. 3-4


\ea
Albaya  ahay  \textbf{aba.}  Tánday  tətalay  a  ləhe.  
\z

albaja  =ahaj  \textbf{aba}  tá-ndaj    tə-tal-aj    a  lɪhɛ



young man   =Pl  \textsc{EXT}  3P.IFV-PRG     3P.IFV-walk-CL  at    bush



‘There were some young men. They were walking in the bush.’


In some presentational clauses both the prototypical existential and the locational existential can co-occur.  Ex. 180 is from the setting of the Disobedient Girl story. Note that this existential clause contains the adverb \textit{ɛtɛ} ‘also.’


Disobedient Girl S. 9-10


\ea
Albaya  \textbf{ava}  \textbf{aba}  ete.  O\textbf{l}o  a\textbf{z}ala  hor.
\z

albaja  \textbf{ava}    \textbf{aba}  ɛtɛ.   ɔ{}-\textbf{l}ɔ    à-\textbf{z=}ala    h\textsuperscript{w}ɔr



young.man  LOC-\textsc{EXT}   \textsc{EXT}  also  3S.PFV-go    3S.PFV-take=to    woman



‘And so, there once was a young man (in a particular place). He went and took a wife.’


The locational existential \textit{ava} ‘there is in’ expresses existence ‘in’ a particular location. This existential is the same as the adpositional verbal extension\textit{ =ava }‘in’ (see \sectref{sec:56}) and locational postclitic \textit{ava} ‘in’ (see \sectref{sec:46}), all of which express the location in something, either physically or figuratively. In some of the examples below, a response is included which also employs the same existential. Note that the existential in ex. 184 carries the directional ‘away from’ (see \sectref{sec:57}).

\ea
Sese  \textbf{ava}  ɗaw?   Ayaw,  sese  \textbf{ava.}
\z

ʃɛʃɛ    \textbf{ava}  ɗaw        ajaw  ʃɛʃɛ  \textbf{ava}



meat  \textsc{EXT}+in  QUEST        yes  meat  \textsc{EXT}+in



‘Is there any meat located here [for sale]?’    ‘Yes, we have meat located here.’


\ea
Baba  ango,  ndahan  \textbf{ava}  ɗaw?   Ndahan  \textbf{ava}  bay;  enje  amətele.
\z

baba  =aŋg\textsuperscript{w}ɔ      ndahaŋ    \textbf{ava}        ɗaw    ndahaŋ  \textbf{ava}  baj    ɛ{}-nʒɛ    amɪ-tɛl-ɛ



father  =2S.POSS    3S        \textsc{EXT}+in   QUEST  3S       \textsc{EXT}+in  \textsc{NEG}   3S.PFV-left DEP-travel-CL



Is your father located here? (lit. your father, is he there?)  No, he is not located here; he went somewhere.


\ea
Ndahan  \textbf{ava.}    
\z

ndahaŋ  \textbf{ava}    



3S    \textsc{EXT}+in    



‘He/she is here.’


\ea
Ndahan  \textbf{ava}  alay.
\z

ndahaŋ  \textbf{ava}  =alaj



3S    \textsc{EXT}+in  =away



‘he/she is located at the place of reference.’ (lit. he is in away)


The possessive existential \textit{aka} ‘there is on’ expresses existence ‘on’ a person (indicating possession or accompaniment).  This existential is the same as the adpositional verbal extension =\textit{aka} ‘on’ (see \sectref{sec:56}) and locational postclitic \textit{aka} ‘on’ (see \sectref{sec:46}), all of which express location on something, wither physically or figuratively. The subject of the possessive existential (the possessed item) is followed by a construction consisting of the indirect object pronominal cliticised to the particle \textit{an-}, in turn followed by the possessive existential \textit{aka} ‘on.’ The particle \textit{an- }is the same particle to which the indirect object pronominal\textit{ }cliticises when there is a suffix on the verb stem (see \sectref{sec:1}) and these elements are found in the same order as they are within the verb complex. Ex. 185 shows a question and response pair.

\ea
Dala  anok\textsuperscript{  }\textbf{aka}  ɗaw?   Ayaw,  dala  anaw  \textbf{aka}\textbf{.}
\z

dala   an=ɔk\textsuperscript{w}     \textbf{aka}     ɗaw    ajaw  dala   an  =aw   \textbf{aka}



money  DAT=2S.IO  \textsc{EXT}+on    QUEST    Yes  money  DAT  =1S.IO  \textsc{EXT}+on



‘Do you have any money [located] with you?’      ‘Yes, I have money [located] on me.’  



(lit. is there money on you?)


\ea
Hor  anan  \textbf{aka  }ana  Mana.
\z

h\textsuperscript{w}ɔr   an=aŋ     \textbf{aka    }ana   Mana



woman  DAT=3S.IO  \textsc{EXT}+on    DAT  Mana



‘He has a wife.’ (lit. a woman to him there is on for Mana)


The existential \textit{aka} can also be used to mean accompaniment (ex. 187).

\ea
Bahay  a  sla  ahay  na,  ndahan  \textbf{aka}  ɗaw?
\z

bahaj  a  ɬa  =ahaj  na  ndahaŋ  \textbf{aka}  ɗaw



chief  GEN  cow  =Pl  PSP  3S  \textsc{EXT}  QUEST



Was the owner of the cows [located] with [you]? (lit. the chief of the cows, was he ‘on’?)


The locational existential \textit{aka}  (ex. 188, 190) can also fill the same role as the verb \textit{ndaj } (ex. 189, see \sectref{sec:62}) to express an action in progress. This usage of \textit{aka} may be due to adoption of a similar particle in Fulfulde, the language of wider communication in the region. The particle \textit{don} in Adamawa Fulfulde has a present progressive and existential use similar to \textit{aka} in Moloko (Noye, 1974: 58, 73; Edward Tong, personal communication).

\ea
ndahaŋ   aka     ɔ{}-zɔm     ɗaf 
\z

ndahaŋ   aka     ɔ{}-zɔm     ɗaf 



3S    \textsc{EXT}+on    3S.IFV-eat  loaf



‘He/she is eating millet loaf.’


\ea
Ánday  ózom  ɗaf.
\z

á-ndaj    ɔ{}-zɔm     ɗaf 



3S.IPV-PROG  3S.IFV-eat  loaf



‘He/she is eating millet loaf.’



Disobedient Girl S.24


\ea
Ndahan  na,  ndahan    aka  njəw  njəw  njəw.   
\z

ndahaŋ   na    ndahaŋ   aka          nzuw  nzuw  nzuw           



3S                   PSP    3S           \textsc{EXT}+on       \textsc{ID}{}-grind



And she, she is grinding some more.


\section{Adverbs }
\hypertarget{RefHeading1211141525720847}{}
Some adverbs modify verbs within the verb phrase (simple or derived, Sections 25 and 26, respectively), others modify the clause as a whole (temporal adverbs, \sectref{sec:27}), and yet others function at the discourse level (\sectref{sec:28}). Note that ideophones can function adverbially to give pictoral vividness to a clause (Doke, 1935). Because they pattern differently than adverbs, they are considered in their own section (\sectref{sec:3.6}).

\subsection{Simple verb phrase-level adverbs}
\hypertarget{RefHeading1211161525720847}{}
Verb phrase adverbs give information concerning the location, quality, quantity, or manner of the action expressed in the verb phrase. These adverbs occur after any adpositional phrases (ex. 191 - 193).\footnote{The first line in each example is the orthographic form. The second is the phonetic form (slow speech) with morpheme breaks.} The negative follows any such adverbs in negative clauses (Chapter 11.2). 


Disobedient Girl S. 4



\ea
Təwasava  \textbf{neken  }kəygehe.    
\z

tə-was                  =ava         \textbf{nɛk}\textbf{\textsuperscript{w}}\textbf{ɛŋ}  kijgɛhɛ    



3P.PFV-cultivate =in       little    like this    



‘They cultivated a little like this.’


\ea
Hərmbəlom  andaɗay  nok  \textbf{gam}\textbf{.}
\z

Hʊrmbʊlɔm  a-ndaɗ-aj  nɔk\textsuperscript{w}  \textbf{gam}



God    3S-love-CL  2S  much



‘God loves you a lot.’


\ea
Názaɗ  a  dəray  ava  \textbf{sawa}\textbf{n.}
\z

ná-zaɗ    a  dəraj  ava    \textbf{sawaŋ}



1S.IFV-carry    in  head  in    without help



‘I can carry it (on my head) by myself!’


Verb phrase adverbs include \textit{dɪrɛŋ} ‘far distance,’ \textit{nɛk}\textit{\textsuperscript{w}}\textit{ɛ}\textit{ŋ} ‘a small quantity’ (ex. 191)\textit{, gam }‘a large quantity’ (ex. 192), \textit{sawaŋ} ‘without help’ (ex. 193) and the modal adverbs \textit{təta} ‘can,’ an adverb of ability (ex. 194 and 195), and \textit{dɛwɛlɛ }‘ought,’ an adverb of necessity (ex. 196). 

\ea
Kázala  \textbf{təta}\textbf{.}
\z

ká-z      =ala    \textbf{təta}



2S.IFV-carry    =to    ability



‘You can carry it.’


\ea
Bahay  ázom  sese  \textbf{təta}\textbf{.}
\z

bahaj  á-zɔm    ʃɛʃɛ  \textbf{təta}



chief  3S.IFV-eat  meat  ability



‘The chief can eat meat.’


\ea
Bahay  ázom  sese  \textbf{dewel}\textbf{e.}
\z

bahaj   á-zɔm    ʃɛʃɛ  \textbf{dɛwɛlɛ}



chief  3S.IFV-eat  meat  necessary



‘The chief must eat meat.’


The simple adverbs expressing location, quantity, quality, and manner can be intensified by reduplication (or lengthening) of a consonant or reduplication of the entire adverb.\footnote{Adverbs of ability and necessity cannot be reduplicated, nor can adverbs which function beyond the verb phrase level. }  Ex. 197 - 200 show the simple adverb with its intensified counterpart. The reduplication of a consonant occurs at the onset of the final syllable (ex. 197 and 198). The entire adverb is reduplicated in ex. 199 and 200. Intensified adverbs cannot be negated. 

\ea
dəren  də\textbf{rr}en 
\z

dɪrɛŋ    dɪ\textbf{rr}ɛŋ 



 ‘far’    ‘very far’


\ea
ɗeɗen     ɗe\textbf{ɗɗ}en
\z

ɗɛɗɛŋ     ɗɛ\textbf{ɗɗ}ɛŋ



 ‘true’    ‘very true’


\ea
gam     gam gam 
\z

‘ a lot     ‘a whole lot’\textit{ }


\ea
nekwen     nekwen nekwen 
\z

nɛk\textsuperscript{w}œŋ     nɛk\textsuperscript{w}œŋ nɛk\textsuperscript{w}œŋ



 ‘little’    ‘a little at a time’


\subsection{  Derived verb phrase-level adverbs}
\hypertarget{RefHeading1211181525720847}{}
Verb phrase adverbs can be derived from nouns by reduplicating the final consonant of the noun and adding [a] (i.e. Ca where the C is the final consonant of the noun). The reduplicated syllable is labelled ‘adverbiser’ (ADV)\footnote{We have not found the term ‘adverbiser’ in the literature. Adverbiser in this work is defined as a derivational morpheme whose presence changes the grammatical class of a stem to become an adverb. } in ex. 201 - 202. Compare the noun and its derived adverb in ex. 201 and 202. Note that the reduplicated consonant in the derived adverb in example 201 is the word-final [x] rather than word-medial [h]. Likewise, example 202 shows [ŋ] rather than [n]. These word-final changes (see \sectref{sec:11}) in the reduplicated consonant indicate that the reduplication occurs after phonological word-final changes are made and that the reduplicated segment is phonologically bound to the noun (see \sectref{sec:12}). 


\ea
zayəh  \textbf{zayəhha}
\z
\ zajəx    zajəx=xa



care      care     =ADV



‘care’    ‘carefully’


\ea
deden    \textbf{d}\textbf{e}\textbf{d}\textbf{enn}\textbf{a}  
\z

dɛdɛŋ    \textbf{d}\textbf{ɛ}\textbf{d}\textbf{ɛ}\textbf{ŋ=ŋa}  



truth    truth    =ADV  



‘truth’    ‘truthfully’  


Note especially ex. 203 and 204 which illustrate that the labialisation prosody on the nouns \textit{rʊbɔk}  and \textit{hʊrʊk}  does not spread rightwards to the adverbiser (otherwise, the reduplicated /k/ would be labialised, see \sectref{sec:2.1}). 

\ea
zar  akar  ɗəw,  ndahan  ava  \textbf{r}\textbf{ə}\textbf{bokka}
\z
\ zar     akar  ɗuw  ndahaŋ  ava  \textbf{rʊbɔk}\textbf{\textsuperscript{w}}\textbf{=ka}



man    theft  also  3S  \textsc{EXT}{}-in  hiding place  =ADV



‘There was (in that place) a thief, hiding.’ 


\ea
nege  slərwle  \textbf{hərəkka}
\z

nɛ-g-ɛ  ɬɪrɛlɛ  \textbf{hʊrʊk}\textbf{\textsuperscript{w}}\textbf{ =ka}



1S-do-CL  work  all day   =ADV



‘I worked all day.’


\subsection{Clause-level adverbs}
\hypertarget{RefHeading1211201525720847}{}
Temporal adverbs modify the clause as a whole and can occur clause initial or verb phrase final(ex. 205 and 206,\footnote{Note that \textit{a }\textit{kɔsɔk}\textit{\textsuperscript{w}}\textit{ɔ ava} ‘in the market’ is a complex adpositional phrase (see \sectref{sec:1146}).} respectively).\footnote{The order of constituents in the verb phrase is given in Chapter 8.} These include ‘today,’ \textit{hadʒaŋ} ‘tomorrow,’ \textit{apaza}\textit{ŋ} ‘yesterday.’ 


\ea
\textbf{Egəne  }nólo  a  kosoko  ava.
\z

\textbf{ɛgɪnɛ}nɔ-lɔ     a  kɔsɔk\textsuperscript{w}ɔ  ava



today  1S.IFV-go  in  market  in



Today I will go to the market. 


\ea
Nólo  a  kosoko  ava\textbf{  }\textbf{hajan.}
\z

nɔ-lɔ   a  kɔsɔk\textsuperscript{w}ɔ  ava  \textbf{hadʒaŋ}



1S.IFV-go  in  market  in  tomorrow



I will go to the market tomorrow.


\subsection{Discourse-level adverbs}
\hypertarget{RefHeading1211221525720847}{}
Discourse adverbs function at the clause combining level. Grammatically they are found verb phrase final. Semantically they tell something of the relation of their clause to what has happened before in the discourse. Discourse adverbs can neither be negated nor intensified by reduplication. They include \textit{ɛʃɛ }‘again’ (same actor, same action, ex. 207), \textit{ɛtɛ }‘also’ (same action, different actor, ex. 208)\textit{, }\textit{fa}\textit{ŋ}\textit{ }‘already’ (expressing Perfect aspect in that the action is performed in the past with effects continuing to the present, ex. 209)\textit{,}\textit{ kʊlɔ}\textit{ }‘already’ or ‘before’ (the action was performed at least once before a particular time, ex. 210). 


\ea
Nóolo  \textbf{ese.}
\z

nɔɔ-lɔ     \textbf{ɛʃɛ}



1S.POT-go    again



 ‘I will go again.’


\ea
Nóolo  \textbf{ete.}
\z

nɔɔ-lɔ     \textbf{ɛtɛ}



1S.POT-go    also



 ‘I will go too.’


\ea
Nege  na  \textbf{fa}\textbf{n.}
\z

nɛ-g-ɛ    na  \textbf{faŋ}



1S.PFV-do{}-CL  3S.DO  already



‘I did it already.’


\ea
Nəmənjar  ndahan  \textbf{k}\textbf{ə}\textbf{l}\textbf{o.}
\z

nə-mənzar    ndahaŋ  \textbf{kʊlɔ}



1S-see  3S  before



‘I have seen him/her before.’


\textit{Uwɗɛ} ‘first’ (ex. 211) indicates that the event expressed in the clause occurs before something else. 


Cicada S. 20


\ea
Náamənjar  na  alay  memele  ga  ndana\textbf{  əwɗ}\textbf{e.}
\z

náá-mənzar    na   =alaj    mɛmɛlɛ  ga  ndana  \textbf{uwɗɛ}



1S.POT-see 3S.DO   =away    tree  ADJ  DEM  before something else



‘First let me go and see that tree that you spoke of.’ (lit. I would like to see that above-mentioned tree first)


\textit{Aɮa}\textit{ ‘}now’ (ex. 212 and 213) adds tension and excitement.


Disobedient Girl S. 21


\ea
Ndahan  bah  məbehe  hay  ahan  amadala  na  kə  ver  aka  \textbf{azla}\textbf{.}
\z

ndahaŋ  bax      mɪ-bɛh-ɛ    haj     =ahaŋ     



3S              \textsc{ID}pour  \textsc{NOM}{}-pour-CL  millet    =3S.POSS  



\textit{ama-d  =ala  na        kə     vɛr         aka  }\textbf{\textit{aɮa}}



DEP-put  =to    3S.DO  on  stone     on     now



‘She poured out the millet to put it onto the grinding stone now.’



Disobedient Girl S. 22


\ea
Njəw  njəw  njəw  aməhaya  \textbf{azla.   }
\z

nzu   nzu    nzu   amə-h=aja  \textbf{aɮa   }



\textsc{ID}grind    DEP-grind=\textsc{PLU}  now   



‘Nzu nzu nzu [she was] grinding now.’


\textit{Ɗuwgɛ} ‘actual’ (to indicate that the events in the clause actually happened, ex. 214).


Snake S. 24


\ea
Ka  nehe  ləbara  a  ma  ndana  \textbf{ɗ}\textbf{əwg}\textbf{e.}      
\z

ka  nɛhɛ  ləbara     a       ma      ndana     \textbf{ɗ}\textbf{uwgɛ}      



like  DEM  news   GEN    word     DEM   actual



‘And so was that  previously mentioned story.’


\textit{Rɛ} ‘counterexpectation’ (ex. 215) indicates that the clause is the opposite to what the hearer might have expected.  


Values S. 50


\ea
Epele  epele  na  me,  Hərmbəlom  anday  agas  ta  a  ahar  ava  \textbf{r}\textbf{e.}
\z

ɛpɛlɛ ɛpɛlɛ   na   mɛ  Hʊrmbʊlɔm   a-ndaj       a-gas   ta  a   ahar  ava   \textbf{rɛ}



in the future  PSP  opinion  God      3S-PROG    3S-catch   3P.DO  to  hand  in     in spite



‘And so in the future in-my-opinion-and-think-about-it-yourself, God is going to accept them [the elders] in his hands, in spite [of what the church people say].’


\section{Ideophones}
\hypertarget{RefHeading1211241525720847}{}
Ideophones are a “vivid representation of an idea in sound” (Doke, 1935: 118). They evoke the ‘idea’ of a sensation or sensory perception (action, movement, colour, sound, smell, or shape). As such they are often onomatopoeic. 

\citet{Newman1968} suggests that ideophones do not comprise a grammatical class of their own, but rather are words from several different classes (including nouns, adjectives, and adverbs) which are grouped together based on phonological and semantic similarities rather than syntax. In Moloko ideophones will be treated as a separate grammatical class since although they may fill the noun, verb, or adverb slot in a clause, ideophones never pattern as typical nouns, verbs, or adverbs. \sectref{sec:29} describes the semantic and phonological features of ideophones; \sectref{sec:30} discusses their syntax and their role in discourse, and \sectref{sec:31} discusses the fact that a clause where an ideophone fills the verb slot can carry zero transitivity.  

\subsection{Semantic and phonological features of ideophones}
\hypertarget{RefHeading1211261525720847}{}
Ideophones carry an idea of a particular state or event – Moloko speakers can imagine the particular situation and the sensation of it when they hear a particular ideophone. The sensation may be a sound (ex. 216), vision (ex. 217), taste (ex. 218), feeling (ex. 219), or even abstract idea (for example, an insult, ex. 220).\footnote{The first line in each example is the orthographic form. The second is the phonetic form (slow speech) with morpheme breaks.} 


\ea
gəɗəgəzl
\z

gəɗəgəɮ



 ‘the noise of something closing or being set down’


\ea
danjəw
\z

dànzúw



 ‘sight of someone walking balancing something on their head’


\ea
podococo
\z

pɔdɔtsɔtsɔ



 ‘taste of sweetness’


\ea
pəyecece
\z

pìj$\text{\textgreek{>e}}$tʃ$\text{\textgreek{>e}}$tʃ$\text{\textgreek{>e}}$



 ‘feeling of coldness’


\ea
kekəf  kəf  kekəf  kəf 
\z

k$\text{\textgreek{<e}}$kɪf  kɪf     k$\text{\textgreek{<e}}$kɪf  kɪf 



 ‘imagination of someone who hasn’t any weight’ (an insult)


Ideophones have specific meanings; compare the following three ideophones in ex. 221 - 223. The ideophones differ in only the final syllable.

\ea
pəvbəw pəvbəw
\z

pəⱱuw pəⱱuw



 ‘sight of rabbit hopping’


\ea
pəvba  pəvba
\z

pəⱱa  pəⱱa



 ‘sound of a whip’


\ea
pəvban  pəvban
\z

pəⱱaŋ  pəⱱaŋ



 ‘sight of the start of a race’


Ideophones do not follow the stress rules for the language (Chapter 2) Some ideophones are stressed on the initial syllable (shown by full vowels in ex. 220). Some ideophones have no full vowel even if one of the syllables is stressed (ex. 216, 224, 227). 

\ea
jəɓ  jəɓ
\z

dʓɪɓ dʓɪɓ



 ‘completely wet’


Moloko ideophones sometimes contain unusual sounds, including the labiodental flap [ⱱ], found only in ideophones that carry a neutral prosody. 

\ea
vbaɓ
\z

ⱱàɓ



 ‘sound of something soft hitting the ground’ (a snake, or a mud wall collapsing)


Ideophones often have reduplicated segments as shown in ex. 226 (see also ex. 218, 219, 220 for additional examples).

\ea
həɓek  həɓek
\z

hìɓ$\text{\textgreek{<e}}$k hìɓ$\text{\textgreek{<e}}$k



 ‘hardly breathing’ (almost dead)


Some ideophones require a context in order for their meaning to be understood clearly; others give a clear meaning even if they are spoken in isolation. For example, if a Moloko speaker hears someone say \textit{nzuw nzuw} (ex. 227), they know that the speaker is talking about someone grinding something on a grinding stone. Likewise see ex. 228, 216, 218 - 220, 225, 226, and 245, all of which carry a distinctive lexical meaning even when spoken in isolation. 

\ea
njəw  njəw
\z

nzùw  nzùw



‘the sound of someone grinding something on a grinding stone’


\ea
pəcəkəɗək
\z

p$\text{\textgreek{'u}}$ts$\text{\textgreek{'u}}$k$\text{\textgreek{'u}}ɗ\text{\textgreek{'u}}$k



‘the sight of a toad hopping’


However, a Moloko speaker will need to understand a wider context to determine the meaning of \textit{dɛrg}\textit{\textsuperscript{w}}\textit{ɛtʃɛk} (ex. 229), which requires a context for the listener to understand the detail of the picture. In the same way, ex. 225 also requires a context to specify its exact meaning (snake falling or wall collapsing).

\ea
dergwecek
\z

dɛrg\textsuperscript{w}ɛtʃɛk



‘sight of someone lifting something onto their head’


\subsection{Syntax of ideophones}
\hypertarget{RefHeading1211281525720847}{}
In a sentence, an ideophone can function as a noun, adverb, or verb. As a noun, the ideophone carries a descriptive picture with certain features.  Ideophones that are lexical nouns (ex. 230 - 232, see also 218 and 219) can function as the head of a noun phrase, but they cannot be pluralised or modified by noun phrase constituents except with the adjectiviser \textit{ga}. In example 232, the ideophone \textit{mbadzak mbadzak mbadzak } ‘something big and reflective’ is the direct object of the clause. The ideophones are bolded in the examples.

Values S. 34


\ea
Ehe  na,  təta  na,  kəw  na,  \textbf{b}\textbf{ə}\textbf{w}\textbf{ɗ}\textbf{er}\textbf{e.}
\z

ɛhɛ  na      təta   na  kuw      na    \textbf{buw}\textbf{ɗ}\textbf{ɛrɛ}



here   PSP    3P    PSP    \textsc{ID}take   PSP    \textsc{ID}foolishness



‘Here, what they are taking is foolishness!’ (lit. they, taking,  foolishness)


Values S. 48

\ea
Kə  wəyen  aka  ehe  \textbf{tezl  t}\textbf{ezlezl.}
\z

kə   wijɛŋ   aka   ɛhɛ    \textbf{tɛɮ-tɛɮɛɮ}



on  earth  on  here    \textsc{ID}hollow



‘Among the people (on this earth) here, [we are like] the sound of a hollow cup bouncing on the ground.’



Snake S. 11


\ea
Námənjar  na,  \textbf{mbajak  mbajak  mbajak}  gogolvon.
\z

ná-mənzar   na  \textbf{mbadzak  mbadzak  mbadzak}  g\textsuperscript{w}ɔg\textsuperscript{w}ɔlvɔŋ



1S.IFV-see  3S.DO  \textsc{ID}something big and reflective  snake 



‘I was seeing it, something big and reflective, a snake!’


When an ideophone functions as an adverb, the ideophone gives information concerning the subject of the clause as well as the manner of the action. Unlike most other adverbs however, ideophones cannot be negated. \tabref{tab:23}. illustrates 11 different adverbial ideophones that collocate with the verb \textit{həm-aj}\textit{ }‘run’ but vary depending on the actor of the clause. 

\begin{tabular}{lll}
\lsptoprule

1 & \textit{zar   a-həm-aj }\textbf{\textit{g}}\textbf{\textit{$\text{\textgreek{`u}}$}}\textbf{\textit{d}}\textbf{\textit{ɔ}}\textbf{\textit{ g}}\textbf{\textit{$\text{\textgreek{`u}}$}}\textbf{\textit{d}}\textbf{\textit{ɔ}}\textbf{\textit{ g}}\textbf{\textit{$\text{\textgreek{`u}}$}}\textbf{\textit{d}}\textbf{\textit{ɔ}}

man  3S-run{}-CL     \textsc{ID}man running & ‘A man runs \textit{gudo gudo gudo}.’\\
2 & \textit{war a-həm-aj       }\textbf{\textit{nz}}\textbf{\textit{$\text{\textgreek{`u}}$}}\textbf{\textit{ɗ}}\textit{ɔ}\textbf{\textit{k nz}}\textbf{\textit{$\text{\textgreek{`u}}$}}\textbf{\textit{ɗ}}\textit{ɔ}\textbf{\textit{k}}

child 3S-run{}-CL    \textsc{ID}child running and jumping & ‘A toddler runs\textit{ }\textit{nzuɗok nzuɗok}.’\\
3 & \textit{albaja  a-həm-aj        }\textbf{\textit{nz}}\textbf{\textit{$\text{\textgreek{`u}}$}}\textbf{\textit{l nz}}\textbf{\textit{$\text{\textgreek{`u}}$}}\textbf{\textit{l}}

youth   3S-run{}-CL      \textsc{ID}youth running & ‘A young man runs \textit{nzul nzul}.’

(also mice run like this)\\
4 & \textit{mød}\textit{œ}\textit{h}\textit{\textsuperscript{w}}\textit{œ}\textit{r  a-həm-aj    təta          baj; }

old person  3S-run{}-CL   ABILITY   \textsc{NEG}

\textit{a-həm-aj      }\textbf{\textit{kərwəɗ wəɗ, kərwəɗ wəɗ}}

3S-run{}-CL      \textsc{ID}insult someone with no stomach & ‘An old person can’t run; he moves\textit{ }\textit{kurwuɗ wuɗ, kurwuɗ wuɗ}.’\\
5 & \textit{ɮɛvɛk  a-həm-aj      }\textbf{\textit{pə}}\textbf{\textit{ⱱ}}\textbf{\textit{ùw  pə}}\textbf{\textit{ⱱ}}\textbf{\textit{ùw}}

rabbit  3S-run{}-CL      \textsc{ID}rabbit hopping & ‘A rabbit runs\textit{ }\textit{pa}\textit{ⱱ}\textit{uw pa}\textit{ⱱ}\textit{uw}.’\\
6 & \textit{ɬa     =ahaj tə-həm-aj   }\textbf{\textit{gr}}\textit{ɪ}\textbf{\textit{p gr}}\textit{ɪ}\textbf{\textit{p}}

cow  =Pl    3P-run{}-CL   \textsc{ID}something heavy running & ‘Cows run\textit{ }\textit{grip grip}.’\\
7 & \textit{dzavar      =ahaj tə-həm-aj    }\textbf{\textit{tsərr}}

guinea fowl=Pl 3P-run{}-CL \textsc{ID}guinea fowl taking off & ‘Guinea fowl run\textit{ }\textit{tsirr}.’ (when they are taking off)\\
8 & \textit{ɛrkɛʃɛ  a-həm-aj    }\textbf{\textit{jɛɗ jɛɗ jɛɗ}}

ostrich  3S-run{}-CL    \textsc{ID}ostrich running & ‘An ostrich runs\textit{ }\textit{yed yed yed}.’\\
9 & \textit{mɔk}\textit{\textsuperscript{w}}\textit{tɔnɔk}\textit{\textsuperscript{w}}\textit{ a-həm-aj }\textbf{\textit{p$\text{\textgreek{'u}}$ts$\text{\textgreek{'u}}$k}}\textit{\textsuperscript{w}}\textbf{\textit{$\text{\textgreek{'u}}ɗ\text{\textgreek{'u}}$k}}\textit{\textsuperscript{w}}\textbf{\textit{, p$\text{\textgreek{'u}}$ts$\text{\textgreek{'u}}$k}}\textit{\textsuperscript{w}}\textbf{\textit{$\text{\textgreek{'u}}ɗ\text{\textgreek{'u}}$k}}\textit{\textsuperscript{w}}

toad           3S-run{}-CL  \textsc{ID}toad hopping & ‘A toad runs\textit{ }\textit{putsukuɗuk, putsukuɗuk}.’\\
10 & \textit{muwta a-həm-aj    }\textbf{\textit{f}}\textbf{\textit{ɪ}}\textbf{\textit{xx}}

truck    3S-run{}-CL  \textsc{ID}truck humming & ‘A truck runs\textit{ }\textit{fɪxx}.’\\
11 & \textit{həmaɗ a-həm-aj     }\textbf{\textit{f}}\textbf{\textit{ɔ}}\textbf{\textit{wwá}}

wind     3S-run{}-CL  \textsc{ID}wind blowing & ‘The wind runs \textit{f}\textit{owwa}.’\\
\lspbottomrule
\end{tabular}

\begin{itemize}
\item \begin{styleTabletitle}
Selected ideophones that co-occur with the verb həmaj  ‘to run’
\end{styleTabletitle}\end{itemize}

When they act as adverbs, ideophones can occupy one of two slots in the clause. When the verb they modify is finite, ideophones will occur at the end of the clause following other adverbs (ex. 233 - 235 and all of the examples in \tabref{tab:23}.). In a narrative, ideophones that function as adverbs can be found wherever the language is vivid. They occur most often at the inciting moment and the peak section of a narrative. The ideophones in each clause are bolded  and the verb phrase is delimited by square brackets. 


\ea
{}[azləgalay ]  avəlo  \textbf{z}o\textbf{r}\textbf{!}
\z

{}[à-ɮəg        =alaj ]  avʊlɔ  \textbf{z}ɔ\textbf{r}



3S.PFV-throw   =away  above  \textsc{ID}throwing



‘She threw [the pestle] up high (movement of throwing).’


\ea
{}[Anday  azlaɓay  ele ]  kəndal,    kəndal,  kəndal.
\z

{}[a-ndaj    a-ɮaɓ-aj    ɛlɛ ]  \textbf{kəndál,  kəndál,  kəndál}



3S-PRG    3S-pound{}-CL  thing  \textsc{ID}pounding millet   



‘She was pounding the [pestle] (threshing millet) pound, pound pound.’


\ea{}
{}[Həmbo  ga  anday  asak  ele  ahan ]  \textbf{wəsekek}\textbf{e.}
\z

{}[hʊmbɔ  ga  a-ndaj    a-sak    ɛlɛ  =ahaŋ ]    \textbf{wùʃɛk}\textbf{ɛ}\textbf{k}\textbf{ɛ}



flour  ADJ  3S-PRG    3S-multiply  thing  =3S.POSS  \textsc{ID}many



‘The flour was multiplying all by itself (lit. its things), sound of multiplying.’


When the verb they modify is non-inflected, the ideophone is the first element of the verb preceding the verb complex (ex. 236 and 237).  This is a special construction that is discussed in \sectref{sec:64.}

\begin{itemize}
\item \begin{styleExampleteference}
Nata  ndahan  \textup{[}\textbf{pək}  \textstyleExampleglossChar{mapata  aka  va}  pərgom  ahay  na.\textup{ ]}
\end{styleExampleteference}
\end{itemize}

nata  ndahaŋ  [\textbf{pək}      \textstyleExampleglossChar{ma-p=ata =aka =va}         pʊrg\textsuperscript{w}ɔm  =ahaj  na ]



also    3S  \textsc{ID} open door or bottle  \textsc{NOM}{}-open=3P.IO=on=\textsc{PRF}    trap    =Pl  PSP



‘He opened the traps for them.’


\ea
Dərlenge  [\textbf{pəyteɗ}  məhəme  ele  ahan ]  ete.
\z

dɪrlɛŋgɛ  [\textbf{píjt}\textbf{$\text{\textgreek{<e}}$}\textbf{ɗ}    mɪ-hɪm-ɛ  ɛlɛ  =ahaŋ ]    ɛtɛ



hyena  \textsc{ID}crawling  \textsc{NOM}{}-run-CL  thing  =3S.POSS  also



‘The hyena, barely escaping, ran home also.’


At the most vivid moments of a discourse, an ideophone can carry the morphosyntactic features of a verb. As a verb the ideophone takes verbal extensions and non-subject pronominals and syntactically fills the verb slot in the verb phrase. Semanticallhy, the main event in a clause is expressed by the ideophone. For example, the ideophone \textit{mək }‘positioning [self] for throwing’ in line 14 of the Snake story (ex. 238) carries the verbal extensions \textit{=ava  }‘in’ and\textit{ =alaj  }‘away.’ Also, the ideophone \textit{təx}\textbf{\textit{ }}‘put on head’ in lines 26 and 27 of the Cicada story (ex. 239) carries the verbal pronominal \textit{an=aŋ}\textit{ }‘to it.’ Ex. 240 also shows an ideophone with the direct object verbal extension \textit{na}.


Snake S. 14


\ea
\textbf{[Mək}  ava  alay. ]  
\z

\textbf{[mək}          =ava   =alaj ]  



\textsc{ID}position [self] for throwing  =in    =away 



‘[I] positioned [myself] \textit{muk}!’



Cicada S. 26


\ea
Albaya  ahay  weley  \textbf{[təh  }anan  dəray  na, ]  abay.
\z

albaja   =ahaj   wɛlɛj\textbf{  [təx       }an=aŋ \textbf{  }      dəraj   na, ]  abaj


youth    =Pl    which   \textsc{ID}put on head   DAT=3S.IO   head   PSP   \textsc{EXT} \textsc{NEG}


‘No one could lift it.’ (lit. whichever young man put his head to the tree (in order to lift it), there was none) 


In an exhortation, the major points may be made more vivid by the use of ideophones. Ex. 240 expresses a major point in the Values exhortation (see \sectref{sec:1.7}). See also ex. 230, 231 which also display this device.


Values S. 22


\ea
Təta  [dəl  na\textbf{,}  ma  Hərmbəlom  nendəye.]  
\z

təta   [dəl     na   ma   Hʊrmbʊlɔm   nɛndijɛ ]  



3P  \textsc{ID}insults  3S.DO  word  God    DEM



‘They insult it, this word of God!’ 


In some cases, the ideophone is the only element in the clause. For example, in the peak episode of the Snake story (lines 8-18, see \sectref{sec:1.4} for the entire text), ideophones are found within many of the clauses. The narrator tells that he took his flashlight, shone it up \textit{tsilar}, saw something \textit{mbadzak}\textbf{\textit{ }}\textit{mbadzak}\textbf{\textit{ }}(big and reflective), a snake! He \textit{mbət} turned off his light, \textit{kaluw} took his spear, \textit{mək} (positioned himself). Penetration \textit{mbəraɓ }! It fell \textit{ⱱ}\textit{aɓ} on the ground. Note that at the climactic moment (line 14, ex. 241), the entire clause is expressed by a single ideophone \textit{mək}, followed by verbal extensions.


S. 14


\ea
\textbf{Mək}  ava  alay.     
\z

\textbf{mək        }  =ava   =alaj   



\textsc{ID}position for throwing  =in    =away 



‘[I] positioned myself \textit{muk}!’


Likewise, in the peak episode of the Cicada text (sentences 25-29, see \sectref{sec:1.6}) ideophones are frequent and at the climactic moment (ex. 242), the ideophone is the only element in the clause. The cicada and young men go \textit{ʃɛŋ }\textbf{\textit{ }}to the tree to to move it. None of the young men \textbf{\textit{t}}\textit{əx }(put) their head to the tree. Then the cicada \textbf{\textit{t}}\textit{əx }(put) his head to the tree. \textit{Kuwna }(he got it)!\textit{ Dɛrg}\textit{\textsuperscript{w}}\textit{ɛtʃɛk}~(he lifted it to his head). In line S. 26 the ideophone \textit{təx} takes the place of the verb in the main clause and in lines S. 28 and 29 the ideophone is the only element in the clause. The entire event in each of those lines is thus expressed by that one word. 


S. 28


\ea
Kəwna.
\z

\textbf{\textit{kuwna}}\textbf{   }


\textsc{ID}getting  



‘[He] got [it].’ 



S. 29


\ea
Dergwecek.~
\z

\textbf{\textit{dɛrg}}\textbf{\textit{\textsuperscript{w}}}\textbf{\textit{ɛtʃɛk}}\textbf{~}


\textsc{ID}lifting onto head



‘[He] lifted [it] onto [his] head.’


\subsection{Clauses with zero transitivity}
\hypertarget{RefHeading1211301525720847}{}
Chapter 9 discusses the semantics of Moloko verbs for different numbers of core grammatical relations. Moloko verbs can have from zero to four grammatical relations, three of which can be coded as part of the verb complex.  Similarly, in clauses where ideophones fill the verb slot, the clause can have from zero to three explicit grammatical relations. The cases where the ideophone clause requires no explicit grammatical relations is a most interesting situation. The clause displays a grammatical transitivity of zero, even though it expresses a semantic event with participants. The use of ideophones makes the moment vivid and draws the listener into the story as if it was present before him/her so that the hearer can see and hear and participate in what is going on. This is a narrative device found in Moloko peak episodes.

Ideophones make up the entire clause in lines S. 28 and 29 at the peak of the Cicada text (example 242 above). On hearing the ideophones \textit{kuwna} and \textit{dɛrg}\textit{\textsuperscript{w}}\textit{ɛtʃɛk}~, the hearer knows that someone has gotten a hold of something, and then lifted it up onto his head to carry it. Two participants are understood, but the actual number of grammatical relations in the clauses is zero. The hearer must infer from the context that it was the cicada (the unexpected participant) who was doing the lifting and carrying. The cicada being so small, the people actually watching the event would not know for sure who was moving the tree either, since it would look like the tree was moving all by itself. Thus the use of ideophones with zero grammatical relations contributes to the visualisation of the story and makes the listener more of an actual participant in the events of the story. 

Likewise, in line S. 21 of the Disobedient Girl story (ex. 244) the clause has no expressed subject, direct or indirect object. The verb /h/ is in nominalised form with no pronominals to indicate participants. Likewise, if a Moloko person hears the ideophone \textit{nzuw nzuw}, he or she knows that someone is grinding something. In the context of the story, the woman is grinding millet, but the millet is expanding to fill the room and eventually will crush the woman. The clause only gives a picture/sound/idea of grinding with gaps in knowledge that the listener must work to fill in for himself, such as who is grinding whom. The listener is thus drawn into the story and made to be a participant in the event, creating vividness.


S. 21



\ea
Njəw  njəw  njəw  aməhaya  azla.
\z

nzuw  nzuw  nzuw           amə-h    =aja     aɮa



\textsc{ID}grind                            DEP-grind  =\textsc{PLU}    now



\textit{Nzu, nzu, nzu }[she] ground [the millet] now.   (lit. \textit{nzu nzu nzu} to grind now)


A third example is found in the Snake story. In lines S. 14 and 15, both the ideophone clause (line 14) and the nominalised form plus ideophone (line 15) have zero grammatical relations (ex. 245). The speaker is making both himself and the snake ‘invisible’ at this peak moment of his story. The effect would be to allow the hearer to imagine himself there right beside the speaker in the darkness, wondering where the snake was, hearing only the sounds of the events. 


S. 14


\ea
Mək  ava  alay.  
\z

mək            =ava   =alaj  



\textsc{ID}take position for throwing  =in    =to   



‘[He] positions himself for throwing [the spear].’



  S. 15



\textit{Mecesle  mbəra}\textit{ɓ.}



\textit{mɛ-t}\textit{ʃ}\textit{ɛ}\textit{ɬ}\textit{{}-ɛ           mbəra}\textit{ɓ}



\textsc{NOM}{}-penetrate-CL    \textsc{ID}penetrate 



‘[The spear] penetrates [the snake].’


\section{Interjections}
\hypertarget{RefHeading1211321525720847}{}
Interjections can form a clause of their own (ex. 246 and 247) or function as a kind of ‘audible’ pause while the speaker is thinking (ex. 248). They can also occur before or after the clause in an exclamation construction (see \sectref{sec:11.5}). Note that some interjections can be reduplicated for emphasis (compare ex. 247 and 249).


\ea
məf 
\z

məf 



 ‘get away! (to put off an animal or a child from continuing to do an undesirable action).’


\ea
\textbf{təd}\textbf{e}       
\z

\textbf{tɪdɛ}       



 ‘good’    


\ea
apazan  nəmənjar,  \textbf{andakaj},  Hawa.
\z

apazaŋ  nə-mənzar    \textbf{andakaj}     Hawa



yesterday  1S.PFV-see    what’s her name    Hawa



‘Yesterday I saw . . . what’s her name . . . Hawa.’ 


\ea
t\textbf{ə}t\textbf{ə}de
\z

‘very good’


\chapter[Noun]{Noun}
\hypertarget{RefHeading1211341525720847}{}
A Moloko noun functions as the head of a noun phrase. A noun phrase can serve as an argument within a clause.  The most prototypical nouns are “those denoting concrete, physical, compact entities made out of durable, solid matter”\footnote{Givón, 1984: 51.} but the class extends also to include a range of more abstract concepts. The morphosyntactic criteria for identifying a noun in Moloko include: 

\begin{itemize}
\item They can be pluralised, taking the plural \textit{=ahaj} (see \sectref{sec:34}).
\end{itemize}

\ea
məze  ahay\footnote{The first line in each example is the orthographic form. The second is the phonetic form (slow speech) with morpheme breaks. }  
\z

mɪʒɛ =ahaj  



person   =Pl 



‘people’


\ea
ayah  ahay
\z

ajax    =ahaj



squirrel  =Pl



‘squirrels’


\begin{itemize}
\item They can take a possessive pronoun (see \sectref{sec:14}).
\end{itemize}

\ea
hor  əwla  
\z

h\textsuperscript{w}ɔr    =uwla  



woman  =1S.POSS



‘my wife’


\ea
slərele  ango
\z

ɬɪrɛlɛ  =aŋg\textsuperscript{w}ɔ



work  =2S.POSS



‘your work’


\begin{itemize}
\item They can be counted (see \sectref{sec:21}).
\end{itemize}

\ea
gəvah  bəlen
\z

gəvax  bɪlɛŋ



field  one



‘one field’


\ea
sla  ahay  kəro
\z

ɬa    =ahaj  kʊrɔ



cow    =Pl  ten



‘ten cows’


\begin{itemize}
\item They can be modified by a demonstrative (see Chapter 18).
\end{itemize}

\ea
war  nehe
\z

war    nɛhɛ



child  DEM



‘this child’


\ea
ma    ndana
\z

word  DEM



‘that word’ (just spoken)


\begin{itemize}
\item They can take the derivational enclitic \textit{ga}  resulting in a derived adjective (Chapter 5.3).
\end{itemize}

\ea
gədan  ga
\z

gədaŋ  ga



strength  ADJ



‘strong’


\ea
ɓərav    ga
\z

heart  ADJ



‘perseverant’


\begin{itemize}
\item They can be modified by a derived adjective (see Chapter 4.3).
\end{itemize}

\ea
memele  malan  ga
\z

mɛmɛlɛ  malaŋ    ga



tree    greatness  ADJ



‘a large tree’


\ea
yam  pəyecece  ga
\z

jam    pijɛtʃɛtʃɛ    ga



water  coldness    ADJ



‘cold water’


\begin{styleTextbodyindent}
Moloko nouns (or noun phrases) carry no overt case markers themselves; the function of the various noun phrases in a clause is indicated by the word order in the clause, pronominal marking in verbs (see \sectref{sec:7.3}), and adpositions (Chapter 5.6). 
\end{styleTextbodyindent}

\section{Phonological structure of the noun stem}
\hypertarget{RefHeading1211361525720847}{}
\citet{Bow1997c} studied syllable patterns in nouns. She found many nouns that are CVC but very few that are CV. However, many CVCV nouns actually contain a reduplicated syllable, ex. 262 - 264.


\ea
dede 
\z

dɛdɛ 



 ‘grandmother’


\ea
sese
\z

ʃɛʃɛ



\textit{ }‘meat’


\ea
baba 
\z

‘father’ 


%%please move \begin{table} just above \begin{tabular
\begin{table}
\caption{(from Bow, 1997c) shows examples of one- to three-syllable noun words of each possible syllable pattern, with and without labialisation and palatalisation prosodies. Syllable pattern is independent of prosody.}
\label{tab:24}
\end{table}

\begin{tabular}{lllllll} & \textbf{Neutral  } & \textbf{Gloss} & \textbf{Labialised} & \textbf{Gloss} & \textbf{Palatalised} & \textbf{Gloss}\\
\lsptoprule
\textbf{CV} & \textit{ɬa} & ‘cow’ &  &  &  & \\
\textbf{CVC} & \textit{fat} & ‘day/sun’ & \textit{h}\textit{\textsuperscript{w}}\textit{ɔɗ} & ‘stomach’ & \textit{dʒɛŋ} & ‘chance’\\
\textbf{V.CV} & \textit{ava} & ‘arrow’ & \textit{ɔk}\textit{\textsuperscript{w}}\textit{ɔ} & ‘fire’ & \textit{ɛlɛ} & ‘eye’\\
\textbf{V.CVC} & \textit{ahar} & ‘hand/arm’ & \textit{ɔtɔs} & ‘hedgehog’ & \textit{ɛnɛŋ} & ‘snake’\\
\textbf{CV.CV} & \textit{ɡala} & ‘yard’ & \textit{sɔnɔ} & ‘joke’ & \textit{dʒɛrɛ} & ‘truth’\\
\textbf{CV.CVC      } & \textit{mavaɗ} & ‘sickle’ & \textit{tɔh}\textit{\textsuperscript{w}}\textit{ɔr} & ‘cheek’ & \textit{pɛmbɛʒ} & ‘blood’\\
\textbf{V.CV.CV} & \textit{adama} & ‘adultery’ & \textit{ɔɓɔlɔ} & ‘yam’ & \textit{ɛtɛmɛ} & ‘onion’\\
\textbf{V.CV.CVC} & \textit{adaŋɡaj} & ‘stick’ & \textit{ɔmbɔɗɔts} & ‘sugar cane’ & \textit{ɛmɛlɛk} & ‘bracelet’\\
\textbf{CV.CV.CV} & \textit{manzara} & ‘termite’ & \textit{mɔzɔŋɡɔ} & ‘chameleon’ & \textit{ʒɛtɛnɛ} & ‘salt’\\
\textbf{CV.CV.CVC} & \textit{maɬalam} & ‘sword’ & \textit{dɔlɔkwɔj} & ‘syphilis’ & \textit{dɛbɛʒɛm} & ‘jawbone’\\
\lspbottomrule
\end{tabular}

\begin{itemize}
\item \begin{styleTabletitle}
 Syllable patterns in noun stems with different prosodies
\end{styleTabletitle}\end{itemize}

There are many Moloko nouns whose first syllable is V. This syllable may be historically an old /a-/ prefix. Nouns with these /a-/ prefixes can only be discovered by comparing Moloko vocabulary with that of other related languages where the nouns do not carry the prefix. \tabref{tab:25}. illustrates three nouns in Moloko and in Mbuko.\footnote{\citet{Mbuagbaw1995}, \citet{Gravina2001}. Judging from the number of nouns in the Moloko database that begin with m, there may be some kind of an old \textit{m-}prefix.  } 

\begin{tabular}{lll}
\lsptoprule

\textbf{Moloko} & \textbf{Mbuko} & \textbf{Gloss}\\
\textit{anzakar} & \textit{nzakar} & ‘chicken’\\
\textit{azʊŋɡʷɔ} & \textit{zʊŋɡʷɔ} & ‘donkey’\\
\textit{ɛtɛmɛ } & \textit{tɛmɛ} & ‘onion’\\
\lspbottomrule
\end{tabular}
\begin{itemize}
\item \begin{styleTabletitle}
/a/ prefix in Moloko compared with Mbuko
\end{styleTabletitle}\end{itemize}

\citet{Bow1997c} discovered that tonal melodies on nouns are different than for verbs (see \sectref{sec:6.7} for verb tone melodies). \tabref{tab:26}. (from Bow, 1997c) shows how the underlying tone melodies are realised on the surface in one, two, and three syllable nouns.  The left column gives examples with no depressor consonants (see \sectref{sec:7}), and the right column contains nouns with depressor consonants which effect different tone melodies. For one syllable nouns, only two tonal melodies are possible (H or L). For two syllable nouns, H, L, HL, or LH are possible. For three syllable nouns, H, L, HL, LH, HLH, and LHL are possible. Note that a surface mid tone can result from two sources. It can be an underlying high tone that has been lowered by a preceding low tone\footnote{Therefore there are no surface LH combinations since an underlying LH will be realised as LM.} or it can be an underlying low tone in a word with no depressor consonants.\footnote{There are also very few examples of ML combinations in the surface form. The only example was [\textit{kɪmɛdʒɛ}], an underlying LHL that had depressor consonants.} 

\begin{tabular}{lllllll} & \multicolumn{3}{l}{\textbf{No depressor consonants}} & \multicolumn{3}{l}{\textbf{Depressor consonants present}}\\
\lsptoprule
\textbf{Underlying tonal melody} & \textbf{Surface tone} & \textbf{Phonetic transcription} & \textbf{Gloss} & \textbf{Surface tone} & \textbf{Phonetic transcription} & \textbf{Gloss}\\
H & H & [tsáf] & ‘shortcut’ & H & [záj] & ‘peace’\\
& HH & [tʃɛtʃɛ] & ‘louse’ & HH & [bɔɮɔm] & ‘cheek’\\
& HHH & [mɔlɔkʷɔ] & ‘Moloko’ & HHH & [dəndárá] & ‘lamp’\\
L & M & [ɗ\={a}f] & ‘loaf’ & L & [ɡàr] & ‘difficulty’\\
& MM & [kər\={a}] & ‘dog’ & LL & [dàndàj] & ‘intestines’\\
& MMM & [mɪtɛnɛŋ] & ‘bottom’ & LLL & [àdàŋɡàj] & ‘stick’\\
HL & HM & [mɛkɛtʃ] & ‘knife’ & HL & [dʒɛrɛ] & ‘truth’\\
& HMM & [átʊkʷɔ] & ‘okra’ & HLL & [mɔɡʷɔdɔkʷ] & ‘hawk’\\
& HHM & [mɔsɔkʷɔj] & ‘vegetable sauce’ & HHL & [ázʊŋɡʷɔ] & ‘donkey’\\
LH & MH & [ɬəmáj] & ‘ear/name’ & LM & [bɔɡʷɔm] & ‘hoe’\\
& MMH & [kɪtɛfɛr] & ‘scoop’ & LLM & [ɡəɡəm\={a}j] & ‘cotton’\\
& MHH & [\={a}mɛlɛk] & ‘bracelet’ & LMH & [ɡɛmbɪrɛ] & ‘dowry’\\
HLH & HMH & [ákʊfɔm] & ‘mouse’ & HLM & [dɛdɪlɛŋ] & ‘black’\\
LHL & MHM & [səsáj\={a}k] & ‘wart’ & LML & [kɪmɛdʒɛ] & ‘clothes’\\
&  &  &  & MHL & [məŋɡáhàk] & ‘crow’\\
\lspbottomrule
\end{tabular}
\begin{itemize}
\item \begin{styleTabletitle}
Tonal melodies on nouns
\end{styleTabletitle}\end{itemize}
\section{Morphological structure of the noun word}
\hypertarget{RefHeading1211381525720847}{}
Moloko noun words are morphologically simple compared with verbs.  A noun can be comprised of just a noun stem,\footnote{We refer to the simplest form as a stem because it can be more complex than a root in that it can have an /a-/ prefix.} a compound noun, or a nominalised verb. 

A noun stem can consist of a simple noun root (ex. 265) or two reduplicated segments (ex. 266). These reduplicated elements actually form two separate phonological words (note the word-final alteration \textit{ŋ} in both segments) but are lexically one item.\footnote{Because there are word-final consonant changes for only /n/ and /h/, it is not known whether all similar reduplications necessarily form two separate phonological words. } 


\ea
hay  
\z

hàj  



 ‘house’


\ea
ndən nden 
\z

ndəŋ ndɛŋ 



 ‘traditional sword’


Nouns can be derived from verbs by a potentially complex process where a prefix, a suffix, and palatalisation are added. The prefix is \textit{m}\textit{ɪ}\textit{{}-}\textit{ }or \textit{mɛ -}, depending on whether the verb has the \textit{/}\textit{a-}\textit{/} prefix or not. The suffix is \textit{{}-ij}\textit{ɛ}\textit{ }or\textit{ -ɛ}, depending on whether the verb root has one or more consonants. The suffix carries palatalisation which palatalises the whole word. The resulting form is an abstract noun which cannot take the plural \textit{=ahaj} but which otherwise has all the characteristics of a noun. This highly productive process is discussed further in \sectref{sec:7.6} but two nominalisations are shown here. In ex. 267 and 268, the underlying form, the 2S imperative, and the nominalised form are given. Ex. 267 is a one-syllable verb with no prefix, and so takes the prefix \textit{mɪ-} and the suffix \textit{{}-ijɛ}. Ex. 268 shows a two consonant root with /a-/ prefix that takes the prefix \textit{mɛ-} and the suffix \textit{{}-ɛ}.

\ea
\textup{/ v}\textup{\textsuperscript{ e}}\textup{ /  v-ɛ      mɪ-v-ijɛ}
\z

pass [2S.IMP]-CL  \textsc{NOM}{}-pass-CL  



‘Pass!’ (spend time)  ‘year’ (lit. passing of time)


\ea
\textup{/a-m l-aj/    məl-aj      mɛ-mɪl-ɛ}
\z

rejoice[2S.IMP]-CL  \textsc{NOM}{}-rejoice-CL



‘Rejoice!’    ‘joy’


Another nominalisation process can be postulated when noun stems and verb roots are compared. This second nominalisation process is irregular and non-productive. \tabref{tab:27}. illustrates a few examples and compares verb roots with their counterpart regular and irregular nominalisations. In each case, the consonants in the nouns in both nominalised forms are the same as those for the underlying verb root. These data show that in the irregular set of nominalisations, there is no set process of nominalisation – in some cases an \textit{a-} prefix is added (see lines 1 and 2); in other cases the prosody is changed to form the irregular nominalised form (from palatalised to neutral in line 4, from neutral to palatalised in lines 3, 5, and 6). 

When the irregular nominalisations are compared with the regular nominalised form in \tabref{tab:27}., it can be seen that the two types of nouns relate to the sense of the verbs in different ways. The regular nominalisation refers to the event or the process itself (stealing, carrying, sending, etc.), whereas the irregular nominalisation denotes some kind of a referent involved in the event (thief, work, hand, etc.). 

\begin{tabular}{lllll}
\lsptoprule

\textbf{Line} & \textbf{Underlying form of verb root} & \textbf{2S imperative} & \textbf{Regular nominalisation} & \textbf{Irregular nominalisation}\\
1 & \textit{/k r/} & kar-aj

\itshape \textup{‘Steal!’} & mɪ-kɛr-ɛ

\itshape \textup{‘stealing’} & \textit{akar}

‘thiefˋ\\
2 & \textit{/h r/} & \textit{har}

‘Carry by hand!’ & \textit{mɪ-hɪr-ɛ}

‘carrying’ & \textit{ahar}

‘hand’\\
3 & \textit{/h r ɓ/} & \textit{hʊrɓ-ɔj}

‘Heat up!’ & \textit{mɪ-hɪrɓ-ɛ}

‘heating’ & \textit{hɛrɛɓ}

‘heat’\\
4 & \textit{/t w/} & \textit{tuw-ɛ}

‘Cry!’ & \textit{mɪ-tuw-ɛ}

‘crying’ & \textit{tuwaj}

‘cry’\\
5 & \textit{/ɬ r/} & \textit{ɬar}

‘Send!’ & \textit{mɪ-ɬɪr-ɛ}

‘sending’ & \textit{ɬərɛlɛ}

‘work’\footnotemark{}\\
6 & \textit{/dz n/} & \textit{dzən-aj}

‘Help!’ & \textit{mɪdʒɛnɛ}

‘helping’ & \textit{dzeŋ}

‘luck’\\
\lspbottomrule
\end{tabular}
\footnotetext{ Probably a compound of \textit{ɬar }‘send/commission’ \textit{+ }\textit{ɛlɛ} ‘thing’ (\sectref{sec:4.3}).}

\begin{itemize}
\item \begin{styleTabletitle}
 Derived nouns
\end{styleTabletitle}\end{itemize}

Two processes denominalise nouns; one forms adjectives (Chapter 4.3) and the other, adverbs (see \sectref{sec:26}). It is not possible to derive a verb from a noun root or stem in Moloko.

\subsection{Subclasses of nouns}
\hypertarget{RefHeading1211401525720847}{}
There are no distinct morphological noun classes in Moloko.  Those nouns with an /a-/ prefix could perhaps be considered a separate class (see \sectref{sec:4.1}), but this phenomenon is more of an interesting historical linguistic phenomenon rather than a marker of synchronically different Moloko noun classes. There appears to be no phonological, grammatical or semantic reason for the prefix or other consequences of the presence versus absence of /a-/. 

The plural construction is discussed in \sectref{sec:33.} Relative to pluralisation patterns, Moloko has four subclasses of nouns. These are concrete nouns (\sectref{sec:34}), mass nouns (\sectref{sec:35}), abstract nouns (\sectref{sec:36}), and irregular nouns (\sectref{sec:37}). The subclasses are distinguished by whether and how they become pluralised. 

\subsection{Plural construction}
\hypertarget{RefHeading1211421525720847}{}
Noun plurals are formed by the addition of the clitic =\textit{ahaj} which follows the noun or the possessive pronoun.{ }The plural clitic carries some features of a separate phonological word and some of a phonologically bound morpheme. The neutral prosody of =\textit{ahaj} does not neutralise the prosody of the word to which it cliticises (ex. 269, 270), which would indicate a separate phonological word (see \sectref{sec:11}). 


\ea
\textup{/atama}\textup{\textsuperscript{e}}\textup{ =ahj/    [F0AE?]   [ɛtɛmɛhaj]}
\z

onion     =Pl           ‘onions’


\ea
\textup{/akfam}\textup{\textsuperscript{o}}\textup{  =ahj/ }    [F0AE?]  \textup{[}\textup{ɔk}\textup{\textsuperscript{w}}\textup{fɔmahaj]}
\z

mouse      =Pl        ‘mice’


Two types of word-final changes indicate that the plural is phonologically bound to the noun. First, word-final changes for /h/ that demonstrate a word break do not occur between a noun and the plural (ex. 251). 

Second, the stem-final deletion of /n/ before the /=ahj/ (shown in \tabref{tab:28}. adapted from Bow 1997c) indicates that the plural is phonologically bound to the noun (criterion e in \sectref{sec:11}). 

\begin{tabular}{lllll} & \textbf{Underlying form} & \textbf{Surface form} &  & \textbf{Gloss}\\
\lsptoprule
\textbf{Neutral } & /ɡ s n/ & [ɡəsaŋ]   [ahaj]   [F0AE?]

‘bull’        ‘Pl’ & [ɡəsahaj] & ‘bulls’\\
\textbf{Labialised} & /t la l n \textsuperscript{o}/ & [tʊlɔlɔŋ] [ahaj]   [F0AE?]

‘heart’      ‘Pl’ & [tʊlɔlɔhaj] & ‘hearts’\\
\textbf{Palatalised} & /da d n \textsuperscript{e}/ & [dɛdɛŋ]   [ahaj]   [F0AE?]

‘truth’       ‘Pl’ & [dɛdɛhaj] & ‘truths’\\
\lspbottomrule
\end{tabular}

\begin{itemize}
\item \begin{styleTabletitle}
Word-final changes of /n/ between noun and plural clitic
\end{styleTabletitle}\end{itemize}

We consider the plural marker to be a type of clitic and not an affix\footnote{\citet{Bow1997c} considered the plural marker to be an affix. } because it does show some evidence of phonological attachment and because it binds to words of different grammatical classes in order to maintain its position at the right edge of the noun phrase permanent attribution construction (see \sectref{sec:43}). The plural  =\textit{ahaj} will cliticise to a noun (ex. 271), possessive pronoun (ex. 272, 273), or pronoun in a permanent attribution construction (ex. 285 in \sectref{sec:5.1}).  The plural modifies the entire construction in a permanent attribution construction. 


\ea
\textup{/ɓ r ɮ n =ahj/        [F0AE?]   [ɓərɮahaj]}\textup{ }
\z

mountain     =Pl          ‘mountains’


\ea
\textup{/g l n        ah n          =ahj/    [F0AE?]   [gəlahahaj]}
\z

kitchen     =3S.POSS     =Pl      ‘his/her kitchens’


\ea
\textup{/plas}\textup{\textsuperscript{e}}\textup{   ah n       =ahj/  [F0AE?]  [pəlɛʃahahaj]}
\z

horse   =3S.POSS    =Pl    ‘his horses’


Note that in adjectivised noun phrases, other constituents must also be pluralised (examples 322 - 324 from Chapter 5.3)

\subsection{Concrete nouns}
\hypertarget{RefHeading1211441525720847}{}
Concrete nouns (see \tabref{tab:29}.) occur in both singular and plural constructions. The plural of these nouns is formed by the addition of the plural clitic \textit{=ahaj}  within the noun phrase, following the head noun (further discussed in \sectref{sec:5.1}). Concrete nouns can also take numerals. 

\begin{tabular}{lll}
\lsptoprule

\textbf{Singular} & \textbf{Plural}\footnotemark{} & \textbf{Plural with numeral}\\
\textit{anzakar}

‘chicken’ & \textit{anzakar}\textit{ }\textit{=ahaj}

‘chickens’ & \textit{anzakar}\textit{ }\textit{=ahaj ɮɔm}

‘five chickens’\\
\textit{ɬəmaj}

‘ear,’ ‘name’ & \textit{ɬəmaj =ahaj}

‘ears’/’names’ & \textit{ɬəmaj =ahaj tʃew}

‘two ears’/’two names’\\
\textit{dzɔg}\textit{\textsuperscript{w}}\textit{ɔ}

‘hat’ & \textit{dzɔg}\textit{\textsuperscript{w}}\textit{ɔ =ahaj}

‘hats’ & \textit{dzɔg}\textit{\textsuperscript{w}}\textit{ɔ =ahaj makar}

‘three hats’\\
\textit{albaja}

‘young man’ & \textit{albaja}\textit{ }\textit{=ahaj}

‘young men’ & \textit{albaja}\textit{ }\textit{=ahaj}\textit{ }\textit{kʊrɔ}

‘ten young men’\\
\textit{d}\textit{ɛ}\textit{d}\textit{ɛ}

‘grandmother’ & \textit{d}\textit{ɛ}\textit{d}\textit{ɛ}\textit{ }\textit{=ahaj}

‘grandmothers’ & \textit{d}\textit{ɛ}\textit{d}\textit{ɛ}\textit{ }\textit{=ahaj}\textit{ }\textit{mʊk}\textit{\textsuperscript{w}}\textit{ɔ}

‘six grandmothers’\\
\lspbottomrule
\end{tabular}
\footnotetext{ Resyllabification occurs with the addition of plural marker. It is the same resyllabification that occurs at the phrase level \citep{Section1110}.}

\begin{itemize}
\item \begin{styleTabletitle}
 Concrete plural
\end{styleTabletitle}\end{itemize}
\subsection{Mass nouns}
\hypertarget{RefHeading1211461525720847}{}
Mass nouns (shown in \tabref{tab:30}.) are non-countable – the singular form refers to a collective or a mass, e.g. \textit{jam} ‘water.’  These nouns, when pluralised, refer to different kinds or varieties of that noun referent. These nouns cannot take numerals but they can be quantified (see \sectref{sec:24}). 

\begin{tabular}{ll}
\lsptoprule

\textbf{Singular} & \textbf{Plural}\\
\textit{jam}

‘water’ & \textit{jam =ahaj}

‘waters’ (in different locations)\\
\textit{ʃ}\textit{ɛ}\textit{ʃ}\textit{ɛ}

meat & \textit{ʃɛʃɛ =ahaj}

‘meats’ (from different animals)\\
\textit{agwødʒɛr}

‘grass’ & \textit{ag}\textit{\textsuperscript{w}}\textit{ød}\textit{ʒ}\textit{ɛr =ahaj}

‘grasses’ (of different species)\\
\lspbottomrule
\end{tabular}

\begin{itemize}
\item \begin{styleTabletitle}
 Mass noun plural
\end{styleTabletitle}\end{itemize}
\subsection{Abstract nouns}
\hypertarget{RefHeading1211481525720847}{}
Abstract nouns are ideas or concepts and as such they are not ‘singular’ or ‘plural.’ In Moloko they do not take \textit{=ahaj}\textit{, }e.g., \textit{fama} ‘intelligence, cleverness,’ \textit{ɬɪrɛlɛ} ‘work.’ Although they cannot be pluralised, they can be quantified (see \sectref{sec:24}). 

\subsection{Irregular nouns}
\hypertarget{RefHeading1211501525720847}{}
Three nouns, all of which refer to basic categories of human beings, have irregular plural forms in that the noun changes in some way when it is pluralised. The singular and plural forms for these nouns are shown in \tabref{tab:31}.. For \textit{h}\textsuperscript{w}\textit{ɔr} ‘woman’ and \textit{zar} ‘man,’ the plural forms resemble the singular but involve insertion of the consonant \textit{w} (\textit{hawər }and \textit{zawər}, respectively). For \textit{war} ‘child’ the plural form is completely suppletive (\textit{babəza}). For each of these three items, there is an alternate plural form which is formed by reduplicating the entire plural root. This alternate form is interchangeable with the corresponding irregular plural form.

\begin{tabular}{lll}
\lsptoprule

\textbf{Singular} & \textbf{Plural} & \textbf{Alternate plural form}\\
\textit{h}\textit{\textsuperscript{w}}\textit{ɔr}

‘woman’ & \textit{hawər =ahaj}

‘women’ & \textit{hawər hawər}

‘women’\\
\textit{zar}

‘man’ & \textit{zawər =ahaj}

‘men’ & \textit{zawər zawər}

‘men’\\
\textit{war}

‘child’ & \textit{babəza =ahaj}

‘children’ & \textit{babəza babəza}

‘children’\\
\lspbottomrule
\end{tabular}

\begin{itemize}
\item \begin{styleTabletitle}
Irregular plurals
\end{styleTabletitle}\end{itemize}
\section{Compounding}
\hypertarget{RefHeading1211521525720847}{}
In a language like Moloko where individual words meld together in normal speech, real compounds are difficult to identify, since two separate nouns can occur together juxtaposed within a noun phrase without a connecting particle (see \sectref{sec:43}). In general, if what might seem to be a compound phonologically can be analysed as separate words in a productive syntactic construction, we interpret them as such. We have found some genuine compound noun stems in Moloko (see also \sectref{sec:4.1}), and proper names are often lexicalised compounds that in terms of their internal structure are structurally like phrases or clauses (\sectref{sec:4.4}). 

The grammatical and phonological criteria used to identify a compound are fourfold:

\begin{itemize}
\item The compound patterns as a single word in whatever class it belongs to, instead of as a phrase (that is, in terms of its outer distribution),
\item    The compound is seen as a unit in the minds of speakers,
\item The compound has a meaning that is more specific than the semantic sum of its parts,
\item The compound exhibits no word final phonological changes that would necessitate more than one phonological word (see \sectref{sec:2.6}); for example, there are no word final changes ([ŋ] and [x]) and prosodies spread over the entire compound.
\end{itemize}
%%please move \begin{table} just above \begin{tabular
\begin{table}
\caption{shows several compounds made from \textit{ɛlɛ} ‘thing,’ placed both before and after another root. The compounds in the table illustrate that compounds can be made from a noun plus another noun root (lines 1-3), or a noun plus a verb root (line 4). Note that when \textit{ɛlɛ} ‘thing’is in a compound, \textit{ɛlɛ}  loses its own palatalisation prosody, an indication that the roots comprise a phonological compound.}
\label{tab:32}
\end{table}

\begin{tabular}{lll}
\lsptoprule

\textbf{Line} & \textbf{Compound noun} & \textbf{Elements}\\
1 & alahar

\itshape \textup{‘weapon, bracelet’} & \textit{ɛlɛ    ahar} 

thing  hand\\
\textit{2} & \textit{ɔlɔk}\textit{\textsuperscript{w}}\textit{ɔ}

‘wood’ & \textit{ɛlɛ    ɔk}\textit{\textsuperscript{w}}\textit{ɔ }

thing  fire\\
\textit{3} & \textit{mɛmɛlɛ}

‘tree’ & \textit{mama ɛlɛ} 

mother  thing\\
\textit{4} & \textit{ɬ}\textit{ɪ}\textit{rɛlɛ}

‘work’ & \textit{ɬar    ɛlɛ }

send  thing\\
\lspbottomrule
\end{tabular}

\begin{itemize}
\item \begin{styleTabletitle}
 Compounds made with ɛlɛ ‘thing’
\end{styleTabletitle}\end{itemize}
%%please move \begin{table} just above \begin{tabular
\begin{table}
\caption{shows two compounds made with \textit{ma} ‘mouth’ or ‘language.’}
\label{tab:33}
\end{table}

\begin{tabular}{ll}
\lsptoprule

\textbf{Compound} & \textbf{Elements}\\
Mahaj

\itshape ‘\textup{door’} & \textit{ma     haj}

mouth house\\
\textit{maɬar }

‘front teeth’ & \textit{ma      aɬar} 

mouth  tooth\\
\lspbottomrule
\end{tabular}
\begin{itemize}
\item \begin{styleTabletitle}
 Compounds made with ma
\end{styleTabletitle}\end{itemize}

A more complex example is \textit{ajva} ‘inside-house.’ It could be analysed as /\textit{a haj ava}/ ‘in house in’; however it distributes not as a locative adpositional phrase, but rather as a noun, in that it can be possessed (ex. 274) and it can be subject of the verb /s/\textit{ }‘want’ (ex. 275).


\ea
Atərava  ayva  ahan.
\z

\textit{a-tər=ava   ajva     =ahaŋ}



3S-enter=in    inside house  =3S.POSS



He goes into his house.


\ea
Asan  ayva  bay.
\z

\textit{a-s=aŋ    ajva    baj}


3S-please=3S.IO  inside house  \textsc{NEG}



He doesn’t want [to go] inside the house. (lit. the inside of the house does not please him)


\section{Proper Names}
\hypertarget{RefHeading1211541525720847}{}
Moloko proper nouns (names of people, tribes, and places) can be morphologically simple but often are compounds. In the case of names for people, the names often indicate something that happened around the time of the baby’s birth.  Names can also be compounds that encode proverbs. Thus, proper names can be simple nouns, compounds, prepositional phrases, verbs, or complete clauses.  \tabref{tab:34}. illustrates some proper names that are compounds, and shows the components of the name where necessary. Lines 1-5 show simple proper names and lines 6-11 show proper names that are compounds. 

\begin{tabular}{lllll}
\lsptoprule

\textbf{Line} & \textbf{Name} & \textbf{Type of name} & \textbf{Components of name}

\textbf{(where applicable)} & \textbf{Meaning}\\
1 & \textit{Dz}\textit{ɛ}\textit{r}\textit{ɛ} & person &  & ‘truth’\\
2 & Gadzəlax & person &  & ‘broken piece of pottery’\\
3 & Ftak & person/village &  & (no meaning outside its name)\\
4 & Mɔk\textsuperscript{w}ijɔ & village &  & (no meaning outside its name)\\
5 & Maɬaj & tribe &  & (no meaning outside its name)\\
6 & \textit{Mʊl}\textit{ɔ}\textit{k}\textit{\textsuperscript{w}}\textit{ɔ} & language & \textit{ma           al}\textit{ɔ}\textit{k}\textit{\textsuperscript{w}}\textit{ɔ} 

language, 1\textsc{Pin}.POSS & ‘our language’ (Moloko)\\
7 & Anzakijma & person & \textit{a-nzak-aj ma} 

3S-find{}-CL     word & ‘here comes trouble’\\
8 & Kɔsijmɪzɛ & person & \textit{kɔs-aj                 mɪʒɛ }

unite[2S.IMP]-CL people & ‘he unites the people’\\
9 & Kavijaka & person & \textit{kə avija        aka }

on sufferring on & ‘in suffering’\\
10 & Aŋgaɗaj & person & \textit{a-ngaɗ-aj}

3S-rejoice-CL & ‘he is joyful’\\
11 & Mərijabaj & person & \textit{məraj   abaj }

shame  \textsc{EXT} \textsc{NEG} & ‘no shame’\\
\lspbottomrule
\end{tabular}

\begin{itemize}
\item \begin{styleTabletitle}
 Proper names
\end{styleTabletitle}\end{itemize}

Twins are usually given special names according to their birth order, \textit{ }\textit{Masaj}\textit{ }‘first twin,’ \textit{Aluwa} ‘second twin.’  A single child after a twin birth is named \textit{Aban}.  

\chapter[Noun phrase]{Noun phrase}
\hypertarget{RefHeading1211561525720847}{}
Moloko, an SVO language, has head initial noun phrases.  Ex. 276 - 279 show a few examples of noun phrases.  A noun (\textit{nafat}  ‘day’ and \textit{l}\textit{ə}\textit{he} ‘bush’ in ex. 276), multiple nouns (\textit{war ɛlɛ haj} ‘millet grain’ in ex. 278 and \textit{war dalaj} ‘girl’ in ex. 279) or free pronoun (\textit{nɛ} ‘me’ ex. 277) is the head of the NP. In the examples in this chapter, the noun phrases are delimited by square brackets. 


\ea
{}[Nafat  enen ]  anday  atalay  a  [ləhe.]
\z

{}[\textit{nafat  ɛnɛŋ }]\textit{  a-ndaj    a-tal-aj    a  }[\textit{lɪhɛ} ]



day    another  3S-PRG    3S-walk{}-CL  at  bush



‘One day, he was walking in the bush.’


\ea
{}[Ne  ahan ]  aməgəye.
\z

{}[\textit{nɛ  =ahaŋ} ]\textit{    amɪ-g-ijɛ }



1S  =3S.POSS  DEP-do-CL



‘It was me (emphatic) that did it.’ 


\ea
Cəcəngehe  na,  [war  elé  hay  bəlen ]   na,  asak  asabay.
\z

tʃɪtʃɪŋgɛhɛ   na   \textit{[}war   ɛlɛ  haj   bɪlɛŋ \textit{]}   na   a-sak       asa-baj


now    PSP    child  eye  millet  one  PSP  3S.IFV-multiply    again-\textsc{NEG}



‘And now, one grain of millet, it doesn’t multiply anymore.’


\ea
Metesle  anga  [war  dalay  ngendəye.]  
\z

\textit{mɛ-tɛɬl-ɛ    aŋga    }[\textit{war    dalaj    ŋgɛndijɛ}.]  



\textsc{NOM}{}-curse  \textsc{POSS}  child  girl  DEM



‘The curse belongs to that young woman.’


In this chapter, noun phrase modifiers and the order of constituents will be discussed (see \sectref{sec:5.1}), using simple noun heads as examples. Then, noun heads are discussed (see \sectref{sec:5.2}). Next, derived adjectives are discussed, which consist of a noun plus the adjectiviser (see \sectref{sec:5.3}). Finally, four kinds of noun plus noun constructions are discussed, the genitive construction (see \sectref{sec:42}), the permanent attribution construction (see \sectref{sec:43}), relative clauses (see \sectref{sec:44}), and coordinated noun phrases (see \sectref{sec:5.5}).

Some things one might expect to see in a noun phrase are not found in Moloko noun phrases, but are accomplished by other constructions. For example, some attributions are expressed at the clause level using an intransitive clause (see \sectref{sec:9.2.1.1.1.1}) or transitive verb with indirect object (see \sectref{sec:66}), and comparison is done through an oblique construction (see \sectref{sec:45}). 

\section{Noun phrase constituents}
\hypertarget{RefHeading1211581525720847}{}\begin{styleTextbodyindent}
A noun head can be modified syntactically by the addition of other full-word or clitic elements. In the examples which follow, the noun phrases are delimited by square brackets. Examples are given in pairs, with the noun phrase in the first of each pair the direct object of the verb. In the second example of each pair, the noun phrase is the predicate in a predicate nominal construction (see \sectref{sec:70}). Note that most of the predicate nominal constructions require the presupposition marker \textit{na} (Chapter 12). The constituents being illustrated are bolded in each example. 
\end{styleTextbodyindent}

\begin{styleTextbodyindent}
A noun modified by the plural marker (ex. 280 and 281, see \sectref{sec:34}).
\end{styleTextbodyindent}


\ea
Nəmənjar  awak \textbf{ahay}. 
\z

nə-mənzar  [awak  \textbf{=ahaj} ]        



1S.IFV  goat  =Pl



‘I see goats.’


\ea
{}[Awak\textbf{  ahay}  na ],  [səlom  \textbf{ahay}  ga.]
\z

{}[awak   \textbf{=ahaj}     na ]   [sʊlɔm   \textbf{=ahaj}   ga ]



goat    =Pl  PSP  good  =Pl  ADJ



‘The goats [are] good.’


A noun modified by a possessive pronoun (ex. 282 and 283, see Sections 14 and 42).

\ea
Nəmənjar  [awak \textbf{ əwla.}]           
\z

\textit{nə-mənzar  }[\textit{awak   }\textbf{\textit{=uwla}}



1S.IFV  goat  =1S.POSS  



‘I see my goat.’


\ea
{}[Awak\textbf{  əwla}  na ],  [səlom  ga. ]
\z

{}[awak   \textbf{=uwla}     na ]   [sʊlɔm   ga ]



goat    =1S.POSS  PSP  good  ADJ



‘My goat [is] good.’


A noun modified by an unspecified pronoun (ex. 284 and 285, see \sectref{sec:17}).

\ea
Nəmənjar  [awak\textbf{  en}\textbf{en.} ]          
\z

nə-mənzar  [awak  \textbf{ɛnɛŋ} ]          



1S.IFV  goat  another



‘I see another goat.’


\ea
{}[Awak\textbf{  enen}  ahay  na ],  [səlom  ahay  ga.] 
\z

{}[awak   \textbf{ɛnɛ}  =ahaj   na ]   [sʊlɔm   =ahaj  ga ] 



goat    other  =Pl  PSP  good  =Pl  ADJ



‘Other goats [are] good.’   


A noun modified by a numeral (ex. 286 and 287, Chapter 3.3).

\ea
Nəmənjar  [awak  \textbf{=}əwla   ahay  \textbf{makar}\textbf{.} ]     
\z

nə-mənzar  [awak  =uwla    =ahaj  \textbf{makar} ]     



1S.IFV  goat  =1S.POSS  =Pl  three



‘I see my three goats.’


\ea
{}[awak  \textbf{=}əwla   ahay  \textbf{makar  ahay  }na ],  [səlom  ahay  ga.]
\z

{}[awak   =uwla     =ahaj   \textbf{makar}   \textbf{=ahaj}   na ]   [sʊlɔm   =ahaj   ga ]



goat    =1S.POSS  =Pl  three  =Pl  PSP  good  =Pl  ADJ



‘My three goats [are] good.’


A noun modified by a derived adjective (ex. 288 and 289, section 5.3).

\ea
Nəmənjar  [awak  ahay  \textbf{malan  ahay  ga.} ]       
\z

nə-mənzar  [awak  =ahaj  \textbf{malaŋ  =ahaj  ga} ]       



1S.IFV  goat  =Pl  great    =Pl  ADJ



‘I see the big goats.’


\ea
{}[awak  ahay  \textbf{malan  ahay  ga}  na],  [səlom  ahay  ga.]
\z

{}[awak  =ahaj  \textbf{malaŋ  =ahaj  ga}   na]   [sʊlɔm   =ahaj   ga ]



goat    =Pl  great    =Pl  ADJ  PSP   good  =Pl  ADJ



‘The big goats [are] good.’


A noun modified by a nominal demonstrative (ex. 290 and 291, Chapter 3.2).

\ea
Nəmənjar  [awak  ahay  makar\textbf{  ngə}\textbf{nd}\textbf{ə}\textbf{ye.}]     
\z

nə-mənzar  [awak  =ahaj  makar  \textbf{ŋgɪndijɛ} ]     



1S.IFV  goat  =Pl  three  DEM



‘I see those three goats.’


\ea
{}[Awak  ahay  makar \textbf{ ng}\textbf{nd}\textbf{əye  }na ],  [səlom  ahay  ga.]
\z

{}[awak   =ahaj   makar   \textbf{ŋgɪndijɛ}   na ]   [sʊlɔm   =ahaj   ga ]



goat    =Pl  three  DEM  PSP  good  =Pl  ADJ



‘Those three goats [are] good.’


A noun modified by a relative clause (ex. 292 and 293, see \sectref{sec:44}).

\ea
Nəmənjar  [awak  əwla  ahay  makar  [\textbf{nok  aməvəlaw.} ] ]     
\z

nə-mənzar  [awak  =uwla    =ahaj  makar  [\textbf{nɔk}\textbf{\textsuperscript{w}}\textbf{  amə-vəl=aw} ] ]     



1S.IFV  goat  =1S.POSS  =Pl  three  2S  DEP-give=1S.IO



‘I see my three goats that you gave to me.’


\ea
{}[awak  əwla  ahay  makar  [\textbf{nok  aməvəlaw }]  na ],  [səlom  ahay  ga.]
\z

{}[awak   =uwla     =ahaj   makar   [\textbf{nɔk}\textbf{\textsuperscript{w}}\textbf{   amə-vəl=aw }]  na ]   [sʊlɔm   =ahaj   ga ]



goat    =1S.POSS  =Pl  three  2S  DEP-give=1S.IO  PSP  good  =Pl  ADJ



‘My three goats that you gave me [are] good.’


A noun modified by a non-numeral quantifier (ex. 294 and 295, see \sectref{sec:24}).

\ea
Nəmənjar  [awak  ahay  \textbf{gam.} ]         
\z

nə-mənzar  [awak  =ahaj  \textbf{gam} ]         



1S.IFV  thing  =Pl  many



‘I see many goats.’


\ea
{}[Awak  ahay  \textbf{gam}  na ],  [səlom  ahay  ga.]
\z

{}[awak   =ahaj   \textbf{gam}   na ]   [sʊlɔm   =ahaj   ga ]



goat    =Pl  many  PSP  good  =Pl  ADJ



‘Many goats [are] good.’


A noun modified by a numeral and the adjectiviser \textit{ga} (ex. 296 and 297, Chapter 5.3).

\ea
Nəmənjar  [awak  ahay  məfaɗ\textbf{  ga.}]         
\z

nə-mənzar  [awak  =ahaj  mʊfaɗ   \textbf{ga}]         



1S.IFV  goat  =Pl  four  ADJ



‘I see the four goats.’


\ea
{}[Awak  ahay  məfaɗ\textbf{  ga}],  [səlom  ahay  ga.]
\z

{}[awak   =ahaj   mʊfaɗ   \textbf{ga}],   [sʊlɔm   =ahaj   ga]



goat    =Pl  four  ADJ  good  =Pl  ADJ



‘The four goats [are] good.’


The constituent order is shown in \figref{fig:8}., followed by illustrative examples. Not all constituents can co-occur in the same clause. There are restrictions on how complex a noun phrase can normally become. Restrictions include the fact that that quantifiers cannot co-occur in the same noun phrase as either derived adjectives or numerals. The order of relative clause and demonstrative does not appear to be strict. Note that nominal demonstratives are in a different position than local adverbial demonstratives.

\begin{tabular}{lllllllll}
\lsptoprule

\textbf{head noun} & possessive & plural & numeral & relative clause & nominal demonstrative & quantifier & ADJ & local adverbial demonstrative\\
\lspbottomrule
\end{tabular}

\begin{itemize}
\item \begin{styleFiguretitle}
Structure of the Moloko noun phrase
\end{styleFiguretitle}\end{itemize}

Modification by possessive pronoun and plural marker. 


\ea
Nəmənjar  [awak  əwla  ahay. ]       
\z

nə-mənzar  [awak  =uwla    =ahaj ]       



1S.IFV  goat  =1S.POSS  =Pl



‘I see my goats.’


\ea
{}[Awak  əwla  ahay  na ],  [səlom  ahay  ga.]
\z

{}[awak   =uwla    =ahaj     na ]   [sʊlɔm   =ahaj   ga ]



goat    =1S.POSS  =Pl  PSP  good  =Pl  ADJ



‘My goats [are] good.’


Modification by nominal demonstrative, relative clause, and plural marker. 

\ea
Nəmənjar  [awak  ahay  ngəndəye  [nok  aməvəlaw. ] ]
\z

nə-mənzar  [awak  =ahaj  ŋgɪndijɛ  [nɔk\textsuperscript{w}  amə-vəl=aw ] ]



1S.IFV  goat  =Pl  DEM  2S  DEP-give=1S.IO



‘I see those goats that you gave me.’


\ea
{}[Awak  əwla  ahay  [nok\textsuperscript{  }aməvəlaw ]  ngəndəye  na ],  [səlom  ahay  ga.]
\z

{}[awak   =uwla     =ahaj   [nɔk\textsuperscript{w}   amə-vəl=aw ]   ŋgɪndijɛ   na ]   [sʊlɔm   =ahaj   ga ]



goat    =1S.POSS  =Pl  2S  DEP-give=1S.IO  DEM  PSP  good  =Pl  ADJ



‘Those goats of mine that you gave me [are] good.’ 


Modification by qnantifier, relative clause, and plural marker. 

\ea
Nəmənjar  [awak  ahay  gam ]  [nok  aməvəlaw  va  na. ]
\z

nə-mənzar  [awak  =ahaj   gam ]   [nɔk\textsuperscript{w}  amə-vəl=aw   =va   na ]



1S.IFV  thing  =Pl  many  2S  DEP-give=1S.IO  =\textsc{PRF}  PSP



‘I see many goats, the ones that you gave me.’


\ea
{}[Awak  əwla  ahay  [nok  aməvəlaw ]  jəyga  na ],  [səlom  ahay  ga. ]
\z

{}[awak   =uwla     =ahaj   [nɔk\textsuperscript{w}   amə-vəl=aw ]   dʒijga   na ]   [sʊlɔm   =ahaj   ga ]



goat    =1S.POSS  =Pl  2S  DEP-give=1S.IO  all  PSP  good  =Pl  ADJ



‘All of my goats that you gave to me [are] good.’


Modification by quantifier, nominal demonstrative, and plural marker. 

\ea
Nəmənjar  [awak  ahay  ngəndəye  jəyga.]       
\z

nə-mənzar  [awak  =ahaj  ŋgɪndijɛ   dʒijga]       



1S.IFV  goat  =Pl  DEM  all  



‘I see all those goats.’


\ea
{}[Awak  ahay  ngəndəye  jəyga  na],  [səlom  ahay  ga,]
\z

{}[awak   =ahaj   ŋgɪndijɛ   dʒijga   na ]   [sʊlɔm   =ahaj   ga ]



goat    =Pl  DEM  all  PSP  good  =Pl  ADJ



‘All of those goats [are] good.’


\section{Noun phrase heads}
\hypertarget{RefHeading1211601525720847}{}
Noun phrases can have a head that is either a simple noun (ex. 306), nominalised verb (see \sectref{sec:38}, ex. 307), or a pronoun (see \sectref{sec:39}, emphatic pronoun in ex. 308). In the examples, the noun phrases are delimited by square brackets and the head is bolded. 


\ea
{}[\textbf{Albaya}  ahay ]  tánday  táwas.
\z

{}[\textbf{albaja}  =ahaj ]    tá-ndaj    tá-was



young man  =Pl    3P -PROG  3P.IFV-cultivate



‘The young men are cultivating.’ 


\ea
{}[\textbf{məzəme}  əwla]  amanday  acsəɓan  ana  Mana.
\z

{}[\textbf{mɪ-ʒum-ɛ}    =uwla]    ama-ndaj  a-tsəɓ=aŋ    ana   Mana



\textsc{NOM}{}-eat-CL  =1S.POSS  DEP-PROG  3S-overwhelm=3S.IO  DAT  Mana



‘[The act of] my eating is irritating Mana.’


\ea
{}[\textbf{Ndahan  }ga]  ánday  áwas.
\z

{}[\textbf{ndahaŋ}  ga]   á-ndaj  á-was



3S    ADJ  3S.IFV-PROG  3S.IFV-cultivate



‘He himself is cultivating.’


\subsection{  Noun phrases with nominalised verb heads}
\hypertarget{RefHeading1211621525720847}{}
When the head noun is a nominalised verb, the other elements in the noun phrase represent clausal arguments of the nominalised verb. The modifying noun represents the direct object Theme of the nominalised verb and the possessive pronoun or noun in a modifying genitive construction represents the subject of the verb. In ex. 309, the noun modifier \textit{ɗaf} ‘loaf’ represents the direct object of the nominalised verb \textit{mɪ-ʒʊm-ɛ} ‘eating’ and the 3P possessive pronoun \textit{=atəta} represents the subject of the nominalised verb, i.e., ‘they are eating loaf.’


\ea
A  [məzəme  ɗaf  atəta ]  ava  na,  tázlapay  bay.
\z

a  [mɪ-ʒʊm-ɛ    ɗaf  =atəta ]    ava  na  tá-ɮap-aj  baj



to  \textsc{NOM}{}-eat-CL    loaf  =3P.POSS  in  PSP  3P.IFV-talk-CL  \textsc{NEG}



‘While eating (lit. in the eating of their loaf), they don’t talk to each other.’


In ex. 310, \textit{mɪ-nd-ijɛ =aŋg}\textit{\textsuperscript{w}}\textit{ɔ} literally ‘your lying down’ indicates that ‘you are lying.’ The possessive pronoun =\textit{aŋg}\textit{\textsuperscript{w}}\textit{ɔ} is the subject of the nominalised verb \textit{mɪ-nd-ijɛ}. In ex. 311, both subject and direct object of the nominalised verb are present. \textit{Mana}, the noun in the genitive construction (see \sectref{sec:42}) codes the subject of the nominalised verb and the ‘body-part’ verbal extension\textit{ va} is the direct object, i.e., ‘Mana is resting his body.’ 


Snake S. 19


\ea
Anjakay  nok\textsuperscript{  }ha  a  slam  [məndəye  ango]  ava.
\z

à-nzak-aj        nɔk\textsuperscript{w   }ha      a      ɬam    [mɪ-nd-ijɛ  =aŋg\textsuperscript{w}ɔ ]   ava



3S.PFV-find{}-CL  2S      until    to  place    \textsc{NOM}{}-sleep{}-CL  =2S.POSS       in



‘It found you even at the place you were sleeping.’ (lit. it found you until in your sleeping place)


\ea
{} [membese va  a  Mana ]    
\z

{}[mɛ-mbɛʃ-ɛ     va   a   Mana ]    



NOM-rest-CL    body   GEN  Mana    



‘Mana’s rest’ (lit. resting body of Mana)


\subsection{  Noun phrases with pronoun heads}
\hypertarget{RefHeading1211641525720847}{}
A free pronoun head is more limited in the number of modifiers that it can take than a lexical noun head. A pronoun head can only be modified by the adjectiviser (ex. 312) or possessive pronoun in emphatic situations (ex. 314, see \sectref{sec:132}). Noun phrases with pronoun heads can not be modified by plural, number, demonstrative, adjective, or relative clause.\footnote{Pronouns can be the subject of a relative clause, see ex. 292 and \sectref{sec:1144.}} The pronoun heads are bolded in the examples. 


\ea
{}[\textbf{Ndahan  ga }]  [aməgəye.]
\z

{}[\textbf{ndahaŋga }]  [amɪ-g-ijɛ ]



3S    ADJ  DEP-do-CL



‘He is the one that did it.’ 


\ea
{}[Amədəye  elele  nəndəye  na],  [\textbf{ne  ga. }]
\z

{}[amɪ-d-ijɛ    ɛlɛlɛ  nɪndijɛ  na ]  [\textbf{nɛ  ga }]



DEP-prepare{}-CL  sauce  DEM  PSP  1S  ADJ



‘The one that prepared the sauce there [was] me.’


\ea
{}[\textbf{Ne  ahan} ]  [aməgəye.]
\z

{}[\textbf{nɛ  =ahaŋ} ]    [amɪ-g-ijɛ ]



1S  =3S.POSS  DEP-do-CL



‘I myself [am] the one that did it.’ 


\ea
{}[\textbf{Ne  ahan} ]  nólo  a  kosoko  ava.
\z

{}[\textbf{nɛ  =ahaŋ} ]    nɔ-lɔ    a  kɔsɔk\textsuperscript{w}ɔ  ava



1S  =3S.POSS  1S.IFV-go  in  market  in



‘I myself am going to the market.’


\section{Derived adjectives}
\hypertarget{RefHeading1211661525720847}{}
All adjectives in Moloko are derived – there is no underived class of adjectives. Adjectives can be derived from nouns by a very productive process in which the morpheme \textit{ga}  follows the noun. All adjectives in Moloko are derived from nouns in this way – there is no separate grammatical class of adjectives.\footnote{There are no comparative adjectives in Moloko – comparison is done by means of a clause construction using a prepositional phrase described in \sectref{sec:1145.}} \tabref{tab:35}. illustrates this process for simple nouns. 

\begin{tabular}{ll}
\lsptoprule

\textbf{Noun} & \textbf{Derived Adjective}\\
\textit{sʊlɔm}

‘goodness’ & \textit{sʊlɔm ga}

‘good’\\
\textit{gədaŋ}

‘force’ & \textit{gədaŋ ga}

‘strong’\\
\textit{dɛdɛŋ}

‘truth’ & \textit{dɛdɛŋ ga}

‘true’\\
\textit{g}\textit{\textsuperscript{w}}\textit{ɔg}\textit{\textsuperscript{w}}\textit{ɛ}\textit{ʒ}

‘redness’ & \textit{g}\textit{\textsuperscript{w}}\textit{ɔg}\textit{\textsuperscript{w}}\textit{ɛ}\textit{ʒ }\textit{ga}

‘red’\\
\textit{dalaj}

‘girl’ & \textit{dalij ga}

‘feminine’\\
\textit{bərav}

‘heart’ & \textit{bərav ga}

‘with ability to support suffering’\footnotemark{}\\
\textit{ɗaz ɗaz}

‘redness’ & \textit{ɗaz ɗaz ga}

‘red’\\
\textit{k}\textit{\textsuperscript{w}}\textit{ʊ}\textit{lɛɗɛɗɛ}

‘smoothness’ & \textit{k}\textit{\textsuperscript{w}}\textit{ʊ}\textit{lɛɗɛɗɛ ga}

‘smooth’\\
\textit{pijɛtʃɛtʃɛ}

‘coldness’ & \textit{pijɛtʃɛtʃɛ ga}

‘cold’\\
\textit{malaŋ}

‘greatness’ & \textit{malaŋ ga}

‘great’ / ‘big’\\
\textit{h}\textit{\textsuperscript{w}}\textit{ʊʃɛʃɛ}

‘smallness’ & \textit{h}\textit{\textsuperscript{w}}\textit{ʊʃɛʃɛ ga}

‘small’\\
\lspbottomrule
\end{tabular}
\footnotetext{ An idiom.}

\begin{itemize}
\item \begin{styleTabletitle}
 Derived adjectives
\end{styleTabletitle}\end{itemize}

Nominalised verbs (see \sectref{sec:7.6}) can be further derived into adjectives by the adjectiviser. The process is illustrated in \tabref{tab:36}..

\begin{tabular}{lll}
\lsptoprule

\textbf{Verb} & \textbf{Nominalised verb} & \textbf{Derived adjective}\\
\textit{ɛ{}-nʒ-ɛ }

3S-sit{}-CL

‘He sat.’ & \textit{mɪ-nʒ-ijɛ}

\textsc{NOM}{}-sit-CL

‘sitting’ (the event) & \textit{mɪ-nʒ-ijɛ        ga}

\textsc{NOM}{}-sit-CL ADJ

‘seated’ (adjective)\\
a-\textit{dar-aj }

3S-plant{}-CL

‘He planted.’ & \textit{mɛ-dɛr-ɛ}

\textsc{NOM}{}-plant{}-CL

‘planting’ (the event) & \textit{mɛ-dɛr-ɛ           ga}

\textsc{NOM}{}-plant-CL ADJ

‘planted’\\
\lspbottomrule
\end{tabular}
\begin{itemize}
\item \begin{styleTabletitle}
Adjectives derived from nominalised verbs
\end{styleTabletitle}\end{itemize}
\subsection{Structure of noun phrase containing \textit{ga}}
\hypertarget{RefHeading1211681525720847}{}
Examples show the adjectivised nouns in complete clauses. In the examples in this section, the adjectiviser \textit{ga}  is bolded and the whole noun phrase construction including \textit{ga}  is delimited by square brackets.


\ea
Nazalay  [awak  gogwez  \textbf{ga.}]\footnote{The first line in each example is the orthographic form. The second is the phonetic form (slow speech) with morpheme breaks.}
\z

nà-z=alaj   [awak   g\textsuperscript{w}ɔg\textsuperscript{w}eʒ    \textbf{ga}  ]



1S.PFV-take=away  goat  redness    ADJ



‘I took a red goat.’


\ea
Tənjakay  [ngəvəray  malan  \textbf{ga} ]  a  ləhe.
\z

tə-nzak-aj    [ŋgəvəraj  malaŋ    \textbf{ga}  ]  a  lɪhɛ



3P-find-CL    spp.of.tree  bigness    ADJ  at  bush



‘They found a big tree (of a specific species) in the bush.’  


\ea
{}[war  enen ]  [cezlere  \textbf{ga}\textbf{ }]
\z

{}[war  ɛnɛŋ  ]   [tʃɛɮɛrɛ    \textbf{ga }]



child    another    disobedient    ADJ



‘Another child [is] disobedient.’


We consider that the adjectiviser is a separate phonological word with semantic scope over the preceding noun phrase.{ }\footnote{\citet{Bow1997c} called this morpheme a noun affix. Also, for simple adjectivised noun constructions, speakers consider the adjectiviser to be part of the same word as the noun that is modified. However, in the absence of evidence for phonological bondedness, we consider \textit{ga} to be a separate phonological word. } There is no undisputable evidence that it is phonologically bound to the noun. Ex. 317 shows noun-final changes /n/ → [ŋ] before \textit{ga}. These changes might be due to assimilation of /n/ to point of articulation of /g/ within a word (see \sectref{sec:2.2}). However, the same change would occur at a word break, with word-final changes to /n/ (see \sectref{sec:3} and criterion b in \sectref{sec:11}).{ }\footnote{We have not no examples of word-final alterations of /h/ before \textit{=ga.}} Also, the prosody of \textit{ga} does not neutralise any prosody on the word to which it is bound. 

The adjectiviser maintains its position at the right edge of a noun phrase regardless of the noun phrase components (ex. 319 - 324). 

\ea
Tákəwala  [kəra  mətece  elé  \textbf{ga.} ]
\z

tá-kuw=ala    [kəra  mɪ-tɛtʃ-ɛ    ɛlɛ  \textbf{ga} ]  



3P.IFV=seek=to  dog  \textsc{NOM}{}-close-CL  eye  ADJ  



‘They look for a puppy that hasn’t opened its eyes yet.’ (lit. a dog closing eyes)


\ea
Ləme  Məloko  ahay  na,  nəmbəɗom  a  dəray  ava  na,
\z

lɪmɛ    Mʊlɔk\textsuperscript{w}ɔ  =ahaj    na   nə-mbʊɗ{}-ɔm        a  dəraj  ava  na



1Pex   Moloko      =Pl  PSP  1.PFV-change-1Pex  in   head  in    PSP  



\textit{ka  }[\textit{kərka}\textit{ɗ}\textit{aw  ahay} \textit{   nə  hərg}\textit{o}\textit{v  ahay }\textbf{\textit{  ga}}\textit{ }]\textit{  a  ɓərzlan  ava  na.}



ka  [kərkaɗaw  =ahaj    nə   hʊrg\textsuperscript{w}ɔv   =ahaj  \textbf{ga} ]    a      ɓərɮaŋ     ava    na



like  monkey      =Pl        with    baboon    =Pl  =AD  in    mountain    in    PSP



‘We the Moloko, we have become like the monkeys and baboons in the mountains’ (lit. we the Moloko, we have changed in the head [to be] like monkeys and baboons on the mountains) 


When the head noun in a phrase that contains the adjectiviser \textit{ga} is pluralised, both the head noun and the noun modifier are pluralised as well. Compare the singular noun phrase in ex. 321 with the pluralised noun phrase in ex. 322 where both the head noun and adjective are pluralised. The same pattern of pluralisation is shown in ex. 323 - 324. Note that the plural is not becoming individually ‘adjectivised.’ but rather the entire noun phrase is adjectivised. Note also that the adjectiviser always maintains its position at the right edge of the noun phrase. 

\ea
Naharalay  [awak  babəɗ\textbf{  ga} ]  a  mogom.
\z

nà-har     =alaj    [awak    babəɗ    \textbf{ga} ]  a  mɔg\textsuperscript{w}ɔm



1S.PFV-carry=away    goat    white    ADJ  at  home



‘I carried the white goat home.’


\ea
Naharala  [awak  ahay  babəɗ  ahay  \textbf{ga} ]  a  mogom.
\z

nà-har     =alaj    [awak  =ahaj  babəɗ  =ahaj  \textbf{ga} ]  a  mɔg\textsuperscript{w}ɔm



1S.PFV-carry=away    goat  =Pl  white  =Pl  ADJ  at  home



‘I carried the white goats home.’


\ea
 [məze  ahay  səlom  ahay  \textbf{ga}   na ],  tázala  təta  bay.
\z

{}[mɪʒɛ  =ahaj   sʊlɔm   =ahaj   \textbf{ga }   na ]   tá-z =ala  təta     baj



person     =Pl     good    =Pl   ADJ   PSP  3P.IFV-take=to  ABILITY  \textsc{NEG}



Good people (lit. people with the quality of goodness), they can’t bring [it]. 



Values S. 49


\ea
Nde  [məze  ahay  gogor  ahay  \textbf{ga}   na ]  ngama.
\z

ndɛ  [mɪʒɛ  =ahaj  g\textsuperscript{w}ɔg\textsuperscript{w}ɔr     =ahaj   \textbf{ga}  na ]   ŋgama



so    person     Pl  elder     =Pl   =  ADJ   PSP  better



So, our elders [had it] better.


Derived adjectives can be negated by following them with the negative \textit{baj}. 

\ea
{}[Agwəjer  mədere  \textbf{ga}  bay   na ],  natoho.  
\z

{}[ag\textsuperscript{w}ødʒɛr  mɪ-dɛr-ɛ    \textbf{ga}  baj  na ]  natɔh\textsuperscript{w}ɔ  



grass  \textsc{NOM}{}-braid-CL  ADJ  \textsc{NEG}  PSP  over there



‘The grass that is not thatched [is] over there’


\ea
{}[Yam  pəyecece  \textbf{ga}  bay   na ] ,  acar  bay.
\z

{}[jam  pijɛtʃɛtʃɛ   \textbf{ga}  baj  na ]   a-tsar    baj



water  coldness  ADJ  \textsc{NEG}  PSP  3S.PFV-taste good  \textsc{NEG}



‘Lukewarm water doesn’t taste good.’ (lit. water not with the quality of coldness, it doesn’t taste good)


\subsection{Functions of noun phrases containing \textit{ga}}
\hypertarget{RefHeading1211701525720847}{}
The morpheme \textit{ga} has two other functions besides adjectiviser.\footnote{These two functions for\textit{ }\textit{ga} do not indicate homophones. We interpret all cases of\textit{ }\textit{ga} as the same morpheme since all instances pattern in exactly the same way even when their function is different. We conclude that the same morpheme is functioning at the noun phrase level as an adjectiviser and at the discourse level in definiteness and emphasis. }  Its function to render a pronoun emphatic is discussed in \sectref{sec:132.} 

The adjectiviser can also function as a discourse demonstrative to make the noun definite and even sometimes emphatic.  A set of examples from the Cicada story illustrates the discourse function. Ex. 327 - 329 are from lines 5, 12 and 18 respectively  (the Cicada story is found in its entirety in \sectref{sec:1.6}). The first mention in the narrative of \textit{ŋgəvəraj} ‘tree of a particular species’ is shown in ex. 327. The tree is introduced as \textit{ŋgəvəraj} \textit{malaŋ}\textit{ }\textit{ga}  ‘a large tree.’ Later on in the narrative, the particular tree that was found is mentioned again (ex. 328 and 329). In these occurrences however, the tree is not modified by an adjective, but the noun is simply marked by \textit{ga }\textit{ }(\textit{ŋgəvəraj ga} ‘this tree of a particular species’ in ex. 328 and\textit{ mɛmɛlɛ ga} ‘the tree’ in ex. 329). In these last two examples, \textit{ga} indicates that ‘tree’ is referring to the particular tree previously mentioned in the discourse. 


Cicada S. 5



\ea
Təlo  tənjakay  [ngəvəray  malan  \textbf{ga }]  a  ləhe.  
\z

tə-lɔ  tə-njak-aj        [ŋgəvəraj      malaŋ    \textbf{ga }]   a    lɪhɛ  



3P.PFV-go   3P.PFV-find-CL   spp. of tree     largeness ADJ   at   bush



‘They went and found a large tree (a particular species) in the bush.’ (lit. a tree with the quality of greatness)



Cicada S. 14


\ea
{}[Ngəvəray  \textbf{ga }]  səlom  ga  aɓəsay  ava  bay.
\z

{}[ŋgəvəraj  \textbf{ga }]    sʊlɔm    ga     aɓəsaj     ava     baj



sp.of.treeADJ   goodness  ADJ     blemish   \textsc{EXT}   \textsc{NEG}



‘This tree is good ;  it has no faults.’



Cicada S. 20


\ea
Náamənjar  na  alay  [memele  \textbf{ ga}  ndana ]  əwɗɛ~
\z

náá-mənzar     na  =alaj   [mɛmɛlɛ  \textbf{ga}   ndana ]  uwɗɛ~



1S.POT-see     3S.DO  =away   tree   ADJ   DEM   first



‘First I want to see this tree that you spoke of.’


In another story about a reconciliation ceremony between two warring parties (the Moloko and the Mbuko), the ceremony requires the cutting in two of a puppy. Which side received which part was a key element to the outcome of the ceremony. In the text, the first mention of \textit{dəraj} ‘the head’ (ex. 330) is marked with \textit{ga}  – it is an expected part of the narrative frame.  When the outcome of the ceremony revealed that the Moloko got the head part (and so ‘won’ the contest) and the Mbuko received the hind parts, both are adjectivised:  \textit{dəraj}\textit{ga} ‘the head’ and\textit{ mɪtɛnɛŋ}\textit{ga}\textit{ ‘}the hindparts’ (ex. 331). Note that ex. 331 consists of two predicate possessive verbless clauses (see \sectref{sec:70}), each with a predicate that is an adjectivised noun. 

\ea
Asa  ləme  nəgəsom  na  [dəray  \textbf{ga}]  na,  [səlom  ga.]
\z

asa  lɪmɛ  nə-gʊs-ɔm    na      [dəraj    \textbf{ga}]    na  [sʊlɔm       ga]



if    1Pex  1.IPV-catch-1Pex    3S.DO  head  ADJ     PSP  goodness     ADJ



If we got the head, [it would be] good.


\ea
{}[Dəray  \textbf{ga} ]  anga  ləme  [mətenen  \textbf{ga} ]  anga  mboko  ahay.  
\z

{}[dəraj  \textbf{ga} ]  aŋga  lɪmɛ    [mɪtɛnɛŋ     \textbf{ga} ]    aŋga  mbɔk\textsuperscript{w}ɔ   =ahaj  



head  ADJ  \textsc{POSS}  1\textsc{Pex}    hindparts  ADJ    \textsc{POSS}  Mbuko  =Pl



‘The head [is] belonging to us; the hindparts [are] belonging to the Mbuko.’ 


Compare ex. 332 and 333 (from lines 1 and 30, respectively of the Disobedient Girl story; shown in its entirety in \sectref{sec:1.5}). The noun \textit{bamba} ‘story,’ when first mentioned in the introduction of the story (ex. 332) is not adjectivised. When the same noun is mentioned again in the conclusion (ex. 333), it is adjectivised \textit{ma bambaga} ‘the story.’ 


Disobedient Girl S. 1


\ea
{}[Bamba ]  [bamba ]  kəlo  dərgoɗ 
\z

{}[bamba ]   [bamba ]  kʊlɔ     dʊrg\textsuperscript{w}ɔɗ 



story        story        under    silo



‘Once upon a time…’ (lit. there’s a story under the silo)



Disobedient Girl S. 39


\ea
Ka  nehe  [ma  bamba\textbf{  ga }]  andavalay.    
\z

ka  nɛhɛ  [ma    bamba \textbf{ga }]  à-ndava=alaj    



like  here  word   story     ADJ     3S.PFV-finsh=away 



‘It is like this the story ends.’  


In the ‘Cows in the Field’ story (not illustrated in this work) \textit{ga} is used to mark the five brothers (previously mentioned) whose field was damaged (and who had to go to the police to resolve the problem, ex. 334 and 335), and the problem (\textit{ma ga }‘that word’) that developed when they couldn’t find justice (ex. 336 and 337). 

\ea
{}[Məlama  ahay  məfaɗ  \textbf{ga} ]  tanday  tágalay  ta  [sla  ahay  na ]  a  Kədəmbor
\z

\textit{[}məlama =ahaj   məfaɗ  \textbf{ga} \textit{ ]}    ta-ndaj   tá-gal-aj          ta  \textit{[}ɬa=ahaj   na\textit{]}   a     Kʊdʊmbɔr


brother     =Pl      four    ADJ  3P+PRG  3P.IFV-drive-CL 3P.DO    cow =Pl       PSP   to    Tokembere



‘The four brothers, they were driving the cows to Tokembere.’ 


\ea
Nəbohom  ta  alay  ləme  [zlom  \textbf{ga} ]
\z

nə-bɔh-ɔm      ta    =alaj  lɪmɛ  [ɮɔm   \textbf{ga} ]



1\textsc{Pex}.PFV-pour-1\textsc{Pex  }  3P.DO =away  1\textsc{Pex}   five  ADJ



‘We gave them [our identity cards], we [were] the five [whose fields were damaged].’ 


\ea
Sen  a slam  na  ava  nendəge  na,  nəmənjorom  [ma  \textbf{ga} ]
\z

ʃɛŋ     a   ɬam   na  ava   nɛndɪgɛ  na  nə-mʊnzɔr-ɔm    [ma     \textbf{ga} ]



\textsc{ID}walk  in  place   PSP  in   DEM   PSP   1\textsc{Pex}.PFV-see   word   ADJ



‘Walking (later), at that place, we saw the problem.’


\ea
Nəbohom  [ma  \textbf{ga} ]  a  brəygad  ava.
\z

nə-bɔh-ɔm      [ma  \textbf{ga} ]  a  brijgad  ava



1\textsc{Pex}.PFV-pour-1\textsc{Pex}    word  ADJ   in   Brigade  in



‘We took the problem to the Brigade.’


The emphatic function of \textit{ga}\footnote{The emphatic function of\textit{=ga } is discussed with respect to pronouns in \sectref{sec:11132.}} mentioned above is even more obvious in the Values exhortation (see \sectref{sec:1.7}). Line 7 in the Values exhortation, shown in ex. 338, alludes to the commandments that \textit{Hʊrmbʊlɔm à-wats=ala kə ɔk}\textit{\textsuperscript{w}}\textit{ɔr aka} ‘God wrote on the stone,’ and line 12 (ex. 339) exhorts the hearer \textit{kɔɔ-gʊs-ɔk ma Hʊrmbʊlɔm  }‘you should accept the word of God.’ Further in the text, the mention of \textit{aŋga} \textit{Hʊrmbʊlɔm ga }‘the very [word] of God himself’ (ex. 340 from line 28) draws attention to the fact that the people don’t accept what God himself wrote on the stone tablets. This time, the marker \textit{ga}  has an emphatic function. 


Values S. 7


\ea
Hərmbəlom  awacala  kə  okor  aka.
\z

Hʊrmbʊlɔm à-wats    =ala   kə   ɔk\textsuperscript{w}ɔr   aka



God    3S.PFV-write  =to  on  stone  on



‘God wrote them on the stone [tablet].’



Values S.12


\ea
Yawa  nde  ele  nehe  ɗəw,  kóogəsok\textsuperscript{  }ma  Hərmbəlɔm.  
\z

jawa   ndɛ   ɛlɛ   nɛhɛ   ɗuw  kɔɔ-g\textsuperscript{w}ʊs-ɔk\textsuperscript{w}     ma   Hʊrmbʊlɔm  



well    so  thing  DEM  also  2S.POT-catch-2P    word  God



‘So, this thing here, you should accept the word of God.’ 



Values S. 28


\ea
 [Anga  Hərmbəlom\textbf{  ga }]  kagas  asabay.
\z

{}[aŋga   Hʊrmbʊlɔm   \textbf{ga }]     kà-gas     asa-baj



\textsc{POSS}  God    ADJ    2S.PFV-catch  again-\textsc{NEG}



‘The very [word] of God himself you will never accept.’ (lit. the quality of belonging to God)


\section{Nouns as modifiers}
\hypertarget{RefHeading1211721525720847}{}
There are three types of constructions where nouns figure in the modification of another head noun in Moloko. They are:


\begin{itemize}
\item Genitive construction. A head noun followed by a genitive noun phrase with the genitive particle \textit{a} (see \sectref{sec:42}, ex. 341).
\item Permanent attribution construction. Two nouns are juxtaposed with no intervening particle (see \sectref{sec:43}, ex. 342).
\item Relative clause (see \sectref{sec:44}, ex. 343). 
\end{itemize}

\ea
{}[war  [a  bahay ] ]
\z

{}[war  [a  bahaj ] ]



child  GEN  chief



‘the chief’s child’


\ea
{}[zar  Məloko ]  
\z

{}[zar    mʊlɔk\textsuperscript{w}ɔ ]  



man    moloko  



‘Moloko man’


\ea
{}[war  [aməgəye  cəɗoy ]  akaray  zana  aloko  apazan.
\z

{}[war  [amɪ-g-ijɛ  tsʊɗoj ]    à-kar-aj    zana  =alɔk\textsuperscript{w}ɔ  apazaŋ



child  DEP-do-CL  wickedness  3S.PFV-steal{}-CL  clothes  1PIN  yesterday



‘The child that did wickedness stole our clothes yesterday.’


\subsection{  Genitive construction}
\hypertarget{RefHeading1211741525720847}{}
The genitive construction follows the head noun in a noun phrase. The genitive noun phrase consists of the genitive particle \textit{a} plus a noun phrase expressing the possessor (ex. 344 and 345). 


\ea
{}[zar  [a  Hawa ] ]
\z

{}[zar    [a  Hawa ] ]



man    GEN  Hawa



‘Hawa’s husband’


\ea
{}[hay  [a  baba  ango ] ]
\z

{}[haj    [a  baba  =aŋg\textsuperscript{w}ɔ  ] ]



house  GEN  father  =2S.POSS



‘your father’s house’


\citet{Bow1997c} remarks that the particle \textit{a} appears to carry the tone HL, with a floating L.\footnote{Note that the genitive particle \textit{a} and the adposition \textit{a} (Sections 1145 and 1146) are homophones. } She demonstrates in ex. 346 that the floating low tone lowers the high tone of the noun (\textit{háj}) to become M. 

\ea
\textup{[ɗ\={a}f ]+  [á]  +  [háj]    [F0AE?]   [ɗəf á h\={a}j]}
\z

‘loaf’    GEN    ‘millet’       ‘millet loaf’


Also, the genitive particle will elide with any word final vowel in a previous word; likewise it will elide with a vowel at the beginning of the following word but the tone effects remain.

In a genitive construction, the relationship of the genitive noun phrase to the head  noun is a temporary attribute of or relationship to the head.\footnote{As compared with the permanent attribution construction \citep{Section1143}.} The semantic relationship between head noun and genitive expresses the same range of semantic notions as the possessive pronoun (see \sectref{sec:141}). In the examples below, the genitive construction expresses ownership (both alienable and inalienable, ex. 347), kinship (ex. 348), partitive (ex. 349), and other looser associations (ex. 350 - 352). When applicable, a corresponding pronominal possessive construction is also given for comparison. 

\begin{itemize}
\item 
{}[\textit{hay  }[\textit{a  Mana} ]      [\textit{hay  }\textit{ə}\textit{wla} ]  
\end{itemize}

{}[haj    [a   Mana ]      [haj   =uwla ]  



house    GEN  Mana      house   =1S.POSS



‘Mana’s house’        ‘the house that I live in’ (not the house I  made)\footnote{‘The house I made’ requires a relative clause: [\textit{haj    }[\textit{uwla      amɪ-hɛr-ɛ  =va }] ] ‘house mine to build.’}


\ea
{}[hor  [a  Mana ] ]      [hor  ahan ]
\z

{}[h\textsuperscript{w}ɔr   [a   Mana ] ]      [h\textsuperscript{w}ɔr  =ahaŋ ]



woman    GEN    Mana      woman  =3S.POSS



‘Mana’s wife’        ‘his wife’


\ea
{}[dəray  [a  Mana ] ]        [dəray  ahan ]
\z

{}[dəraj   [a   Mana ] ]      [dəraj  =ahaŋ ]



head   GEN  Mana      head  =3S.POSS



‘Mana’s head’        ‘his head’


\ea
{}[slərele  [a  Mana ] ]        [slərele    ahan ]
\z

{}[ɬɪrɛlɛ   [a   Mana ] ]      [ɬɪrɛlɛ  =ahaŋ ]



work   GEN  Mana      work  =3S.POSS



‘Mana’s work’        ‘his work’


\ea
{}[pəra  [a  Mala ] ]        [pəra  ahan ]
\z

{}[pəra   [a  Mala ] ]      [pəra    =ahaŋ ]



spirit-place   GEN   Mala      spirit-place  =3S.POSS



‘the spirit-place that Mala worships’    ‘his spirit-place’


\ea
{}[zar  akar  [a  Mana ] ]      [zar  akar  ahan ]
\z

{}[zar   akar   [a   Mana] ]    [zar   akar   =ahaŋ ]



man   thief   GEN   Mana    man  thief  =3S.POSS



‘the man who stole from Mana’    ‘the man who stole from him’


There are several idioms or figurative expressions in Moloko which involve genitive constructions where the head noun in the noun phrase is a body part such as \textit{ma} ‘mouth’ (ex. 353 - 355) or \textit{h}\textit{\textsuperscript{w}}\textit{ɔ}\textit{ɗ}  ‘stomach’ (ex. 356). 

\ea
{}[ma  [a  gəver ] ] 
\z

{}[ma   [a  gɪvɛr ] ] 



mouth  GEN  liver



‘gall bladder’


\ea
{}[ma  [a  gəlan ] ] 
\z

{}[ma   [a  gəlaŋ ] ] 



mouth  GEN  kitchen



‘door to the kitchen’


\ea
{}[ma  [a  savah ] ] 
\z

{}[ma   [a  savax ] ] 



mouth  GEN  rainy season



‘beginning of rainy season’


\ea
Ne  a  [hoɗ  [a  zazay ] ]  ava. 
\z

nɛ  a   [h\textsuperscript{w}ɔɗ  [a   zazaj ] ]   ava 



1S  in  stomach  GEN  peace  in



‘I [am] very peaceful.’ (lit. I, in the centre of peace)


All other modifiers in a genitive construction will modify the genitive noun and not the head noun. In ex. 357, the possessive modifies the genitive noun (my wife) and not the head noun (i.e., not ‘my bride price’). Likewise in ex. 358, the demonstrative modifies the genitive noun (‘this woman’) and not the head noun (i.e., not ‘this bride price’). In ex. 359, it is the genitive noun ‘animals’ that is pluralised and modified by ‘all’, not the head noun ‘chief.’

\ea
{}[Gembere  [a  hor  əwla ]]  adal  anga  ango.
\z

{}[gembɛrɛ  [a  h\textsuperscript{w}ɔr  =uwla]]    a-dal    aŋga  =aŋg\textsuperscript{w}ɔ



bride price  GEN  woman  =1S.POSS  3S-exceed  \textsc{POSS}  =2S.POSS



‘The bride price of my wife exceeded [that] belonging to you.’


\ea
{}[Gembere  [a  hor  nehe ]  na],  acəɓava.
\z

{}[gembɛrɛ  [a  h\textsuperscript{w}ɔr  nɛhɛ]  na]  a-tsəɓ=ava



bride price  GEN  woman  DEM  PSP  3S-overwhelm  =in



‘The bride price of this woman is exhorbitant.’ 


\ea
Angala  [bahay  [a  gənaw  ahan  ahay  a  slala  ga  ava  jəyga.] ]
\z

à-ŋgala     [bahaj  [a   gənaw  =ahaŋ     =ahaj  a  ɬala  ga  ava  dzijga ] ]



3S.PFV-return  chief  GEN  animal  =3S.POSS=Pl  in  village  ADJ  in  all



‘He came back as the chief of all his animals in the village.’ 


\subsection{  Permanent attribution construction}
\hypertarget{RefHeading1211761525720847}{}
In a ‘permanent attribution construction,’ the noun phrase has a head composed of two (or even three) nouns, which acts as a unit within a larger noun phrase. The nouns in a permanent attribution construction do not comprise a compound made of phonologically bound words, but are separate words (prosodies do not spread from one noun to the other, ex. 362, 363, 366, and there are word-final changes in the first noun). Semantically, the second noun in the noun phrase indicates something about the identity of the first noun or gives a permanent attribute of the head noun.\footnote{As compared with the genitive construction which gives a more temporary attribute \citep{Section1142}.} The glosses in each of the examples below confirm this observation. 


\ea
{}[zar  Ftak ]
\z

{}[zar Ftak ]



man    Ftak



‘a man who was born in Ftak’


\ea
{}[zar  akar ]
\z

{}[zar    akar ]



man   theft



‘thief’ (someone who makes his living from stealing)


\ea
{}[zar  jəgwer ]
\z

{}[zar   dʒɪg\textsuperscript{w}ɛr ]



man    shepherd



‘a shepherd’ (paid for his work)


\ea
{}[zar  səlom ]  
\z

{}[zar   sʊlɔm ]  



man    goodness



‘a man who is known for his goodness’


\ea
{}[dalay  zazay ] 
\z

{}[dalaj   zazaj ] 



girl    peace



‘girl of peace’ (peace identifies her)


\ea
{}[zar  madan]  
\z

{}[zar   madaŋ ]  



man    sorcery



‘a known sorcerer’


\ea
{}[zar  slərele\textbf{ }]  
\z

{}[zar   ɬɪrɛlɛ ]  



man    work



‘a man who is known as someone who works hard’


In a noun phrase with the permanent attribution construction as its head noun, other elements in the noun phrase modify the entire head (and not just one of the nouns in the construction, as is the case for the genitive construction, see \sectref{sec:42}). In ex. 367, the plural and the numeral modify the head noun \textit{ndam ɬɪrɛlɛ }and the sense is ‘his three workmen,’ not ‘the man of his three works.’  In ex. 368, the noun phrase has a triple noun head, \textit{war ele haj} ‘millet grain.’ In this noun phrase, the derived adjective \textit{bɪlɛŋga} ‘one,’ the demonstrative \textit{nɛndijɛ  }‘that,’ and the relative clause \textit{nɔk}\textit{\textsuperscript{w}}\textit{ amɛ}\textbf{\textit{{}-}}\textit{ʒ{}-ɛ}  ‘the one that you brought’ all modify the triple noun head \textit{war}\textit{ }\textit{ele}\textit{ }\textit{haj}  ‘millet grain.’ They do not just modify the noun \textit{war} ‘child’ or \textit{haj} ‘millet.’ In the examples below, the noun phrase is delimited by square brackets and the permanent attribution construction is bolded. 

\ea
{}[\textbf{Ndam  sl}\textbf{ə}\textbf{rele  }ahan  ahay  makar.]
\z

{}[\textbf{ndam   ɬɪrɛlɛ}    =ahaŋ     =ahaj   makar ]



people  work  =3S.POSS  =Pl  three



‘his three workmen’



Disobedient Girl 13


\ea
{}[\textbf{War  elé  hay}  bəlen  ga  nendəye  nok  ameze   na ],  kahaya  na  kə  ver  aka.  
\z

{}[\textbf{war     ɛlɛ  }   \textbf{haj}     bɪlɛŋ   ga    nɛndijɛ    nɔk\textsuperscript{w}     amɛ\textbf{{}-}ʒ{}-ɛ ]\textbf{ }     na    



child    eye    millet   one   ADJ    DEM      2S         DEP-take-CL   PSP  



\textit{ka-h=aja   na        kə   vɛr     aka}



2S-grind    3S.DO     on  grinding stone  on



‘That one grain of millet that you took, you should grind it on the grinding stone.’


It is interesting that when dependent and nominalised clauses (see \sectref{sec:7.6} and 7.7) are within permanent attribution  and genitive constructions, the same modal differences seen in \sectref{sec:82} still apply. The nominalised form of the verb functions to give a particular situation a finished idea, with an event that has been accomplished before the point of reference, almost as a state. In contrast, the dependent form of the verb is employed in situations which have an incomplete idea, one that is not yet achieved.

Compare ex. 369 and 370. Ex. 369 (as well as ex. 362) refers to someone whose identity is a shepherd – he is a man who makes his living caring for sheep or other animals. He probably is hired. This more permanent identity or state is expressed through the nominalised form of the verb in a permanent attribution construction. In contrast, example 370 (a relative clause, see \sectref{sec:44}) reflects a man who cares for sheep but being a shepherd isn’t his identity – he has sheep now but may not always have them. It is an incomplete or not completely realised situation expressed through the dependent form of the verb (a relative clause, but similar to the genitive). 

\ea
zar  məjəgwere  
\z
\ zar     mɪ\textbf{{}-}dʒɪg\textsuperscript{w}ɛr-ɛ  



man    \textsc{NOM}{}-shepherd-CL



‘a shepherd-man’ (lit. man shepherding)


\ea
məze  aməjəgwere  təmak  
\z

mɪʒɛ   amɪ-dʒɪg\textsuperscript{w}ɛr-ɛ     təmak  



person  DEP-shepherd-CL  sheep



‘a person that cares for sheep’ (lit. person to care for sheep)


Likewise, compare ex. 371 and 372. In example 371, the dependent verb form is used to give the idea that the person has stolen something from someone, perhaps only once in his life.  In contrast, in ex. 372 (using the irregular nominalised form of the verb, see \sectref{sec:4.2} in a permanent attribution construction) the nominalised form is used to express that the man is a thief by identity or occupation – he steals to make his living. In addition, example 373 shows the man who experienced the theft. The theft has occurred already with ongoing effects to the point of reference and the nominalised form is used. 

\begin{itemize}
\item 
\textit{m}\textit{əze  }\textit{am}\textit{ə}\textit{k}\textit{ə}\textit{re  m}\textit{əze} 
\end{itemize}

\textit{mɪʒɛ  amɪ-kɪr-ɛ  mɪʒɛ} 



person  DEP-steal-CL  person



‘the person that steals’ (lit. person to steal from person)


\ea
zar  akar
\z
\ zar    akar



man    theft



‘a thief’ (lit. man thief)


\begin{itemize}
\item 
\textit{m}\textit{əze  }\textit{m}\textit{ə}\textit{k}\textit{ə}\textit{re  ga} 
\end{itemize}

\textit{mɪʒɛ  mɪ}\textbf{\textit{{}-}}\textit{kɪr-ɛ    ga} 



person  \textsc{NOM}{}-steal-CL  ADJ



‘the person who was robbed’ 


\subsection{  Relative clauses}
\hypertarget{RefHeading1211781525720847}{}
Relative clauses are one of the final elements in a noun phrase (see \sectref{sec:5.1}). The structure of relative clauses in Moloko is shown in \figref{fig:9}. and consists of a pronoun (when necessary), a verb in dependent form (see \sectref{sec:7.7}) and a complement. A relative clause has no pronoun when the head of the relative clause  is the subject of the relative clause. If the head of the relative clause has a grammatical role other than subject, then a pronoun is used.

\begin{tabular}{l}
\lsptoprule

  (pronoun)      dependent verb      complement           (presupposition  marker)\\
\lspbottomrule
\end{tabular}

\begin{itemize}
\item \begin{styleFiguretitle}
Structure of relative clause:
\end{styleFiguretitle}\end{itemize}

The head noun of the relative clause can be either the subject or  the direct object of the relative clause. When the head noun is the subject of the relative clause (ex. 374, 375, 376, 377, 383, 389), there is a gap for subject in the relative clause (marked by ${\emptyset}$ in the examples). For example, the understood subject of the relative clause in ex. 374 is the same as \textit{war dalaj} ‘the girl’ in the noun phrase. In the example, the ${\emptyset}$ is a zero marking where the subject of the clause would otherwise be. There is a gap for subject becuase the subject of the relative clause is the same as the head of the noun phrase that is being modified. The relative clause is bolded and the noun phrase is delimited by square brackets in the examples in this section.


Disobedient Girl S.38



\begin{itemize}
\item 
\textit{Metesle  anga  }[\textit{war  dalay  ngend}\textit{ə}\textit{ye}  \textbf{\textit{amazata  aka  ala}}  \textbf{\textit{av}}\textbf{\textit{əy}}\textbf{\textit{a  nengehe  ana  m}}\textbf{\textit{əz}}\textbf{\textit{e  ahay  na.}}]
\end{itemize}

\textit{mɛtɛ}\textit{ɬ}\textit{ɛ  aŋga  }[\textit{war  dalaj   ŋgɛndijɛ   }${\emptyset}$\textbf{  }\textbf{\textit{ama-z=ata   aka  =ala}} 


\begin{stylefootnotetext}
curse  \textsc{POSS}  child  girl  DEM      DEP-bring=3P.IO on  =to
\end{stylefootnotetext}

\begin{stylefootnotetext}
‘The curse [is] belonging to that girl, (the one) who had brought’
\end{stylefootnotetext}


\textbf{\textit{avija        nɛŋgɛhɛ     ana     mɪʒɛ     =ahaj   na }}]



suffering    DEM    DAT     person    =Pl          PSP


\begin{stylefootnotetext}
‘this suffering to the people.’
\end{stylefootnotetext}

\ea
{}[Ləkwəye  hawər  ahay  na,\textbf{  }\textbf{amanday  a  hay  a  zaw}\textbf{ə}\textbf{r  ahay  ava,}]  səy  kogəsom  ma  a  zawər  a  ləkwəye  ahay.
\z

{}[lʊk\textsuperscript{w}øjɛ  hawər  =ahaj   na  ${\emptyset}$     ama-ndaj   a  haj  a  zawər  =ahaj  ava ]



2P    women  =Pl  PSP          DEP-PROG  in  house  GEN  men  =Pl  in



‘You women, the ones that are living at your husband’s house,



\textit{sij    kɔ-gʊs-om  ma  a  zaw}\textit{ə}\textit{r  a} \textit{lʊk}\textit{\textsuperscript{w}}\textit{øjɛ  =ahaj}



only    2-catch-2P  mouth  GEN  men  2P.POSS  =Pl  



‘you must listen to your husbands.’



Disobedient Girl S.33


\ea
Hərmbəlom  ága  ɓərav  va  kəwaya  [war  dalay  na  \textbf{amecen  sləmay  bay}  ngəndəye.]  \textbf{ }
\z

Hʊrmbʊlɔm  á-g-a         ɓərav   =va  kuwaja     [war      dalaj   na  \textbf{ }



God                 3S.IFV-do  heart   =\textsc{PRF}  because of   child   girl      PSP        



${\emptyset}$   amɛ-tʃɛŋ     ɬəmaj  baj      ŋgɪndijɛ]



DEP-hear   ear       \textsc{NEG}     DEM



‘God had gotten angry because of that girl, that one that was disobedient.’


\ea
Nde  [ləbara  əwla  ga\textbf{  }\textbf{amətar}\textbf{aləkwəye}  \textbf{ma}]  nehe.
\z

ndɛ  [ləbara  =uwla    ga  \textbf{${\emptyset}$}\textbf{  amə-tar  =alʊk}\textbf{\textsuperscript{w}}\textbf{øjɛ  ma}]  nɛhɛ



so  news  =1S.POSS  ADJ      DEP-call=2P.IO    mouth  DEM



‘So, this is my news that I have called you together (to hear).’ (lit. So, my news which called mouth to you [is] this here)


When the head noun is the direct object of the relative clause, the relative clause must contain a subject pronoun. The pronoun  must be inserted before the verb in the relative clause (ex. 378 - 379). It is interesting that this subject pronoun of the relative clause is sometimes a free pronoun (see \sectref{sec:13}; ex. 379, 380, 384) but in other cases is a possessive pronoun (see \sectref{sec:14}; ex. 378).  Ex. 378 and 379 are taken from the same narrative,\footnote{The entire narrative is not included in this work.} but they each use different pronouns for the subject of the relative clause. Ex. 378 uses the 3P possessive pronoun \textit{=atəta}; ex. 379 uses the free pronoun \textit{təta}. In some cases, the relative clause will contain the direct object pronominal \textit{na} following the dependent verb. The DO pronominal represents the noun phrase head. In the examples below, the direct object pronominal \textit{na} is underlined. A gap for the direct object in the relative clause (ex. 379 and 384) is indicated by ${\emptyset}$.  

\ea
Tasan  oko  ana  [hay  \textbf{atəta  am}ə\textbf{g}əye  \textbf{na}\textbf{  va}\textbf{.}]   
\z

tà-s=aŋ            ɔk\textsuperscript{w}ɔ   ana    [haj  \textbf{=atəta     amɪ-g-ijɛ   }\textbf{na}\textbf{   va }]     



3P.PFV-cut=3S.IO  fire   DAT    house  3P.POSS   DEP-do-CL      3S.DO    \textsc{PRF}    



‘They set fire to the house that the others had made.’


\ea
A  slam  a  [hay  təta\textbf{  }\textbf{aməgəye}  \textbf{a  dala  kosoko  ava  }na],  tolo.  
\z

a  ɬam  a  [haj  təta  amɪ-g-ijɛ   ${\emptyset}$  a  dala  kɔsɔk\textsuperscript{w}ɔ  ava   na ]  



to   place  GEN  house  3P  DEP-do-CL    GEN  money  market  in  PSP  



\textit{tɔ-lɔ}



3P.PFV-go



‘To the place of the house that they made in the market, they went.’


\ea
{}[War  háy  ngəndəye  \textbf{nok  ame}\textbf{z}\textbf{e  }\textbf{na}\textbf{  }\textbf{va }]  bəlen  ngəndəye  na,  káahaya  kə  ver  aka.
\z

{}[war  haj      ŋgɪndijɛ   \textbf{nɔk}\textsuperscript{w}\textbf{  amɛ-}\textbf{ʒ}\textbf{{}-ɛ     }\textbf{na}\textbf{     =}\textbf{va}]    bɪlɛŋ  ŋgɪndijɛ  na



child    millet  DEM      2S        DEP-take-CL  3S.DO   =\textsc{PRF}  one        DEM      PSP



\textit{káá-}\textit{h=aj}\textit{a            kə  vɛr             aka}



2S.POT-grind=\textsc{PLU}  on  grinding stone       on



‘That grain that you have taken, that one [grain], grind it on the grinding stone.’


Ex. 381 is more complex since the subject of the relative clause includes the speaker along with the head of the noun phrase (\textit{m}\textit{ɪ}\textit{ʒ}\textit{ɛ ɛnɛŋ =ahaj} ‘some other people’).  The relative clause begins with the 1PEX pronoun \textit{l}\textit{ɪ}\textit{mɛ}. The speaker brought food to those people who helped him to drive the cows. 

\ea
Dəyday  anga  fat  amədeɗe  va  nəngala  a  mogom  waya  amazata  ala  ɗaf  ana  [məze  enen  ahay  \textbf{l}\textbf{ə}\textbf{me  }\textbf{aməngel}\textbf{e  alay  sla  ahay}  jəyga\textbf{  }na.]
\z

dijdaj     aŋga   fat   amɪ-dɛɗ-ɛ  =va  nə-ŋg    =ala  a  mɔg\textsuperscript{w}ɔm   



approximately  \textsc{POSS}  sun  DEP-fall-CL   =\textsc{PRF}  1S.IFV-return  =to  at  home



\textit{waja      ama-z  =ata     =ala    ɗaf     ana    }[\textit{m}\textit{ɪ}\textit{ʒ}\textit{ɛ    ɛnɛŋ   =ahaj}



because  DEP-take=3P.IO  =away  food  DAT  person  another  =Pl



\textbf{\textit{l}}\textbf{\textit{ɪ}}\textbf{\textit{mɛ     am}}\textbf{\textit{ɪ{}-}}\textbf{\textit{ŋgɛl-ɛ       =alaj   ɬa       =ahaj    }}\textit{dʒijga}\textbf{\textit{ }}\textit{  na }]



1Pex  DEP-return-CL  =away  cow  =Pl  all  PSP



‘At sunset, I went home to bring food for all the people that drove the cows [to Tokembere].’ (lit. [it was] approximately [time] belonging to the sun which already fell, I returned because to bring food to some other people that returning all cows) 


In all of the above examples, the head noun can be modified by other modifiers in addition to the relative clause. Sometimes, however, the relative clause itself is the entire noun phrase (ex. 382 - 383). These noun phrases that consist of relative clauses can take no other noun phrase modifiers. Also, they are apparently limited in the type of clause construction that they can occur in. They can only be the predicate of a larger predicate nominal construction (see also \sectref{sec:70}). Ex. 382 and 383 are interrogative constructions with a predicate nominal structure (see \sectref{sec:74}). We found no natural examples where a headless relative clause served as a matrix component in a matrix verbal clause. Ex. 383 is an emphatic construction (which will be discussed in \sectref{sec:78}). 

\ea
{}[Aməzəɗe  dəray  na ]  way?
\z

{}[${\emptyset}$    amɪ-ʒɪɗ{}-ɛ    dəraj  na ]  waj ?



DEP-carry-CL    head  PSP  who



‘Who will win?’ (lit. the one to carry the head, who?) 



Snake S. 7


\ea
 [Amədəvala  okfom  nehe  na ]  almay?
\z

{}[${\emptyset}$   amə-dəv=ala     ɔk\textsuperscript{w}fɔm   nɛhɛ    na ]  almaj



DEP-fall=to     mouse    DEM   PSP  what 



‘What made this mouse fall?’ (lit. to fall this mouse, what?)


Noun phrases with relative clauses can get quite complicated in Moloko even though they only occur in specific places in discourse. In ex. 384, there are two relative clauses together, both modifying the head noun \textit{ɛlɛ} ‘thing.’ In the first (\textit{nɛ ama-h=aŋ} the thing ‘that I told her’) the head of the noun phrase corresponds to the direct object of the verb in the relative clause (marked as ${\emptyset}$ in the example). In the second (\textit{amɪ-dʒ-ijɛ mɛ-g-ɛ}\textit{ }\textit{baj} the thing ‘that I said she should not do’) there is an embedded complement clause within the relative clause (delimited by lines). In this second relative clause, the element that corresponds to the head of the noun phrase is represented by ${\emptyset}$ within the complement clause.


Disobedient Girl S. 29


\ea
Agə  na  va  [ele  \textbf{ne    amahan}\textbf{  }aməjəye  {\textbar}mege  bay {\textbar}  na ]  esəmey.   
\z

à-gə             na       =va    [ɛlɛ     \textbf{nɛ    ama-h=aŋ}\textbf{ }       ${\emptyset}$   



3S.PFV-do     3S.DO   =\textsc{PRF}   thing   1S     DEP-say=3S.IO  



\textit{am}\textit{ɪ}\textit{{}-dʒ-ij}\textit{ɛ}\textit{    }{\textbar}\textit{m}\textit{ɛ}\textit{\`{ }-g-}\textit{ɛ}\textit{        }${\emptyset}$\textit{    baj } {\textbar}\textit{   na } ]\textit{   }\textit{ɛʃɪ}\textit{m}\textit{ɛ}\textit{j}



DEP-say-CL    3S.HOR-do-CL    \textsc{NEG}    PSP    not so



‘She has done the thing that I told her she should not do, not so?’ (lit. she did it, the thing to say to her to say she should not do)


Plural head nouns in noun phrases containing a relative clause have so far only been noted in elicited relative clauses and their interpretation is ambiguous. In these noun phrases, speakers insert the plural \textit{=ahaj}  in one of two places: the plural \textit{=ahaj} can occur immediately following the head noun, or in some instances it may follow the relative clause. Ex. 385 and 386 show the plural in front of the relative clause. 

\ea
{}[Ele  ahay  [\textbf{nok  }\textbf{aməzəɗ}\textbf{e }]  na],  anga  əwla  bay.
\z

{}[ɛlɛ     =ahaj  [\textbf{nɔk}\textsuperscript{w}\textbf{  amɪ-ʒɪɗ{}-ɛ }]   na ]  aŋga  =uwla    baj



thing  =Pl  2S  DEP-take-CL  PSP  \textsc{POSS}  =1S.POSS  \textsc{NEG}



‘The things that you brought [are] not belonging to me.’


\ea
{}[Məze  ahay  [\textbf{am}\textbf{əzə}\textbf{ɗe  dəray }]  na ],  tolo  a  mogom  nə  memle  ga.
\z

{}[mɪʒɛ  =ahaj  [\textbf{amɪ-ʒɪɗ{}-ɛ  dəraj }]  na ]  tɔ-lɔ  a  mɔg\textsuperscript{w}ɔm  nə  mɛmlɛ  ga



person  =Pl  DEP-take{}-CL  head  PSP  3P-go  at  home  with  joy  ADJ



‘The people that won went home with joy.’


When the plural \textit{=ahaj}  occurs after the relative clause (ex. 388), exactly what is pluralised is ambiguous. When plural forms are used in Moloko discourse, which possibility is correct must be already clear from the context. Ex. 387 and 388 give the singular and plural head nouns with following relative clauses. In ex. 387 - 388, the possibilities are chief’s house/ chief’s houses / chiefs’ house / chiefs’ houses,’ depending on if \textit{ndam}, \textit{haj}, \textit{bahaj}, or all three are pluralised. 

\ea
Dala  slərele  asan  ana  [məze  \textbf{am}\textbf{ə}\textbf{here  hay  a  bahay. }] 
\z

dala       ɬərɛlɛ      a-s=aŋ      ana  [mɪʒɛ      ${\emptyset}$\textbf{  amɪ-hɛr-ɛ  haj  a  bahaj }] 



money  work        3S-please=3S.IO   DAT    person    DEP-build-CL  house  GEN  chief   



‘The person (the one) that built the chief’s house wants his wages (lit. work money pleases him).’


\ea
Dala  slərele  asata  ana  [ndam  \textbf{am}ə\textbf{here  hay  a  bahay  }ahay. ]  
\z

dala       ɬɪrɛlɛ    a-s  =ata          



money    work  3S-please=3P.IO        



‘Wages please’



\textit{ana}     [\textit{ndam   ${\emptyset}$}\textbf{\textit{  amɪ-hɛr-ɛ  haj  a  bahaj}}\textit{  =ahaj }]  



DAT  people    DEP-build-CL   house  GEN  chief  =Pl 



‘the people that built the chief’s house/ chief’s houses / chiefs’ house / chiefs’ houses.’ 


The end of the relative clause is sometimes delimited by the presupposition marker \textit{na} (ex. 374,  379, 382, 389, Chapter 12). \textit{Na} indicates that the relative clause contains previously shared (or presupposed) information. \textit{Na} also physically delineates the end of the relative clause. The relative clauses in the following examples are bolded and the noun phrase is delimited by square brackets. In ex. 389, the presupposition marker \textit{na} is underlined. 


Disobedient Girl S. 38


\ea
Metesle  anga  [war  dalay  ngəndəye  [amazata  aka  ala  avəya  nengehe  ana  məze  ahay   na ]. ]
\z

M\textbf{ɛ}t\textbf{ɛ}ɬ\textbf{ɛ  }anga    [war    dalaj  ŋgəndəj\textbf{ɛ} 



\textsc{NOM}{}-curse   \textsc{POSS}   child    girl       DEM   



{}[\textbf{${\emptyset}$  }\textbf{\textit{ama-z=ata     =aka  =ala     avija           nɛŋgɛhɛ  ana    m}}\textbf{\textit{ɪʒ}}\textbf{\textit{ɛ     =ahaj }}]\textbf{\textit{    }}\textbf{\textit{na}} ]\textit{ }



DEP{}-take=3P.IO =on   =to        suffering     DEM      DAT   person    =Pl   PSP



‘The curse belongs to that young woman that brought this suffering onto the people.’  


Any information inside a relative clause is known or presupposed information expected to be shared by the hearer. Relative clauses function in two ways. Firstly, relative clauses may specify the head noun among others. Secondly, in a narrative, relative clauses identify their content as carrying information concerning a key participant in the discourse and may allude to the moral of the story. 

Consider the Disobedient Girl text (see \sectref{sec:1.5} for the full narrative). The moral of the story is to instruct children (especially girls) to be obedient. There are relative clauses in S. 13 (ex. 390), S. 29 (ex. 384), S. 33 (ex. 376), and S. 38 (ex. 374). Note that all but one (ex. 390) of the relative clauses in this narrative concern the moral of the story. The moral of the Disobedient Girl involves suffering of a particular nature that was brought on by a particular girl who disobeyed specific instructions. The instructions that she disobeyed are in a relative clause within the husband’s lament when he finds her (ex. 384 from line 29). The disobedient girl is the head of two relative clauses at the end of the story, one citing her as the reason that God got angry (ex. 376 from line 33) and the other stating that she brought suffering to the Moloko people (ex. 374 from line 38). The only relative clause that does not concern the moral of the story (ex. 390) is from a section in the narrative where the man instructs his wife on how much millet to grind. The man tells her to take one grain of millet. Then he specifies with a relative clause ‘that one grain of millet you have taken.’ This relative clause specifies the one grain of millet (from the other grains in the sack) that will be multiplied for them. 


Disobedient Girl S.13


\ea
Asa  asok  aməhaya  na,  kázaɗ  war  elé  hay  bəlen. Káhaya  na  kə  ver  aka.  [War  elé  hay  bəlen  ga  nəndəye  \textbf{nok  ameze  }na,]  anjaloko  de  pew.
\z

asa  à-s          \textbf{=}ɔk\textsuperscript{  }amə-h            =aja  na   



if     3S.PFV-please=2S.IO  DEP.PFV-grind=\textsc{PLU}    PSP  



‘If you want to grind,’



\textit{ká-zaɗ      war    ɛlɛ  haj       bɪlɛŋ}



2S.IFV-take  child    eye  millet  one



‘you take only one grain.’



{}[\textit{war     ɛ}\textit{l}\textit{ɛ  haj    bɪlɛŋ  ga   ŋɪndijɛ   }\textbf{\textit{nɔk}}\textit{\textsuperscript{w}}\textbf{\textit{      amɛ-}}\textbf{\textit{ʒ}}\textbf{\textit{{}-ɛ    }}\textit{  na }]



child  eye       millet     DEM  ADJ  DEM  2S   DEP-take{}-CL    PSP



‘That one grain that you have taken,’



\textit{ká-h=aja               na  kə  vɛr  aka  à-nz=alɔk}\textit{\textsuperscript{w}}\textit{ɔ    dɛ  pɛw}



2S.IFV-grind=\textsc{PLU}  3S.DO  on  stone     on            3S.PFV-suffice=1\textsc{Pin}   enough  done



‘grind it on the grinding stone, and it will suffice for all of us.’


Note that the relative clauses that contain information about the moral of the story are at the end of the narrative; the noun phrases in S.10 (ex. 391) that introduce her and identify her as disobedient contain no relative clause. 

Disobedient Girl S.10

\ea
Olo  azala  [hor  ndana]  [war  dalay  ga  ndana ]  [cezlere  ga.]
\z

à-lɔ           à-z  =ala      [h\textsuperscript{w}ɔr   ndana]   [war      dalaj  ga    ndana ]   [tʃɛɮɛrɛ       ga ]



3S.PFV-go   3S.PFV-take=to  woman  DEM  child    girl     ADJ  DEM  disobedience ADJ



‘He went and took that above-mentioned woman; that above-mentioned young girl [was] disobedient.’


In the Snake narrative (see \sectref{sec:1.4}), there is only one relative clause. This relative clause shows another function of relative clauses in discourse. The relative clause, \textit{amə-dəv=ala ɔk}\textit{\textsuperscript{w}}\textit{fɔm nɛhɛ }‘the thing that caused the mouse to fall’ in line 7 (ex. 383), contains the first mention (albeit indirect) of the snake who is a central participant in the story and the reason that the story was told. 

\section{Coordinated noun phrases}
\hypertarget{RefHeading1211801525720847}{}
The basic way to coordinate two participants in Moloko is to join two noun phrases by the conjunction \textit{nə }\textit{ }‘with’ (see \sectref{sec:45}). Modifiers will have semantic scope over both of the coordinated elements. In the examples below, the noun phrases will be delimited by square brackets and the conjunctions will be bolded. 


\ea
Ləbara  anga  [ [bahay  a  hay ]  \textbf{nə}  [ndam  slərele  ahan  ahay  makar. ] ]
\z

ləbara  aŋga  [ [bahaj   a   haj ]  \textbf{nə}  [ndam   ɬɪrɛlɛ   =ahaŋ       =ahaj           makar ] ]



news  \textsc{POSS}  chief  GEN  house  with  people  work  =3S.POSS    =Pl    three



‘The story [is] belonging to the chief of the house with his three workmen.’



Values S. 47


\ea
Nəmbəɗom  a  dəray  ava  na,  ka\textbf{ } [ [ [kərkaɗaw  ahay]  \textbf{nə}  [hərgov  ahay ]  ga ]  [a  ɓərzlan  ava   na ]]
\z

nə-mbʊɗ{}-ɔm      a  dəraj  ava  na



1.PFV-change-1Pex  to   head  in    PSP  



‘We have become’



\textit{ka}\textbf{\textit{ }}\textit{ }[ [ [\textit{kərka}\textit{ɗ}\textit{aw  =ahaj }]\textit{  }\textbf{\textit{nə}}\textit{   }[\textit{h}\textit{ʊ}\textit{rg}\textit{\textsuperscript{w}}\textit{ɔv}\textit{ =ahaj}\textit{ }] \textit{ ga} ]\textit{    }[\textit{a      ɓərɮaŋ     ava }]\textit{   na}]



like        monkey    =Pl        with    baboon      =Pl  ADJ  in    mountain    in    PSP



‘like monkeys and baboons in the mountain.’


\ea
{}[Zar  \textbf{nə}  [hor  ahan ] ]  tolo  a  mehele  ava.
\z

{}[zar    \textbf{nə}  [h\textsuperscript{w}ɔr  =ahaŋ ] ]  tɔ-lɔ  a  mɛ-hɛl-ɛ    ava



man    with  woman  =3S.POSS  3P-go  in  \textsc{NOM}{}-unite{}-CL  in



‘A man and his wife went to the meeting.’ 


\section{Adpositional phrase}
\hypertarget{RefHeading1211821525720847}{}
Adpositional phrases function to relate noun phrases to the clause, expressing physical, grammatical, or logical relationships.  Friesen and \citet{Mamalis2008} found two types of adpositional phrases in Moloko; simple and complex. Simple adpositional phrases (see \sectref{sec:45}) consist of an adposition followed by the noun phrase. Complex adpositional phrases (see \sectref{sec:46}) consist of a noun phrase framed by a preposition and a postposition. 

\subsection{Simple adpositional phrase}
\hypertarget{RefHeading1211841525720847}{}
There are seven adpositions in Moloko: \textit{a} ‘to,’ \textit{ana} ‘to’ \textit{nə} ‘with,’ \textit{aka} ‘on,’ \textit{a}\textit{ŋ}\textit{ga} ‘belonging to,’ \textit{afa} ‘at the house of,’ and \textit{ka} ‘like.’  

The preposition \textit{a} ‘at’\footnote{This particle also functions at the noun phrase level as the genitive particle (\sectref{sec:42}). } marks the relationship of location of the event (at, to, in; ex. 395, 396).


Cicada S. 4



\ea
Tənday  tətalay  \textbf{a  }ləhe.
\z

tə-ndajtə-tal-aj    \textbf{a}  lɪhɛ



3P-PRG  3P-walk{}-CL  at  bush



‘They were walking in the bush.’ 


\ea
Olo  \textbf{a}  Marva.
\z

ɔ{}-lɔ    \textbf{a}  Marva



3S.PFV-go  at  Maroua



‘He/she went to Maroua.’ 


The adposition \textit{ana} ‘to’ marks the indirect object which is the place where the action of the verb occurs; the recipient, benefactive, or malefactive (ex. 397, 398, see \sectref{sec:9.2} for a discussion of semantic roles).

\ea
Tolo  na,  tasan  oko  \textbf{ana  }hay  aməgəye  na  va.
\z

tə-lɔ    na  ta-s=aŋ    ɔk\textsuperscript{w}ɔ  \textbf{ana  }haj  amɪ-g-ijɛ   na   =va



3P-go  PSP  3P-cut=3S.DO  fire  DAT  house  DEP-do-CL  3S.DO   =\textsc{PRF}



‘They went and set fire to the house that he had built.’


\ea
Adəkaka  alay  \textbf{ana}  Hərmbəlom.
\z

a-dək\textsuperscript{w}   =aka  =alaj  \textbf{ana}  Hʊrmbʊlɔm



3S-arrive     on  =away  DAT  God



‘It reached God.’


The adposition \textit{nə} ‘with’ marks the instrument (ex. 399) or comitative (accompaniment) relation (ex. 400, 401). It is more natural in Moloko for a plural subject to be expressed via an accompaniment expression (ex. 400) than it is to use a coordinated noun phrase (ex. 401). The adposition is also used to form the verb focus construction (ex. 402, see \sectref{sec:61}).

\ea
Naslay  sla  \textbf{nə  }mekec.
\z

na-ɬ{}-aj  ɬa  \textbf{nə  }mɛkɛtʃ



1S-slay{}-CL  cow  with  knife



‘I kill the cow with a knife.’


\ea
Olo  \textbf{nə}  zar  ahan.
\z

ɔ{}-lɔ     \textbf{nə}   zar   =ahaŋ



3S-go  with  man  =3S.POSS



‘She went with her husband.’


\ea
Zar  \textbf{nə}  hor  ahan  təta  a  mogom.
\z
\ zar    \textbf{nə}  h\textsuperscript{w}ɔr  =ahaŋ    təta  a  mɔg\textsuperscript{w}ɔm



man    with  woman  =3S.POSS  3P  at  home



‘The man and his wife [are] at home.’ 


\ea
Nəskom  awak  \textbf{nə}  məskwəme.
\z

nə-sk\textsuperscript{w}ɔm    awak  \textbf{nə}  mɪ-sk\textsuperscript{w}øm-ɛ



1S.PFV-buy/sell  goat  with  \textsc{NOM}{}-buy/sell-CL



‘I really bought the goat.’ (lit. I bought the goat with buying)


The adposition \textit{nə} ‘with’ also participates in forming comparative constructions in Moloko. When one noun phrase is compared with another, it is done by means of a clause construction using the verb \textit{dal}, ‘overtake.’\footnote{The verb \textit{dal} ‘overtake’ takes subject prefixes and carries aspectual tone. Other constructions can be employed when comparing people (ex. 535) or ideas (line 49 in the Values exhortation).} The standard of comparison (\textit{baba =ahaŋ} ‘his father’ in ex. 403 and 404, and \textit{mədəga =ahaŋ} ‘his older sibling’ in ex. 405) is the direct object of the verb. The quality being compared (\textit{ʃɪbɛr} ‘tallness’ in ex. 403, \textit{gədaŋ}\textit{ }‘strength’ in ex. 404, and \textit{m}\textit{ɪʃɪ}\textit{rɛ ɛlɛ }‘knowledge’ in ex. 405) follows in an adpositional phrase.  

\ea
War  ahan  ádal  baba  ahan  nə  səber.
\z

war =ahaŋ     á-dal       baba   =ahaŋ     nə   ʃɪbɛr



child   =3S.POSS   3S.IFV-overtake   father   =3S.POSS  with  tallness



‘The child is taller than his father.’ (lit. his child surpasses his father with height)  


\ea
War  ahan  ádal  baba  ahan  nə  gədan.
\z

war   =ahaŋ    á-dal     baba   =ahaŋ     nə   gədaŋ



child   =3S.POSS   3S.IFV- overtake   father   =3S.POSS  with  strength



‘The child is stronger than his father.’ 


\ea
War  na,  á-dal  mədəga  ahan  nə  məsəre  ele.
\z

war   na   á-dal        mədəga   =ahaŋ     nə   mɪ{}-ʃɪr-ɛ     ɛlɛ



child   PSP   3S.IFV- overtake    older sibling   =3S.POSS  with   \textsc{NOM}{}-know-CL  thing



‘The child is smarter than his older sibling.’ (lit. the child is greater than his older sibling with respect to knowledge)


No ‘less than’ comparatives were found in the data. Superlative constructions are possible but are not used often in Moloko culture.  Ex. 406 illustrates what people say in an elicitation context.

\ea
Ádal  məze  ahay  jəyga  nə  məsəre  ele  a  lekwel  ava.
\z

á-dal     mɪʒɛ   =ahaj   ʣijga~   nə  mɪ{}-ʃɪr-ɛ    ɛlɛ  a  lɛk\textsuperscript{w}ɛl  ava



3S.IFV- overtake  person  =Pl  all  with  \textsc{NOM}{}-know-CL  thing  at  school  in



‘He/she is the smartest child in his school.’


The adposition \textit{aka}\textit{ ‘}on’ is used with the verb \textit{lɔ} ‘go’ to mark the purpose of a trip (ex. 407).

\ea
Aban  olo  \textbf{aka}  yam.
\z

Abaŋ  ɔ{}-lɔ   \textbf{aka}   jam



Aban  3S-go  on  water



‘Aban goes to get water.’ (lit. she goes on water)


The adposition \textit{aŋga} indicates possession. The predicate possessive construction is discussed in \sectref{sec:70.} In the possessive construction, \textit{aŋga} indicates a possessive relationship between the noun in the adpositional phrase and the other noun phrase in the construction. In ex. 408, \textit{aŋga}  indicates that \textit{dəraj} ‘head’ is possessed by \textit{lɪmɛ} ‘us.’

\ea
{}[Dəray  ga ]  [\textbf{anga}  ləme. ]
\z

{}[dəraj  ga ]    [\textbf{aŋga}  lɪmɛ ]



head  ADJ    \textsc{POSS}  1\textsc{Pex}



‘We got the head.’ (lit. the head, belonging to us)


The adposition \textit{afa} ‘at the house of’plus a noun phrase gives a location at the house of the referent specified in the noun phrase (ex. 409). 

\ea
Nolo  afa  bahay.
\z

nʊ{}-lɔ   afa    bahaj



1S-go  to house of  chief



‘I go to the chief’s house.’


The adposition \textit{ka}\textit{ }‘like’ intoduces an adverbial complement that expresses manner. \textit{Ka} appears twice in ex. 410. In the second instance, \textit{ka} carries the directional extension \textit{ala} ‘towards.’


Values S. 47


\ea
Nəmbəɗom  a  dəray  ava  na,  [\textbf{ka } kərkaɗaw  ahay  nə  hərgov  ahay  ga  a  ɓərzlan  ava  na],  [\textbf{ka}  ala  kəra    na ],  nəsərom  dəray  bay  pat.  
\z

nə-mbʊɗ{}-ɔm    a  dəraj  ava  na



1.PFV-change-1Pex  at   head  in    PSP  



{}[\textbf{\textit{ka  }}\textit{kərka}\textit{ɗ}\textit{aw  =ahaj    nə   h}\textit{ʊ}\textit{rg}\textit{\textsuperscript{w}}\textit{ɔv  =ahaj    ga     a      ɓərɮaŋ     ava    na}]



like  monkey   =Pl  with    baboon  =Pl  ADJ  at  mountain    in    PSP



{}[\textbf{\textit{ka }}\textit{  =ala   kəra    na }],\textit{  nə-s}\textit{ʊ}\textit{r-ɔm     dəraj   baj   pat}



like  =to  dog      PSP  1.PFV-know-1Pex  head  \textsc{NEG}  all



‘We have become (lit. changed in the head) like monkeys and baboons on the mountains, [and] like dogs, we don’t know anything!’


\subsection{   Complex adpositional phrase}
\hypertarget{RefHeading1211861525720847}{}
There are two complex adpositional phrases, each composed of the combination of a preposition and a postposition that surround the noun phrase. The adpositions give locational information. The first, \textit{kə…aka} ‘on’ marks the noun phrase as being a location on which the event expressed by the verb is directed. It can be employed in a physical sense (ex. 411 - 413) or a figurative sense (ex. 414).  


Cicada S. 9



\ea
Kafəɗom  anaw  \textbf{kə  }mahay  əwla\textbf{  aka}\textbf{.}
\z

ka-fʊɗ-ɔm    an=aw    \textbf{kə  }mahaj  =uwla\textbf{    aka}



2.HOR-place-2P  DAT=1S.IO  on  door  =1S.POSS  on



‘You should place  [the tree] at my door.’  


\ea
Enjé  \textbf{kə  }delmete  \textbf{aka  }a  slam  enen.  
\z

ɛ{}-ndʒ{}-ɛ    \textbf{kə  }dɛlmɛtɛ\textbf{    aka  }a  ɬam  ɛnɛŋ  



3S-leave-CL    on  neighbor    on  at  place  another



‘He left to go to his neighbor at some other place.’  


\ea
Azaɗ  oloko  \textbf{kə}  dəray  a  məwta  \textbf{aka}\textbf{.}
\z

à-zaɗ    ɔlɔk\textsuperscript{w}ɔ  \textbf{kə}  dəraj  a  muwta  \textbf{aka}



3S.PFV-carry  wood  on  head  GEN  truck  on



‘He/she carried the wood on top of the truck.’ (lit. on the head of the truck)


\ea
Hərmbəlom  agə  ɓərav  va  \textbf{ka  }war  anga  məze  dedelen  ga\textbf{  aka}\textbf{.}
\z

Hʊrmbʊlɔm  a-gə  ɓərav  =va  \textbf{ka  }war  aŋga  mɪʒɛ  dɛdɛlɛŋ  ga\textbf{  aka}



God    3S-do  heart  =\textsc{PRF}  on  child  \textsc{POSS}  person  black  ADJ  on



‘God was angry with the black man’s child.’  (lit. God did heart on the child that belongs to the black person)


The second complex adpositional phrase, \textit{a…ava} ‘in,’ the prepositional and postposition surround a noun phrase to mark that noun phrase as being a physical location in which the action of the verb is directed (ex. 415 and 416) .

\ea
Olo  \textbf{a}  kosoko  \textbf{ava}\textbf{.}
\z

ɔ{}-lɔ    \textbf{a}  kɔsɔk\textsuperscript{w}ɔ  \textbf{ava}



3S-go  in  market  in



‘He/she goes to market.’


\ea
Afaɗ  dala  \textbf{a}  ombolo  \textbf{ava}\textbf{.}
\z

a-faɗ  dala  \textbf{a}  ambɔlɔ  \textbf{ava}



3S-put  money  in  sack  in



‘He/she put the money into [his] sack.’


The postpositions \textit{aka} ‘on’ and \textit{ava}\textit{ }‘in’ have the same forms as the verb adpositional extensions =\textit{aka} ‘on’ and =\textit{ava}\textit{ }‘in’ (see \sectref{sec:56}). The extensions permit the presence of the complex adpositional phrase which gives further precision concerning the location of the event (ex. 417 and 418\footnote{Even though the verb in this example has verbal extensions, it is not conjugated for subject since it is a climactic point in the story where nominalised forms are often found.  This is discussed further in Sections 7.6 and 1164.  }). In the examples, the postpositions and verbal extensions are both bolded. 

\ea
Afəɗ\textbf{aka}  war  elé  háy  na  \textbf{kə}  ver  \textbf{aka}\textbf{.}
\z

a-fəɗ=\textbf{aka}  war  ɛlɛ  haj  na  \textbf{kə}  vɛr  \textbf{aka}



3S-place  =on  child  eye  millet  PSP  on  stone  on



‘She put the grain of millet on the grinding stone.’



Disobedient Girl S. 26


\ea
Məmət\textbf{ava}  alay  \textbf{a}  ver  \textbf{ava}\textbf{.}
\z

mə-mət  =\textbf{ava  }=alaj  \textbf{a}   vɛr   \textbf{ava}



\textsc{NOM}{}-die  =in  =away  in  room  in



‘She died in the room.’  


\chapter[ Verb root and stem]{ Verb root and stem}
\hypertarget{RefHeading1211881525720847}{}
In addition to analysing the phonology of Moloko, \citet{Bow1997c} studied verb morphology and also produced notes on the grammar of Moloko which were expanded by \citet{Boyd2003}; Friesen and \citet{Mamalis2008} is an analysis of the Moloko verb and verb phrase. This chapter is based on Friesen and \citet{Mamalis2008}, but it has been re-worked, reorganised, and data has been added. 

The verb is the centre of the clause in Moloko.  It expresses the action of an event, or a situation or state.  It may be the only element in a clause, or it may be accompanied by noun phrases or pronouns expressing the subject, the direct object, and the indirect object of the verb, adpositional phrases expressing location, and/or discourse markers. Ideophones (Chapter 3.6) figure greatly in the expression of the action, both when they function as adverbs and when they fill the verb slot in a clause.

Typical of a Chadic language, Moloko has a variety of extensions that modify the sense of the verb stem.{ }\footnote{Note that the term ‘extension’ for Chadic languages has a different use than for Bantu languages. In Chadic languages, ‘extension’ refers to particles or clitics in the verb word or verb phrase. } It has 6 extensions which specify location of the event, direction with respect to centre of reference, and the Perfect.  An underspecified valence system (Chapter 9) allows variable transitivity usage for a given verb. In Moloko, valence-changing operations are not achieved through morphological modifications of the verb (for example with causative, applicative, and passive affixes). Transitivity is a clause-level property that carries a grammatical function. 

The Moloko verb and verb phrase are treated in four separate chapters. We distinguish verb root, stem (both described in Chapter 6), verb word – renamed ‘verb complex’ for Moloko (verb stem plus affixes and extensions, Chapter 7), verb and transitivity types (Chapter 9), and finally verb phrase (Chapter 8). 

\section{The basic verb root and stem}
\hypertarget{RefHeading1211901525720847}{}
\citet{Bow1997c} found that the verb root in Moloko consists of one to four consonants and perhaps a vowel. The verb root by itself never occurs in the language. In discussing the verb in Moloko it is more profitable to consider the verb stem as the most basic lexical unit. The Moloko verb stem itself is already complex. Friesen and \citet{Mamalis2008} determined that in order to pronounce a verb stem in Moloko, a speaker needs to know the following six features: 


\begin{itemize}
\item the consonantal skeleton of the verb root (\sectref{sec:6.2}).
\item if the stem carries the \textit{{}-aj} suffix (\sectref{sec:6.3}).  
\item if the root has an underlying vowel (\sectref{sec:6.4}). 
\item if the stem carries the \textit{a-} prefix (\sectref{sec:6.5}).  
\item the prosody of the stem (labialised, palatalised, or neutral, \sectref{sec:6.6}). 
\item the tone class of the stem (high, low, or toneless, \sectref{sec:6.7}). 
\end{itemize}

The structural arrangement of the six features is diagrammed in \figref{fig:10}.. 


←   tone pattern   →



←←←←←←prosody


\begin{tabular}{lll}
\lsptoprule
\hhline{~-~}
 \textit{a}\textit{{}-}\par & root\par

 C  (C) (C) (VC) & \textit{{}-}\textit{aj}\par\\
\hhline{~-~}
\lspbottomrule
\end{tabular}

\begin{itemize}
\item \begin{styleFiguretitle}
Structure of the verb stem
\end{styleFiguretitle}\end{itemize}
\section{The consonantal skeleton of the root}
\hypertarget{RefHeading1211921525720847}{}
Moloko verb roots are like those of other Afroasiatic languages in that they are built on a consonantal skeleton.  \citet{Bow1997c} found that the verb root consists of one to four consonants, although a skeleton of two consonants is most common.\footnote{Bow’s (1997c) database includes 26 one-consonant verbs, 231 two-consonant verbs, 83 three-consonant verbs, and 10 four-consonant verbs. } That Moloko verb roots are based on a consonantal skeleton can be evidenced by two facts, both of which are illustrated in \tabref{tab:37}. (adapted from Bow, 1997c). Firstly, the consonants display a unique stability when the verb is inflected.\footnote{Note there are consonantal allophones in palatalised and labialised words. } The vowels, on the other hand, change with the prosody of the inflection and whether or not the word carries stress.\footnote{Since stress is phrase-final, the final syllable of these elicited examples will always carry a ‘full’ vowel.} Secondly, there are verb roots that consist simply of one consonant and a prosody. These have no underlying root vowel, but they will acquire their vowels in the inflections. 

The underlying form of a verb stem is defined as the consonantal skeleton plus the optional presence of an underlying vowel, {}-\textit{aj} suffix, and \textit{a-} prefix, potential prosody, and tone (see Sections 6.3 - 6.7). In the examples in \tabref{tab:37}. and in the rest of this section, the underlying form will be given when necessary in addition to the phonetic pronunciation. The tone class is not shown. 

\begin{tabular}{llllll}
\lsptoprule

\textbf{Root type }\textbf{↓} & \textbf{Underlying form of stem} & \textbf{3S Perfective}

\textbf{\textit{a-}} & \textbf{3S Perfective with directional}

\textbf{\textit{a-}}\textbf{       =}\textbf{\textit{ala}} & \textbf{1}\textbf{\textsc{Pin}}\textbf{ Perfective}

\textbf{\textit{mɔ-  {}-ɔk}}\textbf{\textit{\textsuperscript{w}}} & \textbf{Nominalised form}

\textbf{\textit{mɪ-  (-ij)-ɛ}}\\
\textbf{One-consonant}

neutral

palatalised

labialised & /p -j /

/ g\textsuperscript{ e} /

/ l\textsuperscript{  o} / & \textit{a-p-aj}

‘he opens’

\textit{ɛ{}-g-ɛ}

‘he does’

\textit{ɔ{}-lɔ}

‘he goes’ & \textit{a-p=ala}

‘he opens towards’

\textit{a-g=ala}

‘he does towards’

\textit{a-l=ala}

‘he comes towards’ & \textit{mɔ-p-ɔk}\textit{\textsuperscript{w}}

‘we opened’

\textit{mɔ-g}\textit{\textsuperscript{w}}\textit{{}-ɔk}\textit{\textsuperscript{w}}

‘we did’

\textit{mɔ-lɔh-ɔk}\textit{\textsuperscript{w}}\footnotemark{}

‘we went’ & \textit{mɪ-p-ij-ɛ}

‘opening’

\textit{mɪ-g-ij-ɛ}

‘doing’

\textit{mɪ-l-ij-ɛ}

‘going’\\
\textbf{Two-consonant}

neutral

palatalised

labialised & /f ɗ /

/ ɮ g\textsuperscript{ e} /

/ndaɮ -j\textsuperscript{ o} / & \textit{a-fa}\textit{ɗ}

‘he puts’

\textit{ɛ{}-ɮɪg-ɛ}

‘he sows’

\textit{a-ndɔɮ-ɔj}

‘he explodes’ & \textit{a-fəɗ=ala}

‘he puts towards’

\textit{a-}\textit{ɮəg=ala}

‘he sows towards’

\textit{a-ndaɮ=ala}

‘it explodes towards’ & \textit{mʊ-fʊɗ-ɔk}\textit{\textsuperscript{w}}

‘we put’

\textit{mʊ-}\textit{ɮʊg}\textit{\textsuperscript{w}}\textit{{}-ɔk}\textit{\textsuperscript{w}}

‘we sowed’

\textit{mʊ-ndɔ}\textit{ɮ{}-}\textit{ɔk}\textit{\textsuperscript{w}}

‘we exploded’ & \textit{mɪ-fɪɗ-ɛ}

‘putting’

\textit{mɪ-}\textit{ɮ}\textit{ɪg-ɛ}

‘sowing’

\textit{mɪ-ndɛ}\textit{ɮ{}-}\textit{ɛ}

‘exploding’\\
\textbf{Three-consonant}

neutral

palatalised

labialised & /p ɗ k-aj /

/ ts f ɗ \textsuperscript{e} /

/ɓ r ts -j\textsuperscript{ o} / & \textit{a-pəɗək-aj}

‘he wakes’

\textit{ɛ{}-tsəfəɗ-ɛ}

‘he asks’

\textit{ɔ{}-ɓʊrts-ɔj}

‘he pounds’ & \textit{a-pəɗək=ala}

‘he wakes up’

\textit{a-tsəfəɗ=ala}

‘he asks’

\textit{a-ɓərts=ala}

‘he pounds towards’ & \textit{mʊ-pʊɗʊk}\textit{\textsuperscript{w}}\textit{{}-ɔk}\textit{\textsuperscript{w}}

‘we woke up’

\textit{mʊ-tsʊfʊɗ-ɔk}\textit{\textsuperscript{w}}

‘we ask’

\textit{mʊ-ɓʊrts-ɔk}\textit{\textsuperscript{w}}

‘we pound’ & \textit{mɪ-pɪɗɪk-ɛ}

‘waking’

\textit{mɪ-tsɪfɪɗ-ɛ}

‘questioning’

\textit{mɪ-ɓurts-ɛ}

‘pounding’\\
\lspbottomrule
\end{tabular}
\footnotetext{ Irregular form with epenthetic \textit{h} added between vowels. For complete conjugation see appendix \sectref{sec:14.2}). / l\textsuperscript{  o} / is the only single consonant verb root that is labialised. }

\begin{itemize}
\item \begin{styleTabletitle}
Consonantal skeleton of selected verb stems and selected word forms
\end{styleTabletitle}\end{itemize}

Mamalis found that the underlying consonants in a verb root can most easily be identified from the 2S imperative form (\tabref{tab:38}.. from Friesen and Mamalis, 2008). Note that palatalisation will cause an underlying /s/ to be expressed as [ʃ ] (see \sectref{sec:1.2.1}). The same verb stems are included as were in \tabref{tab:37}. as well as a few more. Prosody, underlying vowels (lines 12, 15), and the -\textit{aj} suffix (lines 4-7, 15) can also be seen in the imperative form; these features will be discussed in the sections below.

\begin{tabular}{lllll}
\lsptoprule

\textbf{Line} & \textbf{Prosody} & \textbf{Underlying form showing }

\textbf{consonants in verb root} & \textbf{2S Imperative form} & \textbf{Gloss}\\
\textit{1} & Neutral & /f ɗ / & \textit{faɗ } & ‘put’\\
\textit{2} &  & /g s/ & \textit{gas}~ & ‘catch~‘\\
\textit{3} &  & /m nz r/ & \textit{mənzar }~ & ‘look’\\
\textit{4} &  & /p -j / & \textit{p-aj} & ‘open’\\
5 &  & /t l-aj/ & \textit{tal}\textit{{}-}\textit{aj} & ‘walk’\\
\textit{6} &  & /ɬ-aj/ & \textit{ɬ}\textit{{}-}\textit{aj} & ‘kill (by cutting the throat)’\\
\textit{7} &  & /p ɗ k-aj / & \textit{p}\textit{ə}\textit{ɗak}\textit{{}-}\textit{aj } & ‘wake’\\
\textit{8} & Palatalised & / g\textsuperscript{ e} / & \textit{g-ɛ} & ‘do’\\
\textit{9} &  & /  s\textsuperscript{ e}  / & \textit{ʃ{}-ɛ} & ‘drink’\\
\textit{10} &  & / ɮ g\textsuperscript{ e  }/ & \textit{ɮɪg}\textit{{}-}\textit{ɛ} & ‘bring’\\
\textit{11} &  & / ts f ɗ\textsuperscript{  e}  / & \textit{ts}\textit{ɪ}\textit{f}\textit{ɪ}\textit{ɗ}\textit{{}-}\textit{ɛ} & ‘ask’\\
\textit{12} &  & / ts a n\textsuperscript{ e} / & \textit{tʃɛŋ} & ‘understand’\\
\textit{13} & Labialised & / l \textsuperscript{o} / & \textit{lɔ} & ‘go’\\
\textit{14} &  & /  z m\textsuperscript{ o} / & \textit{zɔm} & ‘eat’\\
\textit{15} &  & /  nd a ɮ -j\textsuperscript{ o} / & \textit{ndɔɮ}\textit{{}-}\textit{ɔj} & ‘explode’\\
\hhline{-~---}
\lspbottomrule
\end{tabular}
\begin{itemize}
\item \begin{styleTabletitle}
Underlying form of selected verb stems and imperative forms
\end{styleTabletitle}\end{itemize}

The consonants in a verb stem in Moloko are remarkably constant. We have found only two irregular verbs where there are changes in the verb consonants. Firstly, the irregular verb /l\textsuperscript{o}/ adds an epenthetic [h ] in some conjugations to break up vowels (the full conjugation of /l\textsuperscript{o}/ is in the appendix, \sectref{sec:14.2}). Secondly, the root-final \textit{ɗ} of the verb /z ɗ/ \textit{ }‘take’ drops off when affixes and clitics are added (ex. 419, 420). This process does not happen with the phonologically similar verb /f ɗ/  ‘put’ (ex. 421, 422). 


\ea
\textup{/z }\textup{ɗ/}\textup{       }=aw   =ala\textup{    →  [zawala]}
\z

take 2S.IMP  =1S.IO   =to      ‘give to me’


\ea
\textup{/z }\textup{ɗ/}\textup{        }=aka    \textup{  →  [zaka]}
\z

take 2S.IMP  =on        ‘give again’ (on top of what you gave before)


\ea
\textup{/f }\textup{ɗ/}\textup{   }=aw  =ala  \textup{  →  [fa}\textup{ɗ}\textup{uwala]}
\z

put 2S.IMP    =1S.IO  =to      ‘put on me’


\ea
\textup{/f }\textup{ɗ/}\textup{         }=aka    \textup{  →  [fa}\textup{ɗ}\textup{aka]}
\z

put 2S.IMP  =on        ‘put again’ (on top of what you put before)


\section{Underlying suffix}
\hypertarget{RefHeading1211941525720847}{}
Moloko verb stems can be divided into two subclasses based on whether an underlying suffix is present or not. Slightly over 70\% of the verb stems in Bow’s (1997c) data take the suffix \textit{{}-aj}, which can have different surface variants depending on the prosody of the stem. 

Friesen and Mamalis found that although the \textit{{}-}\textit{aj} suffix appears to have no semantic value, it does allow certain consonants to be verb root final which would otherwise not be permitted.\footnote{I.e., b, mb, d, nd, dz, nz, g, ŋg, g\textsuperscript{w}, ŋg\textsuperscript{w}, ts, w, j. See discussion on word final consonants in \sectref{sec:119.}}  However, for many verb stems, it appears to at least synchronically be simply a place-holding suffix that drops off whenever other suffixes or extensions are attached to the verb (compare columns 3 and 4 in \tabref{tab:37}.). Ex. 423 and 424 show the same verb complex with (ex. 423) and without (ex. 424) the \textit{{}-}\textit{aj} suffix.\footnote{The first line in each example is the orthographic form. The second is the phonetic form (slow speech) with morpheme breaks.}


\ea
Apay.
\z

a-p-aj



‘It opens.’


\ea
Apala.
\z

a-p=ala



‘It opens towards.’


Verb stems with the underlying suffix but no underlying (i.e., a neutral) prosody take the surface suffix form [{}-aj\textit{ }]; verb stems that are labialised carry the surface form suffix [{}-ɔj].\footnote{Prosody is applied to the verb stem since the -\textit{aj} suffix takes on the prosody of the stem (prosodies spread leftwards, \sectref{sec:2.1}).} With the exception of verbs with the root-final consonant /n/,\footnote{Stems ending in [ŋ] are all palatalised, e.g., \textit{t}\textit{ʃ}\textit{ɛ}\textit{ŋ} ‘understand.’ \textit{tʃ}\textit{ɪ}\textit{d}\textit{ʒɛ}\textit{ŋ} ‘lose.’ \textit{nd}\textit{ʒɛ}\textit{r}\textit{ɛ}\textit{ŋ} ‘groan.’ \textit{mb}\textit{ɛʃɛ}\textit{ŋ} ‘relax.’ \textit{nd}\textit{ɛ}\textit{ɬ}\textit{ɛ}\textit{ŋ} ‘make cold.’ \textit{ɓ}\textit{ɪ}\textit{rɮ}\textit{ɛ}\textit{ŋ} ‘count.’ \textit{mb}\textit{ɛ}\textit{t}\textit{ɛ}\textit{ŋ} ‘put out.’ and \textit{mb}\textit{ɛʒɛ}\textit{ŋ} ‘spoil’.  We interpret these verbs as having /n/ as final consonant because the \textit{ŋ} cannot be interpreted as direct or indirect object and also there are no other stems which end in \textit{ŋ}. } verb stems that are palatalised carry the surface form suffix [{}-ɛ]. We interpret the [{}-ɛ] in palatalised verbs as the palatalised variant of the \textit{{}-}\textit{aj} suffix for two reasons. First, [{}-ɛ] patterns the same way as the -\textit{aj} suffix (dropping off with its prosody whenever another suffix or extension is added). Second, the same rules of restriction of final stem consonants apply for palatalised verb stems as for other verb stems (see \sectref{sec:9}), and so the presence of [{}-ɛ]  allows root-final consonants which would otherwise be restricted. For example, /d/ and /g/ are both not permitted as word-final consonants (\sectref{sec:3}), but the presence of [{}-ɛ] allows verbs like [d-ɛ]  and [g-ɛ]. Examples from verb roots of one, two, and three consonants are shown in \tabref{tab:39}..\footnote{We found no three-consonant palatalised verb stems in the data. Labialised verb stems without the \textit{{}-}\textit{aj} suffix were rare.  }

\begin{tabular}{llll}
\lsptoprule

\textbf{Number of consonants} & \textbf{One-consonant verb root} & \textbf{Two-consonant verb root} & \textbf{Three-consonant verb root}\\
\textbf{Stems with no} \textbf{suffix }

No underlying prosody

Labialised verb stems

Palatalised verb stems &  & \textit{tax       }‘reach out’

\textit{ɮaŋ       }‘begin’ & \textit{mənzar           }\textit{ }‘see’

\textit{təkam             }‘taste’\\
\hhline{-~~~} & \textit{lɔ           }\textit{ }‘go’ & \textit{zɔm        }‘eat’ & \textit{sk}\textit{\textsuperscript{w}}\textit{ɔm          }\textit{ }‘buy/sell’\\
&  & \textit{tʃɛŋ        }‘understand’ & \textit{mbɛɮɛŋ        }\textit{ }‘count’

\textit{mbɛ}\textit{ʒ}\textit{ɛŋ         }\textit{ }‘spoil’\\
\textbf{Stems with suffix}

No underlying prosody

{}[-aj] suffix

Labialised verb stems 

{}[-ɔj] suffix

Palatalised verb stems  

{}[-ɛ] suffix & \textit{l}{}-\textit{aj           }\textit{ }‘dig’

\textit{dz}{}-\textit{aj        }\textit{ }‘say’ & \textit{ha}\textit{ɓ}{}-\textit{aj}\textit{       }‘dance’

\textit{lag-aj}\textit{       }‘accompany’ & \textit{tuwaɗ-aj       }‘cross’

\textit{ɬəɓat-aj         }‘repair’\\
\hhline{-~~~} &  & \textit{tsɔk}\textit{\textsuperscript{w}}{}-\textit{ɔj   }\textit{  }‘undress’

\textit{ɓɔr-ɔj      }\textit{ }‘climb’ & \textit{tʊk}\textit{\textsuperscript{w}}\textit{ɔs-ɔj       }‘cross legs’

\textit{tʊlɔk}\textit{\textsuperscript{w}}\textit{{}-ɔj       }‘drip’\\
& \textit{g-ɛ          }‘do’

\textit{ʒ}\textit{{}-ɛ           }‘smell’ & \textit{tʃ}\textit{ɪ}\textit{k}{}-ɛ\textit{       }‘stand up’

\textit{ɮ}\textit{ɪ}\textit{g-ɛ       }‘plant’ & \\
\hhline{~---}
\lspbottomrule
\end{tabular}

\begin{itemize}
\item \begin{styleTabletitle}
Stems with and without underlying suffix
\end{styleTabletitle}\end{itemize}

Because the suffix surfaces only word-finally, whenever the relevant verb is pronounced in isolation (and is thus phrase-final), the suffix syllable takes the phrase-final stress, necessitating a full vowel. It is therefore pronounced [aj] (see example 425) in verbs with neutral prosody, [ɔj] in labialised verb stems, and [{}-ɛ] in palatalised verb stems).  Whenever the verb is not phrase-final, the vowel drops and an epenthetic schwa occurs, rendering the pronunciation [i] for labialised and neutral prosody verbs (ex. 426) and [ɪ] for palatalised verbs. 


\ea
\textup{[a-paɗ-aj ]}
\z

3S-crunch{}-CL



‘It crunches.’


\ea
\textup{[a-paɗ-ijʃɛʃɛ]}
\z

3S-crunch  meat



‘He eats meat.’


%%please move \begin{table} just above \begin{tabular
\begin{table}
\caption{(adapted from Bow, 1997c and Boyd 2003) illustrates pairs of verb stems that have the same consonantal shape but with and without the -\textit{aj} suffix.}
\label{tab:40}
\end{table}

\begin{tabular}{lll}
\lsptoprule

\textbf{Underlying Form of Stem} & \textbf{Verb Stem} & \textbf{Gloss}\\
/bar/

/bar-aj/ & \textit{ɓár}

\textit{ɓár-aj} & ‘shoot an arrow’

\itshape \textup{‘toss and turn when sick’}\\
/tsar/

/tsar-aj/ & \textit{tsár}

\textit{tsàr-àj} & ‘taste good’

‘tear’\\
/dar/

/dar-aj/ & \textit{dàr}

\textit{dàr-àj} & ‘move’ 

‘plant’\\
/ɗak/

/ɗak-aj/ & \textit{ɗàk}

\textit{ɗàk-aj} & ‘fill up a hole’

‘show’/’tell’\\
/faɗ/

/faɗ-aj/ & \textit{fàɗ}

\textit{fáɗ-aj} & ‘put’

‘fold’\\
/f t/

/fat-aj/ & \textit{fàt }

\textit{fàt-àj} & ‘grow’(plant)

‘lower’\\
/g r/

/gar-aj/ & \textit{gár }

\textit{gár-àj} & ‘grow’ (human)

‘govern’\\
/h ɓ/

/haɓ-aj/ & \textit{hàɓ }

\textit{hàɓ-aj} & {\itshape \textup{‘break’}}

‘dance’\\
/k ɗ/

/kaɗ-aj/ & \textit{káɗ}

\textit{káɗ-áj} & ‘kill’

‘prune’\\
/ɬ r/

/a-ɬar/ & \textit{ɬar}

\textit{ɬar-aj} & ‘send’

‘slide’\\
/mb d/

/mbad-aj/ & \textit{mbàɗ}

\textit{mbáɗ-áj} & ‘change position’

‘swear’\\
/ng r/

ngar-aj / & \textit{ŋgár}

\textit{ŋgàr-aj} & ‘prevent’

‘rip’\\
/s k/

/sak-aj/ & \textit{sák}

\textit{sàk-aj} & ‘multiply’

‘sift’\\
/t r/

/tar-aj/ & \textit{tár}

\textit{tàr-áj} & ‘enter’

‘call’\\
/v r/

/var-aj/ & \textit{vár}

\textit{vàr-àj} & ‘roof’ (a house)

‘chase away’\\
/w l/

/wal-aj/ & \textit{wál}

\textit{wál-áj} & ‘attach’

‘look among things’\\
/w s/

/was-aj/ & \textit{wàs}

\textit{was-aj} & ‘cultivate’

‘populate’\\
\lspbottomrule
\end{tabular}

\begin{itemize}
\item \begin{styleTabletitle}
Verb stems with and without -aj suffix
\end{styleTabletitle}\end{itemize}
\section{Underlying vowel in the root}
\hypertarget{RefHeading1211961525720847}{}
\citet{Bow1997c} noted that no Moloko verb root has more than one underlying internal vowel and many Moloko verb roots have no underlying vowels (see \tabref{tab:38}.).\footnote{Bow, 1997c, page 24. Her database of 350 verb stems has 189 with the internal vowel.}  The presence of an underlying internal vowel in the verb stem (if any) can be determined by studying the second plural imperative. Bow illustrates the following minimal pair. The verb stems  /ts r/ ‘climb’ and /tsar/ ‘taste good’ have identical surface forms in the second person singular imperative (ex. 427, 428) due to stress on the final syllable, which necessitates a full vowel. However, the presence of the underlying vowel can be seen in the second person plural imperative (ex. 429, 430).\footnote{The 2P imperative is formed by adding the suffix -\textit{ɔ}\textit{m} and labialisation prosody.} The verb root for ‘climb’ does not have an underlying vowel, so a schwa is inserted and labialised to become [ʊ] (ex. 429). On the other hand, the verb root for ‘taste good’ has an internal vowel which becomes [ɔ] when labialised (ex. 430).


\ea
\textup{[tsar]  }\textup{  }
\z

‘climb!’ (2S)    


\ea
\textup{[tsar]}
\z

‘taste good!’ (2S)\textit{ }


\ea
\textup{[tsʊr-ɔm]}\textup{    }\textup{  }
\z

‘climb!’ (2P)    


\ea
\textup{[tsɔr-ɔm]}
\z

‘taste good!’ (2P)


%%please move \begin{table} just above \begin{tabular
\begin{table}
\caption{(from Friesen and Mamalis, 2008) shows several other examples. Single consonant roots have no internal vowel (line 1). Two and three-consonant roots may have no internal vowel (lines 2-4) or an internal vowel (lines 5-7). All four-consonant roots have an internal vowel (line 8).}
\label{tab:41}
\end{table}

\begin{tabular}{llllll}
\lsptoprule

\textbf{Line} & \textbf{Presence or absence of internal vowel} & \textbf{2S Imperative} & \textbf{2P Imperative} & \textbf{Consonantal skeleton with stem vowel} & \textbf{Gloss}\\
1 & \textbf{No internal vowel} & \textit{ɬ{}-aj           } & \textit{ɬ{}-ɔm} & /ɬ/ & ‘kill’\\
2 &  & \textit{tar             } & \textit{t}\textit{ʊ}\textit{r-ɔm} & /t r/ & ‘enter’\\
3 &  & \textit{həm-aj        } & \textit{h}\textit{ʊ}\textit{m-ɔm} & /h m{}-aj/ & ‘run’\\
4 &  & \textit{mənzar    } & \textit{m}\textit{ʊ}\textit{nz}\textit{ʊ}\textit{r-ɔm} & /m nz r/ & ‘see’\\
5 & \textbf{Internal vowel} & \textit{tar-aj            } & \textit{tɔr-ɔm} & /tar{}-aj/ & ‘call’\\
6 &  & \textit{ndɔɮ-ɔj   } & \textit{ndɔɮ-ɔm} & / \textsuperscript{w}ndaɮ/ & ‘explode’\\
7 &  & \textit{məndats-aj   } & \textit{m}\textit{ʊ}\textit{ndɔts-ɔm} & /m ndats-aj/ & ‘gather’\\
8 &  & \textit{bədzəgam-aj} & \textit{b}\textit{ʊdzʊg}\textit{\textsuperscript{w}}\textit{ɔm-ɔm} & /b dz ɡam -j/ & ‘crawl’\\
\hhline{-~----}
\lspbottomrule
\end{tabular}

\begin{itemize}
\item \begin{styleTabletitle}
Presence or absence of internal vowel
\end{styleTabletitle}\end{itemize}

Bow discovered that when an underlying vowel exists in the root, it always immediately precedes the final root consonant, so possible verb roots could take the following forms (disregarding affixes):  C, CC, CaC, CCC, CCaC, CCCaC.{ }These ‘full’ vowels will remain full in all inflections of the verb, and will be affected by the prosodies of the forms, resulting in surface [a, ɛ, ɔ, œ].  In syllables where there are no underlying vowels, an epenthetic schwa is inserted between certain consonant clusters to facilitate pronunciation in the inflected forms. On stressed syllables, the schwa will become its full vowel counterpart (see example 427). 

\section{Underlying prefix}
\hypertarget{RefHeading1211981525720847}{}
The verb stems in one class of bi-consonantal verbal stems take subject prefixes with the full vowel /a/ instead of the epenthetic schwa. \citet{Bow1997c} called this a historical \textit{a-} prefix on the verb stem.{ }She reported that 83 out of 231 bi-consonantal verb stems that she studied have the (now frozen)  \textit{a-} prefix. Whether a verb stem has this prefix or not can be determined from the nominalised form. Bow illustrates the presence of this prefix with the minimal pair /a-ndaw/ ‘swallow’ and /ndaw/ ‘insult.’ Ex. 431 and 432 show the nominalised form of the two verb stems.\footnote{The nominalised form has a \textit{m}\textit{ɪ}\textit{{}-} or \textit{mɛ-} prefix, an \textit{{}-ɛ} suffix, and is palatalised (\sectref{sec:7.6}).} The verb stem [mɪ-ndɛw-ɛ] ‘swallow’ does not have the \textit{a-} prefix. The verb stem [mɛ-ndɛw-ɛ] ‘insult’ has the \textit{a-} prefix (shown by the full vowel [ɛ] in the prefix).


\ea
məndewe    
\z

mɪ-ndɛw-ɛ      



\textsc{NOM}{}-swallow-CL    



‘swallowing’      


\ea
mendewe
\z

mɛ-ndɛw-ɛ



\textsc{NOM}{}-insult-CL



‘insulting’


Bow proposed that synchronically, the \textit{a-} prefix verb stems represent a separate class of verb stems. \tabref{tab:42}. (adapted from Bow, 1997c) shows other minimal pairs giving evidence of the presence of the \textit{a-} prefix. Those with \textit{mɛ} in the initial syllable contain the \textit{a-} underlying prefix; those with \textit{m}\textit{ɪ} in the initial syllable do not have the \textit{a-} prefix.

\begin{tabular}{llll}
\lsptoprule

\textbf{Underlying form} & \textbf{Gloss} & \textbf{Nominalised form} & \textbf{Underlying tone of stem}\footnotemark{}\\
 /ndaw -j/ & ‘swallow’ & [mɪ-ndɛw-ɛ] & toneless\\
/a-ndaw -j/ & ‘insult’ & [mɛ-ndɛw-ɛ] & L\\
/ɮ r/ & ‘pierce’ & [mɪ-ɮɪr-ɛ] & H\\
/a-ɮ r/ & ‘kick’ & [mɛ-ɮɪr-ɛ] & L\\
/tsah -j/ & ‘ask’ & [mɪ-tʃɛh-ɛ] & H\\
/ a-tsah -j/ & ‘scar’ & [mɛ-tʃɛh-ɛ] & L\\
/law -j/ & ‘hang’ & [mɪ-lɛw-ɛ] & L\\
/a-law -j/ & ‘mate’ & [mɛ-lɛw-ɛ] & L\\
/k w -j/ & ‘get drunk’ & [mɪ-kuw-ɛ] & L\\
/a-k w -j/ & ‘search’ & [mɛ-kuw-ɛ] & L\\
\lspbottomrule
\end{tabular}
\footnotetext{ Note that the underlying tone of \textit{a-}\textit{ }prefix verb stems is always low (see discussion in \sectref{sec:6.7})}

\begin{itemize}
\item \begin{styleTabletitle}
Minimal pairs showing presence of historical /a-/ prefix.  
\end{styleTabletitle}\end{itemize}

Note that the \textit{a-} prefix carries very little lexical weight; there appears to be no semantic reason for its presence. Contrast is lost between \textit{a-} prefix verb forms and those without the prefix in irrealis mood (see \sectref{sec:53}). Ex. 433 and 434 show that the Potential form for the verbs /a-ndaw/ ‘swallow’ and /ndaw/ ‘insult’ are identical. 


\ea
Káandáway.    
\z

káá-ndaw-aj      



2S.POT-swallow{}-CL     



‘He will swallow.’    


\ea
Káandaway.
\z

káá-ndaw-aj



2S.POT-insult{}-CL



‘He will insult.’


\section{Prosody of verb stem}
\hypertarget{RefHeading1212001525720847}{}
\citet{Bow1997c} found that in their underlying lexical form, Moloko verb stems are either labialised, palatalised, or without a prosody.  The database in the appendix (see \sectref{sec:14.1}) shows that 83 out of 350 verb stems carry a prosody (61 are palatalised and 22 are labialised).\footnote{The effects of labialisation and palatisation are discussed in \sectref{sec:2.1.} Note that there are also some morphological processes where palatalisation or labialisation is a part of the morpheme. For example, palatalisation is part of the formation of the nominalised form (\sectref{sec:7.6}), and labialisation is a part of the 1P and 2P subject forms \citep{Section1149}. } Although prosodies can carry predictable lexical weight in some other related languages,\footnote{All causatives in Muyang involve the palatalisation of the root (Smith, 2002). In Mbuko, the data show a correlation between palatalisation and pluractionality (Gravina, 2001). } in Moloko, labialisation and palatalisation carry very little lexical weight. \tabref{tab:43}. (adapted from Bow, 1997c, with additional data). illustrates several minimal pairs (or near minimal pairs) for prosody. There appears to be no predictable semantic connection between verb stems of differing prosodies. 

\begin{tabular}{llllll}
\lsptoprule

\multicolumn{2}{l}{\textbf{Neutral}} & \multicolumn{2}{l}{\textbf{Labialised}} & \multicolumn{2}{l}{\textbf{Palatalised}}\\
\textit{ɮak-aj} & \textit{‘}suffer pain\textit{’} & \textit{ɮɔk}\textit{\textsuperscript{w}}\textit{{}-ɔj} & \textit{‘}gnaw\textit{’} & \textit{ɮəg-ɛ} & \textit{‘}sow\textit{’}\\
\textit{mbar} & ‘heal’ &  &  & \textit{mbɛ} & \textit{‘}argue\textit{’}\\
\textit{mbas-aj} & \textit{‘}laugh\textit{’} &  &  & \textit{mbɛʃɛŋ} & \textit{‘}rest, breathe\textit{’}\\
\textit{nzar-aj} & ‘comb, separate’ &  &  & \textit{ndʒɛrɛŋ} & \textit{‘}groan\textit{’}\\
\textit{s-aj} & ‘cut’ &  &  & \textit{ʃ{}-ɛ} & \textit{‘}drink\textit{’}\\
\textit{v-aj} & \textit{‘}winnow\textit{’} &  &  & \textit{v-ɛ} & \textit{‘}spend time\textit{’}\\
&  & \textit{tsɔk-ɔj} & \textit{‘}undress\textit{’} & \textit{tʃ}\textit{ɪ}\textit{k-ɛ} & \textit{‘}stand up\textit{’}\\
\textit{dzak-aj} & \textit{‘}lean\textit{’} & \textit{dzɔk}\textit{\textsuperscript{w}}\textit{{}-ɔj} & ‘pack down’ &  & \\
\textit{ɗ}\textit{ak-aj} & \textit{‘}show, tell\textit{’} & \textit{dɔk}\textit{\textsuperscript{w}}\textit{{}-ɔj} & ‘arrive’ &  & \\
\textit{fak}{}-\textit{aj} & \textit{‘}uproot tree\textit{’} & \textit{fɔk}\textit{\textsuperscript{w}}{}-\textit{ɔj} & ‘whistle with lips’ &  & \\
\textit{gaz-aj} & \textit{‘}nod\textit{’} & \textit{gʊz-ɔj} & ‘tan’ &  & \\
\textit{kar-aj} & \textit{‘}steal\textit{’} & \textit{kɔr-ɔj} & ‘put’ &  & \\
\textit{l-aj} & \textit{‘}dig\textit{’} & \textit{lɔ} & ‘go’ &  & \\
\textit{ɬah-aj} & \textit{‘}mix grain with ashes\textit{’} & \textit{ɬɔh}\textit{\textsuperscript{w}}\textit{{}-ɔj} & ‘leave in secret’ &  & \\
\textit{pal-aj} & \textit{‘}choose\textit{’} & \textit{pɔl-ɔj} & ‘scatter’ &  & \\
\textit{saɓ-aj} & \textit{‘}exceed\textit{’} & \textit{sɔɓ-ɔj} & ‘suck’ &  & \\
\textit{sak-aj} & \textit{‘}sift\textit{’} & \textit{sɔk}\textit{\textsuperscript{w}}\textit{{}-ɔj} & ‘whisper’ &  & \\
\textit{sar} & \textit{‘}know\textit{’} & \textit{sɔr-ɔj} & ‘slide’ &  & \\
\textit{təkas-aj} & \textit{‘}cross\textit{’} & \textit{tʊk}\textit{\textsuperscript{w}}\textit{ɔs-ɔj} & ‘fold legs’ &  & \\
\textit{tah}{}-\textit{aj} & \textit{‘}boost\textit{’} & \textit{tɔh}\textit{\textsuperscript{w}}{}-\textit{ɔj} & ‘trace’ &  & \\
\textit{zar-aj} & \textit{‘}linger\textit{’} & \textit{zɔr-ɔj} & ‘notice, inspect’ &  & \\
\lspbottomrule
\end{tabular}

\begin{itemize}
\item \begin{styleTabletitle}
Minimal pairs for prosody of verb stems
\end{styleTabletitle}\end{itemize}

The underlying labialisation and palatalisation prosodies are lost when most suffixes or clitics\footnote{The indirect object pronominal enclitic does not always influence the verb prosody; see \sectref{sec:3.2} and 3.3.2.} are added, compare example 435 and 436 for the verb /s -j\textsuperscript{ e} /\textit{ }‘drink.’  


\ea
Nese.      
\z

nɛ-ʃ{}-ɛ        



1S.PFV-drink-CL



‘I drank.’


\ea
Nasala.
\z

na-s=ala



1S.PFV-drink=to



‘I drank already.’ (lit. I drank towards)


\section{Tone classes}
\hypertarget{RefHeading1212021525720847}{}
\citet{Bow1997c} concluded that verb stems in Moloko belong to one of three underlying tone classes: high (H), low (L), or toneless (Ø). She discovered that the underlying tone of a verb stem can be identified by comparing the 2S imperative with the Potential form. The Potential form has a high tone on a lengthened subject prefix (see \sectref{sec:53}). If the tone melody of the stem is high on both imperative and Potential forms, then that stem has an underlying high tone. If the tone melody is mid or low on both forms due to the presence of depressor consonants (see \sectref{sec:7}), then the stem has underlying low tone.  If the tone melody of the stem syllable is low in the imperative but high following the high tone of the subject prefix in the Potential form, that verb stem is toneless. The high tone of the Potential form of the subject prefix spreads to the toneless stem. For the imperative form of a toneless stem, a default low tone is applied to the stem.  

A minimal triplet is shown in \tabref{tab:44}. (from Friesen and Mamalis, 2008). Line 1 shows a High tone verb stem. The tone on the verb stem is high in both the imperative and Potential forms.{\textbf{ }}Line 2 shows a low tone verb stem with low tone in the imperative form and mid in the Potential form. Line 3 shows a toneless verb stem. This verb stem carries no inherent tone of its own and its surface tone is low in the imperative form and takes the high tone of the prefix in the Potential form. 

\begin{tabular}{lllll}
\lsptoprule

\textbf{Line} & \textbf{Underlying form of stem} & \textbf{Imperative Form} & \textbf{Potential Form } & \textbf{Tone Class}\\
\textbf{1} & /d r/ & \textit{dár}

Burn! & \textit{náá-dár}

‘I will burn’ & \textbf{H}\\
\textbf{2} & /a-dar-aj/ & \textit{dàr-\={a}j}

‘Plant!’ & \textit{náá-d\={a}r-áj}

 ‘I will plant’ & \textbf{L}\\
\textbf{3} & /d r/ & \textit{dàr}

‘Recoil!’ & \textit{náá-dár}

‘I will recoil’ & Ø\\
\lspbottomrule
\end{tabular}

\begin{itemize}
\item \begin{styleTabletitle}
    Tone class contrasts
\end{styleTabletitle}\end{itemize}

Mamalis (Friesen and Mamalis, 2008) studied tone patterns in Moloko verbs. \tabref{tab:45}. (adapted from Friesen and Mamalis, 2008) shows the imperative and Potential forms and the underlying tone patterns for different verb stems. 

\begin{tabular}{lllll}
\lsptoprule

\textbf{CV pattern} & \textbf{Underlying form of stem} & \textbf{Imperative form} & \textbf{Potential (Irrealis) form }

\textbf{(/}náá/- prefix) & \textbf{Tone class}\\
C & /b-j/

‘light’ & [b-àj ] 

‘Light!’ & [náá-b-àj ] 

‘I will light’ & L\\
& /g-ɛ/

‘do’ & [ɡ-ɛ ] 

‘Do!’ & [nɛɛ-ɡ{}-ɛ ] 

‘I will do’ & H\\
\hhline{~----} & /d-ɛ/

‘cook’ & [d-ɛ ] 

‘Cook!’ & [nɛɛ- d-ɛ ] 

‘I will cook’ & L\\
CC & /mb r/

‘heal, cure’ & [mbár ]

‘Heal! ’ & [náá- mbár]

‘I will heal’ & H\\
& /m t/

‘die’ & [m\={a}t ]

‘Die! ’ & [náá-m\={a}t ]

‘I will die’ & L\\
\hhline{~----} & /g s/

‘catch’ & [gàs ]

‘Catch!’ & [náá-gás ]

‘I will catch’ & toneless\\
CaC & /tsar/

‘taste good’ & [ts\={a}r ]

‘Taste good!’ & [náá-ts\={a}r ]

‘I will taste good’ & L\\
a-CaC-aj & /a-pas -j/ 

‘spread out’ & [p\={a}s-áj ]

‘Spread out!’ & [náá- p\={a}s-áj ]

‘I will spread out’ & L\\
CaC-aj & /nzak -j/ 

‘find’ & [nzák-áj ]

‘Find!’ & [náá- nzák-áj ]

‘I will find’ & H\\
& /ndaɗ -j/ 

‘like, love’ & [ndàɗ-\={a}j ]

‘Love!’ & [náá- ndáɗ-\={a}j]

‘I will love’ & toneless\\
CCC-aj & /d b n -j/ 

‘learn’ & [dəbən-\={a}j ]

‘Learn!’ & [náá- dəbən-\={a}j]

‘I will learn’ & L\\
CCCaC-aj & /b dz gam -j/ 

‘crawl’ & [bədzəgàm-\={a}j]

‘Crawl!’ & [náá-bədzəgàm-\={a}j ]

‘I will crawl’ & L\\
\lspbottomrule
\end{tabular}
\begin{itemize}
\item \begin{styleTabletitle}
 Tone patterns for selected verb stems 
\end{styleTabletitle}\end{itemize}

Tone patterns in Moloko verbs are summarised in \tabref{tab:46}. (from Friesen and Mamalis, 2008), which shows the tone pattern on the stem for the imperative and Potential forms for the three underlying tone forms.  All verb stems in each class have the same pattern, as follows (note that the tone in parentheses is the tone on the \textit{{}-}\textit{aj} suffix, if there is one). Tone patterns are influenced by the presence of depressor consonants (see \sectref{sec:47}) and the underlying structure of the verb stem (see \sectref{sec:48}).

\begin{tabular}{lll}
\lsptoprule

\textbf{Underlying tone} & \textbf{Phonetic tone in imperative form} & \textbf{Phonetic tone in Potential form}\\
H & H(H) & H(H)\\
L without depressor consonants in stem & M(H) & HM(H)\\
L with depressor consonants in stem & L(M) & HL(M)\\
Toneless & L(M) & H(H)\\
\lspbottomrule
\end{tabular}
\begin{itemize}
\item \begin{styleTabletitle}
 Summary of tone patterns for the three tone classes
\end{styleTabletitle}\end{itemize}
\subsection{  Effect of depressor consonants}
\hypertarget{RefHeading1212041525720847}{}
\citet{Bow1997c} subdivided the low tone verb stem category phonetically into mid and low surface forms by the presence or absence of one or more of the class of consonants known as depressor consonants (see \sectref{sec:7}).  Depressor consonants in Moloko include all voiced obstruents except implosives and nasals (i.e. \textit{b, d, g, dz, v, ɮ}\textit{, }\textit{z, mb, nd, ŋg}). \citet{Bow1997c} demonstrated that an underlyingly low tone verb with no depressors has a mid tone surface form; with depressors it has a low tone surface form. For verb stems of underlying high tone or toneless verb stems, the presence or absence of depressor consonants makes no difference to the surface form of the melody.  Toneless verb stems take low tone as the default surface form, regardless of depressors. \tabref{tab:47}. (from Bow, 1997c) shows the realisations of surface tone with and without depressor consonants for the most common verb type (underlying form /CaC/ with high tone -\textit{aj} suffix in the 2P.IMP form).

\begin{tabular}{llllll}
\lsptoprule

\textbf{Underlying tonal melody} & \textbf{Depressor consonants} & \textbf{Surface tone} & \textbf{Underlying form of stem} & \textbf{Surface form} & \textbf{Gloss}\\
Toneless & {}- & L & /haɓ-aj/ & [hàɓ-\={a}j] & ‘dance!’\\
& + & L & /daɮ-aj/ & [dàɮ-\={a}j] & ‘join!’\\
L & {}- & M & /pàɗ-aj/ & [p\={a}ɗ-áj] & ‘bite!’\\
& + & L & /ɮàv-aj/ & [ɮàv-\={a}j] & ‘swim!’\\
H & {}- & H & /fáɗ-aj/ & [fáɗ-áj] & ‘fold!’\\
& + & H & /bál-aj/ & [bál-áj] & ‘wash!’\\
\lspbottomrule
\end{tabular}

\begin{itemize}
\item \begin{styleTabletitle}
Effect of depressor consonants; imperative forms
\end{styleTabletitle}\end{itemize}
\subsection{  Effect of underlying form on tone of stem}
\hypertarget{RefHeading1212061525720847}{}
\citet{Bow1997c} found that the components of the underlying form, particularly initial vowel and number of consonants, influence what underlying tone the root has, such that she could predict the underlying tone of certain verb stems with accuracy. \tabref{tab:48}. (from Friesen and Mamalis, 2008) shows the tone of verb stems of different structures, with examples. The following three stem structures are significant with respect to tone:


\begin{itemize}
\item Verb stems with the \textit{a-} prefix (always two-consonant) always have underlying low tone (line 4, \sectref{sec:6.5}). 
\item Verb stems with three or more consonant roots (line 5-6) always have underlying low tone (\sectref{sec:3}). 
\item Non-palatalised verb stems with one-consonant roots (line 1 of \tabref{tab:48}.) always have underlyingly low tone (\sectref{sec:1}). Palatalised verb stems with one-consonant roots may be high or low but not toneless (line 2). 
\end{itemize}

These three categories account for about 45\% of the verb stems in the database of 316 verb stems used by Mamalis (Friesen and Mamalis, 2008). Only two-consonant roots with no \textit{a-} prefix allow all underlying tone patterns (line 3 of \tabref{tab:48}.).

\begin{tabular}{lllll}
\lsptoprule

\textbf{Line} & \textbf{Verb stem structure} & \multicolumn{3}{l}{\textbf{Underlying tone of 316 verb stems}}\\
&  & \textbf{H } & \textbf{L} & \textbf{Toneless}\\
1 & One-consonant 

non-palatalised verb roots &  & 7 verb stems

{}[b-àj ] ‘light’

{}[p-\={a}j ] ‘open’ & \\
\textstyleannotationreference{2} & One-consonant palatalised verb roots & 4 verb stems

{}[ɡ-ɛ ] ‘do’ & 8 verb stems

{}[d-ɛ ] ‘cook’

{}[ʃ-ɛ ] ‘drink’ & \\
3 & 2 consonant verb roots with no \textit{a-} prefix & 36 verb stems

{}[fár ] ‘scratch’   

{}[bál-áj ] ‘wash’ & 49 verb stems

{}[ɡər-\={a}j ] ‘tremble’

{}[f\={a}t ] ‘grow’

{}[tʃɪk-ɛ ] ‘stand’

{}[tsəɗ-áj ] ‘shine’ & 38 verb stems

{}[dàɗ ] ‘fall’

{}[həm-\={a}j ] ‘run’\\
4 & \textit{a-} prefix verb stems 

(all have 2 consonants) &  & 82 verb stems

{}[bàz ] ‘harvest’ & \\
5 & 3 consonant verb roots &  & 58 verb stems

{}[vənàh-\={a}j ] ‘vomit’

{}[ɬəɓ\={a}t-áj ] ‘repair’ & \\
6 & 4 consonant verb roots &  & 12 verb stems

{}[bədzəɡàm-\={a}j ] ‘crawl’ & \\
\lspbottomrule
\end{tabular}

\begin{itemize}
\item \begin{styleTabletitle}
 Underlying tones for different verb stem structures
\end{styleTabletitle}\end{itemize}
\paragraph[Verb Stems with One Root Consonant]{Verb Stems with One Root Consonant}

Verb stems with single consonant verb roots (the -\textit{aj} suffix is added to produce the stem) (cf. lines 1 and 2 of \tabref{tab:48}.) are never toneless.\footnote{One possible exception is /dz-aj/ ‘say,’ which may be toneless.} Non-palatalised verb stems carry only low tone. Palatalised verb stems may be high or low.  The two possible tonal melodies are seen in the following minimal pair (from Friesen and Mamalis, 2008). Example 437 has an underlying high tone; example 438 has an underlying low tone. 


\ea
Njé.    Néenjé.
\z

nʒ-ɛ      néé-nʒ-ɛ



 leave[2S.IMP]{}-CL  1S.\textsc{POT}{}-leave{}-CL



‘leave!’    ‘I will leave.’


\ea
Nje.      Néenje.
\z

nʒ-ɛ      néé-nʒ-ɛ



sit[2S.IMP] -CL  1S.\textsc{POT}{}-sit{}-CL 



‘Sit!’    ‘I will sit.’


Additional examples illustrating underlying stem tone in verb stems with one root consonant are given in \tabref{tab:49}. (from Friesen and Mamalis, 2008). Imperative and Potential forms are given for each example. Stems with and without depressor consonants are included. 

\begin{tabular}{lllll}
\lsptoprule

\multicolumn{2}{l}{\textbf{Syllable pattern and Aspect/mood}} & \textbf{H} & \multicolumn{2}{l}{\textbf{L }}\\
\hhline{---~~} &  &  & {}- depressor consonants & + depressor consonants\\
Palatalised & Imperative & \textit{g-}\textit{ɛ}

‘do, make’ & \textit{ʃ{}-}\textit{ɛ}

‘drink’ & \textit{d-ɛ }

‘prepare’\\
\hhline{-~~~~} & Potential & \textit{kɛɛ-g-ɛ}

‘you will do’ & \textit{kɛɛ-ʃ{}-}\textit{ɛ}

‘you will drink’ & \textit{kɛɛ-d-ɛ}

‘you will prepare’\\
Non-palatalised & Imperative & Ø & \textit{p-\={a}j }

‘open’ & \textit{b-àj }

‘light’\\
\hhline{-~~~~} & Potential &  & \textit{káá-p-\={a}j}

you will open & \textit{káá-b-àj}

‘you will light’\\
\hhline{~-~--}
\lspbottomrule
\end{tabular}

\begin{itemize}
\item \begin{styleTabletitle}
 Tone patterns in stems with one root consonant
\end{styleTabletitle}\end{itemize}
\paragraph{Verb Stems with two root consonants}

Verb stems with two-consonant roots correspond to lines 3 and 4 of \tabref{tab:48}.. \citet{Bow1997c} found that verb stems that have two root consonants and the \textit{a-} prefix all carry low tone (\tabref{tab:50}. adapted from Friesen and Mamalis, 2008).  

\begin{tabular}{llll}
\lsptoprule

\multicolumn{2}{l}{\textbf{Stem structure}} & \multicolumn{2}{l}{\textbf{L}}\\
\multicolumn{2}{l}{} & \textbf{{}- depressor consonants} & \textbf{+ depressor consonants}\\
a-CC & Imperative & Ø & \textit{dàl  }

‘surpass’\\
\hhline{-~~~} & Potential &  & \textit{káá-dàl}

‘you will surpass’\\
a-CC-aj & Imperative & \textit{sɔl-áj }

‘fry’ \footnotemark{} & \textit{gər\={a}j }

‘frighten’\\
\hhline{-~~~} & Potential & \textit{káá-sɔl-áj}

‘you will fry’ & \textit{káá-gər-\={a}j}

‘you will fear’\\
a-CaC-aj (60) & Imperative & \textit{p\={a}s-áj }

‘spread out’ & \textit{dàr-\={a}j }

‘plant’\\
\hhline{-~~~} & Potential & \textit{káá-p\={a}-sáj}

‘you will spread out’ & \textit{káá-dàr-\={a}j}

‘you will plant’\\
\hhline{~---}
\lspbottomrule
\end{tabular}
\footnotetext{ There was only one example of H tone for this structure.}

\begin{itemize}
\item \begin{styleTabletitle}
Tone patterns in a- prefix verbs
\end{styleTabletitle}\end{itemize}

Verb stems with no \textit{a- } prefix may be from any tone class. \tabref{tab:51}. (from Friesen and Mamalis, 2008) shows several examples of two consonant verbs, giving the imperative and Potential verb forms for each of the possibilities. 

\begin{tabular}{lllll}
\lsptoprule

\multicolumn{2}{l}{\textbf{Stem structure}} & \textbf{H} & \textbf{L}\footnotemark{} & \textbf{Toneless}\\
CC & Imperative & \textit{mbár }

‘heal, cure’\footnotemark{} & \textit{m\={a}t }

‘die’ & \textit{gàs }

‘catch’\\
\hhline{-~~~~} & Potential & \textit{káá-mbár  }

‘you will heal’ & \textit{káá-m\={a}t}

‘you will die’ & \textit{káá-gás}

‘you will get’\\
CaC\footnotemark{} & Imperative & Ø & \textit{ts\={a}r }

‘taste good’ & \textit{hàr }

‘make’\\
\hhline{-~~~~} & Potential &  & \textit{káá-ts\={a}r}

‘you will taste good’ & \textit{káá-hár}

‘you will make’\\
CC-aj & Imperative & \textit{ŋgəl-áj }

‘defend’

(only example) & \textit{rəɓ-áj}

‘be beautiful’ & \textit{həm-\={a}j} 

‘run’\\
\hhline{-~~~~} & Potential & \textit{káá-ŋgəl-áj }

‘you will defend’ & \textit{káá-rɓ-áj}

‘you will be beautiful’ & \textit{káá-həm-áj}

‘you will run’\\
CaC-aj & Imperative & \textit{bál-áj }

‘wash’ & \textit{m\={a}k-áj }

‘stop’ & \textit{ɮàw-\={a}j }

‘fear’\\
\hhline{-~~~~} & Potential & \textit{káá-bál-áj}

you will wash & \textit{káá-m\={a}k-áj}

you will leave & \textit{káá-ɮáw-áj}

you will fear\\
\hhline{~----}
\lspbottomrule
\end{tabular}
\addtocounter{footnote}{-3}
\stepcounter{footnote}\footnotetext{ No two-consonant verbs without \textit{a-} prefix with low tone have depressor consonants. }
\stepcounter{footnote}\footnotetext{ Most CC roots that have high tone end in /r/.}
\stepcounter{footnote}\footnotetext{ Note that these are the only structures that have no counterpart \textit{a-} prefix forms.}
\begin{itemize}
\item \begin{styleTabletitle}
Tone patterns in stems with two root consonants with no a-  prefix
\end{styleTabletitle}\end{itemize}
\paragraph[Verb stems with three or more root consonants]{Verb stems with three or more root consonants}

\citet{Bow1997c} determined that verb stems with three (or more) root consonants (cf. lines 5 and 6 of \tabref{tab:48}.) all have underlyingly low tone. The surface tone will be low or mid, depending on the presence or absence of depressor consonants. If the stem carries the -\textit{aj} suffix, the suffix will carry mid tone.  \tabref{tab:52}. (from Friesen and Mamalis, 2008) shows examples of verb stems with three or more consonants in imperative and Potential form.  

\begin{tabular}{llll}
\lsptoprule

\multicolumn{2}{l}{} & \multicolumn{2}{l}{\textbf{L}}\\
\multicolumn{2}{l}{} & \textbf{No depressor consonants} & \textbf{Depressor consonants}\\
CCC & Imperative & \textit{sək}\textit{\textsuperscript{w}}\textit{ɔm} 

‘buy’ & \textit{dzˋʊg}\textit{\textsuperscript{w}}\textit{ɔr }

‘look after’\\
\hhline{-~~~} & Potential & \textit{kɔɔ-s}\textit{ə}\textit{k}\textit{\textsuperscript{w}}\textit{ɔm}

you will buy & \textit{káá-dzəg}\textit{\textsuperscript{w}}\textit{ɔr}

‘you will shepherd’\\
CCaC & Imperative & \textit{təkár }

‘try, taste’ & \textit{mənzàr }

‘see’\\
\hhline{-~~~} & Potential & \textit{káá-təkár}

‘you will try’ & \textit{káá-mənzàr}

‘you will see’\\
CCC-aj & Imperative & \textit{tsəfəɗ-áj }

‘ask’ & \textit{dəbən-\={a}\`{ }j }

‘teach, learn’\\
\hhline{-~~~} & Potential & \textit{káá-tsəfəɗ-áj}

‘you will ask’ & \textit{káá-dəbən-\={a}j}

‘you will learn’\\
CCaC-aj & Imperative & \textit{pəɗək-áj }

‘wake’ & \textit{vənàh-\={a}j }

‘vomit’\\
\hhline{-~~~} & Potential & \textit{páá-ɗək-áj }

‘you will wake’ & \textit{káá-vənàh-\={a}j }

‘you will vomit’\\
CCCaC-aj & Imperative &  & \textit{bədzəɡàm-\={a}j }

‘crawl!’\\
\hhline{-~~~} & Potential &  & \textit{káá-}\textit{bədzəɡàm-\={a}j}

‘you will crawl’\\
\hhline{~---}
\lspbottomrule
\end{tabular}

\begin{itemize}
\item \begin{styleTabletitle}
 Tone patterns in verb stems with three root consonants
\end{styleTabletitle}\end{itemize}

\chapter[The verb complex]{The verb complex}
\hypertarget{RefHeading1212081525720847}{}
Moloko does not have a simple verb word. Rather, Friesen and \citet{Mamalis2008} named this structure the ‘verb complex’ since affixes and extensions attach to the verb stem that comprises a close phonological unit that is not always one phonological word. The verb complex may be made up of from one to three phonological words as defined by prosody spread and word final allophones (Sections 11, 12). 

There are two fundamental aspects of Moloko grammar that are expressed in the verb complex. The first is the concept of the point of reference. The point of reference involves both place and time. Actions in Moloko are usually placed with respect to a set locational point of reference, which in normal speech is usually the speaker. In a narrative or other discourse, the speaker can set the point of reference. Verbs are aligned with respect to the locational point of reference by means of directional verbal extensions (\sectref{sec:57}). These extensions determine the direction of the event with respect to the point of reference, and can be towards the speaker, away from the speaker, or back and forth. Directionals are different from adpositionals (\sectref{sec:56}), since adpositionals align the action with respect to other elements in the immediate context. The temporal point of reference is set in Moloko by mood and the Perfect. Mood involves what is real or not yet experienced in the world shared by the speaker and his or her audience (realis and irrealis, \sectref{sec:53}). The speaker and audience are, as it were, walking backwards into the future.\footnote{I first heard this image at a First Nations languages conference in 2011 to express how many Aboriginal people view time. } What has happened and is happening is ‘visible’ to them (realis) and they move together into the ‘invisible’ world behind them (irrealis). The point of reference will be the time of communication in normal speech. However, again in a narrative or other type of discourse, the speaker can set the point of reference (usually to the time the events took place). The Perfect extension is employed whenever the speaker needs to make sure that the hearer understands that an event is already completed before the point of reference, with ongoing effects to that point. 

Another fundamental concept in Moloko verbs expressed in the verb complex is expectation, also accomplished through mood. The realis world is the realm of the visible or real; it includes the past and what is present as it happens before the speaker and audience and what is shared knowledge or expectations about the world and how it works. It is presented by the speaker as being real or known – events and states that happened, are happening, or which are part of the expected ‘frame’ of a situation. Within the realis world, the distinctions coded in verbs are for events that are complete/accomplished (Perfective, \sectref{sec:51}), incomplete/unachieved (Imperfective, \sectref{sec:52}), in progress (\sectref{sec:62}), repeated (three types, Sections 54, 55, 57). The irrealis world is the realm of desire and will and the unknown world of the future. Within that world, verbs in Moloko are marked as to the degree of desire and perhaps the control the speaker has over the accomplishment of the event. 

There is no system of tense as such in Moloko (Friesen and Mamalis, 2008).\footnote{\citet{Bow1997c} considered tense and mood. } Perfective versus Imperfective aspect is expressed through changes in the tone of the subject prefix (Sections 51 and 52). Irrealis mood is differentiated from realis mood by vowel changes in the subject prefix (\sectref{sec:53}). For the imperative (\sectref{sec:52}), the subject prefix is absent.  

The verb stem as defined in Chapter 6 can take up to two prefixes and only one suffix.  Morphemes on the stem include the subject pronominal affixes (a prefix and a suffix for 1P and 2P subjects, \sectref{sec:49}) and an indirect object pronominal enclitic (\sectref{sec:491}). Two prefixes are derivational – one prefix nominalises the verb (\sectref{sec:7.6}). The other subordinates the entire clause in which it occurs (\sectref{sec:7.7}). 

Reduplication in the root is one of the ways that pluractionals are formed in other Chadic languages (Newman, 1990). Contrary to many Chadic languages, Moloko does not have a productive pluractional. Only a few verb stems take the pluractional extension (used for actions that are made up of repetitive motions, \sectref{sec:57}).\footnote{The only stems which take the pluractional which we have so far identified are \textit{a-h=aja} ‘he/she grinds,’ \textit{a-s=ija} ‘he/she cuts,’ and \textit{d=ija} ‘take many’. } However, two kinds of reduplication of the verb stem in Moloko express the iterative aspect. Reduplication of a consonant in the stem indicates an iterative action that is habitual (\sectref{sec:54}) and reduplication of the entire verb word indicates an iterative action that is intermittent (\sectref{sec:55}).  

The verbal extensions, which include locational and directional information and Perfect aspect, are also described in this chapter (\sectref{sec:7.5}). They and the indirect object pronominal enclitic are discussed as part of the verb complex because they form a close phonological unit with the verb stem, even though they may sometimes be part of a separate phonological word.  

\section{The phonological structure of the verb word}
\hypertarget{RefHeading1212101525720847}{}
The phonological structure of the Moloko verb word is interesting in that, although its elements can each be part of a phonological unit with the verb stem, combinations of different elements can cause the entity to be broken into up to three phonological words. Its complexity is especially located in the post-verbal elements of the verb complex. The subject prefix and verb stem are the only necessary parts of the basic inflected verb complex.\footnote{The structure of the nominalised or dependent forms of the verb is similar. The derivational prefixes are in the same location as the subject prefix. All other affixes and extensions are possible with the exception of the Perfect extension. } All other affixes and extensions are structurally optional and are determined by the context and the lexical requirements of the particular verb. 

Friesen and \citet{Mamalis2008} discovered that Moloko has three types of verb complexes. The first type of verb complex is one phonological word (\figref{fig:11}.), and occurs when there is no plural suffix (see \sectref{sec:49}), no indirect object pronominal enclitic (see \sectref{sec:491}), and no direct object pronominal (see \sectref{sec:50}). In this case, the extensions (see \sectref{sec:7.5}) cliticise directly to the verb stem. 

←←←←←←←←←←←←←Verb word  →→→→→→→→→→→→→→→→

\begin{tabular}{llllll}
\lsptoprule
\hhline{~~-~~~}
subject+ASPECT- & Irrealis- & Verb stem & =adpositional & =directional & =Perfect\\
\hhline{~~-~~~}
\lspbottomrule
\end{tabular}
In the examples, the verb word is delineated by square brackets.


\begin{itemize}
\item \begin{styleFiguretitle}
One phonological word verb complex
\end{styleFiguretitle}\end{itemize}

\ea
Gaka  ala.
\z

{}[g     =aka   =ala ]



do:2S.IMP    =on   =to



‘Put some more on!’\footnote{Note that the verb stem is /g -j\textsuperscript{e}/.  The palatalisation drops with the extensions.} (lit. do on towards)  


\ea
Alala  va.
\z

{}[à-l  =ala  =va ]



3S.PFV-go  =to  =\textsc{PRF}



‘He came back.’ 


The second type necessitates two phonological words – a verb word and an ‘extension word’ – because of the presence of either a direct or indirect object pronominal (or both). The verb word may have either a subject suffix or an indirect object pronominal enclitic (but not both). The structure of this second verb complex is illustrated in \figref{fig:12}..

←←←←←←←←←←←←Verb word  →→→→→→→→{\textbar}  Extension word→→→→→→→→→→→→→→→

\begin{tabular}{llllllll}
\lsptoprule
\hhline{~~-~~~~~}
subject+ASPECT-

derivational prefix- & Irealis- & Verb stem & {}-1P/2P subject\#

{}-indirect object pronominal\# & direct object pronominal & =adpositional & =directional & =Perfect\\
\hhline{~~-~~~~~}
\lspbottomrule
\end{tabular}

\begin{itemize}
\item \begin{styleFiguretitle}
Two phonological word verb complex
\end{styleFiguretitle}\end{itemize}

The word break is initiated by both the direct and indirect object pronominals such that when either is present, there will be a word break. The word break after the 3S indirect object pronominal enclitic is indicated by word-final changes in /n/; in slow speech the 3S indirect object pronominal enclitic /=\textit{an }/ is pronounced [aŋ] (showing word-final changes) even when there are other clitics following the verb word (ex. 441, see \sectref{sec:491}).  The word break before the 3S DO pronominal is indicated by the fact that the 3S DO pronominal does not neutralise the prosody on the verb stem, and does not cause the  /-j/ suffix to drop (ex. 442 and 443, see \sectref{sec:50}).\footnote{The first line in each example is the orthographic form. The second is the phonetic form (slow speech) with morpheme breaks.} 


\ea
Ámbaɗan  aka  alay.
\z

verb word        ‘extension word’



{}[á-mbaɗ  =aŋ ]    [=aka  =alaj ]



3S.PFV-change=3S.IO   =on   =away



‘He/she replied.’ (lit. he changed on away)


\ea
Aslay na.
\z

{}[à-ɬ{}-aj ]    [na ]



3S.PFV-slay-CL  3SDO



‘He killed it.’


\ea
Ege   na.
\z

{}[ɛ-g-ɛ ]    [na ]



3S.PFV-drink  3SDO



‘He did it.’


Ex. 444 shows a direct object pronominal with no indirect object pronominal enclitic and the extensions cliticise to the direct object pronominal. Ex. 445 shows both direct and indirect object pronominals; again the extensions cliticise to the direct object pronominal. Ex. 441 and 446 show an indirect object pronominal enclitic with no direct object pronominal. When there is no direct object pronominal, the extensions form a separate phonological word in and of themselves. 

\ea
Abək  ta  aya  va  məlama  ahan  ahay  jəyga.
\z

verb word      ‘extension word’



{}[a-bək ]   [ta=aja=va ]    məlama  =ahaŋ    =ahaj  dzijga



3S-invite   3PDO=\textsc{PLU}=\textsc{PRF}  brothers  =3P.POSS  =Pl  all



‘He had already invited all of his brothers.’  


\ea
Ákaɗaw  na  va. 
\z

verb word          ‘extension word’



{}[á-kaɗ  =aw]    [na  =va]



3S.PFV-club=1S.IO    3SDO  +\textsc{PRF}



‘He/she has killed it for me.’ 


\ea
Hor  agaw  aka  ala.
\z

verb word          ‘extension word’



h\textsuperscript{w}ɔr    [à-g=aw ]     [=aka   =ala ]



woman  3S.PFV-do=3S.IO   =on   =to



‘The woman liked me [as I liked her].’ (lit. she did to me on toward)  


The third type of verb complex consists of three phonological words (a verb word, an ‘indirect object word,’ and an ‘extension word’). This type occurs when the verb complex has both a subject suffix and an indirect object pronominal enclitic. Phonological rules will not allow two morphemes suffixed or cliticised to the verb; nor can the indirect object pronominal enclitic commence another word. So, the morpheme \textit{an} is inserted and the indirect object pronominal clitic attaches to the inserted morpheme. The overall structure is then as shown in \figref{fig:13}..


←←←←←←←←←Verb word  →→→→→→ {\textbar} Indirect object {\textbar}  Extension word→→→→→→→→→→→→



                  word


\begin{tabular}{lllllllll}
\lsptoprule
\hhline{~~-~~~~~~}
subject+ASPECT- & Irealis- & Verb stem & {}-1P/2P subject\# & an=indirect object pronominale\# & direct object pronominal & =adpositional & =directional & =Perfect\\
\hhline{~~-~~~~~~}
\lspbottomrule
\end{tabular}

\begin{itemize}
\item \begin{styleFiguretitle}
Three phonological word verb complex
\end{styleFiguretitle}\end{itemize}

In ex. 447 and 448, the verb \textit{kə- ɬ-ɔm }has the 2P imperative suffix attached. The indirect object pronominal enclitic and the inserted morpheme \textit{an}. Other extensions must make a third phonological word since there is a word break following the indirect object pronominal enclitic. 


\ea
Kəslom  anan  na  aka  awak.
\z

verb word                 ‘indirect object word’       ‘extension word’



 [kə- ɬ-ɔm ]    [an=aŋ ]    [na  =aka]    awak



2-slay-2P    DAT=3S.IO    3S.DO  =on    goat



‘You (p) kill another goat for him.’ (lit. you slay a goat for him on top of [another time a goat was slain]


\ea
 Kəslom  anan  aka  awak.
\z

verb word                 ‘indirect object word’  ‘extension word’



{}[kə- ɬ-ɔm ]    [an=aŋ  ]    [=aka ]    awak



2-kill-2P    DAT=3S.IO    =on    goat



‘You kill another goat for him.’


The three types of verb complexes seen in Moloko are shown in \tabref{tab:53}..

\begin{tabular}{ll}
\lsptoprule
1 & ←←←←←←←←←←←←←\textbf{Verb word}  →→→→→→→→→→→→→→→→→→→→→→→→→→

\begin{tabular}{llllll}
\lsptoprule
\hhline{~~-~~~}
subject+ASPECT- & Irrealis- & Verb stem & =adpositional & =directional & =Perfect\\
\hhline{~~-~~~}
\lspbottomrule
\end{tabular}
\\
2 & ←←←←←←←←←←←← \textbf{Verb word}  →→→→→→→→→→{\textbar}     \textbf{Extension word }→→→→→→→→→→→→→→→→→→→

\begin{tabular}{llllllll}
\lsptoprule
\hhline{~~-~~~~~}
subject+ASPECT-

derivational prefix- & Irealis- & Verb stem & {}-1P/2P subject\#

{}-indirect object\# & direct object pronominal & =adpositional & =directional & =Perfect\\
\hhline{~~-~~~~~}
\lspbottomrule
\end{tabular}
\\
3 & ←←←←←←←←←\textbf{Verb word}  →→→→→→→→→→→→{\textbar}\textbf{Indirect object} {\textbar}  \textbf{Extension word }→→→→→→→→→→→→→→→→→

                                                                                                        \textbf{word}

\begin{tabular}{lllllllll}
\lsptoprule
\hhline{~~-~~~~~~}
subject+ASPECT- & Irealis- & Verb stem & {}-1P/2P subject\# & an=indirect object pronominal\# & direct object pronominal & =adpositional & =directional & =Perfect\\
\hhline{~~-~~~~~~}
\lspbottomrule
\end{tabular}
\\
\lspbottomrule
\end{tabular}

\begin{itemize}
\item \begin{styleTabletitle}
Three types of verb complexes
\end{styleTabletitle}\end{itemize}
\section{Imperative }
\hypertarget{RefHeading1212121525720847}{}
The 2S imperative form is also the basic citation form of the verb, and the 2S form gives the clearest presentation of the verb stem. The imperative occurs in 2S, 1\textsc{Pin} and 2P forms. The 2S form is simply the verb stem. The plural forms carry suffixes which correspond to their respective subject pronominal suffixes in indicative verb stems (see \sectref{sec:49}). The singular and plural imperative forms are shown in \tabref{tab:54}. (from Friesen and Mamalis, 2008). 

\begin{tabular}{lll}
\lsptoprule

\textbf{2S form} & \textbf{1P inclusive form} & \textbf{2P form}\\
\textit{fa}\textit{ɗ}

‘Put! (2S)’ & \textit{f}\textit{ʊɗ}\textit{{}-}\textit{ɔ}\textit{k}

‘Let’s put! (1\textsc{Pin})’ & \textit{f}\textit{ʊɗ}\textit{{}-}\textit{ɔ}\textit{m}

‘Put! (2P)’\\
\textit{z}\textit{ɔ}\textit{m}~

‘Eat! (2S)’ & \textit{z}\textit{ʊ}\textit{m-}\textit{ɔ}\textit{k}

‘Let’s eat! (1\textsc{Pin})’ & \textit{z}\textit{ʊ}\textit{m-}\textit{ɔ}\textit{m}

‘Eat! (2P)’\\
\textit{ʃ}\textit{{}-}\textit{ɛ}

‘Drink! (2S)’ & \textit{s-}\textit{ɔ}\textit{k}

‘Let’s drink! (1\textsc{Pin})’ & \textit{s-}\textit{ɔ}\textit{m}

‘Drink! (2P)’\\
\textit{fat-aj}

‘Descend! (2S)’ & \textit{f}\textit{ɔ}\textit{t-}\textit{ɔ}\textit{k}

‘Let’s descend! (1\textsc{Pin})’ & \textit{f}\textit{ɔ}\textit{t-}\textit{ɔ}\textit{m}

‘Descend! (2P)’\\
\lspbottomrule
\end{tabular}

\begin{itemize}
\item \begin{styleTabletitle}
Singular and plural imperative forms
\end{styleTabletitle}\end{itemize}
\section{Verb complex pronominals}
\hypertarget{RefHeading1212141525720847}{}
Friesen and \citet{Mamalis2008} showed that the verb complex can carry pronominals that indicate the subject, direct object, and indirect object. These markers in the verb complex are all bound forms. They are called pronominals and not just agreement markers because all of them can be the only indication of their referent in the clause. Because the pronominals are present, there is no need for a noun phrase or free pronoun in the clause. Participants are tracked in discourse solely by pronominals, and free pronouns and noun phrases only occur in discourse to introduce a participant or to switch the referent. 

%%please move \begin{table} just above \begin{tabular
\begin{table}
\caption{lists all the pronominals.  Subject is indicated by a verbal prefix for singular subjects and third person plural.  Plural subjects for first and second person are indicated by a combination of a prefix and a suffix. These subject pronominals (discussed in \sectref{sec:49}) are given in their underlying form because the surface vowel and tone on the prefix is determined by mood and aspect, respectively. Also, the underlying form is given to show the prosody, because the labialisation prosody in the plural subject suffixes will spread over the entire verb stem. The direct object pronominal (\sectref{sec:50}) only occurs for third person singular and plural. The indirect object pronominal (\sectref{sec:491}) cliticises to the right edge of the verb stem and the direct object pronominal follows it. In \tabref{tab:55}., the independent pronouns are also given for comparison since there are similarities between the free pronoun and its corresponding pronominal.}
\label{tab:55}
\end{table}

\begin{tabular}{lllll}
\lsptoprule

\textbf{Person} & \textbf{Pronominal subject affixes} & \textbf{Indirect object pronominal enclitics} & \textbf{Third person direct object pronominals} & \textbf{Independent pronouns}\\
\textbf{1S} & \textit{n-} & =\textit{aw} &  & \textit{n}\textit{ɛ}\\
\textbf{2S} & \textit{k-} & =\textit{ɔk}\textit{\textsuperscript{w}} &  & \textit{nɔk}\textit{\textsuperscript{w}}\\
\textbf{3S} & \textit{a-} / \textit{ma-}\footnotemark{} & \textit{=aŋ} & \textit{na} & \textit{ndahaŋ}\\
\textbf{1P inclusive, }

i.e. speaker (+others)  + hearer & \textit{m-}…-\textit{ɔ}\textit{k}\textit{\textsuperscript{w}} & \textit{=alɔk}\textit{\textsuperscript{w}}\textit{ɔ} &  & \textit{lɔk}\textit{\textsuperscript{w}}\textit{ɔ}\\
\textbf{1P exclusive}

i.e. speaker + others & \textit{n-}…-\textit{ɔ}\textit{m} & \textit{=al}\textit{ɪ}\textit{mɛ} &  & \textit{l}\textit{ɪ}\textit{mɛ}\\
\textbf{2P} & \textit{k-}…-\textit{ɔ}\textit{m} & \textit{=alʊk}\textit{\textsuperscript{w}}\textit{øjɛ} &  & \textit{lʊk}\textit{\textsuperscript{w}}\textit{øjɛ}\\
\textbf{3P} & \textit{t-} & \textit{=ata} & \textit{ta} & \textit{təta}\\
\lspbottomrule
\end{tabular}
\footnotetext{ The third person Hortative subject pronominal, see Table \tabref{tab:65}. in \sectref{sec:1153.}}

\begin{itemize}
\item \begin{styleTabletitle}
Pronominals
\end{styleTabletitle}\end{itemize}
\subsection{  Subject pronominal affixes}
\hypertarget{RefHeading1212161525720847}{}
The subject is always marked on the finite form of the verb, regardless of whether there is a free subject phrase in the clause.\footnote{The presence of both subject pronominal and corresponding noun phrase occurs for pragmatic reasons. }  In fact, the subject pronominal marker in the verb can be the only indication of subject in the entire clause.\footnote{In a non-finite verb form, the subject pronominal is absent and the subject of the clause is either understood from the context or indicated by a free pronoun or noun phrase in the clause (Sections 1160, 7.7, 1164).} As noted in \tabref{tab:56}. and \tabref{tab:57}. (adapted from Friesen and Mamalis, 2008), subject is marked by a prefix or combination of prefix and suffix. In the examples below, the pronominal affixes are highlighted. The prefix carries aspectual tone (see \sectref{sec:7.4}), and the vowel quality is influenced by the prosody on the verb stem (see \sectref{sec:6.6}), the presence of the /a-/ prefix (see \sectref{sec:6.5}), and the mood of the verb (see \sectref{sec:53}). The 1P and 2P suffixes are labialised. This prosody will spread over the entire verb stem. 

\begin{tabular}{lll}
\lsptoprule

\textbf{\textit{Person}} & \textbf{Singular} & \textbf{Plural}\\
\textbf{\textit{1}} & \textbf{nə-}mənzar awak

‘I saw a goat’ & \textbf{\textit{mʊ-}}\textit{mʊnzɔr}\textbf{\textit{{}-ɔk}}\textit{ awak}

‘we (inclusive) saw a goat’\\
\hhline{--~} &  & \textbf{\textit{nʊ-}}\textit{mʊnzɔr}\textbf{\textit{{}-ɔm}}\textit{ awak  }

‘we (exclusive) saw a goat’\\
\textbf{2} & \textbf{\textit{kə-}}\textit{mənzar awak}  

‘you saw a goat’ & \textbf{\textit{kʊ-}}\textit{mʊnzɔr-}\textbf{\textit{ɔm}}\textit{ awak  }

‘you (plural) saw a goat’\\
\textbf{3} & \textbf{\textit{a-}}\textit{mənzar awak} 

‘he/she saw a goat’ & \textbf{\textit{tə-}}\textit{mənzar awak}    

‘they saw a goat’\\
\lspbottomrule
\end{tabular}

\begin{itemize}
\item \begin{styleTabletitle}
Conjugations with subject pronominal affixes for the verb /m nzar/ ‘see’
\end{styleTabletitle}\end{itemize}

\begin{tabular}{lll}
\lsptoprule

\textbf{Person} & \textbf{Singular} & \textbf{Plural}\\
\textbf{1} & \textbf{\textit{nə-}}\textit{həm-aj  }

‘I ran’ & \textbf{\textit{m}}\textbf{\textit{ʊ}}\textbf{\textit{{}-}}\textit{h}\textit{ʊ}\textit{m}\textbf{\textit{{}-ɔk}}    

‘we (inclusive) ran’\\
\hhline{--~} &  & \textbf{\textit{n}}\textbf{\textit{ʊ}}\textbf{\textit{{}-}}\textit{h}\textit{ʊ}\textit{m}\textbf{\textit{{}-ɔm}}    

‘we (exclusive) ran’\\
\textbf{2} & \textbf{\textit{kə-}}\textit{həm-aj}  

‘you ran’ & \textbf{\textit{k}}\textbf{\textit{ʊ}}\textbf{\textit{{}-}}\textit{h}\textit{ʊ}\textit{m}\textbf{\textit{{}-ɔm}}    

‘you (plural) ran’\\
\textbf{3} & \textbf{\textit{a-}}\textit{həm-aj}    

‘he/she ran’ & \textbf{\textit{tə-}}\textit{həm-aj}    

‘they ran’\\
\lspbottomrule
\end{tabular}
\begin{itemize}
\item \begin{styleTabletitle}
Conjugations with subject pronominal affixes for the verb /h m-aj/ ‘run’
\end{styleTabletitle}\end{itemize}

\citet{Bow1997c} found that a prosody on the verb stem will spread leftwards from the verb stem over the singular subject prefixes. The fact that palatalisation and labialisation spread over the subject prefixes indicates that the subject markers are fully bound to the verb stem and are not separate words.  Example 449 presents the palatalised verb / g\textsuperscript{ e}\textit{ }/ ‘do,’ and example 450 presents the labialised verb / l\textsuperscript{o}/ ‘go.’  


\ea
Nege.
\z

{}[nɛ\textbf{{}-}g-ɛ]



1S-do-CL



‘I did.’


\ea
Olo.
\z

{}[ɔ\textbf{{}-}lɔ]



3S-go



‘he/she went.’


\citet{Bow1997c} also discovered that labialisation on the 1P and 2P subject suffixes will spread leftwards from the suffix onto the entire verb word. This fact indicates that these morphemes are fully bound to the verb stem and are not separate words.  The verb / ts k -j\textsuperscript{ e }/ ‘stand’, shown in example 451 in its 1S form, loses its palatalisation and becomes labialised when the (labialised) plural suffixes are added (ex. 452):

\ea
Necəke.
\z

nɛ\textbf{{}-}tʃɪk-ɛ



1S-stand-CL    



‘I stand.’


\ea
Nəcəkom.    
\z

nʊ\textbf{{}-}tsʊk\textsuperscript{w}{}-ɔm



1-stand-1\textsc{Pex}



‘We (ex) stand.’


\citet{Bow1997c} also determined that the subject pronominal prefixes in Moloko appear to be toneless. The aspect of the verbal construction will allocate tone to the pronoun.   In the Imperfective aspect, the pronoun always takes high tone (see \sectref{sec:52}).  In the Perfective aspect, the pronoun copies the first tone of the root if it is low or mid.  If the first tone of the root is high, the pronoun takes on mid tone.

\paragraph[   Indirect object pronominal enclitic]{   Indirect object pronominal enclitic}

An indirect object pronominal enclitic can attach to the verb word to express the indirect object, which is a core argument of the verb. The indirect object in Moloko is the participant that represents the place where the direct object is directed to – the recipient or beneficiary of the action.\footnote{Employing the Agent-Theme-Location analysis developed by De\citet{Lancey1991}, the indirect object in Moloko expresses the semantic LOC (see Chapter 9). The direct object pronominal expresses the semantic Theme – the participant that changes position or state (see \sectref{sec:1150}). } In ex. 453, the verb /dz -j/ ‘help’ takes the indirect object. The indirect object represents the participant who receives the help. 


\ea
Ajənaw.
\z

a-dzən=aw



3S-help  =1S.IO



‘He/she helped me.’


The indirect object pronominal enclitic allows the core indirect object argument to be expressed in a prepositional phrase \textit{ana Mana} ‘to Mana’ (ex. 454). 

\ea
Ajənan ana Mana.
\z

a-dzən  =aŋ  ana  Mana



3S-help  =3S.IO  DAT  Mana



‘He/she helped Mana.’


The indirect object pronominal enclitic can also stand in the place of the prepositional phrase (ex. 455).

\ea
Ajən\textbf{an}.
\z

a-dzən  \textbf{=a}\textbf{ŋ}



3S-help  =3S.IO



‘He/she helped him.’


%%please move \begin{table} just above \begin{tabular
\begin{table}
\caption{(adapted from Friesen and Mamalis, 2008) shows the verb /v l/\textit{ }‘give’ conjugated for the indirect object argument. The indirect object expresses the recipient.}
\label{tab:58}
\end{table}

\begin{tabular}{lll}
\lsptoprule

\textbf{Person} & \textbf{Singular} & \textbf{Plural}\\
\textbf{\textit{1}} & a-vəl\textbf{=aw}

‘he/she gave to me’ & \textit{a-vəl}\textbf{\textit{=alɔk}}\textbf{\textit{\textsuperscript{w}}}\textbf{\textit{ɔ}}

‘he/she gave to us (inclusive)’\\
\hhline{-~~} &  & \textit{a-vəl}\textbf{\textit{=alɪmɛ}}

‘he/she gave to us (exclusive)’\\
\textbf{2} & \textit{a-vəl}\textbf{\textit{=ɔk}}\textit{\textsuperscript{w}}

‘he/she gave to you’ & \textit{a-vəl}\textbf{\textit{=alʊk}}\textbf{\textit{\textsuperscript{w}}}\textbf{\textit{øjɛ}}

‘he/she gave to you (plural)’\\
\textbf{3} & \textit{a-vəl}\textbf{\textit{=aŋ}}

‘he/she gave to him/her’ & \textit{a-vəl}\textbf{\textit{=ata}}

‘he/she gave to them’\\
\lspbottomrule
\end{tabular}

\begin{itemize}
\item \begin{styleTabletitle}
Verb /v l/ ‘give’ conjugated for indirect object pronominal enclitic
\end{styleTabletitle}\end{itemize}

The indirect object pronominal enclitics are phonologically bound to the verb stem  and do not comprise separate words.  When an indirect object pronominal cliticises to the verb stem, there are no word final alternations in the verb stem. Compare the following pairs of examples showing verb stems with and without indirect object pronominal enclitics. Ex. 456 and 457 show that when the indirect object pronominal enclitic is attached (ex. 457), we do not see the word final alternation of /h/ → [x] / \_\#.\footnote{See \sectref{sec:11.} Likewise, we do not see the word-final process of n → [ŋ] /  \_\# between the verb stem and the indirect object pronominal.} 


\ea
Aɓah  zana.
\z

a-ɓax  zana



3S-sew    clothing



‘He/she sews clothing.’


\ea
Aɓahaw  zana.  
\z

a-ɓah=aw    zana  



\textsc{3S}{}-sew=\textsc{1S.IO}   clothing    



‘He/she sews clothing for me.’


Similarly, the example pairs 458 and 459 illustrate that the \textit{{}-aj} suffix is dropped when the indirect object pronominal is present (ex. 459), indicating that the pronominal is phonologically bound to the stem (see \sectref{sec:6.3}). 

\ea
Ajay.
\z

a-dz-aj



3S-speak{}-CL



‘He/she speaks.’


\ea
Ajan.
\z

a-dz=aŋ



3S-speak=3S.IO



‘He/she speaks to him/her.’


The indirect object pronominal enclitic is not phonologically a true suffix, because the prosody of the indirect object pronominal enclitic does not affect the prosody on the verb stem. Compare ex. 460 and 461 which illustrate the verb stem /s/ conjugated with second person singular and plural indirect objects. If the prosody of the indirect object pronominal enclitic affected the verb stem, one would expect that the /s/ in example 461 would be affected by the palatalisation prosody of the plural indirect object pronominal enclitic and be expressed as [ʃ\textit{ }]. 

\ea
Asok  aka  ɗaf.
\z

a-s=ɔk    =aka  ɗaf



3S-please=2S.IO  =on  loaf



‘You  want to have more millet loaves.’ (lit. millet loaf is pleasing to you)  


\ea
Asaləkwəye  aka  ɗaf.
\z

a-s=a l\textup{ʊ}k\textup{ʷ}øj\textup{ɛ  =aka  }ɗaf



3S-please=2P.IO  =on  loaf



‘You want to have more millet loaves.’ (lit. millet loaf is pleasing to you)  


The fact that the indirect object pronominal can attach to verb stems as well as other particles confirms that it is in fact a clitic pronoun. Normally, the indirect object pronominal enclitic attaches directly to the verb stem (ex. 462).  However, if the plural subject pronominal suffix is required on the verb (ex. 463), the indirect object pronominal can no longer attach to the verb, because the verb stem can take only one suffix (see \sectref{sec:7.1}). Instead, the indirect object pronominal cliticises to the particle \textit{an}\textit{.} This particle may be related to \textit{ana}, the dative preposition ‘to.’

\ea
Kasl\textbf{an}  awak.
\z

ka-ɬ\textbf{=aŋ}      awak



2S-slay-3S.IO    goat



‘You slay the goat for him.’ 


\begin{itemize}
\item 
\textit{Kəslom  }\textbf{\textit{ana}}\textbf{\textit{n}}\textit{  awak}\textit{.}
\end{itemize}

kə-ɬ{}-ɔm  \textbf{an=aŋ}    awak



2-slay-2P  to=3S.IO  goat



‘You (plural) slay the goat for him.’  


There is a word break after the indirect object pronominal enclitic (the phonological words are indicated by square brackets in the examples immediately below).  The word break is indicated by the fact that the 3S indirect object pronominal enclitic /=\textit{an}/ in slow speech is pronounced [aŋ] even when there are other clitics following the verb word (see ex. 464 and 465). The word-final [ŋ] will delete in fast speech (see \sectref{sec:10}). These clitics (e.g., the adpositional clitics in these examples, see \sectref{sec:56}) would otherwise attach to the verb (compare with example 466):

\ea
As\textbf{an}  \textbf{aka}  ɗaf.
\z

{}[a-s\textbf{=aŋ} ]\textbf{    }[\textbf{=aka }]    ɗaf



3S-please=3S.IO  =on    loaf



‘He/she wants to have more millet loaves.’ (lit. millet loaf is pleasing to him)  


\ea
Ad\textbf{an}  \textbf{aka}  ɗaf.
\z

{}[a-d\textbf{=aŋ} ]\textbf{    }[\textbf{=aka }]    ɗaf



3S-prepare=3S.IO  =on    loaf



‘She made more loaves of millet for him.’


\ea
Ad\textbf{aka}  ɗaf.
\z

{}[a-d \textbf{=aka }]        ɗaf



3S-prepare=on      loaf



‘She made more loaves of millet.’  


\subsection{  Third person direct object pronominal}
\hypertarget{RefHeading1212181525720847}{}%%please move \begin{table} just above \begin{tabular
\begin{table}
\caption{(from Friesen and Mamalis, 2008) shows the direct object (DO) pronominals. The third person DO pronominals replace or double a full noun phrase in a discourse– the \textit{na} (3S.DO) or \textit{ta} (3P.DO) refer back to something in the immediately preceding context. Ex. 467 and 468 show two clauses that might occur in a discourse. In ex. 468 the \textit{na} refers back to \textit{ɬa} ‘cow’ in ex. 467.}
\label{tab:55}
\end{table}


\ea
Kaslay  sla.      
\z

kà-ɬ{}-aj    ɬa      



2S.PFV-slay{}-CL  cow         



‘You slew the cow.’            


\ea
Kaslay  \textbf{na}\textbf{.}
\z

ka-ɬ{}-aj     \textbf{na}



2S.PFV-slay{}-CL    3S.DO  



‘You slew it.’


A third person DO pronominal can be the only expression of direct object in a clause if its identity is known in the discourse (ex.468, 470, and 474). The only time that a clause will contain both a third person DO pronominal and a noun phrase that co-refer to the direct object in the clause is when a special focus on the direct object is required (‘all his brothers’ in ex. 469, ‘that fruit-bearing tree’ in ex. 476). 


Race Story\footnote{Friesen, 2003.}


\ea
Moktonok  na,  abək  \textbf{ta}  aya  va  məlama  ahan  ahay  jəyga.
\z

mɔk\textsuperscript{w}tɔnɔk\textsuperscript{w}na  a-bək       \textbf{ta}=aja=va  məlama  =ahaŋ    =ahaj  dzijga



toad    PSP  3S-invite  3P=\textsc{PLU}=\textsc{PRF}  brothers  =3P.POSS  =Pl  all



‘The toad, he had already invited all of his brothers.’  


We know that the third person DO pronominals are phonologically separate words (not clitics like the other verbal extensions) because the \textit{{}-}\textit{aj} suffix does not drop when the DO pronominal is added to a clause (ex. 470).  Normally the \textit{{}-}\textit{aj} suffix drops off when extensions or suffixes are added to the clause (ex. 471, see also \sectref{sec:6.3}). 

\ea
Apaɗay  \textbf{na}\textbf{.}      
\z

a-paɗ-aj     \textbf{na}      



3S-crunch{}-CL   3S.DO



‘He/she crunches it.’ 


\ea
Apaɗaka.
\z

a-paɗ  =aka



3S-crunch  =on



‘He/she crunches on.’ 


Another indication that the DO pronominal is phonologically a separate word is that the neutral prosody on the DO pronominal does not affect the prosody of the verb word. Compare ex. 472 and 473. In both examples the verb complex is palatalised in spite of the addition of the DO pronominal. This situation is in contrast to what happens with the Perfect enclitic (see \sectref{sec:58}). 

\ea
Nese.
\z

nɛ-ʃ{}-ɛ



1S-drink{}-CL



‘I drink.’


\ea
Nese  na.
\z

nɛ-ʃ{}-ɛ  na



1S-drink{}-CL  3S.DO



‘I drink it.’ 


A third indication is that word final changes (like word-final /n/ being realised as [ŋ] (see \sectref{sec:11} and example 474) are preserved when followed by \textit{na} or \textit{ta}.

\ea
Nəvəlan  \textbf{na}\textbf{.} 
\z

nə-vəl=aŋ     \textbf{na} 



1S-give=3S.IO   3S.DO



‘I gave it to him.’


The normal slot for the DO pronominal is within the verb complex between the verb stem and the directional extension. In each example below, the verb complex is delineated by square brackets and the third person DO pronominal is bolded.

\ea
Baba  ango  avəlan  \textbf{na}  alay  ana  məze.
\z

baba  =aŋg\textsuperscript{w}ɔ    [a-vəl=aŋ  \textbf{na}=alaj ]    ana  mɪʒɛ



father  =2S.POSS  3S-give=3S.IO  3S.DO=away    DAT  person



‘Your father gave it to that person.’  


Any further verbal extensions will cliticise to a third person DO pronominal. In example 476, the directional extension \textit{=ala} ‘toward’ cliticises to \textit{na}\textit{ }and vowels will elide resulting in the pronunciation [nala]. See also example 469, where the pluractional and perfect extensions \textit{=}\textit{aja}\textit{ }and \textit{=}\textit{va}\textit{ }cliticise to the DO pronominal \textit{ta} to result in the pronunciation [tajava].  


Cicada S. 12


\ea
Tolo  təmənjar  \textbf{na  ala}  mama  ngəvəray  nəndəye.
\z

tɔ{}-lɔ    [tə-mənzar  \textbf{na   =ala }]\textbf{  }mama  ŋgəvəraj   nɪndijɛ



3P-go  3P-see         3S.DO   =to  mother  spp.of.tree  DEM



‘They went and saw that fruit-bearing tree.’


The first and second person direct objects are expressed by free pronouns (see \sectref{sec:131}) or noun phrases. The free pronouns are distributionally and phonologically distinct from the third person direct object pronominals. The free pronouns occur after the verb complex. Note that they occur after the directional extensions in ex. 477 and 478. In each example, the verb complex is delineated by square brackets and the first or second person independent pronoun is bolded.

\ea
Kazalay  \textbf{ne}  a  kosoko  ava  ɗaw?
\z

{}[ka-z  =alaj ]    \textbf{nɛ}  a  kɔsɔk\textsuperscript{w}ɔ  ava  ɗaw



2S-take  =away    1S  in  market  in  QUEST



‘Will you take me to the market?’


\ea
Baba  ango  avəlata  \textbf{nok}  va  a  ahar  atəta  ava
\z

baba  =aŋg\textsuperscript{w}ɔ    [à-vəl=ata ]  \textbf{nɔk}\textbf{\textsuperscript{w}}  =va  a  ahar  =atəta    ava



father  =2S.POSS  3S-give=3P.IO  2S  =\textsc{PRF}  in  hand  3P.POSS  in



\textit{waya  aməmbeɗe  h}\textit{o}\textit{r  atəta.}



waja  amɪ-mbɛɗ-ɛ  h\textsuperscript{w}ɔr  =atəta



because  DEP-change-CL  woman  3P.POSS



‘Your father gave you to them to become a wife [for their relative].’  (lit. your father gave you into their hands because to change their woman)


The 3S pronominal is employed in discourse to track participants (along with the subject and indirect object pronominals, see Sections 49 and 491, respectively). Ex. 479 and 480 are from the Snake story (see \sectref{sec:1.4}).  In S. 4 (ex. 479) the snake is introduced with a noun phrase \textit{g}\textit{\textsuperscript{w}}\textit{ɔg}\textit{\textsuperscript{w}}\textit{ɔlvaŋ} ‘snake’ and in S. 18 (ex. 480) it is referred to by the 3S DO pronominal \textit{na}. 


Snake story S. 4


\ea
Alala  na,  gogolvan  na,  olo  alay.
\z

\textbf{\textit{a-l=la        na}}\textit{  }\textbf{\textit{g}}\textit{\textsuperscript{w}}\textbf{\textit{ɔg}}\textbf{\textit{\textsuperscript{w}}}\textbf{\textit{ɔlvaŋ   na}}\textit{  }\textbf{\textit{ɔ{}-lɔ    =alaj}}



3S-come       PSP  snake      PSP   3S.PFV-go   =away



‘Some time later, the snake went.’



Snake story S. 18


\ea
Ne dəy day  məkəɗe  na  aka
\z

\textit{nɛ dij daj              mɪ-kɪɗ-ɛ      na      =aka} 



1S  approximately     \textsc{NOM}{}-kill-CL  3S.DO   =on



‘I clubbed it to death (approximately).’


In a clause where the referent is clear, the 3S DO pronominal \textit{na} can sometimes be left out in a clause. Ex. 481 is from a narrative not illustrated in this work. In the narrative, the head of the household brings home some things he bought at the market. He tells his workers to carry the things into the house. In his instructions \textit{h}\textit{\textsuperscript{w}}\textit{ɔr-ɔm}\textbf{\textit{ }}\textit{=alaj ajva} ‘carry [all the things] into the house,’ there is no grammatical indication of ‘those things.’ The absence of the DO pronominal is indicated in the clause by the symbol \textit{ø}. In this case, the referent is clear and is not required in the clause.\footnote{The DO pronominal in Moloko does not function in the way Frajzyngier has postulated for some Chadic languages. Frajzyngier and \citet{Shay2008} say that the DO pronoun codes the definiteness of the referent in some Chadic languages. While it is true in Moloko that when the DO pronominal (or any other pronoun) is used, then the referent is definite, the converse is not true. For example, the referent in ex. 481 is definite yet there is no DO pronominal. } 

\ea
Bahay  a  hay  olo  a  kosoko  ava.  Askomala  ele  ahay  gam.  Awəɗakata  ele  ngəndəye  ana  ndam  slərele  ahan  ahay,  awəy, “Horom\textbf{  }alay  ayva!”
\z

bahaj a   haj  ɔ{}-lɔ  a  kɔsɔk\textsuperscript{w}ɔ  ava 



chief  GEN  house  3S-go  in  market  in



‘The head of the house went to the market.’



\textit{a-sk}\textit{\textsuperscript{w}}\textit{[F08D?]m  =ala    ɛlɛ   =ahaj   gam  }



3S-buy  =to    thing  PL  many



‘He bought many things.’



\textit{a-wuɗak  =ata  ɛlɛ   ŋgɛndijɛ  ana  ndam  ɬɪrɛlɛ  =ahaŋ    =ahaj  awij}



3S-divide  =3P.IO  thing  DEM  DAT  people  work  =3S.POSS  PL  3S+say



‘[when he got home], he divided the things among his workmen, saying,’



\textit{h}\textit{\textsuperscript{w}}\textit{ɔr-ɔm     }\textbf{\textit{ø}}\textit{  =alaj    ajva}



carry \textsc{IMP}{}-2P    =away    inside house



‘ “Carry [all the things) into the house.” ’


Likewise, in the Cicada story, the direct object (the tree that the chief wanted by his door) is not grammatically indicated in the clause in S. 16 (ex. 482). Although the referent is definite, there is no grammatical reference to it in the clause. 


Cicada S. 16


\ea
Tàazala\textbf{  }təta  bay. 
\z

tàà-z =ala             \textbf{ø}  təta    baj 



3P.irrealis-take=to        ABILITY  \textsc{NEG}



‘They were not able to bring [the tree].’  


Participants can be made prominent in a clause by doubling the reference to them. In ex. 483 from S. 20 of the Cicada story, the tree that the chief desired is indicated twice in a clause, both by the presence of a noun phrase \textit{mɛmɛlɛ ga ndana }‘that tree that you spoke of’ and also the 3S DO pronominal (bolded in ex. 483). The effect is prominence. 


Cicada S. 20


\ea
Náamənjar  \textbf{na}  alay  \textbf{memele  ga  ndana}  əwɗe.~
\z

náá-mənzar    \textbf{na}  =alaj   \textbf{mɛmɛlɛ  ga   ndana}  uwɗɛ~



1S.POT-see   3S.DO  =away   tree   ADJ  DEM   first



‘“First I want to see the tree that you spoke of.”’


\section{Aspect and mood}
\hypertarget{RefHeading1212201525720847}{}
Friesen and \citet{Mamalis2008} showed that Moloko does not mark verb stems for tense, but uses an aspectual system, looking at events as complete (Perfective, see \sectref{sec:51}) or incomplete (Imperfective, see \sectref{sec:52}). The tonal melody on the subject prefix expresses Perfective or Imperfective aspect. The vowel in the prefix expresses realis or irrealis mood (see \sectref{sec:53}). Reduplication of a consonant in the verb stem indicates habitual iterative aspect (see \sectref{sec:54}). Reduplication of the entire verb stem indicates the intermittent iterative aspect – the intermittent repetition of the same action, possibly by the same actor, over a period of time (see \sectref{sec:55}).\footnote{Another repeated aspect is the pluractional. The pluractional extension in Moloko indicates an action is back and forth, for example \textit{s=ija ‘}sawing’ or \textit{h=aja} ‘grinding’ (\sectref{sec:57}).  }

\subsection{  Perfective}
\hypertarget{RefHeading1212221525720847}{}
The Perfective (\textsc{PFV}) aspect in Moloko is the aspect that presents the event as completed (Friesen and Mamalis, 2008).\footnote{Usually, the term ‘Perfective’ is used to refer to a situation as a whole, whether it is completed at the time of speaking or not.  The situation is viewed in its entirety for Perfective, whereas in Imperfective aspect, the situation is viewed ‘from inside.’ as an ongoing process (Comrie 1976: 3-4; Payne, 1997: 239). \citet{Dixon2012} refers to verbs expressing completed actions as ‘perfect’ and those expressing incomplete actions as ‘imperfect.’ We have used the term ‘Perfective’ for completed actions in Moloko because the Perfect in Moloko \citep{Section1158} collocates with both of these other aspects. } The Perfective aspect is indicated by a phonetic low or mid tone on the subject prefix. Verb stems with underlyingly low tone or toneless verb stems have a phonetic low tone if the verb stem begins with a depressor consonant  (see \sectref{sec:47}), and phonetic mid tone otherwise. Verb stems with underlyingly high tone are unaffected by depressor consonants and so the phonetic tone of the subject prefix is mid. \tabref{tab:59}. (from Friesen and Mamalis, 2008) shows an example from each tone class. 

\begin{tabular}{llll}
\lsptoprule

\textbf{Underlying verb stem} & \textbf{Underlying tone of verb stem} & \textbf{Phonetic tone of Perfective verb word} & \textbf{Gloss}\\
/nz a k -j/ & H & \textit{nə-nzák-áj} & ‘I found’\\
/a-p a s/ & L, no depressor consonants & \textit{n\={a}-p\={a}s-áj} & ‘I spread (something) out’\\
/a-d-a r -j/ & L, with depressor consonants & \textit{nà-dàr-\={a}j} & ‘I planted’\\
/ɮ w -j/ & Toneless & \textit{nə-ɮəw-\={a}j} & ‘I feared’\\
\lspbottomrule
\end{tabular}

\begin{itemize}
\item \begin{styleTabletitle}
Perfective tone
\end{styleTabletitle}\end{itemize}

The default verbal aspect for the main event line in a narrative is Perfective. Perfective verb forms are found in the main eventline clauses expressing the events immediately following the setting sections of narratives. This is seen in the following examples drawn from three different narratives. Ex. 484 is from lines 4-6 of the Snake story, ex. 485 is from a story not illustrated in full, and ex. 486 is from line 6 of the Cicada story. In the examples, Perfective verb forms are bolded. The low tone is marked on the subject pronominal prefix. 


Snake S. 4-6



\ea
Alala  na,  gogolvan  na,  \textbf{olola}\textbf{y}. \textbf{Acar}  a  hay  kəre  ava fo fo fo. Senala  na, okfom  \textbf{adəɗala}  ɓav. 
\z

a-l=ala        na,g\textsuperscript{w}ɔg\textsuperscript{w}ɔlvaŋ   na,         \textbf{ɔ{}-lɔ =alaj}



3S-come       PSP  snake      PSP    3S.PFV-go =away



Some time later, the snake went,



\textbf{\textit{à-tsar}}\textit{              a  haj   kɪrɛ       ava  fɔ fɔ fɔ.}



3S.PFV-climb       in   house   beams  in          \textsc{ID}sound of snake



it climbed into the roof of the house \textit{fɔfɔfɔ},



\textit{ʃɛŋ    =ala  na,  ɔk}\textit{\textsuperscript{w}}\textit{fɔm   }\textbf{\textit{à-dəɗ         =ala}}\textit{      ɓav}



\textsc{ID}go  =to      PSP   mouse  3S.PFV-fall  =to     \textsc{ID}sound of falling 



And walking, a mouse fell \textit{bav}!


\ea
Kəlen  na,  zar  ahan  na,  \textbf{enje}  ele  ahan  ametele.
\z

kɪlɛŋ   na  zar  =ahaŋ    na  ɛ\textbf{{}-nʒ-}\textbf{ɛ}    ɛlɛ    =ahaŋ        amɛ-tɛl-ɛ



next     PSP  man   =3S.POSS  PSP  3S.PFV-leave-CL  thing  =3S.POSS  DEP-walk-CL 



‘Then, her husband went away to walk;



\textbf{\textit{Enje}}\textit{  kə  delmete  aka  a  slam  enen.}



\textit{ɛ}\textbf{\textit{nʒ-}}\textbf{\textit{ɛ}}\textit{               kə    dɛlmɛtɛ  aka   a   ɬam   ɛnɛŋ}



3S.PFV-leave-CL   on   place       on    at  place   another



he left for some place.’  


\begin{stylefootnotetext}
Cicada 6
\end{stylefootnotetext}

\ea
Albaya  ahay  ndana  kəlen  \textbf{tangala}  ala  ma  ana  bahay.
\z

albaja       =ahaj        ndana  kɪlɛŋ   \textbf{tà-ŋgala  }=ala  ma  ana   bahaj



young man    =Pl        DEM  then  3P.PFV-return   =to   word  DAT  chief



‘The above-mentioned young men then took the word (response) to the chief.’


\subsection{  Imperfective}
\hypertarget{RefHeading1212241525720847}{}
In contrast with the Perfective, the Imperfective aspect (\textsc{IFV}) can refer to an event that is incomplete and in the process of happening or to an event that is just about to begin.\footnote{‘Imperfective aspect’ usually refers to a situation ‘from the inside’ and is concerned with the internal structure of the situation (Comrie, 1976: 4). Perhaps ‘incomplete’ would be a better name for this aspect in Moloko; however it does not correspond with imperfect as described by \citet{Dixon2012} in that the action need not begin before the present and be continuing, as \citet[31]{Dixon2012} notes.} The subject prefix for the Imperfective form is always high tone and the tone over the verb stem varies according to the underlying tone of the verb stem. \citet{Bow1997c} noted that the high tone on the prefix spreads to the first syllable of an underlyingly low tone verb. In the examples, the high tone of the Imperfective and low tone of Perfective are marked on the subject pronominal prefix. Ex. 487 - 494 are in pairs to show contrast between the tone of the Imperfective (the first of each pair) and the Perfective (the second of each pair). Compare ex. 487 (Imperfective) and 488 (Perfective). Ex. 487 refers to an event in process of happening (going to the market; already en route).\footnote{There is also a progressive aspect expressed by a complex verb construction (see \sectref{sec:1162}), but the Imperfective verb form alone can give the idea of an action in progress.} 


\ea
K\textbf{ó}lo  amtamay?
\z

k\textbf{ɔ}{}-lɔamtamaj



2S.IFV-go  where



‘Where are you going?’


\ea
K\textbf{o}lo  amtamay?
\z

k\textbf{ɔ}{}-lɔ  amtamaj



2S.PFV-go  where



‘Where were you?’


Ex. 489 and 490 illustrate another Imperfective/Perfective pair. The Imperfective in this case refers to an event in process. 

\ea
N\textbf{á}kaɗ  bərek  cəcəngehe.
\z

n\textbf{á}{}-kàɗ  bɪrɛk  tʃɪtʃɪŋgɛhɛ



1S.IFV-kill  brick  now



‘I am making bricks (now).’ 


\ea
Nakaɗ  bərek  cəcəngehe.
\z

nà-kàɗ  bɪrɛk  tʃɪtʃɪŋgɛhɛ



1S.PFV-kill  brick  now



‘I made bricks just now.’ 


Ex. 366 is an Imperfective that marks an event about to begin (compare with the Perfective in ex. 492). 

\begin{itemize}
\item 
\textit{N}\textbf{\textit{á}}\textit{pas}\textit{a}\textit{y  agaba}\textit{n.}  
\end{itemize}

\textit{n}\textbf{\textit{á-}}\textit{pàs-}\textit{ \={a}}\textit{j    agabaŋ}  



1S.IFV-take.away  sesame



‘I’m about to take away the sesame seeds.’


\begin{itemize}
\item 
\textit{N}\textbf{\textit{a}}\textit{pas}\textit{a}\textit{y  agaban.}  
\end{itemize}

\textit{n}\textbf{\textit{à-}}\textit{pàs-}\textit{ \={a}}\textit{j    agabaŋ}  



1S.PFV-take.away  sesame



‘I took away the sesame seeds.’


Likewise, the Imperfective in ex. 493 illustrates an event about to begin (compared with the Perfective in ex. 494).

\ea
Cəcəngehe  ne  awəy  n\textbf{é}ge  hay  əwla  ete.
\z

tʃɪtʃɪŋgɛhɛ   nɛ   awij~    n\textbf{ɛ}{}-g-ɛ     haj   =uwla     ɛtɛ



now    1S  saying    1S.IFV-do  house  =1S.POSS  also



‘Now I said, “I want to/am going to make a house for myself too.”’


\ea
Cəcəngehe  ne  awəy  n\textbf{e}ge  hay  əwla  ete.
\z

tʃɪtʃɪŋgɛhɛ   nɛ   awij~     n\textbf{ɛ}{}-g-ɛ     haj   =uwla     ɛtɛ



now    1S  saying    1S.PFV-do  house  =1S.POSS  also



‘Now I said, “I made a house for myself too.”’


%%please move \begin{table} just above \begin{tabular
\begin{table}
\caption{(from Friesen and Mamalis, 2008) shows the Imperfective tonal pattern on the same four verb stems as were illustrated in \tabref{tab:59}. for the Perfective.}
\label{tab:60}
\end{table}

\begin{tabular}{llll}
\lsptoprule

\textbf{Underlying verb stem} & \textbf{Underlying tone of verb stem} & \textbf{Phonetic tone of verb word} & \textbf{Gloss}\\
/nz a k-aj/ & H & \textit{nə-nzák-áj} & ‘I’m finding’\\
/a-p a s/ & L, no depressor consonants & \textit{ná-p\={a}s-áj} & ‘I’m spreading (something) out’\\
/a-d-a r-aj/ & L, with depressor consonants & \textit{ná-dàr-\={a}j} & ‘I’m planting’\\
/ɮ w-aj/ & Toneless & \textit{nə-ɮáw-áj} & ‘I’m fearing’\\
\lspbottomrule
\end{tabular}

\begin{itemize}
\item \begin{styleTabletitle}
 Imperfective tone
\end{styleTabletitle}\end{itemize}
%%please move \begin{table} just above \begin{tabular
\begin{table}
\caption{(from Friesen and Mamalis, 2008) summarises the tone patterns for Perfective and Imperfective tone on stems of different structures though the syllable pattern of the stem does not influence the tone pattern for the different aspects.}
\label{tab:61}
\end{table}

\begin{tabular}{llll}
\lsptoprule

\textbf{Underlying tone of verb stem} & \textbf{Structure of verb stem} & \textbf{Perfective  }

\textbf{(lower tone on subject prefix)} & \textbf{Imperfective }

\textbf{(higher tone on subject prefix)}\\
\textbf{H} & CaC-aj & \textit{nə-nzák-áj}

‘I found’ & \textit{nə-nzák-áj}

‘I am finding’\\
\hhline{-~~~} & CC & \textit{n\={a}-mbár}

‘I healed’ & \textit{ná-mbár}

‘I am healing’\\
&  & \textit{n\={a}-ɗák}

‘I blocked up’ & \textit{ná-ɗák}

‘I am blocking up’\\
\textbf{L }

\textbf{no depressor consonants} & a-CaC-aj & \textit{n\={a}-p\={a}s-áj}

‘I took away’ & \textit{ná-p\={a}s-áj}

‘I am taking away’\\
\hhline{-~~~} & CaC-aj & \textit{nə-t\={a}ts-áj}

‘I close’ & \textit{nə-t\={a}ts-áj}

‘I am closing’\\
& CC & \textit{n\={a}-f\={a}ɗ}

‘I put’ & \textit{ná-f\={a}ɗ}

‘I am putting’\\
\textbf{L  }

\textbf{depressor consonants in verb stem} & a-CaC-aj & \textit{nə-dàr-\={a}j}

‘I recoil’ & \textit{nə-dàr-\={a}j}

‘I am recoiling’\\
\hhline{-~~~} & CCaC-aj & \textit{nə-vənàh-\={a}j}

‘I vomitted’ & \textit{nə-vənàh-\={a}j}

‘I am vomiting’\\
\textbf{Toneless} & CaC-aj & \textit{nə-ɮàw-\={a}j}

‘I feared’ & \textit{nə-ɮáw-\={a}j}

‘I am fearing’\\
\hhline{-~~~} & CC & \textit{nà-ndàz}

‘I pierced’ & \textit{ná-ndáz}

‘I am piercing’\\
&  & \textit{nà-dàɗ}

‘I fell’ & \textit{ná-dáɗ}

‘I am falling’\\
\hhline{~~--}
\lspbottomrule
\end{tabular}
\begin{itemize}
\item \begin{styleTabletitle}
Summary of tone patterns in selected verb forms
\end{styleTabletitle}\end{itemize}

In texts, the Imperfective is used whenever the (ongoing) normal state of affairs is being expressed, i.e., the way the world is. Ex. 495 - 498 are general statements; they are not speaking of a particular situation. All the main verbs are Imperfective. 


\ea
Sləreɛle  \textbf{áyə}\textbf{ɗay }məze.
\z

ɬɪrɛlɛ   \textbf{á-jə}\textbf{ɗ{}-aj}      mɪʒɛ



work  3S.IFV-tire{}-CL  person



‘Work tires people out.’  


\ea
Fat  \textbf{ánah}  háy.
\z

fat    \textbf{á-nax}    haj



sun     3S.IFV-ripen  millet



‘The sun ripens the millet.’ 


\ea
\textbf{Káslay  }awak  nə  məsləye.
\z

\textbf{ká-ɬ{}-aj  }  awak  nə  mɪ-ɬ{}-ijɛ



2S.IFV-slay{}-CL  goat  with  \textsc{NOM}{}-slay-CL



‘You slaughter goats by cutting their throat, and not by any other way.’ (lit. you slay a goat with slaying)


\ea
\textbf{Kákaɗ}  okfom  nə  məkəɗe.  \textbf{Káslay  }bay.
\z

\textbf{ká-kaɗ  }  ɔk\textsuperscript{w}fɔm  nə  mɪ-kɪɗ-ɛ      \textbf{ká-ɬ{}-aj}    baj



2S.IFV-kill(club)  mouse  with  \textsc{NOM}{}-kill(club)-CL  2S.IFV-slay-CL  \textsc{NEG}



‘You kill mice by smashing their head; you don’t cut their throats.’ (lit. you kill a mouse with killing; you don’t slay it)


The Imperfective can refer to events that take place at any time, including in the past.  In a story set in the past, the idea of an ongoing event that was the context for another event is encoded using the Imperfective verb form combined with the progressive aspect construction (see \sectref{sec:62}). The Imperfective verb stems are bolded in ex. 499. 

\ea
Asa  təmənjar  zar  Məloko  andalay  \textbf{á}\textbf{səya}  ele  
\z

asa  tə-mənzar    zar  Mʊlɔk\textsuperscript{w}ɔ    a-nd=alaj\textbf{  }\textbf{á}\textbf{{}-s=ija  }    ɛlɛ  



if  3P-see       man    Moloko     3S-PRG=away    3S.IFV-cut=\textsc{PLU}  thing   



\textit{nə}\textit{  zl}\textit{ə}\textit{rg}\textit{o}\textit{  coco  fan  na,}



\textit{nə     }\textit{ɮʊ}\textit{rg}\textit{\textsuperscript{w}}\textit{ɔ}\textit{   tʃ}\textit{ɔ}\textit{tʃ}\textit{ɔ  }\textit{      faŋ       na}



with    axe  \textsc{ID}.cutting    already   PSP



 ‘If they found a Moloko cutting his fields with his axe,’ 



\textit{təlala  təta  gam  na,  tarəbokoy  na  ala  rəbok}\textit{\textsuperscript{  }}\textit{rəbok.}



\textit{tə-l =ala  təta  gam  na  ta-r}\textit{ʊ}\textit{bɔk}\textit{\textsuperscript{w}}\textit{{}-ɔj  na  =ala    r}\textit{ʊ}\textit{bɔk}\textit{\textsuperscript{w}}\textit{ r}\textit{ʊ}\textit{bɔk}\textit{\textsuperscript{w}}



3S-go =to  3P  many  PSP  3P-hide     3S.DO  =to    \textsc{ID}hide hide



‘many came stealthily upon him, hiding, hiding.’


In narratives, the Imperfective is found in the introduction to stories to describe the way things were at the beginning of the story.\footnote{As well as Imperfective, verb forms in the progressive aspect \citep{Section1162} and existentials (which do not inflect for aspect, Chapter 3.4) are found in the setting and conclusion sections of a narrative.} For example, in the Disobedient Girl story, the main verbs in the introduction (lines 1-8) are all Imperfective. The entire story is in \sectref{sec:1.5}; the literal English translation of the introduction is given here with Imperfectives bolded. 

\textit{‘A story under the silo, they say,~the story of the disobedient girl:}

\textit{Long ago, to the Moloko people, God }\textbf{\textit{gives}}\textit{ his blessing. That is, even if they had only sowed a little [millet] like this, it }\textbf{\textit{lasts}}\textit{ them enough for the whole year. While grinding on the grinding stone, they }\textbf{\textit{take}}\textit{ one grain of millet. So, if they }\textbf{\textit{are grinding}}\textit{ it, the flour }\textbf{\textit{multiplies}}\textit{. Just one grain of millet, it }\textbf{\textit{suffices}}\textit{ for them, and there }\textbf{\textit{are leftovers}}\textit{. Because, during its grinding, it }\textbf{\textit{multiplies}}\textit{ on the grinding stone.’}

Imperfectives are also found in the conclusion of the narrative to recount how things turned out at the end of the story. The main verbs in the conclusion of the Disobedient Girl are also Imperfective. The literal English translation of the conclusion (lines 32-38) is given here with Imperfectives bolded (the entire story is in \sectref{sec:1.5}).

\textit{So, ever since that time, finished! The Molokos say that God }\textbf{\textit{gets}}\textit{ angry because of that girl, the disobedient one. Because of all that, God }\textbf{\textit{takes back}}\textit{ his blessing from them. And now, one grain of millet, it }\textbf{\textit{doesn’t multiply}}\textit{ anymore. Putting one grain of millet on the grinding stone, it }\textbf{\textit{doesn’t multiply}}\textit{ }\textit{anymore. You must }\textbf{\textit{put on}}\textit{ a lot. It is like this they say, The curse belongs to that young woman who brought this suffering onto the people. }

When the Imperfective  co-occurs with the Perfect, the verb describes the current state or result of an event (ex. 500, see \sectref{sec:58}).

\ea
Arahə\textbf{va}.
\z

à-rah\textbf{=va}  



3S.PFV-fill=\textsc{PRF}  



‘It is full.’ (it had filled)


\subsection{  Irrealis mood}
\hypertarget{RefHeading1212261525720847}{}
Friesen and \citet{Mamalis2008} showed how mood influences the vowel features of the subject pronominal prefix. Moloko has two moods: realis and irrealis. The main formal feature of the irrealis mood is that the vowel in the subject prefix is lengthened. There are three subtypes of irrealis mood, indicated by tone along with the lengthened subject prefix.\footnote{Only two moods were distinguished in previous documents (Friesen and Mamalis, 2008; Boyd, 2003).   } Tone on the subject prefix has three patterns, and no longer correlates with Perfective or Imperfective aspect in the irrealis mood. Rather, it correlates with the speaker’s desire and will. These three types of mood are called Potential,{ }Hortative, and Possible, respectively. Potential mood expresses an action desired by the speaker that is under his or her influence to perform. It carries a mild hortatory force for second person forms. Hortative mood expresses an action desired by the speaker to be performed by another who is somehow under his or her influence. Possible mood expresses that an action is desired by the speaker but dependent on the will of another.

The difference between the moods is illustrated in the following narrative situations. The first (ex. 501 and 502) illustrates a situation where someone says that he wants the chief to come to him, but he is not sure if the chief will actually come. The fact that the chief’s coming is desired by the speaker but dependent on the will of the chief is expressed by the Possible mood in ex. 501, with falling tone on the lengthened subject prefix (bolded). Compare with the response given in ex. 502, where the speaker is sure that the chief will come. The surety is expressed by the Potential mood, with high tone on the lengthened subject prefix (bolded). 


\ea
Asaw  bahay  məlala\textbf{  }azana  \textbf{áà}lala  ete  ɗaw?
\z

a-s  =aw  bahaj  mə-l    =ala\textbf{   }azana  \textbf{áà-}l=ala  ɛtɛ  ɗaw



3S-please  =1S.IO  chief  3S.HOR-go  =to  maybe  3S+\textsc{PBL}{}-go=to  polite  QUEST



‘I would like the chief to come; maybe he will come (if he wants to).’


\ea
\textbf{Áa}lala. 
\z

\textbf{áá-}l    =ala 



3S.POT-go  =to



‘He will come (I am sure).’


Likewise, in ex. 503, the speaker is expressing his wish that a potential attacker will leave him and his family alone. The falling tone on the lengthened subject prefix (bolded) indicates that the speaker is not sure that the person will leave them alone, but it depends on the will of that person (Possible mood). 

\ea
Adan  bay  \textbf{a}\textbf{á}makay  loko  émbəzen  loko  asabay.
\z

adaŋ  baj  \textbf{áà-}mak-aj    lɔk\textsuperscript{w}ɔ  ɛ{}-mbɪʒɛŋ  lɔk\textsuperscript{w}ɔ  asa-baj



perhaps  \textsc{NEG}  2S+\textsc{PBL}{}-leave{}-CL  1PIN  3S.IFV-ruin  1PIN  again-\textsc{NEG}



‘Perhaps he will leave us alone; he will not ruin us anymore.’


High tone on the lengthened subject prefix indicates Potential mood (an action desired by the speaker that is under his or her influence to perform, ex. 504 and 506). In the examples, the subject prefix is bolded.

\ea
Hajan  \textbf{nóo}l\textbf{o  }a  kosoko  ava.
\z

hadzaŋ  \textbf{nɔɔ}{}-l\textbf{ɔ}    a  kɔsɔk\textsuperscript{w}ɔ  ava



tomorrow  1S.POT-go  in  market  in



‘Tomorrow I will go to the market.’


\ea
Ól\textbf{o.}
\z

áá-l\textbf{ɔ}



3S.POT-go



‘He/she will hopefully go.’ (if I have a say in it)


\ea
\textbf{Káa}zala  təta  bay.
\z

\textbf{káá}{}-z  =ala  təta    baj



2S.POT-take=to  ABILITY  \textsc{NEG}



‘You cannot bring it.’


Low tone on the lengthened subject prefix indicates Hortative mood (an action desired by the speaker to be performed by another who is somehow under his or her influence, ex. 507 - 508).

\ea
\textbf{Moo}lo  a  kosoko  ava.
\z

\textbf{mɔɔ}{}-lɔ    a  kɔsɔk\textsuperscript{w}ɔ  ava



3S.HOR-go    in  market  in



‘He/she should go to the market.’


\ea
\textbf{Koo}zəmom  enen  bay.
\z

\textbf{kɔɔ}{}-zʊm-ɔm    ɛnɛŋ  baj



2P.HOR-eat-2P  another  \textsc{NEG}



‘You (plural) should not eat anything.’ 


\ea
Epeley  epeley  ɗəw  \textbf{nóò}lo  bay  ɗaw?
\z

ɛpɛlɛj  ɛpɛlɛj    ɗuw    \textbf{nɔɔ}{}-lɔ    baj  ɗaw



whenever  whenever  also    1S+\textsc{PBL}{}-go  \textsc{NEG}  QUEST



‘Far in the future also, might I not go perhaps?’


High tone followed by low tone on the lengthened subject prefix indicates Possible mood (an action is desired by the speaker but dependent on the will of another, ex. 509 - 512).

\ea
Aálo.
\z

áà-lɔ



3S+\textsc{PBL}{}-go



‘He/she might go.’ (it is up to him whether he goes, and I don’t know what he is thinking)


\ea
Adan  bay  ɓərav  ahan  \textbf{a}\textbf{á}ndeslen  \textbf{aá}makay  məɗəgele  ahan.
\z

adaŋ baj  ɓərav  =ahaŋ    \textbf{áà}{}-ndɛɬɛŋ  \textbf{áà}{}-m\={a}k-aj    mɪ-ɗɪgɛl-ɛ  =ahaŋ



perhaps  heart  =3S.POSS  3S+\textsc{PBL}{}-cool  3S+\textsc{PBL}{}-leave{}-CL \textsc{NOM}{}-think-CL  =3S.POSS



‘Perhaps his heart will cool, and he might leave behind his anger (lit. his thinking).’


\ea
\textbf{Maá}həzlok  asabay  bay  way.
\z

\textbf{máà}{}-h\textsuperscript{w}ʊɮ{}-ɔk      asa-baj    baj  waj



1Pin+\textsc{PBL}{}-destroy-1Pin  again-\textsc{NEG}  \textsc{NEG}  who



‘Maybe we won’t be destroyed after all.’\footnote{Note that this ‘passive’ idea (to be destroyed) is accomplished through the flexible transitivity system in Moloko. The verb means ‘destroy’ but with the Theme as subject of the verb, the whole clause here expresses a passive idea (Chapter 9). }\textsuperscript{ }


The three irrealis moods are illustrated in \tabref{tab:62}. for the high tone verb /l\textsuperscript{o}/ ‘go.’

\begin{tabular}{lll}
\lsptoprule

\textbf{Mood} & \textbf{2S form} & \textbf{3S form}\\
\textbf{Potential} & \textit{kàà-l=àlà  }

2S.POT-go=to

‘You will come.’(I am sure you will come) & \textit{áá-l= àlà    }

3S.POT-go=to  

‘He/she will come.’ (I am sure he will come)\\
\textbf{Hortative} & \textit{kàà-l=àlá  }

2S.HOR-go=to  

‘You come now!’ (I want you to come) & \textit{mə-l= àlá    }

3S.HOR-go=to  

‘He/she should come.’ (I want him to come)\\
\textbf{Possible} & \textit{káà-l=àlà  }

2S+\textsc{PBL}{}-go=to  

‘I want you to come (but I am not sure if you will).’ & \textit{áà-l=àlà    }

3S+\textsc{PBL}{}-go=to  

‘I want him to come (but am not sure if he will).’\\
\lspbottomrule
\end{tabular}

\begin{itemize}
\item \begin{styleTabletitle}
Mood for the verb /l\textsuperscript{o}/ ‘go’
\end{styleTabletitle}\end{itemize}
%%please move \begin{table} just above \begin{tabular
\begin{table}
\caption{illustrates the low tone verb /tats/ ‘close’ in all of the realis and irrealis forms.}
\label{tab:63}
\end{table}

\begin{tabular}{lll} & \textbf{2S form} & \textbf{Gloss}\\
\lsptoprule
\textbf{Perfective} & \textbf{kə-}t\={a}ts-\={a}j          mahaj

2S.PFV-close-CL door & ‘You closed the door.’\\
\textbf{Imperfective} & \textbf{\textit{kə-}}\textit{t\={a}ts-\={a}j          mahaj}

2S.IFV-close-CL door & ‘You are closing the door.’/ ‘You are about to close the door.’\\
\textbf{Potential} & \textbf{\textit{k}}\textbf{\textit{áá}}\textit{{}-t\={a}ts-\={a}j        mahaj}

2S.POT-close-CL door & ‘I would like you to close the door.’ / ‘You should close the door.’ / ‘You will close the door.’\\
\textbf{Hortative} & \textbf{\textit{k}}\textbf{\textit{àà}}\textit{{}-t\={a}ts-\={a}j        mahaj}

2S.HOR-close-CL door & ‘I strongly suggest you close the door.’ / ‘You should have already closed the door.’\\
\textbf{Possible} & \textbf{\textit{k}}\textbf{\textit{áà}}\textit{{}-t\={a}ts-\={a}j       mahaj}

2S.POT-close-CL door & ‘You might close the door.’ / ‘I want you to close the door but I don’t know if you will.’\\
\lspbottomrule
\end{tabular}
\begin{itemize}
\item \begin{styleTabletitle}
Realis and irrealis forms of /tats/ ‘close’
\end{styleTabletitle}\end{itemize}

In first or third person, the Potential mood indicates some measure of confidence on the part of the speaker that the action will be performed, or the state achieved. To see this, the Imperfective in ex. 513 (with high tone and short vowel on subject prefix) expresses an incomplete action. 


\ea
\textbf{Ná}l\textbf{o  }a  kosoko  ava.
\z

\textbf{ná}{}-l\textbf{ɔ}  a  kɔsɔk\textsuperscript{w}ɔ  ava



1S.IFV-go  in  market  in



‘I am going to the market.’ 


The Potential mood in ex. 514 (with high tone and long vowel on subject prefix) carries the idea of surety (as does ex. 515).

\ea
\textbf{Náa}l\textbf{o  }a  kosoko  ava.
\z

\textbf{náá}{}-l\textbf{ɔ}  a  kɔsɔk\textsuperscript{w}ɔ  ava



1S.POT-go  in  market  in



‘I will go to the market’


\ea
Asa  hay  ango  andava  na  mɛ,  \textbf{áa}rəɓay.
\z

asa  haj  =aŋg\textsuperscript{w}ɔ    a-ndava    na      mɛ  \textbf{áá}{}-rəɓ-aj



if  house  =2S.POSS  3S-finish  PSP   INC  3S.\textsc{POT}{}-be\_beautiful



‘When your house is finished, it will be beautiful.’


%%please move \begin{table} just above \begin{tabular
\begin{table}
\caption{shows a conjugation of the low tone verb /\textit{fat -j }/ ‘descend’ in the Potential form.}
\label{tab:64}
\end{table}

\begin{tabular}{lll}
\lsptoprule

\textbf{Person} & \textbf{Singular} & \textbf{Plural}\\
\textbf{1} & \textbf{\textit{n}}\textbf{\textit{áá}}\textit{{}-f}\textit{\={a}}\textit{t-aj}

1S.POT-descend{}-CL

‘I will go down’ & \textbf{\textit{m}}\textbf{\textit{á}}\textit{{}-f}\textit{ɔ}\textit{t-ɔk}

1\textsc{Pin}.POT-descend-1\textsc{Pin}

‘We will go down.’\\
\hhline{~-~} &  & \textbf{\textit{n}}\textbf{\textit{á}}\textit{{}-f}\textit{ɔ}\textit{t- ɔm}

1\textsc{Pin}.POT-descend-1\textsc{Pin}

‘We (exclusive) will go down.’\\
\textbf{2} & \textbf{\textit{k}}\textbf{\textit{áá}}\textit{{}- f}\textit{\={a}}\textit{t-aj}

2S.POT-descend{}-CL

‘I would like you to go down (you should go down)’ & \textbf{\textit{k}}\textbf{\textit{á}}\textit{{}-f}\textit{ɔ}\textit{t-ɔm}

2P.POT-descend-2P

‘You will all go down.’\\
\textbf{3} & \textbf{\textit{áá}}\textit{{}- f}\textit{\={a}}\textit{t-aj}

3S.POT-descend{}-CL

‘He/she will go down.’ & \textbf{\textit{t}}\textbf{\textit{áá}}\textit{{}-f}\textit{\={a}}\textit{t-aj}

3P.POT-descend{}-CL

‘They will go down.’\\
\lspbottomrule
\end{tabular}

\begin{itemize}
\item \begin{styleTabletitle}
Potential form conjugation
\end{styleTabletitle}\end{itemize}
%%please move \begin{table} just above \begin{tabular
\begin{table}
\caption{\tabref{tab:65}. shows a conjugation of the low tone verb /\textit{fat -j }/ ‘descend’ in the Hortative form. In the Hortative form, the 3S subject prefix is \textit{m}\textit{a}\textit{\`{ }}\textit{a}\textit{\`{ }-}. Compared with the Potential form, the Hortative form is a little stronger in terms of its hortatory force (see \sectref{sec:11.4}).}
\label{tab:65}
\end{table}

\begin{tabular}{lll}
\lsptoprule

\textbf{Person} & \textbf{Singular} & \textbf{Plural}\\
\textbf{1} & \textbf{\textit{nàà}}\textit{{}- f}\textit{à}\textit{t-aj}

1S.HOR-descend{}-CL

‘I should go down’ & \textbf{\textit{m}}\textbf{\textit{à}}\textit{{}-f}\textit{ɔ}\textit{t-ɔk}

1\textsc{Pin}.HOR-descend-1\textsc{Pin}

‘I would like us to go down (we should go down).’\\
\hhline{~-~} &  & \textbf{\textit{n}}\textbf{\textit{à}}\textit{{}-f}\textit{ɔ}\textit{t-ɔm}

1\textsc{Pin}.HOR-descend-1\textsc{Pin}

‘I would like us (ex) to go down (we should go down).’\\
\textbf{2} & \textbf{\textit{kàà}}\textit{{}- f}\textit{à}\textit{t-aj}

2S.HOR-descend{}-CL

‘I would like you to go down (you should go down)’ & \textbf{\textit{k}}\textbf{\textit{àà}}\textit{{}-f}\textit{ɔ}\textit{t-ɔm}

2P.HOR-descend-2P

‘I would like you all to go down (you should go down).’\\
\textbf{3} & \textbf{\textit{màà}}\textit{{}- f}\textit{à}\textit{t-aj}

3S.HOR-descend

‘I would like him to go down (he should go down).’ & \textbf{\textit{tà}}\textbf{\textit{à}}\textit{{}- f}\textit{à}\textit{t-aj}

3P.HOR-descend{}-CL

‘I would like them to go down (they should go down).’\\
\lspbottomrule
\end{tabular}
\begin{itemize}
\item \begin{styleTabletitle}
Hortative form conjugation
\end{styleTabletitle}\end{itemize}
%%please move \begin{table} just above \begin{tabular
\begin{table}
\caption{shows the Possible form of the low tone verb /\textit{fat -j }/ ‘descend’.}
\label{tab:66}
\end{table}

\begin{tabular}{lll}
\lsptoprule

\textbf{\textit{Person}} & \textbf{\textit{Singular}} & \textbf{Plural}\\
\textbf{1} & \textbf{\textit{n}}\textbf{\textit{á}}\textbf{\textit{à}}\textit{{}-f}\textit{à}\textit{t-aj}

1S+\textsc{PBL}{}-descend{}-CL

‘I might go down.’ & \textbf{\textit{m}}\textbf{\textit{á}}\textbf{\textit{à}}\textit{{}-f}\textit{ɔ}\textit{t-ɔk}

1\textsc{Pin}+\textsc{PBL}{}-descend-1\textsc{Pin}

‘We will go down.’\\
\hhline{~-~} &  & \textbf{\textit{n}}\textbf{\textit{á}}\textbf{\textit{à}}\textit{{}-f}\textit{ɔ}\textit{t- ɔm}

1\textsc{Pin}+\textsc{PBL}{}-descend-1\textsc{Pin}

‘We (exclusive) might go down.’\\
\textbf{2} & \textbf{\textit{k}}\textbf{\textit{á}}\textbf{\textit{à}}\textit{{}-f}\textit{à}\textit{t-aj}

2S+\textsc{PBL}{}-descend{}-CL

‘You might go down.’ & \textbf{\textit{k}}\textbf{\textit{á}}\textbf{\textit{à}}\textit{{}-f}\textit{ɔ}\textit{t-ɔm}

2P+\textsc{PBL}{}-descend-2P

‘You might all go down.’\\
\textbf{3} & \textbf{\textit{á}}\textbf{\textit{à}}\textit{{}-f}\textit{à}\textit{t-aj}

3S+\textsc{PBL}{}-descend{}-CL

‘He/she might go down.’ & \textbf{\textit{t}}\textbf{\textit{á}}\textbf{\textit{à}}\textit{{}-f}\textit{à}\textit{t-aj}

3P+\textsc{PBL}{}-descend{}-CL

‘They might go down.’\\
\lspbottomrule
\end{tabular}
\begin{itemize}
\item \begin{styleTabletitle}
Possible form conjugation
\end{styleTabletitle}\end{itemize}

Compare the indicative (ex. 516) and hortatory (ex. 517 and 518) forms of the high tone verb /\textit{z m}/ ‘eat.’ The subject prefixes are bolded. 


\begin{itemize}
\item 
\textbf{\textit{M}}\textbf{\textit{ə}}\textit{z}\textbf{\textit{ə}}\textit{mok  ɗaf}\textit{.}
\end{itemize}

\textbf{\textit{mʊ-}}\textit{zʊm-ɔk}\textit{\textsuperscript{w}}\textit{      ɗaf}



1\textsc{Pin}.IFV-eat-1P    loaf



‘We are eating millet loaves.’


\begin{itemize}
\item 
\textit{Lomala  }\textbf{\textit{máa}}\textit{z}\textbf{\textit{ə}}\textit{mok  ɗaf}\textit{.} 
\end{itemize}

\textit{l-ɔm          =ala  }\textbf{\textit{máá-}}\textit{zʊm-ɔk}\textit{\textsuperscript{w}}\textit{       ɗaf} 



go \textsc{IMP}{}-2P  =to  1\textsc{Pin}.POT-eat-1\textsc{Pin}    loaf



‘Come; I want us to eat food.’ (lit. millet loaf)


\begin{itemize}
\item 
\textit{Lomala  }\textbf{\textit{ma}}\textit{d}\textbf{\textit{ə}}\textit{rok  meher}\textit{.}
\end{itemize}

\textit{l-ɔm  =ala    }\textbf{\textit{mà-}}\textit{d}\textit{ʊ}\textit{r-ɔk}\textit{\textsuperscript{w}}\textit{    mɛhɛr}



go \textsc{IMP}{}-2P  =to    1\textsc{Pin}.HOR-pray-1\textsc{Pin}  forehead



‘Come; I want us to pray together.’


%%please move \begin{table} just above \begin{tabular
\begin{table}
\caption{(from Friesen and Mamalis, 2008) shows the second and third person forms of a verb from each of the tone classes (H, L, toneless) in irrealis and realis moods.}
\label{tab:67}
\end{table}

\begin{tabular}{lllllll}
\lsptoprule

\multicolumn{2}{l}{\textbf{Underlying tone of verb stem}} & \multicolumn{2}{l}{\textbf{Realis }} & \multicolumn{3}{l}{\textbf{Irrealis}}\\
&  & \textbf{Imperfective tone} & \textbf{Perfective tone} & \textbf{Potential} & \textbf{Hortative} & \textbf{Possible}\\
\textbf{H} & \textbf{2S form} & \textit{kə-nzák-\={a}j}

‘you find’ & \textit{kə-nzák-\={a}j}

‘you found’ & \textit{ká-nzák-\={a}j}

‘I would like you to find’ & \textit{kà-nzák-áj }

‘you should find’ & \textit{k}\textit{á}\textit{à-}\textit{nzák-áj }

‘you might find’\\
& \textbf{3S form} & \textit{á-nzák-\={a}j}

‘he finds’ & \textit{à}\textit{{}-nz}\textit{á}\textit{k-\={a}j}

‘he found’ & \textit{á-nzák-\={a}j}

‘I would like him to find’ & \textit{mə-nzák-áj}

‘he should find’ & \textit{m}\textit{á}\textit{à}\textit{{}-nzák-áj}

‘he might find’\\
\textbf{L} & \textbf{2S form} & \textit{kə-tàts-\={a}j}

‘you close’ & \textit{kə-tàts-\={a}j}

‘you closed’ & \textit{ká-tàts-\={a}j}

‘I would like you to close’ & \textit{kà-tàts-\={a}j}

‘you should close’ & \textit{k}\textit{á}\textit{à-}\textit{tàts-\={a}j}

‘you might close’\\
& \textbf{3S form} & \textit{á-tàts-\={a}j}

‘he closes’ & \textit{à}\textit{{}-tàts-\={a}j}

‘he closed’ & \textit{á-tàts-\={a}j}

‘I would like him to close’ & \textit{mə-tàts-\={a}j}

‘he should close’ & \textit{m}\textit{á}\textit{à-}\textit{tàts-\={a}j}

‘he might close’\\
\textbf{toneless} & \textbf{2S form} & \textit{kə-ɮáw-\={a}j}

‘you fear’ & \textit{kə-ɮàw-\={a}j}

‘you feared’ & \textit{ká-ɮáw-\={a}j}

‘I would like you to fear’ & \textit{kà-ɮàw-\={a}j}

‘you should fear’ & \textit{k}\textit{á}\textit{à-}\textit{ɮàw-\={a}j}

‘you might fear’\\
& \textbf{3S form} & \textit{á-ɮáw-\={a}j}

‘he fears’ & \textit{à-ɮàw-\={a}j}

‘he feared’ & \textit{á-ɮáw-\={a}j}

‘I would like him to fear’ & \textit{mà-ɮàw-\={a}j}

‘he should fear’ & \textit{m}\textit{á}\textit{à-}\textit{ɮàw-\={a}j}

‘he might fear’\\
\hhline{~------}
\lspbottomrule
\end{tabular}

\begin{itemize}
\item \begin{styleTabletitle}
 Tone of realis three irrealis verb forms
\end{styleTabletitle}\end{itemize}

Verb forms in irrealis mood occur in Moloko discourse to express events that might occur. In the Cicada text, some young men go out to bring back a tree that was desired by their chief. The men try but can’t bring home the tree (which constitutes contrastive relief for the cicada’s success in the end).  A negative modal statement relates the unsuccessful attempt by the young men (ex. 519, from S.14). The lengthened subject prefix characterising irrealis mood is bolded in ex. 519. 


Cicada S.14



\ea
Albaya  ahay  tolo  amazala  ngəvəray  na,  \textbf{taa}zala  təta  bay.
\z

albaja        =ahaj  tɔ-lɔ     ama-z=ala   ŋgəvəraj        na  \textbf{tàà-}zaɗ\textbf{       =}ala    təta  baj



young man     =Pl       3P-go  DEP-take=to  spp.of tree  PSP  3P.HOR- take=to  ability  \textsc{NEG}



‘And then, the young men left to bring back the tree; but they were not able to bring it.’  


Also, dependent complement clauses represent things that were still future relative to the time of particular events on the eventline (see \sectref{sec:7.7}). They encode desired results that might not necessarily happen as illustrated in the examples below. 


Disobedient Girl S.13


\ea
Asa  asok  \textbf{aməhaya}  na,  kázaɗ  war  elé  hay  bəlen.
\z

asa  à-s          =ɔk\textsuperscript{w}  \textbf{amə-h    =aja}     na      



if     3S.IFV-please=2S.IO  DEP.PFV-grind  =\textsc{PLU}   PSP  



\textit{ká-zaɗ    war  ɛlɛ  haj         bɪlɛŋ}



2S.IFV-take    child  eye  millet  one



‘If you want to grind, you take only one grain.’ 



Cicada S.7


\ea
Agasaka  kə  mahay  ango  aka  \textbf{am}ə\textbf{mbese.}~~  
\z

a-gas  =aka      kə    mahaj  =aŋg\textsuperscript{w}ɔ         aka  \textbf{amɪ-mbɛʃ-ɛ}~~  



3S-catch   =on             on    door        =2S.POSS  on       \textbf{    }DEP-rest{}-CL



‘It would be good for you to have the tree at your door, so that you could rest under it.’ (lit. it catches on your door to rest)


\subsection{  Habitual iterative aspect}
\hypertarget{RefHeading1212281525720847}{}
The habitual iterative aspect\footnote{Friesen and \citet{Mamalis2008} called this ‘repetitive aspect.’} presents the actor(s) performing an action repeatedly as their usual habit. This aspect is formed by the gemination of the onset of the final syllable of the verb word.\footnote{There are no examples in the corpus with verbal extensions.} In a one-consonant root, the root consonant is doubled (ex. 522). The verb words showing this aspect are bolded in each of the examples and the reduplicated consonant is underlined. 


\ea
Kafta  kosoko  zlaba  na,  Məloko  anga  enen  ahay  \textbf{tó}\textbf{ll}\textbf{o}  a  ləhe.
\z

kafta kɔsɔk\textsuperscript{w}ɔ  ɮaba  na  Mʊlɔk\textsuperscript{w}ɔ  aŋga  ɛnɛŋ  =ahaj  \textbf{tɔ-}\textbf{ll}\textbf{ɔ}    a  lɪhɛ



day    market  Dogba  PSP  Moloko  \textsc{POSS}  another  =Pl  3P.IFV-go+\textsc{ITR}  at  bush



‘Each Sunday (the market of Dogba), some Molokos go to their fields (to work).’


In a CC root with no suffix, the first C of the stem is doubled (ex. 523, also see 524).

\ea
Tətərak  ango  nehe  na,  \textbf{k}\textbf{á}\textbf{ff}\textbf{əɗ}  ele  ango  a  mogom waya  azaɗ  merkwe  bay.   
\z

tətərak   =aŋg\textsuperscript{w}ɔ   nɛhɛ   na  \textbf{ká-}\textbf{ff}\textbf{əɗ}      ɛlɛ   =aŋg\textsuperscript{w}ɔ     a   mɔg\textsuperscript{w}ɔm 



shoes   =2S.POSS here   PSP  2S.IFV- put+\textsc{ITR}  thing   =2S.POSS     at   home   



\textit{waja  à-zà}\textit{ɗ}\textit{     m}\textit{ɛ}\textit{rk}\textit{\textsuperscript{w}}\textit{ɛ}\textit{   }\textit{baj}



because  3S.PFV-take  travel  \textsc{NEG}



‘Your shoes there, you should put them on (habitually, repeatedly, day after day) at home, because you can’t travel with them.’ (lit. they don’t take travel)


The fact that the reduplicated consonant is on the onset of the final syllable of the verb word (and not a particular consonant in the verb root) is illustrated by ex. 524 and 525, which show the same verb \textit{/ z m}\textit{\textsuperscript{ o}}\textit{ /} in the 2S and 2P forms. The 2P form has an extra syllable in the verb word because of the 2P subject pronominal suffix. In the 2S form, the reduplicated consonant is \textit{z} - the first consonant of the root. In the 2P form, the reduplicated consonant is \textit{m} – the second consonant of the root. However in both cases, the reduplicated consonant is the consonant at the onset of the final syllable in the verb word.

\ea
A  məjəvoko  ava  na,  \textbf{kó}\textbf{zz}\textbf{om  }ɗaf.  
\z

a   mʊdzʊvɔk\textsuperscript{w}ɔ  ava  na   \textbf{kʊ-}\textbf{zz}\textbf{ʊm  }    ɗaf  



in   feast    in  PSP  2S.IFV-eat+\textsc{ITR}    loaf 



‘During a feast, you eat repeatedly (many times at many people’s houses).’


\ea
A  məjəvoko  ava  na,  \textbf{k}ə\textbf{z}ə\textbf{mm}\textbf{om}  ɗaf.  
\z

a   mʊdzʊvɔk\textsuperscript{w}ɔ  ava  na   \textbf{kʊ -zʊ}\textbf{mm}\textbf{{}-ɔm}     ɗaf  



in   feast     in   PSP  2-\textsc{IFV}{}-eat+\textsc{ITR}{}-2P  loaf 



‘During a feast, you all eat (many times at many people’s houses).’  


Ex. 526 and 527 also show the reduplication of the onset of the final syllable of the verb word with a \textit{/-}\textit{j }/\textit{ }suffix.

\ea
Kosoko  molom  na,  ndam  pəra  ahay  \textbf{té}\textbf{ss}\textbf{e}  gəzom.
\z

kɔsɔk\textsuperscript{w}ɔ   mɔlɔm   na   ndam   pəra   =ahaj   \textbf{tɛ-}\textbf{ʃʃ}\textbf{{}-ɛ }      gʊzɔm



market  home  PSP  person  idol  =Pl  3P.IFV-drink+\textsc{ITR}  wine



‘On market day, the traditionalists drink wine (many people, much wine).’  


\ea
\textbf{Ada}\textbf{rr}\textbf{ay  e}teme  waya  gəvah  gam.
\z

\textbf{à-dà}\textbf{rr}\textbf{{}-\={a}j}    ɛtɛmɛ  waja  gəvax  gam



3S.PFV-plant+\textsc{ITR}  onion  because  field  lots



‘He/she planted many onions because his field was large.’


\subsection{  Intermittent iterative}
\hypertarget{RefHeading1212301525720847}{}
The intermittent iterative\footnote{Friesen and \citet{Mamalis2008} called this aspect simply ‘iterative.’} expresses the idea of the intermittent repetition of the same action, possibly by the same actor, over a period of time.\footnote{Moloko has two other forms that involve repetition of the same actions – the habitual iterative aspect (marked by reduplication of one consonant in the stem, see \sectref{sec:1154}) and the pluractional (marked by a verbal extension\textit{ =aja }or =\textit{ija}, see \sectref{sec:1157}).}  The intermittent iterative is formed by complete reduplication of the verb. Example 528 reflects a remark made by a friend concerning a situation where one duck died, then the owner bought another, and it died, and the situation was repeated four times.  In the examples, the verb complex is delimited by square brackets. 


\begin{itemize}
\item 
\textit{Andəbaba  ango  amət  amat.}
\end{itemize}

\textit{andəbaba =aŋg}\textit{\textsuperscript{w}}\textit{ɔ     }[\textit{a-mət   a-mat }]



duck  =2S.POSS  3S-die   3S-die



‘Your ducks keep dying.’ (lit. your duck dies it dies)


In the elicited example below, the situation is that a group of people has gone to the market and has bought several items from several different vendors. Note that the directional extension \textit{ala} occurs only once, following the second verb.

\ea
A  kosoko  ava  na,  nəskwəmom  nəskwəmom  ala.
\z

a  kɔsɔk\textsuperscript{w}ɔ  ava  na  [nʊ-sk\textsuperscript{w}ʊm-ɔm   nʊ-sk\textsuperscript{w}ʊm-ɔm =ala ]



to  market  in  PSP  1-buy-1\textsc{Pex}  1-buy-1\textsc{Pex}   =to



‘At the market, we buy and buy.’ (lit. at the market, we buy we buy)


\section{Verbal Extensions}
\hypertarget{RefHeading1212321525720847}{}
Friesen and \citet{Mamalis2008} found that the six verbal extensions in Moloko are a class of morphemes that modify the meaning of the verb. They are clitics which cliticise to the right edge of the verbal complex to form a phonological word.  The verb stem and the extensions may be separated syntactically by the indirect object pronominal clitics and third person DO pronominals (see Sections 491 and 50, respectively).  The extensions will trigger the loss of any prosody on the verb stem.

In Moloko there are three categories of verbal extensions.  Adpositionals (=\textit{aka} ‘on’ and =\textit{ava} ‘in’){ }\footnote{These locational extensions are the same as the locational clitics on adpositional phrases; see \sectref{sec:1146.}  } modify the meaning of the verb with particular reference to the location of the action. Directionals (\textit{=ala} ‘toward,’ \textit{=ala }‘away,’ and\textit{ =aja }‘back and forth’ or pluractional) add the idea of movement with respect to a particular point of reference. The third category is the Perfect =\textit{va.} 

\subsection{  Adpositionals}
\hypertarget{RefHeading1212341525720847}{}
There are two adpositional enclitics:\footnote{Friesen and \citet{Mamalis2008} called these ‘locationals.’}\textit{ =aka }‘on, on top of’ and\textit{ =ava }‘in.’  These extensions give the verb an added sense of the location of the action in the discourse.  The extension\textit{ =aka }‘on, on top of’ (ex. 530) resembles the second element of the adposition \textit{kə…aka} ‘on.’  In like manner,\textit{ =ava }‘in’ (ex. 531) resembles the adposition \textit{a…ava}  ‘in’ (see \sectref{sec:46}). The corresponding adpositional phrases often co-occur with the adpositionals. In the examples, the adpositions and adpositionals are bolded. 


\ea
Afəɗ\textbf{aka}  war  elé  hay  na,  \textbf{kə}  ver  \textbf{aka}.
\z

a-fəɗ=\textbf{aka}  war  ɛlɛ  haj  na  \textbf{kə}  vɛr  \textbf{aka}



3S-place  =on  child  eye  millet  PSP  on  stone  on



‘She put the grain of millet on the grinding stone.’



Disobedient Girl S. 26\footnote{Even though the verb in this example has verbal extensions, it is not conjugated for subject since it is a climactic point in the story where nominalised forms are often found (\sectref{sec:7.6.})  }


\ea
Məmət\textbf{ava}  alay  \textbf{a}  ver  \textbf{ava}\textbf{.}
\z

mə-mət=\textbf{ava  }=alaj  \textbf{a}   vɛr   \textbf{ava}



INF-die  =in  =away  in  room  in



‘She died in the room.’  


Adpositional extensions are phonological enclitics at the right edge of the verb.  Friesen and \citet{Mamalis2008} showed them to be phonologically bound to the verb stem because the \textit{{}-}\textit{aj}\textit{ }suffix drops off when the clitic attaches (see \sectref{sec:6.3}). Compare ex. 532 and 533 which illustrate the verb /g -j\textsuperscript{e} / ‘do.’ Note that the \textit{{}-}\textit{aj} suffix in the stem drops off when the extension \textit{=aka }is attached (ex. 533). Another piece of evidence that the extension is phonologically bound to the verb stem is that the palatisation of the verb stem is neutralised by the extension. Ex. 532 has no adpositional extension, and the verb word is palatalised, whereas in ex. 533 the locational extension\textit{ =aka }has neutralised the prosody of the entire verb complex.

\ea
Tege  cəɗoy.
\z

tɛ-g -ɛ  tsʊɗ[F08D?]j



3P-do-CL  trick



They played a trick. (lit. they did trick)


\ea
Tag\textbf{aka}  cəɗoy.
\z

ta-g   =\textbf{aka}   tsʊɗ[F08D?]j



3P-do   =on   trick



‘They played another trick.’ (lit. they did trick ‘on top’ [of when they did it before])


Another piece of evidence that the extensions are phonologically attached to the verb stem is that the word final allophones of /n/ and /h/, that is [ŋ ] and [x ], respectively, do not occur in the word final position in the verb word when the locational is attached. Ex. 534 illustrates that when the extension \textit{=va} cliticises to the verb /r h/ ‘fill,’ word final alterations of /h/ do not occur. These allophones would be expected if the verb stem and Perfect extension were separate words. 

\ea
Arah\textbf{va}  peɗeɗe.
\z

à-rah\textbf{=va}    pɛɗɛɗɛ



3S.PFV-fill=\textsc{PRF}  \textsc{ID}full



‘It had filled right to the rim.’ 


The adpositional does not cliticise to the verb in ex. 535 and 536 since the indirect object pronominal enclitic and plural subject suffix both trigger a word-final boundary (see \sectref{sec:7.1}), rendering the adpositional in a separate phonological word. In the examples, the boundaries of the phonological words are indicated by square brackets. 

\ea
Kanjaw  \textbf{aka}\textbf{.}
\z

{}[ka-nz=aw ]   [=\textbf{aka} ]



2S-sit=1S.IO    =on



‘You are better than me.’ (lit. you sit on me) 


\ea
Nədozlom  \textbf{ava}  a cəveɗ  ava  nə  məze.
\z

{}[nə-dɔɮ-ɔm ]      [\textbf{=ava }]  a  tʃɪvɛɗ  ava  nə  mɪʒɛ



1.PFV-intersect-1\textsc{Pex}   =in    in  road  in  with  person



‘We met a person on the road.’


The extension\textit{ =aka }‘on’ or ‘on top of’ also has the metaphorical meaning of ‘in addition to,’ ‘again,’ or ‘even still’ when the action of the verb occurs ‘on top of’ something that occurred previously; compare the following pair of examples, and note how the\textit{ =aka }in ex. 538 looks backward to another instance of the same action in ex. 537. 

\ea
Dərala.            
\z

dər       =ala            



move 2S.IMP   =to        



‘Come closer~(to me).’        


\ea
Dər\textbf{aka}  ala.
\z

dər       =\textbf{aka}   =ala



move 2S.IMP   =on   =to



‘Come even still closer.’  


Using\textit{ =aka }in a context where the addressee is eating renders the meaning ‘do you want any more ‘on top of’ what you have already eaten?’ (ex. 539).

\ea
Asok  \textbf{aka}  ɗaw?~
\z

a-s    =ɔk\textsuperscript{w}   =\textbf{aka}~  ɗaw~



3S-want  =2S.IO  =on   QUEST



‘Do you want any more?’ (lit. is it pleasing to you on?)  


With the verb \textit{mbaɗ}  ‘change,’ \textit{=aka }gives an idiomatic meaning to mark a change of speaker; that is, he spoke ‘on top of’ what the other person had just said.

\ea
Ambaɗaŋ  \textbf{aka.}
\z

a-mbaɗ  =aŋ   =\textbf{aka}



3S-change  =3S.IO   =on 



‘He/she replied.’ (lit. he changed to him on) 


\subsection{  Directionals}
\hypertarget{RefHeading1212361525720847}{}
Friesen and Mamalis found three directional extensions \textit{=ala}\textit{ }‘towards’ (ex. 541, 417),\textit{ =alaj }‘away from’ (ex. 549), and \textit{=aja/=ija} ‘back and forth movement’ (ex. 544). These directionals occur after the verb word and, if present, after the adpositional extensions as seen in ex. 541 and 542.  The directionals precede the Perfect (see \sectref{sec:58}), as seen in example 544.


\ea
Kazaka  \textbf{ala}  hor  ese.
\z

ka-zaɗ   =aka  =\textbf{ala}     h\textsuperscript{w}ɔr  ɛʃɛ



2S-take   =on   =to    woman  again



‘You take another wife’ (on top of the one you already have).\footnote{The root-final \textit{ɗ} of the verb \textit{zaɗ}\textit{ }\textit{ }‘take’ drops off when affixes and clitics are added (\sectref{sec:6.2}).  } (lit. you take a wife on )


\ea
Təjapata  aka  \textbf{ala}  ana  Məloko  enen  ahay.
\z

tə-dzap=ata     =aka   =\textbf{ala}     ana  Mʊlɔk\textsuperscript{w}ɔ    ɛnɛŋ  =ahaj  



3P-group=3S.IO   =on  =to     DAT   Moloko    another  =Pl  



‘They grouped together again against some of the Molokos’ (point of reference is the Molokos)


\ea
Dəraka \textbf{alay.}  
\z

dər      =aka  =\textbf{alaj}  



move 2S.IMP   =on  =away



‘Move further away~(from me).’



Race story\footnote{Friesen 2003.}


\ea
Moktonok\textsuperscript{  }na,  abək  ta  \textbf{aya}  va  məlama  ahan  ahay  jəyga.
\z

mɔk\textsuperscript{w}tɔnɔk\textsuperscript{w}na  a-bək          ta  =a\textbf{ja  }=va  məlama  =ahaŋ    =ahaj  dzijga



toad    PSP  3S-invite   3P  =\textsc{PLU}  =\textsc{PRF}  brothers  3P.POSS  =Pl  all



‘The toad, he had already invited all of his brothers.’  (this action took place away from the events of the story)


Like the adpositionals, the directionals are phonological clitics at the right edge of the verbal complex.  The presence of the enclitics requires that the \textit{{}-}\textit{aj} suffix be dropped off (the verb stem in example 542 is /dzap -j/ ‘mix’). The neutral prosody of these extensions causes the palatalisation on the verb stem to neutralise. In ex. 545 the verb stem is / nz -j\textsuperscript{ e} / ‘go’ with a 3S surface form of [ɛnʒɛ].  

\begin{itemize}
\item 
\textit{Anj}\textbf{\textit{ala}}\textbf{\textit{.}}
\end{itemize}

\textit{a-nz   =}\textbf{\textit{ala}}



3S-go   =to



‘He/she is coming.’  


Directional extensions orient the event expressed by the verb relative to a centre of reference.  In speech, that point of reference is usually the speaker, so actions are seen as going towards the speaker (=\textit{ala}), away from the speaker (=\textit{a}\textit{laj}), or back and forth (=\textit{aj}\textit{a}). Compare the following examples of the verb /s k\textsuperscript{w} m/ ‘buy/sell’ with a first person subject. When used with the directional \textit{=ala} ‘toward,’ the verb means ‘buy’ (ex. 546). When it is used with the directional\textit{ =alaj }‘away,’ it means ‘sell’ (ex. 547).

\ea
Nəskom\textbf{ala}  awak.
\z

nə-sk\textsuperscript{w}ɔm\textbf{=ala}    awak



1S.PFV-buy/sell=to    goat



‘I bought a goat.’


\ea
Nəskom\textbf{alay}  awak. 
\z

nə-sk\textsuperscript{w}ɔm    \textbf{=alaj}  awak 



1S.PFV-buy/sell  =away  goat 



‘I sold my goat.’


The directional\textit{ =ala }‘toward’ indicates an action that moves toward the centre of reference (see ex. 548 and 550). The directional\textit{ =alaj }‘away’ indicates an action that moves away from that centre (see ex. 549 and 551). Compare the example pairs for  /d r/ ‘move’ (ex. 548 and 549) and for \textit{/}z ɗ/ ‘take’ (ex. 550 and 551). In each example pair, the first shows an action towards the speaker and the second shows an action away from the speaker. 

\ea
Dər\textbf{ala}\textbf{.}            
\z

dər      =\textbf{ala}            



move 2S.IMP   =to        



‘Come closer~(to me).’        


\ea
Dəralay.  
\z

dər      =\textbf{alaj}  



move 2S.IMP   =away



‘Move away~(from me).’


\ea
Z\textbf{ala}  eteme.      
\z
\ zaɗ    =\textbf{ala}    ɛtɛmɛ      



take 2S.IMP  =to    onion      



‘Bring the onion (to me).’


\ea
Z\textbf{alay}  eteme.
\z
\ zaɗ    =\textbf{alaj}    ɛtɛmɛ



take 2S.IMP  =away    onion



‘Take the onion away (from me).’


The third directional\textit{ =aja }or \textit{=}\textit{ija} gives the idea of repetitive movement back and forth. This repetitive back and forth movement is called pluractional.\footnote{A verbal extension or affix is one way of showing pluractional actions in other Chadic languages (Newman, 1990). The other is reduplication of the verb root. Such verb root reduplication is also seen in Moloko for habitual iterative aspect \citep{Section1154} and intermittent iterative aspect \citep{Section1155}. } A few verbs never occur without the pluractional and involve regular back and forth movements like sawing (ex. 552) or grinding (ex. 553). For other verbs, adding the directional adds a back and forth movement to the sense. Ex. 544 above involves the subject going from person to person to invite them to help. 

\ea
Zar  asəya  memele.
\z
\ zar  a-s=ija    mɛmɛlɛ



man    3S-saw=\textsc{PLU}  tree



‘The man saws the tree.’ 


\ea
Aban  ahaya  hay.
\z

Abaŋ   a-h  =aja  haj



Abang  3S-grind  =\textsc{PLU}  millet



‘Abang grinds millet.’


Directionals are a device used in Moloko discourse to help provide cohesion.\footnote{Other discourse devices which function in cohesion include demonstratives (Chapter 3.2), the adjectiviser enclitic =\textit{ga} (Chapter 5.3), the presupposition marker na (Chapter 12), and participant tracking (\sectref{sec:7.3}). } Directionals keep the hearer oriented to the events of a story and how they relate to a particular spatial point of reference (a place or dominant character). The point of reference may remain constant throughout the whole story or it may change during the story. Selected lines from the Cicada text (example 554) illustrate how directionals relate main line events to the point of reference which is the chief (or the place in his compound where he makes the wine). The directionals are bolded in the examples. The presence of the two directionals in ex. 557 and 558 is the only way in the story that we know that the cicada brought the tree back to the chief (until the chief thanks him in line 34).


Cicada S.6 


\ea
Albaya  ahay  ndana  kəlen  təngala\textbf{ala}  ma  ana  bahay.  
\z

albaja =ahaj  ndana  kɪlɛŋ  tə-~ŋgala    \textbf{=ala}  ma  ana  bahaj  



youth  =Pl  DEM  then  3P.PFV-return  =to    word    DAT    chief



‘The above-mentioned young men then took the word (response) to the chief.’ (lit they returned the word to the chief)



S. 12 


\ea
Tolo  tamənjar  na \textbf{ ala}  mama  ngəvəray  nəndəye.
\z

tə-lɔ          tà-mənzar     na      \textbf{=ala}     mama      ŋgəvəraj     nɪndijɛ



3P .PFV-go   3P.HOR-see    3S.DO   =to     mother       species of tree    DEM



They went to see [for the chief] that mother-tree.



S.16


\ea
Kəlen  albaya  ahay  tolo  amaz\textbf{ala}  ngəvəray  na,  taaz\textbf{ala}  təta  bay.
\z

kɪlɛŋ   albaja    =ahaj    tɔ\`{ }-lɔ      ama-z   \textbf{=ala}      ŋgəvəraj            na



then   youth      =Pl      3P.PFV-go  DEPtake  =to       spp.of tree  PSP



tàà-z     \textbf{=ala}  təta    baj



3P.HOR-take   =to      ABILITY  \textsc{NEG}



And then, the young men left to bring back the tree [to the chief]; but they were not able to bring it [to the chief].  



S. 30


\ea
Amag\textbf{ala}  ləmes.  
\z

ama-g   \textbf{=ala}  lɪmɛʃ  



DEP-do  =to  song



He was singing towards [the chief’s house]. (lit. to do towards a song)



S.31


\ea
Sen  \textbf{ala}\textbf{.}  
\z

ʃɛŋ          \textbf{=ala}  



\textsc{ID}walking               =to



‘Walking along, he came [to the chief’s house].’


Sometimes the directional \textit{=ala} ‘towards’ (see see \sectref{sec:57}) can carry a Perfect kind of idea (an event being completed before a temporal reference point with ongoing effects to that time) but which has a slightly different connotation to the Perfect extension \textit{=va}. Compare ex. 559 and 560. Use of the directional \textit{=ala} ‘towards’ (ex. 559) with the verb /z m\textsuperscript{o}\textit{ }/ indicates that the person has already eaten, but at some other location, since the directional gives the idea that food has come to the speaker. Use of the Perfect itself (ex. 560) indicates that the person has finished eating (at the place where he is sitting). As such, the directional\textit{ =ala } may be in the process of becoming grammaticalised for past tense or a subtype of Perfect.

\ea
Nəzəm\textbf{ala  }toho.
\z

nə-zəm  \textbf{=ala  }tɔh\textsuperscript{w}ɔ



1S.PFV-eat  =to  DEM



‘I already ate over there (some other person’s house – before I arrived here).’


\ea
Nəzəm\textbf{va  }pew.
\z

nə-zəm  \textbf{=va  }pɛw



1S.PFV-eat  =\textsc{PRF}  enough



‘I already ate/ I have eaten enough (here in this place since I arrived here).’ 


Likewise, the verb /s k\textsuperscript{w} m\textsuperscript{o} / ‘buy/sell’ is given a Perfect idea when it carries the \textit{=ala} extension. In ex. 546, the goat has come to the speaker. There is no Perfect extension (\textit{=va})  but the idea is accomplished through the directional \textit{=ala}.

\ea
Nəskom  na  ala  awak.
\z

nə-sk\textsuperscript{w}ʊm    na  \textbf{=ala  awak}



1S.PFV-buy/sell  3S.DO  =to  goat



‘I bought the goat (and it is mine now).’


\subsection{  Perfect}
\hypertarget{RefHeading1212381525720847}{}
The final extension is =\textit{va}, the Perfect (Friesen and Mamalis, 2008). The Perfect marks events or states as having occurred prior to a particular point of reference, with ongoing effect that continues to that point of reference (Comrie, 1976). The Perfect extension is bolded in the examples. 


\ea
Ta  awəy, “Ambəɗə\textbf{va  }anga  ləme.
\z

ta awij     à-mbəɗ    =\textbf{va}  aŋga  lɪmɛ



3P  saying    3S.PFV-change  =\textsc{PRF}  \textsc{POSS}  1\textsc{Pex}



‘They said, “It has become ours!”’ (lit. it has changed; belonging to us)


\ea
Nasar  həraf  ɛlɛ  nəngehe  asabay,  
\z

nà-sar      həraf           ɛlɛ       nɪŋgɛhɛ     asa-baj  



1S.PFV-know     medicine     thing    DEM      again-\textsc{NEG}  



\textit{waya  nəl}\textbf{\textit{va}}\textit{  afa  s}\textit{ə}\textit{wp}\textit{ə}\textit{refe.}



\textit{waja  nə-l          =}\textbf{\textit{va}}\textit{       afa    suwpɪrɛfɛ}



because  1S.PFV-go     =\textsc{PRF}    house of    sub prefect



‘I didn’t know how to resolve the problem, because I had already been to the sub-prefect [and he didn’t help me].’  


\ea
Təta  na,  tanjakə\textbf{va}  ɛlɛ  məzəme.  
\z

təta    na  tà-nzak    =\textbf{va}  ɛlɛ  mɪ-ʒɪm-ɛ  



3P    PSP  3P.PFV-find    =\textsc{PRF}  thing  \textsc{NOM}{}-eat-CL



‘And so they had found something to eat.’  


\ea
Arahə\textbf{va}  peɗeɗe.
\z

à-rah\textbf{=va}    pɛɗɛɗɛ



3S.PFV-fill=\textsc{PRF}  \textsc{ID}full



‘It had filled right to the rim.’ 


\ea
Nəzəm\textbf{va}\textbf{.}
\z

nə-zəm\textbf{=va}



1S.PFV-eat=\textsc{PRF}



‘I already ate.’ 


Unlike the other extensions, the Perfect\textit{ }enclitic has two possible positions in the verb phrase. It can either be phonologically bound to the right edge of the verbal complex (see \sectref{sec:7.1}) or to the right edge of the clause (Chapter 8) after the direct object and adpositionals. In ex. 562 - 565, 569, =\textit{va}\textit{ }follows the adpositional and directional extensions in the verb complex and precedes other elements in the verb phrase. In ex. 567 and 570, =\textit{va} occurs at the end of the clause, a rarer construction that presumably occurs to underscore the idea that the event is already finished.


Disobedient Girl S. 17


\ea
Azla  na,  hor  na,  asərkala  afa  təta\textbf{  va}  na,  aməhaya  háy  na,  gam.
\z

aɮa    na  h\textsuperscript{w}ɔr     na  à-sərk          =ala  afa           təta    \textbf{=va}   na



now      PSP  woman  PSP  3S.PFV-HAB=to     at house of     3P  =\textsc{PRF}    PSP



\textit{amə-h  =aja       haj        na  gam}



DEP-grind  =\textsc{PLU}  millet  PSP    a lot



‘Now, that woman, she was in the habit at their place of grinding a lot of millet.’  


The Perfect extension has neutral prosody itself and causes the loss of palatalisation of the verb stem (compare ex. 568{}-569 where the stem is / s -j\textsuperscript{ e}\textit{ }/).  Also, verb stems drop their \textit{{}-a}\textit{j} suffix when this extension is present. These features all confirm that \textit{=va} is an enclitic. In ex. 568 without the Perfect, the verb stem is palatalised. Ex. 568 shows that when the verb carries the Perfect extension, the stem loses its palatalisation. 

\ea
Nese  gəzom.
\z

nɛ-ʃ{}-ɛ       gʊzɔm



1S.PFV-drink-CL    millet.beer



‘I drank millet beer.’


\ea
Nasa\textbf{va  }gəzom.    
\z

nɛ\textbf{{}-}s  {}-a       =\textbf{va  }  gʊzɔm    



1S.PFV-drink  = \textsc{PRF}  millet beer  



‘I drank millet beer already.’  


Notably, palatalisation is lost even when there are intervening words (ex. 570), even though the prosody of these words is unaffected. 

\ea
Nasa  gəzom  \textbf{va}\textbf{.}
\z

nà\textbf{{}-}s  {}-a      gʊzɔm    =\textbf{va}



1S.PFV-drink    millet beer  =\textsc{PRF}



‘I drank millet beer already.’  


Likewise ex. 571 illustrates the loss of palatalisation from the root / g -j\textsuperscript{ e}\textit{ }/ ‘do’when the Perfect is added. 

\ea
Ləho  aga\textbf{v}\textbf{a}\textbf{.}
\z

lʊh\textsuperscript{w}ɔ    à-g-a \textbf{=v}\textbf{a}



late afternoon  3S.PFV-do=\textsc{PRF}



‘It is the cool of the day (after three o’clock).’ (lit. late afternoon has done)  


\citet{Bow1997c} established that the Perfect extension\footnote{\citet{Bow1997c} called it an aspect or tense marker.} carries a floating tone. Its underlying tone is HL. She demonstrates the floating tone using two verbs with different tone melodies; the high tone verb /bal {}-j/ ‘wash’ (ex. 572{}- 573) and the low tone verb /a-dar {}-j/ ‘plant’ (ex. 574 - 575), both with the object noun [háj] ‘millet.’ Ex. 572 and 574 show the two clauses without the Perfect for comparison. Comparing ex. 573 with 575 demonstrates that the floating low tone on the Perfect has lowered the tone of ‘millet’ from high to mid since there is no other low tone apparent that could be responsible for the lowering.  

\ea
Nəbalay  háy.
\z

{}[nə-báláj   háj]  



1S.IFV-wash  millet



‘I wash the millet.’


\ea
Nəbalva  háy.  
\z

{}[nə-bál=vá     h\={a}j]  



1S.PFV-wash=\textsc{PRF}  millet



‘I washed the millet already.’  


\ea
 Nədaray  háy.
\z

{}[nə-dàr\={a}j     háj]  



1S.IFV-plant  millet



‘I plant the millet.’


\ea
Nədarva  háy.  
\z

{}[nə-dàr=v\={a}     h\={a}j]  



1S.PFV-plant=\textsc{PRF}  millet



‘I planted the millet already.’


The Perfect extension can mark information in a relative clause (see \sectref{sec:44}) as having been accomplished before the information in the main clause, with relevance to the point of reference in the main clause (ex. 576). 

\ea
War  elé  háy  ngəndəye  nok\textsuperscript{  }ameze  na  \textbf{va},  bəlen  ngəndəye  na,  káahaya  kə  ver  aka.
\z

war     ɛlɛ  haj      ŋgɪndijɛ   [nɔk\textsuperscript{w}  amɛ-ʒ{}-ɛ     na     =\textbf{va}]     bɪlɛŋ    ŋgɪndijɛ   na



child  eye    millet  DEM      2S        DEP-take-CL   3SDO     =\textsc{PRF}      one        DEM       PSP



\textit{káá-}\textit{h    =aj}\textit{a  kə  vɛr             aka}



2S.POT-grind  =\textsc{PLU}  on  grinding stone     on



‘That grain that you have taken, that one [grain], grind it on the grinding stone.’


When the Perfect co-occurs with Perfective aspect (ex. 562 - 567), it indicates that the event expressed by the verb took place before the point of reference established in the discourse. When the Perfect co-occurs with Imperfective aspect (ex. 577 - 579), the verb is resultative, referring to an ongoing state that is the result of a previous completed event (filling, becoming tired, ripening. or becoming angry).

\ea
Árahə\textbf{va}\textbf{.} 
\z

á-ráh   =\textbf{va} 



3S.IFV-fill   =\textsc{PRF}



‘It is full.’ 


\ea
Mana  áyəɗə\textbf{va.}
\z

Mana      á-jəɗ   =va



Mana      3S-tire  \textsc{PRF}



‘Mana is tired.’


\ea
Háy  ánahə\textbf{va}\textbf{.}
\z

haj      á-nah   =va



millet  3S-ripen  \textsc{PRF}



‘The millet is ripe.’



Disobedient Girl  S.33


\ea
Məloko  ahay  ta  awəy  Hərmbəlom  ága  ɓərav\textbf{  va}\textbf{  }kəwaya  war  dalay  na, amecen  sləmay  bay  ngəndəye.  
\z

mʊlɔk\textsuperscript{w}ɔ   =ahaj  ta  awij~  Hʊrmbʊlɔm    á-g-a       ɓərav  \textbf{=va}  



Moloko    =Pl  3P  say  God            3S.IFV-do    heart      =\textsc{PRF}  



\textit{kuwaja   war     dalaj  na      amɛ-}\textit{tʃ}\textit{ɛ}\textit{ŋ}\textit{      ɬəmaj  baj      ŋgɪndijɛ}



because    child  girl        PSP  DEP-hear  ear       \textsc{NEG}  DEM



‘The Molokos say “God had gotten angry  because of that girl, that one that was disobedient.”’


In narrative discourse, the Perfect verbal extension \textit{=va} marks events that occur prior to the events on the main story line, and which supply flashback information to the story. For example, in the setting of the Disobedient Girl story (S. 2), the Perfect marks God giving his blessing to the people. This blessing preceded the events of the story (ex. 581) and had an ongoing effect at the time of the story. 


Disobedient Girl  S.3


\ea
Zlezle  na,  Məloko  ahay  na,  Hərmbəlom  ávəlata  barka  \textbf{va}      
\z

ɮlɛɮɛ      na   Mʊlɔk\textsuperscript{w}ɔ  =ahaj  na  Hʊrmbʊlɔm  á-vəl         =ata  barka   =\textbf{va}      



long ago       PSP   Moloko  =Pl        PSP   God    3S.IFV-send  =3P.IO  blessing   =\textsc{PRF} 



‘Long ago, to the Moloko people, God had given his blessing.’


In the body of the Disobedient Girl story (S.17; ex. 567), the story flashes back to the woman’s prior situation, using the Perfect, in order to prepare the reader/hearer for what will happen next in the story. In the body of another fable (the race between the giraffe and the toad, Friesen, 2003), the Perfect marks a flashback to a prior action of the toad.

\ea
Macəkəmbay  moktonok  na,  abək  ta  aya  \textbf{va  }  məlama  ahan  ahay  jəyga.
\z

matsəkəmbaj~  mɔk\textsuperscript{w}tɔnɔk\textsuperscript{w}   na  a-bək       ta      =aja  =\textbf{va}  



meantime\textit{      }toad    PSP  3S-invite   3P.DO  =\textsc{PLU}   =\textsc{PRF}  



\textit{məlama  =ahaŋ    =ahaj  dzijga}



brother  =3S.POSS  =Pl  all



‘In the meantime the toad, he had already invited all of his brothers.’  


\section{Nominalised verb form}
\hypertarget{RefHeading1212401525720847}{}
The nominalised verb form\footnote{Friesen and Mamalis called this form the ‘infinitive.’} is derived from a verb stem by the addition of the prefix /\textit{m-}/\textit{ }plus a palatalised suffix [{}-ɛ].\footnote{There is also an irregular nominalisation process that has already been discussed (\sectref{sec:4.2}).  }  Syntactically, the nominalised form can pattern as a noun (see \sectref{sec:59}), and in certain cases it can pattern as a verb, taking some inflectional components such as object suffixes and extensions (see \sectref{sec:60}). In the examples below, both underlying and nominalised forms are given. The nominalised form indicates an event (race, ex. 583;  betrayal, ex. 584) or state (beauty, ex. 585;  coldness, ex. 586).  


\ea
\textup{/h-m -j/     mɪ-hɪm-ɛ}\textup{  }
\z

‘run'     ‘race’


\ea
\textup{/ t}\textup{ʃ}\textup{af}\textup{\textsuperscript{ e}}\textup{ /    mɪ- tʃɛf-ɛ}\textup{ }
\z

‘betray’     ‘betrayal’


\ea
\textup{/r ɓ -j/    mɪ-rɪɓ-ɛ}
\z

‘be beautiful’  ‘beauty’


\ea
\textup{/ ndaɬ -j}\textup{\textsuperscript{e}}\textup{ /    mɪ-ndɛɬ-ɛ}
\z

‘make cold’    ‘coldness’


In the case where a verb stem consists of one single consonant, the nominalised form receives an additional syllable [{}-ijɛ].

\ea
\textup{/dz -j/    mɪ-dʒ-ijɛ}
\z

‘say’    ‘saying’


\ea
\textup{/ s -j}\textup{\textsuperscript{ e}}\textup{ /    mɪ-ʃ{}-ijɛ  }
\z

‘drink’     ‘drinking’


\ea
\textup{/ l}\textup{\textsuperscript{ o}}\textup{ /    mɪ-l-ijɛ}
\z

‘go’      ‘going’


If present, the underlying \textit{a-} prefix in a verb stem shows up in the prefix vowel of the nominalised form. The prefix vowel in an \textit{a-} prefix verb is full; in 590 and 591, this full vowel is realised as [ɛ] due to the palatalisation prosody which is part of the nominalising morphology. Compare with ex. 583 - 586 where [mɪ-] is the prefix for verb stems with no\textit{ a-} prefix. 

\ea
\textup{/a- d a r –aj/    mɛ-dɛr-ɛ}
\z

‘plant’    ‘planting’


\ea
\textup{/a- d l/    mɛ-dɪl-ɛ}
\z

‘overtake’    ‘overtaking’


The tone pattern of the nominalised form reflects the underlying tone of the verb stem.  \tabref{tab:68}. (from Friesen and Mamalis, 2008) illustrates a few nominalised forms that suggest this pattern.

\begin{tabular}{lllll}
\lsptoprule

\textbf{Tone Class} & \textbf{Underlying form} & \textbf{Nominalised form} & \textbf{Imperative} & \textbf{Gloss}\\
\textbf{High tone verb stems} & /nz a k -j / & \textit{mɪ\'{ }-nʒɛk-ɛ} & \textit{nzák-áj} & ‘find’\\
\hhline{-~~~~} & / z m\textsuperscript{ o} / & \textit{mɪ-ʒùm-ɛ} & \textit{zɔm} & ‘eat’\\
\textbf{Low tone verb stems without depressor consonants} & /f ɗ/ & \textit{mɪ-fɪɗ-}\textit{ɛ}\textit{ } & \textit{f\={a}ɗ} & ‘put’\\
\hhline{-~~~~} & /tats -j / & \textit{mɪ-t\={e}tʃ-}\textit{ɛ} & \textit{t\={a}ts-áj} & ‘close’\\
\textbf{Low tone verb stems with depressor consonants} & /v h n -j / & \textit{mɪ-vɪhɪn-}\textit{ɛ}\textit{ } & \textit{vəhən-\={a}j} & ‘vomit’\\
\hhline{-~~~~} & /a-dar -j / & \textit{m}\textit{ɛ}\textit{{}-d}\textit{ɛ}\textit{r-}\textit{ɛ} & \textit{dàr-\={a}j} & ‘plant’\\
\textbf{Toneless verb stems} & /d ɗ / & \textit{mɪ-dɪɗ-}\textit{ɛ} & \textit{dàɗ} & ‘fall’\\
\hhline{-~~~~} & /nd z / & \textit{mɪ-ndɛʒ-}\textit{ɛ} & \textit{ndàz} & ‘pierce’\\
\hhline{~----}
\lspbottomrule
\end{tabular}

\begin{itemize}
\item \begin{styleTabletitle}
 Nominalised form tone patterns
\end{styleTabletitle}\end{itemize}
\subsection{ Nominalised form as noun}
\hypertarget{RefHeading1212421525720847}{}
As a noun, the nominalised form takes modifiers the same as any abstract noun, i.e., quantifier (ex. 593) , numeral (ex. 594), possessive pronoun (ex. 592), demonstrative (ex. 595), adjectiviser (ex. 596 - 598) but not plural (see \sectref{sec:36}). Any argument of the clause can be realised with a nominalisation. The noun phrase is marked off by square brackets and the nominalised form is bolded in the examples. 


\ea
{}[\textbf{M}\textbf{ə}\textbf{h}\textbf{ə}\textbf{me  }aloko   na ],  epeley?~
\z

{}[\textbf{mɪ-hɪm-ɛ}    =alɔk\textsuperscript{w}ɔ    na ],  ɛpɛlɛj~



\textsc{NOM}{}-run-CL  1\textsc{Pin}.POSS  PSP  when



‘When is our race?’ (lit. our running [is] when)



Disobedient Girl S. 4


\ea
Ávata  [\textbf{məvəye}  haɗa. ]
\z

á-v=ata       [\textbf{mɪ-v-ijɛ}      haɗa ]



3S.IFV-spend\_time=3P.IO  \textsc{NOM}{}-spend\_time-CL  many



‘It will last them many years.’  (lit. it will last for them enough lastings\footnote{The nominalised form of the verb ‘last’ has been lexicalized as ‘year.’})


\ea
Ege  [\textbf{m}\textbf{ə}\textbf{v}\textbf{əye  }məko ] ehe,  nawas  háy  əwla.
\z

ɛ{}-g-ɛ  [\textbf{mɪ-v-ijɛ  }  mʊk\textsuperscript{w}ɔ ]  ɛhɛ  na-was    haj  =uwla



3S-do-CL  \textsc{NOM}{}-last-CL  six  here  1S-cultivate  millet  =1S.POSS



‘Six years ago (lit. it did six years), I cultivated my millet.’ 


\ea
{}[\textbf{Med}\textbf{ə}\textbf{re}  nehe   na ],  səlom  ga.
\z

{}[\textbf{mɛ-dɪr-ɛ}    nɛhɛ  na ]  sʊlɔm    ga



\textsc{NOM}{}-plant-CL  DEM  PSP  goodness  ADJ  



‘This planting is good.’


Adjectives can be further derived from a nominalised verb form by adding \textit{ga}, as is true of any noun (Chapter 4.3)\textit{. }Adjectives that are derived from nominalised verbs express resultant states. For example, the peanuts in ex. 596 are already ground, the woman in ex. 597 is already beautiful, the man is already seated in ex. 598. The nominalised forms are bolded in the examples. 

\ea
Nadok  [andəra  məngəlɗe  ga. ]
\z

na-d=ɔk\textsuperscript{w}    [andəra\textbf{    mɪ-ŋgɪlɗ-ɛ}    \textbf{ga }]



1S-prepare=2S.IO  peanut     \textsc{NOM}{}-grind-CL    ADJ



‘I made peanut butter (lit. ground peanuts) for you.’ 


\ea
Avəlaw  [war  dalay  \textbf{mərəɓe}  ga. ]
\z

a-vəl=aw    [war   dalaj  \textbf{mɪ-rɪɓ-ɛ}      ga ]



3S-give=1S.IO  child  female  \textsc{NOM}{}-be\_beautiful-CL  ADJ



‘He/she gave me a beautiful girl.’ 


\ea
Ndahan  [\textbf{mənjəye}  ga. ]
\z

ndahaŋ  [\textbf{mɪ-nʒ-ijɛ}  ga ]



3S    \textsc{NOM}{}-sit-CL  ADJ



‘He/she [is] seated.’ 


It is interesting that noun phrases where the head noun is a nominalised verb behave like a clause when there is a noun modifier. The nominalised verb can be the head of a genitive construction (see \sectref{sec:42}) or a permanent attribution construction (see \sectref{sec:43}) or an argument in another clause (see \sectref{sec:82}). In the genitive construction (ex. 592 and 601), the second noun represents the subject of the verb stem. In the permanent attribution construction (noun plus noun, ex. 599 and 904 from \sectref{sec:82}), the second noun represents the direct object of the nominalised verb. 

\ea
məbeze  háy 
\z

mɪ-bɛʒ-ɛ     haj 



\textsc{NOM}{}-harvest-CL  millet



‘the millet harvest’


\ea
andəra  məngəlɗe  ga
\z

andəra  mɪ-ŋgɪlɗ-ɛ    ga



peanut   \textsc{NOM}{}-grind-CL    ADJ



‘ground peanuts’


\ea
mənjəye  a  Mana    
\z

mɪ-nʒ-ijɛ     a   Mana    



\textsc{NOM}{}-sit-CL    GEN  Mana    



‘Mana’s behaviour’ (lit. the sitting of Mana)


\ea
\textbf{məhəme}  aloko
\z

\textbf{mɪ-hɪm-ɛ}    =alɔk\textsuperscript{w}ɔ



\textsc{NOM}{}-run-C    =1\textsc{Pin}.POSS



‘our race’ (lit. the running of us)


\subsection{ Nominalised form as verb}
\hypertarget{RefHeading1212441525720847}{}
The nominalised form can fill the verb slot in a clause (discussed further in Section  64). Ex. 603 and 604 are full (complete) clauses on the main event line where the verb is in nominalised form. Such clauses are found at the inciting moment and peak of a narrative.  The nominalised form is not conjugated for subject or direct object, but the clause may have a subject (the 3S pronoun \textit{ndahaŋ} in ex. 603) or direct object (\textit{jam} ‘water’ in ex. 603) and other clausal elements. The nominalised form can take verbal extensions (3P indirect object \textit{=ata} and adpositional \textit{=aka} in ex. 603; the adpositional \textit{=ava} and the directional \textit{=alaj} in ex. 604). 


\ea
Ndahan  ngah\textbf{  mangəhata  aka  va}  yam  a  ver  ahan  ava.
\z

ndahaŋ  ŋgah\textbf{  ma-ŋgəh=ata=aka=va  }  jam  a  vɛr  =ahaŋ    ava



3S    hide  \textsc{NOM}{}-hide=3P.IO=on=\textsc{PRF}  water  in  room  =3S.POSS  in



‘He had hidden the water in his room’ (lit. he hide-hiding water in his room)


\ea
Məmətava  alay  a  ver  ava.
\z

mə-mət  =ava  =alaj    a   vɛr   ava



\textsc{NOM}{}-die  =in  =away    in  room  in



‘[She] died in the room.’ (lit. death in the room) 


\subsection{Verb focus construction}
\hypertarget{RefHeading1212461525720847}{}
The nominalised form of a verb is used in an idiomatic construction that functions to bring focus on the verb. The verb focus construction is composed of an inflected verb followed by an adpositional phrase (see \sectref{sec:45}) containing the same verb in nominalised form. Ex. 605 shows the construction \textit{nʊ-sk}\textit{\textsuperscript{w}}\textit{ɔm }\textit{nə mɪ-ʃk}\textit{\textsuperscript{w}}\textit{ø}\textit{m-ɛ } ‘I really did buy it’ (lit. I bought [it] with buying). This construction specifies that the action is done ‘by means of’ or ‘by actually’ doing something (to the exclusion of all other possibilities).  It is used by the speaker to contest a real or implied challenge of the validity of what has been said. In ex. 605, the speaker is saying that he actually bought a particular item, i.e. he didn’t steal it and nobody gave it to him.


\begin{itemize}
\item 
\textit{Awəy,  “Nəskom  }\textbf{\textit{nə  məsk}}\textit{ə}\textbf{\textit{m}}\textbf{\textit{e.”}}
\end{itemize}

\textit{awij    nʊ-sk}\textbf{\textit{\textsuperscript{w}}}\textit{ɔm   }\textbf{\textit{nə   mɪ -sk}}\textbf{\textit{\textsuperscript{w}}}\textit{ø}\textbf{\textit{m-ɛ}}



he/she said  1S-buy    with  \textsc{NOM}{}-buy-CL



‘He said, “I actually bought it.”’ (lit. I bought it with buying) 


\begin{itemize}
\item 
\textit{Káslay  awak  }\textbf{\textit{nə}}\textit{  }\textbf{\textit{m}}\textbf{\textit{əsləye}}\textit{.}
\end{itemize}

\textit{ká-ɬ{}-aj    awak  }\textbf{\textit{nə}}\textit{  }\textbf{\textit{mɪ-ɬ{}-ijɛ}}



2S.IFV-slay{}-CL  goat  with  \textsc{NOM}{}-slay-CL



‘You kill goats by cutting their throat and not by any other way’ (lit. you slay a goat with slaying)


\begin{itemize}
\item 
\textit{Kákaɗ  okfom  }\textbf{\textit{nə  məkəɗe}}\textit{.  Káslay  bay.}
\end{itemize}

\textit{ká-kaɗ    ɔk}\textit{\textsuperscript{w}}\textit{fɔm  }\textbf{\textit{nə  mɪ-kɪɗ-ɛ}}\textit{      ka-ɬ{}-aj    baj}



2S.IFV-kill(club)  mouse  with  \textsc{NOM}{}-kill(club) -CL  2S.IFV-slay-CL  NOT



‘You kill mice by smashing their head; you don’t cut their throats.’  


\begin{itemize}
\item 
\textit{Kándaz  }\textbf{\textit{nə  məndəze  }}\textit{awak  anga  pəra.}
\end{itemize}

\textit{ká-ndaz    }\textbf{\textit{nə  mɪ-ndɪʒ-ɛ    }}\textit{awak  aŋga  pəra}



2S.IFV-kill(pierce)  with  \textsc{NOM}{}-kill(pierce)-CL  goat  \textsc{POSS}  idol



‘You kill a goat for the idols by piercing it (you don’t cut its throat).’ (lit. you kill with killing a goat that belongs to an idol)


\section{Dependent verb forms}
\hypertarget{RefHeading1212481525720847}{}
A dependent verb form is formed by prefixing \textit{am-} to the verb stem, palatalisation, and the suffix \textit{{}-ɛ} (or \textit{{}-ijɛ} for verb roots of one syllable). Historically, this construction may involve the nominalised form (see \sectref{sec:7.6}) preceded by the preposition \textit{a} ‘to.’\footnote{Crosslinguistic studies reveal that locatives can give rise to Imperfectives (Comrie, 1976: 103; Bybee et al., 1994: 142; Heine and Kuteva, 2002: 99). } In any case it acts as a single unit now. \tabref{tab:69}. shows examples of the dependent verb form for stems of each underlying prosody. The table gives the underlying form, the third person singular form, the nominalised form, and the dependent form. 

\begin{tabular}{lllll}
\lsptoprule

\textbf{Underlying form} & \textbf{Gloss} & \textbf{3S form} & \textbf{Nominalised form} & \textbf{Dependent form}\\
/h-m -j/ & ‘run’ & \textit{a-həm-aj} & \textit{mɪ-hɪm-ɛ} & \textit{amɪ-hɪm-ɛ}\\
/ d -j \textsuperscript{e} / & ‘prepare’ & \textit{ɛ{}-d-ɛ} & \textit{mɪ-d-ijɛ} & \textit{amɪ-d-ijɛ}\textit{  }\\
/sk\textsuperscript{w}m/ & ‘buy’/’sell’ & \textit{a-sk}\textit{\textsuperscript{w}}\textit{ɔm} & \textit{mɪ-sk}\textit{\textsuperscript{w}}\textit{ø}\textit{m- ɛ} & \textit{amɪ-sk}\textit{\textsuperscript{w}}\textit{ø}\textit{m- ɛ}\footnotemark{}\\
\lspbottomrule
\end{tabular}
\footnotetext{ Note that the labialised consonant /k\textsuperscript{w}/ keeps its labialisation even when the word is palatalised (\sectref{sec:111}). }

\begin{itemize}
\item \begin{styleTabletitle}
Dependent verb forms
\end{styleTabletitle}\end{itemize}

There are no subject inflections on the dependent verb form; the subject is determined either by the subject of the matrix clause (a gap for subject is marked as Ø in the examples) or (rarely) a subject noun phrase within the dependent clause. The dependent form of the verb may receive object suffixes and extensions (ex. 609, 610, 929, 912, 908).  

The dependent verb form is used when clauses that carry an imperfective or unfinished idea are embedded in other constructions. The clause structure is illustrated in \figref{fig:14}..

\begin{tabular}{lllll}
\lsptoprule

(subject  noun phrase) & \textbf{Dependent verb plus extensions expressing event} & (direct object noun phrase) & (oblique   adpositional phrase) & (adverb)\\
\lspbottomrule
\end{tabular}

\begin{itemize}
\item \begin{styleFiguretitle}
Constituent order in dependent clauses
\end{styleFiguretitle}\end{itemize}

The types of clauses that employ dependent verb forms are: 

\begin{itemize}
\item \textbf{Relative clauses}  (\sectref{sec:44})
\item \textbf{Adverbial clauses}  (\sectref{sec:13.2})
\item \textbf{Complement clauses} (\sectref{sec:13.1})
\end{itemize}

The relative clause is a noun phrase modifier (ex. 609 - 614). In the examples in this section, the dependent verb is bolded and the dependent clause is marked with square brackets.


Disobedient Girl S. 38



\ea
War  dalay  ga  ngendəye  [\textbf{amazata  aka  ala}\textbf{  }avəya  nengehe  ana  məze  ahay  na.]
\z

war  dalaj   ga  ŋgɛndijɛ   [Ø\textbf{   ama-z=ata     =aka  =ala}



child  girl  ADJ  DEM      DEP-carry=3P.IO   =on  =to



\textit{avija        nɛ}\textit{ŋ}\textit{gɛhɛ     ana     mɪʒɛ     =ahaj     na}]



suffering    DEM    DAT     person    =Pl          PSP



‘this female child that had brought (lit. to bring) this suffering to the people.’ 


\ea
Tasan  oko  ana  hay  [ata  \textbf{aməgəye}  \textbf{na  va.} ]
\z

ta-s-aŋ    ɔk\textsuperscript{w}ɔ  ana  haj   [=ata     \textbf{amɪ-g-ijɛ}   \textbf{na   =va} ]



3P-cut=3S.DO  fire  DAT  house    =3P.POSS  DEP-do-CL  3S.DO  \textsc{PRF}



‘They (the attackers) set fire to the house that the others had built (lit. their house to do).’  


Adverbial clauses in Moloko are subordinate temporal clauses that are embedded in the main clause as the first (ex. 611) or last (ex. 612) element. 


Cicada S. 16


\ea
{}[A\textbf{məhaya}  həmbo  na],  anday  asakala  wəsekeke.
\z

{}[Ø  \textbf{amə-h  =aja}  hʊmbɔ  na]  a-ndaj    a-sak    =ala  wuʃɛkɛkɛ



DEP-grind  =\textsc{PLU}  flour  PSP  3S-PRG    3S-multiply  =to  \textsc{ID}multiply



‘While [she] was grinding the flour (lit. to grind flour), [the millet] was multiplying.’ 


\ea
Kəlen  albaya  ahay  tolo  [\textbf{amazala}  ngəvəray  na. ]
\z

kɪlɛŋ  albaja    =ahaj  tɔ-lɔ  [Ø\textbf{   ama-z=ala}  ŋgəvəraj    na ]



then    young men  =Pl  3P-go     DEP-take=to  spp of tree  PSP



‘Then the young men went to try to bring back the tree [to the chief].’


The complement clause can function as the subject (ex. 613) or the direct object (ex. 614) of the matrix verb.

\ea
Asaŋ  [amadata  aka  va  azan. ]
\z

a-s    =aŋ     [Ø   ama-d=ata =aka     =va   azaŋ ]



3S-want  =3SD\={ }    DEP-prepare=3P.IO =on     =\textsc{PRF}  temptation



‘He wanted to tempt them.’ (lit. to prepare a temptation for them [is] pleasing to him)


\ea
Məkəɗ  va  azla  tazlan  a  ləme  [\textbf{aməzləge}  va. ]   
\z

mə-kəɗ  va   aɮa  ta-ɮ  =aŋ     [ alɪmɛ  \textbf{   amɪ-ɮɪg-ɛ}  va ]   



INF-kill  body   now  3P-begin =3S.IO   1\textsc{Pex} POSS  DEP-plant-CL  body  



‘The combat, we started to do it.’ (lit. planting body now, they started us planting bodies)  


\chapter[Verb phrase]{Verb phrase}
\hypertarget{RefHeading1212501525720847}{}
The verb phrase is the last of four chapters that concern the Moloko verb. Chapter 6 explored the structural features of the verb root and stem. Chapter 7 discussed what we have called the verb complex, which is a phonological unit consisting of the verb stem plus the pronominal affixes and enclitics, aspect/mood. markings, and verbal extensions. These components are closely phonologically bound even though they may comprise from one to three phonological words. The chapter also covered derived forms. Chapter 9 described verb types and transitivity. Moloko has a flexible valence system which allows varisations in the transitivity of a given verb with no morphological marking. This chapter concerns the structure and functions of the verb phrase. \sectref{sec:8.1} describes the constituents of the verb phrase and their order. \sectref{sec:8.2} shows auxiliary verb constructions where two verbs form a syntactic unit. 

\section{Verb phrase constituents}
\hypertarget{RefHeading1212521525720847}{}
The verb phrase in Moloko\footnote{This chapter is adapted from Friesen and Mamalis, 2008.} is centred around the verb complex (bolded in \figref{fig:15}., cf. Chapter 7).  Other elements are all optional and occur in the order diagrammed in \figref{fig:15}.. 

\begin{tabular}{l}
\lsptoprule

(Auxiliary)\textbf{Verb complex}  (noun phrase  (Adpositional phrases)    (Adverb)   (Ideophone or

        or ‘body-part’)                negative)             \\
\lspbottomrule
\end{tabular}

\begin{itemize}
\item \begin{styleFiguretitle}
Moloko verb phrase constituents
\end{styleFiguretitle}\end{itemize}

The auxiliary verbs include the progressive (see \sectref{sec:62}), the verb \textit{l}\textit{ɔ} ‘go’ when used as an auxiliary (see \sectref{sec:63}), and the verb stem or ideophone in its construction (see \sectref{sec:64}). 

Direct objects follow immediately after the verb complex and are expressed as noun phrases (underlined in ex. 615 and 628) or ‘body-part’ incorporated nouns (underlined in ex. 616; see \sectref{sec:68.} Prepositional phrases (underlined in ex. 616, 617, 618, 627) and then adverbs (ex. 618; see Chapter 3.5) or ideophone (bolded in ex. 615; see Chapter 3.6) follow after the direct object. The verb phrase is delimited by square brackets in the examples below.


\ea
Həmbo  ga  [anday  asak  ele  ahan  \textbf{wəsekek}\textbf{e.} ]
\z

hʊmbɔ  ga  [a-ndaj    a-sak    ɛlɛ  =ahaŋ    \textbf{wuʃɛkɛkɛ} ]



flour  ADJ  3S-PRG    3S-multiply  thing  =3S.POSS  \textsc{ID}multiply



‘The flour was multiplying all by itself (lit. its things), sound of multiplying.’


\ea
{}[Tandalay  talala  təzləgə   va  ana  Məloko  ahay.]
\z

{}[ta-nd=alaj    ta-l =ala  tə-ɮəg-ə   va  ana  Mʊlɔk\textsuperscript{w}ɔ    =ahaj ]



3P-PRG=away   3P-go =to  3P-throw body  DAT  Moloko    =Pl



‘They were coming and fighting with the Molokos.’ (lit. they were coming they threw body to Molokos)


\ea
{}[Enjé  kə  delmete  aka  a  slam  enen.]  
\z

{}[ɛ-ndʒ{}-ɛ    kə  dɛlmɛtɛ    aka \textbf{  }a  ɬam  ɛnɛŋ ]  



3S-leave-CL    on  neighbor    on  at  place  another



‘He left to go to his neighbor at some other place.’  


\ea
{}[Názaɗ  a  dəray  ava  \textbf{sawa}\textbf{n.}]
\z

{}[ná-zaɗ    a  dəraj  ava   \textbf{sawaŋ} ]



1S.IFV-carry    in  head  in  without help



‘I can carry it (on my head) myself!’


\begin{itemize}
\item \begin{styleExampleteference}
\textup{[}Nəvəlan  yam  ana  Mana  \textbf{zayəhha}\textbf{. }\textbf{\textup{]}}
\end{styleExampleteference}
\end{itemize}
\begin{styleExampleteference}
\textup{[}nə-vəl=aŋ    jam  ana  Mana  \textbf{zajəx=xa }\textbf{\textup{]}}
\end{styleExampleteference}


1S.PFV-give=3S.IO  water  DAT  Mana  care=ADV



‘I gave water to Mana carefully.’


\begin{itemize}
\item 
{}[\textit{Azləgalay  }\textit{a  v}\textit{ə}\textit{lo}\textbf{\textit{  }}\textbf{\textit{zor.}}\textit{ }]
\end{itemize}

{}[\textit{à-ɮəg   =alaj    }\textit{a  v}\textit{ʊ}\textit{lɔ}\textbf{\textit{  }}\textbf{\textit{zɔr}}\textit{ }]



3S-throw =away    at  above  \textsc{ID}throwing



‘She threw [the pestle] up high, movement of throwing.’


\citet[69]{Radford1981} gives diagnostic criteria for determining whether a given string of words is a sentence constituent or not. Following these criteria, all of the above elements are part of the verb phrase as a constituent of the clause. The elements of the verb phrase behave distributionally as a single structural unit that does not permit intrusion of parenthetical elements internally, but rather only at the boundaries. For Moloko, the distribution of adverbs, emphatic pronouns, ideophones, the Perfect enclitic, and the manner of fronting all attest to the unity of the verb phrase as described above. Only the presupposition marker can intrude into the verb phrase, and only in a particular construction. Each of these factors is discussed below. 

Some temporal adverbs can occur first in the clause or last in the verb phrase (ex. 205 and 622), but not in the interior of the verb phrase. Likewise, emphatic pronouns occur first or last in the clause (emphatic pronouns are bolded in ex. 623). 

\ea
\textbf{Eg}\textbf{ə}\textbf{ne}  [nólo  a  kosoko  ava. ]
\z

\textbf{ɛgɪnɛ}  [nɔ-lɔ     a  kɔsɔk\textsuperscript{w}ɔ  ava ]



today  1S.IFV-go  in  market  in



Today I will go to the market. 


\ea
{}[Nólo  a  kosoko  ava  \textbf{eg}\textbf{ə}\textbf{ne.}]
\z

{}[nɔ-lɔ   a  kɔsɔk\textsuperscript{w}ɔ  ava  \textbf{ɛgɪnɛ}]



1S.IFV-go  in  market  in  today



I will go to the market today.


\ea
\textbf{Wa}  [amazaw  ala  ngəvəray  ana  ne  na ]  \textbf{wa}\textbf{y}?
\z

\textbf{wa}      [ama-z  =aw  =ala       ŋgəvəraj     ana     nɛ    na ]     \textbf{waj}



who   DEP-take=1S.IO  =to   spp. of tree     DAT     1S    PSP    who



‘Who can I find to bring me this tree.’


Ideophones have only three slots within the clause: First in clause (ex. 624),\footnote{Note that an ideophone that is first in the clause is sometimes delimited by \textit{na} (ex. 633). } first in verb phrase (ex. 625),\footnote{When the ideophone is first in the verb phrase it necessitates the nominalised form of the verb \citep{Section1164}.} last in verb phrase (ex. 626). The ideophones are bolded in the examples.


Snake S. 13


\ea
\textbf{Kal}\textbf{ə}\textbf{w}  [nazala   ezlere  əwla.]
\z

\textbf{kaluw}       [nà-z          =ala    ɛɮɛrɛ    =uwla ]



\textsc{ID}take quickly    1S.PFV-take  =to   spear    =1S.POSS 



‘I quickly took my spear.’



Cicada S. 15


\ea
Ndahan  [\textbf{g}\textbf{ə}\textbf{dok}   mədəye  gəzom.]  
\z

ndahaŋ  [\textbf{gʊdɔk}\textbf{\textsuperscript{w}}     mɪ-d-ijɛ       gʊzɔm ]  



3S        \textsc{ID} prepare wine  \textsc{NOM}{}-prepare-CL   wine



‘He gudok made wine.’



Snake S. 5


\ea
{}[Acar  a  hay  kəre  ava  \textbf{fo fo f}\textbf{o.}]
\z

{}[à-tsar              a  haj        kɪrɛ       ava  \textbf{fɔ fɔ fɔ} ]



3S.PFV-climb  in   house  beams  in          \textsc{ID}sound of snake



‘[The snake] climbed into the beams in the roof \textit{fof}\textit{o}\textit{fo}.’


The distribution and influence of the Perfect enclitic \textit{=va}  also attests to the unity of the post-verbal elements in the verb phrase. The Perfect enclitic \textit{=va}  (bolded in ex. 627 - 630), can either cliticise to the end of the verb complex (ex. 627) or the end of the entire verb phrase (ex. 628 - 630). The phonological influence of the Perfect extends across the entire verb phrase since its presence in either post-verbal or phrase-final position causes a neutralisation of the prosody on the verb stem (see \sectref{sec:58}).


Values S. 6


\ea
{}[Tahata  na\textbf{  va}  kə  deftere  aka.]
\z

{}[tà-h=ata    na  \textbf{=va}    kə  dɛftɛrɛ  aka ]



3P.PFV-tell=3P.IO  3S.DO  =\textsc{PRF}    on  book  on



‘They have already told them in the book.’ 



Disobedient Girl S. 34


\ea
Waya  ndana  Hərmbəlom  [ázata  aka  barka  ahan \textbf{ va}\textbf{.}]
\z

waja  ndana  Hʊrmbʊlɔm   [á-z  =ata   =aka   barka    =ahaŋ     \textbf{=va} ]



because    DEM   God             3S.IFV-take=3P.IO=on   blessing    =3S.POSS  =\textsc{PRF}



‘Because of the avbove-mentioned, God had taken back his blessing from them.’


\ea
Baba  ango  [avəlata  nok\textbf{  va}  a  ahar  atəta  ava.]
\z

baba  =aŋg\textsuperscript{w}ɔ  [a-vəl=ata    nɔk\textsuperscript{w}  \textbf{=va}  a  ahar  =atəta  ava ]



father  2SP\={ }OSS  3S-give=3P.IO    2S  =\textsc{PRF}  in  hand  3P.POSS  in



‘Your father gave you into their hands [to be a wife for one of them].’


\ea
Nde  hor  na,  [asərkala  afa  təta  \textbf{va} ]\textbf{       } 
\z

ndɛ    h\textsuperscript{w}ɔr  na      [a-sərk =ala  afa           təta     =\textbf{va} ]\textbf{       } 



so    woman  PSP    3S-HAB =to   at.house of  3P.POSS   =\textsc{PRF}     



 ‘Now, the woman had a habit at their house.’ 


Only certain elements in the verb phrase can be fronted in the clause and marked with the presupposition marker \textit{na}. The fact that some elements cannot be fronted indicates that they are closely bound to the verb phrase structure. These elements include the ‘body-part’ incorporated noun (cf. \sectref{sec:9.3}), adverbs which are bound to the negative, and the negative itself (see \sectref{sec:71}).\textbf{\textit{  }}The elements that may be fronted are underlined in ex. 631{}-633 and include direct object (ex. 631), indirect object (ex. 633), oblique (ex. 631), derived adverb (bolded in ex. 632), and ideophone (ex. 633). 


Values S. 13


\ea
A  məsəyon  ava  na  ele  ahay  aməwəsle  na,  [tége  bay.]
\z

a  mɪsijɔŋ   ava   na  ɛlɛ   =ahaj   amɪ-wuɬ{}-ɛ     na     [tɛ-g-ɛ     baj ]



to  mission  in  PSP  thing  =Pl        DEP-forbid-CL  PSP    3P.IFV-do-CL  \textsc{NEG}



‘In the mission, these things that they have forbidden, they don’t do.’



Values S. 39


\ea
Pepenna  na\textbf{,}  [takaɗ  sla.]
\z

pɛpɛŋ=ŋa     na    [tà-kaɗ     ɬa ]



long ago=ADV  PSP    3P.PFV-kill  cow  



‘Long ago, they killed cows.’



Values S. 3


\ea
Səwat  na\textbf{, }  təta  a  məsəyon  na  ava  nəndəye  na,  [pester  áhata ],   “Ey, ele  nehe  na,  kógom  bay!”   
\z

suwat   na   təta   a   mɪsijɔŋ   na  ava     nɪndijɛ  na       



\textsc{ID}disperse  PSP  3P      in  mission  PSP    in  DEM  PSP  



‘As they disperse, those in that mission,’



{}[\textit{pɛʃtɛr  á-h    =ata }]   \textit{ɛ}\textit{j    }\textit{ɛ}\textit{l}\textit{ɛ}\textit{      n}\textit{ɛ}\textit{h}\textit{ɛ}\textit{   na   k}\textit{ɔ}\textit{\'{ }-g}\textit{\textsuperscript{w}}\textit{{}-}\textit{ɔ}\textit{m    baj}



pastor  3S.IFV-tell  =3P.IO   hey    thing  DEM  PSP  2.IFV-do-2P    \textsc{NEG}



‘the Pastor told them, “Hey! These things, don’t do them!”’


Note that in the focus construction (cf. \sectref{sec:12.5}), the presupposition marker\textit{ na }can appear to break up parts of the verb phrase. However the structural unity of the verb phrase unit is not challenged since \textit{na }can occur only once within the verb phrase in this construction and only immediately before the final focussed element. In each of ex. 634{}-635636, the penultimate placing of \textit{na} functions to make the final element of the verb phrase more prominent. In each example, only the verb phrase containing \textit{na }is delimited by square brackets and the part delimited by\textit{ na}  is underlined. In ex. 634,\textit{ na }occurs in the adverbial clause between the direct object (\textit{haj} ‘millet’) and the verb phrase-final adverb (\textit{gam} ‘much’).  In ex. 635, \textit{na }occurs in the matrix clause between the verb complex and the verb phrase-final prepositional phrase (\textit{ka  mahaj =aŋg}\textit{\textsuperscript{w}}\textit{ɔ aka} ‘by your door’). Ex. 635636 shows two verb phrases which both contain \textit{na}. In each case,\textit{ na} occurs immediately before the final element of the verb phrase. 


 Disobedient Girl S. 17 


\ea
Azla  \textbf{na}\textbf{,}  hor  \textbf{na},  asərkala  afa  təta  va  \textbf{na},  [aməhaya  háy  \textbf{na}  gam.]
\z

aɮa  \textbf{na}  h\textsuperscript{w}ɔr  \textbf{na}  [à-sərk         =ala   afa            təta  =va  \textbf{na}]



now  PSP  woman  PSP    3S.PFV-HAB=to    at place of  3P       =\textsc{PRF}  PSP  



‘Now, that woman, she was in the habit at their place’



{}[\textit{amə-h=aja          haj       }\textbf{\textit{na}}\textit{    gam }]



DEP-grind=\textsc{PLU}   millet  PSP   a lot



‘grinding a lot of millet.’



Cicada S. 7 


\ea
Mama  ngəvəray  ava  a  ləhe  \textbf{na},  malan  ga  \textbf{na},  [agasaka  \textbf{na}\textbf{,}  ka  mahay  ango  aka ]  aməmbese.
\z

mama   ŋgəvəraj       ava       a   lɪhɛ        \textbf{na}    malaŋ   ga  \textbf{na}



mother   spp. of tree  \textsc{EXT}   at   bush    PSP  large     ADJ  PSP



‘There is a mother-tree in the bush,  a big one,’



{}[\textit{à-gas  =aka    }\textbf{\textit{na}}\textit{ }\textit{    ka    mahaj    =aŋg}\textit{\textsuperscript{w}}\textit{ɔ    aka} ]\textit{  }\textit{àm}\textit{ɪ}\textit{{}-mbɛ}\textit{ʃ}\textit{{}-ɛ} 



3S.PFV-get=on   PSP   on     door       =2S.POSS      on   DEP-rest-CL



‘[and] it would please you to have that tree at your door, [so that you could] rest [under it].’



Values S. 29 


\ea
Hərmbəlom  \textbf{na},  amaɗaslava  ala  məze  \textbf{na},  ndahan  ese  \textbf{na},  [kagas  ma  Hərmbəlom  na,  asabay]  \textbf{na},  [káagas  \textbf{na}  anga  way?]
\z

Hʊrmbʊlɔm   \textbf{na}  ama-ɗaɬ    =ava  =ala  mɪʒɛ  \textbf{na}   ndahaŋ  ɛʃɛ  \textbf{na}



God      PSP    DEP-multiply  =in  =to   person   PSP  3S     again    PSP  



{}[\textit{ka-gas    ma   H}\textit{ʊ}\textit{rmb}\textit{ʊ}\textit{l}\textit{ɔ}\textit{m  }\textbf{\textit{na}}\textit{  asa-baj }]\textit{         }\textbf{\textit{na}}\textit{ }



2S-catch   word     God          PSP  again-\textsc{NEG}  PSP



{}[\textit{káá-gas           }\textbf{\textit{na}}\textit{ }\textit{     anga     waj }]



2S.POT-catch    PSP   \textsc{POSS}    who



‘And if you will never accept the word of God, the one that multiplied the people, whose word will you accept then?’ (lit. God, the one that multiplied the people, he again, you catch God’s word no longer, you will catch it [word] of whom?)


\section{Auxiliary verb constructions}
\hypertarget{RefHeading1212541525720847}{}
In an auxiliary verb construction in Moloko, two verbs (or a verb plus an ideophone) form a syntactic unit and, consequently, have the same subject. The second verb is the main verb in the construction. Together the two verbs comprise the head of just one clause, with only one set of core participants and obliques that semantically are related to the second (main) verb. 

This section presents three auxiliary verb constructions. In the first two constructions, both main and auxiliary verbs are inflected.  These constructions express progressive aspect (see \sectref{sec:62}) and movement from one place to another (see \sectref{sec:63}).  The third construction consists of a verb stem or ideophone plus the main verb which is in the nominalised form (see \sectref{sec:64}). We consider this third construction to be an auxiliary construction even though the verb stem/ideophone does not carry much of the inflectional information normally associated with auxiliaries (stems and ideophones carry neither subject and object agreement nor aspect and mode marking).\footnote{These criteria for verb auxiliaries are given by Payne, 1997: 84.} However, the verb stem/ideophone construction demonstrates the same structure as the progressive and movement auxiliary constructions and the stem/ideophone functions as an auxiliary in that it adds grammatical information to the main verb.  

\subsection{  Progressive auxiliary}
\hypertarget{RefHeading1212561525720847}{}
Friesen and \citet{Mamalis2008} found that the progressive expresses the idea of an action in progress, an event that doesn’t take place all at once.\footnote{Note that the verb \textit{ndaj}  can occur alone as the main verb of a clause \citep{Section1165}. When it does, the complement expresses the location of the subject. For example, \textit{Hawa a-ndaj a mɔg}\textit{\textsuperscript{w}}\textit{ɔm} ‘Hawa is at home’.}  It is formed with \textit{ndaj} ‘to be’ (see \sectref{sec:65}) plus the main verb.  The auxiliary \textit{ndaj} occurs as the first of two verbs in a verb phrase. The main verb takes all subject affixes and also any inflections or obliques. In the examples, the progressive is bolded and the verb phrase is delimited by square brackets.


\ea
Mala  [\textbf{anday  }ége  slərele. ]
\z

Mala[\textbf{a-ndaj  }  ɛ{}-g-ɛ\textbf{  }  ɬɪrɛlɛ ]



Mala   3S -PRG    3S.IFV-do  work 



‘Mala is working (in the process of doing work).’


\begin{itemize}
\item 
\textit{Mana}\textbf{\textit{  }}[\textbf{\textit{anday}}\textit{  ólo  a  kosoko  ava. }]
\end{itemize}

\textit{Mana}\textbf{\textit{  }}[\textbf{\textit{a-ndaj}}\textit{    ɔ{}-lɔ    a  kɔsɔk}\textit{\textsuperscript{w}}\textit{ɔ  ava }]



Mana  3S-PRG    3S+I\textsc{PFV}{}-go  in  market  in



‘Mana is going to the market.’  (lit. he is currently at…going to the market)


\begin{itemize}
\item 
\textit{Apazan  nanjakay  nok,  }[\textbf{\textit{kanday  }}\textit{kəhaya  hay. }]
\end{itemize}

\textit{apazaŋ  nà-nzak-aj  nɔk}\textit{\textsuperscript{w}}\textit{,   }[\textbf{\textit{ka-ndaj  }}\textit{  kə-h=aja    haj }]



yesterday  1S.PFV-find{}-CL  2S  2S-PRG    2S.PFV-grind=\textsc{PLU}  millet



‘Yesterday when I found you, you were grinding millet.’  


Both of the verbs are marked for subject. In plural forms that take subject prefix and suffix (1P and 2P, ex. 640 and 641), \textit{ndaj} takes subject prefixes only.\footnote{Some Moloko people say that the plural form is \textit{nɔ-ndɔmɔj}, but most people use the reduced form. } 

\ea
{}[\textbf{Nondoy  }nombosom  va. ]
\z

{}[\textbf{nɔ-ndɔj}  nɔ-mbɔs-ɔm   va ]



1P-PRG    1P-rest-1\textsc{Pex}  body



‘We are resting.’ 


\ea
{}[\textbf{Nondoy}  nódorom\textbf{  }amsoko. ]
\z

{}[\textbf{nɔ-ndɔj}     nɔ-dɔr-ɔm\textbf{  }amsɔk\textsuperscript{w}ɔ ]



1P -PRG    1P-plant-1\textsc{Pex}  dry season millet



‘We (exclusive) are planting dry season millet.’


The progressive auxiliary (see \sectref{sec:62}) does not co-occur with the perfect enclitic (see \sectref{sec:58}), nor does the iterative reduplicative construction (see \sectref{sec:55}) combine with the progressive auxiliary.

In discourse, progressive aspect is used to mark an event that is in progress in a Moloko text. It is not necessarily in the background, but indicates durative or ongoing dynamic events. In the Cicada setting, sentences S.3-5, there is a progressive in a tail-head link (see \sectref{sec:81}) showing what the young men were doing when they found the tree (ex. 642). 


Cicada  S.3-5


\ea
Albaya  ahay  aba.  \textbf{T}\textbf{ánday}\textbf{  }tətalay  a  ləhe.  \textbf{T}\textbf{ánday}  tətalay  a  ləhe  na,  tolo  tənjakay  ngəvəray  malan  ga  a  ləhe.
\z

albaja  =ahaj    aba



young man    =Pl     \textsc{EXT}



‘There were some young men.’



\textbf{\textit{t}}\textbf{\textit{á-}}\textbf{\textit{ndaj}}\textbf{\textit{     }}\textit{tə-tal-aj}\textit{    a   lɪhɛ  }



3P.IFV-PRG\textbf{     }3P.IFV-walk{}-CL  at    bush



‘They were walking in the bush.’



\textbf{\textit{t}}\textbf{\textit{á-}}\textbf{\textit{ndaj}}\textit{    }\textit{tə-}\textit{tal-aj}\textit{        a    lɪhɛ      na      }



3P.IFV-PRG   3P.IFV-walk{}-CL  at   bush   PSP    



‘[as] they were walking in the bush,’



\textit{tə-lɔ   tə-nzak-aj  ŋgəvəraj      malaŋ   ga    a     lɪhɛ  }



3P.PFV-go  3P.PFV-find{}-CL    spp. of tree      large   ADJ    at    bush



‘they found a large tree (a particular species) in the bush.’


Also, progressives are used in expository texts that give the ongoing state of the world and show reasons for the way things are. Ex. 643 from the Disobedient Girl story shows the entire reported speech when the husband explains to his wife the way things work for the Moloko. For most of the explanation, the verbs are Imperfective (see \sectref{sec:52}). However, the reason that the millet multiplied – namely, that God used to multiply millet for the Moloko – is given in the final line of his speech. The verb form for the reason is progressive (bolded in the example). Here, the progressive is marking an important ongoing event. 


Disobedient Girl S. 13


\ea
Awəy  hor  golo,  afa  ləme  na,  mənjəye  aləme  na,  kəyga  ehe:  asa  asok  aməhaya  na,  kázaɗ  war  elé  a  háy  bəlen.  War  elé  a  háy  bəlen  ga  nəndəye  nok  amezəɗe  na,  káhaya  na  kə  ver  aka.  Ánjaloko  de  pew.  Ádaloko  ha  ámbaɗ  ese.  Waya  a  məhaya  ahan  ava  na,  Hərmbəlom  \textbf{anday}   ásakaləme  na  aka.
\z

awij



 ‘He said’



\textit{h}\textsuperscript{w}\textit{ɔr}\textit{      }\textit{g}\textsuperscript{w}\textit{ɔlɔ,}\textit{   }\textit{afa        lɪmɛ   na,}\textit{   mɪ-n}\textit{ʒ}\textit{{}-ijɛ     =alɪmɛ         na,}\textit{   kijga     ɛhɛ}



woman   VOC   at place   1\textsc{Pex}   PSP   \textsc{NOM}{}-sit-CL =2\textsc{Pex}.POSS  PSP   like this  here



‘My dear wife, here at our (exclusive) place, it is like this:’



\textit{asa   à-s=ɔk}\textit{\textsuperscript{w}}\textit{      aməh=aja           na,  }



if   3S.IFV-please=2S.IO   DEP-grind=\textsc{PLU}   PSP   



\textit{ká-zaɗ         war     ɛlɛ      a  haj       bɪlɛŋ.}



2S.IFV-take    child    eye  GEN  millet  one



‘If you want to grind, you take only one grain.’  



\textit{war     ɛlɛ     a  haj       bɪlɛŋ   ga    nɪndijɛ  nɔk}\textit{\textsuperscript{w}}\textit{   amɛ-}\textit{ʒ}\textit{ɪɗ{}-ɛ  na,     }



child    eye  GEN  millet  one  ADJ  DEM    2S    DEP-take-CL  PSP     



\textit{ká-h=aja                 na       kə     vɛr         aka}



2S.IFV-grind=\textsc{PLU} 3S.DO on grinding stone on



‘That one grain that you have taken, grind it on the grinding stone,’



\textit{á-nz=alɔk}\textit{\textsuperscript{w}}\textit{ɔ                 dɛ      pɛw }



3S.IFV-suffice=1\textsc{Pin}.IO  just   enough



‘It will suffice for all of us just enough.’



\textit{á-d=alɔk}\textit{\textsuperscript{w}}\textit{ɔ                 ha      á-mbaɗ      ɛ}\textit{ʃ}\textit{ɛ}



3S.IFV-prepare=1\textsc{Pin}.IO       until   3S.IFV-left over  again



‘It will make food for all of us, until there is some left over.’



\textit{waja  a    mə-h=aja    =ahaŋ       ava    na,   }



because  at   \textsc{NOM}{}-grind=\textsc{PLU}   =3S.POSS   in   PSP    



‘Because, all the while you were doing the grinding\textit{,}



\textit{H}\textit{ʊ}\textit{rmb}\textit{ʊ}\textit{l}\textit{ɔ}\textit{m   }\textbf{\textit{a-ndaj}}\textit{         á-sak    =al}\textit{ɪ}\textit{m}\textit{ɛ}\textit{          na      aka}



God            3S-PROG   3S.IFV-multiply=2\textsc{Pex}.IO   3S.DO   on



‘God was multiplying it for us.’


Progressives are also found in the peak section of a narrative where they function to slow down the events and draw the reader into the action. Ex. 644 shows the entire peak section of the Disobedient Girl (shown in its entirety in \sectref{sec:1.5}). In the story, there is a battle between the disobedient girl and the millet itself, which has a supernatural ability to expand. The actions of the millet (which eventually triumph over the woman) and the woman are shown in progressive (S. 23, S. 25)


Disobedient Girl  S.20


\ea
Jo  madala  háy  na    gam.  Ndahan  bah  məbehe  háy  ahan  amadala  na  kə  ver  aka  azla.  Njəw  njəw  njəw \textbf{ }aməhaya  azla.  Həmbo  na  ɗəw  \textbf{anday}  \textbf{ásak  ásak  ásak.  }Ndahan  na,  ndahan  aka  njəw  njəw  njəw.  \textbf{Anday}  \textbf{ahaya}  nə  məzere  ləmes  ga.  Alala  na,  ver  na  árah  mbaf,  nə  həmbo  na,  ɗək  məɗəkaka  alay  ana  hor  na,  nata  ndahan  dəɓəsolək  məmətava  alay  a  hoɗ  a  hay  na  ava.  
\z

dzɔ             ma-d    =ala    haj        na      gam  



\textsc{ID} take    \textsc{NOM}prepare  =to    millet   PSP   a lot



‘She prepared lots of millet.’



S. 21



\textit{ndahaŋ  }\textit{bax       m}\textit{ɪ}\textit{{}-b}\textit{ɛ}\textit{h-}\textit{ɛ}\textit{    }\textit{haj       =ahaŋ      }



3S          \textsc{ID}pour   \textsc{NOM}{}-pour-CL  millet  =3S.POSS  



\textit{ama-d  =ala     na       kə  v}\textit{ɛ}\textit{r       aka  a}\textit{ɮ}\textit{a}



DEPprepare  =to   3S.DO   on   stone  on      now



‘She poured the millet on the grinding stone.’ (lit. she, pouring her millet to put it on the grinding stone)



S. 22



\textit{nzuw  nzuw  nzuw   }\textit{        amə-h    =aja      a}\textit{ɮ}\textit{a}



\textsc{ID} grind                            DEP-grind  =\textsc{PLU}    now



‘\textit{Nzu nzu nzu} [she] is grinding.’ (lit. nzu nzu nzu to grind now)



S. 23



\textit{h}\textit{ʊ}\textit{mb}\textit{ɔ  na}\textit{       }\textit{ɗ}\textit{uw  }\textbf{\textit{a-ndaj}}\textit{      }\textbf{\textit{á-sak  á-sak  á-sak}}



flour  PSP  also  3S-PRG     3S.IFV-multiply IT



‘The flour, it was multiplying multiplying.



S.24



\textit{ndahaŋ            na    ndahaŋ    aka        nzuw  nzuw  nzuw           }



3S                    PSP    3S            \textsc{EXT}+on       \textsc{ID}{}-grind



‘And she, she is grinding some more.’



S. 25



\textbf{\textit{à-ndaj}}\textit{                  }\textbf{\textit{à-h=aja   }}\textit{               nə       m}\textit{ɪ}\textit{{}-}\textit{ʒ}\textit{ɛr-ɛ          l}\textit{ɪ}\textit{mɛ}\textit{ʃ}\textit{    ga}



3S.PFV-PRG  3S.IFV-grind=\textsc{PLU}  with   \textsc{NOM}{}-do\_well-CL  song     ADJ



‘She is grinding while singing well.’



S. 26



\textit{a-l=ala      na  v}\textit{ɛ}\textit{r   na  }\textbf{\textit{á-rəx}}\textit{             mbaf,    nə          h}\textit{ʊ}\textit{mb}\textit{ɔ}\textit{   na}



3S-go=to   PSP  room     PSP  3S.IFV-fill  up to the roof  with  flour  PSP



‘After a while, the room, it fills up to the roof with the flour.’



\textit{ɗ}\textit{ək         mə-}\textit{ɗ}\textit{ək=aka    =alaj  ana      h}\textsuperscript{w}\textit{ɔ}\textit{r  na}



\textsc{ID}{}-stuff    \textsc{NOM}{}-stuff=on  =away     DAT   woman  PSP



‘[It] stuffed [the room] [so there was no place] for the woman [to even breathe].’ (lit. \textit{dik} stuffing for the woman)



\textit{nata   ndahaŋ}\textit{  d}\textit{ʊɓʊ}\textit{s}\textit{ɔ}\textit{l}\textit{ʊ}\textit{k         mə-mət=ava=alaj   }\textit{a  h}\textsuperscript{w}\textit{ɔɗ}\textit{  a  haj  na  ava}



and then  3S          \textsc{ID}{}-collapse/die  \textsc{NOM}die=in =away   at  stomach GEN   house   PSP   in



‘And she collapsed, dying inside the house.’


\subsection{  Movement auxiliary}
\hypertarget{RefHeading1212581525720847}{}
The verb \textit{l}\textit{ɔ} ‘go’ is often found together with a second verb within the same verb phrase to express the idea of movement from one place to another, in order to accomplish the event expressed by the main verb (Friesen and Mamalis, 2008). Both verbs are conjugated, but only the second takes extensions or other verb phrase elements. In the examples, the verb \textit{lɔ} is bolded and the verb phrase is delimited by square brackets. 


Cicada S. 5



\ea
{}[\textbf{Təlo  }tənjakay  ngəvəray  malan  ga  a  ləhe.]
\z

{}[\textbf{tə-lɔ}          tə-nzak-aj        ŋgəvəraj      malaŋ   ga      a    lɪhɛ ]



3P.PFV-go   3P.PFV-find{}-CL   spp. of tree    large   ADJ  to   bush



‘They went and found a large tree (a particular species) in the bush.’



Values S. 18


\ea
{}[\textbf{Olo}  aban  ana  baba  ahan.]
\z

{}[\textbf{ɔ{}-lɔ}   a-b=aŋ     ana   baba   =ahaŋ ]



3S-go  3S-hit=3S.IO  DAT  father  =3S.POSS



‘He goes and hits his father.’



Values S. 19


\ea
{}[\textbf{Olo}  apaɗay  məze  nə  madan.]
\z

{}[\textbf{ɔ{}-lɔ}   a-paɗ{}-aj     mɪʒɛ   nə   madaŋ ]



3S-go  3S-crunch{}-CL  person  with  magic  



‘He goes and eats someone with sorcery.’


\ea
{}[\textbf{Lohom}  komənjɔrom  na  ala  gəvah  na.]   
\z

{}[\textbf{lɔh-ɔm}    kɔ-mʊnzɔr-ɔm   na     ala      gəvax  na ]   



\textsc{IMP}{}-2P   2P-see-2P  3S.DO  to  field  PSP



Go and you will see that field. 


\subsection{  Stem plus ideophone auxiliary}
\hypertarget{RefHeading1212601525720847}{}
Friesen and \citet{Mamalis2008} discovered that pivotal events at the high points in a narrative may be coded with a particular verb phrase construction in which an ideophone or the uninflected stem form of a verb is followed by the main verb in its nominalised form (ex. 649 - 651, see \sectref{sec:60}).  In the stem plus verb construction, the stem and main verb are normally formed from the same verb root. Note that it is the stem that is used in the construction; ex. 651 shows the \textit{{}-aj} suffix. Neither the main verb nor the auxiliary is inflected for subject, and the clause often has no noun phrase to indicate subject (ex. 649, 653, 652). When a subject noun phrase is present, it can only be a full free pronoun (ex. 650 and 651). The main verb can have direct and indirect object pronominals and other extensions (ex. 650 and 651). In the following examples, the verb phrase is delimited by square brackets and the verb stem or ideophone is bolded. 


\ea
{}[\textbf{Bah}  məbehe  kə  ver  aka  azla.]
\z

{}[\textbf{bax}    mɪ-bɛh-ɛ    kə  vɛr  aka  aɮa]



pour   \textsc{NOM}{}-pour-CL  on  stone  on  now  



‘She poured [the grains of millet] on the grinding stone.’\textit{ }(lit. pour, pouring on the grinding stone now)


\ea
Ndahan  [\textbf{ngah}  mangəhata  aka  va  yam  a  ver  ahan  ava.]
\z

ndahaŋ  [\textbf{ŋgax}  ma-ŋgəh=ata=aka=va\textbf{  }  jam  a  vɛr  =ahaŋ    ava ]



3S    hide  \textsc{NOM}{}-hide=3P.IO =on=\textsc{PRF}  water  in  room  =3S.POSS  in



‘He had hidden the water in his room.’


\ea
Ndahan  [\textbf{ngay}  mangaka  alay  pərgom  ahay.] 
\z

ndahaŋ  [\textbf{ŋg-aj}       ma-ŋg           =aka=alaj    pʊrg\textsuperscript{w}ɔm  =ahaj ] 



3S    make\_with\_grass{}-CL  \textsc{NOM}{}-make with grass =on=away    trap    =Pl



‘He made the traps out of grass.’



Disobedient Girl S. 12


\ea
Sen  ala  na  zar  ahan  na,  [\textbf{dək  }mədəkan  na  mənjəye  atəta. ] \textbf{  }
\z

ʃɛŋ    =ala     na   zar   =ahaŋ    na \textbf{  }



\textsc{ID}go     =to  PSP    man   =3S.POSS    PSP\textbf{\textit{    }}



{}[\textbf{\textit{dək    }}\textit{m}\textit{ə}\textit{{}-dək}\textit{=aŋ  na  m}\textit{ɪ{}-nʒ-ijɛ      =atəta }]



instruct  \textsc{NOM}{}-instruct=3P.IO   3S.DO   \textsc{NOM}{}-sit-CL      3P.POSS



‘Then, her husband, instructed her their ways (lit. their sitting).’ 


In the case that there is an ideophone auxiliary (ex. 653 - 656), the ideophone occurs in the same slot as the verb stem auxiliary. Note that the ideophones are from entirely different roots than the verb stems. 


Disobedient Girl S. 20


\ea
{}[\textbf{Jo  madala}  háy  na  gam.]  
\z

{}[\textbf{dzɔ    ma-d    =ala}     haj  na      gam ]  



\textsc{ID} take  \textsc{NOM}prepare  =to  millet       PSP  a lot



‘She prepared lots of millet.’



Disobedient Girl S. 28


\ea
{}[\textbf{Pok  }mapalay  mahay  na],  həmbo  [árah  na  a  hoɗ  a  hay  ava.].       
\z

{}[\textbf{pɔk     }ma-p    =alaj      mahaj  na ]        



\textsc{ID}open  \textsc{NOM}{}-open  =away   door     PSP       



\textit{h}\textit{ʊ}\textit{mbɔ  }[\textit{á-rax         na  a   h}\textsuperscript{w}\textit{ɔɗ      a  haj       ava} ]



flour  3S.IFV-fill  3S.DO  in       stomach  GEN   house  in



‘He opened the door and looked around; the flour filled the house.’


\ea
Ndahan  [\textbf{vəh}  məngwəlva  a  ɗəwer  ahan  ava.]
\z

ndahaŋ  [\textbf{vəh}    mə-ŋg\textsuperscript{w}ul =va    a  ɗuwɛr  =ahaŋ    ava ]



3S    \textsc{ID}return    \textsc{NOM}{}-return =\textsc{PRF}  in  sleep  =3S.POSS  in



‘He had already gone back to sleep.’


\ea
Nata  ndahan  [\textbf{pək}  mapata  aka  va  pərgom  ahay  na.]
\z

nata    ndahaŋ  [\textbf{pək}  ma-p    =ata=aka=va    pʊrg\textsuperscript{w}ɔm  =ahaj  na ]



also    3S  \textsc{ID}open  \textsc{NOM}{}-open=3P.IO=on=\textsc{PRF}  trap  =Pl  PSP



‘He opened the traps.’ 



Disobedient Girl S. 26


\ea
Nata  ndahan  [\textbf{d}\textbf{əɓə}\textbf{sol}\textbf{ə}\textbf{k  }məmətava  alay  a  hoɗ  a  hay  na  ava.]
\z

nata    ndahaŋ  [\textbf{dʊɓʊsɔlʊk      }   mə-mət=ava=alaj   a   h\textsuperscript{w}ɔɗ      a        haj       na     ava ]



and then  3S          \textsc{ID}{}-collapse/die  \textsc{NOM}{}-die=in =away   in   stomach  GEN  house   PSP    in



‘And she collapsed, dying inside the house.’



Disobedient Girl S. 31


\ea
{}[\textbf{Babək  }mələye  na.]
\z

{}[\textbf{babək       }mɪ-l-ijɛ        na ]



\textsc{ID}bury      \textsc{NOM-}bury-CL   3S.DO



‘She was buried.’ (lit. burying it)



Snake S. 18


\ea
Ne  [\textbf{d}\textbf{əy}  \textbf{day  }məkəɗe  na  aka.]
\z

nɛ  [\textbf{dij daj    }            mɪ-kɪɗ-ɛ\textbf{    }  na      =aka ]



1S    approximately     \textsc{NOM}{}-kill-CL   3S.DO   =on



‘I clubbed it to death.’ (lit. I approximately killing it on)


The stem or ideophone plus verb constructions mark significant events at the inciting moment and in the peak of a Moloko narrative. Ex. 652 is from the inciting moment of the Disobedient Girl story when the man instructs his wife. In the peak, the construction is seen when the woman prepares a lot of millet after having decided to disobey him (ex. 653), when she pours a lot of millet on the grinding stone (ex. 653), and when the millet suffocates her and she dies (ex. 657). In the dénouement there is another ideophone plus nominalised form construction when the husband opens the door and finds her (ex. 654). There are no other nominalised forms that fill the main verb slot in this text. 

Because the subject, direct object, and indirect object are optional for this construction, the construction can be used in Moloko discourse as a narrative device to reduce the number of explicit grammatical relations in a clause (cf. Sections 31 and 9.4) to draw the hearer into the action of the moment. The participants become indefinite in the construction and must be inferred by the context. In ex. 649, the construction is completely non-inflected for subject and has zero grammatical relations. The narrative effect is that the hearer only knows that someone is pouring something onto the grinding stone. Likewise, in ex. 658, the verb \textit{mɪ-l-ijɛ }‘bury’ is non-conjugated for subject, making those who buried the dead woman ‘out of sight’ in the narrative. 

\chapter[Verb types and transitivity]{Verb types and transitivity}
\hypertarget{RefHeading1212621525720847}{}
Friesen and \citet{Mamalis2008} reported that Moloko verb lexemes are underspecified with respect to transitivity.  Almost every Moloko verb can occur in clauses which are intransitive, transitive, or bitransitive and therefore cannot be classed as belonging to any one transitivity type.  Further study has allowed some of the ambiguity to be clarified. The unique way that the semantics of the verb are realised by the affixes and extensions is one of the things that shows the genius of the language. 

It is important to understand four important features of Moloko verbs with respect to transitivity. The first is that there are two kinds of transitive constructions in Moloko and an Agent-Theme-Location semantic analysis is necessary to interpret these two constructions (\sectref{sec:9.1}). For transitive clauses, the grammatical relations of Moloko verbs directly and uniformly reflect the semantic picture. Subject expresses Agent. Direct object expresses semantic Theme, the core participant that is literally or metaphorically changes state or position. Indirect object expresses semantic Location (LOC) which can be  (depending on the verb type) either a literal or a metaphorical LOC (recipient or beneficiary).\footnote{This semantic picture holds for bitransitive clauses (Sections 1167 and 1168). For intransitive clauses, the subject can correspond to a range of semantic roles; it can be any one of Agent, Theme, or LOC (\sectref{sec:9.2.1.1.1.1} and 1168).}

The second feature is that most Moloko verbs are ambitransitive – the same verb may occur in clauses that are bitransitive, transitive, or intransitive. Moloko verbs are divided into classes based on the type of transitive and ditransitive construction(s) that the verb has (\sectref{sec:9.2}). The third feature of Moloko verbs with respect to transitivity is that some verbs exhibit noun incorporation (\sectref{sec:9.3}). The final feature of Moloko verbs is that there are clauses with zero transitivity (\sectref{sec:9.4}).

There are no affixes, extensions, or particles that express causative or passive as might be expected in a Chadic language (with the exception of reciprocal, see \sectref{sec:68}).\footnote{Causative verbal extensions, for example, are widespread in Chadic languages (Newman, 1977: 276). } In Moloko, it is the number and type of grammatical relations that a verb has that reflects the semantics of the construction.

\section{Two kinds of transitive clauses}
\hypertarget{RefHeading1212641525720847}{}
Moloko has two kinds of transitive clauses – transitive clauses with subject and direct object (ex. 660 and 661) and transitive clauses with subject and indirect object (ex. 662 and 663). These two grammatically different transitive clauses illustrate that the semantics of Moloko verbs allows three core participants (represented by subject, direct object, and indirect object). Moloko verbs do not have just Agent-Patient semantic frames for events. In this work we follow an Agent-Theme-Location analysis, as developed by De\citet{Lancey1991}, in which ‘Location’ (LOC) has a particular definition. Indirect object always expresses semantic LOC – the participant that represents the place where the Theme is directed to. As such the indirect object can express (depending on the verb type, see Sections 66 - 68) the recipient or beneficiary of the event. Direct object always expresses semantic Theme, the core participant that changes position or state because of the event. Subject in transitive clauses expresses the Agent.

It is the verbal pronominals that best illustrate the grammar of the two types of transitive clauses because the grammatical distinction between direct and indirect object is expressed by a core pronominal (the direct object pronominal and the indirect object pronominal enclitic). For this reason, most of the examples are given in pairs in this chapter. The first example in each pair shows full noun phrase arguments for each core participant.\footnote{Note that the indirect object is inside a prepositional phrase. The indirect object prepositional phrases in Moloko is not a syntactic oblique, but represents a core participant of the event.} The second example in each pair shows the same clause with all core participants represented by verbal pronominals. Pronominals are bolded in the second example in each pair. 

Ex. 660 and 661 show a transitive clause with subject (\textit{Mana}) and direct object (\textit{awak} ‘goat’ in ex. 660, \textit{na} 3S direct object pronominal in ex. 661).


\ea
Mana  aslay  awak.
\z

Manaa\textbf{{}-}ɬ\textbf{{}-}aj    awak



Mana  3S-slay{}-CL  goat



‘Mana slays a goat.’ 


\ea
\textbf{a}slay  \textbf{na}\textbf{.}
\z

\textbf{a-}ɬ\textbf{{}-}aj      \textbf{na}



3S-slay{}-CL      3S.DO



‘He slays it.’ 


Ex. 662 and 663 show a transitive clause with subject (\textit{Mana}) and indirect object (\textit{ana kəra} ‘to dog’ in ex. 662,\textbf{ }=\textit{aŋ} ‘to him’ in ex. 663). 

\ea
Mana  aɓan  ana  kəra.  
\z

Mana   a-ɓ=aŋ    ana   kəra  



Mana  3S-hit=3S.IO  DAT  dog



‘Mana hits a dog.’ (lit. he hits to him to dog)


\ea
Aɓan.    
\z

a-ɓ    =aŋ    



3S-hit  =3S.IO



‘He hits him.’ (lit. he hits to him)


Crosslinguistic studies might lead one to expect a verb like ‘hit’ to take a direct object; however verbs in Moloko require an Agent-Theme-LOC semantic model to explain their behaviour. The indirect object \textit{kəra} ‘dog’ is the semantic LOC – here the recipient of the action – the participant that represents the place where the Theme (the hit) is directed to. The participant that changes position or state in this event (the hit) is implicit in verbs of this type (see \sectref{sec:66}). 

Returning to the transitive clause with subject and direct object (ex. 660 and 661), the direct object \textit{awak} ‘goat’ is the Theme – the participant that changes position or state because of an event (it is slain). 

\section{Verb types}
\hypertarget{RefHeading1212661525720847}{}
Most Moloko verbs are ambitransitive (i.e., labile) in that they can occur in intransitive, transitive, and sometimes bitransitive clauses with no morphological change in the verb complex (except of course the addition of the appropriate pronominals, \sectref{sec:7.3}).\footnote{Some verbs in related Chadic languages can also be ambitransitive. These include Cuvok (Ndokabai, 2006),  Buwal (Viljoen, 2013), and Vame (Kinnaird, 2006). } Nevertheless, they can be divided up into classes that act differently morphologically and syntactically with respect to transitivity. They are classified here as to the maximum number of grammatical relations that the verb can take as well as the type of grammatical relations:


\begin{itemize}
\item Group 1: Verbs that can only be intransitive (\sectref{sec:65})
\item Group 2: Verbs that can be transitive direct object (\sectref{sec:9.2.1.1.1})
\item Group 3: Verbs that can be transitive with indirect object (\sectref{sec:66})
\item Group 4: Verbs that can be bitransitive (\sectref{sec:67})
\item Group 5: Transfer verbs (\sectref{sec:68})
\end{itemize}

Examples are given in pairs in this chapter, first with full noun phrase arguments and then the same clause is given with the noun phrases replaced by pronominals. Both full noun phrases and pronominals are necessary because the centrality of the distinction of verb types in Moloko is more apparent from the pronominals, especially for the indirect object. The indirect object can be expressed with a core pronominal within the verb complex, or a full noun phrase within an adpositional phrase. 

\subsection{  Group 1: Verbs that can only be intransitive }
\hypertarget{RefHeading1212681525720847}{}
Only one verb in Moloko can never take an object (neither direct nor indirect). The locational clause contains the verb \textit{ndaj  }and states that the subject is presently located somewhere (ex. 664, 665). An explicit free noun phrase subject is not always required when this verb is the main predicate since the subject is indicated in the verb prefix; however the adpositional phrase giving the location is required and follows the verb. This same verb functions as a progressive aspect auxiliary (see \sectref{sec:62}).\footnote{It is interesting that the locational extension \textit{aka} is also used to express progressive aspect \citep{Section1156}.}


\ea
Hawa  \textbf{anday  }a  mogom.
\z

Hawa\textbf{a-ndaj}    a  mɔg\textsuperscript{w}ɔm



Hawa  3S-Be located  at  home



‘Hawa is at home.’ 


\ea
\textbf{Anday}  a  mogom.
\z

\textbf{a-ndaj}   a   mɔg\textsuperscript{w}ɔm



3S-be  at  home



‘She is in Maroua.’


\subparagraph[  Group 2: Verbs that can be transitive with direct object]{  Group 2: Verbs that can be transitive with direct object}

Clauses with reflexive-causative verbs can have either one core argument (subject) or two core arguments (subject and direct object). We have never found these verbs in a context where they take an indirect object as third core argument. 

Verbs from this class express reflexive actions when in an intransitive clause (action is to self; ex. 666) and causative actions when in a transitive clause with a direct object (action is to direct object; ex. 667). 


\ea
Mana  enjé  a  mogom. 
\z

Mana   ɛ{}-nʒ-ɛ\'{ }     a   mɔg\textsuperscript{w}ɔm 



Mana  3S.PFV-leave  at  home



‘Mana went home.’ (lit. Mana left to home)  


\ea
Mana  enjé  awak  a  mogom. 
\z

Mana   ɛ{}-nʒ-ɛ\'{ }     awak   a   mɔg\textsuperscript{w}ɔm 



Mana  3S.PFV-leave  goat  at  home



‘Mana took the goat home.’ (lit. Mana left goat to home) 


%%please move \begin{table} just above \begin{tabular
\begin{table}
\caption{presents the morphology and clause structures for sample verbs in this category, across both intransitive and transitive clause constructions.}
\label{tab:70}
\end{table}

\begin{tabular}{ll}
\lsptoprule

\textbf{Intransitive} & \textbf{Transitive}\\
\textit{Hawa }  \textit{ɛ{}-nʒ-}\textit{ɛ}\textit{\'{ } }

Hawa    3S.PFV-leave-CL

‘Hawa is gone.’ (lit. Hawa left)

\textit{ɛ{}-nʒ-}\textit{ɛ}\textit{\'{ } }

3S.PFV-leave-CL

‘She left.’ & \textit{Hawa }  \textit{ɛ{}-nʒ-}\textit{ɛ}\textit{\'{ } }    \textit{awak }  \textit{a }  \textit{m}\textit{ɔ}\textit{g}\textit{\textsuperscript{w}}\textit{ɔ}\textit{m }

Hawa    3S.PFV-leave-CL  goat    at    home

‘Hawa took the goat home.’ 

\textit{ɛ{}-nʒ-}\textit{ɛ}\textit{\'{ } }    \textit{na }  \textit{a }  \textit{m}\textit{ɔ}\textit{g}\textit{\textsuperscript{w}}\textit{ɔ}\textit{m}

3S.PFV-leave-CL  3S.DO   at   home

‘She took it home.’ \\
\textit{Hawa }  \textit{à-həɓ-aj}

Hawa   3S.PFV-dance-CL

‘Hawa danced.’ 

\textit{à-həɓ-aj}

3S.PFV-dance-CL

 ‘She danced.’ & \textit{muwta}  \textit{à-həɓ-aj }    \textit{mɪʒe}

car   3S.PFV-dance-CL   person

 ‘The car shook people up.’ (lit. the car danced people)

\textit{à-həɓ-aj }    \textit{na}

3S.PFV-dance-CL   3S.DO

 ‘It shook him.’\\
\textit{Hawa }  \textit{ɛ{}-tʃɪk-ɛ}

Hawa    3S.PFV-stand-CL

‘Hawa stood up.’

\textit{ɛ{}-tʃɪk-ɛ}

3S.PFV-stand-CL

‘She stood up.’ & \textit{Hawa }  \textit{ɛ{}-tʃɪk-ɛ }    \textit{zar}

Hawa    3S.PFV-stand-CL   man

‘Hawa helped the man to stand up.’ (lit. Hawa stood man)

\textit{ɛ{}-tʃɪk-ɛ }    \textit{na}

3S.PFV-stand-CL   3S.DO

‘She stood him up.’ \\
\textit{Hawa}  \textit{à-jə}\textit{ɗ}\textit{ə =va}

Hawa    3S.PFV-tire  =PERF

‘Hawa is tired.’

\textit{Hawa  }\textit{á-}\textit{jə}\textit{ɗ{}-aj }

Hawa  3S.IFV-tire-CL

‘Hawa can/might get tired.’ (lit. Hawa tires) & \textit{ɬərɛlɛ}  \textit{à}\textit{{}-jə}\textit{ɗ{}-aj }    \textit{Hawa}

work    3S.PFV-tire-CL   Hawa

‘Work tired Hawa out.’\\
\lspbottomrule
\end{tabular}

\begin{itemize}
\item \begin{styleTabletitle}
Reflexive-causative verbs
\end{styleTabletitle}\end{itemize}
\subsection{  Group 3: Verbs that can be transitive with indirect object}
\hypertarget{RefHeading1212701525720847}{}
Some transitive verbs in Moloko never take a direct object but rather have only what we call an indirect object in this work. These verbs express experience, feeling, or emotion. The indirect object expresses the semantic LOC (recipient, beneficiary, experiencer) of the event. A semantic core participant that moves or undergoes a change of state or is in a state (Theme) may be implicit or be lexicalised into the verb. 

The verb \textit{r}\textit{ə}\textit{ɓ{}-aj} ‘to be beautiful’ involves a thing and its quality (ex. 668 and 669), and the person whose opinion or perception is being cited is coded as the indirect object.  In an intransitive clause, the subject (\textit{dalaj}  ‘girl’) is at the state of being beautiful. In a transitive clause (with an indirect object), the subject (\textit{dalaj} ‘girl’) is felt to be beautiful by the indirect object (\textit{=aw} ‘to me’). 


\ea
Dalay  arəɓay.
\z

dalaj  a-rəɓ-aj



girl    3S-be\_beautiful{}-CL



‘The girl is beautiful.’


\ea
Dalay  arəɓaw.
\z

dalaj  a-rəɓ=aw



girl    3S-be beautiful=1S.IO



‘The girl is beautiful to me.’ 


The experience verb  /ts r/ ‘taste good’ is grammatically expressed in ex. 670 as the subject \textit{ɗaf}  ‘millet loaf tastes good at the semantic LOC expressed by the indirect object (the pronominal enclitic \textit{=aw} ‘to me’).

\ea
Ɗaf  acaraw. 
\z

ɗaf     à-tsar      =aw 



millet loaf  3S.PFV-taste good  =1S.IO



‘Millet loaf tasted good to me.’ 


Likewise with the verb /g r -j/  ‘fear’ (ex. 671),  the elephant causes fear at the LOC ‘the children.’

\ea
Mbelele  agarata  ana  babəza  ahay. 
\z

mbɛlɛlɛ   à-gar    =ata  ana  babəza    =ahaj 



elephant  3S.PFV-fear  =3P.IO  DAT  children    =PL



‘The children are afraid of the elephant.’


The verbs /dz n -j/ ‘help’ and /ɓ -j/ ‘hit’ are also in this group of verbs. The receiver of the help or hit is expressed by the indirect object which is affected positively (in the case of help) or negatively (in the case of hit) by the event. For these verbs, the semantic Theme (the hit or the help) never appears as a direct object since it is part of the meaning of these verbs.

%%please move \begin{table} just above \begin{tabular

Table 71 presents examples of verbs of this type.\footnote{An intransitive clause appears to be ungrammatical for the verbs /ɓ -j/ ‘hit’ and /s/ ‘please.’}

\begin{table}
\label{tab:71}
\end{table}

\begin{tabular}{ll}
\lsptoprule

\textbf{Intransitive} & \textbf{Transitive}\\
\textit{Hawa à-r}\textit{ə}\textit{ɓ{}-aj}

Hawa    3S.PFV-be\_beautiful-CL

‘Hawa was beautiful.’

\textit{à-r}\textit{ə}\textit{ɓ{}-aj}

3S.PFV-be\_beautiful-CL

‘She was beautiful.’ & \textit{h}\textsuperscript{w}\textit{ɔr }  \textit{à-r}\textit{ə}\textit{ɓ=aŋ }    \textit{ana }  \textit{Mana}

Hawa  3S.PFV-be\_beautiful  =3S.IO   DAT  Mana

‘The woman was beautiful to Mana.’ 

\textit{à-r}\textit{ə}\textit{ɓ=aŋ}

Hawa  3S.PFV-be\_beautiful  =3S.IO

‘She was beautiful to him.’\\
\textit{ɗaf }    \textit{à-tsar}

millet\_loaf    3S.PFV-taste\_good

‘Millet loaf tasted good.’

\textit{à-tsar}

3S.PFV-taste\_good

‘It tasted good.’ & \textit{ɗaf }     \textit{à-tsar=aŋ }      \textit{ana }  \textit{Mana}

millet\_loat  3S.PFV-taste\_good=3S.IO    DAT    Mana

‘Millet loaf tasted good to Mana.’ 

\textit{à-tsar=aŋ}

3S.PFV-taste\_good=3S.IO

‘It tasted good to him.’\\
\textit{Mana }  \textit{à-gar-aj}

Mana    3S.PFV-fear-CL

‘Mana was afraid.’ 

\textit{à-gar-aj}

3S.PFV-fear-CL

‘He was afraid.’ & \textit{mbɛlɛlɛ }  \textit{à-gar=aŋ }    \textit{ana }  \textit{Mana}

elephant  3S.PFV-fear=3S.IO    DAT   Mana

‘An elephant made Mana afraid.’ 

\textit{à-gar=aŋ}

3S.PFV-fear-CL=3S.IO

‘It made him afraid.’\\
\textit{fat }  \textit{á-war}

sun    3S.IFV-hurt

‘The sun hurts.’

\textit{á-war}

3S.IFV-hurt

‘It hurts.’ & \textit{fat }  \textit{á-wal=aŋ }    \textit{ana }  \textit{Mana}

sun  3S.IFV-hurt=3S.IO   DAT     Mana

‘The sun hurts Mana.’ (lit. The sun hurts to Mana) 

\textit{á-wal=aŋ}

3S.IFV-hurt=3S.IO

‘It hurts him.’\\
\textit{Mana }  \textit{á-ɗas}

Mana   3S.IFV-be\_heavy

‘Mana is honourable.’ (lit. Mana is heavy).

\textit{á-ɗas}

3S.IFV-be\_heavy

‘He is honourable.’ & \textit{Mana }  \textit{á-ɗəs=aŋ }          \textit{ana }   \textit{Hʊrmbʊlɔm}

Mana   3S.IFV-be\_heavy=3S.IO  DAT  God

‘Hawa honours God.’ (lit. Hawa honours to God) 

á\textit{{}-ɗəs=aŋ}

3S.IFV-be\_heavy=3S.IO

‘He honours him.’ \\
\textit{Hawa }  \textit{á-dzən-aj}

Hawa    3S.IFV-help-CL

‘Hawa helps (Hawa is the kind of person who is helpful).’ 

\textit{á-dzən-aj}

3S.IFV-help-CL

‘She is a helpful person.’ & \textit{Hawa }  \textit{á-dzən=aŋ }    \textit{ana }  \textit{Mana}

Hawa    3S.IFV-help =3S.IO  DAT    Mana

‘Hawa helps Mana.’ 

\textit{á-dzən=aŋ }

3S.IFV-help =3S.IO

‘She helps him.’\\
& \textit{Hawa }  \textit{á-ɓ=aŋ }    \textit{ana }  \textit{kəra}

Hawa    3S.IFV-hit =3S.IO  DAT    dog

‘Hawa hits the dog.’  

\textit{á-ɓ=aŋ}

3S.IFV-hit =3S.IO

‘She hits it.’\\
& \textit{ʃ}\textit{ɛ}\textit{ʃ}\textit{ɛ}\textit{ }  \textit{á-s=aŋ }         \textit{ana }  \textit{Mana}

meat    3S.IFV-please =3S.IO  DAT   Mana

‘Meat is pleasing to Mana.’ 

\textit{á-s=aŋ}

3S.IFV-please =3S.IO

‘It pleases him.’\\
\lspbottomrule
\end{tabular}

\begin{itemize}
\item \begin{styleTabletitle}
Verbs that occur in intransitive and transitive-IO clause constructions
\end{styleTabletitle}\end{itemize}
\subsection{ Group 4: Verbs that can be bitransitive }
\hypertarget{RefHeading1212721525720847}{}
Verbs that can be bitransitive with subject, direct object, and indirect object can occur in intransitive clauses (subject only), transitive clauses (subject and direct object), and bitransitive clauses (subject, object and indirect object). When present, the indirect object always expresses the benefactive or malefactive. 

The semantics of transitive and bitransitive clauses is uniform for these verbs – subject always expresses semantic Agent, direct object always expresses semantic Theme, and indirect object always expresses semantic LOC (typically Beneficiary or Maleficiary). Intransitive clauses are more flexible in that the subject can express either Agent or Theme for some verbs. Transitive and bitransitive clauses are discussed for these verbs in \sectref{sec:15.2.4.1} and intransitive clauses are discussed in \sectref{sec:9.2.1.1.1.1.}

\paragraph[Group 4 verbs in transitive and bitransitive clauses]{Group 4 verbs in transitive and bitransitive clauses}

The verb \textit{p-aj} ‘open’ illustrates this verb type. In a transitive clause (ex. 672), the subject (\textit{Mana}) performs the action on the direct object (\textit{mahaj} ‘door’). 


\ea
Mana  apay  mahay.
\z

Mana   à-p-aj    mahaj



Mana  3S.PFV-open{}-CL  door



‘He/she opened the door.’ 


In a bitransitive clause, the action done to the direct object is for the benefit of the indirect object.

\ea
Mana  apan  mahay  ana  Hawa.
\z

Mana   à-p=aŋ      mahaj  ana  Hawa



Mana  3S.PFV-open=3S.IO  door  DAT  Hawa



‘Mana opened the door for Hawa.’ 


The verb \textit{mənzar } ‘see’ occurs in intransitive, transitive, and bitransitive clauses. In a transitive clause (ex. 674), the subject (\textit{Mala}) sees the direct object (\textit{awak} ‘goat’).\footnote{The indirect object ‘goat’ undergoes a change of state from being unseen to being seen at a particular LOC.} 

\ea
Mala  ámənjar  awak.
\z

Malaá-mənzar   awak



Mala  3S.IFV-see  goat



‘Mala sees a goat.’{ }


In a bitransitive clause (ex. 675), the subject (\textit{Mala}) sees the direct object (\textit{awak} ‘goat’) on behalf of the indirect object (\textit{bahaj} ‘chief’). The chief is the metaphorical LOC to which the action is directed.

\ea
Mala  olo  amənjaran  awak  ana  bahay.
\z

Mala  ɔ{}-lɔ    a-mənzar =aŋ  awak  ana  bahaj



Mala  3S-go  3S-see=3SD\={ }AT  goat  DAT  chief



‘Mala went to see a person’s goat in the chief’s place.’{ }


For the verb /h/ ‘say’ (ex. 676), the subject \textit{Mana} says the utterance (expressed by the direct object pronominal \textit{na}) to \textit{Hawa}. 

\ea
Mana  ahan  na  ana  Hawa.
\z

Mana   à-h    =aŋ   na  ana   Hawa



Mana  3S.PFV-say  =3S.IO  3S.DO  DAT  Hawa



‘Mana told it to Hawa.’ 


For some transitive verbs of this type, the indirect object (when present) marks the malefactive of the event. The indirect object will be negatively affected by the event. For the verb \textit{paɗ-aj }‘eat’ in ex. 677 the subject (\textit{awak} ‘goat’) ate the direct object (\textit{haj} ‘millet’), incurring a negative effect on the indirect object (\textit{=aw} ‘to me’).{ } The indirect object expresses the participant who was affected by the loss of the millet who is the possessor of the direct object (the millet that was eaten).{ }\footnote{This phenomenon is also known as possessor raising or external possession. We consider that the semantics for this construction in Moloko are malefactive rather than possessive because a possessive construction can also be employed (without an indirect object): \textit{awak a-paɗ-aj na haj uwla =va, }‘the goat ate my millet’. The construction with an indirect object connects the millet to its owner with less precision than the possessive construction, and concentrates on the loss that the owner incurred (due to the damages done to his millet field) rather than the damage itself.}  

\ea
Awak  apaɗ\textbf{aw}   na  háy  va.
\z

awaka-paɗ\textbf{=aw}    na   haj  =va



goat    3S-crunch=1S.IO   3S.DO   millet  =\textsc{PRF}



‘The goat has eaten my millet.’ (lit. the goat has eaten to me the millet) 


The malefactive also occurs with and for the verbs \textit{mb}\textit{ɪʒɛŋ} ‘ruin’ (ex. 678) and \textit{tʃɛŋ} ‘understand.’ (ex. 679). In ex. 678 the subject (\textit{ɬa =ahaj }‘the cows’) have ruined the direct object (\textit{gəvah} ‘the field’) with a negative effect on the indirect object (\textit{=alɔk}\textit{\textsuperscript{w}}\textit{ɔ} ‘to us’).

\ea
Sla  ahaj  təmbəzaloko  na  gəvah  va.
\z

ɬa       =ahaj   tə-mbəz    =alɔk\textsuperscript{w}ɔ        na     gəvah  =va



cow    =Pl       3P.PFV-ruin  =1\textsc{Pin}.IO  3S.DO  \textit{   }field   =\textsc{PRF}



‘The cows have ruined our field.’ (lit. The cows have ruined to us the field)


Ex. 679 shows a bitransitive clause with the verb \textit{tʃɛŋ} ‘hear.’ The subject (\textit{à-}\textit{ }3S subject pronominal) didn’t understand the direct object (\textit{ma =uwla} ‘my words’) with a negative effect on the indirect object (\textit{=aw} ‘to me’).\footnote{Note that the word final /n/ drops off when the indirect object clitic attaches. } 

\ea
Acaw  aka  va  ma  əwla  bay.
\z

à-ts    =aw     =aka  =va     ma      =uwla      baj



3S.PFV-understand  =1S.IO  =on        =\textsc{PRF}  word   =1S.POSS  \textsc{NEG}



‘He/she didn’t understand my words.’ (lit. he had understood on my words not)


%%please move \begin{table} just above \begin{tabular
\begin{table}
\caption{presents examples of this verb type with benefactive indirect object.}
\label{tab:72}
\end{table}

\begin{tabular}{ll}
\lsptoprule

\textbf{Transitive} & \textbf{Bitransitive}\\
\textit{Hawa }  \textit{à}\textit{{}-bax }        \textit{jam}

Hawa    3S.PFV-pour  water

‘Hawa poured water.’

\textit{à}\textit{{}-bax }        \textit{na}

3S.PFV-pour  3S.DO

‘She poured it.’ & \textit{Hawa }  \textit{à}\textit{{}-bah=aŋ }         \textit{jam}     \textit{ana }  \textit{Mana}

Hawa    3S.PFV-pour=3S.IO  water  DAT  Mana

‘Hawa poured water for Mana.’

\textit{à}\textit{{}-bah=aŋ }      \textit{na}

3S.PFV-pour=3S.IO   3S.DO

‘She poured it for him.’   \\
\textit{Hawa}  \textit{à-p-aj }         \textit{mahaj}

Hawa    3S.PFV- open  door

‘Hawa opened the door.’

\textit{à-p-aj }         \textit{na}

3S.PFV- open  3S.DO

‘She opened it.’ & \textit{Hawa }  \textit{à}\textit{{}-p=aŋ }                  \textit{mahaj }  \textit{ana }  \textit{Mana}

Hawa    3S.PFV-open=3S.IO   door      DAT    Mana

‘Hawa opened the door for Mana.’

\textit{à}\textit{{}-p=aŋ }        \textit{na}

3S.PFV-open=3S.IO  3S.DO

‘She opened it for him.’ \\
\textit{Mana }  \textit{à-ɬ}\textbf{\textit{{}-}}\textit{aj }            \textit{awak }

Mana    3S.PFV-slay-CL  goat

‘Mana slaughtered a goat.’

\textit{à-ɬ}\textbf{\textit{{}-}}\textit{aj }    \textit{na}

3S.PFV-slay-CL    3S.DO

‘He slaughtered it.’ & \textit{Mana }  \textit{à-ɬ}\textbf{\textit{=aŋ}}\textit{ }         \textit{awak}\textit{    ana }  \textit{bahaj}

Mana    3S.PFV-slay=3S.IO      goat       DAT    chief

‘Mana slaughtered the goat for the chief.’ (lit. Mana slay the goat to chief)

\textit{à-ɬ}\textbf{\textit{=aŋ}}\textit{ }       \textit{na}

3S.PFV-slay=3S.IO  3S.DO

‘He slaughtered it for him.’\\
\textit{Hawa }  \textit{ɛ{}-dɛ }    \textit{ɗaf}

Hawa   3S.PFV-make-CL    millet loaf

‘Hawa made millet loaf.’

\textit{ɛ}\textit{{}-dɛ }    \textit{na}

3S.PFV-make-CL   3S.DO

‘She made it.’ & \textit{Hawa }  \textit{à}\textit{{}-d=aŋ }        \textit{ɗaf }       \textit{ana }  \textit{Mana}

Hawa    3S.PFV-make=3S.IO  millet loaf  DAT   Mana

‘Hawa made millet loaf for Mana.’

\textit{à}\textit{{}-d=aŋ }        \textit{na}

3S.PFV-make=3S.IO  3S.DO

‘She made it for him.’\\
\textit{Hawa }  \textit{à-bal-}\textit{aj}\textit{ }    \textit{zana }

Hawa    3S.PFV-wash-CL    clothes

‘Hawa washed clothes.’ 

\textit{à-bal-aj }    \textit{na}

3S.PFV-wash-CL    3S.DO

‘She washed it.’ & \textit{Hawa }  \textit{à-bal-}\textit{aŋ}\textit{ }          \textit{zana     ana }  \textit{Mana}

Hawa    3S.PFV-wash=3S.IO     clothes  DAT   Mana

‘Hawa washed clothes for Mana.’ 

\textit{à-bal-a}\textit{ŋ}\textit{ }         \textit{na}

3S.PFV-wash=3S.IO   3S.DO

‘She washed it for him.’ \\
\textit{Hawa }  \textit{à}\textit{{}-rax }    \textit{tsafəgal}

Hawa    3S.PFV-fill    bucket

‘Hawa poured the bucket.’

\textit{à}\textit{{}-rax }    \textit{na}

3S.PFV-fill    3S.DO

‘She filled it.’ & \textit{Hawa }  \textit{à}\textit{{}-rah=a}\textit{ŋ}\textit{ }        \textit{tsafəgal}  \textit{ana }  \textit{Mana}

Hawa  3S.PFV-slay=3S.IO      bucket    DAT   Mana

‘Hawa poured the bucket for Mana.’

\textit{à}\textit{{}-rah}\textit{=a}\textit{ŋ}\textit{ }    \textit{na}

3S.PFV-fill=3S.IO    3S.DO

‘She filled it for him.’\\
Mala   á-mənzar   awak

Mala   3S.IFV-see   goat

‘Mala sees a goat.’

\textit{á-mənzar  }  \textit{na}

3S.IFV-see    3S.DO

‘He sees it.’ & Mala   a-mənzar =aŋ   awak   ana   bahaj

Mala   3S-see=3S.D\={ }  goat   DAT  chief

‘Mala saw someone’s goat for the chief.’

a-mənzar =aŋ   na 

3S-see=3S.D\={ }   3S.DO 

‘He saw it for him.’\\
\lspbottomrule
\end{tabular}

\begin{itemize}
\item \begin{styleTabletitle}
Group 4 verbs in transitive and bitransitive clauseswhere IO expresses benefactive
\end{styleTabletitle}\end{itemize}
%%please move \begin{table} just above \begin{tabular
\begin{table}
\caption{provides examples of group 4 verbs where IO expresses malefactive.}
\label{tab:73}
\end{table}

\begin{tabular}{ll}
\lsptoprule

\textbf{Transitive} & \textbf{Bitransitive}\\
awak   à-paɗ-aj        haj

goat   3S.\textsc{PFV}{}-crunch{}-CL    millet

‘The goat ate the millet.’

à-paɗ-aj        na

3S.\textsc{PFV}{}-crunch{}-CL    3S.DO

‘He ate it.’ & awak   a-paɗ\textbf{=aw}   na   haj   =va

goat   3S-crunch=1S.IO  3S.DO   millet   =\textsc{PRF}

‘The goat has eaten my millet.’ (lit. the goat has eaten to me the millet)

a-paɗ\textbf{=aw}   na   =va

3S-crunch=1S.IO    3S.DO   =\textsc{PRF}

‘The goat has eaten it to me.’ (the goat has eaten it and I am affected)\\
\textit{avar }  \textit{ɛ}\textit{{}-mb}\textit{ɛʒɛ}\textit{ŋ }  \textit{haj}

rain    3S-ruin      millet

‘The rain ruined the millet.’

\textit{ɛ}\textit{{}-mb}\textit{ɛʒɛ}\textit{ŋ }  \textit{na}

3S-ruin       3S.DO

‘It ruined it.’ & \textit{ɬ}\textit{a }  \textit{à-mbəz=alɔk}\textit{\textsuperscript{w}}\textit{ɔ }    \textit{na }  \textit{gəvah}  \textit{=va}

cow    3S.PFV-ruin=1PIN.IO    3S.DO    field    =PERF

‘The cow has ruined our millet.’

\textit{à-mbəz=alɔk}\textit{\textsuperscript{w}}\textit{ɔ }    \textit{na}  \textit{=va}

3S.PFV-ruin=1PIN.IO    3S.DO    =PERF

‘He has ruined it.’\\
\textit{awak }  \textit{à-zɔm }    \textit{haj}

goat  3S.PFV-eat    millet

‘The goat ate millet.’

\textit{à-zɔm }    \textit{na}

3S.PFV-eat  3S.DO

‘He ate it.’ & \textit{awak }  \textit{à-zɔm=aŋ }    \textit{haj }  \textit{a }  \textit{Mana}

goat    3S.PFV-eat=3S.IO    millet    GEN  Mana

‘The goat ate Mana’s millet.’ (lit. the goat ate to him millet of Mana).’

\textit{à-zɔm=aŋ }    \textit{na}

3S.PFV-eat=3S.IO    3S.DO

‘It ate his thing.’\\
\lspbottomrule
\end{tabular}
\begin{itemize}
\item \begin{styleTabletitle}
Group 4 verbs in transitive and bitransitive clauses where IO expresses malefactive
\end{styleTabletitle}\end{itemize}

Moloko uses a transitive clause with a third person plural subject pronominal when the identity of the Agent is unimportant or unknown in the discourse. The literal meaning of (ex. 680) is ‘They are greeting you,’ but this construction is used even when the person greeting is singular and the speaker knows who it is but doesn’t want to say.\footnote{The verb /h -j/ ‘say’ shows incorporation of the ‘body-part’ noun \textit{ma} ‘word/mouth’ (\sectref{sec:9.3}).} Ex. 681 is from the Disobedient Girl text (see \sectref{sec:1.5}). The example literally means ‘they brought her out’ but the identity of those who carried her is unimportant in the story.


\ea
Tahok  ma.
\z

ta-h=ɔk\textsuperscript{w}  ma



3P-say=2S.IO    mouth



‘You are being greeted.’ (lit. they are saying word to you) 



Disobedient Girl S. 31


\ea
Tazlərav  na  ala    
\z

tà-ɮərav    na  =ala    



3P.PFV-exit 3S.DO  =to 



‘She was brought out [of the house].’ (lit. they brought her out)


\begin{styleHeadingvi}
Group 4 verbs in intransitive clauses
\end{styleHeadingvi}

The meaning of intransitive clauses of group 4 verbs changes depending on whether the subject expresses Agent or Theme (or either, for some verbs) and depending on the aspect of the verb.

There are two possible semantic possibilities for intransitive clauses in Perfective aspect. Subject can be the semantic Agent or the semantic Theme. Some verbs have both possibilities, but for other verbs, subject can express only Agent or only Theme. 

For the verb  ‘prepare,’ the subject of an intransitive clause is the semantic Agent (ex. 684). 


\ea
Hawa  ede. 
\z

Hawa   ɛ{}-d-ɛ 



Hawa  3S.PFV-prepare-CL



‘Hawa made [something].’ 


With \textit{tʃɛŋ} ‘hear,’ an intransitive clause in Perfective aspect (ex. 683) expresses an event where the subject hears and understands (what they hear/understand may not be explicit in the clause). 

\ea
Mana  ecen. 
\z

Mana   ɛ{}-tʃɛŋ 



Mana  3S.PFV-understand



‘Mana heard/understood (something).’


For the verb \textit{p-aj} ‘open,’ the subject of an intransitive clause is the semantic Theme which is affected by the action (ex. 684). 

\ea
Mahay  apay.
\z

mahaj   à-p-aj



door  3S.PFV-open{}-CL



‘The door opened.’ 


There is also a difference between the Imperfective and Perfective in an intransitive clause that doesn’t hold for transitive and bitransitive clauses.\footnote{Intransitive clauses with transfer verbs \citep{Section1168} also show this semantic picture. } In transitive and bitransitive clauses, Imperfective expresses an event that is happening but is incomplete or unfinished (see \sectref{sec:52}). Perfective on the other hand expresses an event that has been completed (see \sectref{sec:51}), and the Perfect expresses that the event was completed prior to a point of reference (see \sectref{sec:58}). In intransitive clauses for these verbs, Imperfective aspect indicates that the subject is at the state of being potentially able to do or submit to the action (more of an irrealis idea) while Perfect is a resultative state. 

For example, an intransitive clause with the verb /p -j/ ‘open’ expresses an event with an unspecified Agent when the verb is Perfective: ‘the door opened’ (ex. 685). 

\ea
Mahay  apay.
\z

mahaj à-p-aj



door  3S.PFV-open{}-CL



‘The door opened.’


Likewise with the verb /b h/ ‘pour,’ water ‘is poured’ (ex. 686).  

\ea
Yam  abah.
\z

jam     à-bax



water  3S.PFV-pour



‘Water poured.’


If the verb is Imperfective, the clause means that the door is able to be opened, i.e., it is not locked (ex. 687).

\ea
Mahay  ápay.
\z

mahaj   á-p-aj



door  3S.IFV-open{}-CL



‘The door opens.’


In the Perfect, the clause means that the door is open (i.e., someone has already opened it, ex. 688).

\ea
Mahay  apava.
\z

mahaj   a-p-a   =va



door  3S-open  =\textsc{PRF}



‘The door is open.’


Imperfective aspect in an intransitive clause presents a situation where a state or capability is expressed. For the verb \textit{mənzar} ‘see,’ an intransitive clause in Imperfective aspect (ex. 689) can have an ablitative sense in that the subject ‘Mala’ is able to see. It can also mean that the subject is visible (subject expresses semantic Theme).

\ea
Mala  ámənjar. 
\z

Mala  á-mənzar 



Mala  3S.IFV-see



‘Mala sees.’ (i.e. he is not blind) \textbf{/ }‘Mala can be seen.’ 


%%please move \begin{table} just above \begin{tabular
\begin{table}
\caption{presents examples of bitransitive Moloko verbs in intransitive clauses. The three columns show Perfective, Imperfective, and Perfect forms of the verbs. Perfective aspect (column 1) expresses either an action that the Agent did (with an unexpressed Theme) or an event that happened to the Theme (with an unexpressed Agent). Imperfective aspect (column 2) indicates readiness of the Agent to do the action or expresses ability of the Theme to submit to the action. The Perfect (column 3) expresses a resultative – a finished action or the state resulting from the event. For some verbs, the subject can express either Agent or Theme. For others, the subject of an intransitive clause can only express Theme.}
\label{tab:74}
\end{table}

\begin{tabular}{llll}
\lsptoprule

\textbf{Verb} & \textbf{Perfective} & \textbf{Imperfective} & \textbf{Perfect}\\
\textit{zɔm }

‘eat’ & \textit{Mana }  \textit{à-}\textit{zɔm }

Mana    3S.PFV-eat

‘Mana ate [something].’ & \textit{Mana }  \textit{á-zɔm }

Mana   3S.IFV-eat 

‘Mana is about to eat [something].’ & \textit{Mana }  \textit{à-}\textit{zəm        =va }

Mana    3S.PFV-eat  =PERF

‘Mana ate [something] already.’\\
&  & \textit{haj }  \textit{á-zɔm }

millet   3S.IFV-eat

‘The millet is eating itself (There are insects in the millet).’ (lit. millet eats) & \textit{haj }  \textit{á-zəm       =va}

millet    3S.IFV-eat  =PERF

‘The millet has been eaten.’ \\
\textit{ɬ}\textbf{\textit{{}-}}\textit{aj}

‘slaughter’ & \textit{Mana }  \textit{à-ɬ}\textbf{\textit{{}-}}\textit{aj }

Mana    3S.PFV-slay-CL

‘Mana slaughtered [something].’ & \textit{Mana }  \textit{á-}\textit{ɬ}\textbf{\textit{{}-}}\textit{aj }

Mana   3S.IFV-slay-CL

‘Mana is about to slaughter [something].’ & \textit{Mana }  \textit{à-ɬ}\textbf{\textit{{}-}}\textit{a           =va}

Mana   3S.PFV-slay  =PERF

‘Mana has slaughtered [something].’\\
&  & \textit{awak }  \textit{á-}\textit{ɬ}\textbf{\textit{{}-}}\textit{aj }

goat   3S.IFV-slay-CL

‘The goat is good for slaughtering.’ & \textit{awak }  \textit{à-ɬ}\textbf{\textit{{}-}}\textit{a           =va}

goat    3S.PFV-slay  =PERF

‘The goat has been slaughtered.’ \\
\textit{ʃ{}-}\textit{ɛ} 

‘drink’ & Mana   \textit{ɛ}\textit{{}-}\textit{ʃ{}-}\textit{ɛ}

Mana    3S.PFV-drink-CL

‘Mana drank [something].’ & Mana   \textit{ɛ{}-}\textit{ʃ{}-}\textit{ɛ}

Mana    3S.IFV-drink-CL

‘Mana is about to drink [something].’ & \\
&  & \textit{jam }  \textit{ɛ{}-}\textit{ʃ{}-}\textit{ɛ }

water    3S.IFV-drink-CL

‘The water is drinkable.’ (lit. water drinks).’ & \textit{jam }  \textit{à-s}\textit{{}-ə             =va}

water   3S.PFV-drink  =PERF

‘The water has been drunk.’\\
\textit{d-ɛ}

‘prepare’ & \textit{Hawa }  \textit{ɛ{}-d-ɛ }

Hawa    3S.PFV-make-CL

‘Hawa prepared [something].’ &  & \\
&  & \textit{ɛlɛlɛ }  \textit{ɛ{}-d-ɛ }

sauce    3S.IFV-make-CL

‘Sauce can be made.’ (lit. sauce makes) & \textit{ɛlɛlɛ }  \textit{à}\textit{{}-d              =va }

sauce   3S.PFV-make  =PERF

‘Sauce is ready to be eaten.’ (lit. sauce was prepared)\\
\textit{bal-aj}

‘wash’ & \textit{Hawa }  \textit{à-bal-aj}

Hawa    3S.PFV-wash-CL

‘Hawa washed [herself].’ & \textit{Hawa }  \textit{á-}\textit{bal-aj}

Hawa    3S.IFV-wash-CL

‘Hawa washes [herself].’ & \textit{Hawa }  \textit{à-bal          =va}

Hawa    3S.PFV-wash  =PERF

‘Hawa is washed.’ \\
&  & \textit{zana }  \textit{á-}\textit{bal-aj}

cloth    3S.IFV-wash-CL

‘The cloth can be washed.’ (lit. the cloth washes) & \textit{zana }  \textit{à-bal          =va}

cloth    3S.PFV-wash  =PERF

‘The cloth is clean.’ (lit. cloth is washed)\\
\textit{p-aj}

‘open’ & \textit{mahaj }  \textit{à-p-aj}

door    3S.PFV-open-CL

‘The door opened.’ & \textit{mahaj }  \textit{á-p-aj}

door    3S.IFV-open-CL

‘The door opens.’ (is able to open) & \textit{mahaj   a-p   =va}

door  3S -open  =PERF

‘The door is open.’\\
\textit{bax}

‘pour’ & \textit{jam }  \textit{à}\textit{{}-bax}

water    3S.PFV-pour

‘Water poured.’ & \textit{jam }  \textit{á-bax}

water    3S.IFV-pour

‘Water is able to be poured.’ (lit. water pours) & \textit{jam }  \textit{a}\textit{{}-bah   =va}

water   3S -pour  =PERF

‘Water is poured out.’\\
\textit{mb}\textit{ɪʒɛ}\textit{ŋ}

‘ruin’ & \textit{haj }  \textit{à-}\textit{mb}\textit{ɪʒɛ}\textit{ŋ}

millet   3S.PFV-ruin

‘The millet ruined.’ & \textit{haj }  \textit{á-mb}\textit{ɪʒɛ}\textit{ŋ}

millet    3S.IFV-ruin

‘The millet is ruining.’ & \textit{haj }  \textit{á-mbəzə    =va}

millet    3S.IFV-ruin  =PERF

‘The millet has ruined.’\\
\textit{rax}

‘fill’ &  & \textit{tsafgal }  \textit{á-rax}

bucket    3S.IFV-fill

‘The bucket will fill.’ (lit. the bucket fills) & \textit{tsafgal }  \textit{à}\textit{{}-rah       =va}

bucket    3S.PFV-fill  =PERF

‘The bucket is full.’ (lit. something filled the bucket)\\
\lspbottomrule
\end{tabular}

\begin{itemize}
\item \begin{styleTabletitle}
Intransitive clauses
\end{styleTabletitle}\end{itemize}
\subsection{  Group 5: Transfer verbs}
\hypertarget{RefHeading1212741525720847}{}
Three transfer verbs in Moloko are notable. They are \textit{dəbən-aj} ‘learn/teach,’ \textit{sk}\textit{\textsuperscript{w}}\textit{ɔm} ‘buy/sell,’ and \textit{v}\textit{ə}\textit{l}  ‘give.’ These verbs are especially labile in terms of their semantic expression in that a transitive clause can have \textit{either} a direct or an indirect object. 

Ex. 690 illustrates the verb \textit{v}\textit{ə}\textit{l}  ‘give’ in a bitransitive clause. The subject (\textit{bahaj} ‘chief’) transfers the direct object (\textit{dalaj}  ‘girl’) to the indirect object (\textit{Mana}).


\ea
Bahay  avəlan  dalay  ana  Mana.
\z

bahaj   à-vəl    =aŋ     dalaj   ana   Mana



chief  3S.PFV-give  =3S.IO  girl  DAT  Mana



‘The chief gave the girl to Mana (in marriage).’ 


When \textit{v}\textit{ə}\textit{l}  ‘give’ occurs in a transitive clause, the second core argument can be either a direct object (ex. 691) or an indirect object (ex. 692).  In ex. 691, the chief is marrying off his daughter to an unspecified suitor. The subject (\textit{bahaj}  ‘chief’) transfers the direct object (\textit{dalaj}  ‘girl’) to someone who is unspecified in the clause. 

\ea
Bahay  ávar  dalay.
\z

bahaj    á-var      dalaj



chief  3S.IFV-give  girl



 ‘The chief is marrying off his daughter [to someone].’ (lit. chief gives girl) 


In ex. 692, the subject (\textit{bahaj}  ‘chief’) transfers something or someone to the indirect object (\textit{Mana}). What he gave would probably be specified in the immediate context, but is out of sight in this clause.

\ea
Bahay  avəlan  ana  Mana.
\z

bahaj   à-vəl    =aŋ   ana   Mana



chief  3S.PFV-give  =3S.IO  DAT  Mana



‘The chief gave [something] to Mana.’


When the verb \textit{v}\textit{ə}\textit{l}  ‘give’ occurs in an intransitive negative clause (Imperfective, ex. 693), it expresses that the subject is at the state of not giving anything to anyone, or not being the giving kind.\footnote{Note the phonological change of the final consonant (\sectref{sec:6.2}). } Without the negative marker, the meaning would probably be ‘the chief is the giving kind.’\footnote{This is a specific example from a text. We have not seen one-participant clauses for this verb type in Perfective aspect. The semantics of one-participant clauses for group four verbs is discussed in \sectref{sec:9.2.1.1.1.1.}}

\ea
Bahay  ávar  bay.
\z

bahaj  á-var       baj



chief  3S.IFV-give  NEG



‘The chief is not the giving kind.’ (lit. chief doesn’t give) 


The verb \textit{dəbən-aj} ‘learn’ or ‘teach’ occurs in transitive and bitransitive clauses.\footnote{We found no clauses with one core participant for this verb. } In bitransitive clauses illustrated by ex. 694, the subject (\textit{bahaj}  ‘chief’) transfers the direct object (\textit{M}\textit{ʊ}\textit{lɔk}\textit{\textsuperscript{w}}\textit{ɔ}\textit{ }‘Moloko language’) to the indirect object (\textit{ana babəza =ahaj} ‘to the children’).\footnote{The indirect object is also expressed as the verbal pronominal extension \textit{=ata} ‘to them.’ The indirect object could also express the beneficiary of the event. } 

\ea
Bahay  adəbənata  Məloko  ana  babəza  ahay. 
\z

bahaja-dəbən=ata   Mʊlɔk\textsuperscript{w}ɔ    ana   babəza     =ahaj 



chief  3S-learn =3P.IO  Moloko    DAT  children    =Pl



‘The chief teaches Moloko to the children.’ 


In transitive clauses with subject and direct object (ex. 695), the subject (\textit{babəza =ahaj} ‘children’) transfers the direct object (\textit{M}\textit{ʊ}\textit{lɔk}\textit{\textsuperscript{w}}\textit{ɔ}\textit{  }‘Moloko language’) to self. 

\ea
Babəza  ahay  tədəbənay  Məloko.
\z

babəza   =ahaj   tə-dəbən-aj   Mʊlɔk\textsuperscript{w}ɔ



children  =Pl  3P-learn{}-CL  Moloko  



‘The children learn Moloko.’ 


Ex. 696 illustrates a transitive clause with subject and indirect object.  The subject (\textit{M}\textit{ʊ}\textit{lɔk}\textit{\textsuperscript{w}}\textit{ɔ}\textit{  }‘Moloko language;’ the semantic Theme) is transferred to the indirect object (\textit{=ɔk}\textit{\textsuperscript{w}}\textit{ }‘to you’).  

\ea
Məloko  adəbənok  na  jajak.
\z

Mʊlɔk\textsuperscript{w}ɔ  a-dəbən=ɔk\textsuperscript{w}  na  dzadzak



Moloko  3S-learn=2S.IO  \textsc{PSP}  fast



‘Moloko is easy for you to learn.’ (lit.Moloko learns to you quickly)


The verb \textit{sk}\textit{\textsuperscript{w}}\textit{ɔm} ‘buy’/’sell’ is also a transfer verb with two semantic LOCs. The event of buy/sell is accomplished through transfer of the Theme from one LOC to another. In a bitransitive clause (ex. 697), the subject (\textit{nə-}  ‘I’) causes the direct object (\textit{awak} ‘goat’) to go to the indirect object (\textit{ana Mana} ‘to Mana’). 

\ea
Nəskoman  awak  ana  Mana.
\z

nə-sk\textsuperscript{w}ɔm =aŋ     awak   ana   Mana



1S-buy/sell  =3S.IO    goat  DAT  Mana



 ‘I sell a goat to Mana.’ 


In a transitive clause with direct object (ex. 698), the subject (\textit{nə-} ‘I’) transfers the direct object (\textit{awak} ‘goat’) to self. We found no intransitive clauses for this verb.

\ea
Nəskomala  awak.
\z

nə-sk\textsuperscript{w}ɔm =ala   awak



1S-buy/sell  =to  goat



‘I bought a goat.’ 


The verb \textit{h-aj} ‘speak’ also appears to be in this class, but we have not found this verb in all contexts. In ex. 699, Mana caused what he said (‘it’) to go to the men.  

\ea
Mana  àhata  na  va  ana  zawər  ahay.
\z

Mana   à-h    =ata   na   =va   ana   zawər   =ahaj



Mana  3S.PFV-speak  =3P.IO  3S.DO  =\textsc{PRF}  DAT  men  =Pl



‘Mana has already told it to the men.”


%%please move \begin{table} just above \begin{tabular
\begin{table}
\caption{presents examples of these transfer verbs in intransitive, transitive, and bitransitive clauses.}
\label{tab:75}
\end{table}

\begin{tabular}{llll}
\lsptoprule

\textbf{Intransitive} & \textbf{Transitive with direct object} & \textbf{Transitive with indirect object} & \textbf{Bitransitive}\\
\textit{Hawa  á-var   }      \textit{baj}

Hawa  3S.IFV-give  NEG

‘Hawa is not the giving kind.’ (lit. Hawa doesn’t give) 

\textit{á-var  }     \textit{baj}

3S.IFV-give   NEG

‘She is not the giving kind.’ & \textit{Hawa}  \textit{á-var  }     \textit{jam}

Hawa   3S.IFV-give  water

‘Hawa gives water [to someone].’ 

\textit{á-var  }     \textit{na}

3S.IFV-give  3S.DO

‘She gives it [to someone].’ & \textit{Hawa à-vəl=aŋ ana Mana}

Hawa 3S.PFV-give=3S.IO DAT Mana

‘Hawa gave [something] to Mana.’

\textit{à-vəl=aŋ }

3S.PFV-give=3S.IO

‘She gave [something] to him.’ & \textit{Hawa à-vəl=aŋ              jam    ana   Mana}

Hawa 3S.PFV-give=3S.IO water DAT Mana

‘Hawa gave water to Mana.’

\textit{à-vəl=aŋ  }    \textit{na}

3S.PFV-give=3S.IO    3S.DO

‘She gave it to him.’ \\
& \textit{babəza =ahaj tə-dəbən-aj Mʊlɔk}\textit{\textsuperscript{w}}\textit{ɔ}

children =Pl 3P-learn{}-CL Moloko

‘The children learn Moloko.’ & \textit{Mʊlɔk}\textit{\textsuperscript{w}}\textit{ɔ a-dəbən=ɔk}\textit{\textsuperscript{w}}\textit{   na   dzadzak}

Moloko  3S-learn=2S.IO \textsc{PSP } fast

‘Moloko is easy for you to learn.’ (lit.Moloko learns to you quickly) & \textit{bahaj    a-dəbən=ata     M}\textit{ʊ}\textit{lɔk}\textit{\textsuperscript{w}}\textit{ɔ }

chief   3S-learn =3P.IO  Moloko 

\textit{ana }  \textit{babəza=ahaj}

DAT   children =Pl

‘The chief teaches Moloko to the children.’\\
& \textit{nə-sk}\textit{\textsuperscript{w}}\textit{ɔm =ala    awak}

1S-buy/sell  =to    goat

‘I bought a goat.’ &  & \textit{nə-sk}\textit{\textsuperscript{w}}\textit{ɔm =aŋ        awak ana   Mana}

1S-buy/sell  =3S.IO  goat   DAT Mana

 ‘I sell a goat to Mana.’\\
\textit{Mana   a-h-aj        baj}

Mana   3S-say-CL  NEG

‘Mana doesn’t say.’ &  &  & \textit{Hawa }    \textit{a-h=aŋ }      \textit{ma     ana   Mana}

Hawa      3S-say=3S.IO mouth DAT Mana

‘Hawa greets Mana.’\\
\lspbottomrule
\end{tabular}

\begin{itemize}
\item \begin{styleTabletitle}
Transfer verbs
\end{styleTabletitle}\end{itemize}

A fourth participant is possible for the verb \textit{v}\textit{ə}\textit{l}  ‘give’ and appears as an oblique adjunct. In ex. 700 the subject (‘you,’ 2S imperative verb) transfers the direct object (\textit{dala} ‘money’) to the indirect object (\textit{ana Mana} ‘to Mana,’ note the indirect object pronominal enclitic \textit{=aŋ}) for the benefit of the person expressed by a possessive pronoun in the oblique prepositional phrase (\textit{kəla} \textit{=uwla} ‘my benefit’). Thus when there is both a Beneficiary and a Recipient (which is the core LOC), a preposition (\textit{kəla}) plus one of the possessive pronouns (see \sectref{sec:14}) mark the benefactive. 


\ea
Vəl\textbf{an}  dala  \textbf{kəla  }\textbf{ə}\textbf{wla}  ana  Mala.
\z

vəl\textbf{=aŋ}  dala  \textbf{kəla    =uwla}    ana  Mala



give=3S.IO  money  for (benefactive)  =1S.POSS  DAT  Mala



‘Give Mala the money for me (lit. my benefit).’


In ex. 701 the subject pronominal (\textit{a-} ‘he’) transfers the direct object (\textit{awak} ‘goat’) to the indirect object (pronominal enclitic \textit{=ɔk}\textit{\textsuperscript{w}} ‘to you’) for the benefit of the pronoun in the oblique (\textit{kəla =uwla} ‘my benefit’).

\ea
Avəl\textbf{ok}  awak  \textbf{kəla  }\textbf{ə}\textbf{wla}  
\z

a-vəl\textbf{=ɔk}\textsuperscript{w}    awak  \textbf{kəla    =uwla}  



3S-give=2S.IO  goat  for (benefactive)  =1S.POSS  



‘He/she gave you the goat on my behalf (lit. my benefit).’


\section{  ‘Body-part’ verbs (noun incorporation)}
\hypertarget{RefHeading1212761525720847}{}
Friesen and \citet{Mamalis2008} identified a unique group of verb constructions. In these constructions, a special, sometimes phonologically reduced noun form that represents a part of the body is incorporated into the verb phrase.  This is a case of noun incorporation where these ‘body-part’ nouns are closely associated with the verb complex and their incorporation changes the lexical characteristics of the verb. These ‘body-part’ nouns include \textit{ma} ‘mouth,’ ex. 702, \sectref{sec:3}, \textit{ɛlɛ} ‘eye,’ ex. 703, \sectref{sec:1}, \textit{ɬ}\textit{əmaj} ‘ear,’ ex. 704, \sectref{sec:2}, and \textit{va} or \textit{har} ‘body,’ ex. 705, 706, Sections 4 and 5, respectively). These nouns can be incorporated into transitive or bitransitive verbs from the types in Sections 67 and 66.


\ea
Ataraŋ  aka  \textbf{ma}  ana  war  ese.
\z

a-tar= aŋ     =aka   \textbf{ma}  ana  war  ɛʃɛ



3S-call=3S.IO  =on  mouth  DAT  child  again



‘He/she calls the child again.’ (lit. he calls mouth to him to the child again)


\ea
Mala  amənjar  \textbf{el}\textbf{é.}
\z

Mala   a-mənzar   \textbf{ɛlɛ}



Mala   3S-see    eye



‘Mala looks around attentively.’ 


\ea
Acaka  va  \textbf{sl}\textbf{əmay}  ana  mama  ahan  bay.
\z

a-ts    =aka  =va  \textbf{ɬ}\textbf{əmaj}   ana  mama  =ahaŋ    baj



3S-hear  =on  =\textsc{PRF}  ear  DAT  mother  =3S.POSS  \textsc{NEG}



‘He/she is disobedient to his mother.’ (he disobeys his mother)\footnote{Note that the word-final /n/ is deleted when the verbal extension is attached \citep{Section1111}.} 


\ea
Tandalay  talala  təzləgə  \textbf{va  }ana  Məloko  ahay.
\z

ta-ndalajta-l =ala  tə-ɮəg-ə   \textbf{va}  ana  Mʊlɔk\textsuperscript{w}ɔ    =ahaj



3P-PRG   3P-go =to  3P-throw body  DAT  Moloko    =Pl



‘They were coming and fighting with the Molokos.’ (lit. they were coming they threw body to Molokos)


\ea
Ma  ango  agəsaw  \textbf{har}\textbf{.}
\z

ma     =aŋg\textsuperscript{w}ɔ    a-gəs     =aw  \textbf{har}



word  =2S.POSS  3S-catch=1S.IO  body



‘It pleases me.’ (lit. it catches body to me)


The body-part noun follows directly after all other elements in the verb complex. It appears to be in the same position as any other noun phrase direct object in the verb phrase (see Chapter 8); however it is in more tightly bound to the verb complex than a noun phrase. The body-part noun does not fill the DO pronominal slot, because verbal extensions that follow the DO pronominal in the Moloko verb complex occur before the body-part (see ex. 702 and 704 which each have an adpositional extension, see \sectref{sec:56}). It is not phonologically bound to the verb since, unlike the Perfect verbal extension \textit{=va} which is part of the verb complex, the ‘body-part’ \textit{va} does not neutralise the prosody on the verb stem (ex. 705). However, the incorporated noun is grammatically closer to the verb complex than a noun phrase direct object would be because the ‘body-part’ can never be separated from the verb complex. The ‘body-part’ can never be fronted in the clause. Nor can the ‘body-part’ be separated from the verb complex by the presupposition marker. Both of these situations can occur for noun phrase direct objects and are illustrated in \sectref{sec:70}, ex. 876 and 877. 

Incorporation of the ‘body-part’ noun never co-occurs with another direct object or with the DO pronominal \textit{na}. A transitive clause with subject, indirect object and incorporated ‘body-part’ noun can occur where the indirect object expresses semantic LOC (sometimes metaphorical).

This section is organised by ‘body-part’ plus verb collocations:


\begin{itemize}
\item \textit{ɛlɛ} ‘eye’ (\sectref{sec:1}). Used with verbs of seeing. 
\item \textit{ɬəmaj} ‘ear’ (\sectref{sec:2}). Collocates with verbs of cognition. 
\item \textit{ma} ‘mouth’ (\sectref{sec:3}). \textit{Ma} also can mean ‘word’ or ‘language.’ Used with verbs of speaking. 
\item \textit{va} ‘body’ (\sectref{sec:4}). \textit{Va} is phonologically reduced from \textit{hərva} ‘body.’ Used to form reciprocal actions. 
\item \textit{har} ‘body’ (\sectref{sec:5}). \textit{Har} is also phonologically reduced from \textit{hərva} ‘body.’ 
\end{itemize}

Note that there are Moloko idioms that employ body parts with the verb \textit{g-ɛ} ‘do’ which may be a case of noun incorporation. To get angry is to ‘do heart’ (ex. 707). 


\ea
Ege  ɓərav.
\z

ɛ{}-g-ɛ   ɓərav



3S-do-CL  heart



‘He/she is angry.’ (lit. he/she does heart)


The idiom for ‘think’ is literally ‘do brain’ (ex. 708). 

\ea
Ge  endeɓ!
\z

g-ɛ       ɛndɛɓ



do[2S \textsc{IMP}] -CL  brain



‘Think!’ (lit. do brain)


\paragraph[ɛlɛ  ‘eye’]{\textit{ɛlɛ}  ‘eye’}

The ‘body-part’ noun \textit{ɛlɛ}  ‘eye’ collocates with some verbs to lexicalise the engagement of the eyes and reduce the focus on what is seen.  This body-part word is used in its full form. For example, the verb \textit{mənzar}\textit{ } normally means ‘see’ (see \tabref{tab:76}.). With the incorporation of \textit{ɛlɛ} (ex. 709 and 710), the verb plus ‘body-part’ construction has a more active experiential meaning in that the subject of the clause (\textit{Mala}) is looking around attentively. Since there can be no direct object, there is no explicit referential object as stimulus – the speaker is vague about what exactly Mala will look at. 


\ea
Mala  amənjar  \textbf{el}\textbf{é.}
\z

Mala   a-mənzar   \textbf{ɛlɛ}



Mala   3S-see    eye



‘Mala looks around attentively.’ 


\ea
Mala  olo  aməmənzəre  \textbf{elé}  a  ləhe.
\z

Mala  ɔ{}-lɔ  amɪ-mɪnʒɪrɛ  \textbf{ɛlɛ}  a  lɪhɛ



Mala  3S-go  DEP-see    eye  at  bush



‘Mala went to see his fields.’ (lit. Mala went to see in the bush)


With the verb \textit{har} ‘carry’ (ex. 711), the addition of \textit{ɛlɛ} also gives an entirely new lexical item – expressing the idea of looking around intensively or studying every square inch (see \tabref{tab:76}.). 

\ea
Nolo  nahar  \textbf{elé}  a  gəvah  əwla  ava  jəyga.
\z

nɔ-lɔ  na-har     \textbf{ɛlɛ}   a   gəvax   =uwla    ava  dzijga



1S-go  1S-carry    eye  in  field  =1S.POSS  in  all



‘I go [and] look around my whole field.’ (lit. I carry eye in my field  all)


%%please move \begin{table} just above \begin{tabular
\begin{table}
\caption{compares examples with and without the ‘body-part.’}
\label{tab:76}
\end{table}

\begin{tabular}{ll}
\lsptoprule

\textbf{Clause without ‘body-part’} & \textbf{Clause with ‘body-part’}\\
\textit{Mana }  \textit{a-mənzar  }  \textit{war}

Mana    3S-see      child

‘Mana sees the child.’ & \textit{a-mənzar  }  \textbf{\textit{ɛlɛ}}

3S-see      eye

‘He/she looks around intently.’\\
\textit{Mana }  \textit{a-har }   \textit{ɛtɛmɛ }  \textit{a }  \textit{dəraj }  \textit{ava}

Mana   3S-carry  onion    in    head    in

‘Mana carries onions on [his] head.’ & \textit{ka-har =aka }  \textbf{\textit{ɛlɛ}}\textit{  a   }\textit{gəvax  =aŋg}\textit{\textsuperscript{w}}\textit{ɔ     ava }\textit{dzijga}

2S-carry=on    eye   in field      =2S.POSS  in   all

‘You look around your whole field.’ \\
\lspbottomrule
\end{tabular}

\begin{itemize}
\item \begin{styleTabletitle}
Selected verbs with and without the incorporation of the ‘body-part’ noun\textbf{ }ɛlɛ ‘eye’
\end{styleTabletitle}\end{itemize}
\paragraph[ɬəmaj ‘ear’]{\textit{ɬ}\textit{əmaj} ‘ear’}

A second ‘body-part’ noun is \textit{ɬ}\textit{əmaj} ‘ear’ which collocates with some cognition verbs.  This body-part noun is used in its full form. Like \textit{ɛlɛ} \textbf{ ‘}eye,’ it adds a new, more active lexical meaning to the verb with which it collocates. 

For example, the normal lexical meaning of the verb \textit{tʃɛŋ }is ‘hear’ or ‘understand’ (ex. 712) and the verb is bitransitive (see \sectref{sec:67}). The incorporation of the ‘body-part’ \textit{ɬ}\textit{əmaj} ‘ear’gives a much more active or intensive idea – not just hear and understand someone, but also listen to them or obey them (ex. 713). The focus is on the fact that the person is benefitting from using his ears to intently listen, rather than on the person speaking or the content of their message. 


\ea
Mana  écen  bay.
\z

Mana   ɛ\'{ }-tʃɛŋ     baj



Mana  3S.IFV-hear  \textsc{NEG}



‘Mana is deaf/doesn’t understand.’ 


\ea
Mana  écen  \textbf{sləmay}  bay.
\z

Mana  ɛ\'{ }-tʃɛŋ     \textbf{ɬəmaj}   baj



Mana  3S.IFV-hear  ear  \textsc{NEG}



‘Mana is deaf/disobedient.’


Examples are in \tabref{tab:77}..

\begin{tabular}{ll}
\lsptoprule

\textbf{Clause without ‘body-part’} & \textbf{Clause with ‘body-part’}\\
\textit{Mana  a-ts=aw =aka        ma             =uwla        baj}

Mana   3S-hear=1S.IO =on  word/mouth  =1S.POSS  NEG

‘Mana didn’t understand my words.’ & \textit{Mana ɛ-t}\textit{ʃ}\textit{ =aka =va      }\textbf{\textit{ɬ}}\textbf{\textit{əmaj}}\textit{ ana    mama =ahaŋ       baj}

Mana 3S-hear =on  =PERF ear    DAT mother =3S.POSS NEG

‘Mana is disobedient to his mother.’ (lit. Mana doesn’t hear ear to his mother)\\
\lspbottomrule
\end{tabular}

\begin{itemize}
\item \begin{styleTabletitle}
Selected verbs of cognition with and without  ɬəmaj ‘ear’ as direct object
\end{styleTabletitle}\end{itemize}
\paragraph[ma ‘mouth’]{\textit{ma} ‘mouth’}

The ‘body-part’ noun \textit{ma} ‘mouth’ (which also means ‘word’ and ‘language’) collocates with some speech verbs. It is found in its full form in the verb plus ‘body-part’ constructions. Ex. 714 shows the verb \textit{h-aj} ‘say’with the ‘body-part’ noun \textit{ma} ‘mouth.’ 


\ea
Tahok  ma.
\z

ta-h=ɔk\textsuperscript{w}    ma



3P-say=2S.IO    mouth



‘You are being greeted.’ (lit. they are saying word to you) 


The example pairs shown in \tabref{tab:78}. illustrate its use with three speaking verbs; \textit{tar-aj} ‘call,’ \textit{h-aj} ‘say’ and \textit{dz-aj} ‘speak.’ Examples are shown with the direct object pronominal \textit{na} (column 1) and with \textit{ma} ‘mouth’ (column 2). With the ‘body-part’ incorporation, there can be no other direct object. 

\begin{tabular}{ll}
\lsptoprule

\textbf{Transitive clause} & \textbf{Clause with ‘body-part’ incorporation}\\
\textit{Mana }  \textit{a-tar-aj}

Mana   3S-call-CL

‘Mana calls out.’

\textit{a-tar-aj}

3S-call-CL

‘He calls out.’ & \textit{Mana }  \textit{a-tar=aŋ }  \textbf{\textit{ma}}\textit{ }         \textit{ana    }Hawa  

Mana    3S-call=3S.IO   mouth/word   DAT   Hawa  

‘Mana calls to Hawa.”

\textit{a-tar=aŋ ma} 

3S-call=3S.IO mouth/word 

‘He calls to her.”\\
\textit{Mana }  \textit{a-h-aj }    \textit{baj}

Mana   3S-say-CL   NEG

‘Mana doesn’t say.’

\textit{a-h-aj }    \textit{baj}

3S-say-CL   NEG

‘He doesn’t say.’ & Mana   \textit{a-h=aŋ }    \textbf{\textit{ma}}\textit{ }         \textit{ana   }Hawa  

Mana  3S-say=3S.IO   mouth/word  DAT  Hawa  

‘Mana  greets Hawa.’

\textit{a-h=aŋ }    \textbf{\textit{ma}}\textit{ }

3S-say=3S.IO   mouth/word  

‘He greets her.’\\
\textit{Mana }  \textit{à-}\textit{dz-aj}

Mana   3S.PFV-speak-CL

‘Mana speaks!’ 

\textit{à-}\textit{dz-aj}

3S.PFV-speak-CL

‘He speaks!’ & Mana    \textit{à-}\textit{dz-aj }      \textbf{\textit{ma}}

Mana    3S.PFV-speak-CL    mouth/word

‘Mana  greets.’ 

\textit{à-}\textit{dz-aj }      \textbf{\textit{ma}}

3S.PFV-speak-CL    mouth/word

‘He greets.’\\
\lspbottomrule
\end{tabular}

\begin{itemize}
\item \begin{styleTabletitle}
Selected speech verbs with and without ma ‘mouth’ as direct object
\end{styleTabletitle}\end{itemize}

A similar creation of new lexical meaning occurs with verbs that are normally not speech verbs but that become speech verbs when they collocate with \textit{ma}. The verbs \textit{sɔk}\textit{\textsuperscript{w}}\textit{{}-ɔj} ‘point,’ \textit{zɔm} ‘eat,’ and \textit{njak-aj}  ‘find’ are shown in \tabref{tab:79}.. The incorporation of \textit{ma} with \textit{sɔk-ɔj} ‘point’ gives a particular manner of communication: \textit{sɔk}\textit{\textsuperscript{w}}\textit{{}-ɔj} \textit{ma} ‘whisper.’ Incorporation of \textit{ma} with the verb \textit{zɔm} ‘eat’ gives the idea of helping someone else to eat. Incorporation of \textit{ma} with \textit{njak-aj}  ‘find’ yields an expression to find trouble. 

\begin{tabular}{ll}
\lsptoprule

\textbf{Transitive clause} & \textbf{Clause with ‘body-part’ incorporation}\\
\textit{Hawa }  \textit{a-sɔk}\textit{\textsuperscript{w}}\textit{{}-ɔj }  \textit{ahar}

Hawa  3S-point-CL  hand

‘Hawa points.’\footnotemark{} & \textit{Hawa a-sɔk}\textit{\textsuperscript{w}}\textit{{}-ɔj }  \textbf{\textit{ma}}

Hawa  3S-point-CL  mouth/word

‘Hawa whispers.’  \\
\textit{Hawa }  \textit{ɔ{}-zɔm }  \textit{ɗaf}

Hawa   3S-eat    millet\_loaf

‘Hawa eats millet loaf.’

\textit{ɔ{}-zɔm }  \textit{na}

3S-eat    3S.DO

‘She eats it.’ & \textit{Hawa }  \textit{a-zəm=aŋ }  \textbf{\textit{ma}}\textit{ }         \textit{ana }    \textit{bahaj}

Hawa   3S-eat=3S.IO   mouth/word  DAT     chief

‘Hawa fed the chief.’ (made him eat)

\textit{a-zəm=aŋ }  \textbf{\textit{ma}}\textit{ }

3S-eat=3S.IO   mouth/word

‘She fed him.’\\
\textit{Hawa }  \textit{a-njak-aj }  \textit{asak }  \textit{=ahaŋ}

Hawa   3S-find-CL    foot   =3S.POSS

‘Hawa gives birth.’ (lit. Hawa finds her feet)\footnotemark{}

\textit{a-njak-aj na}

3S-find-CL    3S.DO

‘She finds it.’ & \textit{Hawa }  \textit{a-njak-aj }  \textbf{\textit{ma}}

Hawa   3S-find-CL    mouth/word

‘Hawa is in trouble.’ (lit. she finds mouth/word)

\textit{a-njak-aj }  \textbf{\textit{ma}}

3S-find-CL    mouth/word

‘Here comes trouble.’ \\
\lspbottomrule
\end{tabular}
\addtocounter{footnote}{-2}
\stepcounter{footnote}\footnotetext{ Perhaps \textit{ahar} ‘hand’ is another ‘body-part’ direct object that acts as semantic Theme. We found no other verbs that collocate with \textit{ahar}. }
\stepcounter{footnote}\footnotetext{ Although asak ‘foot’ is another body part, this is not a case of noun incorporation since \textit{asak} is a noun (in a possession construction with\textit{ =aha}\textit{ŋ}) and not within the verb complex as is \textit{ma} ‘mouth.’ }
\begin{itemize}
\item \begin{styleTabletitle}
Selected non-speech verbs that collocate with ma.
\end{styleTabletitle}\end{itemize}
\paragraph[va ‘body’]{\textit{va} ‘body’}

There are two different phonologically reduced forms of the word \textit{h}\textit{ə}\textit{rva} ‘body’ – \textit{va} and \textit{har}. These reduced forms are only found associated with certain verbs. When collocated with these verbs, the verb plus incorporated ‘body-part’ takes on a new lexical meaning. This is a non-productive process found with only a few verbs.   

The first reduced form of \textit{h}\textit{ə}\textit{rva} ‘body’ is \textit{va.}\footnote{Note that there are three homophones of \textit{va} which one must take care to distinguish:  [\textit{=va }] ‘perfect.’ [\textit{va} ] ‘body.’ and [\textit{ava }] ‘in’.  They all can occur immediately following the verb stem.  } This ‘body-part’ is used for forming reciprocals with plural subjects of a few verbs in a context of killing and loving (\textit{ɮ}\textit{ɪg-ɛ} ‘throw’ ex. 715 - 716, \textit{kaɗ}  ‘kill by clubbing’ ex. 717, and \textit{ndaɗ-aj}  ‘need,’ ex. 718). The ‘body-part’ \textit{va}  indicates that the plural subjects are performing the actions against one another. 


\ea
Tandalay  talala  təzləgə  \textbf{va}  ana  Məloko  ahay.
\z

ta-nd-alajta-l =ala  tɪ-ɮɪg-ə    \textbf{va}  ana  Mʊlɔk\textsuperscript{w}ɔ    =ahaj



3P-PRG   3P-go =to  3P-throw   body  DAT  Moloko    =Pl



‘They were coming and fighting with the Molokos.’ (lit. they were coming they threw body to Molokos)


\ea
Kafta  məze  ahay  təzləgə  \textbf{va}  va  na,  nəwəɗokom  ala  dəray.
\z

kafta  mɪʒɛ  =ahaj  tə-ɮəg-ə    \textbf{va}  =va  na  nu-wuɗɔk\textsuperscript{w}{}-ɔm     =ala  dəraj



day       person  =Pl  3P-throw      body  =\textsc{PRF}  PSP  1P-separate-1\textsc{Pex} =to  head



‘On the day that they had finished fighting each other, we separated as equals.’


\ea
Takaɗ  \textbf{va}\textbf{.}
\z

ta-kaɗ   \textbf{va}



3P-kill  body



‘They kill each other.’ (lit. they kill-by-clubbing body)


The ‘body-part’ \textit{va} ‘body’occurs twice in the clause expressing the reciprocal idea of loving one another in ex. 718 – as incorporated noun and also as the noun phrase within an adpositional phrase (\textit{va} is bolded in the example).

\ea
Kondoɗom  \textbf{va}  a  \textbf{va}  ava.
\z

kɔ-ndɔɗ-ɔm    \textbf{va}  a  \textbf{va}  ava



2P-need-2P    body  in  body  in



‘Love one another.’ (lit. need body in the body)


%%please move \begin{table} just above \begin{tabular
\begin{table}
\caption{compares transitive clauses with a direct object and clauses with the same verbs collocated with the ‘body-part.’ To facilitate comparison between the incorporated ‘body-part’ \textit{va}  and the direct object pronominal extension \textit{na}, the examples in the table are given in pairs. The first example in each pair shows the full noun phrase, and the second example in the pair shows the same clause with only pronominal affixes and extensions. The ‘body-part’ \textit{va} is bolded.}
\label{tab:80}
\end{table}

\begin{tabular}{ll}
\lsptoprule

\textbf{Transitive clause} & \textbf{Clause with ‘body-part’ incorporation}\\
\textit{Mʊl}\textit{ɔk}\textit{\textsuperscript{w}}\textit{ɔ }   \textit{=ahaj  }  \textit{tə-ɮɪg-ɛ  }  \textit{haj}

Moloko     =PL        3P-sow-CL    millet

‘Moloko people sow/throw millet.’

\textit{tə-ɮɪg-ɛ  }  \textit{na}

3P-sow-CL    3S.DO

‘They sow/throw it.’ & \textit{kəra =ahaj tə-ɮɪg-ɛ      }\textbf{\textit{va}}

dog   =PL    3P-sow-CL  body

‘Dogs fight each other.’

\textit{tə-ɮɪg-ɛ      }\textbf{\textit{va}}

3P-sow-CL  body

‘They fight each other.’\\
\textit{babəza  }  \textit{=ahaj  }  \textit{ta-ka}\textit{ɗ}\textit{  }  \textit{kəra}

children   =PL    3P-club    dog

‘The children kill a dog.’

\textit{ta-ka}\textit{ɗ}\textit{  }    \textit{na}

3P-club    3S.DO

‘They kill it.’ & \textit{mɪʒɛ =ahaj  ta-ka}\textit{ɗ}\textit{   }\textbf{\textit{va}}

person =PL   3P-club  body

‘The people kill each other.’

\textit{ta-ka}\textit{ɗ}\textit{    }\textbf{\textit{va}}

3P-club   body

‘They kill each other.’\\
\textit{lɔk}\textit{\textsuperscript{w}}\textit{ɔ }  \textit{na    kɔ-ndɔɗ-ɔm  }        \textit{baba   =alɔk}\textit{\textsuperscript{w}}\textit{ɔ}

1PIN   PSP  1PIN-love-1PIN  father   =1PIN.POSS

‘We (for our part) love our father.’

\textit{kɔ-ndɔɗ-ɔm  }    \textit{na}

1PIN –love-1PIN    3S.DO

‘We love him.’ & \textit{lɔk}\textit{\textsuperscript{w}}\textit{ɔ }  \textit{na }  \textit{kɔ- ndɔɗ-ɔm }     \textbf{\textit{va}}

1PIN    PSP    1PIN –love-1PIN     body

‘We (for our part) love one another.’

\textit{kɔ- ndɔɗ-ɔm }          \textbf{\textit{va}}

1PIN –love-1PIN    body

‘We love one another.’\\
\lspbottomrule
\end{tabular}

\begin{itemize}
\item \begin{styleTabletitle}
Selected verbs with and without the ‘body-part’ va ‘body’
\end{styleTabletitle}\end{itemize}

The verb \textit{zaɗ}  ‘take’ also can incorporate the ‘body-part’ \textit{va}  ‘body.’ The normal lexical meaning of the verb \textit{zaɗ}  is ‘take’ but the combination \textit{zaɗ} \textit{va} (ex. 719 and 720) carries the idea of ‘resemble’ or ‘look like’ and occurs with singular as well as plural subjects. With a plural subject (ex. 720), the clause has a reciprocal idea – the subjects resemble each other. 


\ea
Məlama  ango  azaɗ  \textbf{va}  nə  nok.
\z

məlama   =aŋg\textsuperscript{w}ɔ     a-zaɗ   \textbf{va}  nə  nɔk\textsuperscript{w}



sibling  =2S.POSS  3S-take  body  with  2S



‘Your sibling resembles you.’ (lit. your sibling takes body with you)


\ea
Məlama  ango ahay  jəyga  tazaɗ  \textbf{va}\textbf{.}
\z

məlama   =aŋg\textsuperscript{w}ɔ     =ahaj   dʒijga   ta-zaɗ   \textbf{va}



sibling  =2S.POSS  =Pl  all  3P-take  body



‘All your siblings look alike.’ (lit. siblings take [each other’s] body)


The body part\textit{ }\textit{va } can also collocate with other verbs. For example \textit{ɛ{}-mbeʃɛŋ} means ‘he/she breathes,’ but \textit{ɛ{}-mbeʃɛŋ} \textit{va} means ‘he/she is resting’ (ex. 721).

\ea
Embesen  va  kə  cəveɗ  aka.
\z

ɛ{}-mbɛʃɛŋ  va  kə  tʃɪvɛɗ  aka



3S-breathe  body  on  road  on



‘He rests enroute (to somewhere).’


\paragraph[har ‘body’]{\textit{har} ‘body’}

A second reduced form of \textit{hərva}, \textit{har} ‘body,’ demonstrates another non-productive collocation with some verbs. With the verb \textit{wu}\textit{ɗ}\textit{ak-aj}, which normally means ‘divide,’ the incorporation of \textit{har} gives a new lexical meaning containing the idea of the participants ‘dividing themselves up’ (a reflexive meaning, ex. 722). 


Values S. 16



\ea
Təlala,  a  həlan  ga  ava  ese,  təwəɗakala  \textbf{har}  a  məsyon  ava.
\z

tə-l=ala        a   həlaŋ  ga   ava   ɛʃɛ,   tu-wuɗak=ala   \textbf{har}   a   mʊsjɔŋ   ava



3P-go=to    to  back  ADJ  in  again  3P-divide=to     body  in  mission  in



‘Later, after (lit. at the back of) again, they separate [to go home] from the mission (lit. they divide body).’


With the verb \textit{gas} which normally means ‘catch,’ \textit{har}  gives the lexical idea of pleasing, which is located at the indirect object (ex. 723).

\ea
Membese  va  nə  nok  egəne  na,  agəsaw  \textbf{har}  ava  gam.
\z

mɛ-mbɛʃ-ɛ     va  nə  nɔk\textsuperscript{w}  ɛgɪnɛ  na  a-gəs=aw  \textbf{har}  =va  gam



\textsc{NOM}{}-breathe-CL  body  with  2S  today  PSP  3S-catch=1S.IO  body  =PERF  a lot



‘Spending time with you today pleased me a lot.’ (lit. it catches body to me) 


\section{Clauses with zero grammatical arguments}
\hypertarget{RefHeading1212781525720847}{}
Sometimes in Moloko verbal clauses built around any verb can also carry zero transitivity – clauses where there are no grammatically explicit participants.\footnote{The ideophone clause can also carry zero transitivity \citep{Section1131}. } Nominalised and dependent verb forms are not inflected for subject (see Sections 7.6 and 7.7, respectively). When they also carry no DO or indirect object pronominal, the clause has zero transitivity. The use of verb forms with no grammatical relations has a discourse function to temporarily take participants out of sight. In the Disobedient Girl story peak episode S. 21 (ex. 724), the dependent verb \textit{amə-h=aja} ‘grinding,’ is unconjugated for subject, direct object, and indirect object. The effect is to keep the participants out of sight as the events unfold and increase vividness as the audience is drawn into the story. All the audience hears is the sound of grinding. The millet is expanding, filling the room and the disobedient girl is lost inside it as she is being suffocated by the millet.  


Disobedient Girl S. 21



\ea
Njəw  njəw  njəw  aməhaya  azla.
\z

nzuw  nzuw  nzuw    amə-h=aja        aɮa



\textsc{ID}grind                        DEP-grind=\textsc{PLU}    now



‘Nzu, nzu, nzu [she] ground [the millet] now.’  


Likewise in line S. 15 of the Snake story (ex. 725), the nominalised form of the verb ‘to penetrate’ occurs with neither DO nor indirect object pronominals. This climax moment where the storyteller spears the snake is in a clause with zero transitivity. Participants are out of sight in the discourse. 


Snake story S. 15


\ea
Mecesle  mbəraɓ!
\z

\textit{mɛ-tʃɛɬ-ɛ            mbəraɓ}



\textsc{NOM}{}-penetrate{}-CL      \textsc{ID}penetrate 



‘It penetrated, \textit{mburab}!’


\chapter[ ]{ }
\hypertarget{RefHeading1212801525720847}{}
\chapter[Clause]{Clause}
\hypertarget{RefHeading1212821525720847}{}
Moloko is a SVO language, which means that the default order of clausal constituents in a simple clause is subject, followed by verb (or predicate), and finally object.\footnote{Elements can be fronted only in a special construction described in Chapter 12.}  Clause types in Moloko cannot be very meaningfully discussed apart from verb types, which have been described in Chapter 9. In this chapter the basic structure of declarative clauses for all verb subclasses will be discussed (\sectref{sec:11.1}) followed by the negation constructions (section 11.2), interrogative constructions (\sectref{sec:11.3}), imperative constructions (\sectref{sec:11.4}), and exclamatory constructions (\sectref{sec:11.5}). The constructions discussed in this chapter are each monoclausal, but are specific constructions superimposed on a clause to add a functional element. The \textit{na} construction is also a construction that can be superimposed on a single clause. However since it is more complex, \textit{na} constructions is discussed in a separate chapter (Chapter 12). Clause combining is discussed in \sectref{sec:13.}

\section{Declarative clauses}
\hypertarget{RefHeading1212841525720847}{}
Moloko has two basic types of declarative clauses, depending on whether the clause contains a verb or not. The verbal clause is described in \sectref{sec:69.} Clauses where an existential or an ideophone is the central element are a subtype of verbal clauses.  The special features of their structure have already been discussed in Chapters 3.4 and 3.6, respectively. The non-verbal clauses are described in \sectref{sec:70.} These include predicate nominal, predicate adjective, and predicate possessive clauses.  

There is not a lot of variation in the word order of the elements of the basic clause, but the number of grammatically explicit core participants controls the semantic roles assigned to the subject, direct object, and indirect object (Chapter 9). The presupposition construction (discussed in Chapter 12) can be superimposed upon the basic clause structure, changing the word order. Negation, interrogative, command, and exclamatory clause structures can be further superimposed (Chapters 11.2 and Error: Reference source not found). 

\subsection{  Verbal clause}
\hypertarget{RefHeading1212861525720847}{}
The basic structure of Moloko verbal clauses includes the following elements in the order shown in \figref{fig:16}.. The order of clause constituents for all clause types is always SVO (with V and O being within the verb phrase). The verb phrase (Chapter 8) is the centre of the clause (and also its final element) and can contain information concerning the subject, direct object, indirect object, aspect, mood, direction, location, repetition, and discourse-importance of the event or state expressed by the verb (see Sections 7.3 - 7.5).  Every Moloko verbal clause has a verb complex, and may consist simply of a verb complex. All other elements are optional. The temporal adverb gives locational information concerning the event. If a full subject noun phrase is present, it precedes the verb phrase, and any other core clause constituents follow the verb in the verb phrase (direct object, indirect object, obliques). The subject controls the subject inflections on the verb word. Elements whose inclusion in the clause is optional are in parentheses.

\begin{tabular}{l}
\lsptoprule

(temporal noun phrase)    (subject noun phrase)\textbf{ }    \textbf{Verb phrase}\\
\lspbottomrule
\end{tabular}

\begin{itemize}
\item \begin{styleFiguretitle}
Order of constituents for verbal clause
\end{styleFiguretitle}\end{itemize}

The first element in the clause can be a temporal noun phrase (ex. 619, 726).


\ea
\textbf{Apazan  }albaya  ahay  tolo  a  ləhe.
\z

\textbf{apazaŋ}  albaja    =ahaj    tɔ-lɔ    a  lɪhɛ



yesterday  youth    =Pl      3P.PFV-go  at  bush



‘Yesterday the youths went to the bush.’ 


The subject is expressed by the subject pronominal on the verb (see \sectref{sec:49}). A coreferential noun phrase can be present for discourse functions (ex. 727 and 728). The coreferential noun phrase precedes the verb.  

\begin{itemize}
\item \begin{styleExampleteference}
\textbf{Hawa}  ahəmay.
\end{styleExampleteference}
\end{itemize}
\begin{styleExampleteference}
\textbf{Hawa}   à-həm-aj
\end{styleExampleteference}


Hawa  3S.PFV-run{}-CL



‘Hawa ran.’


\begin{itemize}
\item \begin{styleExampleteference}
\textbf{Ne  ahan}  nozom  na.
\end{styleExampleteference}
\end{itemize}
\begin{styleExampleteference}
\textbf{n}\textbf{ɛ}\textbf{   =ahaŋ}     nɔ-zɔm    na
\end{styleExampleteference}


1S  =3S.POSS  3S.PFV-eat  3S.DO



‘I myself ate it.’


The simplest form of the verbal clause type consists of a verb complex only. A verb complex can stand alone as a clause because, in addition to the verb stem, it contains information on grammatical relations (subject in the subject prefix, direct object and indirect object in a verb extension or suffix). The verb complex also includes directional and (non-core) locational information and indicates whether or not the verb is Perfect (via a verbal extension). It is interesting that the SVO order is maintained in the affixes (s-v-o), as seen in \figref{fig:12}. (from \sectref{sec:7.1}).

The examples below are clauses consisting of just a verb complex. They all have information on the subject (from subject inflections, ex. 729, 731, 732, 733) or the form of the imperative (ex. 730 and 734). Some have information on the direct object (ex. 731{}-734), indirect object (ex. 733), direction of the action (ex. 730, 732, 734), and discourse information (ex. 730).  

\begin{itemize}
\item \begin{styleExampleteference}
Nəhəmay.
\end{styleExampleteference}
\end{itemize}
\begin{styleExampleteference}
nə-həm-aj
\end{styleExampleteference}


1S.PFV-run{}-CL



‘I ran.’


\ea
Dəraka  alay. 
\z

dər    =aka  =alaj 



move  =on  =away



‘Move over again!’


\begin{itemize}
\item \begin{styleExampleteference}
Nozom  na.
\end{styleExampleteference}
\end{itemize}
\begin{styleExampleteference}
nɔ-zɔm    na
\end{styleExampleteference}


1S.PFV-eat    3S.DO



‘I ate it.’


\ea
Nabah  na  alay. 
\z

nà-bax     na   =alaj 



1S.PFV-pour DO  =away



‘I poured it away from myself.’


\begin{itemize}
\item \begin{styleExampleteference}
Nəvəlan  na.
\end{styleExampleteference}
\end{itemize}
\begin{styleExampleteference}
nə-vəl=aŋ    na
\end{styleExampleteference}


1S.PFV-give=3S.IO  3S.DO



‘I gave it to him.’


\begin{itemize}
\item \begin{styleExampleteference}
Zaw  na  ala.
\end{styleExampleteference}
\end{itemize}
\begin{styleExampleteference}\ z    =aw   na   =ala
\end{styleExampleteference}


give 2S.IMP  =1S.IO  3S.DO  =to



‘Give it to me!’


\subsection{  Predicate nominal, predicate adjective, and predicate possessive clauses}
\hypertarget{RefHeading1212881525720847}{}
Predicate nominal (ex. 735 - 737), predicate adjective (ex. 738), and predicate possessive (ex. 739 and 740) clauses lack any verb and consist of a juxtaposition of two noun phrases, in an order shown in \figref{fig:17}.. 

\begin{tabular}{l}
\lsptoprule

Subject noun phrase     Predicate noun phrase\\
\lspbottomrule
\end{tabular}

\begin{itemize}
\item \begin{styleFiguretitle}
Constituent order of predicate nominal/ adjective/ possessive clauses
\end{styleFiguretitle}\end{itemize}

Predicate nominal clauses typically express the notions of proper inclusion, i.e., the clause indicates that the subject is a member of the particular class of items indicated by the predicate (ex. 735); or equation, i.e., the clause indicates that the subject is identical to the predicate (ex. 736 and 737). In the following examples, each noun phrase is delimited by square brackets.


\ea
 [Mana ]  [zar  mehere.]
\z

{}[Mana ]    [zar  mɛ-hɛr-ɛ ]



Mana    man  \textsc{NOM}{}-build-CL



‘Mana [is] a builder.’ (lit.Mana, building-man)


\ea
{}[Sləmay  əwla ]  [Abangay.]
\z

{}[ɬəmaj   =uwla  ]  [Abaŋgaj ]



name  =1S.POSS  Abangay



‘My name [is] Abangay.’


\ea
{}[Zar  nehe ]  [baba  əwla.]
\z

{}[zar   nɛhɛ ]    [baba   =uwla ]



man    DEM    father  =1S.POSS



‘The man [is] my father.’ 


Predicate adjective clauses consist of a subject noun phrase and a derived adjective (Chapter 5.3) as the predicate noun phrase. These clauses express an attribute of the subject.

\ea
{}[Ndahan ]  [malan  ga.]
\z

{}[ndahaŋ ]  [malaŋ     ga ]



3S    largeness  ADJ



‘He/she [is] big.’


Predicate possessive clauses have a subject noun phrase and a possessive prepositional phrase (see \sectref{sec:45}) as the predicate phrase. The participant named in the possessive phrase is expressed via a full noun phrase. These clauses express that the subject noun phrase is associated with the participant named in the possessive phrase. The semantic range for the predicate possessive clauses is the same as that of any possessive or genitive construction (see Sections 141 and 42).

\ea
 [Babəza  ahay  nəndəye]  [\textbf{anga}  bahay. ]
\z

{}[babəza  =ahaj  nɪndijɛ ]    [\textbf{aŋga}  bahaj ]



children  =Pl  DEM    \textsc{POSS}  chief



‘The children here belong to the chief.’ / ‘The children here[are] belonging to the chief.’ 


\ea
{}[Dəray  ga ]  [\textbf{anga}  ləme.]
\z

{}[dəraj  ga ]    [\textbf{aŋga}  lɪmɛ ]



head  ADJ    \textsc{POSS}  1\textsc{Pex}



‘The head belonged to us.’/ ‘The head [was] belonging to us.’ 


For all three of these clause types, the subject may be marked as presupposed (see \sectref{sec:12.2}). For a predicate nominal construction, fronting and marking the predicate with \textit{na} can express inclusion (ex. 735), equation (ex. 741{}-742) or attribution (ex. 738 and 743).

\ea
{}[Zar  mehere   na ],  [Mana.]    
\z

{}[zar    mɛ-hɛr-ɛ ]    na  [Mana ]    



man   \textsc{NOM}{}-build-CL   PSP  Mana        



‘The builder [is] Mana.’ 


\ea
{}[Bahay  a  Laway  na ],  [Ajəva.]
\z

{}[bahaj   a   Lawaj     na  ]  [Adzəva ]



chief  GEN  Lalaway    PSP  Adzava



‘The chief of Lalaway [is] Adzava.’


\ea
{}[Malan  ga   na ],  [ndahan.]
\z

\textit{[}malaŋ   ga\textit{ ]    na  [}ndahaŋ\textit{ ]}


largeness  ADJ    PSP  3S



‘The biggest one [is] him.’ (lit. big, him)


\section{Negation constructions}
\hypertarget{RefHeading1212901525720847}{}
Negation constructions are specific constructions superimposed on a clause to create negation of the entire proposition (clausal negation construction, see \sectref{sec:72}) or negation of one element of the clause (constituent negation, see \sectref{sec:73}). For both, Moloko uses a negative particle \textit{baj} at the end of the clause (see \sectref{sec:71}).

\subsection{Negative particles}
\hypertarget{RefHeading1212921525720847}{}
The all-purpose negative is the particle \textit{baj}\textit{,} which occurs clause-finally (ex. 745, 746) but before any interrogative word (see \sectref{sec:16}). In ex. 744 - 746 the negative is bolded and the negated element is in square brackets. 


\ea
{}[Alala  \textbf{ba}\textbf{y.} ]
\z

{}[à-l    =ala    \textbf{baj} ]



3S.PFV-go  =to    \textsc{NEG}



‘He/she didn’t come.’


\ea
{}[War  ga  ecen  sləmay  \textbf{ba}\textbf{y.}]
\z

{}[war  ga  ɛ{}-tʃɛŋ  ɬəmaj  \textbf{baj} ]



child  ADJ  3S-hear  ear  \textsc{NEG}



‘That child did not obey.’ (lit. that child, he hears ear not)


\ea
{}[Táazləgalay  avəlo  \textbf{ba}\textbf{y.}]
\z

{}[táá-ɮəg    =alaj    avʊlɔ  \textbf{baj} ]



3P.POT-throw   =away    above  \textsc{NEG}



‘They should not throw it too high.’


In ex. 747{}- 748 the negative is clause final and may have sematic scope over the entire proposition. See especially ex. 748 where it is clear that the entire proposition is being negated, and not just the information within the constituent closest to the negative. The meaning is ‘don’t insult a small person.’ If the information in only one constituent was being negated, the meaning would have been ‘insult a person who is not small.’

\ea
{}[Tagaw  ele  lala  \textbf{ba}\textbf{y.}]
\z

{}[ta-g=aw    ɛlɛ   lala  \textbf{baj} \textbf{ }]



3P-do=1S.IO   thing  good  \textsc{NEG}                    



‘They do bad things to me.’ / ‘They don’t do good things to me.’ 


\ea
{}[Kárasay  məze  cəɗew  ga  \textbf{bay.}]
\z

{}[ká-ras-aj     mɪʒɛ   tʃɪɗɛw     ga   \textbf{baj} ]



2S.IFV-minimise{}-CL  person  smallness  ADJ  \textsc{NEG}



‘Don’t insult one of the little people.’ 


\begin{itemize}
\item 
{}[\textit{Anday  dəren  }\textbf{\textit{ba}}\textbf{\textit{y.}}]
\end{itemize}

{}[à-ndaj    dɪrɛŋ  \textbf{baj} \textbf{ }]



3S.PFV-PRG  far  \textsc{NEG}



‘He/she was not far.’  


In ex. 750, \textit{baj} follows a noun phrase within the clause and negates the information expressed within the noun phrase itself; \textit{ɛlɛ lala }\textbf{\textit{baj}} ‘a bad thing.’ 

\ea
Nde  [ele  lala  \textbf{bay }]  kə  təta  aka.
\z

ndɛ     [ɛlɛ   lala    \textbf{baj }]   kə   təta   aka



so    thing  well done  \textsc{NEG}  on  them  on



‘So, a bad thing was upon them.’ 


When relative clauses are negated, the negative may have semantic scope over the entire relative clause (ex. 751, 752). 


Values S. 7


\ea
Ele  ahay  [aməgəye  \textbf{bay} ]  nəngehe  pat  tahata  na  va.  
\z

ɛlɛ     =ahaj  [amɪ-g-ijɛ   \textbf{baj} ]  nɪŋgɛhɛ  pat  ta-h=ata  na  =va  



thing  =Pl  DEP-do-CL  \textsc{NEG}  DEM  all  3P-tell=3P.IO  3S.DO  =\textsc{PRF}



‘All these things that [we] are not supposed to do, they have already told us.’  


\ea
Kəra  [aməmənjere  elé  \textbf{bay}]  táslay  na  gəraw.~
\z

kəra    [amɪ-mɪndʒɛrɛ  ɛlɛ  \textbf{baj} ]  tá-ɬ{}-aj    na  gəraw~



dog    3S-DEP-see  eye  \textsc{NEG}  3P.IFV-slay{}-CL    3S.DO  \textsc{ID}cut through middle



‘The dog that couldn’t see they slew it through the middle.’


The negative can form a compound with some adverbs. \tabref{tab:81}. shows negated and non-negated clauses with four adverbs. The negative \textit{asa-baj} ‘never again’ is a compound of  the adverb \textit{ɛʃɛ}  ‘again’ and \textit{baj}. The evidence of phonological binding is that the adverb \textit{ɛʃɛ}  loses its palatalisation when it compounds with \textit{baj } (line 1 in \tabref{tab:81}.). The other adverbs are considered to be separate phonological words since there are no other indications that the negative is phonologically bound to the adverb since the prosody of other adverbs is not affected (\textit{kʊlɔ} ‘before,’ line 3 in \tabref{tab:81}.). There are word-final changes between the adverb \textit{faŋ} ‘already’ and the negative. For example, line 2 shows that the word-final /n/ in [faŋ] is deleted (a word-final change prompted in some contexts, see \sectref{sec:11}). 

\begin{tabular}{lll}
\lsptoprule

\textbf{Line} & \textbf{Non-negated clause with adverb} & \textbf{Negated clause}\\
1 & \textit{nɔɔ-lɔ       }\textbf{\textit{ɛʃɛ}}

1S.POT-go  \textit{  }again

‘I will go again.’ & \textit{nɔɔ-lɔ       }\textbf{\textit{asa-baj}}

1S.POT-go  \textit{  }again\textit{  }\textsc{NEG}

‘I will not go again.’\\
2 & \textit{nɛ-g-ɛ    na  }\textbf{\textit{faŋ}}

1S.IFV-do-CL  3S.DO  already

‘I have done it already.’ & \textit{nɛ-g-ɛ    na  }\textbf{\textit{fa}}\textbf{\textit{ŋ}}\textit{  }\textbf{\textit{baj}}

1S.IFV-do-CL  3S.DO  already\textit{  }\textsc{NEG}

‘I haven’t done it yet.’\\
3 & \textit{nə-mənzar    ndahaŋ  }\textbf{\textit{kʊlɔ}}

1S-see    3S  before

‘I have seen her before.’ & \textit{nə-mənzar    ndahaŋ  }\textbf{\textit{kʊlɔ}}\textit{  }\textbf{\textit{baj}}

1S-see  \textit{  }3S  before\textit{  }\textsc{NEG}

‘I have never seen her before.’  \\
4 & \textit{káá-z  =ala    }\textbf{\textit{təta}}

2S.POT-take  =to  ABILITY

‘You can bring [it].’ & \textit{káá-z  =ala    }\textbf{\textit{təta}}\textit{     }\textbf{\textit{baj}}

2S.POT-take  =to  ABILITY\textit{   }\textsc{NEG}

‘You can’t bring [it].’  \\
\lspbottomrule
\end{tabular}

\begin{itemize}
\item \begin{styleTabletitle}
Negation of clauses with adverbs
\end{styleTabletitle}\end{itemize}
\subsection{Clausal negation construction}
\hypertarget{RefHeading1212941525720847}{}
For clausal negation, there is no change in word order and no change in clause constituents. A negative clause asserts that some event or state does not hold. Ex. 753 - 768  illustrate various types of clausal negation in Moloko. Each pair of examples represents a positive and a negative assertion for comparison. 

Ex. 753 and 754 illustrate negation of an intransitive clause:


\ea
Ahəmay.           
\z

a-həmaj           



3S-run               



‘He/she runs.’            


\ea
Ahəmay  \textbf{ba}\textbf{y.}
\z

a-həmaj     \textbf{baj}



3S-run    \textsc{NEG}



‘He/she doesn’t run.’


Ex. 755 - 760 illustrate negation of a transitive clause:

\ea
Amənjar  Hawa.        
\z

a-mənzar   Hawa        



3S-see  Hawa             



‘He/she sees Hawa.’           


\ea
Amənjar  Hawa  \textbf{bay.}
\z

a-mənzar   Hawa   \textbf{baj}



3S-see  Hawa  \textsc{NEG}



‘He/she doesn’t see Hawa.’


\ea
Akaɗ  awak.        
\z

a-kaɗ   awak        



3S-kill  goat            



‘He/she kills a goat.’          


\ea
Akaɗ  awak  \textbf{bay.}
\z

a-kaɗ   awak   \textbf{baj}



3S-kill  goat  \textsc{NEG}



‘He/she doesn’t kill a goat.’


\ea
Asaw sese.          
\z

a-s=aw  ʃɛʃɛ          



3S-divide-1S  meat             



‘I want meat.’            


\ea
Asaw  sese  \textbf{ba}\textbf{y.}
\z

a-s=aw     ʃɛʃɛ   \textbf{baj}



3S-divide=1S.IO   meat  \textsc{NEG}



‘I do not want meat.’


Ex. 761 - 764 illustrate negation of existentials. 

\ea
Babəza  əwla  ahay  aba.  
\z

babəza  =uwla    =ahaj  aba  



children  =1S.POSS  =Pl  \textsc{EXT}      



‘I have children.’            


\ea
Babəza  əwla  ahay  \textbf{abay.}
\z

babəza  =uwla    =ahaj  \textbf{abaj}



children  =1S.POSS  =Pl  \textsc{EXT} \textsc{NEG}



‘I have no children.’  


\ea
Dala  anaw  aka.     
\z

dala   an=aw     aka     



money  DAT=1S  \textsc{EXT}+on       



‘I have money.’           


\ea
Dala  anaw  aka\textbf{  ba}\textbf{y.}
\z

dala   an=aw     aka\textbf{    baj}



money  DAT=1S  \textsc{EXT}+on    \textsc{NEG}



‘I have no money.’


Ex. 765 - 768 illustrate negation of a predicate adjective. 

\ea
Ndahan  zləle  ga.      
\z

ndahaŋ   ɮɪlɛ   ga      



3S    richness  ADJ        



‘He/she is rich.’           


\ea
Ndahan  zləle  ga  \textbf{ba}\textbf{y.}
\z

ndahaŋ  ɮɪlɛ   ga   \textbf{baj}



3S    richness  ADJ  \textsc{NEG}



‘He/she is not rich.’


\ea
Ndahan  gədan  ga.       
\z

ndahaŋ   gədaŋ   ga       



3S    strong  ADJ           



‘He/she is strong.’            


\ea
Ndahan  gədan  ga  \textbf{ba}\textbf{y.}
\z

ndahaŋ   gədaŋ   ga   \textbf{baj}



3S    strength  ADJ  \textsc{NEG}



‘He/she is not strong.’


\subsection{Constituent negation}
\hypertarget{RefHeading1212961525720847}{}
Most frequently, it seems that the element closest to the negative that is under the scope of negation, even though a clause-final negative marker can have scope over the whole verb phrase or even over the entire clause. To negate only one constituent in a clause, the clause is sometimes rearranged so that the constituent that is negated is placed in the clause-final position adjacent to the negation particle. Ex. 769 - 771 show a question (ex. 769) with two responses (ex. 770 - 771) where each of the two ambiguous elements is negated. The subject (Mana) is part of the presupposition (marked off by \textit{na} in the question, see \sectref{sec:12.2}). In ex. 770 the oblique is negated and in ex. 771 the entire predicate. The clauses were not restructured since the elements in question were already clause-final. In the following examples, the element that is negated is delimited by square brackets and the negative is bolded.


\ea
Mana  na,  olo  [a  kosoko  ava ]  ɗaw?
\z

Mana   na   ɔ{}-lɔ   [a   kɔsɔk\textsuperscript{w}ɔ   ava ]  ɗaw



Mana  PSP  3S-go  to  market  in  QUEST



‘As for Mana, is he going to the market?’


\ea
Ehe,  olo  [a  kosoko  ava ]  \textbf{bay;  o}lo  afa  bahay.
\z

ɛhɛ     ɔ{}-lɔ   [a   kɔsɔk\textsuperscript{w}ɔ   ava ]  \textbf{baj}    ɔlɔ   afa     bahaj



no    3S-go  to  market  in  \textsc{NEG}  3S-go  house of    chief



‘No, he isn’t going to the market; rather he is going to the chief’s house.’


\ea
Ehe,  olo  [a  kosoko  ava ]  \textbf{bay;  }enjé  a  mogom.
\z

ɛhɛ,   [ɔ-lɔ   a   kɔsɔk\textsuperscript{w}ɔ   ava ]  \textbf{baj}   ɛ{}-nʒ-ɛ     a   mɔg\textsuperscript{w}ɔm



no    3S-go  in  market  in  \textsc{NEG}  3S-stay-CL  at  home



‘No, he isn’t going to the market; rather he is staying at home (or going to the chief’s house).’


Ex. 772 - 775 show some restructuring when different constituents are negated. Ex. 772 illustrates a question and 773 to 775 illustrate three possible answers, each negating a different constituent. Normal SVO structure is maintained for all answers. The responses each use two clauses. The first clause expresses the negation of the element in final position, and the second restates the clause giving the corrected information. In each case the first clause is restructured so as to move the element to be negated to the clause-final position. The response in ex. 773 indicates that the hearer accepts that Mana gave the guitar to someone, but it was not his father. In this clause, \textit{kɪndɛw} ‘guitar’ is realised as the 3S DO pronominal. The response in ex. 774 indicates that Mana gave something to his father, but not a guitar. In this case, the adpositional phrase \textit{ana baba =ahaŋ }‘to his father’ is replaced by the indirect object pronominal so that the negated element \textit{kɪndɛw} ‘guitar’ can be placed next to the negative. 

\ea
Mana  àvəlan  kəndew  ana  baba  ahan  ɗaw?
\z

Mana  à-vəl =aŋ     kɪndɛw  ana   baba  =ahaŋ    ɗaw



Mana     3S.PFV-give=3S.IO  guitar  DAT  father  =3S.POSS  QUEST



‘Did Mana give the guitar to his father?’


\ea
Ehe,  àvəlan  na  [ana  baba  ahan]  \textbf{bay};  àvəlan  na  ana  gəmsodo  ahan.
\z

ɛhɛ    à-vəl =aŋ   na  [ana   baba  =ahaŋ]     \textbf{baj},



no    3S-give=3S.IO  3S.DO  DAT  father  =3S.POSS  \textsc{NEG}



\textit{à-vəl =aŋ     na  ana   g}\textit{ʊ}\textit{msɔdɔ    =ahaŋ }



3S-give=3S.IO  3S.DO  DAT  mother’s brother  =3S.POSS    



‘No, Mana didn’t give it to his father, he gave it to his mother’s brother.’


\ea
Ehe,  àvəlan  [kəndew ]  \textbf{bay},  àvəlan  cecewk.
\z

ɛhɛ    à-vəl =aŋ   [kɪndɛw ]   \textbf{baj},



no    1S-give=3S.IO  guitar    \textsc{NEG}



‘No, Mana didn’t give a guitar to his father,



\textit{à-vəl =aŋ    tʃɛtʃ}\textit{œ}\textit{k}\textit{\textsuperscript{w}}



1S-give=3S.IO  flute



‘he gave him a flute.’ 


The fourth possible reply to the question in ex. 772 negates the subject. Moloko clause structure does not allow the subject to occupy the clause final position; to specifically negate the subject of a clause (ex. 777), a predicate nominal clause structure is used. The predicate is recast as a relative clause (see \sectref{sec:44}) with the presupposed information that someone gave a guitar to his father marked with \textit{na}. The nominal is the negated subject \textit{Mana baj} ‘not Mana.’ 

\ea
Ehe,  aməvəlan  kəndew  ana  baba  ahan  na,  [Mana ]  bay;  aməvəlan  na,  Majay.
\z

ɛhɛ    amə-vəl=aŋ  kɪndɛw  ana  baba  =ahaŋ    na, [Mana ]  baj



no    DEP-give=3S.IO  guitar  DAT  father  =3S.POSS  PSP Mana     \textsc{NEG}



‘No, Mana didn’t give the guitar to his father.’ (lit. the one that gave guitar to his father, not Mana)



\textit{amə-vəl=aŋ    na  Madzaj}



DEP-give=3S.IO  PSP  Madzay



‘The person that gave [it was] Madzay.’ 


Ex. 776 - 777 show a similar restructuring of a verbal clause into a predicate nominal in order to negate the subject of a clause. Ex. 776 shows a question with a verbal clause structure. In order to negate the subject, the clause is restructured to put all of the known information in a predicate that is a relative clause delimited by\textit{ na,} and the negated subject becomes the final nominal (ex. 777). 

\ea
Hawa  àdan  ɗaf  ana  Mana  ɗaw?
\z

Hawa   à-d  =aŋ    ɗaf  ana  Mana  ɗaw



Hawa  3S.PFV-prepare=3S.IO  loaf  DAT  Mana  QUEST



‘Did Hawa prepare food for Mana?’


\ea
Amadan  ɗaf  ana  Mana  na,  [Hawa ]  \textbf{ba}\textbf{y.}
\z

ama-d=aŋ    ɗaf  ana  Mana  na  [Hawa ]   \textbf{baj}



DEP-prepare=3S.IO  loaf  DAT  Mana  PSP  Hawa   \textsc{NEG}



‘The one that prepared the loaf for Mana [was] not Hawa.’


\section{Interrogative constructions}
\hypertarget{RefHeading1212981525720847}{}
Interrogative constructions can be superimposed on top of the other clausal construction types. Like the case for the negation construction (see \sectref{sec:73}), the element closest to the interrogative pronoun or question word seems most frequently under the scope of interrogation. Types of interrogative constructions include content questions (see \sectref{sec:74}), yes/no questions (see \sectref{sec:75}), tag question construction, to clarify a particular statement (see \sectref{sec:76}), rhetorical question constructions (see \sectref{sec:77}), and emphatic question constructions (see \sectref{sec:78}).  

\subsection{  Content question construction}
\hypertarget{RefHeading1213001525720847}{}
Information questions use interrogative pronouns which must be clause-final. The interrogative pronouns (see \sectref{sec:16}) each fill a slot in the clause according to the element they each are questioning. All elements in a clause can be questioned including subject, direct object, indirect object, verb, oblique, and noun modifier. The clause structure will always be arranged such that the element questioned is clause-final. Three main clause structures are employed in order to achieve clause-final interrogative pronouns. \tabref{tab:82}. shows the interrogative forms used for content questions. 

\begin{tabular}{ll}
\lsptoprule

\textbf{Construction} & \textbf{Structure and example}\\
\textbf{Verbal clause structure}

Questions clausal element & \textit{clause –  interrogative word }

{\itshape zar   à-mənzar   \textbf{waj  }}

man  3S.PFV-see  who

‘Who did the man see?’\\
\textbf{Predicate nominal}

Questions subject & \textit{dependent clause marked with }na \textit{– interrogative word}

{\itshape h\textsuperscript{w}ɔr  amɪ-d-ijɛ  ɗaf   na  \textbf{waj}}

woman  DEP-make-CL  loaf  PSP  who

‘Who is making loaf?’ (lit. the woman that is making loaf [is] who?)\\
\textbf{Right-shifted }\textbf{\textit{na}}\textbf{ marked element }

Questions internal element & clause – interrogative word – right-shifted \textit{na} marked element

{\itshape Mala  a-vəl=aŋ  \textbf{al}\textbf{maj  }ana  məlama  =ahaŋ       na }

Mala  3S-give=3S.IO  what  DAT  sibling  =3S.POSS  PSP

‘Mala gave what to his brother?’\\
\lspbottomrule
\end{tabular}

\begin{itemize}
\item \begin{styleTabletitle}
Content information constructions
\end{styleTabletitle}\end{itemize}

The first clause structure that is employed is the verbal clause structure (SV(O)), but with substitution of a question word. The verbal clause structure is rearranged in the same manner as for constituent negation (see \sectref{sec:73}) in order to position the questioned element in the clause-final position so that it is replaced by the interrogative pronoun. Ex. 778 - 782 and 789 are information questions in verbal clauses.

The direct object is questioned in ex. 778. The presupposed information is that the man saw someone. Note that there are no other elements that follow the direct object in the verb phrase. The interrogative pronoun fills the direct object slot (identified by square brackets). The first example in each pair is the interrogative construction. Examples in this section are given in pairs.  The first example in the pair shows the content question. The second example is the clause with the information filled in for comparison.


\ea
Zar  amənjar  [\textbf{way? }]\textbf{ }
\z
\ zar     à-mənzar   [\textbf{waj }]\textbf{ }



man    3S.PFV-see  who



‘Who did the man see?’ 


\ea
Zar  amənjar  [Mana\textbf{.}]
\z
\ zar     à-mənzar   [Mana\textbf{ }]



man    3S.PFV-see  Mana



‘The man saw Mana.’ 


Ex. 780 questions a noun modifier. The presupposed information is that the woman made some kind of sauce, and the question seeks to find out what kind of sauce. The interrogative pronoun \textit{wɛlɛj}\textbf{\textit{ }}‘which’ is within the noun phrase delimited by square brackets in the example. Even though the interrogative pronoun is inside a noun phrase, that noun phrase is clause-final so the interrogative pronoun is the final word in the clause. 

\ea
Hor  ede  [elele  \textbf{weley? }]
\z

h\textsuperscript{w}ɔr    ɛ{}-dɛ    [ɛlɛlɛ  \textbf{wɛlɛj }]



woman  3S-prepare  sauce  which    



‘The woman is making which kind of sauce?’


\ea
Hor  ede  [elele  kəlef\textbf{.}]
\z

h\textsuperscript{w}ɔr    ɛ{}-dɛ    [ɛlɛlɛ  kɪlɛf\textbf{ }]



woman  3S-prepare  sauce  fish    



‘The woman is making fish sauce.’


Ex. 782 questions the direct object of a subordinate clause, in this case a purpose adverbial clause (delimited by square brackets). The presupposed information is that the listener has come to do something. The interrogative pronoun \textit{almaj}  ‘what’ is clause-final since the adverbial clause has no other elements following the direct object. 

\ea
Kəlala  [aməgəye  \textbf{alma}\textbf{y? }]
\z

kə-l    =ala   [amɪ-g-ijɛ   \textbf{almaj }]



2S.PFV-go  =to  DEP-do-CL  what



‘What have you come to do?’ (lit. you have come to do what?)


\ea
Nəlala  [aməgəye  slərele.]
\z

nə-l=ala     [amɪ-g-ijɛ   ɬɪrɛlɛ\textbf{ }]



1S.PFV-go=to  DEP-do{}-CL  work



‘I came to do work.’ 


\ea
Nəlala  [aməjənok.]
\z

nə-l=ala     [amə-dzən-ɔk\textsuperscript{w}]



1S.PFV-go=to  DEP-help-2S



‘I came to help you.’ 


In ex. 785, the indirect object is questioned. The presupposed information is that Mala gave a book to someone. The interrogative pronoun \textit{waj} ‘who,’ is located within a prepositional phrase identified by square brackets.  That prepositional phrase is clause-final, so that again the interrogative pronoun is the final element in the clause. 

\ea
Mala  avəlan  ɗeləywer  [ana  \textbf{way}?\textbf{ ]}
\z

Mala   à-vəl=aŋ     ɗɛlijwɛr     [ana   \textbf{waj ]}



Mala  3S.PFV-give=3S.IO  paper    DAT  who



‘Mala gave the book to whom?’ 


\ea
Mala  avəlan  ɗeləywer  [ana  Hawa\textbf{.]}
\z

Mala   à-vəl=aŋ     ɗɛlijwɛr     [ana   Hawa\textbf{ ]}



Mala  3S.PFV-give=3S.IO  paper    DAT  Hawa



‘Mala gave the book to Hawa.’ 


In ex. 787 and 789, an oblique is questioned. The presupposed information is that the woman plans to go to market sometime. The interrogative pronoun is the temporal element in the clause in ex. 787. While temporal noun phrases can occur clause-initially, the interrogative pronoun is again found in the clause-final position. 

\ea
Hor  olo  a  kosoko  ava  [\textbf{epeley}? ]    
\z

h\textsuperscript{w}ɔr    ɔ{}-lɔ  a  kɔsɔk\textsuperscript{w}ɔ  ava  [\textbf{ɛpɛlɛj} ]    



woman  3S-go  in  market  in  when



‘When is the woman going to market?’


\ea
Hor  olo  a  kosoko  ava  [hajan.]    
\z

h\textsuperscript{w}ɔr    ɔ{}-lɔ  a  kɔsɔk\textsuperscript{w}ɔ  ava  [hadzaŋ ]    



woman  3S-go  in  market  in  tomorrow



‘The woman is going to market tomorrow.’


The elements within adpositional phrases are questioned using the generic location question word \textit{amtamaj }‘where’ (ex. 789). This generic location question word does not need to be located inside an adpositional phrase, eliminating the possibility that the enclitic would follow the interrogative pronoun in the clause allowing the interrogative pronoun to be clause-final. The presupposed information is that the hearer is going somewhere. 

\ea
Kolo  [\textbf{amtama}\textbf{y}?\textbf{ }]        
\z

kɔ-lɔ  [\textbf{amtamaj }]        



2S.PFV  where



‘Where did you go?’


\ea
Nolo  [a  kosoko  ava.]        
\z

nɔ-lɔ  [\textbf{a   }kɔsɔk\textsuperscript{w}ɔ    \textbf{ava}]        



1S.PFV  in  market    in



‘I went to market.’


The second clause structure that is employed is the predicate nominal. The predicate nominal structure is employed for questioning an element of a predicate nominal clause (ex. 791 - 796). The nominal in the following predicate nominal construction is questioned with the interrogative pronoun in a prepositional phrase. The prepositional phrase is delimited by square brackets.

\ea
Mogom  nehe  [anga  \textbf{way}?\textbf{ }]
\z

mɔg\textsuperscript{w}ɔm  nɛhɛ  [aŋga  \textbf{waj }]



house  DEM  \textsc{POSS}  who  



This house here belongs to whom?


\ea
Mogom  nehe  [anga  Mana\textbf{.}]
\z

mɔg\textsuperscript{w}ɔm  nɛhɛ  [aŋga  Mana\textbf{ }]



house  DEM  \textsc{POSS}  Mana  



This house here belongs to Mana.’ (the house here, belonging to Mana)


In ex. 793 and 795, the interrogative word itself is the predicate. 

\ea
Mogom  ango  [\textbf{amtama}\textbf{y}?~]
\z

mɔg\textsuperscript{w}ɔm  =aŋg\textsuperscript{w}ɔ     [\textbf{amtamaj}~]



home  =2S.POSS  where



‘Where is your home?’


\ea
Mogom  əwla  [a\textbf{  }Laway.]
\z

mɔg\textsuperscript{w}ɔm  =uwla     [a\textbf{   }Lawaj~]



home  =1S.POSS  to  Lalawaj



‘My home is in Lalaway.’


\ea
 Bahay  a  slala  aləkwəye\textsuperscript{  }na  [\textbf{wa}\textbf{y}?\textbf{ }]
\z

bahaj  a  ɬala  =alʊk\textsuperscript{w}øjɛ  na  [\textbf{waj }]



chief  GEN  village  2P.POSS  PSP  who  



‘The chief of your village is who?’


\ea
Bahay  a  slala  əwla  na  [Ajəva\textbf{.}]
\z

bahaj  a  ɬala  =uwla    na  [Adzəva\textbf{ }]



chief  GEN  village  1S.POSS  PSP  Adziva  



‘The chief of my village is Adziva.’


The predicate nominal clause is also used for questioning the subject in what would otherwise be a normal verbal clause (paralleling the case for the negative, see \sectref{sec:73}). Ex. 791 and 793 are information questions in a predicate nominal construction. The subject of what would be a verbal clause in a declarative speech act cannot be questioned using the SV(O) verbal clause construction in Moloko, because the clause can never be simply rearranged so that the subject is clause final. For example, it is impossible to question the subject in the following clause, using the SV(O) verbal clause construction.\footnote{Unless the emphatic question construction is used \citep{Section1178}. } 

\ea
Hor  ede  ɗaf.
\z

h\textsuperscript{w}ɔr  ɛ{}-d-ɛ    ɗaf



woman  3S-make-CL  loaf



‘The woman is making loaf’


To question the subject (ex. 798 - 799), the verbal clause must be reformed into a predicate nominal interrogative construction. The clause is reformed into a noun phrase with a relative clause so that the interrogative pronoun questioning the subject can be in clause-final position. 

\ea
Hor  amədəye  ɗaf  na  \textbf{wa}\textbf{y}?
\z

h\textsuperscript{w}ɔr    amɪ-d-ijɛ  ɗaf   na  \textbf{waj}



woman  DEP-make-CL  loaf  PSP  who



‘Who is making loaf?’ (lit. the woman that is making loaf [is] who?)


\ea
 Hor  amədəye  ɗaf  na  \textbf{wel}\textbf{ey}?
\z

h\textsuperscript{w}ɔr    amə-d-ijɛ  ɗaf  na  \textbf{wɛlɛj}



woman  DEP-make-CL  loaf  PSP  which



‘Which woman is making loaf?’ (lit. the woman that is making loaf [is] which one?)


Ex. 800 and 802 show two other predicate nominal clauses that question what would be the subject of an otherwise verbal clause.

\ea
Məze  amanday  aməzəme  ɗaf  na  \textbf{wa}\textbf{y}?
\z

mɪʒɛ   ama-ndaj   amɪ-ʒum-ɛ  ɗaf    na   \textbf{waj}



person  DEP-PRG  DEP-eat-CL  loaf  PSP  who



‘Who is eating loaf?’ (lit. the man that is eating loaf [is] who?)


\ea
Mana  anday  ozom  ɗaf. 
\z

Mana   a-ndaj     a-zɔm    ɗaf 



person  3S-PRG    3S-eat    loaf



‘Mana is eating loaf.’ 


\ea
Aməzəɗe  dəray  na  \textbf{wa}\textbf{y}?
\z

amɪ-ʒɪɗ{}-ɛ    dəraj    na   \textbf{waj}



DEP-take-CL  head  PSP  who



‘Who will win?’ (lit. the one that takes the head [is] who?)


\ea
Mana  azaɗ  dəray.    
\z

Mana  a-zaɗ  dəraj    



Mana  3S- take  head



‘Mana won.’ (lit. Mana took head)


The third structure for content information questions uses a right-shifted \textit{na}{}-marked element. This structure is employed in cases where it is impossible for a questioned verb phrase element to be clause final. In ex. 804, the direct object is questioned. In this case the direct object cannot be clause final since it is necessary to include the information \textit{ana məlama =ahaŋ}\textit{ }‘to his brother,’ and the prepositional phrase must follow the direct object in the verb phrase (Chapter 8).  Thus in the interrogative structure, the interrogative pronoun replaces the direct object and the rest of the clause is put into a post-posed \textit{na}{}-marked phrase (underlined in this example, see \sectref{sec:12.3}). 

\ea
Mala  avəlan  \textbf{al}\textbf{may  }ana  məlama  ahan  na? 
\z

Mala  a-vəl=aŋ    \textbf{al}\textbf{maj  }ana  məlama  =ahaŋ    na 



Mala  3S-give=3S.IO    what  DAT  sibling  =3S.POSS  PSP



‘Mala gave what to his brother?’


\ea
Mala  avəlan  dala  ana  məlama  ahan. 
\z

Mala  a-vəl=aŋ    dala\textbf{  }ana  məlama  =ahaŋ 



Mala  3S-give=3S.IO    money  DAT  sibling  =3S.POSS  



‘Mala gave money to his brother.’


\subsection{  Yes-No question construction}
\hypertarget{RefHeading1213021525720847}{}
Yes/no questions are interrogative clauses which can be answered by a simple ‘yes’ or ‘no’ – they are not asking for content in the reply. Moloko uses the interrogative particle \textit{ɗ}\textit{aw} at the end of what is otherwise a declarative clause to create yes/no interrogatives. Pure yes-no questions can be answered with either yes or no, but in Moloko there is often a degree of expectation to the question.\footnote{Expectation is a central element in understanding Moloko grammar (see \sectref{sec:1153}), as is what constitutes shared information with the hearer (see Chapter 12). Questions are constructed in Moloko with that knowledge and expectation in mind, even when seeking new information. Tag questions are discussed in \sectref{sec:1176.}} When a speaker asks a yes/no question (ex. 806 - 809, 811, 813), they are usually expecting an affirmative reply. 


\ea
Zar na  ndahan  baba  a  Mala  \textbf{ɗ}\textbf{aw}? 
\z
\ zar   na   ndahaŋ   baba   a   Mala   \textbf{ɗ}\textbf{aw} 



man    PSP  3S  father  GEN  Mala  QUEST



‘That man, is he Mala’s father?’


In ex. 807, the speaker expects that Mana is on his way; he is asking for confirmation (but a negative response is always possible). Likewise in ex. 808, he expects that the referent \textit{zar =aŋg}\textit{\textsuperscript{w}}\textit{ɔ} ‘your husband’ is well.

\ea
Mana  na  álala  \textbf{ɗaw}?
\z

Mana   na  á-l=ala    \textbf{ɗaw}



Mana  PSP  3S+IPV-come=to  QUEST



Mana, is he coming?


\ea
Zar  ango  ndahan  aba  \textbf{ɗ}\textbf{aw}?
\z
\ zar     =aŋg\textsuperscript{w}ɔ    ndahaŋ  aba   \textbf{ɗ}\textbf{aw}



man    =2S.POSS  3S  \textsc{EXT}  QUEST



‘Is your husband well?’ (part of a greeting; lit. your husband, does he exist?) 


There is often an even stronger affirmative expectation when the question is negated. Compare the positive and negative pairs of questions (ex. 809 - 814). Some of the negated questions can be used rhetorically (see \sectref{sec:77}), since the speaker already knows that the answer is yes. In the examples, the interrogative particle is bolded.

\ea
Baba  ango,  ndahan  ava  a  mogom  \textbf{ɗaw}?\textbf{  }  
\z

baba  =aŋg\textsuperscript{w}ɔ    ndahaŋ   ava  a  mɔg\textsuperscript{w}ɔm    \textbf{ɗaw  }  



father  =2S.POSS  3S    \textsc{EXT}+in  at  home    QUEST    



‘Is your father in?’


\ea
Baba  ango,  ndahan  ava  a  mogom  bay  \textbf{ɗaw}?\textbf{  }  
\z

baba  =aŋg\textsuperscript{w}ɔ    ndahaŋ ava  a  mɔg\textsuperscript{w}ɔm   baj  \textbf{ɗaw  }  



father  =2S.POSS  3S    \textsc{EXT}+in  at  home    \textsc{NEG}  QUEST    



‘Is your father not in?’


\ea
Ólo  a  kosoko  ava  \textbf{ɗaw}?
\z

ɔ{}-lɔ    a  kɔsɔk\textsuperscript{w}ɔ  ava  \textbf{ɗaw}



3S.IFV-go  in  market  in  QUEST



‘Is he going to the market?’


\ea
Ólo  a  kosoko  ava  bay  \textbf{ɗaw}?
\z

ɔ{}-lɔ    a  kɔsɔk\textsuperscript{w}ɔ  ava  baj  \textbf{ɗaw}



3S.IFV-go  in  market  in  \textsc{NEG}  QUEST



‘Is he not going to the market?’


\ea
Məlama  ango  alala  \textbf{ɗaw}?
\z

məlama   =aŋg\textsuperscript{w}ɔ     à-l=ala     \textbf{ɗaw}



sibling  =2S.POSS  3S.IFV-go=to  QUEST



‘Is your brother coming?’


\ea
Məlama  ango  alala  bay  \textbf{ɗaw}?
\z

məlama   =aŋg\textsuperscript{w}ɔ     à-l=ala     baj  \textbf{ɗaw}



sibling  =2S.POSS  3S.IFV-go=to  \textsc{NEG}  QUEST



‘Is your brother not coming?’


As is the case for the negation construction (see Sections 73, 22.1.1), it could be that the entire proposition in the clause is being questioned. However, it is often the case that only the final constituent is being questioned. Often the clause is restructured when a constituent of the clause is questioned so that the constituent is in final position. In ex. 815 the direct object is fronted and marked as presupposed (it is the topic of discussion) so that the other elements in the clause are questioned (see \sectref{sec:75}). See also ex. 807 where the subject is marked as presupposed.

\ea
Awak  ango  na,  káaslay  na  \textbf{ɗaw}?
\z

awak  =aŋg\textsuperscript{w}ᴐ    na  káá-ɬ{}-aj    na  \textbf{ɗaw}



goat    =2S.POSS  PSP  2S.POT-slay{}-CL  3S.DO  QUEST



‘Your goat, are you going to slaughter it?’


\subsection{  Tag question construction}
\hypertarget{RefHeading1213041525720847}{}
Question tags can be attached at the end of what would otherwise be the construction used for a declarative clause to seek confirmation of a particular statement. In Moloko, a question tag is \textit{kijga baj ɗaw} ‘is that not so?’ The response is \textit{kijga} ‘it is so’ for affirmation. The negative response is \textit{kijga baj }‘it is not so’ with a statement to explain why the negative answer. Some rhetorical questions have a special question tag \textit{ɛ}\textit{ʃ}\textit{ɪ}\textit{m}\textit{ɛ}\textit{j} ‘isn’t that so’ (see \sectref{sec:77}). In the examples below, what is under the scope of questioning is put in square brackets. 


\ea
{}[Kolo  a  Marva  hajan ]  kəyga  bay  daw?          Kəyga.
\z

{}[kɔ-lɔ   a   Marva  hadzaŋ ]   \textbf{kijga    baj   daw    kijga}



2S.IFV-go  at  Maroua  tomorrow  like\_that    \textsc{NEG}  QUEST    like\_that



‘You are going to Maroua tomorrow, not so?’            ‘I will’ (lit. like that)


\ea
{}[Apazan  kolo  a  kosoko  ava ]  \textbf{k}əy\textbf{ga  bay  ɗaw?}
\z

{}[apazaŋ  kɔ-lɔ  a  kɔsɔk\textsuperscript{w}ɔ  ava ]  \textbf{kijga  baj  ɗaw}



yesterday  2S.PFV  in  market  in  like that  \textsc{NEG}  QUEST



‘You went to the market yesterday, right?’


\ea
Nə  alməmar  na,  [avar  abay ]  \textbf{k}əy\textbf{ga}  \textbf{bay  ɗaw?  }\textbf{  }
\z

nə    alməmar   na  [avar   abaj ]     \textbf{kijga}   \textbf{baj   ɗaw    }



with      dry season  PSP  rain  \textsc{EXT} \textsc{NEG}  like-this  \textsc{NEG}  QUEST    



‘In dry season, there is no rain, right?’  


Other question tags are evaluative. Ex. 819 is a question asked in a context where the speaker is examining something physically (perhaps at the market as he is considering to buy it) or analysing and evaluating an event. 

\ea
{}[Səlom ga ]  \textbf{ɗ}\textbf{aw}?
\z

{}[\textit{s}\textit{ʊ}\textit{l}\textit{ɔ}\textit{m   ga} ]\textit{  }\textbf{\textit{ɗ}}\textbf{\textit{aw}}



goodness  ADJ  QUEST



‘’[Is that] good?              


\subsection{  Rhetorical question construction}
\hypertarget{RefHeading1213061525720847}{}
In a rhetorical question, the speaker is not pragmatically asking for information. Rather, the questions can be evaluative, may carry an element of reproach, or may be a mild command. The context gives the rhetorical force. Some rhetorical questions have a special emphatic structure (see \sectref{sec:78}) but many have normal interrogative structure. Rhetorical question constructions can be identical in structure to a content question (ex. 820 - 821, see \sectref{sec:74}) but the speaker is not seeking information. For example, the speaker is not seeking an explanation when he asks \textit{kamaj} ‘why’ in ex. 820. More probably he is making a strong statement, ‘the people had no reason to do this bad thing to me.’ Likewise in ex. 821, the speaker is saying that the listener will listen to no one. 


\ea
Məze  ahay  tagaw  ele  lala  bay  \textbf{kama}\textbf{y}?
\z

mɪʒɛ  =ahaj  ta-g=aw  ɛlɛ  lala  baj  \textbf{kamaj}



person  =Pl  3P-do=1S.IO  thing  good  not  why



‘Why did the people do this bad thing to me?’ (lit. the people did the bad thing to me why?)



Values S. 29


\ea
Hərmbəlom  na,  amaɗaslava  ala  məze  na,  ndaha  ese  na,  kagas  ma  Hərmbəlom  na,  asabay  na,  káagas  na  anga  \textbf{way}?
\z

Hʊrmbʊlɔm  na  ama-ɗaɬ=ava=ala  mɪʒɛ   na  ndahaŋ  ɛʃɛ  na



 God  PSP    DEP-multiply=in=to   person   PSP     3S     again    PSP  



\textit{ka-gas    ma   H}\textit{ʊ}\textit{rmb}\textit{ʊ}\textit{l}\textit{ɔ}\textit{m  na       asa-baj         na  káá-gas            na    a}\textit{ŋ}\textit{ga   }\textbf{\textit{waj}}



2S-catch   word     God             PSP  again-\textsc{NEG}  PSP  2S.POT-catch  PSP \textsc{POSS}  who



‘And if you will never accept the word of God, the one that multiplied the people, whose word will you accept then?’ 


Other rhetorical questions have the same structure as a yes/no tag question\textit{ }(ex. 822 - 821, see \sectref{sec:75}). However either there is no expected answer or the expected answer is the opposite of that for a normal yes/no tag question. For example, during the telling of the text from which ex. 822 is taken, when the storyteller asked the rhetorical question \textit{lala  }\textit{ɗ}\textit{aw} ‘[is that] good?’ the people in the audience replied “\textit{lala baj }” ‘[it is] not good.’ (even though the answer was obvious from the story). In ex. 823, the audience replied “\textit{sʊlɔm ga}” ‘[it is] good’ to the rhetorical question \textit{sʊlɔm ga baj ɗaw} ‘[is that] not good?’

\ea
Kólo  kagas  anga  məze  kək,  lala  \textbf{ɗ}\textbf{aw}?
\z

kɔ-lɔ  kà-gas    aŋga  mɪʒɛ  kək      lala  \textbf{ɗ}\textbf{aw}



2S.IFV-go  2S.PFV-catch  \textsc{POSS}  person  \textsc{ID}catch by throat    good  QUEST



‘If you catch [something] belonging to someone else [and steal it],[ is that] good?’


\ea
Kólo  ele  ango,  səlom  ga  bay  \textbf{ɗaw}?
\z

kɔ{}-lɔ  ɛlɛ  =aŋg\textsuperscript{w}ɔ    sʊlɔm  ga  baj  \textbf{ɗaw}



2S.IFV-go  thing  =2S.POSS  good  ADJ  \textsc{NEG}  QUEST



‘[If] you mind your own business (lit. go to your things), [is that] not good?’


A particular question tag, \textit{ɛ}\textit{ʃ}\textit{ɪ}\textit{m}\textit{ɛ}\textit{j} ‘isn’t that so’ has an element of reproach to it. There is no expected answer to the question in ex. 824. The message is a strong declaration that the speaker had already told something to the hearer. 

\ea
{}[Nahok\textsuperscript{  }ma  fan ]\textbf{  es}ə\textbf{m}\textbf{ey}?
\z

{}[nà-h=ɔk\textsuperscript{w}     ma  faŋ ]  \textbf{ɛʃɪmɛj}



1S.PFV-tell=2S.IO  word  already  isn’t that so



‘I already told you, didn’t I?’


\subsection{  Emphatic question construction}
\hypertarget{RefHeading1213081525720847}{}
Emphatic questions ask emphatically for information. They can be used in a crisis situation where important information is needed immediately. In other contexts, these questions carry imperatival force and are therefore a sub-type of rhetorical questions (see \sectref{sec:77}). The emphatic question construction uses two interrogative pronouns, a reduced emphatic pronoun within the clause in the normal slot for the element questioned, and the other a sometimes reduced pronoun at the end of the clause. 

These reduced interrogative pronouns are \textit{wa} (from \textit{waj} ‘who’) in ex.  825, 827, 828, \textit{maj} and \textit{alma} (from \textit{almaj} ‘what’) in ex. 826 and 829, respectively, \textit{malma} (from \textit{malmaj} ‘what’) in ex. 826, 830, and \textit{mɛmɛ} and \textit{mɛj} (from \textit{mɛmɛj} ‘how’) in ex. 831.


\ea
\textbf{Wa}  aməgok  na  \textbf{wa}\textbf{y}?
\z

\textbf{wa}    amə-g=ɔk  na  \textbf{waj}



who    DEP-do=2SD\={ }  3S.DO  who



‘\textit{What} is wrong?’ / ‘Stop crying!’ (lit. who the one that did that to you, who) 


\ea
Kege  \textbf{may  }ana  war  ga  \textbf{ma}\textbf{y}?  
\z

ka-gɛ  \textbf{maj  }ana  war  ga  \textbf{maj}  



2S-do  what  DAT  child  ADJ  what  



\textit{‘What} are you doing to the child, \textit{what}?’ / ‘Stop doing that!’



Cicada S. 18


\ea
Nánjakay  na  \textbf{wa}  [amazaw  ala  ngəvəray  ana  ne   na ]  \textbf{wa}\textbf{y}?
\z

ná-nzak-aj     na  \textbf{wa}    [ama-z  =aw   =ala  ŋgəvəraj    ana   nɛ    na ]     \textbf{waj}



1S.\textsc{POT}{}-find{}-CL PSP\textsc{ }who   DEP-take  =1S.IO   =to    spp. of tree     DAT 1S    PSP    who



‘\textit{Who} can I find to bring to me this tree for me?  \textit{Who}?’ / ‘\textit{Someone} should be able to bring me this tree.’


\ea
\textbf{Wa}  andaɗay  \textbf{wa}\textbf{y}?
\z

\textbf{wa}    a-ndaɗ-aj   \textbf{waj}



who    3S-love{}-CL  who



‘There is no love [for that one].’ (lit. who loves him?)


\ea
\textbf{Alma}  amədəvala  okfom  na  \textbf{ma}\textbf{y}?
\z

\textbf{alma}  amə-dəv=ala    ɔk\textsuperscript{w}fɔm  na  \textbf{maj}



what  DEP-trip=to    mouse  PSP  what



‘\textit{What} was it that made that mouse fall? \textit{What}?\textit{ }/ \textit{‘}What else [but a snake] makes a mouse fall?’


\ea
\textbf{Malma}  awəlok\textsuperscript{  }\textbf{ma}\textbf{y}?
\z

\textbf{malma}   a-wəl=ɔk\textsuperscript{w}   \textbf{maj}



what  3S-hurt=2S.IO  what



‘\textit{What} is bothering (hurting) you? \textit{What}?’ / ‘\textit{Nothing} should be bothering you.’


\ea
\textbf{Meme}  ege  \textbf{m}\textbf{ey}?
\z

\textbf{mɛmɛ}   ɛ{}-g-ɛ     \textbf{mɛj}



how    3S-do-CL  how?



‘\textit{What} happened?’ / ‘Why did you do that?’ / ‘Stop the foolishness.’ (lit. how did it do?)


\section{Imperative constructions}
\hypertarget{RefHeading1213101525720847}{}
There are several types of imperative constructions in Moloko, which are used in different situations, sometimes to express different degrees of obligation.  So far eight different constructions have been identified, each with a different force of exhortation. They are shown in \tabref{tab:83}.. Some constructions use the imperative mood form of the verb (see \sectref{sec:7.2}), others use Imperfective aspect or irrealis mood or are in the form of a rhetorical question (see \sectref{sec:77}).  \tabref{tab:83}. illustrates all of the imperative constructions for the verb /\textit{l}\textit{ɔ} / ‘go.’ The verb forms are also shown in Perfective and Imperfective aspect (lines 1 and 2) for comparison.

\begin{tabular}{llll}
\lsptoprule

\textbf{Line} &  & \textbf{2S forms} & \textbf{3S forms}\\
\textbf{1} & \textbf{Declarative, Perfective aspect } & \textit{kà-l=àlá  }

2S.PFV-go=to  

‘You came.’ & \textit{à-l= àlá    }

3S.PFV-go=to  

‘He/she came.’\\
\textbf{2} & \textbf{Declarative, Imperfective aspect} & \textit{ká-l=álà  }

2S.IFV-go=to  

‘You come.’ & \textit{á-l=álà    }

3S.IFV-go=to  

‘He/she comes.’\\
\textbf{3} & \textbf{Imperative} & \textit{l=àlá  }

2S.PFV-go=to  

‘Come (now)!’ & \\
\textbf{4} & \textbf{Polite request} & \textit{ká-l=ál}\textit{     ɛtɛ  ɗaw}

2S.IFV-go=to  polite\textit{  QUEST}

‘Please come.’ & \\
\textbf{5} & \textbf{Negative expectation} & \textit{ká-l=ál}\textit{   }\textit{  baj}

2S.IFV-go=to  \textsc{NEG}

‘Don’t come.’ (I don’t expect you to come) & \textit{á-l=ál    baj}

3S.IFV-go=to  \textsc{NEG}

‘He/she is not coming.’ (I don’t expect him to come)\\
\textbf{6} & \textbf{Hortative} & \textit{kàà-l=àlá  }

2S.HOR-go=to  

‘You come now!’ (I want you to come) & \textit{mə-l= àlá    }

3S.HOR-go=to  

‘He/she should come.’ (I want him to come)\\
\textbf{7} & \textbf{Adverb of obligation} & \textit{sij}\textit{       k}\textit{ə}\textit{{}-l=àl  =vá    }

only    2S.PFV-go=to  =\textsc{PRF}  

‘You must come.’ & \textit{sij}\textit{       mə-l=àlá    }

only     3S.HOR-go=to  

‘He/she must come.’\\
\textbf{8} & \textbf{Rhetorical question} & \textit{ká-l=ála  bàj  ɗáw}

2S.IFV-go=to  \textsc{NEG}\textit{  QUEST}

‘You should come.’ (lit. Are you not coming?) & \textit{á-l=álá    bàj   ɗáw}

3S.IFV-go=to  \textsc{NEG}\textit{   QUEST}

‘He should come.’ (lit. Is he not coming?)\\
\lspbottomrule
\end{tabular}

\begin{itemize}
\item \begin{styleTabletitle}
Imperative constructions 
\end{styleTabletitle}\end{itemize}

The imperative form of the verb is used for an immediate command (line 3). The verb is in the imperative mood (see \sectref{sec:7.2}) and can be preceded by a vocative. The addressee is expected to carry out the order in the immediate future as opposed to commands that demand reflection before carrying them out. In hortatory texts, imperatives are not usually found in the body of the exhortation since the hearer is expected to wait until the discourse is finished before carrying out the instructions.  


\ea
Lohom  a  mogom.
\z

lɔh\textsuperscript{w}{}-ɔm  a  mɔg\textsuperscript{w}ɔm



go-2P  at  home



‘Go home!’


\ea
Zəmok  ɗaf.
\z
\ zʊm-ɔk\textsuperscript{w}  ɗaf



eat-1\textsc{Pin}  loaf



‘Let’s eat!’


\ea
Cəke.
\z

tʃɪk-ɛ



stand[2S.IMP] -CL



‘Stand up!’


The word \textit{ɛtɛj} or \textit{ɛtɛ} ‘please’ can be added to other clause types (line 5 in \tabref{tab:83}.) to achieve a milder pragmatic imperative force than the use of the construction without the polite adverb.

\ea
Nde na  asaw  na,  gaw  na\textbf{  etey}?
\z

ndɛ  na   a-s=aw     na   g=aw     na   \textbf{ɛtɛj}



so  PSP   3S-please=1S.IO PSP   do=1S.IO  3S.DO  please



‘So I want that you do that for me, please.’


\ea
Nənjakay  yam\textbf{  ete  }ɗaw?
\z

nə-nzak-aj    jam  \textbf{ɛtɛ}  ɗaw



1S.IFV-find{}-CL  water  please  QUEST



‘Could you please get me some water?’ (lit. can I find water please)


A negated clause in the Imperfective aspect expresses a negative exhortation or statement of expectation (line 5 in \tabref{tab:83}.). In second person (ex. 837), the negative expectation carries a weak hortative force. The speaker is expressing that he/she expects the addressee not to carry out the action. In third person (ex. 838) the negative expectation is not hortatory, but rather simply expresses that the speaker does not expect that the action will be performed. 

\ea
Kámənjar  fabay. 
\z

ká-mənz\={a}r     fá-bàj 



2S.IFV-see    already-\textsc{NEG}



‘Don’t look at it yet.’ (I don’t expect you to look at it).


\ea
á-mənz\={a}r     fá-bàj 
\z

á-mənz\={a}r     fá-bàj 



3S+I\textsc{PFV}{}-see    already-\textsc{NEG}



‘I don’t think he looked at it.’ (I don’t expect that he looked at it).


A clause with a verb in the Hortative mood (line 6 in \tabref{tab:83}., see \sectref{sec:53}) concentrates on the will of the speaker - the speaker wishes the action done. Ex. 839 illustrates this form for 3S. 

\ea
Mamənjar  fabay. 
\z

mà-mənz\={a}r     fá-bàj 



2S.HOR-see    already-\textsc{NEG}



‘He/she shouldn’t look at it yet.’ / ‘Don’t let him/her look at it.’ (I don’t expect him/her to look at it).


An even stronger deontic form is made by the addition of an adverb of obligation (\textit{dɛwɛlɛ} ‘obligation,’ \textit{sij} ‘only’) preceding the clause, with the verb in Hortative mood (line 7 in \tabref{tab:83}.). Imperative forms with an adverb of obligation indicate that the hearer is obligated to do something (he/she has no choice, there is no other way). These forms are used to give an order with insistence, a strong counsel (ex. 840 - 842).

\ea
Səy  koogom  endeɓ.
\z

sij     kɔɔ-g\textsuperscript{w}{}-ɔm   ɛndɛɓ



only    2P-do-2P   wisdom



‘You must be wise (lit. do only wisdom).’


\ea
Dewele  səy  keege  na.
\z

dɛwɛlɛ   sij   kɛɛ-gɛ     na



obligation  only   2S.HOR-do   3S.DO



‘You are obligated to do that.’ (lit. obligation: you must only do it)


\ea
Səy  keege  anga  dewele.
\z

sij     kɛɛ-gɛ     aŋga   dɛwɛlɛ



only   2S.HOR-do   \textsc{POSS}   obligation



‘You must do that obligation.’ (lit. you must only do the thing that belongs to obligation)


\section{ Exclamatory constructions}
\hypertarget{RefHeading1213121525720847}{}
Exclamatory sentences have either an interjection at the initial position (ex. 843) or one of several exclamatory adverbs at the final position (ex. 844 - 847). In the examples, the interjections and exclamatory adverbs are bolded. 


\ea
\textbf{Kay},  nege  na  bay.
\z

\textbf{kaj},    nɛ-g-ɛ    na  baj



EXCL  1S.PFV-do-CL  3S.DO  \textsc{NEG}



‘No, I didn’t do it!’


\ea
Apazan  nok  awəy  Məwsa  álala;  macakəmbay  aməlala  na  ndahan  bay  \textbf{nəy}.
\z

apazaŋ   nɔk\textsuperscript{w}  awij  Muwsa  á-l=ala



yesterday  2S  said  Moses  3S.IFV-come=to



\textit{matsakəmbaj     amə-l=ala  na  ndahaŋ  baj  }\textbf{\textit{nij}}



meanwhile    DEP-come  PSP  3S  \textsc{NEG}  EXCL



‘Yesterday you said that Moses would come; but the one that came was not him after all!’


\ea
Enje  bay  ɗeɗen  \textbf{dey}.
\z

ɛ{}-nʒ-ɛ    baj  ɗɛɗɛŋ  \textbf{dɛj}



3S.PFV-suffice-CL  \textsc{NEG}  truth  EXCL



‘It really wasn’t enough!’


\ea
Gaw  endeɓ  \textbf{d}\textbf{ey}.
\z

g=aw     ɛndɛɓ   \textbf{dɛj}



do 2S.IMP=1S.IO  brain  EXCL



‘Be careful!’ (lit. do brain for me)



Values S. 50


\ea
Epele  epele  na me,  Hərmbəlom  anday  agas  təta  a  ahar  ava \textbf{r}\textbf{e}.
\z

ɛpɛlɛ ɛpɛlɛ   na   mɛ  Hʊrmbʊlɔm   a-ndaj   a-gas         təta   a   ahar  ava   \textbf{rɛ}



in the future  PSP  opinion    God      3S-PROG  3S-catch    3P  in  hand  in         in spite



‘And so in the future (in my opinion), God is going to accept them [the elders] in his hands, in spite [of what the church says].’


\chapter[The na marker and na constructions]{The\textit{ na }marker and\textit{ na }constructions}
\hypertarget{RefHeading1213141525720847}{}
Expectation is a concept that is fundamental for Moloko. Within the irrealis world, this concept has already been discussed (mood, see \sectref{sec:53}). Within the realis world, expectation is shown in other forms. One of these forms is the \textit{na} construction or presupposition construction. Known or expected elements are marked with \textit{na}, which is found at the right edge of the element it modifies.

Knowledge of how the particle \textit{na} works in Moloko is foundational to understanding information flow and interpreting a Moloko text. A very basic knowledge of \textit{na} can be gained from studying the example pair below. Ex. 848 illustrates how a person would tell another person her name during a conversation. However, if the addressee first asked the person to give her name, then ‘name’ will be marked with \textit{na} in the response (ex. 849). Structurally, \textit{na}  isolates or separates some element in a clause or sentence from the rest of the clause. In ex. 849, it separates the predicate \textit{ɬ}\textit{əmaj =uwla} ‘my name’ from the nominal \textit{Abaŋgaj} ‘Abangay.’ In the examples in this chapter, \textit{na} is bolded and the element marked by \textit{na}  is underlined.


\ea
Sləmay  əwla  Abangay.
\z

ɬəmaj   =uwla     Abaŋgaj



name  =1S.POSS  Abangay



‘My name is Abangay.’


\ea
Sləmay  əwla  \textbf{na},  Abangay.
\z

ɬəmaj   =uwla     \textbf{na}   Abaŋgaj



name  =1S.POSS  PSP  Abangay



‘My name is Abangay.’


\textit{Na} is a separate phonological word that positions at the end of a noun phrase (ex. 850), time phrase (ex. 880), discourse particle (ex. 881), or clause (ex. 851) that is being marked. \textit{Na} has semantic scope over the preceding construction. When an element in a clause, or the clause itself, is marked with \textit{na}, it is marked as being known or expected information that is somehow a prerequisite to the information that follows.\footnote{The presupposition marker and the 3S direct object pronominal \citep{Section1150} are homophones and have a similar function to mark previously identified information,.} This structure for marking information as presupposed is a basic organisational structure with  a major function in certain Moloko clause structures and discourse.\footnote{\citet{Bow1997c} called \textit{na} a focus marker. We have found that the function of \textit{na} is not limited to focus. In related languages, a similar particle has often been referred to as a ‘topicalisation’ marker, but the fronting and special marking that \citet{Levinsohn1994} describes as topic marking is only one of the functions of this particle in Moloko.}

\ea
Həmbo  \textbf{na},  anday  ásəkala  azla  wəsekeke.
\z

hʊmbɔ\textbf{na}  à-ndaj      á-sək    =ala     aɮa   wuʃɛkɛkɛ



flour  PSP  3S.PFV-PRG   3S.IFV-multiply  =to  now    \textsc{ID}multiplication



‘The flour, it is multiplying.’



Cicada S. 5


\ea
Tánday  tətalay  a  ləhe  \textbf{na},  tolo  tənjakay  ngəvəray  malan  ga  a  ləhe.
\z

tá-ndaj       tə-tal-aj    a   lɪhɛ     \textbf{na}



 3P.IFV-PRG     3P-walk{}-CL        at    bush       PSP



\textit{tə-lɔ          tə-nzak-aj           }\textit{ŋ}\textit{gəvəraj     mala}\textit{ŋ}\textit{   ga   a  l}\textit{ɪ}\textit{hɛ}



3P.PFV-go    3P.PFV-find{}-CL    spp. of tree  large   ADJ      at    bush



‘[as]They were walking in the bush, they found a large tree (a particular species) in the bush.’


Pragmatic presupposition is defined by \citet[52]{Lambrecht1994} as “the set of presuppositions lexicogrammatically evoked in a sentence which the speaker assumes the hearer already knows or is ready to take for granted at the time the sentence is uttered.”  In Moloko, \textit{na}{}-marked elements indicate information that the speaker shares with the hearer in that the element has been previously mentioned in the discourse,  is the expected part of the situation, is the expected outcome of an event, or is assumed to be common knowledge or a cultural assumption. \textit{Na} -marked elements are the way that the speaker presents any information that he thinks the hearer should not be able to (or would not want to) challenge. 

The partitioning that \textit{na} produces results in the clause being split into two parts:  the presupposition (followed by \textit{na}) and the assertion. The assertion is that part of the sentence which the speaker expects the hearer “to know or take for granted as a result of hearing the sentence uttered” (Lambrecht, 1994: 52), but not necessarily before hearing it.  In the following example groups,\footnote{Adapted from Boyd, 2002.} the first gives the normal SVO clause structure without any \textit{na} -marked element. The rest have \textit{na }{}-marked elements (underlined). In the first triplet, ex. 852 represents a context where there is no specific presupposed information. There is also no \textit{na} marker. Ex. 853 represents a situation where the presupposed information is ‘I like X’ and the topic of the discourse is what is liked. Ex. 854 represents a context where the presupposed information is ‘beans.’

\ea
Hahar  asaw.        
\z

hahar   a-s=aw          



beans   3S-like=1S.IO          



‘I like beans.’ (lit. beans are pleasing to me) 



  Presupposition:     Nothing specific.


\ea
Asaw  \textbf{na},  hahar.      
\z

a-s=aw    \textbf{na}  hahar      



3S-like=1S.IO  PSP  beans      



‘[what] I like [is] beans.’



  Presupposition:     I like something (X). 



Assertion:     X=beans. 



Focus of assertion:   Beans.


\ea
Hahar  \textbf{na}  asaw.      
\z

hahar  \textbf{na}  a-s=aw      



beans  PSP  3S-like=1S.IO      



‘As for beans, I like them.’ 



  Presupposition:     Beans are the topic of this part of the discourse. Beans have some attribute (X).



Assertion :     X=I like them. 



Focus of assertion:   I like them


The rearranging of the construction to front the presupposed information in the clause is shown by another set of examples (855 - 858). Ex. 855 has no specific presupposition (and no \textit{na} marker). Ex. 856 represents a situation where Hawa is presupposed – the hearer knows who she is and Hawa is the topic of discussion. Ex. 857 is similar to 856 except that the relative clause also indicates known information (see \sectref{sec:44}) so the fact that someone prepared the food is also presupposed.  In ex. 858, the presupposed information is ‘someone made the food’ (or ‘X made the food’). 

\ea
Hawa  adan  ɗaf  ana  Mana.
\z

Hawa  a-d-aŋ      ɗaf  ana   Mana



Hawa  3S-prepare=3S.IO    loaf  DAT  Mana



‘Hawa prepared food for Mana.’  



Presupposition:     No specific presupposition



  Assertion:     Hawa prepared food for Mana


\ea
Hawa  \textbf{na},  adan  ɗaf.
\z

Hawa  \textbf{na}  a-d-aŋ      ɗaf



Hawa  PSP  3S-prepare=3S.IO    food



‘Hawa [is] the one who prepared the loaf for him.’



  Presupposition 1:    The hearer knows who Hawa is.



  Presupposition 2:   Hawa is the topic of this section of discourse, or Hawa did something (X).



  Assertion:     X= prepared the food


\ea
Hawa  \textbf{na},  amadan  ɗaf.
\z

Hawa  \textbf{na}  a-ma-d-aŋ    ɗaf



Hawa  PSP  DEPprepare=3S.IO  loaf



‘Hawa [is] the one that prepared the food for him.’



  Presupposition 1:    The hearer knows who Hawa is.



Presupposition 2:    Hawa is the topic of this section of discourse (a contrastive topic).



  Presupposition 3:    Someone (X) prepared the food.



  Assertion:     Hawa is the person who prepared the food


\ea
Amadan  ɗaf  \textbf{na},  Hawa.
\z

a-ma-d-aŋ      ɗaf  \textbf{na}  Hawa



DEP-\textsc{NOM}{}-prepare=3S.IO  food  PSP  Hawa



‘The preparer of his food [is] Hawa.’ 



  Presupposition:     Someone (X) prepared the food



  Assertion:     X=Hawa (the hearer may not know who Hawa is)  


\textit{Na} constructions in Moloko can be divided into five main structural types, depending on which element is presupposed and which element is the assertion. These structural types fit the main ways that \textit{na} constructions function in Moloko discourse. The five structural types are: 


\begin{itemize}
\item \textbf{Presupposition-assertion construction: fronted }\textbf{\textit{na }}\textbf{{}-marked clause}. A whole clause is marked with \textit{na}, separating it from the clause which follows and marking it as presupposed (see \sectref{sec:12.1}). These constructions function in text cohesion. 
\item \textbf{Presupposition-assertion construction: fronted }\textbf{\textit{na }}\textbf{–marked clausal element}. One element in a clause is fronted and delimited by \textit{na}, separating it from the rest of the clause and marking the fronted element as presupposed (see \sectref{sec:12.2}). Such constructions function in tracking participants and marking boundaries in a text. 
\item \textbf{Assertion-presupposition construction: right-shifted }\textbf{\textit{na }}\textbf{{}-marked element}. The element that is marked by \textit{na} is right-shifted to the end of a clause (see \sectref{sec:12.3}).
\item \textbf{The definite construction:}\textbf{ }\textbf{\textit{na }}\textbf{–marked clausal element}. The element that is marked by \textit{na} is in its normal clausal position (see \sectref{sec:12.4}). The definite construction functions to specify the element that is marked by \textit{na} in the text. 
\item \textbf{Presupposition-focus construction}\textbf{: }\textbf{\textit{na}}\textbf{\textit{ }}\textbf{preceeds the final element of the verb phrase. }The final element of a clause is immediately preceded by one or more \textit{na}{}-marked elements (see \sectref{sec:12.5}). This construction makes prominent the final element of the clause. 
\end{itemize}

Note that in the examples,\textit{ na }is always glossed as PSP ‘presupposition marker,’ even if its more specific function in a particular utterance might be argued to be for focus or definiteness, as marking presupposition is its overall function. It is probable that the different functions of\textit{ na }overlap, since structurally, it is often difficult if not impossible to determine whether\textit{ na }is at the end of a noun phrase  or a clause. It is also likely that the functions of\textit{ na }overlap with those of the 3S direct object pronominal (see \sectref{sec:50}) since in certain contexts, it is difficult to determine with certainty whether\textit{ na }is PSP or the 3S DO pronominal. The examples used in the text are chosen to clearly illustrate the function of \textit{na}.

\section{\textit{  }Presupposition-assertion construction: \textit{na }{}-marked clause}
\hypertarget{RefHeading1213161525720847}{}
There are two presupposition-assertion constructions depending on if the entire clause is marked with \textit{na} or if just one clausal element is marked (see \sectref{sec:12.2}). The \textit{na }{}-marked clause presupposition-assertion construction consists of an entire clause marked with \textit{na}  and fronted with respect to another clause (ex. 859 - 861).  The \textit{na }{}-marked clause presupposition-assertion construction functions in discourse in inter-clausal relations and is involved in discourse cohesion. The clause marked with \textit{na} expresses presupposed or shared information, and the main clause that follows contains asserted information. The precise relation between the \textit{na} clause and the main clause is determined by context (see \sectref{sec:13.4}).  In the examples in this section, the \textit{na} -marked clause is underlined. 


Cicada S. 5



\ea
Tánday  tətalay  a  ləhe  \textbf{na},  tolo  tənjakay  ngəvəray  malan  ga  a  ləhe.
\z

tá-ndaj       tə-tal-aj        a   lɪhɛ   \textbf{na}\textbf{ }



 3P.IFV-PRG     3P-walk{}-CL        at    bush       PSP



\textit{t}\textit{ʊ}\textit{{}-lɔ              tə-nzak-aj           }\textit{ŋ}\textit{gəvəraj        mala}\textit{ŋ}\textit{ ga    a    l}\textit{ɪ}\textit{hɛ}



3P.PFV-go   3P.PFV-find{}-CL   spp. of tree  large ADJ    at    bush



‘[As] they were walking in the bush, they found a large tree (a particular species) in the bush.’


\ea
Tənday  táhaya  \textbf{na},\textbf{ } həmbo  ga  ánday  ásak  ele  ahan  wəsekeke.  
\z

tə-ndaj             tá-h=aja \textbf{          na } 



3P.IFV-PRG  3P.IFV-grind   PSP 



\textit{h}\textit{ʊ}\textit{mbɔ  ga  á-ndaj             á-sak             ɛlɛ      =ahaŋ           wuʃɛkɛkɛ    }



flour   ADJ  3S.IFV-PRG  3S.IFV-multiply   thing  =3S.POSS   \textsc{ID}multiply



‘They were grinding it, and the flour was multiplying all by itself, \textit{wushekeke}.’



Disobedient Girl S. 36


\ea
Talay  war  elé  háy  bəlen  kə  ver  aka  \textbf{na},  á\textbf{s}ak  asabay.  
\z

talaj          war        ɛlɛ     haj   bɪlɛŋ  kə     vɛr       aka  \textbf{na}



\textsc{ID} put  child  eye  millet  one  on  stone  on  PSP\textit{  }



\textit{á-}\textbf{\textit{s}}\textit{ak           asa-baj}



3S.IFV-multiply  again-\textsc{NEG}



‘[If] they put one grain on the grinding stone, it doesn’t multiply anymore.’


A \textit{na} -marked clause in Moloko can function adverbially, because it is marked as subordinate (in a way) to the main clause, but it gives no explicit signal as to the nature of the sematic relationship between the two clauses.  The only thing it indicates is that the \textit{na} -marked clause is presented as presupposed, and somehow relevant to the following clause. The relations that \textit{na} clauses are employed in are temporal or logical sequence (see \sectref{sec:79}), simultaneous or coordinated events (see \sectref{sec:80}), and tail-head linking for cohesion (see \sectref{sec:81}).

\subsection{ Temporal or logical sequence}
\hypertarget{RefHeading1213181525720847}{}
The default relation between a \textit{na} -marked clause and the matrix clause in a \textit{na} construction is that there is a sequence (temporal or logical) and the event/state expressed by the \textit{na} -marked clause precedes the event/state in the main clause. Ex. 862 and 863 are both taken from a Moloko legend (Friesen, 2003) where some domestic animals are fleeing their owners because the owners are constantly killing the animals’ children in order to satisfy the demands of the spirits. Ex. 862 shows a reason-result construction. A hen begins the story with her lament expressing the reason why she is fleeing. She first states that ‘they have killed my children,’ then uses a \textit{na} construction to say that because they have killed her children, she is fleeing in anger. The \textit{na} -marked clause repeats the information she just declared in the first clause. This now presupposed information (‘they are killing my children’) is followed by the matrix clause containing the assertion of new information (I am fleeing in anger). Connecting the two clauses in a presupposition-assertion construction influences the hearer to deduce a logical or temporal connection between the two clauses; here reason-result. 


\ea
Tanday taslaw aka babəza ahay va. Nde, taslaw  aka  babəza  ahay  va\textbf{  na},  nəhəmay  mogo  ele  əwla.
\z

ta-nd-aj    ta- ɬ=aw  =aka  babəza  =ahaj  =va



3P-PROG-CL  3P-kill=1S.IO  =on  children  =Pl  =PERF



‘They have killed my children.’



ndɛ  ta-ɬ  =aw   =aka   babəza  =ahaj   =va\textbf{   na} 



3P-kill  =1S.IO  =on  children  =Pl  =\textsc{PRF}  PSP  



‘So, [because] they are killing my children,



\textit{nə-həm-aj  mɔg}\textit{\textsuperscript{w}}\textit{ɔ  ɛlɛ    =uwla}



1S-run{}-CL      anger  thing  =1S.POSS



‘I am running [in] anger.’ (lit. I am running my anger thing). 


Ex. 863 shows a temporal sequence (or perhaps another reason-result construction) from a little later in the same legend. The group of animals is joined by a dog. The dog expresses that whenever a person in the family gets sick, the family will be advised to kill a dog, because dog meat is thought to be especially good to help a sick person get stronger. The dog’s speech uses a \textit{na} construction to express this relation. The \textit{na} -marked clause indicates the condition for the event expressed in the main clause. In this case the clause marked by \textit{na} (‘a person gets sick) is not previously mentioned in the discourse, but rather is a fact of life, a cultural presupposition. 

\ea
Cəje  agan  ana  məze  \textbf{na},  tawəy  kəɗom  kəra.
\z

tʃɪdʒɛ   a-g  =aŋ   ana   mɪʒɛ   \textbf{na}   tawij   kʊɗ-ɔm     kəra



disease  3S-do  =3S.IO  DAT  person  PSP  3P+say  kill \textsc{IMP}{}-2P  dog



‘[If] a person gets sick (lit. sickness does to person), they say, “Kill a dog!” [for the sick person to eat].’ 


Ex. 864 and 865 are from another legend where God used to live very close to people. However one day, a woman did something that made God angry, and so he moved far away from them. The narrator expresses the relation between God becoming angry and his moving away using a \textit{na} construction (ex. 864) where the \textit{na} -marked clause indicates God’s anger (the reason for his leaving) and the main clause indicates the result. 

\ea
Hərmbəlom  na  ɓərav  ahan  atəkam  alay  \textbf{na},  avahay  ele  ahan  botot.
\z

Hʊrmbʊlɔm na  ɓərav   =ahaŋ        a-təkam     =alaj                 \textbf{na} 



God             PSP  heart      =3S.POSS   3S-taste      =away      PSP



‘[It is like this that] God got angry;’ (lit. God, he tasted his heart)



\textit{a-vah-aj  }ɛ\textit{l}ɛ\textit{  =ahaŋ    bɔtɔt}



3S-fly{}-CL  thing  =3S.POSS  \textsc{ID} flying



‘[And so] he went away.’ (lit. God got angry, he flew his thing)


Ex. 865 is from the conclusion where the narrator uses a \textit{na} construction to express a counterexpectation. Although people may seek paradise, they won’t find it because God has gone far away (because of what the woman did). In the \textit{na} construction, the \textit{na} -marked clause expresses what people seek, and the main clause expresses that they won’t find it.

\ea
Mənjokok  egəne  sləlay  mbəlom  \textbf{na},  Hərmbəlom  enjé  dəren.
\z

mə-nzɔk=ɔk\textsuperscript{w}  ɛgɪnɛ  ɬəlaj  mbəlɔm    \textbf{na} 



1\textsc{Pin}{}-seek/find -2\textsc{Pin}  today  root  sky    PSP



‘[Although] today we seek paradise, (lit. we all seek today the root of the sky,)



\textit{Hʊrmbʊlɔm}\textit{  }\textit{ɛ}\textit{{}-}\textit{nʒ-ɛ}\textit{    }\textit{dɪrɛŋ}



God    3S-left-CL  far



God has gone far away.’ (lit. God has gone far away.)


Ex. 866 is from the Values exhortation and illustrates a reason-result connection. There is no connecting conjunction in either of the clauses; however the reader can discern that there is a logical connection between the first clause (marked in five places with\textit{ na,} see \sectref{sec:12.5}) and the second (a rhetorical question, see \sectref{sec:77}).


Values S. 29


\ea
Hərmbəlom  \textbf{na},  amaɗaslava  ala  məze  \textbf{na},  ndahan  ese  \textbf{na},  kagas  ma  Hərmbəlom  \textbf{na},  asabay  \textbf{na},  [káagas  \textbf{na}  anga  way? ]  
\z

Hʊrmbʊlɔm  \textbf{na}  ama-ɗaɬ=ava=ala  mɪʒɛ   \textbf{na}  ndahaŋ  ɛʃɛ  \textbf{na}  



God      PSP    DEP-multiply=in=to   person    PSP     3S     again    PSP  



‘God, the one that multiplied the people,’



\textit{ka-gas    ma   Hʊrmbʊlɔm  }\textbf{\textit{na}}\textit{  }\textit{asa-baj         }\textbf{\textit{na}}\textit{  }



2S-catch   word     God          PSP  again-\textsc{NEG}  PSP   



‘[if] you will never accept the word of God,’



{}[\textit{káá-gas            }\textbf{\textit{na}}\textit{      aŋga     waj }]



2S.POT-catch  PSP   \textsc{POSS}  who



‘whose word will you accept then?’


\subsection{Simultaneous events}
\hypertarget{RefHeading1213201525720847}{}
When the verb in the \textit{na} clause is progressive aspect, the events/states in both clauses are simultaneous. Ex. 867 (from the Leopard story, Friesen, 2003) shows a \textit{na} clause that indicates a presupposed event that is occuring while the event in the main clause happens.\footnote{Ex. 867 is an example of tail-head linking \citep{Section1181}.} The verb \textit{a-ndaj ɛ-tuw-ɛ}  ‘she is crying’ is progressive aspect. Also see ex. 859, 860. 


\ea
Atəwalay  bababa  kəlak  kəlak  kəlak.  Anday  etəwe  \textbf{na},  anjakay  awak.
\z

a-tuw=alaj  bababa  kəlak kəlak kəlak



3S-cry=away    sound of hen



‘She cried, “Bababa kuluk kuluk.”



\textit{a-ndaj  ɛ{}-tuw-ɛ    }\textbf{\textit{na}}\textit{   a-nzak-aj   awak}



3S-PRG  3S-cry-CL  PSP  3S-find{}-CL  goat



‘While she was crying, she found a goat.’ 


\subsection{  Tail-head linking for cohesion}
\hypertarget{RefHeading1213221525720847}{}
In a discourse, the speaker will use several devices to ensure that the hearers can follow what is being said; i.e., to help track participants through the narrative, connect events, and understand logical connections. One of the ways cohesion is achieved in Moloko discourse is by the use of the presupposition marker \textit{na} to mark presupposed (including previously-introduced) information. Cohesion is also created using a special construction that Longacre calls ‘tail-head repetition’ (1976: 204). In this construction, an element previously mentioned in a discourse is repeated in a subsequent sentence in order to provide a cohesive link between new information and the preceding discourse. In Moloko, a clause on the eventline is first asserted and then at the beginning of the next sentence the same propositional content is repeated almost word for word and marked at the end by \textit{na}. Several examples are shown below. Ex. 868 comes from a different retelling of the Disobedient Girl text than is shown in \sectref{sec:1.5.} The final element of \textit{tə-h=aja na kə v}\textit{ɛ}\textit{r aka} ‘they ground it on the grinding stone’ is repeated in the next line and marked with \textit{na} as the first element of the next sentence\textbf{\textit{ }}\textit{tə-ndaj tá-h=aja na} ‘they were grinding it \textit{na.’} In ex. 868 - 873, the clause containing the element to be repeated is delimited by square brackets and the \textit{na}{}-marked clause in the next sentence is underlined. The element that is repeated in both clauses is bolded. 


\ea
Tázaɗ  na  háy,  war  elé  háy  bəlen  na,  [\textbf{təhaya  }na\textbf{  }kə  ver  aka.]\textbf{  }Tənday  \textbf{táhaya  }na,  həmbo  ga  ánday  ásak  ele  ahan  wəsekeke.  \textbf{      }
\z

tá-zaɗ    na     haj,      war      ɛlɛ        haj       bɪlɛŋ  na\textbf{      }



3P.IFV-take  3S.DO  millet  child  eye   millet  one      PSP     



‘They would take one grain of millet;’



{}[\textbf{\textit{tə-h  =aja  na        }}\textit{kə  vɛr  aka}\textit{ }]



3S.IFV-grind=\textsc{PLU}   3S.DO  on     stone  on



‘they ground it on the grinding stone.’



\textit{tə-ndaj             }\textbf{\textit{tá-h  =aja}}\textit{  }\textbf{\textit{na}}\textbf{\textit{   }}\textit{ }



3P.IFV-PRG  3P.IFV-grind  =\textsc{PLU}    PSP   



‘As they were grinding it,



\textit{h}\textit{ʊ}\textit{mbɔ  ga  á-ndaj             á-sak                     ɛlɛ      =ahaŋ       }\textit{wuʃɛkɛkɛ}



flour   ADJ  3S.IFV-PRG  3S.IFV-multiply    thing  =3S.POSS  \textsc{ID}multiply



‘the flour was multiplying all by itself.’


Another tail-head link can be seen a little further in the same narrative in ex. 869. 

\ea
 [Ánday  ásakaka. ]  Ánday  ásakaka  wəsekeke  na,   ver  árəhva  mbaf.
\z

{}[á-ndaj                á-sak                =aka 



3S.IFV-PROG       3S.IFV-multiply  =on



‘It is multiplying.’



\textbf{\textit{á-ndaj           á-sak               =aka}}\textit{    wu}\textit{ʃ}\textit{ɛkɛkɛ    }\textbf{\textit{na}}



3S.IFV-PRG  3S.IFV-multiply =on  \textsc{ID}multiplication  PSP



\textit{vɛr       á-rəh        =va  mbaf }



room             3S.IFV-fill   =\textsc{PRF}  up to the roof



‘As it is multiplying, the room filled completely up.’


Likewise, other tail head links can be seen in ex. 870 (from lines 3 - 5 in the Cicada text), ex. 871 (from lines 9 - 10 in the Snake story), and ex. 872 (from the Leopard story, Friesen, 2003). 


Cicada S. 3


\ea
Albaya  ahay  aba.  [\textbf{Tánday  t}ə\textbf{talay  a  l}\textbf{ə}\textbf{he.} ]  \textbf{Tánday  t}\textbf{ə}\textbf{talay  a  ləhe } \textbf{na},  tolo  tənjakay  ngəvəray  malan  ga  a  ləhe.
\z

albaja  =ahaj  aba   



youth          =Pl  \textsc{EXT}      



S. 4



{}[\textbf{\textit{tá-ndaj      t}}\textit{ə}\textbf{\textit{{}-tal-aj    a lɪhɛ }}]\textbf{\textit{ }}



3P.IFV-PRG     3P.IFV-walk{}-CL  to  bush



‘There were some young men. They were walking in the bush.’



S. 5



\textbf{\textit{tá-ndaj       t}}\textbf{\textit{ə}}\textbf{\textit{{}-tal-aj        a   l}}\textbf{\textit{ɪ}}\textbf{\textit{hɛ}}\textit{     }\textbf{\textit{na}}



 3P.IFV-PRG     3P-walk{}-CL  at    bush       PSP



\textit{tə-lɔ             tə-nzak-aj           }\textit{ŋ}\textit{gəvəraj        mala}\textit{ŋ}\textit{ ga  a    l}\textit{ɪ}\textit{hɛ}



3P.PFV-go  3P.PFV-find{}-CL  spp. of tree  large ADJ      at  bush



‘[As] they were walking in the bush, they found a large tree (a particular species) in the bush.’



Snake S. 9


\ea
Nazala  təystəlam  əwla,  [\textbf{nabay  c}əzlarr.]  \textbf{Nábay}  \textbf{na},  námənjar  na  mbajak  mbajak  mbajak  gogolvan.  
\z

nà-z        =ala  tijstəlam   =uwla    [\textbf{nà-b-aj            }tsəɮarr ]



1S.PFV-take =to     torch   \textit{       }=1S.POSS   1S.PFV-light{}-CL  \textsc{ID}shining the flashlight up



‘I took my flashlight, I shone it up.’



S. 10



\textbf{\textit{ná-b-aj}}\textit{   }\textbf{\textit{na}}\textit{   ná-mənzar          na   mbajak   mbajak    mbajak  g}\textit{\textsuperscript{w}}\textit{ɔg}\textit{\textsuperscript{w}}\textit{ɔlvaŋ} 



1S-light{}-CL    3S.DO   1S.IFV-see  PSP  \textsc{ID}something big and reflective   \textit{snake}



‘I shone [it], I was seeing (I wanted to see / I was trying to see), something big and reflective, a snake!’


\ea
{}[\textbf{At}\textbf{ə}\textbf{walay  }bababa  kəlak  kəlak  kəlak. ]  Anday  \textbf{etəwe}  \textbf{na},  anjakay  awak.
\z

{}[\textbf{a-tuw}=alaj\textbf{    }bababa  kəlak kəlak kəlak ] 



3S-cry=away    sound of hen



‘She cried, “Bababa kuluk kuluk.”



\textit{a-ndaj  }\textbf{\textit{ɛ{}-tuw-ɛ}}\textit{    }\textbf{\textit{na}}\textit{   a-nzak-aj   awak}



3S-PRG  3S-cry-CL  PSP  3S-find-CL  goat



‘As she was crying, she found a goat.’ 


Sometimes the tail and head elements are not identical. For example, the expected (but not overtly-named) result of a previous proposition can be expressed in a subsequent clause and that result marked with \textit{na}. Ex. 873 is from lines 27 and 28 of the Disobedient Girl text shown in Apendix 1.5. The first sentence (\textit{zar =ahaŋ à-ŋgala}) tells of the return of the husband. The next sentence is \textit{pɔk ma-p=alaj mahaj  }\textit{‘}opening the door,’ which is an expected event when a person returns home. The \textit{na }{}-marked clause in the second sentence is presupposed information since although it does not literally repeat the information in the previous sentence, it refers to information which is a natural outcome of it. The construction still provides cohesion to the text because subsequent events are linked together. 


Disobedient Girl S. 27


\ea
{}[Embesen  cacapa  na,  \textbf{zar  ahan  angala.} ]  Pok  mapalay  mahay  \textbf{na},  həmbo  árah  na  a  hoɗ  a  hay  ava.
\z

{}[ɛ{}-mbɛʃɛŋ  tsatsapa        na,     \textbf{zar      =ahaŋ         à-ŋgala} ]



3S-rest       some time     PSP  man  =3S.POSS  3S.PFV-return



After a while, her husband came back.



S. 28



\textit{p}\textit{ɔ}\textit{k        ma-p=alaj             mahaj    }\textbf{\textit{na}}\textit{  }



\textsc{ID}open  \textsc{NOM}{}-open=away  door     PSP  



\textit{h}\textit{ʊ}\textit{mb}\textit{ɔ}\textit{  á-rax           na      a     h}\textit{\textsuperscript{w}}\textit{ɔɗ}\textit{       a        haj    ava}



flour     3S.IFV-fill   3S.DO  at  stomach  GEN  house  in



‘Opening door, the flour filled the stomach (the interior) of the house.’


\section{  Presupposition-assertion construction: \textit{na-}marked clausal element}
\hypertarget{RefHeading1213241525720847}{}
The second type of presupposition-assertion construction occurs when a single clausal element is fronted and marked with \textit{na}. Na marks (occurs immediately after): a) presuppositions and b) topics (including contrastive topics). In both cases the clausal element immediately preceding \textit{na}  is part of an understood presupposition. The part of the clause following \textit{na} is the assertion which contains new information the speaker wants to communicate.

The normal order of elements in a Moloko clause (without \textit{na }) is SVO. \figref{fig:18}. illustrates the constituents in a declarative clause, combining \figref{fig:15}. and \figref{fig:16}. so that the verb phrase constituents are also shown.  

\begin{tabular}{ll}
\lsptoprule

(Discourse particle)   (Subject NP)

(Temporal adverb) & \textbf{Verb phrase}\\
& (Auxiliary)   \textbf{Verb complex       }(noun phrase       (Adpositional phrases)       (Adverb)       (Ideophone)

                     or ‘body-part’)                     \\
\lspbottomrule
\end{tabular}

\begin{itemize}
\item \begin{styleFiguretitle}
Constituents of the clause
\end{styleFiguretitle}\end{itemize}

In a presupposition-assertion construction, one (or more) of the clause or verb phrase elements is marked with \textit{na} and fronted with respect to the subject noun phrase and the verb phrase. The fronted construction is illustrated in \figref{fig:19}.. 

\begin{tabular}{llll}
\lsptoprule

(Discourse particle

or temporal adverb) & Fronted element + \textit{na} & (Subject noun phrase) & \textbf{Verb phrase}\\
\lspbottomrule
\end{tabular}
\begin{itemize}
\item \begin{styleFiguretitle}
Constituent order of Presupposition construction
\end{styleFiguretitle}\end{itemize}

The examples below show the presupposed element can be almost any element of the clause: the subject (ex. 874{}-875), the direct object (ex. 876{}-877), or an oblique (ex. 878 and 879). A discourse conjunction or temporal can also be marked as being presupposed (ex. 896{}-882). In each case, the fronted element is presupposed in the discourse – it is a known or culturally expected participant, location (spatial or temporal), or object. It is noteworthy that neither verbs by themselves, nor an existential word, nor ‘body-part’ incorporated nouns, nor ideophones can be fronted and marked as presupposed. In the following examples, the presupposed element is underlined and the presupposition marker \textit{na} is bolded. 

\textit{Na} -marked element      Assertion  


Cicada S. 19



\ea
Kəlen  bahay  \textbf{na},  olo  kə  mətəɗe  aka.
\z

kɪlɛŋ  bahaj  \textbf{na}    ɔ{}-lɔ          kə  mɪtɪɗɛ  aka



then    chief        PSP    3S.PFV-go        on  cicada  on



‘Then the chief, he went to the cicada.’  


\ea
Həmbo  \textbf{na},  anday  ásəkala  azla  wəsekeke.
\z

hʊmbɔ  \textbf{na}    à-ndaj     á-sək    =ala      aɮa   wuʃɛkɛkɛ



flour  PSP  3S.PFV-PRG   3S.IFV-multiply  =to  now    \textsc{ID} multiplication



‘The flour, it is multiplying.’


\ea
Ele  ahay  nendəye  \textbf{na},  nagala  kəyga  bay.
\z

ɛlɛ\textbf{     }=ahaj   nɛndijɛ \textbf{  na}  nà-g    =ala  kijga  baj



\textstyleExampleglossChar{thing  =Pl        DEM     PSP  1S.PFV-do  =to     like this  }\textstyleExampleglossChar{\textsc{NEG}}\textstyleExampleglossChar{           }



\textstyleExampleglossChar{‘These things, I have never done like this.’}


\ea
Ne  \textbf{na},  kónjokom  ne  asabay. 
\z

nɛ\textbf{    na}    kɔ-nzɔk-ɔm  nɛ  asa-baj 



1S    PSP    2P+IPF-find-2P  1S  again-\textsc{NEG}



‘As for me, you will never find me again.’



Cicada S. 18


\ea
Kə  mahay  aka  \textbf{na},\textbf{  }námbasaka  \textbf{na},  mama  ngəvəray  səlom  ga  lala.
\z

kə  mahaj    aka  \textbf{na}\textbf{  }  ná-mbas      =aka    \textbf{na}      mama   ŋgəvəraj    sʊlɔm  ga     lala



on  door  on  PSP  1S.IFV-rest  =on    PSP  mother  spp. of tree  good   ADJ   well



‘By my door, I will be able to rest well; the mother tree [is] good.’



Values S. 13


\ea
A  məsəyon  ava  \textbf{na},  ele  ahay  aməwəsle  \textbf{na},  tege  bay.  
\z

a   mɪsijɔŋ   ava   \textbf{na}   ɛlɛ   =ahaj   amu-wuɬ{}-ɛ\textbf{   na}   tɛ-g-ɛ    baj  



to  mission  in  PSP  thing  =Pl  DEP-forbid-CL  PSP  3P-do-CL  \textsc{NEG}



‘In the mission, these things that they have forbidden, they don’t do.’


Although the presupposition-assertion construction is structurally a clause level phenomenon, it can function in information structuring at the proposition level both to mark a boundary in a discourse, to set topic, and in participant tracking. When a discourse conjunction or temporal adverb is marked as presupposed (ex. 896 -882), the clause as a whole marks a boundary in the discourse. Such a clause often indicates a time change or an episode boundary. Most of the episodes in the Disobedient Girl story (see \sectref{sec:1.5}) begin with a conjuction marked with \textit{na} (ex. 881) or a \textit{na }{}-marked temporal phrase (ex. 880, 882).  All \textit{na}{}-marked elements are underlined in the examples. 


Disobedient Girl S. 3 (the beginning of the setting)


\ea
Zlezle  \textbf{na},\textbf{  }Məloko  ahay  \textbf{na},  Hərmbəlom  ávəlata  barka  va.      
\z

ɮɛɮɛ        \textbf{na}\textbf{  }Mʊlɔk\textsuperscript{w}ɔ  =ahaj  \textbf{na}  Hʊrmbʊlɔm    á-vəl=ata             barka   =va      



long ago  PSP  Moloko      =Pl   PSP  God    3S.IFV-send=3S.IO  blessing  =\textsc{PRF} 



‘Long ago, to the Moloko people, God had given his blessing. ’



Disobedient Girl S. 9 (the beginning of the\textit{ }episode 1)


\ea
Nde  ehe  \textbf{na},  albaya  ava  aba.                                                   
\z

ndɛ     ɛhɛ\textbf{   na}   albaja           ava    aba                                                   



so     here   PSP   young man  \textsc{EXT}{}-in   \textsc{EXT}                                                            



‘And so, there once was a young  man.’ 



Disobedient Girl S. 27 (the beginning of the dénouement)


\ea
Embesen  cacapa  \textbf{na},\textbf{  }zar  ahan  angala.
\z

ɛ{}-mbɛʃɛŋ  tsatsapa     \textbf{na}\textbf{  }zar    =ahaŋ       à-ŋgala



3S-rest  after some time   PSP   man  =3S.POSS  3S.PFV-return



‘After a while, her husband came back.’ 


The presupposition-assertion construction is also used to mark topic for participant shifts.\footnote{Called ‘subject’ in \citet{Chafe1976}. } The \textit{na }{}-marked element will be the main participant of the clauses that follow it, until there is another \textit{na }{}-marked clause-initial element.\footnote{\citet[151]{Lambrecht1994} says  “what is presupposed in a topic-comment relations is not the topic itself, nor its referent, but the fact that topic referent can be expected to play a role in a given proposition, due to its status as a center of interestor matter of concern in the conversation. It is this property that most clearly distinguishes topic arguments from focus arguments, whose role in the proposition is always unpredictable at the time of utterance…One therefore ought not to say that a topic referent “is presupposed” but that, given its discourse status, it is presupposed to play a role in a given proposition”} \textit{Na} can be thought of as a kind of spotlight, drawing attention to that already-known participant as one to which new or asserted information will be somehow related. Ex. 883 shows lines S. 12, 14, and 15 in the Disobedient Girl text. In S. 12, \textit{zar =ahaŋ} ‘her husband’ is marked with \textit{na}.{ }\footnote{The double \textit{na}{}-marked elements \textit{ʃɛŋ=ala na} ‘later’ and \textit{zar =ahaŋ na} ‘her husband’ function to build up tension (see Section Error: Reference source not found for further discussion).} He is the subject of all of the clauses until \textit{h}\textit{\textsuperscript{w}}\textit{ɔr} ‘the woman’ is marked with \textit{na} in S.14. Then, the woman is the subject of all the clauses until the flour is marked with \textit{na}  in S.23. \textit{Na }{}-marking thus functions here in shifting the spotlight from one participant as topic to another. In these examples, only the \textit{na}{}-marked participants are underlined. 


S.12


\ea
Sen  ala  na,  zar  ahan \textbf{ na},  dək  medakan  na  mənəye  ata.  Hor  \textbf{na},  ambəɗan  aka  awəy, “Ayokon  zar  golo.
\z

ʃɛŋ     =ala   na  zar  =ahaŋ    \textbf{na}



\textsc{ID}go   =to          PSP  man    =3S.POSS    PSP



‘Then her husband’ (lit. going, her husband)    



\textit{dək    mɛ-dak=aŋ            na        mɪ-nʒ-ijɛ     =ata}



\textsc{ID}show   \textsc{NOM}show=3S.IO   3S.DO   \textsc{NOM}{}-sit-CL   =3P.POSS



‘instructed her in their habits.’ (lit. instructing their ways)



S. 14 - 15


\ea
hor  \textbf{na}\textbf{,}  ambəɗan  aka  awəy  ayokon  zar  golo.     
\z

h\textsuperscript{w}ɔr           \textbf{na}  a-mbəɗ=aŋ        =aka 



woman            PSP  3S-change-3S.IO    =on



‘The woman replied.’



S. 15        



\textit{awij    ajɔk}\textit{\textsuperscript{w}}\textit{ɔŋ  zar  g}\textit{\textsuperscript{w}}\textit{ɔlɔ}



3S+say  agreed   man  VOC



‘She said, “Yes, my dear husband.”’~


Marking with \textit{na} can also mark contrastive topic.\footnote{I.e., a section of discourse will be ‘about’ that participant, instead of whatever the preceding section of discourse was about. } Ex. 885, which comes from a Moloko song, marks a participant shift but also functions to contrast the speaker’s situation with others just mentioned in the discourse.\footnote{This is called ‘contrastiveness’ in \citet{Chafe1976}. } 

\ea
Ndam  akar  ahay  ténje  a  avəya  ava.  Ne  \textbf{na},  nénje  nə  memle  ga.
\z

ndam akar  =ahaj  tɛ-nʒ{}-ɛ    a  avija    ava



people  theft  Pl  3P.IFV-sit-CL  in  suffering  in



\textit{nɛ     }\textbf{\textit{na}}\textit{     nɛ-n}\textit{ʒ{}-}\textit{ɛ     nə  mɛmlɛ   ga}



1S    PSP    1S.IFV-sit-CL  with  joy  ADJ



‘(On that day) thieves will be in suffering; [but] as for me, I will rest in joy.’


\section{  Assertion-presupposition construction: right-shifted \textit{na }{}-marked element}
\hypertarget{RefHeading1213261525720847}{}
The assertion-presupposition construction occurs when the (\textit{na }{}-marked) presupposed element is placed after the main clause. This construction is found in concluding statements that explain what has happened in a discourse.\footnote{It is also seen in some information questions \citep{Section1174}.} In ex. 886, from the concluding lines of a narrative, the \textit{na }{}-marked elements that occur in a dependent clause that occurs after the matrix clause explain the problem that the discourse deals with – the fact that cows have destroyed a field.\footnote{Note that the other two occurrences of \textit{na} in this example function in focus (Section Error: Reference source not found) and definiteness (\sectref{sec:12.4}), respectively. } 


\ea
Kógom  ala  na  memey,  sla  ahay  na  aməzəme  gəvah  \textbf{na}.
\z

kɔ-g\textsuperscript{w}{}-ɔm     =ala  na      mɛmɛj   ɬa   =ahaj   na   àmɪ-ʒʊm-ɛ   gəvax   \textbf{na}



2.IFV-do=1Pin  =to  PSP   how   cow  =Pl  PSP  DEP-eat{}-CL  field  PSP



‘What are you going to do [since] the cows ate up the field?’ (lit. you will do how, the cows having eaten the field)


In ex. 887, the \textit{na-}marked final element is a relative clause explaining that the woman had (by her act in the body of the narrative) brought a curse onto the Moloko people. 


Disobedient Girl S. 38


\ea
Metesle  anga  war  dalay  ngəndəye,  amazata  aka  ala  avəya  nengehe  ana  məze  ahay  \textbf{na}.    
\z

mɛ-tɛɬ-ɛ          aŋga    war    dalaj  ŋgɪndijɛ  



\textsc{NOM}{}-curse-CL   \textsc{POSS}   child    girl     DEM   



\textit{ama-z=ata     =aka  =ala   avija           nɛŋgɛhɛ  ana    m}\textit{ɪʒ}\textit{ɛ     =ahaj  }\textbf{\textit{na}}\textit{ }



 DEP{}-take=3P.IO =on   =to        suffering     DEM      DAT   person    =Pl   PSP



‘The curse [is] belonging to that young woman, the one that brought that suffering onto the people.’  


\section{Definite construction: \textit{na }–marked clausal element}
\hypertarget{RefHeading1213281525720847}{}
The Definite construction occurs when a non-fronted noun phrase is marked by \textit{na}. \figref{fig:18}. (in \sectref{sec:12.2}) shows the default order of constituents in a clause. In the definite construction, the \textit{na} –marked element is in its normal clausal position. In this construction, \textit{na}  functions in the realm of definiteness. Definiteness is defined by \citet[79]{Lambrecht1994} as signalling when “the referent of a phrase is assumed by the speaker to be identifiable to the addressee”. While definiteness is a separate function than presupposition, Lambrecht points out that definiteness is related to presupposition in that the definite article is a grammatical symbol for an assumption on the speaker’s part that the hearer is able to identify the definite element in a sentence – the speaker presupposes that the addressee can identify the referent designated by that noun phrase.  

In ex. 888 from the ‘Cows in the Field’ story, the \textit{na} marker is not cliticised at the end of a verb phrase or clausal element but is attached to the noun \textit{gəvax} ‘field’ within an adpositional phrase. This construction is simply identifying the field to be the one that the cows destroyed, definite and previously mentioned in the story, and not some other unidentified field. In the examples in this section, the \textit{na} –marked noun phrase is underlined and the adpositional phrase is delimited by square brackets.


\ea
Təzlərav  ta  ala  va  [a  gəvah  \textbf{na}  ava.]
\z

tə-ɮərav     ta  =ala  =va    [a  gəvax  \textbf{na}  ava]



3P.PFV-move out   3P.DO= to =\textsc{PRF}   to  field   PSP  in



‘They had driven them out of the field.’


Ex. 889 is  from the Disobedient Girl story. The house is marked as definite with \textit{na}. 


Disobedient Girl S. 26


\ea
Nata  ndahan  dəɓəsolək  məmətava  alay  a  hoɗ  [a  hay  \textbf{na}  ava. ]
\z

nata    ndahaŋ  dʊɓʊsɔlʊk         mə-mət=ava=alaj    a     h\textsuperscript{w}ɔɗ      [a        haj       \textbf{na}      ava ]



and then  3S            \textsc{ID}{}-collapse/die  \textsc{NOM}{}-die=in =away   at   stomach   GEN   house   PSP    in



And she collapsed dubusoluk, dying inside the house.


Likewise in ex. 890,  the noun \textit{mɪsijɔŋ} ‘church’ is marked as definite within the adpositional phrase \textit{a mɪsijɔŋ na ava }‘in the church.’ 


Values S. 3


\ea
Səwat  na\textbf{, }  təta  [a  məsəyon  na  ava ]  nəndəye  na,  pester  áhata,   “Ey, ele  nehe  na,  kógom  bay!”   
\z

suwat   na   təta   [a   mɪsijɔŋ   na  ava ]     nɪndijɛ  na       



\textsc{ID}disperse  PSP  3P      in  mission  PSP    in  DEM  PSP  



‘As they disperse, those in that mission,’



\textit{pɛʃtɛr  á-h    =ata}   \textit{ɛ}\textit{j    }\textit{ɛ}\textit{l}\textit{ɛ}\textit{      n}\textit{ɛ}\textit{h}\textit{ɛ}\textit{   na   k}\textit{ɔ}\textit{\'{ }-g}\textit{\textsuperscript{w}}\textit{{}-}\textit{ɔ}\textit{m    baj}



pastor  3S.IFV-tell  =3P.IO   hey    thing  DEM  PSP  2.IFV-do-2P    \textsc{NEG}



‘the Pastor told them, “Hey! These things, don’t do them!”’


Ex. 891 is from line S. 21 of the Snake story. The \textit{na} –marked element \textit{g}\textit{\textsuperscript{w}}\textit{ɔg}\textit{\textsuperscript{w}}\textit{ɔlvaŋ  na} ‘the snake’ follows the verb complex in its normal position of a direct object noun phrase within the verb phrase. 


S 21 


\ea
Alala,  nəzlərav  na  ala  gogolvan  \textbf{na}  a  amata  ava.
\z

a-l  =ala  nə-ɮərav        na         =ala  g\textsuperscript{w}ɔg\textsuperscript{w}ɔlvaŋ  \textbf{na}    a  amata   ava



3S-go=to  1S.PFV-exit   3S.DO  =to  snake        PSP    in   outside  in



‘Sometime later I caused the snake to exit outside.’


\section{Presupposition-focus construction: \textit{na }preceeds the final element of the verb phrase}
\hypertarget{RefHeading1213301525720847}{}
The presupposition-focus construction in Moloko makes prominent the final element of a clause.\footnote{Longacre and \citet[221]{Hwang2012} define prominence as “spotlighting, highlighting, or drawing attention to something.” } \textit{Na} preceeds the final element in the verb phrase.This is the only \textit{na} construction where the \textit{na}{}-marker follows the verb complex. In effect, all of that information that precedes the final element in the clause is marked as presupposed with \textit{na}. The result is that the final element in the clause is highlighted in the discourse.{ }

Ex. 892 is from line S. 20 of the Disobedient Girl text. The placement of \textit{na} postverbally, just before the final element in the clause (\textit{gam} ‘a lot’) functions to highlight that the woman prepared \textit{a lot} of millet. The fact that she prepared a lot of millet instead of just one grain (as she was instructed) is critical to the outcome of the story. An added effect of the \textit{na }plus pause before the final element is to slow down the narrative just a bit, resulting in heightened attention on the final element \textit{gam} ‘a lot.’ In the examples in this section, the prominent final element is bolded and the \textit{na} -marked elements are underlined.


Disobedient Girl S. 20 



\ea
Jo  madala  háy  \textbf{na},  \textbf{gam}\textbf{.}
\z

dzɔ              ma-d=ala            haj   \textbf{na}    \textbf{gam}  



\textsc{ID} take  \textsc{NOM}{}-prepare=to  millet     PSP   a lot



She prepared lots of millet.


Multiple elements in a clause or sentence that are marked with \textit{na} will add even more prominence to the final element. This kind of construction is seen at summation points in a narrative. It is also seen in a hortatory text when the speaker is reiterating his or her argument to make an important point. The many marked elements seem to slow down the action and build up tension towards the final element in the clause, thus putting even more emphasis on the focused item. In ex. 893, the fact that the woman’s habit where she came from was to grind a \textit{large amount} of millet at a time is crucial to the story. Three \textit{na}{}-marked elements (a subject noun phrase, the verb phrase, and the complement without its final element) precede the adverb \textit{gam} ‘a lot.’


Disobedient Girl S. 17 


\ea
Nde  hor  \textbf{na},  asərkala  afa  təta  va  \textbf{na},  aməhaya  háy  \textbf{na},  \textbf{gam.}   
\z

ndɛ  h\textsuperscript{w}ɔr     \textbf{na}  à-sərk=ala             afa              təta   =va      \textbf{na}   



so  woman  PSP     3S.PFV-HAB=to  at place of  3P  =\textsc{PRF}  PSP  



\textit{amə-h=aja           haj        }\textbf{\textit{na}}\textit{  }\textbf{\textit{gam}}



DEP-grind=\textsc{PLU}  millet  PSP  alot



‘The woman, she had been habituated at her house, she ground millet, a lot.’ 


In ex. 894, there are a series of six \textit{na} -marked elements that reiterate some of the main points of argument that the speaker used. The final element \textit{aŋga waj} ‘whose [word]’ is made prominent and the effect is to cause the hearer to think about whose word the people accept (based on their behaviour). 


Values S. 29


\ea
Hərmbəlɔm  \textbf{na},  amaɗaslava  ala  məze  \textbf{na},  ndahan  ese  \textbf{na},  kagas  ma  Hərmbəlom  \textbf{na}  asa  bay  \textbf{na},  ka\'{ }agas  \textbf{na},  \textbf{anga  way}?   
\z

Hʊrmbʊlɔm   \textbf{na}   ama-ɗaɬ=ava=ala     mɪʒɛ     \textbf{na}     



God                 PSP    DEP-multiply=in=to      person   PSP   



\textit{ndaha}\textit{ŋ}\textit{   ɛʃɛ        }\textbf{\textit{na}}\textit{     }\textit{ka-gas       ma       Hʊrmbʊlɔm    }\textbf{\textit{na}}\textit{   }



3S           again    PSP     2S-catch    word      God                PSP   



‘And if you will never accept the word of God, the one that multiplied the people,’ (lit. God, the one that multiplied people, he again, you accept God’s word)



\textit{asa    baj       }\textbf{\textit{na}}\textit{     }\textit{k}\textit{a}\textit{\'{ }}\textit{a}\textit{\'{ }}\textit{{}-gas     }\textbf{\textit{na}}\textit{     }\textbf{\textit{aŋga      waj}}



again  \textsc{NEG}   PSP       2S.POT-catch  PSP    \textsc{POSS}    who



‘\textit{whose} word will you accept then?’ (lit. not anymore, you want to accept,  [word of] who?) 


In both ex. 895 and 896, the final prominent element is \textit{dzijga}  ‘all.’ The effect is to emphasise the totality of the events. In ex. 895, the fact that \textit{all} of the field was destroyed by the cows is important to the story. In ex. 896, the story teller is emphasising that it was important that \textit{everyone} fought against the Mbuko. In fact, people who did not fight were beaten after the skirmish with the Mbuko ended.

\ea
Waya  sla  ahay  \textbf{na},  tozom  gəvah  \textbf{na},\textbf{  j}\textbf{əy}\textbf{ga}   \textbf{anga  l}\textbf{ə}\textbf{me  zlom.}    
\z

waja   ɬa      =ahaj     \textbf{na}  tɔ-zɔm        gəvax  \textbf{na}     



because  cow    =Pl  PSP  3P.PFV-eat   field   PSP   



\textbf{\textit{dzijga}}\textit{   }\textbf{\textit{aŋga    lɪmɛ      ɮɔm}}



all      \textsc{POSS}    1\textsc{Pex}   five



‘Because those cows, they ate all of that field that belonged to the five of us.’ (lit. because the cows, they ate the field, all of it, belonging to us five)


\ea
Nde  \textbf{na},  ləme  \textbf{ɗəw},  nəzləgom  va  \textbf{na},  \textbf{jəyga}.
\z

ndɛ   \textbf{na}  lɪmɛ   \textbf{ɗuw}  nə-ɮʊg-ɔm     va    \textbf{na}   \textbf{dzijga}



so   PSP  1\textsc{Pex}  also  1.PFV-plant-1\textsc{Pex}  body  PSP  all



‘So, we also, we fought (lit. planted body), all of us.’


In ex. 897, two \textit{na }{}-marked elements leave a negative particle highlighted at the end. The fact that the storytellers did not eat the people’s food was important since they would have been expected to eat.

\ea
Nde  kəy  elé  \textbf{na},  nəzəmom  ele  atəta  \textbf{na},  \textbf{ba}\textbf{y}.
\z

ndɛ  kij     ɛlɛ   \textbf{na}  nə-zʊm-ɔm    ɛlɛ   =atəta     \textbf{na}   \textbf{baj}



so  \textsc{ID}looking  eye  PSP  1.PFV-eat-1\textsc{Pex}    thing  3P.POSS  PSP  \textsc{NEG}



‘So, one could see that we had not eaten their food.’ (lit. looking, we ate their thing, not)


In the Disobedient Girl peak, four \textit{na}{}-marked elements precede the expression of the most pivotal event in the narrative – the death of the girl (bolded in ex. 898). 


Disobedient Girl S. 26


\ea
Alala  \textbf{na},  ver  \textbf{na},  árah  mbəf  nə  həmbo  \textbf{na},  ɗək  məɗəkaka  alay  ana  hor  \textbf{na},  \textbf{nata  ndahan  mat  mə-mətava  alay  a  ho}\textbf{ɗ}\textbf{  a  hay  na  ava}.   
\z

a-l=ala   \textbf{na}   vɛr   \textbf{na}  á-rax     mbəf     nə   hʊmbɔ   \textbf{na}~   



3S-come=to  PSP  kitchen  PSP  3S.IFV-fill  to the roof  with  flour  PSP  



\textit{ɗək     mə-ɗək=aka=alaj   ana   h}\textit{\textsuperscript{w}}\textit{ɔr   }\textbf{\textit{na}}



suffocate  \textsc{NOM}{}-suffocate=on=away  DAT  woman  PSP



\textbf{\textit{nata}}\textit{   }\textbf{\textit{ndahaŋ   mat   mə-mət=ava =alaj   a   h}}\textbf{\textit{\textsuperscript{w}}}\textbf{\textit{ɔ}}\textbf{\textit{ɗ}}\textbf{\textit{   a   haj   na   ava}}



then    3S  die  \textsc{NOM}{}-die=in=away  in  stomach  GEN  house  PSP  in



‘Later, the kitchen, it filled to the roof with flour, it suffocated the woman, and then she died in the stomach of that house.’


The 3S DO pronominal \textit{na} (\sectref{sec:50}) is identical to the presupposition marker \textit{na} and some ambiguity can be encountered in contexts where \textit{na} immediately follows a verb that has no locational or directional extensions. Ex. 899{}-900 show two such examples. In ex. 899, even though without a context the underlined \textit{na}  could be interpreted as the 3S DO pronominal for the verb \textit{mədak=aŋ} ‘instructing to him,’ we consider it to be the presupposition marker since there are multiple \textit{na }{}-marked elements in the clause and this final underlined \textit{na }appears immediately before the final focussed element \textit{mɪ-nʒ-ijɛ =ata} ‘their habits.’ 


Disobedient Girl S. 12 


\ea
Sen  ala  \textbf{na}\textbf{,}  zar  ahan  \textbf{na}\textbf{,}  dək  mədakan  \textbf{na}  mənjəye  ata.
\z

ʃɛŋ =ala       \textbf{na}  zar     =ahaŋ     \textbf{na}  dək        mədak  =aŋ    \textbf{na}    mɪ-nʒ-ijɛ     =ata 



\textsc{ID}go =to   PSP   man  =3S.POSS  PSP   \textsc{ID}show  \textsc{NOM}show=3S.IO  PSP  \textsc{NOM}{}-sit-CL =3P.POSS



‘Then her husband instructed her their habits.’(lit. going, her husband, instructing to her, their sitting)


In contrast, in ex. 900 we consider the two underlined \textit{na}  markers to be the 3S DO pronominal since even though there are multiple the \textit{na}{}-marked elements in the clause, these underlined markers are not immediately before the final focussed element. We found no unambiguous instance of the presupposition marker \textit{na} breaking up a verb phrase except for the purpose of isolating the final focussed element in a verb phrase (cf. integrity of the VP, \sectref{sec:8.1}). Thus the first underlined\textit{ na}  is 3S DO for the verb \textit{tozom} ‘they eat.’ Likewise, we found no unambiguous instance of the presupposition marker breaking up a noun phrase in any context and so consider the second underlined\textit{ na} as 3S DO pronominal for the nominalised verb \textit{mɪ-g-ijɛ} ‘doing’ within the noun phrase \textit{mɪ-g-ijɛ na =ahaŋ} ‘his doings.’ The verb and noun phrases in question are each delimited by square brackets in the example. 

\ea
Nde  asa  bahay  a  sla  \textbf{na}\textbf{,}  ndahan  aka  bay  \textbf{na,  }asa  sla  ahay  \textbf{na},  [tozom  na  gəvah ]  \textbf{na},  ɗeɗen  \textbf{na},  ndahan  \textbf{na},  ámənjar  nə  elé  ahan  bay  \textbf{na},  [məgəye  na  ahan ]  \textbf{na}  memey?
\z

ndɛ  asa   bahaj  a  ɬa  \textbf{na}  ndahaŋ  aka   baj  \textbf{na}\textbf{ }



so  if  chief  GEN  cow  PSP  3S  on  \textsc{NEG}  PSP



‘So, if the owner of the cows wasn’t there,’



\textit{asa    ɬa   =ahaj   }\textbf{\textit{na}}\textit{   }[\textit{t}\textit{ɔ}\textit{{}-zɔm    }\textbf{\textit{na}}\textit{  gəvax }]\textit{    }\textbf{\textit{na}}\textit{  }\textit{ɗ}\textit{ɛ}\textit{ɗ}\textit{ɛŋ   }\textbf{\textit{na}}



if  cow  Pl  PSP  3P.PFV-eat  3S.DO  field  PSP  truth  PSP



‘[and] if it is true that the cows really destroyed the fields,’



\textit{ndahaŋ   }\textbf{\textit{na}}\textit{    á-mənzar   nə     ɛlɛ      =ahaŋ      baj    }\textbf{\textit{na}}\textit{ }



3S    PSP  3S.IFV-see  with  thing  =3S.POSS  \textsc{NEG}  PSP



‘[then] he, since he hasn’t seen it for himself,’



{}[\textit{mɪ-g-ijɛ      }\textit{na}\textit{    =ahaŋ }]\textit{    }\textbf{\textit{na}}\textit{  mɛmɛj}



\textsc{NOM}{}-do{}-CL    3S.DO  =3S.POSS  PSP  how



‘what is he supposed to do?’ (lit. his doing, how)


\chapter[Clause combining]{Clause combining}
\hypertarget{RefHeading1213321525720847}{}
In Moloko, clauses combine within sentences and larger structures in one of six main ways: 


\begin{itemize}
\item Complement clause (\sectref{sec:13.1}). The complement clause is an argument within the matrix clause (subject, direct object, or indirect object).
\item Dependent adverbial clause (\sectref{sec:13.2}). A clause giving adverbial information concerning the verb in the matrix clause. 
\item Relative clause. Relative clauses are clauses embedded in a noun phrase within the matrix clause. These are discussed in \sectref{sec:44.}
\item A clause linked to another clause by a conjunction (\sectref{sec:13.3}). 
\item A clause marked with \textit{na} or other expectation marker. The \textit{na}{}-marked clause contains known or presupposed information. This construction is discussed in Chapter 20. 
\item Juxtaposition of two independent clauses (\sectref{sec:13.4}).
\end{itemize}
\section{Complement clauses}
\hypertarget{RefHeading1213341525720847}{}
A complement clause is a clause that is an argument in the matrix clause. Complement clauses in Moloko can contain one of three verb forms: dependent, nominalised or finite. When the complement clause has the same subject as the main clause, the complement clause has a dependent or nominalised verb form (\sectref{sec:82}). When the subject of the complement clause is different than that in the main clause, the verb in the complement clause is finite and the clause has a noun phrase subject (\sectref{sec:83}). 

\subsection{Dependent and nominalised verb complement clauses}
\hypertarget{RefHeading1213361525720847}{}
The complement clause is embedded in the verb phrase as a subject, direct object, or indirect object within the matrix clause.  Nominalised and dependent verb forms collocate with constructions that carry different modal/aspectual values. The nominalised form collocates with constructions that carry a finished, accomplished idea or imply a state,\footnote{The situation refers to something that occurred in the past with ongoing effects to the point of reference.} e.g., \textit{ndav} ‘finish,’ ex. 901 (see \sectref{sec:7.6}). In contrast, the dependent form is found in constructions that have an incomplete or unachieved idea at the time of the matrix situation, e.g., the verb of inception \textit{ɮaŋ} ‘begin’ (ex. 907, see \sectref{sec:7.7}). In the examples in this section, the subject of the complement clause is the same as the subject of the matrix clause (and is indicated by Ø in the examples). The clauses are delimited by square brackets and the verb is bolded.

In ex. 901 the nominalised form is the direct object of the matrix verb \textit{ndav} ‘finish.’ 


\ea
{}[Nəndavalay  [  \textbf{m}ə\textbf{w}əce. ] ]
\z

{}[nə-ndav=alaj    [Ø\textbf{   mu-wutʃ-ɛ} ] ]



1S-finish  =away      \textsc{NOM}{}-write{}-CL



‘I finish writing.’


A clause with the nominalised form can function as an argument of another verb. It is the subject in ex. 902 and the direct object in ex. 903 - 906. In each case, the nominalised form gives a completed idea to the action on the verb.

\ea
 [[\textbf{M}ə\textbf{mb}ə\textbf{ɗe  }ahan   na ],  asaw.]
\z

{}[[\textbf{mɪ-mbɪɗ-ɛ}    =ahaŋ    na ]   a-s=aw ]



\textsc{NOM}{}-remain-CL  =3S.POSS  PSP  3S-please=1S.IO



‘I want the leftovers.’ (lit. its remains pleases me) 


\ea
Bahay  amakay  \textbf{m}əzə\textbf{me  sese.} 
\z

{}[bahaj   à-mak-aj   \textbf{mɪ-ʒʊm-ɛ }  ʃɛʃɛ ]



chief  3S.PFV-leave-CL  \textsc{NOM}{}-eat-CL  meat



‘The chief stopped eating meat.’ (lit. the chief left the eating of meat) 


\ea
{}[Nasar  [\textbf{mədəye}  ɗaf ]  bay.]
\z

{}[na-sar   [\textbf{mɪ-d-ijɛ}      ɗaf ]  baj ]



1S-know  \textsc{NOM}{}-prepare-CL  loaf  not



‘I don’t know how to make millet loaves.’ (lit. I don’t know the preparing of millet loaf).



Disobedient Girl S. 4


\ea
{}[Ávata  [\textbf{m}ə\textbf{v}əye  haɗa.] ]
\z

{}[á-v=ata     [\textbf{mɪ-v-ijɛ}    haɗa ] ]



3S.IFV-last=3P.IO  \textsc{NOM}{}-last-CL  many



‘It will last them many years.’  (lit. it will last for them enough lastings; the nominalised form of the verb ‘last’ has been lexicalized as ‘year’)


\ea
{}[Ege  [\textbf{m}ə\textbf{v}əye  məko ]  ehe,]  [nawas  háy  əwla.]
\z

{}[ɛ-g-ɛ  [\textbf{mɪ-v-ijɛ  }  mʊk\textsuperscript{w}ɔ ]  ɛhɛ ],  [na-was    haj  =uwla ]



3S-do-CL  \textsc{NOM}{}-last-CL  six  here  1S-cultivate  millet  =1S.POSS



‘Six years ago (lit. it did six years), I cultivated my millet.’ 


In contrast, the dependent form is found in adverbial clauses that have an incomplete or unachieved idea at the time of the matrix situation, e.g., the verb of inception \textit{ɮaŋ} ‘begin’ (ex. 907 and 908, see \sectref{sec:7.7}) and habitual aspect (ex. 909 and 910). The writing hasn’t started in ex. 907; the referent isn’t necessarily eating at the moment of ex. 909; the fight was just beginning at the time of ex. 908). In each case, the dependent form is the direct object of the matrix clause.

\ea
{}[Nazlan  [\textbf{am}ə\textbf{w}əce\textbf{.}] ]
\z

{}[na-ɮaŋ    [Ø\textbf{   amu-wutʃ-ɛ }] ]



1S begin       DEP-write-CL



‘I begin to write.’ 


\ea
{}[Tazlan  a  ləme  [\textbf{aməzləge  }va.] ]   
\z

{}[ta-ɮaŋ     a  lɪmɛ  [Ø\textbf{   amɪ-ɮɪg-ɛ}  va ] ]   



3P-begin   at  1\textsc{Pex}    DEP-plant-CL  body  



‘We started to fight.’ (lit. they started to us planting bodies) 


\ea
{}[Asarkva  [\textbf{aməpəɗe  }sese.] ] 
\z

{}[a-sark   =va  [Ø\textbf{   amɪ-pɪɗ-ɛ}   ʃɛʃɛ ] ] 



3S-HAB   =\textsc{PRF}    DEP-crunch-CL  meat



He usually eats meat.’ (lit. He had a habit to eat meat)



Disobedient Girl S. 17


\ea
{}[Hor  na,  asərkala  afa  təta  va\textbf{  na},  [\textbf{aməhaya  }háy  na  gam.] ]    
\z

{}[h\textsuperscript{w}ɔr  na  a-sərk  =ala   afa   təta  va\textbf{  na}    



woman    PSP  3S-HAB =to  at house  3P  \textsc{PRF}  PSP  



{}[Ø  \textbf{\textit{amə}}\textit{{}-}\textbf{\textit{h=aja }}\textit{       haj       na    gam}] ]  



DEP-grind=\textsc{PLU}  millet  PSP  lots



‘The woman, she was used to at their house (before she was married) to grinding a lot of millet.’


Dependent clauses are also found in complement clauses for verbs of desire. For example, the complement clause for the verb  /s/ ‘please’ in ex. 912 -  913 expresses the unrealised object of the desire (with an incomplete or unfinished aspect). Note that the semantic LOC of the desire for the verb /s/ ‘please’ is the understood subject of the complement clause (\textit{=aw} ‘me’ in ex. 912).\footnote{Compare with ex. 922 in \sectref{sec:1183} where the subject of the complement clause is different.} 

\ea
{}[Asaw  [\textbf{am}ə\textbf{p}e\textbf{ɗe}  sese.] ]
\z

{}[a-s=aw [${\emptyset}$  \textbf{amɪ-p}ɛ\textbf{ɗ{}-}ɛ\textbf{ }  ʃɛʃɛ] ]



3S-want=1S.IO  DEP-crunch-CL  meat



‘I want to eat meat.’ (lit. It pleases me to eat meat) 


\ea
{}[Asan  [amadata  aka  va  azan.] ]
\z

{}[a-s=aŋ     [Ø   ama-d=ata   =aka     =va   azaŋ ] ]



3S-want=3SD\={ }    DEP-prepare=3P.IO =on     =\textsc{PRF}  temptation



‘He wanted to tempt them.’ (lit. to prepare a temptation for them [is] pleasing to him)



Race\footnote{Friesen, 2003.}


\ea
{}[A\textbf{s}aw  [\textbf{aməgəye}  ambele  mbele  nə  moktonok. ] ]
\z

{}[a-\textbf{s}=aw  [Ø\textbf{   amɪ-g-ijɛ}    ambɛlɛ mbɛlɛ  nə  mɔk\textsuperscript{w}tɔnɔk\textsuperscript{w} ] ]



3S-want=1S.IO    DEP-do-CL  race    with  toad



‘I want to race with the toad.’ (lit. to do a race with the toad [is] pleasing to me)


\subsection{   Finite complement clauses}
\hypertarget{RefHeading1213381525720847}{}
Finite complement clauses are used with verbs that express propositional attitude, with verbs of speech, and with verbs of desire. The complement clause is the direct object of verbs of these propositional attitude  (‘know,’ ex. 914,\footnote{Note that in this example, the complement clause and the DO pronoun \textit{na} in the matrix clause are co-referential.} ‘think,’ ex. 915, ‘believe,’ ex. 916, ‘doubt,’ ex. 917, ‘forget,’ ex. 918, or ‘worry,’ ex. 919). 


\ea
{}[Nasar  na  va,  [bahay  \textbf{apaɗə}va  sese.] ] 
\z

{}[na-sar na   =va   [bahaj   \textbf{à-paɗ-ə}    =va   ʃɛʃɛ ] ] 



1S-know  DO  =\textsc{PRF}  chief  3S.PFV-crunch  =\textsc{PRF}  meat



‘I know [that] the chief ate meat.’ 


\ea
{}[Nəɗəgalay  na,  [bahay  \textbf{apaɗə}va  sese.] ]
\z

{}[nə-ɗəgal-aj   na   [bahaj   à-paɗə    =va   ʃɛʃɛ ] ]



1S-think{}-CL  PSP  chief  3S.PFV-crunch  =\textsc{PRF}  meat



‘I think [that] the chief ate meat.’ 


\ea
{}[Nafaɗ  na,  [bahay  \textbf{apaɗə}va  sese.] ]
\z

{}[na-faɗ   na   [bahaj   \textbf{à-paɗ-ə    =va}   ʃɛʃɛ ] ]



1S-place  PSP  chief  3S.PFV-crunch  =\textsc{PRF}  meat



‘I believe [that] the chief ate meat.’ (lit. I place, the chief already ate meat) 


\ea
{}[Nəkaɗ  waya  na,  [bahay  \textbf{apaɗə}va  sese.] ]
\z

{}[nəkaɗ   waja   na   [bahaj   \textbf{à-paɗ-ə    =va}   ʃɛʃɛ ] ]



1S-kill  because  PSP  chief  3S.PFV-crunch  =\textsc{PRF}  meat



‘I doubt [that] the chief ate meat.’  (lit. I counsel that) 


\ea
{}[Acəkəzlaw  a  har  ava  [bahay  \textbf{apaɗə}va  sese.] ]
\z

{}[a-tʃəkəɮ=aw   a  har  ava  [bahaj   \textbf{à-paɗ-ə    =va}  ʃɛʃɛ ] ]



3S-forget=1S.IO  in  body  in  chief  3S.PFV-crunch  =\textsc{PRF}  meat



‘I forgot [that] the chief ate meat.’  


\ea
{}[Nazlaway  na,  [bahay  \textbf{apaɗə}va  sese.] ]
\z

{}[na-ɮaw-aj   na   [bahaj   \textbf{à-paɗə    =va   }ʃɛʃɛ ] ]



1S-fear{}-CL  PSP  chief  3S.PFV-crunch  =\textsc{PRF}  meat



‘I am afraid [that] the chief ate meat.’


Indirect speech is often expressed using a complement clause with a finite verb (ex. 920 and 921).

\ea
{}[Ne  awəy  [bahay  \textbf{apa}\textbf{ɗ}\textbf{əva}  sese.] ]
\z

\textit{[}nɛ  awij   \textit{[}bahaj  \textbf{à-pa}\textbf{ɗ}\textbf{ə    =va}  ʃɛʃɛ\textit{ ] ]}


1S  say  chief  3S.PFV-crunch=\textsc{PRF}  meat



‘I said [that] the chief ate meat.’


\ea
{}[Məloko  ahay  baba  ahay  ta  awəy  na,  [Hərmbəlom\textbf{  ege  }ɓərav  va  kə  war  anga  məze  dedelen  ga  aka.] ]
\z

{}[Mʊlɔk\textsuperscript{w}ɔ  =ahaj   baba   =ahaj  ta  awij         na



Moloko     =Pl      father  =Pl     3P say    PSP



‘The Moloko fathers say [that]’



{}[\textit{Hʊrmbʊlɔm   }\textbf{\textit{ɛgɛ }}\textit{             }\textit{ɓ}\textit{ərav   =va       kə     war  aŋga  mɪʒɛ  dɛdɛlɛŋ  ga  aka}] ]



God                 3S.PFV-do  heart    =\textsc{PRF}  on  child     \textsc{POSS}     person    black  ADJ    on



‘God got mad at the black people.’ (lit. God did heart on the child that belongs to black person)


Complement clauses with irrealis verbs are used to express desires and actions connected with the will (‘wish,’ ‘want,’ ‘hope’).  Ex. 922  shows a complement of the verb \textit{a-s=aw } ‘it pleases me.’ The complement shows the object of the desire expressed in the matrix clause. The complement has a different subject than the LOC of the desire in the martrix clause. The verb is finite and the subject is made explicit.\footnote{Compare with ex. 911 from \sectref{sec:1182} which shows a complement with the same subject as the location of the desire in the matrix clause.} Note that almost all of the following examples are elicited, and therefore the first clause is followed by the presupposition marker \textit{na} (Chapter 12). This marker indicates that the first clause contains presupposed (mutually known) information (in this case, the presupposition was established by the elicitation question).

\ea
Asaw  na,  [bahay  \textbf{mapaɗay  }sese.]
\z

a-s=aw   na   [bahaj   \textbf{ma-paɗ-aj}     ʃɛʃɛ ]



3S-please=1S.IO  PSP  chief  3S.\textsc{HOR}{}-crunch{}-CL  meat



‘I want the chief to eat meat.’ (lit. it pleases me, the chief should eat meat). 


In ex. 922 above, the complement clause is the subject of the main verb; in ex. 923, it is the indirect object.\footnote{In Moloko the indirect object uniformly expresses the semantic LOC (e.g., recipient or beneficiary, see Chapter 9). In this case, the metaphorical location of the imploring is its purpose – where the speaker wants to go with his actions towards the chief. The semantic Theme (the chief) is being persuaded to the LOC (eating meat). } 

\ea
Nədəbakay  bahay  na  ana  [\textbf{mazom}  sese.] 
\z

nə-dəbak-aj   bahaj   na   ana   [\textbf{ma-zɔm }  ʃɛʃɛ ] 



1S-implore{}-CL  chief  PSP  DAT  3S.\textsc{HOR}{}-eat  meat



‘I persuaded the chief to eat meat.’ (lit. I implored the chief to he should eat meat)


\section{  Dependent adverbial clauses}
\hypertarget{RefHeading1213401525720847}{}
Adverbial clauses give oblique information concerning the verb in the matrix clause. The adverbial clause containing a dependent verb is embedded in the main clause as the first or last element. Adverbial clauses before the matrix clause (ex. 924 - 926) function to express an event in progress at the time of the matrix event or situation. The entire adverbial clause is delimited by square brackets and the dependent verb is bolded in the examples. The subject of the dependent adverbial clause must be known in the context and will be marked with Ø (ex. 924) or a pronoun (ex. 925 - 926). 


Disobedient Girl S. 5



\ea
{}[[A\textbf{məhaya}  kə  ver  aka  na,]  tázaɗ  war  elé  háy  b\textbf{ə}len.]
\z

{}[[Ø  \textbf{amə-h=aja}  kə  vɛr  aka  na]  tá-zaɗ          war   ɛlɛ  haj     bɪlɛŋ ]



DEP-grind=\textsc{PLU}  on     stone  on  PSP  3P.IFV-take  child   eye  millet  one



‘While grinding (lit. to grind on the stone), they took one grain of millet.’ 


\ea
{}[[Ndahan  \textbf{aməcen  }məbele  a  mbəko  ahay  dəreffefe  na,]  awəy, “Almay?” ]
\z

{}[[ndahaŋ  \textbf{amɪ-tʃɛŋ  }mɪ-bɛl-ɛ    a  mbʊk\textsuperscript{w}ɔ  =ahaj  dɪrɛffɛfɛ    na ]  awij  almaj ]



3S         DEP-hear  \textsc{NOM}{}-move-CL  GEN  Mbuko  =Pl  \textsc{ID}EOmovement  PSP  he said  what



‘He, hearing the movement of the Mbuko (lit. he to hear moving of Mbukos \textit{direfefe}), he said, “What!”’ 



Disobedient Girl S. 16


\ea
 [ [Ndahan  \textbf{amandava}  ɓəl  na,]  zar  ahan  olo  ametele  kə  dəlmete  ahan  aka  a  slam  enen.   
\z

{}[ [ndahaŋ  \textbf{ama-nd=ava}  ɓəl  na],  zar   =ahaŋ           ɔ{}-lɔ           amɛ-tɛl-ɛ   



3S         DEP-sleep=in    \textsc{ID}some  PSP  man  =3S.POSS  3S.PFV-go   DEP-walk-CL  



‘She, sleeping there for some time, her husband went away to walk’



\textit{kə  dɪlmɛtɛ     =ahaŋ      aka    a    ɬam    ɛnɛŋ}



on   neighbor    =3S.POSS  on   at   place   another



‘in the neighborhood to some place.’ 


Adverbial clauses that occur after the main clause (ex. 927{}- 929) occur with verbs of movement (/l\textsuperscript{o}/ ‘go,’ /nz\textsuperscript{e}/ ‘left’).\footnote{We have not found adverbial clauses with other verbs.}{ }The dependent clause expresses the (as yet unachieved) purpose of the going. The adverbial clause does not express whether the purpose was achieved or not - in ex. 927 the reader does not know if he actually bought fish or not, although it is expected. 

\ea
{}[Olo  a  kosoko  ava  [\textbf{aməsək}\textbf{w}\textbf{əme  }kəlef.] ]
\z

{}[ɔ-lɔa   kɔsɔk\textsuperscript{w}ɔ   ava   [${\emptyset}$  \textbf{amɪ-sɪk}\textbf{\textsuperscript{w}}\textbf{ø}\textbf{m-ɛ}   kɪlɛf ] ]



3S-go  in  market  in    DEP-buy/sell-CL  fish



‘He/she went to the market [in order] to buy fish.’


\ea
{}[Kəlen  zar  ahan  na,  enjé  ele  ahan  [\textbf{am}ə\textbf{tele.}] ]
\z

{}[kɪlɛŋ  zar  ahaŋ    na  ɛ{}-nʒ-ɛ    ɛlɛ  ahaŋ      [Ø\textbf{   amɪ-tɛl-ɛ }] ]



then    man  =3S.POSS  PSP  3S-leave-CL  thing  =3S.POSS  DEP-walk-CL



‘Then her husband left to go walking [somewhere].’ 


Likewise, in ex. 929, the reader does not know if the young men actually succeed at bringing back the tree (and indeed the story reveals that they do not succeed, \sectref{sec:1.6}).


Cicada S. 16


\ea
{}[Kəlen  albaya  ahay  tolo  [\textbf{amazala}  ngəvəray  na.] ]
\z

{}[kɪlɛŋ  albaja    =ahaj  tɔ-lɔ  [Ø\textbf{   ama-z=ala}  ŋgəvəraj    na ] ]



then    young men  =Pl  3P-go     DEP-take=to  spp of tree  PSP



‘Then the young men went to try to bring back the tree [to the chief].’


A lengthened prefix vowel in the dependent form in an adverbial clause can also express mood (the desire of the speaker, see \sectref{sec:53}). The lengthened prefix vowel (bolded in ex. 930) expresses potential actions which are not yet complete or even expected, but they are desired by the speaker.

\ea
{}[Bahay  atəta  ahay  ɗəw  tólo  dəren  [\textbf{amaak}\textbf{ə}\textbf{wala}  ele  məzəme.] ]  
\z

{}[bahaj  =atəta     =ahaj  ɗuw  tɔ-lɔ    dɪrɛŋ  



chief    =3P.POSS     =Pl  also  3P.IFV-go  far  



‘Their chiefs also, they will have to travel far’



{}[\textbf{\textit{amaa-kuw=ala}}\textit{  ɛlɛ  mɪ-ʒʊm-ɛ }] ]



DEP+\textsc{PBL}{}-seek=to  thing  \textsc{NOM}{}-eat-CL



‘in order to find something to eat [in a famine].’ 


\section{Clauses linked by conjunctions and conjunctive adverbs}
\hypertarget{RefHeading1213421525720847}{}
The verbs in clauses connected by a conjunction are always finite. The conjunction defines the way that the two linked clauses are related. Conjuctions can be either subordinating or coordinating. Clauses subordinatd by a conjunction can be divided into two types, depending on whether the subordinate clause follows or precedes the main clause (discussed in Sections 84 and  85, respectively). Coordinating conjunctions link clauses that are not syntactically dependent on one another (\sectref{sec:86}). Conjunctive adverbs also function to link clauses (\sectref{sec:87}).

%%please move \begin{table} just above \begin{tabular
\begin{table}
\caption{shows the characteristics of all of the conjunctions and conjunctive adverbs in Moloko.}
\label{tab:84}
\end{table}

\begin{tabular}{lll} & \textbf{Conjunction} & \textbf{Function}\\
\lsptoprule
\textbf{Subordinate conjunction that  }

\textbf{introduces an adverbial clause that follows the matrix clause.}

The adverbial clause is a constituent of the matrix clause. & \textit{bijna} ‘because’ & Demonstrates the proof for the statement in the matrix clause. \\
\hhline{-~~} & \textit{waja} ‘because’ & Indicates the reason for something previously described. The previous clauses function to give a context for the  statement in the \textit{waja} clause\\
& \textit{kuwaja} ‘because’ / ‘that is’ & Introduces clauses (or noun phrases) that give an explanation or reasoning behind the situation expressed in previous clauses. \\
& \textit{ha} ‘until’ & Gives temporal information to that in the matrix clause, indicating that the event in the matrix clause continues up to the point in time or event expressed in the subordinate clause\\
\textbf{Conditional conjunction}

Introdices a conditional clause that precedes the matrix clause. Makes a condition for the matrix clause. & \textit{asa…na} ‘if…PSP’ & Condition is presupposed or a real possibility. \\
\hhline{-~~} & \textit{asa…ɗuw}  ‘if…PSP’ & Condition presents a new possibility.\\
& \textit{ana}\textit{ }\textit{asa} …na ‘to if...PSP’ & Condition presents a strong expectation to be fulfilled.\\
& \textit{azana}\textit{ }\textit{asa} …na ‘maybe if...PSP’ & Condition might be fulfilled.\\
\textbf{Coordinate conjunction }

 Links independent clauses. & \textit{nata} ‘and (then)’ & Marks the clause(s) which contain the most pivotal event(s) in a narrative.\\
\hhline{-~~} & \textit{aɮəna} ‘but’ & Contains an element of counterexpectation with something in the first clause.\\
\textbf{Conjunctive adverb}

Function to express other logical relations between independent clauses. & \textit{kɪlɛŋ} ‘next’ & Returns to the main event line, expressing the next main action, often after a digression from the main event line.\\
\hhline{-~~} & \textit{nd}\textit{ɛ} ‘therefore’ & Marks conclusive statements.\\
& \textit{matsəkəmbaj} ‘meanwhile’ & Relates two clauses with respect to time of the events.\\
\hhline{~--}
\lspbottomrule
\end{tabular}

\begin{itemize}
\item \begin{styleTabletitle}
Subordinating and coordinating conjunctions
\end{styleTabletitle}\end{itemize}
\subsection{  Final adverbial clauses introduced by a subordinating conjunction}
\hypertarget{RefHeading1213441525720847}{}
When an adverbial clause introduced by a subordinating conjunction follows the main clause, the clause supplies new information to the discourse. The manner in which the new information relates to the matrix clause is controlled by the subordinate conjunction. Subordinating conjunctions include \textit{bijna} ‘because,’\textit{ waja} ‘because,’ and \textit{kuwaja} ‘because,’(all involved in reason-result constructions) and \textit{ha} ‘until.’ Sentences in Moloko do not normally have multiple subordinate clauses. In the examples, each of the subordinate clauses is delimited by square brackets.  

\textit{Bijna}\footnote{\textit{Bijna} ‘because’ may be a compound of the negative \textit{baj} and the presupposition marker \textit{na}. }\textit{ } ‘because’ (ex. 931{}-934) is used in result-reason clause constructions that link only two clauses. The clause subordinated by \textit{bijna} demonstrates the proof for the statement in the matrix clause. 


Snake S. 19



\ea
Nəngehe  na,   Hərmbəlom  aloko  ehɛ,  [\textbf{b}əy\textbf{na}  anjakay  nok\textsuperscript{  }ha  a  slam  məndəye  ango  ava.]    
\z

nɪŋgɛhɛ na   Hʊrmbʊlɔm    =alɔk\textsuperscript{w}ɔ    ɛhɛ    



DEM     PSP       God    1\textsc{Pin}.POSS   here  



‘Here, God is [really] with us,’



{}[\textbf{\textit{bijna}}\textit{   à-nzak-aj         nɔk}\textit{\textsuperscript{w}}\textit{  ha       a   ɬam       mɪ-nd-ijɛ  =aŋg}\textit{\textsuperscript{w}}\textit{ɔ     ava }]



because   3S.PFV-find{}-CL  2S     until  in    place  \textsc{NOM}{}-sleep-CL     =2S.POSS  in



‘because he found you even at the place where you slept.’ 


\ea
Náavəlaləkwəye  səloy  [\textbf{b}əy\textbf{na}  kogom  va  slərele  gam.]
\z

náá-vəl=alʊk\textsuperscript{w}øjɛ   sʊlɔj     [\textbf{bijna}  kɔ-g\textsuperscript{w}{}-ɔm =va  ɬɪrɛlɛ  gam ]



1S.POT-give=2P.IO  coin    because  2-do-2P  =\textsc{PRF}  work  much



‘I will give you money because you have done a lot of work.’


\ea
Nàzala  məlama  əwla  a  lopəytal  ava  [\textbf{b}əy\textbf{na}  dəngo  awəlan.]
\z

nà-z=ala    məlama  =uwla    a  lɔpijtal  ava  [\textbf{bijna}  dʊŋg\textsuperscript{w}ɔ   a-wəl=aŋ ]



1S.PFV-take=to  sibling  =1S.POSS  in  hospital  in  because  throat  3S-hurt=3S.IO



‘I took my brother to the hospital because his throat was hurting.’ 



In Cicada S. 14


\ea
Ɗeɗen  bahay,  ngəvəray  ngəndəye  ágasaka  ka  mahay  ango  aka,  [\textbf{b}əy\textbf{na}  ngəvəray  ga  səlom  ga; aɓəsay  ava  bay.]
\z

ɗɛɗɛŋ  bahaj  ŋgəvəraj    ŋgɪndijɛ  á-gas =aka  ka  mahaj  =aŋg\textsuperscript{w}ɔ     aka,



truth     chief  spp. of tree  DEM     3S.IFV-get=on  on   door    =2S.POSS  on



‘True, chief, it is advisable that tree be would fit well by your door,’



{}[\textbf{\textit{bijna}}\textit{      ŋgəvəraj    ga  sʊlɔm  ga   aɓəsaj  ava     baj }]



because   spp. of tree  ADJ   good  ADJ   blemish   \textsc{EXT}  \textsc{NEG}



‘because this tree is good;  it has no faults.’


A clause subordinated by \textit{waja} ‘because’(ex. 935{}-936) indicates the reason for something previously described in the previous clauses. The preceding clauses function to give a context for the statement in the \textit{waja} clause. In ex. 935 (from S. 7 - 8 of the Disobedient Girl story) the \textit{waja} clause explains the reasoning behind the entire paragraph before it (the entire story is in \sectref{sec:1.5}). S. 7 gives the result (one grain of millet would give enough food for a family) and S. 8 gives the reason behind it (because God multiplied the millet while the flour was being ground).

\ea
War  elé  háy  bəlen  fan  na,  ánjata  pew  ha  ámbaɗ  ese.  [\textbf{Waya}  a  məhaya  ahan  ava  na,  ásak  kə  ver  aka  nə  məsəke.]  
\z

war     ɛlɛ   haj   bɪlɛŋ   faŋ   na



child   eye   millet   one    yet   PSP



‘Just one grain of millet,’



\textit{á-nz=ata                    pɛw       ha  á-mbaɗ    ɛʃɛ    }



3S.IFV-suffice=3P.IO       enough   until     3S.IFV-remain   again



‘it sufficed for them, and there were leftovers.’



S. 8



{}[\textbf{\textit{waja}}\textit{   a  mə-h=aja         =ahaŋ      ava    na}



because  in   \textsc{NOM}{}-grind=\textsc{PLU}   =3S.POSS   in   PSP



‘Because, during its grinding,’



\textit{á-sak                     kə  vɛr             aka   nə   mɪ-ʃɪk-ɛ }]



3S.IFV-multiply   on   grinding stone    on    with   \textsc{NOM}{}-multiply-CL



‘it would really multiply on the grinding stone.’ (lit. multiply with multiplying)  


Ex. 936 (part of a story not illustrated in this work) shows another result-reason construction with \textit{waja}.  The clause subordinated by \textit{waja} explains the reason why the speaker didn’t know how to proceed. It was important in the story that the speaker had already visited the sub-prefect.

\ea
Nàsar  həraf  ele  nəngehe  asabay  [\textbf{waya  }nəlva  afa  səwpərefe.]  
\z

nà-sar    həraf    ɛlɛ  nɪŋgɛhɛ  asa-baj  



1S.PFV-know  medicine   thing  DEM       again-\textsc{NEG}  



{}[\textbf{\textit{waja  }}\textit{nə-l             =}\textit{va}\textit{  afa    suwpɪrɛfɛ }]



because  1S.PFV-go  =\textsc{PRF}    house of    sub prefect



‘I didn’t know how to resolve the problem (lit. I never knew the medicine for this particular thing), because I had already been to the sub prefect [and he didn’t help me].’ 


The demonstrative \textit{ndana} in the phrase \textit{waja ndana} ‘refers the hearer to the previously-mentioned clauses to discover the reason behind the statement introduced by \textit{waja ndana}. In the reason-result construction shown in ex. 937 (from the Disobedient Girl story), S. 34 states that God had gotten angry (because of the girl that disobeyed). The \textit{waja ndana} clause in S. 35 identifies that the information in S.34 is the reason for the statement in S.35; it was because of God’s anger that God took back his blessing from the Moloko. 


Disobedient Girl S. 34


\ea
Hərmbəlom  ága  ɓərav  va  kəwaya  war  dalay  na,  amecen  sləmay  baj  ngəndəye.  [\textbf{Waya}  ndana  Hərmbəlom  ázata  aka  barka  ahan  va.]  
\z

Hʊrmbʊlɔm     á-g-a      ɓərav   =va  



God          3S.IFV-do   heart    =\textsc{PRF}       



‘God had gotten angry’



\textit{kuwaja        war    dalaj     na   amɛ-tʃɛŋ   ɬəmaj  baj     ŋgɪndijɛ}



because of  child    girl    PSP  DEP-hear      ear      \textsc{NEG}  DEM



‘because of that girl, that one who was disobedient.’



S. 35\textbf{\textit{  }}



{}[\textbf{\textit{waja}}\textit{   ndana    Hʊrmbʊlɔm   á-z=ata          =aka   barka     =ahaŋ     =va }]



because   DEM     God             3S.IFV-take=3P.IO =on   blessing   =3S.POSS  =\textsc{PRF}



‘Because of that previously-mentioned [event], God had taken back his blessing from them.’


The conjunction \textit{kuwaja}  ‘because’ / ‘that is’ (ex. 938{}-939. also see ex. 937) introduces clauses (or noun phrases) that give an explanation or reasoning behind the situation expressed in previous clauses. \textit{Kuwaja} introduces the conditional construction in ex. 938 (from the Disobedient Girl story S. 3-4) that gave the reasoning behind the blessing that the Molokos experienced in the past. 


Disobedient Girl S. 3


\ea
Zlezle  na,  Məloko  ahay  na,  Hərmbəlom  ávəlata  barka  va.  [\textbf{Kəwaya}  asa  təwasva  nekwen  kəygehe  ɗəw,]  ávata  məvəye  haɗa.  
\z

ɮɛɮɛ  na   Mʊlɔk\textsuperscript{w}ɔ   =ahaj       na



long ago      PSP  Moloko     =Pl           PSP



\textit{Hʊrmbʊlɔm    á-vəl=ata     barka     =va      }



God         3S.IFV-send=3S.IO   blessing  =\textsc{PRF} ˊ  



‘Long ago, to the Moloko people, God had given his blessing.’



S. 4



{}[\textbf{\textit{kuwaja}}\textit{   asa    tə-was              =va  nɛk}\textit{\textsuperscript{w}}\textit{ɛŋ    kijgɛhɛ     ɗuw, }]



that is   if     3P.PFV-cultivate =\textsc{PRF}  little       like this    also



‘That is, even if they had only sowed a little [millet] like this,’



\textit{á-v=ata    mɪ-v-ijɛ         haɗa}  



3S.IFV-pass=3P.IO   \textsc{NOM}{}-pass(year) -CL  a lot



‘it lasted them enough for the whole year.’  


In the conclusion of the same story (S. 34, ex. 939), \textit{kuwaja} introduces a noun phrase with a relative clause that gives the reasoning behind God’s anger.


Disobedient Girl S. 33


\ea
Hərmbəlom  ága  ɓərav  va  [\textbf{k}\textbf{ə}\textbf{waya}  war  dalay  amecen  sləmay  bay  ngəndəye.]     
\z

Hʊrmbʊlɔm   á-g-a         ɓərav     =va   



God    3S.IFV-do    heart   =\textsc{PRF}



{}[\textbf{\textit{kuwaja}}\textit{   war   dalaj   am}\textit{ɛ}\textit{{}-t}\textit{ʃɛ}\textit{ŋ   }\textit{ɬ}\textit{əmaj   baj   ŋg}\textit{ɪ}\textit{ndij}\textit{ɛ }]



because  child  female  DEP-hear  ear  \textsc{NEG}  that



‘God had gotten angry (lit. did heart) because of that girl, the one that didn’t listen.’


The clause introduced by \textit{ha} ‘until’ expresses a literal or metaphorical boundary that marks the termination or cessation of the activity or situation expressed by the matrix clause (ex. 931, 940,\footnote{From the Race story, Friesen, 2003.} 941).

\ea
Kərcece  ahəmay  ahəmay  ahəmay  [\textbf{ha}  ayaɗay  ndele  pəs  pəssa.]
\z

kɪrtʃɛtʃɛ a-həm-aj    a-həm-aj  a-həm-aj   [\textbf{ha}   a-jaɗ-aj     ndɛlɛ pəs pəs  =sa]



giraffe  3S-run-CL IT      until  3S-tire-CL  \textsc{ID}completely tired=ADV



‘The giraffe ran and ran and ran until he was completely tired out.’ 


In ex. 941, the second clause begins with \textit{ha} ‘until’ and gives adverbial information to the matrix clause concerning how that one grain of millet will satisfy their hunger. 


Disobedient Girl S. 7


\ea
War  elé  háy  bəlen  fan  na,  ánjata  pew  [\textbf{ha}  ambəɗ  ese.]       
\z

war    ɛlɛ   haj   bɪlɛŋ   faŋ  na      á-nz        =ata  pɛw       



child     eye  millet  one  already  PSP  3S.IFV-suffice=3P.IO  enough    



{}[\textbf{\textit{ha}}\textit{     a-mbəɗ     ɛʃɛ} ]



until       3S-remain   again



‘One grain of millet, it sufficed for them even to leaving leftovers.’


\subsection{   Conditional construction}
\hypertarget{RefHeading1213461525720847}{}
The subordinating conjuction \textit{asa} ‘if’ subordinates a clause and indicates that the clause expresses a condition required for the main clause. Clauses subordinated by \textit{asa} ‘if’ are neutral with respect to whether the speaker expects the condition to be fulfilled or not. The construction is \textit{asa} plus the conditional clause.  

The end of the subordinate clause is delimited by the presupposition marker \textit{na} or new information marker \textit{ɗuw}. Which marker is employed depends upon the context. If the condition is known or expected in the context, the presupposition marker \textit{na} delimits the condition (ex. 942 - 944).  In the examples of this section, both the subordinating conjunction and presupposition or ‘unexpected’ information marker are bolded, and the subordinate clause is delimited by square brackets.


\ea
{}[\textbf{Asa  }kége  akar  \textbf{na,}]  náaɓok.
\z

{}[\textbf{asa  }  kɛ-g-ɛ     akar   na]  náá-ɓ =ɔk\textsuperscript{w}



if    2S.IFV-do-CL  theft  PSP  1S.POT-beat=2S.IO



‘If you steal, I will beat you.’


\ea
{}[\textbf{Asa  }ások\textsuperscript{  }njəwelek  \textbf{na,}]  kándaɗay  elele  kəlen.
\z

{}[\textbf{asa  }á-s=ɔk\textsuperscript{w}       nʒuwɛlɛk  na]  ká-ndaɗ-aj  ɛlɛlɛ  kɪlɛŋ



if  3S.PFV-cut=2S.IO  leaf (spp)  PSP  2S.IFV-like{}-CL  sauce  then



‘If you like this kind of leaf, you will like this sauce.’  


\ea
 [\textbf{Asa}  taɓan  va  ana  məze  \textbf{na}\textbf{,}]  ləkwəye  na,  gom  ala  sərtəfka  medekal  aləkwəye.
\z

{}[\textbf{asa}  tà-ɓ    =aŋ  =va  ana  mɪʒɛ  na]



if  3P.PFV-hit  =3S.IO  =\textsc{PRF}  DAT  person  PSP    



\textit{lʊk}\textit{\textsuperscript{w}}\textit{øjɛ  na  g-ɔm        =ala  sərtfka     mɛdɛkal   =alʊk}\textit{\textsuperscript{w}}\textit{øjɛ}



2P    PSP   do \textsc{IMP}{}-2P   =to  certificate  medical    2P.POSS  



‘If someone has gotten beaten, make a medical certificate for him.’


If a new idea or another option is added, the subordinated clause is marked by \textit{ɗuw} , the new information marker instead of \textit{na} (ex. 945 and 946) and the meaning of \textit{asa} shifts to more of a concessive idea.


Disobedient Girl S. 4


\ea
{}[\textbf{Asa  }tawasva  nekwen  kəygehe  \textbf{ɗ}\textbf{ə}\textbf{w,}]  ávata  məvəye  haɗa.   
\z

{}[\textbf{asa }  ta-was   =va   nɛk\textsuperscript{w}ɛŋ   kijgɛhɛ  \textbf{ɗuw }]   



if    3P-cultivate  =\textsc{PRF}  little  like this  also  



\textit{á-v=ata     mɪ-v-ijɛ    haɗa}



3S.IFV-last=3P.IO  \textsc{NOM}{}-pass-CL  many



‘Even if they cultivated a little like this, it would last for them many years.’


\ea
{}[\textbf{Asa  }məze  ahay  təcahay  ele  \textbf{ɗ}\textbf{ə}\textbf{w,}]  Hərmbəlom  ecen  asabay.
\z

{}[\textbf{asa }  mɪʒɛ   =ahaj   tə-tsah-aj  ɛlɛ       \textbf{ɗuw }]



if       person    =Pl        3P-ask{}-CL         thing  also



\textit{Hʊrmbʊlɔm     ɛ{}-tʃɛŋ  asa-baj}



God                     3S-hear   again-\textsc{NEG}



‘Even if people ask for anything, God doesn’t hear anymore.’  


Normally the subordinated clause is followed by the main clause (ex. 947), however the clause expressing the condition can be right-shifted in some contexts (ex. 948). The \textit{asa} clause is always delimited by \textit{na}.


Disobedient Girl S. 13


\ea
{}[\textbf{Asa}  àsok  aməhaya  \textbf{na}\textbf{,}]  kázaɗ  war  elé  háy  bəlen.  
\z

{}[\textbf{asa}   à-s=ɔk\textsuperscript{w}      aməh=aja        na\textbf{ }]  



if       3S.IFV-please=2S.IO  DEP-grind=\textsc{PLU}     PSP  



\textit{ká-zaɗ    war     ɛlɛ  haj       bɪlɛŋ}



2S.IFV-take      child    eye  millet  one



‘If you want to grind, you take only one grain.’


\ea
Gəbar  anday  agaw  [\textbf{asa}  bahay  apaɗay  sese  \textbf{na}.]
\z

gəbar  a-ndaj    a-g=aw    [\textbf{asa}  bahaj   à-paɗ-aj  \textbf{    }ʃɛʃɛ   na ]



fear    3S-PROG  3S-do=1S.IO  if  chief  3S.PFV-crunch{}-CL  meat  PSP



‘I am afraid that the chief ate meat.’(lit. fear is doing me if the chief ate meat)  


Other particles co-occurring with the conjunction \textit{asa} ‘if’ can cause a shift in its meaning. Clauses subordinated by the dative marker plus ‘if’ \textit{ana}\textit{ }\textit{asa} have a strong  expectation that the condition will be fulfilled (ex. 949), while clauses subordinated by \textit{azana} \textit{asa} ‘maybe if’ carry the expectation that the condition might be fulfilled, rendering the subordinating clause to have almost a temporal meaning (ex. 950). 

\ea
 [\textbf{Ana  asa  }kege  akar  bay  \textbf{na,}]  náɓok\textsuperscript{  }bay.  
\z

{}[\textbf{ana     asa  }kɛ-g-ɛ    akar  baj  na\textbf{ }]  ná-ɓ=ɔk\textsuperscript{w}    baj  



   DAT  if  2S.PFV-do-CL  theft  \textsc{NEG}  PSP  1S.IFV-beat=2S.IO  \textsc{NEG}



‘If you don’t steal (and I don’t expect you to steal), I won’t beat you.’


\ea
{}[\textbf{Azana  asa}  tanday  təzlaɓay  ele  memey  \textbf{na}\textbf{,}]  təzləgalay  avəlo  bay.
\z

{}[\textbf{azana    asa}     ta-ndaj    tə-ɮaɓ-aj    ɛlɛ    mɛmɛj   na]



maybe    if         3P-PROG  3P-pound{}-CL   thing   how      PSP



\textit{tə-}\textit{ɮ}\textit{əg=alaj                   avʊlɔ     baj}



3P.IFV-throw =away   high up  \textsc{NEG}



‘When something is being pounded, the baton is not thrown too high.’ (lit. if perhaps they are pounding something, they don’t throw the baton too high)


\subsection{   Coordinate constructions}
\hypertarget{RefHeading1213481525720847}{}
Coordinate constructions consists of two independent clauses linked by a coordinate conjunction. The coordinating conjunction specifies the way that the clauses are connected. They include \textit{nata} ‘and then’ and \textit{aɮəna} ‘but.’ In the examples below, the conjunction is bolded and the coordinate clause is delimited by square brackets. 

\textit{Nata} ‘and then’ marks the clauses which contain the most pivotal events in a narrative. Ex. 951 shows the two clauses from the Cicada narrative that are marked with \textit{nata}. These two clauses mark the peak event of the cicada’s success at transporting the tree for the chief. Ex. 952 shows the clause in the peak of the Disobedient Girl narrative that is marked with \textit{nata}. This marked peak event is the death of the girl, the result of her disobedience. 


Cicada S. 25



\ea
{}[\textbf{Nata}  olo,]  albaya  ahay  tolo  sen  na,  albaya  ahay  weley  təh  anan  dəray  na,  abay.  [\textbf{Nata}  mətəde  təh  anan  dəray  ana  ngəvəray  ngəndəye.]
\z

{}[\textbf{nata    }ɔ{}-lɔ ]



and then     3S- go  



‘And then, he went.’



S. 26



\textit{albaja    =ahaj    tɔ-lɔ    ʃɛ}\textit{ŋ}\textit{      na    }



youth     =Pl      3P-go   \textsc{ID}go   PSP  



\textit{albaja   =ahaj   wɛlɛj  təx     an=aŋ         dəraj   na  abaj.}



youth    =Pl    which   \textsc{ID}put   DAT=3S.IO   head   PSP   \textsc{EXT} \textsc{NEG}



‘No one could lift it.’ (lit. the young men went, whichever young man put his head [to the tree] (in order to lift it), there was none)



S. 27



{}[\textbf{\textit{nata}}\textit{  mɪtɪdɛ  təx    an=aŋ    dəraj  ana  ŋgəvəraj    ŋgɪndijɛ }]



and then  cicada  \textsc{ID}put on head  DAT=3S.IO  head  DAT  spp. of tree  DEM



‘And then the cicada put his head to that tree.’ 



Disobedient Girl S. 26


\ea
Alala na,  ver na árah mbəf nə həmbo na, ɗək məɗəkaka alay ana hor na, [\textbf{nata} ndahan mat məmətava alay a hoɗ a hay na ava.]
\z

a-l=ala   na   vɛr   na   á-rax     mbəf     nə   hʊmbɔ   na~   



3S-come=to  PSP  kitchen  PSP  3S.IFV-fill  to the roof  with  flour  PSP  



‘Later, the kitchen, it filled to the roof with flour,’



\textit{ɗək     mə-ɗək=aka=alaj   ana   h}\textit{\textsuperscript{w}}\textit{ɔr   na}



suffocate  \textsc{NOM}{}-suffocate=on=away  DAT  woman  PSP



‘it suffocated the woman,’



{}[\textbf{\textit{nata}}\textit{   ndahaŋ   mat   mə-mət=ava =alaj   a   h}\textit{\textsuperscript{w}}\textit{ɔ}\textit{ɗ}\textit{   a   haj   na   ava}]



then    3S  die  \textsc{NOM}{}-die=in=away  in  stomach  GEN  house  PSP  in



‘and then she died in the stomach of that house.’


\textit{Aɮəna }\footnote{\textit{aɮə}\textit{na} ‘but’ may be a compound of \textit{aɮa}\textit{ }‘now’ and the presupposition marker \textit{na}. } ‘but’{ }indicates that the clause that follows will contain an element of counterexpectation with something in the previous clause (ex. 953{}-955).


Disobedient Girl S. 10 - 11


\ea
Olo  azala  hor.  [\textbf{Azləna}  war  dalay  ndana  cekəzlere  ga.]
\z

ɔ{}-lɔ   a-z=ala   h\textsuperscript{w}ɔr~  [\textbf{aɮəna}  war   dalaj   ndana   tʃɛkɪɮɛrɛ   ga ]



3S-go  3S-take=to  woman  but  child  female  DEM  disobedience  ADJ



‘He went and took a wife, but the above-mentioned girl [was] disobedient.’


\ea
Avəyon  agan  va  gəɓar  ana  Abanga.  Ahəman  alay  nekwen.  [\textbf{A}\textbf{zl}\textbf{əna}  na  me,  ləme  nata  babəza  ahay  na,  ko  məbele  nekwen  ɗəw,  nobəlom  bay.]
\z

avijɔŋ  a-g=aŋ       =va     gəɓar   ana   Abaŋgaj    a-həm=aŋ  =alaj   nɛk\textsuperscript{w}ɛŋ



airplane  3S-do-3S.IO=\textsc{PRF}  fear  DAT  Abangay  3S-run-3S   =to    little



{}[\textbf{\textit{a}}\textbf{\textit{ɮ}}\textbf{\textit{əna}}\textit{   na   mɛ   lɪmɛ   nata   babəza =ahaj   na}



but    PSP  opinion  1\textsc{Pex}  and  children  =Pl  PSP  



\textit{k}\textit{\textsuperscript{w}}\textit{ɔ     mɪ-bɛl-ɛ   nɛk}\textit{\textsuperscript{w}}\textit{ɛŋ   }\textit{ɗ}\textit{uw   nɔ-bʊl-ɔm   baj }]



even  \textsc{NOM}{}-move-CL  little  also  1-move-1\textsc{Pex}  \textsc{NEG}



‘The airplane made Abangay afraid (lit. did fear to her), [so] she ran away a little; but on the other hand, I and the children, we didn’t budge even a little (lit. even a little movement we didn’t move).’ 


\ea
Nahan  ana  hor  əwla  ne  awəy  majaw  ala  yam  aməbele;  [\textbf{a}\textbf{zl}\textbf{əna}  acahay  bay.]    
\z

na-h=aŋ    ana  h\textsuperscript{w}ɔr       =uwla    nɛ awij     



1S-say-3S D  DAT  wife  =1S.POSS  1S saying  



\textit{ma-dz  =aw  =ala  jam  amɪ-bɛl-ɛ}



3S.HOR-help=1S.IO    =to  water  DEP-wash-CL



{}[\textbf{\textit{a}}\textbf{\textit{ɮ}}\textbf{\textit{əna}}\textit{  a-tsah-aj    baj }]



but    3S-obey-CL  \textsc{NEG}



‘I said to my wife that she should bring me water to wash, but she didn’t obey me.’ 


\subsection{    Conjunctive adverbs}
\hypertarget{RefHeading1213501525720847}{}
Conjunctive adverbs are adverbs that function to connect clauses within a larger  context. They include \textit{kɪlɛŋ} ‘next,’ \textit{nd}\textit{ɛ} ‘therefore,’ and \textit{matsəkəmbaj }’meanwhile.’ With the exception of \textit{kɪlɛŋ}, conjunctive adverbs are clause-initial. The examples give some of the surrounding context so that their function can be demonstrated. Many of the examples are from the Disobedient Girl story or the Cicada story. In order to study the larger context for the examples, the stories themselves can be found in Sections  1.5 and 1.6, respectively.

\textit{Kɪlɛŋ} ‘next’ indicates a subsequent mainline event that often follows a digression (often reported speech). This conjunction can either be clause-initial (ex. 957) or it may follow the first argument in the clause (ex. 956). Ex. 956 and 957 are from the Cicada story (\sectref{sec:1.6}). In ex. 956, the \textit{kɪlɛŋ} clause marks the first event in the eventline that concerns the tree. 


Cicada S. 5-6



\ea
Tánday  tətalay  a  ləhe  na,  təlo  tənjakay  ngəvəray  malan  ga  a  ləhe.  [Albaya  ahay  ndana\textbf{  k}\textbf{ə}\textbf{len  }təngalala  ma  ana  bahay.]
\z

tá-ndaj         tə-tal-aj    a    lɪhɛ     na  



3P.IFV-PROG   3P-walk-CL     at   bush   PSP  



‘[As]they were walking in the bush,’



\textit{tə-lɔ             tə-nzak-aj        ŋgəvəraj     malaŋ   ga      a    lɪhɛ}



3P.PFV-go   3P.PFV-find-CL   spp. of tree    large   ADJ      to   bush



‘they found a large tree (a particular species) in the bush.’



{}[\textit{albaja }  \textit{=ahaj  }  \textit{ndana}\textbf{\textit{  k}}\textbf{\textit{ɪ}}\textbf{\textit{lɛŋ  }}\textit{tə-~ŋgala      =ala   }  \textit{ma    }  \textit{ana    }  \textit{bahaj }]



youth    =Pl       DEM  then  3P.PFV-return=to    word    DAT    chief



‘The above-mentioned young men then took the word (response) to the chief.’  


Clauses S. 7 and 8 are shown in ex. 957, where \textit{kɪlɛŋ} coordinates the clause in S. 8 to the eventline after the speech in S. 7.


Cicada S. 7-8\textbf{  }


\ea
Tawəy,  “Bahay,  mama  ngəvəray  ava  a  ləhe  na,  malan  ga  na,  agasaka  na  ka  mahay  ango  aka  aməmbese.”  [\textbf{K}\textbf{ə}\textbf{len  }albaya  ahay  ndana  tolo.]
\z

tawij~ 



3P+say



‘They said,’   



\textit{bahaj,  mama   ŋgəvəraj       ava  a   lɪhɛ   na   malaŋ ga               na}



chief,   mother   spp. of tree  \textsc{EXT}   at   bush    PSP  large   ADJ              PSP



\textit{à-gas  =aka      na     ka  mahaj    =aŋg}\textit{\textsuperscript{w}}\textit{ɔ         aka   amɪ-mbɛʃ-ɛ.}



3S.PFV-get=on  PSP  on     door  =2S.POSS  on  DEP-rest{}-CL



‘“Chief, there is a mother-tree in the bush,  a big one, [and] it would please you to have that tree at your door, so that you could rest under it.”’



{}[\textbf{\textit{kɪlɛŋ}}\textit{  albaja  =ahaj  ndana  tɔ-lɔ }]



next    youth  =Pl  DEM  3P.PFV-go



‘Then those above-mentioned young men went.’ 


A conclusion in a discourse or a concluding remark may be introduced by the conjunctive adverb \textit{nd}\textit{ɛ} ‘so.’ Ex. 958 shows S. 32 - 34 from the conclusion of the Disobedient Girl narrative.  \textit{Nd}\textit{ɛ} introduces the concluding comments concerning the way that the present-day situation for the Molokos has changed from the way it was before the actions of the disobedient girl. Ex. 959 is from the Leopard story (Friesen, 2003) and \textit{nd}\textit{ɛ} marks the clause within the hen’s speech where she makes her concluding decision of what she should do. Ex. 960 shows \textit{nd}\textit{ɛ} marking a concluding statement in an instruction. 


Disobedient Girl S. 32


\ea
{}[\textbf{Nde  }ko  ala  a  ɗəma  ndana    ava  pew.]  Məloko  ahay  tawəy  Hərmbəlom  ága  ɓərav  va  kəwaya  war  dalay  na,  amecen  sləmay  bay  ngəndəye.  Waya  ndana  Hərmbəlom  ázata  aka  barka  ahan  va. 
\z

{}[\textbf{ndɛ  }  k\textsuperscript{w}ɔ   =ala    a    ɗəma     ndana    ava      pɛw ]



so    until    =to    in    time      DEM       in    enough



‘So, ever since the above-mentioned time, it’s done!’



S. 33



\textit{Mʊlɔk}\textit{\textsuperscript{w}}\textit{ɔ  =ahaj  tawij}



Moloko     =Pl       3P+say



‘the Molokos say’ 



\textit{Hʊrmbʊlɔm     á-g-a      ɓərav   =va    }



God          3S.IFV-do   heart    =\textsc{PRF}       



\textit{kuwaja        war    dalaj     na   amɛ-tʃɛŋ   ɬəmaj  baj     ŋgɪndijɛ}



because of  child    girl    PSP  DEP-hear      ear      \textsc{NEG}  DEM



‘“God had gotten angry because of that girl, that one that was disobedient.’



S. 34



\textit{waja   ndana  Hʊrmbʊlɔm   á-z=ata          =aka   barka     =ahaŋ     =va}



because   DEM   God             3S.IFV-take=3P.IO =on   blessing  =3S.POSS  =\textsc{PRF}



‘Because of that mentioned above, God had taken back his blessing from them.”’


\ea
Tanday  taslaw  aka  babəza  ahay  va.  [\textbf{Nde  }taslaw  aka  babəza  ahay  va  na,  nəhəmay  mogo  ele  əwla.]~  
\z

ta-ndaj   ta-ɬ=aw =aka  babəza   =ahaj   =va   



3P-PROG  3P-kill=1S.IO=on  children  =Pl  =\textsc{PRF}  



{}[\textbf{\textit{ndɛ}}\textit{   ta-ɬ=aw   =aka   babəza   =ahaj  =va       na}



so    3P-kill=1S.IO  =on  children  =Pl          =\textsc{PRF}  PSP



\textit{nə-həm-aj   mɔg}\textit{\textsuperscript{w}}\textit{ɔ   ɛlɛ   =uwla. }]\textit{~}



1S-run{}-CL  anger  thing  =1S.POSS



‘They were killing more of my children. So [since] they killed more of my children, I ran away because of my anger (lit. I ran my anger thing).’ 


\ea
Nahok  na  va,  kége  akar  bay.  [Asa\textbf{  }bay  na,]  náaɓok.  [\textbf{Nde  }azləna  kagəva  akar  náaɓok ]  azla.
\z

nà-h=ɔk\textsuperscript{w}    na        =va  kɛ-g-ɛ    akar  baj



1S.PFV-tell=2S.IO    3S.DO   =PER  2S.IFV-do-CL  theft  \textsc{NEG}  



{}[\textit{asa}\textbf{\textit{  }}\textit{  baj  na}]\textit{  náá-}\textit{ɓ}\textit{ =ɔk}\textit{\textsuperscript{w}}



  if    \textsc{NEG}  PSP  1S+ \textsc{POT}{}-beat=2S.IO



{}[\textbf{\textit{ndɛ}}\textit{   a}\textit{ɮ}\textit{əna   kà-gə       =va    akar  náá-}\textit{ɓ=}\textit{ɔk}\textit{\textsuperscript{w}}\textit{ }]\textit{    a}\textit{ɮ}\textit{a}



 so    but  2S+P\={ }FV-do   =PER  theft  1S.POT-beat=2S.IO  now.



‘I already told you, don’t steal, otherwise (lit. if not) I will beat you.  But you have gone and stolen, so I will beat you now.’  


\textit{Matskəmbaj } ‘meanwhile’ indicates that the information in the clause so marked occurred off the main eventline. Ex. 961 is from the Race story (Friesen, 2003). The clause with \textit{matskəmbaj} marks what the toad had done before the race – he had secretly invited his brothers to line the race route so that there would always be a toad ahead of the giraffe. The giraffe ran faster than the toad, but when he stopped running and called out to see how far behind him the toad was, one of the toad’s friends ahead of him would call to him, making him run so hard that he collapsed, thereby losing the race. 

\ea
Paraw  tədəya  məhəme,  ɓərketem, ɓərketem, ɓərketem. Kərcece  enjé  təf  na,  awəy,  “Moktonok  nok  amta?”  Moktonok\textsuperscript{  }awəy, “Ne  ko  ehe.”  Awəy,  “Wa  alma?”  [\textbf{Macəkəmbay}  moktonok  na,  abək  ta  aya  va  məlama  ahan  ahay  jəyga.]  
\z

paraw     tə-d=ija     mɪ-hɪm-ɛ   ɓɪrkɛtɛm, ɓɪrkɛtɛm, ɓɪrkɛtɛm.



\textsc{ID}sudden start   3P-take many =\textsc{PLU}  \textsc{NOM}{}-run{}-CL  \textsc{ID}run     \textsc{ID}run       \textsc{ID}run  



‘\textit{Paraw}, they started the race, running \textit{birketem}, \textit{birketem}, \textit{birketem}.’



\textit{k}\textit{ɪ}\textit{rt}\textit{ʃɛ}\textit{t}\textit{ʃɛ}\textit{   }\textit{ɛ{}-}\textit{n}\textit{ʒ}\textit{{}-}\textit{ɛ}\textit{     təf   na   awij~   m}\textit{ɔ}\textit{k}\textit{\textsuperscript{w}}\textit{t}\textit{ɔ}\textit{n}\textit{ɔ}\textit{k}\textit{\textsuperscript{w}}\textit{   n}\textit{ɔ}\textit{k}\textit{\textsuperscript{w}}\textit{   amta }



giraffe  3S-leave{}-CL  \textsc{ID}far  PSP  3S-say  toad    2S  where



‘The giraffe went far away [along the race route]. He said, “Toad, where are you?”’



\textit{m}\textit{ɔ}\textit{k}\textit{\textsuperscript{w}}\textit{t}\textit{ɔ}\textit{n}\textit{ɔ}\textit{k}\textit{\textsuperscript{w}}\textit{   awəy  n}\textit{ɛ}\textit{   k}\textit{\textsuperscript{w}}\textit{ɔ}\textit{     }\textit{ɛ}\textit{h}\textit{ɛ}\textit{ }



toad    3S-say  1S  no matter  here



‘A toad said, “I am way over here.’ 



\textit{awij   wa   alma    }



3S-say  what  what



‘[The giraffe] said, “What on earth?!” (lit. he said, ‘What what)



{}[\textbf{\textit{matsəkəmbaj}}\textit{   mɔk}\textit{\textsuperscript{w}}\textit{tɔnɔk}\textit{\textsuperscript{w}}\textit{  na  }



meanwhile    toad    PSP  



‘Meanwhile, the toad,’ 



\textit{a-bək   ta   =ja   =va  məlama  =ahaŋ    =ahaj  dzijga }]



3S-invite     3P\-.DO   =\textsc{PLU}  =\textsc{PRF}  brother  3S\-\-\_P\={ }OSS  =Pl  all



‘he had already invited all his brothers [to line up along the race so that there would always be a toad ahead of the giraffe].’


\section{   Juxtaposed clauses}
\hypertarget{RefHeading1213521525720847}{}
Many clauses in a Moloko discourse are independent and are not linked grammatically to a preceding or following clause by a connector or by the presupposition marker \textit{na}. The semantic nature of the connection between these unmarked, juxtaposed clauses is inferred from the context.\footnote{The presupposition marker \textit{na} aids in making a connection between two clauses, because it makes the precision that the first (\textit{na}{}-marked) clause is known information. \textit{Na} constructions have already been discussed in Section Error: Reference source not found.}  A juxtaposed clause can simply re-express the thought in the first clause. In ex. 962, the second clause restates in the negative that God is near. In ex. 963, the second clause makes more precise the general instruction in the first clause. In ex. 964, the second line expands on what the speaker sees about the chief. In the examples in this section, each clause is delimited by square brackets and the juxtaposed clause is bolded.


\ea
{}[Ndahan  bəfa,]  [\textbf{anday  d}ə\textbf{ren  bay.}]    
\z

{}[ndahaŋ bəfa ]    



he                   \textsc{ID}close  



{}[a-ndaj  dɪrɛŋ  baj ]



3S-PROG  far          \textsc{NEG}



‘So, he was close, he was not far.’


\ea
{}[Makay  war; ]  [mapaɗay  sese  ahan.] 
\z

{}[mak-aj     war ]  [ma-paɗ-aj    ʃɛʃɛ  =ahaŋ ] 



leave[2S.IMP]{}-CL  child  3S.\textsc{HOR}{}-crunch{}-CL  meat  3S\textsc{POSS}



‘Leave the child alone; let him eat his meat.’ 


\ea
{}[Nəmənjar  bahay,]  [ndahan  aka  ozom  sese.]
\z

{}[nə-mənzar   bahaj ]     [ndahaŋ   aka   á-zɔm     ʃɛʃɛ ]



1S-see  chief      3S  on  3S.IFV-eat  meat



‘I see the chief, he is eating meat.’ 


Ex. 965 is from S. 8-10 in the peak episode of the Snake story. There is a series of three juxtaposed independent clauses. The second is a restatement of the first. The third follows chronologically. 


Snake S. 8


\ea
{}[Mbaɗala  ehe  na,  nabay  oko,]  [nazala  təystəlam  əwla,]  [nabay  cəzlarr.] 
\z

{}[mbaɗala ɛhɛ na     nà-b-aj            ɔk\textsuperscript{w}ɔ ]       



 then     here PSP  1S.PFV-light{}-CL     fire 



‘Then, I turned on a light,’



S. 9



{}[nà-zaɗ        =ala      tijstəlam       uwla ]



1S.PFV-take=to     torch          =1S.POSS 



‘I took my flashlight,’



S. 10



{}[\textit{nà-b-aj    tsəɮarr} ]                  



1S.PFV-light{}-CL       \textsc{ID}shining the flashlight up



‘I shone it up \textit{tsilar}.’


Two juxtaposed clauses can express a logical or chronological sequence. Ex. 966 illustrates a temporal (or logical) sequence from the Cicada fable. The two clauses are the chief’s command to bring the tree to his door. First (clause 1), the people are to bring the tree and next (clause 2), they are to place it by his door. 


Cicada S. 9


\ea
{}[Kázəɗom  anaw  ala  ngəvəray  ndana  ka  mahay  əwla  aka.]  [\textbf{Káfə}\textbf{ɗo}\textbf{m  anaw  ka  mahay  }ə\textbf{wla  aka}.]
\z

{}[ká-z\textbf{ʊ}ɗ{}-ɔm    an=aw =ala     ŋgəvəraj    ndana  ka   mahaj    =uwla      aka]



2P.POT-take-2P   DAT=1S.IO =to   spp. of tree    DEM      on   door        =1S.POSS   on



‘You will bring that previously mentioned tree to my door for me.’



{}[\textbf{\textit{ká-fʊ}}\textbf{\textit{ɗ{}-ɔm}}\textbf{\textit{             an=aw      ka    mahaj   =uwla       aka}}]



2P.POT-put-2P   D =1S.IO    on      door     =1S.POSS  on



‘You will put it down by my door.’ 


Ex. 967 is a longer sequence from the peak of the Snake story (S. 13 – 18). S. 13 links to the preceding discourse with a \textit{na}{}-marked clause, but there are no conjunctions or discourse particles to indicate how the clauses are linked. He takes his spear (S. 13) hears the penetration (S. 14 - 15),\footnote{S. 16 is a narrator’s comment.} and the snake falls (S. 17). S. 18 is his concluding statement (or realisation) that he had actually killed the snake. 


S. 13


\ea
{}[Ne  mbət  məmbete  oko  əwla  na.]  [Kaləw  nazala  ezlere  əwla.]  \textbf{[Mək  ava  alay.}\textbf{]  }\textbf{[Mecesle  mbəraɓ. ]  [Ele  a  H}\textbf{ə}\textbf{rmb}\textbf{ə}\textbf{lom,  ele  ga  ajənaw  ete  kəl  kəl  kə  ndahan  aka.]  [Ádəɗala  vbaɓ  a  w}\textbf{əyen  }\textbf{ava.]  [Ne  d}\textbf{əy  }\textbf{day  m}\textbf{ə}\textbf{k}\textbf{ə}\textbf{ɗe  na  aka. ]}
\z

{}[nɛ  mbət         mɪ-mbɛt-ɛ           ɔk\textsuperscript{w}ɔ     =uwla      na ]



1S  \textsc{ID}turn off  \textsc{NOM}{}-turn\_off{}-CL     light    =1S.POSS  PSP 



‘I turned off my light.’



{}[\textit{kàluw              nà-zaɗ}          =\textit{ala    ɛɮɛrɛ    =uwla} ]



\textsc{ID}take quickly    1S.PFV-take  =to   spear       =1S.POSS 



‘Quickly I took my spear,’



S. 14      \textbf{    }



\textbf{[}\textbf{\textit{mək   =ava   =alaj}}\textbf{ }]\textbf{\textit{   }}



\textsc{ID}penetrate  =in    =to 



‘Penetration \textit{muk}!’



S. 15



\textbf{[}\textbf{\textit{mɛ-tʃɛɬ-ɛ                 mbəraɓ }}\textbf{]}\textbf{\textit{ }}



\textsc{NOM}{}-penetrate-CL    \textsc{ID}penetrate 



‘It penetrated, \textit{mburab}!’



S. 16



\textbf{[}\textbf{\textit{ɛlɛ    a    Hʊrmbʊlɔm   ɛlɛ  ga  à-dzən=aw     ɛtɛ}}\textit{ }



thing   GEN   God      thing  ADJ   3S.PFV-help=1S.IO    also 



‘God helped me also’



\textbf{\textit{kəl kəl     kə   ndahaŋ   aka }}\textbf{]}



exactly    on    3S          on 



‘[that the spear] went exactly on him.’



S. 17          



\textbf{[}\textbf{\textit{á-dəɗ         =ala     }}\textbf{\textit{ⱱ}}\textbf{\textit{aɓ              a     wijɛŋ      ava }}\textbf{]}



3S.IFV-fall  =to  \textsc{ID}falling on ground  at  ground  on



‘and he fell on the ground \textit{ⱱ}\textit{aɓ}.’



S. 18          



\textbf{[}\textbf{\textit{nɛ   dij daj                mɪ-kɪɗ-ɛ      na      =aka }}\textbf{]}



1S  approximately     \textsc{NOM}{}-kill-CL   3S.DO   =on



‘I clubbed it to death (approximately).’


Two clauses linked by juxtaposition can also express a comparison (ex. 968 and 969). The first clause is a predicate-adjective clause (see \sectref{sec:70}) that expresses the attribute being compared. The second clause establishes the comparison by means of the verb \textit{dal} ‘pass.’

\ea
{}[Kəra  malan  ga,]  [\textbf{adal  pataw}\textbf{.}]
\z

{}[kəra  malaŋ    ga ]    [\textbf{a-dal    pataw }]



dog    largeness  ADJ      3S-pass  cat



‘The dog is bigger than the cat.’ (lit. the dog is big, it is greater than the cat)


\ea
{}[Ne  mədehwer  ga,]  [\textbf{nadal  nok}\textbf{.}]
\z

{}[nɛ  mødœh\textsuperscript{w}œr  ga ],    [\textbf{na-dal  nɔk}\textsuperscript{w}]



1S  old person  ADJ    1S-pass  2S



‘I am older than you.’ (lit. I am old, I am greater than you)


\chapter[Appendix]{Appendix}
\hypertarget{RefHeading1213541525720847}{}\section{List of verbs}
\hypertarget{RefHeading1213561525720847}{}
This list has been adapted from Friesen and \citet{Mamalis2008} and \citet{StarrEtAl2000}. Verbs are listed in their 2S imperative form (citation form).  The table shows syllable structure, prosody, and underlying tone for each verb from Bow’s research (1997c).

\tablehead{
\textbf{2S Imperative} & \textbf{Underlying form} & \textbf{Underlying tone} & \textbf{Tone on Imperative} & \textbf{Gloss}\\
}
\begin{tabular}{lllll}
\lsptoprule
\textit{baj} & /C -j/ & L & L & ‘light’\\
\textit{baɗaj} & /a-CC -j/ & L & LM & ‘marry’\\
\textit{balaj} & /CaC -j/ & H & HH & ‘wash in general’\\
\textit{baɮaj} & /a-CC -j/ & L & LM & ‘weed, breathe’\\
\textit{bataj} & /a-CaC -j/ & L & LM & ‘evaporate’\\
\textit{baz} & /a-CC/ & L? & L & ‘harvest’\\
\textit{bədzakaj} & /CCaC -j/ & L & LLM & ‘dig shallow’\\
\textit{bədzəgamaj} & /CCCaC -j/ & L & LLLM & ‘crawl’\\
\textit{bərkadaj} & /CCCaC -j/ & L & LLM & ‘collect, squeeze’\\
\textit{bərwaɗaj} & /CCCaC -j/ & L & LLM & ‘drive’\\
\textit{bɔkaj} & /a-CC -j\textsuperscript{ o} / & L & LM & ‘cultivate second time, be bald’\\
\textit{bɔlaj} & /a-CaC  {}-j\textsuperscript{ o} / & L? & LM & ‘knead, soak’\\
\textit{ɓaj} & /C -j/ & H & H & ‘hit’ \\
\textit{ɓax} & /CaC/ & L & M & ‘sew’\\
\textit{ɓal} & /CC/ & H & H & ‘stir’\\
\textit{ɓar} & /CC/ & H & H & ‘shoot (arrow)’\\
\textit{ɓaraj} & /CaC -j/ & H & HH & ‘restless when sick’\\
\textit{ɓasaj} & /CC -j/ & toneless & LM & ‘put up with’\\
\textit{ɓɛlɛŋ} & /CaCC\textsuperscript{e}/ & L ? & MH & ‘build up to’\\
\textit{ɓɛɮɛŋ} & / CaCC \textsuperscript{e}/ & L ? & LL & ‘count’\\
\textit{ɓərɮaj} & /CCC -j/ & toneless? & LM & ‘throw a fit ‘\\
\textit{ɓəɬaj} & /CC -j/ & toneless & LM & ‘cough’\\
\textit{ɓɔrɔj} & /a-CaC -j\textsuperscript{ o} / & L & LM & ‘go up, climb’\\
\textit{ɓɔrtsɔj} & /CCC  {}-j\textsuperscript{ o} / & L & MH & ‘first pounding, tear to pieces’\\
\textit{dabaj} & /CC -j/ & toneless & LM & ‘follow, look for, search for’\\
\textit{daɗ} & /CC/ & toneless & L & ‘fall’\\
\textit{dafaj} & /a-CaC -j/ & L & LM & ‘bump’\\
\textit{dal} & /a-CC/ & L & L & ‘go beyond, go past, overtake, pass’\\
\textit{damaj} & /CC -j/ &  & LL? & ‘succeed (at work)’\\
\textit{dar} & /CC/ & toneless & L & ‘recoil ; back away, push away, recoil, approach, move (house)’\\
\textit{dar} & /CC/ & H & H & ‘burn, grill, to get on someone's nerves’\\
\textit{daraj} & /a-CaC -j/ & L & LM & ‘plant, snore, bow low, pray’\\
\textit{daɬaj} & /a-CaC -j/ & L & LM & ‘castrate, sterilize’\\
\textit{daɮaj} & /CaC -j/ & ? & LM & ‘join, tie, cross, intersection’\\
\textit{dav} & /CC/ &  & L & ‘drop, throw, lay eggs’\\
\textit{dɛ} & /C \textsuperscript{e}/ & L & L & ‘cook’\\
\textit{dəbakaj} & /CCaC -j/ & L & LLM & ‘relieve’\\
\textit{dəbanaj} & /CCC -j/ & L & LLM & ‘teach, learn’\\
\textit{dəŋgaɗaj} & /CCC -j/ & L & LLM & ‘lean back’\\
\textit{dija} & /C / =ija &  & HM & ‘take many’\\
\textit{dɔk}\textsuperscript{w}\textit{ɔj} & /a-CaC -j\textsuperscript{ o} / & L & LM & ‘arrive’\\
\textit{ɗak} & /CC/ & H & H? & ‘block up’\\
\textit{ɗakaj} & /CaC -j/ & L & MH & ‘show, tell’\\
\textit{ɗas} & /CC/ & L & M & ‘weigh, respect’\\
\textit{ɗaɬaj} & /a-CaC -j/ & L & MH & ‘reproduce, multiply’\\
\textit{ɗaɮ} & /CC/ & toneless & L & ‘spread for building’\\
\textit{ɗɛ} & /C \textsuperscript{e}/ & L & M & ‘to soak in order to soften’\\
\textit{ɗəgalaj} & /CCaC -j/ & L & LLM & ‘think’\\
\textit{ɗʊg}\textit{\textsuperscript{w}}\textit{ɔtsɔj} & /CCaC -j\textsuperscript{ o} / & L & LLM & ‘stalk’\\
\textit{ɗɔtsɔj} & /a-CaC -j\textsuperscript{ o} / & L & LM & ‘squeeze, juice’\\
\textit{dzaj} & /C -j/ & toneless? & L? & ‘speak’\\
\textit{dzakaj} & /CaC -j/ & toneless & LM & ‘lean’\\
\textit{dzapaj} & /CaC -j/ & toneless & LM & ‘mix, stir’\\
\textit{dzav} & /CC/ & toneless & L & ‘plant’\\
\textit{dzʊɗɔk}\textit{\textsuperscript{w}}\textit{ɔj} & /CCaC -j\textsuperscript{ o} / & L & LLM & ‘mash’\\
\textit{dzʊg}\textit{\textsuperscript{w}}\textit{ɔr} & /CCC \textsuperscript{o}/ & L? & LL & ‘watch, care’\\
\textit{dzənaj} & /CC -j/ & L? & LL & ‘help’\\
\textit{dzɔh}\textit{\textsuperscript{w}}\textit{ɔj} & /a-CaC -j\textsuperscript{ o} / & L & LM & ‘save, economise’\\
\textit{dzɔk}\textit{\textsuperscript{w}}\textit{ɔj} & /a-CC -j\textsuperscript{ o} / & L & LM & ‘pack down’\\
\textit{dzɔrɓɔj} & /CCC -j\textsuperscript{ o} / & toneless? & LM & ‘wash clothes’\\
\textit{faɗ} & /CC/ & L & M & ‘put, down’\\
\textit{faɗaj} & /CaC -j/ & H & HH & ‘fold, create’\\
\textit{fakaj} & /CaC -j/ & L & MH & ‘uproot, knock down tree’\\
\textit{far} & /CC/ & H & H & ‘scratch’\\
\textit{fat} & /CC/ & L & M & ‘grow, sprout’\\
\textit{fataj} & /CaC -j/ & L & MH & ‘lower, go down, land’\\
\textit{fɛ} & /C -j \textsuperscript{e}/ & L & M & ‘blow in an instrument, play an instrument’\\
\textit{fətaɗaj} & /CCaC -j/ & L & MMH & ‘sharpen to a point’\\
\textit{fɔk}\textit{\textsuperscript{w}}\textit{ɔj} & /a-CaC -j\textsuperscript{ o} / & L & MH & ‘whistle with your lips’\\
\textit{gabaj} & /a-CC -j/ & L & LM & ‘constipate’\\
\textit{gar} & /CC/ & H & H & ‘grow up’\\
\textit{garaj} & /CaC -j/ & toneless & LM & ‘own, measure, order’\\
\textit{garaj} & /a-CC -j/ & L & LM & ‘frighten, tremble’\\
\textit{gas} & /CC/ & toneless & L & ‘take hold of, catch, accept, stop, accept, obey’\\
\textit{gazaj} & /a-CaC -j/ & L & LM & ‘nod’\\
\textit{gɛ} & /C -j \textsuperscript{e}/ & H & H & ‘make, do’\\
\textit{gədəgalaj} & /CCCaC -j/ & L & LLLM & ‘get fat’\\
\textit{gədəgaraj} & /CCCaC -j/ & L & LLLM & ‘granulate, weave’\\
\textit{gədzax} & /CCaC/ & L? & LL & ‘pull’\\
\textit{gədzakaj} & /CCaC -j/ & L & LLM & ‘hang’\\
\textit{gədzar} & /CCaC/ & L? & MM & ‘take or steal by force’\\
\textit{gəzamaj} & /CCaC -j/ & L & LLM & ‘lose weight’\\
\textit{g}\textit{\textsuperscript{w}}\textit{ʊɓɔk}\textit{\textsuperscript{w}}\textit{ɔj} & /CCaC -j\textsuperscript{ o} / & L & LLM & ‘bend over’\\
\textit{g}\textit{\textsuperscript{w}}\textit{ɔtsɔj} & /CC -j\textsuperscript{ o} / & toneless & LM & ‘throw, sow’\\
\textit{g}\textit{\textsuperscript{w}}\textit{ɔh}\textit{\textsuperscript{w}}\textit{ɔj} & /a-CaC -j\textsuperscript{ o} / & L & LM & ‘brush’\\
\textit{g}\textit{\textsuperscript{w}}\textit{ɔlɔj} & /a-CaC -j\textsuperscript{ o} / & L & LM & ‘to silence’\\
\textit{g}\textit{\textsuperscript{w}}\textit{ɔrɔj} & /a-CaC -j\textsuperscript{ o} / & L & LM & ‘strip leaves from stalk’\\
\textit{g}\textit{\textsuperscript{w}}\textit{ɔrtsɔj} & /CCC -j\textsuperscript{ o} / & toneless? & LM & ‘sniff, slurp’\\
\textit{g}\textit{\textsuperscript{w}}\textit{ʊvɔj} & /a-CC -j\textsuperscript{ o} / & L & LM & ‘rot meat or skin to flavor food’\\
\textit{g}\textit{\textsuperscript{w}}\textit{ʊzɔj} & /a-CC -j\textsuperscript{ o} / & L & LM & ‘tan’\\
\textit{haj} & /C -j/ & H & H & ‘say’\\
\textit{haɓ} & /CC/ & L & M & ‘break’\\
\textit{haɓaj} & /CaC -j/ & toneless & LM & ‘dance’\\
\textit{hakaj} & /CaC -j/ & L & MH & ‘push’\\
\textit{halaj} & /CaC -j/ & H & HH & ‘gather, organise’\\
\textit{hamaj} & /CaC -j/ & H & HH & ‘pay certain debt’\\
\textit{həmaj} & /CC -j/ & toneless & LM & ‘run’\\
\textit{har} & /CaC/ & toneless & L & ‘make, build’\\
\textit{har} & /CaC/ & L & M & ‘carry , move’\\
\textit{hərkaj} & /CCC -j/ & toneless? & LM & ‘beg’\\
\textit{haɬ} & /CC/ & L & M & ‘swell, blow up, abcess, boil’\\
\textit{haja} & /C/ =aja & ? & HM & ‘crush, grind with stone’\\
\textit{hədzəgaɗaj} & /CCCaC -j/ & L? & MMMH & ‘limp’\\
\textit{həraɗ} & /CCC/ &  & MM & ‘jump, pull out’\\
\textit{h}\textit{\textsuperscript{w}}\textit{ʊrɓɔj} & /CCC -j\textsuperscript{ o} / & toneless? & LM & ‘dissolve’\\
\textit{h}\textit{\textsuperscript{w}}\textit{ʊɮɔj} & /a-CC -j\textsuperscript{ o} / & L & LM & ‘rot’\\
\textit{jaɗaj} & /CaC -j/ & L & MH & ‘tire’\\
\textit{jamaj} & /CaC -j/ & H & HH & ‘spin’\\
\textit{kaɓaj} & /a-CaC -j/ & L & MH & ‘cook/stir quickly next to fire’\\
\textit{kaɗ} & /CC/ & L & M & ‘kill, beat’\\
\textit{kaɗaj} & /a-CaC -j/ & L & MH & ‘prune, close eyes of dead’\\
\textit{kapaj} & /CaC -j/ & L & MH & ‘roughcast (plaster)’\\
\textit{karaj} & /CaC -j/ & H & HH & ‘steal’\\
\textit{kaɬ} & /CC/ & L & M & ‘wait, watch’\\
\textit{kəɓətsaj} & /CCC -j/ & L & MMH & ‘snap’\\
\textit{kəɓətsaj} & /CCC -j/ & L & MMH & ‘blink quickly’\\
\textit{kətsawaj} & /CCaC -j/ & L & LLM & ‘trap, seize’\\
\textit{kərɗawaj} & /CCCaC -j/ & L & LLM & ‘scrape’\\
\textit{kərɗaj} & /CCC -j/ & L & MH & ‘chew’\\
\textit{kərkaj} & /CCC -j/ & L & MH & ‘kneel’\\
\textit{kərwaj} & /CCC -j/ & toneless? & LM & ‘cultivate second time’\\
\textit{kətaj} & /CC -j/ & toneless & LM & ‘punish’\\
\textit{kuwaj} & /a-CC -j/ & L & LM & ‘search’\\
\textit{kuwaj} & /CC -j/ & ? & MH? & ‘inebriate’\\
\textit{k}\textit{\textsuperscript{w}}\textit{ʊmbʊh}\textit{\textsuperscript{w}}\textit{ɔj} & /CCC -j\textsuperscript{ o} / & L & LLM’ & ‘wrap’\\
\textit{k}\textit{\textsuperscript{w}}\textit{ɔlɔj} & /a-CaC -j\textsuperscript{ o} / & L & MH & ‘dry’\\
\textit{k}\textit{\textsuperscript{w}}\textit{ɔrɔj} & /a-CaC -j\textsuperscript{ o} / & L & LM & ‘put’\\
\textit{k}\textit{\textsuperscript{w}}\textit{ʊrsɔj} & /CCC -j\textsuperscript{ o} / & L & MH & ‘sweep’\\
\textit{k}\textit{\textsuperscript{w}}\textit{ʊrtɔj} & /CCC -j\textsuperscript{ o} / & toneless? & LM & ‘undress, peel’\\
\textit{k}\textit{\textsuperscript{w}}\textit{ʊrɔj} & /a-CC -j\textsuperscript{ o} / & L & LM & ‘mount’\\
\textit{laj} & /C -j/ & L & M & ‘dig’\\
\textit{lagaj} & /CaC -j/ & toneless & LM & ‘accompany’\\
\textit{lamaj} & /CaC -j/ & H & HH & ‘touch’\\
\textit{lawaj} & /CaC -j/ & L & MH & ‘hang’\\
\textit{lawaj} & /a-CaC -j/ & L & MH & ‘mate’\\
\textit{lɔ} & /Ca \textsuperscript{o}/ & H & H & ‘go’\\
\textit{ɬaj} & /C -j/ & L & M & ‘hunt, slit throat’\\
\textit{ɬahaj} & /CaC -j/ & H & HH & ‘mix grain and ashes to prevent insects from eating seeds’\\
\textit{ɬapaj} & /a-CaC -j/ & L & MH & ‘plait’\\
\textit{ɬar} & /CC/ & H & H & ‘send’\\
\textit{ɬaraj} & /a-CaC -j/ & L & MH & ‘slip, slide’\\
\textit{ɬəɓataj} & /CCaC -j/ & L & MMH & ‘fix, repair’\\
\textit{ɬɔh}\textit{\textsuperscript{w}}\textit{ɔj} & /a-CaC -j\textsuperscript{ o} / & L & MH & ‘leave in secret, go shamefully’\\
\textit{ɬɔh}\textit{\textsuperscript{w}}\textit{ɔj} & /a-CaC -j\textsuperscript{ o} / & L & MH & ‘take leaves off stalk’\\
\textit{ɮaɓaj} & /CaC -j/ & toneless & LM & ‘pound, beat, help up (boost), shelf’\\
\textit{ɮax} & /CaC/ & toneless & L & ‘cry (dog, cock)’\\
\textit{ɮəkaj} & /a-CC -j/ & L & LM & ‘suffer, pain, sorrow’\\
\textit{ɮaŋ} & /CC/ & L & L & ‘start, beginning’\\
\textit{ɮapaj} & /CaC -j/ & toneless & LM & ‘discuss’\\
\textit{ɮar} & /CC/ & H & H & ‘pierce, inject’\\
\textit{ɮar} & /a-CC/ & L & L & ‘kick’\\
\textit{ɮavaj} & /a-CaC -j/ & L & LM & ‘swim’\\
\textit{ɮawaj} & /CaC -j/ & toneless & LM & ‘fear’\\
\textit{ɮɪgɛ} & /CC -j \textsuperscript{e}/ & L? & LL & ‘throw, sow’\\
\textit{ɮərav} & /CCC/ & L? & LL & ‘go out, appear’\\
\textit{ɮɔk}\textit{\textsuperscript{w}}\textit{ɔj} & /a-CaC -j\textsuperscript{ o} / & L & LM & ‘gnaw’\\
\textit{ɮɔk}\textit{\textsuperscript{w}}\textit{ɔj} & /a-CaC -j\textsuperscript{ o} / & L & LM & ‘squeeze out’\\
\textit{makaj} & /CaC -j/ & L & MH & ‘stop, let go, shut up’\\
\textit{malaj} & /CaC -j/ & L & MH & ‘leave’\\
\textit{məlaj} & /a-CC -j/ & L & LM & ‘enjoy, to be happy, happiness’\\
\textit{mərtsaj} & /CCC -j/ & L & MH & ‘put horizontally, horizontal’\\
\textit{mat} & /CC/ & L & M & ‘die’\\
\textit{məndatsaj} & /CCaC -j/ & L & LLM & ‘gather’\\
\textit{mənzar } & /CCaC/ &  & LL & ‘see, resemble’\\
\textit{mbaɗ} & /CC/ & toneless & L & ‘transform, turn, change, become’\\
\textit{mbaɗaj} & /CaC -j/ & H & HH & ‘swear, jump’\\
\textit{mbahaj} & /CaC -j/ & toneless & LM & ‘call’\\
\textit{mbar} & /CC/ & H & H & ‘heal, care for’\\
\textit{mbasaj} & /a-CaC -j/ & L & LM & ‘smile, laugh’\\
\textit{mbaɮ} & /CC/ & toneless & L & ‘destroy’\\
\textit{mbɛ} & /C -j \textsuperscript{e}/ & L & M & ‘argue, scold’\\
\textit{mbɛ}\textit{ʃ}\textit{ɛŋ} & /CaCC \textsuperscript{e}/ & ? & HM & ‘rest, breathe, live, last’\\
\textit{mbɛtɛŋ} & /CaCC \textsuperscript{e}/ & ? & HM & ‘put out, extinguish’\\
\textit{mbərəmaj} & /CCC -j/ & toneless? & LM & ‘blink slowly, break violently’\\
\textit{mbərɮaj} & /CCC -j/ & toneless? & LM & ‘pass’\\
\textit{mbərwaj} & /CCC -j/ & toneless? & LM & ‘destroy violently’\\
\textit{mbɪʒɛŋ} & /CCC \textsuperscript{e}/ &  & LL & ‘spoil’\\
\textit{mbɔtsɔj} & /a-CaC -j\textsuperscript{ o} / & L & LM & ‘beat lightly’\\
\textit{mbʊlɗɔj} & /CCC -j\textsuperscript{ o} / & toneless? & LM & ‘skin, peel’\\
\textit{mbɔmɔj} & /a-CaC -j\textsuperscript{ o} / & L & LM & ‘gather with a stick’\\
\textit{mbʊrtsɔj} & /CCC -j\textsuperscript{ o} / & toneless? & LM & ‘untie’\\
\textit{nax} & /CaC/ & L & M & ‘ripen, mature’\\
\textit{ndaja} & /C / =aja & ? & ? & ‘in progress’\\
\textit{ndaɗaj} & /CaC -j/ & toneless & LM & ‘like, want’\\
\textit{ndahaj} & /CaC -j/ & toneless & LN & ‘reprimand, scold’\\
\textit{ndar} & /CC/ & H & H & ‘weave’\\
\textit{ndavaj} & /CaC -j/ & H & HH & ‘finish’\\
\textit{ndawaj} & /CaC -j/ & toneless & LM & ‘swallow’\\
\textit{ndawaj} & /a-CaC -j/ & L & LM & ‘insult, hurt’\\
\textit{ndaz} & /CC/ & toneless & L & ‘pierce’\\
\textit{ndəraj} & /CC -j/ &  & LL & ‘stay, leave’\\
\textit{ndɛɬɛŋ} & /CaCC \textsuperscript{e}/ & ? & HM & ‘make cold, cold’\\
\textit{ndəlkadaj} & /CCCaC -j/ & L & LLM & ‘lick’\\
\textit{ndʊrdɔj} & /CCC -j\textsuperscript{ o}/ & L & MH & ‘stretch’\\
\textit{ndɔɮɔj} & /CaC -j\textsuperscript{ o} / & H & HH & ‘explode’\\
\textit{nzahaj} & /a-CaC -j/ & L & LM & ‘roast’\\
\textit{nzakaj} & /CaC -j/ & H & HH & ‘find, receive, succeed, hope’\\
\textit{nzaraj} & /a-CaC -j/ & L & MH & ‘comb, separate’\\
\textit{ndʒɛ} & /C-j \textsuperscript{e}/ & H & H & ‘suffice, leave’\\
\textit{ndʒɛ} & /C -j \textsuperscript{e}/ & L & L & ‘sit, stay, live, attain, seated’\\
\textit{ndʒɛrɛŋ} & /CaCC \textsuperscript{e}/ & ? & HM & ‘groan, push baby in delivery’\\
\textit{ŋgaj} & /C -j/ & L & L & ‘to work with wood or grasses to make something’\\
\textit{ŋgəɗaj} & /a-CC -j/ & L & LM & ‘burn’\\
\textit{ŋgax} & /CaC/ & toneless & L & ‘hide, cover, protect’\\
\textit{ŋgal} & /CaC/ & toneless & L & ‘return’\\
\textit{ŋgəlaj} & /CC -j/ & H & HH 

(the only non CaC/-aj HH verb) & ‘defend’\\
\textit{ŋgar} & /CC/ & H & H & ‘prevent’\\
\textit{ŋgaraj} & /a-CaC -j/ & L & LM & ‘tear’\\
\textit{ŋgərɮaj} & /CCC -j/ & toneless? & LM & ‘to be in conflict’\\
\textit{ŋgaɮaj} & /CaC -j/ & toneless & LM & ‘show, introduce’\\
\textit{ŋgaz} & /CC/ & toneless & L & ‘flow, leak’\\
\textit{ŋgəɗatsaj} & /CCaC -j/ & L & MMH & ‘butt with horns’\\
\textit{ŋgərɗasaj} & /CCCaC -j/ & L & LLM & ‘wrinkle the skin’\\
\textit{ŋgərwaj} & /CCC -j\textsuperscript{ o} / & toneless? & LM & ‘break, tear away’\\
\textit{ŋg}\textit{\textsuperscript{w}}\textit{əɗaɬaj} & /CCaC -j\textsuperscript{ o} / & L & LLM & ‘simmer’\\
\textit{paj} & /C -j/ & L & M & ‘open’\\
\textit{pətsaj} & /CC -j/ & L & MH & ‘bring’\\
\textit{paɗaj} & /a-CaC -j/ & L & MH & ‘bite, chew’\\
\textit{pahaj} & /a-CaC -j/ & L & MH & ‘speak badly of someone for one's own interest’\\
\textit{palaj} & /CaC -j/ & H & HH & ‘choose’\\
\textit{pəl}\textit{ɗ}\textit{aj} & /CCC -j/ & L & MH & ‘shell’\\
\textit{pəlɬaj} & /CCC -j/ & L & MH & ‘split in half’\\
\textit{pamaj} & /a-CaC -j/ & L & MH & ‘fan’\\
\textit{par} & /CC/ & H & H & ‘pay’\\
\textit{pəraj} & /a-CC -j/ & L & LM & ‘spray’\\
\textit{pərtaj} & /CCC -j/ & L & MH & ‘remove forcibly’\\
\textit{pasaj} & /a-CaC -j/ & L & MH & ‘take away’\\
\textit{paɬ} & /CC/ & L & M & ‘break’\\
\textit{paɮaj} & /a-CaC -j/ & L & LM & ‘decimate, kill many’\\
\textit{pataj} & /CaC -j/ & L & MH & ‘wipe, rub’\\
\textit{pətsahaj} & /CCaC -j/ & L & MMH & ‘remove insides’\\
\textit{pəɗakaj} & /CCaC -j/ & L & MMH & ‘wake’\\
\textit{pəɗakaj} & /CCaC -j/ & L & LLM & ‘chop’\\
\textit{pəsakaj} & /CCaC -j/ & L & MMH & ‘detach’\\
\textit{pɔtsɔj} & /CaC -j\textsuperscript{ o} / &  &  & ‘wear something small (small article of clothing of leather)’\\
\textit{pɔlɔj} & /a-CaC -j\textsuperscript{ o} / & L & LM & ‘scatter’\\
\textit{rəɓaj} & /CC -j/ & L & MH & ‘be beautiful’\\
\textit{rax} & /CC/ & H? & H? & ‘fill up’\\
\textit{rax} & /CC/ & L? & M? & ‘pluck’\\
\textit{rʊtsɔj} & /CC -j\textsuperscript{ o} / & L & MH & ‘block up’\\
\textit{saɓaj} & /CaC -j/ & L & MH & ‘exceed’\\
\textit{sahaj} & /a-CaC -j/ & L & MH & ‘slander’\\
\textit{sak} & /CC/ & H & H? & ‘multiply’\\
\textit{sakaj} & /a-CaC -j/ & L & MH & ‘sift’\\
\textit{səlɗaj} & /CCC -j/ & toneless? & LM & ‘cross ankles’\\
\textit{sar} & /CC/ & H & H & ‘know’\\
\textit{sərkaj} & /CCC -j/ & L & MH & ‘get used to’\\
\textit{səɓataj} & /CCaC -j/ & L & MMH & ‘trick, tempt’\\
\textit{sədaraj} & /CCaC -j/ & L & LLM & ‘misbehave’\\
\textit{sk}\textit{\textsuperscript{w}}\textit{ɔm} & /CCC \textsuperscript{o}/ & L? & MH & ‘buy, sell, pay’\\
\textit{sija} & /C/ =aja & ? & HM & ‘cut’\\
\textit{sɔɓɔj} & /a-CaC -j\textsuperscript{ o} / & L & MH & ‘suck’\\
\textit{sɔk}\textit{\textsuperscript{w}}\textit{ɔj} & /a-CaC -j\textsuperscript{ o} / & L & MH & ‘whisper’\\
\textit{sʊlɔj} & /a-CC -j\textsuperscript{ o} / & L & MH & ‘cook on fire’\\
\textit{sɔrɔj} & /CaC -j\textsuperscript{ o} / & toneless & LM & ‘slide’\\
\textit{ʃɛ} & /C -j \textsuperscript{e}/ & L & M & ‘drink’\\
\textit{tatsaj} & /CaC -j/ & L & MH & ‘close’\\
\textit{taɗ} & /CC/ & L & M & ‘fall’\\
\textit{taf} & /CC/ & L & M & ‘spit’\\
\textit{tax} & /CaC/ & toneless & L & ‘pile’\\
\textit{tax} & /CaC/ & L & M & ‘reach out’\\
\textit{tahaj} & /CaC -j/ & L & MH & ‘boost’\\
\textit{talaj} & /CaC -j/ & H & HH & ‘take a walk’\\
\textit{tam} & /CC/ & H & H & ‘save’\\
\textit{tapaj} & /CaC -j/ & L & MH & ‘stick’\\
\textit{tar} & /CC/ & H & H & ‘enter’\\
\textit{taraj} & /CaC -j/ & L & MH & ‘call’\\
\textit{tərɗaj} & /CCC -j/ & L & MH & ‘tie off’\\
\textit{taɬaj} & /a-CaC -j/ & L & MH & ‘curse’\\
\textit{təkam} & /CCaC/ & L? & MH & ‘taste’\\
\textit{təkaraj} & /CCaC -j/ & L? & MH & ‘try, invite’\\
\textit{təkasaj} & /CCaC -j/ & L & LLM & ‘cross’\\
\textit{tʊk}\textit{\textsuperscript{w}}\textit{ɔsɔj} & /CCaC -j\textsuperscript{ o} / & L & MMH & ‘fold (legs)’\\
\textit{təlɓawaj} & /CCCaC -j/ & L & LLM & ‘be sticky’\\
\textit{tʊlɔk}\textit{\textsuperscript{w}}\textit{ɔj} & /CCaC -j\textsuperscript{ o} / & L & LLM & ‘drip’\\
\textit{təmbaɗaj} & /CCaC -j/ & L & LLM & ‘twist’\\
\textit{təmbalaj} & /CCaC -j/ & L & LLM & ‘shake out stones’\\
\textit{tʊvalaj} & /CCaC -j/ & L & LLM & ‘hunt’\\
\textit{tuwaɗaj} & /CCaC -j/ & L & LLM & ‘cross’\\
\textit{tuwɛ} & /CC -j \textsuperscript{e}/ & L & MH & ‘cry’\\
\textit{tʊɗɔj} & /CC -j\textsuperscript{ o} / & L & MH & ‘wind, roll’\\
\textit{tɔh}\textit{\textsuperscript{w}}\textit{ɔj} & /a-CaC -j\textsuperscript{ o} / & L & MH & ‘trace’\\
\textit{tɔk}\textit{\textsuperscript{w}}\textit{ɔj} & /a-CaC -j\textsuperscript{ o} / & L & MH & ‘tap’\\
\textit{tɔsɔj} & /a-CaC -j\textsuperscript{ o} / & L & MH & ‘bud, uproot’\\
\textit{tsaɓaj} & /a-CaC -j/ & L & MH & ‘skewer’\\
\textit{tsaɗaj} & /a-CaC -j/ & L & MH & ‘smooth’\\
\textit{tsaɗaj} & /CC -j/ & L & MH & ‘clear’\\
\textit{tsaɗaj} & /a-CC -j/ & L & LM & ‘castrate’\\
\textit{tsahaj} & /CaC -j/ & L & M & ‘get water’\\
\textit{tsahaj} & /CaC -j/ & H & HH & ‘ask’\\
\textit{tsahaj} & /a-CaC -j/ & L & MH & ‘scar’\\
\textit{tsapaj} & /CaC -j/ & L & MH & ‘drape, double’\\
\textit{tsar} & /CC/ & H & H & ‘climb’\\
\textit{tsar} & /CaC/ & L & M & ‘taste good’\\
\textit{tsaraj} & /CaC -j/ & H & HH & ‘tear up’\\
\textit{tsarɮaj} & /CCC -j/ & toneless? & LM & ‘fold (legs)’\\
\textit{tsaɮaj} & /a-CaC -j/ & L & MH & ‘pierce, cut’\\
\textit{tsaɮaj} & /CaC -j/ & toneless & LM & ‘ have a headache’\\
\textit{tsʊɗɔkaj} & /CCaC -j\textsuperscript{ o} / & L & MMH & ‘crouch, squat’\\
\textit{tsəfaɗaj} & /CCC -j/ & L & MMH & ‘ask’\\
\textit{tsəkafaj} & /CCaC -j/ & L & MMH & ‘get up’\\
\textit{tsəkalaj} & /CCaC -j/ & L & MMH & ‘assemble, unite’\\
\textit{tsəkaɮaj} & /CCC -j/ & L & LLM & ‘forget’\\
\textit{tsʊlɔk}\textsuperscript{w}\textit{ɔj} & /CCC -j\textsuperscript{ o} / & toneless? & LM & ‘peel’\\
\textit{tsəɮahaj} & /CCaC -j/ & L & LLM & ‘cut, chop’\\
\textit{tsɔk}\textsuperscript{w}\textit{ɔj} & /CaC -j\textsuperscript{ o} / & L & MH & ‘undress’\\
\textit{tʃɛ} & /C -j \textsuperscript{e}/ & H & H & ‘be small’\\
\textit{tʃɛfɛ} & /CaC -j \textsuperscript{e}/ & L? & MH & ‘betray’\\
\textit{tʃɛŋ} & /CC \textsuperscript{e}/ & H & H & ‘understand’\\
\textit{tʃɪdʒɛŋ} & /CCC \textsuperscript{e}/ &  & LL & ‘lose , get lost’\\
\textit{tʃɪkɛ} & /CC -j \textsuperscript{e}/ & L & MH & ‘stand up, standing’\\
\textit{vaj} & /C -j/ & L & L? & ‘winnow’\\
\textit{vahaj} & /a-CaC -j/ & L & LM & ‘fly’\\
\textit{vakaj} & /a-CaC -j/ & L & LM & ‘burn, grill’\\
\textit{vər} & /CC / & L? & L & ‘give’\\
\textit{vəlaj} & /CC -j/ & H? & HH & ‘boil’\\
\textit{var} & /a-CC/ & L & L & ‘build roof’\\
\textit{varaj} & /a-CaC -j/ & L & LM & ‘chase out’\\
\textit{vərɗaj} & /CCC -j/ & toneless? & LM & ‘boil’\\
\textit{vasaj} & /a-CaC -j/ & L & LM & ‘wipe out, cancel’\\
\textit{vawaj} & /CaC -j/ & toneless & LM & ‘twist, hang, twisted, lunacy, madness’\\
\textit{vɛ} & /C -j \textsuperscript{e}/ & L & L & ‘spend time, year’\\
\textit{vənahaj} & /CCaC -j/ & L & LLM & ‘vomit’\\
\textit{watsaj} & /CaC -j/ & H & HH & ‘write’\\
\textit{waɗaj} & /a-CaC -j/ & L & MH & ‘spread out’\\
\textit{wahaj} & /a-CaC -j/ & L & MH & ‘waste’\\
\textit{wal} & /CC/ & H & H & ‘attach, tie’\\
\textit{walaj} & /CaC -j/ & H & HH & ‘dismantle’\\
\textit{waraj} & /CaC -j/ & H & HH & ‘to take upon oneself’\\
\textit{was} & /CC/ & L & M & ‘cultivate, weed, shave’\\
\textit{wasaj} & /CaC -j/ & H & HH & ‘populate’\\
\textit{waɬ} & /CC/ & L & M & ‘is forbidden’\\
\textit{waɬaj} & /CaC -j/ & H & HH & ‘melt, liquidize’\\
\textit{waɮaj} & /a-CaC -j/ & L & LM & ‘shine’\\
\textit{wazaj} & /a-CaC -j/ & L & LM & ‘shake, shine light around’\\
\textit{wɛ} & /C {}-j\textsuperscript{e}/ & L & M & ‘give birth, be born’\\
\textit{wurkaj} & /CCC -j/ & L & MH & ‘pay’\\
\textit{wutsaɗaj} & /CCaC -j/ & L & MMH & ‘shine’\\
\textit{wuɗakaj} & /CCaC -j/ & L & MMH & ‘share, divide’\\
\textit{wuɗɔj} & /CC -j\textsuperscript{ o} / & L & MH & ‘populate’\\
\textit{wulɗɔj} & /CCC -j\textsuperscript{ o} / & toneless? & LM & ‘devour’\\
\textit{wuɮaj} & /CC -j/ & toneless & LM & ‘publish, announce’\\
\textit{zaɗ} & /CC/ & L & L & ‘take, carry’\\
\textit{zaraj} & /CaC -j/ & H & HH & ‘linger’\\
\textit{zərɗaj} & /CCC -j/ & toneless? & LM & ‘watch intently’\\
\textit{zəmbaɗaj} & /CCaC -j/ & L & LLM & ‘glorify’\\
\textit{zɔk}\textit{\textsuperscript{w}}\textit{aj} & /CaC -j\textsuperscript{ o} / & toneless & LM & ‘try’\\
\textit{zɔm} & /CC \textsuperscript{o}/ & H & H & ‘eat’\\
\textit{zʊrɔj} & /a-CC -j\textsuperscript{ o} / & L & LM & ‘notice, inspect’\\
\textit{ʒɛ} & /C -j \textsuperscript{e}/ & H & H & ‘smell, stink’\\
\lspbottomrule
\end{tabular}
\section{Verb paradigms}
\hypertarget{RefHeading1213581525720847}{}
\textbf{\textit{z}}\textbf{\textit{ɔ}}\textbf{\textit{m}}\textbf{ ‘eat’ (high tone)}

\begin{tabular}{llll}
\lsptoprule

\textbf{Nominalised Form} & \textbf{Dependent Form} & \multicolumn{2}{l}{\textbf{Imperative}}\\
\textit{m}\textit{ɪ}\textit{{}-ʒ}\textit{ʊ}\textit{m-}\textit{ɛ} & \textit{á}\textit{m}\textit{ɪ}\textit{{}-ʒ}\textit{ʊ}\textit{m-}\textit{ɛ } & \textbf{2S} & \textit{z}\textit{ɔ}\textit{m}\\
\hhline{--~~} &  & \textbf{1P}\textbf{IN} & \textit{z}\textit{ʊ}\textit{m-}\textit{ɔ}\textit{k}\textit{\textsuperscript{w}}\\
&  & \textbf{2P} & \textit{z}\textit{ʊ}\textit{m-}\textit{ɔm}\\
\hhline{~~--}
\lspbottomrule
\end{tabular}
\begin{tabular}{llllll} & \textbf{Perfective} & \textbf{Imperfective} & \textbf{Potential} & \textbf{Hortatory} & \textbf{Possible}\\
\lsptoprule
\textbf{1S} & \textit{nɔ-}\textit{z}\textit{ɔ}\textit{m} & \textit{nɔ-}\textit{z}\textit{ɔ}\textit{m} & \textit{nɔɔ-}\textit{z}\textit{ɔ}\textit{m} & \textit{nɔɔ-}\textit{z}\textit{ɔ}\textit{m} & \textit{nɔɔ-}\textit{z}\textit{ɔ}\textit{m}\\
\textbf{2S} & \textit{kɔ-}\textit{z}\textit{ɔ}\textit{m} & \textit{kɔ-}\textit{z}\textit{ɔ}\textit{m} & \textit{kɔɔ-}\textit{z}\textit{ɔ}\textit{m} & \textit{k}\textit{ɔɔ{}-}\textit{z}\textit{ɔ}\textit{m } & \textit{kɔɔ-}\textit{z}\textit{ɔ}\textit{m}\\
\textbf{3S} & \textit{ɔ{}-}\textit{z}\textit{ɔ}\textit{m} & \textit{ɔ{}-}\textit{z}\textit{ɔ}\textit{m} & \textit{ɔɔ{}-}\textit{z}\textit{ɔ}\textit{m} & \textit{m}\textit{ɔɔ{}-}\textit{z}\textit{ɔ}\textit{m} & \textit{ɔɔ{}-}\textit{z}\textit{ɔ}\textit{m}\\
\textbf{1P}\textbf{IN} & \textit{m}\textit{ʊ{}-}\textit{z}\textit{ʊ}\textit{m-}\textit{ɔ}\textit{k}\textit{\textsuperscript{w}} & \textit{m}\textit{ʊ{}-}\textit{z}\textit{ʊ}\textit{m-}\textit{ɔ}\textit{k}\textit{\textsuperscript{w}} & \textit{m}\textit{ɔ{}-}\textit{z}\textit{ʊ}\textit{m-}\textit{ɔ}\textit{k}\textit{\textsuperscript{w}} & \textit{mɔ-}\textit{z}\textit{ʊ}\textit{m-}\textit{ɔ}\textit{k}\textit{\textsuperscript{w}} & \textit{m}\textit{ɔɔ{}-}\textit{z}\textit{ʊ}\textit{m-}\textit{ɔ}\textit{k}\textit{\textsuperscript{w }}\\
\textbf{1P}\textbf{EX} & \textit{n}\textit{ʊ{}-}\textit{z}\textit{ʊ}\textit{m-}\textit{ɔ}\textit{m} & \textit{n}\textit{ʊ{}-}\textit{z}\textit{ʊ}\textit{m-}\textit{ɔ}\textit{m} & \textit{n}\textit{ɔ{}-}\textit{z}\textit{ʊ}\textit{m-}\textit{ɔ}\textit{m} & \textit{n}\textit{ɔ{}-}\textit{z}\textit{ʊ}\textit{m-}\textit{ɔ}\textit{m} & \textit{n}\textit{ɔɔ{}-}\textit{z}\textit{ʊ}\textit{m-}\textit{ɔ}\textit{m}\\
\textbf{2P} & \textit{k}\textit{ʊ{}-}\textit{z}\textit{ʊ}\textit{m-}\textit{ɔ}\textit{m} & \textit{k}\textit{ʊ{}-}\textit{z}\textit{ʊ}\textit{m-}\textit{ɔ}\textit{m} & \textit{k}\textit{ɔ{}-}\textit{z}\textit{ʊ}\textit{m-}\textit{ɔ}\textit{m} & \textit{k}\textit{ɔ{}-}\textit{z}\textit{ʊ}\textit{m-}\textit{ɔ}\textit{m} & \textit{k}\textit{ɔɔ{}-}\textit{z}\textit{ʊ}\textit{m-}\textit{ɔ}\textit{m}\\
\textbf{3P} & \textit{tɔ-}\textit{z}\textit{ɔ}\textit{m} & \textit{tɔ-}\textit{z}\textit{ɔ}\textit{m} & \textit{tɔɔ-}\textit{z}\textit{ɔ}\textit{m} & \textit{t}\textit{ɔɔ{}-}\textit{z}\textit{ɔ}\textit{m} & \textit{tɔɔ-}\textit{z}\textit{ɔ}\textit{m}\\
\lspbottomrule
\end{tabular}
\textbf{\textit{ʃɛ}}\textbf{ ‘drink’ (Low tone)}

\begin{tabular}{llll}
\lsptoprule

\textbf{Nominalised Form} & \textbf{Dependent Form} & \multicolumn{2}{l}{\textbf{Imperative}}\\
\textit{m}\textit{ɪ}\textit{{}-}\textit{ʃ{}-}\textit{ɪ}\textit{j}\textit{ɛ} & \textit{ám}\textit{ɪ}\textit{{}-}\textit{ʃ{}-}\textit{ɪ}\textit{j}\textit{ɛ} & \textbf{2S} & \textit{ʃ}\textit{ɛ}\\
\hhline{--~~} &  & \textbf{1P}\textbf{EX} & \textit{s}\textit{ɔ}\textit{k}\textit{\textsuperscript{w}}\\
&  & \textbf{2P} & \textit{s}\textit{ɔm}\\
\hhline{~~--}
\lspbottomrule
\end{tabular}
\begin{tabular}{llllll} & \textbf{Perfective} & \textbf{Imperfective} & \textbf{Potential} & \textbf{Hortatory} & \textbf{Possible}\\
\lsptoprule
\textbf{1S} & \textit{nɛ-}\textit{ʃ}\textit{ɛ} & \textit{nɛ-}\textit{ʃ}\textit{ɛ} & \textit{nɛɛ-}\textit{ʃ}\textit{ɛ} & \textit{n}\textit{ɛɛ{}-}\textit{ʃ}\textit{ɛ} & \textit{nɛɛ-}\textit{ʃ}\textit{ɛ}\\
\textbf{2S} & \textit{kɛ-}\textit{ʃ}\textit{ɛ} & \textit{kɛ-}\textit{ʃ}\textit{ɛ} & \textit{kɛɛ-}\textit{ʃ}\textit{ɛ} & \textit{k}\textit{ɛɛ{}-}\textit{ʃ}\textit{ɛ} & \textit{kɛɛ-}\textit{ʃ}\textit{ɛ}\\
\textbf{3S} & \textit{ɛ{}-}\textit{ʃ}\textit{ɛ} & \textit{ɛ{}-}\textit{ʃ}\textit{ɛ} & \textit{ɛɛ{}-}\textit{ʃ}\textit{ɛ} & \textit{m}\textit{ɛ{}-}\textit{ʃ}\textit{ɛ} & \textit{ɛɛ{}-}\textit{ʃ}\textit{ɛ}\\
\textbf{1P}\textbf{IN} & \textit{m}\textit{ɔ{}-}\textit{s}\textit{{}-}\textit{ɔ}\textit{k}\textit{\textsuperscript{w}} & \textit{m}\textit{ɔ{}-}\textit{s}\textit{{}-}\textit{ɔ}\textit{k}\textit{\textsuperscript{w}} & \textit{m}\textit{ɔɔ{}-}\textit{s}\textit{{}-}\textit{ɔ}\textit{k}\textit{\textsuperscript{w}} & \textit{mɔ-}\textit{s-}\textit{ɔ}\textit{k}\textit{\textsuperscript{w}} & \textit{m}\textit{ɔɔ{}-}\textit{s}\textit{{}-}\textit{ɔ}\textit{k}\textit{\textsuperscript{w }}\\
\textbf{1P}\textbf{EX} & \textit{n}\textit{ɔ{}-}\textit{s}\textit{{}-}\textit{ɔ}\textit{m} & \textit{n}\textit{ɔ{}-}\textit{s}\textit{{}-}\textit{ɔ}\textit{m} & \textit{n}\textit{ɔɔ{}-}\textit{s}\textit{{}-}\textit{ɔ}\textit{m} & \textit{n}\textit{ɔ{}-}\textit{s}\textit{{}-}\textit{ɔ}\textit{m} & \textit{n}\textit{ɔɔ{}-}\textit{s}\textit{{}-}\textit{ɔ}\textit{m}\\
\textbf{2P} & \textit{k}\textit{ɔ{}-}\textit{s}\textit{{}-}\textit{ɔ}\textit{m} & \textit{k}\textit{ɔ{}-}\textit{s}\textit{{}-}\textit{ɔ}\textit{m} & \textit{k}\textit{ɔɔ{}-}\textit{s}\textit{{}-}\textit{ɔ}\textit{m} & \textit{k}\textit{ɔ{}-}\textit{s}\textit{{}-}\textit{ɔ}\textit{m} & \textit{k}\textit{ɔɔ{}-}\textit{s}\textit{{}-}\textit{ɔ}\textit{m}\\
\textbf{3P} & \textit{tɛ-}\textit{ʃ}\textit{ɛ} & \textit{tɛ-}\textit{ʃ}\textit{ɛ} & \textit{tɛɛ-}\textit{ʃ}\textit{ɛ} & \textit{t}\textit{ɛɛ{}-}\textit{ʃ}\textit{ɛ} & \textit{tɛɛ-}\textit{ʃ}\textit{ɛ}\\
\lspbottomrule
\end{tabular}
\textbf{\textit{h}}\textbf{\textit{əm}}\textbf{\textit{aj}}\textbf{ ‘run’ (toneless)}

\begin{tabular}{llll}
\lsptoprule

\textbf{Nominalised Form} & \textbf{Dependent Form} & \multicolumn{2}{l}{\textbf{Imperative}}\\
\textit{m}\textit{ɪhɪmɛ} & \textit{ám}\textit{ɪhɪmɛ} & \textbf{2S} & \textit{h}\textit{əm\={a}j}\\
\hhline{--~~} &  & \textbf{1P}\textbf{EX} & \textit{h}\textit{ʊmɔk}\textit{\textsuperscript{w}}\\
&  & \textbf{2P} & \textit{h}\textit{ʊmɔm}\\
\hhline{~~--}
\lspbottomrule
\end{tabular}
\begin{tabular}{llllll} & \textbf{Perfective} & \textbf{Imperfective} & \textbf{Potential} & \textbf{Hortatory} & \textbf{Possible}\\
\lsptoprule
\textbf{1S} & \textit{n}\textit{ə{}-həm-\={a}j } & \textit{n}\textit{á-həm-\={a}j} & \textit{n}\textit{áá-həm-\={a}j} & \textit{n}\textit{àà-həm-\={a}j} & \textit{n}\textit{áà-həm-\={a}j}\\
\textbf{2S} & \textit{k}\textit{ə{}-həm-\={a}j} & \textit{k}\textit{á-həm-\={a}j} & \textit{k}\textit{áá-həm-\={a}j} & \textit{k}\textit{àà-həm-\={a}j} & \textit{k}\textit{áà-həm-\={a}j}\\
\textbf{3S} & \textit{à-həm-\={a}j} & \textit{á-həm-\={a}j} & \textit{áá-həm-\={a}j} & \textit{m}\textit{àà-həm-\={a}j} & \textit{áà-həm-\={a}j}\\
\textbf{1P}\textbf{IN} & \textit{m}\textit{ʊ{}-}\textit{h}\textit{ʊ}\textit{m-}\textit{ɔ}\textit{k}\textit{\textsuperscript{w}} & \textit{m}\textit{ʊ{}-}\textit{h}\textit{ʊ}\textit{m-}\textit{ɔ}\textit{k}\textit{\textsuperscript{w}} & \textit{m}\textit{ɔ{}-}\textit{h}\textit{ʊ}\textit{m-}\textit{ɔ}\textit{k}\textit{\textsuperscript{w}} & \textit{mɔ-həm-ɔ}\textit{k}\textit{\textsuperscript{w}} & \textit{m}\textit{ɔɔ{}- həm-ɔ}\textit{k}\textit{\textsuperscript{w }}\\
\textbf{1P}\textbf{EX} & \textit{n}\textit{ʊ{}-}\textit{h}\textit{ʊ}\textit{m-}\textit{ɔ}\textit{m} & \textit{n}\textit{ʊ{}-}\textit{h}\textit{ʊ}\textit{m-}\textit{ɔ}\textit{m} & \textit{n}\textit{ɔ{}-}\textit{h}\textit{ʊ}\textit{m-}\textit{ɔ}\textit{m} & \textit{n}\textit{ɔ{}-həm-ɔ}\textit{m} & \textit{n}\textit{ɔɔ{}- həm-ɔ}\textit{m}\\
\textbf{2P} & \textit{k}\textit{ʊ{}-}\textit{h}\textit{ʊ}\textit{m-}\textit{ɔ}\textit{m} & \textit{k}\textit{ʊ{}-}\textit{h}\textit{ʊ}\textit{m-}\textit{ɔ}\textit{m} & \textit{k}\textit{ɔ{}-}\textit{h}\textit{ə}\textit{m-}\textit{ɔ}\textit{m} & \textit{k}\textit{ɔ{}-həm-ɔ}\textit{m} & \textit{k}\textit{ɔɔ{}- həm-ɔ}\textit{m}\\
\textbf{3P} & \textit{t}\textit{ə{}-həm-\={a}j} & \textit{t}\textit{á-həm-\={a}j} & \textit{t}\textit{áá-həm-\={a}j} & \textit{t}\textit{àà-həm-\={a}j} & \textit{t}\textit{áà-həm-\={a}j}\\
\lspbottomrule
\end{tabular}
\textbf{\textit{l}}\textbf{\textit{ɔ}}\textbf{ ‘go’(Low tone Irregular)}

\begin{tabular}{llll}
\lsptoprule

\textbf{Nominalised Form} & \textbf{Dependent Form} & \multicolumn{2}{l}{\textbf{Imperative}}\\
\textit{mí-}\textit{l-}\textit{íj}\textit{ɛ} & \textit{ámí-}\textit{l-}\textit{íj}\textit{ɛ} & \textbf{2S} & \textit{l}\textit{ɔ}\\
\hhline{--~~} &  & \textbf{1P}\textbf{IN} & \textit{t}\textit{ɔk}\textit{\textsuperscript{w}}\textit{ɔ}\\
&  & \textbf{2P} & \textit{l}\textit{ɔh}\textit{\textsuperscript{w}}\textit{ɔm}\\
\hhline{~~--}
\lspbottomrule
\end{tabular}
\begin{tabular}{llllll} & \textbf{Perfective} & \textbf{Imperfective} & \textbf{Potential} & \textbf{Hortatory} & \textbf{Possible}\\
\lsptoprule
\textbf{1S} & \textit{nɔ-lɔ} & \textit{nɔ-}\textit{l}\textit{ɔ } & \textit{nɔɔ-}\textit{l}\textit{ɔ} & \textit{nɔɔ-lɔ} & \textit{nɔɔ-lɔ}\\
\textbf{2S} & \textit{kɔ-lɔ} & \textit{kɔ-}\textit{l}\textit{ɔ} & \textit{kɔɔ-}\textit{l}\textit{ɔ} & \textit{k}\textit{ɔɔ{}-lɔ} & \textit{kɔɔ-lɔ}\\
\textbf{3S} & \textit{ɔ{}-lɔ} & \textit{ɔ{}-}\textit{l}\textit{ɔ} & \textit{ɔɔ{}-}\textit{l}\textit{ɔ} & \textit{m}\textit{ɔɔ{}-lɔ} & \textit{ɔɔ{}-lɔ}\\
\textbf{1P}\textbf{IN} & \textit{m}\textit{ ʊ-}\textit{t}\textit{ɔ}\textit{{}-}\textit{ k}\textit{\textsuperscript{w}}\textit{ɔ} & \textit{m}\textit{ɔ{}-}\textit{t}\textit{ɔ}\textit{{}-}\textit{k}\textit{\textsuperscript{w}}\textit{ɔ} & \textit{m}\textit{ɔɔ{}-}\textit{t}\textit{ɔ}\textit{{}-}\textit{k}\textit{\textsuperscript{w}}\textit{ɔ} & \textit{mɔɔ-}\textit{t}\textit{ɔ}\textit{{}-}\textit{k}\textit{\textsuperscript{w}}\textit{ɔ} & \textit{m}\textit{ɔɔ{}-}\textit{t}\textit{ɔ}\textit{k}\textit{\textsuperscript{w}}\textit{ɔ}\textit{\textsuperscript{ }}\\
\textbf{1P}\textbf{EX} & \textit{n}\textit{ʊ{}-}\textit{l}\textit{ɔ}\textit{{}-h}\textit{ɔ}\textit{m } & \textit{n}\textit{ɔ{}-}\textit{l}\textit{ɔh}\textit{\textsuperscript{w}}\textit{{}-ɔ}\textit{m} & \textit{n}\textit{ɔɔ{}-}\textit{l}\textit{ɔh}\textit{\textsuperscript{w}}\textit{{}-ɔ}\textit{m} & \textit{n}\textit{ɔɔ{}-}\textit{l}\textit{ɔ}\textit{{}-h}\textit{ɔ}\textit{m} & \textit{n}\textit{ɔɔ{}-}\textit{l}\textit{ɔ}\textit{{}-h}\textit{ɔ}\textit{m}\\
\textbf{2P} & \textit{k}\textit{ʊ}\textit{{}-}\textit{l}\textit{ɔ}\textit{{}-h}\textit{ɔ}\textit{m} & \textit{k}\textit{ɔ{}-}\textit{l}\textit{ɔh}\textit{\textsuperscript{w}}\textit{{}-ɔ}\textit{m} & \textit{k}\textit{ɔɔ{}-}\textit{l}\textit{ɔh}\textit{\textsuperscript{w}}\textit{{}-ɔ}\textit{m} & \textit{k}\textit{ɔɔ{}-}\textit{l}\textit{ɔ}\textit{{}-h}\textit{ɔ}\textit{m} & \textit{k}\textit{ɔɔ{}-}\textit{l}\textit{ɔ}\textit{{}-h}\textit{ɔ}\textit{m}\\
\textbf{3P} & \textit{tɔ-lɔ} & \textit{tɔ-}\textit{l}\textit{ɔ} & \textit{tɔɔ-}\textit{l}\textit{ɔ} & \textit{t}\textit{ɔɔ{}-lɔ} & \textit{tɔɔ-lɔ}\\
\lspbottomrule
\end{tabular}
\chapter[References]{References}
\hypertarget{RefHeading1213601525720847}{}\section{References cited in this work}
\hypertarget{RefHeading1213621525720847}{}
Blama, Tchari.  1980.  Essai d’inventaire préliminaire des unités langues dans l’extrême nord du Cameroun. Yaoundé:  Université de Yaoundé, Département de Langues Africaines et Linguistiques.  

Bow, Catherine.  1997a.  Classification of Moloko.  Yaoundé:  SIL. Ms. \url{http://silcam.org/languages/languagepage.php?languageid=187} 

Bow, Catherine. 1997b. Labialisation and palatalisation in Moloko. Yaoundé: SIL. Ms. \url{http://silcam.org/languages/languagepage.php?languageid=187}

Bow, Catherine. 1997c.  A description of Moloko phonology.  Yaoundé:  SIL. Ms.

http://www.silcam.org/documents/melokwo\_bow1997\_2141\_p.pdf

Bow, Catherine.  1999.  The vowel system of Moloko, MA Thesis.  University of Melbourne. \url{http://silcam.org/languages/languagepage.php?languageid=187}  

Boyd, Virginia.  2001.  Trois textes Molokos. Yaoundé:  SIL. Ms.

Boyd, Virginia.  2002.  Initial analysis of the pitch system of Moloko nouns.  Yaoundé:  SIL. Ms. \url{http://silcam.org/languages/languagepage.php?languageid=187}

Boyd, Virginia.  2003.  A grammar of Moloko. Ms.

Bradley, Karen M. 1992.  Melokwo survey report.  Yaoundé:  SIL. Ms. \url{http://silcam.org/languages/languagepage.php?languageid=187}

Chafe, Wallace L. 1976. Givenness, Contrastiveness, Definiteness, Subjects, Topics and Point of View\textit{. }In Charles N. Li (ed.), \textit{Subject and Topic}, 27-55. New York: Academic Press. 

Comrie, Bernard. 1976. \textit{Aspect. An introduction to the study of verbal aspect and related problems}. Cambridge Textbooks in Linguistics. Cambridge: Cambridge University Press. 

de Colombel, Véronique.  1982.  Esquisse d’une classification de 18 langues tchadiques du Nord-Cameroun.  In Hermann Jungraithmayr (ed.). \textit{The Chad Languages in the Hamitosemitic-Nigritic border area – Papers of the Marburg Symposium, 1979.}  Berlin:  Verlag von Dietrich Reimer.  pp. 103-122. 

DeLancey, Scott. 1991. Event construal and case role assignment. \textit{Proceedings of the Seventeenth Annual Meeting of the Berkeley Linguistics Society: General Session and Parasession on the Grammar of Event Structure}. pp. 338-353.

\begin{styleTableheader}
Dieu, Michel \& Patrick Renaud (ed.).  1983. \textit{Atlas Linguistique du Cameroun.}  Paris: CERDOTOLA—Agence de Coopération Culturelle et Technique.
\end{styleTableheader}

Dixon, Robert. M. 2012. \textit{Basic Linguistic Theory Volume 3: Further Grammatical Topics} (Vol. 3). Oxford: Oxford University Press.

\begin{styleTableheader}
Dixon, Robert. M. W. 2003. Demonstratives. A cross-linguistic typology. \textit{Studies in Language }27(1): 61-112.
\end{styleTableheader}

\begin{styleTableheader}
Doke, Clement. M. 1935. \textit{Bantu linguistic terminology}. London: Longmans, Green.
\end{styleTableheader}

Frajzyngier, Zygmunt., \& Shay, E. 2008. Language-internal versus contact-induced change: the split coding of person and number: A Stefan Elders question. \textit{Journal of Language Contact}, \textit{2}(1), 274-296.

Frajzyngier, Zygmunt. 2012. \textit{A grammar of Wandala}. De Gruyter Mouton.

Friesen, Dianne.  2001.  Proposed segmental orthography of Moloko.  Yaoundé:  SIL. Ms. \url{http://silcam.org/languages/languagepage.php?languageid=187}

Friesen, Dianne.  2003.  Deux histoires Molokos sur l’unité et la solidarité.  Yaoundé:  SIL. Ms.

Friesen, Dianne, \& Mamalis, Megan, 2008. The Moloko verb phrase. SIL Electronic Working Papers ~\url{http://www.sil.org/silewp/abstract.asp?ref=2008-003}

\begin{styleexamplegloss}
Goldsmith, John A.  1990.  \textit{Autosegmental and Metrical Phonology}.  Oxford: Basil Blackwell.
\end{styleexamplegloss}

\begin{styleTableheader}
Gravina, Richard.  2001.  The verb phrase in Mbuko.  Yaoundé: SIL. Ms.
\end{styleTableheader}

\begin{styleTableheader}
Gravina, Richard. 2003. Topic and focus in Mbuko discourse. Yaoundé: SIL, Ms.
\end{styleTableheader}

Heine, Bernd and Tania Kuteva. 2002. \textit{World Lexicon of Grammaticalization}. Cambridge: Cambridge University Press.

Hollingsworth, Kenneth. 1991.  Tense and aspect in Mofu-Gudur.  In: Stephen C. Anderson and Bernard Comrie, eds., \textit{Tense and Aspect in Eight Languages of Cameroon}, Dallas:  SIL and University of Texas at Arlington.  pp. 239-255.

Holmaka, Marcel \& Boyd, Virginia, ed. 2002.  \textit{Ceje m}\textit{ə}\textit{lam ula} (La Maladie de mon Frère) (raconté par Oumarou Moïze).  Yaoundé: SIL. Ms.

Holmaka, Marcel, ed. 2002.  \textit{Asak ma megel k}\textit{ə}\textit{ra }(Histoire de la chasse avec mon chien) (raconté par Toukour Tadjiteke). Yaoundé: SIL. Ms.

Hyman, Larry M. 2007. Niger-Congo verb extensions: Overview and discussion. \textit{Selected Proceedings of the 37}\textit{\textsuperscript{th}}\textit{ Annual Conference on African Linguistics}, ed. Doris L. Payne and Jaime Peña, 149-163. Somerville, MA: Cascadilla Proceedings Project. 

\begin{styleReferences}
Keenan, Edward L. 1985. Relative clauses. In \textit{Language Typology and Syntactic Description}, \textit{volume II.}\textit{ Complex Constructions}, ed. by Timothy Shopen. Cambridge: Cambridge University Press. pp. 141-70.
\end{styleReferences}

Kinnaird, William J. \& Kinnaird, Anni. M. 1998. Ouldeme narrative discourse. Yaoundé:  SIL. Ms.

Kinnaird, Willian J. 2006. The Vamé verbal system, Yaoundé: SIL, Ms. 

Lambrecht, Knud. 1994. \textit{Information structure and sentence form. Topic, focus, and the mental representations of discourse referents}. Cambridge: Cambridge University Press.

Levinsohn, Stephen H. 1994. Discontinuities in coherent texts. In: \textit{Discourse features of ten languages of west-centralAfrica}, ed. by Stephen H. Levinsohn, Dallas: SIL, pp. 3-14.

Lewis, M. Paul, Gary F. Simons \& Charles D. Fennig. 2009. \textit{Ethnologue: Languages of the world} (Vol. 9). Dallas, TX: SIL international.  \href{http://www.ethnologue.com/show_family.asp?subid=90341}{http://www.ethnologue.com/} .

Longacre, Robert E., 1976. \textit{An anatomy of speech notions.} Lisse, Belgium: The Peter de Ridder Press.

Longacre, Robert E., 1996. \textit{The Grammar of discourse}. Second edition, New York: Plenum Press. 

Longacre, Robert E. and Shin Ja Hwang, 2012. \textit{Holistic discourse analysis.} Dallas, Texas: SIL International. 

Mbuagbaw, Tanyi E. 1995.  \textit{Léxique Mbuko provisoire}.  Yaoundé: CABTA.

Ndokobai, Dadak. 2006. \textit{Morphologie verbale du cuvok, une ~langue tchadique du Cameroun}. Mémoire de Diplome d'Etudes Approfondies, Faculté des Arts Lettres et Sciences Humaines, Université de Yaoundé I

Newman, Paul. 1968. Ideophones from a syntactic point of view.\textit{ Journal of West African Linguistics} 2:107-117.

Newman, Paul. 1973. Grades, vowel-tone classes and extensions in the Hausa verbal system. \textit{Studies in African Linguistics} 4(3): 297-346.

Newman, Paul, 1977. Chadic extensions and pre-dative verb forms in Hausa. \textit{Studies in African Linguistics }8(3): 275-297.

Newman, Paul. 1990. \textit{Nominal and Verbal Plurality in Chadic}. Dordrecht: Foris Publications. 

Olson, Kenneth S. and Hajek, John. 2004. A cross-linguistic lexicon of the labial flap. \textit{Linguistic Discovery} 2(2) \url{http://linguistic-discovery.dartmouth.edu/cgi-bin/WebObjects/Journals.woa/2/xmlpage/1/article/262} 

Oumar, Abraham~\& Boyd, Virginia, eds. 2002.  \textit{Mədeye alele azəbat a Məloko va }et \textit{Məkeceker ava amədəye ɗaf} (Deux textes procédurals).  Yaoundé:  SIL. Ms.

\begin{styleReferences}
Payne, Thomas. 1997. \textit{Describing morphosyntax: a guide for field linguists}. New York: Cambridge University Press.
\end{styleReferences}

\begin{styleReferences}
Radford, Andrew, 1981. \textit{Transformational Syntax. A student’s guide to Chomsky’s extended standard theory}. Cambridge: Cambridge University Press. 
\end{styleReferences}

Roberts, James, S. 2001. Phonological features of Central Chadic languages. In: Ngessimo M. Mutaka and Sammy B. Chumbow (eds.), \textit{Research Mate in African Linguistics: Focus on Cameroon}, Grammatische Analysen Afrikanischer Sprachen, 17. Köln: Rüdiger Köppe Verlag, pp. 93-118.

Rossing, Melvin Olaf, 1978.  Mafa-Mada:  A comparative study of Chadic languages in North Cameroon.  Doctor of Philosophy Thesis, University of Wisconsin.  

Shuh, Russell.  1998.  \textit{A Grammar of Miya}.  University of California Publications in Linguistics, vol. 130.  Berkley and Los Angeles:  University of California Press.  

Smith, Tony. 1999. Muyang phonology. Ms. \url{http://www.silcam.org/documents/muyang_smith1999_2354_p.pdf} 

Smith, Tony.  2002.  The Muyang verb phrase. Yaoundé: SIL. Ms. \url{http://silcam.org/languages/languagepage.php?languageid=200} 

\begin{styleReferences}
Starr, Alan. 1997. Usage des langues et des attitudes sociolinguistiques—cas des locuteurs de melokwo. Yaoundé: SIL
\end{styleReferences}

Starr, Alan, Boyd, Virginia \& Bow, Catherine. 2000. Lexique provisionnelle Moloko-Français, Yaoundé:  SIL.

Viljoen, Melanie H. 2013. A grammatical description of the Buwal language. Doctor of Philosophy Thesis. La Trobe University.

Wolff, Ekkehard.  1981.  Vocalisation patterns, prosodies, and Chadic reconstructions. In: \textit{Studies in African Linguistics}, Supplement 8, 144 -148.

Yip, Moira. 2002. \textit{Tone}. Cambridge: Cambridge University Press

\section{Bibliography of other materials written in the language}
\hypertarget{RefHeading1213641525720847}{}
These materials can be obtained from SIL, B.P. 1299, Yaounde, Cameroun.

\textbf{Primers}

\textit{Afa Mala} (At Mala’s house, Primer 1)

\textit{Mənjəye ata Aha}\textit{laj}\textit{ nə Tosoloj} (The life of Ahalay and Tosoloy, Primer 2)

Lire et ecrire Moloko (transfer primer from French)

\textit{Deftel ngam ekkitugo winndugo e janŋgugo wolde Molko} (transfer primer from Fulfulde)

\textbf{Easy readers}

\textit{Awak} (The goat)

\textit{Kosoko a Lalaway} (Lalaway market)

\textit{Ləbara a mbele mbele a moktonok nə}\textit{ kərcece} (Story of the race between the toad and the giraffe)

\textit{Ma asak a ma Məloko }(Moloko alphabet)

\textit{Mabamba tədo} (Tale of the leopard)

\textbf{Texts}

Boyd, Virginia.  2001.  Trois Textes Molokos.  Yaoundé:  SIL. Ms.

Friesen, Dianne.  2003.  Deux Histoires Molokos sur l’unité et la solidarité.  Yaoundé:  SIL. Ms.

Holmaka, Marcel \& Boyd, Virginia (ed.).  2002.  \textit{Ceje M}\textit{ə}\textit{lam Ula} (La Maladie de mon Frère) (raconté par Oumarou Moïze).  Yaoundé: SIL. Ms.  

Holmaka, Marcel, ed.  2002.  \textit{Asak ma megel k}\textit{ə}\textit{ra }(Histoire de la chasse avec mon chien) (raconté par Toukour Tadjiteke). Yaoundé: SIL. Ms.

Oumar, Abraham~\& Boyd, Virginia (eds.)  2002.  \textit{M}\textit{ə}\textit{deye alele az}\textit{ə}\textit{bat a M}\textit{ə}\textit{loko va }et \textit{M}\textit{ə}\textit{keceker ava }\textit{amədəye ɗ}\textit{~af} (Deux Textes procédurals).  Yaoundé:  SIL. Ms.

\setcounter{page}{1}\chapter[Appendix 1 Moloko{}-English Lexicon]{Appendix 1 Moloko-English Lexicon}
\hypertarget{RefHeading1213661525720847}{}\begin{multicols}{2}

\thepage{}

\end{multicols}
\begin{styleLetterParagraph}
\textstyleLetterv{A  {}-  a}
\end{styleLetterParagraph}

\begin{multicols}{2}
\begin{styleEntryParagraph}
\textstyleLexeme{a\nobreakdash-}\textstylefstandard{   }\textstylePartofspeech{vpfx.} \textstyleDefinitionn{3S}\textstylefstandard{ }\textstyleDefinitionn{subject}\textstylefstandard{.}
\end{styleEntryParagraph}

\begin{styleEntryParagraph}
\textstyleLexeme{a}\textstylefstandard{   }\textstylePartofspeech{adp.} \textstyleDefinitionn{to}\textstylefstandard{.}
\end{styleEntryParagraph}

\begin{styleEntryParagraph}
\textstyleLexeme{a…ava}\textstylefstandard{   }\textstylePartofspeech{adp.} \textstyleDefinitionn{in}\textstylefstandard{.}
\end{styleEntryParagraph}

\begin{styleEntryParagraph}
\textstyleLexeme{aba}\textstylefstandard{   }\textstylePartofspeech{ext.} \textstyleDefinitionn{there is}\textstylefstandard{.}
\end{styleEntryParagraph}

\begin{styleEntryParagraph}
\textstyleLexeme{abalak}\textstylefstandard{   }\textstylePartofspeech{n.} \textstyleDefinitionn{hangar to give shade in front of a house}\textstylefstandard{.} 
\end{styleEntryParagraph}

\begin{styleEntryParagraph}
\textstyleLexeme{Aban}\textstylefstandard{   }\textstylePartofspeech{n.pr.} \textstyleDefinitionn{name of child following twins}\textstylefstandard{.} \textstyleflabel{Cf.: }\textstylefvernacular{Masay, Aləwa}\textstylefstandard{.}
\end{styleEntryParagraph}

\begin{styleEntryParagraph}
\textstyleLexeme{abangay}\textstylefstandard{   }\textstylePartofspeech{n.} large bright star; planet \textstyleDefinitionn{Venus}\textstylefstandard{.}
\end{styleEntryParagraph}

\begin{styleIndentedParagraph}
\textstyleSubentry{abangay dedew}\textstylefstandard{   }\textstylePartofspeech{n.} \textstyleDefinitionn{star of the morning}\textstylefstandard{.}
\end{styleIndentedParagraph}

\begin{styleIndentedParagraph}
\textstyleSubentry{abangay aləho}\textstylefstandard{   }\textstylePartofspeech{n.} \textstyleDefinitionn{star of the night}\textstylefstandard{.} 
\end{styleIndentedParagraph}

\begin{styleEntryParagraph}
\textstyleLexeme{abay}\textstylefstandard{   }\textstylePartofspeech{ext. } \textit{there is not}\textstylefstandard{.}
\end{styleEntryParagraph}

\begin{styleEntryParagraph}
\textstyleLexeme{abəlgamay}\textstylefstandard{   }\textstylefstandard{\textit{ID.}}\textstylePartofspeech{n.} \textstyleDefinitionn{the way a sick person walks.}
\end{styleEntryParagraph}

\begin{styleEntryParagraph}
\textstyleLexeme{aɓalan}\textstylefstandard{   }\textstylePartofspeech{n.} \textstyleDefinitionn{goat horn}\textstylefstandard{.}
\end{styleEntryParagraph}

\begin{styleEntryParagraph}
\textstyleLexeme{aɓəsay}\textstylefstandard{   }\textstylePartofspeech{n.} \textstyleDefinitionn{blemish}\textstylefstandard{.}
\end{styleEntryParagraph}

\begin{styleEntryParagraph}
\textstyleLexeme{adama}\textstylefstandard{   }\textstylePartofspeech{n.} \textstyleDefinitionn{adultery}\textstylefstandard{.} 
\end{styleEntryParagraph}

\begin{styleEntryParagraph}
\textstyleLexeme{adamay}\textstylefstandard{   }\textstylePartofspeech{n.} \textit{spouse’s sibling}\textstyleExamplefreetransn{\textup{.}}
\end{styleEntryParagraph}

\begin{styleEntryParagraph}
\textstyleLexeme{adangay}\textstylefstandard{   }\textstylePartofspeech{n.} \textit{stick}\textstyleExamplefreetransn{.}
\end{styleEntryParagraph}

\begin{styleEntryParagraph}
\textstyleLexeme{adan bay}\textstylefstandard{ }\textstylePartofspeech{adv.} \textit{perhaps}\textstyleExamplefreetransn{.}
\end{styleEntryParagraph}

\begin{styleEntryParagraph}
\textstyleLexeme{afa}\textstylefstandard{   }\textstylePartofspeech{adp.} \textstyleDefinitionn{at the house of}\textstylefstandard{.} 
\end{styleEntryParagraph}

\begin{styleEntryParagraph}
\textstyleLexeme{agaban}\textstylefstandard{   }\textstylePartofspeech{n.} \textstyleDefinitionn{sesame seeds/plant}\textstylefstandard{.}
\end{styleEntryParagraph}

\begin{styleEntryParagraph}
\textstyleLexeme{agwazlak}\textstylefstandard{   }\textstylePartofspeech{n.} \textstyleDefinitionn{rooster}\textstylefstandard{.}
\end{styleEntryParagraph}

\begin{styleEntryParagraph}
\textstyleLexeme{agwəjer}\textstylefstandard{   }\textstylePartofspeech{n.} \textstyleDefinitionn{grass}\textstylefstandard{.} 
\end{styleEntryParagraph}

\begin{styleEntryParagraph}
\textstyleLexeme{ahakay}\textstylefstandard{   }\textstylePartofspeech{adv.} \textstyleDefinitionn{here}\textstylefstandard{.}
\end{styleEntryParagraph}

\begin{styleEntryParagraph}
\textstyleLexeme{ahan}\textstylefstandard{ }\textstylefstandard{\textit{n}}\textstylePartofspeech{clitic. 3S possessive.}
\end{styleEntryParagraph}

\begin{styleEntryParagraph}
\textstyleLexeme{ahar}\textstylefstandard{   }\textstylePartofspeech{n.} \textstyleDefinitionn{hand}\textstylefstandard{.}
\end{styleEntryParagraph}

\begin{styleIndentedParagraph}
\textstyleSubentry{baba ahar}\textstylefstandard{   }\textstylePartofspeech{n.}\textstyleDefinitionn{thumb}\textstylefstandard{.}
\end{styleIndentedParagraph}

\begin{styleIndentedParagraph}
\textstyleSubentry{war ahar}\textstylefstandard{   }\textstylePartofspeech{n.} \textstyleDefinitionn{finger}\textstylefstandard{.}
\end{styleIndentedParagraph}

\begin{styleIndentedParagraph}
\textstyleSubentry{bəbəza}\textstyleSubentry{ ahar ahay}\textstylefstandard{   }\textstylePartofspeech{n.} \textstyleDefinitionn{fingers}\textstylefstandard{.}
\end{styleIndentedParagraph}

\begin{styleEntryParagraph}
\textstyleLexeme{ahay}\textstylefstandard{   }\textstylePartofspeech{nclitic.} \textstyleDefinitionn{plural}\textstylefstandard{.}
\end{styleEntryParagraph}

\begin{styleEntryParagraph}
\textstyleLexeme{aka}\textstylefstandard{   }\textstylefstandard{\textit{v}}\textstylePartofspeech{clitic.} \textstyleDefinitionn{on (top of)}\textstylefstandard{.} 
\end{styleEntryParagraph}

\begin{styleEntryParagraph}
\textstyleLexeme{akar}\textstylefstandard{   }\textstylePartofspeech{n.} \textstyleDefinitionn{theft}\textstylefstandard{.}
\end{styleEntryParagraph}

\begin{styleEntryParagraph}
\textstyleLexeme{ala}\textstylefstandard{   }\textstylePartofspeech{vclitic.} towards
\end{styleEntryParagraph}

\begin{styleEntryParagraph}
\textstyleLexeme{alahar}\textstylefstandard{   }\textstylePartofspeech{n.} \textstyleDefinitionn{weapon, bracelet}\textstylefstandard{.}
\end{styleEntryParagraph}

\begin{styleEntryParagraph}
\textstyleLexeme{alay}\textstylefstandard{   }\textstylePartofspeech{vclitic.} away
\end{styleEntryParagraph}

\begin{styleEntryParagraph}
\textstyleLexeme{albaya}\textstylefstandard{   }\textstylePartofspeech{n.} \textstyleDefinitionn{young man}\textstylefstandard{.}
\end{styleEntryParagraph}

\begin{styleEntryParagraph}
\textstyleLexeme{almamar}\textstylefstandard{   }\textstylePartofspeech{n.} \textstyleDefinitionn{dry season}\textstylefstandard{.}
\end{styleEntryParagraph}

\begin{styleEntryParagraph}
\textstyleLexeme{aloko}\textstylefstandard{   }\textstylefstandard{\textit{nclitic}}\textstylePartofspeech{.1P}\textstylePartofspeech{IN}\textstyleDefinitionn{ possessive}\textstylefstandard{.}
\end{styleEntryParagraph}

\begin{styleEntryParagraph}
\textstyleLexeme{aloko}\textstylefstandard{   }\textstylefstandard{\textit{vclitic}}\textstylePartofspeech{.1P}\textstylePartofspeech{IN}\textstyleDefinitionn{ indirect object}\textstylefstandard{.}
\end{styleEntryParagraph}

\begin{styleEntryParagraph}
\textstyleLexeme{aləkwəye}\textstylefstandard{\textit{   nclitic}}\textstylePartofspeech{.} \textstyleDefinitionn{2P possessive}\textstylefstandard{.}
\end{styleEntryParagraph}

\begin{styleEntryParagraph}
\textstyleLexeme{aləkwəye}\textstylefstandard{\textit{   vclitic}}\textstylePartofspeech{.} \textstyleDefinitionn{2P indirect object}\textstylefstandard{.}
\end{styleEntryParagraph}

\begin{styleEntryParagraph}
\textstyleLexeme{aləme}\textstylefstandard{   }\textstylefstandard{\textit{nclitic}}\textstylePartofspeech{.1P}\textstylePartofspeech{EX}\textstyleDefinitionn{ possessive}\textstylefstandard{.}
\end{styleEntryParagraph}

\begin{styleEntryParagraph}
\textstyleLexeme{aləme}\textstylefstandard{   }\textstylefstandard{\textit{vclitic}}\textstylePartofspeech{.1P}\textstylePartofspeech{EX}\textstyleDefinitionn{ indirect object}\textstylefstandard{.}
\end{styleEntryParagraph}

\begin{styleEntryParagraph}
\textstyleLexeme{Aləwa}\textstylefstandard{   }\textstylePartofspeech{n.pr.} \textit{name of the second twin}\textstylefstandard{.} \textstyleflabel{Cf.: }\textstylefvernacular{Masay}\textstylefstandard{.}
\end{styleEntryParagraph}

\begin{styleEntryParagraph}
\textstyleLexeme{almay}\textstylefstandard{   }\textstylePartofspeech{pn. what}\textstylefstandard{.}
\end{styleEntryParagraph}

\begin{styleEntryParagraph}
\textstyleLexeme{amar}\textstylefstandard{   }\textstylePartofspeech{n.} \textstyleDefinitionn{oil.}
\end{styleEntryParagraph}

\begin{styleEntryParagraph}
\textstyleLexeme{amata}\textstylefstandard{   }\textstylePartofspeech{n.}\textstyleDefinitionn{outside.}
\end{styleEntryParagraph}

\begin{styleEntryParagraph}
\textstyleLexeme{ambay}\textstylefstandard{   }\textstylePartofspeech{n.} \textstyleDefinitionn{manioc}\textstylefstandard{.}
\end{styleEntryParagraph}

\begin{styleEntryParagraph}
\textstyleLexeme{ambəlak}\textstylefstandard{   }\textstylePartofspeech{n.} \textstyleDefinitionn{cut, sore}\textstylefstandard{.}
\end{styleEntryParagraph}

\begin{styleEntryParagraph}
\textstyleLexeme{amtamay}\textstylefstandard{   }\textstylePartofspeech{pn.} \textstyleDefinitionn{where}\textstylefstandard{.}
\end{styleEntryParagraph}

\begin{styleEntryParagraph}
\textstyleLexeme{an}\textstylefstandard{   }\textstylePartofspeech{vclitic. }\textstyleDefinitionn{3S indirect object}\textstylefstandard{.}
\end{styleEntryParagraph}

\begin{styleEntryParagraph}
\textstyleLexeme{ana}\textstylefstandard{   }\textstylePartofspeech{adp.} \textstyleDefinitionn{to}\textstylefstandard{.} 
\end{styleEntryParagraph}

\begin{styleEntryParagraph}
\textstyleLexeme{andakay}\textstylefstandard{   }\textstylePartofspeech{interj.} \textstyleDefinitionn{what’s his/her name}\textstylefstandard{.}
\end{styleEntryParagraph}

\begin{styleEntryParagraph}
\textstyleLexeme{andəbaba}\textstylefstandard{   }\textstylePartofspeech{n.} \textstyleDefinitionn{duck}\textstylefstandard{.}
\end{styleEntryParagraph}

\begin{styleEntryParagraph}
\textstyleLexeme{andəra}\textstylefstandard{   }\textstylePartofspeech{n.} \textstyleDefinitionn{peanut}\textstylefstandard{.}
\end{styleEntryParagraph}

\begin{styleEntryParagraph}
\textstyleLexeme{anga}\textstylefstandard{   }\textstylePartofspeech{adp.} \textstyleDefinitionn{possessive}\textstylefstandard{.}
\end{styleEntryParagraph}

\begin{styleEntryParagraph}
\textstyleLexeme{ango}\textstylefstandard{   }\textstylefstandard{\textit{nclitic}}\textstylePartofspeech{.} \textstyleDefinitionn{2S possessive}\textstylefstandard{.}
\end{styleEntryParagraph}

\begin{styleEntryParagraph}
\textstyleExamplev{angolay}\textstylefstandard{\textit{ v}}\textstylePartofspeech{.} \textstyleDefinitionn{take courage}\textstylefstandard{.}
\end{styleEntryParagraph}

\begin{styleEntryParagraph}
\textstyleLexeme{angwərzla}\textstylefstandard{   }\textstylePartofspeech{n.} \textstyleDefinitionn{sparrow}\textstylefstandard{.}
\end{styleEntryParagraph}

\begin{styleEntryParagraph}
\textstyleLexeme{anjakar}\textstylefstandard{   }\textstylePartofspeech{n.} \textstyleDefinitionn{chicken}\textstylefstandard{.}
\end{styleEntryParagraph}

\begin{styleEntryParagraph}
\textstyleLexeme{apazan}\textstylefstandard{   }\textstylePartofspeech{adv.} \textstyleDefinitionn{yesterday}\textstylefstandard{.}
\end{styleEntryParagraph}

\begin{styleEntryParagraph}
\textstyleLexeme{asa}\textstylefstandard{   }\textstylePartofspeech{conj.} \textstyleDefinitionn{if}\textstylefstandard{.}
\end{styleEntryParagraph}

\begin{styleEntryParagraph}
\textstyleLexeme{asabay}\textstylefstandard{   }\textstylePartofspeech{adv.} \textstyleDefinitionn{never again}\textstylefstandard{.}
\end{styleEntryParagraph}

\begin{styleEntryParagraph}
\textstyleLexeme{asak}\textstylefstandard{   }\textstylePartofspeech{n.} \textstyleDefinitionn{foot, leg}\textstylefstandard{.}
\end{styleEntryParagraph}

\begin{styleEntryParagraph}
\textstyleLexeme{asara}\textstylefstandard{   }\textstylePartofspeech{n.} \textstyleDefinitionn{Westerner}\textstylefstandard{.}
\end{styleEntryParagraph}

\begin{styleEntryParagraph}
\textstyleLexeme{asəbo}\textstylefstandard{   }\textstylePartofspeech{adv.} \textstyleDefinitionn{below.}
\end{styleEntryParagraph}

\begin{styleEntryParagraph}
\textstyleLexeme{aslar}\textstylefstandard{   }\textstylePartofspeech{n.} \textstyleDefinitionn{tooth}\textstylefstandard{.}
\end{styleEntryParagraph}

\begin{styleEntryParagraph}
\textstyleLexeme{ata}\textstylefstandard{   }\textstylePartofspeech{vclitic. }\textstyleDefinitionn{3P indirect object}\textstylefstandard{.}
\end{styleEntryParagraph}

\begin{styleEntryParagraph}
\textstyleLexeme{atəko}\textstylefstandard{   }\textstylePartofspeech{n. }\textstyleDefinitionn{okra}\textstylefstandard{.}
\end{styleEntryParagraph}

\begin{styleEntryParagraph}
\textstyleLexeme{atəta}\textstylefstandard{   }\textstylePartofspeech{nclitic. }\textstyleDefinitionn{3P possessive}\textstylefstandard{.}
\end{styleEntryParagraph}

\begin{styleEntryParagraph}
\textstyleLexeme{ava}\textstylefstandard{   }\textstylePartofspeech{n.} \textstyleDefinitionn{arrow}\textstylefstandard{.}
\end{styleEntryParagraph}

\begin{styleEntryParagraph}
\textstyleLexeme{ava}\textstylefstandard{   }\textstylePartofspeech{vclitic. }\textstyleDefinitionn{in}\textstylefstandard{.}
\end{styleEntryParagraph}

\begin{styleEntryParagraph}
\textstyleLexeme{ava}\textstylefstandard{   }\textstylePartofspeech{adp.} \textstyleDefinitionn{in}\textstylefstandard{.}
\end{styleEntryParagraph}

\begin{styleEntryParagraph}
\textstyleLexeme{ava}\textstylefstandard{   }\textstylePartofspeech{ext.} \textstyleDefinitionn{there is (in a place)}\textstylefstandard{.}
\end{styleEntryParagraph}

\begin{styleEntryParagraph}
\textstyleLexeme{avar}\textstylefstandard{   }\textstylePartofspeech{n.} \textstyleDefinitionn{rain}\textstylefstandard{.}
\end{styleEntryParagraph}

\begin{styleEntryParagraph}
\textstyleLexeme{avəlo}\textstylefstandard{   }\textstylePartofspeech{adv.} \textstyleDefinitionn{above.}
\end{styleEntryParagraph}

\begin{styleEntryParagraph}
\textstyleLexeme{avəya}\textstylefstandard{   }\textstylePartofspeech{n.} \textstyleDefinitionn{suffering}\textstylefstandard{.}
\end{styleEntryParagraph}

\begin{styleEntryParagraph}
\textstyleLexeme{{}-aw}\textstylefstandard{   }\textstylePartofspeech{vclitic. }\textstyleDefinitionn{1S indirect object}\textstylefstandard{.}
\end{styleEntryParagraph}

\begin{styleEntryParagraph}
\textstyleLexeme{awak}\textstylefstandard{   }\textstylePartofspeech{n.} \textstyleDefinitionn{goat}\textstylefstandard{.}
\end{styleEntryParagraph}

\begin{styleEntryParagraph}
\textstyleLexeme{awəy}\textstylefstandard{   }\textstylePartofspeech{v.} \textstyleDefinitionn{saying}\textstylefstandard{.}
\end{styleEntryParagraph}

\begin{styleEntryParagraph}
\textstyleLexeme{ayah}\textstylefstandard{   }\textstylePartofspeech{n.} \textstyleDefinitionn{squirrel}\textstylefstandard{.}
\end{styleEntryParagraph}

\begin{styleEntryParagraph}
\textstyleLexeme{ayaw}\textstylefstandard{   }\textstylePartofspeech{adv.} \textstyleDefinitionn{yes}\textstylefstandard{.}
\end{styleEntryParagraph}

\begin{styleEntryParagraph}
\textstyleLexeme{ayokon}\textstylefstandard{   }\textstylePartofspeech{adv. }\textstyleDefinitionn{agreed}\textstylefstandard{.}
\end{styleEntryParagraph}

\begin{styleEntryParagraph}
\textstyleLexeme{ayva}\textstylefstandard{   }\textstylePartofspeech{n. }\textstyleDefinitionn{inside house}\textstylefstandard{.}
\end{styleEntryParagraph}

\begin{styleEntryParagraph}
\textstyleLexeme{azana}\textstylefstandard{   }\textstylePartofspeech{adv.} \textstyleDefinitionn{perhaps}\textstylefstandard{.}
\end{styleEntryParagraph}

\begin{styleEntryParagraph}
\textstyleLexeme{azan}\textstylefstandard{   }\textstylePartofspeech{n.} \textstyleDefinitionn{temptation, trap}\textstylefstandard{.}
\end{styleEntryParagraph}

\begin{styleEntryParagraph}
\textstyleLexeme{azay}\textstylefstandard{   }\textstylePartofspeech{n.} \textstyleDefinitionn{excrement, faeces.}
\end{styleEntryParagraph}

\begin{styleEntryParagraph}
\textstyleLexeme{  azay}\textstylefstandard{ }\textstyleLexeme{andəra }\textstylePartofspeech{n.} \textstyleDefinitionn{deep-fried pastry made from peanuts after the oil is removed.}
\end{styleEntryParagraph}

\begin{styleEntryParagraph}
\textstyleLexeme{azəɓat}\textstylefstandard{   }\textstylePartofspeech{n.} \textstyleDefinitionn{a dish made of bean leaves}\textstylefstandard{.}
\end{styleEntryParagraph}

\begin{styleEntryParagraph}
\textstyleLexeme{azlam}\textstylefstandard{   }\textstylePartofspeech{n.} \textstyleDefinitionn{vulture}\textstylefstandard{.}
\end{styleEntryParagraph}

\begin{styleEntryParagraph}
\textstyleLexeme{azla}\textstylefstandard{  }\textstylePartofspeech{adv.} \textstyleDefinitionn{now}\textstylefstandard{.}
\end{styleEntryParagraph}

\begin{styleEntryParagraph}
\textstyleLexeme{azla na}\textstylefstandard{   }\textstylePartofspeech{conj.} \textstyleDefinitionn{but}\textstylefstandard{.}
\end{styleEntryParagraph}
\end{multicols}
\begin{multicols}{2}
\end{multicols}
\begin{styleLetterParagraph}
\textstyleLetterv{B  {}-  b }
\end{styleLetterParagraph}

\begin{multicols}{2}
\begin{styleEntryParagraph}
\textstyleLexeme{baba}\textstylefstandard{   }\textstylePartofspeech{n.} \textstyleDefinitionn{father}\textstylefstandard{.}
\end{styleEntryParagraph}

\begin{styleEntryParagraph}
\textstyleLexeme{babək}\textstylefstandard{   }\textstylePartofspeech{ID.} \textstyleDefinitionn{idea of burying}\textstylefstandard{.}
\end{styleEntryParagraph}

\begin{styleEntryParagraph}
\textstyleLexeme{babəza}\textstylefstandard{   }\textstylePartofspeech{n.} \textstyleDefinitionn{children}\textstylefstandard{.}
\end{styleEntryParagraph}

\begin{styleEntryParagraph}
\textstyleLexeme{baɗay}\textstylefstandard{   }\textstylePartofspeech{v.} \textstyleDefinitionn{marry}\textstylefstandard{.}
\end{styleEntryParagraph}

\begin{styleEntryParagraph}
\textstyleLexeme{bah}\textstylefstandard{   }\textstylePartofspeech{v.} \textstyleDefinitionn{pour}\textstylefstandard{.}
\end{styleEntryParagraph}

\begin{styleEntryParagraph}
\textstyleLexeme{bahay}\textstylefstandard{   }\textstylePartofspeech{n.} \textstyleDefinitionn{chief}\textstylefstandard{.}
\end{styleEntryParagraph}

\begin{styleEntryParagraph}
\textstyleLexeme{bakaka}\textstylefstandard{   }\textstylePartofspeech{ID.} \textstyleDefinitionn{spicy hot taste}\textstylefstandard{.}
\end{styleEntryParagraph}

\begin{styleEntryParagraph}
\textstyleLexeme{bal}\textstylefstandard{   }\textstylePartofspeech{v.} \textstyleDefinitionn{move}\textstylefstandard{.}
\end{styleEntryParagraph}

\begin{styleEntryParagraph}
\textstyleLexeme{balon}\textstylefstandard{   }\textstylePartofspeech{n.} \textstyleDefinitionn{soccer ball/soccer}\textstylefstandard{.}
\end{styleEntryParagraph}

\begin{styleEntryParagraph}
\textstyleLexeme{balay}\textstylefstandard{   }\textstylePartofspeech{v.} \textstyleDefinitionn{wash}\textstylefstandard{.}
\end{styleEntryParagraph}

\begin{styleEntryParagraph}
\textstyleLexeme{bamba}\textstylefstandard{   }\textstylePartofspeech{n.} \textstyleDefinitionn{story}\textstylefstandard{.}
\end{styleEntryParagraph}

\begin{styleEntryParagraph}
\textstyleLexeme{barka}\textstylefstandard{   }\textstylePartofspeech{n.} \textstyleDefinitionn{blessing (from Fulfuldé)}\textstylefstandard{.}
\end{styleEntryParagraph}

\begin{styleEntryParagraph}
\textstyleLexeme{baskwar}\textstylefstandard{   }\textstylePartofspeech{n.} \textstyleDefinitionn{bicycle}\textstylefstandard{.}
\end{styleEntryParagraph}

\begin{styleEntryParagraph}
\textstyleLexeme{batay}\textstylefstandard{   }\textstylePartofspeech{v.} \textstyleDefinitionn{evaporate}\textstylefstandard{.}
\end{styleEntryParagraph}

\begin{styleEntryParagraph}
\textstyleLexeme{bay}\textstylefstandard{   }\textstylePartofspeech{NEG.} \textstyleDefinitionn{not}\textstylefstandard{.}
\end{styleEntryParagraph}

\begin{styleEntryParagraph}
\textstyleLexeme{bay}\textstylefstandard{   }\textstylePartofspeech{v.} \textstyleDefinitionn{light}\textstylefstandard{.}
\end{styleEntryParagraph}

\begin{styleEntryParagraph}
\textstyleLexeme{baya}\textstylefstandard{   }\textstylePartofspeech{n.}\textstyleDefinitionn{one time, occasion}\textstylefstandard{.}
\end{styleEntryParagraph}

\begin{styleEntryParagraph}
\textstyleLexeme{baybojo}\textstylefstandard{   }\textstylePartofspeech{n.} \textstyleDefinitionn{lizard}\textstylefstandard{.}
\end{styleEntryParagraph}

\begin{styleEntryParagraph}
\textstyleLexeme{baz}\textstylefstandard{   }\textstylePartofspeech{v.} \textstyleDefinitionn{harvest}\textstylefstandard{.}
\end{styleEntryParagraph}

\begin{styleEntryParagraph}
\textstyleLexeme{bazlay}\textstylefstandard{   }\textstylePartofspeech{v.} \textstyleDefinitionn{breathe}\textstylefstandard{.}
\end{styleEntryParagraph}

\begin{styleEntryParagraph}
\textstyleLexeme{beke}\textstylefstandard{   }\textstylePartofspeech{n.} \textstyleDefinitionn{slave}\textstylefstandard{.}
\end{styleEntryParagraph}

\begin{styleEntryParagraph}
\textstyleLexeme{bəfa}\textstylefstandard{   }\textstylePartofspeech{ID.} \textstyleDefinitionn{idea of being close}\textstylefstandard{.}
\end{styleEntryParagraph}

\begin{styleEntryParagraph}
\textstyleLexeme{bəjakay}\textstylefstandard{   }\textstylePartofspeech{v.} \textstyleDefinitionn{dig shallow}\textstylefstandard{.}
\end{styleEntryParagraph}

\begin{styleEntryParagraph}
\textstyleLexeme{bəjəgamay}\textstylefstandard{   }\textstylePartofspeech{v.} \textstyleDefinitionn{crawl}\textstylefstandard{.}
\end{styleEntryParagraph}

\begin{styleEntryParagraph}
\textstyleLexeme{bəlay}\textstylefstandard{   }\textstylePartofspeech{n.} \textstyleDefinitionn{sea,}
\end{styleEntryParagraph}

\begin{styleEntryParagraph}
\textstyleLexeme{bəlen}\textstylefstandard{   }\textstylePartofspeech{num.} \textstyleDefinitionn{one}\textstylefstandard{.}
\end{styleEntryParagraph}

\begin{styleEntryParagraph}
\textstyleLexeme{bərkaday}\textstylefstandard{   }\textstylePartofspeech{v.} \textstyleDefinitionn{collect, squeeze}\textstylefstandard{.}
\end{styleEntryParagraph}

\begin{styleEntryParagraph}
\textstyleLexeme{bərwaɗay}\textstylefstandard{   }\textstylePartofspeech{v.} \textstyleDefinitionn{drive}\textstylefstandard{.}
\end{styleEntryParagraph}

\begin{styleEntryParagraph}
\textstyleLexeme{bəway}\textstylefstandard{   }\textstylePartofspeech{n.} \textstyleDefinitionn{baboon}\textstylefstandard{.}
\end{styleEntryParagraph}

\begin{styleEntryParagraph}
\textstyleLexeme{bəwce}\textstylefstandard{   }\textstylePartofspeech{n.} \textstyleDefinitionn{mat}\textstylefstandard{.}
\end{styleEntryParagraph}

\begin{styleEntryParagraph}
\textstyleLexeme{bəwɗere}\textstylefstandard{   }\textstylePartofspeech{ID. idea of }\textstyleDefinitionn{foolishness}\textstylefstandard{.}
\end{styleEntryParagraph}

\begin{styleEntryParagraph}
\textstyleLexeme{bəyaw}\textstylefstandard{   }\textstylePartofspeech{n.} \textstyleDefinitionn{next year}\textstylefstandard{.}
\end{styleEntryParagraph}

\begin{styleEntryParagraph}
\textstyleLexeme{bəyna}\textstylefstandard{   }\textstylePartofspeech{conj. }\textstyleDefinitionn{because}\textstylefstandard{.}
\end{styleEntryParagraph}

\begin{styleEntryParagraph}
\textstyleLexeme{bokay}\textstylefstandard{   }\textstylePartofspeech{v.} \textstyleDefinitionn{cultivate a second time; be bald}\textstylefstandard{.}
\end{styleEntryParagraph}

\begin{styleEntryParagraph}
\textstyleLexeme{bolay}\textstylefstandard{   }\textstylePartofspeech{v.} \textstyleDefinitionn{knead, soak}\textstylefstandard{.}
\end{styleEntryParagraph}

\begin{styleEntryParagraph}
\textstyleLexeme{botot}\textstylefstandard{   }\textstylePartofspeech{ID.} \textstyleDefinitionn{idea of flying away}\textstylefstandard{.}
\end{styleEntryParagraph}

\begin{styleEntryParagraph}
\textstyleLexeme{bozlom}\textstylefstandard{   }\textstylePartofspeech{n.} \textstyleDefinitionn{cheek}\textstylefstandard{.}
\end{styleEntryParagraph}
\end{multicols}
\begin{styleLetterParagraph}
\textstyleLetterv{Ɓ  {}-  ɓ}
\end{styleLetterParagraph}

\begin{multicols}{2}
\begin{styleEntryParagraph}
\textstyleLexeme{ɓah}\textstylefstandard{   }\textstylePartofspeech{v.} \textstyleDefinitionn{sew}\textstylefstandard{.}
\end{styleEntryParagraph}

\begin{styleEntryParagraph}
\textstyleLexeme{ɓal}\textstylefstandard{   }\textstylePartofspeech{v.} \textstyleDefinitionn{stir}\textstylefstandard{.}
\end{styleEntryParagraph}

\begin{styleEntryParagraph}
\textstyleLexeme{ɓalay}\textstylefstandard{   }\textstylePartofspeech{v.} \textstyleDefinitionn{build}\textstylefstandard{.}
\end{styleEntryParagraph}

\begin{styleEntryParagraph}
\textstyleLexeme{ɓar}\textstylefstandard{   }\textstylePartofspeech{v.} \textstyleDefinitionn{ shoot an arrow}\textstylefstandard{.}
\end{styleEntryParagraph}

\begin{styleEntryParagraph}
\textstyleLexeme{ɓaray}\textstylefstandard{   }\textstylePartofspeech{v.} \textstyleDefinitionn{toss and turn while sick}\textstylefstandard{.} 
\end{styleEntryParagraph}

\begin{styleEntryParagraph}
\textstyleLexeme{ɓasay}\textstylefstandard{   }\textstylePartofspeech{v.} \textstyleDefinitionn{tolerate}\textstylefstandard{.}
\end{styleEntryParagraph}

\begin{styleEntryParagraph}
\textbf{ɓ}\textbf{a}\textbf{vb}\textstylefstandard{   }\textstylePartofspeech{ID. }\textstyleDefinitionn{sound/idea of something falling}\textstylefstandard{.}
\end{styleEntryParagraph}

\begin{styleDoublecolumnSection}
\textbf{ɓ}\textbf{a}\textbf{vb}\textbf{aw}\textstylefstandard{  }\textstylefstandard{\textit{ID.}}\textstylePartofspeech{ sound or }\textstyleDefinitionn{idea of men running}
\end{styleDoublecolumnSection}

\begin{styleEntryParagraph}
\textstyleLexeme{ɓay}\textstylefstandard{   }\textstylePartofspeech{v.} \textstyleDefinitionn{hit}\textstylefstandard{.}
\end{styleEntryParagraph}

\begin{styleEntryParagraph}
\textstyleLexeme{ɓelen}\textstylefstandard{   }\textstylePartofspeech{v.} \textstyleDefinitionn{build up to}\textstylefstandard{.}
\end{styleEntryParagraph}

\begin{styleEntryParagraph}
\textstyleLexeme{ɓezlen}\textstylefstandard{   }\textstylePartofspeech{v.} \textstyleDefinitionn{count}\textstylefstandard{.} 
\end{styleEntryParagraph}

\begin{styleEntryParagraph}
\textstyleLexeme{ɓəl}\textstylefstandard{   }\textstylePartofspeech{ID.} \textstyleDefinitionn{some}\textstylefstandard{.}
\end{styleEntryParagraph}

\begin{styleEntryParagraph}
\textstyleLexeme{ɓəra}\textstylefstandard{   }\textstylePartofspeech{n.} \textstyleDefinitionn{granary}\textstylefstandard{.}
\end{styleEntryParagraph}

\begin{styleEntryParagraph}
\textstyleLexeme{ɓərav}\textstylefstandard{   }\textstylePartofspeech{n.} \textstyleDefinitionn{heart, self.}
\end{styleEntryParagraph}

\begin{styleEntryParagraph}
\textstyleLexeme{ɓərketem}\textstylefstandard{ }\textstyleLexeme{ɓərketem}\textstylefstandard{  }\textstylePartofspeech{ID.} \textstyleDefinitionn{idea/sound of race}\textstylefstandard{.}
\end{styleEntryParagraph}

\begin{styleEntryParagraph}
\textstyleLexeme{ɓərzlan}\textstylefstandard{   }\textstylePartofspeech{n.} \textstyleDefinitionn{mountain}\textstylefstandard{.}
\end{styleEntryParagraph}

\begin{styleEntryParagraph}
\textstyleLexeme{ɓərzlay}\textstylefstandard{   }\textstylePartofspeech{v.} \textstyleDefinitionn{throw a fit}\textstylefstandard{.}
\end{styleEntryParagraph}

\begin{styleEntryParagraph}
\textstyleLexeme{ɓəslay}\textstylefstandard{   }\textstylePartofspeech{v.} \textstyleDefinitionn{cough}\textstylefstandard{.}
\end{styleEntryParagraph}

\begin{styleEntryParagraph}
\textstyleLexeme{ɓoray}\textstylefstandard{   }\textstylePartofspeech{v.} \textstyleDefinitionn{climb}\textstylefstandard{.}
\end{styleEntryParagraph}

\begin{styleEntryParagraph}
\textstyleLexeme{ɓorcay}\textstylefstandard{   }\textstylePartofspeech{v.} \textstyleDefinitionn{first pounding, tear to pieces}\textstylefstandard{.}
\end{styleEntryParagraph}
\end{multicols}
\begin{multicols}{2}
\end{multicols}
\begin{styleLetterParagraph}
\textstyleLetterv{C  {}-  c}
\end{styleLetterParagraph}

\begin{multicols}{2}
\begin{styleEntryParagraph}
\textstyleLexeme{caɓay}\textstylefstandard{   }\textstylePartofspeech{v.} \textstyleDefinitionn{skewer}\textstylefstandard{.}
\end{styleEntryParagraph}

\begin{styleEntryParagraph}
\textstyleLexeme{caɗay}\textstylefstandard{   }\textstylePartofspeech{v.} \textstyleDefinitionn{smooth}
\end{styleEntryParagraph}

\begin{styleEntryParagraph}
\textstyleLexeme{caɗay}\textstylefstandard{   }\textstylePartofspeech{v.} \textstyleDefinitionn{clear}\textstylefstandard{.}
\end{styleEntryParagraph}

\begin{styleEntryParagraph}
\textstyleLexeme{caɗay}\textstylefstandard{   }\textstylePartofspeech{v.} \textstyleDefinitionn{castrate}\textstylefstandard{.}
\end{styleEntryParagraph}

\begin{styleEntryParagraph}
\textstyleLexeme{cafgal}\textstylefstandard{   }\textstylePartofspeech{n.} \textstyleDefinitionn{bucket}\textstylefstandard{.}
\end{styleEntryParagraph}

\begin{styleEntryParagraph}
\textstyleLexeme{cahay}\textstylefstandard{   }\textstylePartofspeech{v.} \textstyleDefinitionn{get water}\textstylefstandard{.}
\end{styleEntryParagraph}

\begin{styleEntryParagraph}
\textstyleLexeme{cahay}\textstylefstandard{   }\textstylePartofspeech{v.} \textstyleDefinitionn{ask}\textstylefstandard{.}
\end{styleEntryParagraph}

\begin{styleEntryParagraph}
\textstyleLexeme{cahay}\textstylefstandard{   }\textstylePartofspeech{v.} \textstyleDefinitionn{scarify}\textstylefstandard{.}
\end{styleEntryParagraph}

\begin{styleEntryParagraph}
\textstyleLexeme{cacapa}\textstylefstandard{   }\textstylePartofspeech{ID.} \textstyleDefinitionn{idea of later on.}
\end{styleEntryParagraph}

\begin{styleEntryParagraph}
\textstyleLexeme{capay}\textstylefstandard{   }\textstylePartofspeech{v.} \textstyleDefinitionn{drape, double}\textstylefstandard{.}
\end{styleEntryParagraph}

\begin{styleEntryParagraph}
\textstyleLexeme{car}\textstylefstandard{   }\textstylePartofspeech{v.} \textstyleDefinitionn{climb}\textstylefstandard{.}
\end{styleEntryParagraph}

\begin{styleEntryParagraph}
\textstyleLexeme{car}\textstylefstandard{   }\textstylePartofspeech{v.} \textstyleDefinitionn{taste good}\textstylefstandard{.}
\end{styleEntryParagraph}

\begin{styleEntryParagraph}
\textstyleLexeme{caray}\textstylefstandard{   }\textstylePartofspeech{v.} \textstyleDefinitionn{tear up}\textstylefstandard{.}
\end{styleEntryParagraph}

\begin{styleEntryParagraph}
\textstyleLexeme{carzlay}\textstylefstandard{   }\textstylePartofspeech{v.} \textstyleDefinitionn{fold legs}\textstylefstandard{.}
\end{styleEntryParagraph}

\begin{styleEntryParagraph}
\textstyleLexeme{caslay}\textstylefstandard{   }\textstylePartofspeech{v.} \textstyleSensenumber{~}\textstyleDefinitionn{pierce}\textstylefstandard{.}
\end{styleEntryParagraph}

\begin{styleEntryParagraph}
\textstyleLexeme{caway}\textstylefstandard{   }\textstylePartofspeech{v.} \textstyleDefinitionn{cut off head}\textstylefstandard{.}
\end{styleEntryParagraph}

\begin{styleEntryParagraph}
\textstyleLexeme{caway}\textstylefstandard{   }\textstylePartofspeech{v.} \textstyleDefinitionn{grow}\textstylefstandard{.}
\end{styleEntryParagraph}

\begin{styleEntryParagraph}
\textstyleLexeme{cazlay}\textstylefstandard{   }\textstylePartofspeech{v.} \textstyleDefinitionn{pierce, cut}\textstylefstandard{.} 
\end{styleEntryParagraph}

\begin{styleEntryParagraph}
\textstyleLexeme{cazlay}\textstylefstandard{   }\textstylePartofspeech{v.} \textstyleDefinitionn{have a headache}\textstylefstandard{.} 
\end{styleEntryParagraph}

\begin{styleEntryParagraph}
\textstyleLexeme{ce}\textstylefstandard{   }\textstylePartofspeech{v.} \textstyleDefinitionn{lack, be insufficient}\textstylefstandard{.}
\end{styleEntryParagraph}

\begin{styleEntryParagraph}
\textstyleLexeme{cece}\textstylefstandard{   }\textstylePartofspeech{n.} \textstyleDefinitionn{all}\textstylefstandard{.}
\end{styleEntryParagraph}

\begin{styleEntryParagraph}
\textstyleLexeme{cece}\textstylefstandard{   }\textstylePartofspeech{n.} \textstyleDefinitionn{louse}\textstylefstandard{.}
\end{styleEntryParagraph}

\begin{styleEntryParagraph}
\textstyleLexeme{cecekem}\textstylefstandard{   }\textstylePartofspeech{n.} \textstyleDefinitionn{first}\textstylefstandard{.}
\end{styleEntryParagraph}

\begin{styleEntryParagraph}
\textstyleLexeme{cecew}\textstylefstandard{   }\textstylePartofspeech{n.} \textstyleDefinitionn{friend}\textstylefstandard{.}
\end{styleEntryParagraph}

\begin{styleEntryParagraph}
\textstyleLexeme{cecewk}\textstylefstandard{   }\textstylePartofspeech{n.} \textstyleSensenumber{~}\textstyleDefinitionn{flute}\textstylefstandard{.}
\end{styleEntryParagraph}

\begin{styleEntryParagraph}
\textstyleLexeme{cefe}\textstylefstandard{   }\textstylePartofspeech{v.} \textstyleDefinitionn{betray}\textstylefstandard{.}
\end{styleEntryParagraph}

\begin{styleEntryParagraph}
\textstyleLexeme{celelew}\textstylefstandard{   }\textstylePartofspeech{n.} \textstyleDefinitionn{chain}\textstylefstandard{.}
\end{styleEntryParagraph}

\begin{styleEntryParagraph}
\textstyleLexeme{cen}\textstylefstandard{   }\textstylePartofspeech{v.} \textstyleDefinitionn{hear, understand}\textstylefstandard{.}
\end{styleEntryParagraph}

\begin{styleEntryParagraph}
\textstyleLexeme{cew}\textstylefstandard{   }\textstylePartofspeech{num.} \textstyleDefinitionn{two}\textstylefstandard{.}
\end{styleEntryParagraph}

\begin{styleEntryParagraph}
\textstyleLexeme{cezlere}\textstylefstandard{   }\textstylePartofspeech{n.} \textstyleDefinitionn{disobedience}\textstylefstandard{.}
\end{styleEntryParagraph}

\begin{styleEntryParagraph}
\textstyleLexeme{cə}\textstyleLexeme{ɓ}\textstyleLexeme{ay}\textstylefstandard{   }\textstylePartofspeech{v.} \textstyleDefinitionn{overwhelm}\textstylefstandard{.}
\end{styleEntryParagraph}

\begin{styleEntryParagraph}
\textstyleLexeme{cəcəngehe}\textstylefstandard{   }\textstylePartofspeech{adv.} \textstyleDefinitionn{now}\textstylefstandard{.}
\end{styleEntryParagraph}

\begin{styleEntryParagraph}
\textstyleLexeme{cəɗew}\textstylefstandard{   }\textstylePartofspeech{n.} \textstyleDefinitionn{smallness}\textstylefstandard{.}
\end{styleEntryParagraph}

\begin{styleEntryParagraph}
\textstyleLexeme{cəɗoy}\textstylefstandard{   }\textstylePartofspeech{n.} \textstyleDefinitionn{trick}\textstylefstandard{.}
\end{styleEntryParagraph}

\begin{styleEntryParagraph}
\textstyleLexeme{cəɗokay}\textstylefstandard{   }\textstylePartofspeech{v.} \textstyleDefinitionn{crouch, squat}\textstylefstandard{.}
\end{styleEntryParagraph}

\begin{styleEntryParagraph}
\textstyleLexeme{cəfəɗay}\textstylefstandard{   }\textstylePartofspeech{v.} \textstyleDefinitionn{ask for}\textstylefstandard{.}
\end{styleEntryParagraph}

\begin{styleEntryParagraph}
\textstyleLexeme{cəje}\textstylefstandard{   }\textstylePartofspeech{n.} \textstyleDefinitionn{disease}\textstylefstandard{.}
\end{styleEntryParagraph}

\begin{styleEntryParagraph}
\textstyleLexeme{cəjen}\textstylefstandard{   }\textstylePartofspeech{v.} \textstyleDefinitionn{lose, get lost}\textstylefstandard{.}
\end{styleEntryParagraph}

\begin{styleEntryParagraph}
\textstyleLexeme{cəjen}\textstylefstandard{   }\textstylePartofspeech{n.} \textstyleDefinitionn{mortar}\textstylefstandard{.}
\end{styleEntryParagraph}

\begin{styleEntryParagraph}
\textstyleLexeme{cəkafay}\textstylefstandard{   }\textstylePartofspeech{v.} \textstyleDefinitionn{get up}\textstylefstandard{.}
\end{styleEntryParagraph}

\begin{styleEntryParagraph}
\textstyleLexeme{cəkalay}\textstylefstandard{   }\textstylePartofspeech{v.} \textstyleDefinitionn{assemble, unite}\textstylefstandard{.}
\end{styleEntryParagraph}

\begin{styleEntryParagraph}
\textstyleLexeme{cəke}\textstylefstandard{   }\textstylePartofspeech{v.} \textstyleDefinitionn{stand}\textstylefstandard{.}
\end{styleEntryParagraph}

\begin{styleEntryParagraph}
\textstyleLexeme{cəkele}\textstylefstandard{   }\textstylePartofspeech{n.} \textstyleDefinitionn{price}\textstylefstandard{.}
\end{styleEntryParagraph}

\begin{styleEntryParagraph}
\textstyleLexeme{cəkəzlay}\textstylefstandard{   }\textstylePartofspeech{v.} \textstyleDefinitionn{forget}\textstylefstandard{.}
\end{styleEntryParagraph}

\begin{styleEntryParagraph}
\textstyleLexeme{cəlokoy}\textstylefstandard{   }\textstylePartofspeech{v.} \textstyleDefinitionn{peel}\textstylefstandard{.}
\end{styleEntryParagraph}

\begin{styleEntryParagraph}
\textstyleLexeme{cərr}\textstylefstandard{   }\textstylePartofspeech{ID.} \textit{idea of }\textstyleDefinitionn{guinea fowl running}\textstylefstandard{.}
\end{styleEntryParagraph}

\begin{styleEntryParagraph}
\textstyleLexeme{cəveɗ}\textstylefstandard{   }\textstylePartofspeech{n.} \textstyleDefinitionn{road}\textstylefstandard{.}
\end{styleEntryParagraph}

\begin{styleEntryParagraph}
\textstyleLexeme{cəzlahay}\textstylefstandard{   }\textstylePartofspeech{v.} \textstyleDefinitionn{cut, chop}\textstylefstandard{.}
\end{styleEntryParagraph}

\begin{styleEntryParagraph}
\textbf{cəzlar}\textstylefstandard{   }\textstylePartofspeech{ID.} \textstyleDefinitionn{idea of shining upwards}\textstylefstandard{.}
\end{styleEntryParagraph}

\begin{styleEntryParagraph}
\textstyleLexeme{coco}\textstylefstandard{   }\textstylePartofspeech{ID.} \textstyleDefinitionn{sound/idea of cutting with axe}\textstylefstandard{.}
\end{styleEntryParagraph}

\begin{styleEntryParagraph}
\textstyleLexeme{cokoy}\textstylefstandard{   }\textstylePartofspeech{v.} \textstyleDefinitionn{undress}\textstylefstandard{.}
\end{styleEntryParagraph}

\begin{styleEntryParagraph}
\textstyleLexeme{cokor}\textstylefstandard{   }\textstylePartofspeech{n.} \textstyleDefinitionn{fish net}\textstylefstandard{.}
\end{styleEntryParagraph}
\end{multicols}
\begin{styleLetterParagraph}
\textstyleLetterv{D  {}-  d}
\end{styleLetterParagraph}

\begin{multicols}{2}
\begin{styleEntryParagraph}
\textstyleLexeme{dabay}\textstylefstandard{   }\textstylePartofspeech{v.} \textstyleDefinitionn{follow}\textstylefstandard{.}
\end{styleEntryParagraph}

\begin{styleEntryParagraph}
\textstyleLexeme{daɗ}\textstylefstandard{   }\textstylePartofspeech{v.} \textstyleDefinitionn{fall}\textstylefstandard{.}
\end{styleEntryParagraph}

\begin{styleEntryParagraph}
\textstyleLexeme{dafay}\textstylefstandard{   }\textstylePartofspeech{v.}\textstyleDefinitionn{bump}\textstylefstandard{.}
\end{styleEntryParagraph}

\begin{styleEntryParagraph}
\textstyleLexeme{dal}\textstylefstandard{   }\textstylePartofspeech{v.} \textstyleDefinitionn{surpass}\textstylefstandard{.}
\end{styleEntryParagraph}

\begin{styleEntryParagraph}
\textstyleLexeme{dala}\textstylefstandard{   }\textstylePartofspeech{n.} \textstyleDefinitionn{money}\textstylefstandard{.}
\end{styleEntryParagraph}

\begin{styleEntryParagraph}
\textstyleLexeme{dalay}\textstylefstandard{   }\textstylePartofspeech{n.} \textstyleDefinitionn{girl}\textstylefstandard{.}
\end{styleEntryParagraph}

\begin{styleEntryParagraph}
\textstyleLexeme{damay}\textstylefstandard{   }\textstylePartofspeech{v.} \textstyleDefinitionn{succeed}\textstylefstandard{.}
\end{styleEntryParagraph}

\begin{styleEntryParagraph}
\textstyleLexeme{danday}\textstylefstandard{   }\textstylePartofspeech{n.} \textstyleDefinitionn{intestines}\textstylefstandard{.}
\end{styleEntryParagraph}

\begin{styleEntryParagraph}
\textstyleLexeme{danjəw}\textstylefstandard{   }\textstylePartofspeech{ID.} \textstyleDefinitionn{idea of someone balancing something on head}\textstylefstandard{.}
\end{styleEntryParagraph}

\begin{styleEntryParagraph}
\textstyleLexeme{dar}\textstylefstandard{   }\textstylePartofspeech{v.} \textstyleDefinitionn{fake.}
\end{styleEntryParagraph}

\begin{styleEntryParagraph}
\textstyleLexeme{dar}\textstylefstandard{   }\textstylePartofspeech{v.} \textstyleDefinitionn{withdraw, recoil}\textstylefstandard{.}
\end{styleEntryParagraph}

\begin{styleEntryParagraph}
\textstyleLexeme{dar}\textstylefstandard{   }\textstylePartofspeech{v.} \textstyleDefinitionn{burn.}
\end{styleEntryParagraph}

\begin{styleEntryParagraph}
\textstyleLexeme{daray}\textstylefstandard{   }\textstylePartofspeech{v.} \textstyleDefinitionn{plant, snore}\textstylefstandard{.}
\end{styleEntryParagraph}

\begin{styleEntryParagraph}
\textstyleLexeme{daslay}\textstylefstandard{   }\textstylePartofspeech{v.} \textstyleDefinitionn{castrate, sterilize}\textstylefstandard{.}
\end{styleEntryParagraph}

\begin{styleEntryParagraph}
\textstyleLexeme{dav}\textstylefstandard{   }\textstylePartofspeech{v.} \textstyleSensenumber{~}\textstyleDefinitionn{drop.}
\end{styleEntryParagraph}

\begin{styleEntryParagraph}
\textstyleLexeme{daz}\textstylefstandard{   }\textstylePartofspeech{adv.} \textstyleDefinitionn{one complete year}\textstylefstandard{.}
\end{styleEntryParagraph}

\begin{styleEntryParagraph}
\textstyleLexeme{dazlay}\textstylefstandard{   }\textstylePartofspeech{v. }\textstyleDefinitionn{join, tie}\textstylefstandard{.}
\end{styleEntryParagraph}

\begin{styleEntryParagraph}
\textstyleLexeme{de}\textstylefstandard{   }\textstylePartofspeech{v.} \textit{cook}, \textstyleDefinitionn{prepare}\textstylefstandard{.}
\end{styleEntryParagraph}

\begin{styleEntryParagraph}
\textstyleLexeme{debezem}\textstylefstandard{   }\textstylePartofspeech{n.} \textstyleDefinitionn{jawbone}\textstylefstandard{.}
\end{styleEntryParagraph}

\begin{styleEntryParagraph}
\textstyleLexeme{dede}\textstylefstandard{   }\textstylePartofspeech{n.} \textstyleDefinitionn{grandmother}\textstylefstandard{.}
\end{styleEntryParagraph}

\begin{styleEntryParagraph}
\textstyleLexeme{dedew}\textstylefstandard{   }\textstylePartofspeech{n.} \textstyleDefinitionn{morning.}
\end{styleEntryParagraph}

\begin{styleEntryParagraph}
\textstyleLexeme{dedewe}\textstylefstandard{   }\textstylePartofspeech{n.} \textstyleDefinitionn{egret}\textstylefstandard{.}
\end{styleEntryParagraph}

\begin{styleEntryParagraph}
\textstyleLexeme{dedəlen}\textstylefstandard{   }\textstylePartofspeech{n.} \textstyleDefinitionn{blackness}\textstylefstandard{.}
\end{styleEntryParagraph}

\begin{styleEntryParagraph}
\textstyleLexeme{deftere}\textstylefstandard{   }\textstylePartofspeech{n.} \textstyleSensenumber{~}\textstyleDefinitionn{book}\textstylefstandard{ (from Fulfuldé).}
\end{styleEntryParagraph}

\begin{styleEntryParagraph}
\textstyleLexeme{dergwecik}\textstylefstandard{   }\textstylePartofspeech{ID.} idea of \textstyleDefinitionn{lifting on head.}
\end{styleEntryParagraph}

\begin{styleEntryParagraph}
\textstyleLexeme{dewele}\textstylefstandard{   }\textstylePartofspeech{n.} \textstyleDefinitionn{obligation}\textstylefstandard{.}
\end{styleEntryParagraph}

\begin{styleEntryParagraph}
\textstyleLexeme{dey}\textstylefstandard{   }\textstylePartofspeech{adv.} \textstyleDefinitionn{emphasis}\textstylefstandard{.}
\end{styleEntryParagraph}

\begin{styleEntryParagraph}
\textstyleLexeme{dəbakay}\textstylefstandard{   }\textstylePartofspeech{v.} \textstyleDefinitionn{persuade, relieve}\textstylefstandard{.}
\end{styleEntryParagraph}

\begin{styleEntryParagraph}
\textstyleLexeme{dəbənay}\textstylefstandard{   }\textstylePartofspeech{v.} \textstyleDefinitionn{learn, teach}\textstylefstandard{.}
\end{styleEntryParagraph}

\begin{styleEntryParagraph}
\textstyleLexeme{dəbo}\textstylefstandard{   }\textstylePartofspeech{num.1000}\textstylefstandard{.}
\end{styleEntryParagraph}

\begin{styleEntryParagraph}
\textstyleLexeme{dəɓəsolək}\textstylefstandard{   }\textstylePartofspeech{ID.} \textit{idea of} \textstyleDefinitionn{collapsing, dying.}
\end{styleEntryParagraph}

\begin{styleEntryParagraph}
\textstyleLexeme{dəgolay}\textstylefstandard{   }\textstylePartofspeech{n.} \textstyleDefinitionn{thigh}\textstylefstandard{.}
\end{styleEntryParagraph}

\begin{styleEntryParagraph}
\textstyleLexeme{dəl}\textstylefstandard{   }\textstylePartofspeech{ID.} \textit{idea of} \textstyleDefinitionn{insulting.}
\end{styleEntryParagraph}

\begin{styleEntryParagraph}
\textstyleLexeme{dəlmete}\textstylefstandard{   }\textstylePartofspeech{n.}\textstyleDefinitionn{neighbour}\textstylefstandard{.}
\end{styleEntryParagraph}

\begin{styleEntryParagraph}
\textstyleLexeme{dəlov}\textstylefstandard{   }\textstylePartofspeech{n.} \textstyleDefinitionn{lake}\textstylefstandard{.}
\end{styleEntryParagraph}

\begin{styleEntryParagraph}
\textstyleLexeme{dəndara}\textstylefstandard{   }\textstylePartofspeech{n.} \textstyleDefinitionn{lamp}\textstylefstandard{.}
\end{styleEntryParagraph}

\begin{styleEntryParagraph}
\textstyleLexeme{dəngaɗay}\textstylefstandard{   }\textstylePartofspeech{v.} \textstyleDefinitionn{lean back}\textstylefstandard{.}
\end{styleEntryParagraph}

\begin{styleEntryParagraph}
\textstyleLexeme{dəngo}\textstylefstandard{   }\textstylePartofspeech{n.} \textstyleSensenumber{\textit{neck, voice.}}
\end{styleEntryParagraph}

\begin{styleEntryParagraph}
\textstyleLexeme{dəray}\textstylefstandard{   }\textstylePartofspeech{n.} \textstyleDefinitionn{head}\textstylefstandard{.}
\end{styleEntryParagraph}

\begin{styleEntryParagraph}
\textstyleLexeme{dəreffefe}\textstylefstandard{   }\textstylePartofspeech{ID.} \textstyleDefinitionn{sound/idea of movement}\textstylefstandard{.}
\end{styleEntryParagraph}

\begin{styleEntryParagraph}
\textstyleLexeme{dəren}\textstylefstandard{   }\textstylePartofspeech{adv.} \textstyleDefinitionn{far}\textstylefstandard{.}
\end{styleEntryParagraph}

\begin{styleEntryParagraph}
\textstyleLexeme{dəres}\textstylefstandard{   }\textstylePartofspeech{ID.} \textstyleDefinitionn{idea of many}\textstylefstandard{.}
\end{styleEntryParagraph}

\begin{styleEntryParagraph}
\textstyleLexeme{dərlenge}\textstylefstandard{   }\textstylePartofspeech{n.} \textstyleDefinitionn{hyena}\textstylefstandard{.}
\end{styleEntryParagraph}

\begin{styleEntryParagraph}
\textstyleLexeme{dəwa}\textstylefstandard{   }\textstylePartofspeech{n.} \textstyleDefinitionn{debt}\textstylefstandard{.}
\end{styleEntryParagraph}

\begin{styleEntryParagraph}
\textstyleLexeme{dəwlay}\textstylefstandard{   }\textstylePartofspeech{n.} \textstyleDefinitionn{millet drink}\textstylefstandard{.}
\end{styleEntryParagraph}

\begin{styleEntryParagraph}
\textstyleLexeme{Dəwlek}\textstylefstandard{   }\textstyleSensenumber{1)~}\textstylePartofspeech{n.} \textstyleDefinitionn{Thursday market day in the village of Doulek}\textstylefstandard{.}
\end{styleEntryParagraph}

\begin{styleEntryParagraph}
\textstyleLexeme{dəwnəya}\textstylefstandard{   }\textstylePartofspeech{n.} \textstyleDefinitionn{earth}\textstylefstandard{.}
\end{styleEntryParagraph}

\begin{styleEntryParagraph}
\textstyleLexeme{dəy day}\textstylefstandard{   }\textstylePartofspeech{ID.} \textstyleDefinitionn{approximately}\textstylefstandard{.}
\end{styleEntryParagraph}

\begin{styleEntryParagraph}
\textstyleLexeme{dəya}\textstylefstandard{   }\textstylePartofspeech{v.} \textstyleDefinitionn{take many}\textstylefstandard{.}
\end{styleEntryParagraph}

\begin{styleEntryParagraph}
\textstyleLexeme{dokay}\textstylefstandard{   }\textstylePartofspeech{v.} \textstyleDefinitionn{arrive}\textstylefstandard{.}
\end{styleEntryParagraph}

\begin{styleEntryParagraph}
\textstyleLexeme{dolokoy}\textstylefstandard{   }\textstylePartofspeech{n.} \textstyleDefinitionn{syphilis}\textstylefstandard{.}
\end{styleEntryParagraph}

\begin{styleEntryParagraph}
\textstyleLexeme{dozloy}\textstylefstandard{   }\textstylePartofspeech{v.} \textstyleDefinitionn{intersect, meet}\textstylefstandard{.} 
\end{styleEntryParagraph}
\end{multicols}
\begin{styleLetterParagraph}
\textstyleLetterv{Ɗ -}\textstyleLetterv{ }\textstyleLexeme{ɗ}
\end{styleLetterParagraph}

\begin{multicols}{2}
\begin{styleEntryParagraph}
\textstyleLexeme{ɗaf}\textstylefstandard{   }\textstylePartofspeech{n.} \textstyleDefinitionn{millet porridge, food}\textstylefstandard{.}
\end{styleEntryParagraph}

\begin{styleEntryParagraph}
\textstyleLexeme{ɗak}\textstylefstandard{   }\textstylePartofspeech{v.} \textstyleDefinitionn{plug}\textstylefstandard{.}
\end{styleEntryParagraph}

\begin{styleEntryParagraph}
\textstyleLexeme{ɗakay}\textstylefstandard{   }\textstylePartofspeech{v.} \textstyleDefinitionn{indicate.}
\end{styleEntryParagraph}

\begin{styleEntryParagraph}
\textstyleLexeme{ɗas}\textstylefstandard{   }\textstylePartofspeech{v.} \textstyleDefinitionn{be heavy/honourable}\textstylefstandard{.} 
\end{styleEntryParagraph}

\begin{styleEntryParagraph}
\textstyleLexeme{ɗaslay}\textstylefstandard{   }\textstylePartofspeech{v.} \textstyleDefinitionn{multiply}\textstylefstandard{.}
\end{styleEntryParagraph}

\begin{styleEntryParagraph}
\textstyleLexeme{ɗaw}\textstylefstandard{   }\textstylePartofspeech{pn.} \textit{question}\textstyleSensenumber{\textit{ marker}}\textstylefstandard{\textit{.}}
\end{styleEntryParagraph}

\begin{styleEntryParagraph}
\textstyleLexeme{ɗaz ɗaz}\textstylefstandard{   }\textstylePartofspeech{n.} \textstyleDefinitionn{redness.}
\end{styleEntryParagraph}

\begin{styleEntryParagraph}
\textstyleLexeme{ɗazl}\textstylefstandard{   }\textstylePartofspeech{v.} \textstyleDefinitionn{spread for building}\textstylefstandard{.}
\end{styleEntryParagraph}

\begin{styleEntryParagraph}
\textstyleLexeme{ɗe}\textstylefstandard{   }\textstylePartofspeech{v.} \textstyleDefinitionn{flourish, soak in order to soften}\textstylefstandard{.}
\end{styleEntryParagraph}

\begin{styleEntryParagraph}
\textstyleLexeme{ɗeɗen}\textstylefstandard{   }\textstylePartofspeech{n.} \textstyleDefinitionn{truth}\textstylefstandard{.}
\end{styleEntryParagraph}

\begin{styleEntryParagraph}
\textstyleLexeme{ɗeɗew}\textstylefstandard{   }\textstylePartofspeech{n.}\textstyleSensenumber{~}\textstyleDefinitionn{pot}\textstylefstandard{.}
\end{styleEntryParagraph}

\begin{styleEntryParagraph}
\textstyleLexeme{ɗeləywel}\textstylefstandard{ }\textstylePartofspeech{n.paper}\textstylefstandard{.}
\end{styleEntryParagraph}

\begin{styleEntryParagraph}
\textstyleLexeme{ɗen}\textstylefstandard{   }\textstylePartofspeech{ID.idea of} \textstyleDefinitionn{putting}\textstylefstandard{.}
\end{styleEntryParagraph}

\begin{styleEntryParagraph}
\textstyleLexeme{ɗəgalay}\textstylefstandard{   }\textstylePartofspeech{v.} \textstyleDefinitionn{think}\textstylefstandard{.}
\end{styleEntryParagraph}

\begin{styleEntryParagraph}
\textstyleLexeme{ɗəgocoy}\textstylefstandard{   }\textstylePartofspeech{v.} \textstyleDefinitionn{stalk}\textstylefstandard{.}\textstyleExamplev{ }
\end{styleEntryParagraph}

\begin{styleEntryParagraph}
\textstyleLexeme{ɗəgom}\textstylefstandard{   }\textstylePartofspeech{n.} \textstyleDefinitionn{nape}\textstylefstandard{.}
\end{styleEntryParagraph}

\begin{styleEntryParagraph}
\textstyleLexeme{ɗəma}\textstylefstandard{   }\textstylePartofspeech{n.} \textstyleDefinitionn{time}\textstylefstandard{.}
\end{styleEntryParagraph}

\begin{styleEntryParagraph}
\textstyleLexeme{ɗəw}\textstylefstandard{   }\textstylePartofspeech{adv.} \textstyleDefinitionn{also}\textstylefstandard{.}
\end{styleEntryParagraph}

\begin{styleEntryParagraph}
\textstyleLexeme{ɗəwa}\textstylefstandard{   }\textstylePartofspeech{n.} \textstyleDefinitionn{milk}\textstylefstandard{, }\textstylefstandard{\textit{breast}}\textstylefstandard{.}
\end{styleEntryParagraph}

\begin{styleEntryParagraph}
\textstyleLexeme{ɗəwer}\textstylefstandard{   }\textstylePartofspeech{n.} \textstyleDefinitionn{sleep}\textstylefstandard{.}
\end{styleEntryParagraph}

\begin{styleEntryParagraph}
\textstyleLexeme{ɗəwge}\textstylefstandard{   }\textstylePartofspeech{adv.} \textstyleDefinitionn{actually}\textstylefstandard{.}
\end{styleEntryParagraph}

\begin{styleEntryParagraph}
\textstyleLexeme{ɗocay}\textstylefstandard{   }\textstylePartofspeech{v.} \textstyleDefinitionn{squeeze, juice}\textstylefstandard{.}
\end{styleEntryParagraph}
\end{multicols}
\begin{styleLetterParagraph}
\textstyleLetterv{E  {}-  e}
\end{styleLetterParagraph}

\begin{multicols}{2}
\begin{styleEntryParagraph}
\textstyleLexeme{eɗəyen}\textstylefstandard{   }\textstylePartofspeech{n.} \textstyleDefinitionn{bird}\textstylefstandard{.}
\end{styleEntryParagraph}

\begin{styleEntryParagraph}
\textstyleLexeme{eɗongwereɗ}\textstylefstandard{   }\textstylePartofspeech{n.}\textstyleDefinitionn{type of tree}\textstylefstandard{.}
\end{styleEntryParagraph}

\begin{styleEntryParagraph}
\textstyleLexeme{egəne}\textstylefstandard{   }\textstylePartofspeech{adv.} \textstyleDefinitionn{today}\textstylefstandard{.}
\end{styleEntryParagraph}

\begin{styleEntryParagraph}
\textstyleLexeme{ehe}\textstylefstandard{   }\textstylePartofspeech{adp.} \textstyleDefinitionn{here}\textstylefstandard{.}
\end{styleEntryParagraph}

\begin{styleEntryParagraph}
\textstyleLexeme{ehe}\textstylefstandard{   }\textstylePartofspeech{adv.} \textstyleDefinitionn{no}\textstylefstandard{.}
\end{styleEntryParagraph}

\begin{styleEntryParagraph}
\textstyleLexeme{ehwəɗe}\textstylefstandard{   }\textstylePartofspeech{n.} \textstyleDefinitionn{nail, claw.}
\end{styleEntryParagraph}

\begin{styleEntryParagraph}
\textstyleLexeme{ele}\textstylefstandard{   }\textstylePartofspeech{n.} \textstyleDefinitionn{eye}\textstylefstandard{.} 
\end{styleEntryParagraph}

\begin{styleEntryParagraph}
\textstyleLexeme{ele}\textstylefstandard{   }\textstylePartofspeech{n.} \textstyleDefinitionn{thing}\textstylefstandard{.}
\end{styleEntryParagraph}

\begin{styleEntryParagraph}
\textstyleLexeme{elele}\textstylefstandard{   }\textstylePartofspeech{n.} \textstyleDefinitionn{leaf ; sauce made from edible leaves.}
\end{styleEntryParagraph}

\begin{styleEntryParagraph}
\textstyleLexeme{eleməzləɓe}\textstylefstandard{   }\textstylePartofspeech{n.} \textstyleDefinitionn{termites}\textstylefstandard{.}
\end{styleEntryParagraph}

\begin{styleEntryParagraph}
\textstyleLexeme{eləmene}\textstylefstandard{   }\textstylePartofspeech{n.} \textstyleDefinitionn{treasure}\textstylefstandard{.}
\end{styleEntryParagraph}

\begin{styleEntryParagraph}
\textstyleLexeme{emelek}\textstylefstandard{   }\textstylePartofspeech{n.} \textstyleDefinitionn{bracelet}\textstylefstandard{.}
\end{styleEntryParagraph}

\begin{styleEntryParagraph}
\textstyleLexeme{endeɓ}\textstylefstandard{   }\textstylePartofspeech{n.} \textstyleDefinitionn{brain ; wisdom}\textstylefstandard{.}
\end{styleEntryParagraph}

\begin{styleEntryParagraph}
\textstyleLexeme{enen}\textstylefstandard{   }\textstylePartofspeech{n.} \textstyleDefinitionn{snake}\textstylefstandard{.}
\end{styleEntryParagraph}

\begin{styleEntryParagraph}
\textstyleLexeme{enen}\textstylefstandard{   }\textstylePartofspeech{pn.} \textstyleDefinitionn{another}\textstylefstandard{.}
\end{styleEntryParagraph}

\begin{styleEntryParagraph}
\textstyleLexeme{engeren}\textstylefstandard{   }\textstylePartofspeech{n.} \textstyleDefinitionn{insect}\textstylefstandard{.}
\end{styleEntryParagraph}

\begin{styleEntryParagraph}
\textstyleLexeme{epeley}\textstylefstandard{   }\textstylePartofspeech{pn.} \textstyleDefinitionn{when}\textstylefstandard{.}
\end{styleEntryParagraph}

\begin{styleIndentedParagraph}
\textstyleSubentry{epele pele}\textstylefstandard{ }\textstylefstandard{\textit{ID.}}\textstylefstandard{   }\textstyleDefinitionn{in the future, forever}\textstylefstandard{.}
\end{styleIndentedParagraph}

\begin{styleEntryParagraph}
\textstyleLexeme{ercece}\textstylefstandard{   }\textstylePartofspeech{n.} \textstyleDefinitionn{compassion}\textstylefstandard{.}
\end{styleEntryParagraph}

\begin{styleEntryParagraph}
\textstyleLexeme{erkece}\textstylefstandard{   }\textstylePartofspeech{n.} \textstyleDefinitionn{ostrich}\textstylefstandard{.}
\end{styleEntryParagraph}

\begin{styleEntryParagraph}
\textstyleLexeme{ese}\textstylefstandard{   }\textstylePartofspeech{adv.} \textstyleDefinitionn{again}\textstylefstandard{.}
\end{styleEntryParagraph}

\begin{styleEntryParagraph}
\textstyleLexeme{esew}\textstylefstandard{   }\textstylePartofspeech{n.} \textstyleDefinitionn{laziness}\textstylefstandard{.}
\end{styleEntryParagraph}

\begin{styleEntryParagraph}
\textstyleLexeme{esəmey}\textstylefstandard{   }\textstylePartofspeech{adv.} \textstyleDefinitionn{not so ?}\textstylefstandard{.}
\end{styleEntryParagraph}

\begin{styleEntryParagraph}
\textstyleLexeme{eslesleɓ}\textstylefstandard{   }\textstylePartofspeech{n.} \textstyleDefinitionn{saliva}\textstylefstandard{.}
\end{styleEntryParagraph}

\begin{styleEntryParagraph}
\textstyleLexeme{eslesleɗ}\textstylefstandard{   }\textstylePartofspeech{n.} \textstyleDefinitionn{egg}\textstylefstandard{.}
\end{styleEntryParagraph}

\begin{styleEntryParagraph}
\textstyleLexeme{ete}\textstylefstandard{   }\textstylePartofspeech{adv.} \textstyleDefinitionn{also}\textstylefstandard{.}
\end{styleEntryParagraph}

\begin{styleEntryParagraph}
\textstyleLexeme{eteme}\textstylefstandard{   }\textstylePartofspeech{n.} \textstyleDefinitionn{onion}\textstylefstandard{.}
\end{styleEntryParagraph}

\begin{styleEntryParagraph}
\textstyleLexeme{etew}\textstylefstandard{   }\textstylePartofspeech{n.} \textstyleDefinitionn{hawk}\textstylefstandard{.}
\end{styleEntryParagraph}

\begin{styleEntryParagraph}
\textstyleLexeme{etey}\textstylefstandard{   }\textstylePartofspeech{adv.} \textstyleDefinitionn{polite demand}
\end{styleEntryParagraph}

\begin{styleEntryParagraph}
\textstyleLexeme{eyeweɗ}\textstylefstandard{   }\textstylePartofspeech{n.} \textstyleDefinitionn{whip}\textstylefstandard{.}
\end{styleEntryParagraph}

\begin{styleEntryParagraph}
\textstyleLexeme{eyewk}\textstylefstandard{   }\textstylePartofspeech{n.} \textstyleDefinitionn{ground nut.}
\end{styleEntryParagraph}

\begin{styleEntryParagraph}
\textstyleLexeme{ezeweɗ}\textstylefstandard{   }\textstylePartofspeech{n.} \textstyleDefinitionn{cord}\textstylefstandard{.}
\end{styleEntryParagraph}

\begin{styleEntryParagraph}
\textstyleLexeme{ezewk}\textstylefstandard{   }\textstylePartofspeech{n.} \textstyleDefinitionn{misfortune}\textstylefstandard{.}
\end{styleEntryParagraph}

\begin{styleEntryParagraph}
\textstyleLexeme{ezlegweme}\textstylefstandard{   }\textstylePartofspeech{n.} \textstyleDefinitionn{camel}\textstylefstandard{.}
\end{styleEntryParagraph}

\begin{styleEntryParagraph}
\textstyleLexeme{ezlere}\textstylefstandard{   }\textstylePartofspeech{n.} \textstyleDefinitionn{spear.}
\end{styleEntryParagraph}
\end{multicols}
\begin{styleLetterParagraph}
\textstyleLexeme{ə}\textstyleLetterv{\textmd{  {}-  }}\textstyleLexeme{ə }
\end{styleLetterParagraph}

\begin{multicols}{2}
\begin{styleEntryParagraph}
\textstyleLexeme{əwɗe}\textstylefstandard{   }\textstylePartofspeech{adv.} \textstyleDefinitionn{first.}
\end{styleEntryParagraph}

\begin{styleEntryParagraph}
\textstyleLexeme{əwfaɗ}\textstylefstandard{   }\textstylePartofspeech{num.} \textstyleDefinitionn{four.}
\end{styleEntryParagraph}

\begin{styleEntryParagraph}
\textstyleLexeme{əwla}\textstylefstandard{   }\textstylePartofspeech{nclitic.} \textstyleDefinitionn{1S possessive.}
\end{styleEntryParagraph}
\end{multicols}
\begin{styleLetterParagraph}
\textstyleLetterv{F  {}-  f}
\end{styleLetterParagraph}

\begin{multicols}{2}
\begin{styleEntryParagraph}
\textstyleLexeme{fabay}\textstylefstandard{   }\textstylePartofspeech{NEG.} \textstyleDefinitionn{not yet}\textstylefstandard{.}
\end{styleEntryParagraph}

\begin{styleEntryParagraph}
\textstyleLexeme{faɗ}\textstylefstandard{   }\textstylePartofspeech{v.} \textstyleDefinitionn{put, set down}\textstylefstandard{.}
\end{styleEntryParagraph}

\begin{styleEntryParagraph}
\textstyleLexeme{faɗay}\textstylefstandard{   }\textstylePartofspeech{v.} \textstyleDefinitionn{fold}\textstylefstandard{.}
\end{styleEntryParagraph}

\begin{styleEntryParagraph}
\textstyleLexeme{fakay}\textstylefstandard{   }\textstylePartofspeech{v.} \textit{u}\textstyleDefinitionn{proot a tree}\textstylefstandard{.}
\end{styleEntryParagraph}

\begin{styleEntryParagraph}
\textstyleLexeme{fan}\textstylefstandard{   }\textstylePartofspeech{adv.} \textstyleDefinitionn{already}\textstylefstandard{.}
\end{styleEntryParagraph}

\begin{styleEntryParagraph}
\textstyleLexeme{far}\textstylefstandard{   }\textstylePartofspeech{v.} \textstyleDefinitionn{scratch}\textstylefstandard{.}
\end{styleEntryParagraph}

\begin{styleEntryParagraph}
\textstyleLexeme{fat}\textstylefstandard{   }\textstylePartofspeech{n.} \textstyleDefinitionn{sun, daytime}\textstylefstandard{.}
\end{styleEntryParagraph}

\begin{styleEntryParagraph}
\textstyleLexeme{fat}\textstylefstandard{   }\textstylePartofspeech{v.} \textstyleDefinitionn{germinate}\textstylefstandard{.}
\end{styleEntryParagraph}

\begin{styleEntryParagraph}
\textstyleLexeme{fatay}\textstylefstandard{   }\textstylePartofspeech{v.} \textstyleDefinitionn{descend}\textstylefstandard{.}
\end{styleEntryParagraph}

\begin{styleEntryParagraph}
\textstyleLexeme{fe}\textstylefstandard{   }\textstylePartofspeech{v.} \textstyleDefinitionn{play a wind instrument}\textstylefstandard{.}
\end{styleEntryParagraph}

\begin{styleEntryParagraph}
\textstyleLexeme{fefen}\textstylefstandard{   }\textstylePartofspeech{n.} \textstyleDefinitionn{millet leaf}\textstylefstandard{.}
\end{styleEntryParagraph}

\begin{styleEntryParagraph}
\textstyleLexeme{fenge}\textstylefstandard{   }\textstylePartofspeech{n.} \textstyleDefinitionn{termite mound}\textstylefstandard{.}
\end{styleEntryParagraph}

\begin{styleEntryParagraph}
\textstyleLexeme{fətaɗay}\textstylefstandard{   }\textstylePartofspeech{v.} \textstyleDefinitionn{sharpen to a point}\textstylefstandard{.}
\end{styleEntryParagraph}

\begin{styleEntryParagraph}
\textstyleLexeme{fəhh}\textstylefstandard{   }\textstylePartofspeech{ID.}\textstyleDefinitionn{sound/idea of truck engine humming}\textstylefstandard{.} 
\end{styleEntryParagraph}

\begin{styleEntryParagraph}
\textstyleLexeme{Fətak}\textstylefstandard{   }\textstylePartofspeech{n.} \textstyleDefinitionn{name of a village and a clan of Moloko}\textstylefstandard{.} 
\end{styleEntryParagraph}

\begin{styleEntryParagraph}
\textstyleLexeme{fofofo}\textstylefstandard{   }\textstylePartofspeech{ID.}\textstyleDefinitionn{sound/idea of a snake slithering}\textstylefstandard{.}
\end{styleEntryParagraph}

\begin{styleEntryParagraph}
\textstyleLexeme{fokoy}\textstylefstandard{   }\textstylePartofspeech{v.} \textstyleDefinitionn{whistle}\textstylefstandard{.}
\end{styleEntryParagraph}

\begin{styleEntryParagraph}
\textstyleLexeme{fowwa}\textstylefstandard{   }\textstylePartofspeech{ideo  }\textstyleDefinitionn{sound/idea of wind blowing}\textstylefstandard{.} 
\end{styleEntryParagraph}
\end{multicols}
\begin{styleLetterParagraph}
\textstyleLetterv{G  {}-  g}
\end{styleLetterParagraph}

\begin{multicols}{2}
\begin{styleEntryParagraph}
\textstyleLexeme{ga}\textstylefstandard{   }\textstylePartofspeech{nclitic.} \textstyleDefinitionn{adjectiviser}\textstylefstandard{.}
\end{styleEntryParagraph}

\begin{styleEntryParagraph}
\textstyleLexeme{gabay}\textstylefstandard{   }\textstylePartofspeech{v.} \textstyleDefinitionn{constipate}\textstylefstandard{.}
\end{styleEntryParagraph}

\begin{styleEntryParagraph}
\textstyleLexeme{gala}\textstylefstandard{   }\textstylePartofspeech{n.} \textstyleDefinitionn{yard}\textstylefstandard{.}
\end{styleEntryParagraph}

\begin{styleEntryParagraph}
\textstyleLexeme{galay}\textstylefstandard{   }\textstylePartofspeech{v.} \textstyleDefinitionn{herd, chase}\textstylefstandard{.}
\end{styleEntryParagraph}

\begin{styleEntryParagraph}
\textstyleLexeme{gam}\textstylefstandard{   }\textstylePartofspeech{quant.} \textstyleDefinitionn{much}\textstylefstandard{.} 
\end{styleEntryParagraph}

\begin{styleEntryParagraph}
\textstyleLexeme{gar}\textstylefstandard{   }\textstylePartofspeech{n.} \textstyleDefinitionn{difficulty}
\end{styleEntryParagraph}

\begin{styleEntryParagraph}
\textstyleLexeme{gar}\textstylefstandard{   }\textstylePartofspeech{v.} \textstyleDefinitionn{grow}\textstylefstandard{.}
\end{styleEntryParagraph}

\begin{styleEntryParagraph}
\textstyleLexeme{garay}\textstylefstandard{   }\textstylePartofspeech{v.} \textstyleDefinitionn{command, frighten}\textstylefstandard{.} 
\end{styleEntryParagraph}

\begin{styleEntryParagraph}
\textstyleLexeme{gas}\textstylefstandard{   }\textstylePartofspeech{v.} \textstyleDefinitionn{catch, accept}\textstyleSensenumber{.}
\end{styleEntryParagraph}

\begin{styleEntryParagraph}
\textstyleLexeme{gazay}\textstylefstandard{   }\textstylePartofspeech{v.} \textstyleDefinitionn{nod}\textstylefstandard{.}
\end{styleEntryParagraph}

\begin{styleEntryParagraph}
\textstyleLexeme{ge}\textstylefstandard{   }\textstylePartofspeech{v.} \textstyleDefinitionn{do}\textstylefstandard{.} 
\end{styleEntryParagraph}

\begin{styleEntryParagraph}
\textstyleLexeme{gembəre}\textstylefstandard{   }\textstylePartofspeech{n.} \textstyleDefinitionn{bride price}\textstylefstandard{.}
\end{styleEntryParagraph}

\begin{styleEntryParagraph}
\textstyleLexeme{gəɓar}\textstylefstandard{   }\textstylePartofspeech{n.} \textstyleDefinitionn{fear}\textstylefstandard{.}
\end{styleEntryParagraph}

\begin{styleEntryParagraph}
\textstyleLexeme{gəɓokoy}\textstylefstandard{   }\textstylePartofspeech{v.} \textstyleDefinitionn{bend over}\textstylefstandard{.}
\end{styleEntryParagraph}

\begin{styleEntryParagraph}
\textstyleLexeme{gədan}\textstylefstandard{   }\textstylePartofspeech{n.} \textstyleDefinitionn{strength}\textstylefstandard{.}
\end{styleEntryParagraph}

\begin{styleEntryParagraph}
\textstyleLexeme{gədəgalay}\textstylefstandard{   }\textstylePartofspeech{v.} \textstyleDefinitionn{get fat}\textstylefstandard{.}
\end{styleEntryParagraph}

\begin{styleEntryParagraph}
\textstyleLexeme{gədəgar}\textstylefstandard{   }\textstylePartofspeech{v.} \textstyleDefinitionn{granulate, weave}\textstylefstandard{.}
\end{styleEntryParagraph}

\begin{styleEntryParagraph}
\textstyleLexeme{gədo gədo}\textstylefstandard{ }\textstyleLexeme{gədo}\textstylefstandard{  }\textstylePartofspeech{ID.}\textstyleDefinitionn{sight/idea of man running}\textstylefstandard{.}
\end{styleEntryParagraph}

\begin{styleEntryParagraph}
\textstyleLexeme{gədok}\textstylefstandard{   }\textstylePartofspeech{ID.} \textstyleDefinitionn{make beer.}
\end{styleEntryParagraph}

\begin{styleEntryParagraph}
\textstyleLexeme{gəɗəgəzl}\textstylefstandard{   }\textstylePartofspeech{ID.}\textstyleDefinitionn{idea of setting down something heavy.}
\end{styleEntryParagraph}

\begin{styleEntryParagraph}
\textstyleLexeme{gəgəmay}\textstylefstandard{   }\textstylePartofspeech{n.} \textstyleDefinitionn{cotton}\textstylefstandard{.}
\end{styleEntryParagraph}

\begin{styleEntryParagraph}
\textstyleLexeme{gəgoro}\textstylefstandard{   }\textstylePartofspeech{n.} \textstyleDefinitionn{ram}\textstylefstandard{.}
\end{styleEntryParagraph}

\begin{styleEntryParagraph}
\textstyleLexeme{gəjah}\textstylefstandard{   }\textstylePartofspeech{v.} \textstyleDefinitionn{pull}\textstylefstandard{.}
\end{styleEntryParagraph}

\begin{styleEntryParagraph}
\textstyleLexeme{gəjakay}\textstylefstandard{   }\textstylePartofspeech{v.}\textstyleDefinitionn{hang}\textstylefstandard{.}
\end{styleEntryParagraph}

\begin{styleEntryParagraph}
\textstyleLexeme{gəjar}\textstylefstandard{   }\textstylePartofspeech{v.} \textstyleDefinitionn{take or steal by force}\textstylefstandard{.}
\end{styleEntryParagraph}

\begin{styleEntryParagraph}
\textstyleLexeme{gəlan}\textstylefstandard{   }\textstylePartofspeech{n.} \textstyleDefinitionn{kitchen}\textstylefstandard{.}
\end{styleEntryParagraph}

\begin{styleEntryParagraph}
\textstyleLexeme{gəlan}\textstylefstandard{   }\textstylePartofspeech{n.} \textstyleDefinitionn{threshing floor}\textstylefstandard{.}
\end{styleEntryParagraph}

\begin{styleEntryParagraph}
\textstyleLexeme{gəlo}\textstylefstandard{ }\textstylefstandard{\textit{n}}\textstylePartofspeech{.}\textstyleSensenumber{~}\textstyleDefinitionn{left}\textstylefstandard{.} 
\end{styleEntryParagraph}

\begin{styleEntryParagraph}
\textstyleLexeme{gəlo}\textstylefstandard{   }\textstylePartofspeech{n.}\textstyleSensenumber{~}\textstyleDefinitionn{firstborn son}\textstylefstandard{.} 
\end{styleEntryParagraph}

\begin{styleEntryParagraph}
\textstyleLexeme{gəmsodo}\textstylefstandard{   }\textstylePartofspeech{n.} \textstyleDefinitionn{maternal uncle}\textstylefstandard{.}
\end{styleEntryParagraph}

\begin{styleEntryParagraph}
\textstyleLexeme{gənaw}\textstylefstandard{   }\textstylePartofspeech{n.} \textstyleDefinitionn{animal}\textstylefstandard{.}
\end{styleEntryParagraph}

\begin{styleEntryParagraph}
\textstyleLexeme{gəraw}\textstylefstandard{   }\textstylePartofspeech{ID.} \textstyleDefinitionn{idea of cutting something through the middle}\textstylefstandard{.}
\end{styleEntryParagraph}

\begin{styleEntryParagraph}
\textstyleLexeme{gərəp gərəp }\textstylefstandard{  }\textstylePartofspeech{ID.} \textstyleDefinitionn{sight/idea of something heavy running (cows)}\textstylefstandard{.}
\end{styleEntryParagraph}

\begin{styleEntryParagraph}
\textstyleLexeme{gəsan}\textstylefstandard{   }\textstylePartofspeech{n.} \textstyleDefinitionn{bull}\textstylefstandard{.}
\end{styleEntryParagraph}

\begin{styleEntryParagraph}
\textstyleLexeme{gəvah}\textstylefstandard{   }\textstylePartofspeech{n.} \textit{cultivated} \textstyleDefinitionn{field}\textstylefstandard{.}
\end{styleEntryParagraph}

\begin{styleEntryParagraph}
\textstyleLexeme{gəver}\textstylefstandard{   }\textstylePartofspeech{n.} \textstyleDefinitionn{liver}\textstylefstandard{.}
\end{styleEntryParagraph}

\begin{styleEntryParagraph}
\textstyleLexeme{gəvoy}\textstylefstandard{   }\textstylePartofspeech{v.} \textstyleDefinitionn{rot meat to flavour food}\textstylefstandard{.}
\end{styleEntryParagraph}

\begin{styleEntryParagraph}
\textstyleLexeme{gəzamay}\textstylefstandard{   }\textstylePartofspeech{v.}\textstyleDefinitionn{lose weight}\textstylefstandard{.}
\end{styleEntryParagraph}

\begin{styleEntryParagraph}
\textstyleLexeme{gəzo}\textstylefstandard{   }\textstylePartofspeech{n.} \textstyleDefinitionn{hip}\textstylefstandard{.}
\end{styleEntryParagraph}

\begin{styleEntryParagraph}
\textstyleLexeme{gəzom}\textstylefstandard{   }\textstylePartofspeech{n.} \textit{millet} \textstyleDefinitionn{beer. }\textstyleflabel{ }
\end{styleEntryParagraph}

\begin{styleEntryParagraph}
\textstyleLexeme{gobay}\textstylefstandard{   }\textstylePartofspeech{n.} \textstyleDefinitionn{a lot}\textstylefstandard{.}
\end{styleEntryParagraph}

\begin{styleEntryParagraph}
\textstyleLexeme{gocoy}\textstylefstandard{   }\textstylePartofspeech{v.} \textstyleDefinitionn{throw, sow}\textstylefstandard{.}
\end{styleEntryParagraph}

\begin{styleEntryParagraph}
\textstyleLexeme{gogolvon}\textstylefstandard{   }\textstylePartofspeech{n.} \textstyleDefinitionn{snake}\textstylefstandard{.}
\end{styleEntryParagraph}

\begin{styleEntryParagraph}
\textstyleLexeme{gogor}\textstylefstandard{   }\textstylePartofspeech{n.} \textstyleDefinitionn{elder}\textstylefstandard{.}
\end{styleEntryParagraph}

\begin{styleEntryParagraph}
\textstyleLexeme{gogwez}\textstylefstandard{   }\textstylePartofspeech{n.} \textstyleDefinitionn{redness}\textstylefstandard{.}
\end{styleEntryParagraph}

\begin{styleEntryParagraph}
\textstyleLexeme{gohoy}\textstylefstandard{   }\textstylePartofspeech{v.} \textstyleDefinitionn{brush}\textstylefstandard{.}
\end{styleEntryParagraph}

\begin{styleEntryParagraph}
\textstyleLexeme{goloy}\textstylefstandard{   }\textstylePartofspeech{v.} \textstyleDefinitionn{silence}\textstylefstandard{.}
\end{styleEntryParagraph}

\begin{styleEntryParagraph}
\textstyleLexeme{golo}\textstylefstandard{   }\textstylePartofspeech{n.voc.} \textstyleDefinitionn{dear}\textstylefstandard{.}
\end{styleEntryParagraph}

\begin{styleEntryParagraph}
\textstyleLexeme{goroy}\textstylefstandard{   }\textstylePartofspeech{v.}\textstyleDefinitionn{strip leaves from stalk}\textstylefstandard{.}
\end{styleEntryParagraph}

\begin{styleEntryParagraph}
\textstyleLexeme{gorcoy}\textstylefstandard{   }\textstylePartofspeech{v.}\textstyleDefinitionn{sniff, slurp}\textstylefstandard{.}
\end{styleEntryParagraph}

\begin{styleEntryParagraph}
\textstyleLexeme{goro}\textstylefstandard{   }\textstylePartofspeech{n.} \textstyleDefinitionn{kola nut}\textstylefstandard{.}
\end{styleEntryParagraph}

\begin{styleEntryParagraph}
\textstyleLexeme{gwədar}\textstylefstandard{   }\textstylePartofspeech{n.} \textstyleDefinitionn{youngest child}\textstylefstandard{.}
\end{styleEntryParagraph}

\begin{styleEntryParagraph}
\textstyleLexeme{gwədeɗek}\textstylefstandard{   }\textstylePartofspeech{n.} \textstyleDefinitionn{frog}\textstylefstandard{.}
\end{styleEntryParagraph}

\begin{styleEntryParagraph}
\textstyleLexeme{gwəla}\textstylefstandard{   }\textstylePartofspeech{n.} \textit{son}
\end{styleEntryParagraph}

\begin{styleEntryParagraph}
\textstyleLexeme{gwəlek}\textstylefstandard{   }\textstylePartofspeech{n.} \textstyleDefinitionn{small axe}\textstylefstandard{.}
\end{styleEntryParagraph}

\begin{styleEntryParagraph}
\textstyleLexeme{gwəzoy}\textstylefstandard{   }\textstylePartofspeech{v.} \textstyleDefinitionn{tan, treat animal skin}\textstylefstandard{.} 
\end{styleEntryParagraph}
\end{multicols}
\begin{styleLetterParagraph}
\textstyleLetterv{H  {}-  h}
\end{styleLetterParagraph}

\begin{multicols}{2}
\begin{styleEntryParagraph}
\textstyleLexeme{ha}\textstylefstandard{  }\textstylePartofspeech{adp.} \textstyleDefinitionn{until}\textstylefstandard{.}
\end{styleEntryParagraph}

\begin{styleEntryParagraph}
\textstyleLexeme{haɓ}\textstylefstandard{   }\textstylePartofspeech{v.} \textstyleDefinitionn{break}\textstylefstandard{.}
\end{styleEntryParagraph}

\begin{styleEntryParagraph}
\textstyleLexeme{haɓay}\textstylefstandard{   }\textstylePartofspeech{v.} \textstyleDefinitionn{dance}\textstylefstandard{.}
\end{styleEntryParagraph}

\begin{styleEntryParagraph}
\textstyleLexeme{hadak}\textstylefstandard{   }\textstylePartofspeech{n.} \textstyleDefinitionn{thorn}\textstylefstandard{.}
\end{styleEntryParagraph}

\begin{styleEntryParagraph}
\textstyleLexeme{haɗa}\textstylefstandard{   }\textstylePartofspeech{quant.} \textit{enough, }\textstyleDefinitionn{many}\textstylefstandard{.}
\end{styleEntryParagraph}

\begin{styleEntryParagraph}
\textstyleLexeme{hahar}\textstylefstandard{   }\textstylePartofspeech{n.} \textstyleDefinitionn{straw granary}\textstylefstandard{.}
\end{styleEntryParagraph}

\begin{styleEntryParagraph}
\textstyleLexeme{hahar}\textstylefstandard{   }\textstylePartofspeech{n.} \textstyleDefinitionn{bean}\textstylefstandard{.}
\end{styleEntryParagraph}

\begin{styleEntryParagraph}
\textstyleLexeme{hajan}\textstylefstandard{   }\textstylePartofspeech{adv.} \textstyleDefinitionn{tomorrow}\textstylefstandard{.} 
\end{styleEntryParagraph}

\begin{styleEntryParagraph}
\textstyleLexeme{hakay}\textstylefstandard{   }\textstylePartofspeech{v.}\textstyleDefinitionn{push}\textstylefstandard{.}
\end{styleEntryParagraph}

\begin{styleEntryParagraph}
\textstyleLexeme{halay}\textstylefstandard{   }\textstylePartofspeech{v.} \textstyleDefinitionn{gather}\textstylefstandard{.}
\end{styleEntryParagraph}

\begin{styleEntryParagraph}
\textstyleLexeme{hamay}\textstylefstandard{   }\textstylePartofspeech{v.} \textstyleDefinitionn{pay a debt}\textstylefstandard{.}
\end{styleEntryParagraph}

\begin{styleEntryParagraph}
\textstyleLexeme{hambar}\textstylefstandard{   }\textstylePartofspeech{n.} \textstyleDefinitionn{skin}\textstylefstandard{.}
\end{styleEntryParagraph}

\begin{styleEntryParagraph}
\textstyleLexeme{har}\textstylefstandard{   }\textstylePartofspeech{n.} \textstyleSensenumber{~}\textstyleDefinitionn{body}\textstylefstandard{.}
\end{styleEntryParagraph}

\begin{styleEntryParagraph}
\textstyleLexeme{har}\textstylefstandard{   }\textstylePartofspeech{v.} \textstyleDefinitionn{construct}\textstylefstandard{.}
\end{styleEntryParagraph}

\begin{styleEntryParagraph}
\textstyleLexeme{har}\textstylefstandard{   }\textstylePartofspeech{v.} \textstyleSensenumber{~}\textstyleDefinitionn{collect}\textstylefstandard{.}
\end{styleEntryParagraph}

\begin{styleEntryParagraph}
\textstyleLexeme{hara}\textstylefstandard{   }\textstylePartofspeech{n.} \textstyleDefinitionn{iron, metal}\textstylefstandard{.}
\end{styleEntryParagraph}

\begin{styleEntryParagraph}
\textstyleLexeme{hara}\textstylefstandard{   }\textstylePartofspeech{n.} \textstyleDefinitionn{hour}\textstylefstandard{.}
\end{styleEntryParagraph}

\begin{styleEntryParagraph}
\textstyleLexeme{harac}\textstylefstandard{   }\textstylePartofspeech{n.} \textstyleDefinitionn{scorpion}\textstylefstandard{.}
\end{styleEntryParagraph}

\begin{styleEntryParagraph}
\textstyleLexeme{hasl}\textstylefstandard{   }\textstylePartofspeech{v.} \textstyleDefinitionn{swell}\textstylefstandard{.}
\end{styleEntryParagraph}

\begin{styleEntryParagraph}
\textstyleLexeme{hay}\textstylefstandard{   }\textstylePartofspeech{n.} \textstyleDefinitionn{millet}\textstylefstandard{.}
\end{styleEntryParagraph}

\begin{styleEntryParagraph}
\textstyleLexeme{hay}\textstylefstandard{   }\textstylePartofspeech{n.} \textit{h}\textstyleDefinitionn{ouse.}
\end{styleEntryParagraph}

\begin{styleEntryParagraph}
\textstyleLexeme{hay}\textstylefstandard{   }\textstylePartofspeech{v.} \textstyleDefinitionn{greet someone}\textstylefstandard{.}
\end{styleEntryParagraph}

\begin{styleEntryParagraph}
\textstyleLexeme{haya}\textstylefstandard{   }\textstylePartofspeech{v.} \textstyleDefinitionn{grind}\textstylefstandard{.}
\end{styleEntryParagraph}

\begin{styleEntryParagraph}
\textstyleLexeme{azak}\textstylefstandard{   }\textstylePartofspeech{n.} \textstyleDefinitionn{smoke}\textstylefstandard{.}
\end{styleEntryParagraph}

\begin{styleEntryParagraph}
\textstyleLexeme{heɓek heɓek}\textstylefstandard{   }\textstylePartofspeech{ID.} \textstyleDefinitionn{hardly breathing}\textstylefstandard{.}
\end{styleEntryParagraph}

\begin{styleEntryParagraph}
\textstyleLexeme{hehen}\textstylefstandard{   }\textstylePartofspeech{n.} \textstyleDefinitionn{owl}\textstylefstandard{.}
\end{styleEntryParagraph}

\begin{styleEntryParagraph}
\textstyleLexeme{hereɓ}\textstylefstandard{   }\textstylePartofspeech{n.} \textstyleDefinitionn{heat}\textstylefstandard{.}
\end{styleEntryParagraph}

\begin{styleEntryParagraph}
\textstyleLexeme{heyew}\textstylefstandard{   }\textstylePartofspeech{n.} \textstyleDefinitionn{grasshopper}\textstylefstandard{.}
\end{styleEntryParagraph}

\begin{styleEntryParagraph}
\textstyleLexeme{hədo}\textstylefstandard{   }\textstylePartofspeech{n.} \textstyleDefinitionn{wall}\textstylefstandard{.}
\end{styleEntryParagraph}

\begin{styleEntryParagraph}
\textstyleLexeme{həjəgaɗay}\textstylefstandard{   }\textstylePartofspeech{v.} \textstyleDefinitionn{limp}\textstylefstandard{.}
\end{styleEntryParagraph}

\begin{styleEntryParagraph}
\textstyleLexeme{həlan}\textstylefstandard{   }\textstylePartofspeech{n.} \textit{place} \textstyleDefinitionn{behind}\textstylefstandard{.}
\end{styleEntryParagraph}

\begin{styleEntryParagraph}
\textstyleLexeme{həlef}\textstylefstandard{   }\textstylePartofspeech{n.} \textstyleDefinitionn{hoe}\textstylefstandard{.}
\end{styleEntryParagraph}

\begin{styleEntryParagraph}
\textstyleLexeme{həlfe}\textstylefstandard{   }\textstylePartofspeech{n.} \textstyleDefinitionn{seeds}\textstylefstandard{.}
\end{styleEntryParagraph}

\begin{styleEntryParagraph}
\textstyleLexeme{həmaɗ}\textstylefstandard{   }\textstylePartofspeech{n.} \textstyleDefinitionn{wind}\textstylefstandard{.}
\end{styleEntryParagraph}

\begin{styleEntryParagraph}
\textstyleLexeme{həmay}\textstylefstandard{   }\textstylePartofspeech{v.} \textstyleDefinitionn{run}\textstylefstandard{.}
\end{styleEntryParagraph}

\begin{styleEntryParagraph}
\textstyleLexeme{həmbo}\textstylefstandard{   }\textstylePartofspeech{n.} \textstyleDefinitionn{flour}\textstylefstandard{.}
\end{styleEntryParagraph}

\begin{styleEntryParagraph}
\textstyleLexeme{hənder}\textstylefstandard{   }\textstylePartofspeech{n.} \textstyleDefinitionn{nose}\textstylefstandard{.}
\end{styleEntryParagraph}

\begin{styleEntryParagraph}
\textstyleLexeme{həraɗ}\textstylefstandard{   }\textstylePartofspeech{v.} \textstyleDefinitionn{jump, pull out}\textstylefstandard{.}
\end{styleEntryParagraph}

\begin{styleEntryParagraph}
\textstyleLexeme{həraf}\textstylefstandard{   }\textstylePartofspeech{n.}\textstyleSensenumber{~}\textstyleDefinitionn{medicine}\textstylefstandard{.}
\end{styleEntryParagraph}

\begin{styleEntryParagraph}
\textstyleLexeme{hərɓoy}\textstylefstandard{   }\textstylePartofspeech{v.} \textstyleDefinitionn{heat up, dissolve}\textstylefstandard{.} \textstyleflabel{Cf.: }\textstylefvernacular{hereɓ}\textstylefstandard{.}
\end{styleEntryParagraph}

\begin{styleEntryParagraph}
\textstyleLexeme{hərdedem}\textstylefstandard{   }\textstylePartofspeech{n.} \textstyleDefinitionn{knee}\textstylefstandard{.}
\end{styleEntryParagraph}

\begin{styleEntryParagraph}
\textstyleLexeme{hərdesl}\textstylefstandard{   }\textstylePartofspeech{n.} \textstyleDefinitionn{grave}\textstylefstandard{.}
\end{styleEntryParagraph}

\begin{styleEntryParagraph}
\textstyleLexeme{hərəngezl}\textstylefstandard{   }\textstylePartofspeech{n.} \textstyleDefinitionn{joint}\textstylefstandard{.}
\end{styleEntryParagraph}

\begin{styleEntryParagraph}
\textstyleLexeme{hərgov}\textstylefstandard{   }\textstylePartofspeech{n.} \textstyleDefinitionn{baboon}\textstylefstandard{.}
\end{styleEntryParagraph}

\begin{styleEntryParagraph}
\textstyleLexeme{hərkay}\textstylefstandard{   }\textstylePartofspeech{v.} \textstyleDefinitionn{beg}\textstylefstandard{.}
\end{styleEntryParagraph}

\begin{styleEntryParagraph}
\textstyleLexeme{hərmbəlom}\textstylefstandard{   }\textstylePartofspeech{n.} \textstyleDefinitionn{god, sky}\textstylefstandard{.}
\end{styleEntryParagraph}

\begin{styleEntryParagraph}
\textstyleLexeme{hərnek}\textstylefstandard{   }\textstylePartofspeech{n.} \textstyleDefinitionn{tongue}\textstylefstandard{.}
\end{styleEntryParagraph}

\begin{styleEntryParagraph}
\textstyleLexeme{hərnje}\textstylefstandard{   }\textstylePartofspeech{n.} \textit{hate, }\textstyleDefinitionn{quarrel}\textstylefstandard{.}
\end{styleEntryParagraph}

\begin{styleEntryParagraph}
\textstyleLexeme{hərov}\textstylefstandard{   }\textstylePartofspeech{n.} \textstyleDefinitionn{fig tree}\textstylefstandard{.}
\end{styleEntryParagraph}

\begin{styleEntryParagraph}
\textstyleLexeme{hərva}\textstylefstandard{   }\textstylePartofspeech{n.} \textstyleDefinitionn{body}\textstylefstandard{.}
\end{styleEntryParagraph}

\begin{styleEntryParagraph}
\textstyleLexeme{hərzloy}\textstylefstandard{   }\textstylePartofspeech{v.} \textstyleDefinitionn{rot}\textstylefstandard{.}
\end{styleEntryParagraph}

\begin{styleEntryParagraph}
\textstyleLexeme{hoɗ}\textstylefstandard{   }\textstylePartofspeech{n.} \textstyleDefinitionn{stomach}\textstylefstandard{.}
\end{styleEntryParagraph}

\begin{styleEntryParagraph}
\textstyleLexeme{hohom}\textstylefstandard{   }\textstylePartofspeech{n.} \textstyleDefinitionn{beetle}\textstylefstandard{.}
\end{styleEntryParagraph}

\begin{styleEntryParagraph}
\textstyleLexeme{holombo}\textstylefstandard{   }\textstylePartofspeech{num.} \textstyleDefinitionn{nine}\textstylefstandard{.}
\end{styleEntryParagraph}

\begin{styleEntryParagraph}
\textstyleLexeme{homboh}\textstylefstandard{   }\textstylePartofspeech{n.} \textstyleDefinitionn{pardon}\textstylefstandard{.}
\end{styleEntryParagraph}

\begin{styleEntryParagraph}
\textstyleLexeme{hor}\textstylefstandard{   }\textstylePartofspeech{n.} \textstyleDefinitionn{woman, wife}\textstylefstandard{.}
\end{styleEntryParagraph}

\begin{styleEntryParagraph}
\textstyleLexeme{  hawər ahay}\textstyleLexeme{\textmd{ }}\textstyleLexeme{\textmd{\textit{n. women.}}}
\end{styleEntryParagraph}

\begin{styleEntryParagraph}
\textstyleLexeme{hwə}\textstyleLexeme{ɗ}\textstyleLexeme{a}\textstylefstandard{   }\textstylePartofspeech{n.}\textstyleDefinitionn{dregs}\textstylefstandard{.}
\end{styleEntryParagraph}

\begin{styleEntryParagraph}
\textstyleLexeme{hwəlen}\textstylefstandard{   }\textstylePartofspeech{n.} \textstyleDefinitionn{back}\textstylefstandard{.}
\end{styleEntryParagraph}

\begin{styleEntryParagraph}
\textstyleLexeme{hwəsese}\textstylefstandard{   }\textstylePartofspeech{n.} \textstyleDefinitionn{smallness}\textstylefstandard{.}
\end{styleEntryParagraph}

\begin{styleEntryParagraph}
\textstyleLexeme{hwəter}\textstylefstandard{   }\textstylePartofspeech{n.} \textstyleDefinitionn{tail}\textstylefstandard{.}
\end{styleEntryParagraph}

\begin{styleEntryParagraph}
\textstyleLexeme{hwəzlay}\textstylefstandard{   }\textstylePartofspeech{v.} \textstyleDefinitionn{destroy}\textstylefstandard{.} 
\end{styleEntryParagraph}
\end{multicols}
\begin{styleLetterParagraph}
\textstyleLetterv{J  {}-  j}
\end{styleLetterParagraph}

\begin{multicols}{2}
\begin{styleEntryParagraph}
\textstyleLexeme{jajak}\textstylefstandard{   }\textstylePartofspeech{adv.} \textstyleDefinitionn{fast}\textstylefstandard{.}
\end{styleEntryParagraph}

\begin{styleEntryParagraph}
\textstyleLexeme{jajay}\textstylefstandard{   }\textstylePartofspeech{n.} \textstyleDefinitionn{dawn, light}\textstylefstandard{.}
\end{styleEntryParagraph}

\begin{styleEntryParagraph}
\textstyleLexeme{jakay}\textstylefstandard{   }\textstylePartofspeech{v.}\textstyleDefinitionn{lean}\textstylefstandard{.}
\end{styleEntryParagraph}

\begin{styleEntryParagraph}
\textstyleLexeme{japay}\textstylefstandard{   }\textstylePartofspeech{v.} \textstyleDefinitionn{mix}\textstylefstandard{.}
\end{styleEntryParagraph}

\begin{styleEntryParagraph}
\textstyleLexeme{jav}\textstylefstandard{   }\textstylePartofspeech{v.} \textstyleDefinitionn{plant.}
\end{styleEntryParagraph}

\begin{styleEntryParagraph}
\textstyleLexeme{javar}\textstylefstandard{   }\textstylePartofspeech{n.} \textstyleDefinitionn{guinea fowl}\textstylefstandard{.}
\end{styleEntryParagraph}

\begin{styleEntryParagraph}
\textstyleLexeme{jay}\textstylefstandard{   }\textstylePartofspeech{v.}\textstyleDefinitionn{speak}\textstylefstandard{.}
\end{styleEntryParagraph}

\begin{styleEntryParagraph}
\textstyleLexeme{jeɗ jeɗ jeɗ}\textstylefstandard{   }\textstylePartofspeech{ID.} s\textstylefstandard{\textit{ight/idea of ostrich running}}\textstylefstandard{.}
\end{styleEntryParagraph}

\begin{styleEntryParagraph}
\textstyleLexeme{jegwer}\textstylefstandard{   }\textstylePartofspeech{n.} \textstyleDefinitionn{limpness}\textstylefstandard{.}
\end{styleEntryParagraph}

\begin{styleEntryParagraph}
\textstyleLexeme{jen}\textstylefstandard{   }\textstylePartofspeech{n.} \textstyleDefinitionn{luck}\textstylefstandard{.}
\end{styleEntryParagraph}

\begin{styleEntryParagraph}
\textstyleLexeme{jere}\textstylefstandard{   }\textstylePartofspeech{n.} \textstyleDefinitionn{truth}
\end{styleEntryParagraph}

\begin{styleEntryParagraph}
\textstyleLexeme{jəbe}\textstylefstandard{   }\textstylePartofspeech{n.} \textstyleDefinitionn{tribe}\textstylefstandard{.}
\end{styleEntryParagraph}

\begin{styleEntryParagraph}
\textstyleLexeme{jəb jəb}\textstylefstandard{   }\textstylePartofspeech{ID.} \textstyleDefinitionn{completely wet}\textstylefstandard{.}
\end{styleEntryParagraph}

\begin{styleEntryParagraph}
\textstyleLexeme{jəɗokoy}\textstylefstandard{   }\textstylePartofspeech{v.} \textstyleDefinitionn{mash}\textstylefstandard{.}
\end{styleEntryParagraph}

\begin{styleEntryParagraph}
\textstyleLexeme{jəgəlen}\textstylefstandard{   }\textstylePartofspeech{n.} \textstyleDefinitionn{stable}\textstylefstandard{.}
\end{styleEntryParagraph}

\begin{styleEntryParagraph}
\textstyleLexeme{jəgor}\textstylefstandard{   }\textstylePartofspeech{n.} \textstyleDefinitionn{shepherd; stake}\textstylefstandard{.}
\end{styleEntryParagraph}

\begin{styleEntryParagraph}
\textstyleLexeme{jəgor}\textstylefstandard{   }\textstylePartofspeech{v.} \textstyleDefinitionn{shepherd}\textstylefstandard{.}
\end{styleEntryParagraph}

\begin{styleEntryParagraph}
\textstyleLexeme{jənay}\textstylefstandard{   }\textstylePartofspeech{v.} \textstyleDefinitionn{help}\textstylefstandard{.}
\end{styleEntryParagraph}

\begin{styleEntryParagraph}
\textstyleLexeme{jəway}\textstylefstandard{   }\textstylePartofspeech{n.} \textstyleDefinitionn{fly}\textstylefstandard{.}
\end{styleEntryParagraph}

\begin{styleEntryParagraph}
\textstyleLexeme{jəwk jəwk}\textstylefstandard{   }\textstylePartofspeech{adv.} \textstyleDefinitionn{suddenly}\textstylefstandard{.}
\end{styleEntryParagraph}

\begin{styleEntryParagraph}
\textstyleLexeme{jəyga}\textstylefstandard{   }\textstylePartofspeech{quant.} \textstyleDefinitionn{all}\textstylefstandard{.} 
\end{styleEntryParagraph}

\begin{styleEntryParagraph}
\textstyleLexeme{jo}\textstylefstandard{   }\textstylePartofspeech{ID.} \textstyleflabel{take.}
\end{styleEntryParagraph}

\begin{styleEntryParagraph}
\textstyleLexeme{jogo}\textstylefstandard{   }\textstylePartofspeech{n.} \textstyleDefinitionn{hat}\textstylefstandard{.}
\end{styleEntryParagraph}

\begin{styleEntryParagraph}
\textstyleLexeme{johoy}\textstylefstandard{   }\textstylePartofspeech{v.} \textstyleDefinitionn{save, economize}\textstylefstandard{.}
\end{styleEntryParagraph}

\begin{styleEntryParagraph}
\textstyleLexeme{jokoy}\textstylefstandard{   }\textstylePartofspeech{v.} \textstyleDefinitionn{pack down}\textstylefstandard{.}
\end{styleEntryParagraph}

\begin{styleEntryParagraph}
\textstyleLexeme{jorɓoy}\textstylefstandard{   }\textstylePartofspeech{v.} \textstyleDefinitionn{wash clothes}\textstylefstandard{.} 
\end{styleEntryParagraph}
\end{multicols}
\begin{styleLetterParagraph}
\textstyleLetterv{K  {}-  k}
\end{styleLetterParagraph}

\begin{multicols}{2}
\begin{styleEntryParagraph}
\textstyleLexeme{k\nobreakdash-}\textstylefstandard{   }\textstylePartofspeech{vpfx.}\textstyleDefinitionn{2S subject}\textstylefstandard{.}
\end{styleEntryParagraph}

\begin{styleEntryParagraph}
\textstyleLexeme{kə…aka}\textstylefstandard{   }\textstylePartofspeech{adp.}\textstyleDefinitionn{on}\textstylefstandard{.}
\end{styleEntryParagraph}

\begin{styleEntryParagraph}
\textstyleLexeme{ka}\textstylefstandard{   }\textstylePartofspeech{adv.} \textstyleDefinitionn{like}\textstylefstandard{.}
\end{styleEntryParagraph}

\begin{styleIndentedParagraph}
\textstyleSubentry{ka nehe}\textstylefstandard{   }\textstylePartofspeech{dem.} \textstyleDefinitionn{like this}\textstylefstandard{.}
\end{styleIndentedParagraph}

\begin{styleIndentedParagraph}
\textstyleLexeme{ka ngəhe}\textstylefstandard{   }\textstylePartofspeech{dem.} \textstyleDefinitionn{like this here}\textstylefstandard{.}
\end{styleIndentedParagraph}

\begin{styleEntryParagraph}
\textstyleLexeme{kaɓay}\textstylefstandard{   }\textstylePartofspeech{v.} \textstyleDefinitionn{cook or stir quickly next to fire}\textstylefstandard{.}
\end{styleEntryParagraph}

\begin{styleEntryParagraph}
\textstyleLexeme{kaɗ}\textstylefstandard{   }\textstylePartofspeech{v.} \textstyleDefinitionn{kill by clubbing}\textstylefstandard{.}
\end{styleEntryParagraph}

\begin{styleEntryParagraph}
\textstyleLexeme{kaɗay}\textstylefstandard{   }\textstylePartofspeech{v.} \textstyleDefinitionn{prune}\textstylefstandard{.}
\end{styleEntryParagraph}

\begin{styleEntryParagraph}
\textbf{kaləw}\textbf{\textit{  }}\textstylefstandard{   }\textstylePartofspeech{ID. } \textstyleDefinitionn{quickly}\textstylefstandard{.}
\end{styleEntryParagraph}

\begin{styleEntryParagraph}
\textstyleLexeme{kamay}\textstylefstandard{   }\textstylePartofspeech{pn.} \textstyleDefinitionn{why}\textstylefstandard{.}
\end{styleEntryParagraph}

\begin{styleEntryParagraph}
\textstyleLexeme{kapay}\textstylefstandard{   }\textstylePartofspeech{v.} be \textstyleDefinitionn{roughcast}\textstylefstandard{.}
\end{styleEntryParagraph}

\begin{styleEntryParagraph}
\textstyleLexeme{karay}\textstylefstandard{   }\textstylePartofspeech{v.} \textstyleDefinitionn{steal}\textstylefstandard{.}
\end{styleEntryParagraph}

\begin{styleIndentedParagraph}
\textstyleSubentry{akar}\textstylefstandard{   }\textstylePartofspeech{n.} \textstyleDefinitionn{theft}\textstylefstandard{.}
\end{styleIndentedParagraph}

\begin{styleEntryParagraph}
\textstyleLexeme{kasl}\textstylefstandard{   }\textstylePartofspeech{v.} \textstyleDefinitionn{wait ; watch over}\textstylefstandard{.}
\end{styleEntryParagraph}

\begin{styleEntryParagraph}
\textstyleLexeme{kay}\textstylefstandard{   }\textstylePartofspeech{interj.}\textstyleDefinitionn{exclamation when surprised.}
\end{styleEntryParagraph}

\begin{styleEntryParagraph}
\textstyleLexeme{kekəɓ\nobreakdash-kekeɓ}\textstylefstandard{   }\textstylePartofspeech{ID.} \textstyleDefinitionn{sharpness}\textstylefstandard{.}
\end{styleEntryParagraph}

\begin{styleEntryParagraph}
\textstyleLexeme{kəɓəcay}\textstylefstandard{   }\textstylePartofspeech{v.} \textstyleDefinitionn{snap}\textstylefstandard{.}
\end{styleEntryParagraph}

\begin{styleEntryParagraph}
\textstyleLexeme{kəɓəcay}\textstylefstandard{   }\textstylePartofspeech{v.} \textstyleDefinitionn{blink quickly}\textstylefstandard{.}
\end{styleEntryParagraph}

\begin{styleEntryParagraph}
\textstyleLexeme{kəcaway}\textstylefstandard{   }\textstylePartofspeech{v.} \textstyleDefinitionn{trap, seize}\textstylefstandard{.}
\end{styleEntryParagraph}

\begin{styleEntryParagraph}
\textstyleLexeme{kək }\textstylefstandard{  }\textstylePartofspeech{ID.} \textstyleDefinitionn{idea of catching someone by the throat}\textstylefstandard{.}
\end{styleEntryParagraph}

\begin{styleEntryParagraph}
\textstyleLexeme{kəkef  kəf }\textstylefstandard{  }\textstylePartofspeech{ID.} \textstyleDefinitionn{idea of someone who hasn’t any weight (an insult)}\textstylefstandard{.}
\end{styleEntryParagraph}

\begin{styleEntryParagraph}
\textstyleLexeme{kəlakasl}\textstylefstandard{   }\textstylePartofspeech{n.} \textstyleDefinitionn{bone}\textstylefstandard{.}
\end{styleEntryParagraph}

\begin{styleEntryParagraph}
\textstyleLexeme{kəlbawak}\textstylefstandard{   }\textstylePartofspeech{n.} \textstyleDefinitionn{bird}\textstylefstandard{.}
\end{styleEntryParagraph}

\begin{styleEntryParagraph}
\textstyleLexeme{kəlef}\textstylefstandard{   }\textstylePartofspeech{n.} \textstyleDefinitionn{fish}\textstylefstandard{.}
\end{styleEntryParagraph}

\begin{styleEntryParagraph}
\textstyleLexeme{kəlen}\textstylefstandard{   }\textstylePartofspeech{n.} \textstyleDefinitionn{seer}\textstylefstandard{.}
\end{styleEntryParagraph}

\begin{styleEntryParagraph}
\textstyleLexeme{kəlen}\textstylefstandard{   }\textstylePartofspeech{disc.} \textstyleDefinitionn{then}\textstylefstandard{.}
\end{styleEntryParagraph}

\begin{styleEntryParagraph}
\textstyleLexeme{kəl kəl}\textstylefstandard{   }\textstylePartofspeech{ID.} \textstyleDefinitionn{exactly}\textstylefstandard{.}
\end{styleEntryParagraph}

\begin{styleEntryParagraph}
\textstyleLexeme{kəla}\textstylefstandard{   }\textstylePartofspeech{conj.} \textstyleDefinitionn{Benefactive}\textstylefstandard{.}
\end{styleEntryParagraph}

\begin{styleEntryParagraph}
\textstyleLexeme{kəlo}\textstylefstandard{  }\textstylePartofspeech{adv.} \textstyleDefinitionn{before}\textstylefstandard{.}
\end{styleEntryParagraph}

\begin{styleEntryParagraph}
\textstyleLexeme{kəmbohoy}\textstylefstandard{   }\textstylePartofspeech{v.} \textstyleDefinitionn{wrap}\textstylefstandard{.}
\end{styleEntryParagraph}

\begin{styleEntryParagraph}
\textstyleLexeme{kəmeje}\textstylefstandard{   }\textstylePartofspeech{n.} \textstyleDefinitionn{clothes}\textstylefstandard{.}
\end{styleEntryParagraph}

\begin{styleEntryParagraph}
\textstyleLexeme{kəndal}\textstylefstandard{   }\textstylePartofspeech{ID.}\textstyleDefinitionn{sound/idea of pounding millet}\textstylefstandard{.}
\end{styleEntryParagraph}

\begin{styleEntryParagraph}
\textstyleLexeme{kəndew}\textstylefstandard{   }\textstylePartofspeech{n.} \textstyleDefinitionn{stringed instrument}\textstylefstandard{.}
\end{styleEntryParagraph}

\begin{styleEntryParagraph}
\textstyleLexeme{kəra}\textstylefstandard{   }\textstylePartofspeech{n.} \textstyleDefinitionn{dog}\textstylefstandard{.}
\end{styleEntryParagraph}

\begin{styleEntryParagraph}
\textstyleLexeme{kəramba}\textstylefstandard{   }\textstylePartofspeech{n.} \textstyleDefinitionn{crocodile}\textstylefstandard{.}
\end{styleEntryParagraph}

\begin{styleEntryParagraph}
\textstyleLexeme{kəray}\textstylefstandard{   }\textstylePartofspeech{adv.}\textstyleDefinitionn{everywhere}\textstylefstandard{.}
\end{styleEntryParagraph}

\begin{styleEntryParagraph}
\textstyleLexeme{kərcece}\textstylefstandard{   }\textstylePartofspeech{n.} \textstyleDefinitionn{giraffe}\textstylefstandard{.}
\end{styleEntryParagraph}

\begin{styleEntryParagraph}
\textstyleLexeme{kərɗay}\textstylefstandard{   }\textstylePartofspeech{v.} \textstyleDefinitionn{chew}\textstylefstandard{.}
\end{styleEntryParagraph}

\begin{styleEntryParagraph}
\textstyleLexeme{kərɗaway}\textstylefstandard{   }\textstylePartofspeech{v.} \textstyleDefinitionn{scrape}\textstylefstandard{.}
\end{styleEntryParagraph}

\begin{styleEntryParagraph}
\textstyleLexeme{kəre}\textstylefstandard{   }\textstylePartofspeech{n.} \textstyleDefinitionn{rafter}\textstylefstandard{.}
\end{styleEntryParagraph}

\begin{styleEntryParagraph}
\textstyleLexeme{kərkaɗaw}\textstylefstandard{   }\textstylePartofspeech{n.} \textstyleDefinitionn{monkey.}
\end{styleEntryParagraph}

\begin{styleEntryParagraph}
\textstyleLexeme{kərkay}\textstylefstandard{   }\textstylePartofspeech{v.} \textstyleDefinitionn{kneel}\textstylefstandard{.}
\end{styleEntryParagraph}

\begin{styleEntryParagraph}
\textstyleLexeme{kərkayah}\textstylefstandard{   }\textstylePartofspeech{n.} \textstyleDefinitionn{turtle}\textstylefstandard{.}
\end{styleEntryParagraph}

\begin{styleEntryParagraph}
\textstyleLexeme{kəro}\textstylefstandard{   }\textstylePartofspeech{num.} \textstyleDefinitionn{ten}\textstylefstandard{.}
\end{styleEntryParagraph}

\begin{styleEntryParagraph}
\textstyleLexeme{kəroy}\textstylefstandard{   }\textstylePartofspeech{v.} \textstyleDefinitionn{mount}\textstylefstandard{.}
\end{styleEntryParagraph}

\begin{styleEntryParagraph}
\textstyleLexeme{kərpasla}\textstylefstandard{   }\textstylePartofspeech{n.} \textstyleDefinitionn{wings}\textstylefstandard{.}
\end{styleEntryParagraph}

\begin{styleEntryParagraph}
\textstyleLexeme{kərsay}\textstylefstandard{   }\textstylePartofspeech{n.} \textstyleDefinitionn{cucumber}\textstylefstandard{.}
\end{styleEntryParagraph}

\begin{styleEntryParagraph}
\textstyleLexeme{kərsoy}\textstylefstandard{   }\textstylePartofspeech{v.} \textstyleDefinitionn{sweep}\textstylefstandard{.}
\end{styleEntryParagraph}

\begin{styleEntryParagraph}
\textstyleLexeme{kərtoy}\textstylefstandard{   }\textstylePartofspeech{v.} \textstyleDefinitionn{undress, peel}\textstylefstandard{.}
\end{styleEntryParagraph}

\begin{styleEntryParagraph}
\textstyleLexeme{kərway}\textstylefstandard{   }\textstylePartofspeech{v.} \textstyleDefinitionn{cultívate second time}\textstylefstandard{.}
\end{styleEntryParagraph}

\begin{styleEntryParagraph}
\textstyleLexeme{kərwəɗ wəɗ kərwəɗ wəɗ}\textstylefstandard{   }\textstylePartofspeech{ID.} \textstyleDefinitionn{sight/idea of an old person trying to run}\textstylefstandard{.}
\end{styleEntryParagraph}

\begin{styleEntryParagraph}
\textstyleLexeme{kətay}\textstylefstandard{   }\textstylePartofspeech{v.} \textstyleDefinitionn{punish}\textstylefstandard{.}
\end{styleEntryParagraph}

\begin{styleEntryParagraph}
\textstyleLexeme{kətefer}\textstylefstandard{   }\textstylePartofspeech{n.} \textstyleDefinitionn{scoop}\textstylefstandard{.}
\end{styleEntryParagraph}

\begin{styleEntryParagraph}
\textstyleLexeme{kəway}\textstylefstandard{   }\textstylePartofspeech{v.} \textstyleDefinitionn{look for}\textstylefstandard{.}
\end{styleEntryParagraph}

\begin{styleEntryParagraph}
\textstyleLexeme{kəway}\textstylefstandard{   }\textstylePartofspeech{v.} \textstyleDefinitionn{get drunk}\textstylefstandard{.}
\end{styleEntryParagraph}

\begin{styleEntryParagraph}
\textstyleLexeme{kəwaya}\textstylefstandard{   }\textstylePartofspeech{conj.}\textstyleDefinitionn{because, that is}\textstylefstandard{.}
\end{styleEntryParagraph}

\begin{styleEntryParagraph}
\textstyleLexeme{kəwna}\textstylefstandard{   }\textstylePartofspeech{ID.} \textstyleDefinitionn{idea of grasping}\textstylefstandard{.}
\end{styleEntryParagraph}

\begin{styleEntryParagraph}
\textstyleLexeme{kəy}\textstylefstandard{   }\textstylePartofspeech{ID.} \textstyleDefinitionn{idea of looking}\textstylefstandard{.}
\end{styleEntryParagraph}

\begin{styleEntryParagraph}
\textstyleLexeme{kəya}\textstylefstandard{   }\textstylePartofspeech{n.} \textstyleDefinitionn{moon}\textstylefstandard{.}
\end{styleEntryParagraph}

\begin{styleEntryParagraph}
\textstyleLexeme{kəyga}\textstylefstandard{   }\textstylePartofspeech{dem.} \textstyleDefinitionn{like that}\textstylefstandard{.}
\end{styleEntryParagraph}

\begin{styleEntryParagraph}
\textstyleLexeme{kəygehe}\textstylefstandard{   }\textstylePartofspeech{dem.} \textstyleDefinitionn{like this}\textstylefstandard{.}
\end{styleEntryParagraph}

\begin{styleEntryParagraph}
\textstyleLexeme{ko}\textstylefstandard{   }\textstylePartofspeech{adv.} \textstyleDefinitionn{even}\textstylefstandard{.}
\end{styleEntryParagraph}

\begin{styleEntryParagraph}
\textstyleLexeme{kokofoy}\textstylefstandard{   }\textstylePartofspeech{n.} \textstyleDefinitionn{newborn baby}\textstylefstandard{.}
\end{styleEntryParagraph}

\begin{styleEntryParagraph}
\textstyleLexeme{kokolo}\textstylefstandard{   }\textstylePartofspeech{n.} \textstyleDefinitionn{leprosy}\textstylefstandard{.}
\end{styleEntryParagraph}

\begin{styleEntryParagraph}
\textstyleLexeme{kokor}\textstylefstandard{   }\textstylePartofspeech{n.} \textstyleDefinitionn{gourd.}
\end{styleEntryParagraph}

\begin{styleEntryParagraph}
\textstyleLexeme{koloy}\textstylefstandard{   }\textstylePartofspeech{v.} \textstyleDefinitionn{dry}\textstylefstandard{.}
\end{styleEntryParagraph}

\begin{styleEntryParagraph}
\textstyleLexeme{kondon}\textstylefstandard{   }\textstylePartofspeech{n.} \textstyleDefinitionn{banana}\textstylefstandard{.}
\end{styleEntryParagraph}

\begin{styleEntryParagraph}
\textstyleLexeme{koroy}\textstylefstandard{   }\textstylePartofspeech{v.} \textstyleDefinitionn{put}\textstylefstandard{.}
\end{styleEntryParagraph}

\begin{styleEntryParagraph}
\textstyleLexeme{kosoko}\textstylefstandard{   }\textstylePartofspeech{n.} \textstyleDefinitionn{market}\textstylefstandard{.} 
\end{styleEntryParagraph}

\begin{styleEntryParagraph}
\textstyleLexeme{kweɗe kweɗe}\textstylefstandard{   }\textstylePartofspeech{n.} \textstyleDefinitionn{shakers}\textstylefstandard{.}
\end{styleEntryParagraph}

\begin{styleEntryParagraph}
\textstyleLexeme{kwəcesl}\textstylefstandard{   }\textstylePartofspeech{n.} \textstyleDefinitionn{viper}\textstylefstandard{.}
\end{styleEntryParagraph}

\begin{styleEntryParagraph}
\textstyleLexeme{kwəleɗeɗe}\textstylefstandard{   }\textstylePartofspeech{n.} \textstyleDefinitionn{smooth.}
\end{styleEntryParagraph}

\begin{styleEntryParagraph}
\textstyleLexeme{kwəsay}\textstylefstandard{   }\textstylePartofspeech{n.} \textstyleDefinitionn{haze}\textstylefstandard{.}
\end{styleEntryParagraph}
\end{multicols}
\begin{styleLetterParagraph}
\textstyleLetterv{L  {}-  l }
\end{styleLetterParagraph}

\begin{multicols}{2}
\begin{styleEntryParagraph}
\textstyleLexeme{lagay}\textstylefstandard{   }\textstylePartofspeech{v.} \textstyleDefinitionn{accompany}\textstylefstandard{.}
\end{styleEntryParagraph}

\begin{styleEntryParagraph}
\textstyleLexeme{lala}\textstylefstandard{   }\textstylePartofspeech{adv.}\textstyleDefinitionn{good}\textstylefstandard{.}
\end{styleEntryParagraph}

\begin{styleEntryParagraph}
\textstyleLexeme{lamay}\textstylefstandard{   }\textstylePartofspeech{v.} \textstyleDefinitionn{touch}\textstylefstandard{.}
\end{styleEntryParagraph}

\begin{styleEntryParagraph}
\textstyleLexeme{lamba}\textstylefstandard{   }\textstylePartofspeech{n.} \textstyleDefinitionn{number}\textstylefstandard{.}
\end{styleEntryParagraph}

\begin{styleEntryParagraph}
\textstyleLexeme{laway}\textstylefstandard{   }\textstylePartofspeech{v.} \textstyleDefinitionn{hang}\textstylefstandard{.}
\end{styleEntryParagraph}

\begin{styleEntryParagraph}
\textstyleLexeme{laway}\textstylefstandard{   }\textstylePartofspeech{v.} \textstyleDefinitionn{mate with}\textstylefstandard{.}
\end{styleEntryParagraph}

\begin{styleEntryParagraph}
\textstyleLexeme{lay}\textstylefstandard{   }\textstylePartofspeech{v.} \textstyleDefinitionn{dig}\textstylefstandard{.}
\end{styleEntryParagraph}

\begin{styleEntryParagraph}
\textstyleLexeme{layaw}\textstylefstandard{   }\textstylePartofspeech{n.} \textstyleDefinitionn{large squash}\textstylefstandard{.}
\end{styleEntryParagraph}

\begin{styleEntryParagraph}
\textstyleLexeme{lekwel}\textstylefstandard{   }\textstylePartofspeech{n.} \textstyleDefinitionn{school}\textstylefstandard{.}
\end{styleEntryParagraph}

\begin{styleEntryParagraph}
\textstyleLexeme{ləbara}\textstylefstandard{   }\textstylePartofspeech{n.} \textstyleDefinitionn{news}\textstylefstandard{.}
\end{styleEntryParagraph}

\begin{styleEntryParagraph}
\textstyleLexeme{ləhe}\textstylefstandard{   }\textstylePartofspeech{n.} \textstyleDefinitionn{bush, fields}\textstylefstandard{.}
\end{styleEntryParagraph}

\begin{styleEntryParagraph}
\textstyleLexeme{ləho}\textstylefstandard{   }\textstylePartofspeech{n.} \textstyleDefinitionn{evening}\textstylefstandard{.} 
\end{styleEntryParagraph}

\begin{styleEntryParagraph}
\textstyleLexeme{ləkwəye}\textstylefstandard{   }\textstylePartofspeech{pn.} \textstyleDefinitionn{2P}\textstylefstandard{.}
\end{styleEntryParagraph}

\begin{styleEntryParagraph}
\textstyleLexeme{ləme}\textstylefstandard{   }\textstylePartofspeech{pn.} \textstyleDefinitionn{1P}\textstylePartofspeech{EX}\textstylefstandard{.}
\end{styleEntryParagraph}

\begin{styleEntryParagraph}
\textstyleLexeme{ləmes}\textstylefstandard{   }\textstylePartofspeech{n.} \textstyleDefinitionn{song}\textstylefstandard{.}
\end{styleEntryParagraph}

\begin{styleEntryParagraph}
\textstyleLexeme{ləpəre}\textstylefstandard{   }\textstylePartofspeech{n.} \textstyleDefinitionn{needle}\textstylefstandard{.}
\end{styleEntryParagraph}

\begin{styleEntryParagraph}
\textstyleLexeme{ləvan}\textstylefstandard{   }\textstylePartofspeech{n.} \textstyleDefinitionn{night}\textstylefstandard{.}
\end{styleEntryParagraph}

\begin{styleEntryParagraph}
\textstyleLexeme{lo}\textstylefstandard{   }\textstylePartofspeech{v.} \textstyleDefinitionn{go.}
\end{styleEntryParagraph}

\begin{styleEntryParagraph}
\textstyleLexeme{loko}\textstylefstandard{   }\textstylePartofspeech{pn.} \textstyleDefinitionn{1P}\textstylePartofspeech{IN}\textstylefstandard{.}
\end{styleEntryParagraph}

\begin{styleEntryParagraph}
\textstyleLexeme{lolokoy}\textstylefstandard{   }\textstylePartofspeech{n.} \textstyleDefinitionn{mouse trap}\textstylefstandard{.}
\end{styleEntryParagraph}
\end{multicols}
\begin{styleLetterParagraph}
\textstyleLetterv{M  {}-  m}
\end{styleLetterParagraph}

\begin{multicols}{2}
\begin{styleEntryParagraph}
\textstyleLexeme{ma}\textstylefstandard{   }\textstylePartofspeech{n.} \textstyleDefinitionn{mouth, language, word}\textstylefstandard{.}
\end{styleEntryParagraph}

\begin{styleEntryParagraph}
\textstyleLexeme{maɓasl}\textstylefstandard{   }\textstylePartofspeech{n.} \textstyleDefinitionn{pumpkin}\textstylefstandard{.}
\end{styleEntryParagraph}

\begin{styleEntryParagraph}
\textstyleLexeme{macəkəmbay}\textstylefstandard{   }\textstylePartofspeech{conj.} \textstyleDefinitionn{meanwhile}\textstylefstandard{.}
\end{styleEntryParagraph}

\begin{styleEntryParagraph}
\textstyleLexeme{madan}\textstylefstandard{   }\textstylePartofspeech{n.} \textstyleDefinitionn{sorcery}\textstylefstandard{.}
\end{styleEntryParagraph}

\begin{styleEntryParagraph}
\textstyleLexeme{madəras}\textstylefstandard{   }\textstylePartofspeech{n.} \textstyleDefinitionn{pig}\textstylefstandard{.}
\end{styleEntryParagraph}

\begin{styleEntryParagraph}
\textstyleLexeme{mahaw}\textstylefstandard{   }\textstylePartofspeech{n.} \textstyleDefinitionn{snake.}
\end{styleEntryParagraph}

\begin{styleEntryParagraph}
\textstyleLexeme{mahay}\textstylefstandard{   }\textstylePartofspeech{n.} \textstyleDefinitionn{door}\textstylefstandard{.}
\end{styleEntryParagraph}

\begin{styleEntryParagraph}
\textstyleLexeme{makala}\textstylefstandard{   }\textstylePartofspeech{n.} \textstyleDefinitionn{donut}\textstylefstandard{.}
\end{styleEntryParagraph}

\begin{styleEntryParagraph}
\textstyleLexeme{makar}\textstylefstandard{   }\textstylePartofspeech{num.} \textstyleDefinitionn{three}\textstylefstandard{.}
\end{styleEntryParagraph}

\begin{styleEntryParagraph}
\textstyleLexeme{makay}\textstylefstandard{   }\textstylePartofspeech{v.} \textstyleDefinitionn{leave, let go}\textstylefstandard{.}
\end{styleEntryParagraph}

\begin{styleEntryParagraph}
\textstyleLexeme{malay}\textstylefstandard{   }\textstylePartofspeech{v.} \textstyleDefinitionn{leave}\textstylefstandard{.}
\end{styleEntryParagraph}

\begin{styleEntryParagraph}
\textstyleLexeme{malan}\textstylefstandard{   }\textstylePartofspeech{n.} \textstyleDefinitionn{greatness}\textstylefstandard{.}
\end{styleEntryParagraph}

\begin{styleEntryParagraph}
\textstyleLexeme{malgamay}\textstylefstandard{   }\textstylePartofspeech{n.} \textstyleDefinitionn{jawbone}\textstylefstandard{.}
\end{styleEntryParagraph}

\begin{styleEntryParagraph}
\textstyleLexeme{malmay}\textstylefstandard{   }\textstylePartofspeech{pn.} \textstyleDefinitionn{what?}
\end{styleEntryParagraph}

\begin{styleEntryParagraph}
\textstyleLexeme{mama}\textstylefstandard{   }\textstylePartofspeech{n.} \textstyleDefinitionn{mother}\textstylefstandard{.}
\end{styleEntryParagraph}

\begin{styleEntryParagraph}
\textstyleLexeme{mana}\textstylefstandard{   }\textstylePartofspeech{n.} \textstyleDefinitionn{so and so}\textstylefstandard{.}
\end{styleEntryParagraph}

\begin{styleEntryParagraph}
\textstyleLexeme{mangasl}\textstylefstandard{   }\textstylePartofspeech{n.} \textstyleDefinitionn{fiancé}\textstylefstandard{.}
\end{styleEntryParagraph}

\begin{styleEntryParagraph}
\textstyleLexeme{manjara}\textstylefstandard{   }\textstylePartofspeech{n.} \textstyleDefinitionn{termite}\textstylefstandard{.}
\end{styleEntryParagraph}

\begin{styleEntryParagraph}
\textstyleLexeme{manjaw}\textstylefstandard{   }\textstylePartofspeech{n.} \textstyleDefinitionn{donut made from ground nuts}\textstylefstandard{.}
\end{styleEntryParagraph}

\begin{styleEntryParagraph}
\textstyleLexeme{marasl}\textstylefstandard{   }\textstylePartofspeech{n.} \textstyleDefinitionn{hail}\textstylefstandard{.}
\end{styleEntryParagraph}

\begin{styleEntryParagraph}
\textstyleLexeme{margaba}\textstylefstandard{   }\textstylePartofspeech{n.} \textstyleDefinitionn{ant}\textstylefstandard{.}
\end{styleEntryParagraph}

\begin{styleEntryParagraph}
\textstyleLexeme{Masay}\textstylefstandard{   }\textstylePartofspeech{n.pr.} \textstyleDefinitionn{name of first twin}\textstylefstandard{.} \textstyleflabel{Cf.: }\textstylefvernacular{Aləwa}\textstylefstandard{.}
\end{styleEntryParagraph}

\begin{styleEntryParagraph}
\textstyleLexeme{maslalam}\textstylefstandard{   }\textstylePartofspeech{n.} \textstyleDefinitionn{sword}\textstylefstandard{.}
\end{styleEntryParagraph}

\begin{styleEntryParagraph}
\textstyleLexeme{maslar}\textstylefstandard{   }\textstylePartofspeech{n.} \textstyleDefinitionn{front teeth}\textstylefstandard{.}
\end{styleEntryParagraph}

\begin{styleEntryParagraph}
\textstyleLexeme{mat}\textstylefstandard{   }\textstylePartofspeech{v.} \textstyleDefinitionn{die}\textstylefstandard{.}
\end{styleEntryParagraph}

\begin{styleEntryParagraph}
\textstyleLexeme{mataɓasl}\textstylefstandard{   }\textstylePartofspeech{n.} \textstyleDefinitionn{cloud}\textstylefstandard{.}
\end{styleEntryParagraph}

\begin{styleEntryParagraph}
\textstyleLexeme{mavaɗ}\textstylefstandard{   }\textstylePartofspeech{n.} \textstyleDefinitionn{sickle}\textstylefstandard{.}
\end{styleEntryParagraph}

\begin{styleEntryParagraph}
\textstyleLexeme{mawar}\textstylefstandard{   }\textstylePartofspeech{n.} \textstyleDefinitionn{tamarind}\textstylefstandard{.} 
\end{styleEntryParagraph}

\begin{styleEntryParagraph}
\textstyleLexeme{may}\textstylefstandard{   }\textstylePartofspeech{n.} \textstyleDefinitionn{hunger}\textstylefstandard{.}
\end{styleEntryParagraph}

\begin{styleEntryParagraph}
\textstyleLexeme{may}\textstylefstandard{   }\textstylePartofspeech{pn.} \textstyleDefinitionn{what? (emphatic).}
\end{styleEntryParagraph}

\begin{styleEntryParagraph}
\textstyleLexeme{mazlərpapan}\textstylefstandard{   }\textstylePartofspeech{n.} \textstyleDefinitionn{spider}\textstylefstandard{.}
\end{styleEntryParagraph}

\begin{styleEntryParagraph}
\textstyleLexeme{mazloko}\textstylefstandard{   }\textstylePartofspeech{n.} \textstyleDefinitionn{lion}\textstylefstandard{.}
\end{styleEntryParagraph}

\begin{styleEntryParagraph}
\textstyleLexeme{mba}\textstylefstandard{   }\textstylePartofspeech{ID.}\textstyleDefinitionn{a short time.}
\end{styleEntryParagraph}

\begin{styleEntryParagraph}
\textstyleLexeme{mbaɗ}\textstylefstandard{   }\textstylePartofspeech{v.} \textstyleDefinitionn{change}\textstylefstandard{.} 
\end{styleEntryParagraph}

\begin{styleEntryParagraph}
\textstyleLexeme{mbaɗay}\textstylefstandard{   }\textstylePartofspeech{v.} \textstyleDefinitionn{swear}\textstylefstandard{.} 
\end{styleEntryParagraph}

\begin{styleEntryParagraph}
\textstyleLexeme{mbahay}\textstylefstandard{   }\textstylePartofspeech{v.} \textstyleDefinitionn{call}\textstylefstandard{.}
\end{styleEntryParagraph}

\begin{styleEntryParagraph}
\textstyleLexeme{mbaf}\textstylefstandard{   }\textstylePartofspeech{ID.} \textstyleDefinitionn{idea of full up to the roof.}
\end{styleEntryParagraph}

\begin{styleEntryParagraph}
\textstyleLexeme{mbajak}\textstylefstandard{   }\textstylePartofspeech{ID.} \textstyleDefinitionn{something big and reflective}\textstylefstandard{.}
\end{styleEntryParagraph}

\begin{styleEntryParagraph}
\textstyleLexeme{mbar}\textstylefstandard{   }\textstylePartofspeech{v.} \textstyleDefinitionn{heal}\textstylefstandard{.}
\end{styleEntryParagraph}

\begin{styleEntryParagraph}
\textstyleLexeme{mbasay}\textstylefstandard{   }\textstylePartofspeech{v.}\textstyleSensenumber{~}\textstyleDefinitionn{smile}\textstylefstandard{.}
\end{styleEntryParagraph}

\begin{styleEntryParagraph}
\textstyleLexeme{mbat}\textstylefstandard{   }\textstylePartofspeech{v.} \textstyleDefinitionn{turn off}\textstylefstandard{.}
\end{styleEntryParagraph}

\begin{styleEntryParagraph}
\textstyleLexeme{mbay}\textstylefstandard{   }\textstylePartofspeech{n.} \textstyleDefinitionn{manioc}\textstylefstandard{.}
\end{styleEntryParagraph}

\begin{styleEntryParagraph}
\textstyleLexeme{mbay}\textstylefstandard{   }\textstylePartofspeech{v.} \textstyleDefinitionn{follow}\textstylefstandard{.}
\end{styleEntryParagraph}

\begin{styleEntryParagraph}
\textstyleLexeme{mbazl}\textstylefstandard{   }\textstylePartofspeech{v.} \textstyleDefinitionn{demolish}\textstylefstandard{.}
\end{styleEntryParagraph}

\begin{styleEntryParagraph}
\textstyleLexeme{mbe}\textstylefstandard{   }\textstylePartofspeech{v.} \textstyleDefinitionn{argue, scold}\textstylefstandard{.}
\end{styleEntryParagraph}

\begin{styleEntryParagraph}
\textstyleLexeme{mbeɗem}\textstylefstandard{   }\textstylePartofspeech{n.} \textstyleDefinitionn{centre, middle}\textstylefstandard{.}
\end{styleEntryParagraph}

\begin{styleEntryParagraph}
\textstyleLexeme{mbəlele}\textstylefstandard{   }\textstylePartofspeech{n.}\textstyleDefinitionn{elephant.}
\end{styleEntryParagraph}

\begin{styleEntryParagraph}
\textstyleLexeme{mbesen}\textstylefstandard{   }\textstylePartofspeech{v.} \textstyleDefinitionn{rest, breathe.}
\end{styleEntryParagraph}

\begin{styleEntryParagraph}
\textstyleLexeme{mbeten}\textstylefstandard{   }\textstylePartofspeech{v.} \textstyleDefinitionn{extinguish}\textstylefstandard{.}
\end{styleEntryParagraph}

\begin{styleEntryParagraph}
\textstyleLexeme{mbezlen}\textstylefstandard{   }\textstylePartofspeech{v.} \textstyleDefinitionn{count}\textstylefstandard{.}
\end{styleEntryParagraph}

\begin{styleEntryParagraph}
\textstyleLexeme{mbəlɗoy}\textstylefstandard{   }\textstylePartofspeech{v.} \textstyleDefinitionn{skin, peel}\textstylefstandard{.}
\end{styleEntryParagraph}

\begin{styleEntryParagraph}
\textstyleLexeme{mbəra}\textstyleLexeme{ɓ}\textstylefstandard{   }\textstylePartofspeech{ID.} \textstyleDefinitionn{idea of penetration}\textstylefstandard{.}
\end{styleEntryParagraph}

\begin{styleEntryParagraph}
\textstyleLexeme{mbəramay}\textstylefstandard{ }\textstylePartofspeech{v.} \textstyleDefinitionn{blink slowly, break violently}\textstylefstandard{.}
\end{styleEntryParagraph}

\begin{styleEntryParagraph}
\textstyleLexeme{mbərcay}\textstylefstandard{   }\textstylePartofspeech{v.} \textstyleDefinitionn{untie}\textstylefstandard{.}
\end{styleEntryParagraph}

\begin{styleEntryParagraph}
\textstyleLexeme{mbərkala}\textstylefstandard{   }\textstylePartofspeech{n.} \textstyleDefinitionn{red millet}\textstylefstandard{.}
\end{styleEntryParagraph}

\begin{styleEntryParagraph}
\textstyleLexeme{mbərlom}\textstylefstandard{   }\textstylePartofspeech{n.} \textstyleDefinitionn{throat}\textstylefstandard{.}
\end{styleEntryParagraph}

\begin{styleEntryParagraph}
\textstyleLexeme{mbərway}\textstylefstandard{   }\textstylePartofspeech{v.} \textstyleDefinitionn{destroy violently.}
\end{styleEntryParagraph}

\begin{styleEntryParagraph}
\textstyleLexeme{mbərzlay}\textstylefstandard{   }\textstylePartofspeech{v.} \textstyleDefinitionn{pass}\textstylefstandard{.}
\end{styleEntryParagraph}

\begin{styleEntryParagraph}
\textstyleLexeme{mbəzen}\textstylefstandard{   }\textstylePartofspeech{v.} \textstyleDefinitionn{ruin}\textstylefstandard{.}
\end{styleEntryParagraph}

\begin{styleEntryParagraph}
\textstyleLexeme{mbocoy}\textstylefstandard{   }\textstylePartofspeech{v.} \textstyleDefinitionn{beat lightly}\textstylefstandard{.}
\end{styleEntryParagraph}

\begin{styleEntryParagraph}
\textstyleLexeme{Mboko}\textstylefstandard{   }\textstylePartofspeech{n.pr.} \textstyleDefinitionn{Mbuko people/language}\textstylefstandard{.}
\end{styleEntryParagraph}

\begin{styleEntryParagraph}
\textstyleLexeme{mbomoy}\textstylefstandard{   }\textstylePartofspeech{v.} \textstyleDefinitionn{gather with a stick}\textstylefstandard{.}
\end{styleEntryParagraph}

\begin{styleEntryParagraph}
\textstyleLexeme{mebebek}\textstylefstandard{   }\textstylePartofspeech{n.} \textstyleDefinitionn{bat}\textstylefstandard{.}
\end{styleEntryParagraph}

\begin{styleEntryParagraph}
\textstyleLexeme{mecekweɗ}\textstylefstandard{   }\textstylePartofspeech{n.} \textstyleDefinitionn{larva, worm}\textstylefstandard{.}
\end{styleEntryParagraph}

\begin{styleEntryParagraph}
\textstyleLexeme{medəlengwez}\textstylefstandard{   }\textstylePartofspeech{n.} \textstyleDefinitionn{leopard}\textstylefstandard{.}
\end{styleEntryParagraph}

\begin{styleEntryParagraph}
\textstyleLexeme{meher}\textstylefstandard{   }\textstylePartofspeech{n.} \textstyleDefinitionn{forehead}\textstylefstandard{.}
\end{styleEntryParagraph}

\begin{styleEntryParagraph}
\textstyleLexeme{mekec}\textstylefstandard{   }\textstylePartofspeech{n.} \textstyleDefinitionn{knife}\textstylefstandard{.}
\end{styleEntryParagraph}

\begin{styleEntryParagraph}
\textstyleLexeme{mekəlewez}\textstylefstandard{   }\textstylePartofspeech{n.} \textstyleDefinitionn{mongoose}\textstylefstandard{.}
\end{styleEntryParagraph}

\begin{styleEntryParagraph}
\textstyleLexeme{Meme}\textstylefstandard{   }\textstylePartofspeech{n.} \textstyleDefinitionn{market day in the village of Meme}\textstylefstandard{.}
\end{styleEntryParagraph}

\begin{styleEntryParagraph}
\textstyleLexeme{memele}\textstylefstandard{   }\textstylePartofspeech{n.} \textstyleDefinitionn{tree}\textstylefstandard{.}
\end{styleEntryParagraph}

\begin{styleEntryParagraph}
\textstyleLexeme{memey}\textstylefstandard{   }\textstylePartofspeech{pn.} \textstyleDefinitionn{how}\textstylefstandard{.}
\end{styleEntryParagraph}

\begin{styleEntryParagraph}
\textstyleLexeme{memle}\textstylefstandard{   }\textstylePartofspeech{n.} \textstyleDefinitionn{joy}\textstylefstandard{.}
\end{styleEntryParagraph}

\begin{styleEntryParagraph}
\textstyleLexeme{mepetəpete}\textstylefstandard{   }\textstylePartofspeech{n.} \textstyleDefinitionn{butterfly}\textstylefstandard{.}
\end{styleEntryParagraph}

\begin{styleEntryParagraph}
\textstyleLexeme{merkwe}\textstylefstandard{   }\textstylePartofspeech{n.} \textstyleDefinitionn{stranger, traveler}\textstylefstandard{.}
\end{styleEntryParagraph}

\begin{styleEntryParagraph}
\textstyleLexeme{mesesewk}\textstylefstandard{   }\textstylePartofspeech{n.} \textstyleDefinitionn{ant}\textstylefstandard{.}
\end{styleEntryParagraph}

\begin{styleEntryParagraph}
\textstyleLexeme{meslenen}\textstylefstandard{   }\textstylePartofspeech{pn.} \textstyleDefinitionn{no one}\textstylefstandard{.}
\end{styleEntryParagraph}

\begin{styleEntryParagraph}
\textstyleLexeme{metesle}\textstylefstandard{   }\textstylePartofspeech{n.} \textstyleDefinitionn{curse}\textstylefstandard{.}
\end{styleEntryParagraph}

\begin{styleEntryParagraph}
\textstyleLexeme{mey}\textstylefstandard{   }\textstylePartofspeech{pn.} \textstyleDefinitionn{how (emphatic)}\textstylefstandard{.}
\end{styleEntryParagraph}

\begin{styleEntryParagraph}
\textstyleLexeme{mədara}\textstylefstandard{   }\textstylePartofspeech{n.} \textstyleDefinitionn{fire}\textstylefstandard{.}
\end{styleEntryParagraph}

\begin{styleEntryParagraph}
\textstyleLexeme{mədegen}\textstylefstandard{   }\textstylePartofspeech{n.} \textstyleDefinitionn{cold/flu}\textstylefstandard{.}
\end{styleEntryParagraph}

\begin{styleEntryParagraph}
\textstyleLexeme{mədehwer}\textstylefstandard{   }\textstylePartofspeech{n.} \textstyleDefinitionn{old person}\textstylefstandard{.}
\end{styleEntryParagraph}

\begin{styleEntryParagraph}
\textstyleLexeme{mədəga}\textstylefstandard{   }\textstylePartofspeech{n.} \textstyleDefinitionn{older sibling}\textstylefstandard{.}
\end{styleEntryParagraph}

\begin{styleEntryParagraph}
\textstyleLexeme{mədəger}\textstylefstandard{   }\textstylePartofspeech{n.} \textstyleDefinitionn{hoe}\textstylefstandard{.}
\end{styleEntryParagraph}

\begin{styleEntryParagraph}
\textstyleLexeme{mədəra}\textstylefstandard{   }\textstylePartofspeech{n.} \textstyleDefinitionn{bicep}\textstylefstandard{.}
\end{styleEntryParagraph}

\begin{styleEntryParagraph}
\textstyleLexeme{məf}\textstylefstandard{   }\textstylePartofspeech{interj.} \textstyleDefinitionn{get away}\textstylefstandard{!}
\end{styleEntryParagraph}

\begin{styleEntryParagraph}
\textstyleLexeme{məfaɗ}\textstylefstandard{   }\textstylePartofspeech{num.} \textstyleDefinitionn{four}\textstylefstandard{.}
\end{styleEntryParagraph}

\begin{styleEntryParagraph}
\textstyleLexeme{məjəvoko}\textstylefstandard{   }\textstylePartofspeech{n.} \textstyleDefinitionn{celebration (lit. planting fire)}\textstylefstandard{.}
\end{styleEntryParagraph}

\begin{styleEntryParagraph}
\textstyleLexeme{mək}\textstylefstandard{   }\textstylePartofspeech{ID.} \textstyleDefinitionn{idea of positioning self for throwing something (spear)}\textstylefstandard{.}
\end{styleEntryParagraph}

\begin{styleEntryParagraph}
\textstyleLexeme{məko}\textstylefstandard{   }\textstylePartofspeech{num.} \textstyleDefinitionn{six}\textstylefstandard{.}
\end{styleEntryParagraph}

\begin{styleEntryParagraph}
\textstyleLexeme{məlama}\textstylefstandard{   }\textstylePartofspeech{n.} \textstyleDefinitionn{sibling.}
\end{styleEntryParagraph}

\begin{styleEntryParagraph}
\textstyleLexeme{məlay}\textstylefstandard{   }\textstylePartofspeech{v. enjoy}
\end{styleEntryParagraph}

\begin{styleEntryParagraph}
\textstyleLexeme{Məloko}\textstylefstandard{   }\textstylePartofspeech{n.pr.} \textstyleDefinitionn{Moloko people/language}\textstylefstandard{.}
\end{styleEntryParagraph}

\begin{styleEntryParagraph}
\textstyleLexeme{məndacay}\textstylefstandard{   }\textstylePartofspeech{v.} \textstyleDefinitionn{gather}\textstylefstandard{.}
\end{styleEntryParagraph}

\begin{styleEntryParagraph}
\textstyleLexeme{məndəye}\textstylefstandard{   }\textstylePartofspeech{n.} \textstyleSensenumber{\textit{day}}\textstylefstandard{.}
\end{styleEntryParagraph}

\begin{styleEntryParagraph}
\textstyleLexeme{məndocay}\textstylefstandard{   }\textstylePartofspeech{v.} \textstyleSensenumber{\textit{gather}}\textstylefstandard{.}
\end{styleEntryParagraph}

\begin{styleEntryParagraph}
\textstyleLexeme{məngahak}\textstylefstandard{   }\textstylePartofspeech{n.} \textstyleDefinitionn{crow}\textstylefstandard{.}
\end{styleEntryParagraph}

\begin{styleEntryParagraph}
\textstyleLexeme{məngamak}\textstylefstandard{   }\textstylePartofspeech{n.} \textstyleDefinitionn{wild cat}\textstylefstandard{.}
\end{styleEntryParagraph}

\begin{styleEntryParagraph}
\textstyleLexeme{mənjaɗ}\textstylefstandard{   }\textstylePartofspeech{adp.} \textstyleDefinitionn{without}\textstylefstandard{.}
\end{styleEntryParagraph}

\begin{styleEntryParagraph}
\textstyleLexeme{mənjar}\textstylefstandard{   }\textstylePartofspeech{v.} \textstyleSensenumber{\textit{see}}\textstylefstandard{.}
\end{styleEntryParagraph}

\begin{styleEntryParagraph}
\textstyleLexeme{mənjəye}\textstylefstandard{   }\textstylePartofspeech{n.} \textstyleSensenumber{\textit{habits}}\textstylefstandard{.}
\end{styleEntryParagraph}

\begin{styleEntryParagraph}
\textstyleLexeme{məpapar}\textstylefstandard{   }\textstylePartofspeech{n.} \textstyleDefinitionn{grass fence}\textstylefstandard{.}
\end{styleEntryParagraph}

\begin{styleEntryParagraph}
\textstyleLexeme{məray}\textstylefstandard{   }\textstylePartofspeech{n.}\textstyleSensenumber{~}\textstyleDefinitionn{shame}\textstylefstandard{.}
\end{styleEntryParagraph}

\begin{styleEntryParagraph}
\textstyleLexeme{mərcay}\textstylefstandard{   }\textstylePartofspeech{v.}\textstyleSensenumber{~}\textstyleDefinitionn{put horizontally}\textstylefstandard{.}
\end{styleEntryParagraph}

\begin{styleEntryParagraph}
\textstyleLexeme{məsek}\textstylefstandard{   }\textstylePartofspeech{n.} \textstyleDefinitionn{pot}\textstylefstandard{.}
\end{styleEntryParagraph}

\begin{styleEntryParagraph}
\textstyleLexeme{mətenen}\textstylefstandard{   }\textstylePartofspeech{n.} \textstyleDefinitionn{bottom}\textstylefstandard{.}
\end{styleEntryParagraph}

\begin{styleEntryParagraph}
\textstyleLexeme{mətəde}\textstylefstandard{   }\textstylePartofspeech{n.} \textstyleDefinitionn{cicada}\textstylefstandard{.}
\end{styleEntryParagraph}

\begin{styleEntryParagraph}
\textstyleLexeme{mətəmbətəmbezl}\textstylefstandard{   }\textstyleDefinitionn{viper}\textstylefstandard{.}
\end{styleEntryParagraph}

\begin{styleEntryParagraph}
\textstyleLexeme{mətəmey}\textstylefstandard{   }\textstylePartofspeech{pn.} \textstyleDefinitionn{how much/how many}\textstylefstandard{.}
\end{styleEntryParagraph}

\begin{styleEntryParagraph}
\textstyleLexeme{mətərak}\textstylefstandard{   }\textstylePartofspeech{n.} \textstyleDefinitionn{pap, hot drink made with rice}
\end{styleEntryParagraph}

\begin{styleEntryParagraph}
\textstyleLexeme{məvəye}\textstylefstandard{   }\textstylePartofspeech{n.} \textstyleSensenumber{\textit{year}}\textstylefstandard{.}
\end{styleEntryParagraph}

\begin{styleEntryParagraph}
\textstyleLexeme{məwta}\textstylefstandard{   }\textstylePartofspeech{n.} \textstyleDefinitionn{truck}\textstylefstandard{.}
\end{styleEntryParagraph}

\begin{styleEntryParagraph}
\textstyleLexeme{məyek}\textstylefstandard{   }\textstylePartofspeech{n.} \textstyleDefinitionn{deer}\textstylefstandard{.}
\end{styleEntryParagraph}

\begin{styleEntryParagraph}
\textstyleLexeme{məze}\textstylefstandard{   }\textstylePartofspeech{n.} \textstyleDefinitionn{person}
\end{styleEntryParagraph}

\begin{styleEntryParagraph}
\textstyleLexeme{məzlelem}\textstylefstandard{   }\textstylePartofspeech{n.} \textstyleDefinitionn{trumpet}\textstylefstandard{.}
\end{styleEntryParagraph}

\begin{styleEntryParagraph}
\textstyleLexeme{mogo}\textstylefstandard{   }\textstylePartofspeech{n.} \textstyleDefinitionn{anger}\textstylefstandard{.}
\end{styleEntryParagraph}

\begin{styleEntryParagraph}
\textstyleLexeme{mogodok}\textstylefstandard{   }\textstylePartofspeech{n.} \textstyleDefinitionn{hawk}\textstylefstandard{.}
\end{styleEntryParagraph}

\begin{styleEntryParagraph}
\textstyleLexeme{mogom}\textstylefstandard{   }\textstylePartofspeech{n.} \textstyleDefinitionn{house, home}\textstylefstandard{.}
\end{styleEntryParagraph}

\begin{styleEntryParagraph}
\textstyleLexeme{Mokəyo}\textstylefstandard{   }\textstylePartofspeech{n.pr.}\textstyleDefinitionn{ Market day of the village of Mok}\textstyleLexeme{\textmd{\textit{əyo}}}\textstylefstandard{.}
\end{styleEntryParagraph}

\begin{styleEntryParagraph}
\textstyleLexeme{moktonok}\textstylefstandard{   }\textstylePartofspeech{n.} \textstyleDefinitionn{toad}\textstylefstandard{.}
\end{styleEntryParagraph}

\begin{styleEntryParagraph}
\textstyleLexeme{molo}\textstylefstandard{   }\textstylePartofspeech{n.} \textstyleDefinitionn{vulture}\textstylefstandard{.}
\end{styleEntryParagraph}

\begin{styleEntryParagraph}
\textstyleLexeme{molo}\textstylefstandard{   }\textstylePartofspeech{n.} \textstyleDefinitionn{twin}\textstylefstandard{.}
\end{styleEntryParagraph}

\begin{styleEntryParagraph}
\textstyleLexeme{molom}\textstylefstandard{   }\textstylePartofspeech{n.} \textstyleDefinitionn{home market day}
\end{styleEntryParagraph}

\begin{styleEntryParagraph}
\textstyleLexeme{mombərkotok}\textstylefstandard{   }\textstylePartofspeech{n.} \textstyleDefinitionn{fish}
\end{styleEntryParagraph}

\begin{styleEntryParagraph}
\textstyleLexeme{mongom}\textstylefstandard{   }\textstylePartofspeech{n.}\textstyleDefinitionn{horn}
\end{styleEntryParagraph}

\begin{styleEntryParagraph}
\textstyleLexeme{mongoro}\textstylefstandard{   }\textstylePartofspeech{n.} \textstyleDefinitionn{mango}\textstylefstandard{.}
\end{styleEntryParagraph}

\begin{styleEntryParagraph}
\textstyleLexeme{morkoyo}\textstylefstandard{   }\textstylePartofspeech{n.} \textstyleDefinitionn{oldest child.}
\end{styleEntryParagraph}

\begin{styleEntryParagraph}
\textstyleLexeme{mosokoy}\textstylefstandard{   }\textstylePartofspeech{n.} \textstyleDefinitionn{vegetable sauce}\textstylefstandard{.}
\end{styleEntryParagraph}

\begin{styleEntryParagraph}
\textstyleLexeme{mozongo}\textstylefstandard{   }\textstylePartofspeech{n.} \textstyleDefinitionn{chameleon}\textstylefstandard{.}
\end{styleEntryParagraph}
\end{multicols}
\begin{styleLetterParagraph}
\textstyleLetterv{N  {}-  n}
\end{styleLetterParagraph}

\begin{multicols}{2}
\begin{styleEntryParagraph}
\textstyleLexeme{n\nobreakdash-}\textstylefstandard{   }\textstylePartofspeech{vpfx.} \textstyleDefinitionn{1S subject}\textstylefstandard{.}
\end{styleEntryParagraph}

\begin{styleEntryParagraph}
\textstyleLexeme{na}\textstylefstandard{   }\textstylePartofspeech{disc.} \textstyleDefinitionn{presupposition marker}\textstylefstandard{.}
\end{styleEntryParagraph}

\begin{styleEntryParagraph}
\textstyleLexeme{na}\textstylefstandard{   }\textstylefstandard{\textit{vclitic}}\textstylePartofspeech{.} \textstyleDefinitionn{3S direct obj}\textstylefstandard{.}
\end{styleEntryParagraph}

\begin{styleEntryParagraph}
\textstyleLexeme{nə}\textstylefstandard{   }\textstylePartofspeech{conj.} \textstyleDefinitionn{with}\textstylefstandard{.}
\end{styleEntryParagraph}

\begin{styleEntryParagraph}
\textstyleLexeme{nah}\textstylefstandard{   }\textstylePartofspeech{v.} \textstyleDefinitionn{ripen}\textstylefstandard{.}
\end{styleEntryParagraph}

\begin{styleEntryParagraph}
\textstyleLexeme{nata}\textstylefstandard{   }\textstylePartofspeech{conj.} \textstyleDefinitionn{and ; marks the climactic moment in a narrative.}
\end{styleEntryParagraph}

\begin{styleEntryParagraph}
\textstyleLexeme{nday}\textstylefstandard{   }\textstylePartofspeech{v.}\textstyleDefinitionn{be in process of}\textstylefstandard{.}
\end{styleEntryParagraph}

\begin{styleEntryParagraph}
\textstyleLexeme{ndaɓay}\textstylefstandard{   }\textstylePartofspeech{v.} \textstyleDefinitionn{wet, whip.}
\end{styleEntryParagraph}

\begin{styleEntryParagraph}
\textstyleLexeme{ndaɗay}\textstylefstandard{   }\textstylePartofspeech{v.} \textstyleDefinitionn{want, love}\textstylefstandard{.}
\end{styleEntryParagraph}

\begin{styleEntryParagraph}
\textstyleLexeme{ndahay}\textstylefstandard{   }\textstylePartofspeech{v.} \textstyleDefinitionn{reprimand, scold}\textstylefstandard{.}
\end{styleEntryParagraph}

\begin{styleEntryParagraph}
\textstyleLexeme{ndahan}\textstylefstandard{   }\textstylePartofspeech{pn.} \textstyleDefinitionn{3S}\textstylefstandard{.}
\end{styleEntryParagraph}

\begin{styleEntryParagraph}
\textstyleLexeme{ndam}\textstylefstandard{   }\textstylePartofspeech{n.} \textstyleDefinitionn{people}\textstylefstandard{.}
\end{styleEntryParagraph}

\begin{styleEntryParagraph}
\textstyleLexeme{ndana}\textstylefstandard{   }\textstylePartofspeech{dem. }\textstyleDefinitionn{this}
\end{styleEntryParagraph}

\begin{styleEntryParagraph}
\textstyleLexeme{ndar}\textstylefstandard{   }\textstylePartofspeech{v.} \textstyleDefinitionn{weave}\textstylefstandard{.}
\end{styleEntryParagraph}

\begin{styleEntryParagraph}
\textstyleLexeme{ndavay}\textstylefstandard{   }\textstylePartofspeech{v.} \textstyleSensenumber{\textit{fi}}\textstyleDefinitionn{nish}\textstylefstandard{.} 
\end{styleEntryParagraph}

\begin{styleEntryParagraph}
\textstyleLexeme{ndawan}\textstylefstandard{   }\textstylePartofspeech{adv. }\textstyleDefinitionn{maybe}\textstylefstandard{.}
\end{styleEntryParagraph}

\begin{styleEntryParagraph}
\textstyleLexeme{ndaway}\textstylefstandard{   }\textstylePartofspeech{v.} \textstyleDefinitionn{insult}\textstylefstandard{.}
\end{styleEntryParagraph}

\begin{styleEntryParagraph}
\textstyleLexeme{ndaway}\textstylefstandard{   }\textstylePartofspeech{v.} \textstyleDefinitionn{swallow}\textstylefstandard{.}
\end{styleEntryParagraph}

\begin{styleEntryParagraph}
\textstyleLexeme{ndaz}\textstylefstandard{   }\textstylePartofspeech{v.} \textstyleDefinitionn{kill by piercing}\textstylefstandard{.}
\end{styleEntryParagraph}

\begin{styleEntryParagraph}
\textstyleLexeme{nde}\textstylefstandard{   }\textstylePartofspeech{v.} \textstyleDefinitionn{lie down}\textstylefstandard{.} 
\end{styleEntryParagraph}

\begin{styleEntryParagraph}
\textstyleLexeme{nde}\textstylefstandard{   }\textstylePartofspeech{conj.} \textstyleDefinitionn{therefore}\textstylefstandard{.}
\end{styleEntryParagraph}

\begin{styleEntryParagraph}
\textstyleLexeme{ndeslen}\textstylefstandard{   }\textstylePartofspeech{v.} \textstyleDefinitionn{make cold}\textstylefstandard{.}
\end{styleEntryParagraph}

\begin{styleEntryParagraph}
\textstyleLexeme{ndəlkaday}\textstylefstandard{   }\textstylePartofspeech{v.} \textstyleDefinitionn{lick}\textstylefstandard{.}
\end{styleEntryParagraph}

\begin{styleEntryParagraph}
\textstyleLexeme{nd}\textstyleLexeme{ə}\textstyleLexeme{n nden}\textstylefstandard{   }\textstylePartofspeech{n.} \textstyleDefinitionn{traditional sword}\textstylefstandard{.}
\end{styleEntryParagraph}

\begin{styleEntryParagraph}
\textstyleLexeme{ndəray}\textstylefstandard{   }\textstylePartofspeech{v. }\textstyleDefinitionn{stay, leave}\textstylefstandard{.}
\end{styleEntryParagraph}

\begin{styleEntryParagraph}
\textstyleLexeme{ndərdoy}\textstylefstandard{   }\textstylePartofspeech{v. }\textstyleDefinitionn{stretch}\textstylefstandard{.}
\end{styleEntryParagraph}

\begin{styleEntryParagraph}
\textstyleLexeme{ndozlay}\textstylefstandard{   }\textstylePartofspeech{v.} \textstyleDefinitionn{explode}\textstylefstandard{.}
\end{styleEntryParagraph}

\begin{styleEntryParagraph}
\textstyleLexeme{ne}\textstylefstandard{   }\textstylePartofspeech{pn.} \textstyleDefinitionn{1S}\textstylefstandard{.} 
\end{styleEntryParagraph}

\begin{styleEntryParagraph}
\textstyleLexeme{nehe}\textstylefstandard{   }\textstylePartofspeech{dem.} \textstyleDefinitionn{here}\textstylefstandard{.}
\end{styleEntryParagraph}

\begin{styleEntryParagraph}
\textstyleLexeme{nekwen}\textstylefstandard{   }\textstylePartofspeech{quant. a } \textstyleDefinitionn{small amount}\textstylefstandard{.}
\end{styleEntryParagraph}

\begin{styleEntryParagraph}
\textstyleLexeme{nendəye}\textstylefstandard{   }\textstylePartofspeech{dem. there.}
\end{styleEntryParagraph}

\begin{styleEntryParagraph}
\textstyleLexeme{nəngehe}\textstylefstandard{   }\textstylePartofspeech{dem. there.}
\end{styleEntryParagraph}

\begin{styleEntryParagraph}
\textstyleLexeme{ngala}\textstylefstandard{   }\textstylePartofspeech{v.} \textit{co}\textstyleDefinitionn{me back.}
\end{styleEntryParagraph}

\begin{styleEntryParagraph}
\textstyleLexeme{ngama}\textstylefstandard{   }\textstylePartofspeech{adv.} \textstyleDefinitionn{better}\textstylefstandard{.}
\end{styleEntryParagraph}

\begin{styleEntryParagraph}
\textstyleLexeme{ngar}\textstylefstandard{   }\textstylePartofspeech{v.} \textstyleDefinitionn{prevent}\textstylefstandard{.}
\end{styleEntryParagraph}

\begin{styleEntryParagraph}
\textstyleLexeme{ngaray}\textstylefstandard{   }\textstylePartofspeech{v.} \textstyleDefinitionn{rip}\textstylefstandard{.}
\end{styleEntryParagraph}

\begin{styleEntryParagraph}
\textstyleLexeme{ngay}\textstylefstandard{   }\textstylePartofspeech{v.} \textstyleDefinitionn{set, work with wood or grasses}\textstylefstandard{.}
\end{styleEntryParagraph}

\begin{styleEntryParagraph}
\textstyleLexeme{ngaz}\textstylefstandard{   }\textstylePartofspeech{v.} \textstyleDefinitionn{flow, leak}\textstylefstandard{.}
\end{styleEntryParagraph}

\begin{styleEntryParagraph}
\textstyleLexeme{ngazlay}\textstylefstandard{   }\textstylePartofspeech{v.} \textstyleDefinitionn{introduce}\textstylefstandard{.}
\end{styleEntryParagraph}

\begin{styleEntryParagraph}
\textstyleLexeme{ngəɗacay}\textstylefstandard{   }\textstylePartofspeech{v.} \textstyleDefinitionn{butt with horns}\textstylefstandard{.}
\end{styleEntryParagraph}

\begin{styleEntryParagraph}
\textstyleLexeme{ngəɗay}\textstylefstandard{   }\textstylePartofspeech{v.}\textstyleDefinitionn{burn}\textstylefstandard{.}
\end{styleEntryParagraph}

\begin{styleEntryParagraph}
\textstyleLexeme{ngəhe}\textstylefstandard{   }\textstylePartofspeech{dem.} \textstyleDefinitionn{this particular one here}\textstylefstandard{.}
\end{styleEntryParagraph}

\begin{styleEntryParagraph}
\textstyleLexeme{ngəm\nobreakdash-ngam}\textstylefstandard{   }\textstylePartofspeech{n.} \textstyleDefinitionn{mouse trap}\textstylefstandard{.}
\end{styleEntryParagraph}

\begin{styleEntryParagraph}
\textstyleLexeme{ngəlay}\textstylefstandard{   }\textstylePartofspeech{v.} \textstyleDefinitionn{defend}\textstylefstandard{.}
\end{styleEntryParagraph}

\begin{styleEntryParagraph}
\textstyleLexeme{ngəlday}\textstylefstandard{   }\textstylePartofspeech{v. grind (peanuts).}
\end{styleEntryParagraph}

\begin{styleEntryParagraph}
\textstyleLexeme{ngərɗasay}\textstylefstandard{   }\textstylePartofspeech{v.} \textstyleDefinitionn{wrinkle the skin}\textstylefstandard{.}
\end{styleEntryParagraph}

\begin{styleEntryParagraph}
\textstyleLexeme{ngərkaka}\textstylefstandard{   }\textstylePartofspeech{n.} \textstyleDefinitionn{heron}\textstylefstandard{.}
\end{styleEntryParagraph}

\begin{styleEntryParagraph}
\textstyleLexeme{ngərway}\textstylefstandard{   }\textstylePartofspeech{v.} \textstyleDefinitionn{break, tear away}\textstylefstandard{.}
\end{styleEntryParagraph}

\begin{styleEntryParagraph}
\textstyleLexeme{ngərzlay}\textstylefstandard{   }\textstylePartofspeech{v.} \textstyleDefinitionn{be in conflict}\textstylefstandard{.} 
\end{styleEntryParagraph}

\begin{styleEntryParagraph}
\textstyleLexeme{ngəvəray}\textstylefstandard{   }\textstylePartofspeech{n.} \textstyleDefinitionn{tree}\textstylefstandard{.}
\end{styleEntryParagraph}

\begin{styleEntryParagraph}
\textstyleLexeme{ngomna}\textstylefstandard{   }\textstylePartofspeech{n.} \textstyleDefinitionn{government}\textstylefstandard{.}
\end{styleEntryParagraph}

\begin{styleEntryParagraph}
\textstyleLexeme{ngwəɗaslay}\textstylefstandard{   }\textstylePartofspeech{v.} \textstyleDefinitionn{simmer}\textstylefstandard{.}
\end{styleEntryParagraph}

\begin{styleEntryParagraph}
\textstyleLexeme{njahay}\textstylefstandard{   }\textstylePartofspeech{v.} \textstyleDefinitionn{roast}\textstylefstandard{.}
\end{styleEntryParagraph}

\begin{styleEntryParagraph}
\textstyleLexeme{njakay}\textstylefstandard{   }\textstylePartofspeech{v.}\textstyleSensenumber{~}\textstyleDefinitionn{find.}
\end{styleEntryParagraph}

\begin{styleEntryParagraph}
\textstyleLexeme{njaray}\textstylefstandard{   }\textstylePartofspeech{v.}\textstyleSensenumber{~}\textstyleDefinitionn{comb, separate}\textstylefstandard{.}
\end{styleEntryParagraph}

\begin{styleEntryParagraph}
\textstyleLexeme{njavar}\textstylefstandard{   }\textstylePartofspeech{n.} \textstyleDefinitionn{young man over 18}\textstylefstandard{.}
\end{styleEntryParagraph}

\begin{styleEntryParagraph}
\textstyleLexeme{nje}\textstylefstandard{   }\textstylePartofspeech{v.} \textstyleDefinitionn{leave}\textstylefstandard{.}
\end{styleEntryParagraph}

\begin{styleEntryParagraph}
\textstyleLexeme{nje}\textstylefstandard{   }\textstylePartofspeech{v.} \textstyleDefinitionn{sit, suffice}\textstylefstandard{.}
\end{styleEntryParagraph}

\begin{styleEntryParagraph}
\textstyleLexeme{njeren}\textstylefstandard{   }\textstylePartofspeech{v.} \textstyleDefinitionn{groan}\textstylefstandard{.}
\end{styleEntryParagraph}

\begin{styleEntryParagraph}
\textstyleLexeme{njəɗa}\textstylefstandard{   }\textstylePartofspeech{n.} \textstyleDefinitionn{power}\textstylefstandard{.}
\end{styleEntryParagraph}

\begin{styleEntryParagraph}
\textstyleLexeme{njəl njəl}\textstylefstandard{   }\textstylePartofspeech{ID.} \textstyleDefinitionn{sight/idea of youth running}\textstylefstandard{.}
\end{styleEntryParagraph}

\begin{styleEntryParagraph}
\textstyleLexeme{njəɗok njəɗok}\textstylefstandard{   }\textstylePartofspeech{ID.} \textstyleDefinitionn{sight/idea of child running}\textstylefstandard{.}
\end{styleEntryParagraph}

\begin{styleEntryParagraph}
\textstyleLexeme{njəwelek}\textstylefstandard{   }\textstylePartofspeech{n.} \textstyleDefinitionn{leaf for making a sauce}\textstylefstandard{.}
\end{styleEntryParagraph}

\begin{styleEntryParagraph}
\textstyleLexeme{njəw njəw njəw}\textstylefstandard{  }\textstylePartofspeech{ID.} \textstyleDefinitionn{idea of grinding}\textstylefstandard{.}
\end{styleEntryParagraph}

\begin{styleEntryParagraph}
\textstyleLexeme{nok}\textstylefstandard{   }\textstylePartofspeech{pn.} \textstyleDefinitionn{2S}\textstylefstandard{.}
\end{styleEntryParagraph}
\end{multicols}
\begin{styleLetterParagraph}
\textstyleLetterv{O  {}-  o}
\end{styleLetterParagraph}

\begin{multicols}{2}
\begin{styleEntryParagraph}
\textstyleLexeme{obor}\textstylefstandard{   }\textstylePartofspeech{n.} \textstyleDefinitionn{lust} 
\end{styleEntryParagraph}

\begin{styleEntryParagraph}
\textstyleLexeme{oɓolo}\textstylefstandard{   }\textstylePartofspeech{n.} \textstyleDefinitionn{yam}\textstylefstandard{.}
\end{styleEntryParagraph}

\begin{styleEntryParagraph}
\textstyleLexeme{ocom}\textstylefstandard{   }\textstylePartofspeech{n.} \textstyleDefinitionn{hyrax}\textstylefstandard{.}
\end{styleEntryParagraph}

\begin{styleEntryParagraph}
\textstyleLexeme{ogəro}\textstylefstandard{   }\textstylePartofspeech{n.} \textstyleDefinitionn{gold}\textstylefstandard{.}
\end{styleEntryParagraph}

\begin{styleEntryParagraph}
\textstyleLexeme{ok}\textstylefstandard{   }\textstylefstandard{\textit{v}}\textstylePartofspeech{clitic.}\textstyleDefinitionn{ 2S indirect object}\textstylefstandard{.}
\end{styleEntryParagraph}

\begin{styleEntryParagraph}
\textstyleLexeme{{}-ok}\textstylefstandard{   }\textstylefstandard{\textit{v}}\textstylePartofspeech{sfx. 1P}\textstylePartofspeech{IN}\textstylePartofspeech{,}\textstyleDefinitionn{ 2P subject}\textstylefstandard{.}
\end{styleEntryParagraph}

\begin{styleEntryParagraph}
\textstyleLexeme{aw}\textstylefstandard{\textit{   vclitic}}\textstylePartofspeech{. }\textstyleDefinitionn{1S indirect object}\textstylefstandard{.}
\end{styleEntryParagraph}

\begin{styleEntryParagraph}
\textstyleLexeme{okfom}\textstylefstandard{   }\textstylePartofspeech{n.} \textstyleDefinitionn{mouse}\textstylefstandard{.}
\end{styleEntryParagraph}

\begin{styleEntryParagraph}
\textstyleLexeme{oko}\textstylefstandard{   }\textstylePartofspeech{n.} \textstyleDefinitionn{fire}\textstylefstandard{.}
\end{styleEntryParagraph}

\begin{styleEntryParagraph}
\textstyleLexeme{okor}\textstylefstandard{   }\textstylePartofspeech{n.} \textstyleDefinitionn{rock}\textstylefstandard{.}
\end{styleEntryParagraph}

\begin{styleEntryParagraph}
\textstyleLexeme{okos}\textstylefstandard{   }\textstylePartofspeech{n.} \textstyleDefinitionn{fat}\textstylefstandard{.}
\end{styleEntryParagraph}

\begin{styleEntryParagraph}
\textstyleLexeme{oloko}\textstylefstandard{   }\textstylePartofspeech{n.} \textstyleDefinitionn{wood}\textstylefstandard{.}
\end{styleEntryParagraph}

\begin{styleEntryParagraph}
\textstyleLexeme{\nobreakdash-om}\textstylefstandard{   }\textstylePartofspeech{vsfx.} \textstyleDefinitionn{1P}\textstyleDefinitionn{EX}\textstyleDefinitionn{/2P subject}\textstylefstandard{.}
\end{styleEntryParagraph}

\begin{styleEntryParagraph}
\textstyleLexeme{omboɗoc}\textstylefstandard{   }\textstylePartofspeech{n.} \textstyleDefinitionn{sugar cane}\textstylefstandard{.}
\end{styleEntryParagraph}

\begin{styleEntryParagraph}
\textstyleLexeme{ombolo}\textstylefstandard{   }\textstylePartofspeech{n.} \textstyleDefinitionn{sack; thousand francs.}
\end{styleEntryParagraph}

\begin{styleEntryParagraph}
\textstyleLexeme{omom}\textstylefstandard{   }\textstylePartofspeech{n.} \textstyleDefinitionn{honey}\textstylefstandard{.}
\end{styleEntryParagraph}

\begin{styleEntryParagraph}
\textstyleLexeme{  }\textstylefvernacular{war omom }\textstylefvernacular{\textmd{\textit{n.}}}\textstylefvernacular{ }\textstylefvernacular{\textmd{\textit{bee}}}\textstylefstandard{.}
\end{styleEntryParagraph}

\begin{styleEntryParagraph}
\textstyleLexeme{omsoko}\textstylefstandard{   }\textstylePartofspeech{n.} \textit{sorghum}, \textstyleDefinitionn{dry season millet}\textstylefstandard{.}
\end{styleEntryParagraph}

\begin{styleEntryParagraph}
\textstyleLexeme{ongolo}\textstylefstandard{   }\textstylePartofspeech{n.} \textstyleDefinitionn{liar}\textstylefstandard{.}
\end{styleEntryParagraph}

\begin{styleEntryParagraph}
\textstyleLexeme{opongo}\textstylefstandard{   }\textstylePartofspeech{n.} \textstyleDefinitionn{mushroom}\textstylefstandard{.}
\end{styleEntryParagraph}

\begin{styleEntryParagraph}
\textstyleLexeme{oroh}\textstylefstandard{   }\textstylePartofspeech{n.} \textstyleDefinitionn{pus}\textstylefstandard{.}
\end{styleEntryParagraph}

\begin{styleEntryParagraph}
\textstyleLexeme{orov}\textstylefstandard{   }\textstylePartofspeech{n.} \textstyleDefinitionn{thorny tree}\textstylefstandard{.}
\end{styleEntryParagraph}

\begin{styleEntryParagraph}
\textstyleLexeme{otos}\textstylefstandard{   }\textstylePartofspeech{n.} \textstyleDefinitionn{hedgehog}\textstylefstandard{.}
\end{styleEntryParagraph}

\begin{styleEntryParagraph}
\textstyleLexeme{ovolom}\textstylefstandard{   }\textstylePartofspeech{n.} \textstyleDefinitionn{ladle}\textstylefstandard{.}
\end{styleEntryParagraph}

\begin{styleEntryParagraph}
\textstyleLexeme{ozəngo}\textstylefstandard{   }\textstylePartofspeech{n.} \textstyleDefinitionn{donkey}\textstylefstandard{.}
\end{styleEntryParagraph}

\begin{styleEntryParagraph}
\textstyleLexeme{ozlərgo}\textstylefstandard{   }\textstylePartofspeech{n.} \textstyleDefinitionn{axe}\textstylefstandard{.}
\end{styleEntryParagraph}
\end{multicols}
\begin{styleLetterParagraph}
\textstyleLetterv{P  {}-  p}
\end{styleLetterParagraph}

\begin{multicols}{2}
\begin{styleEntryParagraph}
\textstyleLexeme{paɗay}\textstylefstandard{   }\textstylePartofspeech{v.}\textstyleDefinitionn{crunch}\textstylefstandard{.}
\end{styleEntryParagraph}

\begin{styleEntryParagraph}
\textstyleLexeme{pahay}\textstylefstandard{   }\textstylePartofspeech{v.}\textstyleDefinitionn{speak badly of someone for one’s own interest}\textstylefstandard{.}
\end{styleEntryParagraph}

\begin{styleEntryParagraph}
\textstyleLexeme{pahav}\textstylefstandard{   }\textstylePartofspeech{n.} \textstyleDefinitionn{lungs}\textstylefstandard{.}
\end{styleEntryParagraph}

\begin{styleEntryParagraph}
\textstyleLexeme{palay}\textstylefstandard{   }\textstylePartofspeech{v. }\textstyleDefinitionn{choose}\textstylefstandard{.}
\end{styleEntryParagraph}

\begin{styleEntryParagraph}
\textstyleLexeme{pamay}\textstylefstandard{   }\textstylePartofspeech{v. }\textstyleDefinitionn{fan.}
\end{styleEntryParagraph}

\begin{styleEntryParagraph}
\textstyleLexeme{par}\textstylefstandard{   }\textstylePartofspeech{v.} \textstyleDefinitionn{pay}\textstylefstandard{.}
\end{styleEntryParagraph}

\begin{styleEntryParagraph}
\textstyleLexeme{pasay}\textstylefstandard{   }\textstylePartofspeech{v.} \textstyleDefinitionn{detach, spread out}\textstylefstandard{.}
\end{styleEntryParagraph}

\begin{styleEntryParagraph}
\textstyleLexeme{pasl}\textstylefstandard{   }\textstylePartofspeech{v.} \textstyleDefinitionn{break}\textstylefstandard{.}
\end{styleEntryParagraph}

\begin{styleEntryParagraph}
\textstyleLexeme{pat}\textstylefstandard{   }\textstylePartofspeech{adv.} \textstyleDefinitionn{all}\textstylefstandard{.}
\end{styleEntryParagraph}

\begin{styleEntryParagraph}
\textstyleLexeme{Patatah}\textstylefstandard{   }\textstylePartofspeech{n.} \textstyleDefinitionn{Wednesday market}\textstylefstandard{.}
\end{styleEntryParagraph}

\begin{styleEntryParagraph}
\textstyleLexeme{pataw}\textstylefstandard{   }\textstylePartofspeech{n.} \textstyleDefinitionn{cat}\textstylefstandard{.}
\end{styleEntryParagraph}

\begin{styleEntryParagraph}
\textstyleLexeme{patay}\textstylefstandard{   }\textstylePartofspeech{v.} \textstyleDefinitionn{wipe, rub}\textstylefstandard{.}
\end{styleEntryParagraph}

\begin{styleEntryParagraph}
\textstyleLexeme{pay}\textstylefstandard{   }\textstylePartofspeech{v.} \textstyleDefinitionn{open}\textstylefstandard{.}
\end{styleEntryParagraph}

\begin{styleEntryParagraph}
\textstyleLexeme{pazlay}\textstylefstandard{   }\textstylePartofspeech{v.} \textstyleDefinitionn{decimate, kill many}\textstylefstandard{.}
\end{styleEntryParagraph}

\begin{styleEntryParagraph}
\textstyleLexeme{peɗeɗe}\textstylefstandard{   }\textstylefstandard{\textit{ID.}} \textstyleDefinitionn{fullness.}
\end{styleEntryParagraph}

\begin{styleEntryParagraph}
\textstyleLexeme{peɗewk}\textstylefstandard{   }\textstylePartofspeech{n.} \textstyleDefinitionn{razor}\textstylefstandard{.}
\end{styleEntryParagraph}

\begin{styleEntryParagraph}
\textstyleLexeme{pembez}\textstylefstandard{   }\textstylePartofspeech{n.} \textstyleDefinitionn{blood}\textstylefstandard{.}
\end{styleEntryParagraph}

\begin{styleEntryParagraph}
\textstyleLexeme{pepen}\textstylefstandard{   }\textstylePartofspeech{n. }\textstyleDefinitionn{immediately}\textstylefstandard{.}
\end{styleEntryParagraph}

\begin{styleIndentedParagraph}
\textstyleSubentry{pepenna}\textstylefstandard{   }\textstyleSensenumber{\textit{adv}}\textstyleSensenumber{. }\textstylePartofspeech{long ago}\textstylefstandard{.} 
\end{styleIndentedParagraph}

\begin{styleEntryParagraph}
\textstyleLexeme{pew}\textstylefstandard{   }\textstylePartofspeech{adv.} \textstyleDefinitionn{enough}\textstylefstandard{.}
\end{styleEntryParagraph}

\begin{styleEntryParagraph}
\textstyleLexeme{pəcahay}\textstylefstandard{   }\textstylePartofspeech{v.} \textstyleDefinitionn{remove insides}\textstylefstandard{.}
\end{styleEntryParagraph}

\begin{styleEntryParagraph}
\textstyleLexeme{pəcay}\textstylefstandard{   }\textstylePartofspeech{v.} \textstyleDefinitionn{bring}\textstylefstandard{.}
\end{styleEntryParagraph}

\begin{styleEntryParagraph}
\textstyleLexeme{pəcəkəɗək}\textstylefstandard{   }\textstylePartofspeech{ID.} \textstyleDefinitionn{sight/idea of a toad hopping}\textstylefstandard{.}
\end{styleEntryParagraph}

\begin{styleEntryParagraph}
\textstyleLexeme{pəɗakay}\textstylefstandard{   }\textstylePartofspeech{v.} \textstyleDefinitionn{wake up}\textstylefstandard{.}
\end{styleEntryParagraph}

\begin{styleEntryParagraph}
\textstyleLexeme{pəɗakay}\textstylefstandard{   }\textstylePartofspeech{v.} \textstyleDefinitionn{chop}\textstylefstandard{.}
\end{styleEntryParagraph}

\begin{styleEntryParagraph}
\textstyleLexeme{pəɗak}\textstylefstandard{   }\textstylePartofspeech{v.} \textstyleDefinitionn{melt}\textstylefstandard{.}
\end{styleEntryParagraph}

\begin{styleEntryParagraph}
\textstyleLexeme{pəɗe}\textstylefstandard{   }\textstylePartofspeech{n.} \textstyleDefinitionn{hole}\textstylefstandard{.}
\end{styleEntryParagraph}

\begin{styleEntryParagraph}
\textstyleLexeme{pək}\textstylefstandard{  }\textstylePartofspeech{ID. }\textstyleDefinitionn{sound/idea of bottle opening}\textstylefstandard{.}
\end{styleEntryParagraph}

\begin{styleEntryParagraph}
\textstyleLexeme{pəlɗay}\textstylefstandard{   }\textstylePartofspeech{v.} \textstyleDefinitionn{shell}\textstylefstandard{.}
\end{styleEntryParagraph}

\begin{styleEntryParagraph}
\textstyleLexeme{pəlslay}\textstylefstandard{   }\textstylePartofspeech{v. }\textstyleDefinitionn{split in half}\textstylefstandard{.}
\end{styleEntryParagraph}

\begin{styleEntryParagraph}
\textstyleLexeme{pəles}\textstylefstandard{   }\textstylePartofspeech{n.} \textstyleDefinitionn{horse}\textstylefstandard{.}
\end{styleEntryParagraph}

\begin{styleEntryParagraph}
\textstyleLexeme{pəra}\textstylefstandard{   }\textstylePartofspeech{n.} \textstyleDefinitionn{spirit, idol}\textstylefstandard{.}
\end{styleEntryParagraph}

\begin{styleEntryParagraph}
\textstyleLexeme{pəraɗ}\textstylefstandard{   }\textstylePartofspeech{n.} \textstyleDefinitionn{large rock}\textstylefstandard{.}
\end{styleEntryParagraph}

\begin{styleEntryParagraph}
\textstyleLexeme{pəray}\textstylefstandard{   }\textstylePartofspeech{v.} \textstyleDefinitionn{spray}\textstylefstandard{.}
\end{styleEntryParagraph}

\begin{styleEntryParagraph}
\textstyleLexeme{pərgom}\textstylefstandard{   }\textstylePartofspeech{n.} \textstyleDefinitionn{trap}\textstylefstandard{.}
\end{styleEntryParagraph}

\begin{styleEntryParagraph}
\textstyleLexeme{pərtay}\textstylefstandard{   }\textstylePartofspeech{v.} \textstyleDefinitionn{remove forcibly}\textstylefstandard{.}
\end{styleEntryParagraph}

\begin{styleEntryParagraph}
\textstyleLexeme{pəsakay}\textstylefstandard{   }\textstylePartofspeech{v.} \textstyleDefinitionn{detach}\textstylefstandard{.}
\end{styleEntryParagraph}

\begin{styleEntryParagraph}
\textstyleLexeme{pəvban}\textstylefstandard{   }\textstylefstandard{\textit{ID.}}\textstyleDefinitionn{start of a race}\textstylefstandard{.}
\end{styleEntryParagraph}

\begin{styleEntryParagraph}
\textstyleLexeme{pəvbəw pəvbəw}\textstylefstandard{   }\textstylefstandard{\textit{ID.}}\textstyleDefinitionn{ sight/idea of rabbit hopping}\textstylefstandard{.}
\end{styleEntryParagraph}

\begin{styleEntryParagraph}
\textstyleLexeme{pəyecece}\textstylefstandard{   }\textstylefstandard{\textit{ID.}} \textstyleDefinitionn{coldness}\textstylefstandard{.}
\end{styleEntryParagraph}

\begin{styleEntryParagraph}
\textstyleLexeme{pəyteɗ}\textstylefstandard{   }\textstylefstandard{\textit{ID.}} \textstyleDefinitionn{idea of barely escaping}\textstylefstandard{.}
\end{styleEntryParagraph}

\begin{styleEntryParagraph}
\textstyleLexeme{pok}\textstylefstandard{   }\textstylePartofspeech{ID. }\textstyleDefinitionn{idea of opening door.}
\end{styleEntryParagraph}

\begin{styleEntryParagraph}
\textstyleLexeme{pocoy}\textstylefstandard{   }\textstylePartofspeech{v.} \textstyleDefinitionn{wear small leather article of clothing}\textstylefstandard{.}
\end{styleEntryParagraph}

\begin{styleEntryParagraph}
\textstyleLexeme{poɗococo}\textstylefstandard{   }\textstylefstandard{\textit{ID.}} \textstyleDefinitionn{sweetness}\textstylefstandard{.}
\end{styleEntryParagraph}

\begin{styleEntryParagraph}
\textstyleLexeme{poloy}\textstylefstandard{   }\textstylePartofspeech{v.} \textstyleDefinitionn{scatter}\textstylefstandard{.}
\end{styleEntryParagraph}
\end{multicols}
\begin{styleLetterParagraph}
\textstyleLetterv{R  {}-  r}
\end{styleLetterParagraph}

\begin{multicols}{2}
\begin{styleEntryParagraph}
\textstyleLexeme{rah}\textstylefstandard{   }\textstylePartofspeech{v.} \textit{fill,} \textstyleDefinitionn{satisfy}\textstylefstandard{.}
\end{styleEntryParagraph}

\begin{styleEntryParagraph}
\textstyleLexeme{rah}\textstylefstandard{   }\textstylePartofspeech{v.} \textstyleDefinitionn{pluck}\textstylefstandard{.}
\end{styleEntryParagraph}

\begin{styleEntryParagraph}
\textstyleLexeme{rasay}\textstylefstandard{   }\textstylePartofspeech{v.} \textstyleDefinitionn{minimize}\textstylefstandard{.}
\end{styleEntryParagraph}

\begin{styleEntryParagraph}
\textstyleLexeme{re}\textstylefstandard{   }\textstylePartofspeech{adv.} \textstyleDefinitionn{in spite of}\textstylefstandard{.}
\end{styleEntryParagraph}

\begin{styleEntryParagraph}
\textstyleLexeme{reke}\textstylefstandard{   }\textstylePartofspeech{n.} \textstyleDefinitionn{sugar cane}\textstylefstandard{.}
\end{styleEntryParagraph}

\begin{styleEntryParagraph}
\textstyleLexeme{rəbok}\textstylefstandard{   }\textstylePartofspeech{n.} \textstyleDefinitionn{hiding place.}\textstyleLexeme{ }
\end{styleEntryParagraph}

\begin{styleEntryParagraph}
\textstyleLexeme{rəbok}\textstylefstandard{ }\textstyleLexeme{rəbok}\textstylefstandard{  }\textstylePartofspeech{ID.}\textit{ idea of }\textstyleDefinitionn{hiding.}\textstyleLexeme{ }
\end{styleEntryParagraph}

\begin{styleEntryParagraph}
\textstyleLexeme{rəbokay}\textstylefstandard{   }\textstylePartofspeech{v.} \textstyleDefinitionn{hide.}\textstyleLexeme{ }
\end{styleEntryParagraph}

\begin{styleEntryParagraph}
\textstyleLexeme{rəɓay}\textstylefstandard{   }\textstylePartofspeech{v.}\textstyleDefinitionn{be beautiful}\textstylefstandard{.}
\end{styleEntryParagraph}

\begin{styleEntryParagraph}
\textstyleLexeme{rəcoy}\textstylefstandard{   }\textstylePartofspeech{v.} \textstyleDefinitionn{block up}\textstylefstandard{.}
\end{styleEntryParagraph}
\end{multicols}
\begin{styleLetterParagraph}
\textstyleLetterv{S  {}-  s}
\end{styleLetterParagraph}

\begin{multicols}{2}
\begin{styleEntryParagraph}
\textstyleLexeme{saɓay}\textstylefstandard{   }\textstylePartofspeech{v.} \textstyleDefinitionn{exceed}\textstylefstandard{.}
\end{styleEntryParagraph}

\begin{styleEntryParagraph}
\textstyleLexeme{sahay}\textstylefstandard{   }\textstylePartofspeech{v.} \textstyleDefinitionn{slander}\textstylefstandard{.}
\end{styleEntryParagraph}

\begin{styleEntryParagraph}
\textstyleLexeme{sak}\textstylefstandard{   }\textstylePartofspeech{v.} \textstyleDefinitionn{multiply}\textstylefstandard{.}
\end{styleEntryParagraph}

\begin{styleEntryParagraph}
\textstyleLexeme{sakay}\textstylefstandard{   }\textstylePartofspeech{v.} \textstyleDefinitionn{sift}\textstylefstandard{.}
\end{styleEntryParagraph}

\begin{styleEntryParagraph}
\textstyleLexeme{sar}\textstylefstandard{   }\textstylePartofspeech{v.} \textstyleDefinitionn{know}\textstylefstandard{.}
\end{styleEntryParagraph}

\begin{styleEntryParagraph}
\textstyleLexeme{savah}\textstylefstandard{   }\textstylePartofspeech{n.} \textstyleDefinitionn{rainy season}\textstylefstandard{.}
\end{styleEntryParagraph}

\begin{styleEntryParagraph}
\textstyleLexeme{say}\textstylefstandard{   }\textstylePartofspeech{v.} \textstyleDefinitionn{cut, want}\textstylefstandard{.}
\end{styleEntryParagraph}

\begin{styleEntryParagraph}
\textstyleLexeme{sawan}\textstylefstandard{   }\textstylePartofspeech{adv.} \textstyleDefinitionn{without help}\textstylefstandard{.}
\end{styleEntryParagraph}

\begin{styleEntryParagraph}
\textstyleLexeme{se}\textstylefstandard{   }\textstylePartofspeech{v.} \textstyleDefinitionn{drink}\textstylefstandard{.}
\end{styleEntryParagraph}

\begin{styleEntryParagraph}
\textstyleLexeme{se}\textstyleLexeme{ɓet}\textstyleLexeme{əy}\textstylefstandard{   }\textstylePartofspeech{v. repair.}
\end{styleEntryParagraph}

\begin{styleEntryParagraph}
\textstyleLexeme{sede}\textstylefstandard{   }\textstylePartofspeech{n.} \textstyleDefinitionn{witness}\textstylefstandard{.}
\end{styleEntryParagraph}

\begin{styleEntryParagraph}
\textstyleLexeme{sen}\textstylefstandard{   }\textstylefstandard{\textit{ID.}} \textstyleDefinitionn{idea of going}\textstylefstandard{.}
\end{styleEntryParagraph}

\begin{styleEntryParagraph}
\textstyleLexeme{serəya}\textstylefstandard{   }\textstylePartofspeech{n.} \textstyleDefinitionn{judgement}\textstylefstandard{.}
\end{styleEntryParagraph}

\begin{styleEntryParagraph}
\textstyleLexeme{sese}\textstylefstandard{   }\textstylePartofspeech{n.} \textstyleDefinitionn{meat}\textstylefstandard{.}
\end{styleEntryParagraph}

\begin{styleEntryParagraph}
\textstyleLexeme{səber}\textstylefstandard{   }\textstylePartofspeech{n.} \textstyleDefinitionn{height}\textstylefstandard{.}
\end{styleEntryParagraph}

\begin{styleEntryParagraph}
\textstyleLexeme{səɓatay}\textstylefstandard{   }\textstylePartofspeech{v.} \textstyleDefinitionn{trick, tempt}\textstylefstandard{.}
\end{styleEntryParagraph}

\begin{styleEntryParagraph}
\textstyleLexeme{sədaray}\textstylefstandard{   }\textstylePartofspeech{v.} \textstyleDefinitionn{misbehave}\textstylefstandard{.}
\end{styleEntryParagraph}

\begin{styleEntryParagraph}
\textstyleLexeme{səkar}\textstylefstandard{   }\textstylePartofspeech{n.} \textstyleSensenumber{\textit{spirit being.}}
\end{styleEntryParagraph}

\begin{styleEntryParagraph}
\textstyleLexeme{səkat}\textstylefstandard{   }\textstylePartofspeech{n.} \textstyleDefinitionn{hundred}\textstylefstandard{.}
\end{styleEntryParagraph}

\begin{styleEntryParagraph}
\textstyleLexeme{səkom}\textstylefstandard{   }\textstylePartofspeech{v.} \textstyleDefinitionn{buy/sell}
\end{styleEntryParagraph}

\begin{styleEntryParagraph}
\textstyleLexeme{səkoy}\textstylefstandard{   }\textstylePartofspeech{n.} \textstyleDefinitionn{clan}\textstylefstandard{.}
\end{styleEntryParagraph}

\begin{styleEntryParagraph}
\textstyleLexeme{səlɗay}\textstylefstandard{   }\textstylePartofspeech{v.} \textstyleDefinitionn{cross ankles}\textstylefstandard{.}
\end{styleEntryParagraph}

\begin{styleEntryParagraph}
\textstyleLexeme{səlek}\textstylefstandard{   }\textstylePartofspeech{n.} \textstyleDefinitionn{jealousy}\textstylefstandard{.}
\end{styleEntryParagraph}

\begin{styleEntryParagraph}
\textstyleLexeme{səlewk}\textstylefstandard{   }\textstylePartofspeech{n.} \textstyleDefinitionn{broom}\textstylefstandard{.}
\end{styleEntryParagraph}

\begin{styleEntryParagraph}
\textstyleLexeme{səlom}\textstylefstandard{   }\textstylePartofspeech{n.} \textstyleDefinitionn{good}\textstylefstandard{.}
\end{styleEntryParagraph}

\begin{styleEntryParagraph}
\textstyleLexeme{səloy}\textstylefstandard{   }\textstylePartofspeech{n.}\textstyleSensenumber{~}\textstyleDefinitionn{money}\textstylefstandard{.}
\end{styleEntryParagraph}

\begin{styleEntryParagraph}
\textstyleLexeme{səloy}\textstylefstandard{   }\textstylePartofspeech{v.}\textstyleSensenumber{~}\textstyleDefinitionn{cook on fire}\textstylefstandard{.}
\end{styleEntryParagraph}

\begin{styleEntryParagraph}
\textstyleLexeme{səmbetewk}\textstylefstandard{   }\textstylePartofspeech{n.} \textstyleDefinitionn{hair}
\end{styleEntryParagraph}

\begin{styleEntryParagraph}
\textstyleLexeme{sənewk}\textstylefstandard{   }\textstylePartofspeech{n.}\textstyleSensenumber{~}\textstyleDefinitionn{shadow, spirit}\textstylefstandard{.}
\end{styleEntryParagraph}

\begin{styleEntryParagraph}
\textstyleLexeme{sərkay}\textstylefstandard{   }\textstylePartofspeech{v.} \textstyleDefinitionn{do something habitually}\textstylefstandard{.}
\end{styleEntryParagraph}

\begin{styleEntryParagraph}
\textstyleLexeme{səsayak}\textstylefstandard{   }\textstylePartofspeech{n.} \textstyleDefinitionn{wart}\textstylefstandard{.}
\end{styleEntryParagraph}

\begin{styleEntryParagraph}
\textstyleLexeme{səsəre}\textstylefstandard{   }\textstylePartofspeech{num.} \textstyleDefinitionn{seven}\textstylefstandard{.}
\end{styleEntryParagraph}

\begin{styleEntryParagraph}
\textstyleLexeme{səwat}\textstylefstandard{   }\textstylePartofspeech{ID. idea of }\textstyleDefinitionn{dispersing}\textstylefstandard{.} 
\end{styleEntryParagraph}

\begin{styleEntryParagraph}
\textstyleLexeme{səwse}\textstylefstandard{   }\textstylePartofspeech{n. }\textstyleDefinitionn{thanks}\textstylefstandard{.} 
\end{styleEntryParagraph}

\begin{styleEntryParagraph}
\textstyleLexeme{səy}\textstylefstandard{   }\textstylePartofspeech{conj.} \textstyleDefinitionn{except}
\end{styleEntryParagraph}

\begin{styleEntryParagraph}
\textstyleLexeme{səya}\textstylefstandard{   }\textstylePartofspeech{v.} \textstyleDefinitionn{cut}\textstylefstandard{.} 
\end{styleEntryParagraph}

\begin{styleEntryParagraph}
\textstyleLexeme{səyfa}\textstylefstandard{   }\textstylePartofspeech{n.} \textstyleDefinitionn{life}\textstylefstandard{.}
\end{styleEntryParagraph}

\begin{styleEntryParagraph}
\textstyleLexeme{səy\nobreakdash-say}\textstylefstandard{   }\textstylePartofspeech{n.} \textstyleDefinitionn{5 francs}\textstylefstandard{.}
\end{styleEntryParagraph}

\begin{styleEntryParagraph}
\textstyleLexeme{sla}\textstylefstandard{   }\textstylePartofspeech{n.} \textstyleDefinitionn{cow}\textstylefstandard{.}
\end{styleEntryParagraph}

\begin{styleEntryParagraph}
\textstyleLexeme{slahay}\textstylefstandard{   }\textstylePartofspeech{v.} \textstyleDefinitionn{mix grain with ashes to prevent insects from eating seeds}\textstylefstandard{.} 
\end{styleEntryParagraph}

\begin{styleEntryParagraph}
\textstyleLexeme{slala}\textstylefstandard{   }\textstylePartofspeech{n.} \textstyleDefinitionn{village}\textstylefstandard{.}
\end{styleEntryParagraph}

\begin{styleEntryParagraph}
\textstyleLexeme{slalakar}\textstylefstandard{   }\textstylePartofspeech{num.} \textstyleDefinitionn{eight}\textstylefstandard{.}
\end{styleEntryParagraph}

\begin{styleEntryParagraph}
\textstyleLexeme{slam}\textstylefstandard{   }\textstylePartofspeech{n.} \textstyleDefinitionn{place}\textstylefstandard{.   }
\end{styleEntryParagraph}

\begin{styleEntryParagraph}
\textstyleLexeme{slapay}\textstylefstandard{   }\textstylePartofspeech{v. }\textstyleDefinitionn{braid}\textstylefstandard{.}
\end{styleEntryParagraph}

\begin{styleEntryParagraph}
\textstyleLexeme{slar}\textstylefstandard{   }\textstylePartofspeech{v.} \textstyleDefinitionn{send}\textstylefstandard{.}
\end{styleEntryParagraph}

\begin{styleEntryParagraph}
\textstyleLexeme{slaray}\textstylefstandard{   }\textstylePartofspeech{v.} \textstyleDefinitionn{slide}\textstylefstandard{.}
\end{styleEntryParagraph}

\begin{styleEntryParagraph}
\textstyleLexeme{slay}\textstylefstandard{   }\textstylePartofspeech{v.} \textstyleDefinitionn{slay}\textstylefstandard{.   }
\end{styleEntryParagraph}

\begin{styleEntryParagraph}
\textstyleLexeme{sləɓatay}\textstylefstandard{   }\textstylePartofspeech{v.} \textstyleDefinitionn{repair}\textstylefstandard{.}
\end{styleEntryParagraph}

\begin{styleEntryParagraph}
\textstyleLexeme{sləlay}\textstylefstandard{   }\textstylePartofspeech{n.} \textstyleDefinitionn{root}\textstylefstandard{.}
\end{styleEntryParagraph}

\begin{styleEntryParagraph}
\textstyleLexeme{sləmay}\textstylefstandard{   }\textstylePartofspeech{n.} \textstyleDefinitionn{ear, name}\textstylefstandard{.}
\end{styleEntryParagraph}

\begin{styleEntryParagraph}
\textstyleLexeme{slərah}\textstylefstandard{   }\textstylePartofspeech{n.} \textstyleDefinitionn{board}\textstylefstandard{.}
\end{styleEntryParagraph}

\begin{styleEntryParagraph}
\textstyleLexeme{slərele}\textstylefstandard{   }\textstylePartofspeech{n.} \textstyleDefinitionn{work}\textstylefstandard{.}
\end{styleEntryParagraph}

\begin{styleEntryParagraph}
\textstyleLexeme{slohoy}\textstylefstandard{   }\textstylePartofspeech{v.} \textstyleDefinitionn{leave in secret}\textstylefstandard{.}
\end{styleEntryParagraph}

\begin{styleEntryParagraph}
\textstyleLexeme{slohoy}\textstylefstandard{   }\textstylePartofspeech{v.}\textstyleDefinitionn{take leaves off stalk}\textstylefstandard{.}
\end{styleEntryParagraph}

\begin{styleEntryParagraph}
\textstyleLexeme{sloko}\textstylefstandard{   }\textstylePartofspeech{n.} \textstyleDefinitionn{earring}\textstylefstandard{.}
\end{styleEntryParagraph}

\begin{styleEntryParagraph}
\textstyleLexeme{soɓoy}\textstylefstandard{   }\textstylePartofspeech{v.} \textstyleDefinitionn{suck}\textstylefstandard{.}
\end{styleEntryParagraph}

\begin{styleEntryParagraph}
\textstyleLexeme{sokoy}\textstylefstandard{   }\textstylePartofspeech{v.} \textstyleDefinitionn{whisper}\textstylefstandard{.}
\end{styleEntryParagraph}

\begin{styleEntryParagraph}
\textstyleLexeme{solay}\textstylefstandard{   }\textstylePartofspeech{v.} \textstyleDefinitionn{fry}\textstylefstandard{.}
\end{styleEntryParagraph}

\begin{styleEntryParagraph}
\textstyleLexeme{sono}\textstylefstandard{   }\textstylePartofspeech{n.} \textstyleDefinitionn{joke}\textstylefstandard{.}
\end{styleEntryParagraph}

\begin{styleEntryParagraph}
\textstyleLexeme{soroy}\textstylefstandard{   }\textstylePartofspeech{v.} \textstyleDefinitionn{slide}\textstylefstandard{.}
\end{styleEntryParagraph}
\end{multicols}
\begin{styleLetterParagraph}
\textstyleLetterv{T  {}-  t}
\end{styleLetterParagraph}

\begin{multicols}{2}
\begin{styleEntryParagraph}
\textstyleLexeme{ta}\textstylefstandard{   }\textstylePartofspeech{vclitic. }\textstyleDefinitionn{3P} \textit{direct object}.
\end{styleEntryParagraph}

\begin{styleEntryParagraph}
\textstyleLexeme{t\nobreakdash-}\textstylefstandard{   }\textstylePartofspeech{vpfx.} \textstyleDefinitionn{3p}\textstylefstandard{.}
\end{styleEntryParagraph}

\begin{styleEntryParagraph}
\textstyleLexeme{tacay}\textstylefstandard{   }\textstylePartofspeech{v.} \textstyleDefinitionn{close}\textstylefstandard{.}
\end{styleEntryParagraph}

\begin{styleEntryParagraph}
\textstyleLexeme{taɗ}\textstylefstandard{   }\textstylePartofspeech{v.} \textstyleDefinitionn{fall}\textstylefstandard{.}
\end{styleEntryParagraph}

\begin{styleEntryParagraph}
\textstyleLexeme{taf}\textstylefstandard{   }\textstylePartofspeech{v.} \textstyleDefinitionn{spit}\textstylefstandard{.}
\end{styleEntryParagraph}

\begin{styleEntryParagraph}
\textstyleLexeme{tah}\textstylefstandard{   }\textstylePartofspeech{v.} \textstyleDefinitionn{pile something}\textstylefstandard{.}
\end{styleEntryParagraph}

\begin{styleEntryParagraph}
\textstyleLexeme{tah}\textstylefstandard{   }\textstylePartofspeech{v.} \textstyleDefinitionn{reach out}\textstylefstandard{.}
\end{styleEntryParagraph}

\begin{styleEntryParagraph}
\textstyleLexeme{tahay}\textstylefstandard{   }\textstylePartofspeech{v.} \textstyleDefinitionn{boost}\textstylefstandard{.}
\end{styleEntryParagraph}

\begin{styleEntryParagraph}
\textstyleLexeme{talay}\textstylefstandard{   }\textstylePartofspeech{v.}\textstyleDefinitionn{walk}\textstylefstandard{.}
\end{styleEntryParagraph}

\begin{styleEntryParagraph}
\textstyleLexeme{tam}\textstylefstandard{   }\textstylePartofspeech{v.} \textstyleDefinitionn{save}\textstylefstandard{.}
\end{styleEntryParagraph}

\begin{styleEntryParagraph}
\textstyleLexeme{tapay}\textstylefstandard{   }\textstylePartofspeech{v.} \textstyleDefinitionn{stick}\textstylefstandard{.}
\end{styleEntryParagraph}

\begin{styleEntryParagraph}
\textstyleLexeme{tar}\textstylefstandard{   }\textstylePartofspeech{v.} \textstyleDefinitionn{enter}\textstylefstandard{.}
\end{styleEntryParagraph}

\begin{styleEntryParagraph}
\textstyleLexeme{taray}\textstylefstandard{   }\textstylePartofspeech{v.} \textstyleDefinitionn{call}\textstylefstandard{.}
\end{styleEntryParagraph}

\begin{styleEntryParagraph}
\textstyleLexeme{taslay}\textstylefstandard{   }\textstylePartofspeech{v.}\textstyleSensenumber{ ~}\textstyleDefinitionn{curse}\textstylefstandard{.}
\end{styleEntryParagraph}

\begin{styleEntryParagraph}
\textstyleLexeme{tenjew}\textstylefstandard{   }\textstylePartofspeech{n.} \textstyleDefinitionn{mosquito}\textstylefstandard{.}
\end{styleEntryParagraph}

\begin{styleEntryParagraph}
\textstyleLexeme{tere}\textstylefstandard{   }\textstylePartofspeech{n.} \textstyleDefinitionn{another, a different one}\textstylefstandard{.}
\end{styleEntryParagraph}

\begin{styleEntryParagraph}
\textstyleLexeme{ter\nobreakdash-tere}\textstylefstandard{   }\textstylePartofspeech{ID. idea of something} \textstyleDefinitionn{different}\textstylefstandard{.}
\end{styleEntryParagraph}

\begin{styleEntryParagraph}
\textstyleLexeme{tezeh}\textstylefstandard{   }\textstylePartofspeech{n.} \textstyleDefinitionn{boa}\textstylefstandard{.}
\end{styleEntryParagraph}

\begin{styleEntryParagraph}
\textstyleLexeme{tezl tezlezl}\textstylefstandard{   }\textstylePartofspeech{ID. }\textstyleDefinitionn{idea of hollowness}\textstylefstandard{.}
\end{styleEntryParagraph}

\begin{styleEntryParagraph}
\textstyleLexeme{təde}\textstylefstandard{   }\textstylePartofspeech{n.} \textstyleDefinitionn{good}\textstylefstandard{.}
\end{styleEntryParagraph}

\begin{styleEntryParagraph}
\textstyleLexeme{tədo}\textstylefstandard{   }\textstylePartofspeech{n.} \textstyleDefinitionn{leopard}\textstylefstandard{.}
\end{styleEntryParagraph}

\begin{styleEntryParagraph}
\textstyleLexeme{təɗoy}\textstylefstandard{   }\textstylePartofspeech{v.} \textstyleDefinitionn{roll, wind}\textstylefstandard{.}
\end{styleEntryParagraph}

\begin{styleEntryParagraph}
\textstyleLexeme{təf}\textstylefstandard{   }\textstylePartofspeech{ID.} \textstyleDefinitionn{idea of going far}\textstylefstandard{.}
\end{styleEntryParagraph}

\begin{styleEntryParagraph}
\textstyleLexeme{təh}\textstylefstandard{   }\textstylePartofspeech{ID.} \textstyleDefinitionn{idea of putting on head}\textstylefstandard{.}
\end{styleEntryParagraph}

\begin{styleEntryParagraph}
\textstyleLexeme{təkam}\textstylefstandard{   }\textstylePartofspeech{v.} \textstyleDefinitionn{taste}\textstylefstandard{.}
\end{styleEntryParagraph}

\begin{styleEntryParagraph}
\textstyleLexeme{təkaray}\textstylefstandard{   }\textstylePartofspeech{v.} \textstyleDefinitionn{try, invite}\textstylefstandard{.}
\end{styleEntryParagraph}

\begin{styleEntryParagraph}
\textstyleLexeme{təkasay}\textstylefstandard{   }\textstylePartofspeech{v.} \textstyleDefinitionn{cross}\textstylefstandard{.}
\end{styleEntryParagraph}

\begin{styleEntryParagraph}
\textstyleLexeme{təkosoy}\textstylefstandard{   }\textstylePartofspeech{v. }\textstyleDefinitionn{fold, cross}\textstylefstandard{.}
\end{styleEntryParagraph}

\begin{styleEntryParagraph}
\textstyleLexeme{təkwərak}\textstylefstandard{   }\textstylePartofspeech{n.} \textstyleDefinitionn{partridge}\textstylefstandard{.}
\end{styleEntryParagraph}

\begin{styleEntryParagraph}
\textstyleLexeme{təlɓaway}\textstylefstandard{   }\textstylePartofspeech{v.} \textstyleDefinitionn{be sticky}\textstylefstandard{.}
\end{styleEntryParagraph}

\begin{styleEntryParagraph}
\textstyleLexeme{təlokoy}\textstylefstandard{   }\textstylePartofspeech{v. }\textstyleDefinitionn{drip}\textstylefstandard{.}
\end{styleEntryParagraph}

\begin{styleEntryParagraph}
\textstyleLexeme{təmak}\textstylefstandard{   }\textstylePartofspeech{n.} \textstyleDefinitionn{sheep}\textstylefstandard{.}
\end{styleEntryParagraph}

\begin{styleEntryParagraph}
\textstyleLexeme{təmbaɗay}\textstylefstandard{   }\textstylePartofspeech{v.} \textstyleDefinitionn{twist}\textstylefstandard{.}
\end{styleEntryParagraph}

\begin{styleEntryParagraph}
\textstyleLexeme{təmbalay}\textstylefstandard{   }\textstylePartofspeech{v.}\textstyleDefinitionn{shake out stones}\textstylefstandard{.}
\end{styleEntryParagraph}

\begin{styleEntryParagraph}
\textstyleLexeme{tərɗay}\textstylefstandard{   }\textstylePartofspeech{v.} \textstyleDefinitionn{tie off}\textstylefstandard{.}
\end{styleEntryParagraph}

\begin{styleEntryParagraph}
\textstyleLexeme{təta}\textstylefstandard{   }\textstylePartofspeech{pn.} \textstyleDefinitionn{3p}\textstylefstandard{.}
\end{styleEntryParagraph}

\begin{styleEntryParagraph}
\textstyleLexeme{təta}\textstylefstandard{   }\textstylePartofspeech{adv. }\textstyleDefinitionn{is able to}\textstylefstandard{.}
\end{styleEntryParagraph}

\begin{styleEntryParagraph}
\textstyleLexeme{tətərak}\textstylefstandard{   }\textstylePartofspeech{n.} \textstyleDefinitionn{shoes}\textstylefstandard{.}
\end{styleEntryParagraph}

\begin{styleEntryParagraph}
\textstyleLexeme{təvalay}\textstylefstandard{   }\textstylePartofspeech{v.} \textstyleDefinitionn{hunt}\textstylefstandard{.}
\end{styleEntryParagraph}

\begin{styleEntryParagraph}
\textstyleLexeme{təwaɗay}\textstylefstandard{   }\textstylePartofspeech{v. go a}\textstyleDefinitionn{cross}\textstylefstandard{.}
\end{styleEntryParagraph}

\begin{styleEntryParagraph}
\textstyleLexeme{təway}\textstylefstandard{  }\textstylePartofspeech{v.} \textstyleDefinitionn{cry}\textstylefstandard{.}
\end{styleEntryParagraph}

\begin{styleEntryParagraph}
\textstyleLexeme{təwe}\textstylefstandard{  }\textstylePartofspeech{n.} \textstyleDefinitionn{cry}\textstylefstandard{.}
\end{styleEntryParagraph}

\begin{styleEntryParagraph}
\textstyleLexeme{toho}\textstylefstandard{   }\textstylePartofspeech{dem.} \textstyleDefinitionn{far}\textstylefstandard{.}
\end{styleEntryParagraph}

\begin{styleEntryParagraph}
\textstyleLexeme{tohoy}\textstylefstandard{   }\textstylePartofspeech{v.} \textstyleDefinitionn{trace}\textstylefstandard{.}
\end{styleEntryParagraph}

\begin{styleEntryParagraph}
\textstyleLexeme{tokoy}\textstylefstandard{   }\textstylePartofspeech{v.} \textstyleDefinitionn{tap}\textstylefstandard{.} 
\end{styleEntryParagraph}

\begin{styleEntryParagraph}
\textstyleLexeme{Tokombere}\textstylefstandard{   }\textstylePartofspeech{n.pr.} \textstyleDefinitionn{Tuesday market}\textstylefstandard{.}
\end{styleEntryParagraph}

\begin{styleEntryParagraph}
\textstyleLexeme{tololon}\textstylefstandard{   }\textstylePartofspeech{n.} \textstyleDefinitionn{heart, chest.}
\end{styleEntryParagraph}

\begin{styleEntryParagraph}
\textstyleLexeme{tosoy}\textstylefstandard{   }\textstylePartofspeech{v.} \textstyleDefinitionn{bud, uproot}\textstylefstandard{.}
\end{styleEntryParagraph}
\end{multicols}
\begin{styleLetterParagraph}
\textstyleLetterv{V  {}-  v}
\end{styleLetterParagraph}

\begin{multicols}{2}
\begin{styleEntryParagraph}
\textstyleLexeme{va}\textstylefstandard{   }\textstylePartofspeech{vclitic.} \textstyleDefinitionn{Perfect}\textstylefstandard{.}
\end{styleEntryParagraph}

\begin{styleEntryParagraph}
\textstyleLexeme{va}\textstylefstandard{   }\textstylePartofspeech{n.} \textstyleDefinitionn{body}\textstylefstandard{.}
\end{styleEntryParagraph}

\begin{styleEntryParagraph}
\textstyleLexeme{vahay}\textstylefstandard{   }\textstylePartofspeech{v.} \textstyleDefinitionn{fly away}\textstylefstandard{.}
\end{styleEntryParagraph}

\begin{styleEntryParagraph}
\textstyleLexeme{vakay}\textstylefstandard{   }\textstylePartofspeech{v.}\textstyleSensenumber{~}\textstyleDefinitionn{burn}\textstylefstandard{.}
\end{styleEntryParagraph}

\begin{styleEntryParagraph}
\textstyleLexeme{var}\textstylefstandard{   }\textstylePartofspeech{v.} \textit{put on a }\textstyleDefinitionn{roof}\textstylefstandard{.}
\end{styleEntryParagraph}

\begin{styleEntryParagraph}
\textstyleLexeme{varay}\textstylefstandard{   }\textstylePartofspeech{v.} \textstyleDefinitionn{chase away}\textstylefstandard{.}
\end{styleEntryParagraph}

\begin{styleEntryParagraph}
\textstyleLexeme{vərɗay}\textstylefstandard{   }\textstylePartofspeech{v.} \textstyleDefinitionn{boil}\textstylefstandard{.}
\end{styleEntryParagraph}

\begin{styleEntryParagraph}
\textstyleLexeme{vasay}\textstylefstandard{   }\textstylePartofspeech{v.} \textstyleDefinitionn{wipe out, cancel}\textstylefstandard{.}
\end{styleEntryParagraph}

\begin{styleEntryParagraph}
\textstyleLexeme{vaway}\textstylefstandard{   }\textstylePartofspeech{v.} \textstyleDefinitionn{twist, hang}\textstylefstandard{.}
\end{styleEntryParagraph}

\begin{styleEntryParagraph}
\textstyleLexeme{vay}\textstylefstandard{   }\textstylePartofspeech{v. }\textstyleDefinitionn{winnow}\textstylefstandard{.}
\end{styleEntryParagraph}

\begin{styleDoublecolumnSection}
\textbf{vbaɓ}\textstylefstandard{   }\textstylefstandard{\textit{ID. }}\textstylePartofspeech{sound or }\textstyleDefinitionn{idea of something soft hitting the ground (a snake, or a mud wall)}
\end{styleDoublecolumnSection}

\begin{styleDoublecolumnSection}
\textbf{vbəvbəvbə}\textstylefstandard{   }\textstylefstandard{\textit{ID.}}\textstylePartofspeech{ }\textstyleDefinitionn{rapidly}
\end{styleDoublecolumnSection}

\begin{styleEntryParagraph}
\textstyleLexeme{ve}\textstylefstandard{   }\textstylePartofspeech{v.} \textstyleDefinitionn{spend time}\textstylefstandard{.}
\end{styleEntryParagraph}

\begin{styleEntryParagraph}
\textstyleLexeme{ver}\textstylefstandard{   }\textstylePartofspeech{n.} \textstyleDefinitionn{room}\textstylefstandard{.}
\end{styleEntryParagraph}

\begin{styleEntryParagraph}
\textstyleLexeme{ver}\textstylefstandard{   }\textstylePartofspeech{n.} \textstyleDefinitionn{grinding stone}\textstylefstandard{.}
\end{styleEntryParagraph}

\begin{styleEntryParagraph}
\textstyleLexeme{vəɗ vaɗ}\textstylefstandard{   }\textstylePartofspeech{n.} \textstyleDefinitionn{all night}\textstylefstandard{.}
\end{styleEntryParagraph}

\begin{styleEntryParagraph}
\textstyleLexeme{vəlalay}\textstylefstandard{   }\textstylePartofspeech{v.} \textstyleDefinitionn{oyster}\textstylefstandard{.}
\end{styleEntryParagraph}

\begin{styleEntryParagraph}
\textstyleLexeme{vəlay}\textstylefstandard{   }\textstylePartofspeech{v.} \textstyleDefinitionn{boil}\textstylefstandard{.}
\end{styleEntryParagraph}

\begin{styleEntryParagraph}
\textstyleLexeme{vənahay}\textstylefstandard{   }\textstylePartofspeech{v.} \textstyleDefinitionn{vomit}\textstylefstandard{.}
\end{styleEntryParagraph}

\begin{styleEntryParagraph}
\textstyleLexeme{vər}\textstylefstandard{   }\textstylePartofspeech{v.} \textstyleDefinitionn{give}\textstylefstandard{.}
\end{styleEntryParagraph}

\begin{styleEntryParagraph}
\textstyleLexeme{vəy}\textstylefstandard{   }\textstylePartofspeech{n.} \textstyleDefinitionn{rib}\textstylefstandard{.}
\end{styleEntryParagraph}

\begin{styleIndentedParagraph}
\textstyleSubentry{vəy}\textstyleLexeme{mete}\textstylefstandard{   }\textstylePartofspeech{n.} \textstyleDefinitionn{neighbour}\textstylefstandard{.}
\end{styleIndentedParagraph}

\begin{styleEntryParagraph}
\textstyleLexeme{vəya}\textstylefstandard{   }\textstylePartofspeech{n.} \textstyleDefinitionn{rainy season}\textstylefstandard{.}
\end{styleEntryParagraph}\end{multicols}
\begin{styleLetterParagraph}
\textstyleLetterv{W  {}-  w }
\end{styleLetterParagraph}

\begin{multicols}{2}
\begin{styleEntryParagraph}
\textstyleLexeme{wacay}\textstylefstandard{   }\textstylePartofspeech{v.} \textstyleDefinitionn{write}\textstylefstandard{.}
\end{styleEntryParagraph}

\begin{styleEntryParagraph}
\textstyleLexeme{waɗay}\textstylefstandard{   }\textstylePartofspeech{v.} \textstyleDefinitionn{spread out}\textstylefstandard{.}
\end{styleEntryParagraph}

\begin{styleEntryParagraph}
\textstyleLexeme{wahay}\textstylefstandard{   }\textstylePartofspeech{v.} \textstyleDefinitionn{waste}\textstylefstandard{.}
\end{styleEntryParagraph}

\begin{styleEntryParagraph}
\textstyleLexeme{wal}\textstylefstandard{   }\textstylePartofspeech{v.} \textstyleDefinitionn{attach, hurt}\textstylefstandard{.}
\end{styleEntryParagraph}

\begin{styleEntryParagraph}
\textstyleLexeme{walay}\textstylefstandard{   }\textstylePartofspeech{v.} \textit{dismantle}\textstylefstandard{.}
\end{styleEntryParagraph}

\begin{styleEntryParagraph}
\textstyleLexeme{war}\textstylefstandard{   }\textstylePartofspeech{n.} \textstyleDefinitionn{child}
\end{styleEntryParagraph}

\begin{styleEntryParagraph}
\textstyleLexeme{  babəza ahay }\textstyleLexeme{\textmd{n. children}}\textstylefstandard{.}
\end{styleEntryParagraph}

\begin{styleEntryParagraph}
\textstyleLexeme{waray}\textstylefstandard{   }\textstylePartofspeech{v.} \textstyleDefinitionn{take upon oneself}\textstylefstandard{.}
\end{styleEntryParagraph}

\begin{styleEntryParagraph}
\textstyleLexeme{was}\textstylefstandard{   }\textstylePartofspeech{v.} \textstyleDefinitionn{cultivate}\textstylefstandard{.}
\end{styleEntryParagraph}

\begin{styleEntryParagraph}
\textstyleLexeme{wasay}\textstylefstandard{   }\textstylePartofspeech{v.} \textstyleDefinitionn{populate}\textstylefstandard{.}
\end{styleEntryParagraph}

\begin{styleEntryParagraph}
\textstyleLexeme{wasl}\textstylefstandard{   }\textstylePartofspeech{v.}\textstylefstandard{ }\textstylefstandard{\textit{be forbidden.}}
\end{styleEntryParagraph}

\begin{styleEntryParagraph}
\textstyleLexeme{way}\textstylefstandard{   }\textstylePartofspeech{pn.} \textstyleDefinitionn{who}\textstylefstandard{.}
\end{styleEntryParagraph}

\begin{styleEntryParagraph}
\textstyleLexeme{waya}\textstylefstandard{   }\textstylePartofspeech{conj.}\textstyleDefinitionn{because}\textstylefstandard{.}
\end{styleEntryParagraph}

\begin{styleEntryParagraph}
\textstyleLexeme{wazay}\textstylefstandard{   }\textstylePartofspeech{v.} \textstyleDefinitionn{shake}\textstylefstandard{.}
\end{styleEntryParagraph}

\begin{styleEntryParagraph}
\textstyleLexeme{wazlay}\textstylefstandard{   }\textstylePartofspeech{v.} \textstyleDefinitionn{shine.}
\end{styleEntryParagraph}

\begin{styleEntryParagraph}
\textstyleLexeme{we}\textstylefstandard{   }\textstylePartofspeech{v.} \textstyleDefinitionn{give birth}\textstylefstandard{.}
\end{styleEntryParagraph}

\begin{styleEntryParagraph}
\textstyleLexeme{weley}\textstylefstandard{   }\textstylefstandard{\textit{p}}\textstylePartofspeech{n.} \textstyleDefinitionn{which.}
\end{styleEntryParagraph}

\begin{styleEntryParagraph}
\textstyleLexeme{wewer}\textstylefstandard{   }\textstylePartofspeech{n.} \textstyleDefinitionn{cunning}\textstylefstandard{.}
\end{styleEntryParagraph}

\begin{styleEntryParagraph}
\textstyleLexeme{wəcaɗay}\textstylefstandard{   }\textstylePartofspeech{v.} \textstyleDefinitionn{shine}\textstylefstandard{.}
\end{styleEntryParagraph}

\begin{styleEntryParagraph}
\textstyleLexeme{wəɗakay}\textstylefstandard{   }\textstylePartofspeech{v.} \textstyleDefinitionn{divide, share}\textstylefstandard{.} 
\end{styleEntryParagraph}

\begin{styleEntryParagraph}
\textstyleLexeme{wəɗoy}\textstylefstandard{   }\textstylePartofspeech{v.} \textstyleDefinitionn{populate}\textstylefstandard{.}
\end{styleEntryParagraph}

\begin{styleEntryParagraph}
\textstyleLexeme{wəldoy}\textstylefstandard{   }\textstylePartofspeech{v.} \textstyleDefinitionn{devour}\textstylefstandard{.}
\end{styleEntryParagraph}

\begin{styleEntryParagraph}
\textstyleLexeme{wəle}\textstylefstandard{   }\textstylePartofspeech{n.} \textstyleDefinitionn{potash}\textstylefstandard{.}
\end{styleEntryParagraph}

\begin{styleEntryParagraph}
\textstyleLexeme{wərkay}\textstylefstandard{   }\textstylePartofspeech{v.} \textstyleDefinitionn{pay}\textstylefstandard{.}
\end{styleEntryParagraph}

\begin{styleEntryParagraph}
\textstyleLexeme{wərge}\textstylefstandard{   }\textstylePartofspeech{n.} \textstyleDefinitionn{bad spirit}\textstylefstandard{.}
\end{styleEntryParagraph}

\begin{styleEntryParagraph}
\textstyleLexeme{wərsla}\textstylefstandard{   }\textstylePartofspeech{n.} \textstyleDefinitionn{butter}\textstylefstandard{.}
\end{styleEntryParagraph}

\begin{styleEntryParagraph}
\textstyleLexeme{wərzla}\textstylefstandard{   }\textstylePartofspeech{n.} \textstyleDefinitionn{star}\textstylefstandard{.}
\end{styleEntryParagraph}

\begin{styleEntryParagraph}
\textstyleLexeme{wəse}\textstylefstandard{   }\textstylePartofspeech{n.} \textstyleDefinitionn{thank you}\textstylefstandard{.}
\end{styleEntryParagraph}

\begin{styleEntryParagraph}
\textstyleLexeme{wəsekeke}\textstylefstandard{   }\textstylePartofspeech{ID.} \textstyleDefinitionn{sight/idea of something multiplying}\textstylefstandard{.}
\end{styleEntryParagraph}

\begin{styleEntryParagraph}
\textstyleLexeme{wəyen}\textstylefstandard{   }\textstylePartofspeech{n.} \textstyleDefinitionn{land.}
\end{styleEntryParagraph}

\begin{styleEntryParagraph}
\textstyleLexeme{wəzlay}\textstylefstandard{   }\textstylePartofspeech{v.} \textit{p}\textstyleDefinitionn{ublish, announce.}
\end{styleEntryParagraph}
\end{multicols}
\begin{styleLetterParagraph}
\textstyleLetterv{Y  {}-  y}
\end{styleLetterParagraph}

\begin{multicols}{2}
\begin{styleEntryParagraph}
\textstyleLexeme{\nobreakdash-ya}\textstylefstandard{   }\textstylePartofspeech{nsfx.} \textstyleDefinitionn{respectful vocative}\textstylefstandard{.}
\end{styleEntryParagraph}

\begin{styleEntryParagraph}
\textstyleLexeme{yaɗay}\textstylefstandard{   }\textstylePartofspeech{v.} \textstyleDefinitionn{tire}\textstylefstandard{.}
\end{styleEntryParagraph}

\begin{styleEntryParagraph}
\textstyleLexeme{yam}\textstylefstandard{   }\textstylePartofspeech{n.} \textstyleDefinitionn{water}\textstylefstandard{.}
\end{styleEntryParagraph}

\begin{styleEntryParagraph}
\textstyleLexeme{yamay}\textstylefstandard{   }\textstylePartofspeech{v.} \textstyleDefinitionn{spin}\textstylefstandard{.}
\end{styleEntryParagraph}

\begin{styleEntryParagraph}
\textstyleLexeme{Yerəyma}\textstylefstandard{   }\textstylePartofspeech{n.prince~}\textstyleDefinitionn{; Monday market}\textstylefstandard{.}
\end{styleEntryParagraph}
\end{multicols}
\begin{styleLetterParagraph}
\textstyleLetterv{Z  {}-  z}
\end{styleLetterParagraph}

\begin{multicols}{2}
\begin{styleEntryParagraph}
\textstyleLexeme{zaɗ}\textstylefstandard{   }\textstylePartofspeech{v.}\textstyleDefinitionn{take, carry}
\end{styleEntryParagraph}

\begin{styleEntryParagraph}
\textstyleLexeme{zana}\textstylefstandard{   }\textstylePartofspeech{n.} \textstyleSensenumber{\textit{clothes, cloth}}\textstylefstandard{\textit{.}}
\end{styleEntryParagraph}

\begin{styleEntryParagraph}
\textstyleLexeme{zar}\textstylefstandard{   }\textstylePartofspeech{n.} \textstyleDefinitionn{male ; husband}\textstylefstandard{.}
\end{styleEntryParagraph}

\begin{styleEntryParagraph}
\textstyleLexeme{  zawər ahay}\textstyleLexeme{\textmd{ }}\textstyleLexeme{\textmd{\textit{n. men}}}
\end{styleEntryParagraph}

\begin{styleEntryParagraph}
\textstyleLexeme{zaray}\textstylefstandard{   }\textstylePartofspeech{v.} \textstyleDefinitionn{linger}\textstylefstandard{.}
\end{styleEntryParagraph}

\begin{styleEntryParagraph}
\textstyleLexeme{zay}\textstylefstandard{   }\textstylePartofspeech{n.} \textstyleSensenumber{\textit{peace, wholeness}}\textstylefstandard{\textit{.}}
\end{styleEntryParagraph}

\begin{styleEntryParagraph}
\textstyleLexeme{zayəh}\textstylefstandard{   }\textstylePartofspeech{n.} \textstyleSensenumber{\textit{care}}\textstylefstandard{.}
\end{styleEntryParagraph}

\begin{styleEntryParagraph}
\textstyleLexeme{zazay}\textstylefstandard{   }\textstylePartofspeech{n.} \textstyleSensenumber{\textit{peace, wholeness}}\textstylefstandard{\textit{.}}
\end{styleEntryParagraph}

\begin{styleEntryParagraph}
\textstyleLexeme{ze}\textstylefstandard{   }\textstylePartofspeech{v.} \textit{smell}\textstyleSensenumber{.}
\end{styleEntryParagraph}

\begin{styleEntryParagraph}
\textstyleLexeme{zetene}\textstylefstandard{   }\textstylePartofspeech{n.} \textstyleDefinitionn{salt}\textstylefstandard{.}
\end{styleEntryParagraph}

\begin{styleEntryParagraph}
\textstyleLexeme{zəgogom}\textstylefstandard{   }\textstylePartofspeech{n.} \textstyleDefinitionn{tree (sp.)}\textstylefstandard{.}
\end{styleEntryParagraph}

\begin{styleEntryParagraph}
\textstyleLexeme{zəmbaɗay}\textstylefstandard{   }\textstylePartofspeech{v.} \textstyleDefinitionn{glorify}\textstylefstandard{.}
\end{styleEntryParagraph}

\begin{styleEntryParagraph}
\textstyleLexeme{zənof}\textstylefstandard{   }\textstylePartofspeech{n.} \textstyleSensenumber{~}\textstyleDefinitionn{naivety, kindness}\textstylefstandard{.}
\end{styleEntryParagraph}

\begin{styleEntryParagraph}
\textstyleLexeme{zən zan}\textstylefstandard{   }\textstylePartofspeech{n.} \textstyleDefinitionn{mouse}\textstylefstandard{.}
\end{styleEntryParagraph}

\begin{styleEntryParagraph}
\textstyleLexeme{zən zen}\textstylefstandard{   }\textstylePartofspeech{n.} \textstyleDefinitionn{darkness}\textstylefstandard{.}
\end{styleEntryParagraph}

\begin{styleEntryParagraph}
\textstyleLexeme{zən zon}\textstylefstandard{   }\textstylePartofspeech{n.} \textstyleDefinitionn{gourd}\textstylefstandard{.}
\end{styleEntryParagraph}

\begin{styleEntryParagraph}
\textstyleLexeme{zəraka}\textstylefstandard{   }\textstylePartofspeech{n.} \textstyleDefinitionn{river}\textstylefstandard{.}
\end{styleEntryParagraph}

\begin{styleEntryParagraph}
\textstyleLexeme{zərɗay}\textstylefstandard{   }\textstylePartofspeech{v.} \textstyleDefinitionn{watch intently}\textstylefstandard{.}
\end{styleEntryParagraph}

\begin{styleEntryParagraph}
\textstyleLexeme{zəroy}\textstylefstandard{   }\textstylePartofspeech{v.} \textstyleDefinitionn{notice, inspect.}
\end{styleEntryParagraph}

\begin{styleEntryParagraph}
\textstyleLexeme{zəva}\textstylefstandard{   }\textstylePartofspeech{n.} \textstyleDefinitionn{net}\textstylefstandard{.}
\end{styleEntryParagraph}

\begin{styleEntryParagraph}
\textstyleLexeme{Zlaba}\textstylefstandard{   }\textstylePartofspeech{n.} \textstyleSensenumber{\textit{~}}\textstyleSensenumber{\textit{Sunday market}}\textstylefstandard{.}
\end{styleEntryParagraph}

\begin{styleEntryParagraph}
\textstyleLexeme{zlaɓay}\textstylefstandard{   }\textstylePartofspeech{v.} \textstyleDefinitionn{pound/crush}\textstylefstandard{.}
\end{styleEntryParagraph}

\begin{styleEntryParagraph}
\textstyleLexeme{zlah}\textstylefstandard{   }\textstylePartofspeech{v.} \textstyleDefinitionn{cry (dog or rooster)}\textstylefstandard{.}
\end{styleEntryParagraph}

\begin{styleEntryParagraph}
\textstyleLexeme{zlakay}\textstylefstandard{   }\textstylePartofspeech{v.} \textstyleDefinitionn{suffer pain}\textstylefstandard{.}
\end{styleEntryParagraph}

\begin{styleEntryParagraph}
\textstyleLexeme{zlan}\textstylefstandard{   }\textstylePartofspeech{v.} \textstyleDefinitionn{start}\textstylefstandard{.}
\end{styleEntryParagraph}

\begin{styleEntryParagraph}
\textstyleLexeme{zlapay}\textstylefstandard{   }\textstylePartofspeech{v.}\textstyleDefinitionn{talk with someone}\textstylefstandard{.}
\end{styleEntryParagraph}

\begin{styleEntryParagraph}
\textstyleLexeme{zlar}\textstylefstandard{   }\textstylePartofspeech{v.} \textstyleDefinitionn{pierce}\textstylefstandard{.}
\end{styleEntryParagraph}

\begin{styleEntryParagraph}
\textstyleLexeme{zlar}\textstylefstandard{   }\textstylePartofspeech{v.} \textstyleDefinitionn{kick}\textstylefstandard{.}
\end{styleEntryParagraph}

\begin{styleEntryParagraph}
\textstyleLexeme{zlavay}\textstylefstandard{   }\textstylePartofspeech{v.} \textstyleDefinitionn{swim}\textstylefstandard{.}
\end{styleEntryParagraph}

\begin{styleEntryParagraph}
\textstyleLexeme{zlaway}\textstylefstandard{   }\textstylePartofspeech{v.} \textstyleDefinitionn{fear}\textstylefstandard{.}
\end{styleEntryParagraph}

\begin{styleEntryParagraph}
\textstyleLexeme{zlevek}\textstylefstandard{   }\textstylePartofspeech{n.} \textstyleDefinitionn{rabbit}\textstylefstandard{.}
\end{styleEntryParagraph}

\begin{styleEntryParagraph}
\textstyleLexeme{zlezle}\textstylefstandard{   }\textstylefstandard{\textit{ID.}} \textstyleDefinitionn{time long ago}\textstylefstandard{.}
\end{styleEntryParagraph}

\begin{styleEntryParagraph}
\textstyleLexeme{zləge}\textstylefstandard{   }\textstylePartofspeech{v.} \textstyleDefinitionn{throw, plant.}
\end{styleEntryParagraph}

\begin{styleEntryParagraph}
\textstyleLexeme{zlək zlak}\textstylefstandard{   }\textstylePartofspeech{n.}\textstyleDefinitionn{termite}\textstylefstandard{.}
\end{styleEntryParagraph}

\begin{styleEntryParagraph}
\textstyleLexeme{zləle}\textstylefstandard{   }\textstylePartofspeech{n.} \textstyleDefinitionn{richness}\textstylefstandard{.}
\end{styleEntryParagraph}

\begin{styleEntryParagraph}
\textstyleLexeme{zlərav}\textstylefstandard{   }\textstylePartofspeech{v.} \textstyleDefinitionn{remove}\textstylefstandard{.}
\end{styleEntryParagraph}

\begin{styleEntryParagraph}
\textstyleLexeme{zləray}\textstylefstandard{   }\textstylePartofspeech{v.} \textstyleDefinitionn{go out, appear}\textstylefstandard{.}
\end{styleEntryParagraph}

\begin{styleEntryParagraph}
\textstyleLexeme{zlərgo}\textstylefstandard{   }\textstylePartofspeech{v.} \textstyleDefinitionn{axe}\textstylefstandard{.}
\end{styleEntryParagraph}

\begin{styleEntryParagraph}
\textstyleLexeme{zlokoy}\textstylefstandard{   }\textstylePartofspeech{v.} \textstyleDefinitionn{gnaw}\textstylefstandard{.}
\end{styleEntryParagraph}

\begin{styleEntryParagraph}
\textstyleLexeme{zlokoy}\textstylefstandard{   }\textstylePartofspeech{v.} \textstyleDefinitionn{squeeze out}\textstylefstandard{.}
\end{styleEntryParagraph}

\begin{styleEntryParagraph}
\textstyleLexeme{zlom}\textstylefstandard{   }\textstylePartofspeech{num.} \textstyleDefinitionn{five}\textstylefstandard{.}
\end{styleEntryParagraph}

\begin{styleEntryParagraph}
\textstyleLexeme{zokoy}\textstylefstandard{   }\textstylePartofspeech{v.} \textstyleDefinitionn{try}\textstylefstandard{.}
\end{styleEntryParagraph}

\begin{styleEntryParagraph}
\textstyleLexeme{zom}\textstylefstandard{   }\textstylePartofspeech{v.} \textstyleDefinitionn{eat.}
\end{styleEntryParagraph}

\begin{styleEntryParagraph}
\textstyleLexeme{zor}\textstylefstandard{   }\textstylePartofspeech{ID.} \textstyleDefinitionn{sight/idea of something thrown up high}
\end{styleEntryParagraph}
\end{multicols}

\setcounter{page}{1}\chapter[Appendix 2. English{}-Moloko Lexicon]{Appendix 2. English-Moloko Lexicon}
\hypertarget{RefHeading1213681525720847}{}\begin{styleLetterParagraph}
\thepage{}\textstyleLetterv{A  {}-  a}
\end{styleLetterParagraph}

\begin{multicols}{2}
\end{multicols}
\begin{multicols}{2}
\begin{styleEntryParagraph}
\textstyleDefinitionn{able to}\textstyleLexeme{     təta}\textstylefstandard{.}
\end{styleEntryParagraph}

\begin{styleEntryParagraph}
\textstyleDefinitionn{above}\textstyleLexeme{     avəlo}\textstyleDefinitionn{.}
\end{styleEntryParagraph}

\begin{styleEntryParagraph}
\textstyleDefinitionn{accept, catch}\textstyleLexeme{    gas}\textstyleSensenumber{.}
\end{styleEntryParagraph}

\begin{styleEntryParagraph}
\textstyleDefinitionn{accompany}\textstyleLexeme{     lagay}\textstylefstandard{.}
\end{styleEntryParagraph}

\begin{styleEntryParagraph}
\textstyleDefinitionn{actually}\textstyleLexeme{     ɗəwge}\textstylefstandard{.}
\end{styleEntryParagraph}

\begin{styleEntryParagraph}
\textstyleDefinitionn{adultery}\textstyleLexeme{     adama}\textstylefstandard{.} 
\end{styleEntryParagraph}

\begin{styleEntryParagraph}
\textstyleDefinitionn{again}\textstyleLexeme{     ese}\textstylefstandard{.}
\end{styleEntryParagraph}

\begin{styleEntryParagraph}
\textstyleDefinitionn{agreed}\textstyleLexeme{     ayokon}\textstylefstandard{.}
\end{styleEntryParagraph}

\begin{styleEntryParagraph}
\textstyleDefinitionn{all}\textstyleLexeme{     cece, jəyga, pat}\textstylefstandard{.}
\end{styleEntryParagraph}

\begin{styleEntryParagraph}
\textstyleDefinitionn{all night}\textstyleLexeme{     vəɗ vaɗ}\textstylefstandard{.}
\end{styleEntryParagraph}

\begin{styleEntryParagraph}
\textstyleDefinitionn{already}\textstyleLexeme{     fan}\textstylefstandard{.}
\end{styleEntryParagraph}

\begin{styleEntryParagraph}
\textstyleDefinitionn{also}\textstyleLexeme{     ɗəw}\textstylefstandard{.}
\end{styleEntryParagraph}

\begin{styleEntryParagraph}
\textstyleDefinitionn{also, as well}\textstyleLexeme{     ete}\textstylefstandard{.}
\end{styleEntryParagraph}

\begin{styleEntryParagraph}
\textstyleDefinitionn{and}\textstyleLexeme{     nata}\textstyleDefinitionn{.}
\end{styleEntryParagraph}

\begin{styleEntryParagraph}
\textstyleDefinitionn{anger}\textstyleLexeme{     mogo}\textstylefstandard{.}
\end{styleEntryParagraph}

\begin{styleEntryParagraph}
\textstyleDefinitionn{animal}\textstyleLexeme{     gənaw}\textstylefstandard{.}
\end{styleEntryParagraph}

\begin{styleEntryParagraph}
\textstyleDefinitionn{announce, }\textit{p}\textstyleDefinitionn{ublish}\textstyleLexeme{     wəzlay}\textstyleDefinitionn{.}
\end{styleEntryParagraph}

\begin{styleEntryParagraph}
\textstyleDefinitionn{another}\textstyleLexeme{   enen}\textstylefstandard{.}
\end{styleEntryParagraph}

\begin{styleEntryParagraph}
\textstyleDefinitionn{ant species}\textstyleLexeme{ }\textstyleLexeme{margaba,}\textstyleLexeme{ mesesewk}\textstylefstandard{.}
\end{styleEntryParagraph}

\begin{styleEntryParagraph}
\textstyleDefinitionn{appear, go out}\textstyleLexeme{     zləray}\textstylefstandard{.}
\end{styleEntryParagraph}

\begin{styleEntryParagraph}
\textstyleDefinitionn{argue, scold}\textstyleLexeme{     mbe}\textstylefstandard{.}
\end{styleEntryParagraph}

\begin{styleEntryParagraph}
\textstyleDefinitionn{arrive}\textstyleLexeme{     dokay}\textstylefstandard{.}
\end{styleEntryParagraph}

\begin{styleEntryParagraph}
\textstyleDefinitionn{arrow}\textstyleLexeme{     ava}\textstylefstandard{.}
\end{styleEntryParagraph}

\begin{styleEntryParagraph}
\textstyleDefinitionn{ask}\textstyleLexeme{     cahay}\textstylefstandard{.}
\end{styleEntryParagraph}

\begin{styleEntryParagraph}
\textstyleDefinitionn{ask for}\textstyleLexeme{     cəfəɗay}\textstylefstandard{.}
\end{styleEntryParagraph}

\begin{styleEntryParagraph}
\textstyleDefinitionn{assemble, unite}\textstyleLexeme{     cəkalay}\textstylefstandard{.}
\end{styleEntryParagraph}

\begin{styleEntryParagraph}
\textstyleDefinitionn{at the house of}\textstyleLexeme{     afa}\textstylefstandard{.} 
\end{styleEntryParagraph}

\begin{styleEntryParagraph}
\textstyleDefinitionn{attach, hurt}\textstyleLexeme{     wal}\textstylefstandard{.}
\end{styleEntryParagraph}

\begin{styleEntryParagraph}
\textstyleDefinitionn{axe}\textstyleLexeme{   ozlərgo,  zlərgo}\textstylefstandard{.}
\end{styleEntryParagraph}

\begin{styleEntryParagraph}
\textstyleDefinitionn{axe, small}\textstyleLexeme{     gwəlek}\textstylefstandard{.}
\end{styleEntryParagraph}
\end{multicols}
\begin{multicols}{2}
\end{multicols}
\begin{styleLetterParagraph}
\textstyleLetterv{B  {}-  b }
\end{styleLetterParagraph}

\begin{multicols}{2}
\begin{styleEntryParagraph}
\textstyleDefinitionn{baboon}\textstyleLexeme{     bəway}\textstylefstandard{.}
\end{styleEntryParagraph}

\begin{styleEntryParagraph}
\textstyleDefinitionn{baboon}\textstyleLexeme{     hərgov}\textstylefstandard{.}
\end{styleEntryParagraph}

\begin{styleEntryParagraph}
\textstyleDefinitionn{back}\textstyleLexeme{     hwəlen}\textstylefstandard{.}
\end{styleEntryParagraph}

\begin{styleEntryParagraph}
\textstyleDefinitionn{banana}\textstyleLexeme{     kondon}\textstylefstandard{.}
\end{styleEntryParagraph}

\begin{styleEntryParagraph}
\textstyleDefinitionn{bat}\textstyleLexeme{     mebebek}\textstylefstandard{.}
\end{styleEntryParagraph}

\begin{styleEntryParagraph}
\textstyleDefinitionn{be bald, cultivate a second time}\textstyleLexeme{     bokay}\textstylefstandard{.}
\end{styleEntryParagraph}

\begin{styleEntryParagraph}
\textstyleDefinitionn{be beautiful}\textstyleLexeme{     rəɓay}\textstylefstandard{.}
\end{styleEntryParagraph}

\begin{styleEntryParagraph}
\textstyleDefinitionn{be heavy/honourable}\textstyleLexeme{     ɗas}\textstylefstandard{.} 
\end{styleEntryParagraph}

\begin{styleEntryParagraph}
\textstyleDefinitionn{be in conflict}\textstyleLexeme{     ngərzlay}\textstylefstandard{.} 
\end{styleEntryParagraph}

\begin{styleEntryParagraph}
\textstyleDefinitionn{be in process of}\textstyleLexeme{     nday}\textstylefstandard{.}
\end{styleEntryParagraph}

\begin{styleEntryParagraph}
\textstyleDefinitionn{be insufficient, lack}\textstyleLexeme{     ce}\textstylefstandard{.}
\end{styleEntryParagraph}

\begin{styleEntryParagraph}
\textit{be} \textstyleDefinitionn{roughcast}\textstyleLexeme{     kapay}\textstylefstandard{.}
\end{styleEntryParagraph}

\begin{styleEntryParagraph}
\textstyleDefinitionn{be sticky}\textstyleLexeme{     təlɓaway}\textstylefstandard{.}
\end{styleEntryParagraph}

\begin{styleEntryParagraph}
\textstyleDefinitionn{bean}\textstyleLexeme{     hahar}\textstylefstandard{.}
\end{styleEntryParagraph}

\begin{styleEntryParagraph}
\textstyleDefinitionn{beat lightly}\textstyleLexeme{     mbocoy}\textstylefstandard{.}
\end{styleEntryParagraph}

\begin{styleEntryParagraph}
\textstyleDefinitionn{because}\textstyleLexeme{     bəyna, waya}\textstylefstandard{.}
\end{styleEntryParagraph}

\begin{styleEntryParagraph}
\textstyleDefinitionn{because, that is}\textstyleLexeme{     kəwaya}\textstylefstandard{.}
\end{styleEntryParagraph}

\begin{styleEntryParagraph}
\textstyleDefinitionn{become drunk}\textstyleLexeme{     kəway}\textstylefstandard{.}
\end{styleEntryParagraph}

\begin{styleEntryParagraph}
\textstylefvernacular{\textmd{\textit{bee}}}\textstylefvernacular{     war omom}\textstylefstandard{.}
\end{styleEntryParagraph}

\begin{styleEntryParagraph}
\textstyleDefinitionn{before}\textstyleLexeme{     kəlo}\textstylefstandard{.}
\end{styleEntryParagraph}

\begin{styleEntryParagraph}
\textstyleDefinitionn{beetle}\textstyleLexeme{     hohom}\textstylefstandard{.}
\end{styleEntryParagraph}

\begin{styleEntryParagraph}
\textstyleDefinitionn{beg}\textstyleLexeme{     hərkay}\textstylefstandard{.}
\end{styleEntryParagraph}

\begin{styleEntryParagraph}
\textstyleDefinitionn{behind}\textstyleLexeme{     həlan}\textstylefstandard{.}
\end{styleEntryParagraph}

\begin{styleEntryParagraph}
\textstyleDefinitionn{below}\textstyleLexeme{     asəbo}\textstyleDefinitionn{.}
\end{styleEntryParagraph}

\begin{styleEntryParagraph}
\textstyleDefinitionn{bend over}\textstyleLexeme{     gəɓokoy}\textstylefstandard{.}
\end{styleEntryParagraph}

\begin{styleEntryParagraph}
\textstyleDefinitionn{benefit of}\textstyleLexeme{     kəla}\textstylefstandard{.}
\end{styleEntryParagraph}

\begin{styleEntryParagraph}
\textstyleDefinitionn{betray}\textstyleLexeme{     cefe}\textstylefstandard{.}
\end{styleEntryParagraph}

\begin{styleEntryParagraph}
\textstyleDefinitionn{better}\textstyleLexeme{     ngama}\textstylefstandard{.}
\end{styleEntryParagraph}

\begin{styleEntryParagraph}
\textstyleDefinitionn{bicep}\textstyleLexeme{     mədəra}\textstylefstandard{.}
\end{styleEntryParagraph}

\begin{styleEntryParagraph}
\textstyleDefinitionn{bicycle}\textstyleLexeme{     baskwar}\textstylefstandard{.}
\end{styleEntryParagraph}

\begin{styleEntryParagraph}
\textstyleDefinitionn{bird species}\textstyleLexeme{     kəlbawak, eɗəyen}\textstylefstandard{.}
\end{styleEntryParagraph}

\begin{styleEntryParagraph}
\textstyleDefinitionn{birth}\textstyleLexeme{     we}\textstylefstandard{.}
\end{styleEntryParagraph}

\begin{styleEntryParagraph}
\textstyleDefinitionn{blackness}\textstyleLexeme{     dedəlen}\textstylefstandard{.}
\end{styleEntryParagraph}

\begin{styleEntryParagraph}
\textstyleDefinitionn{blemish}\textstyleLexeme{     aɓəsay}\textstylefstandard{.}
\end{styleEntryParagraph}

\begin{styleEntryParagraph}
\textstyleDefinitionn{blessing    }\textstyleLexeme{barka}\textstylefstandard{.}
\end{styleEntryParagraph}

\begin{styleEntryParagraph}
\textstyleDefinitionn{blink quickly}\textstyleLexeme{     kəɓəcay}\textstylefstandard{.}
\end{styleEntryParagraph}

\begin{styleEntryParagraph}
\textstyleDefinitionn{blink slowly, break violently}\textstyleLexeme{     mbəramay}\textstylefstandard{.}
\end{styleEntryParagraph}

\begin{styleEntryParagraph}
\textstyleDefinitionn{block up}\textstyleLexeme{     rəcoy}\textstylefstandard{.}
\end{styleEntryParagraph}

\begin{styleEntryParagraph}
\textstyleDefinitionn{blood}\textstyleLexeme{     pembez}\textstylefstandard{.}
\end{styleEntryParagraph}

\begin{styleEntryParagraph}
\textstyleDefinitionn{boa}\textstyleLexeme{     tezeh}\textstylefstandard{.}
\end{styleEntryParagraph}

\begin{styleEntryParagraph}
\textstyleDefinitionn{board}\textstyleLexeme{     slərah}\textstylefstandard{.}
\end{styleEntryParagraph}

\begin{styleEntryParagraph}
\textstyleDefinitionn{body}\textstyleLexeme{     hərva}\textstylefstandard{.}
\end{styleEntryParagraph}

\begin{styleEntryParagraph}
\textstyleDefinitionn{body}\textstyleLexeme{     har}\textstylefstandard{.}
\end{styleEntryParagraph}

\begin{styleEntryParagraph}
\textstyleDefinitionn{body}\textstyleLexeme{       va}\textstylefstandard{.}
\end{styleEntryParagraph}

\begin{styleEntryParagraph}
\textstyleDefinitionn{boil}\textstyleLexeme{     vəlay, vərɗay}\textstylefstandard{.}
\end{styleEntryParagraph}

\begin{styleEntryParagraph}
\textstyleDefinitionn{bone}\textstyleLexeme{     kəlakasl}\textstylefstandard{.}
\end{styleEntryParagraph}

\begin{styleEntryParagraph}
\textstyleDefinitionn{book}\textstylefstandard{     }\textstyleLexeme{deftere}\textstylefstandard{.}
\end{styleEntryParagraph}

\begin{styleEntryParagraph}
\textstyleDefinitionn{boost}\textstyleLexeme{     tahay}\textstylefstandard{.}
\end{styleEntryParagraph}

\begin{styleEntryParagraph}
\textstyleDefinitionn{bottom}\textstyleLexeme{     mətenen}\textstylefstandard{.}
\end{styleEntryParagraph}

\begin{styleEntryParagraph}
\textstyleDefinitionn{bracelet}\textstyleLexeme{     emelek}\textstylefstandard{.}
\end{styleEntryParagraph}

\begin{styleEntryParagraph}
\textstyleDefinitionn{bracelet, weapon}\textstyleLexeme{     alahar}\textstylefstandard{.}
\end{styleEntryParagraph}

\begin{styleEntryParagraph}
\textstyleDefinitionn{braid}\textstyleLexeme{     slapay}\textstylefstandard{.}
\end{styleEntryParagraph}

\begin{styleEntryParagraph}
\textstyleDefinitionn{brain; wisdom}\textstyleLexeme{     endeɓ}\textstylefstandard{.}
\end{styleEntryParagraph}

\begin{styleEntryParagraph}
\textstyleDefinitionn{break}\textstyleLexeme{   haɓ,  pasl}\textstylefstandard{.}
\end{styleEntryParagraph}

\begin{styleEntryParagraph}
\textstyleDefinitionn{break, tear away}\textstyleLexeme{     ngərway}\textstylefstandard{.}
\end{styleEntryParagraph}

\begin{styleEntryParagraph}
\textstyleDefinitionn{break violently, blink slowly}\textstyleLexeme{     mbəramay}\textstylefstandard{.}
\end{styleEntryParagraph}

\begin{styleEntryParagraph}
\textstylefstandard{\textit{breast, }}\textstyleDefinitionn{milk}\textstyleLexeme{     ɗəwa}\textstylefstandard{.}
\end{styleEntryParagraph}

\begin{styleEntryParagraph}
\textstyleDefinitionn{breathe}\textstyleLexeme{     bazlay}\textstylefstandard{.}
\end{styleEntryParagraph}

\begin{styleEntryParagraph}
\textstyleDefinitionn{breathe, rest}\textstyleLexeme{     mbesen}\textstyleDefinitionn{.}
\end{styleEntryParagraph}

\begin{styleEntryParagraph}
\textstyleDefinitionn{bride price}\textstyleLexeme{     gembere}\textstyleDefinitionn{.}
\end{styleEntryParagraph}

\begin{styleEntryParagraph}
\textstyleDefinitionn{bring}\textstyleLexeme{     pəcay}\textstylefstandard{.}
\end{styleEntryParagraph}

\begin{styleEntryParagraph}
\textstyleDefinitionn{broom}\textstyleLexeme{     səlewk}\textstylefstandard{.}
\end{styleEntryParagraph}

\begin{styleEntryParagraph}
\textstyleDefinitionn{brush}\textstyleLexeme{     gohoy}\textstylefstandard{.}
\end{styleEntryParagraph}

\begin{styleEntryParagraph}
\textstyleDefinitionn{bucket}\textstyleLexeme{     cafgal}\textstylefstandard{.}
\end{styleEntryParagraph}

\begin{styleEntryParagraph}
\textstyleDefinitionn{bud, uproot}\textstyleLexeme{     tosoy}\textstylefstandard{.}
\end{styleEntryParagraph}

\begin{styleEntryParagraph}
\textstyleDefinitionn{build}\textstyleLexeme{     ɓalay}\textstylefstandard{.}
\end{styleEntryParagraph}

\begin{styleEntryParagraph}
\textstyleDefinitionn{build up to}\textstyleLexeme{     ɓelen}\textstylefstandard{.}
\end{styleEntryParagraph}

\begin{styleEntryParagraph}
\textstyleDefinitionn{bull}\textstyleLexeme{     gəsan}\textstylefstandard{.}
\end{styleEntryParagraph}

\begin{styleEntryParagraph}
\textstyleDefinitionn{bump}\textstyleLexeme{     dafay}\textstylefstandard{.}
\end{styleEntryParagraph}

\begin{styleEntryParagraph}
\textstyleDefinitionn{burn}\textstyleLexeme{   dar,  ngəɗay, vakay}\textstylefstandard{.}
\end{styleEntryParagraph}

\begin{styleEntryParagraph}
\textstyleDefinitionn{bush, fields}\textstyleLexeme{     ləhe}\textstylefstandard{.}
\end{styleEntryParagraph}

\begin{styleEntryParagraph}
\textstyleDefinitionn{but}\textstyleLexeme{     azla na}\textstylefstandard{.}
\end{styleEntryParagraph}

\begin{styleEntryParagraph}
\textstyleDefinitionn{butt with horns}\textstyleLexeme{     ngəɗacay}\textstylefstandard{.}
\end{styleEntryParagraph}

\begin{styleEntryParagraph}
\textstyleDefinitionn{butter}\textstyleLexeme{     wərsla}\textstylefstandard{.}
\end{styleEntryParagraph}

\begin{styleEntryParagraph}
\textstyleDefinitionn{butterfly}\textstyleLexeme{     mepetəpete}\textstylefstandard{.}
\end{styleEntryParagraph}

\begin{styleEntryParagraph}
\textstyleDefinitionn{buy/sell}\textstyleLexeme{     səkom}\textstylePartofspeech{.} 
\end{styleEntryParagraph}
\end{multicols}
\begin{styleLetterParagraph}
\textstyleLetterv{C  {}-  c}
\end{styleLetterParagraph}

\begin{multicols}{2}
\begin{styleEntryParagraph}
\textstyleDefinitionn{call}\textstyleLexeme{     mbahay, taray}\textstylefstandard{.}
\end{styleEntryParagraph}

\begin{styleEntryParagraph}
\textstyleDefinitionn{camel}\textstyleLexeme{     ezlegweme}\textstylefstandard{.}
\end{styleEntryParagraph}

\begin{styleEntryParagraph}
\textstyleDefinitionn{cancel, wipe out}\textstyleLexeme{     vasay}\textstylefstandard{.}
\end{styleEntryParagraph}

\begin{styleEntryParagraph}
\textstyleSensenumber{\textit{care}}\textstyleLexeme{     zayəh}\textstylefstandard{.}
\end{styleEntryParagraph}

\begin{styleEntryParagraph}
\textstyleDefinitionn{carry, take}\textstyleLexeme{     zaɗ}\textstylePartofspeech{.}
\end{styleEntryParagraph}

\begin{styleEntryParagraph}
\textstyleDefinitionn{castrate}\textstyleLexeme{     caɗay}\textstylefstandard{.}
\end{styleEntryParagraph}

\begin{styleEntryParagraph}
\textstyleDefinitionn{castrate, sterilize}\textstyleLexeme{     daslay}\textstylefstandard{.}
\end{styleEntryParagraph}

\begin{styleEntryParagraph}
\textstyleDefinitionn{cat}\textstyleLexeme{     pataw}\textstylefstandard{.}
\end{styleEntryParagraph}

\begin{styleEntryParagraph}
\textstyleDefinitionn{cat, wild}\textstyleLexeme{     məngamak}\textstylefstandard{.}
\end{styleEntryParagraph}

\begin{styleEntryParagraph}
\textstyleDefinitionn{catch, accept}\textstyleLexeme{     gas}\textstyleSensenumber{.}
\end{styleEntryParagraph}

\begin{styleEntryParagraph}
\textstyleDefinitionn{celebration (lit. planting fire)    }\textstyleLexeme{məjəvoko}\textstylefstandard{.}
\end{styleEntryParagraph}

\begin{styleEntryParagraph}
\textstyleDefinitionn{centre, middle}\textstyleLexeme{     mbeɗem}\textstylefstandard{.}
\end{styleEntryParagraph}

\begin{styleEntryParagraph}
\textstyleDefinitionn{chain}\textstyleLexeme{     celelew}\textstylefstandard{.}
\end{styleEntryParagraph}

\begin{styleEntryParagraph}
\textstyleDefinitionn{chameleon}\textstyleLexeme{   mozongo}\textstylefstandard{.}
\end{styleEntryParagraph}

\begin{styleEntryParagraph}
\textstyleDefinitionn{change}\textstyleLexeme{     mbaɗ}\textstylefstandard{.} 
\end{styleEntryParagraph}

\begin{styleEntryParagraph}
\textstyleDefinitionn{chase, herd}\textstyleLexeme{     galay}\textstylefstandard{.}
\end{styleEntryParagraph}

\begin{styleEntryParagraph}
\textstyleDefinitionn{chase away}\textstyleLexeme{     varay}\textstylefstandard{.}
\end{styleEntryParagraph}

\begin{styleEntryParagraph}
\textstyleDefinitionn{cheek}\textstyleLexeme{     bozlom}\textstylefstandard{.}
\end{styleEntryParagraph}

\begin{styleEntryParagraph}
\textstyleDefinitionn{chest, heart}\textstyleLexeme{     tololon}\textstyleDefinitionn{.}
\end{styleEntryParagraph}

\begin{styleEntryParagraph}
\textstyleDefinitionn{chew}\textstyleLexeme{     kərɗay}\textstylefstandard{.}
\end{styleEntryParagraph}

\begin{styleEntryParagraph}
\textstyleDefinitionn{chicken}\textstyleLexeme{     anjakar}\textstylefstandard{.}
\end{styleEntryParagraph}

\begin{styleEntryParagraph}
\textstyleDefinitionn{chief}\textstyleLexeme{     bahay}\textstylefstandard{.}
\end{styleEntryParagraph}

\begin{styleEntryParagraph}
\textstyleDefinitionn{child}\textstyleLexeme{     war}\textstylefstandard{.}
\end{styleEntryParagraph}

\begin{styleEntryParagraph}
\textstyleDefinitionn{child, oldest}\textstyleLexeme{     morkoyo}\textstylePartofspeech{.} 
\end{styleEntryParagraph}

\begin{styleEntryParagraph}
\textstyleDefinitionn{child, youngest}\textstyleLexeme{     gwədar}\textstylefstandard{.}
\end{styleEntryParagraph}

\begin{styleEntryParagraph}
\textstyleDefinitionn{children}\textstyleLexeme{     babəza ahay}\textstylefstandard{.}
\end{styleEntryParagraph}

\begin{styleEntryParagraph}
\textstyleDefinitionn{choose}\textstyleLexeme{     palay}\textstylefstandard{.}
\end{styleEntryParagraph}

\begin{styleEntryParagraph}
\textstyleDefinitionn{chop}\textstyleLexeme{     pəɗakay}\textstylefstandard{.}
\end{styleEntryParagraph}

\begin{styleEntryParagraph}
\textstyleDefinitionn{chop, cut}\textstyleLexeme{     cəzlahay}\textstylefstandard{.}
\end{styleEntryParagraph}

\begin{styleEntryParagraph}
\textstyleDefinitionn{cicada}\textstyleLexeme{     mətəde}\textstylefstandard{.}
\end{styleEntryParagraph}

\begin{styleEntryParagraph}
\textstyleDefinitionn{clan}\textstyleLexeme{     səkoy}\textstylefstandard{.}
\end{styleEntryParagraph}

\begin{styleEntryParagraph}
\textstyleDefinitionn{claw, nail  }\textstyleLexeme{ ehwəɗe}\textstylePartofspeech{.} 
\end{styleEntryParagraph}

\begin{styleEntryParagraph}
\textstyleDefinitionn{clear}\textstyleLexeme{     caɗay}\textstylefstandard{.}
\end{styleEntryParagraph}

\begin{styleEntryParagraph}
\textstyleDefinitionn{climb}\textstyleLexeme{     ɓoray}\textstylefstandard{.}
\end{styleEntryParagraph}

\begin{styleEntryParagraph}
\textstyleDefinitionn{climb}\textstyleLexeme{     car}\textstylefstandard{.}
\end{styleEntryParagraph}

\begin{styleEntryParagraph}
\textstyleDefinitionn{close}\textstyleLexeme{     tacay}\textstylefstandard{.}
\end{styleEntryParagraph}

\begin{styleEntryParagraph}
\textstyleDefinitionn{clothes}\textstyleLexeme{     kəmeje}\textstylefstandard{.}
\end{styleEntryParagraph}

\begin{styleEntryParagraph}
\textstyleSensenumber{\textit{clothes, cloth}}\textstyleLexeme{     zana}\textstylefstandard{.}
\end{styleEntryParagraph}

\begin{styleEntryParagraph}
\textstyleDefinitionn{cloud}\textstyleLexeme{     mataɓasl}\textstylefstandard{.}
\end{styleEntryParagraph}

\begin{styleEntryParagraph}
\textstyleDefinitionn{coin (5 francs)}\textstyleLexeme{ səy\nobreakdash-say}\textstylefstandard{.}
\end{styleEntryParagraph}

\begin{styleEntryParagraph}
\textstyleDefinitionn{cold/flu}\textstyleLexeme{     mədegen}\textstylefstandard{.}
\end{styleEntryParagraph}

\begin{styleEntryParagraph}
\textstyleDefinitionn{collect}\textstyleLexeme{     har}\textstylefstandard{.}
\end{styleEntryParagraph}

\begin{styleEntryParagraph}
\textstyleDefinitionn{collect, squeeze}\textstyleLexeme{     bərkaday}\textstylefstandard{.}
\end{styleEntryParagraph}

\begin{styleEntryParagraph}
\textstyleDefinitionn{comb, separate}\textstyleLexeme{ njaray}\textstylefstandard{.}
\end{styleEntryParagraph}

\begin{styleEntryParagraph}
\textit{co}\textstyleDefinitionn{me back}\textstyleLexeme{     ngala}\textstyleDefinitionn{.}
\end{styleEntryParagraph}

\begin{styleEntryParagraph}
\textstyleDefinitionn{command, frighten}\textstyleLexeme{     garay}\textstylefstandard{.} 
\end{styleEntryParagraph}

\begin{styleEntryParagraph}
\textstyleDefinitionn{compassion}\textstyleLexeme{     ercece}\textstylefstandard{.}
\end{styleEntryParagraph}

\begin{styleEntryParagraph}
\textstyleDefinitionn{constipate}\textstyleLexeme{     gabay}\textstylefstandard{.}
\end{styleEntryParagraph}

\begin{styleEntryParagraph}
\textstyleDefinitionn{construct}\textstyleLexeme{     har}\textstylefstandard{.}
\end{styleEntryParagraph}

\begin{styleEntryParagraph}
\textit{cook}, \textstyleDefinitionn{prepare}\textstyleLexeme{     de}\textstylefstandard{.}
\end{styleEntryParagraph}

\begin{styleEntryParagraph}
\textstyleDefinitionn{cook on fire}\textstyleLexeme{     səloy}\textstylefstandard{.}
\end{styleEntryParagraph}

\begin{styleEntryParagraph}
\textstyleDefinitionn{cook or stir quickly next to fire}\textstyleLexeme{     kaɓay}\textstylefstandard{.}
\end{styleEntryParagraph}

\begin{styleEntryParagraph}
\textstyleDefinitionn{cord}\textstyleLexeme{     ezeweɗ}\textstylefstandard{.}
\end{styleEntryParagraph}

\begin{styleEntryParagraph}
\textstyleDefinitionn{cotton}\textstyleLexeme{     gəgəmay}\textstylefstandard{.}
\end{styleEntryParagraph}

\begin{styleEntryParagraph}
\textstyleDefinitionn{cough}\textstyleLexeme{     ɓəslay}\textstylefstandard{.}
\end{styleEntryParagraph}

\begin{styleEntryParagraph}
\textstyleDefinitionn{count}\textstyleLexeme{     ɓezlen, mbezlen}\textstylefstandard{.} 
\end{styleEntryParagraph}

\begin{styleEntryParagraph}
\textstyleDefinitionn{cow}\textstyleLexeme{     sla}\textstylefstandard{.}
\end{styleEntryParagraph}

\begin{styleEntryParagraph}
\textstyleDefinitionn{crawl}\textstyleLexeme{     bəjəgamay}\textstylefstandard{.}
\end{styleEntryParagraph}

\begin{styleEntryParagraph}
\textstyleDefinitionn{crocodile}\textstyleLexeme{     kəramba}\textstylefstandard{.}
\end{styleEntryParagraph}

\begin{styleEntryParagraph}
\textstyleDefinitionn{cross}\textstyleLexeme{     təkasay}\textstylefstandard{.}
\end{styleEntryParagraph}

\begin{styleEntryParagraph}
\textstyleDefinitionn{cross, fold}\textstyleLexeme{     təkosoy}\textstylefstandard{.}
\end{styleEntryParagraph}

\begin{styleEntryParagraph}
\textstyleDefinitionn{cross ankles}\textstyleLexeme{     səlɗay}\textstylefstandard{.}
\end{styleEntryParagraph}

\begin{styleEntryParagraph}
\textstyleDefinitionn{crouch, squat}\textstyleLexeme{     cəɗokay}\textstylefstandard{.}
\end{styleEntryParagraph}

\begin{styleEntryParagraph}
\textstyleDefinitionn{crow}\textstyleLexeme{     məngahak}\textstylefstandard{.}
\end{styleEntryParagraph}

\begin{styleEntryParagraph}
\textstyleDefinitionn{crunch}\textstyleLexeme{   paɗay}\textstylefstandard{.}
\end{styleEntryParagraph}

\begin{styleEntryParagraph}
\textstyleDefinitionn{crush, pound}\textstyleLexeme{     zlaɓay}\textstylefstandard{.}
\end{styleEntryParagraph}

\begin{styleEntryParagraph}
\textstyleDefinitionn{cry (noun)    }\textstyleLexeme{təwe}\textstylefstandard{.}
\end{styleEntryParagraph}

\begin{styleEntryParagraph}
\textstyleDefinitionn{cry (verb)    }\textstyleLexeme{təway}\textstylefstandard{.}
\end{styleEntryParagraph}

\begin{styleEntryParagraph}
\textstyleDefinitionn{cry (dog or rooster)    }\textstyleLexeme{zlah}\textstylefstandard{.}
\end{styleEntryParagraph}

\begin{styleEntryParagraph}
\textstyleDefinitionn{cucumber}\textstyleLexeme{ kərsay}\textstylefstandard{.}
\end{styleEntryParagraph}

\begin{styleEntryParagraph}
\textstyleDefinitionn{cultivate}\textstyleLexeme{     was}\textstylefstandard{.}
\end{styleEntryParagraph}

\begin{styleEntryParagraph}
\textstyleDefinitionn{cultívate second time}\textstyleLexeme{     kərway}\textstylefstandard{.}
\end{styleEntryParagraph}

\begin{styleEntryParagraph}
\textstyleDefinitionn{cultivate a second time; be bald}\textstyleLexeme{     bokay}\textstylefstandard{.}
\end{styleEntryParagraph}

\begin{styleEntryParagraph}
\textit{cultivated} \textstyleDefinitionn{field}\textstyleLexeme{     gəvah}\textstylefstandard{.}
\end{styleEntryParagraph}

\begin{styleEntryParagraph}
\textstyleDefinitionn{cunning}\textstyleLexeme{     wewer}\textstylefstandard{.}
\end{styleEntryParagraph}

\begin{styleEntryParagraph}
\textstyleDefinitionn{curse}\textstyleLexeme{     taslay}\textstylefstandard{.}
\end{styleEntryParagraph}

\begin{styleEntryParagraph}
\textstyleDefinitionn{cut}\textstyleLexeme{     səya}\textstylefstandard{.} 
\end{styleEntryParagraph}

\begin{styleEntryParagraph}
\textstyleDefinitionn{cut, chop}\textstyleLexeme{     cəzlahay}\textstylefstandard{.}
\end{styleEntryParagraph}

\begin{styleEntryParagraph}
\textstyleDefinitionn{cut, sore}\textstyleLexeme{     ambəlak}\textstylefstandard{.}
\end{styleEntryParagraph}

\begin{styleEntryParagraph}
\textstyleDefinitionn{cut, pierce}\textstyleLexeme{     cazlay}\textstylefstandard{.} 
\end{styleEntryParagraph}

\begin{styleEntryParagraph}
\textstyleDefinitionn{cut, want}\textstyleLexeme{     say}\textstylefstandard{.}
\end{styleEntryParagraph}

\begin{styleEntryParagraph}
\textstyleDefinitionn{cut off head}\textstyleLexeme{     caway}\textstylefstandard{.}
\end{styleEntryParagraph}
\end{multicols}
\begin{styleLetterParagraph}
\textstyleLetterv{D  {}-  d}
\end{styleLetterParagraph}

\begin{multicols}{2}
\begin{styleEntryParagraph}
\textstyleDefinitionn{dance}\textstyleLexeme{     haɓay}\textstylefstandard{.}
\end{styleEntryParagraph}

\begin{styleEntryParagraph}
\textstyleDefinitionn{darkness}\textstyleLexeme{     zən zen}\textstylefstandard{.}
\end{styleEntryParagraph}

\begin{styleEntryParagraph}
\textstyleDefinitionn{dawn, light}\textstyleLexeme{     jajay}\textstylefstandard{.}
\end{styleEntryParagraph}

\begin{styleEntryParagraph}
\textstyleSensenumber{\textit{day}}\textstyleLexeme{     məndəye}\textstylefstandard{.}
\end{styleEntryParagraph}

\begin{styleEntryParagraph}
\textstyleDefinitionn{dear}\textstyleLexeme{     golo}\textstylefstandard{.}
\end{styleEntryParagraph}

\begin{styleEntryParagraph}
\textstyleDefinitionn{debt}\textstyleLexeme{     dəwa}\textstylefstandard{.}
\end{styleEntryParagraph}

\begin{styleEntryParagraph}
\textstyleDefinitionn{decimate, kill many}\textstyleLexeme{     pazlay}\textstylefstandard{.}
\end{styleEntryParagraph}

\begin{styleEntryParagraph}
\textstyleDefinitionn{deer}\textstyleLexeme{     məyek}\textstylefstandard{.}
\end{styleEntryParagraph}

\begin{styleEntryParagraph}
\textstyleDefinitionn{defend}\textstyleLexeme{     ngəlay}\textstylefstandard{.}
\end{styleEntryParagraph}

\begin{styleEntryParagraph}
\textstyleDefinitionn{demolish}\textstyleLexeme{     mbazl}\textstylefstandard{.}
\end{styleEntryParagraph}

\begin{styleEntryParagraph}
\textstyleDefinitionn{descend}\textstyleLexeme{     fatay}\textstylefstandard{.}
\end{styleEntryParagraph}

\begin{styleEntryParagraph}
\textstyleDefinitionn{destroy}\textstyleLexeme{     hwəzlay}\textstylefstandard{.} 
\end{styleEntryParagraph}

\begin{styleEntryParagraph}
\textstyleDefinitionn{destroy violently}\textstyleLexeme{     mbərway}\textstyleDefinitionn{.}
\end{styleEntryParagraph}

\begin{styleEntryParagraph}
\textstyleDefinitionn{detach}\textstyleLexeme{     pəsakay}\textstylefstandard{.}
\end{styleEntryParagraph}

\begin{styleEntryParagraph}
\textstyleDefinitionn{detach, spread out}\textstyleLexeme{     pasay}\textstylefstandard{.}
\end{styleEntryParagraph}

\begin{styleEntryParagraph}
\textstyleDefinitionn{devour}\textstyleLexeme{     wəldoy}\textstylefstandard{.}
\end{styleEntryParagraph}

\begin{styleEntryParagraph}
\textstyleDefinitionn{die}\textstyleLexeme{     mat}\textstylefstandard{.}
\end{styleEntryParagraph}

\begin{styleEntryParagraph}
\textstyleDefinitionn{different}\textstyleLexeme{     tere}\textstylefstandard{.}
\end{styleEntryParagraph}

\begin{styleEntryParagraph}
\textstyleDefinitionn{difficulty}\textstyleLexeme{     gar}\textstylePartofspeech{.} 
\end{styleEntryParagraph}

\begin{styleEntryParagraph}
\textstyleDefinitionn{dig}\textstyleLexeme{     lay}\textstylefstandard{.}
\end{styleEntryParagraph}

\begin{styleEntryParagraph}
\textstyleDefinitionn{dig shallow}\textstyleLexeme{     bəjakay}\textstylefstandard{.}
\end{styleEntryParagraph}

\begin{styleEntryParagraph}
\textstyleDefinitionn{disease}\textstyleLexeme{     cəje}\textstylefstandard{.}
\end{styleEntryParagraph}

\begin{styleEntryParagraph}
\textit{dismantle}\textstyleLexeme{ walay}\textstylefstandard{.}
\end{styleEntryParagraph}

\begin{styleEntryParagraph}
\textstyleDefinitionn{disobedience}\textstyleLexeme{     cezlere}\textstylefstandard{.}
\end{styleEntryParagraph}

\begin{styleEntryParagraph}
\textstyleDefinitionn{divide, share}\textstyleLexeme{     wəɗakay}\textstylefstandard{.} 
\end{styleEntryParagraph}

\begin{styleEntryParagraph}
\textstyleDefinitionn{do}\textstyleLexeme{     ge}\textstylefstandard{.} 
\end{styleEntryParagraph}

\begin{styleEntryParagraph}
\textstyleDefinitionn{dog}\textstyleLexeme{     kəra}\textstylefstandard{.}
\end{styleEntryParagraph}

\begin{styleEntryParagraph}
\textstyleDefinitionn{donkey}\textstyleLexeme{     ozəngo}\textstylefstandard{.}
\end{styleEntryParagraph}

\begin{styleEntryParagraph}
\textstyleDefinitionn{donut}\textstyleLexeme{     makala}\textstylefstandard{.}
\end{styleEntryParagraph}

\begin{styleEntryParagraph}
\textstyleDefinitionn{donut made from ground nuts}\textstyleLexeme{     manjaw}\textstylefstandard{.}
\end{styleEntryParagraph}

\begin{styleEntryParagraph}
\textstyleDefinitionn{door}\textstyleLexeme{     mahay}\textstylefstandard{.}
\end{styleEntryParagraph}

\begin{styleEntryParagraph}
\textstyleDefinitionn{double, drape}\textstyleLexeme{     capay}\textstylefstandard{.}
\end{styleEntryParagraph}

\begin{styleEntryParagraph}
\textstyleDefinitionn{drape, double}\textstyleLexeme{     capay}\textstylefstandard{.}
\end{styleEntryParagraph}

\begin{styleEntryParagraph}
\textstyleDefinitionn{dregs}\textstyleLexeme{     hwə}\textstyleLexeme{ɗ}\textstyleLexeme{a}\textstylefstandard{.}
\end{styleEntryParagraph}

\begin{styleEntryParagraph}
\textstyleDefinitionn{drink}\textstyleLexeme{     se}\textstylefstandard{.}
\end{styleEntryParagraph}

\begin{styleEntryParagraph}
\textstyleDefinitionn{drip}\textstyleLexeme{     təlokoy}\textstylefstandard{.}
\end{styleEntryParagraph}

\begin{styleEntryParagraph}
\textstyleDefinitionn{drive}\textstyleLexeme{     bərwaɗay}\textstylefstandard{.}
\end{styleEntryParagraph}

\begin{styleEntryParagraph}
\textstyleDefinitionn{drop}\textstyleLexeme{     dav}\textstylePartofspeech{.} 
\end{styleEntryParagraph}

\begin{styleEntryParagraph}
\textstyleDefinitionn{dry}\textstyleLexeme{     koloy}\textstylefstandard{.}
\end{styleEntryParagraph}

\begin{styleEntryParagraph}
\textstyleDefinitionn{dry season}\textstyleLexeme{     almamar}\textstylefstandard{.}
\end{styleEntryParagraph}

\begin{styleEntryParagraph}
\textstyleDefinitionn{duck}\textstyleLexeme{     andəbaba}\textstylefstandard{.}
\end{styleEntryParagraph}
\end{multicols}
\begin{styleLetterParagraph}
\textstyleLetterv{E  {}-  e}
\end{styleLetterParagraph}

\begin{multicols}{2}
\begin{styleEntryParagraph}
\textstyleDefinitionn{ear, name}\textstyleLexeme{     sləmay}\textstylefstandard{.}
\end{styleEntryParagraph}

\begin{styleEntryParagraph}
\textstyleDefinitionn{earring}\textstyleLexeme{     sloko}\textstylefstandard{.}
\end{styleEntryParagraph}

\begin{styleEntryParagraph}
\textstyleDefinitionn{earth}\textstyleLexeme{     dəwnəya}\textstylefstandard{.}
\end{styleEntryParagraph}

\begin{styleEntryParagraph}
\textstyleDefinitionn{eat}\textstyleLexeme{     zom}\textstyleDefinitionn{.}
\end{styleEntryParagraph}

\begin{styleEntryParagraph}
\textstyleDefinitionn{economize, save}\textstyleLexeme{     johoy}\textstylefstandard{.}
\end{styleEntryParagraph}

\begin{styleEntryParagraph}
\textstyleDefinitionn{egg}\textstyleLexeme{     eslesleɗ}\textstylefstandard{.}
\end{styleEntryParagraph}

\begin{styleEntryParagraph}
\textstyleDefinitionn{egret}\textstyleLexeme{     dedewe}\textstylefstandard{.}
\end{styleEntryParagraph}

\begin{styleEntryParagraph}
\textstyleDefinitionn{eight}\textstyleLexeme{     slalakar}\textstylefstandard{.}
\end{styleEntryParagraph}

\begin{styleEntryParagraph}
\textstyleDefinitionn{elder}\textstyleLexeme{     gogor}\textstylefstandard{.}
\end{styleEntryParagraph}

\begin{styleEntryParagraph}
\textstyleDefinitionn{elephant}\textstyleLexeme{     mbəlele}\textstyleDefinitionn{.}
\end{styleEntryParagraph}

\begin{styleEntryParagraph}
\textstyleDefinitionn{emphasis}\textstyleLexeme{     dey}\textstylefstandard{.}
\end{styleEntryParagraph}

\begin{styleEntryParagraph}
\textstylePartofspeech{enjoy}\textstyleLexeme{     məlay}
\end{styleEntryParagraph}

\begin{styleEntryParagraph}
\textstyleDefinitionn{enough}\textstyleLexeme{     pew}\textstylefstandard{!}
\end{styleEntryParagraph}

\begin{styleEntryParagraph}
\textit{enough, }\textstyleDefinitionn{many}\textstyleLexeme{     haɗa}\textstylefstandard{.}
\end{styleEntryParagraph}

\begin{styleEntryParagraph}
\textstyleDefinitionn{enter}\textstyleLexeme{     tar}\textstylefstandard{.}
\end{styleEntryParagraph}

\begin{styleEntryParagraph}
\textstyleDefinitionn{evaporate}\textstyleLexeme{     batay}\textstylefstandard{.}
\end{styleEntryParagraph}

\begin{styleEntryParagraph}
\textstyleDefinitionn{even}\textstyleLexeme{     ko}\textstylefstandard{.}
\end{styleEntryParagraph}

\begin{styleEntryParagraph}
\textstyleDefinitionn{evening}\textstyleLexeme{ ləho}\textstylefstandard{.} 
\end{styleEntryParagraph}

\begin{styleEntryParagraph}
\textstyleDefinitionn{everywhere}\textstyleLexeme{     kəray}\textstylefstandard{.}
\end{styleEntryParagraph}

\begin{styleEntryParagraph}
\textstyleDefinitionn{exceed}\textstyleLexeme{     saɓay}\textstylefstandard{.}
\end{styleEntryParagraph}

\begin{styleEntryParagraph}
\textstyleDefinitionn{except}\textstyleLexeme{     səy}\textstylePartofspeech{.} 
\end{styleEntryParagraph}

\begin{styleEntryParagraph}
\textstyleDefinitionn{exclamation when surprised}\textstyleLexeme{     kay}\textstyleDefinitionn{.}
\end{styleEntryParagraph}

\begin{styleEntryParagraph}
\textstyleDefinitionn{excrement, faeces}\textstyleLexeme{     azay}\textstyleDefinitionn{.}
\end{styleEntryParagraph}

\begin{styleEntryParagraph}
\textstyleDefinitionn{existential}\textstyleLexeme{     aba}\textstylefstandard{.}
\end{styleEntryParagraph}

\begin{styleEntryParagraph}
\textstyleDefinitionn{existential}\textstyleLexeme{     abay}\textstylefstandard{.}
\end{styleEntryParagraph}

\begin{styleEntryParagraph}
\textstyleDefinitionn{existential}\textstyleLexeme{     ava}\textstylefstandard{.}
\end{styleEntryParagraph}

\begin{styleEntryParagraph}
\textstyleDefinitionn{explode}\textstyleLexeme{     ndozlay}\textstylefstandard{.}
\end{styleEntryParagraph}

\begin{styleEntryParagraph}
\textstyleDefinitionn{extinguish}\textstyleLexeme{     mbeten}\textstylefstandard{.}
\end{styleEntryParagraph}

\begin{styleEntryParagraph}
\textstyleDefinitionn{eye}\textstyleLexeme{     ele}\textstylefstandard{.} 
\end{styleEntryParagraph}
\end{multicols}
\begin{styleLetterParagraph}
\textstyleLetterv{F  {}-  f}
\end{styleLetterParagraph}

\begin{multicols}{2}
\begin{styleEntryParagraph}
\textstyleDefinitionn{faeces, excrement}\textstyleLexeme{     azay}\textstyleDefinitionn{.}
\end{styleEntryParagraph}

\begin{styleEntryParagraph}
\textstyleDefinitionn{fake}\textstyleLexeme{     dar}\textstyleDefinitionn{.}
\end{styleEntryParagraph}

\begin{styleEntryParagraph}
\textstyleDefinitionn{fall}\textstyleLexeme{     daɗ, taɗ}\textstylefstandard{.}
\end{styleEntryParagraph}

\begin{styleEntryParagraph}
\textstyleDefinitionn{fan}\textstyleLexeme{     pamay}\textstyleDefinitionn{.}
\end{styleEntryParagraph}

\begin{styleEntryParagraph}
\textstyleDefinitionn{far}\textstyleLexeme{     toho}\textstylefstandard{.}
\end{styleEntryParagraph}

\begin{styleEntryParagraph}
\textstyleDefinitionn{far away}\textstyleLexeme{     dəren}\textstylefstandard{.}
\end{styleEntryParagraph}

\begin{styleEntryParagraph}
\textstyleDefinitionn{fast}\textstyleLexeme{       jajak}\textstylefstandard{.}
\end{styleEntryParagraph}

\begin{styleEntryParagraph}
\textstyleDefinitionn{fat}\textstyleLexeme{     okos}\textstylefstandard{.}
\end{styleEntryParagraph}

\begin{styleEntryParagraph}
\textstyleDefinitionn{father}\textstyleLexeme{     baba}\textstylefstandard{.}
\end{styleEntryParagraph}

\begin{styleEntryParagraph}
\textstyleDefinitionn{fatten}\textstyleLexeme{     gədəgalay}\textstylefstandard{.}
\end{styleEntryParagraph}

\begin{styleEntryParagraph}
\textstyleDefinitionn{fear (noun)    }\textstyleLexeme{gəɓar}\textstylefstandard{.}
\end{styleEntryParagraph}

\begin{styleEntryParagraph}
\textstyleDefinitionn{fear (verb)}\textstyleLexeme{     zlaway}\textstylefstandard{.}
\end{styleEntryParagraph}

\begin{styleEntryParagraph}
\textstyleDefinitionn{fiancé}\textstyleLexeme{     mangasl}\textstylefstandard{.}
\end{styleEntryParagraph}

\begin{styleEntryParagraph}
\textstyleDefinitionn{fields, bush}\textstyleLexeme{     ləhe}\textstylefstandard{.}
\end{styleEntryParagraph}

\begin{styleEntryParagraph}
\textstyleDefinitionn{fig tree}\textstyleLexeme{     hərov}\textstylefstandard{.}
\end{styleEntryParagraph}

\begin{styleEntryParagraph}
\textit{fill,} \textstyleDefinitionn{satisfy}\textstyleLexeme{     rah}\textstylefstandard{.}
\end{styleEntryParagraph}

\begin{styleEntryParagraph}
\textstyleDefinitionn{find}\textstyleLexeme{     njakay}\textstyleDefinitionn{.}
\end{styleEntryParagraph}

\begin{styleIndentedParagraph}
\textstyleDefinitionn{finger}\textstyleSubentry{     war ahar}\textstylefstandard{.}
\end{styleIndentedParagraph}

\begin{styleIndentedParagraph}
\textstyleDefinitionn{fingers}\textstyleSubentry{     bəbəza}\textstyleSubentry{ ahar ahay}\textstylefstandard{.}
\end{styleIndentedParagraph}

\begin{styleEntryParagraph}
\textstyleSensenumber{\textit{fi}}\textstyleDefinitionn{nish}\textstyleLexeme{     ndavay}\textstylefstandard{.} 
\end{styleEntryParagraph}

\begin{styleEntryParagraph}
\textstyleDefinitionn{fire}\textstyleLexeme{     oko, mədara}\textstylefstandard{.}
\end{styleEntryParagraph}

\begin{styleEntryParagraph}
\textstyleDefinitionn{first (adv)}\textstyleLexeme{     əwɗe}\textstyleDefinitionn{.}
\end{styleEntryParagraph}

\begin{styleEntryParagraph}
\textstyleDefinitionn{first}\textstyleLexeme{     cecekem}\textstylefstandard{.}
\end{styleEntryParagraph}

\begin{styleEntryParagraph}
\textstyleDefinitionn{first pounding, tear to pieces}\textstyleLexeme{     ɓorcay}\textstylefstandard{.}
\end{styleEntryParagraph}

\begin{styleEntryParagraph}
\textstyleDefinitionn{fish}\textstyleLexeme{     kəlef}\textstylefstandard{.}
\end{styleEntryParagraph}

\begin{styleEntryParagraph}
\textstyleDefinitionn{fish net}\textstyleLexeme{     cokor}\textstylefstandard{.}
\end{styleEntryParagraph}

\begin{styleEntryParagraph}
\textstyleDefinitionn{fish species}\textstyleLexeme{     mombərkotok}\textstylePartofspeech{.} 
\end{styleEntryParagraph}

\begin{styleEntryParagraph}
\textstyleDefinitionn{five}\textstyleLexeme{     zlom}\textstylefstandard{.}
\end{styleEntryParagraph}

\begin{styleEntryParagraph}
\textstyleDefinitionn{flour}\textstyleLexeme{     həmbo}\textstylefstandard{.}
\end{styleEntryParagraph}

\begin{styleEntryParagraph}
\textstyleDefinitionn{flourish, soak in order to soften}\textstyleLexeme{     ɗe}\textstylefstandard{.}
\end{styleEntryParagraph}

\begin{styleEntryParagraph}
\textstyleDefinitionn{flow, leak    }\textstyleLexeme{ngaz}\textstylefstandard{.}
\end{styleEntryParagraph}

\begin{styleEntryParagraph}
\textstyleDefinitionn{flu, cold}\textstyleLexeme{     mədegen}\textstylefstandard{.}
\end{styleEntryParagraph}

\begin{styleEntryParagraph}
\textstyleDefinitionn{flute}\textstyleLexeme{     cecewk}\textstylefstandard{.}
\end{styleEntryParagraph}

\begin{styleEntryParagraph}
\textstyleDefinitionn{fly}\textstyleLexeme{     jəway}\textstylefstandard{.}
\end{styleEntryParagraph}

\begin{styleEntryParagraph}
\textstyleDefinitionn{fly away}\textstyleLexeme{     vahay}\textstylefstandard{.}
\end{styleEntryParagraph}

\begin{styleEntryParagraph}
\textstyleDefinitionn{fold}\textstyleLexeme{     faɗay}\textstylefstandard{.}
\end{styleEntryParagraph}

\begin{styleEntryParagraph}
\textstyleDefinitionn{fold, cross}\textstyleLexeme{     təkosoy}\textstylefstandard{.}
\end{styleEntryParagraph}

\begin{styleEntryParagraph}
\textstyleDefinitionn{fold legs}\textstyleLexeme{     carzlay}\textstylefstandard{.}
\end{styleEntryParagraph}

\begin{styleEntryParagraph}
\textstyleDefinitionn{follow}\textstyleLexeme{   dabay, mbay}\textstylefstandard{.}
\end{styleEntryParagraph}

\begin{styleEntryParagraph}
\textstyleDefinitionn{food, millet porridge}\textstyleLexeme{     ɗaf}\textstylefstandard{.}
\end{styleEntryParagraph}

\begin{styleEntryParagraph}
\textstyleDefinitionn{foot, leg}\textstyleLexeme{     asak}\textstylefstandard{.}
\end{styleEntryParagraph}

\begin{styleEntryParagraph}
\textstylefstandard{\textit{forbid      }}\textstyleLexeme{wasl}\textstylefstandard{\textit{.}}
\end{styleEntryParagraph}

\begin{styleEntryParagraph}
\textstyleDefinitionn{forget}\textstyleLexeme{     cəkəzlay}\textstylefstandard{.}
\end{styleEntryParagraph}

\begin{styleEntryParagraph}
\textstyleDefinitionn{forehead}\textstyleLexeme{     meher}\textstylefstandard{.}
\end{styleEntryParagraph}

\begin{styleEntryParagraph}
\textstyleDefinitionn{four}\textstyleLexeme{     əwfaɗ, məfaɗ}\textstyleDefinitionn{.}
\end{styleEntryParagraph}

\begin{styleEntryParagraph}
\textstyleDefinitionn{friend}\textstyleLexeme{     cecew}\textstylefstandard{.}
\end{styleEntryParagraph}

\begin{styleEntryParagraph}
\textstyleDefinitionn{frighten, command}\textstyleLexeme{    garay}\textstylefstandard{.} 
\end{styleEntryParagraph}

\begin{styleEntryParagraph}
\textstyleDefinitionn{frog}\textstyleLexeme{     gwədeɗek}\textstylefstandard{.}
\end{styleEntryParagraph}

\begin{styleEntryParagraph}
\textstyleDefinitionn{fry}\textstyleLexeme{ solay}\textstylefstandard{.}
\end{styleEntryParagraph}
\end{multicols}
\begin{styleLetterParagraph}
\textstyleLetterv{G  {}-  g}
\end{styleLetterParagraph}

\begin{multicols}{2}
\begin{styleEntryParagraph}
\textstyleDefinitionn{gather}\textstyleLexeme{     halay, məndacay, məndocay}\textstylefstandard{.}
\end{styleEntryParagraph}

\begin{styleEntryParagraph}
\textstyleDefinitionn{gather with a stick}\textstyleLexeme{     mbomoy}\textstylefstandard{.}
\end{styleEntryParagraph}

\begin{styleEntryParagraph}
\textstyleDefinitionn{germinate}\textstyleLexeme{     fat}\textstylefstandard{.}
\end{styleEntryParagraph}

\begin{styleEntryParagraph}
\textstyleDefinitionn{get away}\textstylefstandard{\textit{!}}\textstylefstandard{    }\textstyleLexeme{məf}
\end{styleEntryParagraph}

\begin{styleEntryParagraph}
\textstyleDefinitionn{get lost, lose}\textstyleLexeme{     cəjen}\textstylefstandard{.}
\end{styleEntryParagraph}

\begin{styleEntryParagraph}
\textstyleDefinitionn{get up}\textstyleLexeme{     cəkafay}\textstylefstandard{.}
\end{styleEntryParagraph}

\begin{styleEntryParagraph}
\textstyleDefinitionn{get water}\textstyleLexeme{     cahay}\textstylefstandard{.}
\end{styleEntryParagraph}

\begin{styleEntryParagraph}
\textstyleDefinitionn{giraffe}\textstyleLexeme{     kərcece}\textstylefstandard{.}
\end{styleEntryParagraph}

\begin{styleEntryParagraph}
\textstyleDefinitionn{girl}\textstyleLexeme{     dalay}\textstylefstandard{.}
\end{styleEntryParagraph}

\begin{styleEntryParagraph}
\textstyleDefinitionn{give}\textstyleLexeme{     vər}\textstylefstandard{.}
\end{styleEntryParagraph}

\begin{styleEntryParagraph}
\textstyleDefinitionn{glorify}\textstyleLexeme{     zəmbaɗay}\textstylefstandard{.}
\end{styleEntryParagraph}

\begin{styleEntryParagraph}
\textstyleDefinitionn{gnaw}\textstyleLexeme{     zlokoy}\textstylefstandard{.}
\end{styleEntryParagraph}

\begin{styleEntryParagraph}
\textstyleDefinitionn{go}\textstyleLexeme{     lo}\textstyleDefinitionn{.}
\end{styleEntryParagraph}

\begin{styleEntryParagraph}
\textstylePartofspeech{go a}\textstyleDefinitionn{cross}\textstyleLexeme{     təwaɗay}\textstylefstandard{.}
\end{styleEntryParagraph}

\begin{styleEntryParagraph}
\textstyleDefinitionn{go out, appear}\textstyleLexeme{     zləray}\textstylefstandard{.}
\end{styleEntryParagraph}

\begin{styleEntryParagraph}
\textstyleDefinitionn{goat}\textstyleLexeme{     awak}\textstylefstandard{.}
\end{styleEntryParagraph}

\begin{styleEntryParagraph}
\textstyleDefinitionn{goat horn}\textstyleLexeme{     aɓalan}\textstylefstandard{.}
\end{styleEntryParagraph}

\begin{styleEntryParagraph}
\textstyleDefinitionn{god, sky}\textstyleLexeme{     hərmbəlom}\textstylefstandard{.}
\end{styleEntryParagraph}

\begin{styleEntryParagraph}
\textstyleDefinitionn{gold}\textstyleLexeme{     ogəro}\textstylefstandard{.}
\end{styleEntryParagraph}

\begin{styleEntryParagraph}
\textstyleDefinitionn{good}\textstyleLexeme{     lala, səlom, təde}\textstylefstandard{.}
\end{styleEntryParagraph}

\begin{styleEntryParagraph}
\textstyleDefinitionn{gourd}\textstyleLexeme{     kokor , zən zon}\textstylefstandard{.}
\end{styleEntryParagraph}

\begin{styleEntryParagraph}
\textstyleDefinitionn{government}\textstyleLexeme{     ngomna}\textstylefstandard{.}
\end{styleEntryParagraph}

\begin{styleEntryParagraph}
\textstyleDefinitionn{granary}\textstyleLexeme{     ɓəra}\textstylefstandard{.}
\end{styleEntryParagraph}

\begin{styleEntryParagraph}
\textstyleDefinitionn{granary for straw}\textstyleLexeme{   hahar}\textstylefstandard{.}
\end{styleEntryParagraph}

\begin{styleEntryParagraph}
\textstyleDefinitionn{grandmother}\textstyleLexeme{     dede}\textstylefstandard{.}
\end{styleEntryParagraph}

\begin{styleEntryParagraph}
\textstyleDefinitionn{granulate, weave}\textstyleLexeme{     gədəgar}\textstylefstandard{.}
\end{styleEntryParagraph}

\begin{styleEntryParagraph}
\textstyleDefinitionn{grass}\textstyleLexeme{     agwəjer}\textstylefstandard{.} 
\end{styleEntryParagraph}

\begin{styleEntryParagraph}
\textstyleDefinitionn{grass fence}\textstyleLexeme{     məpapar}\textstylefstandard{.}
\end{styleEntryParagraph}

\begin{styleEntryParagraph}
\textstyleDefinitionn{grasshopper}\textstyleLexeme{     heyew}\textstylefstandard{.}
\end{styleEntryParagraph}

\begin{styleEntryParagraph}
\textstyleDefinitionn{grave}\textstyleLexeme{     hərdesl}\textstylefstandard{.}
\end{styleEntryParagraph}

\begin{styleEntryParagraph}
\textstyleDefinitionn{greatness}\textstyleLexeme{     malan}\textstylefstandard{.}
\end{styleEntryParagraph}

\begin{styleEntryParagraph}
\textstyleDefinitionn{greet someone}\textstyleLexeme{     hay}\textstylefstandard{.}
\end{styleEntryParagraph}

\begin{styleEntryParagraph}
\textstyleDefinitionn{grind}\textstyleLexeme{     haya}\textstylefstandard{.}
\end{styleEntryParagraph}

\begin{styleEntryParagraph}
\textstylePartofspeech{grind (peanuts)    }\textstyleLexeme{ngəlday}\textstylePartofspeech{.}
\end{styleEntryParagraph}

\begin{styleEntryParagraph}
\textstyleDefinitionn{grinding stone}\textstyleLexeme{     ver}\textstylefstandard{.}
\end{styleEntryParagraph}

\begin{styleEntryParagraph}
\textstyleDefinitionn{groan}\textstyleLexeme{   njeren}\textstylefstandard{.}
\end{styleEntryParagraph}

\begin{styleEntryParagraph}
\textstyleDefinitionn{ground nut}\textstyleLexeme{     eyewk}\textstyleDefinitionn{.}
\end{styleEntryParagraph}

\begin{styleEntryParagraph}
\textstyleDefinitionn{grow}\textstyleLexeme{     caway}\textstylefstandard{.}
\end{styleEntryParagraph}

\begin{styleEntryParagraph}
\textstyleDefinitionn{grow}\textstyleLexeme{     gar}\textstylefstandard{.}
\end{styleEntryParagraph}

\begin{styleEntryParagraph}
\textstyleDefinitionn{guinea fowl}\textstyleLexeme{     javar}\textstylefstandard{.}
\end{styleEntryParagraph}
\end{multicols}
\begin{styleLetterParagraph}
\textstyleLetterv{H  {}-  h}
\end{styleLetterParagraph}

\begin{multicols}{2}
\begin{styleEntryParagraph}
\textstyleSensenumber{\textit{habits}}\textstyleLexeme{     mənjəye}\textstylefstandard{.}
\end{styleEntryParagraph}

\begin{styleEntryParagraph}
\textstyleDefinitionn{habitually do something}\textstyleLexeme{   sərkay}\textstylefstandard{.}
\end{styleEntryParagraph}

\begin{styleEntryParagraph}
\textstyleDefinitionn{hail}\textstyleLexeme{     marasl}\textstylefstandard{.}
\end{styleEntryParagraph}

\begin{styleEntryParagraph}
\textstyleDefinitionn{hair}\textstyleLexeme{     səmbetewk}\textstylePartofspeech{.} 
\end{styleEntryParagraph}

\begin{styleEntryParagraph}
\textstyleDefinitionn{hand}\textstyleLexeme{     ahar}\textstylefstandard{.}
\end{styleEntryParagraph}

\begin{styleEntryParagraph}
\textstyleDefinitionn{hang}\textstyleLexeme{   gəjakay,   laway}\textstylefstandard{.}
\end{styleEntryParagraph}

\begin{styleEntryParagraph}
\textstyleDefinitionn{hang, twist}\textstyleLexeme{    vaway}\textstylefstandard{.}
\end{styleEntryParagraph}

\begin{styleEntryParagraph}
\textstyleDefinitionn{hangar to give shade in front of a house}\textstyleLexeme{   abalak}\textstylefstandard{.} 
\end{styleEntryParagraph}

\begin{styleEntryParagraph}
\textstyleDefinitionn{harvest}\textstyleLexeme{     baz}\textstylefstandard{.}
\end{styleEntryParagraph}

\begin{styleEntryParagraph}
\textstyleDefinitionn{hat}\textstyleLexeme{     jogo}\textstylefstandard{.}
\end{styleEntryParagraph}

\begin{styleEntryParagraph}
\textit{hate, }\textstyleDefinitionn{quarrel}\textstyleLexeme{     hərnje}\textstylefstandard{.}
\end{styleEntryParagraph}

\begin{styleEntryParagraph}
\textstyleDefinitionn{have a headache}\textstyleLexeme{     cazlay}\textstylefstandard{.} 
\end{styleEntryParagraph}

\begin{styleEntryParagraph}
\textstyleDefinitionn{hawk}\textstyleLexeme{     }\textstyleLexeme{etew}\textstyleLexeme{, mogodok}\textstylefstandard{.}
\end{styleEntryParagraph}

\begin{styleEntryParagraph}
\textstyleDefinitionn{haze}\textstyleLexeme{     kwəsay}\textstylefstandard{.}
\end{styleEntryParagraph}

\begin{styleEntryParagraph}
\textstyleDefinitionn{head}\textstyleLexeme{     dəray}\textstylefstandard{.}
\end{styleEntryParagraph}

\begin{styleEntryParagraph}
\textstyleDefinitionn{heal}\textstyleLexeme{     mbar}\textstylefstandard{.}
\end{styleEntryParagraph}

\begin{styleEntryParagraph}
\textstyleDefinitionn{hear, understand}\textstyleLexeme{     cen}\textstylefstandard{.}
\end{styleEntryParagraph}

\begin{styleEntryParagraph}
\textstyleDefinitionn{heart, self}\textstyleLexeme{     ɓərav}\textstyleDefinitionn{.}
\end{styleEntryParagraph}

\begin{styleEntryParagraph}
\textstyleDefinitionn{heat}\textstyleLexeme{     hereɓ}\textstylefstandard{.}
\end{styleEntryParagraph}

\begin{styleEntryParagraph}
\textstyleDefinitionn{hedgehog}\textstyleLexeme{     otos}\textstylefstandard{.}
\end{styleEntryParagraph}

\begin{styleEntryParagraph}
\textstyleDefinitionn{heart, chest}\textstyleLexeme{     tololon}\textstyleDefinitionn{.}
\end{styleEntryParagraph}

\begin{styleEntryParagraph}
\textstyleDefinitionn{heat up, dissolve    }\textstyleLexeme{hərɓoy}\textstylefstandard{.}
\end{styleEntryParagraph}

\begin{styleEntryParagraph}
\textstyleDefinitionn{height}\textstyleLexeme{     səber}\textstylefstandard{.}
\end{styleEntryParagraph}

\begin{styleEntryParagraph}
\textstyleDefinitionn{help}\textstyleLexeme{     jənay}\textstylefstandard{.}
\end{styleEntryParagraph}

\begin{styleEntryParagraph}
\textstyleDefinitionn{herd, chase}\textstyleLexeme{     galay}\textstylefstandard{.}
\end{styleEntryParagraph}

\begin{styleEntryParagraph}
\textstyleDefinitionn{here}\textstyleLexeme{     ahakay, ehe, nehe}\textstylefstandard{.}
\end{styleEntryParagraph}

\begin{styleEntryParagraph}
\textstyleDefinitionn{heron}\textstyleLexeme{     ngərkaka}\textstylefstandard{.}
\end{styleEntryParagraph}

\begin{styleEntryParagraph}
\textstyleDefinitionn{hide}\textstyleLexeme{     rəbokay}\textstyleDefinitionn{.}\textstyleLexeme{ }
\end{styleEntryParagraph}

\begin{styleEntryParagraph}
\textstyleDefinitionn{hiding place}\textstyleLexeme{     rəbok}\textstyleDefinitionn{.}\textstyleLexeme{ }
\end{styleEntryParagraph}

\begin{styleEntryParagraph}
\textstyleDefinitionn{hip}\textstyleLexeme{     gəzo}\textstylefstandard{.}
\end{styleEntryParagraph}

\begin{styleEntryParagraph}
\textstyleDefinitionn{hit}\textstyleLexeme{     ɓay}\textstylefstandard{.}
\end{styleEntryParagraph}

\begin{styleEntryParagraph}
\textstyleDefinitionn{hoe}\textstyleLexeme{     həlef, }\textstyleLexeme{mədəger}\textstylefstandard{.}
\end{styleEntryParagraph}

\begin{styleEntryParagraph}
\textstyleDefinitionn{hole}\textstyleLexeme{     pəɗe}\textstylefstandard{.}
\end{styleEntryParagraph}

\begin{styleEntryParagraph}
\textstyleDefinitionn{home}\textstyleLexeme{     mogom}\textstylefstandard{.}
\end{styleEntryParagraph}

\begin{styleEntryParagraph}
\textstyleDefinitionn{honey}\textstyleLexeme{     omom}\textstylefstandard{.}
\end{styleEntryParagraph}

\begin{styleEntryParagraph}
\textstyleDefinitionn{horn}\textstyleLexeme{     mongom}
\end{styleEntryParagraph}

\begin{styleEntryParagraph}
\textstyleDefinitionn{horse}\textstyleLexeme{     pəles}\textstylefstandard{.}
\end{styleEntryParagraph}

\begin{styleEntryParagraph}
\textstyleDefinitionn{hot drink made with rice, pap}\textstyleLexeme{     mətərak}\textstyleDefinitionn{.}
\end{styleEntryParagraph}

\begin{styleEntryParagraph}
\textstyleDefinitionn{hour}\textstyleLexeme{     hara}\textstylefstandard{.}
\end{styleEntryParagraph}

\begin{styleEntryParagraph}
\textit{h}\textstyleDefinitionn{ouse}\textstyleLexeme{       hay}\textstyleDefinitionn{.}
\end{styleEntryParagraph}

\begin{styleEntryParagraph}
\textstyleDefinitionn{how}\textstyleLexeme{     memey}\textstylefstandard{.}
\end{styleEntryParagraph}

\begin{styleEntryParagraph}
\textstyleDefinitionn{how (emphatic)    }\textstyleLexeme{mey}\textstylefstandard{.}
\end{styleEntryParagraph}

\begin{styleEntryParagraph}
\textstyleDefinitionn{how much/how many}\textstyleLexeme{     mətəmey}\textstylefstandard{.}
\end{styleEntryParagraph}

\begin{styleEntryParagraph}
\textstyleDefinitionn{hundred}\textstyleLexeme{     səkat}\textstylefstandard{.}
\end{styleEntryParagraph}

\begin{styleEntryParagraph}
\textstyleDefinitionn{hunger}\textstyleLexeme{     may}\textstylefstandard{.}
\end{styleEntryParagraph}

\begin{styleEntryParagraph}
\textstyleDefinitionn{hunt}\textstyleLexeme{     təvalay}\textstylefstandard{.}
\end{styleEntryParagraph}

\begin{styleEntryParagraph}
\textstyleDefinitionn{hurt, attach}\textstyleLexeme{     wal}\textstylefstandard{.}
\end{styleEntryParagraph}

\begin{styleEntryParagraph}
\textstyleDefinitionn{husband, male}\textstyleLexeme{     zar}\textstylefstandard{.}
\end{styleEntryParagraph}

\begin{styleEntryParagraph}
\textstyleDefinitionn{hyena}\textstyleLexeme{     dərlenge}\textstylefstandard{.}
\end{styleEntryParagraph}

\begin{styleEntryParagraph}
\textstyleDefinitionn{hyrax}\textstyleLexeme{     ocom}\textstylefstandard{.}
\end{styleEntryParagraph}
\end{multicols}
\begin{styleLetterParagraph}
\textstyleLetterv{I  {}-  i}\textstyleLexeme{ }
\end{styleLetterParagraph}

\begin{multicols}{2}
\begin{styleEntryParagraph}
\textstylePartofspeech{idea of} \textstyleDefinitionn{approximately}\textstyleLexeme{     dəy day}\textstylefstandard{.}
\end{styleEntryParagraph}

\begin{styleEntryParagraph}
\textstyleDefinitionn{idea of barely escaping}\textstyleLexeme{ pəyteɗ}\textstylefstandard{.}
\end{styleEntryParagraph}

\begin{styleEntryParagraph}
\textstyleDefinitionn{idea of being close    }\textstyleLexeme{bəfa}\textstylefstandard{.}
\end{styleEntryParagraph}

\begin{styleEntryParagraph}
\textstylePartofspeech{idea of being} \textstyleDefinitionn{completely wet}\textstyleLexeme{     jəb jəb}\textstylefstandard{.}
\end{styleEntryParagraph}

\begin{styleEntryParagraph}
\textstyleDefinitionn{idea of burying    }\textstyleLexeme{babək}\textstylefstandard{.}
\end{styleEntryParagraph}

\begin{styleEntryParagraph}
\textstyleDefinitionn{idea of catching someone by the throat}\textstyleLexeme{   kək}\textstylefstandard{.}
\end{styleEntryParagraph}

\begin{styleEntryParagraph}
\textstylefstandard{\textit{idea of}} \textstyleDefinitionn{coldness}\textstyleLexeme{     pəyecece}\textstylefstandard{.}
\end{styleEntryParagraph}

\begin{styleEntryParagraph}
\textit{idea of }\textstyleDefinitionn{collapsing, dying    }\textstyleLexeme{dəɓəsolək}\textstyleDefinitionn{.}
\end{styleEntryParagraph}

\begin{styleEntryParagraph}
\textstyleDefinitionn{idea of cutting something through the middle}\textstyleLexeme{ gəraw}\textstylefstandard{.}
\end{styleEntryParagraph}

\begin{styleEntryParagraph}
\textstylePartofspeech{idea of }\textstyleDefinitionn{dispersing}\textstyleLexeme{     səwat}\textstylefstandard{.} 
\end{styleEntryParagraph}

\begin{styleEntryParagraph}
\textstylePartofspeech{idea of} \textstyleDefinitionn{exactly}\textstyleLexeme{     kəl kəl}\textstylefstandard{.}
\end{styleEntryParagraph}

\begin{styleEntryParagraph}
\textstyleDefinitionn{idea of flying away}\textstyleLexeme{     botot}\textstylefstandard{.}
\end{styleEntryParagraph}

\begin{styleEntryParagraph}
\textstylePartofspeech{idea of }\textstyleDefinitionn{foolishness    }\textstyleLexeme{bəwɗere}\textstylefstandard{.}
\end{styleEntryParagraph}

\begin{styleIndentedParagraph}
\textstylefstandard{\textit{idea of }}\textstyleDefinitionn{forever, in the future}\textstyleSubentry{     epele pele}\textstylefstandard{.}
\end{styleIndentedParagraph}

\begin{styleEntryParagraph}
\textstyleDefinitionn{idea of full up to the roof}\textstyleLexeme{     mbaf}\textstyleDefinitionn{.}
\end{styleEntryParagraph}

\begin{styleEntryParagraph}
\textstylefstandard{\textit{idea of}} \textstyleDefinitionn{fullness}\textstyleLexeme{   peɗeɗe}\textstyleDefinitionn{.}
\end{styleEntryParagraph}

\begin{styleEntryParagraph}
\textstyleDefinitionn{idea of going}\textstyleLexeme{     sen}\textstylefstandard{.}
\end{styleEntryParagraph}

\begin{styleEntryParagraph}
\textstyleDefinitionn{idea of going far}\textstyleLexeme{   təf}\textstylefstandard{.}
\end{styleEntryParagraph}

\begin{styleEntryParagraph}
\textstyleDefinitionn{idea of grasping}\textstyleLexeme{     kəwna}\textstylefstandard{.}
\end{styleEntryParagraph}

\begin{styleEntryParagraph}
\textstyleDefinitionn{idea of grinding}\textstyleLexeme{     njəw njəw njəw}\textstylefstandard{.}
\end{styleEntryParagraph}

\begin{styleEntryParagraph}
\textstylePartofspeech{idea of} \textstyleDefinitionn{guinea fowl running}\textstyleLexeme{ cərr}\textstylefstandard{.}
\end{styleEntryParagraph}

\begin{styleEntryParagraph}
\textstylePartofspeech{idea of} \textstyleDefinitionn{hardly breathing}\textstyleLexeme{     heɓek heɓek}\textstylefstandard{.}
\end{styleEntryParagraph}

\begin{styleEntryParagraph}
\textit{idea of }\textstyleDefinitionn{hiding}\textstyleLexeme{     rəbok}\textstylefstandard{ }\textstyleLexeme{rəbok}\textstyleDefinitionn{.}\textstyleLexeme{ }
\end{styleEntryParagraph}

\begin{styleEntryParagraph}
\textstyleDefinitionn{idea of hollowness}\textstyleLexeme{     tezl tezlezl}\textstylefstandard{.}
\end{styleEntryParagraph}

\begin{styleEntryParagraph}
\textit{idea of} \textstyleDefinitionn{insulting}\textstyleLexeme{     dəl}\textstyleDefinitionn{.}
\end{styleEntryParagraph}

\begin{styleEntryParagraph}
\textstyleDefinitionn{idea of later on}\textstyleLexeme{     cacapa}\textstyleDefinitionn{.}
\end{styleEntryParagraph}

\begin{styleEntryParagraph}
\textit{idea of} \textstyleDefinitionn{lifting on head}\textstyleLexeme{     dergwecik}\textstyleDefinitionn{.}
\end{styleEntryParagraph}

\begin{styleEntryParagraph}
\textstyleDefinitionn{idea of long ago}\textstyleLexeme{     zlezle}\textstylefstandard{.}
\end{styleEntryParagraph}

\begin{styleEntryParagraph}
\textstyleDefinitionn{idea of looking}\textstyleLexeme{     kəy}\textstylefstandard{.}
\end{styleEntryParagraph}

\begin{styleEntryParagraph}
\textstylePartofspeech{idea of} \textstyleDefinitionn{making beer}\textstyleLexeme{ gədok}\textstyleDefinitionn{.}
\end{styleEntryParagraph}

\begin{styleEntryParagraph}
\textstyleDefinitionn{idea of many}\textstyleLexeme{ dəres}\textstylefstandard{.}
\end{styleEntryParagraph}

\begin{styleEntryParagraph}
\textstyleDefinitionn{idea of opening door}\textstyleLexeme{     pok}\textstyleDefinitionn{.}
\end{styleEntryParagraph}

\begin{styleEntryParagraph}
\textstyleDefinitionn{idea of penetration}\textstyleLexeme{   mbəra}\textstyleLexeme{ɓ}\textstylefstandard{.}
\end{styleEntryParagraph}

\begin{styleEntryParagraph}
\textstyleDefinitionn{idea of positioning self for throwing spear  }\textstyleLexeme{mək}\textstylefstandard{.}
\end{styleEntryParagraph}

\begin{styleEntryParagraph}
\textstylePartofspeech{idea of} \textstyleDefinitionn{putting down}\textstyleLexeme{     ɗen}\textstylefstandard{.}
\end{styleEntryParagraph}

\begin{styleEntryParagraph}
\textstyleDefinitionn{idea of putting on head}\textstyleLexeme{     təh}\textstylefstandard{.}
\end{styleEntryParagraph}

\begin{styleEntryParagraph}
\textstylePartofspeech{idea of }\textstyleDefinitionn{quickly}\textbf{     kaləw}\textstylefstandard{.}
\end{styleEntryParagraph}

\begin{styleDoublecolumnSection}
\textstylefstandard{\textit{idea of }}\textstyleDefinitionn{rapidly  }\textbf{vbəvbəvbə}
\end{styleDoublecolumnSection}

\begin{styleEntryParagraph}
\textstyleDefinitionn{idea of redness}\textstyleLexeme{     ɗaz ɗaz}\textstyleDefinitionn{.}
\end{styleEntryParagraph}

\begin{styleEntryParagraph}
\textstyleDefinitionn{idea of setting down something heavy}\textstyleLexeme{   gəɗəgəzl}\textstyleDefinitionn{.}
\end{styleEntryParagraph}

\begin{styleEntryParagraph}
\textstylePartofspeech{idea of }\textstyleDefinitionn{sharpness}\textstyleLexeme{     kekəɓ\nobreakdash-kekeɓ}\textstylefstandard{.}
\end{styleEntryParagraph}

\begin{styleEntryParagraph}
\textstyleDefinitionn{idea of shining upwards}\textbf{     cəzlar}\textstylefstandard{.}
\end{styleEntryParagraph}

\begin{styleEntryParagraph}
\textstylePartofspeech{idea of }\textstyleDefinitionn{a short time}\textstyleLexeme{     mba}\textstyleDefinitionn{.}
\end{styleEntryParagraph}

\begin{styleEntryParagraph}
\textstylePartofspeech{idea of} \textstyleDefinitionn{some}\textstyleLexeme{     ɓəl}\textstylefstandard{.}
\end{styleEntryParagraph}

\begin{styleEntryParagraph}
\textstyleDefinitionn{idea of someone balancing something on head    }\textstyleLexeme{danjəw}\textstylefstandard{.}
\end{styleEntryParagraph}

\begin{styleEntryParagraph}
\textstyleDefinitionn{idea of someone who hasn’t any weight (an insult)  }\textstyleLexeme{kəkef  kəf}\textstylefstandard{.}
\end{styleEntryParagraph}

\begin{styleEntryParagraph}
\textstylePartofspeech{idea of} \textstyleDefinitionn{something big and reflective}\textstyleLexeme{   mbajak}\textstylefstandard{.}
\end{styleEntryParagraph}

\begin{styleEntryParagraph}
\textstylePartofspeech{idea of something} \textstyleDefinitionn{different}\textstyleLexeme{   ter\nobreakdash-tere}\textstylefstandard{.}
\end{styleEntryParagraph}

\begin{styleEntryParagraph}
\textstylePartofspeech{idea of} \textstyleDefinitionn{spicy hot taste    }\textstyleLexeme{bakaka}\textstylefstandard{.}
\end{styleEntryParagraph}

\begin{styleEntryParagraph}
\textstylefstandard{\textit{idea of the }}\textstyleDefinitionn{start of a race}\textstyleLexeme{     pəvban}\textstylefstandard{.}
\end{styleEntryParagraph}

\begin{styleEntryParagraph}
\textstylefstandard{\textit{idea of}} \textstyleDefinitionn{sweetness}\textstyleLexeme{     poɗococo}\textstylefstandard{.}
\end{styleEntryParagraph}

\begin{styleEntryParagraph}
\textstylePartofspeech{idea of} \textstyleflabel{taking}\textstyleLexeme{     jo}\textstyleflabel{.}
\end{styleEntryParagraph}

\begin{styleEntryParagraph}
\textstylefstandard{\textit{idea of}} \textstyleDefinitionn{the way a sick person walks}\textstyleLexeme{   abəlgamay}\textstyleDefinitionn{.}
\end{styleEntryParagraph}

\begin{styleEntryParagraph}
\textstyleDefinitionn{idea/sight of child running}\textstyleLexeme{     njəɗok njəɗok}\textstylefstandard{.}
\end{styleEntryParagraph}

\begin{styleEntryParagraph}
\textstyleDefinitionn{idea/sight of man running}\textstyleLexeme{     gədo gədo}\textstylefstandard{ }\textstyleLexeme{gədo}\textstylefstandard{.}
\end{styleEntryParagraph}

\begin{styleEntryParagraph}
\textstyleDefinitionn{idea/sight of old person trying to run}\textstyleLexeme{   kərwəɗ wəɗ kərwəɗ wəɗ}\textstylefstandard{.}
\end{styleEntryParagraph}

\begin{styleEntryParagraph}
\textstylefstandard{\textit{idea/}} s\textstylefstandard{\textit{ight of ostrich running}}\textstyleLexeme{     jeɗ jeɗ jeɗ}\textstylefstandard{.}
\end{styleEntryParagraph}

\begin{styleEntryParagraph}
\textstylefstandard{\textit{idea/}}\textstyleDefinitionn{sight of rabbit hopping}\textstyleLexeme{ pəvbəw pəvbəw}\textstylefstandard{.}
\end{styleEntryParagraph}

\begin{styleEntryParagraph}
\textstyleDefinitionn{idea/sight of something heavy running (cows)  }\textstyleLexeme{gərəp gərəp}\textstylefstandard{.}
\end{styleEntryParagraph}

\begin{styleEntryParagraph}
\textstyleDefinitionn{idea/sight of something multiplying}\textstyleLexeme{     wəsekeke}\textstylefstandard{.}
\end{styleEntryParagraph}

\begin{styleEntryParagraph}
\textstyleDefinitionn{idea/sight of something thrown up high}\textstyleLexeme{     zor}\textstylefstandard{.}
\end{styleEntryParagraph}

\begin{styleEntryParagraph}
\textstyleDefinitionn{idea/ sight of a toad hopping}\textstyleLexeme{     pəcəkəɗək}\textstylefstandard{   .}
\end{styleEntryParagraph}

\begin{styleEntryParagraph}
\textstyleDefinitionn{idea/sight of youth running}\textstyleLexeme{     njəl njəl}\textstylefstandard{.}
\end{styleEntryParagraph}

\begin{styleEntryParagraph}
\textstyleDefinitionn{idea/sound of bottle opening}\textstyleLexeme{     pək}\textstylefstandard{.}
\end{styleEntryParagraph}

\begin{styleEntryParagraph}
\textstyleDefinitionn{idea/sound of cutting with axe}\textstyleLexeme{     coco}\textstylefstandard{.}
\end{styleEntryParagraph}

\begin{styleDoublecolumnSection}
\textstyleDefinitionn{idea/sound of men running}\textbf{   ɓ}\textbf{a}\textbf{vb}\textbf{aw}\textstylefstandard{\textit{.}}
\end{styleDoublecolumnSection}

\begin{styleEntryParagraph}
\textit{idea/}\textstyleDefinitionn{sound of movement    }\textstyleLexeme{dəreffefe}\textstylefstandard{.}
\end{styleEntryParagraph}

\begin{styleEntryParagraph}
\textstyleDefinitionn{idea/sound of pounding millet}\textstyleLexeme{     kəndal}\textstylefstandard{.}
\end{styleEntryParagraph}

\begin{styleEntryParagraph}
\textstyleDefinitionn{idea/sound of race}\textstyleLexeme{     ɓərketem}\textstylefstandard{ }\textstyleLexeme{ɓərketem}\textstylefstandard{.}
\end{styleEntryParagraph}

\begin{styleEntryParagraph}
\textstyleDefinitionn{idea/sound of snake slithering}\textstyleLexeme{   fofofo}\textstylefstandard{.}
\end{styleEntryParagraph}

\begin{styleEntryParagraph}
\textstyleDefinitionn{idea/sound of something falling}\textbf{   ɓ}\textbf{a}\textbf{vb}\textstylefstandard{.}
\end{styleEntryParagraph}

\begin{styleDoublecolumnSection}
\textstyleDefinitionn{idea/sound of something soft hitting the ground (a snake, or a mud wall)  }\textbf{vbaɓ}
\end{styleDoublecolumnSection}

\begin{styleEntryParagraph}
\textstyleDefinitionn{idea/sound of truck engine humming}\textstyleLexeme{   fəhh}\textstylefstandard{.} 
\end{styleEntryParagraph}

\begin{styleEntryParagraph}
\textstyleDefinitionn{idea/sound of wind blowing}\textstyleLexeme{   fowwa}\textstylefstandard{.} 
\end{styleEntryParagraph}

\begin{styleEntryParagraph}
\textstyleDefinitionn{idol, spirit}\textstyleLexeme{     pəra}\textstylefstandard{.}
\end{styleEntryParagraph}

\begin{styleEntryParagraph}
\textstyleDefinitionn{if}\textstyleLexeme{     asa}\textstylefstandard{.}
\end{styleEntryParagraph}

\begin{styleEntryParagraph}
\textstyleDefinitionn{immediately}\textstyleLexeme{     pepen}\textstylefstandard{.}
\end{styleEntryParagraph}

\begin{styleEntryParagraph}
\textstyleDefinitionn{in}\textstyleLexeme{     ava, }\textstyleLexeme{a…ava}\textstylefstandard{.}
\end{styleEntryParagraph}

\begin{styleEntryParagraph}
\textstyleDefinitionn{in spite of}\textstyleLexeme{     re}\textstylefstandard{.}
\end{styleEntryParagraph}

\begin{styleEntryParagraph}
\textstyleDefinitionn{indicate}\textstyleLexeme{     ɗakay}\textstyleDefinitionn{.}
\end{styleEntryParagraph}

\begin{styleEntryParagraph}
\textstyleDefinitionn{insect}\textstyleLexeme{     engeren}\textstylefstandard{.}
\end{styleEntryParagraph}

\begin{styleEntryParagraph}
\textstyleDefinitionn{inside house}\textstyleLexeme{     ayva}\textstylefstandard{.}
\end{styleEntryParagraph}

\begin{styleEntryParagraph}
\textstyleDefinitionn{inspect, notice}\textstyleLexeme{     zəroy}\textstyleDefinitionn{.}
\end{styleEntryParagraph}

\begin{styleEntryParagraph}
\textstyleDefinitionn{insult}\textstyleLexeme{     ndaway}\textstylefstandard{.}
\end{styleEntryParagraph}

\begin{styleEntryParagraph}
\textstyleDefinitionn{intersect, meet}\textstyleLexeme{     dozloy}\textstylefstandard{.} 
\end{styleEntryParagraph}

\begin{styleEntryParagraph}
\textstyleDefinitionn{intestines}\textstyleLexeme{     danday}\textstylefstandard{.}
\end{styleEntryParagraph}

\begin{styleEntryParagraph}
\textstyleDefinitionn{introduce}\textstyleLexeme{     ngazlay}\textstylefstandard{.}
\end{styleEntryParagraph}

\begin{styleEntryParagraph}
\textstyleDefinitionn{instrument, stringed}\textstyleLexeme{     kəndew}\textstylefstandard{.}
\end{styleEntryParagraph}

\begin{styleEntryParagraph}
\textstyleDefinitionn{invite, try}\textstyleLexeme{     təkaray}\textstylefstandard{.}\textstyleDefinitionn{ }
\end{styleEntryParagraph}

\begin{styleEntryParagraph}
\textstyleDefinitionn{iron, metal}\textstyleLexeme{     hara}\textstylefstandard{.}
\end{styleEntryParagraph}
\end{multicols}
\begin{styleLetterParagraph}
\textstyleLetterv{J  {}-  j}
\end{styleLetterParagraph}

\begin{multicols}{2}
\begin{styleEntryParagraph}
\textstyleDefinitionn{jawbone}\textstyleLexeme{     debezem, malgamay}\textstylefstandard{.}
\end{styleEntryParagraph}

\begin{styleEntryParagraph}
\textstyleDefinitionn{jealousy}\textstyleLexeme{     səlek}\textstylefstandard{.}
\end{styleEntryParagraph}

\begin{styleEntryParagraph}
\textstyleDefinitionn{join, tie}\textstyleLexeme{     dazlay}\textstylefstandard{.}
\end{styleEntryParagraph}

\begin{styleEntryParagraph}
\textstyleDefinitionn{joint}\textstyleLexeme{     hərəngezl}\textstylefstandard{.}
\end{styleEntryParagraph}

\begin{styleEntryParagraph}
\textstyleDefinitionn{joke}\textstyleLexeme{     sono}\textstylefstandard{.}
\end{styleEntryParagraph}

\begin{styleEntryParagraph}
\textstyleDefinitionn{joy}\textstyleLexeme{     memle}\textstylefstandard{.}
\end{styleEntryParagraph}

\begin{styleEntryParagraph}
\textstyleDefinitionn{judgement}\textstyleLexeme{     serəya}\textstylefstandard{.}
\end{styleEntryParagraph}

\begin{styleEntryParagraph}
\textstyleDefinitionn{juice, squeeze}\textstyleLexeme{     ɗocay}\textstylefstandard{.}
\end{styleEntryParagraph}

\begin{styleEntryParagraph}
\textstyleDefinitionn{jump, pull out}\textstyleLexeme{     həraɗ}\textstylefstandard{.}
\end{styleEntryParagraph}
\end{multicols}
\begin{styleLetterParagraph}
\textstyleLetterv{K  {}-  k}
\end{styleLetterParagraph}

\begin{multicols}{2}
\begin{styleEntryParagraph}
\textstyleDefinitionn{kick}\textstyleLexeme{     zlar}\textstylefstandard{.}
\end{styleEntryParagraph}

\begin{styleEntryParagraph}
\textstyleDefinitionn{kill by clubbing}\textstyleLexeme{     kaɗ}\textstylefstandard{.}
\end{styleEntryParagraph}

\begin{styleEntryParagraph}
\textstyleDefinitionn{kill by piercing}\textstyleLexeme{     ndaz}\textstylefstandard{.}
\end{styleEntryParagraph}

\begin{styleEntryParagraph}
\textstyleDefinitionn{kill many, decimate}\textstyleLexeme{    pazlay}\textstylefstandard{.}
\end{styleEntryParagraph}

\begin{styleEntryParagraph}
\textstyleDefinitionn{kindness, naivety}\textstyleLexeme{     zənof}\textstylefstandard{.}
\end{styleEntryParagraph}

\begin{styleEntryParagraph}
\textstyleDefinitionn{kitchen}\textstyleLexeme{     gəlan}\textstylefstandard{.}
\end{styleEntryParagraph}

\begin{styleEntryParagraph}
\textstyleDefinitionn{knead, soak}\textstyleLexeme{ bolay}\textstylefstandard{.}
\end{styleEntryParagraph}

\begin{styleEntryParagraph}
\textstyleDefinitionn{knee}\textstyleLexeme{     hərdedem}\textstylefstandard{.}
\end{styleEntryParagraph}

\begin{styleEntryParagraph}
\textstyleDefinitionn{kneel}\textstyleLexeme{     kərkay}\textstylefstandard{.}
\end{styleEntryParagraph}

\begin{styleEntryParagraph}
\textstyleDefinitionn{knife}\textstyleLexeme{     mekec}\textstylefstandard{.}
\end{styleEntryParagraph}

\begin{styleEntryParagraph}
\textstyleDefinitionn{know}\textstyleLexeme{     sar}\textstylefstandard{.}
\end{styleEntryParagraph}

\begin{styleEntryParagraph}
\textstyleDefinitionn{kola nut}\textstyleLexeme{     goro}\textstylefstandard{.}
\end{styleEntryParagraph}
\end{multicols}
\begin{styleLetterParagraph}
\textstyleLetterv{L  {}-  l }
\end{styleLetterParagraph}

\begin{multicols}{2}
\begin{styleEntryParagraph}
\textstyleDefinitionn{lack, be insufficient}\textstyleLexeme{     ce}\textstylefstandard{.}
\end{styleEntryParagraph}

\begin{styleEntryParagraph}
\textstyleDefinitionn{ladle}\textstyleLexeme{     ovolom}\textstylefstandard{.}
\end{styleEntryParagraph}

\begin{styleEntryParagraph}
\textstyleDefinitionn{lake}\textstyleLexeme{     dəlov}\textstylefstandard{.}
\end{styleEntryParagraph}

\begin{styleEntryParagraph}
\textstyleDefinitionn{lamp}\textstyleLexeme{     dəndara}\textstylefstandard{.}
\end{styleEntryParagraph}

\begin{styleEntryParagraph}
\textstyleDefinitionn{land}\textstyleLexeme{     wəyen}\textstyleDefinitionn{.}
\end{styleEntryParagraph}

\begin{styleEntryParagraph}
\textstyleDefinitionn{language, word, mouth}\textstyleLexeme{     ma}\textstylefstandard{.}
\end{styleEntryParagraph}

\begin{styleEntryParagraph}
\textstyleDefinitionn{larva, worm}\textstyleLexeme{     mecekweɗ}\textstylefstandard{.}
\end{styleEntryParagraph}

\begin{styleEntryParagraph}
\textstyleDefinitionn{laziness}\textstyleLexeme{     esew}\textstylefstandard{.}
\end{styleEntryParagraph}

\begin{styleEntryParagraph}
\textstyleDefinitionn{leaf for making a sauce}\textstyleLexeme{     njəwelek}\textstylefstandard{.}
\end{styleEntryParagraph}

\begin{styleEntryParagraph}
\textstyleDefinitionn{leaf ; sauce made from edible leaves}\textstyleLexeme{   elele}\textstyleDefinitionn{.}
\end{styleEntryParagraph}

\begin{styleEntryParagraph}
\textstyleDefinitionn{leak, flow    }\textstyleLexeme{ngaz}\textstylefstandard{.}
\end{styleEntryParagraph}

\begin{styleEntryParagraph}
\textstyleDefinitionn{lean}\textstyleLexeme{     jakay}\textstylefstandard{.}
\end{styleEntryParagraph}

\begin{styleEntryParagraph}
\textstyleDefinitionn{lean back}\textstyleLexeme{     dəngaɗay}\textstylefstandard{.}
\end{styleEntryParagraph}

\begin{styleEntryParagraph}
\textstyleDefinitionn{learn, teach}\textstyleLexeme{     dəbənay}\textstylefstandard{.}
\end{styleEntryParagraph}

\begin{styleEntryParagraph}
\textstyleDefinitionn{leave, let go}\textstyleLexeme{     makay}\textstylefstandard{.}
\end{styleEntryParagraph}

\begin{styleEntryParagraph}
\textstyleDefinitionn{leave, stay}\textstyleLexeme{     ndəray}\textstylefstandard{.}
\end{styleEntryParagraph}

\begin{styleEntryParagraph}
\textstyleDefinitionn{leave}\textstyleLexeme{     malay}\textstylefstandard{.}
\end{styleEntryParagraph}

\begin{styleEntryParagraph}
\textstyleDefinitionn{leave in secret}\textstyleLexeme{     slohoy}\textstylefstandard{.}
\end{styleEntryParagraph}

\begin{styleEntryParagraph}
\textstyleDefinitionn{left}\textstyleLexeme{     gəlo}\textstylefstandard{.} 
\end{styleEntryParagraph}

\begin{styleEntryParagraph}
\textstyleDefinitionn{left (gone)    }\textstyleLexeme{nje}\textstylefstandard{.}
\end{styleEntryParagraph}

\begin{styleEntryParagraph}
\textstyleDefinitionn{leg, foot}\textstyleLexeme{     asak}\textstylefstandard{.}
\end{styleEntryParagraph}

\begin{styleEntryParagraph}
\textstyleDefinitionn{leopard}\textstyleLexeme{     medəlengwez, tədo}\textstylefstandard{.}
\end{styleEntryParagraph}

\begin{styleEntryParagraph}
\textstyleDefinitionn{leprosy}\textstyleLexeme{     kokolo}\textstylefstandard{.}
\end{styleEntryParagraph}

\begin{styleEntryParagraph}
\textstyleDefinitionn{liar}\textstyleLexeme{     ongolo}\textstylefstandard{.}
\end{styleEntryParagraph}

\begin{styleEntryParagraph}
\textstyleDefinitionn{lick}\textstyleLexeme{     ndəlkaday}\textstylefstandard{.}
\end{styleEntryParagraph}

\begin{styleEntryParagraph}
\textstyleDefinitionn{lie down}\textstyleLexeme{     nde}\textstylefstandard{.} 
\end{styleEntryParagraph}

\begin{styleEntryParagraph}
\textstyleDefinitionn{life}\textstyleLexeme{     səyfa}\textstylefstandard{.}
\end{styleEntryParagraph}

\begin{styleEntryParagraph}
\textstyleDefinitionn{light, dawn}\textstyleLexeme{     jajay}\textstylefstandard{.}
\end{styleEntryParagraph}

\begin{styleEntryParagraph}
\textstyleDefinitionn{light}\textstyleLexeme{     bay}\textstylefstandard{.}
\end{styleEntryParagraph}

\begin{styleEntryParagraph}
\textstyleDefinitionn{like}\textstyleLexeme{     ka}\textstylefstandard{.}
\end{styleEntryParagraph}

\begin{styleEntryParagraph}
\textstyleDefinitionn{like that}\textstyleLexeme{     kəyga}\textstylefstandard{.}
\end{styleEntryParagraph}

\begin{styleEntryParagraph}
\textstyleDefinitionn{like this}\textstyleLexeme{    kəygehe, }\textstyleSubentry{ka nehe, }\textstyleLexeme{ka ngəhe}\textstylefstandard{.}
\end{styleEntryParagraph}

\begin{styleEntryParagraph}
\textstyleDefinitionn{limp}\textstyleLexeme{     həjəgaɗay}\textstylefstandard{.}
\end{styleEntryParagraph}

\begin{styleEntryParagraph}
\textstyleDefinitionn{limpness}\textstyleLexeme{     jegwer}\textstylefstandard{.}
\end{styleEntryParagraph}

\begin{styleEntryParagraph}
\textstyleDefinitionn{linger}\textstyleLexeme{     zaray}\textstylefstandard{.}
\end{styleEntryParagraph}

\begin{styleEntryParagraph}
\textstyleDefinitionn{lion}\textstyleLexeme{     mazloko}\textstylefstandard{.}
\end{styleEntryParagraph}

\begin{styleEntryParagraph}
\textstyleDefinitionn{liver}\textstyleLexeme{     gəver}\textstylefstandard{.}
\end{styleEntryParagraph}

\begin{styleEntryParagraph}
\textstyleDefinitionn{lizard}\textstyleLexeme{     baybojo}\textstylefstandard{.}
\end{styleEntryParagraph}

\begin{styleIndentedParagraph}
\textstylePartofspeech{long ago}\textstyleSubentry{     pepenna}\textstylefstandard{.} 
\end{styleIndentedParagraph}

\begin{styleEntryParagraph}
\textstyleDefinitionn{look for}\textstyleLexeme{     kəway}\textstylefstandard{.}
\end{styleEntryParagraph}

\begin{styleEntryParagraph}
\textstyleDefinitionn{lose, get lost}\textstyleLexeme{     cəjen}\textstylefstandard{.}
\end{styleEntryParagraph}

\begin{styleEntryParagraph}
\textstyleDefinitionn{lose weight}\textstyleLexeme{     gəzamay}\textstylefstandard{.}
\end{styleEntryParagraph}

\begin{styleEntryParagraph}
\textstyleDefinitionn{lots}\textstyleLexeme{     gobay}\textstylefstandard{.}
\end{styleEntryParagraph}

\begin{styleEntryParagraph}
\textstyleDefinitionn{louse}\textstyleLexeme{     cece}\textstylefstandard{.}
\end{styleEntryParagraph}

\begin{styleEntryParagraph}
\textstyleDefinitionn{love, want}\textstyleLexeme{     ndaɗay}\textstylefstandard{.}
\end{styleEntryParagraph}

\begin{styleEntryParagraph}
\textstyleDefinitionn{luck}\textstyleLexeme{     jen}\textstylefstandard{.}
\end{styleEntryParagraph}

\begin{styleEntryParagraph}
\textstyleDefinitionn{lungs}\textstyleLexeme{     pahav}\textstylefstandard{.}
\end{styleEntryParagraph}

\begin{styleEntryParagraph}
\textstyleDefinitionn{lust}\textstyleLexeme{     obor}\textstylePartofspeech{.}
\end{styleEntryParagraph}
\end{multicols}
\begin{styleLetterParagraph}
\textstyleLetterv{M  {}-  m}
\end{styleLetterParagraph}

\begin{multicols}{2}
\begin{styleEntryParagraph}
\textstyleDefinitionn{make cold}\textstyleLexeme{     ndeslen}\textstylefstandard{.}
\end{styleEntryParagraph}

\begin{styleEntryParagraph}
\textstyleDefinitionn{male, man, husband}\textstyleLexeme{     zar}\textstylefstandard{.}
\end{styleEntryParagraph}

\begin{styleEntryParagraph}
\textstyleDefinitionn{man, young}\textstyleLexeme{     albaya}\textstylefstandard{.}
\end{styleEntryParagraph}

\begin{styleEntryParagraph}
\textstyleDefinitionn{man (young, over 18)    }\textstyleLexeme{njavar}\textstylefstandard{.}
\end{styleEntryParagraph}

\begin{styleEntryParagraph}
\textstyleDefinitionn{mango}\textstyleLexeme{     mongoro}\textstylefstandard{.}
\end{styleEntryParagraph}

\begin{styleEntryParagraph}
\textstyleDefinitionn{manioc}\textstyleLexeme{     ambay}\textstylefstandard{.}
\end{styleEntryParagraph}

\begin{styleEntryParagraph}
\textstyleDefinitionn{many, }\textit{enough}\textstyleLexeme{     haɗa}\textstylefstandard{.}
\end{styleEntryParagraph}

\begin{styleEntryParagraph}
\textstyleDefinitionn{market}\textstyleLexeme{     kosoko}\textstylefstandard{.} 
\end{styleEntryParagraph}

\begin{styleEntryParagraph}
\textstyleDefinitionn{market day at home}\textstyleLexeme{     molom}\textstyleDefinitionn{.}
\end{styleEntryParagraph}

\begin{styleEntryParagraph}
\textstyleDefinitionn{market day at the village of Doule    }\textstyleLexeme{  Dəwlek}\textstylefstandard{.}
\end{styleEntryParagraph}

\begin{styleEntryParagraph}
\textstyleDefinitionn{market day at the village of Mok}\textstyleLexeme{\textmd{\textit{əyo}}}\textstyleLexeme{   Mokəyo}\textstylefstandard{.}
\end{styleEntryParagraph}

\begin{styleEntryParagraph}
\textstyleDefinitionn{market day in the village of Meme}\textstyleLexeme{   Meme}\textstylefstandard{.}
\end{styleEntryParagraph}

\begin{styleEntryParagraph}
\textstyleDefinitionn{marry}\textstyleLexeme{     baɗay}\textstylefstandard{.}
\end{styleEntryParagraph}

\begin{styleEntryParagraph}
\textstyleDefinitionn{mash}\textstyleLexeme{     jəɗokoy}\textstylefstandard{.}
\end{styleEntryParagraph}

\begin{styleEntryParagraph}
\textstyleDefinitionn{mat}\textstyleLexeme{     bəwce}\textstylefstandard{.}
\end{styleEntryParagraph}

\begin{styleEntryParagraph}
\textstyleDefinitionn{mate with}\textstyleLexeme{     laway}\textstylefstandard{.}
\end{styleEntryParagraph}

\begin{styleEntryParagraph}
\textstyleDefinitionn{maybe}\textstyleLexeme{     ndawan}\textstylefstandard{.}
\end{styleEntryParagraph}

\begin{styleEntryParagraph}
\textstyleDefinitionn{Mbuko people/language}\textstyleLexeme{     Mboko}\textstylefstandard{.}
\end{styleEntryParagraph}

\begin{styleEntryParagraph}
\textstyleDefinitionn{meanwhile}\textstyleLexeme{     macəkəmbay}\textstylefstandard{.}
\end{styleEntryParagraph}

\begin{styleEntryParagraph}
\textstyleDefinitionn{meat}\textstyleLexeme{     sese}\textstylefstandard{.}
\end{styleEntryParagraph}

\begin{styleEntryParagraph}
\textstyleDefinitionn{medicine}\textstyleLexeme{     həraf}\textstylefstandard{.}
\end{styleEntryParagraph}

\begin{styleEntryParagraph}
\textstyleDefinitionn{meet, intersect}\textstyleLexeme{     dozloy}\textstylefstandard{.} 
\end{styleEntryParagraph}

\begin{styleEntryParagraph}
\textstyleDefinitionn{melt}\textstyleLexeme{     pəɗak}\textstylefstandard{.}
\end{styleEntryParagraph}

\begin{styleEntryParagraph}
\textstyleLexeme{\textmd{\textit{men}}}\textstyleLexeme{     zawər ahay}\textstyleLexeme{\textmd{\textit{. }}}
\end{styleEntryParagraph}

\begin{styleEntryParagraph}
\textstyleDefinitionn{metal, iron}\textstyleLexeme{     hara}\textstylefstandard{.}
\end{styleEntryParagraph}

\begin{styleEntryParagraph}
\textstyleDefinitionn{middle, centre}\textstyleLexeme{    mbeɗem}\textstylefstandard{.}
\end{styleEntryParagraph}

\begin{styleEntryParagraph}
\textstyleDefinitionn{milk}\textstylefstandard{, }\textstylefstandard{\textit{breast}}\textstyleLexeme{     ɗəwa}\textstylefstandard{.}
\end{styleEntryParagraph}

\begin{styleEntryParagraph}
\textstyleDefinitionn{millet}\textstyleLexeme{       hay}\textstylefstandard{.}
\end{styleEntryParagraph}

\begin{styleEntryParagraph}
\textstyleDefinitionn{millet}, \textstyleDefinitionn{dry season; }\textit{sorghum}\textstyleLexeme{     omsoko}\textstylefstandard{.}
\end{styleEntryParagraph}

\begin{styleEntryParagraph}
\textstyleDefinitionn{millet, red    }\textstyleLexeme{ mbərkala}\textstylefstandard{.}
\end{styleEntryParagraph}

\begin{styleEntryParagraph}
\textit{millet} \textstyleDefinitionn{beer}\textstyleLexeme{     gəzom}\textstyleDefinitionn{. }\textstyleflabel{ }
\end{styleEntryParagraph}

\begin{styleEntryParagraph}
\textstyleDefinitionn{millet drink}\textstyleLexeme{     dəwlay}\textstylefstandard{.}
\end{styleEntryParagraph}

\begin{styleEntryParagraph}
\textstyleDefinitionn{millet leaf}\textstyleLexeme{     fefen}\textstylefstandard{.}
\end{styleEntryParagraph}

\begin{styleEntryParagraph}
\textstyleDefinitionn{millet porridge, food}\textstyleLexeme{     ɗaf}\textstylefstandard{.}
\end{styleEntryParagraph}

\begin{styleEntryParagraph}
\textstyleDefinitionn{minimize}\textstyleLexeme{     rasay}\textstylefstandard{.}
\end{styleEntryParagraph}

\begin{styleEntryParagraph}
\textstyleDefinitionn{misbehave}\textstyleLexeme{     sədaray}\textstylefstandard{.}
\end{styleEntryParagraph}

\begin{styleEntryParagraph}
\textstyleDefinitionn{misfortune}\textstyleLexeme{     ezewk}\textstylefstandard{.}
\end{styleEntryParagraph}

\begin{styleEntryParagraph}
\textstyleDefinitionn{mix}\textstyleLexeme{     japay}\textstylefstandard{.}
\end{styleEntryParagraph}

\begin{styleEntryParagraph}
\textstyleDefinitionn{mix grain with ashes to prevent insects from eating seeds}\textstyleLexeme{     slahay}\textstylefstandard{.} 
\end{styleEntryParagraph}

\begin{styleEntryParagraph}
\textstyleDefinitionn{Moloko people/language}\textstyleLexeme{     Məloko}\textstylefstandard{.}
\end{styleEntryParagraph}

\begin{styleEntryParagraph}
\textstyleDefinitionn{Monday market, prince}\textstyleLexeme{     Yerəyma}\textstylefstandard{.}
\end{styleEntryParagraph}

\begin{styleEntryParagraph}
\textstyleDefinitionn{money}\textstyleLexeme{     dala, səloy}\textstylefstandard{.}
\end{styleEntryParagraph}

\begin{styleEntryParagraph}
\textstyleDefinitionn{mongoose}\textstyleLexeme{     mekəlewez}\textstylefstandard{.}
\end{styleEntryParagraph}

\begin{styleEntryParagraph}
\textstyleDefinitionn{monkey}\textstyleLexeme{     kərkaɗaw}\textstyleDefinitionn{.}
\end{styleEntryParagraph}

\begin{styleEntryParagraph}
\textstyleDefinitionn{moon}\textstyleLexeme{     kəya}\textstylefstandard{.}
\end{styleEntryParagraph}

\begin{styleEntryParagraph}
\textstyleDefinitionn{morning}\textstyleLexeme{     dedew}\textstyleDefinitionn{.}
\end{styleEntryParagraph}

\begin{styleEntryParagraph}
\textstyleDefinitionn{mortar}\textstyleLexeme{     cəjen}\textstylefstandard{.}
\end{styleEntryParagraph}

\begin{styleEntryParagraph}
\textstyleDefinitionn{mosquito}\textstyleLexeme{     tenjew}\textstylefstandard{.}
\end{styleEntryParagraph}

\begin{styleEntryParagraph}
\textstyleDefinitionn{mother}\textstyleLexeme{     mama}\textstylefstandard{.}
\end{styleEntryParagraph}

\begin{styleEntryParagraph}
\textstyleDefinitionn{mount}\textstyleLexeme{     kəroy}\textstylefstandard{.}
\end{styleEntryParagraph}

\begin{styleEntryParagraph}
\textstyleDefinitionn{mountain}\textstyleLexeme{     ɓərzlan}\textstylefstandard{.}
\end{styleEntryParagraph}

\begin{styleEntryParagraph}
\textstyleDefinitionn{mouse}\textstyleLexeme{     okfom}\textstylefstandard{.}
\end{styleEntryParagraph}

\begin{styleEntryParagraph}
\textstyleDefinitionn{mouse species}\textstyleLexeme{     zən zan}\textstylefstandard{.}
\end{styleEntryParagraph}

\begin{styleEntryParagraph}
\textstyleDefinitionn{mouse trap}\textstyleLexeme{     lolokoy, ngəm\nobreakdash-ngam}\textstylefstandard{.}
\end{styleEntryParagraph}

\begin{styleEntryParagraph}
\textstyleDefinitionn{mouth, language, word}\textstyleLexeme{     ma}\textstylefstandard{.}
\end{styleEntryParagraph}

\begin{styleEntryParagraph}
\textstyleDefinitionn{move}\textstyleLexeme{     bal}\textstylefstandard{.}
\end{styleEntryParagraph}

\begin{styleEntryParagraph}
\textstyleDefinitionn{much}\textstyleLexeme{     gam}\textstylefstandard{.} 
\end{styleEntryParagraph}

\begin{styleEntryParagraph}
\textstyleDefinitionn{multiply}\textstyleLexeme{     ɗaslay, sak}\textstylefstandard{.}
\end{styleEntryParagraph}

\begin{styleEntryParagraph}
\textstyleDefinitionn{mushroom}\textstyleLexeme{     opongo}\textstylefstandard{.}
\end{styleEntryParagraph}
\end{multicols}
\begin{styleLetterParagraph}
\textstyleLetterv{N  {}-  n}
\end{styleLetterParagraph}

\begin{multicols}{2}
\begin{styleEntryParagraph}
\textstyleDefinitionn{nail, claw  }\textstyleLexeme{ ehwəɗe}\textstylePartofspeech{.} 
\end{styleEntryParagraph}

\begin{styleEntryParagraph}
\textstyleDefinitionn{naivety, kindness}\textstyleLexeme{     zənof}\textstylefstandard{.}
\end{styleEntryParagraph}

\begin{styleEntryParagraph}
\textstyleDefinitionn{name, ear}\textstyleLexeme{     sləmay}\textstylefstandard{.}
\end{styleEntryParagraph}

\begin{styleEntryParagraph}
\textstyleDefinitionn{name of child following twins}\textstyleLexeme{     Aban}\textstylefstandard{.}
\end{styleEntryParagraph}

\begin{styleEntryParagraph}
\textstyleDefinitionn{name of first twin}\textstyleLexeme{     Masay}\textstylefstandard{.}
\end{styleEntryParagraph}

\begin{styleEntryParagraph}
\textit{name of second twin}\textstyleLexeme{   Aləwa}\textstylefstandard{.} 
\end{styleEntryParagraph}

\begin{styleEntryParagraph}
\textstyleDefinitionn{name of a village and a clan of Moloko}\textstyleLexeme{   Fətak}\textstylefstandard{.} 
\end{styleEntryParagraph}

\begin{styleEntryParagraph}
\textstyleDefinitionn{nape}\textstyleLexeme{     ɗəgom}\textstylefstandard{.}
\end{styleEntryParagraph}

\begin{styleEntryParagraph}
\textstyleSensenumber{\textit{neck, voice}}\textstyleLexeme{     dəngo}\textstyleSensenumber{\textit{.}}
\end{styleEntryParagraph}

\begin{styleEntryParagraph}
\textstyleDefinitionn{needle}\textstyleLexeme{     ləpəre}\textstylefstandard{.}
\end{styleEntryParagraph}

\begin{styleIndentedParagraph}
\textstyleDefinitionn{neighbour}\textstyleSubentry{     }\textstyleLexeme{dəlmete,}\textstyleSubentry{ vəy}\textstyleLexeme{mete}\textstylefstandard{.}
\end{styleIndentedParagraph}

\begin{styleEntryParagraph}
\textstyleDefinitionn{net}\textstyleLexeme{     zəva}\textstylefstandard{.}
\end{styleEntryParagraph}

\begin{styleEntryParagraph}
\textstyleDefinitionn{never again}\textstyleLexeme{     asabay}\textstylefstandard{.}
\end{styleEntryParagraph}

\begin{styleEntryParagraph}
\textstyleDefinitionn{newborn baby}\textstyleLexeme{     kokofoy}\textstylefstandard{.}
\end{styleEntryParagraph}

\begin{styleEntryParagraph}
\textstyleDefinitionn{news}\textstyleLexeme{     ləbara}\textstylefstandard{.}
\end{styleEntryParagraph}

\begin{styleEntryParagraph}
\textstyleDefinitionn{next year}\textstyleLexeme{     bəyaw}\textstylefstandard{.}
\end{styleEntryParagraph}

\begin{styleEntryParagraph}
\textstyleDefinitionn{night}\textstyleLexeme{     ləvan}\textstylefstandard{.}
\end{styleEntryParagraph}

\begin{styleEntryParagraph}
\textstyleDefinitionn{nine}\textstyleLexeme{     holombo}\textstylefstandard{.}
\end{styleEntryParagraph}

\begin{styleEntryParagraph}
\textstyleDefinitionn{no}\textstyleLexeme{     ehe}\textstylefstandard{.}
\end{styleEntryParagraph}

\begin{styleEntryParagraph}
\textstyleDefinitionn{no one}\textstyleLexeme{     meslenen}\textstylefstandard{.}
\end{styleEntryParagraph}

\begin{styleEntryParagraph}
\textstyleDefinitionn{nod}\textstyleLexeme{     gazay}\textstylefstandard{.}
\end{styleEntryParagraph}

\begin{styleEntryParagraph}
\textstyleDefinitionn{nose}\textstyleLexeme{     hənder}\textstylefstandard{.}
\end{styleEntryParagraph}

\begin{styleEntryParagraph}
\textstyleDefinitionn{not}\textstyleLexeme{     bay}\textstylefstandard{.}
\end{styleEntryParagraph}

\begin{styleEntryParagraph}
\textstyleDefinitionn{not so?    }\textstyleLexeme{esəmey}\textstylefstandard{.}
\end{styleEntryParagraph}

\begin{styleEntryParagraph}
\textstyleDefinitionn{not yet}\textstyleLexeme{     fabay}\textstylefstandard{.}
\end{styleEntryParagraph}

\begin{styleEntryParagraph}
\textstyleDefinitionn{notice, inspect}\textstyleLexeme{     zəroy}\textstyleDefinitionn{.}
\end{styleEntryParagraph}

\begin{styleEntryParagraph}
\textstylePartofspeech{noun clitic.} \textstyleDefinitionn{plural}\textstyleLexeme{     ahay}\textstylefstandard{.}
\end{styleEntryParagraph}

\begin{styleEntryParagraph}
\textstylePartofspeech{noun clitic.} \textstyleDefinitionn{1S possessive    }\textstyleLexeme{əwla}\textstyleDefinitionn{.}
\end{styleEntryParagraph}

\begin{styleEntryParagraph}
\textstylefstandard{\textit{noun clitic}}\textstylePartofspeech{.} \textstyleDefinitionn{2S possessive}\textstyleLexeme{     ango}\textstylefstandard{.}
\end{styleEntryParagraph}

\begin{styleEntryParagraph}
\textstylefstandard{\textit{noun}}\textstylefstandard{ }\textstylePartofspeech{clitic. 3S possessive}\textstyleLexeme{     ahan}\textstylePartofspeech{.}
\end{styleEntryParagraph}

\begin{styleEntryParagraph}
\textstylefstandard{\textit{noun }}\textstylePartofspeech{clitic 1P}\textstylePartofspeech{EX}\textstyleDefinitionn{ possessive}\textstyleLexeme{   aləme}\textstylefstandard{.}
\end{styleEntryParagraph}

\begin{styleEntryParagraph}
\textstylefstandard{\textit{noun }}\textstylePartofspeech{clitic 1P}\textstylePartofspeech{IN}\textstyleDefinitionn{ possessive}\textstyleLexeme{   aloko}\textstylefstandard{.}
\end{styleEntryParagraph}

\begin{styleEntryParagraph}
\textstylefstandard{\textit{noun clitic}}\textstylePartofspeech{.} \textstyleDefinitionn{2P possessive}\textstyleLexeme{     aləkwəye}\textstylefstandard{.}
\end{styleEntryParagraph}

\begin{styleEntryParagraph}
\textstylePartofspeech{noun clitic }\textstyleDefinitionn{3P possessive}\textstyleLexeme{     atəta}\textstylefstandard{.}
\end{styleEntryParagraph}

\begin{styleEntryParagraph}
\textstylePartofspeech{noun clitic} \textstyleDefinitionn{adjectiviser    }\textstyleLexeme{ga}\textstylefstandard{.}
\end{styleEntryParagraph}

\begin{styleEntryParagraph}
\textstylePartofspeech{noun suffix} \textstyleDefinitionn{respectful vocative}\textstyleLexeme{   \nobreakdash-ya}\textstylefstandard{.}
\end{styleEntryParagraph}

\begin{styleEntryParagraph}
\textstyleDefinitionn{now}\textstyleLexeme{     cəcəngehe, azla}\textstylefstandard{.}
\end{styleEntryParagraph}

\begin{styleEntryParagraph}
\textstyleDefinitionn{number}\textstyleLexeme{     lamba}\textstylefstandard{.}
\end{styleEntryParagraph}
\end{multicols}
\begin{styleLetterParagraph}
\textstyleLetterv{O  {}-  o}
\end{styleLetterParagraph}

\begin{multicols}{2}
\begin{styleEntryParagraph}
\textstyleDefinitionn{obligation}\textstyleLexeme{     dewele}\textstylefstandard{.}
\end{styleEntryParagraph}

\begin{styleEntryParagraph}
\textstyleDefinitionn{oil}\textstyleLexeme{     amar}\textstyleDefinitionn{.}
\end{styleEntryParagraph}

\begin{styleEntryParagraph}
\textstyleDefinitionn{okra}\textstyleLexeme{     atəko}\textstylefstandard{.}
\end{styleEntryParagraph}

\begin{styleEntryParagraph}
\textstyleDefinitionn{old person}\textstyleLexeme{     mədehwer}\textstylefstandard{.}
\end{styleEntryParagraph}

\begin{styleEntryParagraph}
\textstyleDefinitionn{older sibling}\textstyleLexeme{     mədəga}\textstylefstandard{.}
\end{styleEntryParagraph}

\begin{styleEntryParagraph}
\textstyleDefinitionn{on}\textstyleLexeme{     kə…aka}\textstylefstandard{.}
\end{styleEntryParagraph}

\begin{styleEntryParagraph}
\textstyleDefinitionn{one}\textstyleLexeme{     bəlen}\textstylefstandard{.}
\end{styleEntryParagraph}

\begin{styleEntryParagraph}
\textstyleDefinitionn{one complete year}\textstyleLexeme{     daz}\textstylefstandard{.}
\end{styleEntryParagraph}

\begin{styleEntryParagraph}
\textstyleDefinitionn{one time, occasion}\textstyleLexeme{     baya}\textstylefstandard{.}
\end{styleEntryParagraph}

\begin{styleEntryParagraph}
\textstyleDefinitionn{onion}\textstyleLexeme{     eteme}\textstylefstandard{.}
\end{styleEntryParagraph}

\begin{styleEntryParagraph}
\textstyleDefinitionn{open}\textstyleLexeme{ pay}\textstylefstandard{.}
\end{styleEntryParagraph}

\begin{styleEntryParagraph}
\textstyleDefinitionn{ostrich}\textstyleLexeme{     erkece}\textstylefstandard{.}
\end{styleEntryParagraph}

\begin{styleEntryParagraph}
\textstyleDefinitionn{outside}\textstyleLexeme{     amata}\textstyleDefinitionn{.}
\end{styleEntryParagraph}

\begin{styleEntryParagraph}
\textstyleDefinitionn{overwhelm}\textstyleLexeme{     cə}\textstyleLexeme{ɓ}\textstyleLexeme{ay}\textstylefstandard{.}
\end{styleEntryParagraph}

\begin{styleEntryParagraph}
\textstyleDefinitionn{owl}\textstyleLexeme{     hehen}\textstylefstandard{.}
\end{styleEntryParagraph}

\begin{styleEntryParagraph}
\textstyleDefinitionn{oyster}\textstyleLexeme{     vəlalay}\textstylefstandard{.}
\end{styleEntryParagraph}
\end{multicols}
\begin{styleLetterParagraph}
\textstyleLetterv{P  {}-  p}
\end{styleLetterParagraph}

\begin{multicols}{2}
\begin{styleEntryParagraph}
\textstyleDefinitionn{pack down}\textstyleLexeme{     jokoy}\textstylefstandard{.}
\end{styleEntryParagraph}

\begin{styleEntryParagraph}
\textstyleDefinitionn{pap, hot drink made with rice}\textstyleLexeme{     mətərak}\textstyleDefinitionn{.}
\end{styleEntryParagraph}

\begin{styleEntryParagraph}
\textstylePartofspeech{paper}\textstyleLexeme{     ɗeləywel}\textstylefstandard{.}
\end{styleEntryParagraph}

\begin{styleEntryParagraph}
\textstyleDefinitionn{pardon}\textstyleLexeme{     homboh}\textstylefstandard{.}
\end{styleEntryParagraph}

\begin{styleEntryParagraph}
\textstyleDefinitionn{partridge}\textstyleLexeme{     təkwərak}\textstylefstandard{.}
\end{styleEntryParagraph}

\begin{styleEntryParagraph}
\textstyleDefinitionn{pass}\textstyleLexeme{     mbərzlay}\textstylefstandard{.}
\end{styleEntryParagraph}

\begin{styleEntryParagraph}
\textstyleDefinitionn{pay}\textstyleLexeme{     par, wərkay}\textstylefstandard{.}
\end{styleEntryParagraph}

\begin{styleEntryParagraph}
\textstyleDefinitionn{pay a debt}\textstyleLexeme{     hamay}\textstylefstandard{.}
\end{styleEntryParagraph}

\begin{styleEntryParagraph}
\textstyleSensenumber{\textit{peace, wholeness}}\textstyleLexeme{     zay, zazay}\textstylefstandard{\textit{.}}
\end{styleEntryParagraph}

\begin{styleEntryParagraph}
\textstyleDefinitionn{peanut}\textstyleLexeme{     andəra}\textstylefstandard{.}
\end{styleEntryParagraph}

\begin{styleEntryParagraph}
\textstyleDefinitionn{peanut cookie, deep fried}\textstyleLexeme{     azay}\textstylefstandard{ }\textstyleLexeme{andəra}\textstyleDefinitionn{.}
\end{styleEntryParagraph}

\begin{styleEntryParagraph}
\textstyleDefinitionn{peel}\textstyleLexeme{     cəlokoy}\textstylefstandard{.}
\end{styleEntryParagraph}

\begin{styleEntryParagraph}
\textstyleDefinitionn{peel, skin}\textstyleLexeme{     mbəlɗoy}\textstylefstandard{.}
\end{styleEntryParagraph}

\begin{styleEntryParagraph}
\textstyleDefinitionn{peel off, undress}\textstyleLexeme{     kərtoy}\textstylefstandard{.}
\end{styleEntryParagraph}

\begin{styleEntryParagraph}
\textstyleDefinitionn{people}\textstyleLexeme{     ndam}\textstylefstandard{.}
\end{styleEntryParagraph}

\begin{styleEntryParagraph}
\textstyleDefinitionn{Perfect    }\textstyleLexeme{va}\textstylefstandard{.}
\end{styleEntryParagraph}

\begin{styleEntryParagraph}
\textstyleDefinitionn{perhaps}\textstyleLexeme{     azana, adan bay}\textstylefstandard{.}
\end{styleEntryParagraph}

\begin{styleEntryParagraph}
\textstyleDefinitionn{person}\textstyleLexeme{     məze}\textstylePartofspeech{.} 
\end{styleEntryParagraph}

\begin{styleEntryParagraph}
\textstyleDefinitionn{persuade, relieve}\textstyleLexeme{     dəbakay}\textstylefstandard{.}
\end{styleEntryParagraph}

\begin{styleEntryParagraph}
\textstyleDefinitionn{pierce}\textstyleLexeme{     caslay, zlar}\textstylefstandard{.}
\end{styleEntryParagraph}

\begin{styleEntryParagraph}
\textstyleDefinitionn{pierce, cut}\textstyleLexeme{     cazlay}\textstylefstandard{.} 
\end{styleEntryParagraph}

\begin{styleEntryParagraph}
\textstyleDefinitionn{pig}\textstyleLexeme{     madəras}\textstylefstandard{.}
\end{styleEntryParagraph}

\begin{styleEntryParagraph}
\textstyleDefinitionn{pile something}\textstyleLexeme{     tah}\textstylefstandard{.}
\end{styleEntryParagraph}

\begin{styleEntryParagraph}
\textstyleDefinitionn{place}\textstyleLexeme{     slam}\textstylefstandard{.   }
\end{styleEntryParagraph}

\begin{styleEntryParagraph}
\textstyleDefinitionn{plant}\textstyleLexeme{     jav}\textstyleDefinitionn{.}
\end{styleEntryParagraph}

\begin{styleEntryParagraph}
\textstyleDefinitionn{plant, snore}\textstyleLexeme{     daray}\textstylefstandard{.}
\end{styleEntryParagraph}

\begin{styleEntryParagraph}
\textstyleDefinitionn{plant, throw}\textstyleLexeme{     zləge}\textstyleDefinitionn{.}
\end{styleEntryParagraph}

\begin{styleEntryParagraph}
\textstyleDefinitionn{play a wind instrument}\textstyleLexeme{     fe}\textstylefstandard{.}
\end{styleEntryParagraph}

\begin{styleEntryParagraph}
\textstyleDefinitionn{pluck}\textstyleLexeme{     rah}\textstylefstandard{.}
\end{styleEntryParagraph}

\begin{styleEntryParagraph}
\textstyleDefinitionn{plug}\textstyleLexeme{     ɗak}\textstylefstandard{.}
\end{styleEntryParagraph}

\begin{styleEntryParagraph}
\textstyleDefinitionn{polite demand}\textstyleLexeme{     etey}\textstylePartofspeech{.} 
\end{styleEntryParagraph}

\begin{styleEntryParagraph}
\textstyleDefinitionn{populate}\textstyleLexeme{     wasay, wəɗoy}\textstylefstandard{.}
\end{styleEntryParagraph}

\begin{styleEntryParagraph}
\textstyleDefinitionn{possessed by}\textstyleLexeme{     anga}\textstylefstandard{.}
\end{styleEntryParagraph}

\begin{styleEntryParagraph}
\textstyleDefinitionn{pot}\textstyleLexeme{     ɗeɗew, məsek}\textstylefstandard{.}
\end{styleEntryParagraph}

\begin{styleEntryParagraph}
\textstyleDefinitionn{potash}\textstyleLexeme{     wəle}\textstylefstandard{.}
\end{styleEntryParagraph}

\begin{styleEntryParagraph}
\textstyleDefinitionn{power}\textstyleLexeme{     njəɗa}\textstylefstandard{.}
\end{styleEntryParagraph}

\begin{styleEntryParagraph}
\textstyleDefinitionn{prepare, }\textit{cook}\textstyleLexeme{     de}\textstylefstandard{.}
\end{styleEntryParagraph}

\begin{styleEntryParagraph}
\textstyleDefinitionn{presupposition marker}\textstyleLexeme{     na}\textstylefstandard{.}
\end{styleEntryParagraph}

\begin{styleEntryParagraph}
\textstyleDefinitionn{prevent}\textstyleLexeme{     ngar}\textstylefstandard{.}
\end{styleEntryParagraph}

\begin{styleEntryParagraph}
\textstyleDefinitionn{price}\textstyleLexeme{     cəkele}\textstylefstandard{.}
\end{styleEntryParagraph}

\begin{styleEntryParagraph}
\textstylePartofspeech{pronoun} \textstyleDefinitionn{2S}\textstyleLexeme{     nok}\textstylefstandard{.}
\end{styleEntryParagraph}

\begin{styleEntryParagraph}
\textstylePartofspeech{pronoun.} \textstyleDefinitionn{3S}\textstyleLexeme{     ndahan}\textstylefstandard{.}
\end{styleEntryParagraph}

\begin{styleEntryParagraph}
\textstylePartofspeech{pronoun.} \textstyleDefinitionn{1S    }\textstyleLexeme{ne}\textstylefstandard{.} 
\end{styleEntryParagraph}

\begin{styleEntryParagraph}
\textstylePartofspeech{pronoun} \textstyleDefinitionn{1P}\textstylePartofspeech{EX}\textstyleDefinitionn{    }\textstyleLexeme{ləme}\textstylefstandard{.}
\end{styleEntryParagraph}

\begin{styleEntryParagraph}
\textstylePartofspeech{pronoun} \textstyleDefinitionn{1P}\textstylePartofspeech{IN}\textstyleDefinitionn{    }\textstyleLexeme{loko}\textstylefstandard{.}
\end{styleEntryParagraph}

\begin{styleEntryParagraph}
\textstylePartofspeech{pronoun} \textstyleDefinitionn{2P}\textstyleLexeme{     ləkwəye}\textstylefstandard{.}
\end{styleEntryParagraph}

\begin{styleEntryParagraph}
\textstylePartofspeech{pronoun} \textstyleDefinitionn{3P}\textstyleLexeme{     təta}\textstylefstandard{.}
\end{styleEntryParagraph}

\begin{styleEntryParagraph}
\textstyleDefinitionn{pound, crush}\textstyleLexeme{     zlaɓay}\textstylefstandard{.}
\end{styleEntryParagraph}

\begin{styleEntryParagraph}
\textstyleDefinitionn{pour}\textstyleLexeme{     bah}\textstylefstandard{.}
\end{styleEntryParagraph}

\begin{styleEntryParagraph}
\textstylePartofspeech{prince,}\textstyleDefinitionn{ Monday market}\textstyleLexeme{     Yerəyma}\textstylefstandard{.}
\end{styleEntryParagraph}

\begin{styleEntryParagraph}
\textstyleDefinitionn{prune}\textstyleLexeme{     kaɗay}\textstylefstandard{.}
\end{styleEntryParagraph}

\begin{styleEntryParagraph}
\textit{p}\textstyleDefinitionn{ublish, announce}\textstyleLexeme{     wəzlay}\textstyleDefinitionn{.}
\end{styleEntryParagraph}

\begin{styleEntryParagraph}
\textstyleDefinitionn{pull}\textstyleLexeme{     gəjah}\textstylefstandard{.}
\end{styleEntryParagraph}

\begin{styleEntryParagraph}
\textstyleDefinitionn{pull out, jump}\textstyleLexeme{     həraɗ}\textstylefstandard{.}
\end{styleEntryParagraph}

\begin{styleEntryParagraph}
\textstyleDefinitionn{pumpkin}\textstyleLexeme{     maɓasl}\textstylefstandard{.}
\end{styleEntryParagraph}

\begin{styleEntryParagraph}
\textstyleDefinitionn{punish}\textstyleLexeme{     kətay}\textstylefstandard{.}
\end{styleEntryParagraph}

\begin{styleEntryParagraph}
\textstyleDefinitionn{pus}\textstyleLexeme{     oroh}\textstylefstandard{.}
\end{styleEntryParagraph}

\begin{styleEntryParagraph}
\textstyleDefinitionn{push}\textstyleLexeme{     hakay}\textstylefstandard{.}
\end{styleEntryParagraph}

\begin{styleEntryParagraph}
\textstyleDefinitionn{put}\textstyleLexeme{     koroy}\textstylefstandard{.}
\end{styleEntryParagraph}

\begin{styleEntryParagraph}
\textstyleDefinitionn{put, set down}\textstyleLexeme{     faɗ}\textstylefstandard{.}
\end{styleEntryParagraph}

\begin{styleEntryParagraph}
\textstyleDefinitionn{put horizontally}\textstyleLexeme{     mərcay}\textstylefstandard{.}
\end{styleEntryParagraph}

\begin{styleEntryParagraph}
\textit{put on a }\textstyleDefinitionn{roof}\textstyleLexeme{     var}\textstylefstandard{.}
\end{styleEntryParagraph}
\end{multicols}
\begin{styleLetterParagraph}
\textstyleLetterv{Q  {}-  q}
\end{styleLetterParagraph}

\begin{multicols}{2}
\begin{styleEntryParagraph}
\textstyleDefinitionn{quarrel}\textit{, hate}\textstyleLexeme{    hərnje}\textstylefstandard{.}
\end{styleEntryParagraph}

\begin{styleEntryParagraph}
\textit{question}\textstyleSensenumber{\textit{ marker}}\textstyleLexeme{     ɗaw}\textstylefstandard{.}
\end{styleEntryParagraph}
\end{multicols}
\begin{styleLetterParagraph}
\textstyleLetterv{R  {}-  r}
\end{styleLetterParagraph}

\begin{multicols}{2}
\begin{styleEntryParagraph}
\textstyleDefinitionn{rabbit}\textstyleLexeme{     zlevek}\textstylefstandard{.}
\end{styleEntryParagraph}

\begin{styleEntryParagraph}
\textstyleDefinitionn{rafter}\textstyleLexeme{     kəre}\textstylefstandard{.}
\end{styleEntryParagraph}

\begin{styleEntryParagraph}
\textstyleDefinitionn{rain}\textstyleLexeme{     avar}\textstylefstandard{.}
\end{styleEntryParagraph}

\begin{styleEntryParagraph}
\textstyleDefinitionn{rainy season}\textstyleLexeme{     savah, vəya}\textstylefstandard{.}
\end{styleEntryParagraph}

\begin{styleEntryParagraph}
\textstyleDefinitionn{ram}\textstyleLexeme{     gəgoro}\textstylefstandard{.}
\end{styleEntryParagraph}

\begin{styleEntryParagraph}
\textstyleDefinitionn{razor}\textstyleLexeme{     peɗewk}\textstylefstandard{.}
\end{styleEntryParagraph}

\begin{styleEntryParagraph}
\textstyleDefinitionn{reach out}\textstyleLexeme{     tah}\textstylefstandard{.}
\end{styleEntryParagraph}

\begin{styleEntryParagraph}
\textstyleDefinitionn{redness}\textstyleLexeme{     gogwez}\textstylefstandard{.}
\end{styleEntryParagraph}

\begin{styleEntryParagraph}
\textstyleDefinitionn{recoil, withdraw}\textstyleLexeme{     dar}\textstylefstandard{.}
\end{styleEntryParagraph}

\begin{styleEntryParagraph}
\textstyleDefinitionn{relieve, persuade}\textstyleLexeme{     dəbakay}\textstylefstandard{.}
\end{styleEntryParagraph}

\begin{styleEntryParagraph}
\textstyleDefinitionn{remove}\textstyleLexeme{     zlərav}\textstylefstandard{.}
\end{styleEntryParagraph}

\begin{styleEntryParagraph}
\textstyleDefinitionn{remove forcibly}\textstyleLexeme{     pərtay}\textstylefstandard{.}
\end{styleEntryParagraph}

\begin{styleEntryParagraph}
\textstyleDefinitionn{remove insides}\textstyleLexeme{     pəcahay}\textstylefstandard{.}
\end{styleEntryParagraph}

\begin{styleEntryParagraph}
\textstylePartofspeech{repair}\textstyleLexeme{     }\textstyleLexeme{seɓetəy, }\textstyleLexeme{sləɓatay}\textstylePartofspeech{.}
\end{styleEntryParagraph}

\begin{styleEntryParagraph}
\textstyleDefinitionn{reprimand, scold}\textstyleLexeme{     ndahay}\textstylefstandard{.}
\end{styleEntryParagraph}

\begin{styleEntryParagraph}
\textstyleDefinitionn{rest, breathe}\textstyleLexeme{     mbesen}\textstyleDefinitionn{.}
\end{styleEntryParagraph}

\begin{styleEntryParagraph}
\textstyleDefinitionn{rib}\textstyleLexeme{     vəy}\textstylefstandard{.}
\end{styleEntryParagraph}

\begin{styleEntryParagraph}
\textstyleDefinitionn{richness}\textstyleLexeme{     zləle}\textstylefstandard{.}
\end{styleEntryParagraph}

\begin{styleEntryParagraph}
\textstyleDefinitionn{rip}\textstyleLexeme{     ngaray}\textstylefstandard{.}
\end{styleEntryParagraph}

\begin{styleEntryParagraph}
\textstyleDefinitionn{ripen}\textstyleLexeme{     nah}\textstylefstandard{.}
\end{styleEntryParagraph}

\begin{styleEntryParagraph}
\textstyleDefinitionn{river}\textstyleLexeme{     zəraka}\textstylefstandard{.}
\end{styleEntryParagraph}

\begin{styleEntryParagraph}
\textstyleDefinitionn{road}\textstyleLexeme{     cəveɗ}\textstylefstandard{.}
\end{styleEntryParagraph}

\begin{styleEntryParagraph}
\textstyleDefinitionn{roast}\textstyleLexeme{     njahay}\textstylefstandard{.}
\end{styleEntryParagraph}

\begin{styleEntryParagraph}
\textstyleDefinitionn{rock}\textstyleLexeme{     okor}\textstylefstandard{.}
\end{styleEntryParagraph}

\begin{styleEntryParagraph}
\textstyleDefinitionn{rock, large}\textstyleLexeme{     pəraɗ}\textstylefstandard{.}
\end{styleEntryParagraph}

\begin{styleEntryParagraph}
\textstyleDefinitionn{roll, wind}\textstyleLexeme{     təɗoy}\textstylefstandard{.}
\end{styleEntryParagraph}

\begin{styleEntryParagraph}
\textstyleDefinitionn{room}\textstyleLexeme{     ver}\textstylefstandard{.}
\end{styleEntryParagraph}

\begin{styleEntryParagraph}
\textstyleDefinitionn{rooster}\textstyleLexeme{     agwazlak}\textstylefstandard{.}
\end{styleEntryParagraph}

\begin{styleEntryParagraph}
\textstyleDefinitionn{root}\textstyleLexeme{     sləlay}\textstylefstandard{.}
\end{styleEntryParagraph}

\begin{styleEntryParagraph}
\textstyleDefinitionn{rot}\textstyleLexeme{     hərzloy}\textstylefstandard{.}
\end{styleEntryParagraph}

\begin{styleEntryParagraph}
\textstyleDefinitionn{rot meat to flavour food}\textstyleLexeme{     gəvoy}\textstylefstandard{.}
\end{styleEntryParagraph}

\begin{styleEntryParagraph}
\textstyleDefinitionn{rub, wipe}\textstyleLexeme{     patay}\textstylefstandard{.}
\end{styleEntryParagraph}

\begin{styleEntryParagraph}
\textstyleDefinitionn{ruin}\textstyleLexeme{     mbəzen}\textstylefstandard{.}
\end{styleEntryParagraph}

\begin{styleEntryParagraph}
\textstyleDefinitionn{run}\textstyleLexeme{     həmay}\textstylefstandard{.}
\end{styleEntryParagraph}\end{multicols}
\begin{styleLetterParagraph}
\textstyleLetterv{S  {}-  s}
\end{styleLetterParagraph}

\begin{multicols}{2}
\begin{styleEntryParagraph}
\textstyleDefinitionn{sack; thousand francs}\textstyleLexeme{     ombolo}\textstyleDefinitionn{.}
\end{styleEntryParagraph}

\begin{styleEntryParagraph}
\textstyleDefinitionn{saliva}\textstyleLexeme{     eslesleɓ}\textstylefstandard{.}
\end{styleEntryParagraph}

\begin{styleEntryParagraph}
\textstyleDefinitionn{salt}\textstyleLexeme{     zetene}\textstylefstandard{.}
\end{styleEntryParagraph}

\begin{styleEntryParagraph}
\textstyleDefinitionn{satisfy, fill}\textstyleLexeme{     rah}\textstylefstandard{.}
\end{styleEntryParagraph}

\begin{styleEntryParagraph}
\textstyleDefinitionn{sauce made from edible leaves, leaf}\textstyleLexeme{   elele}\textstyleDefinitionn{.}
\end{styleEntryParagraph}

\begin{styleEntryParagraph}
\textstyleDefinitionn{sauce made of bean leaves}\textstyleLexeme{     azəɓat}\textstylefstandard{.}
\end{styleEntryParagraph}

\begin{styleEntryParagraph}
\textstyleDefinitionn{save}\textstyleLexeme{     tam}\textstylefstandard{.}
\end{styleEntryParagraph}

\begin{styleEntryParagraph}
\textstyleDefinitionn{save, economize}\textstyleLexeme{     johoy}\textstylefstandard{.}
\end{styleEntryParagraph}

\begin{styleEntryParagraph}
\textstyleDefinitionn{saying}\textstyleLexeme{     awəy}\textstylefstandard{.}
\end{styleEntryParagraph}

\begin{styleEntryParagraph}
\textstyleDefinitionn{scarify}\textstyleLexeme{     cahay}\textstylefstandard{.}
\end{styleEntryParagraph}

\begin{styleEntryParagraph}
\textstyleDefinitionn{scatter}\textstyleLexeme{     poloy}\textstylefstandard{.}
\end{styleEntryParagraph}

\begin{styleEntryParagraph}
\textstyleDefinitionn{school}\textstyleLexeme{     lekwel}\textstylefstandard{.}
\end{styleEntryParagraph}

\begin{styleEntryParagraph}
\textstyleDefinitionn{scold, argue}\textstyleLexeme{     mbe}\textstylefstandard{.}
\end{styleEntryParagraph}

\begin{styleEntryParagraph}
\textstyleDefinitionn{scold, reprimand}\textstyleLexeme{     ndahay}\textstylefstandard{.}
\end{styleEntryParagraph}

\begin{styleEntryParagraph}
\textstyleDefinitionn{scoop}\textstyleLexeme{     kətefer}\textstylefstandard{.}
\end{styleEntryParagraph}

\begin{styleEntryParagraph}
\textstyleDefinitionn{scorpion}\textstyleLexeme{     harac}\textstylefstandard{.}
\end{styleEntryParagraph}

\begin{styleEntryParagraph}
\textstyleDefinitionn{scrape}\textstyleLexeme{     kərɗaway}\textstylefstandard{.}
\end{styleEntryParagraph}

\begin{styleEntryParagraph}
\textstyleDefinitionn{scratch}\textstyleLexeme{     far}\textstylefstandard{.}
\end{styleEntryParagraph}

\begin{styleEntryParagraph}
\textstyleDefinitionn{sea}\textstyleLexeme{     bəlay}\textstylePartofspeech{.}
\end{styleEntryParagraph}

\begin{styleEntryParagraph}
\textstyleSensenumber{\textit{see}}\textstyleLexeme{     mənjar}\textstylefstandard{.}
\end{styleEntryParagraph}

\begin{styleEntryParagraph}
\textstyleDefinitionn{seeds}\textstyleLexeme{     həlfe}\textstylefstandard{.}
\end{styleEntryParagraph}

\begin{styleEntryParagraph}
\textstyleDefinitionn{seer}\textstyleLexeme{     kəlen}\textstylefstandard{.}
\end{styleEntryParagraph}

\begin{styleEntryParagraph}
\textstyleDefinitionn{seize, trap}\textstyleLexeme{     kəcaway}\textstylefstandard{.}
\end{styleEntryParagraph}

\begin{styleEntryParagraph}
\textstyleDefinitionn{self, heart}\textstyleLexeme{     ɓərav}\textstyleDefinitionn{.}
\end{styleEntryParagraph}

\begin{styleEntryParagraph}
\textstyleDefinitionn{sell/buy}\textstyleLexeme{     səkom}\textstylePartofspeech{.} 
\end{styleEntryParagraph}

\begin{styleEntryParagraph}
\textstyleDefinitionn{send}\textstyleLexeme{     slar}\textstylefstandard{.}
\end{styleEntryParagraph}

\begin{styleEntryParagraph}
\textstyleDefinitionn{separate, comb}\textstyleLexeme{ njaray}\textstylefstandard{.}
\end{styleEntryParagraph}

\begin{styleEntryParagraph}
\textstyleDefinitionn{sesame seeds/plant}\textstyleLexeme{     agaban}\textstylefstandard{.}
\end{styleEntryParagraph}

\begin{styleEntryParagraph}
\textstyleDefinitionn{set, work with wood or grasses}\textstyleLexeme{     ngay}\textstylefstandard{.}
\end{styleEntryParagraph}

\begin{styleEntryParagraph}
\textstyleDefinitionn{set down, put}\textstyleLexeme{     faɗ}\textstylefstandard{.}
\end{styleEntryParagraph}

\begin{styleEntryParagraph}
\textstyleDefinitionn{seven}\textstyleLexeme{     səsəre}\textstylefstandard{.}
\end{styleEntryParagraph}

\begin{styleEntryParagraph}
\textstyleDefinitionn{sew}\textstyleLexeme{     ɓah}\textstylefstandard{.}
\end{styleEntryParagraph}

\begin{styleEntryParagraph}
\textstyleDefinitionn{shadow, spirit}\textstyleLexeme{     sənewk}\textstylefstandard{.}
\end{styleEntryParagraph}

\begin{styleEntryParagraph}
\textstyleDefinitionn{shake}\textstyleLexeme{     wazay}\textstylefstandard{.}
\end{styleEntryParagraph}

\begin{styleEntryParagraph}
\textstyleDefinitionn{shake out stones}\textstyleLexeme{     təmbalay}\textstylefstandard{.}
\end{styleEntryParagraph}

\begin{styleEntryParagraph}
\textstyleDefinitionn{shakers}\textstyleLexeme{     kweɗe kweɗe}\textstylefstandard{.}
\end{styleEntryParagraph}

\begin{styleEntryParagraph}
\textstyleDefinitionn{shame}\textstyleLexeme{   məray}\textstylefstandard{.}
\end{styleEntryParagraph}

\begin{styleEntryParagraph}
\textstyleDefinitionn{share, divide}\textstyleLexeme{    wəɗakay}\textstylefstandard{.} 
\end{styleEntryParagraph}

\begin{styleEntryParagraph}
\textstyleDefinitionn{sharpen to a point}\textstyleLexeme{     fətaɗay}\textstylefstandard{.}
\end{styleEntryParagraph}

\begin{styleEntryParagraph}
\textstyleDefinitionn{sheep}\textstyleLexeme{     təmak}\textstylefstandard{.}
\end{styleEntryParagraph}

\begin{styleEntryParagraph}
\textstyleDefinitionn{shell}\textstyleLexeme{     pəlɗay}\textstylefstandard{.}
\end{styleEntryParagraph}

\begin{styleEntryParagraph}
\textstyleDefinitionn{shepherd; stake}\textstyleLexeme{     jəgor}\textstylefstandard{.}
\end{styleEntryParagraph}

\begin{styleEntryParagraph}
\textstyleDefinitionn{shepherd}\textstyleLexeme{     jəgor}\textstylefstandard{.}
\end{styleEntryParagraph}

\begin{styleEntryParagraph}
\textstyleDefinitionn{shine}\textstyleLexeme{     wazlay}\textstyleDefinitionn{.}
\end{styleEntryParagraph}

\begin{styleEntryParagraph}
\textstyleDefinitionn{shine}\textstyleLexeme{     wəcaɗay}\textstylefstandard{.}
\end{styleEntryParagraph}

\begin{styleEntryParagraph}
\textstyleDefinitionn{shoes}\textstyleLexeme{     tətərak}\textstylefstandard{.}
\end{styleEntryParagraph}

\begin{styleEntryParagraph}
\textstyleDefinitionn{shoot an arrow}\textstyleLexeme{     ɓar}\textstylefstandard{.}
\end{styleEntryParagraph}

\begin{styleEntryParagraph}
\textstyleDefinitionn{sibling}\textstyleLexeme{     məlama}\textstyleDefinitionn{.}
\end{styleEntryParagraph}

\begin{styleEntryParagraph}
\textit{sibling, spouse’s}\textstyleLexeme{    adamay}\textstyleExamplefreetransn{.}
\end{styleEntryParagraph}

\begin{styleEntryParagraph}
\textstyleDefinitionn{sickle}\textstyleLexeme{     mavaɗ}\textstylefstandard{.}
\end{styleEntryParagraph}

\begin{styleEntryParagraph}
\textstyleDefinitionn{sift}\textstyleLexeme{     sakay}\textstylefstandard{.}
\end{styleEntryParagraph}

\begin{styleEntryParagraph}
\textstyleDefinitionn{silence}\textstyleLexeme{     goloy}\textstylefstandard{.}
\end{styleEntryParagraph}

\begin{styleEntryParagraph}
\textstyleDefinitionn{simmer}\textstyleLexeme{     ngwəɗaslay}\textstylefstandard{.}
\end{styleEntryParagraph}

\begin{styleEntryParagraph}
\textstyleDefinitionn{sit, suffice}\textstyleLexeme{     nje}\textstylefstandard{.}
\end{styleEntryParagraph}

\begin{styleEntryParagraph}
\textstyleDefinitionn{six}\textstyleLexeme{     məko}\textstylefstandard{.}
\end{styleEntryParagraph}

\begin{styleEntryParagraph}
\textstyleDefinitionn{skewer    }\textstyleLexeme{caɓay}\textstylefstandard{.}
\end{styleEntryParagraph}

\begin{styleEntryParagraph}
\textstyleDefinitionn{skin}\textstyleLexeme{     hambar}\textstylefstandard{.}
\end{styleEntryParagraph}

\begin{styleEntryParagraph}
\textstyleDefinitionn{skin, peel}\textstyleLexeme{     mbəlɗoy}\textstylefstandard{.}
\end{styleEntryParagraph}

\begin{styleEntryParagraph}
\textstyleDefinitionn{sky, god}\textstyleLexeme{     hərmbəlom}\textstylefstandard{.}
\end{styleEntryParagraph}

\begin{styleEntryParagraph}
\textstyleDefinitionn{slander}\textstyleLexeme{     sahay}\textstylefstandard{.}
\end{styleEntryParagraph}

\begin{styleEntryParagraph}
\textstyleDefinitionn{slave}\textstyleLexeme{     beke}\textstylefstandard{.}
\end{styleEntryParagraph}

\begin{styleEntryParagraph}
\textstyleDefinitionn{slay}\textstyleLexeme{     slay}\textstylefstandard{.   }
\end{styleEntryParagraph}

\begin{styleEntryParagraph}
\textstyleDefinitionn{sleep}\textstyleLexeme{     ɗəwer}\textstylefstandard{.}
\end{styleEntryParagraph}

\begin{styleEntryParagraph}
\textstyleDefinitionn{slide}\textstyleLexeme{     slaray}\textstylefstandard{.}
\end{styleEntryParagraph}

\begin{styleEntryParagraph}
\textstyleDefinitionn{slide}\textstyleLexeme{     soroy}\textstylefstandard{.}
\end{styleEntryParagraph}

\begin{styleEntryParagraph}
\textstyleDefinitionn{slurp, sniff}\textstyleLexeme{     gorcoy}\textstylefstandard{.}
\end{styleEntryParagraph}

\begin{styleEntryParagraph}
\textstyleDefinitionn{small amount}\textstyleLexeme{     nekwen}\textstylefstandard{.}
\end{styleEntryParagraph}

\begin{styleEntryParagraph}
\textstyleDefinitionn{smallness}\textstyleLexeme{     cəɗew, hwəsese}\textstylefstandard{.}
\end{styleEntryParagraph}

\begin{styleEntryParagraph}
\textit{smell}\textstyleLexeme{     ze}\textstyleSensenumber{.}
\end{styleEntryParagraph}

\begin{styleEntryParagraph}
\textstyleDefinitionn{smile}\textstyleLexeme{     mbasay}\textstylefstandard{.}
\end{styleEntryParagraph}

\begin{styleEntryParagraph}
\textstyleDefinitionn{smoke}\textstyleLexeme{     azak}\textstylefstandard{.}
\end{styleEntryParagraph}

\begin{styleEntryParagraph}
\textstyleDefinitionn{smooth}\textstyleLexeme{     caɗay}\textstylePartofspeech{.} 
\end{styleEntryParagraph}

\begin{styleEntryParagraph}
\textstyleDefinitionn{smoothness}\textstyleLexeme{     kwəleɗeɗe}\textstyleDefinitionn{.}
\end{styleEntryParagraph}

\begin{styleEntryParagraph}
\textstyleDefinitionn{snake}\textstyleLexeme{   }\textstyleLexeme{enen,}\textstyleLexeme{ gogolvon,  mahaw}\textstyleDefinitionn{.}
\end{styleEntryParagraph}

\begin{styleEntryParagraph}
\textstyleDefinitionn{snap}\textstyleLexeme{     kəɓəcay}\textstylefstandard{.}
\end{styleEntryParagraph}

\begin{styleEntryParagraph}
\textstyleDefinitionn{sniff, slurp}\textstyleLexeme{     gorcoy}\textstylefstandard{.}
\end{styleEntryParagraph}

\begin{styleEntryParagraph}
\textstyleDefinitionn{snore, plant}\textstyleLexeme{     daray}\textstylefstandard{.}
\end{styleEntryParagraph}

\begin{styleEntryParagraph}
\textstyleDefinitionn{so and so}\textstyleLexeme{     mana}\textstylefstandard{.}
\end{styleEntryParagraph}

\begin{styleEntryParagraph}
\textstyleDefinitionn{soak, knead}\textstyleLexeme{ bolay}\textstylefstandard{.}
\end{styleEntryParagraph}

\begin{styleEntryParagraph}
\textstyleDefinitionn{soak in order to soften, flourish}\textstyleLexeme{     ɗe}\textstylefstandard{.}
\end{styleEntryParagraph}

\begin{styleEntryParagraph}
\textstyleDefinitionn{soccer ball/soccer  }\textstyleLexeme{ balon}\textstylefstandard{.}
\end{styleEntryParagraph}

\begin{styleEntryParagraph}
\textit{son}\textstyleLexeme{     gwəla}\textit{.}
\end{styleEntryParagraph}

\begin{styleEntryParagraph}
\textstyleDefinitionn{son, firstborn}\textstyleLexeme{     gəlo}\textstylefstandard{.} 
\end{styleEntryParagraph}

\begin{styleEntryParagraph}
\textstyleDefinitionn{song}\textstyleLexeme{     ləmes}\textstylefstandard{.}
\end{styleEntryParagraph}

\begin{styleEntryParagraph}
\textstyleDefinitionn{sorcery}\textstyleLexeme{     madan}\textstylefstandard{.}
\end{styleEntryParagraph}

\begin{styleEntryParagraph}
\textstyleDefinitionn{sore, cut}\textstyleLexeme{    ambəlak}\textstylefstandard{.}
\end{styleEntryParagraph}

\begin{styleEntryParagraph}
\textit{sorghum}, \textstyleDefinitionn{dry season millet}\textstyleLexeme{     omsoko}\textstylefstandard{.}
\end{styleEntryParagraph}

\begin{styleEntryParagraph}
\textstyleDefinitionn{sow, throw}\textstyleLexeme{     gocoy}\textstylefstandard{.}
\end{styleEntryParagraph}

\begin{styleEntryParagraph}
\textstyleDefinitionn{sparrow}\textstyleLexeme{     angwərzla}\textstylefstandard{.}
\end{styleEntryParagraph}

\begin{styleEntryParagraph}
\textstyleDefinitionn{speak}\textstyleLexeme{     jay}\textstylefstandard{.}
\end{styleEntryParagraph}

\begin{styleEntryParagraph}
\textstyleDefinitionn{speak badly of someone for one’s own interest}\textstyleLexeme{ pahay}\textstylefstandard{.}
\end{styleEntryParagraph}

\begin{styleEntryParagraph}
\textstyleDefinitionn{spear}\textstyleLexeme{     ezlere}\textstyleDefinitionn{.}
\end{styleEntryParagraph}

\begin{styleEntryParagraph}
\textstyleDefinitionn{spend time}\textstyleLexeme{     ve}\textstylefstandard{.}
\end{styleEntryParagraph}

\begin{styleEntryParagraph}
\textstyleDefinitionn{spider}\textstyleLexeme{     mazlərpapan}\textstylefstandard{.}
\end{styleEntryParagraph}

\begin{styleEntryParagraph}
\textstyleDefinitionn{spin}\textstyleLexeme{     yamay}\textstylefstandard{.}
\end{styleEntryParagraph}

\begin{styleEntryParagraph}
\textstyleSensenumber{\textit{spirit being}}\textstyleLexeme{     səkar}\textstyleSensenumber{\textit{.}}
\end{styleEntryParagraph}

\begin{styleEntryParagraph}
\textstyleDefinitionn{spirit, bad    }\textstyleLexeme{ wərge}\textstylefstandard{.}
\end{styleEntryParagraph}

\begin{styleEntryParagraph}
\textstyleDefinitionn{spirit, idol}\textstyleLexeme{     pəra}\textstylefstandard{.}
\end{styleEntryParagraph}

\begin{styleEntryParagraph}
\textstyleDefinitionn{spirit, shadow}\textstyleLexeme{     sənewk}\textstylefstandard{.}
\end{styleEntryParagraph}

\begin{styleEntryParagraph}
\textstyleDefinitionn{spit}\textstyleLexeme{     taf}\textstylefstandard{.}
\end{styleEntryParagraph}

\begin{styleEntryParagraph}
\textstyleDefinitionn{split in half}\textstyleLexeme{     pəlslay}\textstylefstandard{.}
\end{styleEntryParagraph}

\begin{styleEntryParagraph}
\textstyleDefinitionn{spray}\textstyleLexeme{     pəray}\textstylefstandard{.}
\end{styleEntryParagraph}

\begin{styleEntryParagraph}
\textstyleDefinitionn{spread for building}\textstyleLexeme{     ɗazl}\textstylefstandard{.}
\end{styleEntryParagraph}

\begin{styleEntryParagraph}
\textstyleDefinitionn{spread out}\textstyleLexeme{     waɗay}\textstylefstandard{.}
\end{styleEntryParagraph}

\begin{styleEntryParagraph}
\textstyleDefinitionn{spread out, detach}\textstyleLexeme{     pasay}\textstylefstandard{.}
\end{styleEntryParagraph}

\begin{styleEntryParagraph}
\textstyleDefinitionn{squash, large}\textstyleLexeme{     layaw}\textstylefstandard{.}
\end{styleEntryParagraph}

\begin{styleEntryParagraph}
\textstyleDefinitionn{squat, crouch}\textstyleLexeme{     cəɗokay}\textstylefstandard{.}
\end{styleEntryParagraph}

\begin{styleEntryParagraph}
\textstyleDefinitionn{squeeze out}\textstyleLexeme{     zlokoy}\textstylefstandard{.}
\end{styleEntryParagraph}

\begin{styleEntryParagraph}
\textstyleDefinitionn{squeeze, collect}\textstyleLexeme{     bərkaday}\textstylefstandard{.}
\end{styleEntryParagraph}

\begin{styleEntryParagraph}
\textstyleDefinitionn{squeeze, juice}\textstyleLexeme{     ɗocay}\textstylefstandard{.}
\end{styleEntryParagraph}

\begin{styleEntryParagraph}
\textstyleDefinitionn{squirrel}\textstyleLexeme{     ayah}\textstylefstandard{.}
\end{styleEntryParagraph}

\begin{styleEntryParagraph}
\textstyleDefinitionn{stable}\textstyleLexeme{     jəgəlen}\textstylefstandard{.}
\end{styleEntryParagraph}

\begin{styleEntryParagraph}
\textstyleDefinitionn{stake, shepherd}\textstyleLexeme{     jəgor}\textstylefstandard{.}
\end{styleEntryParagraph}

\begin{styleEntryParagraph}
\textstyleDefinitionn{stalk}\textstyleLexeme{     ɗəgocoy}\textstylefstandard{.}\textstyleExamplev{ }
\end{styleEntryParagraph}

\begin{styleEntryParagraph}
\textstyleDefinitionn{stand}\textstyleLexeme{     cəke}\textstylefstandard{.}
\end{styleEntryParagraph}

\begin{styleEntryParagraph}
\textstyleDefinitionn{star}\textstyleLexeme{     wərzla}\textstylefstandard{.}
\end{styleEntryParagraph}

\begin{styleEntryParagraph}
\textit{star, large and bright; planet }\textstyleDefinitionn{Venus}\textstyleLexeme{    abangay}\textstylefstandard{.}
\end{styleEntryParagraph}

\begin{styleIndentedParagraph}
\textstyleDefinitionn{star of the morning}\textstyleSubentry{     abangay dedew}\textstylefstandard{.}
\end{styleIndentedParagraph}

\begin{styleIndentedParagraph}
\textstyleDefinitionn{star of the night}\textstyleSubentry{     abangay aləho}\textstylefstandard{.} 
\end{styleIndentedParagraph}

\begin{styleEntryParagraph}
\textstyleDefinitionn{start}\textstyleLexeme{     zlan}\textstylefstandard{.}
\end{styleEntryParagraph}

\begin{styleEntryParagraph}
\textstyleDefinitionn{stay, leave}\textstyleLexeme{     ndəray}\textstylefstandard{.}
\end{styleEntryParagraph}

\begin{styleEntryParagraph}
\textstyleDefinitionn{steal}\textstyleLexeme{     karay}\textstylefstandard{.}
\end{styleEntryParagraph}

\begin{styleEntryParagraph}
\textstyleDefinitionn{sterilize, castrate}\textstyleLexeme{     daslay}\textstylefstandard{.}
\end{styleEntryParagraph}

\begin{styleEntryParagraph}
\textit{stick (noun)    }\textstyleLexeme{adangay}\textstyleExamplefreetransn{.}
\end{styleEntryParagraph}

\begin{styleEntryParagraph}
\textstyleDefinitionn{stick (verb)}\textstyleLexeme{     tapay}\textstylefstandard{.}
\end{styleEntryParagraph}

\begin{styleEntryParagraph}
\textstyleDefinitionn{stir}\textstyleLexeme{     ɓal}\textstylefstandard{.}
\end{styleEntryParagraph}

\begin{styleEntryParagraph}
\textstyleDefinitionn{stomach}\textstyleLexeme{     hoɗ}\textstylefstandard{.}
\end{styleEntryParagraph}

\begin{styleEntryParagraph}
\textstyleDefinitionn{story}\textstyleLexeme{     bamba}\textstylefstandard{.}
\end{styleEntryParagraph}

\begin{styleEntryParagraph}
\textstyleDefinitionn{stranger, traveler}\textstyleLexeme{     merkwe}\textstylefstandard{.}
\end{styleEntryParagraph}

\begin{styleEntryParagraph}
\textstyleDefinitionn{strength}\textstyleLexeme{     gədan}\textstylefstandard{.}
\end{styleEntryParagraph}

\begin{styleEntryParagraph}
\textstyleDefinitionn{stretch}\textstyleLexeme{     ndərdoy}\textstylefstandard{.}
\end{styleEntryParagraph}

\begin{styleEntryParagraph}
\textstyleDefinitionn{strip leaves from stalk}\textstyleLexeme{     goroy}\textstylefstandard{.}
\end{styleEntryParagraph}

\begin{styleEntryParagraph}
\textstyleDefinitionn{succeed}\textstyleLexeme{     damay}\textstylefstandard{.}
\end{styleEntryParagraph}

\begin{styleEntryParagraph}
\textstyleDefinitionn{suck}\textstyleLexeme{     soɓoy}\textstylefstandard{.}
\end{styleEntryParagraph}

\begin{styleEntryParagraph}
\textstyleDefinitionn{suddenly}\textstyleLexeme{     jəwk jəwk}\textstylefstandard{.}
\end{styleEntryParagraph}

\begin{styleEntryParagraph}
\textstyleDefinitionn{suffer pain}\textstyleLexeme{     zlakay}\textstylefstandard{.}
\end{styleEntryParagraph}

\begin{styleEntryParagraph}
\textstyleDefinitionn{suffering}\textstyleLexeme{     avəya}\textstylefstandard{.}
\end{styleEntryParagraph}

\begin{styleEntryParagraph}
\textstyleDefinitionn{suffice, sit}\textstyleLexeme{     nje}\textstylefstandard{.}
\end{styleEntryParagraph}

\begin{styleEntryParagraph}
\textstyleDefinitionn{sugar cane}\textstyleLexeme{     omboɗoc, reke}\textstylefstandard{.}
\end{styleEntryParagraph}

\begin{styleEntryParagraph}
\textstyleDefinitionn{sun, daytime}\textstyleLexeme{     fat}\textstylefstandard{.}
\end{styleEntryParagraph}

\begin{styleEntryParagraph}
\textstyleSensenumber{\textit{Sunday market}}\textstyleLexeme{     Zlaba}\textstylefstandard{.}
\end{styleEntryParagraph}

\begin{styleEntryParagraph}
\textstyleDefinitionn{surpass}\textstyleLexeme{     dal}\textstylefstandard{.}
\end{styleEntryParagraph}

\begin{styleEntryParagraph}
\textstyleDefinitionn{swallow}\textstyleLexeme{     ndaway}\textstylefstandard{.}
\end{styleEntryParagraph}

\begin{styleEntryParagraph}
\textstyleDefinitionn{swear}\textstyleLexeme{     mbaɗay}\textstylefstandard{.} 
\end{styleEntryParagraph}

\begin{styleEntryParagraph}
\textstyleDefinitionn{sweep}\textstyleLexeme{     kərsoy}\textstylefstandard{.}
\end{styleEntryParagraph}

\begin{styleEntryParagraph}
\textstyleDefinitionn{swell}\textstyleLexeme{     hasl}\textstylefstandard{.}
\end{styleEntryParagraph}

\begin{styleEntryParagraph}
\textstyleDefinitionn{swim}\textstyleLexeme{     zlavay}\textstylefstandard{.}
\end{styleEntryParagraph}

\begin{styleEntryParagraph}
\textstyleDefinitionn{sword}\textstyleLexeme{     maslalam}\textstylefstandard{.}
\end{styleEntryParagraph}

\begin{styleEntryParagraph}
\textstyleDefinitionn{sword, traditional}\textstyleLexeme{ nd}\textstyleLexeme{ə}\textstyleLexeme{n nden}\textstylefstandard{.}\textstyleDefinitionn{ }
\end{styleEntryParagraph}

\begin{styleEntryParagraph}
\textstyleDefinitionn{syphilis}\textstyleLexeme{     dolokoy}\textstylefstandard{.}
\end{styleEntryParagraph}\end{multicols}
\begin{styleLetterParagraph}
\textstyleLetterv{T  {}-  t}
\end{styleLetterParagraph}

\begin{multicols}{2}
\begin{styleEntryParagraph}
\textstyleDefinitionn{tail  }\textstyleLexeme{ hwəter}\textstylefstandard{.}
\end{styleEntryParagraph}

\begin{styleEntryParagraph}
\textstyleDefinitionn{take or steal by force}\textstyleLexeme{     gəjar}\textstylefstandard{.}
\end{styleEntryParagraph}

\begin{styleEntryParagraph}
\textstyleDefinitionn{take, carry}\textstyleLexeme{     zaɗ}\textstylePartofspeech{.}
\end{styleEntryParagraph}

\begin{styleEntryParagraph}
\textstyleDefinitionn{take courage}\textstyleExamplev{     angolay}\textstylefstandard{.}
\end{styleEntryParagraph}

\begin{styleEntryParagraph}
\textstyleDefinitionn{take leaves off stalk}\textstyleLexeme{     slohoy}\textstylefstandard{.}
\end{styleEntryParagraph}

\begin{styleEntryParagraph}
\textstyleDefinitionn{take many}\textstyleLexeme{     dəya}\textstylefstandard{.}
\end{styleEntryParagraph}

\begin{styleEntryParagraph}
\textstyleDefinitionn{take upon oneself}\textstyleLexeme{     waray}\textstylefstandard{.}
\end{styleEntryParagraph}

\begin{styleEntryParagraph}
\textstyleDefinitionn{talk with someone}\textstyleLexeme{     zlapay}\textstylefstandard{.}
\end{styleEntryParagraph}

\begin{styleEntryParagraph}
\textstyleDefinitionn{tamarind}\textstyleLexeme{     mawar}\textstylefstandard{.} 
\end{styleEntryParagraph}

\begin{styleEntryParagraph}
\textstyleDefinitionn{tan (treat animal skin)}\textstyleLexeme{     gwəzoy}\textstylefstandard{.} 
\end{styleEntryParagraph}

\begin{styleEntryParagraph}
\textstyleDefinitionn{tap}\textstyleLexeme{     tokoy}\textstylefstandard{.} 
\end{styleEntryParagraph}

\begin{styleEntryParagraph}
\textstyleDefinitionn{taste}\textstyleLexeme{     təkam}\textstylefstandard{.}
\end{styleEntryParagraph}

\begin{styleEntryParagraph}
\textstyleDefinitionn{taste good}\textstyleLexeme{     car}\textstylefstandard{.}
\end{styleEntryParagraph}

\begin{styleEntryParagraph}
\textstyleDefinitionn{teach, learn}\textstyleLexeme{     dəbənay}\textstylefstandard{.}
\end{styleEntryParagraph}

\begin{styleEntryParagraph}
\textstyleDefinitionn{tear away, break}\textstyleLexeme{    ngərway}\textstylefstandard{.}
\end{styleEntryParagraph}

\begin{styleEntryParagraph}
\textstyleDefinitionn{tear to pieces, first pounding}\textstyleLexeme{     ɓorcay}\textstylefstandard{.}
\end{styleEntryParagraph}

\begin{styleEntryParagraph}
\textstyleDefinitionn{tear up}\textstyleLexeme{     caray}\textstylefstandard{.}
\end{styleEntryParagraph}

\begin{styleEntryParagraph}
\textstyleDefinitionn{teeth, front}\textstyleLexeme{     maslar}\textstylefstandard{.}
\end{styleEntryParagraph}

\begin{styleEntryParagraph}
\textstyleDefinitionn{tempt, trick}\textstyleLexeme{     səɓatay}\textstylefstandard{.}
\end{styleEntryParagraph}

\begin{styleEntryParagraph}
\textstyleDefinitionn{temptation, trap}\textstyleLexeme{     azan}\textstylefstandard{.}
\end{styleEntryParagraph}

\begin{styleEntryParagraph}
\textstyleDefinitionn{ten}\textstyleLexeme{     kəro}\textstylefstandard{.}
\end{styleEntryParagraph}

\begin{styleEntryParagraph}
\textstyleDefinitionn{termite mound}\textstyleLexeme{     fenge}\textstylefstandard{.}
\end{styleEntryParagraph}

\begin{styleEntryParagraph}
\textstyleDefinitionn{termite species}\textstyleLexeme{     manjara, zlək zlak}\textstylefstandard{.}
\end{styleEntryParagraph}

\begin{styleEntryParagraph}
\textstyleDefinitionn{termites}\textstyleLexeme{     eleməzləɓe}\textstylefstandard{.}
\end{styleEntryParagraph}

\begin{styleEntryParagraph}
\textstyleDefinitionn{thanks}\textstyleLexeme{     səwse, wəse}\textstylefstandard{.}
\end{styleEntryParagraph}

\begin{styleEntryParagraph}
\textstyleDefinitionn{that is, because}\textstyleLexeme{     kəwaya}\textstylefstandard{.}
\end{styleEntryParagraph}

\begin{styleEntryParagraph}
\textstyleDefinitionn{theft}\textstyleLexeme{     akar}\textstylefstandard{.}
\end{styleEntryParagraph}

\begin{styleEntryParagraph}
\textstyleDefinitionn{then}\textstyleLexeme{     kəlen}\textstylefstandard{.}
\end{styleEntryParagraph}

\begin{styleEntryParagraph}
\textstylePartofspeech{there}\textstyleLexeme{     nendəye, nəngehe}\textstylePartofspeech{.}
\end{styleEntryParagraph}

\begin{styleEntryParagraph}
\textstyleDefinitionn{therefore}\textstyleLexeme{     nde}\textstylefstandard{.}
\end{styleEntryParagraph}

\begin{styleEntryParagraph}
\textstyleDefinitionn{thigh}\textstyleLexeme{     dəgolay}\textstylefstandard{.}
\end{styleEntryParagraph}

\begin{styleEntryParagraph}
\textstyleDefinitionn{thing}\textstyleLexeme{     ele}\textstylefstandard{.}
\end{styleEntryParagraph}

\begin{styleEntryParagraph}
\textstyleDefinitionn{think}\textstyleLexeme{     ɗəgalay}\textstylefstandard{.}
\end{styleEntryParagraph}

\begin{styleEntryParagraph}
\textstyleDefinitionn{this}\textstyleLexeme{     ndana}\textstylePartofspeech{.}
\end{styleEntryParagraph}

\begin{styleEntryParagraph}
\textstyleDefinitionn{this particular one here}\textstyleLexeme{     ngəhe}\textstylefstandard{.}
\end{styleEntryParagraph}

\begin{styleEntryParagraph}
\textstyleDefinitionn{thorn}\textstyleLexeme{     hadak}\textstylefstandard{.}
\end{styleEntryParagraph}

\begin{styleEntryParagraph}
\textstyleDefinitionn{thousand francs, sack}\textstyleLexeme{     ombolo}\textstyleDefinitionn{.}
\end{styleEntryParagraph}

\begin{styleEntryParagraph}
\textstyleDefinitionn{three}\textstyleLexeme{     makar}\textstylefstandard{.}
\end{styleEntryParagraph}

\begin{styleEntryParagraph}
\textstyleDefinitionn{threshing floor}\textstyleLexeme{     gəlan}\textstylefstandard{.}
\end{styleEntryParagraph}

\begin{styleEntryParagraph}
\textstyleDefinitionn{throat}\textstyleLexeme{     mbərlom}\textstylefstandard{.}
\end{styleEntryParagraph}

\begin{styleEntryParagraph}
\textstylePartofspeech{thousand}\textstyleLexeme{     dəbo}\textstylefstandard{.}
\end{styleEntryParagraph}

\begin{styleEntryParagraph}
\textstyleDefinitionn{throw, plant}\textstyleLexeme{     zləge}\textstyleDefinitionn{.}
\end{styleEntryParagraph}

\begin{styleEntryParagraph}
\textstyleDefinitionn{throw, sow}\textstyleLexeme{     gocoy}\textstylefstandard{.}
\end{styleEntryParagraph}

\begin{styleEntryParagraph}
\textstyleDefinitionn{throw a fit}\textstyleLexeme{     ɓərzlay}\textstylefstandard{.}
\end{styleEntryParagraph}

\begin{styleIndentedParagraph}
\textstyleDefinitionn{thumb}\textstyleSubentry{     baba ahar}\textstylefstandard{.}
\end{styleIndentedParagraph}

\begin{styleEntryParagraph}
\textstyleDefinitionn{Thursday, market day in the village of Doulek}\textstyleLexeme{  Dəwlek}\textstylefstandard{.}
\end{styleEntryParagraph}

\begin{styleEntryParagraph}
\textstyleDefinitionn{tie, join}\textstyleLexeme{     dazlay}\textstylefstandard{.}
\end{styleEntryParagraph}

\begin{styleEntryParagraph}
\textstyleDefinitionn{tie off}\textstyleLexeme{     tərɗay}\textstylefstandard{.}
\end{styleEntryParagraph}

\begin{styleEntryParagraph}
\textstyleDefinitionn{time}\textstyleLexeme{     ɗəma}\textstylefstandard{.}
\end{styleEntryParagraph}

\begin{styleEntryParagraph}
\textstyleDefinitionn{tire out    }\textstyleLexeme{ yaɗay}\textstylefstandard{.}
\end{styleEntryParagraph}

\begin{styleEntryParagraph}
\textstyleDefinitionn{to}\textstyleLexeme{     a, ana}\textstylefstandard{.}
\end{styleEntryParagraph}

\begin{styleEntryParagraph}
\textstyleDefinitionn{toad}\textstyleLexeme{     moktonok}\textstylefstandard{.}
\end{styleEntryParagraph}

\begin{styleEntryParagraph}
\textstyleDefinitionn{today}\textstyleLexeme{     egəne}\textstylefstandard{.}
\end{styleEntryParagraph}

\begin{styleEntryParagraph}
\textstyleDefinitionn{tolerate}\textstyleLexeme{     ɓasay}\textstylefstandard{.}
\end{styleEntryParagraph}

\begin{styleEntryParagraph}
\textstyleDefinitionn{tomorrow}\textstyleLexeme{     hajan}\textstylefstandard{.} 
\end{styleEntryParagraph}

\begin{styleEntryParagraph}
\textstyleDefinitionn{tongue}\textstyleLexeme{     hərnek}\textstylefstandard{.}
\end{styleEntryParagraph}

\begin{styleEntryParagraph}
\textstyleDefinitionn{tooth}\textstyleLexeme{     aslar}\textstylefstandard{.}
\end{styleEntryParagraph}

\begin{styleEntryParagraph}
\textstyleDefinitionn{toss and turn while sick}\textstyleLexeme{     ɓaray}\textstylefstandard{.} 
\end{styleEntryParagraph}

\begin{styleEntryParagraph}
\textstyleDefinitionn{touch}\textstyleLexeme{     lamay}\textstylefstandard{.}
\end{styleEntryParagraph}

\begin{styleEntryParagraph}
\textstyleDefinitionn{trace}\textstyleLexeme{     tohoy}\textstylefstandard{.}
\end{styleEntryParagraph}

\begin{styleEntryParagraph}
\textstyleDefinitionn{trap}\textstyleLexeme{     pərgom}\textstylefstandard{.}
\end{styleEntryParagraph}

\begin{styleEntryParagraph}
\textstyleDefinitionn{trap, seize}\textstyleLexeme{     kəcaway}\textstylefstandard{.}
\end{styleEntryParagraph}

\begin{styleEntryParagraph}
\textstyleDefinitionn{trap, temptation}\textstyleLexeme{    azan}\textstylefstandard{.}
\end{styleEntryParagraph}

\begin{styleEntryParagraph}
\textstyleDefinitionn{traveler, stranger}\textstyleLexeme{     merkwe}\textstylefstandard{.}
\end{styleEntryParagraph}

\begin{styleEntryParagraph}
\textstyleDefinitionn{treasure}\textstyleLexeme{     eləmene}\textstylefstandard{.}
\end{styleEntryParagraph}

\begin{styleEntryParagraph}
\textstyleDefinitionn{tree}\textstyleLexeme{     memele}\textstylefstandard{.}
\end{styleEntryParagraph}

\begin{styleEntryParagraph}
\textstyleDefinitionn{tree, thorny}\textstyleLexeme{ orov}\textstylefstandard{.}
\end{styleEntryParagraph}

\begin{styleEntryParagraph}
\textstyleDefinitionn{tree species}\textstyleLexeme{     eɗongwereɗ, ngəvəray , zəgogom}\textstylefstandard{.}
\end{styleEntryParagraph}

\begin{styleEntryParagraph}
\textstyleDefinitionn{tribe}\textstyleLexeme{     jəbe}\textstylefstandard{.}
\end{styleEntryParagraph}

\begin{styleEntryParagraph}
\textstyleDefinitionn{trick}\textstyleLexeme{     cəɗoy}\textstylefstandard{.}
\end{styleEntryParagraph}

\begin{styleEntryParagraph}
\textstyleDefinitionn{trick, tempt}\textstyleLexeme{     səɓatay}\textstylefstandard{.}
\end{styleEntryParagraph}

\begin{styleEntryParagraph}
\textstyleDefinitionn{truck}\textstyleLexeme{     məwta}\textstylefstandard{.}
\end{styleEntryParagraph}

\begin{styleEntryParagraph}
\textstyleDefinitionn{trumpet}\textstyleLexeme{     məzlelem}\textstylefstandard{.}
\end{styleEntryParagraph}

\begin{styleEntryParagraph}
\textstyleDefinitionn{truth}\textstyleLexeme{     ɗeɗen, jere}\textstylePartofspeech{.}
\end{styleEntryParagraph}

\begin{styleEntryParagraph}
\textstyleDefinitionn{try}\textstyleLexeme{     zokoy}\textstylefstandard{.}
\end{styleEntryParagraph}

\begin{styleEntryParagraph}
\textstyleDefinitionn{try, invite}\textstyleLexeme{     təkaray}\textstylefstandard{.}
\end{styleEntryParagraph}

\begin{styleEntryParagraph}
\textstyleDefinitionn{Tuesday market}\textstyleLexeme{     Tokombere}\textstylefstandard{.}
\end{styleEntryParagraph}

\begin{styleEntryParagraph}
\textstyleDefinitionn{turn off}\textstyleLexeme{     mbat}\textstylefstandard{.}
\end{styleEntryParagraph}

\begin{styleEntryParagraph}
\textstyleDefinitionn{turtle}\textstyleLexeme{     kərkayah}\textstylefstandard{.}
\end{styleEntryParagraph}

\begin{styleEntryParagraph}
\textstyleDefinitionn{twin}\textstyleLexeme{     molo}\textstylefstandard{.}
\end{styleEntryParagraph}

\begin{styleEntryParagraph}
\textstyleDefinitionn{twist}\textstyleLexeme{     təmbaɗay}\textstylefstandard{.}
\end{styleEntryParagraph}

\begin{styleEntryParagraph}
\textstyleDefinitionn{twist, hang}\textstyleLexeme{     vaway}\textstylefstandard{.}
\end{styleEntryParagraph}

\begin{styleEntryParagraph}
\textstyleDefinitionn{two}\textstyleLexeme{     cew}\textstylefstandard{.}
\end{styleEntryParagraph}
\end{multicols}
\begin{styleLetterParagraph}
\textstyleLetterv{U  {}-  u}
\end{styleLetterParagraph}

\begin{multicols}{2}
\begin{styleEntryParagraph}
\textstyleDefinitionn{uncle, maternal}\textstyleLexeme{     gəmsodo}\textstylefstandard{.}
\end{styleEntryParagraph}

\begin{styleEntryParagraph}
\textstyleDefinitionn{understand, hear}\textstyleLexeme{     cen}\textstylefstandard{.}
\end{styleEntryParagraph}

\begin{styleEntryParagraph}
\textstyleDefinitionn{undress}\textstyleLexeme{     cokoy}\textstylefstandard{.}
\end{styleEntryParagraph}

\begin{styleEntryParagraph}
\textstyleDefinitionn{undress, peel off}\textstyleLexeme{     kərtoy}\textstylefstandard{.}
\end{styleEntryParagraph}

\begin{styleEntryParagraph}
\textstyleDefinitionn{unite, assemble}\textstyleLexeme{     cəkalay}\textstylefstandard{.}
\end{styleEntryParagraph}

\begin{styleEntryParagraph}
\textstyleDefinitionn{untie}\textstyleLexeme{     mbərcay}\textstylefstandard{.}
\end{styleEntryParagraph}

\begin{styleEntryParagraph}
\textstyleDefinitionn{until}\textstyleLexeme{     ha}\textstylefstandard{.}
\end{styleEntryParagraph}

\begin{styleEntryParagraph}
\textstyleDefinitionn{uproot, bud}\textstyleLexeme{     tosoy}\textstylefstandard{.}
\end{styleEntryParagraph}

\begin{styleEntryParagraph}
\textit{u}\textstyleDefinitionn{proot a tree}\textstyleLexeme{     fakay}\textstylefstandard{.}
\end{styleEntryParagraph}
\end{multicols}
\begin{styleLetterParagraph}
\textstyleLetterv{V  {}-  v}
\end{styleLetterParagraph}

\begin{multicols}{2}
\begin{styleEntryParagraph}
\textstyleDefinitionn{vegetable sauce}\textstyleLexeme{     mosokoy}\textstylefstandard{.}
\end{styleEntryParagraph}

\begin{styleEntryParagraph}
\textstyleDefinitionn{Venus,}\textit{ large and bright star}\textstyleLexeme{    abangay}\textstylefstandard{.}
\end{styleEntryParagraph}

\begin{styleEntryParagraph}
\textstylefstandard{\textit{verb clitic}} \textstyleDefinitionn{3S direct object    }\textstyleLexeme{na}\textstylefstandard{.}
\end{styleEntryParagraph}

\begin{styleEntryParagraph}
\textstylePartofspeech{verb clitic}\textstyleDefinitionn{3P} \textit{direct object    }\textstyleLexeme{ta}.
\end{styleEntryParagraph}

\begin{styleEntryParagraph}
\textstylefstandard{\textit{v}}\textstylefstandard{\textit{erb}}\textstylefstandard{ }\textstylefstandard{\textit{clitic}}\textstylePartofspeech{ }\textstyleDefinitionn{1S indirect object}\textstyleLexeme{   aw}\textstylefstandard{.}
\end{styleEntryParagraph}

\begin{styleEntryParagraph}
\textstylefstandard{\textit{v}}\textstylefstandard{\textit{erb}}\textstylefstandard{ }\textstylefstandard{\textit{clitic}}\textstylePartofspeech{ }\textstyleDefinitionn{2S indirect object}\textstyleLexeme{   ok}\textstylefstandard{.}
\end{styleEntryParagraph}

\begin{styleEntryParagraph}
\textstylefstandard{\textit{v}}\textstylefstandard{\textit{erb}}\textstylefstandard{ }\textstylefstandard{\textit{clitic}}\textstylePartofspeech{ }\textstyleDefinitionn{3S indirect object}\textstyleLexeme{   an}\textstylefstandard{.}
\end{styleEntryParagraph}

\begin{styleEntryParagraph}
\textstylefstandard{\textit{verb }}\textstylePartofspeech{clitic1P}\textstylePartofspeech{EX}\textstyleDefinitionn{ indirect object}\textstyleLexeme{   aləme}\textstylefstandard{.}
\end{styleEntryParagraph}

\begin{styleEntryParagraph}
\textstylefstandard{\textit{verb }}\textstylePartofspeech{clitic1P}\textstylePartofspeech{IN}\textstyleDefinitionn{ indirect object}\textstyleLexeme{   aloko}\textstylefstandard{.}
\end{styleEntryParagraph}

\begin{styleEntryParagraph}
\textstylefstandard{\textit{v}}\textstylefstandard{\textit{erb}}\textstylefstandard{ }\textstylefstandard{\textit{clitic}} \textstyleDefinitionn{2P indirect object}\textstyleLexeme{   aləkwəye.}\textstylefstandard{\textit{   }}\textstylefstandard{.}
\end{styleEntryParagraph}

\begin{styleEntryParagraph}
\textstylefstandard{\textit{v}}\textstylefstandard{\textit{erb}}\textstylefstandard{ }\textstylefstandard{\textit{clitic}}\textstylePartofspeech{ }\textstyleDefinitionn{3P indirect object}\textstyleLexeme{   ata}\textstylefstandard{.}
\end{styleEntryParagraph}

\begin{styleEntryParagraph}
\textstylefstandard{\textit{v}}\textstylefstandard{\textit{erb}}\textstylefstandard{ }\textstylefstandard{\textit{clitic}} \textit{away}\textstyleLexeme{     alay.}\textstylefstandard{   }
\end{styleEntryParagraph}

\begin{styleEntryParagraph}
\textstylefstandard{\textit{v}}\textstylefstandard{\textit{erb}}\textstylefstandard{ }\textstylefstandard{\textit{clitic}}\textstylePartofspeech{ }\textstyleDefinitionn{in}\textstyleLexeme{     ava}\textstylefstandard{.}
\end{styleEntryParagraph}

\begin{styleEntryParagraph}
\textstylefstandard{\textit{v}}\textstylefstandard{\textit{erb}}\textstylefstandard{ }\textstylefstandard{\textit{clitic}} \textstyleDefinitionn{on (top of)    }\textstyleLexeme{aka}\textstylefstandard{.} 
\end{styleEntryParagraph}

\begin{styleEntryParagraph}
\textstylefstandard{\textit{v}}\textstylefstandard{\textit{erb}}\textstylefstandard{ }\textstylefstandard{\textit{clitic}} \textit{towards}\textstyleLexeme{     ala}\textstylefstandard{.}
\end{styleEntryParagraph}

\begin{styleEntryParagraph}
\textstylePartofspeech{verb clitic} \textstyleDefinitionn{Perfect    }\textstyleLexeme{va}\textstylefstandard{.}
\end{styleEntryParagraph}

\begin{styleEntryParagraph}
\textstylePartofspeech{verb prefix} \textstyleDefinitionn{1S/P subject}\textstyleLexeme{     n\nobreakdash-}\textstylefstandard{.}
\end{styleEntryParagraph}

\begin{styleEntryParagraph}
\textstylePartofspeech{verb prefix }\textstyleDefinitionn{2S/P subject    }\textstyleLexeme{k\nobreakdash-}\textstylefstandard{.}
\end{styleEntryParagraph}

\begin{styleEntryParagraph}
\textstylePartofspeech{verb prefix} \textstyleDefinitionn{3S}\textstylefstandard{ }\textstyleDefinitionn{subject}\textstyleLexeme{    a\nobreakdash-}\textstylefstandard{.}
\end{styleEntryParagraph}

\begin{styleEntryParagraph}
\textstylePartofspeech{verb prefix} \textstyleDefinitionn{3P  subject  }\textstyleLexeme{t\nobreakdash-}\textstylefstandard{.}
\end{styleEntryParagraph}

\begin{styleEntryParagraph}
\textstylePartofspeech{verb suffix} \textstyleDefinitionn{1P}\textstylePartofspeech{EX}\textstyleLexeme{ }\textstyleLexeme{\textmd{\textit{subject}}}\textstyleLexeme{    \nobreakdash-om}\textstylefstandard{.}
\end{styleEntryParagraph}

\begin{styleEntryParagraph}
\textstylefstandard{\textit{verb }}\textstylePartofspeech{suffix 1P}\textstylePartofspeech{IN}\textstyleDefinitionn{/2P subject}\textstyleLexeme{ -ok}\textstylefstandard{.}
\end{styleEntryParagraph}

\begin{styleEntryParagraph}
\textstyleDefinitionn{village}\textstyleLexeme{     slala}\textstylefstandard{.}
\end{styleEntryParagraph}

\begin{styleEntryParagraph}
\textstyleDefinitionn{viper}\textstyleLexeme{     mətəmbətəmbezl, }\textstyleLexeme{kwəcesl}\textstylefstandard{.}
\end{styleEntryParagraph}

\begin{styleEntryParagraph}
\textstyleSensenumber{\textit{voice, neck}}\textstyleLexeme{     dəngo}\textstyleSensenumber{\textit{.}}
\end{styleEntryParagraph}

\begin{styleEntryParagraph}
\textstyleDefinitionn{vomit}\textstyleLexeme{     vənahay}\textstylefstandard{.}
\end{styleEntryParagraph}

\begin{styleEntryParagraph}
\textstyleDefinitionn{vulture}\textstyleLexeme{     azlam, }\textstyleLexeme{molo}\textstylefstandard{.}
\end{styleEntryParagraph}\end{multicols}
\begin{styleLetterParagraph}
\textstyleLetterv{W  {}-  w }
\end{styleLetterParagraph}

\begin{multicols}{2}
\begin{styleEntryParagraph}
\textstyleDefinitionn{wait, watch over}\textstyleLexeme{     kasl}\textstylefstandard{.}
\end{styleEntryParagraph}

\begin{styleEntryParagraph}
\textstyleDefinitionn{wake up}\textstyleLexeme{     pəɗakay}\textstylefstandard{.}
\end{styleEntryParagraph}

\begin{styleEntryParagraph}
\textstyleDefinitionn{walk}\textstyleLexeme{     talay}\textstylefstandard{.}
\end{styleEntryParagraph}

\begin{styleEntryParagraph}
\textstyleDefinitionn{wall}\textstyleLexeme{     hədo}\textstylefstandard{.}
\end{styleEntryParagraph}

\begin{styleEntryParagraph}
\textstyleDefinitionn{want, cut}\textstyleLexeme{     say}\textstylefstandard{.}
\end{styleEntryParagraph}

\begin{styleEntryParagraph}
\textstyleDefinitionn{want, love}\textstyleLexeme{     ndaɗay}\textstylefstandard{.}
\end{styleEntryParagraph}

\begin{styleEntryParagraph}
\textstyleDefinitionn{wart}\textstyleLexeme{     səsayak}\textstylefstandard{.}
\end{styleEntryParagraph}

\begin{styleEntryParagraph}
\textstyleDefinitionn{wash}\textstyleLexeme{     balay}\textstylefstandard{.}
\end{styleEntryParagraph}

\begin{styleEntryParagraph}
\textstyleDefinitionn{wash clothes}\textstyleLexeme{     jorɓoy}\textstylefstandard{.} 
\end{styleEntryParagraph}

\begin{styleEntryParagraph}
\textstyleDefinitionn{waste}\textstyleLexeme{     wahay}\textstylefstandard{.}
\end{styleEntryParagraph}

\begin{styleEntryParagraph}
\textstyleDefinitionn{watch intently}\textstyleLexeme{     zərɗay}\textstylefstandard{.}
\end{styleEntryParagraph}

\begin{styleEntryParagraph}
\textstyleDefinitionn{watch over, wait}\textstyleLexeme{     kasl}\textstylefstandard{.}
\end{styleEntryParagraph}

\begin{styleEntryParagraph}
\textstyleDefinitionn{water}\textstyleLexeme{     yam}\textstylefstandard{.}
\end{styleEntryParagraph}

\begin{styleEntryParagraph}
\textstyleDefinitionn{weapon, bracelet}\textstyleLexeme{     alahar}\textstylefstandard{.}
\end{styleEntryParagraph}

\begin{styleEntryParagraph}
\textstyleDefinitionn{wear small leather article of clothing}\textstyleLexeme{   pocoy}\textstylefstandard{.}
\end{styleEntryParagraph}

\begin{styleEntryParagraph}
\textstyleDefinitionn{weave}\textstyleLexeme{     ndar}\textstylefstandard{.}
\end{styleEntryParagraph}

\begin{styleEntryParagraph}
\textstyleDefinitionn{weave, granulate}\textstyleLexeme{    gədəgar}\textstylefstandard{.}
\end{styleEntryParagraph}

\begin{styleEntryParagraph}
\textstyleDefinitionn{Wednesday market}\textstyleLexeme{     Patatah}\textstylefstandard{.}
\end{styleEntryParagraph}

\begin{styleEntryParagraph}
\textstyleDefinitionn{Westerner}\textstyleLexeme{     asara}\textstylefstandard{.}
\end{styleEntryParagraph}

\begin{styleEntryParagraph}
\textstyleDefinitionn{wet, whip}\textstyleLexeme{     ndaɓay}\textstyleDefinitionn{.}
\end{styleEntryParagraph}

\begin{styleEntryParagraph}
\textstylePartofspeech{what}\textstyleLexeme{     almay, malmay}\textstylefstandard{.}
\end{styleEntryParagraph}

\begin{styleEntryParagraph}
\textstyleDefinitionn{what (emphatic)    }\textstyleLexeme{may}\textstyleDefinitionn{.}
\end{styleEntryParagraph}

\begin{styleEntryParagraph}
\textstyleDefinitionn{what’s his/her name}\textstyleLexeme{     andakay}\textstylefstandard{.}
\end{styleEntryParagraph}

\begin{styleEntryParagraph}
\textstyleDefinitionn{when}\textstyleLexeme{     epeley}\textstylefstandard{.}
\end{styleEntryParagraph}

\begin{styleEntryParagraph}
\textstyleDefinitionn{where}\textstyleLexeme{     amtamay}\textstylefstandard{.}
\end{styleEntryParagraph}

\begin{styleEntryParagraph}
\textstyleDefinitionn{which}\textstyleLexeme{     weley}\textstyleDefinitionn{.}
\end{styleEntryParagraph}

\begin{styleEntryParagraph}
\textstyleDefinitionn{whip}\textstyleLexeme{     eyeweɗ}\textstylefstandard{.}
\end{styleEntryParagraph}

\begin{styleEntryParagraph}
\textstyleDefinitionn{whip, wet}\textstyleLexeme{     ndaɓay}\textstyleDefinitionn{.}
\end{styleEntryParagraph}

\begin{styleEntryParagraph}
\textstyleDefinitionn{whisper}\textstyleLexeme{     sokoy}\textstylefstandard{.}
\end{styleEntryParagraph}

\begin{styleEntryParagraph}
\textstyleDefinitionn{whistle}\textstyleLexeme{     fokoy}\textstylefstandard{.}
\end{styleEntryParagraph}

\begin{styleEntryParagraph}
\textstyleDefinitionn{who}\textstyleLexeme{     way}\textstylefstandard{.}
\end{styleEntryParagraph}

\begin{styleEntryParagraph}
\textstyleSensenumber{\textit{wholeness, peace}}\textstyleLexeme{     zay, zazay}\textstylefstandard{\textit{.}}
\end{styleEntryParagraph}

\begin{styleEntryParagraph}
\textstyleDefinitionn{why}\textstyleLexeme{     kamay}\textstylefstandard{.}
\end{styleEntryParagraph}

\begin{styleEntryParagraph}
\textstyleDefinitionn{wife, woman}\textstyleLexeme{     hor}\textstylefstandard{.}
\end{styleEntryParagraph}

\begin{styleEntryParagraph}
\textstyleDefinitionn{wind}\textstyleLexeme{     həmaɗ}\textstylefstandard{.}
\end{styleEntryParagraph}

\begin{styleEntryParagraph}
\textstyleDefinitionn{wind, roll}\textstyleLexeme{     təɗoy}\textstylefstandard{.}
\end{styleEntryParagraph}

\begin{styleEntryParagraph}
\textstyleDefinitionn{wings}\textstyleLexeme{     kərpasla}\textstylefstandard{.}
\end{styleEntryParagraph}

\begin{styleEntryParagraph}
\textstyleDefinitionn{winnow}\textstyleLexeme{     vay}\textstylefstandard{.}
\end{styleEntryParagraph}

\begin{styleEntryParagraph}
\textstyleDefinitionn{wipe, rub}\textstyleLexeme{     patay}\textstylefstandard{.}
\end{styleEntryParagraph}

\begin{styleEntryParagraph}
\textstyleDefinitionn{wipe out, cancel}\textstyleLexeme{     vasay}\textstylefstandard{.}
\end{styleEntryParagraph}

\begin{styleEntryParagraph}
\textstyleDefinitionn{wisdom, brain}\textstyleLexeme{     endeɓ}\textstylefstandard{.}
\end{styleEntryParagraph}

\begin{styleEntryParagraph}
\textstyleDefinitionn{with}\textstyleLexeme{     }\textstyleLexeme{nə}\textstylefstandard{.}
\end{styleEntryParagraph}

\begin{styleEntryParagraph}
\textstyleDefinitionn{withdraw, recoil}\textstyleLexeme{     dar}\textstylefstandard{.}
\end{styleEntryParagraph}

\begin{styleEntryParagraph}
\textstyleDefinitionn{without}\textstyleLexeme{     mənjaɗ}\textstylefstandard{.}
\end{styleEntryParagraph}

\begin{styleEntryParagraph}
\textstyleDefinitionn{without help}\textstyleLexeme{     sawan}\textstylefstandard{.}
\end{styleEntryParagraph}

\begin{styleEntryParagraph}
\textstyleDefinitionn{witness}\textstyleLexeme{     sede}\textstylefstandard{.}
\end{styleEntryParagraph}

\begin{styleEntryParagraph}
\textstyleDefinitionn{woman, wife}\textstyleLexeme{     hor}\textstylefstandard{.}
\end{styleEntryParagraph}

\begin{styleEntryParagraph}
\textstyleLexeme{\textmd{\textit{women    }}}\textstyleLexeme{hawər ahay}\textstyleLexeme{\textmd{. }}
\end{styleEntryParagraph}

\begin{styleEntryParagraph}
\textstyleDefinitionn{wood}\textstyleLexeme{     oloko}\textstylefstandard{.}
\end{styleEntryParagraph}

\begin{styleEntryParagraph}
\textstyleDefinitionn{work}\textstyleLexeme{ slərele}\textstylefstandard{.}
\end{styleEntryParagraph}

\begin{styleEntryParagraph}
\textstyleDefinitionn{work with wood or grasses, set}\textstyleLexeme{     ngay}\textstylefstandard{.}
\end{styleEntryParagraph}

\begin{styleEntryParagraph}
\textstyleDefinitionn{worm, larva}\textstyleLexeme{     mecekweɗ}\textstylefstandard{.}
\end{styleEntryParagraph}

\begin{styleEntryParagraph}
\textstyleDefinitionn{wrap}\textstyleLexeme{     kəmbohoy}\textstylefstandard{.}
\end{styleEntryParagraph}

\begin{styleEntryParagraph}
\textstyleDefinitionn{wrinkle the skin}\textstyleLexeme{     ngərɗasay}\textstylefstandard{.}
\end{styleEntryParagraph}

\begin{styleEntryParagraph}
\textstyleDefinitionn{write}\textstyleLexeme{     wacay}\textstylefstandard{.}
\end{styleEntryParagraph}
\end{multicols}
\begin{styleLetterParagraph}
\textstyleLetterv{Y  {}-  y}
\end{styleLetterParagraph}

\begin{multicols}{2}
\begin{styleEntryParagraph}
\textstyleDefinitionn{yam}\textstyleLexeme{     oɓolo}\textstylefstandard{.}
\end{styleEntryParagraph}

\begin{styleEntryParagraph}
\textstyleDefinitionn{yard}\textstyleLexeme{     gala}\textstylefstandard{.}
\end{styleEntryParagraph}

\begin{styleEntryParagraph}
\textstyleSensenumber{\textit{year}}\textstyleLexeme{     məvəye}\textstylefstandard{.}
\end{styleEntryParagraph}

\begin{styleEntryParagraph}
\textstyleDefinitionn{yes}\textstyleLexeme{     ayaw}\textstylefstandard{.}
\end{styleEntryParagraph}

\begin{styleEntryParagraph}
\textstyleDefinitionn{yesterday}\textstyleLexeme{     apazan}\textstylefstandard{.}
\end{styleEntryParagraph}
\end{multicols}
\begin{multicols}{2}
\end{multicols}

\setcounter{page}{1}\chapter[Index]{Index}
\hypertarget{RefHeading1213701525720847}{}\begin{multicols}{2}
\begin{styleindexi}
Attribution
\end{styleindexi}

\begin{styleindexii}
Comparative constructions  125
\end{styleindexii}

\begin{styleindexii}
Derived adjective  108, 182
\end{styleindexii}

\begin{styleindexii}
Expressed using verb  200, 208
\end{styleindexii}

\begin{styleindexii}
Ideophones  87
\end{styleindexii}

\begin{styleindexii}
Permanent atttibution  116
\end{styleindexii}

\begin{styleindexi}
Buwal language  198
\end{styleindexi}

\begin{styleindexi}
Clitics
\end{styleindexi}

\begin{styleindexii}
Adpositionals  171
\end{styleindexii}

\begin{styleindexii}
Plural  99
\end{styleindexii}

\begin{styleindexii}
Possessive pronoun  58
\end{styleindexii}

\begin{styleindexii}
Verb clitics  146
\end{styleindexii}

\begin{styleindexi}
Cohesion
\end{styleindexi}

\begin{styleindexii}
Anaphoric referencing  56, 67, 72
\end{styleindexii}

\begin{styleindexii}
Na-marking  246, 247
\end{styleindexii}

\begin{styleindexii}
Participant tracking  149, 157, 254
\end{styleindexii}

\begin{styleindexii}
Point of reference  175
\end{styleindexii}

\begin{styleindexii}
Tail-head linking  250
\end{styleindexii}

\begin{styleindexi}
Cuvok language  49, 198
\end{styleindexi}

\begin{styleindexi}
Deixis
\end{styleindexi}

\begin{styleindexii}
Definiteness  256
\end{styleindexii}

\begin{styleindexii}
Demonstrative function of \textit{ga}  111
\end{styleindexii}

\begin{styleindexii}
Demonstratives and demonstrationals  65
\end{styleindexii}

\begin{styleindexii}
Directional  173
\end{styleindexii}

\begin{styleindexii}
Locational  124, 126, 171
\end{styleindexii}

\begin{styleindexii}
Pronouns and pro-forms  56
\end{styleindexii}

\begin{styleindexii}
Proper Names  101
\end{styleindexii}

\begin{styleindexi}
Derivational processes
\end{styleindexi}

\begin{styleindexii}
Noun to adjective  108
\end{styleindexii}

\begin{styleindexii}
Noun to adverb  83
\end{styleindexii}

\begin{styleindexii}
Verb to noun  96, 97, 180
\end{styleindexii}

\begin{styleindexi}
Dugwor language  4
\end{styleindexi}

\begin{styleindexi}
Focus and prominence
\end{styleindexi}

\begin{styleindexii}
Definiteness  111
\end{styleindexii}

\begin{styleindexii}
Discourse peak  57, 85, 89, 183, 192, 196, 220, 257, 272
\end{styleindexii}

\begin{styleindexii}
Ideophones  90
\end{styleindexii}

\begin{styleindexii}
Local adverbial demonstratives  71
\end{styleindexii}

\begin{styleindexii}
\textit{Na} marker  244
\end{styleindexii}

\begin{styleindexii}
Stem plus ideophone auxiliary  194
\end{styleindexii}

\begin{styleindexii}
Topicalisation  244
\end{styleindexii}

\begin{styleindexii}
Verb focus construction  183
\end{styleindexii}

\begin{styleindexi}
Fulfulde language  3, 4, 81
\end{styleindexi}

\begin{styleindexi}
Gemjek language  4
\end{styleindexi}

\begin{styleindexi}
Gemzek language  49
\end{styleindexi}

\begin{styleindexi}
Giziga language  3, 4
\end{styleindexi}

\begin{styleindexi}
Mbuko language  4, 49, 95, 137
\end{styleindexi}

\begin{styleindexi}
Muyang language  4, 28, 29, 49, 137
\end{styleindexi}

\begin{styleindexi}
Noun class
\end{styleindexi}

\begin{styleindexii}
A-prefux  98
\end{styleindexii}

\begin{styleindexi}
Plurality
\end{styleindexi}

\begin{styleindexii}
Noun phrase plural  110
\end{styleindexii}

\begin{styleindexii}
Noun plurals  98
\end{styleindexii}

\begin{styleindexii}
Numerals and quantifiers  75
\end{styleindexii}

\begin{styleindexii}
Verb plurals  149, 150
\end{styleindexii}

\begin{styleindexi}
Tense, mood, and aspect
\end{styleindexi}

\begin{styleindexii}
Aspect in complement clauses  261
\end{styleindexii}

\begin{styleindexii}
Aspect in intransitive clauses  207
\end{styleindexii}

\begin{styleindexii}
Completion  261
\end{styleindexii}

\begin{styleindexii}
Habitual iterative aspect  170
\end{styleindexii}

\begin{styleindexii}
Imperfective  207
\end{styleindexii}

\begin{styleindexii}
Imperfective aspect  160
\end{styleindexii}

\begin{styleindexii}
Inception  262
\end{styleindexii}

\begin{styleindexii}
Intermittent iterative  171
\end{styleindexii}

\begin{styleindexii}
Irrealis mood  164
\end{styleindexii}

\begin{styleindexii}
Mood in adverbial clauses  266
\end{styleindexii}

\begin{styleindexii}
Mood in noun phrase  117
\end{styleindexii}

\begin{styleindexii}
Perfect  84, 177, 188, 208
\end{styleindexii}

\begin{styleindexii}
Perfective aspect  158, 207
\end{styleindexii}

\begin{styleindexii}
Pluractional  175
\end{styleindexii}

\begin{styleindexii}
Progressive  190
\end{styleindexii}

\begin{styleindexi}
Transitivity  197
\end{styleindexi}

\begin{styleindexii}
Clauses with zero transitivity   90, 219
\end{styleindexii}

\begin{styleindexii}
Noun incorporation  212
\end{styleindexii}

\begin{styleindexi}
Vame language  49, 198
\end{styleindexi}

\begin{styleindexi}
Verb classification
\end{styleindexi}

\begin{styleindexii}
Structural  128
\end{styleindexii}

\begin{styleindexii}
Transitivity  198
\end{styleindexii}\end{multicols}

\setcounter{page}{1}

\begin{verbatim}%%move bib entries to  localbibliography.bib
\end{verbatim}