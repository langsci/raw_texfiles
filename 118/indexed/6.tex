\chapter[ Verb root and stem]{ Verb root and stem}\label{chap:6}
\hypertarget{RefHeading1211881525720847}{}
In addition to analysing the phonology of Moloko, \citet{Bow1997c} studied verb morphology and also produced notes on the grammar of Moloko which were expanded by \citet{Boyd2003}; \citet{FriesenMamalis2008} is an analysis of the Moloko verb and verb phrase. The next four chapters are based on \citet{FriesenMamalis2008}, but the data and analysis have been re-worked, reorganised, and expanded. 

The verb is the centre of the clause in Moloko.  It expresses the action of an event, or a situation or state.  It may be the only element in a clause, or it may be accompanied by noun phrases or pronouns expressing the subject, the direct object, and the indirect object of the verb, adpositional phrases expressing location, and/or discourse markers. Ideophones\is{Ideophone} (\sectref{sec:3.6}) figure greatly in the expression of the action, both when they function as adverbs and when they fill the verb slot in a clause.

Typical of a Chadic language, Moloko has a variety of extensions that modify the sense of the verb stem.\footnote{Note that the term ‘extension’ for Chadic languages has a different use than for Bantu languages. In Chadic languages, ‘extension’ refers to particles or clitics in the verb word or verb phrase.} It has 6 extensions which specify location of the event, direction with respect to centre of reference, and the Perfect.  An underspecified valence system (\chapref{chap:9}) allows variable transitivity usage for a given verb. In Moloko, valence-changing operations are not achieved through morphological modifications of the verb (for example with causative, applicative, and passive affixes). Transitivity is a clause-level property that carries a grammatical function. 

Because of its complexity, the Moloko verb and verb phrase are treated in four separate chapters. We distinguish verb root, stem (both described in \chapref{chap:6}), verb word -- renamed ‘verb complex’ for Moloko (verb stem plus affixes and extensions, \chapref{chap:7}), verb phrase (\chapref{chap:8}), and finally verb and transitivity types (\chapref{chap:9}). 

\section{The basic verb root and stem}\label{sec:6.1}%%\is{Verb classification!Structural}
\is{Verb classification|(}
\hypertarget{RefHeading1211901525720847}{}
\citet{Bow1997c} found that the verb root in Moloko consists of one to four consonants and perhaps a vowel. The verb root by itself never occurs in the language. In discussing the verb in Moloko it is more profitable to consider the verb stem as the most basic lexical unit. The Moloko verb stem itself is already complex. \citet{FriesenMamalis2008} determined that in order to pronounce a verb stem in Moloko, a speaker needs to know the following six features: 


\begin{itemize}
\item the consonantal skeleton of the verb root (\sectref{sec:6.2}).
\item if the stem carries the /-j/ suffix (\sectref{sec:6.3}).  
\item if the root has an underlying vowel (\sectref{sec:6.4}). 
\item if the stem carries the \textit{a-} prefix (\sectref{sec:6.5}).  
\item the prosody of the stem (labialised, palatalised, or neutral, \sectref{sec:6.6}). 
\item the tone class of the stem (high, low, or toneless, \sectref{sec:6.7}). 
\end{itemize}

The structural arrangement of the six features is diagrammed in \figref{fig:10}.. 

\begin{figure}
\begin{tabular}{l|c|l}

\multicolumn{3}{c}{$\leftarrow$ \hfill   tone pattern \hfill  $\rightarrow$}\\
\multicolumn{3}{c}{$\leftarrow\leftarrow\leftarrow\leftarrow\leftarrow\leftarrow$\hfill prosody}\\
\cline{2-2}
\itshape a- & root & \itshape -aj\\
            & C (C) (C) (VC) &   \\\cline{2-2}
\end{tabular}
\caption{Structure of the verb stem}\label{fig:10}
\end{figure}

\section{The consonantal skeleton of the root}\label{sec:6.2}
\hypertarget{RefHeading1211921525720847}{}
Moloko verb roots are like those of other Afroasiatic languages in that they are built on a consonantal skeleton. \citet{Bow1997c} found that the verb root consists of one to four consonants, although a skeleton of two consonants is most common.\footnote{Bow’s (\citeyear{Bow1997c}) database includes 26 one-consonant verbs, 231 two-consonant verbs, 83 three-consonant verbs, and 10 four-consonant verbs. } That Moloko verb roots are based on a consonantal skeleton can be evidenced by two facts, both of which are illustrated in \tabref{tab:37}. (adapted from \citealt{Bow1997c}). Firstly, the consonants display a unique stability when the verb is inflected.\footnote{Note there are consonantal allophones in palatalised and labialised words. } The vowels, on the other hand, change with the prosody of the inflection and whether or not the word carries stress.\footnote{Since stress is phrase-final, the final syllable of these elicited examples will always carry a ‘full’ vowel.} Secondly, there are verb roots that consist simply of one consonant and a prosody. These have no underlying root vowel, but they will acquire their vowels in the inflections. 

The underlying form of a verb stem is defined as the consonantal skeleton plus the optional presence of an underlying vowel, /-j/ suffix, and \textit{a-} prefix, potential prosody, and tone (see Sections \ref{sec:6.3}--\ref{sec:6.7}). In the examples in \tabref{tab:37} and in the rest of this section, the underlying form will be given when necessary in addition to the phonetic pronunciation. The tone class is not shown. 

\begin{table}
\resizebox{\textwidth}{!}{
\begin{tabular}{lp{2cm}lp{3.25cm}p{2.5cm}p{2.5cm}}
\lsptoprule
Root type $\downarrow$ & {Underlying form of stem} & {\SSS Perfective} {\textit{a-}} & {\SSS Perfective with directional} {\textit{a-}}{    =}{\textit{ala}} & {\oldstylenums{1}}{\textsc{Pin}}{ Perfective} {\textit{mo- {}-ok}} & {Nominalised form} {\textit{mə- (-əy)-e}}\\

\midrule
\multicolumn{6}{c}{One-consonant}\\\midrule               
                                                                                                                
neutral   & /p -j /      & \textit{a-p-ay}  &  \textit{a-p=ala}  & \textit{mo-p-ok}   &  \textit{mə-p-əy-e}\\
          &          &  ‘he opened’ &            ‘he opened towards’     &    ‘we opened’ & ‘opening’ \\
palatalised & / g\textsuperscript{ e} /& \textit{e-g-e}   &  \textit{a-g=ala}    & \textit{mo-g-ok} & \textit{mə-g-əy-e}\\
	  &	   & ‘he did’ &‘he did towards’        &  ‘we did’ & ‘doing’\\
labialised & / l\textsuperscript{  o} / &    \textit{o-lo}     &    \textit{a-l=ala} &  \textit{mo-loh-ok}\footnote{Irregular form with epenthetic \textit{h} added between vowels. For complete conjugation see Appendix~\ref{sec:13.2}. / l \textsuperscript{o} / is the only single consonant verb root that is labialised.} & \textit{mə-l-əy-e}\\
	   &				&  ‘he went’             & ‘he came towards’ &    ‘we went’ &  ‘going’\\
\midrule
\multicolumn{6}{c}{Two-consonant}\\\midrule               
neutral     & /f  ɗ /    & \textit{a-fa}\textit{ɗ} & \textit{a-fəɗ=ala}           & \textit{mə-fəɗ-ok}    & \textit{mə-fəɗ-e}\\
            &    &  ‘he put’              &  ‘he put towards’           & ‘we put’    & ‘putting’\\
palatalised &  / ɮ g\textsuperscript{ e} /    &  \textit{e-zləg-e}     &  \textit{a-zləg=ala} & \textit{mə-zləg-ok} & \textit{mə-zləg-e}\\
            &                                 &  ‘he sowed’              &  ‘he sowed towards’           & ‘we sowed’    & ‘sowing’\\
labialised  &  /ndaɮ -j\textsuperscript{ o} / &  \textit{a-ndozl-oy}  &  \textit{a-ndazl=ala}  & \textit{mə-ndozl-ok}  & \textit{mə-ndezl-e}\\
            &                                 &  ‘he exploded’          &  ‘it exploded towards’       & ‘we exploded’  & ‘exploding’\\
\midrule
\multicolumn{6}{c}{Three-consonant}\\\midrule

neutral         &     /p ɗ k-aj /   &  \textit{a-pəɗək-ay}     & \textit{a-pəɗək=ala}  & \textit{mə-pəɗək-ok}  & \textit{mə-pəɗək-e}\\
                &                          &  ‘he woke’              & ‘he woke up’    & ‘we woke up’     & ‘waking’\\
palatalised     &     / ts f ɗ \textsuperscript{e} /      &  \textit{e-cəfəɗ-e}   & \textit{a-cəfəɗ=ala} & \textit{mə-cəfəɗ-ok}     & \textit{mə-cəfəɗ-e}\\
                &    &  ‘he asked’               & ‘he asked’             & ‘we asked’   & ‘questioning’\\
labialised      &     /ɓ r ts -j \textsuperscript{o} /    &  \textit{o-ɓərc-oy}   & \textit{a-ɓərc=ala}  & \textit{mə-ɓərc-ok}     & \textit{mə-ɓərc-e}\\
                &                                         &  ‘he pounded’             & \mbox{‘he pounded towards’}   & ‘we pounded’    & ‘pounding’\\
\lspbottomrule
\end{tabular}}
\caption{Consonantal skeleton of selected verb stems and selected word forms\label{tab:37}}
\end{table}

Mamalis found that the underlying consonants in a verb root can most easily be identified from the {\twoS} imperative form (\tabref{tab:38} from \citealt{FriesenMamalis2008}). Note that palatalisation will cause an underlying /s/ to be expressed as [ʃ] (see \sectref{sec:2.2.3}). The same verb stems are included as were in \tabref{tab:37} as well as a few more. Prosody, underlying vowels (lines 12, 15), and the /-j/ suffix (lines 4-7, 15) can also be seen in the imperative form; these features will be discussed in the sections below.

\begin{table}
\begin{tabularx}{\textwidth}{lXll}
\lsptoprule
{Line} & {\raggedright Underlying form showing consonants in verb root} & {\twoS Imperative form} & {Gloss}\\\midrule
\multicolumn{4}{c}{Neutral prosody}\\\midrule
{1} & /f ɗ / & \textit{faɗ} & ‘put’\\
{2} &   /g s/ & \textit{gas} & ‘catch’\\
{3} &   /m nz r/ & \textit{mənjar} & ‘look’\\
{4} &   /p -j / & \textit{p-ay} & ‘open’\\
{5} &  /t l-aj/ & \textit{tal{}-ay} & ‘walk’\\
{6} &   /ɬ-aj/ & \textit{sl-ay} & ‘kill (by cutting \\
    &          &                                   & the throat)’\\
{7} &   /p ɗ k-aj / & \textit{pəɗak{}-ay } & ‘wake’\\
\midrule
\multicolumn{4}{c}{Palatalised prosody}\\\midrule
{8} &  / g \textsuperscript{e} / & \textit{g-e} & ‘do’\\
{9} &   /  s \textsuperscript{e}  / & \textit{s-e} & ‘drink’\\
{10} &   / ɮ g \textsuperscript{e}/ & \textit{zləg{}-e} & ‘bring’\\
{11} &   / ts f ɗ \textsuperscript{e}  / & \textit{cəfəɗ{}-e} & ‘ask’\\
{12} &   / ts a n \textsuperscript{e} / & \textit{cen} & ‘understand’\\
\midrule\multicolumn{4}{c}{Labialised prosody}\\\midrule
{13} &   / l \textsuperscript{o} / & \textit{lo} & ‘go’\\
{14} &   /  z m \textsuperscript{o} / & \textit{zom} & ‘eat’\\
{15} &   /  nd a ɮ -j \textsuperscript{o} / & \textit{ndozl-oy} & ‘explode’\\
\lspbottomrule
\end{tabularx}
\caption{Underlying form of selected verb stems and imperative forms}\label{tab:38}
\end{table}

\largerpage
The consonants in a verb stem in Moloko are remarkably constant. We have found only two irregular verbs where there are changes in the verb consonants. Firstly, the irregular verb /l\textsuperscript{o}/ adds an epenthetic [h] in some conjugations to break up vowels (the full conjugation of /l\textsuperscript{o}/ is in  Appendix~\ref{sec:13.2}). Secondly, the root-final \textit{ɗ} of the verb /z ɗ/  ‘take’ drops off when affixes and clitics are added (\ref{ex:6:1}, \ref{ex:6:2}). This process does not happen with the phonologically similar verb /f ɗ/  ‘put’ (\ref{ex:6:3}, \ref{ex:6:4}). 


\ea \label{ex:6:1}
/z ɗ/ \hspace{29pt} =aw \hspace{5pt}  =ala  \hspace{1pt} → \hspace{3pt} [zawala]\\
\glt  take[{\twoS}.{\IMP}]  ={\oneS}.{\IO}   =to   \hspace{25pt}   ‘give to me’\\
\z

\ea \label{ex:6:2}
/z ɗ/ \hspace{31pt} =aka   \hspace{24pt}    → \hspace{3pt} [zaka]\\
\glt  take[{\twoS}.{\IMP}]  =on    \hspace{50pt}     ‘give again’ (on top of what you gave before)
\z

\ea \label{ex:6:3}
/f ɗ/ \hspace{27pt} =aw \hspace{5pt} =ala  \hspace{1pt}   → \hspace{3pt} [faɗuwala]\\
\glt  put[{\twoS}.{\IMP}]    ={\oneS}.{\IO}  =to   \hspace{25pt}    ‘put on me’
\z

\ea \label{ex:6:4}
/f ɗ/ \hspace{29pt} =aka   \hspace{24pt}    → \hspace{3pt} [faɗaka]\\
\glt  put[{\twoS}.{\IMP}]  =on    \hspace{50pt}     ‘put again’ (on top of what you put before)
\z

\section{Underlying suffix}\label{sec:6.3}
\hypertarget{RefHeading1211941525720847}{}
Moloko verb stems can be divided into two subclasses based on whether an underlying suffix is present or not. Slightly over 70\% of the verb stems in Bow’s (\citeyear{Bow1997c}) data take the suffix /-j/, which can have different surface variants depending on the prosody of the stem. 

\citeyear{FriesenMamalis2008} found that although the /-j/ suffix appears to have no semantic value, it does allow certain consonants to be verb root final which would otherwise not be permitted.\footnote{I.e., [\textit{b, mb, d, nd, dz, nz, g, ŋg, gʷ, ŋgʷ, ts, w, j}]. See discussion on word final consonants in \sectref{sec:2.5.1}.}  However, for many verb stems, it appears to at least synchronically be simply a place-holding suffix that drops off whenever other suffixes or extensions are attached to the verb (compare columns 3 and 4 in \tabref{tab:37}.). \REF{ex:6:5} and \REF{ex:6:6} show the same verb complex with \REF{ex:6:5} and without \REF{ex:6:6} the /-j/ suffix.\footnote{The first line in each example is the orthographic form. The second is the phonetic form (slow speech) with morpheme breaks.}


\ea \label{ex:6:5}
Apay.\\
\gll  a-p-aj\\
      \oldstylenums{3}\textsc{s}-open-{\CL}\\
\glt  ‘It opens.’
\z

\clearpage
\ea \label{ex:6:6}
Apala.\\
\gll  a-p=ala\\
      \oldstylenums{3}\textsc{s}-open=to\\
\glt  ‘It opens towards.’
\z

Verb stems with the underlying suffix but no underlying (i.e., a neutral) prosody take the surface suffix form [{}-aj]; verb stems that are labialised carry the surface form suffix [{}-ɔj].\footnote{Prosody is applied to the verb stem since the -\textit{aj} suffix takes on the prosody of the stem (prosodies spread leftwards, \sectref{sec:2.1}).} With the exception of verbs with the root-final consonant /n/,\footnote{Stems ending in \textit{n} are all palatalised, e.g., \textit{cen} ‘understand’, \textit{cəjen} ‘lose’, \textit{njeren} ‘groan’, \textit{mbesen} ‘relax’, \textit{ndeslen} ‘make cold’, \textit{ɓərzlen} ‘count’, \textit{mbeten} ‘put out’, and \textit{mbezen} ‘spoil’.  We interpret these verbs as having /n/ as final consonant because the \textit{n} cannot be interpreted as direct or indirect object and also there are no other stems which end in \textit{n}. } verb stems that are palatalised carry the surface form suffix [{}ɛ]. We interpret the [{}-ɛ] in palatalised verbs as the palatalised variant of the /-j/ suffix for two reasons. First, [{}-ɛ] patterns the same way as the /-j/ suffix (dropping off with its prosody whenever another suffix or extension is added). Second, the same rules of restriction of final stem consonants apply for palatalised verb stems as for other verb stems (see \sectref{sec:2.5.1}), and so the presence of [{}-ɛ]  allows root-final consonants which would otherwise be restricted. For example, /d/ and /g/ are both not permitted as word-final consonants (\sectref{sec:2.2.4}), but the presence of [{}-ɛ] allows verbs like [d-ɛ]  and [g-ɛ]. Examples from verb roots of one, two, and three consonants are shown in \tabref{tab:39}.\footnote{We found no three-consonant palatalised verb stems in the data. Labialised verb stems without the /\textit{{}-j}/ suffix were rare.}

\begin{table}
\resizebox{\textwidth}{!}{\begin{tabular}{llll}
\lsptoprule
{Number of } & {One-consonant } & {Two-consonant } & {Three-consonant}\\
{consonants} & {verb root} & {verb root} & {verb root}\\
\midrule
\multicolumn{4}{c}{Stems with no suffix }\\\midrule 
No underlying prosody & & \textit{tah} ‘reach out’ & \textit{mənjar} ‘see’\\
 & & \textit{zlan} ‘begin’ & \textit{təkam} ‘taste’\\
\\
Labialised verb stems  & \textit{lo} ‘go’ & \textit{zom} ‘eat’ & \textit{səkom} ‘buy/sell’\\
\\
Palatalised verb stems &  & \textit{cen} ‘understand’ & \textit{mbezlen} ‘count’\\
& & & \textit{mbezen} ‘spoil’\\
\midrule
\multicolumn{4}{c}{Stems with suffix}\\\midrule
No underlying prosody & \textit{l{}-ay} ‘dig’ & \textit{haɓ{}-ay} ‘dance’ & \textit{təwaɗ-ay} ‘cross’\\
{}\textit{-ay} suffix & \textit{j-ay} ‘say’ & \textit{lag-ay} ‘accompany’ & \textit{sləɓat-ay} ‘repair’\\
\\
Labialised verb stems & & \textit{cok-oy} ‘undress’ & \textit{təkos-oy} ‘cross legs’\\
{}\textit{-oy} suffix & & \textit{ɓor-oy} ‘climb’ & \textit{təlok-oy} ‘drip’\\
\\
Palatalised verb stems  & \textit{g-e} ‘do’ & \textit{cək{}-e} ‘stand up’\\
{}\textit{-e} suffix  & \textit{z-e} ‘smell’ & \textit{zləg-e} ‘plant’  \\
\lspbottomrule
\end{tabular}}
\caption{Stems with and without underlying suffix\label{tab:39}}
\end{table}

Because the suffix surfaces only word-finally, whenever the relevant verb is pronounced in isolation (and is thus phrase-final), the suffix syllable takes the phrase-final stress, necessitating a full vowel. It is therefore pronounced [aj] (see example \ref{ex:6:7}) in verbs with neutral prosody, [ɔj] in labialised verb stems, and [{}ɛ] in palatalised verb stems).  Whenever the verb is not phrase-final, the vowel drops and an epenthetic schwa occurs, rendering the pronunciation [i] for labialised and neutral prosody verbs \REF{ex:6:8} and [ɪ] for palatalised verbs. 

\ea \label{ex:6:7}
\textup{[a-paɗ-aj]}\\
      \oldstylenums{3}\textsc{s}-crunch{}-{\CL}\\
\glt  ‘It crunches.’
\z

\largerpage
\ea \label{ex:6:8}
\textup{[a-paɗ-ij  ʃɛʃɛ]}\\
      \oldstylenums{3}\textsc{s}-crunch-{\CL}  meat\\
\glt  ‘He eats meat.’
\z

\tabref{tab:40} (adapted from \citealt{Bow1997c} and \citealt{Boyd2003}) illustrates the phonetic pronunciation including tone of pairs of verb stems that have the same consonantal shape but with and without the /-j/ suffix.

\begin{table}
\begin{tabular}{lll}
\lsptoprule

{Underlying Form of Stem} & {Verb Stem} & {Gloss}\\\midrule
/bar/ & [ɓár] & ‘shoot an arrow’\\
/bar-aj/ & [ɓár-áj] & ‘toss and turn when sick’\\

/tsar/ & [tsár] & ‘taste good’\\
/tsar-aj/ & [tsàr-àj] & ‘tear’\\

/dar/ & [dàr] & ‘move’ \\
/dar-aj/ & [dàr-àj] & ‘plant’\\

/ɗak/ & [ɗàk] & ‘fill up a hole’\\
/ɗak-aj/ & [ɗàk-áj] & ‘show’/‘tell’\\

/faɗ/ & [fàɗ] & ‘put’\\
/faɗ-aj/ & [fáɗ-áj] & ‘fold’\\

/f t/ & [fàt] & ‘grow’ (plant)\\
/fat-aj/ & [fàt-àj] & ‘lower’\\

/g r/ & [gár] & ‘grow’ (human)\\
/gar-aj/ & [gár-àj] & ‘govern’\\

/h ɓ/ & [hàɓ] & ‘break’\\
/haɓ-aj/ & [hàɓ-àj] & ‘dance’\\

/k ɗ/ & [káɗ] & ‘kill’\\
/kaɗ-aj/ & [káɗ-áj] & ‘prune’\\

/ɬ r/ & [ɬár] & ‘send’\\
/a-ɬar/ & [ɬàr-áj] & ‘slide’\\

/mb d/ & [mbàɗ] & ‘change position’\\
/mbad-aj/ & [mbáɗ-áj] & ‘swear’\\

/ng r/ & [ŋgár] & ‘prevent’\\
/ngar-aj / & [ŋgàr-àj] & ‘rip’\\

/s k/ & [sák] & ‘multiply’\\
/sak-aj/ & [sàk-áj] & ‘sift’\\

/t r/ & [tár] & ‘enter’\\
/tar-aj/ & [tàr-áj] & ‘call’\\

/v r/ & [vár] & ‘roof’ (a house)\\
/var-aj/ & [vàr-àj] & ‘chase away’\\

/w l/ & [wál] & ‘attach’\\
/wal-aj/ & [wál-áj] & ‘look among things’\\

/w s/ & [wàs] & ‘cultivate’\\
/was-aj/ & [wás-áj] & ‘populate’\\
\lspbottomrule
\end{tabular}
\caption{Verb stems with and without /\textit{-j}/ suffix\label{tab:40}}
\end{table}

\section{Underlying vowel in the root}\label{sec:6.4}
\hypertarget{RefHeading1211961525720847}{}
\largerpage \citet{Bow1997c} noted that no Moloko verb root has more than one underlying internal vowel and many Moloko verb roots have no underlying vowels (see \tabref{tab:38}.).\footnote{\citealt{Bow1997c}, page 24. Her database of 350 verb stems has 189 with the internal vowel.}  The presence of an underlying internal vowel in the verb stem (if any) can be determined by studying the second plural imperative. Bow illustrates the following minimal pair. The verb stems  /ts r/ ‘climb’ and /tsar/ ‘taste good’ have identical surface forms in the second person singular imperative (\ref{ex:6:9}, \ref{ex:6:10}) due to stress on the final syllable, which necessitates a full vowel. However, the presence of the underlying vowel can be seen in the second person plural imperative (\ref{ex:6:11}, \ref{ex:6:12}).\footnote{The {\twoP} imperative is formed by adding the suffix -\textit{om} and labialisation prosody.} The verb root for ‘climb’ does not have an underlying vowel, so a schwa is inserted and labialised to become [ʊ] \REF{ex:6:11}. On the other hand, the verb root for ‘taste good’ has an internal vowel which becomes [ɔ] when labialised \REF{ex:6:12}.

\ea \label{ex:6:9}
[tsar] \\
\glt  ‘climb!’ ({\twoS})    
\z

\ea \label{ex:6:10}
[tsar]\\
\glt  ‘taste good!’ ({\twoS})
\z

\ea \label{ex:6:11}
[tsʊr-ɔm]\\
\glt  ‘climb!’ ({\twoP})    
\z

\ea \label{ex:6:12}
[tsɔr-ɔm]\\
\glt  ‘taste good!’ ({\twoP})
\z

\tabref{tab:41} (from \citealt{FriesenMamalis2008}) shows several other examples. Single consonant roots have no internal vowel (line 1). Two and three-consonant roots may have no internal vowel (lines 2-4) or an internal vowel (lines 5-7). All four-consonant roots have an internal vowel (line 8).

\begin{table}
\begin{tabular}{lllll}
\lsptoprule
{Line}  & {\twoS Imperative} & {\oldstylenums{2}\textsc{p} Imperative} & {Consonantal skeleton } & {Gloss}\\
 & & & {with stem vowel} &\\
\midrule
\multicolumn{5}{c}{No internal vowel}\\\midrule
1  & \textit{sl-ay} & \textit{sl-om} & /ɬ/ & ‘kill’\\
2  & \textit{tar} & \textit{tər-om} & /t r/ & ‘enter’\\
3  & \textit{həm-ay} & \textit{həm-om} & /h m{}-aj/ & ‘run’\\
4  & \textit{mənjar} & \textit{mənjər-om} & /m nz r/ & ‘see’\\\midrule\multicolumn{5}{c}{Internal vowel}\\\midrule
5 & \textit{tar-ay} & \textit{tor-om} & /tar{}-aj/ & ‘call’\\
6 & \textit{ndozl-oy} & \textit{ndozl-om} & /ndaɮ \textsuperscript{o}/ & ‘explode’\\
7 & \textit{məndac-ay} & \textit{məndoc-om} & /m ndats-aj/ & ‘gather’\\
8 & \textit{bəjəgam-ay} & \textit{bəjəgom-om} & /b dz gam -j/ & ‘crawl’\\
\lspbottomrule
\end{tabular}
\caption{Presence or absence of internal vowel\label{tab:41}}
\end{table}

Bow discovered that when an underlying vowel exists in the root, it always immediately precedes the final root consonant, so possible verb roots could take the following forms (disregarding affixes):  C, CC, CaC, CCC, CCaC, CCCaC.{ }These ‘full’ vowels will remain full in all inflections of the verb, and will be affected by the prosodies of the forms, resulting in surface [a, ɛ, ɔ, œ].  In syllables where there are no underlying vowels, an epenthetic schwa is inserted between certain consonant clusters to facilitate pronunciation in the inflected forms. On stressed syllables, the schwa will become its full vowel counterpart (see example \ref{ex:6:9}). 

\section{Underlying prefix}\label{sec:6.5}
\hypertarget{RefHeading1211981525720847}{}
The verb stems in one class of bi-consonantal verbal stems take subject prefixes with the full vowel /a/ instead of the epenthetic schwa. \citet{Bow1997c} called this a historical \textit{a-} prefix on the verb stem. She reported that 83 out of 231 bi-consonantal verb stems that she studied have the (now frozen)  \textit{a-} prefix. Whether a verb stem has this prefix or not can be determined from the nominalised form. Bow illustrates the presence of this prefix with the minimal pair /a-ndaw/ ‘swallow’ and /ndaw/ ‘insult.’ \REF{ex:6:13} and \REF{ex:6:14} show the nominalised form of the two verb stems.\footnote{The nominalised form has a \textit{mə{}-} or \textit{me-} prefix, an \textit{{}-e} suffix, and is palatalised (\sectref{sec:7.6}).} The verb stem \textit{məndewe} ‘swallow’ does not have the \textit{a-} prefix. The verb stem \textit{mendewe} ‘insult’ has the \textit{a-} prefix (shown by the full vowel \textit{e} in the prefix).

\ea \label{ex:6:13}
mənd\'{e}we\\
\gll  mɪ-ndɛw-ɛ \\     
      {\NOM}{}-swallow-{\CL}\\
\glt  ‘swallowing’      
\z

\ea \label{ex:6:14}
mendewe\\
\gll  mɛ-ndɛw-ɛ\\
      {\NOM}{}-insult-{\CL}\\
\glt  ‘insulting’
\z

Bow proposed that synchronically, the \textit{a-} prefix verb stems represent a separate class of verb stems. \tabref{tab:42}. (adapted from \citealt{Bow1997c}) shows the phonetic representation of minimal pairs giving evidence of the presence of the \textit{a-} prefix. Those with [mɛ-] in the initial syllable contain the \textit{a-} underlying prefix; those with [mɪ-] in the initial syllable do not have the \textit{a-} prefix.

\begin{table}\caption{Minimal pairs showing presence of historical /a-/ prefix}\label{tab:42}
\resizebox{\textwidth}{!}{\begin{tabular}{llll}
\lsptoprule
{Underlying form} & {Gloss} & {Nominalised form} & {Underlying tone of stem}\footnote{Note that the underlying tone of \textit{a-} prefix verb stems is always low (see discussion in \sectref{sec:6.7})}\\\midrule
 /ndaw -j/ & ‘swallow’ & [mɪ-ndɛw-ɛ] & toneless\\
/a-ndaw -j/ & ‘insult’ & [mɛ-ndɛw-ɛ] & L\\
/ɮ r/ & ‘pierce’ & [mɪ-ɮɪr-ɛ] & H\\
/a-ɮ r/ & ‘kick’ & [mɛ-ɮɪr-ɛ] & L\\
/tsah -j/ & ‘ask’ & [mɪ-tʃɛh-ɛ] & H\\
/ a-tsah -j/ & ‘scar’ & [mɛ-tʃɛh-ɛ] & L\\
/law -j/ & ‘hang’ & [mɪ-lɛw-ɛ] & L\\
/a-law -j/ & ‘mate’ & [mɛ-lɛw-ɛ] & L\\
/k w -j/ & ‘get drunk’ & [mɪ-kuw-ɛ] & L\\
/a-k w -j/ & ‘search’ & [mɛ-kuw-ɛ] & L\\
\lspbottomrule
\end{tabular}}
\end{table}

\largerpage
Note that the \textit{a-} prefix carries very little lexical weight; there appears to be no semantic reason for its presence. Contrast is lost between \textit{a-} prefix verb forms and those without the prefix in irrealis mood\is{Tense, mood, and aspect!Irrealis mood} (see \sectref{sec:7.4.3}). \REF{ex:6:15} and \REF{ex:6:16} show that the Potential form for the verbs /a-ndaw/ ‘swallow’ and /ndaw/ ‘insult’ are identical. 

\ea \label{ex:6:15}
Káandáway.\\
\gll  káá-ndaw-aj\\
      {\twoS}+{\POT}-swallow{}-{\CL}\\
\glt  ‘He will swallow.’    
\z

\ea \label{ex:6:16}
Káandaway.\\
\gll  káá-ndaw-aj\\
      {\twoS}+{\POT}-insult{}-{\CL}\\
\glt  ‘He will insult.’
\z

\section{Prosody of verb stem}\label{sec:6.6}\is{Prosody (labialisation or palatalization)|(}
\hypertarget{RefHeading1212001525720847}{}
\largerpage[2] \citet{Bow1997c} found that in their underlying lexical form, Moloko verb stems are either labialised, palatalised, or without a prosody.  The database in Appendix~\ref{sec:13.1} shows that 83 out of 350 verb stems carry a prosody (61 are palatalised and 22 are labialised).\footnote{The effects of labialisation and palatisation are discussed in \sectref{sec:2.1}. Note that there are also some morphological processes where palatalisation or labialisation is a part of the morpheme, for example, palatalisation is part of the formation of the nominalised form (\sectref{sec:7.6}), and labialisation is a part of the 1P and {\twoP} subject forms \sectref{sec:7.3.1}.} Although prosodies can carry predictable lexical weight in some other related languages,\footnote{All causatives in \ili{Muyang}\il{Muyang} involve the palatalisation of the root \citep{Smith2002}. In \ili{Mbuko}\il{Mbuko}, the data show a correlation between palatalisation and pluractionality \citep{Gravina2001}.} in Moloko, labialisation and palatalisation carry very little lexical weight. \tabref{tab:43} (adapted from \citealt{Bow1997c}, with additional data) illustrates the phonetic pronunciation of several minimal pairs (or near minimal pairs) for prosody. There appears to be no predictable semantic connection between verb stems of differing prosodies. 

\begin{sidewaystable}
\begin{tabular}{llllll}
\lsptoprule
\multicolumn{2}{l}{{Neutral}} & \multicolumn{2}{l}{{Labialised}} & \multicolumn{2}{l}{{Palatalised}}\\\midrule\relax
[ɮak-aj] & ‘suffer pain’ & [ɮɔkʷ{}-ɔj] & ‘gnaw’ & [ɮɪg-ɛ] & ‘sow’\\\relax
[mbar] & ‘heal’ &  &  & [mb-ɛ] & ‘argue’\\\relax
[mbas-aj] & ‘laugh’ &  &  & [mbɛʃɛŋ] & ‘rest, breathe’\\\relax
[nzar-aj] & ‘comb, separate’ &  &  & [ndʒɛrɛŋ] & ‘groan’\\\relax
[s-aj] & ‘cut’ &  &  & [ʃ{}-ɛ] & ‘drink’\\\relax
[v-aj] & ‘winnow’ &  &  & [v-ɛ] & ‘spend time’\\\relax
&  & [tsɔk-ɔj] & ‘undress’ & [tʃɪk-ɛ] & ‘stand up’\\\relax
[dzak-aj] & ‘lean’ & [dzɔkʷ{}-ɔj] & ‘pack down’ &  & \\\relax
[ɗak-aj] & ‘show, tell’ & [dɔkʷ{}-ɔj] & ‘arrive’ &  & \\\relax
[fak-aj] & ‘uproot tree’ & [fɔkʷ-ɔj] & ‘whistle with lips’ &  & \\\relax
[gaz-aj] & ‘nod’ & [gʊz-ɔj] & ‘tan’ &  & \\\relax
[kar-aj] & ‘steal’ & [kɔr-ɔj] & ‘put’ &  & \\\relax
[l-aj] & ‘dig’ & [lɔ] & ‘go’ &  & \\\relax
[ɬah-aj] & ‘mix grain with ashes’ & [ɬɔhʷ-ɔj] & ‘leave in secret’ &  & \\\relax
[pal-aj] & ‘choose’ & [pɔl-ɔj] & ‘scatter’ &  & \\\relax
[saɓ-aj] & ‘exceed’ & [sɔɓ-ɔj] & ‘suck’ &  & \\\relax
[sak-aj] & ‘sift’ & [sɔkʷ-ɔj] & ‘whisper’ &  & \\\relax
[sar] & ‘know’ & [sɔr-ɔj] & ‘slide’ &  & \\\relax
[təkas-aj] & ‘cross’ & [tʊkʷɔs-ɔj] & ‘fold legs’ &  & \\\relax
[tah-aj] & ‘boost’ & [tɔhʷ-ɔj] & ‘trace’ &  & \\\relax
[zar-aj] & ‘linger’ & [zɔr-ɔj] & ‘notice, inspect’ &  & \\
\lspbottomrule
\end{tabular}
\caption{Minimal pairs for prosody of verb stems\label{tab:43}}
\end{sidewaystable}

The underlying labialisation and palatalisation prosodies are lost when most suffixes or clitics\footnote{The indirect object pronominal enclitic does not always influence the verb prosody; see \sectref{sec:7.3.2} and 2.6.1.3.} are added, compare example \REF{ex:6:17} and \REF{ex:6:18} for the verb /s -j \textsuperscript{e} / ‘drink.’  

\ea \label{ex:6:17}
Nese.\\
\gll  n\`ɛ-ʃ{}-ɛ\\
      {\oneS}+{\PFV}-drink-{\CL}\\
\glt  ‘I drank.’
\z

\ea \label{ex:6:18}
Nasala.\\
\gll  nà-s=ala\\
      {\oneS}+{\PFV}-drink=to\\
\glt  ‘I drank already.’ (lit. I drank towards)
\z
\is{Prosody (labialisation or palatalization)|)}
\section{Tone classes}\label{sec:6.7}
\hypertarget{RefHeading1212021525720847}{}
\citet{Bow1997c} concluded that verb stems in Moloko belong to one of three underlying tone classes: high (H), low (L), or toneless (Ø). She discovered that the underlying tone of a verb stem can be identified by comparing the {\twoS} imperative with the Potential form. The Potential form has a high tone on a lengthened subject prefix (see \sectref{sec:7.4.3}). If the tone melody of the stem is high on both imperative and Potential forms, then that stem has an underlying high tone. If the tone melody is mid or low on both forms due to the presence of depressor consonants (see \sectref{sec:2.4.1}), then the stem has underlying low tone.  If the tone melody of the stem syllable is low in the imperative but high following the high tone of the subject prefix in the Potential form, that verb stem is toneless. The high tone of the Potential form of the subject prefix spreads to the toneless stem. For the imperative form of a toneless stem, a default low tone is applied to the stem.  

A minimal triplet is shown in \tabref{tab:44} (from \citealt{FriesenMamalis2008}). Line 1 shows a High tone verb stem. The tone on the verb stem is high in both the imperative and Potential forms\is{Tense, mood, and aspect!Irrealis mood|(}. Line 2 shows a low tone verb stem with low tone in the imperative form and mid in the Potential form. Line 3 shows a toneless verb stem. This verb stem carries no inherent tone of its own and its surface tone is low in the imperative form and takes the high tone of the prefix in the Potential form. 

\begin{table}
\begin{tabular}{llllc}
\lsptoprule
{Line} & {Underlying } & {Imperative Form} & {Potential Form} & {Tone Class}\\
& {form of stem}\\\midrule
{1} & /d r/ & [dár] & [náá-dár] & {H}\\

& & ‘Burn!’ & ‘I will burn’ \\
{2} & /a-dar-aj/ & [dàr-\={a}j] & [náá-d\={a}r-áj] & {L}\\

& & ‘Plant!’ &  ‘I will plant’ & \\
{3} & /d r/ & [dàr] & [náá-dár] & Ø\\

& & ‘Recoil!’ & ‘I will recoil’ \\
\lspbottomrule
\end{tabular}
\caption{Tone class contrasts \label{tab:44}}
\end{table}

Mamalis (\citealt{FriesenMamalis2008}) studied tone patterns in Moloko verbs. \tabref{tab:45} (adapted from \citealt{FriesenMamalis2008}) shows the imperative and Potential forms and the underlying tone patterns for different verb stems. 

\begin{table}
\resizebox{\textwidth}{!}{\begin{tabular}{llllc}
\lsptoprule
{CV pattern} & {Underlying} & {Imperative} & {Potential (Irrealis) form} &  {Tone class}\\
& {form of stem} & {form} & (/náá/- prefix) &\\
\midrule
C & /b-j/ & [b-àj ] & [náá-b-àj ] & L\\
& ‘light’ & ‘Light!’  &  ‘I will light’ \\
\hhline{~----}& /g-ɛ/ & [g-\'{ɛ} ] & [n\'{ɛ}\'{ɛ}-g{}-\'{ɛ} ] & H\\
& ‘do’  & ‘Do!’ & ‘I will do’ \\
\hhline{~----} & /d-ɛ/ & [d-\`{ɛ} ] & [n\'{ɛ}\'{ɛ}- d-\`{ɛ} ] & L\\
& ‘cook’ & ‘Cook!’ & ‘I will cook’ \\\midrule
CC & /mb r/ & [mbár ] & [náá- mbár] & H\\
& ‘heal, cure’ & ‘Heal! ’ & ‘I will heal’ \\
\hhline{~----}& /m t/ & [m\={a}t ] & [náá-m\={a}t ] & L\\
& ‘die’ & ‘Die! ’ & ‘I will die’ \\
\hhline{~----} & /g s/ & [gàs ] & [náá-gás ] & toneless\\
& ‘catch’ & ‘Catch!’ & ‘I will catch’ \\\midrule
CaC & /tsar/ & [ts\={a}r ]  & [náá-ts\={a}r ] & L\\
& ‘taste good’ & ‘Taste good!’ & ‘I will taste good’ \\\midrule
a-CaC-aj & /a-pas -j/ & [p\={a}s-áj ] & [náá- p\={a}s-áj ] & L\\
& ‘spread out’ & ‘Spread out!’ & ‘I will spread out’ \\\midrule
CaC-aj & /nzak -j/ & [nzák-áj ] & [náá- nzák-áj ] & H\\
& ‘find’ & ‘Find!’ & ‘I will find’ \\
\hhline{~----}& /ndaɗ -j/ & [ndàɗ-\={a}j ] & [náá- ndáɗ-\={a}j] & toneless\\
& ‘like, love’ & ‘Love!’ & ‘I will love’ \\\midrule
CCC-aj & /d b n -j/ & [d\`{ə}b\`{ə}n-\={a}j ] & [náá- d\'{ə}b\`{ə}n-\={a}j] & L\\
& ‘learn’ & ‘Learn!’ & ‘I will learn’ \\\midrule
CCCaC-aj & /b dz gam -j/ & [b\`{ə}dz\`{ə}gàm-\={a}j] & [náá-b\`{ə}dz\`{ə}gàm-\={a}j ] & L\\
& ‘crawl’ & ‘Crawl!’ & ‘I will crawl’ \\
\lspbottomrule
\end{tabular}}
\caption{Tone patterns for selected verb stems \label{tab:45}}
\end{table}
\is{Tense, mood, and aspect!Irrealis mood|)}
Tone patterns in Moloko verbs are summarised in \tabref{tab:46} (from \citealt{FriesenMamalis2008}), which shows the tone pattern on the stem for the imperative and Potential forms for the three underlying tone forms.  All verb stems in each class have the same pattern, as follows (note that the tone in parentheses is the tone on the /-j/ suffix, if there is one). Tone patterns are influenced by the presence of depressor consonants (see \sectref{sec:6.7.1}) and the underlying structure of the verb stem (see \sectref{sec:6.7.2}).

\begin{table}
\resizebox{\textwidth}{!}{\begin{tabular}{lll}
\lsptoprule
{Underlying tone} & {Phonetic tone } & {Phonetic tone }\\
& {in imperative form} & {in Potential form}\\\midrule
H & H(H) & H(H)\\
L without depressor consonants in stem & M(H) & HM(H)\\
L with depressor consonants in stem & L(M) & HL(M)\\
Toneless & L(M) & H(H)\\
\lspbottomrule
\end{tabular}}
\caption{Summary of tone patterns for the three tone classes\label{tab:46}}
\end{table}

\subsection{Effect of depressor consonants}\label{sec:6.7.1}
\hypertarget{RefHeading1212041525720847}{}
\largerpage \citet{Bow1997c} subdivided the low tone verb stem category phonetically into mid and low surface forms by the presence or absence of one or more of the class of consonants known as depressor consonants (see \sectref{sec:2.4.1}).  Depressor consonants in Moloko include all voiced obstruents except implosives and nasals (i.e. [b, d, g, dz, v, ɮ, \textit{}z, mb, nd, ŋg]). \citet{Bow1997c} demonstrated that an underlyingly low tone verb with no depressors has a mid tone surface form; with depressors it has a low tone surface form. For verb stems of underlying high tone or toneless verb stems, the presence or absence of depressor consonants makes no difference to the surface form of the melody.  Toneless verb stems take low tone as the default surface form, regardless of depressors. \tabref{tab:47} (from \citealt{Bow1997c}) shows the realisations of surface tone with and without depressor consonants for the most common verb type (underlying form /CaC/ with high tone /-j/ suffix in the {\twoP}.{\IMP} form).

\begin{table}
\begin{tabular}{llllll}
\lsptoprule
{Underlying } & {Depressor} & {Surface } & {Underlying } & {Surface } & {Gloss}\\
{tonal melody} & {consonants} & {tone} & {form of stem} & {form} &\\\midrule
Toneless & -- & L & /haɓ-aj/ & [hàɓ-\={a}j] & ‘dance!’\\
& + & L & /daɮ-aj/ & [dàɮ-\={a}j] & ‘join!’\\\midrule
L & -- & M & /pàɗ-aj/ & [p\={a}ɗ-áj] & ‘bite!’\\
& + & L & /ɮàv-aj/ & [ɮàv-\={a}j] & ‘swim!’\\\midrule
H & -- & H & /fáɗ-aj/ & [fáɗ-áj] & ‘fold!’\\
& + & H & /bál-aj/ & [bál-áj] & ‘wash!’\\
\lspbottomrule
\end{tabular}
\caption{Effect of depressor consonants; imperative forms\label{tab:47}}
\end{table}

\subsection{  Effect of underlying form on tone of stem}\label{sec:6.7.2}
\hypertarget{RefHeading1212061525720847}{}
\largerpage \citet{Bow1997c} found that the components of the underlying form, particularly initial vowel and number of consonants, influence what underlying tone the root has, such that she could predict the underlying tone of certain verb stems with accuracy. \tabref{tab:48} (from \citealt{FriesenMamalis2008}) shows the tone of verb stems of different structures, with examples. The following three stem structures are significant with respect to tone:


\begin{itemize}
\item Verb stems with the \textit{a-} prefix (always two-consonant) always have underlying low tone (line 4, \sectref{sec:6.5}). 
\item Verb stems with three or more consonant roots (line 5-6) always have underlying low tone (\sectref{sec:6.7.2.3}). 
\item Non-palatalised verb stems with one-consonant roots (line 1 of \tabref{tab:48}) always have underlyingly low tone (\sectref{sec:6.7.2.1}). Palatalised verb stems with one-consonant roots may be high or low but not toneless (line 2). 
\end{itemize}

These three categories account for about 45\% of the verb stems in the database of 316 verb stems used by Mamalis (\citealt{FriesenMamalis2008}). Only two-consonant roots with no \textit{a-} prefix allow all underlying tone patterns (line 3 of \tabref{tab:48}).

\begin{sidewaystable}
\begin{tabular}{lllll}
\lsptoprule
{Line} & {Verb stem structure} & \multicolumn{3}{c}{{Underlying tone of 316 verb stems}}\\\cmidrule(lr){3-5}
&  & {H } & {L} & {Toneless}\\\midrule
1 & One-consonant &  & 7 verb stems & \\
& non-palatalised  & & {}[b-àj ] ‘light’\\
& verb roots & & {}[p-\={a}j ] ‘open’\\\midrule
{2} & One-consonant  & 4 verb stems & 8 verb stems & \\
& palatalised verb  & {}[g-\'{ɛ} ] ‘do’ & {}[d-\`{ɛ} ] ‘cook’\\
& roots & & {}[ʃ-\={ɛ} ] ‘drink’ \\\midrule
3 & 2 consonant  & 36 verb stems & 49 verb stems & 38 verb stems\\
& verb roots with  & {}[fár ] ‘scratch’  & {}[g\`{ə}r-\={a}j ] ‘tremble’ & {}[dàɗ ] ‘fall’\\
& no \textit{a-} prefix & {}[bál-áj ] ‘wash’ & {}[f\={a}t ] ‘grow’ & {}[h\`{ə}m-\={a}j ] ‘run’\\
& & & {}[tʃ\={ɪ}k-\'{ɛ} ] ‘stand’ & \\
& & & {}[ts\={ə}ɗ-áj ] ‘shine’ \\\midrule
4 & \textit{a-} prefix verb stems &  & 82 verb stems \\
& (all have  & & {}[bàz ] ‘harvest’ & \\
& 2 consonants)& &\\\midrule
5 & 3 consonant  &  & 58 verb stems\\
& verb roots & & {}[v\`{ə}nàh-\={a}j ] ‘vomit’\\
& & & {}[ɬ\={ə}ɓ\={a}t-áj ] ‘repair’ & \\\midrule
6 & 4 consonant &  & 12 verb stems\\
& verb roots & & {}[b\`{ə}dz\`{ə}gàm-\={a}j ] ‘crawl’ & \\
\lspbottomrule
\end{tabular}
\caption{Underlying tones for different verb stem structures\label{tab:48}}
\end{sidewaystable}

\subsubsection{Verb stems with one root consonant}\label{sec:6.7.2.1}

Verb stems with single consonant verb roots (the /-j/ suffix is added to produce the stem) (cf. lines 1 and 2 of \tabref{tab:48}) are never toneless.\footnote{One possible exception is /dz-aj/ ‘say,’ which may be toneless.} Non-palatalised verb stems carry only low tone. Palatalised verb stems may be high or low.  The two possible tonal melodies are seen in the following minimal pair (from \citealt{FriesenMamalis2008}). Example \REF{ex:6:19} has an underlying high tone; example \REF{ex:6:20} has an underlying low tone. 

\ea \label{ex:6:19}
Njé. \hspace{63pt}   Néenjé.\\
\gll  nʒ-\'{ɛ}  \hspace{10pt}    néé-nʒ-\'{ɛ}\\
      leave[{\twoS}.{\IMP}]{}-{\CL} \hspace{5pt} {\oneS}+{\POT}-leave{}-{\CL}\\
\glt  ‘leave!’  \hspace{50pt}  ‘I will leave.’
\z

\ea \label{ex:6:20}
Nje.  \hspace{50pt}    Néenje.\\
\gll  nʒ-\`{ɛ}  \hspace{10pt}    néé-nʒ-\`{ɛ}\\
      sit[{\twoS}.{\IMP}]-{\CL} \hspace{5pt} {\oneS}+{\POT}-sit{}-{\CL}\\
\glt  ‘Sit!’ \hspace{50pt}   ‘I will sit.’
\z

Additional examples illustrating underlying stem tone in verb stems with one root consonant are given in \tabref{tab:49} (from \citealt{FriesenMamalis2008}). Imperative and Potential forms are given for each example. Stems with and without depressor consonants are included. 

\begin{table}
\begin{tabular}{lllll}
\lsptoprule
\multicolumn{2}{l}{{Syllable pattern and }} & {H} & \multicolumn{2}{l}{{L}}\\
{Aspect/mood} &  &  & -- depressor  & + depressor\\
& & & consonants &  consonants\\ \midrule
Palatalised & Imperative & [g-\={ɛ}] & [ʃ-\={ɛ}] & [d-\`{ɛ}] \\
& & ‘do, make’ & ‘drink’ & ‘prepare’\\
 & Potential & [k\'{ɛ}\'{ɛ}-g-\'{ɛ}] & [k\'{ɛ}\'{ɛ}-ʃ{}-\={ɛ}] & [k\'{ɛ}\'{ɛ}-d-\`{ɛ}]\\
& & ‘you will do’ & ‘you will drink’ & ‘you will prepare’\\\midrule
Non- & Imperative & Ø & [p-\={a}j]  & [b-àj] \\
palatalised & & & ‘open’ & ‘light’\\
& Potential &  & [káá-p-\={a}j] & [káá-b-àj] \\
& & & ‘you will open’ & ‘you will light’\\
\lspbottomrule
\end{tabular}
\caption{Tone patterns in stems with one root consonant\label{tab:49}}
\end{table}

\subsubsection{Verb Stems with two root consonants}\label{sec:6.7.2.2}

Verb stems with two-consonant roots correspond to lines 3 and 4 of \tabref{tab:48}. \citet{Bow1997c} found that verb stems that have two root consonants and the \textit{a-} prefix all carry low tone (\tabref{tab:50} adapted from \citealt{FriesenMamalis2008}).  

\begin{table}
\begin{tabular}{llll}
\lsptoprule
\multicolumn{2}{l}{{Stem structure}} & \multicolumn{2}{l}{{L}}\\
\multicolumn{2}{l}{} & {-- depressor consonants} & {+ depressor consonants}\\\midrule
/a-CC/ & Imperative & Ø & [dàl] \\
& & & ‘surpass’\\
 & Potential &  & [káá-dàl]\\
& & & ‘you will surpass’\\\midrule
/a-CC-j/ & Imperative & [s\={ɔ}l-áj]  & [g\`{ə}r\={a}j] \\
& & ‘fry’ \footnote{There was only one example of H tone for this structure.} & ‘frighten’\\
 & Potential & [káá-s\={ɔ}l-áj]  & [káá-g\`{ə}r-\={a}j]\\
& & ‘you will fry’ & ‘you will fear’\\\midrule
/a-CaC-j/ (60) & Imperative & [p\={a}s-áj]  & [dàr-\={a}j] \\
& & ‘spread out’ & ‘plant’\\
 & Potential & [káá-p\={a}-sáj] & [káá-dàr-\={a}j]\\
& & ‘you will spread out’ & ‘you will plant’\\
\lspbottomrule
\end{tabular}
\caption{Tone patterns in a- prefix verbs\label{tab:50}}
\end{table}

Verb stems with no \textit{a- } prefix may be from any tone class. \tabref{tab:51} (\citealt{FriesenMamalis2008}) shows several examples of two consonant verbs, giving the imperative and Potential verb forms for each of the possibilities. 

\begin{table}
\begin{tabular}{lllll}
\lsptoprule
\multicolumn{2}{l}{{Stem structure}} & {H} & {L}\footnote{No two-consonant verbs without \textit{a-} prefix with low tone have depressor consonants.} & {Toneless}\\\midrule
/CC/ & Imperative & [mbár]  & [m\={a}t]  & [gàs] \\
& & ‘heal, cure’\footnote{Most CC roots that have high tone end in /r/.} & ‘die’ & ‘catch’\\
 & Potential & [káá-mbár]   & [káá-m\={a}t] & [káá-gás]\\
& & ‘you will heal’ & ‘you will die’ & ‘you will get’\\\midrule
/CaC/\footnote{Note that these are the only structures that have no counterpart \textit{a-} prefix forms.} & Imperative & Ø & [ts\={a}r]   & [hàr] \\
& & & ‘taste good’ & ‘make’\\
 & Potential &  & [káá-ts\={a}r] & [káá-hár]\\
&  & & ‘you will taste good’ & ‘you will make’\\\midrule
/CC-j/ & Imperative & [ŋg\'{ə}l-áj]  & [r\={ə}ɓ-áj] & [h\`{ə}m-\={a}j] \\
& & ‘defend’ & ‘be beautiful’ & ‘run’\\
& & (only example) \\
 & Potential & [káá-ŋg\'{ə}l-áj]  & [káá-rɓ-áj] & [káá-həm-áj]\\
& & ‘you will defend’ & ‘you will be beautiful’ & ‘you will run’\\\midrule
/CaC-j/ & Imperative & [bál-áj]  & [m\={a}k-áj]  & [ɮàw-\={a}j] \\
& & ‘wash’ & ‘stop’ & ‘fear’\\
 & Potential & [káá-bál-áj] & [káá-m\={a}k-áj] & [káá-ɮáw-áj]\\
& & ‘you will wash’ & ‘you will leave’ & ‘you will fear’\\
\lspbottomrule
\end{tabular}
\caption{Tone patterns in stems with two root consonants with no a-  prefix\label{tab:51}}
\end{table}

\subsubsection{Verb stems with three or more root consonants}\label{sec:6.7.2.3}

\citet{Bow1997c} determined that verb stems with three (or more) root consonants (cf. lines 5 and 6 of  \tabref{tab:48}) all have underlyingly low tone. The surface tone will be low or mid, depending on the presence or absence of depressor consonants. If the stem carries the -\textit{aj} suffix, the suffix will carry mid tone.  \tabref{tab:52} (from \citealt{FriesenMamalis2008}) shows examples of verb stems with three or more consonants in imperative and Potential form.  

\begin{table}
\begin{tabular}{llll}
\lsptoprule
\multicolumn{2}{l}{} & \multicolumn{2}{l}{{L}}\\
\multicolumn{2}{l}{} & {No depressor consonants} & {Depressor consonants}\\\midrule
/CCC/ & Imperative & [s\={ʊ}kʷ\'{ɔ}m] & [dz\`{ʊ}gʷ\`{ɔ}r] \\
& & ‘buy’ & ‘look after’\\
 & Potential & [k\'{ɔ}\'{ɔ}-s\={ʊ}kʷ\'{ɔ}m] & [káá-dz\`{ʊ}gʷ\`{ɔ}r]\\
& & ‘you will buy’ & ‘you will shepherd’\\\midrule
/CCaC/ & Imperative & [t\={ə}kár]   & [m\`{ə}nzàr] \\
& & ‘try, taste’ & ‘see’\\
 & Potential & [káá-t\={ə}kár] & [káá-m\`{ə}nzàr]\\
& & ‘you will try’ & ‘you will see’\\\midrule
/CCC-j/ & Imperative & [ts\={ə}f\={ə}ɗ-áj]  & [d\`{ə}b\`{ə}n-\={a}j] \\
& & ‘ask’ & ‘teach, learn’\\
 & Potential & [káá-ts\={ə}f\={ə}ɗ-áj] & [káá-d\`{ə}b\`{ə}n-\={a}j]\\
& & ‘you will ask’ & ‘you will learn’\\\midrule
/CCaC-j/ & Imperative & [p\`{ə}ɗ\`{ə}k-áj]  & [v\`{ə}nàh-\={a}j] \\
& & ‘wake’ & ‘vomit’\\
 & Potential & [káá-p\={ə}ɗ\={ə}k-áj]  & [káá-v\`{ə}nàh-\={a}j] \\ 
& & ‘you will wake’ & ‘you will vomit’\\\midrule
/CCCaC-j/ & Imperative &  & [b\`{ə}dz\`{ə}gàm-\={a}j] \\
& & & ‘crawl!’\\
 & Potential &  & [káá-b\`{ə}dz\`{ə}gàm-\={a}j]\\
& & & ‘you will crawl’\\
\lspbottomrule
\end{tabular}
\caption{Tone patterns in verb stems with three root consonants\label{tab:52}}
\end{table}
\is{Verb classification|)}
