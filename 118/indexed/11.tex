\chapter[The na marker and na constructions]{The \textit{na} marker and \textit{na} constructions}\label{chap:11}\is{Focus and prominence!Na@\textit{Na} marker|(}\is{Presupposition constructions|(}
\hypertarget{RefHeading1213141525720847}{}
Knowledge of how the particle \textit{na} works in Moloko is foundational to understanding information flow and interpreting a Moloko text. Expectation is a concept that is fundamental for Moloko. Within the irrealis world, this concept has already been discussed (mood, see \sectref{sec:7.4.3}). Within the realis world, expectation is shown in other forms. One of these forms is the \textit{na} construction or presupposition construction. Known or expected elements are marked with \textit{na}, which is found at the right edge of the element it modifies.

A very basic knowledge of \textit{na} can be gained from studying the example pair below. \REF{ex:11:1} illustrates how a person would tell another person her name during a conversation. However, if the addressee first asked the person to give her name, then ‘name’ will be marked with \textit{na} in the response \REF{ex:11:2}. Structurally, \textit{na}  isolates or separates some element in a clause or sentence from the rest of the clause. In \REF{ex:11:2}, it separates the predicate \textit{sl}\textit{əmay =əwla} ‘my name’ from the nominal \textit{Abangay}. In the examples in this chapter, \textit{na} is bolded and the element marked by \textit{na}  is underlined.

\ea \label{ex:11:1}
Sləmay  əwla  Abangay.\\
\gll  ɬəmaj   =uwla     Abaŋgaj\\
      name  ={\oneS}.{\POSS}  Abangay\\
\glt  ‘My name is Abangay.’
\z

\ea \label{ex:11:2}
\underline{Sləmay  əwla  \textbf{na}},  Abangay.\\
\gll \ulp{ɬəmaj}{~~~~}   \ulp{=uwla}{~~~~}  \ule{\textbf{na}}   Abaŋgaj\\
      name  ={\oneS}.{\POSS} {\PSP}  Abangay\\
\glt  ‘My name is Abangay.’
\z
\largerpage
\textit{Na} is a separate phonological word that positions at the end of a noun phrase \REF{ex:11:2}--\REF{ex:11:3}, time phrase \REF{ex:11:33}, discourse particle \REF{ex:11:34}, or clause \REF{ex:11:4} that is being marked. \textit{Na} has semantic scope over the preceding construction. When an element in a clause, or the clause itself, is marked with \textit{na}, it is marked as being known or expected information that is somehow a prerequisite to the information that follows.\footnote{The presupposition marker and the \oldstylenums{3}\textsc{s} direct object pronominal \sectref{sec:7.3.2} are homophones. Both function (in different ways) to mark previously identified information.} This structure for marking information as presupposed is a basic organisational structure with  a major function in certain Moloko clause structures and discourse.\footnote{\citet{Bow1997c} called \textit{na} a focus marker. We have found that the function of \textit{na} is not limited to focus. In related languages, a similar particle has often been referred to as a ‘topicalisation’\is{Focus and prominence!Topicalisation} marker, but the fronting and special marking that \citet{Levinsohn1994} describes as topic marking is only one of the functions of this particle in Moloko.}

\ea \label{ex:11:3}
\underline{Həmbo  \textbf{na}},  anday  ásəkala  azla  wəsekeke.\\
\gll  \ulp{hʊmbɔ}{~~~~} \ule{\textbf{na}}  à-ndaj      á-sək    =ala     aɮa   wuʃɛkɛkɛ\\
      flour  {\PSP}  \oldstylenums{3}\textsc{s}+{\PFV}-{\PRG}   \oldstylenums{3}\textsc{s}+{\IFV}-multiply  =to  now    \textsc{id}multiplication\\
\glt  ‘The flour, it is multiplying.’
\z

\ea \label{ex:11:4}
\corpussource{Cicada, S. 5}\\
\underline{Tánday  t\'ətalay  a  ləhe  \textbf{na}},\\  
\gll  \ulp{tá-ndaj}{~~~~~~~~} \ulp{t\'ə-tal-aj}{~~~~~~} \ulp{a}{~~~~} \ulp{lɪhɛ}{~~~~} \ule{\textbf{na}}\\ 
      \oldstylenums{3}\textsc{p}+{\IFV}-{\PRG}  \oldstylenums{3}\textsc{p}+{\IFV}-walk{}-{\CL}  at  bush  {\PSP}\\
\glt ‘[As] they were walking in the bush,’\\
      
      \medskip
tolo  tənjakay  agwazla  malan  ga  a  ləhe.\\      
\gll t\`{ə}-lɔ t\`{ə}-nzak-aj agʷaɮa malaŋ ga a lɪhɛ\\
     \oldstylenums{3}\textsc{p}+{\PFV}-go \oldstylenums{3}\textsc{p}+{\PFV}-find{}-{\CL} {spp. of tree}  large {\ADJ} at bush\\
\glt  ‘they went and found a large tree (a particular species) in the bush.’
\z

Pragmatic presupposition is defined by \citet[52]{Lambrecht1994} as “the set of presuppositions lexicogrammatically evoked in a sentence which the speaker assumes the hearer already knows or is ready to take for granted at the time the sentence is uttered.”  In Moloko, \textit{na}{}-marked elements indicate information that the speaker shares with the hearer in that the element has been previously mentioned in the discourse, is the expected part of the situation, is the expected outcome of an event, or is assumed to be common knowledge or a cultural assumption. \textit{Na} -marked elements are the way that the speaker presents any information that he thinks the hearer should not be able to (or would not want to) challenge. 

\largerpage
The partitioning that \textit{na} produces results in the clause being split into two parts:  the presupposition (followed by \textit{na}) and the assertion. The assertion is that part of the sentence which the speaker expects “the hearer knows or is ready to take for granted at the time the sentence is uttered” \citep[52]{Lambrecht1994}, but not necessarily before hearing it.  In the following example groups,\footnote{Adapted from \citealt{Boyd2002}.} the first gives the normal SVO clause structure without any \textit{na}{}-marked element. The rest have \textit{na}-marked elements (underlined). In the first triplet, \REF{ex:11:5} represents a context where there is no specific presupposed information (and there is no \textit{na} marker). \REF{ex:11:6} represents a situation where the presupposed information (marked with  \textit{na}) is ‘I like X’ and the topic of the discourse is what is liked. \REF{ex:11:7} represents a context where the presupposed information is ‘beans.’

\ea \label{ex:11:5}
Hahar  asaw. \\
\gll  hahar   a-s=aw \\
      beans   \oldstylenums{3}\textsc{s}-like={\oneS}.{\IO}\\
\glt  ‘I like beans.’ (lit. beans are pleasing to me) \\
  Presupposition:     Nothing specific.
\z

\ea \label{ex:11:6}
\underline{Asaw  \textbf{na}},  hahar. \\
\gll  \ulp{a-s=aw}{~~~~~~~~~~~~} \ule{\textbf{na}}  hahar\\
      \oldstylenums{3}\textsc{s}-like={\oneS}.{\IO} {\PSP}  beans\\
\glt  ‘[what] I like [is] beans.’\\
  Presupposition:     I like something (X). \\
Assertion:     X=beans. \\
Focus of assertion:   Beans.
\z

\ea \label{ex:11:7}
\underline{Hahar  \textbf{na}}  asaw. \\
\gll  \ulp{hahar}{~~~~~}  \ule{\textbf{na}}  a-s=aw\\
      beans  {\PSP} \oldstylenums{3}\textsc{s}-like={\oneS}.{\IO}\\
\glt  ‘As for beans, I like them.’ \\
  Presupposition 1:     Beans are the topic of this part of the discourse. 
  Presupposition 2:     Beans have some attribute (X).\\
Assertion :     X=I like them. \\
Focus of assertion:   I like them.
\z

The rearranging of the construction to front the presupposed information in the clause is shown by another set of examples (\ref{ex:11:8}--\ref{ex:11:11}). \REF{ex:11:8} has no specific presupposition (and no \textit{na} marker). \REF{ex:11:9} represents a situation where Hawa is presupposed -- the hearer knows who she is and Hawa is the topic of discussion. \REF{ex:11:10} is similar to \REF{ex:11:9} except that the relative clause also indicates known information (see \sectref{sec:5.4.3}) so the fact that someone prepared the food is also presupposed.  In \REF{ex:11:11}, the presupposed information is ‘someone made the food’ (or ‘X made the food’). 

\clearpage
\ea \label{ex:11:8}
Hawa  adan  ɗaf  ana  Mana.\\
\gll  Hawa  a-d=aŋ      ɗaf  ana   Mana\\
      Hawa  \oldstylenums{3}\textsc{s}-prepare=\oldstylenums{3}\textsc{s}.{\IO}    {millet loaf}  {\DAT} Mana\\
\glt  ‘Hawa prepared millet loaf for Mana.’  \\
Presupposition:     No specific presupposition.\\
  Assertion:     Hawa prepared millet loaf for Mana.
\z

\ea \label{ex:11:9}
\underline{Hawa  \textbf{na}},  adan  ɗaf.\\
\gll  \underline{Hawa  \textbf{na}}  a-d=aŋ      ɗaf\\
      {Hawa  {\PSP}}  \oldstylenums{3}\textsc{s}-prepare=\oldstylenums{3}\textsc{s}.{\IO}    {millet loaf}\\
\glt  ‘Hawa [is] the one who prepared the millet loaf for him.’\\
  Presupposition 1:    The hearer knows who Hawa is.\\
  Presupposition 2:   Hawa is the topic of this section of discourse, or Hawa did something (X).\\
  Assertion:     X= prepared the millet.
\z

\ea \label{ex:11:10}
\underline{Hawa  \textbf{na}},  amadan  ɗaf.\\
\gll  \ulp{Hawa}{~~~~}  \ule{\textbf{na}}  ama-d=aŋ    ɗaf\\
      Hawa  {\PSP}  {\DEP}-prepare=\oldstylenums{3}\textsc{s}.{\IO}  {millet loaf}\\
\glt  ‘Hawa [is] the one that prepared the millet loaf for him.’\\
  Presupposition 1:    The hearer knows who Hawa is.\\
Presupposition 2:    Hawa is the topic of this section of discourse (a contrastive topic).\\
  Presupposition 3:    Someone (X) prepared the millet loaf.\\
  Assertion:     Hawa is the person who prepared the millet loaf.
\z

\ea \label{ex:11:11}
\underline{Amadan  ɗaf  \textbf{na}},  Hawa.\\
\gll  \ulp{ama-d=aŋ}{~~~~~~~~~~~~~~~~~~~~~~~~~}      \ulp{ɗaf}{~~~}  \ule{\textbf{na}}  Hawa\\
     {\DEP}-prepare=\oldstylenums{3}\textsc{s}.{\IO}  {millet loaf}  {\PSP}  Hawa\\
\glt  ‘The preparer of his millet loaf [is] Hawa.’ \\
  Presupposition:     Someone (X) prepared the millet loaf.\\
  Assertion:     X=Hawa (the hearer may not know who Hawa is).  
\z

\textit{Na} constructions in Moloko can be divided into five main structural types, depending on which element is presupposed and which element is the assertion. These structural types fit the main ways that \textit{na} constructions function in Moloko discourse. The five structural types are: 

\begin{enumerate}
\item \textbf{Presupposition-assertion construction: fronted \textit{na}{}-marked clause} (\sectref{sec:11.1}). A whole clause is marked with \textit{na}, separating it from the clause which follows and marking it as presupposed. These constructions function in text cohesion. 
\item \textbf{Presupposition-assertion construction: fronted \textit{na}–marked clausal element} (\sectref{sec:11.2}). One element in a clause is fronted and delimited by \textit{na}, separating it from the rest of the clause and marking the fronted element as presupposed. Such constructions function in tracking participants and marking boundaries in a text. 
\item \textbf{Assertion-presupposition construction: right-shifted \textit{na}{}-marked element} (\sectref{sec:11.3}). The element that is marked by \textit{na} is right-shifted to the end of a clause. This construction is found in concluding statements. 
\item \textbf{The definite construction: \textit{na}–marked clausal element} (\sectref{sec:11.4}). The element that is marked by \textit{na} is in its normal clausal position. The definite construction functions to specify the element that is marked by \textit{na} in the text. 
\item \textbf{Presupposition-focus construction: \textit{na} preceeds the final element of the verb phrase} (\sectref{sec:11.5}). The final element of a clause is immediately preceded by one or more \textit{na}{}-marked elements. This construction makes prominent the final element of the clause. 
\end{enumerate}

Note that in the examples, \textit{na} is always glossed as {\PSP} ‘presupposition marker,’ even if its more specific function in a particular utterance might be argued to be for focus or definiteness, as marking presupposition is its overall function. It is probable that the different functions of \textit{na} overlap, since structurally, it is often difficult if not impossible in some cases to determine whether \textit{na} is at the end of a noun phrase  or a clause. It is also likely that the functions of \textit{na} overlap with those of the \oldstylenums{3}\textsc{s} direct object pronominal (see \sectref{sec:7.3.2}) since in certain contexts, it is difficult to determine with certainty whether \textit{na} is {\PSP} or the \oldstylenums{3}\textsc{s} \DO pronominal. The examples used in the text are chosen to clearly illustrate the function of \textit{na}.

\section{Presupposition-assertion construction: \textit{na}-marked clause}\label{sec:11.1}
\hypertarget{RefHeading1213161525720847}{}
There are two presupposition-assertion constructions depending on if the entire clause is marked with \textit{na} or if just one clausal element is marked (see \sectref{sec:11.2}). The \textit{na}{}-marked clause presupposition-assertion construction consists of an entire clause marked with \textit{na}  and fronted with respect to another clause (\ref{ex:11:12}--\ref{ex:11:14}).  The \textit{na}{}-marked clause presupposition-assertion construction functions in discourse in inter-clausal relations and is involved in discourse cohesion. The clause marked with \textit{na} expresses presupposed or shared information, and the main clause that follows contains asserted information. The precise relation between the \textit{na} clause and the main clause is determined by context (see \sectref{sec:12.4}).  In the examples in this section, the \textit{na}-marked clause is underlined. 

\ea \label{ex:11:12}
\corpussource{Cicada, S. 5}\\
\underline{Tánday  t\'ətalay  a  ləhe  \textbf{na}},\\  
\gll  \ulp{tá-ndaj}{~~~~~~~}  \ulp{t\'ə-tal-aj}{~~~~~~~~~~~~~}  \ulp{a}{~~~} \ulp{lɪhɛ}{~~~~~} \ule{\textbf{na}}\\ 
      \oldstylenums{3}\textsc{p}+{\IFV}-{\PRG} \oldstylenums{3}\textsc{p}+{\IFV}-walk{}-{\CL} at bush {\PSP}\\  
\glt ‘[As] they were walking in the bush,’\\
      
      \medskip
tolo  tənjakay  agwazla  malan  ga  a  ləhe.\\      
\gll t\`{ə}-lɔ t\`{ə}-nzak-aj agʷaɮa malaŋ ga a lɪhɛ\\
     \oldstylenums{3}\textsc{p}+{\PFV}-go \oldstylenums{3}\textsc{p}+{\PFV}-find{}-{\CL} {spp. of tree}  large {\ADJ} at bush\\
\glt  ‘[As] they were walking in the bush, they went and found a large tree (a particular species) in the bush.’
\z

\ea \label{ex:11:13}
\underline{T\'ənday  táhaya  \textbf{na}},  həmbo  ga\\ 
\gll  \ulp{t\'ə-ndaj}{~~~~~~~~} \ulp{tá-h=aja}{~~~~~~~~~~~~~~~~} \ule{\textbf{na}} hʊmbɔ  ga \\ 
      \oldstylenums{3}\textsc{p}+{\IFV}-{\PRG}  \oldstylenums{3}\textsc{p}+{\IFV}-grind={\PLU}    {\PSP}     flour   {\ADJ} \\ 
\glt ‘They were grinding it, [and] the flour’ \\     
      
      \medskip
ánday  ásak  ele  ahan  wəsekeke. \\      
\gll á-ndaj á-sak   ɛlɛ      =ahaŋ  wuʃɛkɛkɛ\\
     \oldstylenums{3}\textsc{s}+{\IFV}-{\PRG}  \oldstylenums{3}\textsc{s}+{\IFV}-multiply   thing  =\oldstylenums{3}\textsc{s}.{\POSS}   \textsc{id}multiply\\
\glt  ‘was multiplying all by itself, \textit{wəsekeke}.’
\z

\ea \label{ex:11:14}
\corpussource{Disobedient Girl, S. 36}\\
\underline{Talay  war  elé  háy  bəlen  kə  ver  aka  \textbf{na}},  ásak  asabay. \\
\gll  \ulp{talaj}{~~~~~~} \ulp{war}{~~~} \ulp{ɛlɛ}{~~~~} \ulp{haj}{~~~~} \ulp{bɪlɛŋ}{~~~~} \ulp{kə}{~~~~~} \ulp{vɛr}{~~~~} \ulp{aka}{~~~~~}  \ule{\textbf{na}} á-sak  asa-baj\\
      \textsc{id}put  child  eye  millet  one  on  stone  on  {\PSP} \oldstylenums{3}\textsc{s}+{\IFV}-multiply  again-{\NEG}\\
\glt  ‘[If] they put one grain on the grinding stone, it doesn’t multiply anymore.’
\z

\largerpage
A \textit{na}{}-marked clause in Moloko can function adverbially, because it is marked as subordinate (in a way) to the main clause, but it gives no explicit signal as to the nature of the sematic relationship between the two clauses.  The only thing it indicates is that the \textit{na}{}-marked clause is presented as presupposed, and somehow relevant to the following clause. The relations that \textit{na} clauses are employed in are temporal or logical sequence (see \sectref{sec:11.1.1}), simultaneous or coordinated events (see \sectref{sec:11.1.2}), and tail-head linking for cohesion (see \sectref{sec:11.1.3}).

\subsection{Temporal or logical sequence}\label{sec:11.1.1}
\hypertarget{RefHeading1213181525720847}{}
The default relation between a \textit{na}-marked clause and the matrix clause in a \textit{na} construction is that there is a sequence (temporal or logical) and the event/state expressed by the \textit{na}{}-marked clause precedes the event/state in the main clause. \REF{ex:11:15} and \REF{ex:11:16} are both taken from a Moloko legend (from the Leopard story,\citealt{Friesen2003}) where some domestic animals are fleeing their owners because the owners are constantly killing the animals’ children in order to satisfy the demands of the spirits. \REF{ex:11:15} shows a reason-result construction.\footnote {It is also an example of tail-head linking, see \sectref{sec:11.1.3}.}  A hen begins the story with her lament expressing the reason why she is fleeing. She first states, “They have killed my children," then uses a \textit{na} construction to say that \textit{because} they have killed her children, she is fleeing in anger. The \textit{na}{}-marked clause repeats the information she just declared in the first clause. This now presupposed information (‘they are killing my children’) is followed by the matrix clause containing the assertion of new information (I am fleeing in anger). Connecting the two clauses in a presupposition-assertion construction influences the hearer to deduce a logical or temporal connection between the two clauses; here reason-result. 


\ea \label{ex:11:15}
Tanday taslaw aka babəza ahay va.\\   
\gll  ta-nd-aj    ta-ɬ=aw  =aka  babəza  =ahaj  =va \\
      \oldstylenums{3}\textsc{p}-{\PROG}-{\CL}  \oldstylenums{3}\textsc{p}-kill={\oneS}.{\IO}  =on  children  =Pl  {=\PRF} \\    
\glt ‘They have killed my children.’\\

      \medskip
\underline{Nde, taslaw  aka  babəza  ahay  va \textbf{na}},\\      
\gll \ulp{ndɛ}{~~~}  \ulp{ta-ɬ}{~~~~~}  \ulp{=aw}{~~~~} \ulp{=aka}{~~~~~~} \ulp{babəza}{~~~~}  \ulp{=ahaj}{~~~~}   \ulp{=va}{~~~~} \ule{\textbf{na}}\\
     so   \oldstylenums{3}\textsc{p}-kill  ={\oneS}.{\IO}  =on  children  =Pl  ={\PRF} {\PSP} \\
\glt ‘So, [because] they are killing my children,’\\
      
      \medskip
nəhəmay  mogo  ele  əwla.\\      
\gll nə-həm-aj  mɔgʷɔ  ɛlɛ    =uwla\\ 
     {\oneS}-run{}-{\CL}      anger  thing  ={\oneS}.{\POSS}\\
\glt  ‘I am running [in] anger.’ (lit. I am running my anger thing). 
\z

\largerpage
\REF{ex:11:16} shows a temporal sequence (or perhaps another reason-result construction) from a little later in the same legend. The group of animals is joined by a dog. The dog expresses that whenever a person in the family gets sick, the family will be advised to kill a dog, because dog meat is thought to be especially good to help a sick person get stronger. The dog’s speech uses a \textit{na} construction to express this relation. The \textit{na}-marked clause indicates the condition for the event expressed in the main clause. In this case the clause marked by \textit{na}{} (‘a person gets sick’) is not previously mentioned in the discourse, but rather is a fact of life, a cultural presupposition. 

\ea \label{ex:11:16}
\underline{Cəje  agan  ana  məze  \textbf{na}},  tawəy,  “Kəɗom  kəra."\\
\gll  \ulp{tʃɪdʒɛ}{~~~~~}   \ulp{a-g}{~~~~~}  \ulp{=aŋ}{~~~~~}   \ulp{ana}{~~~~~}   \ulp{mɪʒɛ}{~~~~~}   \ule{\textbf{na}}   tawij   kʊɗ-ɔm     kəra\\
      disease  \oldstylenums{3}\textsc{s}-do  =\oldstylenums{3}\textsc{s}.{\IO}  {\DAT} person  {\PSP}  \oldstylenums{3}\textsc{p}+said  kill[{\IMP}]-{\twoP}  dog\\
\glt  ‘[If] a person gets sick (lit. sickness does to person), they say, “Kill a dog!” [for the sick person to eat].’ 
\z

\REF{ex:11:17} and \REF{ex:11:18} are from another legend that talks about how God used to live very close to people. However one day, a woman did something that made God angry, and so he moved far away from them. The narrator expresses the relation between God becoming angry and his moving away using a \textit{na} construction \REF{ex:11:17} where the \textit{na}-marked clause indicates God’s anger (the reason for his leaving) and the main clause indicates the result (he went away). 

\ea \label{ex:11:17}
\underline{Hərmbəlom  na  ɓərav  ahan  atəkam  alay  \textbf{na}},  avahay  ele  ahan  botot.\\
\gll  \ulp{Hʊrmbʊlɔm}{~~~~~} \ulp{na}{~~~~~}  \ulp{ɓərav}{~~~~~}   \ulp{=ahaŋ}{~~~~~} \ulp{a-təkam}{~~~~~} \ulp{=alaj}{~~~~~}  \ule{\textbf{na}} a-vah-aj ɛlɛ\\ 
      God  {\PSP}  heart =\oldstylenums{3}\textsc{s}.{\POSS} \oldstylenums{3}\textsc{s}-taste  =away {\PSP} \oldstylenums{3}\textsc{s}-fly{}-{\CL}  thing\\  
      
      \medskip
\gll =ahaŋ bɔtɔt\\      
     =\oldstylenums{3}\textsc{s}.{\POSS}  \textsc{id}flying\\ 
\glt  ‘God (for his part) got angry; [and so] he went away.’ (lit. God, he tasted his heart, he flew his thing)
\z

\REF{ex:11:18} is from the conclusion of the same legend where the narrator uses a \textit{na} construction to express a counterexpectation. Although people may seek paradise, they won’t find it because God has gone far away (because of what the woman did). In the \textit{na} construction, the \textit{na}{}-marked clause expresses what people seek, and the main clause expresses that they won’t find it.

\clearpage
\ea \label{ex:11:18}
\underline{Mənjokok  egəne  sləlay  mbəlom  \textbf{na}},  Hərmbəlom  enjé  dəren.\\
\gll  \ulp{mə-nzɔk-ɔkʷ}{~~~~~~~~~~~~}  \ulp{ɛgɪnɛ}{~~~~~}  \ulp{ɬəlaj}{~~~~~}  \ulp{mbəlɔm}{~~~~~}    \ule{\textbf{na}}  Hʊrmbʊlɔm ɛ-nʒ-ɛ\\ 
      \oldstylenums{1}\textsc{Pin}{}-seek/find-\oldstylenums{2}\textsc{Pin}  today  root  sky    {\PSP}  God  \oldstylenums{3}\textsc{s}-left-{\CL}\\  
      
      \medskip
\gll dɪrɛŋ\\      
     far\\ 
\glt  ‘[Although] today we seek paradise, God has gone far away.’ (lit. we seek today the root of the sky, God has gone far away.)
\z

\REF{ex:11:19} is from the Values exhortation and illustrates a reason-result connection. There is no connecting conjunction in either of the clauses; however the reader can discern that there is a logical connection between the first clause ‘[If] you will ever accept the word of God' (marked in five places with \textit{na,} see \sectref{sec:11.5}) and the second ‘whose word will you accept [then]?' (a rhetorical question, see \sectref{sec:10.3.4}).


\ea \label{ex:11:19}\corpussource{Values, S. 29}\\
\underline{Hərmbəlom  \textbf{na}},  \underline{amaɗaslava  ala  məze  \textbf{na}},  \underline{ndahan  ese  \textbf{na}},\\    
\gll  \ulp{Hʊrmbʊlɔm}{~~~~~}  \ule{\textbf{na}} \ulp{ama-ɗaɬ=ava=ala}{~~~~~}  \ulp{mɪʒɛ}{~~~~~}   \ule{\textbf{na}} \ulp{ndahaŋ}{~~~~~}  \ulp{ɛʃɛ}{~~~~~}  \ule{\textbf{na}}\\ 
      God  {\PSP} {\DEP}-multiply=in=to   person {\PSP} \oldstylenums{3}\textsc{s}     again    {\PSP}\\    
\glt ‘God, the one who multiplied the people, him again’ \\
      
      \medskip
\underline{kagas  ma  Hərmbəlom  \textbf{na}},  \underline{asabay  \textbf{na}},\\      
\gll \ulp{ka-gas}{~~~~~}    \ulp{ma}{~~~~~}   \ulp{Hʊrmbʊlɔm}{~~~~~} \ule{na} \ulp{asa-baj}{~~~~~} \ule{\textbf{na}}\\ 
     {\twoS}-catch   word     God {\PSP}  again-{\NEG}  {\PSP} \\
\glt ‘[if] you no longer accept the word of God,’\\ 

      \medskip
[káagas  \textbf{na}  anga  way?]  \\      
\gll [{káá-gas} \textbf{na} aŋga waj] \\ 
     {\twoS}+{\POT}-catch  {\PSP}   {\POSS}  who  \\
\glt  ‘[then] you will never accept anyone's word.' (lit. whose [word] will you accept?)\\
\z

\subsection{Simultaneous events}\label{sec:11.1.2}\is{Tense, mood, and aspect!Progressive|(}
\hypertarget{RefHeading1213201525720847}{}
When the verb in the \textit{na} clause is progressive aspect, the events/states in both clauses are simultaneous. \REF{ex:11:20} (from the Leopard story, \citealt{Friesen2003}) shows a \textit{na} clause that indicates a presupposed event that is occuring while the event in the main clause happens.\footnote{\REF{ex:11:20} is an example of tail-head linking (\sectref{sec:11.1.3}) where the example is repeated.} The verb \textit{anday etəwe}  ‘she is crying’ is progressive aspect. Also see \REF{ex:11:12}, \REF{ex:11:13}. 

\ea \label{ex:11:20}
Atəwalay  “Bababa  kəlak  kəlak  kəlak.”  \underline{Anday  etəwe  \textbf{na}},  anjakay  awak.\\
\gll  a-tuw=alaj  {bababa  kəlak kəlak kəlak}  \ulp{a-ndaj}{~~~~~~}  \ulp{ɛ-tuw-ɛ}{~~~~~} \ule{\textbf{na}} {a-nzak-aj}\\ 
      \oldstylenums{3}\textsc{s}-cry=away    {sound of hen}      \oldstylenums{3}\textsc{s}-{\PRG}  \oldstylenums{3}\textsc{s}-cry-{\CL}  {\PSP}  \oldstylenums{3}\textsc{s}-find{}-{\CL}\\  
     
      \medskip
\gll {awak}\\      
     goat\\
\glt  ‘She cried, “\textit{Bababa kəlak kəlak kəlak}.” As she was crying, she found a goat.’ 
\z
\is{Tense, mood, and aspect!Progressive|)}
\subsection{Tail-head linking for cohesion}\label{sec:11.1.3}\is{Cohesion!Tail-head linking}
\hypertarget{RefHeading1213221525720847}{}
In a discourse, the speaker will use several devices to ensure that the hearers can follow what is being said; i.e., to help track participants through the narrative, connect events, and understand logical connections. One of the ways cohesion is achieved in Moloko discourse is by the use of the presupposition marker \textit{na} to mark presupposed (including previously-introduced) information. Cohesion is also created using a special construction that Longacre calls “tail-head repetition” \citep[204]{Longacre1976}. In this construction, an element previously mentioned in a discourse is repeated in a subsequent sentence in order to provide a cohesive link between new information and the preceding discourse. In Moloko, a clause on the eventline is first asserted and then at the beginning of the next sentence the same propositional content may be repeated almost word for word and marked at the end by \textit{na}. Several examples are shown below. \REF{ex:11:21} comes from a different retelling of the Disobedient Girl text than is shown in \sectref{sec:1.5}. The final element of \textit{təhaya na kə ver aka} ‘they ground it on the grinding stone’ is repeated in the next line and marked with \textit{na} as the first element of the next sentence \textit{tənday táhaya na} ‘they were grinding it \textit{na}.’ In (\ref{ex:11:21}--\ref{ex:11:26}), the clause containing the element to be repeated is delimited by square brackets and the \textit{na}{}-marked clause in the next sentence is underlined. The element that is repeated in both clauses is bolded. 

\ea \label{ex:11:21}
Tázaɗ  na  háy,  war  elé  háy  bəlen  na,\\  
\gll  tá-zaɗ  na  haj,  war ɛlɛ   haj  bɪlɛŋ  na \\       
      \oldstylenums{3}\textsc{p}+{\IFV}-take  \oldstylenums{3}\textsc{s}.{\DO}  millet  child  eye   millet  one {\PSP} \\         
\glt ‘They would take one grain of millet;’\\

\medskip 
[\textbf{t\'əhaya} na kə  ver  aka.]\\  
\gll [\textbf{t\'ə-h}  \textbf{=aja}  na kə  vɛr  aka] \\
     \oldstylenums{3}\textsc{s}+{\IFV}-grind ={\PLU}   \oldstylenums{3}\textsc{s}.{\DO} on stone  on\\
\glt ‘they ground it on the grinding stone.’\\

\medskip
\underline{T\'ənday  \textbf{táhaya} na},\\
\gll \ulp{t\'ə-ndaj}{~~~~~~~~} \ulp{\textbf{tá-h}}{~~~~~~~~~~~~~~~}  \ulp{\textbf{=aja}}{~~~~~} \ule{na}\\
     \oldstylenums{3}\textsc{p}+{\IFV}-{\PRG}  \oldstylenums{3}\textsc{p}+{\IFV}-grind  ={\PLU}    {\PSP}\\
\glt  ‘As they were grinding it,’\\ 

\medskip
 həmbo  ga  ánday  ásak  ele  ahan  wəsekeke. \\
\gll hʊmbɔ ga á-ndaj á-sak ɛlɛ =ahaŋ wuʃɛkɛkɛ\\
     flour   {\ADJ}  \oldstylenums{3}\textsc{s}+{\IFV}-{\PRG}  \oldstylenums{3}\textsc{s}+{\IFV}-multiply    thing  =\oldstylenums{3}\textsc{s}.{\POSS}  \textsc{id}multiply\\
\glt ‘the flour was multiplying all by itself \textit{wəsekeke}.’
\z

Another tail-head link can be seen a little further in the same narrative in \REF{ex:11:22}. 

\ea \label{ex:11:22}\relax
 [\textbf{Ánday  ásakaka}.]\\     
\gll  {}[\textbf{á-ndaj  á-sak  =aka}]\\  
      \oldstylenums{3}\textsc{s}+{\IFV}-{\PROG} \oldstylenums{3}\textsc{s}+{\IFV}-multiply  =on \\  
\glt ‘It is multiplying.’\\

\medskip
\underline{\textbf{Ánday  ásakaka}  wəsekeke  na},\\
\gll \ulp{\textbf{á-ndaj}}{~~~~~~~~~} \ulp{\textbf{á-sak}}{~~~~~~~~~~~~~~~~~~~} \ulp{\textbf{=aka}}{~~~~~} \ulp{wuʃɛkɛkɛ}{~~~~~~~~~~~~} \ule{na}\\ 
     \oldstylenums{3}\textsc{s}+{\IFV}-{\PRG}  \oldstylenums{3}\textsc{s}+{\IFV}-multiply =on  \textsc{id}multiplication  {\PSP} \\
\glt ‘As it is multiplying \textit{wəsekeke},’\\

\medskip
ver  árəhva  mbaf.\\
\gll vɛr á-rəh =va mbaf\\
     room  \oldstylenums{3}\textsc{s}+{\IFV}-fill   ={\PRF}  {up to the roof}\\
\glt  ‘the room filled completely up \textit{mbaf}.’
\z

Likewise, other tail head links can be seen in \REF{ex:11:23} (from lines 3-5 in the Cicada text), \REF{ex:11:24} (from lines 9-10 in the Snake story), and \REF{ex:11:25} (from the Leopard story, \citealt{Friesen2003}). 

\ea \label{ex:11:23}
\corpussource{Cicada, S. 3}\\
Albaya  ahay  aba.\\   
\gll  albaja  =ahaj  aba   \\
      youth   =Pl  {\EXT} \\
\glt ‘There were some young men.’\\ 

\medskip
\corpussource{Cicada, S. 4}\\\relax
[\textbf{Tánday  t\'ətalay  a  ləhe.}] \\
\gll  [\textbf{{tá-ndaj}} \textbf{{t\'ə-tal-aj}} \textbf{{a}} \textbf{{lɪhɛ}}]\\
      \oldstylenums{3}\textsc{p}+{\IFV}-{\PRG} \oldstylenums{3}\textsc{p}+{\IFV}-walk-{\CL}  to  bush\\
\glt  ‘They were walking in the bush.’\\

\medskip
\corpussource{Cicada, S. 5}\\
\underline{\textbf{Tánday  t\'ətalay  a  ləhe} na},  tolo  tənjakay  agwazla  malan  ga  a  ləhe.\\
\gll  \ulp{\textbf{tá-ndaj}}{~~~~~~~} \ulp{\textbf{t\'{ə}-tal-aj}}{~~~~~} \ulp{\textbf{a}}{~~~~~} \ulp{\textbf{lɪhɛ}}{~~~~~} \ule{na}\\ 
      \oldstylenums{3}\textsc{p}+{\IFV}-{\PRG} \oldstylenums{3}\textsc{p}-walk-{\CL}  at bush {\PSP}\\ 
\glt ‘[As] they were walking in the bush,’ \\     
      
      \medskip
      
\gll {tə-lɔ} {tə-nzak-aj} {agʷaɮa} {malaŋ} {ga} {a} {lɪhɛ}\\
     \oldstylenums{3}\textsc{p}+{\PFV}-go  \oldstylenums{3}\textsc{p}+{\PFV}-find{}-{\CL}  {spp. of tree}  large {\ADJ}      at  bush\\
\glt  ‘they went and found a large tree (a particular species) in the bush.’
\z

\ea \label{ex:11:24}\corpussource{Snake, S. 9}\\
Nazala  təystəlam  əwla.\\    
\gll  nà-z =ala  tijstəlam   =uwla  \\  
      {\oneS}+{\PFV}-take =to  torch   ={\oneS}.{\POSS} \\  
\glt ‘I took my flashlight.’\\

      \medskip
[\textbf{Nabay}  cəzlarr.]\\      
\gll [\textbf{nà-b-aj} tsəɮarr]\\
     {\oneS}+{\PFV}-light{}-{\CL}  {\textsc{id}shining the flashlight up}\\
\glt  ‘I shone it up \textit{cəzlarr}.’\\

\medskip
\corpussource{Snake, S. 10}\\
\underline{\textbf{Nábay}  na},  námənjar  na  mbajak  mbajak  mbajak\\  
\gll  \ulp{\textbf{ná-b-aj}}{~~~~~~~~~~~~~} \ule{na} ná-mənzar na {mbajak   mbajak    mbajak}\\ 
      {\oneS}+{\IFV}-light{}-{\CL}    \oldstylenums{3}\textsc{s}.{\DO}   {\oneS}+{\IFV}-see  {\PSP}  {\textsc{id}something big and reflective}\\   
\glt ‘[As] I shone [it], I was seeing it, something big and reflective \textit{mbajak},’\\
      
      \medskip
gogolvan.\\      
\gll {gʷɔgʷɔlvaŋ} \\
     snake\\
\glt  ‘a snake!’
\z

\ea \label{ex:11:25}
{}[\textbf{Atəwalay} “Bababa  kəlak  kəlak  kəlak."]\\  
\gll  {}[\textbf{a-tuw=alaj} {bababa  kəlak kəlak kəlak}]\\ 
      \oldstylenums{3}\textsc{s}-cry=away {sound of hen}\\ 
\glt ‘She cried, “\textit{Bababa kəlak  kəlak  kəlak}.”’

\medskip
\underline{Anday  \textbf{etəwe}  na},  anjakay  awak.\\
\gll \ulp{a-ndaj}{~~~~~} \ulp{\textbf{ɛ-tuw-ɛ}}{~~~~~} \ule{na} a-nzak-aj   awak\\
     \oldstylenums{3}\textsc{s}-{\PRG}  \oldstylenums{3}\textsc{s}-cry-{\CL}  {\PSP}  \oldstylenums{3}\textsc{s}-find-{\CL}  goat\\
\glt  ‘As she was crying, she found a goat.’ 
\z

\largerpage
Sometimes the tail and head elements are not identical. For example, the expected (but not overtly-named) result of a previous proposition can be expressed in a subsequent clause and that result marked with \textit{na}. \REF{ex:11:26} is from lines 27 and 28 of the Disobedient Girl text shown in Section \sectref{sec:1.5}. The first sentence (\textit{zar ahan angala}) tells of the return of the husband. The next sentence is \textit{pok mapalay mahay} ‘opening the door,’ which is an expected event when a person returns home. The \textit{na}{}-marked clause in the second sentence is presupposed information since although it does not literally repeat the information in the previous sentence, it refers to information which is a natural outcome of it. The construction still provides cohesion to the text because subsequent events are linked together. 

\ea \label{ex:11:26}
\corpussource{Disobedient Girl, S. 27}\\
{}[Embesen  cacapa  na,  \textbf{zar  ahan  angala.}]\\ 
\gll  {}[ɛ{}-mbɛʃɛŋ  tsatsapa  na, \textbf{zar} \textbf{=ahaŋ}  \textbf{à-ŋgala}]\\
      \oldstylenums{3}\textsc{s}-rest {some time} {\PSP}  man  =\oldstylenums{3}\textsc{s}.{\POSS} \oldstylenums{3}\textsc{s}+{\PFV}-return\\
\glt  'After a while, her husband came back.'\\
\medskip
\corpussource{Disobedient Girl, S. 28}\\
 \underline{Pok  mapalay  mahay  na},  həmbo  árah  na  a  hoɗ  a  hay  ava.\\
\gll  \ulp{pɔk}{~~~~~~~}  \ulp{ma-p=alaj}{~~~~~~~~~~~~}   \ulp{mahaj}{~~~~~} \ule{na}  hʊmbɔ  á-rax   na   a  hʷɔɗ\\ 
      \textsc{id}open  {\NOM}{}-open=away  door {\PSP}  flour \oldstylenums{3}\textsc{s}+{\IFV}-fill \oldstylenums{3}\textsc{s}.{\DO}  at  stomach\\ 
      
      \medskip
\gll a  haj    ava\\
     {\GEN}  house  in\\
\glt  ‘Opening the door, the flour filled the stomach (the interior) of the house.’
\z

\section{Presupposition-assertion construction: \textit{na}-marked clausal element}\label{sec:11.2}\is{Cohesion!Na-marking}
\hypertarget{RefHeading1213241525720847}{}
The second type of presupposition-assertion construction occurs when a single clausal element is fronted and marked with \textit{na}. \textit{Na} marks (occurs immediately after): a) presuppositions and b) topics (including contrastive topics). In both cases the clausal element immediately preceding \textit{na}  is part of an understood presupposition. The part of the clause following \textit{na} is the assertion which contains new information the speaker wants to communicate.

The normal order of elements in a Moloko clause (without \textit{na}) is SVO. \figref{fig:18}. illustrates the constituents in a declarative clause, combining \figref{fig:15}. and \figref{fig:16}. so that the verb phrase constituents are also shown.  

\begin{figure}
\resizebox{\textwidth}{!}{\noindent\frame{\begin{tabular}{llllllll}
(Discourse particle) &   (Subject NP) & \textbf{Verb phrase} & & & & & \\
(Temporal adverb)  & & & & & & & \\
& & (Auxiliary)  & \textbf{Verb complex} & (Noun phrase    &   (Adpositional phrases) & (Adverb) & (Ideophone\is{Ideophone})\\
& &            &                        & or ‘body-part’) & & &(Negative) \\
\end{tabular}}}
\caption{Constituents of the clause\label{fig:18}}
\end{figure}

In a presupposition-assertion construction, one (or more) of the clause or verb phrase elements is marked with \textit{na} and fronted with respect to the subject noun phrase and the verb phrase. The fronted construction is illustrated in \figref{fig:19}. 

\begin{figure}
\resizebox{\textwidth}{!}{\frame{\begin{tabular}{llll}
(Discourse particle
or temporal adverb) & Fronted element + \textit{na} & (Subject noun phrase) & \textbf{Verb phrase}\\
\end{tabular}}}
\caption{Constituent order of Presupposition construction\label{fig:19}}
\end{figure}

The examples below show the presupposed element can be almost any element of the clause: the subject (\ref{ex:11:27}--\ref{ex:11:28}), the direct object (\ref{ex:11:29}--\ref{ex:11:30}), or an oblique (\ref{ex:11:31} and \ref{ex:11:32}). A discourse conjunction or temporal can also be marked as being presupposed (\ref{ex:11:33}--\ref{ex:11:35}). In each case, the fronted element is presupposed in the discourse -- it is a known or culturally expected participant, location (spatial or temporal), or object. It is noteworthy that neither verbs by themselves, nor an existential word, nor ‘body-part’ incorporated nouns, nor ideophones can be fronted and marked as presupposed. In the following examples, the presupposed element is underlined and the presupposition marker \textit{na} is bolded.  The \textit{na}-marked element and the assertion are marked in \REF{ex:11:27}.
 \\

%%\todo[inline]{How is this to be type set? Please check the pdf. It should be like a corpussource but the 'Na-marked element' should line up with 'Kəlen bahay na' in the example line and 'Assertion' should line up with 'olo  kə  mətəɗe  aka'}

\textit{Na}-marked element  \hspace{40pt}    Assertion
\ea \label{ex:11:27}
\corpussource{Cicada, S. 19}\\
Kəlen  \underline{bahay  \textbf{na}}, \hspace{30pt}  olo  kə  mətəɗe  aka.\\
\gll  kɪlɛŋ  \ulp{bahaj}{~~~~~}  \ule{\textbf{na}} \hspace{35pt}   \`{ɔ}-lɔ          kə  mɪtɪɗɛ  aka\\
      then    chief {\PSP} {}   \oldstylenums{3}\textsc{s}+{\PFV}-go        on  cicada  on\\
\glt  ‘Then the chief, he went to the cicada.’  
\z

\ea \label{ex:11:28}
\underline{Həmbo  \textbf{na}},  anday  ásəkala  azla  wəsekeke.\\
\gll  \ulp{hʊmbɔ}{~~~}  \ule{\textbf{na}}    à-ndaj     á-sək    =ala      aɮa   wuʃɛkɛkɛ\\
      flour  {\PSP}  \oldstylenums{3}\textsc{s}+{\PFV}-{\PRG}   \oldstylenums{3}\textsc{s}+{\IFV}-multiply  =to  now    \textsc{id}multiplication\\
\glt  ‘The flour, it is multiplying \textit{wəsekeke}.’
\z

\ea \label{ex:11:29}
\underline{Ele  ahay  nendəye  \textbf{na}},  nagala  kəyga  bay.\\
\gll  \ulp{ɛlɛ}{~~~~~} \ulp{=ahaj}{~~~~~}   \ulp{nɛndijɛ}{~~~~~} \ule{\textbf{na}}  nà-g    =ala  kijga  baj\\
      thing  =Pl        {\DEM}     {\PSP}  {\oneS}+{\PFV}-do  =to     {like this}  {\NEG}\\
\glt ‘These things, I have never done like this.’
\z

\ea \label{ex:11:30}
\underline{Ne  \textbf{na}},  kónjokom  ne  asabay. \\
\gll  \ulp{nɛ}{~~~~~} \ule{\textbf{na}}    k\'ɔ-nz\'ɔk-\'ɔm  nɛ  asa-baj \\
      {\oneS}    {\PSP}    {\twoP}+{\IFV}-find-{\twoP}  {\oneS}  again-{\NEG}\\
\glt  ‘[As for] me, you will never find me again.’
\z

\ea \label{ex:11:31}
\corpussource{Cicada, S. 18}\\
\underline{Kə  mahay  aka  \textbf{na}}, \underline{námbasaka  \textbf{na}},  mama  agwazla  səlom  ga  lala.\\
\gll  \ulp{kə}{~~~~~}  \ulp{mahaj}{~~~~~}    \ulp{aka}{~~~~~}  \ule{\textbf{na}} \ulp{ná-mbas}{~~~~~} \ulp{=aka}{~~~~~} \ule{\textbf{na}}  mama   agʷaɮa    sʊlɔm  ga \\    
      on  door  on  {\PSP}  {\oneS}+{\IFV}-rest  =on    {\PSP}  mother  {spp. of tree}  good   {\ADJ} \\  
      
      \medskip
\gll lala\\
     well\\
\glt  ‘By my door, I will be able to rest well; the mother tree [is] good.’
\z

\ea \label{ex:11:32}
\corpussource{Values, S. 13}\\
\underline{A  məsəyon  ava  \textbf{na}},  \underline{ele  ahay  aməwəsle  \textbf{na}},  tege  bay.  \\
\gll  \ulp{a}{~~~~~}   \ulp{mʊsijɔŋ}{~~~~~}   \ulp{ava}{~~~~~}   \ule{\textbf{na}}   \ulp{ɛlɛ}{~~~~~}   \ulp{=ahaj}{~~~~~}   \ulp{amu-wuɬ-ɛ}{~~~~~} \ule{\textbf{na}}   tɛ-g-ɛ    baj  \\
      at  mission  in  {\PSP}  thing  =Pl  {\DEP}-forbid-{\CL}  {\PSP}  \oldstylenums{3}\textsc{p}-do-{\CL}  {\NEG}\\
\glt  ‘In the church, these things that are forbidden, they don’t do.’
\z

Although the presupposition-assertion construction is structurally a clause level phenomenon, it can function in information structuring at the proposition level both to mark a boundary in a discourse, to set topic, and in participant tracking\is{Cohesion!Participant tracking}. When a discourse conjunction or temporal adverb is marked as presupposed (\ref{ex:11:33}--\ref{ex:11:35}, see also \ref{ex:11:49} from \sectref{sec:11.5}), the clause as a whole marks a boundary in the discourse. Such a clause often indicates a time change or an episode boundary. Most of the episodes in the Disobedient Girl story (see \sectref{sec:1.5}) begin with a conjuction marked with \textit{na} \REF{ex:11:34} or a \textit{na}-marked temporal phrase (\ref{ex:11:33}, \ref{ex:11:35}).  All \textit{na}-marked elements are underlined in the examples. 

\ea \label{ex:11:33}
\corpussource{Disobedient Girl, S. 3 (the beginning of the setting)}\\
\underline{Zlezle  \textbf{na}}, \underline{Məloko  ahay  \textbf{na}},  Hərmbəlom  ávəlata  barka  va.  \\
\gll  \ulp{ɮɛɮɛ}{~~~~~~~~~~~}  \ule{\textbf{na}} \ulp{Mʊlɔkʷɔ}{~~~~~}  \ulp{=ahaj}{~~~~~}  \ule{\textbf{na}}  Hʊrmbʊlɔm    á-vəl=ata             barka=va  \\
      {long ago}  {\PSP}  Moloko      =Pl   {\PSP}  God    \oldstylenums{3}\textsc{s}+{\IFV}-send=\oldstylenums{3}\textsc{s}.{\IO}  blessing={\PRF}\\ 
\glt  ‘Long ago, to the Moloko people, God had given to them his blessing.’
\z

 
\ea \label{ex:11:34}
\corpussource{Disobedient Girl, S. 9 (the beginning of episode 1)}\\
\underline{Nde  ehe  \textbf{na}},  albaya  ava  aba. \\  
\gll  \ulp{ndɛ}{~~~~~}     \ulp{ɛhɛ}{~~~~~} \ule{\textbf{na}}   albaja           ava    aba\\
      so     here   {\PSP}   {young man}  {\EXT}+in   {\EXT}  \\
\glt  ‘And so, there once was a young  man.’ 
\z

\ea \label{ex:11:35}
\corpussource{Disobedient Girl, S. 27 (the beginning of the dénouement)}\\
\underline{Embesen  cacapa  \textbf{na}}, zar  ahan  angala.\\
\gll  \ulp{\`ɛ-mbɛʃɛŋ}{~~~~~}  \ulp{tsatsapa}{~~~~~~~~~~~~~~}     \ule{\textbf{na}} zar    =ahaŋ       à-ŋgala\\
      \oldstylenums{3}\textsc{s}-rest  {after some time}   {\PSP}   man  =\oldstylenums{3}\textsc{s}.{\POSS}  \oldstylenums{3}\textsc{s}+{\PFV}-return\\
\glt  ‘After a while, her husband came back.’ 
\z

The presupposition-assertion construction is also used to mark topic for participant shifts.\footnote{Called ‘subject’ in \citet{Chafe1976}.} The \textit{na}{}-marked element will be the main participant of the clauses that follow it, until there is another \textit{na}{}-marked clause-initial element. \citet[151]{Lambrecht1994} says,  

\begin{quote}
“what is presupposed in a topic-comment relations is not the topic itself, nor its referent, but the fact that topic referent can be expected to play a role in a given proposition, due to its status as a center of interest or matter of concern in the conversation. It is this property that most clearly distinguishes topic arguments from focus arguments, whose role in the proposition is always unpredictable at the time of utterance…One therefore ought not to say that a topic referent “is presupposed” but that, given its discourse status, it is presupposed to play a role in a given proposition.” 
\end{quote}

\textit{Na} can be thought of as a kind of spotlight, drawing attention to that already-known participant as one to which new or asserted information will be somehow related. \REF{ex:11:36} shows lines S. 12, 14, and 15 in the Disobedient Girl text. In S. 12, \textit{zar ahan} ‘her husband’ is marked with \textit{na}.\footnote{The double \textit{na}{}-marked elements \textit{senala na} ‘later’ and \textit{zar ahan na} ‘her husband’ function to build up tension (see \sectref{sec:11.5} for further discussion).} He is the subject of all of the clauses until \textit{hor} ‘the woman’ is marked with \textit{na} in S.14. Then, the woman is the subject of all the clauses until the flour is marked with \textit{na} in S.23. \textit{Na}{}-marking thus functions here in shifting the spotlight from one participant as topic to another. In these examples, only the \textit{na}{}-marked participants are underlined. 

 
\ea \label{ex:11:36}
\corpussource{Disobedient Girl, S. 12}\\
Sen  ala  na,  \underline{zar  ahan \textbf{na}},  dək  medakan  na  mənəye  ata.  \\
\gll  ʃɛŋ     =ala   na  \ulp{zar}{~~~~~}  \ulp{=ahaŋ}{~~~~~}    \ule{\textbf{na}}      dək    mɛ-dak=aŋ na\\ 
      \textsc{id}go   =to  {\PSP}  man    =\oldstylenums{3}\textsc{s}.{\POSS}    {\PSP}      \textsc{id}show   {\NOM}-show=\oldstylenums{3}\textsc{s}.{\IO}   \oldstylenums{3}\textsc{s}.{\DO}\\   
      
      \medskip
\gll  mɪ-nʒ-ijɛ     =atəta\\
      {\NOM}{}-sit-{\CL}   =\oldstylenums{3}\textsc{p}.{\POSS}\\
\glt  ‘Then her husband instructed her in their habits.’  (lit. going, her husband instructing their sitting)    
\z

\ea \label{ex:11:37}
\corpussource{Disobedient Girl, S. 14-15}\\
\underline{Hor  \textbf{na}},  ambəɗan  aka  awəy, “Ayokon  zar  golo."\\
\gll  \ulp{hʷɔr}{~~~~~}  \ule{\textbf{na}}  a-mbəɗ=aŋ        =aka       awij    ajɔkʷɔŋ  zar  gʷɔlɔ\\
      woman {\PSP}  \oldstylenums{3}\textsc{s}-change=\oldstylenums{3}\textsc{s}.{\IO}    =on      said  agreed   man  {\VOC}\\
\glt  ‘The woman replied. She said, “Yes, my dear husband.”’
\z

Marking with \textit{na} can also mark contrastive topic; i.e., a section of discourse will be ‘about’ that participant, instead of whatever the preceding section of discourse was about. \REF{ex:11:38}, which comes from a Moloko song, marks a participant shift but also functions to contrast the speaker’s situation with others just mentioned in the discourse.\footnote{This is called ‘contrastiveness’ in \citet{Chafe1976}.} 

\ea \label{ex:11:38}
Ndam  akar  ahay  ténje  a  avəya  ava.\\  
\gll  ndam akar  =ahaj  tɛ-nʒ{}-ɛ    a  avija    ava \\ 
      people  theft  =Pl  \oldstylenums{3}\textsc{p}+{\IFV}-sit-{\CL}  at  suffering  in\\ 
      \glt ‘(On that day) thieves will be in suffering;’\\
      
      \medskip
\underline{Ne  \textbf{na}},  nénje  nə  memle  ga.\\
\gll \ulp{nɛ}{~~~~~} \ule{\textbf{na}} n\'ɛ-nʒ{}-ɛ     nə  mɛmlɛ   ga\\
     {\oneS} {\PSP} {\oneS}+{\IFV}-sit-{\CL}  with  joy  {\ADJ}\\
\glt  ‘[but] as for me, I will rest in joy.’
\z

\section{Assertion-presupposition construction: right-shifted \textit{na}-marked element}\label{sec:11.3}
\hypertarget{RefHeading1213261525720847}{}
The assertion-presupposition construction occurs when the (\textit{na}{}-marked) presupposed element is placed after the main clause. This construction is found in concluding statements that explain what has happened in a discourse.\footnote{It is also seen in some information questions \sectref{sec:10.3.1}.} In \REF{ex:11:39}, from the concluding lines of a narrative, the \textit{na}{}-marked elements that occur in a dependent clause that occurs after the matrix clause explain the problem that the discourse deals with -- the fact that cows have destroyed a field.\footnote{Note that the other two occurrences of \textit{na} in this example function in focus (\sectref{sec:11.5}) and definiteness (\sectref{sec:11.4}), respectively.} 

\ea \label{ex:11:39}
Kógom  ala  na  memey,  \underline{sla  ahay  na  aməzəme  gəvah  \textbf{na}}.\\
\gll  k\'ɔ-gʷ{}-ɔm     =ala  na      mɛmɛj   \ulp{ɬa}{~~~~~}   \ulp{=ahaj}{~~~~~}   \ulp{na}{~~~~~}   \ulp{amɪ-ʒʊm-ɛ}{~~~~~}   \ulp{gəvax}{~~~~~}   \ule{\textbf{na}}\\
      2+{\IFV}-do-\oldstylenums{1}\textsc{Pin}  =to  {\PSP}   how   cow  =Pl  {\PSP}  {\DEP}-eat{}-{\CL}  field  {\PSP}\\
\glt  ‘What are you going to do [since] the cows ate up the field?’ (lit. you will do how, the cows having eaten the field)
\z

In \REF{ex:11:40}, the \textit{na}-marked final element is a relative clause explaining the main point of the narrative -- that the woman had brought a curse onto the Moloko people by what she had done. 

\ea \label{ex:11:40}
\corpussource{Disobedient Girl, S. 38}\\
Metesle  anga  war  dalay  ngəndəye, \\ 
\gll  mɛ-tɛɬ-ɛ  aŋga war dalaj  ŋgɪndijɛ \\ 
      {\NOM}{}-curse-{\CL}   {\POSS}   child  girl  {\DEM} \\
\glt ‘The curse [is] belonging to that young woman,’\\      
      
      \medskip
\underline{amazata  aka  ala  avəya  nengehe  ana  məze  ahay  \textbf{na}}. \\      
\gll \ulp{ama-z=ata}{~~~~~~~~~} \ulp{=aka}{~~~~~}  \ulp{=ala}{~~~~~} \ulp{avija}{~~~~~~~} \ulp{nɛŋgɛhɛ}{~~~~~}  \ulp{ana}{~~~~~} \ulp{mɪʒɛ}{~~~~~}  \ulp{=ahaj}{~~~~~} \ule{\textbf{na}}\\
     {\DEP}-take=\oldstylenums{3}\textsc{p}.{\IO} =on   =to suffering {\DEM}      {\DAT}  person    =Pl   {\PSP}\\
\glt  ‘the one that brought this suffering onto the people.’  
\z

\section{Definite construction: \textit{na}-marked clausal element}\label{sec:11.4}
\hypertarget{RefHeading1213281525720847}{}
The Definite construction occurs when a non-fronted noun phrase is marked by \textit{na}. \figref{fig:18} (from \sectref{sec:11.2}) shows the default order of constituents in a clause. In the definite construction, the \textit{na}–marked element is in its normal clausal position. In this construction, \textit{na}  functions in the realm of definiteness. Definiteness\is{Deixis!Definiteness|(} is defined by \citet[79]{Lambrecht1994} as signalling when “the referent of a phrase is assumed by the speaker to be identifiable to the addressee.” While definiteness is a separate function than presupposition, Lambrecht points out that definiteness is related to presupposition in that the definite article is a grammatical symbol for an assumption on the speaker’s part that the hearer is able to identify the definite element in a sentence -- the speaker presupposes that the addressee can identify the referent designated by that noun phrase.  

In \REF{ex:11:41} from the \textit{Cows in the Field} story, the \textit{na} marker is attached to the noun \textit{gəvah} ‘field’ within an adpositional phrase. This construction is simply identifying the field to be the one that the cows destroyed, definite and previously mentioned in the story, and not some other unidentified field. In the examples in this section, the \textit{na}–marked noun phrase is underlined and the adpositional phrase is delimited by square brackets.

\ea \label{ex:11:41}
Təzlərav  ta  ala  va  [a  \underline{gəvah  \textbf{na}}  ava.]\\
\gll  t\`ə-ɮərav     ta  =ala  =va    [a  \ulp{gəvax}{~~~~~}  \ule{\textbf{na}}  ava]\\
      {\oldstylenums{3}\textsc{p}+{\PFV}-move out}   \oldstylenums{3}\textsc{p}.{\DO} =to ={\PRF}   at  field   {\PSP}  in\\
\glt  ‘They had driven them out of the field.’
\z

\REF{ex:11:42} is  from the Disobedient Girl story. Her house is marked as definite with \textit{na}. 

\ea \label{ex:11:42}
\corpussource{Disobedient Girl, S. 26}\\
Nata  ndahan  dəɓəsolək  məmətava  alay\\  
\gll  nata    ndahaŋ  dʊɓʊsɔlʊk  mə-mət=ava =alaj \\   
      {and then}  \oldstylenums{3}\textsc{s} \textsc{id}collapse/die  {\NOM}{}-die=in =away \\
\glt ‘And she collapsed \textit{dəɓəsolək}, dying’\\      
      
      \medskip
a  hoɗ  [a  \underline{hay  \textbf{na}}  ava.]\\      
\gll a     hʷɔɗ      [a  \ulp{haj}{~~~~~}       \ule{\textbf{na}}      ava]\\
     at   stomach {\GEN} house   {\PSP}    in\\
\glt  ‘inside the house.'
\z

Likewise in \REF{ex:11:43}, the noun \textit{məsəyon} ‘church’ is marked as definite within the adpositional phrase \textit{a məsəyon na ava} ‘in the church.’ 

\newpage 
\ea \label{ex:11:43}
\corpussource{Values, S. 3}\\
Səwat  na,  təta  [a  \underline{məsəyon  \textbf{na}}  ava]  nəndəye  na,\\  
\gll  suwat   na   təta   [a   \ulp{mʊsijɔŋ}{~~~~~}   \ule{\textbf{na}}  ava]     nɪndijɛ  na  \\     
      \textsc{id}disperse  {\PSP}  \oldstylenums{3}\textsc{p}      at  mission  {\PSP}    in  {\DEM}  {\PSP}  \\    
      \glt ‘As the people go home from church,' (lit. disperse, they in the mission there)\\
 
 
      \medskip
pester  áhata,   “Ey, ele  nehe  na,  kógom  bay!” \\      
\gll pɛʃtɛr  á-h    =ata   ɛj ɛlɛ nɛhɛ na   k\'{ɔ}-gʷ-ɔm    baj\\
     pastor  \oldstylenums{3}\textsc{s}+{\IFV}-tell  =\oldstylenums{3}\textsc{p}.{\IO}   hey    thing  {\DEM}  {\PSP}  2+{\IFV}-do-{\twoP}    {\NEG}\\
\glt  ‘the Pastor told them, “Hey! These things here, don’t do them!”’
\z

\REF{ex:11:44} is from line S. 21 of the Snake story. The \textit{na}–marked element \textit{gogolvan  na} ‘the snake’ follows the verb complex in its normal position of a direct object noun phrase within the verb phrase. 

\ea \label{ex:11:44}
\corpussource{Snake, S. 21}\\
Alala,  nəzlərav  na  ala  \underline{gogolvan  \textbf{na}}  a  amata  ava.\\
\gll  a-l  =ala  n\`ə-ɮərav  na =ala  \ulp{gʷɔgʷɔlvaŋ}{~~~~~}  \ule{\textbf{na}}    a  amata   ava\\
      \oldstylenums{3}\textsc{s}-go =to  {\oneS}+{\PFV}-exit   \oldstylenums{3}\textsc{s}.{\DO}  =to  snake        {\PSP}    at   outside  in\\
\glt  ‘Sometime later I took the snake outside.’
\z

\section{Presupposition-focus construction: \textit{na} preceeds the final element of the verb phrase}\label{sec:11.5}\sectionmark{Presupposition-focus construction}
\hypertarget{RefHeading1213301525720847}{}
\is{Deixis!Definiteness|)}
The presupposition-focus construction in Moloko makes prominent the final element of a clause.\footnote{\citet[221]{LongacreHwang2012} define prominence as “spotlighting, highlighting, or drawing attention to something.”} \textit{Na} preceeds the final element in the verb phrase. This is the only \textit{na} construction where the \textit{na}{}-marker follows the verb complex but is not clause final. In effect, all of that information that precedes the final element in the clause is marked as presupposed with \textit{na}. The result is that the final element in the clause is highlighted in the discourse.

\REF{ex:11:45} is from line S. 20 of the Disobedient Girl text. The placement of \textit{na} postverbally, just before the final element in the verb phrase (\textit{gam} ‘a lot’) functions to highlight that the woman prepared \textit{a lot} of millet. The fact that she prepared a lot of millet instead of just one grain (as she was instructed) is critical to the outcome of the story. An added effect of the \textit{na} plus pause before the final element is to slow down the narrative just a bit\is{Focus and prominence!Discourse peak}, resulting in heightened attention on the final element \textit{gam} ‘a lot.’ In the examples in this section, the prominent final element is bolded and the \textit{na}-marked elements are underlined.


\ea \label{ex:11:45}\corpussource{Disobedient Girl, S. 20}\\
\underline{Jo  madala  háy  \textbf{na}},  \textbf{gam}.\\
\gll  \ulp{dzɔ}{~~~~~~~~}  \ulp{ma-d=ala}{~~~~~~~~~~~~~}  \ulp{haj}{~~~~~} \ule{\textbf{na}}  \textbf{gam}\\
      \textsc{id}take  {\NOM}-prepare=to  millet  {\PSP} {a lot}\\
\glt  ‘She prepared lots of millet.'
\z

Multiple elements in a clause or sentence that are marked with \textit{na} will add even more prominence to the final element. This kind of construction is seen at summation points in a narrative. It is also seen in a hortatory text when the speaker is reiterating his or her argument to make an important point. The many marked elements slow down the discourse and build up tension towards the final element in the clause, thus putting even more emphasis on the focused item. In \REF{ex:11:46}, the fact that the woman’s habit where she came from was to grind a \textit{large amount} of millet at a time is crucial to the story. Three \textit{na}-marked elements (a subject noun phrase, the verb phrase, and the complement without its final element) precede the adverb \textit{gam} ‘a lot.’


\ea \label{ex:11:46}\corpussource{Disobedient Girl, S. 17}\\
Nde  \uline{hor  \textbf{na}},  \uline{asərkala  afa  təta  va  \textbf{na}},\\
\gll  ndɛ  \ulp{hʷɔr}{~~~~~}  \ule{\textbf{na}}{~~~~~}  \ulp{à-sərk=ala}{~~~~~~~~~~~~~~~~~~}  \ulp{afa}{~~~~~~~~~~~~~~~~}  \ulp{təta}{~~~~~}  \ulp{=va}{~~~~~}  \ule{\textbf{na}}{~~} \\ 
      so woman {\PSP} \oldstylenums{3}\textsc{s}+{\PFV}-{habitually}=to {at place of} \oldstylenums{3}\textsc{p} ={\PRF} {\PSP}\\ 
\glt ‘Now that woman, she was in the habit at their house,’\\      
      
      \medskip
\underline{aməhaya  háy  \textbf{na}},  \textbf{gam.}\\      
\gll \ulp{amə-h=aja}{~~~~~~~}  \ulp{haj}{~~~~~} \ule{\textbf{na}} \textbf{gam}\\
     {\DEP}-grind={\PLU}  millet {\PSP} alot\\
\glt  ‘[of] grinding \textit{a lot} of millet.’  
\z

In \REF{ex:11:47} from the Values exhortation, there are a series of six \textit{na}-marked elements that reiterate some of the main points of argument that the speaker used. The final element \textit{anga way} ‘whose [word]’ is made prominent and the effect is to cause the hearer to think about whose word the people accept (based on their behaviour). 

\newpage 
\ea \label{ex:11:47}\corpussource{Values, S. 29}\\
\underline{Hərmbəlɔm  \textbf{na}},  \underline{amaɗaslava  ala  məze  \textbf{na}},  \underline{ndahan  ese  \textbf{na}},\\    
\gll  \ulp{Hʊrmbʊlɔm}{~~~~~}   \ule{\textbf{na}}   \ulp{ama-ɗaɬ=ava=ala}{~~~~~}     \ulp{mɪʒɛ}{~~~~~}     \ule{\textbf{na}}      \ulp{ndahaŋ}{~~~~~} \ulp{ɛʃɛ}{~~~~~} \ule{\textbf{na}}\\  
      God  {\PSP}    {\DEP}-multiply=in=to  person  {\PSP}  \oldstylenums{3}\textsc{s}  again  {\PSP}\\  
\glt ‘God, the one that multiplied the people, him again,’ \\

\medskip
\underline{kagas  ma  Hərmbəlom  \textbf{na}  asabay  \textbf{na}},\\
\gll \ulp{ka-gas}{~~~~~} \ulp{ma}{~~~~~} \ulp{Hʊrmbʊlɔm}{~~~~~} \ule{\textbf{na}}  \ulp{asa-baj}{~~~~~~~}  \ule{\textbf{na}}\\
     {\twoS}-catch  word God  {\PSP} again-{\NEG}   {\PSP}  \\
\glt ‘[if] you catch God's word no longer,’\\

\medskip
\underline{k\'{a}agas  \textbf{na}},  \textbf{anga  way}?\\ 
\gll \ulp{k\'{a}\'{a}-gas}{~~~~~~~~~} \ule{\textbf{na}} \textbf{aŋga}      \textbf{waj}\\
     {\twoS}+{\POT}-catch  {\PSP} {\POSS}    who\\
\glt ‘You won't accept anyone's word!' (lit. you will catch it [word] of whom?’) \\ 
\z

In both \REF{ex:11:48} and \REF{ex:11:49}, the final prominent element is \textit{jəyga}  ‘all.’ The effect is to emphasise the totality of the events. In \REF{ex:11:48}, the fact that \textit{all} of the field was destroyed by the cows is important to the story. In \REF{ex:11:49}, the story teller is emphasising that it was important that \textit{everyone} fought against the \ili{Mbuko}. In fact, people who did not fight were beaten after the skirmish with the \ili{Mbuko} ended.

\ea \label{ex:11:48}
Waya  \underline{sla  ahay  \textbf{na}},  \underline{tozom  gəvah  \textbf{na}}, \textbf{jəyga}   \textbf{anga  ləme  zlom.}\\
\gll  waja   \ulp{ɬa}{~~~~~}  \ulp{=ahaj}{~~~~~}  \ule{\textbf{na}}  \ulp{t\`{ɔ}-zɔm}{~~~~~~~} \ulp{gəvax}{~~~~~} \ule{\textbf{na}}  \textbf{{dzijga}}  \textbf{aŋga}    \textbf{lɪmɛ}      \textbf{ɮɔm}\\
      because  cow    =Pl  {\PSP}  \oldstylenums{3}\textsc{p}+{\PFV}-eat   field   {\PSP}      all      {\POSS}    \oldstylenums{1}\textsc{Pex}   five \\
\glt  ‘Because those cows, they ate \textit{all} of that field that belonged to the five of us.’ (lit. because the cows, they ate the field, all of it, belonging to us five)
\z

\ea \label{ex:11:49}
\underline{Nde  \textbf{na}},  \underline{ləme  \textbf{ɗəw}},  \underline{nəzləgom  va  \textbf{na}},  \textbf{jəyga}.\\
\gll  \ulp{ndɛ}{~~~~~~}   \ule{\textbf{na}}  \ulp{lɪmɛ}{~~~~~~}   \ule{\textbf{ɗuw}}  \ulp{n\`ə-ɮʊg-ɔm}{~~~~~~~~~~~~~~~}     \ulp{va}{~~~~~~}    \ule{\textbf{na}}   \textbf{dzijga}\\
     so   {\PSP}  \oldstylenums{1}\textsc{Pex}  also  \oldstylenums{1}+{\PFV}-plant-\oldstylenums{1}\textsc{Pex}  body  {\PSP}  all\\
\glt  ‘So, we also, we fought (lit. planted body), \textit{all of us}.’
\z

In \REF{ex:11:50}, two \textit{na}{}-marked elements leave a negative particle highlighted at the end of the clause. The fact that the storytellers did not eat the people’s food was important since they would have been expected to eat.
\newpage 

\ea \label{ex:11:50}
Nde  \underline{kəy  elé  \textbf{na}},  \underline{nəzəmom  ele  ata  \textbf{na}},  \textbf{bay}.\\
\gll  ndɛ  \ulp{kij}{~~~~~~~~~~~~~}     \ulp{ɛlɛ}{~~~~~~}   \ule{\textbf{na}}  \ulp{n\`ə-zʊm-ɔm}{~~~~~~~~}    \ulp{ɛlɛ}{~~~~~~}   \ulp{=atəta}{~~~~~~}     \ule{\textbf{na}}   \textbf{baj}\\
      so  \textsc{id}looking  eye  {\PSP}  \oldstylenums{1}+{\PFV}-eat-\oldstylenums{1}\textsc{Pex}    thing  =\oldstylenums{3}\textsc{p}.{\POSS}  {\PSP}  {\NEG}\\
\glt  ‘So, one could see that we had \textit{not} eaten their food.’ (lit. looking, we ate their thing, not)
\z

In the Disobedient Girl peak, four \textit{na}{}-marked elements precede the expression of the most pivotal event in the narrative -- the death of the girl (expressed in a clause that is bolded in \ref{ex:11:51}). 

\ea \label{ex:11:51}
\corpussource{Disobedient Girl, S. 26}\\
\underline{Alala  \textbf{na}},  \underline{ver  \textbf{na}},  \underline{árah  mbəf  nə  həmbo  \textbf{na}},\\ 
\gll  \ulp{a-l=ala}{~~~~~~~~}   \ule{\textbf{na}}   \ulp{vɛr}{~~~~~~~~~}   \ule{\textbf{na}}  \ulp{á-rax}{~~~~~~~~~}     \ulp{mbəf}{~~~~~~~~~~~}  \ulp{nə}{~~~~~~} \ulp{hʊmbɔ}{~~~~~~} \ule{\textbf{na}}\\      
      \oldstylenums{3}\textsc{s}-go=to  {\PSP} room  {\PSP} \oldstylenums{3}\textsc{s}+{\IFV}-fill  {to the roof}  with  flour  {\PSP} \\      
\glt ‘Later, the room, it filled up to the roof with the flour,’\\

\medskip
\underline{ɗək  məɗəkaka  alay  ana  hor  \textbf{na}},\\
\gll \ulp{ɗək}{~~~~~~~~~~} \ulp{mə-ɗək=aka=alaj}{~~~~~~~~~~~~~~~}   \ulp{ana}{~~~~~~}   \ulp{hʷɔr}{~~~~~~} \ule{\textbf{na}}\\
     suffocate  {\NOM}{}-suffocate=on=away  {\DAT} woman  {\PSP}   \\ 
\glt ‘[the flour] suffocated the woman,’ \\

\medskip
\textbf{nata  ndahan  dəɓəsolək  məmətava  alay  a  hoɗ} \textbf{a  hay  na  ava}.\\
\gll \textbf{{nata}}  \textbf{ndahaŋ}   \textbf{dʊɓʊsɔlʊkʷ}   \textbf{mə-mət=ava=alaj}   \textbf{a}   \textbf{hʷɔɗ} \textbf{a}   \textbf{haj}   \textbf{na}   \textbf{ava}\\
     then    \oldstylenums{3}\textsc{s}  \textsc{id}collapse/die  {\NOM}{}-die=in=away  at  stomach {\GEN}  house  {\PSP}  in\\
\glt  ‘and then she collapsed \textit{dəɓəsolək}, dying inside the house.’
\z

The \oldstylenums{3}\textsc{s} \DO pronominal \textit{na} (\sectref{sec:7.3.2}) is identical to the presupposition marker \textit{na} and some ambiguity can be encountered in contexts where \textit{na} immediately follows a verb that has no locational or directional extensions (which follow the DO pronominal but would precede a PSP marker). (\ref{ex:11:52}--\ref{ex:11:53}) show two such examples. In \REF{ex:11:52},  the underlined \textit{na}  could be interpreted as the presupposition marker since there are multiple \textit{na}{}-marked elements in the clause and this final underlined \textit{na} appears immediately before the final (presumably) focussed element \textit{mənjəye ata} ‘their habits.’ On the other hand, \textit{na} could be  the \oldstylenums{3}\textsc{s} \DO pronominal for the verb \textit{mədakan} ‘instructing to him,’ since the verb is in a construction which marks significant events (see stem plus ideophone auxiliary \sectref{sec:8.2.3}), so it is the event of the husband instructing his wife that is highlighted by the preceding \textit{na}-marked elements.

\newpage 
\ea \label{ex:11:52}
\corpussource{Disobedient Girl, S. 12}\\
Sen  ala  \textbf{na},  zar  ahan  \textbf{na},  dək  mədakan  \underline{\textbf{na}}  mənjəye  ata.\\
\gll  ʃɛŋ =ala       \textbf{na}  zar     =ahaŋ     \textbf{na}  dək        mə-dak  =aŋ    \underline{\textbf{na}}    mɪ-nʒ-ijɛ\\     
      \textsc{id}go =to   {\PSP}   man  =\oldstylenums{3}\textsc{s}.{\POSS}  {\PSP}   \textsc{id}show  {\NOM}-show =\oldstylenums{3}\textsc{s}.{\IO}  {\PSP}  {\NOM}{}-sit-{\CL}\\ 
      
      \medskip 
\gll =atəta \\
     =\oldstylenums{3}\textsc{p}.{\POSS}\\
\glt  ‘Then her husband instructed her their habits.’ (lit. going, her husband, instructing to her, their sitting)
\z

In  \REF{ex:11:53} the situation is more clear. We consider the two underlined \textit{na}  markers to be the \oldstylenums{3}\textsc{s} DO pronominal since even though there are multiple \textit{na}{}-marked elements in the clause, these underlined markers are neither at the end of the noun phrase (as they would be if they were the definite marker), nor are they immediately before the final focussed element (as they would be if this was a presupposition-focus construction). The verb and noun phrases in question are each delimited by square brackets in the example. We found no unambiguous instance of the presupposition marker \textit{na} breaking up a verb phrase except for the purpose of isolating the final focussed element in a verb phrase (cf. integrity of the \VP, \sectref{sec:8.1}). Thus the first underlined \textit{na}  is \oldstylenums{3}\textsc{s} \DO for the verb \textit{tozom} ‘they eat.’ It is doubling the direct object noun phrase gəvax ‘field.' Likewise, we found no unambiguous instance of the presupposition marker breaking up a noun phrase in any context and so consider the second underlined \textit{na} as \oldstylenums{3}\textsc{s} \DO pronominal for the nominalised verb \textit{məgəye} ‘doing’ within the noun phrase \textit{məgəye na ahan} ‘his doings.’  

\largerpage
\ea \label{ex:11:53}
Nde  asa  bahay  a  sla  \textbf{na},  ndahan  aka  bay  \textbf{na,} asa  sla  ahay  \textbf{na},\\ 
\gll  ndɛ  asa   bahaj  a  ɬa  \textbf{na}  ndahaŋ  aka   baj  \textbf{na} asa ɬa  =ahaj \textbf{na}\\ 
      so  if  chief  {\GEN}  cow  {\PSP}  \oldstylenums{3}\textsc{s}  on  {\NEG}  {\PSP} if  cow  =Pl  {\PSP} \\ 
\glt ‘So, if the owner of the cows wasn’t there, [and] that the cows’ 

\medskip
 [tozom  \underline{na}  gəvah]  \textbf{na},  ɗeɗen  \textbf{na},  ndahan  \textbf{na},\\  
\gll [{t\`{ɔ}-zɔm} \underline{na} {gəvax}] \textbf{na} {ɗɛɗɛŋ} \textbf{na} {ndahaŋ} \textbf{na}\\  
     \oldstylenums{3}\textsc{p}+{\PFV}-eat  \oldstylenums{3}\textsc{s}.{\DO}  field  {\PSP}  truth  {\PSP} \oldstylenums{3}\textsc{s} {\PSP} \\ 
\glt ‘really destroyed the fields  is true (lit. if the cows ate the field true), [then] he,’

\medskip
ámənjar  nə  elé  ahan  bay  \textbf{na},  [məgəye  \underline{na}  ahan]  \textbf{na}  memey?\\
\gll á-mənzar nə ɛlɛ  =ahaŋ baj \textbf{na} [{mɪ-g-ijɛ} \underline{na} {=ahaŋ}] \textbf{na}\\ 
     \oldstylenums{3}\textsc{s}+{\IFV}-see  with  thing  =\oldstylenums{3}\textsc{s}.{\POSS}  {\NEG}  {\PSP} {\NOM}-do{}-{\CL} \oldstylenums{3}\textsc{s}.{\DO}  =\oldstylenums{3}\textsc{s}.{\POSS}  {\PSP}\\  
     
     \medskip
\gll {mɛmɛj}\\
     how\\
\glt  ‘[since] he hasn’t seen it for himself, what is he supposed to do?’ (lit. his doing, how)
\z
\is{Focus and prominence!Na@\textit{Na} marker|)}\is{Presupposition constructions|)}