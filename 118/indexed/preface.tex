\addchap{Foreword}

Documentary work on small and threatened languages has received increased attention in recent decades, to the point that even members of the general public may be aware of the notion of “endangered language.” While language documentation itself --  the collection and possible archiving of primary audio and video recordings of language, tagged with various types of metadata and typically also (partially) transcribed and translated into a language of wider communication -- is valuable for community and scholarly reasons, the importance of developing additional analytical and interpretive works, based in part or in whole on such documentary materials, must not be discounted. The latter assist multiple communities -- ranging from the speakers themselves, to scholars, educators, government officers, journalists and media enterprises, and even the general public -- to appreciate the intricate intellectual, cultural, and creative achievements and knowledge of the speakers and the cultures built with these languages.

It is thus my pleasure to recommend this very fine grammar on Moloko, a little-studied Chadic (Afro-Asiatic) language, spoken by upwards of 10,000 people in Cameroon. Its principal author lived in the Moloko region of Cameroon for nearly a decade, studying the Moloko language and collaborating directly with community members. From my own experiences, I can attest that it is no small endeavor to produce any reference grammar, much less a linguistically sophisticated one like this. The quality of the grammar clearly reflects Dianne Friesen’s substantive and deep knowledge of the language, as well as her persistence in the face of many significant obstacles to see this published grammar come to fruition.

The work is a rich treasure trove, giving insight not just into the workings of the Moloko linguistic system, but also into cultural issues. The presentation notably fronts several translated and analyzed Moloko texts which, in themselves, give us glimpses of Moloko thought and life. Throughout, the grammar then often illustrates claims about grammatical phenomena using examples drawn from these texts. This enables the reader to evaluate the claims and data in their larger communicative context. The analytical chapters discuss intricate phonological phenomena including word-level palatalization and labialization “prosodies,” \nohyphens{lexical} matters including how semantic distinctions in the verbal lexicon affect morphosyntactic patterns, multiple syntactic issues that help reveal (as Friesen puts it) the “genius” of the language, and various discourse phenomena. The work concludes with a bilingual lexicon and indices, enhancing its use as a reference work. 

After having consulted with Dianne Friesen across several years on the content, analysis, and exposition of many parts of this grammar, it is supremely evident to me that this work is grounded in extensive collaboration and dialogue between the principle author and members of the Moloko community. It also reflects respectful consideration of analyses reported in manuscripts and publications produced by previous researchers, and it is enriched by an understanding of Chadic phenomena more generally. It also is grounded in typological and theoretical knowledge of linguistic patterns beyond Chadic. As a whole, the work reflects some of the best practices in scholarly research and practice around small and little-studied languages. 

The various collaborators and contributors to this published grammar are to be thoroughly congratulated for the quality and excellence of their work. It is also my hope that this grammar will stand as testament and encouragement to others working on minority languages of the real possibility of seeing their knowledge be “put to paper” in a way that becomes useful and is of benefit to others. Attention to the details, while holding onto the big vision, matter. Grit makes a difference. Persistence produces results. Do not be discouraged in doing well.
\vskip\baselineskip
\noindent Doris Payne\hfill
\parbox[t]{4cm}{\raggedleft Eugene, Oregon\\June 7, 2016}
% This book has no preface