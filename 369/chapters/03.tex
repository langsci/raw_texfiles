\documentclass[output=paper,colorlinks,citecolor=brown]{langscibook}
\ChapterDOI{10.5281/zenodo.7525096}
\author{Mary A. Kato\affiliation{Universidade Estadual de Campinas / UNICAMP} and Maria Eugenia Lammoglia Duarte\affiliation{Universidade Federal do Rio de Janeiro / UFRJ}}
\title[The partial loss of free inversion and of referential null subjects in BP]{The partial loss of free inversion and of referential null subjects in Brazilian Portuguese}
\abstract{Brazilian Portuguese (BP) has been considered a \emph{Partial Null Subject language} with the following properties: optional referential null subjects (RNS), null generic subjects, and null expletives. The aim of this paper is to discuss the nature of the optionality of RNSs. The case of the null generic subjects, partially attested in BP, and of null expletives are not under discussion. Using the macro-parametric view of the NS Parameter, we will make a joint discussion of both the possibility of RNSs and of free inversion in present BP. With regard to the latter, we will propose first that partial loss of free inversion has to be relativized in terms of prosodic weight. With regard to the former, we propose that optional RNS is felicitous when a sentence has a linear V2 (non-Germanic) pattern at PF, with the presence of a short cliticized element, which includes the subject pronoun. In both cases, we claim that BP has a filter at PF (Avoid V1). }

%move the following commands to the "local..." files of the master project when integrating this chapter
% \usepackage{tabularx}
% \usepackage{langsci-basic}
% \usepackage{langsci-optional}
% \usepackage{langsci-gb4e}
% \bibliography{localbibliography}
% %\newcommand{\orcid}[1]{}
% \pagenumbering{arabic}
% \setcounter{page}{53}


\IfFileExists{../localcommands.tex}{
  \addbibresource{../localbibliography.bib}
  \usepackage{langsci-optional}
\usepackage{langsci-gb4e}
\usepackage{langsci-lgr}

\usepackage{listings}
\lstset{basicstyle=\ttfamily,tabsize=2,breaklines=true}

%added by author
% \usepackage{tipa}
\usepackage{multirow}
\graphicspath{{figures/}}
\usepackage{langsci-branding}

  
\newcommand{\sent}{\enumsentence}
\newcommand{\sents}{\eenumsentence}
\let\citeasnoun\citet

\renewcommand{\lsCoverTitleFont}[1]{\sffamily\addfontfeatures{Scale=MatchUppercase}\fontsize{44pt}{16mm}\selectfont #1}
  
  %% hyphenation points for line breaks
%% Normally, automatic hyphenation in LaTeX is very good
%% If a word is mis-hyphenated, add it to this file
%%
%% add information to TeX file before \begin{document} with:
%% %% hyphenation points for line breaks
%% Normally, automatic hyphenation in LaTeX is very good
%% If a word is mis-hyphenated, add it to this file
%%
%% add information to TeX file before \begin{document} with:
%% %% hyphenation points for line breaks
%% Normally, automatic hyphenation in LaTeX is very good
%% If a word is mis-hyphenated, add it to this file
%%
%% add information to TeX file before \begin{document} with:
%% \include{localhyphenation}
\hyphenation{
affri-ca-te
affri-ca-tes
an-no-tated
com-ple-ments
com-po-si-tio-na-li-ty
non-com-po-si-tio-na-li-ty
Gon-zá-lez
out-side
Ri-chárd
se-man-tics
STREU-SLE
Tie-de-mann
}
\hyphenation{
affri-ca-te
affri-ca-tes
an-no-tated
com-ple-ments
com-po-si-tio-na-li-ty
non-com-po-si-tio-na-li-ty
Gon-zá-lez
out-side
Ri-chárd
se-man-tics
STREU-SLE
Tie-de-mann
}
\hyphenation{
affri-ca-te
affri-ca-tes
an-no-tated
com-ple-ments
com-po-si-tio-na-li-ty
non-com-po-si-tio-na-li-ty
Gon-zá-lez
out-side
Ri-chárd
se-man-tics
STREU-SLE
Tie-de-mann
}
  % \togglepaper[3]%%chapternumber
}{}


\begin{document}
\maketitle
\section{Introduction}
\subsection{The problem}
Brazilian Portuguese (BP) has been known to have partially lost the properties of the Null Subject Parameter (NSP), as conceived in its macro-parametric view,\footnote{Cf. \citet{rizzi_issues_1982},\citet{chomsky:1981}} since the middle of the 19\textsuperscript{th} century.\footnote{Cf. \citet{tarallo_diagnosticando_1993}; \citet{duarte1995}; \citet{kato_strong_1999, kato2000a, kato_o_2017}}
Two major explanations can be given for this partiality:

a) the changes are still in progress, and the partiality has to do with the fact that the changes are not completed yet (see \figref{03:Fig1}).\footnote{Cf. \citet[58--59]{cyrino2000}  propose a referential hierarchy that guides changes concerning pronominalization. Under their hypothesis, [+N, + human] arguments, the speaker and the addressee, are the highest in the referential hierarchy, and a proposition, the lowest. [$-$human] entities are in between, with the [-animate] entity interacting with [+animate/+human] ones. The feature [+/$-$specific] interacts with all the other features. This explains why the change towards overt pronouns affects 1st and 2nd persons first, whereas 3rd person, exhibiting [+/$-$human] referents shows a slower increase of overt pronominal subjects, as shown in \figref{03:Fig1}}

\begin{figure}
    \includegraphics[scale=.40]{figures/kato_fig1.png}
    \caption{Null subjects (vs overt) in theater popular plays across two centuries \citep{duarte_pronome_1993}}
    \label{03:Fig1}
\end{figure}

b) the partial aspects of the change have to do with the fact that there are partial NS languages, among which BP,\footnote{Cf.  \citet{holmberg_control_2010}.} with \emph{optional} referential human and non-human NSs (\ref{ex:03:kato:1a}--\ref{ex:03:kato:1b} and \ref{ex:03:kato:2a}--\ref{ex:03:kato:2b}), null expletives (\ref{ex:03:kato:3a}--\ref{ex:03:kato:3b}) with existential and weather verbs, and, we add, free inversion restricted to unaccusative verbs (\ref{ex:03:kato:4a}-\ref{ex:03:kato:4b}) reanalyzed as existentials.\footnote{Cf. \citet{kato_loss_1993, kato2000b, kato2002a}. According to the latter, unaccusatives have been reanalyzed in BP as existentials, the reason why the verb is always in the 3\textsuperscript{rd} person singular.}
\ea\label{ex:03:kato:1}
  \ea\label{ex:03:kato:1a}
 \gll Os pais\textsubscript{i}    passam aos     filhos     o que \textbf{eles}\textsubscript{i} têm, né?\\
          the parents pass      to.the children that  which  they have right\\
          \glt ‘The parents transmit to their children what they know, see?’
%1A
   \ex\label{ex:03:kato:1b}
 \gll Meu marido\textsubscript{i} foi  quase   preso     aí      no      fort porque \textbf{Ø}\textsubscript{i} foi    mergulhar.\\
          my   husband was almost arrested there in.the fort because  {}  went dive.\\
         \glt ‘My husband was almost arrested at the fort because he went for a dive.’

   %1B
   \z
\z
 %Exemplo 2
\ea\label{ex:03:kato:2}
  \ea\label{ex:03:kato:2a}
 \gll Escola pública\textsubscript{i} nunca é boa    opção porque   \textbf{elas}\textsubscript{i} são ruins.\\
           school public    never  is good option because they are bad\\
\glt ‘Public schools are never a good option because they are bad.’
%2A
   \ex\label{ex:03:kato:2b}
 \gll O sistema  público\textsubscript{i} é totalmente diferente de empresas privadas. \textbf{Ø}\textsubscript{i} não funciona da    mesma maneira.\\
          the system public   is totally        different of companies private   {}  not  works   of.the same way\\
        \glt ‘The public system is totally different from private companies. It does not work the same way.’
%2B
  \z
\z
 %Exemplo 3
\ea\label{ex:03:kato:3}
  \ea\label{ex:03:kato:3a}
 \gll Quando eu cheguei aqui \textbf{Ø}\textsubscript{exp} tinha  uns   tiros,        por exemplo.\\
          when    I    arrived  here    {}    had    some shotguns for instance\\
          \glt ‘When I arrived here there were shotguns for instance.’
 %3A
   \ex\label{ex:03:kato:3b}
 \gll Aqui \textbf{Ø}\textsubscript{exp} não chove muito.\\
         here    {}      not  rains    lot\\
        \glt ‘It doesn’t rain very often here.’
 %3B
   \z
\z
%Exemplo 4
\ea\label{ex:03:kato:4}
  \ea[]{\gll Chegou os ovos.\\
          arrived\textsubscript{3ps} the eggs\\
          \glt ‘There arrived the eggs.’}\label{ex:03:kato:4a}
%4A
   \ex[*]{\gll Riu       a     plateia.		 \\
           laughed the audience     \\
           \glt ‘The audience laughed.’}\label{ex:03:kato:4b}
%4B
  \ex[*]{\gll Deu uma carta pra ela  o  Pedro \\
 gave a     letter to  her the Pedro  \\
 \glt ‘Pedro gave her a letter.’}\label{ex:03:kato:4c}
  \z
\z

%4C


\is{Cognition} %add "Cogntion" to subject index for this page

\il{Latin} %add "Latin" to language index for this page

The optionality shown in \REF{ex:03:kato:1} and \REF{ex:03:kato:2} does not mean that NSs are frequent in spoken BP. Recent empirical research \citep{duarte_sociolinguistics_2020} shows that BP is losing crucial properties of Romance NS languages, such as anti-c-command relation and c-command relation between subjects, contexts where a null subject is categorical in European Portuguese (EP). In BP, the anti-c-command environment shows the lowest rates of null subjects in a postposed main clause (around 11\% of the data) whereas the latter, with a subordinate clause following its main clause, already reaches 40\%, very distant rates from those attested for EP, 93\% and 94\% of null subjects, respectively. This gradual result for E-language confirms that the value of the NSP has already been reset in BP. As for restricted “free” inversion, we will show that, in spite of some possible contexts, it is restricted to thetic sentences with unaccusatives and that the variation SV/VS in the same context is in course, preferably with definite DPs with [+human] semantic feature, an important step in the change.

\subsection{The Aim}

The aim of this study is to show that variation/optionality in the properties of the NS parameter (NSP) in BP, attested today in writing and planned speech, has a stylistic or prosodic character \citep{kato2013a}, and does not constitute morphological or syntactic \emph{doublets}, in the sense of \citet{aronoff1976} and \citet{kroch_morphosyntactic_1994}.\footnote{“Syntactic heads, we believe, behave like morphological formatives generally in being subject to the well-known `Blocking Effect' \citep{aronoff1976}, which excludes morphological doublets, and more generally, it seems, any coexisting formatives that are not functionally differentiated. This  exclusion, however, does not mean, either for morphology or for syntax, that languages never exhibit doublets. Rather it means that doublets are always reflections of unstable competition between mutually exclusive grammatical options.” \citep[181]{kroch_morphosyntactic_1994}} \sectref{03:sec:kato:2} will discuss the optional character of free inversion, and \sectref{03:sec:kato:3} will discuss the optional possibility of Null Subjects (NSs).  \sectref{03:sec:kato:4} will synthesize  what was described in the previous sections to see if we can sketch a common explanation in terms of trigger or consequence(s) of the changes. In the last section we will draw the conclusions. Our diachronic data come from popular plays, written in Rio de Janeiro; synchronic data have been collected in recent interviews recorded in Rio de Janeiro between 2009--2010 (available at \url{www.corporaport.letras.ufrj.br}).

\section{The partial loss of free inversion}\label{03:sec:kato:2}

\subsection{Free inversion in Romance}\label{03:sec:kato:2.1}

Comparing BP with other NS Romance languages, \citet{kato2000a, kato2000b} noticed that free inversion   is more easily found when the objects are clitics. First, according to \citet{bentivoglio1978}, inversion is easily found in Spanish when the complements are clitics. The same is found to be true in Italian by \citet{beninca1988}:\\

\ea\label{ex:03:kato:5} %Exemplo 5
   \ea\label{ex:03:kato:5a}
 \gll \textbf{Lo}    instaló \textbf{Esteban}. \\
           it.\textsubscript{CL} installed Esteban     \\
           \glt ‘Esteban has installed it.’
%5A
   \ex\label{ex:03:kato:5b}
 \gll Quería hacer=\textbf{lo} Juan. \\
           wanted do=it John	\\
           \glt ‘John wanted to do it.’
           %5B
   \z
\z
%Exemplo 6
\ea\label{ex:03:kato:6}
  \ea[]{\gll L' ha  mangiata  \textbf{la} \textbf{mamma}.\\
           it.\textsubscript{CL} has eaten  the mommy  \\
           \glt ‘Mommy has eaten it.’}\label{ex:03:kato:6a}
%6A
   \ex[?]{\gll Há mangiato la torta \textbf{la} \textbf{mamma}. \\
          has  eaten       the pie  the mommy \\
          \glt ‘Mommy has eaten the pie.’}\label{ex:03:kato:6b}
 %6B
   \z
\z

In BP, such constraint has been aggravated by the fact that it has lost part of its clitics, particularly those belonging to the 3\textsuperscript{rd} person paradigm, which made the right side of the verb heavier. The examples in (\ref{ex:03:kato:7a}, \ref{ex:03:kato:7c} and \ref{ex:03:kato:7e})  show that the sentences are ungrammatical in BP because this variety does not dispose of clitics, and in (\ref{ex:03:kato:7b}, \ref{ex:03:kato:7d} and \ref{ex:03:kato:7f}), they are ill-formed because the right hand side of the verb is too heavy.

\ea\label{ex:03:kato:7} %Exemplo 7
   \ea\label{ex:03:kato:7a}
 \gll Comprou-\textbf{lha }           o  Pedro. \\
     bought-her.\textsubscript{CL}-it.\textsubscript{CL}    the Peter     \\ 	
     \glt ‘Peter bought it.’\jambox{(EP BP 19th century;  *BP 20th century)}%\todo{is the meaning of \textit{lha} `her' or `it' or both?}
%7a
   \ex\label{ex:03:kato:7b}
 \gll Comprou \textbf{ela} \textbf{para} \textbf{ele}   o Pedro. \\
           bought     it   for     him the Peter	\\ 
           \glt ‘Peter bought it for him.’ \jambox{(*EP  *BP 20th century)}
 %7b
    \ex\label{ex:03:kato:7c}
 \gll Comprou-\textbf{lhe}       um perfume   o Pedro.\\
           bought-3sg.\textsc{dat}   a     perfume     the Peter	\\ 
           \glt ‘Peter bought a perfume.’ \jambox{(EP  *BP 20th century)}
%7C
    \ex\label{ex:03:kato:7d}
 \gll Comprou \textbf{pra} \textbf{ela} um perfume  o  Pedro. \\
    bought     for  her  a   perfume    the Peter\\ 
    \glt ‘Peter bought her a perfume.’ \jambox{(*EP  *BP 20th century)}
%7D
    \ex\label{ex:03:kato:7e}
 \gll Comprou-\textbf{o}  para ela  o   Pedro. \\
           bought  for her the Peter 	\\	
           \glt ‘Peter bought it for her.’ \jambox{(EP *BP 20th century)}
%7E
    \ex\label{ex:03:kato:7f}
 \gll Comprou \textbf{ele} \textbf{pra} \textbf{ela} o Pedro. \\
           bought     it   for     her the Peter	\\
           \glt ‘Peter bought it for her.’	\jambox{(*EP  *BP 20th century)}
           %5B
   \z
\z

This shows that free inversion in Romance NS languages is constrained by phonological weight, a fact that made \citet{zubizarreta_prosody_1998} propose that predicate inversion in Romance results from a general predicate movement, which is constrained by prosody, called \emph{P-movement}. The existence of clitics and their light nature explains why predicates with clitics are found in Romance inversion. However in BP, such a constraint has been aggravated not only by the loss of 3\textsuperscript{rd} person clitics, which are replaced by weak pronouns optionally null in anaphoric contexts (cf. \citealt{kato_loss_1993}), but also by the fact that 2\textsuperscript{nd} person clitic \emph{te} is in variation with the weak pronoun \emph{você} ‘you', which is originally an address form \emph{Vossa Mercê} ‘Your Grace', that fully grammaticalized as a pronoun in BP and used in nominative, accusative, and oblique functions \citep{lopes2016}. Since its implementation through the 20\textsuperscript{th} century, there has been a competition with canonical 2\textsuperscript{nd} person \emph{tu} ‘you' and the accusative and oblique forms associated with it. Today, although their distribution is mainly diatopic, \emph{você} outnumbers \emph{tu}. \tabref{03:table1} shows the changes in the paradigm and in the placement of the remaining clitics (cf. \citealt{kato_distribution_1993}).\footnote{See \citet{nunes2019}, for whom the null object in BP is an object agreement.}

\begin{table}
\caption{2\textsuperscript{nd} and 3\textsuperscript{rd} person accusative pronominal paradigms (clitics and weak pronouns)}
\label{03:table1}
 \begin{tabular}{ll}
  \lsptoprule
  BP  19th century & BP 20th century \\
  \midrule
a. A Maria ama-me. & a’. A Maria me-ama \\
\multicolumn{2}{l}{‘Mary loves me.’}\\
\tablevspace
b. A Maria ama-te.    & b.' A Maria te-ama \\
& \hphantom{b.' }A Maria ama você.\\
\multicolumn{2}{l}{‘Mary loves you.’}\\

  \lspbottomrule
 \end{tabular}
\end{table}

\subsection{V1 vs V2 in BP free inversion}

As we realized that the partial loss of free inversion was triggered by the expressive loss of clitics, we considered the possibility that it was independent of the loss of the NS.  However, in a research project on spoken BP, \citet{kato2002b} and \citet{kato_semantic_2003} concluded that this variety of Portuguese rejects V1 structures in free inversion, with transitive and intransitive verbs filling the preverbal position, when possible, with a short (light) item, even a discursive one. The authors associated this restriction to the new prosodic rhythm of the language, a consequence of the resetting of the NSP.

\ea\label{ex:03:kato:8} %Exemplo 8
   \ea[]{\gll \textbf{Ali}  \ul{vai} \ul{a} \ul{Maria}. \\
     there  goes the Maria\\
     \glt ‘There goes Maria.’}\label{ex:03:kato:8a}
%9a

   \ex[*]{\gll Vai   \textbf{ali}    a Maria. \\
           goes there the Maria	\\
           \glt ‘There goes Maria.’}\label{ex:03:kato:8b}
 %9b

    \ex[]{\gll \textbf{Lá}    vem      o bonde.\\
          there comes the tram	\\
           \glt ‘There comes the tram.’}\label{ex:03:kato:8c}
%9C

    \ex[?]{\gll Vem \textbf{lá} o bonde.  \\
    comes there the tram \\
    \glt ‘There comes the tram.’}\label{ex:03:kato:8d}
%9D
   \z
\z

\citet{pilati_aspectos_2006} proposed that inversion is obtained more easily in BP when a deictic or a locative element satisfies the EPP, occupying the verb initial position:

\ea\label{ex:03:kato:9} %Exemplo 9
   \ea[*]{\gll \ul{Dormem} \textbf{aqui} \textbf{as} \textbf{crianças}. \\
             sleep       here the children\\
        \glt ‘The children sleep here.’}\label{ex:03:kato:9a}

%10A
   \ex[]{\gll \textbf{Aqui} \underline{dormem} \textbf{as} \textbf{crianças}. \\
            here  sleep      the children\\
        \glt ‘The children sleep here.’}\label{ex:03:kato:9b}

           %10B
   \z
\z

\citet{buthers2012} proposed that, as BP has become \textit{a partial NS language} with optional referential NSs, there has been an increasing tendency to avoid null expletives in VS structures, and locatives have become grammaticalized as lexical expletives, just like \emph{there} in English, which licenses VS even with transitive verbs with lower frequency:

\ea\label{ex:03:kato:10}%Exemplo 10
   \ea\label{ex:03:kato:10a}
 \gll \textbf{Lá}    vai     \textbf{o}  \textbf{time} \textbf{de} \textbf{futebol}. \\
          there goes  the team of soccer \\
         \glt ‘There goes the soccer team.’
 %11A
   \ex\label{ex:03:kato:10b}
 \gll \textbf{Aqui} constrói \textbf{um} \textbf{país}.\\
           here  builds     a    country\\
         \glt ‘Here a country is built.’
 %11B
   \z
\z

As such a constraint increases, the pattern XP V (YP) also does, affecting especially impersonal constructions, which still allow null expletives:

\ea\label{ex:03:kato:11} %Exemplo 11
   \ea\label{ex:03:kato:11a}
 \gll Ø\textsubscript{exp} Chove em São Paulo. \\
        {}  rains  in   São Paulo \\
         \glt ‘It rains in São Paulo.’
 %11A
   \ex\label{ex:03:kato:11b}
 \gll \textbf{São Paulo} chove\\
           {São Paulo} rains\\
         \glt ‘It rains in São Paulo.’
 %11B
   \ex\label{ex:03:kato:11c}
 \gll Ø\textsubscript{exp} Faz frio em Curitiba.\\
          {} does cold in Curitiba\\
         \glt ‘It is cold in Curitiba.’
 %11C
   \ex\label{ex:03:kato:11d}
 \gll \textbf{Curitiba} faz frio. \\
           Curitiba does cold\\
         \glt ‘It is cold in Curitiba.’
 %11D
   \z
\z

\section{The partial loss of NSs in Brazilian Portuguese}\label{03:sec:kato:3}

\subsection{The ongoing loss of referential NSs in BP and their recovery through instruction}
While in the 19\textsuperscript{th} century and the beginning of the 20\textsuperscript{th} century BP, the data attested in the plays shows BP as a consistent NS language (see \ref{ex:03:kato:12}), from the decade of 1950 and on, BP has been described as having lost the referential NS in most contexts (see \ref{ex:03:kato:13}) and preserved the non-referential null expletive (see \ref{ex:03:kato:14}).\footnote{See \citet{duarte1995}; \citet{figueiredo_silva_posicao_1996}; \citet{modesto_null_2000} \emph{inter alia}}

\ea\label{ex:03:kato:12} %Exemplo 12
   \ea\label{ex:03:kato:12a}\gll Ontem       Ø\textsubscript{1ps} comprei-\textbf{lhe}\textsubscript{i}  o   hábito       com que     Ø\textsubscript{i 3ps}  andará vestido.\\
            yesterday   Ø\textsubscript{1ps} bought-\textbf{him}\textsubscript{i}  the costume   with which Ø\textsubscript{i 3ps} be.\textsc{fut} dressed			 \\ 
           \glt ‘Yesterday I bought him the costume he will wear.’ \jambox{(\emph{O noviço}, Martins Pena, 1845)}
%13A
   \ex\label{ex:03:kato:12b}
 \gll  Ø\textsubscript{2ps}  Terá          o   cavalo  que Ø\textsubscript{2ps} deseja.	\\
             Ø\textsubscript{2ps} have.\textsc{fut} the horse    that Ø\textsubscript{2ps} wish \\ 
            \glt ‘You will have the horse you wish.’ \jambox{(\emph{O simpatico Jeremias}, 1918, Gastão Tojeiro)}
%13B
   \z
\z

\ea\label{ex:03:kato:13} %Exemplo 13
   \ea\label{ex:03:kato:13a}
 \gll \textbf{Você} não  entende      meu coração porque \textbf{você} ‘tá sempre olhando pro      céu, procurando chuva. \\
            you  not  understand  my heart       because you  are always looking  at.the  sky looking\_for rain \\
            \glt ‘You don´t understand my heart because you're always looking at the sky in search of rain.’
%14A
   \ex\label{ex:03:kato:13b}
 \gll  Se \textbf{eu}   ficasse aqui \textbf{eu} ia          querer ser      a    madrinha. \\
             if  I     stayed  here  I   would   want   be.   the godmother \\ 
            \glt ‘If I stayed here I would like to be the godmother.’ \jambox{(\emph{No coração do Brasil}, M. Falabella, 1992)}

%14B
   \ex\label{ex:03:kato:13c}
 \gll Agora \textbf{ele} não vai mais poder dizer as coisas que \textbf{ele} queria dizer.\\
            now    he  \textsc{neg}  \textsc{fut}  \textsc{neg} can    say    the things that he wanted say \\ 
          \glt ‘Now he will no longer be able to say everything he would like to.’ \jambox{(\emph{No coração do Brasil}, M. Falabella, 1992)}
%14C
   \z
\z

% \largerpage[2]
\ea\label{ex:03:kato:14} %Exemplo 14
   \ea\label{ex:03:kato:14a}
 \gll O   dia está bonito     e     Ø\textsubscript{exp} \textbf{haverá}          muita gente. \\
           the day is   beautiful and     {}     \textsc{exist}.\textsc{fut} much people \\ 
           \glt ‘The day is fair and there will be a lot of people.’ \jambox{(\emph{Os irmãos das almas}, Martins Pena, 1845)}
%15A
   \ex\label{ex:03:kato:14b}
 \gll E Ø\textsubscript{exp} \textbf{tem} o quarto da empregada {lá fora}.\footnotemark{} \\
            and   {}   has  the room of.the maid    outside \\ 
            \glt ‘And there is a maid’s room outside.’ \jambox{(\emph{Um elefante no caos}, Millôr Fernandes, 1955)}
%15B
   \z
\footnotetext{Existential \emph{haver} ‘there is/are' has been replaced in speech by the possessive \emph{ter} ‘to have', which keeps the possessive meaning and the innovative existential meaning, as in \emph{Na esquina tem uma livraria} ‘On the corner has a bookstore'; \emph{ter} has the advantage to allow the projection of Spec, TP, as in \emph{Eu tenho uma livraria na esquina} ‘I have a bookstore on the corner', which suits the change in progress (\citealt{duarte_o_2003}).}
\z
\clearpage
Let us see what happens in the acquisition of BP considering such a change. As claimed by \citet{lightfoot_development_1999}, children’s core grammar does not have \emph{doublets}, containing only the innovative form. Confirming this claim, \citet{simoes_null_2000} shows that Brazilian children do not have NSs for referential subjects as in \REF{ex:03:kato:15}, keeping them for expletives only as in \REF{ex:03:kato:16}.\\

\ea\label{ex:03:kato:15} %Exemplo 15
   \ea\label{ex:03:kato:15a}
 \gll \textbf{Eu} to botando Ø (null object).\\
            I   am throwing\\
            \glt ‘I am throwing it (out).’
 %16A
   \ex\label{ex:03:kato:15b}
 \gll Não   quer Ø, \textbf{tu}  não    quer Ø?\\
            \textsc{neg} want {}   you \textsc{neg} want \\
            \glt ‘She/he doesn't want it, you dont' want it?’
%16B
   \ex\label{ex:03:kato:15c}
 \gll \textbf{Ela} anda a cavalo, anda de moto, \textbf{ela} anda. \\
            she rides a horse rides of motorcycle she walks  \\ 
            \glt ‘She rides a horse, rides a motorcycle, she gets around.’ \jambox{(André, 2;4) (exs. from \citealt{simoes_null_2000})}
 %16C
   \z
\z

\ea\label{ex:03:kato:16} %Exemplo 16
   \ea\label{ex:03:kato:16a}
 \gll Ø\textsubscript{expl} Tem     dois aviões. \\
                  {}  have.3\textsc{sg} two airplanes	\\
          \glt ‘There are two planes.’

%17A
   \ex\label{ex:03:kato:16b}
 \gll  Ø\textsubscript{expl} É esse que cabe.\\
                      {}  is  this that fits	\\ 
                        \glt ‘This is the one that fits.’ \jambox{(André, 2;4) (exs from \citealt{simoes_null_2000}}
 %17B
   \z
\z

In BP, however, NSs are shown to be recovered by instruction at school (\citealt{magalhaes_aprendendo_2000}) (See \tabref{03:table2}).

\begin{table}
\caption{Null subjects recovered through schooling \citep{magalhaes_aprendendo_2000}}
\label{03:table2}
 \begin{tabularx}{\textwidth}{Qrrr}
  \lsptoprule
  Referential Subjects  & 1st grade & 3rd/4th grades & 7th/8th grades \\
  \midrule
        Overt Pronominal Subjects & 97.89\% & 78.0\% &50.38\%\\
        Null Subjects & 2.11\%&22.0\%&49.62\%\\
  \lspbottomrule
 \end{tabularx}
\end{table}

With only 2,11\% of NSs in the first grade, Magalhães shows that optionality of NSs results from schooling. At the 7\textsuperscript{th} grade, adolescents start behaving like literate adults.

\ea\label{ex:03:kato:17}%Exemplo 17
   \ea\label{ex:03:kato:17a}
 \gll Ø\textsubscript{1ps} vou pedir uma ordem      ao      médico porque \textbf{eu}\textsubscript{1} não aguento ver você   sofrer   mais. \\
           (I)   \textsc{fut} ask a prescription to.the doctor because  I    not stand     to.see you  suffer  anymore	   \\ 
          \glt ‘I'm going to ask the doctor for a prescription because I can't stand to see  you suffer anymore.’ \jambox{(7th grade)}
 %18A
   \ex\label{ex:03:kato:17b}
 \gll \textbf{Eu}\textsubscript{1} estou de castigo,      porque Ø\textsubscript{1ps} briguei com minha irmã e    Ø\textsubscript{1ps}  não vou   poder jogar     futebol hoje\\
              I      am   of punishment because (I) argued with my sister   and  (I)    not \textsc{fut} can play  soccer today.	\\ 
          \glt ‘I'm have been punished because I argued with my sister and won't be able to play soccer today.’ \jambox{(7th grade)}
%18B

   \z

\z

The NS acquired through schooling is not part of the child’s core grammar, and it can be said that NS in the writing of literate adults is part of a second grammar in \emph{the periphery} of his/her I-language.\footnote{According to \citet{chomsky_language_1988}, the adult’s I-language may contain an extended periphery, with old forms, or even a mixture with a second language, like what heritage speakers tend to do.} The variation/optionality that we find in students’ and literate Brazilians’\footnote{Research on the speech of literate adults show that acquired/learned null subjects, 3\textsuperscript{rd} person clitics, existential \emph{haver} ‘there is/are ', and so many other features that are not in the primary acquisition data are not carried over to their spontaneous speech  (\citealt{duarte1995}; \citealt{freire_os_2000}; \citealt{duarte_o_2003}, among many others).} writing is like the phenomenon of code-switching, and the effect is stylistic. Looking at \tabref{03:table2}, at the distribution of 3\textsuperscript{rd} person NSs in the speech of Brazilian adults, we can see that it is much below what we have with Europeans, which shows a decline in progress. But in writing, Brazilians show a recovery of more than 20\% in the use of NSs, compared to their speech.

\begin{table}
\caption{Null subjects recovered through schooling \citep{magalhaes_aprendendo_2000}}
\label{03:table3}
 \begin{tabular}{rrrr}
  \lsptoprule
  \multicolumn{2}{c}{Speech} &\multicolumn{2}{c}{Writing} \\
  \cmidrule(lr){1-2}\cmidrule(lr){3-4}
EP (1980)&BP (2010)&EP (2000)&BP (2000)\\
\midrule
303/417 &331/1179& 227/244&119/241\\
73\%&28\%&93\%&49\%\\
  \lspbottomrule
 \end{tabular}
\end{table}

Notice that the variation between NSs and pronominal subjects in the Brazilian adult is very similar to that of 7\textsuperscript{th} graders, whereas NSs are frequent in speech, as well as, in writing in spoken and written EP. Considering that the overall rate of null subjects in BP is around 28\%, we can say that schooling shows a relative ``success'' by reaching about half of what EP writing reveals. The optionality is illustrated in BP standard writing.

\ea\label{ex:03:kato:18} %Exemplo 18
   \ea\label{ex:03:kato:18a}
 \gll Durante a   solenidade,  [o prefeito]\textsubscript{i} anunciou   um pacote   de obras  que Ø\textsubscript{i} autorizou  para a cidade. \\
            during  the ceremony    the mayor announced a    package of works that   {} authorized for the city\\
            \glt ‘During the ceremony, the mayour announced a package of works he authorized for the city.’
 %19A
   \ex\label{ex:03:kato:18b}
 \gll  Ele\textsubscript{i} explicou que à tarde         ele\textsubscript{i} vai         avaliar        todas as alternativas. \\
            he explained that at afternoon he  \textsc{fut} evaluate  all    the alternatives\\
        \glt ‘He explained that he would evaluate all the alternatives in the afternoon.’
%19B
   \ex\label{ex:03:kato:18c}
 \gll A França\textsubscript{i}  se       prepara   para o ataque ao      inimigo\textsubscript{2}. Ø\textsubscript{i} Sabe    que ele\textsubscript{2}    se      aproxima, mas Ø\textsubscript{i} não sabe     exatamente quando Ø\textsubscript{2} aparecerá. \\
            the France itself  prepares for the attack    to.the enemy    {}      knows that 3\textsc{sg} itself approaches   but     {}  not  knows  exactly       when    {}     appear.\textsc{fut}\\
            \glt ‘France is getting ready for the attack to the enemy, but does not know when it will appear.’
%19C
   \ex\label{ex:03:kato:18d}
 \gll O governo\textsubscript{i}       não considera o    fato de que os possíveis desvios                  e     exageros de que    ele\textsubscript{i}   tanto       se queixa são criticados na      própria imprensa.\\
            the government not considers  the fact  of that the possible misappropriations and exaggerations of which 3\textsc{sg} so\_much  \textsc{refl} complains are criticized in.the very   press\\
            \glt ‘The government does not consider the fact that possible misappropriations and exaggerations about
            which it so often complains are criticized by the press itself.’
%19D

   \z

\z

Concluding this section, we can say that NSs in the I-language of literate Brazilians is a function of stylistic prescriptions imposed by instruction.

\subsection{Optionality cases independent of competing grammars}

With the ongoing change from [+NS] to [$-$NS], \citet{duarte1995} found out that when the finite verb was preceded by a short item, the subject pronoun could be \emph{optionally} null, although overt pronouns are preferred.

\ea\label{ex:03:kato:19} %Exemplo 19
   \ea\label{ex:03:kato:19a}
 \gll (Eu) \textbf{Já}       trabalhava   naquela época.\\
             (I)   already worked.1\textsc{sg} at.that   time \\
            \glt ‘I already worked at that time.’
%20A
   \ex\label{ex:03:kato:19b}\gll (Cê)     \textbf{Nunca}  ouviu       falar     nele?\footnotemark{}\\
             (You) never    heard.2\textsc{sg}  talk  about.him\\
            \glt ‘You never head talk about him.’
%20B
   \ex\label{ex:03:kato:19c}
 \gll (Ele) \textbf{Não} aguentou o   tranco. \\
            (He) Not stood.3\textsc{sg}   the pressure\\
            \glt ‘He didn't stand the pressure.’
%20C
   \ex\label{ex:03:kato:19d}
 \gll (Eu) \textbf{Me}   tornei      independente.\\
             (I)  \textsc{refl}.\textsc{cl} became.1\textsubscript{sg} independent\\
            \glt ‘I became independent.’
%20D

   \z

\z
\footnotetext{The short or light elements we refer to are clitics, negation, light adverbs, located inside the TP, usually between Spec,TP and V. Adverbs and other elements located in the left periphery of the sentence, like interrogative and relative pronouns are contexts where null subjects are almost completely lost.}

We have tested the following group of sentences, and noticed that they are all well-formed except for \REF{ex:03:kato:20a}.

\ea\label{ex:03:kato:20} %Exemplo 20
   \ea[?]{\gll Ø  Como cenouras orgânicas.\\
   {} eat.\textsc{1sg}  carrots     organic     \\
   \glt ‘I eat organic carrots.’} \label{ex:03:kato:20a}
%21A
   \ex[]{\gll \textbf{Eu}  como  cenouras orgânicas.\\
                I    eat.1\textsc{sg}    carrots    organic   \\
                \glt ‘I eat organic carrots.’} \label{ex:03:kato:20b}
 %21B
   \ex[]{\gll Ø \textbf{Não} como    cenouras orgânicas.\\
	    {} not   eat.1\textsc{sg}  carrots     organic 	\\
      \glt ‘I don’t eat organic carrots.’} \label{ex:03:kato:20c}
 %21C
   \ex[]{\gll Ø \textbf{Só}   como    cenouras orgânicas.\\
                  {} only eat.1\textsc{sg}  carrots    organic   	\\
                   \glt ‘I only eat organic carrots.’} \label{ex:03:kato:20d}
 %21D

   \z

\z

The only difference between \REF{ex:03:kato:20a} and the others is that the former starts with the main verb while others are introduced by a short element (cf. \citealt{duarte1995}), avoiding a V1 sentence pattern.  Just as we showed a prosodic constraint disfavoring V1 in free inversion, we can say here that ordinary affirmative sentences in BP favor V2\footnote{This ``linear'' V2 order should not be confused with the structural Germanic V2.} as a general pattern. But it seems that in BP the first sentential element does not have to be the subject pronoun, but, as seen in \REF{ex:03:kato:20}, it can be some short element that cliticizes to the verb.

The usual cases of verb in initial position are in answers, but the derivation of such answers have to do with movement of the verb to the Focus initial position, followed by remnant erasure of the predicate (cf. \citealt{kato2013b}):

\ea\label{ex:03:kato:21}
 \gll -- Você come cenouras cruas?  \\
     {} you eat       carrots    raw       \\
     \glt \hphantom{-- }‘Do you eat raw carrots?’\\

 \gll -- [\textsubscript{Focus} Como [\textsubscript{TP} eu [ \sout{como cenouras cruas}]]] \\
          {} {}  eat\\
      \glt \hphantom{-- }‘I do.’
%Exemplo 21
\z
\ea\label{ex:03:kato:22}\gll -- O avião chegou no horário?\\
           {} the plane arrived on time?            \\
           \glt \hphantom{-- }‘Has the plane arrived on time?’\\

 \gll -- [\textsubscript{Focus} Chegou  [\textsubscript{TP} ele chegou  no horario]\\
     {} {} arrived\\
     \glt \hphantom{-- }‘It has.’
%Exemplo 22
\z


Traditionally, when the notion of the NS Parameter was introduced, it was conceived that the NS was a \emph{pro}, which had to be licensed and identified \citep{rizzi_issues_1982}. More recently, inspired by an old idea of \citegen{perlmutter_deep_1971}, who conceived NSs as resulting from a pronominal deletion process, \citet{holmberg_is_2005} and \citet{roberts_deletion_2010} adhered to this idea. In this paper, borrowing ideas from \citet{kato_restricoes_2014,kato_pre-verbal_2018}, we will also conceive the referential NSs of prototypical NS languages as deleted \emph{weak} pronouns.\footnote{\citet{kato_strong_1999} used the idea of weak \emph{vs.} strong pronouns from \citet{cardinaletti1999}.} The notion of licensing, required by the concept of \emph{pro}, does not allow the idea of optionality. But the notion of deletion can place the phenomenon at PF, the level that defines stylistic rules according to \citet{chomsky_filters_1977}.

\section{From triggers to consequences}\label{03:sec:kato:4}
\subsection{The role of rich morphology as the licensing condition (the trigger) for the referential NS}

That rich morphology is a licensing condition for the null subject in ``consistent'' NS languages \citep{roberts_introduction_2010} has been one of the most prevalent hypotheses, with diachronic evidence to support it. The changes that occurred in Old French \citep{adams1987, roberts_verbs_1993} and in BP \citep{duarte_pronome_1993,kato_strong_1999} support the rich Agr hypothesis, as it was the reduction of the Agr paradigm that led to the general loss of \emph{pro} in the former and the loss of referential \emph{pro} in the latter.  Similar facts have been attested for Caribbean Spanish, which lost the NS in all contexts, like Dominican Spanish \citep{toribio_dialectal_1994}, or in referential contexts like Puerto Rican Spanish.

The empirical facts that support the Agr identification hypothesis come from \citet{duarte_pronome_1993}, who shows that in the main dialects of BP the grammaticalization of the old address form \emph{você}  (from \emph{Vossa Mercê} = `Your Grace'), which is associated with the 3\textsuperscript{rd} person verb form, led to an inflectional paradigm in which the three-person distinction was lost.  \emph{Você} is now in competition with 2\textsuperscript{nd} person pronoun \emph{tu} ‘you', which also lost its canonical distinctive ending in speech. The inflectional reduction has been aggravated by the grammaticalization of another nominal expression \emph{a gente} ‘the people', which entered the BP pronominal system in competition with \emph{nós} ‘we', very similar to French \emph{on/nous}, and also requires 3\textsuperscript{rd} person singular agreement (see \cite{lopes2016}).

\begin{table}
\caption{Evolution of the pronominal and inflectional paradigm in BP in two centuries \citep{duarte_pronome_2018}}
\label{03:table4}
 \begin{tabular}{lllll}
  \lsptoprule
   & Nominative&Paradigm 1& Paradigm 2 &Paradigm 3\\
   & Pronouns & 19th C. & 20th C./1 &20th C./2\\
  \midrule
1PS&eu&cant\textit{o}&cant\textit{o}&cant\textit{o}\\
\tablevspace
1PP & nós & canta\textit{mos} &canta\textit{mos} & canta\textit{mos}\\
& \textit{a gente} & -- & cantaØ & cantaØ\\
\tablevspace
2PS & tu & canta\textit{s} & canta\textit{s}& canta\textit{s} \\
& você&--&cantaØ&cantaØ\\
\tablevspace
2PP & vós & canta\textit{is} & -- & -- \\
& vocês &canta\textit{m}&canta\textit{m}&canta(\textit{m})\\
\tablevspace
3PS&ele, ela&cantaØ&cantaØ&cantaØ\\
\tablevspace
3PP&eles, elas&canta\textit{m}&canta\textit{m}&canta(\textit{m})\\
  \lspbottomrule
 \end{tabular}
\end{table}


For \citet{galves_o_1993}, the consequence of such changes in the BP inflectional paradigm was a change in the strength of the AGR head, making it [-person]. One possible assumption regarding the partial loss of referential null subjects in BP was that the change was triggered by the loss of its rich inflectional morphology \citep{duarte1995,duarte_pronome_2018} and the acquisition of new free weak pronouns instead of a regular pronominal inflection \citep{kato_strong_1999}. See \tabref{03:table5} for the inflections in contemporary paradigm and the reduction in the realization of the personal pronouns.

\begin{table}
\caption{Personal pronoun contemporary paradigm reductions}
\label{03:table5}
 \begin{tabular}{llll}
  \lsptoprule
1. Corr-\textbf{o}\textsubscript{1PS} & \textbf{eu}\textsubscript{1PS} & [ô] & corro\\
2. Corr-\textbf{}Ø\textsubscript{2PS} & \textbf{você}\textsubscript{2PS} & [cê] & corre\\
3. Corr-\textbf{Ø}\textsubscript{3PS} & \textbf{ele}\textsubscript{3PS} & [ei] & corre\\
4. Corr-\textbf{Ø}\textsubscript{1PP} & \textbf{a gente}\textsubscript{1PP} & -- & corre\\
5. Corr-\textbf{m}\textsubscript{2PP} & \textbf{vocês}\textsubscript{2PP} & [ceis] & correm\\
6. Corr-\textbf{m}\textsubscript{3PP} & \textbf{eles}\textsubscript{3PP} & [eis] & correm\\
  \lspbottomrule
 \end{tabular}
\end{table}



\subsection{The loss of the rich inflection paradigm in BP and its new linear V2 sentential pattern}

Until the 19\textsuperscript{th} century, BP had a rich inflectional paradigm with one ending for each person, which qualified it as a type of \emph{weak} pronominal-like clitic (cf. \citealt{kato_strong_1999}). We propose that, as a consequence of the loss of some of the bound person inflection, we started having the spell-out of free weak pronouns in sentence initial position. As a consequence, we had a parallel \emph{change in terms of sentential prosody}. While, before the change, examples with NSs exhibited a V1 sentential pattern, they now appeared as a V2 pattern with overt subject pronouns.

\begin{table}
\caption{Changes in BP sentential patterns}
\label{03:table6}
 \begin{tabular}{ll}
  \lsptoprule
  19th c. BP: [V1] & 20th c. BP: [V2] \\
  \midrule
{\gll a. Comi maçã ontem \\
   {} ate.1\textsc{sg} apples yesterday\\} & {\gll a.$'$ Eu comi maçã ontem\\
                                                {} I ate apples yesterday\\}\\
   \tablevspace
   {\gll b. Fala.2\textsc{sg} francês?\\
   {} speak French\\}& {\gll b.$'$ Você/tu fala francês?\\
                    {} you speak French?\\}\\
   \tablevspace
   {\gll c. Vamos viajar  amanhã\\
   {} go.1\textsc{pl} travel.\textsc{inf} tomorrow\\} & {\gll c.$'$ A gente vai viajar  amanhã\\
                          {} the people go travel.\textsc{inf} tomorrow\\}\\
   \hphantom{c. }‘We are travelling tomorrow' & \hphantom{c.' }‘We are travelling tomorrow'\\
  \lspbottomrule
 \end{tabular}
\end{table}


The same happened with free inversion, which is more acceptable when the first position is occupied by some constituent, resulting in a linear V2 pattern \citep{kato_pre-verbal_2018}.


\begin{table}
\caption{V1 and V2 with free inversion}
\label{03:table7}
 \begin{tabular}{ll}
  \lsptoprule
  V1 & V2 \\
  \midrule
{\gll a.* Dormiram as crianças aqui.\\
      {}   slept         the children here\\} &  {\gll a.' Aqui dormiram as crianças.\\
                                        {} here  slept         the children\\}\\
   \tablevspace
   {\gll b.* Vai    lá      a    Maria.\\
   {} goes there the Mary\\}	& {\gll b.' Lá vai a Maria.\\
                          {} there goes the Mary\\}\\
   \tablevspace
   {\gll c.?  Chove em São Paulo.\\
    {} rains    in   São Paulo\\} & {\gll c.' São Paulo chove.\\
                              {} São Paulo rains\\}\\
  \lspbottomrule
 \end{tabular}
\end{table}

We can conclude that BP is a sort of partial NS language with remaining NSs and free inversion, both strongly constrained by prosodic factors in both cases. Regarding the changes that BP underwent, we can say that morphology was the trigger to change the value of the parameter, but prosody was the consequence.

\section{Conclusions}

In this work, we proposed that a stylistic prosodic rule affected both free inversion and NS constructions in BP.

\ea\label{ex:03:kato:23} %Exemplo 23
   \ea[]{\gll \textbf{Lá}    vem     \textbf{a}  \textbf{Maria}.     (V2)\\
   there comes the Mary     \\
   \glt ‘There comes Mary.’}\label{ex:03:kato:23a}
%23A
   \ex[*]{\gll Vem    lá      \textbf{a}   \textbf{Maria}. 	 (V1) \\
  comes there the Mary   \\
  \glt ‘There comes Mary.’}\label{ex:03:kato:23b}
 %23B
   \ex[]{\gll Você é americano?          (V2)\\
	     you are American\\
	     \glt ‘Are you American?’}\label{ex:03:kato:23c}
 %23C
   \ex[*]{\gll  É americano?   (V1)\\
    are American\\
    \glt ‘Are you American?’}\label{ex:03:kato:23d}
 %23D
   \z

\z

We may consider that, at the PF interface, languages have filters regarding their rhythm. To account for the preference for certain forms at this stage of the change in course in PB, a constraint of the form \emph{“Avoid V1”} will be proposed.  This constraint has nothing to do with an XP constituent in Spec of C with the verb in C, as in V2 Germanic languages, but with a prosodic requirement. This means that the initial element can be a head or an XP.

We may conjecture that the rhythmic pattern acts as a sort of parameter, just like morphology. The child would probably be more sensitive to prosody (first) than to morphology.

\section*{Acknowledgement}
We would like to thank Marcello Marcelino for his usual revision of our articles.

\printbibliography[heading=subbibliography,notkeyword=this]

\end{document}
