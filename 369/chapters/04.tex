\documentclass[output=paper,colorlinks,citecolor=brown]{langscibook}
\ChapterDOI{10.5281/zenodo.7525098}
\author{Karina High \affiliation{University of Texas at Austin}}

\title{The antipassive as a Romance phenomenon: A case study of Italian}
\abstract{This study focuses on the Italian pronominal verbs \textit{lamentarsi} ‘lament/complain’, \textit{ricordarsi} ‘remember/remind’, \textit{vantarsi} ‘praise/boast’ and their transitive counterparts and analyzes their distribution from the 13\textsuperscript{th} to the 21\textsuperscript{st} century across different syntactic environments, with particular attention to logical object expressions. It explores the possibility of an antipassive (AP) analysis, thereby adding a Romance perspective to the growing research of the historical development of the AP. The pronominal constructions of the sample that select an oblique complement display structural characteristics typical of the AP. Namely, they contain a demoted logical object, are structurally intransitive and semantically transitive, mark the oblique using the preposition \textit{di}, display a detransitivizing “AP morpheme” \textit{si}, and have a transitive counterpart. For all three verb pairs, there is initially a high frequency of AP constructions (13\textsuperscript{th}-15\textsuperscript{th} centuries), followed by a decrease in favor of transitive constructions with a direct object complement.}

% \usepackage{tabularx}
% \usepackage{langsci-basic}
% \usepackage{langsci-optional}
% \usepackage{langsci-gb4e}

% % PACKAGES IN PHONETICS & PHONOLOGY
% \usepackage{tipa}
% \let\ipa\textipa
% \usepackage{vowel}
% \usepackage{ot-tableau}

% % PACKAGES FOR IMAGES
% \usepackage{graphicx}

% % MISCELLANEOUS PACKAGES
% \usepackage{lastpage}
% \usepackage{hyperref}


% \let\ipa\textipa
% \let\eachwordone=\it
% \bibliography{localbibliography}
% \pagenumbering{arabic}
% \setcounter{page}{76}

\IfFileExists{../localcommands.tex}{
  \addbibresource{../localbibliography.bib}
  % add all extra packages you need to load to this file

\usepackage{tabularx,multicol}
\usepackage{url}
\urlstyle{same}

\usepackage{listings}
\lstset{basicstyle=\ttfamily,tabsize=2,breaklines=true}

\usepackage{langsci-basic}
\usepackage{langsci-optional}
\usepackage{langsci-lgr}
\usepackage{langsci-osl}
% \usepackage{./langsci/styles/langsci-lgr}
% \usepackage{./langsci/styles/langsci-osl}
% \usepackage{langsci-gb4e}

\usepackage{tikz}
\usetikzlibrary{patterns,calc}
\pgfdeclarepatternformonly{south east lines}{\pgfqpoint{-0pt}{-0pt}}{\pgfqpoint{3pt}{3pt}}{\pgfqpoint{3pt}{3pt}}{
    \pgfsetlinewidth{0.6pt}
    \pgfpathmoveto{\pgfqpoint{0pt}{3pt}}
    \pgfpathlineto{\pgfqpoint{3pt}{0pt}}
    \pgfpathmoveto{\pgfqpoint{.2pt}{-.2pt}}
    \pgfpathlineto{\pgfqpoint{-.2pt}{.2pt}}
    \pgfpathmoveto{\pgfqpoint{3.2pt}{2.8pt}}
    \pgfpathlineto{\pgfqpoint{2.8pt}{3.2pt}}
    \pgfusepath{stroke}}
    
\usepackage{stmaryrd}
\usepackage{wasysym}
\usepackage{multirow}
\usepackage{caption}
\usepackage{subcaption}
\usepackage{mathrsfs}
\usepackage{qtree}

\usepackage{linguex}


  %pminos do not split footnotes
% \interfootnotelinepenalty=10000 %Footnote in Laporte chapters has to be split SN


%\DeclareIndexNameFormat{default}{%
%\nameparts{#1}%
%\usebibmacro{index:name}%
%{\index[names]}%
%{\namepartfamily}%
%{\namepartgiveni}%
% {}% L1
% {}% L2
%{\namepartprefix}% generates spurious space L3
%{\namepartsuffix}% generates spurious space L4
%}

%  {\DeclareIndexNameFormat{default}{%
%     \usebibmacro{index:name}{\index[names]}{#1}{#3}{#5}{#7}}}

%\DeclareIndexNameFormat{default}{%
%  \usebibmacro{index:name}{\sindex[nom]}{#1}{#3}{#5}{#7}}

%\DeclareIndexNameFormat{default}{%
%  \usebibmacro{index:name}{\sindex[person]}{#1}{#3}{#5}{#7}}
%\DeclareIndexNameFormat{default}{%
%\nameparts{#1} \usebibmacro{index:name}{\sindex[person]]}{\namepartfamily}{‌​\namepartgiven}{\nam‌​epartprefix}{\namepa‌​rtsuffix}}

%\newcommand{\smiley}{:)}

%\renewbibmacro*{index:name}[5]{%
%\usebibmacro{index:entry}{#1}%
%{\iffieldundef{usera}{}{\thefield{usera}\actualoperator}\mkbibindexname{#2}{#3}{#4}{#5}}}

% \newcommand{\noop}[1]{}

%remove for final
%\overfullrule=1mm

\newcommand{\tobi}[2]}}
\renewcommand{\S}[1]{\tobi{#1}{\textsc{*}}}

% this volume references
% puts: [this volume]
% already defined: \citetv
%\newcommand{\citepv}[1]{(\citeauthor{#1} \citeyear*{#1} [this volume])}
\newcommand{\citealtv}[1]{\citeauthor{#1} \citeyear*{#1} [this volume]}

%parentheses around example number
\newcommand{\pref}[1]{(\ref{#1})}

% in-text examples

\newcommand{\lnex}[1]{\textit{#1}} %target lang word
\newcommand{\lnlit}[1]{(lit.: `#1')} %literal reading
\newcommand{\lnlat}[1]{(#1)} % latinization
\newcommand{\lntrans}[1]{`#1'} %translation
\newcommand{\lnexl}[2]%
{\lnex{#1}{} \lnlat{#2}} % ex with latinization
\newcommand{\lnexlat}[3]{\lnex{#1}{} \lnlat{#2}{} \lntrans{#3}} % ex with latinization and tranl.

%ch01
\newcommand{\co}[1]{\mbox{\textbf{#1}}}

%ch09

\newcommand{\cyrbulg}[1]{\begin{otherlanguage*}{bulgarian}#1\end{otherlanguage*}}


%ch10
\newcommand{\nlp}{{\small NLP}}
\newcommand{\mwe}{{\small MWE}}
\newcommand{\rae}{{\small RAE}}
\newcommand{\lvc}{{\small LVC}}
\newcommand{\pos}{{\small P}o{\small S}}
%\newcommand{\todo}[1]{ \textcolor{red}{#1} }

%\renewcommand{\labelenumi}{\theenumi}
%\ainamefmt{{vv}{ll}{, ff}{, jj}} % fullname

\newcommand{\biberror}[1]{{\color{red}#1}}

\newcommand{\osenovaitem}{--~}
  %% hyphenation points for line breaks
%% Normally, automatic hyphenation in LaTeX is very good
%% If a word is mis-hyphenated, add it to this file
%%
%% add information to TeX file before \begin{document} with:
%% %% hyphenation points for line breaks
%% Normally, automatic hyphenation in LaTeX is very good
%% If a word is mis-hyphenated, add it to this file
%%
%% add information to TeX file before \begin{document} with:
%% %% hyphenation points for line breaks
%% Normally, automatic hyphenation in LaTeX is very good
%% If a word is mis-hyphenated, add it to this file
%%
%% add information to TeX file before \begin{document} with:
%% \include{localhyphenation}
\hyphenation{
    Beck-man
    Ngu-yen
    back-chan-nel
    back-chan-nels
    mo-not-o-nous
    ste-reo-typ-i-cal
}

\hyphenation{
    Beck-man
    Ngu-yen
    back-chan-nel
    back-chan-nels
    mo-not-o-nous
    ste-reo-typ-i-cal
}

\hyphenation{
    Beck-man
    Ngu-yen
    back-chan-nel
    back-chan-nels
    mo-not-o-nous
    ste-reo-typ-i-cal
}

  % \togglepaper[3]%%chapternumber
}{}



\begin{document}
\maketitle

\section{Introduction}
This study examines the distribution of a particular class of pronominal verbs and their transitive counterparts in Italian from the 13\textsuperscript{th} to 21\textsuperscript{st} centuries and explores diachronic and synchronic evidence for the antipassive (AP) construction. The verbs in question are \textit{lamentare/lamentarsi} ‘lament, complain, moan’, \textit{ricordare/ricordarsi} ‘remember, remind’, and \textit{vantare/vantarsi} ‘praise, boast’.

\newpage

As shown in (\ref{ex:Intro Oblique}), the pronominal verb is semantically transitive and is characterized by the realization of the logical object\footnote{I am using the term ``logical object'' following \citet{polinsky_13._2017}. Others, such as \citet{creissels_origin_2012}, \citet{janic_slavonic_2013}, and \citet{sanso_where_2017, sanso_sources_2019}, refer to it as `patient'.} as an oblique complement, while its transitive form in (\ref{ex:Intro Direct}) selects a direct object complement.\footnote{It is also possible to find \textit{ricordarsi} followed by a direct object as an alternative to the construction in (\ref{ex:Intro Oblique}). The use of this particular construction increases over time and is more frequent than the construction in (\ref{ex:Intro Oblique}) in the 21\textsuperscript{st} century. Examples of this construction are found in (\ref{ex:Ricord PRO+NP}) and (\ref{ex:Disc:PRO+NP}) and are further examined in the Discussion.}

\begin{exe}
\ex \label{Intro Examples} \begin{xlist}
    \ex\label{ex:Intro Oblique}
    \gll Dopo aver cercato dappertutto \textbf{si} \textbf{ricordò} \ul{\textbf{del}} \ul{sogno} e corse in gardino, vicino al fiume, dove dormendo l' aveva veduta.\\
    after	have.\textsc{aux.inf}	search.\textsc{pst.ptcp}	everywhere \textsc{se}.\textsc{3sg} remember.\textsc{pfv.pst.3sg}		of.\textsc{def.det.msg}	dream.\textsc{msg}		and run.\textsc{pfv.pst.3sg} 	in	garden.\textsc{msg}	near	to.\textsc{def.det.m.sg}	river.\textsc{msg} where sleeping her have.\textsc{aux.ipfv.pst.3sg} see.\textsc{pst.ptcp}\\ \jambox*{(\textit{I racconti delle fate}, 1876)}
    \glt ‘After having searched everywhere, he remembered the dream and ran into the garden, near the river where sleeping, he had seen her.'\footnote{Unless otherwise indicated, all translations in this paper are my own.}
    \ex\label{ex:Intro Direct}
    \gll Chiunque \textbf{ricordi} \ul{la} \ul{vita}	\ul{italiana} \ul{al}	\ul{principio} \ul{del}	\ul{secolo} non potrà non sottoscrivere a questo  apprezzamento.\\
    whoever	remember.\textsc{sbjv.prs.3sg} \textsc{def.det.fsg} life.\textsc{fsg} Italian.\textsc{fsg} to.\textsc{def.det.msg}	beginning.\textsc{msg} of.\textsc{def.det.msg} century.\textsc{msg} \textsc{neg} can.\textsc{fut.3sg} \textsc{neg}		subscribe.\textsc{inf} to this.\textsc{msg} comment.\textsc{msg}\\ \jambox*{(\textit{Pensiero e azione del risorgimento}, 1943)}
    \glt ‘Whoever remembers the Italian life at the start of the century, cannot not subscribe to this comment.’
\end{xlist}
\end{exe}

The effect in (\ref{ex:Intro Oblique}) is a change in the valency of the verb, as the number of arguments is reduced. A similar type of valency-reducing strategy called the AP has been studied in ergative languages and increasingly, in accusative languages. In AP constructions, the logical object is realized as a non-core argument or is omitted (but remains presupposed).

This paper analyzes how such semantically-related transitive and pronominal verbs pattern diachronically and if the diachronic perspective provides evidence for the AP construction.

The organization of the paper is as follows. In \sectref{Romance se}, I first discuss current research about Romance pronominal verbs and the Romance clitic \textsc{se}, as well as typological and historical work on the AP. In \sectref{methods}, I describe the data sources, the data collection and coding process. In \sectref{analysis}, I present my findings. In \sectref{discussion}, I relate these to a discussion of the AP and examine the possibility of an AP analysis and the mechanisms of diachronic change. Finally, I conclude with some remarks about the main findings and ideas for future research.

\section{Romance \textsc{se} and previous analyses} \label{Romance se}
The pronominal verbs of this study attest to the heterogeneous nature of Romance \textsc{se}, which is traditionally labeled reflexive. Accounts from the prescriptive tradition struggle to find a classification of pronominal verbs that captures the polyfunctionality of this clitic. Analyses have been proposed to describe different uses of Romance \textsc{se}.
For instance, \citet{melis_classement_1985, melis_voie_1990,melis_pronominal_1990} expanded on the classical grouping for French pronominals, while \citet{nishida_spanish_1994} identified uses of Spanish \textit{se} as an overt aspectual class marker. For Italian, Cennamo's extensive work sheds light on several phenomena regarding the diachronic development of the Late Latin/Early Romance reflexive \textit{se/sibi}. Among others, she identified the use of the pleonastic reflexives \textit{se/sibi} with intransitive verbs as markers for Split Intransitivity \citeyearpar{cennamo_late_1999} and studied the expansion of the domain of reference \citeyearpar{cennamo_lestensione_1993} and the continuum of prototypical and less prototypical/grammaticalized uses of \textit{se, sibi, suus} in Late Latin Christian inscriptions \citeyearpar{cennamo_se_1991}.

Evidence has been discovered that suggests the existence of the AP construction in Romance. For Spanish, \citet{masullo_antipassive_1992} proposes the derivation of an AP \textit{se} as a direct object in the deep structure, which is then incorporated into the verb; for Slavic and Romance, \citet[vii]{medova_reflexive_2009} argues that “the reflexive clitic \textit{se} is an AP morpheme of the sort known from the ergative languages” and proposes a parallel derivation for inherent reflexives and APs.

Typological studies of the AP, such as \citet{polinsky_13._2017}, present and discuss various manifestations of this construction and weave together characteristics shared across ergative and accusative languages, which can serve as a set of diagnostics to identify the AP. To date, the historical work on the AP has been limited to non-Romance languages; for instance,  \citet{terrill_development_1997},  \citet{creissels_origin_2012},  \citet{janic_slavonic_2013}, and \citet{sanso_where_2017, sanso_sources_2019} identified the reflexive construction as one of several sources of the AP marker. These studies, however, do not look for supporting evidence from Romance; moreover, the current research on the Romance AP does not adopt a diachronic perspective.

\section{Data and methods} \label{methods}

The data of this study is drawn from online databases and annotated corpora, as well as collections of texts that are accessible online. The sources include the corpus \textit{Opera del Vocabolario della lingua italiana} (OVI), \textit{Tesoro della Lingua Italiana delle Origini} (TLIO), the \textit{Biblioteca dei Classici Italiani}, \textit{IntraText}, the \textit{Corpus Diacronico dell’Italiano Scritto} (DiaCORIS), and the \textit{Corpus di Italiano Scritto} (CORIS/CODIS). Together they cover a period from the 13\textsuperscript{th} until the 21\textsuperscript{st} century and include literary (e.g., novels, poems, plays, operas) and non-literary texts (e.g., religious texts, journalistic writings, essays, and correspondence).

For each verb pair I randomly selected 60 tokens per century or if there were not sufficient data, I included all occurrences available. This was the case for \textit{vantar(si)}, which records only 40 tokens for the 13\textsuperscript{th} century. Finite and non-finite verb forms are equally included in the dataset. I excluded passive forms and a handful of tokens that had prominent non-Tuscan characteristics, which is illustrated in (\ref{ex:non-tu}), an excerpt in the Venetian dialect from Carlo Goldoni's comedy, \textit{I Rusteghi}.

\begin{exe}
\ex \label{ex:non-tu}
    \gll Cossa songio? un tartaro? una bestia? \textbf{De} \ul{cossa} \textbf{ve} \textbf{podeu} \textbf{lamentar}? Le cosse oneste le me piase anca a mi.\\
    what be.\textsc{prs.1sg} \textsc{indf.det.msg} Tartar.\textsc{msg} \textsc{indf.det.fsg} beast.\textsc{fsg} of what \textsc{se}.\textsc{2pl} can.\textsc{prs.2pl} lament.\textsc{inf} \textsc{def.det.fpl} thing.\textsc{fpl} honest.\textsc{fpl} they.\textsc{f} me please.\textsc{prs.3sg} also to me\\  \jambox*{(\textit{I Rusteghi}, 1760)}
    \glt ‘What do you think of me? A Tartar? A brute? What have you to complain of? I don’t object to honest pleasures.’ (from Goldoni 1961: 109)
\end{exe}

The queries resulted in a total of 1600 tokens. In order to isolate and study the distribution of the verb pairs in general and the verbs with a logical object, I grouped the data into two large categories based on the presence or absence of the clitic \textit{si}. These categories I labeled TR, referring to verbs without \textit{si} (e.g., \textit{lamentare}), and PRO, referring to verbs with \textit{si} (e.g., \textit{lamentarsi}). In addition to author, title, and date, I also coded transitivity (Intrans/AP/Trans), type of phrasal complement (NP, CP, PP, Null), meaning, and type of logical object (IO/DO).\footnote{I also coded auxiliary selection for compound tenses. However, the pattern was exceptionless: PRO verbs selected the auxiliary \textit{essere} ‘to be’ (like reflexive verbs), while TR verbs selected the auxiliary \textit{avere} ‘to have’ (like transitive verbs).}

\section{Analysis} \label{analysis}

In the dataset, PRO forms are more common overall at 72\% (1153 tokens) compared with TR forms at 28\% (447 tokens). The overall distribution is heavily influenced by the fact that the PRO forms represent 73\%--91\% of the data for six centuries, from the 13\textsuperscript{th}--18\textsuperscript{th} centuries, but decline thereafter. Figure \ref{fig:disttime} shows the trends over time for the three verb pairs. For each verb pair, the solid line represents the PRO form and the dotted line represents the TR form and together total to 100\%. In addition, the shadowed area displays the mean percent TR for all verbs over time.

\begin{figure}[h]

    \includegraphics[width=\textwidth]{figures/High_fig1.png}
    \caption{Distribution across time}
    \label{fig:disttime}
\end{figure}

Starting from the 16\textsuperscript{th} through the 18\textsuperscript{th} century, the trendline documents an increase in the TR construction and a decrease in the PRO construction until they reach almost equal distribution in the 19\textsuperscript{th} century. The individual verb pairs differ slightly with respect to their distribution across time. In the 19\textsuperscript{th} century, \textit{vantare} becomes more frequent than \textit{vantarsi} (TR~=~68.3\%), \textit{ricordare} barely surpasses \textit{ricordarsi} (TR~=~51.7\%), and \textit{lamentarsi} drops in frequency but continues to be more common than \textit{lamentare} (PRO~=~68.3\%). This trend continues into the 21\textsuperscript{st} century.\footnote{The unexpected dip in the trendline for \textit{vantare}, which occurs in 20\textsuperscript{th} century, may be due to noise in the data.}

\subsection{\textit{Lamentar(si)}}

As represented in \tabref{tab:1:lament}, \textit{lamentare} most frequently occurs in constructions involving a direct object complement (43.16\%) and intransitive constructions with a null object complement (37.89\%), while \textit{lamentarsi} most often selects a null object complement (40.67\%) or and indirect object complement (37.08\%).

\begin{table}
\caption{Syntactic environments of \textit{lamentar(si)} (13\textsuperscript{th}--21\textsuperscript{st} c.)}
\label{tab:1:lament}
\small
 \begin{tabularx}{\textwidth}{lrrrYYrr}
  \lsptoprule
            & NP & PP & Null & Finite CP & Non-finite CP & Other & Total \\
  \midrule
  \textit{lamentare}  &  \cellcolor{lsLightGray}\textbf{43.16\%}   &    3.16\%  &    \cellcolor{lsLightGray}\textbf{37.89}\%      & 15.79\%  &   -- & -- & 100\%\\
  \textit{lamentarsi}  &   0.67\% &   \cellcolor{lsLightGray}\textbf{37.08}\%  &    \cellcolor{lsLightGray}\textbf{40.67}\%     & 17.53\% &   2.70\% & 1.35\% & 100\%\\
  \lspbottomrule
 \end{tabularx}
\end{table}

Over time, there is an increase in the selection of a finite CP for both verbs. The most frequent TR construction in the 13\textsuperscript{th} century is the intransitive construction with a null object complement (70\%), which is surpassed by the direct object complement in the 21\textsuperscript{st} century (78.95\%). \textit{Lamentarsi} most commonly selects an indirect object or a null object complement in the 13\textsuperscript{th} century, at 44\% and 46\% respectively. By the 21\textsuperscript{st} century, the frequency of the PP complement has decreased considerably (21.95\%), while that of the null construction has increased (58.54\%).
The PRO and TR verbs overlap significantly in meaning. With or without an object, they most commonly have the meaning ‘mourn, lament, complain (about)’, as in (\ref{ex:Lament Oblique}) for PRO and (\ref{ex:Lament Direct}) for TR. As in (\ref{ex:Lament Oblique}), the object is introduced mostly by the preposition \textit{di} ‘of’ or less often by \textit{per} ‘for’. As an intransitive, the non-pronominal form can also refer to the act of emitting sound while lamenting or suffering, that is, ‘wail, moan’. This latter meaning is not shared with the PRO verb.

\largerpage
\begin{exe}
\ex \label{Lamentarsi Set1} \begin{xlist}
    \ex \label{ex:Lament Oblique}
    \gll E molto \textbf{si} \textbf{lamentava} \textbf{di} \ul{Guerrino}, cioè, \ul{\textbf{della}} \ul{sua} 	\ul{morte} \ul{e} \ul{di} \ul{Bernardo} \ul{suo} \ul{fratello}, ch' era preso, ma	non	sapeva dove s' era, s' egli era	preso o morto.\\
    and much \textsc{se}.\textsc{3sg} lament.\textsc{ipfv.pst.3sg} of Guerrino that.is of.\textsc{def.det.msg} his.\textsc{fsg} death.\textsc{fsg} and of Bernardo his.\textsc{msg} brother.\textsc{msg} who be.\textsc{aux.ipfv.pst.3sg} take.\textsc{pst.ptcp} but \textsc{neg} know.\textsc{ipfv.pst.3sg} where \textsc{se}.\textsc{3sg} be.\textsc{aux.ipfv.pst.3sg} if he be.\textsc{aux.ipfv.pst.3sg} take.\textsc{pst.ptcp} or dead.\textsc{msg}\\ \jambox*{(\textit{I reali di Francia}, 1491)}
    \clearpage
    \glt ‘And much did he [i.e., Gherardo] mourn Guerrino, that is, his death, and Bernard his brother, who was taken captive, but he did not know where he was, if he had been taken captive or was dead.’
    \ex\label{ex:Lament Direct}
    \gll Quella dispicca un vol sopra il pollone D' un vecchio salcio, e colassù \textbf{lamenta} \ul{Il} \ul{suo} \ul{timor} pe’ tenerelli aspetti:\\
    that.\textsc{fsg} take.off.\textsc{prs.3sg} \textsc{indf.det.msg} flight.\textsc{msg} on \textsc{def.det.msg} shoot.\textsc{msg} of \textsc{indf.det.msg} old.\textsc{msg} willow.\textsc{msg} and up.there lament.\textsc{prs.3sg} \textsc{det.det.msg} his.\textsc{msg} dread.\textsc{msg} for tender.\textsc{mpl} aspect.\textsc{mpl}\\  \jambox*{(\textit{Il primo bacio}, 18\textsuperscript{th} c.)}
    \glt ‘And she [i.e., the hen] flies off up to the branch of the old willow and from there laments her dread for her tender little ones.’
\end{xlist}
\end{exe}

These examples document the most common environments of \textit{lamentar(si)} involving a logical object. As indicated in \tabref{tab:1:lament}, less common constructions include ones where the PRO verb selects an NP complement as in (\ref{ex:Lament PRO+NP}) and the TR verb occurs with an PP complement as in (\ref{ex:Lament TR+PP}), as well as constructions with the impersonal \textit{si}, reflexive \textit{si}, and adjectival phrase complements.

\begin{exe}
\ex \label{Lamentarsi Set2} \begin{xlist}
    \ex PRO + NP \\ \label{ex:Lament PRO+NP}
    \gll Udite tucti comunamente come Dio omnipotente \textbf{si} \textbf{lamenta} \ul{chi} l' ofende, et duramente li riprende di ciò che tucte criature, segondo le loro nature, connosceno lo lor criatore meglo che l' omo a tucte hore;\\
    hear.\textsc{prs.2pl} all.\textsc{mpl} together how God almighty.\textsc{msg} \textsc{se}.\textsc{3sg} lament.\textsc{prs.3sg} who him offend.\textsc{prs.3sg} and harshly them.\textsc{m} reprimand.\textsc{prs.3sg} of that which all.\textsc{fpl} creature.\textsc{fpl} according.to \textsc{def.det.fpl} their nature.\textsc{fpl} know.\textsc{prs.3pl} \textsc{def.det.msg} their creator.\textsc{msg} better than \textsc{def.det.msg} man.\textsc{msg} at all.\textsc{fpl} hour.\textsc{fpl} \\ \jambox*{(\textit{Quindici segni del giudizio}, 1270)}
    \glt ‘Hear all together how God Almighty laments those who offend him and harshly reprimands them with respect to the fact that all creatures according to their nature know their creator better than man at any hour;’
    \newpage
    \ex TR + PP\\ \label{ex:Lament TR+PP}
    \gll Vedro=gli in un voler tutti dispor=si con meco a \textbf{lamentar} \ul{\textbf{della}} {\ul{mie}\footnotemark} \ul{pena}, fin che’ pianeti aràn fatti lor corsi, […].\\
    see.\textsc{fut.1sg}=him/them.\textsc{m} in \textsc{indf.det.msg} want all.\textsc{mpl} arrange.\textsc{inf}=si with with.me to lament.\textsc{inf} of.\textsc{def.det.fsg} my.\textsc{fsg} trouble.\textsc{fsg} until that planet.\textsc{mpl} have.\textsc{aux.fut.3pl} make.\textsc{pst.ptcp} their course.\textsc{mpl} [...]\\ \jambox*{(\textit{Rime varie}, 15\textsuperscript{th} c.)}
    \glt ‘I will see him/them all wanting to prepare themselves to lament my trouble/s with me, as long as the planets are running their course [...]’
\end{xlist}
\end{exe}
\footnotetext{[sic]}

The sentences in (\ref{ex:Lament PRO+NP}) and (\ref{ex:Lament TR+PP}) illustrate the diversity of constructions in the dataset, particularly in texts before the 16\textsuperscript{th} century. The variability suggests that this is a period of change.

\subsection{\textit{Ricordar(si)}}

This verb pair has the meaning ‘remember/remind’. Generally, with a null object complement, \textit{ricordarsi} and \textit{ricordare} have the definition of ‘remember’. This use appears to have become a fixed expression, similar to the English ‘as far as I remember’. With an indirect object (the addressee of the act of reminding) and a direct object, \textit{ricordare} means ‘remind/recall’. With only the direct object, it can also have the meaning ‘recount, record’, as for example in the context of historical events.

The general distribution for \textit{ricordar(si)} is represented in Table \ref{tab:1:ricord}. While both verbs select a finite complementizer phrase (usually introduced by \textit{che} ‘that’) at similar frequencies, the most frequent complements of the TR and PRO verbs are the direct object and the indirect object, respectively.


\begin{table}
\caption{Syntactic environments of \textit{ricordar(si)} (13\textsuperscript{th}--21\textsuperscript{st})}
\label{tab:1:ricord}
\small
 \begin{tabularx}{\textwidth}{lrrrYYrr}
  \lsptoprule
            & NP & PP & Null & Finite CP & Non-finite CP & Other & Total\\
  \midrule
  \textit{ricordare}  &   \cellcolor{lsLightGray}\textbf{65.00\%}  &    2.00\%  &    8.00\%      & \cellcolor{lsLightGray}\textbf{21.00\%}  &   3.50\% & 0.50\% & 100\%\\
  \textit{ricordarsi}  &   7.94\% &   \cellcolor{lsLightGray}\textbf{45.59\%}  &    12.35\%     & \cellcolor{lsLightGray}\textbf{23.82\%} &   9.71\% & 0.59\% & 100\% \\
  \lspbottomrule
 \end{tabularx}
\end{table}

Already the earliest data from the 13\textsuperscript{th} show a similar pattern.The data records few cases of the intransitive construction with a null object complement for \textit{ricordare} until the 19\textsuperscript{th} century. The construction accounts for 23\% of the TR constructions in the 21\textsuperscript{st} century, however. The most notable change for \textit{ricordarsi} is the increase in frequency of the NP complement, as seen in (\ref{ex:Ricord PRO+NP}).

\begin{exe}
\ex PRO + NP\\ \label{ex:Ricord PRO+NP}
    \gll Ogni volta che \textbf{si} \textbf{ricorda} \ul{quel} \ul{nome} \ul{glorioso}, pieghi i ginocchi del suo cuore;\\
    each time.\textsc{fsg} that \textsc{se}.\textsc{3sg} remember.\textsc{prs.3sg} that.\textsc{msg} name.\textsc{msg} glorious.\textsc{msg} bend.\textsc{prs.sbjv.3sg}	\textsc{def.det.mpl} knee.\textsc{mpl} of.\textsc{def.det.msg} his.\textsc{msg} heart.\textsc{msg} \\ \jambox*{(\textit{Il Concilio di Lione II}, 1274)}
    \glt ‘Each time that one remembers that glorious name, one bends the knees of one’s heart.’
\end{exe}

In contrast to the PRO + PP construction, the PRO construction in (\ref{ex:Ricord PRO+NP}) maps onto the transitive construction \textit{ricordare qualcosa a qualcuno}, meaning that the logical object is realized as an NP. It is already present in 13\textsuperscript{th}-century data with two occurrences (4.65\%) and increases gradually until it represents 38.1\% of PRO constructions in the 21\textsuperscript{st} century. It appears to be in competition with the PRO + PP construction, which decreases significantly in the last two centuries of the data and only represents 33.33\% of PRO.

Some cases of \textit{ricordare} selecting an indirect object complement are recorded in the earlier periods of the data, as seen in (\ref{ex:Ricord ne}). The indirect object complement is marked in (\ref{ex:Ricord ne}) by the partitive clitic \textit{ne}, which refers to a PP headed by the preposition \textit{di} ‘of’.

\begin{exe}
\ex \label{ex:Ricord ne}
    \gll Ma vendichi alle molte volte grandemente, a tal otta che a pena \ul{ne} \textbf{ricorda} a chi l' ha fatto- ma a noi non esce di mente mai.\\
    but avenge.\textsc{prs.sbjv.3sg} to.\textsc{def.det.fpl} many.\textsc{fpl} time.\textsc{fpl} greatly at such time.\textsc{fsg} that to trouble.\textsc{fsg} of.it remind.\textsc{prs.3sg} to who it.\textsc{m} have.\textsc{aux.prs.3sg} do.\textsc{pst.ptcp} but to us \textsc{neg} leave.\textsc{prs.3sg} from mind.\textsc{fsg} never \\ \jambox*{(\textit{Il Libro de' Vizî e delle Virtudi}, 1292)}
    \glt ‘But he [i.e., God] punishes harshly many times, at such a time that he who did it hardly remembers it, but we never forget.’
\end{exe}

\subsection{\textit{Vantar(si)}}

In most cases, the PRO and TR verbs mean ‘boast, praise’. In early texts, during the period of medieval courts and chivalry, \textit{vantarsi} can have the meaning ‘pledge (oneself)’, which refers to a future exploit rather than a current or past one. As a pledge made to a person (or being), it follows the schema \textit{vantarsi a} + person/being. This meaning is unique to the PRO verb, which underlines its close relationship with the reflexive use of \textsc{se}. An example is given in (\ref{ex:Vantare a}).

\begin{exe}
\ex PRO + NP \\ \label{ex:Vantare a}
    \gll Ma Parmenione \ul{che} \ul{d'} \ul{adestrare} \ul{Biancifiore} \ul{a} \ul{casa} \ul{del} \ul{novello} \ul{sposo} \textbf{s'} \textbf{era} al paone \textbf{vantato}, [...] con Alcipiades [...], e con alcuni altri giovani nobili della città, [...], al freno di Biancifiore vennero, [...].\\
    but Parmenione that to ride.beside.\textsc{inf} Biancifiore to house.\textsc{fsg} of.\textsc{def.det.msg} new.\textsc{msg} bridegroom.\textsc{msg} \textsc{se}.\textsc{3sg} be.\textsc{aux.ipfv.pst.3sg} to.\textsc{def.det.msg} peacock.\textsc{msg} boast.\textsc{pst.ptcp} [...] with Alcipiades [...] and with some.\textsc{mpl} other.\textsc{mpl} young.\textsc{mpl} noble.\textsc{mpl} of.\textsc{def.det.fsg} city.\textsc{fsg} [...] to.\textsc{def.det.msg} rein.\textsc{msg} of Biancifiore come.\textsc{pfv.pst.3pl} [...]\\ \jambox*{(\textit{Filocolo} 4.163, 1336)}
    \glt ‘but Parmenione, who had pledged before the peacock to ride beside Biancifiore to the bridegroom's house, [...], and so along with Alcibiades, [...], and other young nobles of the city, [...], he came up to Biancifiore's reins, [...].’ (from Boccaccio 1985: 370)
\end{exe}

The most common complements of \textit{vantar(si)} are found in \tabref{tab:1:vant}. In 90\% of TR occurrences, \textit{vantare} selects an NP complement, while \textit{vantarsi} displays an array of complements. The most frequent is the non-finite complementizer phrase (40.60\%), as in the expression \textit{vantarsi di fare qualcosa} ‘boast about doing something’. This is followed by the PP complement at 33.51\%.

\begin{table}
\caption{Syntactic environments of \textit{vantar(si)} (13\textsuperscript{th}--21\textsuperscript{st} c.)}
\label{tab:1:vant}
\small
 \begin{tabularx}{\textwidth}{lrrrYYrr}
  \lsptoprule
            & NP & PP & Null & Finite CP & Non-finite CP & Other & Total \\
  \midrule
  \textit{vantare}  &   \cellcolor{lsLightGray}\textbf{90.20\%}  &    -  &    2.61\%      & 1.96\% & 3.92\%  & 1.31\% & 100\%\\
  \textit{vantarsi}  &   1.63\% &   \cellcolor{lsLightGray}\textbf{33.51\%}  &    13.62\%     & 8.17\% &   \cellcolor{lsLightGray}\textbf{40.60\%} & 2.00\% & 100\%\\
  \lspbottomrule
 \end{tabularx}
\end{table}

For both \textit{vantare} and \textit{vantarsi}, the dataset also revealed copula constructions, which are included in \tabref{tab:1:vant} in the category ``Other''. The attributes introduced in these constructions are nouns (e.g., \textit{inventore del poema eroicomico} ‘inventor of mock-heroic poetry’), past participles (e.g., \textit{nato} ‘born’), and adjectives (e.g., \textit{opportuno} ‘timely, ready’), as in (\ref{ex:Vantare ADJ}). They appear to derive from elliptical constructions meaning ‘boast about being’ or ‘pridefully claim to be’.

\begin{exe}
\ex \label{ex:Vantare ADJ}
    \gll Ecco l’ astuccio, Di pelli rilucenti ornato e d’ oro, Sdegnar la turba, e gli occhi tuoi primiero Occupar di sua mole: esso \ul{a} \ul{cent’} \ul{usi} \ul{Opportuno} \textbf{si} \textbf{vanta}; e ad esso in grembo, Atta agli orecchi, ai denti, ai peli, all’ ugne, Vien forbita famiglia.\\
    here.it.is \textsc{def.det.msg} case.\textsc{msg} of skin.\textsc{fpl} shining.\textsc{fpl} adorn.\textsc{pst.ptcp} and of gold.\textsc{msg} disdain.\textsc{inf} \textsc{def.det.fsg} throng.\textsc{fsg} and \textsc{def.det.msg} eye.\textsc{mpl} your.\textsc{mpl} first.\textsc{msg} hold.\textsc{inf} of its.\textsc{f} bulk.\textsc{fsg} it.\textsc{m} to one.hundred usage.\textsc{mpl} appropriate.\textsc{msg} \textsc{se}.\textsc{3sg} boast.\textsc{prs.3sg} and to it.\textsc{m} in lap.\textsc{msg} apt.\textsc{fsg} to.\textsc{mpl} ear.\textsc{mpl} to.\textsc{mpl} tooth.\textsc{mpl} to.\textsc{mpl} hair.\textsc{mpl} to.\textsc{fpl} nail.\textsc{fpl} come.\textsc{prs.3sg} polished.\textsc{fsg} family.\textsc{fsg} \\ \jambox*{(\textit{Il giorno}, 1763)}
    \glt ‘I see, the throng disdaining with bulk that catches first thine eye, the case adorn'd with glossy skin and gold, whose boast is to be ready for a thousand needs, for in its lap a polish'd family it bears; apt are they for the ears, the teeth, the hair, the nails.’ (from Parini 1977: 68)
\end{exe}

From the 13\textsuperscript{th} century on, the direct object is the most common complement of TR forms. There are some cases of finite and non-finite complementizer phrases throughout the dataset, but they are not frequent in general. By contrast, the indirect object and the null object are the most frequent PRO complements in the 13\textsuperscript{th} century. The PRO construction involving the non-finite complementizer increases over time, from 15.79\% in the 13\textsuperscript{th} century to 37.50\% in the 21\textsuperscript{st} century. Interestingly, \textit{vantarsi} is the verb most consistently used across centuries in the intransitive + null object construction.

\section{Discussion} \label{discussion}

Constructions involving a logical object complement are an area of significant change between the 13\textsuperscript{th} and 21\textsuperscript{st} centuries and account for 49\% of the dataset. There are two main ways that a logical object is encoded by these verbs, namely as an oblique or as a direct object. In the former case, the PRO verb selects a PP, introduced by \textit{di} ‘of’, as in (\ref{ex:Disc:PRO+PP}). In the latter case, the TR verb selects an NP complement, as in (\ref{ex:Disc:TR+NP}). For one verb, \textit{ricordarsi}, there is a third option, in which the PRO form selects an NP, as in (\ref{ex:Disc:PRO+NP}).
%\newpage

\begin{exe} %This example is a widow :( 
\ex \label{ex:Disc} \begin{xlist}
    \ex PRO + PP \label{ex:Disc:PRO+PP} \\
    \gll Il Signor Chiari \textbf{si} \textbf{vantava} \ul{\textbf{d'}} \ul{uno} \ul{stile} \ul{pindarico} \ul{e} \ul{sublime};\\
    \textsc{def.det.msg} Mr.\textsc{msg} Chiari \textsc{se}.\textsc{3sg} boast.\textsc{ipfv.pst.3sg} of \textsc{indf.det.msg} style.\textsc{msg} Pindaric.\textsc{msg} and sublime.\textsc{msg}\\ \jambox*{(\textit{L'amore delle tre Melarance colle alusioni al Goldoni e al Chiari}, 1835)}
    \glt ‘Mr. Chiari boasted of a Pindaric and sublime style.’
    \ex TR + NP \label{ex:Disc:TR+NP} \\
    \gll Una volta almeno gli Italiani potevano \textbf{vantare} \ul{il} \ul{bel} \ul{cielo} \ul{d'} \ul{Italia}.\\
    \textsc{indf.det.fsg} time\textsc{fsg} at.least \textsc{def.det.mpl} Italian.\textsc{mpl} can.\textsc{ipfv.pst.3pl} praise.\textsc{inf} \textsc{def.det.msg} beautiful.\textsc{msg} sky.\textsc{msg} of Italy \\ \jambox*{(\textit{L'umiltà nazionale}, 1871)}
    \glt ‘At least one time, the Italians were able to praise the beautiful sky of Italy.’
    \ex PRO + NP \label{ex:Disc:PRO+NP} \\
    \gll e benché molti intendano meglio di me questa materia, penso non di meno di poter=ne significar il mio parere, e tanto più quanto \textbf{mi} \textbf{ricordo} \ul{il} \ul{danno} che averebbe potuto far=mi lo sfrenato amor di dir il vero, di che non mi son pentito;\\
    and although many.\textsc{mpl} understand.\textsc{sbjv.prs.3pl} better of me this.\textsc{fsg} matter.\textsc{fsg} think.\textsc{prs.1sg} \textsc{neg} of less of can.\textsc{inf}=of.it express.\textsc{inf} \textsc{def.det.msg} my.\textsc{msg} opinion.\textsc{msg} and so.much more as.much.as \textsc{se}.\textsc{1sg} remember.\textsc{prs.1sg} \textsc{def.det.msg} damage.\textsc{msg} \textsc{obj.rel} have.\textsc{aux.cond.3sg} can.\textsc{pst.ptcp} do.\textsc{inf}=me \textsc{def.det.msg} wild.\textsc{msg} love.\textsc{msg} of say.\textsc{inf} \textsc{def.det.msg} truth.\textsc{msg} of that \textsc{neg} \textsc{se}.\textsc{1sg} be.\textsc{aux.prs.1sg} repent.\textsc{pst.ptcp}\\ \jambox*{(\textit{Della dissimulazione onesta}, 1641)}
    \newpage
    \glt ‘And although many understand this matter better than me, I think at least that I am able to express my point of view, and even more so since I remember the damage that the/my wild love for speaking the truth could have caused me, of which I have not repented.’
\end{xlist}
\end{exe}

In (\ref{ex:Disc:PRO+PP}), the clitic \textit{si} does not fit into the classifications offered by previous prescriptive and descriptive accounts. Its use is close to the “inherent \textit{se}” or “inherently reflexive \textit{se}”, as described by  \citet[426]{nishida_spanish_1994} and by \citet[8]{medova_reflexive_2009}, respectively, which are illustrated in (\ref{ex:Inherent se}) and (\ref{ex:Inherent refl}):

\begin{exe}
\ex ``Inherent \textit{se}": Spanish \textit{arrepentirse} -- *\textit{arrepentir} \\ \label{ex:Inherent se}
    \gll Juan \textbf{se} arrepintió de haber=lo hecho.\\
    Juan \textsc{se}.\textsc{3sg} repent.\textsc{pfv.pst.3sg} of have.\textsc{inf=obj.3msg} have.\textsc{pst.ptcp}\\
    \glt ‘John repented (himself) having done it.’ \citep[426]{nishida_spanish_1994}
\end{exe}

\begin{exe}
\ex ``Inherently reflexive \textit{se}": Italian \textit{accorgersi} -- *\textit{accorgere} \\ \label{ex:Inherent refl}
    \gll Paolo non \textbf{si} è accorto di niente.\\
    Paolo \textsc{neg} \textsc{se}.\textsc{3sg} be.\textsc{aux.prs.3sg} notice.\textsc{pst.ptcp} of nothing\\
    \glt ‘Paolo hasn't noticed anything.’ \citep[76]{sorace_unaccusativity_1993}
\end{exe}

\begin{exe}
\ex Italian \textit{ricordarsi di -- ricordare} \\ \label{ex:PRO TR}
    \gll [...] \textbf{si} \textbf{ricordò} \ul{\textbf{del}} \ul{sogno} e corse in gardino, vicino al fiume, dove dormendo l' aveva veduta.\\
    [...] \textsc{se}.\textsc{3sg} remember.\textsc{pfv.pst.3sg} of.\textsc{def.det.msg} dream.\textsc{msg} and run.\textsc{pfv.pst.3sg} in garden.\textsc{msg} near to.\textsc{def.det.msg} river.\textsc{msg} where sleeping her have.\textsc{aux.ipfv.pst.3sg} see.\textsc{pst.ptcp} \\ \jambox*{(\textit{I racconti delle fate}, 1876)}
    \glt ‘[...] he remembered the dream and ran into the garden, near the river where sleeping, he had seen her.’
\end{exe}

Similar to \textit{ricordarsi} in (\ref{ex:PRO TR}), the pronominal verbs in (\ref{ex:Inherent se}) and (\ref{ex:Inherent refl}) display a pronoun \textit{se/si} that cannot be interpreted as the object of the verb, direct or indirect. The difference between (\ref{ex:Inherent se}), (\ref{ex:Inherent refl}) and (\ref{ex:PRO TR}) lies in the existence of a transitive counterpart for \textit{ricordarsi}, i.e., \textit{ricordare}. According to \citet[8]{medova_reflexive_2009}, the inherently reflexive \textit{se} distinguishes itself from the true reflexive by the absence of a corresponding transitive form. However, the expression in (\ref{ex:PRO TR}) cannot be interpreted as a true reflexive nor does it correspond to the characteristics of an inherently reflexive verb, with out a transitive counterpart. The process observed in (\ref{ex:PRO TR}) is the demotion of the logical object to a non-core argument along the hierarchy of grammatical roles. This is a characteristic of the AP construction, which may also entirely suppress the logical object. Examples of this may be found in the intransitive PRO constructions that are not true reflexives and do not select an overt complement.\footnote{Such cases are coded to have a null object complement.} Interestingly, the use of intransitive TR constructions decreases over time for \textit{lamentare} and \textit{vantare}, while the corresponding PRO constructions increase or stay consistent across time. While \textit{ricordar(si)} features the intransitive construction less frequently overall, its frequency declines for \textit{ricordarsi} and increases for \textit{ricordare} in recent centuries.

Additional diagnostics, as described by \citet{polinsky_13._2017}, serve to better examine evidence for the AP. As is the case for AP constructions, the PRO verbs of this study are transitive in meaning, although they are syntactically intransitive (through the presence of \textit{si}). In terms of morphology, the AP has bearings on case-marking, whereby the non-core status of the object is signaled by case inflection, e.g., an oblique case. Romance case inflection is greatly depleted since Classical Latin, but the use of prepositions increased and their functions were extended to cover functions previously fulfilled by the case system. As seen for (\ref{ex:PRO TR}), the non-core argument status is thus marked by the preposition \textit{di}. Evidence from Chukchi (Chukotko-Kamchatkan: Russia) suggests that the logical object can also be left unexpressed without a great loss in meaning \citep[7]{polinsky_13._2017}, which may explain the presence of PRO constructions in the dataset that follow the pattern PRO + Null and that are not true reflexives. As observed for other languages that exhibit an AP, it displays a type of “verbal affixation” \citep[7]{polinsky_13._2017}, which may serve as a more general detransitivizing affix, found in other contexts marking reflexive/reciprocal, middle, passive, and aspect, among others. These characteristics are reflected in Romance \textsc{se}, which is similarly polyfunctional. It also functions as a marker of the anticausative, reflexive/reciprocal, etc. and detransitivizes transitive constructions.

The AP has a pragmatic effect in that it places the subject in a position of prominence, while demoting the object to a place of less prominence. This is called ``subject prominence” or ``agent foregrounding” \citep[9]{polinsky_13._2017}. The prominence of the subject in the Italian pronominal verbs of this study is indicated not only by the demotion of the logical object, but also by the presence of the clitic \textit{si}, which refers back to the subject, therefore highlighting its position. This concept is further analyzed below with respect to the mechanisms underlying the development of AP morphology.

\hspace*{-2.2pt}The historical perspective of this phenomenon is represented in \figref{fig:distlobj}, which traces the distribution of the logical object for seven constructions across time.\footnote{Cases such as \textit{lamentare} + PP, \textit{lamentarsi} + NP, \textit{ricordare} + PP, and \textit{vantarsi} + NP are excluded from \figref{fig:distlobj}, as they account for only 2\% of the data. Also, the construction \textit{vantare} + PP does not appear in the data.}

\begin{figure}[h]

    \includegraphics[width=\textwidth]{figures/High_fig2.png}
    \caption{Distribution of logical object across time}
    \label{fig:distlobj}
\end{figure}

The earliest periods of the corpus denote a stark contrast between the frequencies of the PRO verbs and the TR verbs. According to \figref{fig:distlobj}, the PRO or AP construction is strongly preferred until at least the 17\textsuperscript{th} century, when \textit{vantare} + NP surpasses \textit{vantarsi} + PP. In the same period, \textit{lamentarsi} + PP continues to dominate at $>$90\% and \textit{ricordarsi} + PP, while still more frequent than \textit{ricordare} + NP, continues its gradual decline that had started in the 15\textsuperscript{th} century. This decrease of \textit{ricordarsi} + PP is accompanied by an increase in \textit{ricordarsi} + NP, which surpasses the AP construction in the 21\textsuperscript{st} century, however. \textit{Ricordare} + NP and \textit{ricordarsi} + PP are at close to equal distribution in the 18\textsuperscript{th} century and the TR verb stabilizes as the preferred construction. There is a sharp decline in \textit{lamentarsi} + PP from the 18\textsuperscript{th} to the 19\textsuperscript{th} century, which consequently is less frequent than \textit{lamentare} + NP for the first time. From the 19\textsuperscript{th} until the 21\textsuperscript{st} century, the abrupt changes in the trendlines point towards noisy or insufficient data; nonetheless, the trend started in earlier centuries continues – TR constructions are preferred for expressing logical objects. This is a considerable change from the 13\textsuperscript{th} century, when the PRO constructions dominated at $>$70\% of logical object expressions.

While the data convincingly display this change, it is harder to pinpoint the period in which these AP constructions may have developed. The AP constructions are already represented in the earliest stages of the corpus. However, the presence of the detransitivizing \textit{si} signals a connection with the reflexive/reciprocal construction, which is a well-studied source of AP markers.  \citet{sanso_where_2017, sanso_sources_2019} identified the reflexive/reciprocal construction as one of four sources of the AP marker across a 120-language sample and terms it the ``best-documented polysemy pattern involving AP constructions” \citeyearpar[193]{sanso_where_2017}. \citet{creissels_origin_2012} reconstructed the Proto-West-Mande suffix *-i as the source of a detransitivizing suffix that grammaticalized into the reflexive pronoun \textit{í} for Mandinka, among other Mande languages, and functions as an AP marker for some verbs \citeyearpar[15]{creissels_origin_2012}. This is also observed and studied by \citet{janic_slavonic_2013} for Slavonic (specifically, Polish and Russian). An earlier paper by \citet{terrill_development_1997}, focusing on the development of AP in Australian languages, examined the diachronic processes by which the verbal morphology of reflexive constructions is reanalyzed and extended to AP constructions, first to a pragmatic AP and then to a structural AP. Her proposal sheds light on a sequence of mechanisms that underlie this change, which could account for the development found in this paper’s Italian data.

As with the AP constructions in the Pama-Nyungan languages described by \citep{terrill_development_1997}, the PRO verbs of this sample share verbal morphology with the reflexive construction. Terrill suggests that AP constructions develop from reflexive constructions via extension of their pragmatic function. Reflexive constructions display low transitivity; not only are they semantically and syntactically less transitive than the corresponding non-reflexive verbs, they also tend to have low-transitivity verbs, non-agent subjects, and ``non-distinct'' objects \citep[81]{terrill_development_1997}, and their agent and object are coreferential. In AP constructions, the patient similarly has low prominence and the verbs display lowered transitivity, but the agent and object are not coreferential. By extending the verbal morphology from the reflexive environment to the AP environment, a similar pragmatic situation is maintained, although the agent and object are no longer coreferential. It is plausible that a similar mechanism operated in the extension of the function of reflexive \textit{si} to the AP construction. Support for this is found in the dataset, as seen in (\ref{ex:Lament ambig}), where there is an ambiguous reading between a reflexive -- a woman (\textit{ella}) bemoaning herself -- or an AP -- a woman grieving [someone]:

\begin{exe}
\ex \label{ex:Lament ambig}
    \gll [...] onde io veggendo ritornare alquante donne da lei, udio dicere loro parole di questa gentilissima, com' ella \textbf{si} \textbf{lamentava};\\
     [...] so.that I see.\textsc{ger} return.\textsc{inf} some.\textsc{fpl} woman.\textsc{fpl} from her hear.\textsc{pfv.pst.1sg} say.\textsc{inf} them word.\textsc{fpl} of this.\textsc{fsg} gracious.\textsc{sup.fsg} how she \textsc{se}.\textsc{3sg} lament.\textsc{ipfv.pst.3sg} \\ \jambox*{(\textit{La vita nuova}, 1292)}
    \glt `Seeing some ladies come away from her I heard them describe Beatrice's lamentations [lit. how she grieved].' (from Alighieri 1964: 71)
\end{exe}

The context of this excerpt reveals that the woman is grieving the death of her father and therefore provides evidence for an AP reading. Terrill also suggests that after a first extension of the reflexive to the similar pragmatic situation of the AP construction, its function is reanalyzed as both reflexive and AP. This is considered a pragmatic AP. In a third stage, a new construction emerges, the syntactic AP. It maintains the pragmatic AP’s structure, but its pragmatic function becomes secondary, demoted by the structural function.

In this account of the development of the AP, the transition from one stage to another is facilitated by one or more shared characteristics—first pragmatic/semantic, then structural. The data appear to mirror this development. However, it does not explain the decrease in frequency of the AP construction with a logical object. This may be due to the loss of the pragmatic function, and the syntactic AP may have subsequently competed with the transitive construction. Despite this development, the constructions in which \textit{lamentarsi} and \textit{vantarsi} select a null object complement, as in (\ref{ex:Lament ambig}), represent a significant percentage of each verb pair’s occurrences in the 21\textsuperscript{st} century, at 58\% and 20.83\% respectively. They may be remnants of AP constructions, which have been lexicalized. I also propose that \textit{ricordarsi} + NP, which is already present in the 13\textsuperscript{th}-century data and becomes more frequent than \textit{ricordarsi} + PP in the 21\textsuperscript{st} century, existed as a competitor to the AP construction. As the pragmatic function receded, the AP morpheme \textit{si} may have been reanalyzed as a dative reflexive, a function which was already present at this point. This developed into the dominant PRO construction in the 21\textsuperscript{st} century. However, this is an initial, tentative explanation of the diachronic processes triggering change in this dataset, which would require further data to answer more definitively.

The changing and at times ambiguous meanings of the verbs provide another perspective in this historical narrative. It is possible to find contexts in which the reading of \textit{lamentar(si)} is ambiguous, as the boundary between lamenting oneself (implies inner torment or other suffering) and complaining (implies dissatisfaction) can be vague. Also, the act of remembering is almost inextricable from the (unconscious) act of reminding oneself of something. As for \textit{vantarsi}, the more reflexive meaning of ‘pledge (oneself)’ in the 13\textsuperscript{th} century may have provided the starting point from which the pragmatic AP construction developed. With the presence of the reflexive meaning and the AP construction in the 13\textsuperscript{th} c., it is at least possible to suggest that the emergence of the AP was underway.

\section{Conclusion} \label{conclusion}

In response to the questions laid out in the introduction, the PRO verbs and their TR counterparts display a great deal of variation in terms of their distribution across time and syntactic environments. The overall distribution reveals an important trend: the PRO forms are more frequent overall, but experience a decline starting in the 16\textsuperscript{th} century for \textit{vantarsi}. This development is reflected in the distribution of the logical object constructions, where constructions with PRO forms were preferred early on as well. However, the TR forms start to dominate from the 17\textsuperscript{th} century onwards, which suggests a decline of the AP construction.

I propose three lines of inquiry that could deepen and broaden this study: analyzing the diachronic relationship between the semantic roles of these verbs and their argument structure, examining dialectal variation in Italo-Romance, and determining if there are similar patterns across Romance languages.

As suggested in \sectref{discussion}, the low transitivity of \textit{lamentar(si)}, \textit{ricordar(si)}, and \textit{vantar(si)} facilitated the extension of reflexive verbal morphology to the AP construction. This is reflected in the semantic roles of these verbs: Experiencer subjects with low agentivity (and volition) and Theme objects that are little or not affected by the action. Change over time of semantic roles could account for diachronic variation of constructions and support the proposal for the emergence of the AP construction.

For the purpose of this study, I excluded data that presented prominent non-Tuscan features. A further study could include these data and examine the extent to which interdialectal contact shapes the use and distribution of the AP construction.
The presence of verb pairs with similar characteristics in other Romance languages, such as French \textit{(se) vanter (de)} `praise, boast’, suggests that a similar pattern might exist more broadly in Romance. It remains to be examined if the AP construction affects the same classes of verbs and if its distribution follows a comparable trajectory across time. Additional diachronic studies in other Romance languages examining these constructions could provide further evidence in favor of the AP construction as a Romance phenomenon, while also tracing its emergence back to a common source. This would be a valuable contribution to the historical research of the AP, which has tended to focus on ergative languages and other accusative languages.


\section*{Abbreviations}
\begin{tabularx}{.45\textwidth}{lQ}
1 & first person\\
2 & second person\\
3 & third person\\
\textsc{AP} & antipassive \\
\textsc{art} & article \\
\textsc{aux} & auxiliary\\
\textsc{cond} & conditional\\
\textsc{CP} & complementizer phrase \\
\textsc{def} & definite\\
\textsc{det} & determiner\\
\textsc{ger} & gerund\\
\textsc{f} & feminine\\
\textsc{fsg} & feminine singular\\
\textsc{fut} & future\\
\textsc{indf} & indefinite\\
\textsc{inf} & infinitive\\
\end{tabularx}
\begin{tabularx}{.45\textwidth}{lQ}
\textsc{ipfv} & imperfective\\
\textsc{m} & masculine\\
\textsc{msg} & masculine singular\\
\textsc{neg} & negation, negative\\
\textsc{NP} & noun phrase \\
\textsc{pfv} & perfective\\
\textsc{pl} & plural\\
\textsc{PP} & prepositional phrase \\
\textsc{PRO} & pronominal \\
\textsc{prs} & present\\
\textsc{pst} & past\\
\textsc{ptcp} & participle\\
\textsc{sbjv} & subjunctive\\
\textsc{sg} & singular\\
\textsc{sup} & superlative\\
\textsc{TR} & transitive \\
\end{tabularx}

\section*{Acknowledgements}
I wish to thank Cinzia Russi for her helpful feedback on data and drafts of this article, as well as the three anonymous reviewers for their valuable comments. Any errors remain my own.

\section*{Corpora}
\smallskip Opera del Vocabolario della lingua Italiana (OVI, \url{https://artfl-project.uchicago.edu/content/ovi}).\\
\smallskip Corpus Tesoro della lingua Italiana delle Origini (TLIO, \url{http://tlioweb.ovi.cnr.it/(S(nir13gk3cbh0jfugkf2xow3a))/CatForm01.aspx}).\\
\smallskip Corpus di italiano scritto (CORIS/CODIS, \url{http://corpora.dslo.unibo.it/coris_ita.html}).\\
\smallskip Biblioteca dei Classici Italiani (\url{http://www.classicitaliani.it/}).\\
\smallskip IntraText Digital Library (IntraText, \url{http://www.intratext.com/}).\\
\smallskip Corpus Diacronico dell’Italiano Scritto (DiaCORIS, \url{http://corpora.dslo.unibo.it/DiaCORIS/}).


\section*{Primary Sources}
\smallskip Accetto, Torquato. 1641. \textit{Della dissimulazione onesta}. (IntraText)\\
\smallskip Alighieri, Dante. 1292. \textit{Vita nuova}. (OVI)\\
\smallskip Alighieri, Dante. 1964. \textit{The new life. La vita nuova. Translated with an introduction by William Anderson.} Trans. by William Anderson (Penguin classics). Baltimore: Penguin Books.\\
\smallskip Boccaccio, Giovanni. 1336. \textit{Filocolo}. (IntraText)\\
\smallskip Boccaccio, Giovanni. 1985. \textit{Il Filocolo / Giovanni Boccaccio; translated by Donald Cheney with the collaboration of Thomas G. Bergin.} Trans. by Donald Cheney and Thomas G. Bergin. New York: Garland Pub.\\
\smallskip Collodi, Carlo. 1876. \textit{I racconti delle fate}. (Biblioteca dei Classici Italiani)\\
\smallskip Collodi, Carlo. 1871. \textit{L’Umiltà nazionale. Il Fanfulla}. (Biblioteca dei Classici Italiani)\\
\smallskip da Barberino, Andrea. 1491. \textit{I reali di Francia}. (IntraText)\\
\smallskip Giamboni, Bono. 1292. \textit{Il Libro de’ Vizî e delle Virtudi}. (IntraText)\\
\smallskip Giamboni, Bono. 2013. \textit{Le Livre des vices et des vertus}. Trans. by Sylvain Trousselard \& Elisabetta Vianello (Textes littéraires du Moyen Âge 23). Paris: Classiques Garnier.\\
\smallskip Giambullari, Bernardo. 15th. \textit{Rime varie}. (Biblioteca dei Classici Italiani)\\
\smallskip Goldoni, Carlo. 1760. \textit{I Rusteghi}. (Biblioteca dei Classici Italiani)\\
\smallskip Goldoni, Carlo. 1961. \textit{Three comedies}. Trans. by Clifford Bax, I.M. Rawson, Eleanor Farjeon \& Herbert Farjeon (Oxford library of Italian classics xxvii, 293 p.) London: Oxford University Press.\\
\smallskip Gozzi, Carlo. 1761. \textit{L'amore delle tre Melarance colle alusioni al Goldoni e al Chiari}. (Biblioteca dei Classici Italiani)\\
\smallskip Papa Gregorio X. 1274. \textit{Secondo Concilio di Lione}. (IntraText)\\
\smallskip Parini, Giuseppe. 1763. \textit{Il giorno}. (IntraText)\\
\smallskip Parini, Giuseppe. 18th. \textit{Il primo bacio}. (Biblioteca dei Classici Italiani)\\
\smallskip Parini, Giuseppe. 1977. \textit{The day: morning, midday, evening, night: a poem}. Trans. by H. M. Bower. Westport, Conn.: Hyperion Press.\\
\smallskip Salvatorelli, Luigi. 1943. \textit{Pensiero e azione del risorgimento}. (DiaCORIS)\\
\smallskip Anon. 1270. \textit{Quindici segni del giudizio}. (OVI)
\printbibliography[heading=subbibliography,notkeyword=this]

\end{document}
