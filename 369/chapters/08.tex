\documentclass[output=paper,colorlinks,citecolor=brown,draft,draftmode]{langscibook}
\ChapterDOI{10.5281/zenodo.7525106}

\author{Monica Alexandrina Irimia\orcid{https://orcid.org/0000-0003-3733-8163}\affiliation{University of Modena and Reggio Emilia}}
\title{Oblique DOM and co-occurrence restrictions: How many types?}





\abstract{This paper examines co-occurrence restrictions involving oblique \textsc{dom} in (standard and leísta) Spanish and Romanian. Even a limited set of data reveals at least six puzzles, some of which are novel, ranging from differences in the syntactic behavior of oblique \textsc{dom} on clitics as opposed to full DPs to unsystematicity of repair strategies. It is shown that the \textit{narrow local} domain where the relevant ([\textsc{person}]) features are licensed plays a role in these patterns, beyond the split Agree/Case. } %Change Latin filler with your own abstract.

% % PACKAGES IN PHONETICS & PHONOLOGY
% %In order to type phonetic fonts, you have to use this code in your preambule
% %Need to remove from Preamble
% \usepackage{tipa}
% \let\ipa\textipa
% \usepackage{vowel}
% \newcommand{\BlankCell}{}
% \usepackage{ot-tableau}

% % PACKAGES IN SYNTAX: Use this code to have the capability to make syntactic trees
% \usepackage{forest}
% \usepackage[noeepic]{qtree}

% % PACKAGES FOR IMAGES
% \usepackage{graphicx}

% % MISCELLANEOUS PACKAGES
% \usepackage{lastpage}
% \usepackage{hyperref}

% %move the following commands to the "local\ldots\," files of the master project when integrating this chapter
% %DO NOT CHANGE THIS SOURCE SEQUENCE %%%% Skip to 'custom footer for preprints'
% \usepackage{tabularx}
% \usepackage{langsci-basic}
% \usepackage{langsci-optional}
% \usepackage{langsci-gb4e}
% \bibliography{localbibliography}
% \newcommand{\orcid}[1]{}
% \pagenumbering{arabic}
% \setcounter{page}{170}

\IfFileExists{../localcommands.tex}{
  \addbibresource{../localbibliography.bib}
  % add all extra packages you need to load to this file

\usepackage{tabularx,multicol}
\usepackage{url}
\urlstyle{same}

\usepackage{listings}
\lstset{basicstyle=\ttfamily,tabsize=2,breaklines=true}

\usepackage{langsci-basic}
\usepackage{langsci-optional}
\usepackage{langsci-lgr}
\usepackage{langsci-osl}
% \usepackage{./langsci/styles/langsci-lgr}
% \usepackage{./langsci/styles/langsci-osl}
% \usepackage{langsci-gb4e}

\usepackage{tikz}
\usetikzlibrary{patterns,calc}
\pgfdeclarepatternformonly{south east lines}{\pgfqpoint{-0pt}{-0pt}}{\pgfqpoint{3pt}{3pt}}{\pgfqpoint{3pt}{3pt}}{
    \pgfsetlinewidth{0.6pt}
    \pgfpathmoveto{\pgfqpoint{0pt}{3pt}}
    \pgfpathlineto{\pgfqpoint{3pt}{0pt}}
    \pgfpathmoveto{\pgfqpoint{.2pt}{-.2pt}}
    \pgfpathlineto{\pgfqpoint{-.2pt}{.2pt}}
    \pgfpathmoveto{\pgfqpoint{3.2pt}{2.8pt}}
    \pgfpathlineto{\pgfqpoint{2.8pt}{3.2pt}}
    \pgfusepath{stroke}}
    
\usepackage{stmaryrd}
\usepackage{wasysym}
\usepackage{multirow}
\usepackage{caption}
\usepackage{subcaption}
\usepackage{mathrsfs}
\usepackage{qtree}

\usepackage{linguex}


  %pminos do not split footnotes
% \interfootnotelinepenalty=10000 %Footnote in Laporte chapters has to be split SN


%\DeclareIndexNameFormat{default}{%
%\nameparts{#1}%
%\usebibmacro{index:name}%
%{\index[names]}%
%{\namepartfamily}%
%{\namepartgiveni}%
% {}% L1
% {}% L2
%{\namepartprefix}% generates spurious space L3
%{\namepartsuffix}% generates spurious space L4
%}

%  {\DeclareIndexNameFormat{default}{%
%     \usebibmacro{index:name}{\index[names]}{#1}{#3}{#5}{#7}}}

%\DeclareIndexNameFormat{default}{%
%  \usebibmacro{index:name}{\sindex[nom]}{#1}{#3}{#5}{#7}}

%\DeclareIndexNameFormat{default}{%
%  \usebibmacro{index:name}{\sindex[person]}{#1}{#3}{#5}{#7}}
%\DeclareIndexNameFormat{default}{%
%\nameparts{#1} \usebibmacro{index:name}{\sindex[person]]}{\namepartfamily}{‌​\namepartgiven}{\nam‌​epartprefix}{\namepa‌​rtsuffix}}

%\newcommand{\smiley}{:)}

%\renewbibmacro*{index:name}[5]{%
%\usebibmacro{index:entry}{#1}%
%{\iffieldundef{usera}{}{\thefield{usera}\actualoperator}\mkbibindexname{#2}{#3}{#4}{#5}}}

% \newcommand{\noop}[1]{}

%remove for final
%\overfullrule=1mm

\newcommand{\tobi}[2]}}
\renewcommand{\S}[1]{\tobi{#1}{\textsc{*}}}

% this volume references
% puts: [this volume]
% already defined: \citetv
%\newcommand{\citepv}[1]{(\citeauthor{#1} \citeyear*{#1} [this volume])}
\newcommand{\citealtv}[1]{\citeauthor{#1} \citeyear*{#1} [this volume]}

%parentheses around example number
\newcommand{\pref}[1]{(\ref{#1})}

% in-text examples

\newcommand{\lnex}[1]{\textit{#1}} %target lang word
\newcommand{\lnlit}[1]{(lit.: `#1')} %literal reading
\newcommand{\lnlat}[1]{(#1)} % latinization
\newcommand{\lntrans}[1]{`#1'} %translation
\newcommand{\lnexl}[2]%
{\lnex{#1}{} \lnlat{#2}} % ex with latinization
\newcommand{\lnexlat}[3]{\lnex{#1}{} \lnlat{#2}{} \lntrans{#3}} % ex with latinization and tranl.

%ch01
\newcommand{\co}[1]{\mbox{\textbf{#1}}}

%ch09

\newcommand{\cyrbulg}[1]{\begin{otherlanguage*}{bulgarian}#1\end{otherlanguage*}}


%ch10
\newcommand{\nlp}{{\small NLP}}
\newcommand{\mwe}{{\small MWE}}
\newcommand{\rae}{{\small RAE}}
\newcommand{\lvc}{{\small LVC}}
\newcommand{\pos}{{\small P}o{\small S}}
%\newcommand{\todo}[1]{ \textcolor{red}{#1} }

%\renewcommand{\labelenumi}{\theenumi}
%\ainamefmt{{vv}{ll}{, ff}{, jj}} % fullname

\newcommand{\biberror}[1]{{\color{red}#1}}

\newcommand{\osenovaitem}{--~}
  %% hyphenation points for line breaks
%% Normally, automatic hyphenation in LaTeX is very good
%% If a word is mis-hyphenated, add it to this file
%%
%% add information to TeX file before \begin{document} with:
%% %% hyphenation points for line breaks
%% Normally, automatic hyphenation in LaTeX is very good
%% If a word is mis-hyphenated, add it to this file
%%
%% add information to TeX file before \begin{document} with:
%% %% hyphenation points for line breaks
%% Normally, automatic hyphenation in LaTeX is very good
%% If a word is mis-hyphenated, add it to this file
%%
%% add information to TeX file before \begin{document} with:
%% \include{localhyphenation}
\hyphenation{
    Beck-man
    Ngu-yen
    back-chan-nel
    back-chan-nels
    mo-not-o-nous
    ste-reo-typ-i-cal
}

\hyphenation{
    Beck-man
    Ngu-yen
    back-chan-nel
    back-chan-nels
    mo-not-o-nous
    ste-reo-typ-i-cal
}

\hyphenation{
    Beck-man
    Ngu-yen
    back-chan-nel
    back-chan-nels
    mo-not-o-nous
    ste-reo-typ-i-cal
}

  % \togglepaper[3]%%chapternumber
}{}


\begin{document}
\maketitle
\section{Oblique DOM and co-occurrence restrictions}\label{sec:irimia:section1Intro}

A defining trait of several Romance languages is the presence of object splits, under the broader phenomenon known as \textit{differential object marking} (\textsc{dom)}. The particular \textsc{dom} subtype we are concerned with here uses oblique morphology (henceforth \textit{oblique} \textsc{dom}).\footnote{See \citeauthor{boss91} (\citeyear{boss91}, \citeyear{boss98}), \citeauthor{torrego98dependency} (\citeyear{torrego98dependency}), \citeauthor{cornil00} (\citeyear{cornil00}),  \citeauthor{ais2003} (\citeyear{ais2003}), \citeauthor{rodmondon07} (\citeyear{rodmondon07}),  \citeauthor{tig11} (\citeyear{tig11}), \citeauthor{lopez12} (\citeyear{lopez12}),
\citeauthor{ormromero13} (\citeyear{ormromero13}), \citeauthor{manzfranco16} (\citeyear{manzfranco16}), \citeauthor{hillandmardale2021} (\citeyear{hillandmardale2021}), a.o. We assume an accusative syntax for oblique \textsc{dom}.} For example,
in (standard) Spanish \REF{ex:irimia:SpanishAnimDOMPrep} or Romanian \REF{ex:irimia:RomanianDOM} a  human D(irect) O(bject) DP needs to be introduced by a preposition, as opposed to the inanimate DOs in \REF{ex:irimia:SpanishInanimNoDOM} or \REF{ex:irimia:RomanianNoDOM}. The split extends to DO clitics too, as documented for leísta Spanish, with the contrast in \REF{ex:irimia:LeistaSpanishCliticnoDOM} vs. \REF{ex:irimia:LeistaSpanishCliticDOM} from \citeauthor{ormromero07} (\citeyear{ormromero07}; ex. 15a, b, adapted).



\ea \label{ex:irimia:SpanishAnimDOMPrep}
\gll Vi \fbox {\bf {*(a)}} la ni\~na. \\
see.\sc {pst.1sg} \sc {dat=dom} the girl \\
\glt `I saw the girl.'
\z

\ea \label{ex:irimia:SpanishInanimNoDOM}
\gll Vi \bf {(*a)} el libro. \\
see.{\sc pst.1sg} {\sc dat=dom} the book\\ \jambox*{({Spanish})}
\glt `I saw the book.'
\z

\ea \label{ex:irimia:RomanianDOM}
\gll Nu v\u ad \fbox {\bf {*(pe)}} nimeni.\\
\sc{neg} see.\sc{1sg} \sc{loc=dom} nobody\\
\glt `I can't see anybody.'
\z


\ea \label{ex:irimia:RomanianNoDOM}
\gll Nu v\u ad \bf {(*pe)} copaci. \\
{\sc neg} see.{\sc 1sg} {\sc loc=dom} trees \\ \jambox*{({Romanian})}
\glt `I can't see trees.'
\z


\ea \label{ex:irimia:LeistaSpanishCliticnoDOM}
\gll Lo vi.\\
\sc{cl.3m.sg.acc} see.\sc{pst.1sg}\\
\glt `I saw it/him.'
\z


\ea \label{ex:irimia:LeistaSpanishCliticDOM}
\gll \fbox{\textbf{Le}} vi. \\
\sc{cl.3m.sg.dat=dom} see.\sc{pst.1sg} \\ \jambox*{({Leísta Spanish})}
\glt `I saw him.'
\z

A salient, although less discussed, property of oblique \textsc{dom} are the co-occur\-rence restrictions it gives rise to. For example, \citeauthor{ormromero07} (\citeyear{ormromero07})\footnote{See also \citeauthor{bleam} (\citeyear{bleam}), \citeauthor{zdrojewski} (\citeyear{zdrojewski}), \citeauthor{ormromero13} (\citeyear{ormromero13}, \citeyear{OrmRomero2013c}, \citeyear{OrmRomero2013b}) among others.} have shown that Cl\textsubscript{\textsc{obl=dom}}\footnote{In order to individuate oblique \textsc{dom} on clitics (as in \REF{ex:irimia:LeistaSpanishCliticDOM}) from oblique \textsc{dom} on full nominals (as in \REF{ex:irimia:SpanishAnimDOMPrep} or \REF{ex:irimia:RomanianDOM}), we encode the former as Cl\textsubscript{\textsc{obl=dom}} and the latter as DP\textsubscript{\textsc{obl=dom}}. We also collapse the locative and the dative under the broader category `oblique'.} bans the presence of an I(ndirect) O(bject) dative clitic, as in \REF{ex:irimia:LeistaSpanishCliticDOMPCC}.


\ea {Leísta Spanish} (\citeauthor{ormromero07} \citeyear{ormromero07}; ex. 16a, b, glosses adapted) \label{ex:irimia:LeistaSpanishCliticnoDOMnoPCC}
\ea[] {
\gll \Checkmark Te lo di.\\
\sc{2cl.dat} \sc{3cl.acc} give.\sc{pst.1sg} \\
\glt `I gave it to you.'
}
\ex[*]{
\gll Te \fbox{\textbf{le}} di. \\
\sc{2cl.dat} \sc{cl.3m.sg.dat=dom} give.\sc{pst.1sg} \\
\glt Intended: `I gave him to you.' \label{ex:irimia:LeistaSpanishCliticDOMPCC}
}
\z
\z

Co-occurrence restrictions provide important insights into the nature of \textsc{dom}. However, even in the initial, pioneering observations, it became immediately clear that they are not uniform. This paper touches on precisely this issue. The contribution is two-fold; on the empirical side, it is interested in the landscape of these phenomena, using (standard and leísta) Spanish and standard Romanian.\footnote{The data come from native speaker judgments, and  from 20 native speaker consultants each for Spanish and Romanian, and 3 for leísta Spanish.} Even a limited set of data reveals at least six puzzles, some of which are novel. Besides the differences between (leísta) Spanish Cl\textsubscript{\textsc{obl=dom}} and  DP\textsubscript{\textsc{obl=dom}}   (\citeauthor{ormromero07} \citeyear{ormromero07}, \sectref{sec:irimia:Section2DOMPCC} and \sectref{sec:irimia:Section3SomeProblems}), we touch on other problems such as: i) differences in the behavior of possessor vs. goal dative clitics with Romanian \textsc{dom} ({\sectref{sec:irimia:Section3SomeProblems}}); ii) splits between DP\textsubscript{\textsc{obl=dom}} and \textsc{dom} negative quantifiers (\sectref{sec:irimia:Section4SAgreeCase}); iii) lack of systematicity of accusative clitic doubling as a repair strategy on  Romanian \textsc{dom} (\sectref{sec:irimia:Section4SAgreeCase}). On the theoretical side, \sectref{sec:irimia:Section3SomeProblems} and \sectref{sec:irimia:Section4SAgreeCase} also show that the split Agree/Case is not sufficient to derive the data. \sectref{sec:irimia:ObliqueDOMandlicensingPositionSection5} explores the proposal that the \textit{narrow local} domain where the relevant ([\textsc{person}]) features need to be licensed plays a role in these types of co-occurrence restrictions. \sectref{sec:irimia:ConclusionsSec6} contains the conclusions.

\section{Oblique \textsc{DOM} and the \textsc{PCC}}\label{sec:irimia:Section2DOMPCC}

In a pioneering analysis of co-occurrence restrictions triggered by oblique \textsc{dom}, \citeauthor{ormromero07} (\citeyear{ormromero07}) reduced the ungrammaticality of examples such as \REF{ex:irimia:LeistaSpanishCliticDOMPCC} to principles behind the better known \textsc{P}(erson) \textsc{C}(ase) \textsc{C}(onstraint) or \textit{Me-Lui} phenomena. Across Romance, the latter have been extensively discussed for clitic clusters, following seminal work by \citeauthor{Perlmutter1971} (\citeyear{Perlmutter1971})
and \citeauthor{bonet1991} (\citeyear{bonet1991}).\footnote{See also \citeauthor{albizu1997} (\citeyear{albizu1997}), \citeauthor{anag2003} (\citeyear{anag2003}), \citeauthor{BejRez2003} (\citeyear{BejRez2003}), among others.} \\
\indent  The standard Spanish examples below illustrate the so-called \textit{strong} \textsc{PCC}. The ungrammaticality of \REF{ex:irimia:StrongPCCORexample} is triggered by the DO (direct object) clitic that has a person feature (1\textsuperscript{st}) which is hierarchically higher than the person feature of the IO clitic (3\textsuperscript{rd}), as schematically summarized in \REF{ex:irimia:StrongPCCformulation}. The ungrammaticality is avoided in \REF{NoPCCORExample}, as this time the DO is 3\textsuperscript{rd} person, while the IO is 1\textsuperscript{st} person.

\ea  \ul{Strong \textsc{PCC}: If DATIVE, then ACC = 3\textsuperscript{rd} person} \label{ex:irimia:StrongPCCformulation}
\z

\ea {Standard Spanish strong PCC}  (\citeauthor{ormromero07} \citeyear{ormromero07}) \label{ex:irimia:StrongPCCanditsgrammaticalityORexamples}
\ea[*] {
\gll Pedro \textbf{le/se} \textbf{me} envía. \\
Pedro \textsc{cl.3sg.dat} \textsc{cl.1sg.acc} send.\textsc{3sg.subj}\\
\glt Intended: `Pedro sends me to him.' \label{ex:irimia:StrongPCCORexample}
}
\ex[]{
\gll Pedro \textbf{me} \textbf{lo} envía. \\
Pedro \textsc{cl.1sg.dat} \textsc{cl.3sg.acc} send.\textsc{3sg.subj} \\
\glt `Pedro sends him/it to me.' \label{NoPCCORExample}
}
\z
\z

Although initial accounts investigated a morphological explanation for the (strong) \textsc{PCC}, subsequent research (\citeauthor{albizu1997} \citeyear{albizu1997}, \citeauthor{anag2003} \citeyear{anag2003}, \citeauthor{BejRez2003} \citeyear{BejRez2003}, \citeauthor{prem19} \citeyear{prem19}, among others.)
underpinned its clear \textit{syntactic} source. A general idea in syntactic accounts has been that the \textsc{PCC} involves more than one category which requires \textit{licensing} in the syntax, in a local configuration containing just one relevant licenser. To briefly cite two analyses, for \citeauthor{anag2003} (\citeyear{anag2003}) 1\textsuperscript{st} and 2\textsuperscript{nd} persons contain a [\textsc{person}] feature, which requires licensing just like the [\textsc{person}] feature introduced by all (inflectional) datives. \citeauthor{BejRez2003} (\citeyear{BejRez2003}) similarly assume an obligatory \textsc{person licensing condition} affecting speaker and hearer-related categories. \\
\indent \citeauthor{ormromero07} (\citeyear{ormromero07}, \citeyear{ormromero13}, \citeyear{OrmRomero2013c}, \citeyear{OrmRomero2013b}) follow the premises of intervention based syntactic accounts for \textsc{PCC} to explain co-occurrence restrictions induced by Cl\textsubscript{\textsc{obl=dom}} as in \REF{ex:irimia:LeistaSpanishCliticDOMPCC}. The reasoning goes as follows: Differential morphology on the DO clitic in \REF{ex:irimia:LeistaSpanishCliticDOMPCC} signals grammaticalized animacy, which requires \textit{obligatory licensing} via object agreement. A constraint is active which prohibits the verb from entering into other agreement operations, besides object agreement. This is formalized as the  \textsc{O}(bject) \textsc{A}(greement) \textsc{C}(onstraint) in \REF{ex:irimia:OAC}:

\ea \textsc{OAC}\label{ex:irimia:OAC} (\citeauthor{ormromero07} \citeyear{ormromero07}:50): {If the verbal complex encodes object agreement, no other argument can be licensed through verbal agreement}.
\z

In fact, for \citeauthor{ormromero07} (\citeyear{ormromero07}, \citeyear{ormromero13}, \citeyear{OrmRomero2013c}, \citeyear{OrmRomero2013b}), the \textsc{OAC} is the unifying factor behind all types of \textsc{PCC}. In oblique \textsc{dom}, grammaticalized animacy requires obligatory licensing but is relevant on all persons (including 3\textsuperscript{rd} person), and thus will block \textit{any} type of inflectional dative (clitic), which equally requires licensing. Moreover, the hypothesis that grammaticalized animacy, signalled by  oblique \textsc{dom}, requires special syntactic licensing appears to find support elsewhere. For example, in Romanian a DP\textsubscript{\textsc{obl=dom}} results in ungrammaticality (for all the consultants in this study) in a context which also contains a Cl\textsubscript{\textsc{dat}} interpreted as a possessor, irrespective of the person specification of the latter, as in \REF{ex:irimia:RomDOMUngramWithPossClitic}. Grammaticality is restored if oblique \textsc{dom} is removed \REF{ex:irimia:NOPCCnoDOMwithDativePossClitic}.

\ea {Romanian}: *Cl\textsubscript{\textsc{dat=poss}} DP\textsubscript{\textsc{obl=dom}} \underline{(\textsc{dom} blocked under possessor Cl\textsubscript{\textsc{dat}})} \label{ex:irimia:RomanianDOMPossessorDativeClitic}
\ea[*] {
\gll \textit{\c Si/*mi}-(l) ajut\u a \fbox{\textbf{pe}} prieten. \\
\textsc{cl.3sg.refl.dat/1sg.dat}-\textsc{cl.3m.sg.acc} help.\textsc{3sg} \textsc{loc=dom} friend
\\
\glt Intended: `He helps his own/my friend.' (`helps the friend to me') \label{ex:irimia:RomDOMUngramWithPossClitic}
}
\ex[ ]{
\gll \Checkmark {\textit{\^Işi} /\Checkmark \textit{îmi}} ajut\u a prieten-u-l. \\
{\textsc{cl.3sg.refl.dat}/\textsc{cl.1sg.dat}} help.\textsc{3sg} friend-\textsc{m.sg}-\textsc{def.m.sg} \\
\glt `He helps his own/my friend.' (i.e.,`helps the friend to himself/me') \label{ex:irimia:NOPCCnoDOMwithDativePossClitic} \footnote{Cl\textsubscript{\textsc{acc}} (\textit{-l)} doubling the DP\textsubscript{\textsc{obl=dom}} in \REF{ex:irimia:RomDOMUngramWithPossClitic} triggers alternation in the shape of Cl\textsubscript{\textsc{dat=poss}} in \REF{ex:irimia:RomanianDOMPossessorDativeClitic}.
}
}
\z
\z

\section{Some problems}\label{sec:irimia:Section3SomeProblems}

However, several problems immediately became apparent. If grammaticalized animacy, which requires obligatory licensing, is what triggers  oblique \textsc{dom}, one reasonable assumption is that DP\textsubscript{\textsc{obl=dom}} in \REF{ex:irimia:SpanishAnimDOMPrep} (from standard or leísta Spanish) should also trigger \textsc{PCC} effects. But this expectation is \textit{not} (fully) borne out. \\
\indent In \REF{ex:irimia:FullNominalDOMnoPCCwithDativeClitic} we see that DP\textsubscript{\textsc{obl=dom}}  is well formed with Cl\textsubscript{\textsc{dat}} (irrespectively of the latter's person feature). This contrasts with examples like \REF{ex:irimia:LeistaSpanishCliticDOMPCC}, repeated in \REF{ex:irimia:LeistaSpanishCliticDOMPCCRepeated}.

\ea {Spanish}: Oblique \textsc{dom} PCC on full nominals vs. clitics
\label{ex:irimia:PCCwithDOMClandFullNominals}
\ea[] {
\gll \Checkmark \textit{Te/me} enviaron \fbox {\textbf{a}} todos los enfermos. \\
\textsc{cl.2/1sg.dat} send.\textsc{pst.3pl} \textsc{dat=dom} all \textsc{def} {sick people}.\textsc{m.pl} \\ \jambox*{(Leísta/Standard)}
\glt `They have sent all the sick people to you/me.' \label{ex:irimia:FullNominalDOMnoPCCwithDativeClitic}
}
\ex[*]{
\gll \textit{Te/me} \fbox{\textbf{le}} di. \\
\sc{2/1cl.dat} \sc{cl.3m.sg.dat=dom} give.\sc{pst.1sg} \\ \jambox*{({Leísta})}\footnote{In standard Spanish (and Romanian) 3\textsuperscript{rd} person DO clitics only allow accusative morphology; thus, they do not grammaticalize animacy. Leísta varieties allow both DP\textsubscript{\textsc{obl=dom}} and Cl\textsubscript{\textsc{obl=dom}}. }
\glt Intended: `I gave him to you/me.' \label{ex:irimia:LeistaSpanishCliticDOMPCCRepeated}
}

\z
\z


DP\textsubscript{\textsc{obl=dom}} is also possible with an IO DP introduced by the (dative/locative) preposition \textit{a},\footnote{Thus, indicating that the effect is not due to haplology (the need to avoid two \textit{a-} sequences).} as in \REF{ex:irimia:ObliqueDOMonFullNominalswithFullNomDat} from \citeauthor{ormromero13} (\citeyear{ormromero13}). Crucially, in both leísta and standard Spanish, DP\textsubscript{\textsc{obl=dom}}  becomes \textit{ungrammatical} with an IO DP which is also \textit{doubled} by a  dative clitic. Thus, the example in \REF{ex:irimia:FullNominalDOMwithDativeDOMCliticDoubledDativeNotPossible} is grammatical (to the speakers tested here) only if the differential marker is removed.

\ea
\ea[\Checkmark]{
\gll Enviaron \fbox {\textbf{a}} todos los enfermos \textit{a} \textit{la} \textit{doctora}\\
send.\textsc{pst.3pl} \textsc{dat=dom} all the {sick people}.\textsc{m.pl} \textsc{dat} the doctor \\
\glt `They have sent all the sick people to the doctor.' \label{ex:irimia:ObliqueDOMonFullNominalswithFullNomDat}
}
\ex []{
\gll
\textit{Le} enviaron \fbox {(*\textbf{a})} todos los enfermos \textit{a} \textit{la} \textit{doctora}.\\
\textsc{cl.3dat} send.\textsc{pst.3pl} \textsc{loc/dat=dom} all.\textsc{m.pl} \textsc{def.m.pl} {sick people}.\textsc{m.pl} \textsc{dat} \textsc{def.f.sg} doctor \\
\glt Intended: `They have sent all the sick people to the doctor.' \label{ex:irimia:FullNominalDOMwithDativeDOMCliticDoubledDativeNotPossible}
}
\z
\z

Complex problems are the norm in Romanian, too. In \REF{ex:irimia:RomDOMUngramWithPossClitic} DP\textsubscript{\textsc{obl=dom}} is ungrammatical with a Cl\textsubscript{\textsc{dat=poss}}. But there are (at least) two twists in the data. On the one hand, other types of dative clitics are tolerated by DP\textsubscript{\textsc{obl=dom}}. The sentence in \REF{ex:irimia:RomanianObliqueDOMwithGoalDativeCliticOK} contains a \textit{goal} dative clitic and a DP\textsubscript{\textsc{obl=dom}} and is \textit{grammatical}, irrespectively of the person of the former:

\ea
{Romanian -- \Checkmark oblique DOM with goal dative clitic} \\
\gll \Checkmark\textit{Mi/\c ti/i} (l)-au prezentat \fbox {\textbf{pe}} student. \\
\textsc{cl.1/2/3sg.dat} \textsc{cl.3msg.acc}-have introduced \textsc{loc=dom} student \\ 
\glt `They have introduced the student to me/you$_\textsc{sg}$/him.'\footnote{\label{fn:irimia:9}In these contexts a DP\textsubscript{\textsc{dat}} is also possible: \textit{I\textsubscript{i} (l)-au prezentat pe student profesorului\textsubscript{i}\textsubscript{[professor.\textsc{dat}]}). } } (cf. \ref{ex:irimia:RomDOMUngramWithPossClitic}) \label{ex:irimia:RomanianObliqueDOMwithGoalDativeCliticOK}
\z

On the other hand, there are also configurations where a Cl\textsubscript{\textsc{dat}}-doubled IO\textsubscript{\textsc{dat}} \textit{outputs ungrammaticality} with DP\textsubscript{\textsc{obl=dom}}, even if the former is interpreted as a goal. In \REF{ex:irimia:RomanianDOMClDoubledDativeUngrammatical} we present a relevant example from \citeauthor{cornil2020} (\citeyear{cornil2020}). In a sense, such sentences mirror the Spanish one in \REF{ex:irimia:FullNominalDOMwithDativeDOMCliticDoubledDativeNotPossible}, with \textit{a difference}: In Romanian, ``\textsc{PCC} effects'' arise when DP\textsubscript{\textsc{obl=dom}} \textit{binds into} the Cl\textsubscript{\textsc{dat}} doubled IO (cf. \ref{ex:irimia:RomanianObliqueDOMwithGoalDativeCliticOK}/fn.\ref{fn:irimia:9}).



\ea {Romanian}  (\citeauthor{cornil2020} \citeyear{cornil2020}, ex. 4; glosses adapted) \\
\gll Comisia (*\textit{le})-a repartizat \fbox{\textbf{pe}} mai mul\c ti\textsubscript{i} medici reziden\c ti \textit{unor} foşti profesori de-ai lor\textsubscript{i}.\\
board.\textsc{def.f.sg} \textsc{cl.3pl.dat}-has assigned \textsc{loc=dom} more many.\textsc{m} medical residents some.\textsc{dat.pl} former.\textsc{m} professors of theirs \\
\glt Intended: `The board assigned several medical residents to some former professors of theirs.' \label{ex:irimia:RomanianDOMClDoubledDativeUngrammatical}
\z

Besides the removal of \textsc{dom} (a repair strategy equally available in standard and/or leísta Spanish), Romanian provides a second repair strategy for examples such as \REF{ex:irimia:RomanianDOMClDoubledDativeUngrammatical}, namely accusative clitic doubling of DP\textsubscript{\textsc{dom}}.\footnote{Not all varieties of Spanish allow clitic doubling of DP\textsubscript{\textsc{obl=dom}}. See further remarks in \sectref{sec:irimia:ObliqueDOMandlicensingPositionSection5}.} This is seen in the \textit{grammatical} sentence \REF{ex:irimia:ClDoubleDOMnoPCCwithClDoubleDat}, which contains a Cl\textsubscript{\textsc{dat}}-doubled IO, and a DP\textsubscript{\textsc{obl=dom}} which is \textit{clitic doubled} using the accusative form of the clitic (cf. \ref{ex:irimia:RomanianDOMClDoubledDativeUngrammatical}). A puzzle, however, is that Cl\textsubscript{\textsc{acc}}-doubling of DP\textsubscript{\textsc{obl=dom}} is \textit{not} a repair strategy in contexts that contain a dative clitic interpreted as a possessor, no matter whether a possesor dative DP is also present or not. Example \REF{ex:irimia:RomDOMUngramWithPossClitic} is adapted here as \REF{ex:irimia:RomDOMUngramWithPossCliticRepeated}.

\ea  {Romanian}: DP\textsubscript{\textsc{obl=dom}} and clitic doubled IOs \\
\ea {\Checkmark}  Cl\textsubscript{\textsc{dat}} DP\textsubscript{\textsc{dat}\textsubscript{i}} \ldots\, Cl\textsubscript{\textsc{acc}} DP\textsubscript{\textsc{obl=dom}\textsubscript{i}} \hfill (\citeauthor{cornil2020} \citeyear{cornil2020}, ex. 6; adapted) \\
{
\gll Comisia \textit{i} \ul{\textit{l}\textbf{}}-a repartizat \fbox{\textbf{pe}} fiecare\textsubscript{i} rezident \textit{unei} foste profesoare a lui\textsubscript{i}. \\
board.\textsc{def.f.sg} \textsc{cl.3sg.dat} \textsc{cl.3sg.m.acc}-has assigned \textsc{loc=dom} each  resident some.\textsc{dat.sg.f} former.\textsc{f.dat} professor.\textsc{f.dat} \textsc{lk} his\\
\glt `The board assigned each  resident to a former professor of his.' \label{ex:irimia:ClDoubleDOMnoPCCwithClDoubleDat}
}
\ex \textsc{*} Cl\textsubscript{\textsc{dat=poss}} (DP\textsubscript{\textsc{dat)}} \ldots\, Cl\textsubscript{\textsc{acc}} DP\textsubscript{\textsc{obl=dom}} \\{
\gll \textit{* I}-\ul{\textit{l}} ajut\u a \fbox{\textbf{pe}} prieten (lui Ion). \\
\textsc{cl.3sg.dat}-\textsc{cl.3m.sg.acc} help.\textsc{3sg} \textsc{loc=dom} friend \textsc{dat.3sg.m} Ion
\\
\glt Intended: `He helps his/Ion's friend.'  \label{ex:irimia:RomDOMUngramWithPossCliticRepeated}

}
\z
\z

% \subsection{Summary}\label{subsec:irimia:Sec2Summary}%no orphaned subsections

In general, as we can see from these limited sets of data, the co-occurrence restrictions on oblique \textsc{dom} are extremely complex and still uncharted. A modest goal here is, first of all, empirical - trying to map which domains are relevant, and where the cross-linguistic similarities and differences are to be found. Let us first summarize the five (related) puzzles we have identified (see also \tabref{tab:1:Summary2FivePuzzles}):

\ea {Oblique DOM and co-occurrence restrictions: Five puzzles}
\z

\begin{itemize}
\item {Puzzle\textsubscript{1}}: Assuming that DP\textsubscript{\textsc{obl=dom}} grammaticalizes animacy, it should trigger a PCC effect with dative clitics, similarly to Cl\textsubscript{\textsc{obl=dom}}. Why is this prediction not borne out? What is the reason for this contrast, which we repeat in \REF{ex:irimia:Puzzle1}?
\end{itemize}


\ea {Puzzle\textsubscript{1}}: * Cl\textsubscript{\textsc{dat}}  \ldots\, Cl\textsubscript{\textsc{obl=dom}} ({Leísta Spanish} \ref{ex:irimia:LeistaSpanishCliticDOMPCC}, \ref{ex:irimia:LeistaSpanishCliticDOMPCCRepeated})
vs. \\
\indent \hskip 1.6cm \Checkmark Cl\textsubscript{\textsc{dat}}  \ldots\, DP\textsubscript{\textsc{obl=dom}} ({Spanish, Romanian} \ref{ex:irimia:FullNominalDOMnoPCCwithDativeClitic}, \ref{ex:irimia:RomanianObliqueDOMwithGoalDativeCliticOK}) \label{ex:irimia:Puzzle1}
\z

\begin{itemize}
\item {Puzzle\textsubscript{2}}:
Why does Spanish DP\textsubscript{\textsc{obl=dom}} produce a \textsc{PCC} effect with an IO which is doubled by a dative clitic, as represented in \REF{ex:irimia:Puzzle2}?
\end{itemize}

\ea {Puzzle\textsubscript{2}}: * Cl\textsubscript{\textsc{dat}} DP\textsubscript{\textsc{dat}} \ldots\, DP\textsubscript{\textsc{obl=dom}} ({Leísta/Standard} \ref{ex:irimia:FullNominalDOMwithDativeDOMCliticDoubledDativeNotPossible}) \label{ex:irimia:Puzzle2}
\z

\begin{itemize}
\item {Puzzle\textsubscript{3}}:
Why does the restriction under Puzzle\textsubscript{2} obtain in Romanian (only) when
DP\textsubscript{\textsc{obl=dom}} binds into a Cl\textsubscript{\textsc{dat}}-doubled IO\textsubscript{\textsc{io}}, as summarized in
\REF{ex:irimia:Puzzle3New}?
\end{itemize}

\ea {Puzzle\textsubscript{3}}: * Cl\textsubscript{\textsc{dat}} DP\textsubscript{\textsc{dat}\textsubscript{i}}\ \ldots\, DP\textsubscript{\textsc{obl=dom}\textsubscript{i}} ({Romanian} \ref{ex:irimia:RomanianDOMClDoubledDativeUngrammatical}) vs. \\
\indent \hskip 1.6cm \Checkmark Cl\textsubscript{\textsc{dat}} DP\textsubscript{\textsc{dat}}\ \ldots\, DP\textsubscript{\textsc{obl=dom}} ({Romanian} \ref{ex:irimia:RomanianObliqueDOMwithGoalDativeCliticOK})
\label{ex:irimia:Puzzle3New}
\z

\begin{itemize}
\item {Puzzle\textsubscript{4}}: Why is Cl\textsubscript{\textsc{dat=poss}} distinct from other dative clitics in that it triggers PCC effects in interaction with DP\textsubscript{\textsc{obl=dom}} in Romanian?
\end{itemize}

\ea {Puzzle\textsubscript{4}}: *Cl\textsubscript{\textsc{dat=poss}}  \ldots\, DP\textsubscript{\textsc{obl=dom}} ({Romanian} \ref{ex:irimia:RomDOMUngramWithPossClitic},  \ref{ex:irimia:RomDOMUngramWithPossCliticRepeated}) vs. \\
\indent \hskip 1.6cm \Checkmark Cl\textsubscript{\textsc{dat=goal}} \ldots\, DP\textsubscript{\textsc{obl=dom}} ({Romanian} \ref{ex:irimia:RomanianObliqueDOMwithGoalDativeCliticOK})
\z

\begin{itemize}

\item {Puzzle\textsubscript{5}}: Why is the accusative clitic double of DP\textsubscript{\textsc{obl=dom}} a repair strategy in contexts containing a clitic doubled IO goal, but not a possessor dative in Romanian? This is summarized in \REF{ex:irimia:Puzzle4}.

\end{itemize}

\ea {Puzzle\textsubscript{5}}: *Cl\textsubscript{\textsc{dat=poss}} (DP\textsubscript{\textsc{dat=poss}}) \ldots\, Cl\textsubscript{\textsc{acc}} DP\textsubscript{\textsc{obl=dom}} ({Romanian} \ref{ex:irimia:RomDOMUngramWithPossClitic},  \ref{ex:irimia:RomDOMUngramWithPossCliticRepeated}) \\
\indent \hskip 1.6cm \Checkmark Cl\textsubscript{\textsc{dat}} DP\textsubscript{\textsc{dat}} \ldots\, Cl\textsubscript{\textsc{acc}} DP\textsubscript{\textsc{obl=dom}} ({Romanian} \ref{ex:irimia:ClDoubleDOMnoPCCwithClDoubleDat})
\label{ex:irimia:Puzzle4}


\z

\section{Agree vs. Case} \label{sec:irimia:Section4SAgreeCase}
\largerpage
Previous work has mostly been concerned with Puzzle\textsubscript{1}, namely the contrast between Cl\textsubscript{\textsc{obl=dom}} which gives rise to \textsc{PCC} effects with Cl\textsubscript{\textsc{dat}} in \ref{ex:irimia:LeistaSpanishCliticDOMPCC} (\ref{ex:irimia:LeistaSpanishCliticDOMPCCRepeated}), and DP\textsubscript{\textsc{obl=dom}}, which does not (\ref{ex:irimia:FullNominalDOMnoPCCwithDativeClitic}, \ref{ex:irimia:RomanianObliqueDOMwithGoalDativeCliticOK}). As mentioned in \sectref{sec:irimia:Section2DOMPCC} and \sectref{sec:irimia:Section3SomeProblems}, \citeauthor{ormromero07} (\citeyear{ormromero07}, \citeyear{ormromero13}, \citeyear{OrmRomero2013c}, \citeyear{OrmRomero2013b}, et subseq.) attribute the ungrammaticality of examples like \ref{ex:irimia:LeistaSpanishCliticDOMPCC} (\ref{ex:irimia:LeistaSpanishCliticDOMPCCRepeated}) to the OAC in \REF{ex:irimia:OAC}. Grammaticalized animacy spelled out by Cl\textsubscript{\textsc{obl=dom}} in  \ref{ex:irimia:LeistaSpanishCliticDOMPCC} (\ref{ex:irimia:LeistaSpanishCliticDOMPCCRepeated}) requires obligatory object agreement on the verb, blocking the licensing of any other argument through verbal agreement. Thus, Cl\textsubscript{\textsc{dat}}, which equally needs licensing, remains unlicensed causing ungrammaticality.

But, then, what is the status of grammaticalized animacy on full nominal \textsc{dom} (DP\textsubscript{\textsc{obl=dom}}), which is equally signaled via oblique morphology? \citeauthor{ormromero07} (\citeyear{ormromero07}: 338) provide the following explanation for this contrast: ``whatever rule or principle is involved in A-insertion (\textit{in DP\textsubscript{\textsc{obl=dom}}, our note}) it has to be independent of object agreement.'' In later works, \citeauthor{ormromero13} (\citeyear{ormromero13}) associate Cl\textsubscript{\textsc{obl=dom}} in \REF{ex:irimia:LeistaSpanishCliticDOMPCCRepeated} with licensing in terms of Agree, while DP\textsubscript{\textsc{obl=dom}} (i.e., prepositional \textit{a-}\textsc{DOM}, as in \ref{ex:irimia:SpanishAnimDOMPrep} or \ref{ex:irimia:FullNominalDOMnoPCCwithDativeClitic}) involves licensing in terms of Case.

The Agree/Case divide can also, potentially, explain why examples such as \REF{ex:irimia:ObliqueDOMonFullNominalswithFullNomDat} are \textit{grammatical}. The intuition is that the IO DP introduced by the preposition \textit{a} (`a la doctora') does not have a Case feature (it is a lexical dative, instead). Thus, it cannot compete for Case licensing with the Case feature in oblique \textsc{dom} on full nominals. In \REF{ex:irimia:FullNominalDOMwithDativeDOMCliticDoubledDativeNotPossible}, instead, the IO DP\textsubscript{\textsc{dat}} is doubled by a dative clitic. The latter contains a Case feature, which competes for licensing with the Case feature in DP\textsubscript{\textsc{obl=dom}}, introduced by the \textit{a}-preposition. This is {puzzle}\textsubscript{2}.


% \subsection{Some more problems}\label{sec:irimia:subsection4.1AgreeCaseMoreProblems} %no orphaned subsections

In \sectref{sec:irimia:Section2DOMPCC} and \sectref{sec:irimia:Section3SomeProblems} we have also seen  the data are truly complex and refined. The question is whether we can extend the split Agree/Case to all the patterns examined here. One problem is Puzzle\textsubscript{4} from Romanian, which sets aside the dative possessor clitic from other types of dative clitics, as repeated in  \REF{ex:irimia:Puzzle3Repeated}.

\ea {Puzzle\textsubscript{4}}: *Cl\textsubscript{\textsc{dat=poss}}  \ldots\, DP\textsubscript{\textsc{obl=dom}} ({Romanian} \ref{ex:irimia:RomDOMUngramWithPossClitic},  \ref{ex:irimia:RomDOMUngramWithPossCliticRepeated}) vs. \\
\indent \hskip 1.5cm \Checkmark Cl\textsubscript{\textsc{dat=goal}} \ldots\, DP\textsubscript{\textsc{obl=dom}} ({Romanian} \ref{ex:irimia:RomanianObliqueDOMwithGoalDativeCliticOK})
\label{ex:irimia:Puzzle3Repeated}
\z

\largerpage
Here, the explanation would have to be that Cl\textsubscript{\textsc{dat=poss}} needs licensing in terms of Agree, while other dative clitics either stay unlicensed or require licensing in terms of Case (or the other way around). The non-trivial question is what independent evidence would motivate this assumption. Similarly problematic is the contrast between \REF{ex:irimia:RomanianObliqueDOMwithGoalDativeCliticOK} and \REF{ex:irimia:RomanianDOMClDoubledDativeUngrammatical}. In what sense is this  a matter of Case vs. Agree?

There is yet another complex issue regarding the licensing of DP\textsubscript{\textsc{obl=dom}} in terms of Case. A less discussed fact is that not all types of DP\textsubscript{\textsc{obl=dom}} trigger co-occurrence restrictions. For example, \textsc{dom}-ed \textit{Neg(ative) Q(uantifier)s} (more easily) escape them. This is clearly seen in the contrast in \REF{ex:irimia:SpanishDOMinExistandNegQSsentences} from Spanish. In Romanian, the data are even more subtle. If NegQ\textsubscript{\textsc{obl=dom}} might be problematic to some speakers with \textit{assign/distribute}-type predicates (Class A), irrespectively of binding, as in \REF{ex:irimia:NoPCCwithNegQDOMRomanian}, \textit{introduce}-type predicates (Class B) seem to be fine in \REF{ex:irimia:NoPCCwithNegQDOMRomanianPresent}, as expected. But then, if oblique \textsc{dom} and clitic doubled datives compete for Case, leading to \textsc{PCC} in \REF{ex:irimia:ObliqueDOMonFullNominalswithFullNomDatRepeated2} and \REF{ex:irimia:RomanianDOMClDoubledDativeUngrammaticalRepeated}, why is the \textsc{PCC} avoided in \REF{ex:irimia:NoPCCwithNegQDOMSpanish}?

\ea \label{ex:irimia:SpanishDOMinExistandNegQSsentences}
\ea[*] {\gll \textit{Le} enviaron \fbox {\textbf{a}} todos los enfermos \textit{a} \textit{la} \textit{doctora}.\\
\textsc{cl.3dat} send.\textsc{pst.3pl} \textsc{dat=dom} all.\textsc{m.pl} \textsc{def.m.pl} {sick people}.\textsc{m.pl} \textsc{dat} \textsc{def.f.sg} doctor \\
\glt Intended: `They have sent all the sick people to the doctor.'  \label{ex:irimia:ObliqueDOMonFullNominalswithFullNomDatRepeated2}
}
\ex[]
{\gll No \textit{le} enviaron \fbox {\textbf{a}} nadie \textit{a} \textit{la} \textit{doctora}.\\
\textsc{neg} \textsc{cl.3sg.dat} send.\textsc{pst.3pl} \textsc{dat=dom} nobody \textsc{dat} the doctor \\ 
\glt `They haven't sent anybody to the doctor.' \jambox*{({Spanish})}
\label{ex:irimia:NoPCCwithNegQDOMSpanish}
}
\z
\z

\ea
\ea[*] {\gll Comisia \textit{le}-a repartizat \fbox{\textbf{pe}} mai mul\c ti\textsubscript{i} medici reziden\c ti \textit{unor} foşti profesori de-ai lor\textsubscript{i}.\\
board.\textsc{def.f.sg} \textsc{cl.3pl.dat}-has assigned \textsc{loc=dom} more many.\textsc{m} medical residents some.\textsc{dat.pl} former.\textsc{m} professors of-\textsc{lk} theirs \\ 
\glt Intended: `The board assigned several medical residents to some former professors of theirs.'\label{ex:irimia:RomanianDOMClDoubledDativeUngrammaticalRepeated} {(\citeauthor{cornil2020} \citeyear{cornil2020}, ex. 4; adapted)}\footnote{As \citeauthor{cornil2020} (\citeyear{cornil2020}) also notices, the problem is not the putative absence of clitic doubling on DP\textsubscript{\textsc{dom}}. DP\textsubscript{\textsc{dom}} is grammatical without clitic doubling for all the speakers consulted here.}
}
\ex[\textsuperscript{?}]
{\gll Comisia nu \textit{i}-a repartizat \fbox{\textbf{pe}} nimeni \textit{profesorului}. \\
board.\textsc{def} \textsc{neg} \textsc{cl.3sg.dat}-has assigned \textsc{loc=dom} nobody professor.\textsc{def.dat} \\
\glt `The board hasn't assigned anybody to the professor.'\footnote{As \textsc{dom} is obligatory on \textit{nimeni}, the only repair here is the removal of Cl\textsubscript{\textsc{dat}} double (-\textit{i}). Also,  \REF{ex:irimia:RomanianDOMClDoubledDativeUngrammaticalRepeated} and \REF{ex:irimia:NoPCCwithNegQDOMRomanian} show that these co-occurrence restrictions are not simply a matter of DP\textsubscript{\textsc{obl=dom}} binding into Cl\textsubscript{\textsc{dat}}-doubled IO; NegQ\textsubscript{\textsc{obl=dom}} is not involved in such  operation in \REF{ex:irimia:NoPCCwithNegQDOMRomanian}.
}\label{ex:irimia:NoPCCwithNegQDOMRomanian}
}
\ex[]
{\gll Comisia nu \textit{i}-a prezentat \fbox{\textbf{pe}} nimeni \textit{profesorului}. \\
board.\textsc{def} \textsc{neg} \textsc{cl.3sg.dat}-has introduced \textsc{loc=dom} nobody professor.\textsc{def.dat} \\
\glt `The board hasn't introduced anybody to the professor.' \hfill(Romanian)
\label{ex:irimia:NoPCCwithNegQDOMRomanianPresent}
}
\z
\z

Assuming that differential marking on NegQs is not active syntactically is a non-starter.  NegQ\textsubscript{\textsc{obl=dom}} \textit{is} blocked under other configurations which do \textit{not} permit differential marking. One such case is the medio-passive \textsc{se}. The two examples below are ungrammatical in both Spanish \REF{ex:irimia:SpanishnoDOMNegQundrMPSE} and Romanian \REF{ex:irimia:RomaniannoDOMNegQundrMPSE}.

\ea \label{ex:irimia:noDOMunderMPse}{DOM under medio-passive \textit{se}: Spanish and Romanian} \\
\ea[]{
\gll No se encerr\'o \fbox{\textbf{(*a)}} nadie.\\
\textsc{neg} \textsc{se} {locked up}.\textsc{3sg} \textsc{dat=dom} nobody \\ \jambox*{({Spanish})\footnote{In Spanish, such examples are possible under an \textit{impersonal} reading. The difference between the medio-passive and the impersonal reading is clearer with a plural direct object, as the impersonal reading blocks plural agreement. See \citeauthor{Mendikoetxea2008} (\citeyear{Mendikoetxea2008}), a.o. for further discussion.
}}
\glt Intended: `Nobody was/got locked up.' \label{ex:irimia:NoNegQDOMunderMPse} \label{ex:irimia:SpanishnoDOMNegQundrMPSE}
}
\ex[]{
\gll Nu se invit\u a \fbox{\textbf{(*pe)}} nimeni.\\
\textsc{neg} \textsc{se} invite.\textsc{3sg} \textsc{loc=dom} nobody \\ \jambox*{({Romanian})}
\glt Intended: `Nobody is/gets invited.'     \label{ex:irimia:RomaniannoDOMNegQundrMPSE}
}

\z
\z

Moreover, in Romanian, NegQ\textsubscript{\textsc{obl=dom}}  is still ungrammatical in a structure which contains a dative clitic interpreted as a possessor. In \REF{ex:irimia:NegQUngrammaticalWithPossDative}, we have forced a possessor reading of the dative clitic (i.e., \textit{he didn't help anybody of his}). The consultants judge this example ungrammatical/degraded, contrary to \REF{ex:irimia:NoPCCwithNegQDOMRomanianPresent}.

\ea[*/??]{\gll Nu \textit{şi}-a ajutat \fbox{\textbf{pe}} nimeni dintre ai s\u ai.\\
\textsc{neg} \textsc{cl.3sg.dat}-has helped \textsc{loc=dom} nobody from \textsc{lk.def.m.pl} his.\textsc{pl} \\ 
\glt Intended: `He hasn't helped anybody of his.' \jambox*{({Romanian})} } \label{ex:irimia:NegQUngrammaticalWithPossDative}
\z


In order to explain such examples, NegQ\textsubscript{\textsc{obl=dom}} will need to be Case licensed in some contexts (\ref{ex:irimia:noDOMunderMPse}, etc.), but caseless in others (\ref{ex:irimia:NoPCCwithNegQDOMSpanish}, etc.). We thus have yet another problem, as summarized under Puzzle\textsubscript{6}:
\largerpage[2]

\begin{itemize}
\item {Puzzle\textsubscript{6}}: Why does NegQ\textsubscript{\textsc{obl=dom}} (more easily) escape the PCC in configurations involving clitic doubled IO\textsubscript{\textsc{dat}}, as summarized in \REF{ex:irimia:Puzzle5}?
\end{itemize}

\ea {Puzzle}\textsubscript{6}: \Checkmark Cl\textsubscript{\textsc{dat}} DP\textsubscript{\textsc{dat}} \ldots\, Neg Q\textsubscript{\textsc{dom}} (\ref{ex:irimia:NoPCCwithNegQDOMSpanish}, \ref{ex:irimia:NoPCCwithNegQDOMRomanianPresent}) \\
\indent \hskip 1.5cm  *Cl\textsubscript{\textsc{dat}} DP\textsubscript{\textsc{dat}} \ldots\, DP\textsubscript{\textsc{dom}} (\ref{ex:irimia:FullNominalDOMwithDativeDOMCliticDoubledDativeNotPossible}, \ref{ex:irimia:RomanianDOMClDoubledDativeUngrammaticalRepeated})
\label{ex:irimia:Puzzle5}
\z



\begin{table}[b]
\caption{Six puzzles}
\label{tab:1:Summary2FivePuzzles} %this label will be use for referencing this table in the prose of your document. NOTE you may label your items in whichever way helps you remember them.
\small
 \begin{tabularx}{\textwidth}{lQll} %in the second argument you determine the alignment of your columns. l=left, c=center, and r=right.
 % we will now add a column :) make sure to add another 'r' on the argument portion next to tabular.
  \lsptoprule
            & Content & Language  & Repair  \\%and ampersand '&' will represent a column separation every time you finish a row make sure to add double backslash \\
  \midrule
  Puzzle\textsubscript{1} & \textbf{no Cl\textsubscript{\textsc{dom}} with Cl\textsubscript{\textsc{dat}}} &   leísta   &   remove Cl\textsubscript{\textsc{dom}}/  Cl\textsubscript{\textsc{dat}}   \\
  & *Cl\textsubscript{\textsc{dat}}  \ldots\, Cl\textsubscript{\textsc{obl=dom}} (\ref{ex:irimia:LeistaSpanishCliticDOMPCC}, \ref{ex:irimia:LeistaSpanishCliticDOMPCCRepeated}) & &  \\
  & \Checkmark Cl\textsubscript{\textsc{dat}}  \ldots\, DP\textsubscript{\textsc{obl=dom}} (\ref{ex:irimia:FullNominalDOMnoPCCwithDativeClitic}, \ref{ex:irimia:RomanianObliqueDOMwithGoalDativeCliticOK}) & & \\
  \tablevspace
  Puzzle\textsubscript{2}  & \textbf{no DP\textsubscript{\textsc{dom}} with Cl\textsubscript{\textsc{dat}}-doubled }   & Spanish/  &  remove DP\textsubscript{\textsc{dom}}/DP\textsubscript{\textsc{dat}}     \\
  & *Cl\textsubscript{\textsc{dat}} DP\textsubscript{\textsc{dat}} \ldots\, DP\textsubscript{\textsc{dom}}
(\ref{ex:irimia:FullNominalDOMwithDativeDOMCliticDoubledDativeNotPossible}, \ref{ex:irimia:RomanianDOMClDoubledDativeUngrammatical}) & Romanian & Cl\textsubscript{\textsc{dat}}/DP\textsubscript{\textsc{dat}}/\\
& & (binding) & Cl\textsubscript{\textsc{acc}}-double DP\textsubscript{\textsc{dom}} \\
& & & (latter -- Romanian)\\
  \tablevspace
  Puzzle\textsubscript{3}  & \Checkmark \textbf{Cl\textsubscript{\textsc{dat}} DP\textsubscript{\textsc{dat}}\ldots\,   DP\textsubscript{\textsc{dom}}}   &  Romanian &       \\
 & \textbf{if no DP\textsubscript{\textsc{dom}} binding into IO} &  & \\
& *Cl\textsubscript{\textsc{dat}} DP\textsubscript{\textsc{dat}\textsubscript{i}}\ \ldots\, DP\textsubscript{\textsc{obl=dom}\textsubscript{i}} (\ref{ex:irimia:RomanianDOMClDoubledDativeUngrammatical}) & &  clitic-double\textsubscript{\textsc{acc}} DP\textsubscript{\textsc{dom}}/\\
&  \Checkmark Cl\textsubscript{\textsc{dat}} DP\textsubscript{\textsc{dat}}\ \ldots\, DP\textsubscript{\textsc{obl=dom}} (\ref{ex:irimia:RomanianObliqueDOMwithGoalDativeCliticOK}) & & remove Cl\textsubscript{\textsc{obl=dom}}\\
  \tablevspace
  Puzzle\textsubscript{4}  &  \textbf{no Cl\textsubscript{\textsc{dat=poss}} with DP\textsubscript{\textsc{dom}}} & Romanian &  remove DP\textsubscript{\textsc{obl=dom}}         \\
  & *Cl\textsubscript{\textsc{dat=poss}}\ldots\, DP\textsubscript{\textsc{obl=dom}}(\ref{ex:irimia:RomDOMUngramWithPossClitic},\ref{ex:irimia:RomDOMUngramWithPossCliticRepeated})
& & \\
& \Checkmark Cl\textsubscript{\textsc{dat}}  \ldots\, DP\textsubscript{\textsc{dom}}
(\ref{ex:irimia:RomanianObliqueDOMwithGoalDativeCliticOK})
& & \\
  \tablevspace
  Puzzle\textsubscript{5}  & \textbf{Cl\textsubscript{\textsc{acc}} of \textsc{dom} not a repair with Cl\textsubscript{\textsc{poss}}} & Romanian  & remove DP\textsubscript{\textsc{dom}}       \\
  & *Cl\textsubscript{\textsc{dat=poss}}  \ldots\, Cl\textsubscript{\textsc{acc}} DP\textsubscript{\textsc{dom}}
(\ref{ex:irimia:RomDOMUngramWithPossClitic},\ref{ex:irimia:RomDOMUngramWithPossCliticRepeated})\\
& \Checkmark Cl\textsubscript{\textsc{dat}} DP\textsubscript{\textsc{dat}} \ldots\, Cl\textsubscript{\textsc{acc}} DP\textsubscript{\textsc{dom}}
(\ref{ex:irimia:ClDoubleDOMnoPCCwithClDoubleDat})
& & \\
  \tablevspace
   Puzzle\textsubscript{6}  & \textbf{Neg Q\textsubscript{\textsc{dom}} OK with Cl\textsubscript{\textsc{dat}} DP\textsubscript{\textsc{dat}}}
   &  Spanish/  &     \\
   & \Checkmark Cl\textsubscript{\textsc{dat}} DP\textsubscript{\textsc{dat}} \ldots\, Neg Q\textsubscript{\textsc{dom}}   (\ref{ex:irimia:NoPCCwithNegQDOMSpanish}, \ref{ex:irimia:NoPCCwithNegQDOMRomanianPresent}) &  \makecell[tl]{Romanian --\\class B verbs}  &\\
   & *Cl\textsubscript{\textsc{dat}} DP\textsubscript{\textsc{dat}\textsubscript{i}} \ldots\, DP\textsubscript{\textsc{dom}\textsubscript{i}} (\ref{ex:irimia:FullNominalDOMwithDativeDOMCliticDoubledDativeNotPossible}, \ref{ex:irimia:RomanianDOMClDoubledDativeUngrammatical}) &  & \\
  \lspbottomrule
 \end{tabularx}
\end{table}


In \tabref{tab:1:Summary2FivePuzzles} we summarize the six puzzles. In \sectref{sec:irimia:ObliqueDOMandlicensingPositionSection5} we explore a solution which (also) takes into account the \textit{position} in which a certain category needs licensing.




\section{Oblique DOM and licensing positions}\label{sec:irimia:ObliqueDOMandlicensingPositionSection5}

\subsection{Oblique DOM and the possessor dative}\label{subsec:irimia:ObliqueDOMPossessorClSubsection5.1}

As the facts are clearer, let's start with the problems involving the possessor clitic. Puzzle\textsubscript{4} showed that oblique DP\textsubscript{\textsc{obl=dom}} is ungrammatical in configurations which contain a Cl\textsubscript{\textsc{dat=poss}}, as in \REF{ex:irimia:RomDOMUngramWithPossClitic}, repeated in \REF{ex:irimia:RomDOMUngramWithPossCliticRepeated3}. However, there are further quirks in the data. For all speakers, such structures significantly improve or are perfectly grammatical if DP\textsubscript{\textsc{obl=dom}} is dislocated to the left periphery, as in \REF{ex:irimia:RomDOMGrammaticalwithPossCliticIfDOMDislocated}. Moreover, if Cl\textsubscript{\textsc{dat}} is not interpreted as a possessor on DP\textsubscript{\textsc{obl=dom}}, the structure is again grammatical. This is illustrated in \REF{ex:irimia:RomDOMGrammaticalifPossCliticonOtherElelement}, where the possessor is interpreted on the PP-adjunct.\footnote{For lack of a more adequate notation, we indicate this connectedness via a subscript index.} In fact, a possessor reading on DP\textsubscript{\textsc{obl=dom}} would be ungrammatical, as already shown in \REF{ex:irimia:NegQUngrammaticalWithPossDative}.


\ea Puzzle\textsubscript{\textsc{4}}:  *Cl\textsubscript{\textsc{dat=poss}\textsubscript{\textit{i}}}  \ldots\, DP\textsubscript{\textsc{dom}\textsubscript{\textit{i}}}
(\ref{ex:irimia:RomDOMUngramWithPossClitic}, \ref{ex:irimia:RomDOMUngramWithPossCliticRepeated3}
) vs. \\
\hskip 1.5cm \Checkmark [\textsubscript{\textsc{cp}} DP\textsubscript{\textsc{dom}\textsubscript{\textit{i}}} \ldots\,[\textsubscript{C\textsuperscript{0}}\ldots\, [Cl\textsubscript{\textsc{dat=poss}\textsubscript{\textit{i}}}  \ldots\, ] ] ] \REF{ex:irimia:RomDOMGrammaticalwithPossCliticIfDOMDislocated}
\\
\hskip 1.5cm \Checkmark Cl\textsubscript{\textsc{dat=poss}\textsubscript{\textit{i}}}  \ldots\, DP\textsubscript{\textsc{dom}\textsubscript{\textit{j}}} \ldots\, XP\textsubscript{\textit{i}}  \REF{ex:irimia:RomDOMGrammaticalifPossCliticonOtherElelement}
\\

\ea {*\gll \textit{\c Si\textsubscript{\textit{i}}/*mi}\textsubscript{\textit{i}}-(l) ajut\u a \fbox{\textbf{pe}} prieten\textsubscript{\textit{i}}. \\
\textsc{cl.3sg.refl.dat/1sg.dat}-\textsc{cl.3m.sg.acc} help.\textsc{3sg} \textsc{loc=dom} friend
\\
\glt Intended: `He is helping his own/my friend.'  \label{ex:irimia:RomDOMUngramWithPossCliticRepeated3}
}
\ex {
\gll \fbox{\textbf{Pe}} prieteni\textsubscript{\textit{i}}, Ion \textit{şi}\textsubscript{\textit{i}}-i ajut\u a. \\
\textsc{loc=dom} friends, Ion \textsc{cl.dat.3sg.refl}-\textsc{cl.3m.pl.acc} helps \\
\glt `His own friends, Ion helps them.' \label{ex:irimia:RomDOMGrammaticalwithPossCliticIfDOMDislocated}
}
\ex {\gll Nu \textit{şi}\textsubscript{\textit{i}}-a trimis \fbox{\textbf{pe}} nimeni\textsubscript{\textit{*i}} în ajutor\textsubscript{\textit{i}}. \\
\textsc{neg} \textsc{cl.3sg.refl.dat}-has sent \textsc{loc=dom} nobody in help \\ 
\glt Lit. `He hasn't sent anybody to/as his own aid.' \\
\# `He hasn't sent anybody of his as an aid.' \jambox*{({Romanian})}
\label{ex:irimia:RomDOMGrammaticalifPossCliticonOtherElelement}
}
\z
\z

This also implies that local, narrow domains \textit{do} matter. As mentioned, we follow accounts which link oblique \textsc{dom} to a specification beyond Case. For simplicity, we encode it as a [\textsc{person}] feature (\citeauthor{cornil00} \citeyear{cornil00},   \citeauthor{rodmondon07} \citeyear{rodmondon07}, \citeauthor{rich08} \citeyear{rich08}), which needs obligatory licensing in the syntax. The dative possessor clitic also encodes a [\textsc{person}] feature, which equally needs licensing. The data also indicate that this is a type of dative possessor clitic which is generated DP-internally and then raises to its spell-out position.\footnote{\citeauthor{landau1999} (\citeyear{landau1999}), \citeauthor{diaconescu2004} (\citeyear{diaconescu2004}), a.o.}

\hspace*{-1pt}The more specific problem with examples such as \REF{ex:irimia:RomDOMUngramWithPossCliticRepeated3} is that the two [\textsc{person}] features are \textit{too local} in the same KP, as represented in \figref{ex:irimia:TreewithPersonTooLocal}. Additionally, in the local domain that contains these two [\textsc{person}] features, there is only one [\textsc{person}] licenser, on the functional projection we label here $\alpha$ (following \citeauthor{lopez12} \citeyear{lopez12}). Crash can be avoided, if one of the [\textsc{person}] features can be removed from this local domain, for example via dislocation to/direct merge in the left periphery, as in \REF{ex:irimia:RomDOMGrammaticalwithPossCliticIfDOMDislocated}. Here, the [\textsc{person}] feature can be licensed by a [\textsc{person}]-related functional projection in the C\textsuperscript{0} domain, while the [\textsc{person}]-related specification in the possessor clitic is licensed by $\alpha$\textsubscript{1} head. Another possibility is to have the two [\textsc{person}] features on different categories, as in \REF{ex:irimia:RomDOMGrammaticalifPossCliticonOtherElelement}; here, as schematically shown in \figref{ex:irimia:TreeWithPossDatinPPNoPCCeffect}, the \textit{Possessor}-related [\textsc{person}] feature is generated inside the PP, while the object DP contains a separate [\textsc{person}] feature. As we show in \sectref{subsec:irimia:DOMandCliticDoubledDativesSubsection5.3} and \sectref{sec:irimia:DOMNegQuantifiers}, depending on the narrow domain in which each of these [\textsc{person}] features is checked, crash can be avoided.\footnote{ \citeauthor{oneahole2017} (\citeyear{oneahole2017}) and
\citeauthor{onea2018} (\citeyear{onea2018}) derive ungrammaticality in examples like \REF{ex:irimia:RomDOMUngramWithPossCliticRepeated3} on the hypothesis that both oblique \textsc{dom} and the possessor clitic need licensing in a position above VP. As we see in this paper, this seems to be too coarse; there are instances (e.g., \ref{ex:irimia:RomDOMGrammaticalifPossCliticonOtherElelement} in the relevant interpretation) where these two categories do not produce ungrammaticality, indicating that some other factor is at play too. }

\begin{figure}
\centering
\caption{[\textsc{person}] categories too local}
\label{ex:irimia:TreewithPersonTooLocal}
% \small
% \begin{tikzpicture}
% \Tree [\ldots\,..$\upsilon$P {$\upsilon$} [\ldots\,.$\alpha$\textsubscript{1}P {$\alpha$\textsubscript{1}  \\ \textsc{\textbf{person}}} [.VP {V} [.\textbf{KP-\textit{Person}}\textbf{!!} {\textit{Cl-\textit{Person}\textsubscript{Poss}}} [.KP\textsubscript{DOM} {\textit{Person}\textsubscript{DOM}} [.DP ] ] ] ] ] ]
% \end{tikzpicture}
\begin{forest}
[\ldots\,$\upsilon$P
    [$\upsilon$]
    [\ldots\,$\alpha$\textsubscript{1}P
        [{$\alpha$\textsubscript{1}  \\ \textsc{\textbf{person}}}]
        [VP
          [V]
          [\textbf{KP-\textit{Person}}\textbf{!!}
            [\textit{Cl-\textit{Person}\textsc{poss}}]
            [KP\textsc{dom}
              [\textit{Person}\textsc{dom}]
              [DP]
            ]
          ]
        ]
    ]
]
\end{forest}
\end{figure}

\begin{figure}
\caption{[\textsc{person}] categories in separate domains}
\label{ex:irimia:TreeWithPossDatinPPNoPCCeffect}
% \small
% \begin{tikzpicture}
% \Tree [\ldots\,.VP {\textit{DP-Person}\textsubscript{DOM}}  [.VP {V} [.\textit{Person-}PossP\\şi- şi- în-ajutor ] ] ]
% \end{tikzpicture}
\begin{forest}
[\ldots\,VP
    [\textit{DP-Person}\textsubscript{DOM}]
    [VP
      [V]
      [{\textit{Person-}PossP\\şi-}
        [şi-]
        [în-ajutor]
      ]
      ]
]
\end{forest}
\end{figure}



\subsection{Puzzle\textsubscript{1}: Oblique \textsc{dom} on clitics and interaction with IO clitics}\label{subsec:irimia:CLiticDOMandIOCliticSubsection5.2}

Thus, the \textit{position} in which [\textsc{person}] features are licensed is relevant. But this begs the question about possible [\textsc{person}]-licensing positions. The literature contains a variety of proposals. As already mentioned above, \citeauthor{lopez12} (\citeyear{lopez12}) assumes that (oblique) \textsc{dom} is licensed\footnote{Note that for \citeauthor{lopez12} (\citeyear{lopez12}), oblique \textsc{dom} involves licensing in terms of Case. As we have outlined some shortcomings of this hypothesis, we take \textsc{dom} to involve the licensing of a ([\textsc{person}]) feature beyond Case. This way we obtain better results both empirically and formally.
} in an intermediate position between VP and $\upsilon$\textsuperscript{0},\footnote{One important piece of evidence for a licensing position below $\upsilon$\textsuperscript{0} comes from the absence of binding effects into the EA from DP\textsubscript{\textsc{obl=dom}} (see \citeauthor{lopez12} \citeyear{lopez12}: 41--46 for exemplification).}\label{ex:irimia:footnoteLopezpositions} which we denote by $\alpha$\textsubscript{1}\textsuperscript{0}. Yet, \citeauthor{belletti2005} (\citeyear{belletti2005}), \citeauthor{ciucivara2009} (\citeyear{ciucivara2009}), and \citeauthor{stegovec2020} (\citeyear{stegovec2020}), among others, have identified a [\textsc{person}] (animacy) licensing field above $\upsilon$P, which is especially relevant for clitics. A third explicit proposal is that oblique \textsc{dom} on DPs has $\upsilon$\textsuperscript{0} as a licenser (\citeauthor{rodmondon07} \citeyear{rodmondon07}, a.o.).\footnote{Of course, a [\textsc{person}]-licensing field is also available in the CP. In fact, there are Romance varieties where DP\textsubscript{\textsc{dom}} is only possible on XPs that are overtly dislocated to the left periphery. See especially \citeauthor{belletti18} (\citeyear{belletti18}) for Italian or \citeauthor{EscandellVidal2009} (\citeyear{EscandellVidal2009}, et subseq.) for Balearic Catalan.} The three [\textsc{person}] licensers are illustrated in \figref{ex:irimia:TreeWithAllLicensingPositions}. Importantly, what the data at hand show is that \textit{all} these positions and licensers are relevant in their own way.

Let's turn now to the [\textsc{person}] field above $\upsilon$P. We assume that this area is involved in the licensing of oblique \textsc{dom} on clitics, as seen in leísta varieties of Spanish. {Puzzle}\textsubscript{1} is precisely concerned with the ungrammaticality of an oblique \textsc{dom} clitic in the context of an IO clitic. Crucially, this effect does not arise when a full nominal is differentially marked. We repeat the relevant examples in \REF{ex:irimia:Puzzle1Repeated2}:



\ea \textsc{puzzle\textsubscript{1}}: * Cl\textsubscript{\textsc{dat}}  \ldots\, Cl\textsubscript{\textsc{obl=dom}} ({Leísta Spanish} \ref{ex:irimia:LeistaSpanishCliticDOMPCC}, \ref{ex:irimia:LeistaSpanishCliticDOMPCCRepeated})
\textit{vs} \\
\indent \hskip 1.4cm \Checkmark Cl\textsubscript{\textsc{dat}}  \ldots\, DP\textsubscript{\textsc{obl=dom}} ({Spanish, Romanian} \ref{ex:irimia:FullNominalDOMnoPCCwithDativeClitic}, \ref{ex:irimia:RomanianObliqueDOMwithGoalDativeCliticOK})  \label{ex:irimia:Puzzle1Repeated2}\\
\ea[*]{
\gll \textit{Te/me} \fbox{\textbf{le}} di. \\
\sc{2/1cl.dat} \sc{cl.3m.sg.dat=dom} give.\sc{pst.1sg} \\
\glt Intended: `I gave him to you/me.' \label{ex:irimia:LeistaSpanishCliticDOMPCCRepeated3}
}
\ex[\Checkmark]{\gll \textit{Te/me} enviaron \fbox {\textbf{a}} todos los enfermos. \\
\textsc{cl.2/1sg.dat} send.\textsc{pst.3pl} \textsc{dat=dom} all the {sick people}.\textsc{m.pl} \\ 
\glt `They have sent all the sick people to you/me.'\jambox*{({Leísta Spanish})} \label{ex:irimia:FullNominalDOMnoPCCwithDativeCliticRepeated3}
}
\z
\z

As the PCC effects induced by Cl\textsubscript{\textsc{obl=dom}} are different from those of DP\textsubscript{\textsc{obl=dom}} and given the problems with an analysis under the split Case/Agree, let's see what we obtain as a result of licensing position. As DP\textsubscript{\textsc{obl=dom}} gets licensed in  $\alpha$\textsubscript{1}, it must be the case that Cl\textsubscript{\textsc{obl=dom}} is licensed in a different position. We propose that this is the [\textsc{person}] domain above $\upsilon$P, what we abbreviate as $\alpha$\textsubscript{2}\textsuperscript{0} (see the tree in \figref{ex:irimia:TreeWithAllLicensingPositions}).\footnote{Tests similar to the ones alluded to in fn. (16) actually show that Cl\textsubscript{\textsc{obl=dom}} can be found above the EA, as opposed to DP\textsubscript{\textsc{obl=dom}}, which is found below $\upsilon$\textsuperscript{0}.} The problem with \REF{ex:irimia:LeistaSpanishCliticDOMPCCRepeated3} is that the same local $\alpha$\textsubscript{2} domain also hosts the dative clitic encoding a [\textsc{person}] feature equally needing licensing. As there is only one [\textsc{person]} licenser available, namely $\alpha$\textsubscript{2}\textsuperscript{0}, the derivation will crash, as in \figref{ex:irimia:TwoPersonCliticsinAlpha2Tree}. On the other hand, the two [\textsc{person}] features in \REF{ex:irimia:FullNominalDOMnoPCCwithDativeCliticRepeated3} can be licensed by two licensers found in different domains, as in \figref{ex:irimia:DPDOMnoPCCEffectwithDativeClitic}: $\alpha$\textsubscript{2} for the [\textsc{person}] feature in the dative clitic, and $\alpha$\textsubscript{1} for the [\textsc{person}] feature in DP\textsubscript{\textsc{obl=dom}}. This latter structure is also seen with Romanian ditransitives as in \REF{ex:irimia:RomanianObliqueDOMwithGoalDativeCliticOK} (remember that these are either Class B verbs or configurations in which DP\textsubscript{\textsc{obl=dom}} does not bind into the IO\textsubscript{\textsc{dat}}).\footnote{As expected, binding from the IO into DP\textsubscript{\textsc{obl=dom}} \textit{is} possible, indicating that in these configurations the IO is higher (and if containing a [\textsc{person}] feature, it has an independent licenser for it, which does not interact with oblique \textsc{dom}).} Therefore, we also have part of the answer to {Puzzle}\textsubscript{3}.


\begin{figure}[t]
\caption{[\textsc{person}] licensing positions}
\label{ex:irimia:TreeWithAllLicensingPositions}
% \scriptsize
% \begin{tikzpicture}
% \Tree
%  [\ldots\,.$\alpha$\textsubscript{2}P { Cl\textsubscript{DO} \textsuperscript{0} } [.$\alpha$\textsubscript{2}P { Cl\textsubscript{IO} \textsuperscript{0} } [.$\alpha$\textsubscript{2}' { $\alpha$\textsubscript{2} \\ \textsc{\textbf{person}} } [.$\upsilon$P { Obj } [.$\upsilon$P { EA } [.$\upsilon$' {$\upsilon$ \textsuperscript{0} \\ \textbf{[\textsc{acc - person}}] } [\ldots\,.$\alpha$\textsubscript{1}P {\textsc{dom} }  [.$\alpha$\textsubscript{1}' {$\alpha$\textsubscript{1}\textsuperscript{0} \\ \textsc{\textbf{person}} } [.ApplP {IO} [.Appl' {Appl\textsuperscript{0} } [.VP
%  {V} [.\textit{t}$_{DOM}$ ] ] ] ] ] ] ] ] ] ] ] ]
% \end{tikzpicture}
\begin{forest}
[\ldots\,$\alpha$\textsubscript{2}P
    [Cl\textsubscript{DO} \textsuperscript{0}]
    [$\alpha$\textsubscript{2}P
      [Cl\textsubscript{IO} \textsuperscript{0}]
      [$\alpha$\textsubscript{2}$\prime$
      [$\alpha$\textsubscript{2} \\ \textsc{\textbf{person}}]
        [$\upsilon$P
          [Obj]
          [$\upsilon$P
            [EA]
            [$\upsilon\prime$
              [$\upsilon$ \textsuperscript{0} \\ \textbf{[\textsc{acc - person}]}]
              [\ldots\,$\alpha$\textsubscript{1}P
                  [\textsc{dom}]
                  [$\alpha$\textsubscript{1}$\prime$
                    [$\alpha$\textsubscript{1}\textsuperscript{0} \\ \textsc{\textbf{person}}]
                    [ApplP
                      [IO]
                      [Appl$\prime$
                        [Appl\textsuperscript{0}]
                        [VP
                          [V]
                          [\textit{t}$_{\textsc{dom}}$ ]
                        ]
                      ]
                    ]
                  ]
              ]
            ]
          ]
        ]
      ]
    ]
]
\end{forest}
\end{figure}

\begin{figure} %\scriptsize
\caption{[\textsc{person}] licensing above \textit{v}P}
\label{ex:irimia:DPDOMnoPCCEffectwithDativeClitic}
% \begin{tikzpicture}
% \Tree
% [\ldots\,.$\alpha$\textsubscript{2}P {Cl\textsubscript{IO} \textsuperscript{0}\\\textit{Person} }  [.$\alpha$\textsubscript{2}' {$\alpha$\textsubscript{2}\textsuperscript{0} \\ \textsc{\textbf{person}} } [\ldots\,$\upsilon$P {$\upsilon$} [.$\alpha$\textsubscript{1}P {DP\textsubscript{\textsc{dom}} \\\textit{Person}} [.$\alpha$\textsubscript{1}' {$\alpha$\textsubscript{1}\\\textsc{\textbf{person}} } [.VP
%  {V} [.\textit{t}$_{Dom}$ ] ] ] ] ] ] ]
%  \end{tikzpicture}
\begin{forest}
[\ldots\,$\alpha$\textsubscript{2}P
    [{Cl\textsubscript{IO} \textsuperscript{0}\\\textit{Person}}]
    [$\alpha$\textsubscript{2}$\prime$
      [{$\alpha$\textsubscript{2}\textsuperscript{0} \\ \textsc{\textbf{person}}}]
      [\ldots\,$\upsilon$P
        [$\upsilon$]
        [$\alpha$\textsubscript{1}P
          [{DP\textsubscript{\textsc{dom}} \\\textit{Person}}]
          [$\alpha$\textsubscript{1}$\prime$
            [{$\alpha$\textsubscript{1}\\\textsc{\textbf{person}}}]
              [VP
                [V]
                [\textit{t}$_{\textsc{dom}}$ ]
              ]
            ]
          ]
        ]
      ]
  ]
]
\end{forest}
\end{figure}

\begin{figure} %\scriptsize
\caption{DOM doubling by accusative clitic above \textit{v}P}
\label{ex:irimia:ACCClDoubleRepairStrategy}
% \begin{tikzpicture}
% \Tree
%  [\ldots\,.$\alpha$\textsubscript{2}P {\textsc{dom}\\ \textit{Person}} [\ldots\,.$\alpha$\textsubscript{2} {Cl \textsubscript{\textsc{acc}}\\\textsc{\textbf{person}} }  [.$\upsilon$P {$\upsilon$\textsuperscript{0}} [\ldots\,.$\alpha$\textsubscript{1}P {$\alpha$\textsubscript{1}\textsuperscript{0} \\ \textsc{\textbf{person}} } [.ApplP {IO} [.Appl' {Appl\textsuperscript{0}\\\textit{Person} } [.VP
%  {V} [.\textit{t}$_{dom}$ ] ] ] ]  ] ] ] ]
%  \end{tikzpicture}
\begin{forest}
[\ldots\,$\alpha$\textsubscript{2}P
    [{\textsc{dom}\\ \textit{Person}}]
    [\ldots\,$\alpha$\textsubscript{2}
        [{Cl\textsubscript{\textsc{acc}}\\\textsc{\textbf{person}}}]
        [$\upsilon$P
          [$\upsilon$\textsuperscript{0}]
          [\ldots\,$\alpha$\textsubscript{1}P
              [{$\alpha$\textsubscript{1}\textsuperscript{0} \\ \textsc{\textbf{person}}}]
              [ApplP
                [IO]
                [Appl$\prime$
                  [{Appl\textsuperscript{0}\\\textit{Person}}]
                  [VP
                    [V]
                    [\textit{t}$_{\textsc{dom}}$]
                  ]
                ]
              ]
          ]
        ]
    ]
]
\end{forest}
 \end{figure}


\begin{figure} %\scriptsize
\caption{Two [\textsc{person}] categories to be licensed by α$_2$ – clash}
\label{ex:irimia:TwoPersonCliticsinAlpha2Tree}
% \begin{tikzpicture}
% \Tree [\ldots\,.$\alpha$\textsubscript{2}P { Cl\textsubscript{DO} \textsuperscript{0}\\\textit{Person} } [.$\alpha$\textsubscript{2}P {Cl\textsubscript{IO} \textsuperscript{0}\\\textit{Person} } [.$\alpha$\textsubscript{2}' { $\alpha$\textsubscript{2} \\ \textsc{\textbf{person!!}} } [.$\upsilon$P { Obj } \ldots\, ] ] ] ]
% \end{tikzpicture}
\begin{forest}
[\ldots\,$\alpha$\textsubscript{2}P
    [Cl\textsubscript{DO} \textsuperscript{0}\\\textit{Person}]
    [$\alpha$\textsubscript{2}P
      [Cl\textsubscript{IO} \textsuperscript{0}\\\textit{Person}]
      [$\alpha$\textsubscript{2}$\prime$
        [$\alpha$\textsubscript{2} \\ \textsc{\textbf{person!!}}]
        [$\upsilon$P
          [Obj]
          [\ldots\,]
        ]
      ]
    ]
]
\end{forest}
\end{figure}




\subsection{Oblique \textsc{dom} and clitic doubled datives}\label{subsec:irimia:DOMandCliticDoubledDativesSubsection5.3}

Let's see now the explanation to {Puzzle}\textsubscript{2} which involves ungrammaticality of DP\textsubscript{\textsc{obl=dom}} with a dative IO which is clitic doubled by dative clitic (in Spanish and in Romanian configurations where DP\textsubscript{\textsc{obl=dom}} binds into the clitic doubled dative, see \tabref{tab:1:Summary2FivePuzzles}). In these structures \figref{ex:irimia:PCCeffectDOMwithClDoubledDative}, DP\textsubscript{\textsc{obl=dom}} contains a [\textsc{person}] feature needing licensing. Dative clitic doubling involves the introduction of a [\textsc{person}] feature on the (low) Appl head,\footnote{The evidence discussed by \citeauthor{lopez12} (\citeyear{lopez12}: 41--46) indicates that DP\textsubscript{\textsc{obl=dom}} binds into the IO, and not the other way around. Thus here the DP\textsubscript{\textsc{obl=dom}} is (interpreted) higher than the IO. } which equally needs licensing. As there is only one licenser available, namely $\alpha$\textsubscript{1}\textsuperscript{0}, the derivation will crash.

\begin{figure} %\scriptsize
\caption{Licensing of DOM animate negative quantifiers}
\label{ex:irimia:NeQnoPCCTree}
% \begin{tikzpicture}
% \Tree
% [\ldots\,.$\upsilon$P { \textsc{dom-negQ}\\\textit{Person} } [.$\upsilon$P { EA } [.$\upsilon$' { $\upsilon$ \textsuperscript{0} \\ { \textbf[\textsc{acc}}, \textsc{person}] } [\ldots\,.$\alpha$\textsubscript{1}P { }  [.$\alpha$\textsubscript{1}' {$\alpha$\textsubscript{1}\textsuperscript{0} \\ \textsc{\textbf{person}} } [.ApplP {IO} [.Appl' { Appl\textsuperscript{0}\\ \textit{Person} } [.VP
%  {V} [.\textit{t}$_{DOM - NegQ}$ ] ] ] ] ] ] ] ] ]
% \end{tikzpicture}
\begin{forest}
[\ldots\,$\upsilon$P
    [\textsc{dom-negQ}\\\textit{Person}]
    [$\upsilon$P
      [EA]
      [$\upsilon$$\prime$
        [$\upsilon$\textsuperscript{0} \\ \textbf{[\textsc{acc}, \textsc{person}]}]
        [\ldots\,$\alpha$\textsubscript{1}P
            [-]
            [$\alpha$\textsubscript{1}$\prime$
              [$\alpha$\textsubscript{1}\textsuperscript{0} \\ \textsc{\textbf{person}}]
              [ApplP
                [IO]
                [Appl$\prime$
                  [Appl\textsuperscript{0}\\ \textit{Person}]
                  [VP
                    [V]
                    [\textit{t}$_{DOM - NegQ}$ ]
                  ]
                ]
              ]
            ]
        ]
      ]
    ]
]
\end{forest}
\end{figure}

\begin{figure} %\scriptsize
\caption{Two [\textsc{person}] categories to be licensed by α$_1$ – clash}
\label{ex:irimia:PCCeffectDOMwithClDoubledDative}
% \begin{tikzpicture}
% \Tree
% [\ldots\,.$\alpha$\textsubscript{1}P {\textsc{dom}\\\textit{Person} }  [.$\alpha$\textsubscript{1}' {$\alpha$\textsubscript{1}\textsuperscript{0} \\ \textsc{\textbf{person!!}} } [.ApplP {IO} [.Appl' {Appl\textsuperscript{0}\\\textit{Person} } [.VP
%  {V} [.\textit{t}$_{dom}$ ] ] ] ] ] ]
% \end{tikzpicture}
\begin{forest}
[\ldots\,$\alpha$\textsubscript{1}P
    [\textsc{dom}\\\textit{Person}]
    [$\alpha$\textsubscript{1}$\prime$
      [$\alpha$\textsubscript{1}\textsuperscript{0} \\ \textsc{\textbf{person!!}}]
      [ApplP
        [IO]
        [Appl$\prime$
          [Appl\textsuperscript{0}\\\textit{Person}]
          [VP
            [V]
            [\textit{t}$_{\textsc{dom}}$ ]
          ]
        ]
      ]
    ]
]
\end{forest}
\end{figure}


In Romanian such configurations have a repair strategy which consists in accusative clitic doubling of DP\textsubscript{\textsc{dom}}, as in \REF{ex:irimia:ClDoubleDOMnoPCCwithClDoubleDat}, part of {Puzzle}\textsubscript{5}. The PCC effect is avoided as accusative clitic doubling, which involves the licensing of a [\textsc{person}] feature in $\alpha$\textsubscript{2}, removes oblique \textsc{dom} from the domain of $\alpha$\textsubscript{1} (see also \citeauthor{cornil2020} \citeyear{cornil2020}).\footnote{Clitic doubled \textsc{dom} allows binding into the EA (as opposed to DP\textsubscript{\textsc{dom}} which is not clitic doubled), indicating a position above $\upsilon$P. See also \citeauthor{hillandmardale2021} (\citeyear{hillandmardale2021}), a.o., for discussion.} Thus $\alpha$\textsubscript{1}\textsuperscript{0} can license the [\textsc{person}] feature on the clitic doubled dative, as shown in \figref{ex:irimia:ACCClDoubleRepairStrategy}. As in dative possessor contexts, nominal \textsc{dom} and the dative clitic are \textit{too} local in the KP on first merge,\footnote{Only movement/direct merge in the CP \REF{ex:irimia:RomDOMGrammaticalwithPossCliticIfDOMDislocated} can break this too local relationship. This indicates that C\textsuperscript{0} introduces its own [\textsc{person}] zone, separate from the [\textsc{person}] zone below it. } accusative clitic is not a repair strategy, and examples like \REF{ex:irimia:RomDOMUngramWithPossCliticRepeated} are ungrammatical.

\subsection{DOM on negative quantifiers}\label{sec:irimia:DOMNegQuantifiers}

Let's turn now to Puzzle\textsubscript{6}. The question is why NegQ\textsubscript{\textsc{obl=dom}} can avoid a \textsc{PCC} effect with clitic doubled datives  as opposed to DP\textsubscript{\textsc{obl=dom}} in examples like \REF{ex:irimia:NoPCCwithNegQDOMSpanishRepeated}:

\ea Puzzle\textsubscript{6}:
\Checkmark Cl\textsubscript{\textsc{dat}} DP\textsubscript{\textsc{dat}} \ldots\, Neg Q\textsubscript{\textsc{dom}} (\ref{ex:irimia:NoPCCwithNegQDOMSpanish}, \ref{ex:irimia:NoPCCwithNegQDOMRomanianPresent}) \hfill vs \\
\hskip 1.6cm *Cl\textsubscript{\textsc{dat}} DP\textsubscript{\textsc{dat}\textsubscript{(i)}} \ldots\, DP\textsubscript{\textsc{dom}\textsubscript{(i)}} (\ref{ex:irimia:FullNominalDOMwithDativeDOMCliticDoubledDativeNotPossible}, \ref{ex:irimia:RomanianDOMClDoubledDativeUngrammatical})\\
\gll No \textit{le} enviaron \fbox {\textbf{a}} nadie \textit{a} \textit{la} \textit{doctora}.\\
\textsc{neg} \textsc{cl.3sg.dat} send.\textsc{pst.3pl} \textsc{dat=dom} nobody \textsc{dat} \textsc{def.f.sg} doctor \\ \jambox*{({Spanish)}}
\glt `They haven't sent anybody to the doctor.'  \label{ex:irimia:NoPCCwithNegQDOMSpanishRepeated}
\z

Although the explanation is more tentative, one possibility is to relate this to intrinsic properties of NegQ\textsubscript{\textsc{obl=dom}}, which trigger raising higher than $\upsilon$\textsuperscript{0}. For one,  NegQ\textsubscript{\textsc{obl=dom}} carries emphatic accent, related to a focus feature,\footnote{See \citeauthor{Giannak2020} (\citeyear{Giannak2020}), a.o., for discussion.} which forces raising. Therefore, animate NegQ has its accusative Case (and subsequently its [\textsc{person}] feature) licensed by $\upsilon$\textsuperscript{0}; \textsc{[person]}
on clitic-doubled datives is licensed by $\alpha$\textsubscript{1}\textsuperscript{0}, as shown in \figref{ex:irimia:NeQnoPCCTree}. This, however, would predict that examples like \REF{ex:irimia:NoPCCwithNegQDOMRomanian} should always be grammatical. Although none of the consultants judged \REF{ex:irimia:NoPCCwithNegQDOMRomanian} as ungrammatical as \REF{ex:irimia:RomanianDOMClDoubledDativeUngrammaticalRepeated}, for some speakers these examples were not fully perfect either. Therefore, further research is clearly needed into this point, as well as into the more precise difference between Class A and Class B verbs (\ref{ex:irimia:NoPCCwithNegQDOMRomanian} vs. \ref{ex:irimia:NoPCCwithNegQDOMRomanianPresent}) and the effect of binding.

\newpage
Finally, raising to Spec, $\upsilon$\textsuperscript{0} is not a repair strategy in contexts such as \REF{ex:irimia:NegQUngrammaticalWithPossDative} for the same reasons mentioned above. And it does not work under medio-passive \textsc{se} in \REF{ex:irimia:ObliqueDOMonFullNominalswithFullNomDatRepeated2} or \REF{ex:irimia:NoPCCwithNegQDOMSpanish} either; \textsc{se}\textsubscript{\textsc{mp}} involves the removal of structural \textit{accusative}
case. 

In \tabref{tab:2:Summary2FivePuzzlesExplanation} we summarize the results obtained in this section.

\begin{table}
\caption{Six puzzles and their explanations}
\label{tab:2:Summary2FivePuzzlesExplanation} %this label will be use for referencing this table in the prose of your document. NOTE you may label your items in whichever way helps you remember them.
\small
 \begin{tabularx}{\textwidth}{lQQ} %in the second argument you determine the alignment of your columns. l=left, c=center, and r=right.
 % we will now add a column :) make sure to add another 'r' on the argument portion next to tabular.
  \lsptoprule
            & Content & Explanation  \\%and ampersand '&' will represent a column separation every time you finish a row make sure to add double backslash \\
  \midrule
  Puzzle\textsubscript{1} & \textbf{no Cl\textsubscript{\textsc{dom}} with Cl\textsubscript{\textsc{dat}}} &   both need licensing from $\alpha$\textsubscript{2}\textsuperscript{0} \\
  & *Cl\textsubscript{\textsc{dat}}  \ldots\, Cl\textsubscript{\textsc{obl=dom}} (\ref{ex:irimia:LeistaSpanishCliticDOMPCC}, \ref{ex:irimia:LeistaSpanishCliticDOMPCCRepeated}) & \figref{ex:irimia:TwoPersonCliticsinAlpha2Tree} in \sectref{subsec:irimia:CLiticDOMandIOCliticSubsection5.2}\\
  \tablevspace
  Puzzle\textsubscript{2}  & \textbf{no DP\textsubscript{\textsc{dom}} with Cl\textsubscript{\textsc{dat}}-doubled DP\textsubscript{\textsc{dat}}}   &  both need licensing from $\alpha$\textsubscript{1}\textsuperscript{0}    \\
  & *Cl\textsubscript{\textsc{dat}\textsubscript{i}} DP\textsubscript{\textsc{dat}\textsubscript{i}} \ldots\, DP\textsubscript{\textsc{dom}}
(\ref{ex:irimia:FullNominalDOMwithDativeDOMCliticDoubledDativeNotPossible}, \ref{ex:irimia:RomanianDOMClDoubledDativeUngrammatical}) &  \figref{ex:irimia:PCCeffectDOMwithClDoubledDative} in \sectref{subsec:irimia:DOMandCliticDoubledDativesSubsection5.3} \\
  \tablevspace
  Puzzle\textsubscript{3}  & \Checkmark \textbf{Cl\textsubscript{\textsc{dat}} DP\textsubscript{\textsc{dat}}\ldots\,   DP\textsubscript{\textsc{dom}}}   & Cl\textsubscript{\textsc{dat}} DP\textsubscript{\textsc{dat}} above DP\textsubscript{\textsc{dom}} \& \\
 & \textbf{if no DP\textsubscript{\textsc{dom}} binding into IO} & Cl\textsubscript{\textsc{dat}}DP\textsubscript{\textsc{dat}} licensed independently  \\
& *Cl\textsubscript{\textsc{dat}} DP\textsubscript{\textsc{dat}\textsubscript{i}}\ \ldots\, DP\textsubscript{\textsc{obl=dom}\textsubscript{i}} (\ref{ex:irimia:RomanianDOMClDoubledDativeUngrammatical}) &  \figref{ex:irimia:DPDOMnoPCCEffectwithDativeClitic} in \sectref{subsec:irimia:DOMandCliticDoubledDativesSubsection5.3} \\
\tablevspace
  Puzzle\textsubscript{4}  &  \textbf{no Cl\textsubscript{\textsc{dat=poss}} with DP\textsubscript{\textsc{dom}}} & both too local in the same KP \\
  & *Cl\textsubscript{\textsc{dat=poss}}  \ldots\, DP\textsubscript{\textsc{dom}}
(\ref{ex:irimia:RomDOMUngramWithPossClitic},  \ref{ex:irimia:RomDOMUngramWithPossCliticRepeated})
& \figref{ex:irimia:TreewithPersonTooLocal} in \sectref{subsec:irimia:ObliqueDOMPossessorClSubsection5.1} \\
  \tablevspace
  Puzzle\textsubscript{5}  & \textbf{Cl\textsubscript{\textsc{acc}} of \textsc{dom} not a repair with Cl\textsubscript{\textsc{poss}}} & both too local in the same KP \\
  & *Cl\textsubscript{\textsc{dat=poss}}  \ldots\, Cl\textsubscript{\textsc{acc}} DP\textsubscript{\textsc{dom}}
(\ref{ex:irimia:RomDOMUngramWithPossClitic},  \ref{ex:irimia:RomDOMUngramWithPossCliticRepeated}) & discussion in \sectref{subsec:irimia:CLiticDOMandIOCliticSubsection5.2} and \sectref{subsec:irimia:DOMandCliticDoubledDativesSubsection5.3} \\
 \tablevspace
   Puzzle\textsubscript{6}  & \textbf{Neg Q\textsubscript{\textsc{dom}} OK with Cl\textsubscript{\textsc{dat}} DP\textsubscript{\textsc{dat}}}
   &  NegQ\textsubscript{\textsc{dom}} licensed by $\upsilon$\textsuperscript{0}  \\
   & \Checkmark Cl\textsubscript{\textsc{dat}} DP\textsubscript{\textsc{dat}} \ldots\, Neg Q\textsubscript{\textsc{dom}} (\ref{ex:irimia:NoPCCwithNegQDOMSpanish}, \ref{ex:irimia:NoPCCwithNegQDOMRomanianPresent}) & \figref{ex:irimia:NeQnoPCCTree} in \sectref{subsec:irimia:DOMandCliticDoubledDativesSubsection5.3} and \sectref{sec:irimia:DOMNegQuantifiers} \\
  \lspbottomrule
 \end{tabularx}
\end{table}





\section{Conclusions}\label{sec:irimia:ConclusionsSec6}

This short paper has examined co-occurrence restrictions with oblique \textsc{dom} from (leísta and standard) Spanish and Romanian. The complexity and richness of an otherwise rather limited set of data give rise to six puzzles, which prove hard to reduce just to the split Agree/Case. We have found that an important factor behind these patterns is also the \textit{narrow local} domain where the relevant [\textsc{person}] features are licensed. Obviously, oblique \textsc{dom} is part of many other co-occurrence restrictions, for example with variants of the Pan-Romance \textsc{se}, begging the question of how all these effects can be further unified.

\section*{Abbreviations}
\begin{tabularx}{.54\textwidth}{lQ}
\textsc{acc} & accusative  \\
\textsc{cl}& clitic  \\
\textsc{dat}&dative  \\
\textsc{def}&definite  \\
\textsc{do}&direct object  \\
\textsc{dom}&differential object marking   \\
\textsc{f}&feminine  \\
\textsc{io}&indirect object  \\
\textsc{lk}&linker  \\
\end{tabularx}
\begin{tabularx}{.45\textwidth}{lQ}
\textsc{loc}&locative  \\
\textsc{m}&masculine  \\
\textsc{neg}&negative  \\
\textsc{obl}&oblique  \\
\textsc{pl}&plural  \\
\textsc{pst}&past  \\
\textsc{refl}&reflexive  \\
\textsc{sg}&singular  \\
\textsc{subj}&subject  \\
\end{tabularx}

\section*{Acknowledgements}
We would like to thank Ion Giurgea as well as the audiences at LSRL 50 and SLE 2021 for the discussion and very useful feedback. All errors are our own.

\printbibliography[heading=subbibliography,notkeyword=this]

\end{document}
