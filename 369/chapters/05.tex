\documentclass[output=paper,colorlinks,citecolor=brown]{langscibook}
\ChapterDOI{10.5281/zenodo.7525100}
\author{{Irene Fernández-Serrano}\orcid{https://orcid.org/0000-0003-0343-0586}\affiliation{Universitat Aut\`onoma de Barcelona}}
\title{The role of SE in Spanish agreement variation}
\abstract{This paper analyses Spanish agreement variation in non-paradigmatic SE structures. It is argued that in European Spanish the attested alternation between agreement and lack of agreement is part of a single grammar, i.e. a case of intra-speaker optionality. To support this claim it is shown that neither definiteness nor Case assignment are responsible for the lack of agreement pattern. The proposal combines two basic ingredients: the special featural configuration of SE \citep{Mendikoetxea1999, DAlessandro2007} and the parametrization of the order of syntactic operations \citep{Obataetal2015,Obata_Epstein2016}. This analysis reflects the asymmetries with respect to Italian and Icelandic data and is compatible with a similar case of variation in Spanish \textsc{dat-nom} psych-verb structures.}

%move the following commands to the "local..." files of the master project when integrating this chapter
% \usepackage{tabularx}
% \usepackage{langsci-basic}
% \usepackage{langsci-optional}
% \usepackage{langsci-gb4e}
% \bibliography{localbibliography}
% \newcommand{\orcid}[1]{}
% \pagenumbering{arabic}
% \setcounter{page}{98}
% \usetikzlibrary{positioning, tikzmark, arrows, arrows.meta}


\IfFileExists{../localcommands.tex}{
  \addbibresource{../localbibliography.bib}
  \usepackage{langsci-optional}
\usepackage{langsci-gb4e}
\usepackage{langsci-lgr}

\usepackage{listings}
\lstset{basicstyle=\ttfamily,tabsize=2,breaklines=true}

%added by author
% \usepackage{tipa}
\usepackage{multirow}
\graphicspath{{figures/}}
\usepackage{langsci-branding}

  
\newcommand{\sent}{\enumsentence}
\newcommand{\sents}{\eenumsentence}
\let\citeasnoun\citet

\renewcommand{\lsCoverTitleFont}[1]{\sffamily\addfontfeatures{Scale=MatchUppercase}\fontsize{44pt}{16mm}\selectfont #1}
  
  %% hyphenation points for line breaks
%% Normally, automatic hyphenation in LaTeX is very good
%% If a word is mis-hyphenated, add it to this file
%%
%% add information to TeX file before \begin{document} with:
%% %% hyphenation points for line breaks
%% Normally, automatic hyphenation in LaTeX is very good
%% If a word is mis-hyphenated, add it to this file
%%
%% add information to TeX file before \begin{document} with:
%% %% hyphenation points for line breaks
%% Normally, automatic hyphenation in LaTeX is very good
%% If a word is mis-hyphenated, add it to this file
%%
%% add information to TeX file before \begin{document} with:
%% \include{localhyphenation}
\hyphenation{
affri-ca-te
affri-ca-tes
an-no-tated
com-ple-ments
com-po-si-tio-na-li-ty
non-com-po-si-tio-na-li-ty
Gon-zá-lez
out-side
Ri-chárd
se-man-tics
STREU-SLE
Tie-de-mann
}
\hyphenation{
affri-ca-te
affri-ca-tes
an-no-tated
com-ple-ments
com-po-si-tio-na-li-ty
non-com-po-si-tio-na-li-ty
Gon-zá-lez
out-side
Ri-chárd
se-man-tics
STREU-SLE
Tie-de-mann
}
\hyphenation{
affri-ca-te
affri-ca-tes
an-no-tated
com-ple-ments
com-po-si-tio-na-li-ty
non-com-po-si-tio-na-li-ty
Gon-zá-lez
out-side
Ri-chárd
se-man-tics
STREU-SLE
Tie-de-mann
}
  % \togglepaper[3]%%chapternumber
}{}

\begin{document}
\maketitle

\section{Introduction}

This paper investigates lack of agreement in Spanish SE structures when the Internal Argument (IA) is not animate (and therefore does not require Differential Object Marking (DOM)\footnote{\label{05:fn:fernandez:1}The requirements for DPs to be DOM-marked in Spanish are much more complex and still subject to debate (see \citealt{Leonetti2004}, \citealt{Rodriguez-Mondonedo2007}, \citealt{Lopez2012} among many others).}) and remains in postverbal position.\footnote{As different authors have noted (see \citealt{OrtegaSantos2008} and references therein) number mismatches with postverbal subjects are pervasive across languages. I leave this matter aside since it is beyond the scope of this paper.} The basic asymmetry is presented in \REF{ex:05:1}:

\ea \label{ex:05:1}
\ea \label{ex:05:1a}
        Agreeing SE \\
        \gll Se discutieron los resultados.  \\
            SE discuss.\textsc{3pl.pst} the results. \\
        \glt `The results were discussed / Someone discussed the results.'
\ex \label{ex:05:1b}
        Non-agreeing SE \\
        \gll Se discutió los resultados.  \\
            SE discuss.\textsc{3sg.pst} the results. \\
        \glt `The results were discussed / Someone discussed the results.'
\z
\z

In \REF{ex:05:1a} we see the ``standard'' pattern that I refer to as ``agreeing SE'',\footnote{I want to highlight that I shorten the whole terms ``agreeing SE pattern" and ``non-agreeing SE pattern." I do not want to imply with these labels that the asymmetry relies on agreement with the clitic.} where the verb \textit{discutieron} `discussed' agrees with the IA \textit{los resultados} `the results'. This structure is subject to variation, reflected in \REF{ex:05:1b}, where there is a lack of agreement: the verb \textit{discutió} shows 3rd singular inflection despite the fact that the IA is plural.

The goal of this paper is to give a syntactic analysis for the asymmetry exemplified in \REF{ex:05:1}. I defend that this analysis has to account for two main aspects of the phenomenon: (i) intra-speaker optionality and (ii) number/person agreement asymmetry. While (ii) has been widely explored and discussed in the literature, (i) is a novel hypothesis.

For (i), I maintain that the attested non-agreeing pattern (\citealt{RaposoUriag1996,DAlessandro2007,Mendikoetxea1999,OrmazabalRomero2019,SanchezLopez2002}, among many others) freely alternates with its agreeing counterpart and that this alternation belongs to a single grammar. To defend this idea, I provide evidence from oral interviews extracted from a corpus that shows that a single speaker may produce one pattern or another indistinctly. Syntactically, I argue that there are no specific properties, such as the shape of the IA, responsible for the lack of agreement.

Regarding (ii), it has been shown that the asymmetry only affects number, since person agreement is always banned in SE structures \citep{Lopez2007}. Previous analyses have related this fact to a person restriction on the IAs by means of a Multiple Agree or similar mechanisms \citep{DAlessandro2007,Lopez2007}. However, I show that these approaches do not capture the fact that strong pronouns are always banned in Spanish SE structures, as opposed to other languages such as Italian or Icelandic where the restriction only holds for 1st and 2nd person pronouns.

The gist of the analysis is not new: it is based on defective intervention effects \citep{Chomsky2001a} created by an oblique element that can be avoided via movement \citep{Sigurdsson1992,Sirg_Holm2008}. The novelty is that Spanish SE has not been treated as a intervener before, neither as a case of intra-speaker optionality. In particular, I follow the idea that this optionality is part of syntax \citep{Biberauer_Richards2006} and adopt \citegen{Obataetal2015} and \citegen{Obata_Epstein2016} proposal that different grammatical outputs are explained by the timing of syntactic operations.\footnote{This is a formulation within the Minimalist Program of an old idea already present in the P\&P framework (for instance in treatments of wh- or verb movement). See \citet{Georgi2014} and ref. therein.}
I keep the intuition that the specific featural configuration of SE has an impact on agreement, mainly blocking the possibility of person agreement with the IA. The new twist is that in Spanish the number asymmetry may be explained via cliticization, since Multiple Agree does not capture the impossibility of strong pronoun licensing.

The paper is structured as follows: \sectref{05:sec:fernandez:2} introduces SE structures and the variation data. In \sectref{05:theory}, I present in more detail the puzzles that SE presents for Case and Agreement, and I review some previous approaches. \sectref{05:proposal} focuses on the proposal and a possible extension to \textsc{dat-nom} Spanish psych-verbs agreement variation. \sectref{05:sec:fernandez:5} concludes the paper.

\section{The data}\label{05:sec:fernandez:2}

\subsection{SE structures in Spanish}

Non-paradigmatic \textit{se} (SE) has been considered a defective clitic in Spanish because it does not have number or person inflection.\footnote{I focus on a very specific instance of \textit{SE}. Since it is well-known that this clitic is involved in a varied range of structures (reflexive, inchoative, aspectual, etc.), not only in Spanish but in Romance languages in general (see for instance \citealt{Mendikoetxea2012} and \citealt{MacDonald2017}).} It has been described as 3rd singular clitic, but its exact $\phi$-featural content differs among proposals (see \citealt{Torrego2008} and references therein). Here, I analyse SE as a clitic with a valued 3rd person feature and an underspecified number feature, following \citet{DAlessandro2007}.

SE structures are characterized by showing only one overt DP in IA position, which traditionally has been considered to be the subject when there is agreement \REF{ex:05:SEbasicacopy}, or the direct object, when there is no agreement. These configurations have been called ``passive SE'' and ``impersonal SE'' respectively (see \citealt{Mendikoetxea1999, SanchezLopez2002} and references therein).

I adhere to the perspective of more recent proposals that cast doubts on a double derivation and consider that there exists only one SE structure with different agreement outcomes \citep{PujalteSaab2014, OrmazabalRomero2019, Gallego2016, Gallego2019}. There are at least two main reasons for supporting this view. On the one hand, regarding interpretation, in current Spanish both patterns are virtually equivalent: SE ``absorbs'' the External theta-role \citep{Cinque1988d}
and the subject gets an ``arbitrary'' or ``indefinite'' interpretation (\citealt{RaposoUriag1996}, among others).\footnote{It is beyond the scope of this paper to focus on the semantics of SE structure; however, it is important to note that roughly speaking, in SE structures, the subject is either unknown by the speaker or the speakers do not want to make it explicit. In English, the closer translation is by the pronoun `one', although I also use passives or an arbitrary `they' in the examples.} The comparison between a regular transitive sentence and a SE sentence is exemplified in \REF{ex:05:SEactpas}:

\ea \label{ex:05:SEactpas}
    \ea \label{ex:05:SEactpasa}
        \gll Las investigadoras discutieron los resultados. \\
            the researchers discuss.\textsc{3pl.pst} the results.\\
        \glt `The researchers discussed the results.'
    \ex \label{ex:05:SEactpasb}
        \gll Se discutieron/discutió los resultados.  \\
            SE discuss.\textsc{3pl.pst} the results. \\
        \glt `The results were discussed / Someone discussed the results.'
\z \z

On the other hand, as I show in the next sections, there seems to be no specific syntactic conditions that lead us to think that there exist two different derivations. Before moving to the theoretical aspects, I introduce variation data in more detail.

\subsection{Variation: non-agreeing SE}

In this section, I present the variation data and defend the hypothesis that for some speakers, agreement with the IA in SE structures is optional. In particular, I show that not only are
the two variants used by one speaker, but they also show no asymmetries in syntactic properties, such as verb aspect or definiteness of the IA.

In \citet{Arias_Fernandez2020}, we highlight that the precise distribution of non-agreeing SE remains a mystery, even though the phenomenon has traditionally been present in the literature on Spanish grammar \citep[see][]{SanchezLopez2002}. Consider for instance the examples reported in ALPI (\textit{Atlas Lingüístico de la Península Ibérica}):

\newpage
\ea\label{ex:05:ALPI} \citet[data from ALPI;][23]{DeBenito2010}
    \ea\label{ex:05:ALPIa}
        \gll Se necesita obreros. \\
        SE need.\textsc{3sg} workers. \\
        \glt ‘Workers are needed.’
    \ex \label{ex:05:ALPIb}
        \gll Se vende patatas. \\
         SE sell.\textsc{3sg} potatoes. \\
        \glt ‘Potatoes are sold.’
    \ex \label{ex:05:ALPIc}
        \gll En el huerto se podía plantar rosales. \\
       In the orchard SE could.\textsc{3sg} plant.\textsc{inf} rose-bushes. \\
        \glt ‘In the orchard rose bushes could be planted.’
\z \z

The fact that non-agreeing SE has been normatively banned and
considered an error from oral speech may be responsible for the lack of a comprehensive empirical study and, at the same time, prove that the phenomenon exists and is pervasive to some extent (\citealt[36]{SanchezLopez2002}). By way of illustration, the works that mention the distribution generally describe it as more frequent in American varieties (\citealt{Mendikoetxea1999} and references therein) without further specification. The rich linguistic diversity of American varieties makes it dubious that there are no specific aspects such as language contact that may influence the phenomenon. I am not in a position of filling this gap, but it is worth noting that data from new corpora challenge some of the traditional ideas. Let me sketch two of them.

Firstly, the incidence of the phenomenon in different sociolinguistic contexts should be explored. In particular, I want to highlight that there are plenty of examples from the press \citep[see][]{Arias_Fernandez2020}, suggesting that it is not so straightforward to relegate the phenomenon to oral speech. Some of these examples from different dialectal areas are shown in \REF{ex:05:NOW}:

\ea\label{ex:05:NOW} Data from NOW corpus, \citep{Davies}
    \ea Mexico \label{ex:05:NOWa}\\
        \gll Se descubrió las verdaderas causas de su renuncia.\\
        SE discover.\textsc{3sg.pst} the real reasons of his resignation. \\
        \glt ‘The real reasons for his resignation were discovered.’
    \ex Venezuela \label{ex:05:NOWb}\\
        \gll Aún no se tiene datos específicos de los daños. \\
        yet no SE have.\textsc{3sg} data.\textsc{pl}  specific of the damages. \\
        \glt ‘There are no specific data from the damages yet.’
    \ex Bolivia \label{ex:05:NOWc} \\
        \gll En la propuesta técnica se consideró estos aspectos. \\
        in the proposal technical SE consider.\textsc{3sg.pst} these aspects. \\
        \glt ‘These aspects were not considered in the technical proposal.’
    \ex Argentina \label{ex:05:NOWd} \\
        \gll  Hasta el día de hoy, no se sabe las causas exactas de su muerte. \\
        until the day of today no SE know.\textsc{3sg} the reasons exact of his death.\\
        \glt ‘To this day the exact cause of his death is not known.’
\z \z

Secondly, some syntactic properties have been identified as being responsible for lack of agreement. One of them is the degree of definiteness of the DP, NPs being the most likely to appear in non-agreeing SE (\citealt[1677]{Mendikoetxea1999}). Notice that the examples in \REF{ex:05:NOWa}, \REF{ex:05:NOWc} and \REF{ex:05:NOWd} already challenge this point (see also \cite{DeMello1995}). In the same vein, imperfective verbal aspect has been pointed out as favouring agreement (\citealt[1678]{Mendikoetxea1999}). Again, the examples in \REF{ex:05:NOWa} and \REF{ex:05:NOWc} prove that even if this is a tendency, the phenomenon is possible with perfective verbal aspect.

Even if the prevailing view is on the right track, meaning that the phenomenon is more extended in American varieties, that does not preclude that it is not present in European Spanish, as the examples in \REF{ex:05:ALPI} reveal. This is corroborated by the data collected in COSER \citep{COSER}. This corpus contains transcriptions of interviews with elder speakers from rural areas of Spain. Consider \REF{ex:05:COSER}:

\ea\label{ex:05:COSER} Data from COSER corpus, \citep{COSER}
    \ea\label{ex:05:COSERa}
        \gll no se echaba esos compuestos que se echan en la comida \\
        no SE put.impfv.3sg those compounds that SE put.3pl in the food \\
        \glt ‘They(arb) didn’t put those compounds that are put in the food’
    \ex
        \gll Y con manteca, se hacía unas gachas y eso alimenta…\\
        and with butter SE make.impfv.3sg some oatmeal and that feeds \\
        \glt ‘And with butter, they(arb) made oatmeal and that is nourishing’
    \ex
        \gll También se cultiva muchas cebollas \\
        also SE cultivate.3sg much onions \\
        \glt ‘A lot of onions were also cultivated’
    \ex\label{ex:05:COSERd}
        \gll se corta los trozos gordos y aluego se hacen trocitos \\
        SE cut.3sg the pieces big and then SE do.3pl pieces \\
        \glt ‘It is chopped in big pieces and then little pieces are made’
\z \z

Examples \REF{ex:05:COSERa} and \REF{ex:05:COSERd} are especially relevant since they contain agreeing and non-agreeing SE, illustrating that the same speaker may alternate between both options. Its also worth pointing out that these examples differ from the ones from ALPI in \REF{ex:05:ALPI} in that they are instances of spontaneous speech. This is most likely the reason why all the examples from ALPI contain a bare NP, while we see that in oral speech examples with DPs are also possible. In fact, only one third of the non-agreeing SE examples gathered from the corpus contained bare NPs \citep{Arias_Fernandez2020}.

This evidence aligns with the facts indicated above about the data in \REF{ex:05:NOW} from American varieties. However, to avoid the risk of overgeneralization, I am going to consider only the last pieces of evidence for my analysis; that is to say, I restrict my proposal to oral European Spanish data. In sum, the key aspect that I attempt to reflect is that there are two possible agreement outcomes in SE contexts that can freely alternate regardless of the aspect of the verb or the shape of the IA.


\section{The puzzle of SE} \label{05:theory}

\subsection{Agreement}\label{05:sec:fernandez:3.1}
This section reviews the theoretical challenges that SE structures present regarding agreement and Case. The basic asymmetry is reminded in \REF{ex:05:basicasymcopy} below. In \REF{ex:05:SEbasicacopy}, there is number agreement between T(ense) and the IA \textit{los resultados} `the results'; while in \REF{ex:05:SEbasicvcopy}, the verb is inflected in 3rd person singular, thus there is no agreement with the IA.

\ea \label{ex:05:basicasymcopy}
\ea \label{ex:05:SEbasicacopy}
        Agreeing SE \\
        \gll Se discutieron los resultados.  \\
            SE discuss.\textsc{3pl.pst} the results. \\
        \glt `The results were discussed / Someone discussed the results.'
\ex \label{ex:05:SEbasicvcopy}
        Non-agreeing SE \\
        \gll Se discutió los resultados.  \\
            SE discuss.\textsc{3sg.pst} the results. \\
        \glt `The results were discussed / Someone discussed the results.'
\z \z

The first question about these patterns concerns what $\phi$-features are involved in agreement. It is clear that there is an asymmetry in number agreement between \REF{ex:05:SEbasicacopy} and \REF{ex:05:SEbasicvcopy}, whereas there is no apparent problem regarding person: the IA is 3rd person singular and the verb shows 3rd person singular morphology. However, this point needs a more careful consideration, since, as it is well-known, 3rd person morphology may reflect default valuation (see \cite{Preminger2014}, among many others).

The following examples show that person verbal morphology is not possible in SE contexts \REF{ex:05:SEpersona} as opposed to non-SE (regular active) contexts \REF{ex:05:SEpersonb}:

\ea \label{ex:05:SEperson} \citet[127]{Lopez2007}
\ea[*]{\gll Se vimos unos lingüistas en el mercado ayer.  \\
            SE see.\textsc{1pl.pst} some linguists in the market yesterday. \\
        \glt Intended meaning: `Some of us linguists were seen in the market.'}\label{ex:05:SEpersona}
\ex[]{\gll Unos lingüistas vimos/visteis/vieron un gran melón. \\
            some linguists see\textsc{1pl/2pl/3pl.pst} a big watermelon. \\
        \glt `Some (of us/of you) linguists saw a big watermelon.'}\label{ex:05:SEpersonb}
\z \z

As we see in \REF{ex:05:SEpersonb}, a person mismatch between the verb and the 3rd person subject is allowed in Spanish,\footnote{This phenomenon is referred to as ``unagreement'', ``anti-agreement'' or ``disagreement'', see \citet{Hohn2015} for a detailed overview.} but that option is excluded with SE. \cite{Lopez2007} takes this as evidence that the IA must be 3rd person or, in other words, that it can never be 1st or 2nd person.

This constraint, known as ``person restriction'' \citep{Burzio1986}, has typically been  described for Italian \citep{DAlessandro2007, Pescarini2018} and for Icelandic quirky subject structures (\citealt{Sigurdsson1992} and following work):

\ea \label{ex:05:PRIT} \citet[89]{DAlessandro2007}
\ea[*]{\gll In televisione si vediamo spesso noi.  \\
            in television SE see.\textsc{1pl} often we. \\
        \glt `One often sees us on TV.’}\label{ex:05:PRITa}
\ex[]{\gll In televisione si vediamo spesso Maria/lui.  \\
            in television SE see.\textsc{1pl} often Maria/he.\\
        \glt `One often sees Maria/him on TV.’}\label{ex:05:PRITb}
\z \z

\newpage
\ea \label{ex:05:PRICEL} \citet[254]{Sirg_Holm2008}
\ea[*]{\gll  Honum líki\dh{} \TH i\dh. \\
             he.\textsc{dat} like.\textsc{2pl} you.\textsc{pl}. \\
        \glt `He likes you.'}\label{ex:05:PRICELa}
\ex[]{\gll Honum líka  \TH eir.  \\
            me.\textsc{dat} like.\textsc{3pl.pst} they. \\
        \glt `He likes them.'}\label{ex:05:PRICELb}
\z \z

In Italian, the person restriction arises precisely in
SE (\textit{si} in Italian) contexts as shown in \REF{ex:05:PRIT}. From the comparison with this language (see \REF{ex:05:SEitpers}--\REF{ex:05:SESPpers} below), it can be concluded that such restriction does not hold for Spanish SE since third person pronouns are also banned
(\citealt[248]{OrdonezTrevino2016}):\footnote{\citet{Rivero2004} shows that certain contexts in Spanish, structures involving specific psychological verbs with inherent \textit{SE} morphology seem to be subject to the person restriction. She argues against a Multiple Agree perspective to account for such scenarios, which I leave aside in this paper.}

\ea \label{ex:05:SEitpers} \citet[80]{Pescarini2018}
\ea[]{\gll Lui si vede spesso in televisione. \\
            He SE see.\textsc{3sg} often in television. \\
        \glt `One often sees him on TV.'}\label{ex:05:SEitpersa}
\ex[*]{\gll Tu si vedi spesso in televisione. \\
            You SE see.\textsc{2sg} often in television. \\
         \glt `One sees you often on TV.'}\label{ex:05:SEitpersb}
\z \z

\ea \label{ex:05:SESPpers}
\ea[*]{\gll Se ve él {a menudo} en televisión \\
            SE see.\textsc{3sg} he often in television \\
        \glt `One often sees him on TV'}\label{ex:05:SESPpersa}
\ex[*]{\gll Se ve/s tú {a menudo} en televisión \\
            SE see.\textsc{3sg/2sg} you often in television \\
         \glt `One sees you often on TV'}\label{ex:05:SESPpersb}
\z \z

Different authors have considered that this restriction is due to the fact that the Probe, T, establishes a relationship with both the IA and the clitic \citep{DAlessandro2007, Lopez2007}. This is possible by means of a Multiple Agree \citep{Hiraiwa2001} link, where one Probe is able to agree with more than one Goal:

\newpage
\ea \label{ex:05:esqMA} Multiple Agree (\citealt[69]{Hiraiwa2001}) \\
\ConnectTail{\ConnectTail{$\upalpha$}[A]}[B]  > \ConnectHead[2ex]{$\upbeta$}[A] > \ConnectHead [3ex]{$\upgamma$}[B]
\z
\vspace{3mm}

As argued by \citet{DAlessandro2007}, the Multiple Agree approach correctly captures the person restriction found in Icelandic and Italian, following a condition on feature specification \citep{Anagnos2005}. This condition bans Multiple Agree if the two Goals do not share the same feature value. In this case, since SE is 3rd person, the DP must also be 3rd person for the derivation to converge (see \citealt[§3]{DAlessandro2007} for a detailed discussion).\footnote{In \citegen{Lopez2007} system, the Probe targets a Complex Dependency that has been previously formed between the two Goals. This dependency requires a similar condition on feature coincidence than the one on Multiple Agree, hence I do not treat these proposals separately.}

Since, as we saw in \REF{ex:05:SESPpers}, there is no person restriction in Spanish SE, an alternative analysis to Multiple Agree should be considered. In this sense, I suggest that Spanish \textsc{nom} pronouns require a $\phi$-relationship with a Probe, which is prevented in SE contexts. In non-agreeing SE, T does not reach the IA at all, while in agreeing SE, there is only number agreement and consequently, pronouns can not  be licensed. This will become more clear in \sectref{05:proposal}.

%So far I have highlighted two properties of SE structures in Spanish: (i) person agreement is never possible; (ii) NOM strong pronouns are banned. The latter is directly linked to Case, which is the second topic to be discussed in this section.

\subsection{Case}
%Traditionally, two different SE structures have been differentiated. Agreeing-SE structures are claimed to be passives, where the agreeing IA is a NOM argument, while the non-agreeing-SE is considered an impersonal, where the IA is a regular ACC object. Different authors have cast doubts about this dichotomy (Ordóñez 2013, PujalteSaab 2014, \citep{OrmazabalRomero2019, Gallego2016,Gallego2019} and consider that there is only one structure with two different agreeing patterns.

Non-agreeing SE (traditionally considered ``impersonal SE'') has been analysed by arguing that T agrees either with SE \citep{RaposoUriag1996, Lopez2007, PujalteSaab2014, OrmazabalRomero2019} or with a null \textit{pro} \citep{Otero1986, Cinque1988d, Bosque&Reixach2009, Torrego2008}, while the IA is assigned \textsc{acc} Case. In this section, I present evidence, following ideas by \citet{OrdonezTrevino2016}, that challenge the assumption that the IA of non-agreeing SE is always \textsc{acc}.

There are only two tests to check for \textsc{acc} in Spanish: the presence of DOM (``a'' preceding animate objects), and the possibility of paraphrasing the argument by an \textsc{acc} clitic (pronominalization). In SE contexts, DOM is compulsory when the IA is a definite+animate argument as in \REF{ex:05:DOMa}.\footnote{see fn. \ref{05:fn:fernandez:1}} In the case of pronouns, they must be also DOM-marked, but the phrase can be dropped since there is obligatory clitic doubling, as we see in \REF{ex:05:DOMb}:

\ea\label{ex:05:DOM}
    \ea \label{ex:05:DOMa}
        \gll Se ve *(a) María en televisión. \\
        SE see.\textsc{3sg} \textsc{dom} Maria in television. \\
        \glt `One sees Mary on TV.'
    \ex \label{ex:05:DOMb}
        \gll Se *(te) ve (a ti) en televisión. \\
        SE you.\textsc{acc} see.\textsc{3sg} \textsc{dom} you.\textsc{obl} in television. \\
        \glt `One sees you on TV.'
\z \z

DOM seems a robust test for supporting that the IA of non-agreeing SE is \textsc{acc}. However, some authors challenge this evidence by showing that in SE contexts, DOM arguments are prone to be pronominalized by a \textsc{dat} clitic (\textit{le}), instead of an \textsc{acc} (\textit{lo}), even in non-leísta dialects \citep{Mendikoetxea1999,OrdonezTrevino2016}.\footnote{``Leísta'' speakers use the dative clitic \textit{le} instead of the accusative clitic when the object is animate (\citealt{Fernandez-Ordonez1999}, among many others).} Consider the following data from Mexican Spanish where there is no \textit{leísmo}, as \REF{ex:05:leísmo} shows, but the dative clitic appears in combination with SE:

\ea\label{ex:05:leísmo}         \citet[240]{OrdonezTrevino2016}\footnote{Boldface and italics in original, translation is mine.}
    \ea  \label{ex:05:leísmoa}
        \gll \textbf{A} Juan/Sara \textbf{lo}/{\textbf{la}} vieron cantando.\\
        \textsc{dom} Juan/Sara \textsc{cl.acc.3msg/cl.acc.3fsg}  see.\textsc{3pl.pst} singing. \\
        \glt `They saw Juan/Sara while s/he was singing.'
    \ex \label{ex:05:leísmob}
        \gll \textbf{A} Juan/Sara \textit{se} \textbf{\textit{le}} vio cantando.\\
        \textsc{dom} Juan/Sara SE \textsc{cl.dat.3sg}  see.\textsc{3sg.pst} singing. \\
        \glt `One saw Juan/Sara while s/he was singing.'
\z \z

Let me now show what happens when the IA does not accept DOM-marking. The only test that allows us to do that is pronominalization. If these arguments were assigned \textsc{acc} we would expect pronominalization to be possible. However, as different authors have pointed out \citep{Torrego2008, OrdonezTrevino2016}, this does not seem to be the case:\footnote{To some members of the audience of LSRL50 wondered about the factors that make \REF{ex:05:*selob}  worse than \REF{ex:05:*seloa}. I agree with this judgement, but the reason for this contrast is not obvious.
}

\ea\label{ex:05:*selo} \citet[243]{OrdonezTrevino2016}
    \ea \label{ex:05:*seloa}
        \gll {*} Esos libros se los/les prohibió en el franquismo. \\
        { } those books SE \textsc{acc.3mpl/dat.3pl}  prohibited.\textsc{3sg} in the franquismo. \\
        \glt `Those books were banned during Franco years.'
    \ex \label{ex:05:*selob} \citet[788]{Torrego2008}, from \citet{Ordonez2004} \\
        \gll {*} El arroz, se lo come cada domingo en este hostal. \\
        { } the rice SE \textsc{acc.3mpl} eat.\textsc{3sg} every Sunday in this hostel. \\
        \glt `In this hostel they eat rice every Sunday.'
\z \z

Note that the IA, here fronted, is not repleaceable by either an \textsc{acc} or a \textsc{dat} clitic \REF{ex:05:*selob}. The conclusion is that the IA cannot be a clitic, regardless of whether the speaker is ``leísta'' or speaks a dialect in which the ``se le'' effect of \REF{ex:05:leísmob} applies.\footnote{A reviewer wonders if DOM may be a precondition for the appearance of the clitic, which would be coherent with an analysis of DOM à la \citet{Lopez2012}. In fact, \citet{Mendikoetxea1999} already notes that pronominalization of non-animate IAs seems to be favoured if they are DOM-marked, although some examples with non-DOM objects are also attested. For the latter, \citegen{OrmazabalRomero2019} analysis can be adopted, by which the clitic is the spell-out of agreement between T and the IA. What is important for the main discussion is that neither of those cases seem to support that the IA gets \textsc{acc} Case.}

In sum, in this section I have outlined two puzzles that non-agreeing SE pre\-sents. Firstly, that person agreement is never possible and that there is no person restriction on the IA, meaning that all \textsc{nom} pronouns are banned. Secondly, the IA does not seem to receive \textsc{acc} Case, especially when the IA is a non-DOM object. This evidence leads me to suggest an analysis where the IA in non-agreeing SE structures gets valuation by default since there is no relationship, either regarding Case or agreement with T (or \textit{v}).\footnote{For the sake of simplicity, I am not considering the role of \textit{v*}. For these type of structures that show a T-IA relationship, it can be assumed that \textit{v} is $\phi$-defective, meaning that it does not trigger Spell-Out, or assign Case to its complement \citep{Chomsky2001a}.}


\section{Proposal: SE intervention and how to avoid it} \label{05:proposal}

In this section, I formulate an analysis that takes into account the pieces of the puzzle outlined so far. The gist of my proposal for non-agreeing SE is that the clitic creates an intervention effect that blocks agreement with the IA. The optional alternation with the agreeing version follows from applying a relative timing of AGREE and MOVE \citep{Obataetal2015,Obata_Epstein2016}. This analysis becomes more transparent when comparing SE contexts with the variation found in \textsc{dat-nom} configurations in Spanish, to which I devote the last part of the section. Let me explore these points in turn.

In the previous sections, I have highlighted that lack of agreement is not a consequence of the specific shape of the IA  when it is not animate. This claim allows us to discard an analysis based on definiteness effects (see for instance \citealt{Belletti1988a}) and leads us to think that there is another factor that prevents the Probe from reaching the expected Goal. The hypothesis that I put forward here is that it is SE itself that creates this ``barrier''.

This type of blocking effect has been extensively studied in the literature, known as ``defective intervention''  \citep{Chomsky2000}. This effect arises when an element is visible for the Probe's search but is not a suitable Goal itself:

\ea \label{ex:05:esqintervention} Defective Intervention (\citealt[69]{Hiraiwa2001}) \\
\ConnectTail{*$\upalpha$} > $\upbeta$ > \ConnectHead [2ex] {$\upgamma$}
\z
\vspace{3mm}

As we see in \REF{ex:05:esqintervention}, where > stands for c-command, if the Probe $\upalpha$ finds the intervener $\upbeta$ before reaching the Goal $\upgamma$, $\upbeta$ prevents agreement between $\upalpha$ and $\upgamma$. In SE contexts, $\upbeta$ is the clitic SE and $\upgamma$ the IA.

The paradigmatic example of this effect is found in Icelandic quirky subject configurations, where the \textsc{dat} argument blocks agreement with the \textsc{nom} IA in biclausal contexts \citep{Sigurdsson1992, Boeckx2000, Holmberg&Hroasdottir2003, Sirg_Holm2008, Preminger2014}.

\ea \label{ex:05:Icelraising}
        \citet[998]{Holmberg&Hroasdottir2003}
	\ea  \label{ex:05:Icelraisinga}
		\gll Mér vir{\dh}ast [hestarnir vera seinir].  \\
		me.\textsc{dat} seem.\textsc{3pl} the-horses.\textsc{nom} be slow. \\
		\glt`It seems to me that the horses are slow.’
	\ex \label{ex:05:Icelraisingb}
		\gll  {\TH}a{\dh} vir{\dh}ist/*vir{\dh}ast einhverjum manni [hestarnir vera seinir.] \\
		\textsc{expl} seem.\textsc{3sg}/*seem.\textsc{3pl} some man.\textsc{dat} the.horses.\textsc{nom} be slow.\\
			\glt `A man finds the horses slow.'
\z \z

As we see in \REF{ex:05:Icelraisingb}, when the \textsc{dat} is \textit{in situ}, between the verb and the \textsc{nom}, there is lack of agreement; while this is avoided in \REF{ex:05:Icelraisinga}, where the \textsc{dat} has raised above the verb. \citet{Sirg_Holm2008} show that these types of agreement configurations are also subject to variation in Icelandic, distinguishing three dialects: one with only agreement; a second one with only lack of agreement; and a third where speakers accept both variants. They propose a competing grammars approach for the latter, since speakers seem to alternate between the other two dialects.

There is a crucial difference between Icelandic variation and the one I present here, namely, there seems to be no dialect in Spanish where lack of agreement is always compulsory. In this sense, I want to defend the possibility of a true optionality within the same dialect (grammar), following \citet{Biberauer_Richards2006}. Consequently, I am not arguing that there are two distinct dialects or that a speaker can be bi-dialectal, but that both outcomes of agreement are possible as part of the same grammar.\footnote{An important matter left for future research is what is the role of preferences in grammar since agreeing SE is much more frequent than non-agreeing SE.}

Coming back to the specifics of SE configurations, remember that I assume that SE is endowed with a valued 3rd person feature and a underspecified number feature \citep{DAlessandro2007}. The crucial idea is that SE does not lack the feature number, and therefore, number is visible for the Probe. My hypothesis is that SE triggers the same effect as the \textsc{dat} and makes it impossible for T to reach the IA, as we see in \REF{ex:05:analysisb}:\footnote{The exact position where SE is first-merged is tangencial to my discussion. I assume that it is c-commanded by T and higher than the IA. It can be either an EA position in Spec,\textit{v} (following \citealt{RaposoUriag1996, DAlessandro2007, Torrego2008}) or heading Voice (see \citealt{MacDonald2017} and ref. therein)}

%\ea \label{ex:05:analysisb}
 %   Non-agreeing SE \\
%    \ConnectTail{T} \dots \ConnectHead[2ex]{SE}\textsubscript{[p:3][n]} \dots IA
%\z

\ea \label{ex:05:analysisb}
Non-agreeing SE \\
\includegraphics[width=3cm]{figures/FS_NAS.jpg}
\z

%\vspace{3mm}

A direct consequence of this analysis is that, if there is no T-IA relationship, the IA cannot be assigned Case. However, I would like to suggest that this is not necessarily a negative consequence, but it correctly predicts the asymmetry in licensing between non-animate DPs and pronouns that I presented in section 3.1. (see \REF{ex:05:SESPpers} repeated here as \REF{ex:05:SESPperscopy}).

\ea \label{ex:05:SESPperscopy}
\ea[*]{\gll Se ve él {a menudo} en televisión. \\
            SE see.\textsc{3sg} he often in television. \\
        \glt `One often sees him on TV.'}
\ex[*]{\gll Se ve/s tú {a menudo} en televisión. \\
            SE see.\textsc{3sg/2sg} you often in television. \\
         \glt `One sees you often on TV.'}
\z \z

It is an old observation that in Spanish and in other Romance languages such as Catalan, person agreement is the key licensing factor for \textsc{nom}, which makes the appearance of NOM pronouns impossible in \textit{SE} structures \citep{Bianchi2001,Bianchi2003,Rigau1991}. The question is then how non DOM-marked IAs of SE sentences are licensed.\footnote{We assume here the analysis of \citet[248]{OrdonezTrevino2016} that DOM objects receive inherent Case.} Either they are licensed by other means, such as focus, (\citealt{Belletti,Rossello2000,Etxepare&Gallego2019}, among others) or they receive another non-morphologically realized Case from \textit{v}. The latter option seems more promising considering the parallelism with QS structures. A specific flavour of \textit{v}, capable of assigning \textsc{nom} has been argued to appear in such structures \citep{Boeckx2008,Lopez2007,Gallego2018}.

Turning now to agreeing SE, it is plausible that in this case the intervention effect is avoided exactly as in Icelandic, i.e. by the movement of the intervener. Note that SE must cliticize on T \citep{Cinque1988d,DAlessandro2007}; therefore, SE must always raise.\footnote{A reviewer wonders about infinitive contexts in which SE is enclitic. This question, although very relevant, requires a more detailed description of the appearance of SE in bi-clausal contexts which for reasons of space I cannot develop here (see \citealt[1705--1715]{Mendikoetxea1999}; \citealt[43--49]{SanchezLopez2002}).} If we assume that rule-ordering is not predetermined, and that MOVE and AGREE may feed one another indistinctly (see \cite{Georgi2014}), two outcomes are possible. Agreeing SE is the result of SE cliticizing on T before Agree takes place (see \REF{ex:05:analysis}), whereas in non-agreeing SE cliticization happens after Agree (cf. \REF{ex:05:analysisb} above).

\ea \label{ex:05:analysis}
    Agreeing SE \\
    \ConnectTail{SE-T} \dots <SE> \dots \ConnectHead[2ex]{IA}
\z

\vspace{3mm}

In sum, I propose that both \REF{ex:05:analysis} and \REF{ex:05:analysisb} are possible derivations in a speaker's grammar, the only difference between them is the order of the operations MOVE and AGREE \citep{Obataetal2015, Obata_Epstein2016}.\footnote{A reviewer is worried about this kind of analysis in that it revamps the notion of extrinsic rule-ordering. It is indeed an old idea that goes back to \citet{Chomsky1965}, but it is not clear to me that it should not have a role in current Minimalist theories. \citet{Georgi2014} provides a very thorough argumentation about the topic and convincingly shows that intrinsic ordering cannot predict all attested orderings of Merge and Agree to which she concludes that grammar requires both types of rule-ordering. It is also interesting to point out that OT is based on extrinsic rule ordering (as \citealt[253]{Georgi2014} also notes), which is not incompatible with the Minimalist framework (see \citealt{BroekhuisWoolford2010}).} Find the asymmetry summarized in \REF{ex:05:order}:

\ea \label{ex:05:order}
    \ea Agreeing SE: MOVE > AGREE
    \ex Non-agreeing SE: AGREE > MOVE
\z \z

It remains to be discussed how this analysis accounts for the impossibility of person agreement in SE contexts (see \sectref{05:sec:fernandez:3.1}). My hypothesis is that the specific featural configuration of SE is again responsible for this behaviour. When SE cliticizes on T before Agree, the person feature of SE values the person feature of T. This is not possible with number since the underspecified number feature of SE cannot provide any value. Consequently, when T probes, it only needs to check number with the IA. The steps are schematized in \REF{ex:05:clitic}:

\begin{exe}
	 \ex \label{ex:05:clitic}
	 \begin{xlisti}
		\ex  SE-T\textsubscript{[P:3][N:]} \dots <SE> \dots IA\textsubscript{[P:3][N:PL]} \hfill SE cliticization
		\ex SE-T\textsubscript{[P:3][N:\textbf{PL}]}  <SE>  ... IA\textsubscript{[P:3][N:PL]} \hfill T-IA agreement
	\end{xlisti}
\end{exe}

This possibility raises non-trivial questions about cliticization that must be addressed in the future.

\subsection{Extension: non-agreeing \textsc{dat} structures}

Before I conclude, I want to extend my analysis to a similar case of agreement variation in Spanish found in \textsc{dat}-\textsc{nom} contexts. As argued in \citet{Arias_Fernandez2020}, some Spanish speakers optionally obviate agreement (see \REF{ex:05:DPVb}), otherwise compulsory, (see \REF{ex:05:DPVa}) between the verb and the IA of \textsc{dat}-\textsc{nom} psych-verb structures.

\ea \label{ex:05:DPV}
\ea \label{ex:05:DPVa}
        \gll No me gustan las funerarias  \\
            no me.\textsc{dat} like.\textsc{3pl} the funeral-homes \\
        \glt `I don't like funeral homes'
\ex \label{ex:05:DPVb}
        \gll No me gusta las funerarias  \\
            no me.\textsc{dat} like.\textsc{3sg} the funeral-homes \\
        \glt `I don't like funeral homes'
\z \z

In this context, analyzing the lack of agreement in \REF{ex:05:DPVb} as a result of defective intervention is more straightforward since the intervener is a \textsc{dat} argument, exactly as is the case in Icelandic (see \REF{ex:05:Icelraising}). The difference with this language is, again, that Spanish does not show a person restriction on the IA, but a ban on \textsc{nom} pronouns when there is no agreement, as we see in \REF{ex:05:DATPRb}:

\ea \label{ex:05:DATPR}
    \ea \label{ex:05:DATPRa}
        \gll Le gustas tú. \\
         him/her.\textsc{dat} like.\textsc{2sg} you.\textsc{sg}. \\
        \glt `S/he likes you.'
    \ex \label{ex:05:DATPRb}
        \gll {*}Le gusta tú/nosotros/ellos. \\
         { }him/her.\textsc{dat} like.\textsc{3sg} you.\textsc{sg}/we/they. \\
        \glt `S/he likes you/us/them.'
\z\z

Therefore, the same analysis presented for SE can be adopted for this case, as \REF{ex:05:analysisDAT} reflects:

\ea \label{ex:05:analysisDAT}
    \ea Agreement: MOVE > AGREE \\
      \textsc{dat} \dots \ConnectTail{T} \dots <\textsc{dat}> \dots \ConnectHead[2ex]{IA}
      \vspace{3mm}
    \ex Lack of agreement: AGREE > MOVE \\
    \ConnectTail{T} \dots \ConnectHead[2ex]{\textsc{dat}} \dots IA
\z\z

\vspace{3mm}

It is important to note that, while in agreeing SE only number agreement with the IA is allowed, in \textsc{dat}-\textsc{nom} contexts there is full $\phi$-agreement between T and the IA (see \REF{ex:05:DATPRa}). That said, this contrast is not necessarily problematic for the analysis since, unlike SE, \textsc{dat} clitics do not cliticize on T
\citep[128]{DAlessandro2007}. This would prevent any interference between T and the IA once the \textsc{dat} has moved above T. In fact, some authors maintain that the \textsc{dat} of psych-verbs does not land in Spec,T at any point of the derivation because it is not a subject vis-à-vis Icelandic \citep{Gutierrez-Bravo2006, Fabregasetal2017}.

In summary, I have proposed that SE and \textsc{dat} clitics may trigger defective intervention effects in Spanish if they are \textit{in situ} when Agree takes place. At the same time, the different agreement outcomes (full or partial agreement) depend on the featural configuration of these clitics. This analysis tries to shed some light on the role of optionality in grammar and in syntactic theory.

\section{Conclusion}\label{05:sec:fernandez:5}

This paper has examined variation in SE structures in Spanish with a special focus on non-agreeing SE. Empirically, I have shown that non-agreeing SE seems to be acceptable with definite DPs and that it optionally alternates with the agreeing version. This seems to be accurate at least for European Spanish and more investigation is needed to include American varieties.
Theoretically, I have maintained that there is only one SE structure with two possible agreement outcomes. Evidence for this comes from the impossibility of having \textsc{acc} clitics in SE contexts.

The analysis I have put forward considers SE as a defective intervener that blocks agreement if Agree happens before SE cliticizes. On the contrary, agreement is possible if SE raises before Agree. I have shown that this analysis correctly predicts that, in Spanish, there is no person restriction regarding the IA and that this analysis is consistent with variation in \textsc{dat}-\textsc{nom} contexts.

Further research is required to assess the hypotheses presented throughout the paper about the mechanism of strong pronouns licensing and \textsc{nom} Case assignment in Spanish, the impact of cliticization, and the role of optionality within grammar.



%agreement across DOM

%This analysis allows me to explain the optionality as part of grammar following the idea that the relative timing of syntactic operations can be parametrized \citep{Obataetal2015,Obata_Epstein2016}. If SE must cliticize on T \citep{DAlessandro2007}, the crucial difference is if Agree takes place before or after cliticization.

%Following the assumption that there is only one SE and the hypothesis that it is the element responsible for blocking agreement with the IA, the question is how it is possible that sometimes it blocks only person, allowing number agreement with the IA, and others it seems to block agreement completely, triggering 3rd singular morphology on the verb. My hypothesis is that cliticization explains this dual behaviour. When SE cliticizes in T, it only prevents person from probing, and gets a similar effect than MA but dispensing with the person restriction on the IA. On the other hand, when it remains in situ, it blocks agreement completely.

%This proposal keeps \citet[34]{DAlessandro2007}'s intuition that "SE seems to be halfway between a lexical a functional element". I also assume \cite{DAlessandro2007}'s proposal on the featural configuration of the clitic, a valued 3rd person feature and an underspecified number feature, which is responsible for this hybrid nature.




\section*{Abbreviations}
\begin{tabularx}{.45\textwidth}{lQ}
\textsc{acc} & Accusative \\
\textsc{dat} & Dative \\
\textsc{dom} & Differential Object Marking \\
\textsc{fsg} & feminine singular \\
\end{tabularx}
\begin{tabularx}{.45\textwidth}{lQ}
\textsc{ia} & Internal Argument \\
\textsc{msg} & masculine singular \\
\textsc{nom} & Nominative \\
\textsc{spec} & Specifier \\
\textsc{t} & Tense \\
\end{tabularx}


\section*{Acknowledgements}
Some of the ideas presented in this paper were developed during a research stay in the University of Utrecht. I am grateful to Roberta D'Alessandro for her guidance and to Ángel Gallego for his comments on a previous version of this paper. I also want to thank the audience of LSRL50 for their useful suggestions and two anonymous reviewers for their comments and remarks for further research. This research has been supported by the Spanish Government: (i) “Redes de variación microparamétricas en las lenguas románicas” (FFI2017-87140-C4-1-P, PIs: Á. Gallego \& J. Mateu); (ii) Predoctoral Grant FPU2016; (iii) mobility grant for predoctoral stays FPU/EST2018.

\printbibliography[heading=subbibliography,notkeyword=this]

\end{document}
