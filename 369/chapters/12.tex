\documentclass[output=paper,colorlinks,citecolor=brown,draftmode]{langscibook}
\ChapterDOI{10.5281/zenodo.7525114}


\author{Annie Helms\orcid{https://orcid.org/0000-0001-9960-3351}\affiliation{University of California, Berkeley}}


\title{Sociophonetic analysis of mid front vowel production in Barcelona}

\abstract{Studies aimed at observing Spanish contact-induced changes in Catalan among Spanish-Catalan bilinguals in Catalonia have evidenced both assimilation and dissimilation in the production of Catalan mid front vowels. However, the general lack of studies aimed at observing Catalan contact-induced changes in Spanish is an oversight of bidirectionality as an expectation of language contact. Accordingly, the present study uses both Catalan and Spanish mid front vowel production data from Barcelona to investigate the roles of age, gender, and language dominance in the processes of assimilation (Catalan cognate effects in Spanish) and dissimilation (distinctly produced cross-linguistic mid front vowel categories). While no cognate effect was observed in Spanish, younger speakers maintain less distinction between Catalan mid front vowel categories than older speakers, and females have less overlap between Spanish /e/ and Catalan /\textepsilon/ than males. These results are consistent with a male-led change towards greater assimilation and more overlapping productions of all three mid front vowel categories. While cognitive factors such as language dominance and cognate status are central to models of bilingual phonetic representation, it is paramount to situate the bilingual individual within the context of the community and acknowledge the external social factors which also mediate variation in acquisition and production.

%KEYWORDS: Spanish, Catalan, bilingualism, sociophonetic variation, vowels
}

% PACKAGES IN PHONETICS & PHONOLOGY
%In order to type phonetic fonts, you have to use this code in your preamble
%Need to remove from Preamble
% \usepackage{tipa}
% \let\ipa\textipa
% \usepackage{vowel}
% \newcommand{\BlankCell}{}
% \usepackage{ot-tableau}

% PACKAGES IN SYNTAX: Use this code to have the capability to make syntactic trees
% \usepackage{forest}
% \usepackage[noeepic]{qtree}

% % PACKAGES FOR IMAGES
% \usepackage{graphicx}

% % MISCELLANEOUS PACKAGES
% \usepackage{lastpage}
% \usepackage{hyperref}

% %move the following commands to the "local..." files of the master project when integrating this chapter
% %DO NOT CHANGE THIS SOURCE SEQUENCE %%%% Skip to 'custom footer for preprints'
% \usepackage{tabularx}
% \usepackage{langsci-basic}
% \usepackage{langsci-optional}
% \usepackage{langsci-gb4e}
% \bibliography{localbibliography}
% \newcommand{\orcid}[1]{}
% \pagenumbering{arabic}
% \setcounter{page}{256}

\IfFileExists{../localcommands.tex}{
  \addbibresource{../localbibliography.bib}
  % add all extra packages you need to load to this file

\usepackage{tabularx,multicol}
\usepackage{url}
\urlstyle{same}

\usepackage{listings}
\lstset{basicstyle=\ttfamily,tabsize=2,breaklines=true}

\usepackage{langsci-basic}
\usepackage{langsci-optional}
\usepackage{langsci-lgr}
\usepackage{langsci-osl}
% \usepackage{./langsci/styles/langsci-lgr}
% \usepackage{./langsci/styles/langsci-osl}
% \usepackage{langsci-gb4e}

\usepackage{tikz}
\usetikzlibrary{patterns,calc}
\pgfdeclarepatternformonly{south east lines}{\pgfqpoint{-0pt}{-0pt}}{\pgfqpoint{3pt}{3pt}}{\pgfqpoint{3pt}{3pt}}{
    \pgfsetlinewidth{0.6pt}
    \pgfpathmoveto{\pgfqpoint{0pt}{3pt}}
    \pgfpathlineto{\pgfqpoint{3pt}{0pt}}
    \pgfpathmoveto{\pgfqpoint{.2pt}{-.2pt}}
    \pgfpathlineto{\pgfqpoint{-.2pt}{.2pt}}
    \pgfpathmoveto{\pgfqpoint{3.2pt}{2.8pt}}
    \pgfpathlineto{\pgfqpoint{2.8pt}{3.2pt}}
    \pgfusepath{stroke}}
    
\usepackage{stmaryrd}
\usepackage{wasysym}
\usepackage{multirow}
\usepackage{caption}
\usepackage{subcaption}
\usepackage{mathrsfs}
\usepackage{qtree}

\usepackage{linguex}


  %pminos do not split footnotes
% \interfootnotelinepenalty=10000 %Footnote in Laporte chapters has to be split SN


%\DeclareIndexNameFormat{default}{%
%\nameparts{#1}%
%\usebibmacro{index:name}%
%{\index[names]}%
%{\namepartfamily}%
%{\namepartgiveni}%
% {}% L1
% {}% L2
%{\namepartprefix}% generates spurious space L3
%{\namepartsuffix}% generates spurious space L4
%}

%  {\DeclareIndexNameFormat{default}{%
%     \usebibmacro{index:name}{\index[names]}{#1}{#3}{#5}{#7}}}

%\DeclareIndexNameFormat{default}{%
%  \usebibmacro{index:name}{\sindex[nom]}{#1}{#3}{#5}{#7}}

%\DeclareIndexNameFormat{default}{%
%  \usebibmacro{index:name}{\sindex[person]}{#1}{#3}{#5}{#7}}
%\DeclareIndexNameFormat{default}{%
%\nameparts{#1} \usebibmacro{index:name}{\sindex[person]]}{\namepartfamily}{‌​\namepartgiven}{\nam‌​epartprefix}{\namepa‌​rtsuffix}}

%\newcommand{\smiley}{:)}

%\renewbibmacro*{index:name}[5]{%
%\usebibmacro{index:entry}{#1}%
%{\iffieldundef{usera}{}{\thefield{usera}\actualoperator}\mkbibindexname{#2}{#3}{#4}{#5}}}

% \newcommand{\noop}[1]{}

%remove for final
%\overfullrule=1mm

\newcommand{\tobi}[2]}}
\renewcommand{\S}[1]{\tobi{#1}{\textsc{*}}}

% this volume references
% puts: [this volume]
% already defined: \citetv
%\newcommand{\citepv}[1]{(\citeauthor{#1} \citeyear*{#1} [this volume])}
\newcommand{\citealtv}[1]{\citeauthor{#1} \citeyear*{#1} [this volume]}

%parentheses around example number
\newcommand{\pref}[1]{(\ref{#1})}

% in-text examples

\newcommand{\lnex}[1]{\textit{#1}} %target lang word
\newcommand{\lnlit}[1]{(lit.: `#1')} %literal reading
\newcommand{\lnlat}[1]{(#1)} % latinization
\newcommand{\lntrans}[1]{`#1'} %translation
\newcommand{\lnexl}[2]%
{\lnex{#1}{} \lnlat{#2}} % ex with latinization
\newcommand{\lnexlat}[3]{\lnex{#1}{} \lnlat{#2}{} \lntrans{#3}} % ex with latinization and tranl.

%ch01
\newcommand{\co}[1]{\mbox{\textbf{#1}}}

%ch09

\newcommand{\cyrbulg}[1]{\begin{otherlanguage*}{bulgarian}#1\end{otherlanguage*}}


%ch10
\newcommand{\nlp}{{\small NLP}}
\newcommand{\mwe}{{\small MWE}}
\newcommand{\rae}{{\small RAE}}
\newcommand{\lvc}{{\small LVC}}
\newcommand{\pos}{{\small P}o{\small S}}
%\newcommand{\todo}[1]{ \textcolor{red}{#1} }

%\renewcommand{\labelenumi}{\theenumi}
%\ainamefmt{{vv}{ll}{, ff}{, jj}} % fullname

\newcommand{\biberror}[1]{{\color{red}#1}}

\newcommand{\osenovaitem}{--~}
  %% hyphenation points for line breaks
%% Normally, automatic hyphenation in LaTeX is very good
%% If a word is mis-hyphenated, add it to this file
%%
%% add information to TeX file before \begin{document} with:
%% %% hyphenation points for line breaks
%% Normally, automatic hyphenation in LaTeX is very good
%% If a word is mis-hyphenated, add it to this file
%%
%% add information to TeX file before \begin{document} with:
%% %% hyphenation points for line breaks
%% Normally, automatic hyphenation in LaTeX is very good
%% If a word is mis-hyphenated, add it to this file
%%
%% add information to TeX file before \begin{document} with:
%% \include{localhyphenation}
\hyphenation{
    Beck-man
    Ngu-yen
    back-chan-nel
    back-chan-nels
    mo-not-o-nous
    ste-reo-typ-i-cal
}

\hyphenation{
    Beck-man
    Ngu-yen
    back-chan-nel
    back-chan-nels
    mo-not-o-nous
    ste-reo-typ-i-cal
}

\hyphenation{
    Beck-man
    Ngu-yen
    back-chan-nel
    back-chan-nels
    mo-not-o-nous
    ste-reo-typ-i-cal
}

  % \togglepaper[3]%%chapternumber
}{}


\begin{document}
\maketitle

\section{Introduction}

In Barcelona, Spanish and Catalan have been in close contact for centuries and many instances of lexical and phonological imposition and borrowing have been recorded \citep{sole2003language,sole2006llengues,arnal2011linguistic}. However, the nearly exclusive focus on the variable acquisition of Catalan in Spain has furthered a long-standing asymmetry favoring the study of the Spanish-influence in Catalan over Catalan-influence in Spanish \citep[18]{sole2003language}. This trend in the literature, exemplified by \citet[][22]{arnal2011linguistic} who states that ``in the current situation of generalized bilingualism in Catalonia, the change caused by contact does not affect Spanish, but rather only affects Catalan", runs counter to the expectation of bidirectionality in language contact situations with widespread bilingualism \citep{thomason1988contact,davidson2020asymmetry}. Furthermore, the lack of cross-linguistic production data in the literature limits the conclusions that can be drawn about outcomes of language contact, especially regarding the processes of assimilation and dissimilation theorized by the Speech Learning Model \citep[SLM;][]{flege1995second} and the revised Speech Learning Model \citep[SLM-r;][]{flege2021revised}.

A variable studied among Spanish-Catalan bilinguals in which this asymmetry is often present is the production of mid vowels in Catalan. The mid front  and mid back vowels of Catalan are contrasting, yielding minimal pairs (e.g. /net/ `grandson' and /n{\textepsilon}t/ `clean’; /os/ `bear' and /{\textopeno}s/ `bone'). Regarding the mid front vowels, with which this study is concerned, Spanish /e/ is produced lower (higher F1) than the Catalan /e/, but higher (lower F1) and more fronted (higher F2) than the Catalan /\textepsilon/ (\figref{fig:vowel_chart}).
    \begin{figure}
        \includegraphics[scale=0.54]{figures/helms_fig1.png}
        \caption{Vowel Spaces of Spanish \citep[][237]{ladefoged2015course} and Catalan \citep[][62]{carbonell1999catalan}}
        \label{fig:vowel_chart}
    \end{figure}
However, the most significant sources of variability among the three vowel categories are found across F1, not F2 \citep{bosch2011variability,cortes2019weighing,recasens2006dispersion,simonet2011production}. Certain factors, such as cognate status, language dominance, and age have been linked to the variable production of these vowels cross-linguistically. Therefore, the production of these mid front vowels in bilingual settings have been the focus of many studies as they provide an opportunity to study outcomes of language contact, phonological representations of bilinguals, and the social factors that mediate these processes.

\subsection{Models of phonological acquisition and representation}

The Speech Learning Model \citep[SLM;][]{flege1995second} and the Revised Speech Learning Model \citep[SLM-r;][]{flege2021revised} postulate that bilingual speakers do not maintain separate phonetic systems for each language, rather, the two systems co-exist in a mutual phonetic space and may influence one another. Whereas the SLM focuses on between-group differences and whether L2 speakers are able to produce target L1 categories, the SLM-r shifts focus to the individual and an individual's differentiation of L1 and L2 categories based on quantity and quality of input of L1 and L2.

\subsubsection{Assimilatory outcomes}

Both the SLM and SLM-r postulate that if phonetic differences between an L2 category and an L1 category are not perceived by a bilingual individual, the formation of a new category will be blocked. Blockage of category formation results in assimilation, by which the speaker may produce a composite L1-L2 category \citep{evans2007plasticity,kendall2012variation}. The acoustic properties of this composite category are ``defined by the statistical regularities present in the combined distributions of the perceptually linked L1 and L2 sounds" \citep[][41]{flege2021revised}. A number of factors, such as the quality and quantity of L1 and L2 input the bilingual receives in their lifetime  \citep{flege2002interactions,yeni2000pronunciation}, individual cognitive differences \citep{lev2013low}, or relative language activation \citep{grosjean2001bilingual}, will affect the overall acoustic profile of this composite sound. Additionally, assimilation may occur between two L2 categories when sufficient acoustic differences are not perceived or produced within the contrast.

\subsubsection{Dissimilatory outcomes}

Alternatively, the SLM and SLM-r describe the process of dissimilation, which may occur in the inventory of a bilingual that is able to perceive sufficient acoustic difference between the L1 and L2 categories, or between two L2 categories, thus preventing assimilation from occurring. In this case, the individual maintains  distinct categories in each language. Categories may ``deflect'' one another to maintain contrast in the shared phonetic space \citep{baker2005interaction,flege2021revised}, yielding categories that differ from those of a monolingual speaker. However, evidence of cross-linguistic dissimilation is less abundant in the literature as many studies report perception or production data from only one of the languages in question.

\subsection{Cognate effects}

Through a cognate effect, the phonology of cognate words in the non-target language may be activated during speech, potentially affecting the production of a word in the target language \citep{costa2000cognate,colome2010words}. Assimilation may be realized through cognate effects, where the influence of the L1 on L2 category production is strengthened in words that are cognate between the L1 and L2. \citet{amengual2016cross} provides evidence for a cognate effect in the production of Catalan mid back vowels in Mallorca, where productions of Catalan /\textopeno/ with incongruent Spanish cognates are raised, evidencing assimilation towards Spanish /o/.

\subsection{Language dominance}

Language dominance is a measure of linguistic history, linguistic attitudes, language proficiency, and language use, and is often a predictor of category production and perception of bilinguals \citep{birdsong2012bilingual}. Under the SLM and SLM-r \citep{flege1995second,flege2021revised}, language dominance, through language exposure and experience, can predict whether assimilation or dissimilation may occur. \citet{amengual2016perception} provides evidence for language dominance as a predictor of assimilation, where Spanish-dominant bilinguals assimilated both Catalan /e/ and Catalan /\textepsilon/ to Spanish /e/. Alternatively, \citet{bosch2011variability} find that systematic and consistent exposure to Catalan (i.e., greater Catalan-dominance) contributes to the production of distinct Catalan mid front vowel categories by bilinguals raised in Spanish-dominant homes, thus providing evidence for dissimilation between two L2 categories. No cross-linguistic comparisons were performed in the study, thus it is unknown whether or not language dominance additionally contributed to cross-linguistic dissimilation between Catalan /e/ and Spanish /e/ among these bilinguals.

\subsection{Age and gender}

Within the variationist sociolinguistic framework, social factors such as age and gender are correlated with sound changes in progress. Under the apparent-time construct \citep{bailey2004real}, generational sound change can be observed by comparing the speech of older speakers from that of younger speakers, where the speakers pertain to separate generations \citep[][45--46]{labov1994principles}. Patterns of language use across gender often are consistent with the Gender Paradox, where ``women conform more closely than men to sociolinguistic norms that are overtly prescribed, but conform less than men when they are not" \citep[][292--293]{labov2001principles}. Therefore, in both changes from above and below the level of conscious awareness, women tend to be the leaders of change. The combination of this principle with the apparent-time construct results in the methodological practice of treating younger women's speech patterns as suggestive of possible community-wide changes in progress \citep[][279]{labov2001principles}.

In Barcelona, age is additionally a correlate to access to explicit language instruction as the Catalonian Linguistic Normalization Law of 1983 has yielded a generational divide between those that have and have not had access to Catalan instruction in school. Although most studies of Spanish and Catalan production in Catalonia focus on the effects of language dominance and other cognitive factors, \citet{cortes2019weighing} examine the language of preschool-aged children and their parents in three neighborhoods of Barcelona. They observe that the children's ability to produce Catalan mid vowels is most affected by the language environment (i.e., strength of Spanish-influence in neighborhood), whereas the production of adults is more affected by personal relationships and connections maintained in the present and the past. Despite the relative lack of research on the role of age and gender in vowel production in this bilingual community, but due to increasing immigration and the documented mid front vowel merger in progress in some areas in Barcelona \citep{mora2012}, I predict that a potential change in progress, if observed, would be led by younger female speakers and be advancing in the direction of vowel assimilation.

\subsection{The present study}

The present study uses cross-linguistic production data to address the following research questions relating social factors to processes of assimilation and dissimilation. First, how do the factors of gender, age, and language dominance mediate a cognate effect from Catalan in the production of Spanish /e/, demonstrating assimilation? I hypothesize that a cognate effect will occur less among less Catalan-dominant speakers, less in females than males, and less in younger speakers -- despite exposure to new Catalan educational policies -- as these speakers are less likely to have maintained the Catalan mid front vowel contrast, thereby inhibiting a cognate effect. Secondly, how do these social factors mediate the degree to which Spanish /e/ overlaps with each Catalan mid front vowel? I hypothesize that decreased Catalan-dominance will contribute to greater overlap (assimilation), and that female speakers and younger speakers will additionally produce categories with less acoustic distinction.

\section{Methodology}

\subsection{Subject population}
    Seventeen participants were recruited with flyers posted at the University of Barcelona and were stratified according to age and gender. Two generations are represented, one group between 18--25 years old and the other between 40--65 years old. All participants are bilingual in Spanish and Catalan and have lived in Barcelona for the past 10 years. All participants were connected to the University, or had been connected in the past, which may yield a similar exposure to Catalan in a professional setting across participants.
    \begin{table}
        \centering
        \begin{tabular}{lrr}
        \lsptoprule
        Participant Group & Count & Language Dominance\\
        \midrule
        Younger Women & 6 & +55.7 (17.2)\\
        Older Women & 4 & +50.8 (77.5)\\
        Younger Men & 4 & +1.0 (47.0)\\
        Older Men & 3 & +93.9 (10.4)\\
        \lspbottomrule
        \end{tabular}
        \caption{Number of participants and mean language dominance scores (with standard deviations) across the four social cells}
        \label{tab:participants}
    \end{table}

    Each participant was assigned a Catalan-Spanish dominance score (minimum: $-$218; maximum: +218) after completing the Bilingual Language Profile \citep{birdsong2012bilingual}, where a more positive dominance score is correlated with greater dominance in Catalan and a more negative dominance score is correlated with greater dominance in Spanish. \tabref{tab:participants} shows the distribution of participants across the four participant groups, as well as the mean language dominance score for each group. The relatively low number of speakers that are more Spanish-dominant (n = 4) and the imbalance of their distribution across the four social cells prevent the use of a categorical dominance score without crossing language dominance with other social factors. Instead, dominance will remain a continuous factor in the present analysis and will not interact with either gender or age. Although the sample size is relatively small, 3--5 participants per cell is the statistical minimum to reflect group tendencies more than individual idiosyncrasies \citep[][31]{tagliamonte2006analysing}. At the time of data collection, no participant reported ever having any history of speech or hearing disorders.

\subsection{Materials and procedure}
\largerpage
	All experimental sessions took place in an empty classroom at the University of Barcelona. Participants were asked if they would prefer to interact with the researcher in Spanish or in Catalan. All communication, including the consent form, questionnaire, and  sociolinguistic interview, were subsequently conducted in the preferred language. First, the participants were instructed to read and sign the consent form and complete the Bilingual Language Profile \citep{birdsong2012bilingual}, adapted as a Qualtrics survey \citep{qualtrics2005qualtrics}. Next, the participants engaged in a sociolinguistic interview (data not analyzed in the present study), followed by two elicited production tasks, the first in Spanish and the second in Catalan. The productions of these token stimuli were recorded using a Zoom H4N Multitrack Recorder and Comica Lavalier microphone.

	The Spanish word list\footnote{Spanish and Catalan word lists available at \url{https://anniehelms.github.io/lsrl50_supplemental/}} used in the elicited production task was stratified according to cognate status: 20 words have congruent Catalan cognates (e.g., Sp. conc[e]pto, Cat. conc[e]pte `concept'); 20 words have incongruent Catalan cognates (e.g., Sp. inter[e]s, Cat. inter[\textepsilon]s `interest'); and 20 words have no Catalan cognate (e.g., Sp. mad[e]ra, Cat. fusta `wood'). In order to determine the target vowel for each Catalan cognate, an online dictionary with transcriptions of the Barcelona variety of Catalan was consulted \citep{alcover2002diccionari}. The Catalan word list consisted only of the 40 congruent and incongruent Catalan cognates. According to the online corpus NIM \citep{guasch2013nim}, all words from the Spanish word list have a relative frequency of at least 10 parts per million (ppm), and all words from the Catalan word list have a relative frequency of at least 5 ppm. In each word list, all target vowels occur in stressed syllables. Additionally, Spanish words where /e/ is followed by a palatal consonant, or either an /x/ or an /r/, were excluded, as these segments either lower or raise the F1 of /e/ \citep[][115]{hualde2013sonidos}.
	Before data collection began in Barcelona, four trained linguists who are native speakers of Catalan and/or Spanish participated in a pilot study. After the experiment, none of the participants were able to identify the sound of interest, so to reduce the duration of the experiment, neither word list included filler tokens. Each word list was randomized and all participants saw the same list orders appear on a tablet in the form of isolated words.

\subsection{Acoustic analysis}
\largerpage
    A total of 1,020 Spanish mid front vowels (17 participants x 20 words x 3 cognate levels) and a total of 680 Catalan mid front vowels (17 participants x 20 words x 2 target vowels) were submitted to acoustic analysis. For the Spanish data, time-aligned, word- and phoneme-segmented Praat TextGrid files were generated using Montreal Forced Aligner \citep{mcauliffe2017montreal} with a Spanish dictionary \citep{morgan2017dict}. The TextGrids were hand-corrected in Praat \citep{boersma6praat}, and a Praat script \citep{riebold2013vowel} was used to extract measurements for F1, F2, and F3 at the midpoint of each stressed /e/ phone marked in the TextGrid in order to minimize co-articulation effects upon the formant measurements. The same procedure was carried out for the 680 Catalan mid front vowels, and vowel categories were classified following the target vowels in the word list. The F1 and F2 measurements for all mid front vowels of Spanish and Catalan were normalized across vocal tract length, using the Lammert and Narayanan $\Delta$F normalization method \citep{johnson2020deltaf}, which can be calculated using only a subset of vowels from the acoustic space.

\subsection{Statistical analysis}

    To analyze a possible mid front vowel merger in Catalan, a Pillai score was calculated for the two mid front Catalan vowels for each speaker in the data set and measures of F1 were submitted to a mixed effects linear regression model. Although normally both Pillai scores and measures of Euclidean distance are employed to analyze possible mergers, the two Catalan mid front vowels predominantly differ across F1, so the regression model of F1 measures provides roughly the same information as Euclidean distance. The Pillai score is a measure of the degree of overlap between vowel categories and is calculated for each speaker from multivariate analysis of variance (\textsc{manova}) models fitted with F1 and F2 measurements by vowel category \citep{nycz2013best}. The Pillai scores from each speaker were calculated using a custom function and submitted to a fixed effects linear regression model using the \texttt{glm()} function in R \citep{R}. The model includes a main effect of \textsc{language dominance} and a two-way interaction term of \textsc{age} and \textsc{gender}. The mixed effects linear regression model predicting F1 was built using the \texttt{lmerTest} package \citep{lmertest}. This model serves as another indicator of a possible mid front vowel merger, and additionally provides information about variation occurring across this formant axis. The model contains a two-way interaction of \textsc{language dominance} and \textsc{target catalan vowel}, a three-way interaction of \textsc{gender}, \textsc{age}, and \textsc{target catalan vowel}, and random intercepts of \textsc{participant} and \textsc{token word}.

    In order to observe a possible cognate effect within productions of Spanish /e/, F1 measures were submitted to a mixed effects linear regression model. The model included a two-way interaction of \textsc{language dominance} and \textsc{catalan cognate vowel}, a three-way interaction between \textsc{gender}, \textsc{age}, and \textsc{catalan cognate vowel}, and random intercepts of \textsc{token word} and \textsc{participant}. Additionally, to observe the impact of social factors on the degree of overlap between Spanish /e/ and Catalan /e/, and between Spanish /e/ and Catalan /\textepsilon/, Pillai scores were calculated for each participant for each vowel category comparison. Scores were submitted to separate fixed effects linear regression models with the two-way interaction between \textsc{age} and \textsc{gender} and the main effect of \textsc{language dominance}. For these regression models, and all previous models, the \texttt{emmeans} package \citep{emmeans} was used to calculate Cohen's \emph{d} effect sizes for pairwise comparisons and to perform necessary post-hoc tests using a Tukey pairwise comparison. The \texttt{heplots} package \citep{heplots} was used to calculate partial eta-squared ($\eta_{p}^{2}$) effect sizes for fixed effects models, and the \texttt{r2glmm} package \citep{r2glmm} was used to calculate marginal R-squared ($R^{2}$) effect sizes for mixed effects models.

\section{Results}
\subsection{Catalan production}

   \begin{table}
        \centering
        \begin{tabular}{lrrrrl}
        \lsptoprule
         & Estimate & Std. Error & \emph{t}-value & \emph{p}-value &\\
        \midrule
        (Intercept) & 0.3669835 & 0.0822150 &  4.464& 0.000774 &\textasteriskcentered\textasteriskcentered\textasteriskcentered \\
        Younger & $-$0.1727733 & 0.0900524 & $-$1.919 & 0.079134 &\textasteriskcentered\\
        Male & $-$0.1587978 & 0.1126758 & $-$1.409& 0.184121 & \\
        Dom. & 0.0003611 & 0.0008584 &  0.421 &0.681425 &\\
        Younger: Male & 0.2068923 & 0.1626608 &  1.272& 0.227497& \\
        \lspbottomrule
        \end{tabular}
        \caption{Regression coefficients for fixed effects linear model predicting overlap between Catalan /e/ and Catalan /\textepsilon/ (Pillai scores) across the two-way interaction of \textsc{age} and \textsc{gender} and the main effect of \textsc{language dominance} (Dom.). The intercept is the overlap of older female speakers with a language dominance score of 0.}
        \label{tab:pillai_cat}
    \end{table}
    To look for evidence of assimilation via a cognate effect, it must first be determined if /e/ and /\textepsilon/ are produced as distinct vowel categories. An initial visual examination of the Catalan mid front vowels (\figref{fig:cat_plot}) suggests that older females produce more contrasting vowels than other participants. To investigate this observation further, Pillai scores and productions along F1 were analyzed. The coefficients of the regression model of individuals' Pillai scores (\tabref{tab:pillai_cat}) indicate that neither \textsc{age}, \textsc{gender}, or \textsc{language dominance} significantly impact the degree of overlap between the two mid front vowel categories. However, the main effect of \textsc{age} is approaching significance, where younger speakers produce Catalan mid front vowels with greater overlap than older speakers ($\beta$ = $-$0.17, $\eta_{p}^{2}$ = 0.565, \emph{p} = 0.079).
    \begin{figure}
        \centering
        \includegraphics[width=\textwidth]{figures/helms_fig2.png}
        \caption{Vowel space plot showing distribution of three vowel categories in acoustic space, and displayed across social factors of age and gender. Ellipses are drawn 1 SD from the mean. Formant values appear in normalized units, derived originally from raw hertz values.}
        \label{fig:cat_plot}
    \end{figure}

    Regression coefficients for the mixed effect linear regression model predicting Catalan F1 (\tabref{tab:cat_f1}) indicate significant main effects of \textsc{language dominance}, \textsc{target catalan vowel}, \textsc{gender}, and a significant interaction of \textsc{target catalan vowel} and \textsc{age}. A post-hoc Tukey pairwise comparison of \textsc{target catalan vowel} and \textsc{age} reveals that /e/ is produced significantly higher than /\textepsilon/ for older speakers ($\beta$ = 0.0771, \emph{d} = 1.474, \emph{p} $<$ 0.001) and for younger speakers ($\beta$ = 0.0433, \emph{d} = 0.828, \emph{p} $<$ 0.05). Additionally, while the acoustic distinction between the /\textepsilon/ of older speakers and the /\textepsilon/ of younger speakers is not significant, older speakers produce /e/ higher than younger speakers ($\beta$ = 0.0885, \emph{d} = 1.691, \emph{p} $<$ 0.01). Main effects of \textsc{language dominance} ($\beta$ = 0.00059, $R^{2}$ = 0.047, \emph{p} $<$ 0.05) and \textsc{gender} ($\beta$ = $-$0.095, $R^{2}$ = 0.068, \emph{p} $<$ 0.01) indicate that mid front vowels are produced lower with increasing Catalan-dominance, and that male productions are higher than female productions. Although these social factors influence Catalan production, they do not impact the degree of acoustic distinction between the two mid front vowel categories.
    \begin{table}
        \begin{tabular}{lrrrrl}
        \lsptoprule
             & Estimate & Std. Error & \emph{t}-value & \emph{p}-value &\\
        \midrule
        (Intercept) & 5.092e-01 & 2.220e-02 & 22.934 &  < 2e-16 & \textasteriskcentered\textasteriskcentered\textasteriskcentered \\
        Dom. & 5.939e-04 & 2.098e-04 &  2.830 & 0.010688 & \textasteriskcentered\\
        /e/ & $-$9.065e-02 & 1.654e-02 &  $-$5.480 & 5.06e-07 & \textasteriskcentered\textasteriskcentered\textasteriskcentered\\
        Male & $-$9.454e-02&  2.754e-02 & $-$3.433& 0.002790 &\textasteriskcentered\textasteriskcentered\\
        Younger & 4.266e-02&  2.201e-02&   1.938& 0.067620 & \textasteriskcentered\\
        Dom.: /e/ &2.003e-05 & 1.019e-04&   0.197& 0.844256&\\
        Male: Younger & 2.405e-02 & 3.976e-02  & 0.605 & 0.552480&\\
        /e/: Male & 2.514e-02 & 1.338e-02&  1.879& 0.060672 & \textasteriskcentered\\
        /e/: Younger &3.895e-02 & 1.069e-02&   3.643& 0.000292& \textasteriskcentered\textasteriskcentered\textasteriskcentered \\
        /e/: Male: Younger & $-$1.028e-02 & 1.931e-02  & $-$0.532& 0.594572&\\
        \lspbottomrule
        \end{tabular}
        \caption{Regression coefficients for mixed effects linear model predicting Catalan F1, with a three-way interaction of \textsc{age}, \textsc{gender}, and \textsc{target vowel} and a two-way interaction of \textsc{language dominance} (Dom.) and \textsc{target vowel}. The intercept is older, female speakers producing Catalan /\textepsilon/ with a language dominance score of 0.}
        \label{tab:cat_f1}
    \end{table}


\subsection{Spanish production}
    From the Spanish word lists, F1 measures of Spanish /e/ were submitted to a mixed effects linear regression model and the output of the model appears in  \tabref{tab:span_f1}. The model output indicates a significant main effect of \textsc{language dominance} ($\beta$ = 0.00052, $R^{2}$ = 0.040, \emph{p} $<$ 0.01), where speakers that are more Catalan-dominant produce Spanish /e/ lower than speakers that are less Catalan-dom\-i\-nant. The model also reveals a significant main effect of \textsc{age} ($\beta$ = 0.049, $R^{2}$ = 0.033, \emph{p} $<$ 0.05), where younger speakers have lower productions than older speakers.  Additionally, the main effect of \textsc{gender} is approaching significance, where males produce /e/ higher than females ($\beta$ = 0.047, $R^{2}$ = 0.020, \emph{p} = 0.0524). Importantly, there is no significant interaction containing levels of the factor \textsc{catalan cognate vowel}, which could indicate a cognate effect from Catalan in the production of Spanish /e/. Accordingly, it seems that these participants do not evidence assimilation via a cognate effect to Catalan mid front vowel categories in their production of Spanish /e/.
    \begin{table}
        \begin{tabular}{lrrrrl}
        \lsptoprule
             & Estimate & Std. Error & \emph{t}-value & \emph{p}-value &\\
        \midrule
        (Intercept) & 4.097e-01 & 1.716e-02&  23.872 &  <2e-16 &\textasteriskcentered\textasteriskcentered\textasteriskcentered \\
        Dom. & 5.186e-04 & 1.737e-04&  2.986 &  0.0071 & \textasteriskcentered\textasteriskcentered\\
        /e/ & $-$2.370e-03 & 1.073e-02& $-$0.221 &  0.8253& \\
        NC & $-$5.334e-03 & 1.073e-02 & $-$0.497 &  0.6194&\\
        Male & $-$4.690e-02 & 2.280e-02&  $-$2.057 &  0.0524 & \cdot\\
        Younger & 4.903e-02 & 1.822e-02&  2.691 &  0.0138& \textasteriskcentered\\
        Dom.: /e/ & $-$8.966e-05 & 9.308e-05&  $-$0.963&   0.3357 & \\
        Dom.: NC&9.668e-06 & 9.308e-05&  0.104&   0.9173 &\\
        Male: Younger &3.510e-02 & 3.291e-02&   1.067&   0.2984 &\\
        /e/: Male& $-$2.240e-03 & 1.222e-02& $-$0.183&   0.8546&\\
        NC: Male & 3.753e-03 & 1.222e-02&  0.307 &  0.7588& \\
        /e/: Younger & 1.125e-02 & 9.765e-03&   1.152&   0.2497 &\\
        NC: Younger & 4.063e-03 & 9.765e-03&   0.416 &  0.6774&\\
        /e/: Male: Younger & 7.526e-03 & 1.764e-02  &0.427 &  0.6697 &\\
        NC: Male: Younger & 1.032e-02 & 1.764e-02  & 0.585 &  0.5585 &\\
        \lspbottomrule
        \end{tabular}
        \caption{Regression coefficients for mixed effects linear model predicting Spanish F1 with a three-way interaction of \textsc{age}, \textsc{gender}, and \textsc{catalan cognate vowel} (NC = non-cognate) and a two-way interaction of \textsc{language dominance} (Dom.) and \textsc{catalan cognate vowel}. The intercept is older, female speakers with a language dominance score of 0 producing Spanish /e/ where the Catalan cognate vowel is /\textepsilon/.}
        \label{tab:span_f1}
    \end{table}

\subsection{Cross-linguistic production}
\largerpage
    To observe how social factors mediate the degree of overlap between Spanish /e/ and each Catalan mid front vowel, Pillai scores measuring the degree of overlap between Spanish /e/ and each Catalan mid front vowel were calculated for each participant. The values were submitted to fixed effects linear regression models with main effects of \textsc{language dominance} and two-way interactions of \textsc{age} and \textsc{gender}. The regression coefficients for the overlap between Spanish /e/ and Catalan /e/ appear in  \tabref{tab:pillai_e}. The model output indicates that social factors do not impact the degree of overlap of these two categories. In other words, all participants regardless of age, gender, or language dominance, demonstrate a similar degree of overlap of Spanish /e/ and Catalan /e/. As a relative measure, Pillai scores do not convey a specific percentage of overlap in productions. However, as scores can range from 0 (merged) to 1 (unmerged), the fairly low $\beta$-coefficient of the intercept (0.156) suggests that the categories have a considerable degree of overlap.\newpage

    \begin{table}
        \begin{tabular}{l ccccl}
        \lsptoprule
             & Estimate & Std. Error & \emph{t}-value & \emph{p}-value &\\
        \midrule
        (Intercept) & 0.1557052 & 0.0552384  & 2.819 &  0.0155& \textasteriskcentered \\
        Younger & 0.0274665 & 0.0605042 &  0.454 &  0.6580&\\
        Male & $-$0.0125400 & 0.0757043 & $-$0.166 &  0.8712& \\
        Dom.& 0.0001259 & 0.0005767 &  0.218 &  0.8309& \\
        Younger: Male & $-$0.1134477 & 0.1092881 & $-$1.038 &  0.3197&\\
        \lspbottomrule
        \end{tabular}
        \caption{Regression coefficients for the fixed linear effects model predicting the degree of overlap between Spanish /e/ and Catalan /e/ (Pillai scores) across the two-way interaction of \textsc{age} and \textsc{gender} and the main effect of \textsc{language dominance} (Dom.). The intercept is the overlap of productions by older, female speakers with a language dominance score of 0.}
        \label{tab:pillai_e}
    \end{table}

    \begin{table}
        \begin{tabular}{l rrrrl}
        \lsptoprule
             & Estimate & Std. Error & \emph{t}-value & \emph{p}-value &\\
        \midrule
        (Intercept) & 0.5244873 & 0.0604239 &  8.680& 1.62e-06& \textasteriskcentered\textasteriskcentered\textasteriskcentered \\
        Younger & $-$0.1223421&  0.0661840&  $-$1.849  & 0.0893&\textasteriskcentered\\
        Male & $-$0.2384856 & 0.0828110&  $-$2.880 &  0.0138& \textasteriskcentered\\
        Dom. & 0.0012737 & 0.0006309 &  2.019 &  0.0664& \textasteriskcentered\\
        Younger: Male & 0.0144009 & 0.1195474 &  0.120&   0.9061&\\
        \lspbottomrule
        \end{tabular}
        \caption{Regression coefficients for the fixed linear effects model predicting the degree of overlap between Spanish /e/ and Catalan /\textepsilon/ (Pillai scores) across the two-way interaction of \textsc{age} and \textsc{gender} and the main effect of \textsc{language dominance} (Dom.). The intercept is the overlap of productions by older, female speakers with a language dominance of 0.}
        \label{tab:pillai_eh}
    \end{table}

    \largerpage
    The regression coefficients for the model predicting the overlap of Spanish /e/ and Catalan /\textepsilon/ are shown in Table \ref{tab:pillai_eh}. In this model output, the main effect of \textsc{gender} is significant ($\beta$ = 0.238, $R^{2}$ = 0.99, \emph{p} $<$ 0.05), indicating that male speakers produce the two categories with more overlap, relative to female speakers. Additionally, the factors of \textsc{age} and \textsc{language dominance} are approaching significance, where younger speakers would evidence more overlap than older speakers ($\beta$ = 0.122, $R^{2}$ = 0.99, \emph{p} = 0.0893) and more Catalan-dominant speakers would evidence less overlap ($\beta$ = 0.0013, $R^{2}$ = 0.99, \emph{p} = 0.0664). The $\beta$-coefficient of the intercept, compared with that of the previously analyzed model, suggests that these participants have greater overlap of Spanish /e/ and Catalan /e/, than of Spanish /e/ and Catalan /\textepsilon/, an observation that is supported visually in \figref{fig:cat_plot}.

\section{Discussion}
In order to address the first research question, whether there is a cognate effect from Catalan in the production of Spanish /e/, the Catalan data were first analyzed. The results of this analysis indicate that the categories of Catalan /e/ and Catalan /\textepsilon/ have not yet fully merged for these participants, as each category was produced significantly differently across F1 by both older speakers and younger speakers. However, the results of the regression model of individual Pillai scores indicate that the degree of category overlap may be increasing in apparent-time. Since significant acoustic distinction was observed, it could be possible to find cognate effects in the Spanish production data. However, the factor of \textsc{catalan cognate vowel} was not a significant main effect of F1 of Spanish /e/, nor was it involved in any significant interactions, suggesting that there is no observable cognate effect, regardless of the social profile of the participants analyzed in this study.

The second research question regards the variation due to social factors in the degree of overlap between Spanish /e/ and Catalan /e/, and also between Spanish /e/ and Catalan /\textepsilon/. While the overlap between Spanish /e/ and Catalan /e/ was not seen to be socially-mediated, \textsc{gender} was a significant predictor of the degree of overlap between Spanish /e/ and Catalan /\textepsilon/, where the male participants tend to evidence more overlap between the two categories than the female participants. Contrary to the research hypothesis, neither \textsc{age} nor \textsc{language dominance} significantly impacted the degree of overlap, though both factors approached statistical significance.

Regarding the role of age, older speakers maintained less overlap of Catalan mid front categories and greater acoustic distinction across F1, thus demonstrating adherence to prescriptive norms, where Catalan /e/ and Catalan /\textepsilon/ are two distinct phones. This result is in line with the research hypothesis, despite the educational changes that occurred in Catalonia following the Catalonian Linguistic Normalization law of 1983. Though exposure to Catalan instruction in school may be greater for younger speakers, the large number of immigrants that are L2 Catalan speakers attending public schools alongside native Catalan speakers may contribute to increased exposure to exemplars with a merger. Therefore, exposure to Catalan under the new educational policies does not deterministically cause younger speakers to fully adhere to prescriptive norms. These findings therefore support the SLM-r's assertion that ``[q]uality of input has been largely ignored in L2 speech research [in favor of quantity] even though it may well determine the extent to which L2 learners differ from native speakers" \citep[][32]{flege2021revised}.

With regard to the role of gender, the female participants had less overlap between Spanish /e/ and Catalan /\textepsilon/ than the male participants, thus conforming more to the prescriptive norm than males. That gender did not also impact the overlap of Spanish /e/ and Catalan /e/ could be attributed to the lack of salience of this distinction; one participant of this study mentioned during the sociolinguistic interview that students are taught in school that Spanish /e/ and Catalan /e/ are the same sound. Though the factor of age was only marginally significant, a visual examination of Figure \ref{fig:cat_plot} suggests that the younger speakers tend to conflate all three vowel categories whereas the older speakers mainly conflate Spanish /e/ and Catalan /e/. With greater statistical power, perhaps age would surface as a significant predictor of overlap, in which case the data would be consistent with a male-led change in progress. Alternatively, the lack of a significant age effect, coupled with female speakers' greater adherence to the prescriptive norm than male speakers, is consistent with a case of stable variation. However, since prior studies \citep[e.g.,][]{mora2012} have documented the merger as recent and ongoing, I will proceed considering the merger as a possible change in progress.

\hspace*{-1.7pt}Whereas females are often the leaders of community-wide change \citep{labov2001principles}, a male-led change could suggest that the social meaning indexed by a production of Spanish /e/ that is conflated with Catalan mid front vowels is stratified by gender. For example, it could be that this variant is an index of solidarity or a Catalan-identity marker \citep[akin to lateral velarization and other phonetic phenomena;][]{davidson2019covert} that is generally associated with males and does not provide social gain if used by females \citep{chappell2016social}. Similarly, because bilinguals that use this variant would only have one mid front vowel category cross-linguistically, the variant could also index some trait associated with metropolitan bilingualism or a unique Barcelona identity that is a blend of Spanish and Catalan identities \citep{newman2015language}. Of course, these presently speculative accounts can be empirically tested in future perception research that aims to uncover the social meanings that listeners afford to the assimilation of Spanish /e/ and Catalan /\textepsilon/.

Models of second language acquisition and representation, such as the SLM, SLM-r, PAM-L2, and L2LP, predict language dominance to be an important factor in category formation. Although language dominance did not surface as a significant predictor of production in this study, the participant group was considerably homogeneous in terms of dominance, where only 4 participants were scored as Spanish-dominant by the BLP. A larger data set with more variability in language dominance could be more revealing of the importance of input in language production. Of the three models, the SLM-r should make the most relevant predictions for the data as the participants are early bilinguals rather than na{\"i}ve listeners or adult learners and the bias towards Catalan-dominance in the data set means that the majority of participants are attempting to produce a single L2 category (Spanish /e/) rather than an L2 category contrast (Catalan /e/ and /\textepsilon/). Under this theory, the quality and quantity of input that participants receive of Catalan /e/ does not allow for sufficient differentiation from Spanish /e/, yielding category assimilation in their production as was seen in the Pillai scores.

Based on SLM-r, any differences in production based on gender, ethnicity, age, or other social factors must necessarily be derived from differences in L2 input. However, the connection between variants and social meaning central to variationist sociolinguistic theory is not accounted for in the SLM-r. \citet{chappell2021learners} attempt to reconcile this disconnect by proposing a unified framework of theoretical models of second language learning, exemplar theory, and indexical meaning to explain variable outcomes in second language perception. The present data also support this unified approach for production, where variable productions may be influenced by the mapping of social meaning onto said variants, in addition to the L2 input received. While cognitive factors such as language dominance and cognate status are central to models of bilingual phonetic representation, it is paramount to situate the bilingual individual within the context of the community and acknowledge the external social factors which also mediate variation in acquisition and production.

\section{Conclusions}
This study found evidence for age and gender as predictors of assimilation among productions of Spanish and Catalan mid front vowels. Younger speakers evidence less acoustic distinction between Catalan mid front vowels than older speakers, and males produce Spanish /e/ and Catalan /\textepsilon/ with more overlap than females. These findings contribute a variationist sociolinguistic analysis to the literature of bilingual production of mid front vowels, demonstrating the importance of viewing models of category acquisition and phonetic representation through the lens of social factors, in conjunction with cognitive factors. Future studies of contact varieties of Spanish found in bilingual settings within Barcelona, both in the realms of production and perception, will further reveal the impact of contemporary Catalan linguistic policies, and the social meaning indexed by variation within the production of these (vocalic) and other linguistic variables.

\section*{Acknowledgements}
    I would like to thank Justin Davidson, Ernesto Guti{\'e}rrez Topete, Ana Bel{\'e}n Redondo Campillos, Bernat Bardagil i M{\'a}s, Yamel Nu{\~n}ez, Antonio Torres Torres, the audience members at LSRL50, and two anonymous reviewers for their contributions and insightful feedback. All remaining errors are my own.

\printbibliography[heading=subbibliography,notkeyword=this]

\end{document}
