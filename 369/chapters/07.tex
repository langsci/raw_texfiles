\documentclass[output=paper,colorlinks,citecolor=brown,draft,draftmode]{langscibook}
\ChapterDOI{10.5281/zenodo.7525104}
\author{Kryzzya Gómez\orcid{}\affiliation{LLING-Université de Nantes /CNRS} and Maia Duguine\orcid{}\affiliation{IKER-CNRS} and  Hamida Demirdache\orcid{}\affiliation{LLING-Université de Nantes /CNRS}}
\title{Overt vs. null subjects in infinitival constructions in Colombian Spanish}

\abstract{Standard approaches predict complementary distribution between referentially free (overt/null) subjects and referentially dependent PRO-type null subjects. This generalization is challenged by Colombian Spanish non-finite adjuncts, which allow both overt subjects and referentially free null subjects. We uncover an intricate pattern of distribution and interpretation along two criteria: Obligatory \textit{vs}. Non-Obligatory Control (OC \textit{vs.} NOC) and whether the controllee is silent or an oblig\-a\-to\-ry-controlled overt pronoun (covert \textit{vs}. so-called ``overt PRO"). The distribution of sloppy readings with null subjects provides arguments for analyzing NOC as DP-ellipsis. We also show that both covert and overt PRO display the canonical diagnostics of OC except in one context. While they both only allow bound variable construals under ellipsis, overt PRO also allows co-reference when its controller is associated with focus. This paradox follows on the assumption that while both null and overt anaphors must be syntactically bound, only null anaphors are necessarily semantically bound.}


 \IfFileExists{../localcommands.tex}{
  \addbibresource{../localbibliography.bib}
  \usepackage{langsci-optional}
\usepackage{langsci-gb4e}
\usepackage{langsci-lgr}

\usepackage{listings}
\lstset{basicstyle=\ttfamily,tabsize=2,breaklines=true}

%added by author
% \usepackage{tipa}
\usepackage{multirow}
\graphicspath{{figures/}}
\usepackage{langsci-branding}

  
\newcommand{\sent}{\enumsentence}
\newcommand{\sents}{\eenumsentence}
\let\citeasnoun\citet

\renewcommand{\lsCoverTitleFont}[1]{\sffamily\addfontfeatures{Scale=MatchUppercase}\fontsize{44pt}{16mm}\selectfont #1}
  
  %% hyphenation points for line breaks
%% Normally, automatic hyphenation in LaTeX is very good
%% If a word is mis-hyphenated, add it to this file
%%
%% add information to TeX file before \begin{document} with:
%% %% hyphenation points for line breaks
%% Normally, automatic hyphenation in LaTeX is very good
%% If a word is mis-hyphenated, add it to this file
%%
%% add information to TeX file before \begin{document} with:
%% %% hyphenation points for line breaks
%% Normally, automatic hyphenation in LaTeX is very good
%% If a word is mis-hyphenated, add it to this file
%%
%% add information to TeX file before \begin{document} with:
%% \include{localhyphenation}
\hyphenation{
affri-ca-te
affri-ca-tes
an-no-tated
com-ple-ments
com-po-si-tio-na-li-ty
non-com-po-si-tio-na-li-ty
Gon-zá-lez
out-side
Ri-chárd
se-man-tics
STREU-SLE
Tie-de-mann
}
\hyphenation{
affri-ca-te
affri-ca-tes
an-no-tated
com-ple-ments
com-po-si-tio-na-li-ty
non-com-po-si-tio-na-li-ty
Gon-zá-lez
out-side
Ri-chárd
se-man-tics
STREU-SLE
Tie-de-mann
}
\hyphenation{
affri-ca-te
affri-ca-tes
an-no-tated
com-ple-ments
com-po-si-tio-na-li-ty
non-com-po-si-tio-na-li-ty
Gon-zá-lez
out-side
Ri-chárd
se-man-tics
STREU-SLE
Tie-de-mann
}
  % \togglepaper[3]%%chapternumber
}{}





\begin{document}
\maketitle

\section{Introduction}
According to the standard theory, overt nominative subjects and referentially null subjects of the \textit{pro}-type are licensed by finite INFL/T.
As such, they are excluded from non-finite clauses, and thus not expected to alternate with PRO in this position \citep{williams80, chomsky81, rizzi86, lasnik&uriagereka88, miller02}.
This set of standard assumptions is stated in (\ref{07:StandGen}), adapted from \citet{szabolcsi09}. See also \citet{rigau95, mensching00, barbosa19, livitz11, corbalan18}, a.o. and references therein for extensive discussion.

\ea \label{07:StandGen} Infinitival clauses do not allow:
\begin{enumerate}[(i)]
 \item \label{07:StandGen1} Overt (nominative) subjects.
 \item \label{07:StandGen2} Referentially free null subjects (\textit{pro}).
 \item \label{07:StandGen3} Overt controlled subjects.\footnote{We borrow this terminology from \citet{szabolcsi09} and \citet{livitz11} respectively.}
 \end{enumerate}
\z

However, as has been reported in the literature, the predictions of the standard theory in (\ref{07:StandGen}) are empirically falsified. In particular, overt subjects alternating with controlled null (PRO-type) subjects have been shown to be allowed in many languages. The pioneering work by \citet{piera87} and \citet{lipski94} for instance shows that complement and adjunct infinitives allow overt subjects in different varieties of Spanish (\ref{07:sp2})--(\ref{07:sp3}):\footnote{For discussion of this pattern in Spanish, see also \citet{rigau95, torrego98nominative, mensching00, zagona02, pereztattam07, schulte07, herbeck11, corbalan18, gonzalez20}. See also \citet{szabolcsi09} on Hungarian, Italian, European Spanish, Portuguese, Romanian and Modern Hebrew, \citet{sundaresan&mcfadden09} on Tamil, \citet{borer89} on Korean and \citet{duguine13} on Basque.}


% More generally, this approach predicts the complementary distribution between null PRO and overt DPs in non-finite vs. finite constructions, which has been challenged in the literature on the basis of languages where overt subjects occur within infinitive clauses.

%\renewcommand{\eachwordone}{\itshape}
\begin{exe}
\ex \label{07:sp2}
\langinfo{(European) Spanish}{}{\citealt[160]{piera87}} \\
\gll Julia quería [telefonear ella]. \\
Julia wanted phone.\textsc{inf} she.\textsc{nom} \\
\glt ‘Julia wanted to be her who phones.’
%\end{exe}

%%\renewcommand{\eachwordone}{\itshape}
%\begin{exe}
\ex \label{07:sp3}
\langinfo{(Colombian) Spanish}{}{\citealt[215]{lipski94}} \\
\gll Antes de yo salir de mi país. \\
before of I.\textsc{nom} exit.\textsc{inf} of my country \\
\glt ‘Before I left my country.’
\end{exe}

Even more strikingly, although it is theoretically impossible for an infinitival subject that exhibits all the diagnostics of Obligatory Control (OC) to be overt, overt obligatory-controlled pronominal subjects are attested cross-linguistically. Following \citet{livitz11}, we henceforth refer to the latter as ``overt PRO'' (see also \citealt{szabolcsi09}).

The novel contribution of this paper is twofold. First, it extends the empirical domain of overt PRO discussed in the literature solely in the context of complement clauses to adjunct clauses. Second, it applies a battery of diagnostic tests to establish whether overt pronominals in infinitival subject positions instantiate (or not) OC (in line with \citealt{duguine13}).

The focus of this paper is on non-finite adjunct clauses which allow overt preverbal subjects. We show that overt and null subjects in infinitival adjunct clauses in (Andean) Colombian Spanish (henceforth CS) exemplify three systematic patterns of exception to the standard generalizations in (\ref{07:StandGen}).\footnote{CS infinitival adjuncts allow overt subjects in preverbal position, a pattern commonly found across Caribbean varieties of Spanish \citep{suner86, dauphinais&ortiz-lopez16, gonzalez20}. In Contemporary European varieties, overt subjects are typically restricted to the postverbal position \citep{RAEandASALE09}. In Old Spanish, however, they were accepted in preverbal position \citep{mensching00, corbalan18}, while in other colloquial varieties, there appears to be no preference for the pre/postverbal position \citep{mensching00, gallego10, herbeck11}. Scholars moreover agree that the overt subject appearing in nonfinite contexts in Spanish  bears nominative case regardless of its pre/postverbal position \citep{piera87, schütze97, mensching00}.}

The first pattern can be observed in infinitives introduced by the temporal-causal preposition \textit{al}. It allows both overt DPs (pronouns or R-expressions) and null subjects of the PRO-type, which as we shall establish in \sectref{07:al-infinitives}, meet all the diagnostics of OC.\footnote{See \citet{rico16} on the temporal-causal properties of the complementizer \textit{al}.} This pattern is illustrated in (\ref{07:sp4}). Overt subjects (also) allowed in \textit{al}-infinitives violate the ban on overt subjects in (\ref{07:StandGen}\ref{07:StandGen1}).

%\renewcommand{\eachwordone}{\itshape}
\begin{exe}
\ex \label{07:sp4}
\gll Juan$_i$ sería feliz [al José$_k$/ él$_{i/k}$/ PRO$_{i/*k}$ dejar la casa]. \\
Juan be.\textsc{cond} happy in.the José he.\textsc{nom} {} leave.\textsc{inf} the house \\
\glt ‘Juan would be happy on leaving the house.’
\end{exe}


The second pattern is instantiated in infinitive adjunct clauses headed by the preposition \textit{sin} ‘without’. The latter allow for both overt subjects (pronouns or R-expressions) and \textit{pro}-type subjects, as illustrated below.\footnote{Here we use PRO and \textit{pro} as descriptive terms to conventionally refer to referentially dependent vs. free null subjects.}

%\renewcommand{\eachwordone}{\itshape}
\begin{exe}
\ex \label{07:sp5}
\gll María$_i$ dejó de trabajar [sin Rosa$_k$/ ella$_{i/k}$/ pro$_{i/k}$ decir nada].\\
María stopped of work.\textsc{inf} without Rosa she.\textsc{nom} {} say.\textsc{inf} nothing \\
\glt ‘María stopped working without (Rosa/her) saying anything.’
\end{exe}

The reference of the null infinitival subject in (\ref{07:sp5}) is free and (\ref{07:sp5}), as such, violates the ban on referentially free null subjects (\ref{07:StandGen}\ref{07:StandGen2}), in addition to the ban on overt subjects (\ref{07:StandGen}\ref{07:StandGen1}). As discussed in \sectref{07:sin-infinitives}, we follow \citet{hornstein99} in assuming null subjects in Non-Obligatory Control (NOC) constructions to be \textit{pro} rather than PRO. Importantly, however, we provide an argument from the distribution of sloppy readings for generalizing \citeauthor{duguine13}’s (\citeyear{duguine13, duguine14}) DP-ellipsis analysis of \textit{pro} to NOC contexts.
%revoir cette idée
The last pattern is that of infinitive complements selected by the preposition \textit{para} ‘for’, illustrated below:

%\renewcommand{\eachwordone}{\itshape}
\begin{exe}
\ex \label{07:sp6}
\gll Juan$_i$ se fue [para *María$_k$/ él$_{i/*j}$/ PRO$_{i/*j}$ estar feliz].\\
Juan \textsc{cl.3} left for María he.\textsc{nom} {} be.\textsc{inf} happy \\
\glt ‘Juan left in order to be happy.’
\end{exe}

Besides the expected PRO-like null subject, \textit{para}-infinitives also allow overt subjects. But unlike in the two previous cases, the reference of the overt subject is not free -- that is, only pronominal subjects anaphorically dependent on a local c-commanding antecedent are allowed to be overt. \textit{Para}-infinitives thus violate the ban on overt controlled subjects (\ref{07:StandGen}\ref{07:StandGen3}), and as a consequence, the ban on overt subjects (\ref{07:StandGen}\ref{07:StandGen1}), too. Following \citet{livitz11}, we call the overt controlled subject pronoun in (\ref{07:sp6}) ``overt PRO''. Although overt PRO in complement clauses has drawn pointed attention in the literature \citep{piera87, mensching00, szabolcsi09, livitz11, livitz14}, this is to our knowledge the first discussion of overt PRO in adjunct clauses (though see \citealt{duguine13} for a first approach in Basque).

\hspace*{-4pt}The present paper explores the distribution and interpretation of subjects across these three patterns of infinitival adjunct clauses.
We provide novel arguments showing that null and overt PRO do not pattern alike according to the standard tests for pronominal interpretation, namely, the ellipsis and association-with-focus tests. As is well-known, while the relation holding between a pronoun and its antecedent comes out as either Bound Variable Anaphora (BVA) or coreference under these tests, the relation between null PRO and its antecedent/controller comes out as BVA exclusively, as illustrated by the paradigm in (\ref{07:e7}) from \citet[133--134]{fodor75}.\footnote{Here we present this contrast as it is observed in the association-with-focus test. \sectref{07:sectionPara} presents the same contrast in the VP-ellipsis test.}  The contrast in the interpretation of (\ref{07:e7a}) with a PRO subject in the gerund and (\ref{07:e7b}) with a possessive pronominal subject instead is fleshed out in (\ref{07:e8})--(\ref{07:e9}) adapted from \citet[135]{fodor75}. Note that gerundive complements are OC constructions \citep{hornstein99}.

\ea \label{07:e7}
\ea \label{07:e7a} Only Churchill remembers [PRO giving the speech about blood, toil, tears, and sweat].
\ex \label{07:e7b} Only Churchill remembers [his giving the speech about blood, toil, tears, and sweat].
\z

\ex \label{07:e8}
 Churchill$_i$ remembers PRO$_i$ giving the speech about blood, toil, tears, and sweat
 \ea \label{07:e8a} and no other ($\lambda$ x (x remembers x giving the speech)). \hfill \textit{BVA}

 \z
\z
\ea \label{07:e9}
Churchill$_i$ remembers his$_i$ giving the speech about blood, toil, tears, and sweat
 \ea \label{07:e9a} and no other ($\lambda$ x(x remembers x giving the speech)). \hfill \textit{BVA}
 \ex \label{07:e9b} and no other ($\lambda$ x(x remembers Churchill giving the speech))
 \\
 \hfill \textit{Coreference} (his=Churchill)\footnote{The notation here indicates that ‘his’ and ‘Churchill’ have the same referent.}
 \z
\z

Example (\ref{07:e8a}) can only be understood to mean that no individual other than Churchill remembers himself giving the speech. This is captured by assuming that the VP property attributed to Churchill (and asserted to hold of no other individual) is that of remembering \textit{oneself} giving the speech, where PRO is interpreted at LF as a variable bound by its antecedent (via $\lambda$-abstraction), that is, as semantically bound (cf. \citealt{HeimandKratzer98, buring05}).

Now, (\ref{07:e7b}) with the possessive pronoun allows the very same BV reading as in (\ref{07:e9a}), but can also be construed as in (\ref{07:e9b}). Here, the VP property attributed to Churchill (and asserted to hold of no other individual) is that of remembering \textit{Churchill} giving the speech. This construal arises because the overt pronoun can be understood as coreference with the NP Churchill rather than being semantically bound. That the reading in (\ref{07:e9b}) is unavailable with PRO is taken to show that the relation that holds between (OC) PRO and its antecedent can only be BVA, not coreference anaphora.

Importantly, however, as we shall see in \sectref{07:para-infinitives}, overt PRO, unlike null PRO, patterns like a standard pronoun on the association-with-focus test, in allowing for both coreferential and BV construals, but just like null PRO on the ellipsis test, in allowing solely BV construals. There is therefore an unexpected and seemingly paradoxical distribution of interpretations. We show that these facts follow straightforwardly from the Anaphor Generalizations that we state in (\ref{07:AG10}):
\ea\label{07:AG10}
{The Anaphor Generalizations:}
\begin{enumerate}[(i)]
    \item Both null and overt anaphors need to be syntactically bound.
    \item Overt anaphors can be semantically bound, null anaphors \textit{must} be semantically bound.
\end{enumerate}
\z

\largerpage
This paper is organized as follows. \sectref{07:section2} establishes the properties of both overt and null subjects in light of the tests distinguishing OC from NOC across the three types of adjunct infinitival clauses introduced above. \sectref{07:section3} provides an analysis of NOC, as instantiated by the \textit{sin}-pattern. Taking as point of departure \citeauthor{hornstein99}'s (\citeyear{hornstein99}) characterization of NOC null subjects as \textit{pro}, we extend \citeauthor{duguine13}'s (\citeyear{duguine13, duguine14}) DP-ellipsis analysis of \textit{pro} to NOC contexts, providing a compelling argument from the distribution of sloppy readings of null subjects. \sectref{07:sectionPara} discusses what we call the Overt vs. Covert PRO paradox and develops an account in terms of the Anaphor Generalization in (\ref{07:AG10}). \sectref{07:sectionConclusion} concludes the paper.

\section{Diagnostics for (Non) Obligatory Control}
\label{07:section2}
This section seeks to establish a more fine-grained description of the interpretive properties of the null subject across our three classes of adjunct clauses. A battery of tests has been put forth in the literature to diagnose whether control is obligatory or not \citep{williams80, hornstein99, landau00, landau13, baltin15}. OC and NOC can be distinguished as follows. Obligatorily controlled PRO requires a local and c-commanding antecedent for the null subject and, moreover, only allows BV readings under the two tests for pronominal interpretation (ellipsis and association-with-focus). In contrast, the null subject of NOC constructions does not require an antecedent (and if there is one, it need not be local or c-commanding), and allows both BV and coreferential construals under the ellipsis and association-with-focus tests.

Crucially, we apply these tests not only to the null subject of our three non-finite adjunct clauses but also to the overt pronominal subject appearing in \textit{para}-infinitives which, recall from (\ref{07:sp6}), displays restrictions on its interpretation similar to that of (null) PRO.
The picture that emerges is quite an intricate one, with three levels of variation: (i) adjuncts differ from one another with respect to whether they enforce OC or not (cf. also \citealt{landau13}); (ii) while besides null PRO, certain OC adjuncts allow overt subjects (pronouns or R-expressions) (e.g.\textit{ al}-infinitives), others only allow overt PRO (e.g. \textit{para}-infinitives); and (iii) overt and null PRO yield conflicting results with respect to the tests for pronominal interpretation (and as such with respect to the diagnostics for OC vs. NOC) (e.g. \textit{para}-infinitives).

\subsection{\textit{Al}-infinitives}
\label{07:al-infinitives}
Example (\ref{07:sp4}) above shows clearly that while the overt subject of an \textit{al}-infinitive is referentially free, its null subject is referentially dependent. The diagnostic tests teasing apart OC vs. NOC, applied in (\ref{07:sp12}) to (\ref{07:sp15}), show that \textit{al}-infinitives with silent subjects involve OC.\footnote{The judgments for sloppy/strict readings in Colombian Spanish reported in this paper are taken from an experimental protocol carried out with 36 native speakers and using a Truth Value Judgment task \citep{gomezinprogress}. This protocol showed that both \textit{al}-infinitives, and \textit{para}-infinitives in \sectref{07:para-infinitives}, were systematically rejected on a strict reading, but accepted on a sloppy reading  -- in contrast to \textit{sin}-infinitives (\sectref{07:sin-infinitives}, \fnref{07:sinfn}). \label{07:alfn}}

% We also found a different pattern, namely 5 speakers accepted strict readings for both \textit{al}-infinitives and \textit{para}-infinitives. We didn't include this sample in our analysis since they were statistically insignificant.}



%\renewcommand{\eachwordone}{\itshape}
\begin{exe}
\ex\label{07:sp12}
\gll [El papá de Juan$_k$]$_i$ sería feliz [al [Ø]$_{i/*k}$ dejar la casa].\\
the dad of Juan be.\textsc{cond} happy in.the {} leave.\textsc{inf} the house\\
\glt ‘Juan’s dad would be happy once he left the house.’
\end{exe}

%\renewcommand{\eachwordone}{\itshape}
\begin{exe}
\ex\label{07:sp13}
\gll Juan$_i$ sabe que [Pedro$_j$ vendrá  [al [Ø]$_{j/*i}$ terminar los estudios]].\\
Juan knows that Pedro come.\textsc{fut.3sg}  in.the {} finish.\textsc{inf} the studies\\
\glt ‘Juan knows that Pedro will come once he finishes studying.’
\end{exe}


%\renewcommand{\eachwordone}{\itshape}
\begin{exe}
\ex\label{07:sp14}
\gll Juan$_i$ sería feliz [al [Ø]$_i$ dejar la casa]  y María$_k$ también.\\
Juan be.\textsc{inf} happy in.the {} leave.\textsc{inf} the house and María also\\
\glt ‘Juan would be happy when leaving the house and María would too.’
\ea \cmark Sloppy reading \textit{(BVA)}\\
 María will be happy when María leaves the house.
 \ex \xmark Strict reading \textit{(Coreference)}\\
 María will be happy when Juan leaves the house.
 \z
\end{exe}

%\renewcommand{\eachwordone}{\itshape}
\begin{exe}
\ex\label{07:sp15}
\gll Sólo Léa$_i$ se  cayó  [al [Ø]$_i$ subir      al     tren].\\
only Léa \textsc{cl.3} fell.down in.the {} board.\textsc{inf} in.the train\\
\glt ‘Only Léa fell when she was taking the train.’
\ea \cmark\textit{BVA}  \\
No, Karla also fell when she herself was taking the train.\\
Karla ($\lambda$ x(x also fell when x was taking the train))
\ex \xmark\textit{Coreference}\\
No, Karla also fell when Léa was taking the train. \\
 Karla ($\lambda$ x(x also fell when she was taking the train)) \\
(she=Léa)
\z
\end{exe}

Example (\ref{07:sp12}) shows that the null infinitival subject needs to be c-commanded by its antecedent, and (\ref{07:sp13}) shows that long-distance antecedents are not allowed. The ellipsis test in (\ref{07:sp14}) shows that the null subject allows a sloppy (that is, BV) reading, but crucially not a strict (coreferential) reading. The association-with-focus test in (\ref{07:sp15}) confirms that it only allows a BV construal.\footnote{Spanish doesn’t admit VP-ellipsis, but allows ellipsis of larger structures \citep{dagnac10,  saab10}.}
In sum, \textit{al}-in\-fin\-i\-tives with null subjects display all the diagnostic properties of OC PRO.

\subsection{\textit{Sin}-infinitives}
\label{07:sin-infinitives}
Whereas \textit{sin} and \textit{al}-infinitives pattern alike in allowing overt subjects (be it R-expressions or pronouns (\ref{07:sp5})), \textit{sin}-infinitives with null subjects differ radically from \textit{al}-infinitives in allowing null subjects that need not be controlled, leading us to characterize the latter as \textit{pro}. Applying the diagnostics for OC vs. NOC in (\ref{07:sp16}) to (\ref{07:sp19}) allows us to confirm this characterization.

%\renewcommand{\eachwordone}{\itshape}
\begin{exe}
\ex\label{07:sp16}
\gll La editorial publicó [el libro de María$_i$]$_j$  sin [Ø]$_i$ haber terminado las correcciones.\\
the publishing.house published the book of María without {} have.\textsc{inf} finished the corrections \\
\glt ‘The publishing house published María’s book without her finishing the corrections.’
\end{exe}

%\renewcommand{\eachwordone}{\itshape}
\begin{exe}
\ex\label{07:sp17}
\gll Juan$_i$ sabe [que se abrieron las puertas [sin [Ø]$_{i/k}$ dar la autorización]]. \\
Juan knows that \textsc{cl.3} opened the doors without {} give.\textsc{inf} the permission \\
\glt ‘Juan knows that doors were opened without him giving permission.’
\end{exe}


%\renewcommand{\eachwordone}{\itshape}
\begin{exe}
\ex \label{07:sp18}
\gll María$_i$   dejó de trabajar    [sin ella$_i$/ [Ø]$_i$  decir nada] y Rosa$_k$ también]. \\
María stopped of work.\textsc{inf} without she.\textsc{nom} {} say.\textsc{inf} nothing and Rosa too\\
\glt ‘María stopped working without her saying anything and Rosa did too.’
\ea\cmark Sloppy reading \textit{(BVA)}\\
And Rosa also stopped working without Rosa saying anything.
\ex\cmark Strict reading \textit{(Coreference)}\\
And Rosa also stopped working without María saying anything.
 \z
\end{exe}

%\renewcommand{\eachwordone}{\itshape}
\begin{exe}
\ex \label{07:sp19}
\gll Sólo María$_i$ dejó de trabajar [sin [Ø]$_i$ firmar la autorización].\\
only María stopped of work.\textsc{inf} without {} sign.\textsc{inf} the authorization\\
\glt ‘Only María stopped working without her authorizing signature.’
\ea \cmark\textit{BVA}\\
No, Daniela also stopped working without her own signed authorization.\\
Daniela (λx(x stopped working without x’s signed authorization)).
\ex \cmark\textit{Coreference}\\
No, Daniela also stopped working without María’s signed authorization. (In a context where María$_i$ was the only person that could sign the authorization).\\
Daniela (λy(y stopped working without her authorization))\\ (her=María).
 \z
\end{exe}

\textit{Sin}-infinitives exhibit all the properties of NOC. (\ref{07:sp16}) shows that their null subject need not be c-commanded by its antecedent, and (\ref{07:sp17}) that it need not be locally bound since long-distance antecedents are permitted. The ellipsis and association-with-focus tests in (\ref{07:sp18}) and (\ref{07:sp19}) show that their null subject allows for both BV and coreferential readings. These findings confirm that these silent subjects are of the \textit{pro}-type, not of the (OC) PRO type.\footnote{The experiment \citep{gomezinprogress} (see \fnref{07:alfn}) showed that in contrast to both \textit{para}-infinitives and \textit{al}-infinitives, strict readings in \textit{sin}-infinitives, just like sloppy readings, were accepted systematically by speakers. \label{07:sinfn}}
%The results also showed higher scores for sloppy readings than for strict readings.

% strict readings in \textit{sin}-infinitives, just like sloppy readings, were accepted systematically by speakers.
%(that is, 13/36 people accepted this reading).

\subsection{\textit{Para}-infinitives}
\label{07:para-infinitives}
Just as in \textit{al}-infinitives, the null subject in \textit{para}-infinitives is referentially dependent, which we took to suggest it should also be PRO (see (\ref{07:sp4})/(\ref{07:sp6})).\footnote{See \fnref{07:alfn}.} Unlike in \textit{al}-infinitives, however, and as indicated in example (\ref{07:sp6}), the overt subject in these constructions is also referentially dependent, leading us to characterize it as overt PRO. We apply below the diagnostic tests for OC vs. NOC to both overt PRO and covert PRO:

%\renewcommand{\eachwordone}{\itshape}
\begin{exe}
\ex\label{07:sp20}
\gll [El hermano de Juan$_k$]$_i$ se fue [para él$_{i/*k}$/ [Ø]$_{i/*k}$  estar feliz].\\
the brother of Juan \textsc{cl.3} left   for he.\textsc{nom} {} be.\textsc{inf}  happy \\
\glt ‘Juan’s brother left (in order for him) to be happy.’
\end{exe}

%\renewcommand{\eachwordone}{\itshape}
\begin{exe}
\ex\label{07:sp21}
\gll [Pedro$_k$ sabe [que Juan$_i$ se fue [para él$_{i/*k}$/ [Ø]$_{i/*k}$ estar feliz]]].\\
Pedro knows that Juan \textsc{cl.3} left for he.\textsc{nom} {} be.\textsc{inf}  happy\\
\glt ‘Pedro knows that Juan left (in order for him) to be happy.
\end{exe}


%\renewcommand{\eachwordone}{\itshape}
\begin{exe}
\ex\label{07:sp22}
\gll [Juan$_i$ se fue [para él$_{i/*k}$/ [Ø]$_{i/*k}$ estar feliz] y María$_k$ también].\\
Juan \textsc{cl.3} left for he.\textsc{nom} {} be.\textsc{inf} happy and María also\\
\glt ‘Juan left to be happy and María did too.\\
\ea\cmark Sloppy reading \textit{(BVA)}\\
 María left in order for María to be happy.
\ex\xmark Strict reading \textit{(Coreference)}\\
María left in order for Juan to be happy.
\z
\end{exe}

Overt and covert controlled subjects pattern exactly the same under these three tests. They need to be c-commanded by their antecedent (\ref{07:sp20}), locally bound (\ref{07:sp21}), and only allow BV readings under ellipsis (\ref{07:sp22}). In sum, both types of subjects uniformly exhibit OC properties. These results confirm that the null subject is OC PRO, and that the overt subject is its overtly realized counterpart ``Overt PRO'' \citep{piera87, mensching00, livitz11, livitz14}.
Crucially, however, the association-with-focus test in (\ref{07:sp23})/(\ref{07:sp24}) yields dissonant results:

%\renewcommand{\eachwordone}{\itshape}
\begin{exe}
\ex\label{07:sp23}
\gll Sólo María$_i$ hizo trampa [para [Ø]$_i$ ganar el primer lugar]. \\
only María made trap    for  {} win.\textsc{inf}  the first place \\
\glt ‘Only María cheated in order for herself to win the first place’.
\ea \cmark\textit{BVA}\\
No, Daniela also cheated in order for herself to win. \\
    Daniela ($\lambda$ y (y also cheated in order for y to win)).
\ex \xmark\textit{Coreference}\\
No, Daniela also cheated in order for María to win.\\
 Daniela ($\lambda$ y (y also cheated in order for her to win))\\
(her= María)\\
\z
\end{exe}

The statement in (\ref{07:sp23}) with a null (by hypothesis, PRO) subject can only be denied in one way -- that is, on its BVA construal, as expected. This result converges with those from the three previous tests in signaling OC.
In contrast, the statement in (\ref{07:sp24}) with an overt (by hypothesis, PRO) subject can be denied in either of two ways -- that is, on either its BVA or coreferential construal.

%\renewcommand{\eachwordone}{\itshape}
\begin{exe}
\ex\label{07:sp24}
\gll Sólo María$_i$ hizo trampa [para ella$_i$ ganar el primer lugar].\\
only María made trap for she.\textsc{nom} win.\textsc{inf} the first place\\
\glt ‘Only María cheated in order for herself to win the first place.’
\ea \cmark\textit{BVA}\\
No, Daniela also cheated in order for herself to win. \\
  Daniela ($\lambda$ y (y also cheated in order for y to win))\\
\ex \cmark\textit{Coreference}\\
No, Daniela also cheated in order for María to win.\\
 Daniela ($\lambda$ y (y also cheated in order for her to win))\\
(her= María)\\
\z
\end{exe}

The availability of both readings here for overt PRO is surprising, given that it is characteristic of NOC -- and more generally of pronouns, see discussion of Fodor’s examples (\ref{07:e7})--(\ref{07:e9}) --  and that the results of all the previous tests converge on an OC diagnostic.

\subsection{Interim summary}
\label{07:interim}
The results obtained in this section are summarized in \tabref{07:tab:1:infinitivals}.
\begin{table}
%\resizebox{12cm}{!} {
\caption{Null vs. overt subjects in CS infinitival adjuncts.}
\label{07:tab:1:infinitivals}
\small
\begin{tabularx}{\textwidth}{lllllQl}
  \lsptoprule
               & \multicolumn{2}{c}{\textit{Al}-infinitives} & \multicolumn{2}{c}{\textit{Sin}-infinitives} & \multicolumn{2}{c}{\textit{Para}-infinitives}  \\
\cmidrule(lr){2-3}\cmidrule(lr){4-5}\cmidrule(lr){6-7}
               & Overt & Null                     & Overt & Null                      & Overt     & Null                    \\
\midrule
Distribution   & DP    & PRO                        & DP    & \textit{pro}                         & Overt PRO & PRO                       \\
Interpretation & Free  & Dependent                  & Free  & Free                        & Dependent (except in association-with-focus) & Dependent                 \\
  \lsptoprule
\end{tabularx}
%}
\end{table}

A complex picture emerges from these findings. First, nonfinite adjuncts divide up into three classes. The first class, instantiated by \textit{al}-infinitives, displays OC properties with null subjects while allowing for overt subjects. The second class, instantiated by \textit{sin}-infinitives, displays the properties of NOC, systematically allowing for referentially free subjects -- be they silent or lexical. And the third class, instantiated by \textit{para}-infinitives, exhibits the properties of OC fully with null PRO, but not overt PRO.
Second, overt subjects do not display homogeneous behavior across OC adjuncts. In particular, while  \textit{al}-infinitives allow either overt pronouns or R-expressions in subject position (alongside null PRO), \textit{para}-infinitives only allow overt PRO (alongside null PRO).
Finally, regarding the results of the OC/NOC tests in \textit{para}-infinitives, a surprising contrast has emerged. On the one hand, null PRO displays all the characteristic properties of OC since it must be locally c-commanded by its antecedent, and it only yields BV readings under the ellipsis and association-with-focus tests. On the other hand, overt PRO displays most, but not all the characteristics of OC. It must be locally c-commanded by its antecedent, and it only yields BV readings under the ellipsis test. Crucially, however, it yields both coreference and BV readings under the association-with-focus test, patterning this time like a non-obligatorily controlled subject. These unexpected findings are recapitulated in (\ref{07:px25}):

\ea\label{07:px25}
{Overt vs. Covert PRO paradox}
\begin{enumerate}[(i)]
    \item Both null and overt PRO only allow BV interpretations under the ellipsis test.
    \item Overt PRO, unlike null PRO, also allows coreferential interpretations under the association-with-focus test.
\end{enumerate}
    Why do null PRO and overt PRO pattern differently (yield conflicting results) with respect to the two standard tests for pronominal interpretation? In particular, why does overt PRO, unlike null PRO,
    also allow co-reference interpretations,
    and why so only under the association-with-focus test, but not under the ellipsis test?
\z
We address this paradox in \sectref{07:sectionPara} below. First, however, we turn to the analysis of NOC as instantiated in \textit{sin}-infinitives.


\section{Non Obligatory Control in \textit{sin}-infinitives}
\label{07:section3}
\textit{Sin}-infinitives exhibit a uniform NOC pattern. Assuming with \citet{hornstein99} that a NOC null subject is not PRO, but \textit{pro}, we develop here an analysis that not only explains why overt DPs and \textit{pro} subjects alternate in these constructions, but also correctly predicts that they yield sloppy readings otherwise unavailable with (overt) pronouns.
We put forth an analysis of NOC in terms of DP-ellipsis, in line with ellipsis approaches to null arguments in \textit{pro}-drop languages (cf. a.o. \citealt{oku98,saito07, duguine13, duguine14, takahashi14}). \textit{Sin}-infinitives are thus constructions in which DPs (R-expressions or pronouns) surface in subject position and, under the right conditions, can also undergo ellipsis.

\subsection{A DP-ellipsis analysis of NOC subjects}
\label{07:section3.1}
This section shows that the \textit{pro}-like null subjects of \textit{sin}-infinitives can behave as complex R-expressions, displaying non-pronominal readings.  Consequently, they are better analyzed as elided DPs than as null pronouns.
The crucial piece of data is given below. Consider a conversation like (\ref{07:sp26}) where the subject of the \textit{sin}-infinitive in \REF{07:sp26B} is null:

%\renewcommand{\eachwordone}{\itshape}
\begin{exe}
\ex\label{07:sp26}
\begin{xlista}
\ex
\gll\label{07:sp26a}María$_i$ dejó de trabajar [sin [su$_i$ jefa]$_j$ decirle$_i$ nada].\\
María stopped of work.\textsc{inf} without \textsc{poss} boss say.\textsc{inf}.\textsc{3sg} nothing\\
\glt ‘María stopped working without her boss saying anything to her.’
\ex
\gll\label{07:sp26b}\label{07:sp26B}Y Ana$_k$ también dejó de trabajar [sin [Ø] decirle$_k$ nada].\\
and Ana also stopped of work.\textsc{inf} without {} say.\textsc{inf}.\textsc{3sg} nothing\\
\glt ‘And Ana also stopped working without saying anything to her.’
\end{xlista}
\end{exe}


The sentence in (\ref{07:sp26b}) is ambiguous. The null subject can refer either to María's boss or to Ana's boss and the sentence can thus be interpreted as in (\ref{07:sp27a}) or (\ref{07:sp27b}) respectively:

\ea\label{07:sp27}
María stopped working without her boss saying nothing to her.
\ea\label{07:sp27a}\cmark Strict-like reading\\
And Ana also stopped working without María's boss saying anything to her.\\
\ex\label{07:sp27b} \cmark Sloppy-like reading\\
 And Ana also stopped working without her own boss saying anything to her.
\z
 \z

The strict reading in (\ref{07:sp27a}) can be explained if the null subject in (\ref{07:sp26B}) is a covert pronominal expression, corefering with the DP \textit{su jefa} ‘her boss’ introduced in the previous discourse in (\ref{07:sp27}). However, since Ana's boss has not been introduced in the prior discourse context, a pronominal expression can logically not give rise to the sloppy interpretation in (\ref{07:sp27b}). This is confirmed by (\ref{07:sp28}): substituting an overt pronoun for the silent subject in (\ref{07:sp26B}) can give rise to the strict interpretation in (\ref{07:sp27a}), but not to the sloppy one in (\ref{07:sp27b}):

%\renewcommand{\eachwordone}{\itshape}
\begin{exe}
\ex\label{07:sp28}
\begin{xlista}
\ex
\gll\label{07:sp28A}María$_i$ dejó de trabajar [sin [su$_i$ jefa]$_j$ decirle$_i$ nada].\\
María stopped of work.\textsc{inf} without \textsc{poss} boss say.\textsc{inf}.\textsc{3sg} nothing\\
\glt ‘María stopped working without her boss saying nothing to her.’\\
\ex
\gll\label{07:sp28B}Y Ana$_k$ también dejó de trabajar [sin ella$_j$ decirle$_k$ nada].\\
and Ana also stopped of work.\textsc{inf} without she.\textsc{nom} say.\textsc{inf}.\textsc{3sg} nothing\\
\glt ‘And Ana also stopped working without her saying nothing to her.’
\end{xlista}
\end{exe}


The sloppy reading for the sentence in (\ref{07:sp26B}) can however be accounted for if we postulate that, rather than a pronoun, the null subject is the covert complex R-expression \textit{su jefa} ‘her boss’. This analysis would make it possible for the possessive pronoun \textit{su} inside the (covert) R-expression to be bound by the higher subject \textit{Ana}, giving rise to the targeted sloppy reading in (\ref{07:sp27b}). The ambiguity of (\ref{07:sp29}), where the null subject has been overtly spelled out as the DP \textit{su jefa}, and which allow both the sloppy reading in (\ref{07:sp27b}) and the strict reading in (\ref{07:sp27a}), corroborates our analysis:

%\renewcommand{\eachwordone}{\itshape}
\begin{exe}
\ex\label{07:sp29}
\begin{xlista}
\ex
\gll\label{07:sp29A}María$_i$ dejó de trabajar [sin [su$_i$ jefa]$_j$ decirle$_i$ nada].\\
María stopped of work.\textsc{inf} without \textsc{poss} boss say.\textsc{inf}.\textsc{3sg} nothing\\
\glt ‘María stopped working without her boss saying nothing to her.’\\
\ex
\gll\label{07:sp29B}Y Ana$_k$ también dejó de trabajar [sin [su jefa] decirle$_k$ nada].\\
and Ana also stopped of work.\textsc{inf} without \textsc{poss} boss say.\textsc{inf}.\textsc{3sg} nothing\\
\glt ‘And Ana also stopped working without her boss saying nothing to her.’
\end{xlista}
\end{exe}


That (\ref{07:sp29B}) with an overt R-expression in subject position and (\ref{07:sp26B}) with a silent subject yielding the same sloppy reading suggests that, rather than a null pronominal, the null subject of (\ref{07:sp26B}) is a syntactically null complex R-expression embedding the pronoun \textit{su}.
Interestingly, these data parallel some data pointed out by \citet{duguine13, duguine14} in the realm of pro-drop in finite clauses in Spanish. Duguine shows that (\ref{07:sp30B}), with a null subject in a finite context, allows the two interpretations in (\ref{07:sp31}):

%\renewcommand{\eachwordone}{\itshape}
\begin{exe}
\ex\label{07:sp30}
\begin{xlista}
\ex
\gll\label{07:sp30A}María$_i$ cree [que [su$_i$ jefa]$_j$ le$_i$ exigirá mucho trabajo].\\
María believes that \textsc{poss} boss \textsc{cl.3sg.dat} require.\textsc{fut.3sg} much work\\ \hfill (adapted from Duguine 2014:520)
\glt ‘María believes that her boss will require a lot of work from her.’\\
\ex
\gll\label{07:sp30B}Y Ana$_k$ espera [que [Ø] le$_k$ dejará los fines de semana libre].\\
and Ana hopes that {} \textsc{cl.3sg.dat} leave.\textsc{fut.3sg} the ends of week free\\
\glt ‘And Ana hopes [Ø] will leave her the weekends free.’
\end{xlista}
\end{exe} % traduction

\ea \label{07:sp31}
\ea\label{07:sp31a}\cmark Strict reading\\
And Ana hopes that María’s boss will leave her the weekends free. \\
\ex\label{07:sp31b}\cmark Sloppy reading\\
And Ana hopes that her (own) boss will leave her the weekends free.
\z
\z

As already mentioned with regards to (\ref{07:sp26}), postulating that the silent embedded subject is a pronominal incorrectly predicts that (\ref{07:sp30}) should be unambiguous, allowing only the strict reading in (\ref{07:sp31a}), exactly as is the case for (\ref{07:sp32})  where an overt pronoun has been substituted for the silent subject.

%\renewcommand{\eachwordone}{\itshape}
\begin{exe}
\ex\label{07:sp32}
\gll Y Ana$_k$ espera [que ella$_j$ le$_k$ dejará los fines de semana libre].\\
and Ana hopes that she \textsc{cl.3sg.dat} leave.\textsc{fut.3sg} the ends of week free\\
\glt ‘And Ana hopes she will leave her the weekends free.’
\ea\cmark Strict reading\\
%And Ana hopes that she will leave her the weekends free.\\
\ex\xmark Sloppy reading\\
%And Ana hopes that her(own) boss will leave her the weekends free.\\
\z
\end{exe}


\hspace*{-1mm}The lack of ambiguity in (\ref{07:sp32})  has the same logical explanation as above: There is no prior discourse antecedent with which the pronoun could corefer that would allow the sloppy interpretation in (\ref{07:sp31b}).
In contrast, substituting the full R-ex\-pres\-sion \textit{su jefa} for the silent subject does give rise to the intended sloppy reading, as illustrated in (\ref{07:sp33}):


%\renewcommand{\eachwordone}{\itshape}
\begin{exe}
\ex\label{07:sp33}
\gll Y Ana$_k$ espera [que [su$_k$ jefa]$_j$ le$_k$ dejará los fines de semana libre].\\
and Ana hopes that \textsc{poss} boss \textsc{cl.3sg.dat} leave.\textsc{fut.3sg} the ends of week free\\
\glt ‘And Ana hopes her (own) boss will leave her the weekends free.’
\ea\cmark Sloppy reading\\
%Ana hopes that her (own) boss will leave her the weekends free.\\
\ex\cmark Strict reading\\
%And Ana hopes that she will leave her the weekends free\\
\z
\end{exe}


As already pointed out with respect to (\ref{07:sp26B}), the ambiguity of (\ref{07:sp30B}), on a par with that of (\ref{07:sp33}), suggests that null arguments can be null complex R-expressions. Based on this observation, \citet{duguine13, duguine14} proposes that null (finite) subjects in Spanish (and other languages such as Basque) are elided DPs. Under this view, the interpretations in (\ref{07:sp31a}) and (\ref{07:sp31b}) come out as strict and sloppy readings (respectively), arising under ellipsis.
Duguine’s analysis builds on previous literature on null arguments in East-Asian languages such as Korean or Japanese. As is indeed well-known, null arguments in these languages allow non-pronominal interpretations (cf. \citealt{oku98, kim99, saito07, takahashi14}). This is illustrated in Japanese (\ref{07:jp34}), where the null object in the second conjunct can yield not only a pronominal/strict, but also an anaphoric/sloppy reading:

%\renewcommand{\eachwordone}{\itshape}
\begin{exe}
\ex
\label{07:jp34}
\langinfo{Japanese}{}{Takahashi 2014} \\
\gll Taroo$_i$-wa zibun$_i$-o semeta-ga, Ken$_j$-wa [Ø] kabatta.\\
Taroo-\textsc{top} self-\textsc{acc} blamed-while Ken-\textsc{top} {} defended\\
\glt Lit. While Taroo blamed self, Ken defended [Ø].\\
\ea\label{07:jp34a}\cmark Strict reading\\
Ken defended Taroo.
\ex\label{07:jp34b} \cmark Sloppy reading\\
Ken defended himself.
\z
\end{exe}

The sloppy reading in (\ref{07:jp34b}) indicates that the null object has the properties of a locally bound anaphor. Sloppy readings represent a major challenge for analyses of null arguments assuming \textit{pro} as a primitive -- e.g. \citet{chomsky82, rizzi86} -- since in the configuration in (\ref{07:jp34b}), the null pronominal would be locally bound, violating Binding Condition A (see (\ref{07:jp35a})). This problem disappears, however, with an analysis in terms of ellipsis -- here ellipsis of an anaphor, as indicated in (\ref{07:jp35b}) (cf. \citealt{oku98, kim99, saito07, takahashi14}):

\ea\label{07:jp35}
\ea\label{07:jp35a}\xmark Sloppy reading \\
\gll \textit{Ken$_j$-wa} \textit{pro} \textit{kabatta}.\\
Ken-\textsc{top} {} defended\\
\ex \cmark Strict reading \\
\gll\label{07:jp35b}\textit{Ken$_j$-wa} \st{zibun-o} \textit{kabatta}.\\
Ken-\textsc{top} self-\textsc{acc} defended \\
\z
\z

In sum, the availability of sloppy readings for the null subjects of \textit{sin}-infinitives provides a compelling argument for a DP-ellipsis analysis of NOC.

\subsection{\text{Parallelism conditions on DP-ellipsis}}
Building on \citeauthor{fox00}'s (\citeyear{fox00}) Conditions on NP-Parallelism, Duguine (\citeyear{duguine13, duguine14}) develops an analysis of argument ellipsis which accounts for the varieties of readings null arguments give rise to. Under this account, an elided DP must satisfy either of the conditions in (\ref{07:DP36}):

\ea \label{07:DP36}
{DP-Parallelism} (adapted from NP-Parallelism; \citealt[117]{fox00})\\
DPs in the elided constituent and their antecedents must either:
\ea have the same referential value (Referential Parallelism), or\\
\ex be linked by identical dependencies  (Structural Parallelism).\\
\z
\z

The Conditions on DP-Parallelism will provide a formal account of the strict and sloppy readings that NOC null subjects display. The relevant piece of data in (\ref{07:sp26}) and its possible interpretations in (\ref{07:sp27}) are repeated in (\ref{07:sp37})--(\ref{07:sp38}):


%\renewcommand{\eachwordone}{\itshape}
\begin{exe}
\ex\label{07:sp37}
\begin{xlista}
\ex
\gll\label{07:sp37A}María$_i$ dejó de trabajar [sin [su$_i$ jefa]$_j$ decirle$_i$ nada].\\
María stopped of work.\textsc{inf} without \textsc{poss} boss say.\textsc{inf}.\textsc{3sg} nothing\\
\glt ‘María stopped working without her boss saying nothing to her.’\\
\ex
\gll\label{07:sp37B}Y Ana$_k$ también dejó de trabajar [sin [Ø] decirle$_k$ nada].\\
and Ana also stopped of work.\textsc{inf} without  {} say.\textsc{inf}.\textsc{3sg} nothing\\
\glt ‘And Ana also stopped working without saying nothing to her.’
\end{xlista}
\end{exe}


\ea\label{07:sp38}
\ea\label{07:sp38a}\cmark Strict-like reading\\
And Ana also stopped working without María’s boss saying nothing to her.\\
\ex\label{07:sp38b}\cmark Sloppy-like reading\\
And Ana also stopped working without her own boss saying nothing to her.
\z
 \z

Now, to derive the strict reading in (\ref{07:sp38a}), the condition on Referential Parallelism must be satisfied. The latter requires the null subject in (\ref{07:sp37B}) to share the same referential value as its discourse antecedent, \textit{su jefa} in (\ref{07:sp37A}). We can achieve this by postulating the representation in (\ref{07:sp39b}) where the pronoun \textit{ella} ‘her’, corefering with its discourse antecedent \textit{su jefe}, undergoes ellipsis:

\ea\label{07:sp39}
\ea\label{07:sp39a}María$_i$ dejó de trabajar [sin [su$_i$ jefa]$_j$ decirle$_i$ nada].\\
\ex\label{07:sp39b}Y Ana$_k$ también dejó de trabajar [sin \st{ella$_j$} decirle$_k$ nada]. \\
\z
\z


On the other hand, to derive the sloppy reading in (\ref{07:sp38b}), the condition on Structural Parallelism must be satisfied. The latter requires identical binding dependencies across both (\ref{07:sp37A}) and (\ref{07:sp37B}). We can achieve this by postulating the derivation in (\ref{07:sp40b}): the complex DP \textit{su jeja} occupying the embedded subject  position and  containing a pronoun bound by the matrix subject undergoes ellipsis,  yielding the surface structure in (\ref{07:sp37B}) under the reading in (\ref{07:sp38b}).

%(\ref{07:sp40b}), where the complex DP \textit{su jefa} occupying the embedded subject position and containing a bound pronoun undergoes ellipsis.

\ea\label{07:sp40}
\ea\label{07:sp40a}María$_i$ dejó de trabajar [sin [su$_i$ jefa]$_j$ decirle$_i$ nada].\\
\ex\label{07:sp40b}Y Ana$_k$ también dejó de trabajar [sin \st{[su$_k$ jefa]$_m$} decirle$_k$ nada]. \\
\z
\z
(\ref{07:sp40a}) an (\ref{07:sp40b}) display identical anaphoric dependencies between the matrix subject, the possessive pronoun and the (dative) clitic, thus satisfying the condition on Structural Parallelism and licensing ellipsis in (\ref{07:sp40b}).

\subsection{\text{Interim conclusion}}
The DP-ellipsis analysis of non-obligatorily controlled null subjects in \textit{sin}-in\-fin\-i\-tives elegantly predicts that they allow sloppy readings otherwise unavailable with overt pronouns. It also automatically explains why overt subjects in \textit{sin}-infinitives freely alternate with referentially free null subjects. That is to say, ellipsis presumes this alternation to be possible in the first place: DP-ellipsis targets overt DPs and, as such, can only occur where overt DPs can surface, in the same way that VP-ellipsis can only occur where overt VPs can surface.


\section{Obligatory Control:  the overt vs. covert PRO puzzle \textit{(para}-infinitives)}
\label{07:sectionPara}
\sectref{07:para-infinitives} showed that null vs. overt PRO in \textit{para}-infinitives converge in all but one of their properties: While under both the ellipsis and focus particle tests the former only allows BV interpretations, the latter also allows coreferential readings but under the association-with-focus test only. This is summarized in \tabref{07:tab:2:para-infinitives}.

\begin{table}
%\resizebox{10 cm}{!}{
\caption{Null vs. overt PRO in para-infinitives.}
\label{07:tab:2:para-infinitives}
 \begin{tabularx}{\textwidth}{lQll}
  \lsptoprule
    &    & Overt PRO       & Null PRO\\
   \midrule
BVA         & Ellipsis test               & yes       & yes       \\
                             & Association-with-focus test & yes       & yes       \\
Coreference & Ellipsis test               & no        & no        \\
                             & Association-with-focus test & yes       & no        \\
  \lspbottomrule
 \end{tabularx}
%}
\end{table}

We now tackle the questions raised in \sectref{07:interim}, regarding how to reconcile our contradictory findings and solve the paradox that these unexpected results give rise to, repeated in (\ref{07:px41}) (see also (\ref{07:px25})):

\ea\label{07:px41}
{Overt vs. Covert PRO paradox}
\begin{enumerate}[(i)]
    \item Both null and overt PRO only allow BV interpretations under the ellipsis test.
    \item Overt PRO, unlike null PRO, also allows coreferential interpretations under the association-with-focus test.
\end{enumerate}
\z
Why do null PRO and overt PRO pattern differently (yield conflicting results) with respect to the two standard tests for pronominal interpretation? In particular, why does overt PRO, unlike null PRO, also allow coreference interpretations, and why so only under the association-with-focus test, but not under the ellipsis test?


First of all, it should be pointed out that the conflicting patterns of interpretation that arise with overt PRO also arise with overt anaphors such as \textit{himself} in English or \textit{se} in French. Indeed, both have also been reported to allow coreferential readings alongside BVA under the association-with-focus test. This state of affairs is illustrated in (\ref{07:ef42}) through (\ref{07:ef45}), taken from \citet{sportiche14} (based on a remark due to M. Prinzhorn about German).\footnote{For discussion of these issues, see \citet{buring05} and references therein, as well as \fnref{07:fn11}.}


\ea\label{07:ef42}
\ea {Only Pierre shaves himself}.
\ex
\gll {Seul} {Pierre} {se} {rase}.\\
only Pierre \textsc{cl.3} shave\\
\glt ‘Only Pierre shaves himself.’
\z
\z

That both statements are ambiguous between a BV and a coreferential reading is shown by the fact they can be denied in two different ways:

\ea\label{07:ef43}
\ea {No, I shave myself too}.\\
\ex
\gll {Non}, {moi} {aussi} {je} {me} {rase}. \\
no I also  I \textsc{cl}.1sg\textsc{acc} shave \\
\glt   ‘No, I shave myself too.’\\
    →VP property: $\lambda$ x(x shave x) \hfill  \textit{BVA}
\z
\z

\ea\label{07:ef44}
\ea {No, I shave him too}.\\
\ex
\gll {Non}, {moi} {aussi} {je} {le}  {rase}. \\
no I also  I \textsc{cl}.3sg\textsc{acc} shave \\
\glt   ‘No, I shave him too.’\\
→VP property: $\lambda$ x(x shave y) with y = Pierre \hfill  \textit{Coreference}
\z
\z

Furthermore, \citet{sportiche14} points out that VP-ellipsis does not give the same result as association with \textit{only} since it does not allow a coreferential interpretation for anaphors. For instance, the second conjunct in (\ref{07:ef45a}) or (\ref{07:ef45b}) below cannot be interpreted as meaning “Pierre shaved Jean”:

\ea\label{07:ef45}
\ea\label{07:ef45a}{Jean} {s’est} {rasé} {et} {Pierre} {aussi}.\\
Jean \textsc{cl.3.}is shaved and Pierre also\\
\glt ‘Jean shave himself and Pierre did too.’
\ex\label{07:ef45b} {Jean shaved himself and Pierre did, too}.
\z
\z

The contrast above suggests that the conflicting patterns of interpretation that overt PRO yields is not a surprising state of affairs, but rather appears to reflect a more general property of overt anaphors, be it self-anaphors or overt PRO. We thus put forth the following generalizations to account for the Overt vs. Covert PRO paradox in (\ref{07:AG46}):\footnote{\citet[141]{buring05} discusses the wrong prediction made for reflexives in association-with-focus constructions (namely, that they should only allow BV construals, contrary to fact), concluding with a suggestion similar in spirit to ours: ``As far as I know, this wrong prediction has not been addressed in the pertinent literature. The only immediate way to capture this behavior would seem to be to reformulate Binding Condition A so as to require that reflexives be \textit{either} semantically or syntactically bound within their local domain, accepting the fact that Binding Conditions A and B are simply not on a par".
This suggestion, however, does not carry over straightforwardly to ellipsis contexts where, as discussed by Büring, the pattern is more complex. Roughly, strict readings are generally impossible in coordinated ellipsis (\ref{07:fn_ex_i}), but possible in subordinated ellipsis (\ref{07:fn_ex_ii}), a generalization advocated by \citet{hestvik92}:
\begin{exe}
\ex \label{07:fn_ex_i} John defended himself, and Peter too.	\hfill	(sloppy)
\ex \label{07:fn_ex_ii} John defended himself better than Peter. \hfill	(strict or sloppy)
\end{exe}
As Hestvik points out, the crucial factor at play in subordinated ellipsis –but lacking in coordinated ellipsis– is that the matrix antecedent of the reflexive on the strict reading c-commands the ellipsis site and, as such, can bind the anaphor in the elided VP.

This contrast is in keeping with the generalization in (\ref{07:AG46}\ref{07:AG46i}) since the anaphor in the ellipsis site can satisfy the syntactic binding requirement in (\ref{07:fn_ex_ii}) but not in (\ref{07:fn_ex_i}).
\label{07:fn11}}

\ea \label{07:AG46}
{The Anaphor Generalizations}
\begin{enumerate}[(i)]
    \item \label{07:AG46i} Both null and overt anaphors need to be syntactically bound.
    \item \label{07:AG46ii} Overt anaphors can be semantically bound, null anaphors \textit{must} be semantically bound.
\end{enumerate}
\z

Example (\ref{07:AG46}) requires null anaphoric expressions such as PRO to be both syntactically bound (that is, coindexed with a c-commanding DP) and semantically bound (that is, interpreted at LF as a variable bound by a predicate abstractor/$\lambda$-operator), while only enforcing syntactic binding for overt anaphors, which include self-anaphors like \textit{se} and \textit{himself}, as well as overt PRO.
Consider first the ellipsis context in (\ref{07:sp22}), repeated here as (\ref{07:sp47}):

%%\renewcommand{\eachwordone}{\itshape}
\begin{exe}
\ex\label{07:sp47}
\gll Juan$_i$ se fue [para él$_{i}$/ [Ø]$_{i}$ estar feliz] y María$_k$ también.\\
Juan \textsc{cl.3} left for he {} be.\textsc{inf} happy and María also\\
\glt ‘Juan left to be happy and María did too.’\\
\ea\label{07:sp47a}\cmark Sloppy reading \textit{(BVA)}\\
María left in order for María to be happy.\\
\ex\label{07:sp47b}\xmark Strict reading \textit{(Coreference)}\\
María left in order for Juan to be happy.\\
\z
\end{exe}


Both overt and null PRO only allow the BV construal in (\ref{07:sp47a}). This is accounted for by the syntactic binding requirement in (\ref{07:AG46}\ref{07:AG46i}) which anaphors, whether they are overt or not, are required to satisfy. The sloppy/BV reading is available because the representation in (\ref{07:sp48a}) satisfies (\ref{07:AG46}\ref{07:AG46i}) since in each conjunct, the matrix subject locally binds (c-commands)  the embedded overt/null PRO.

\ea \label{07:sp48}
\ea\label{07:sp48a}\cmark Juan$_i$ se fue [para él$_{i}$/[Ø]$_{i}$ estar feliz] y María$_k$ también <se fue [para ella$_k$/[Ø]$_{k}$ estar feliz]>.
\ex\label{07:sp48b}\xmark Juan$_i$ se fue [para él$_{i}$/[Ø]$_{i}$ estar feliz]  y María$_k$ también <se fue [para él$_{i}$/[Ø]$_{i}$ estar feliz]>.
\z
\z


In contrast, the strict/coreference reading in (\ref{07:sp47b}) is unavailable because the representation in (\ref{07:sp48b}) fails to satisfy (\ref{07:AG46}\ref{07:AG46i}) since the overt/null PRO embedded in the second conjunct is not bound (c-commanded) by the matrix antecedent in the first conjunct.


We turn next to the association-with-focus paradigm in (\ref{07:sp23})--(\ref{07:sp24}), repeated below. This time, (\ref{07:AG46}\ref{07:AG46i}) does not filter out the coreferential reading in (\ref{07:sp49}) since both overt and null PRO can be bound by the matrix antecedent, thus satisfying (\ref{07:AG46}\ref{07:AG46i}):

\ea\label{07:sp49}
\gll Sólo María$_i$ hizo trampa [para ella$_i$/[Ø]$_i$ ganar el primer lugar]. \\
only María made trap for she win.\textsc{inf} the first place \\
\glt ‘Only María cheated in order for herself to win the first place’.
\z

(\ref{07:AG46}\ref{07:AG46ii}), however, is at play here, explaining why null and overt PRO cannot be interpreted alike in this context:

%\renewcommand{\eachwordone}{\itshape}
\begin{exe}
\ex \label{07:sp50}
\gll Sólo María$_i$ hizo trampa [para [Ø]$_i$ ganar el primer lugar]. \\
only María made trap for {} win.\textsc{inf} the first place \\
\glt ‘Only María cheated in order for herself to win the first place’.
\ea\label{07:sp50a}\cmark\textit{BVA}\\
No, Daniela also cheated in order for herself to win. \\
  Daniela (λy (y also cheated in order for y to win)).\\
\ex\label{07:sp50b}\xmark\textit{Coreference}\\
No, Daniela also cheated in order for María to win.\\
 Daniela (λy (y also cheated in order for her to win))\\
(her= María)\\
\z
\end{exe}


%\renewcommand{\eachwordone}{\itshape}
\begin{exe}
\ex\label{07:sp51}
\gll Sólo María$_i$ hizo trampa [para ella$_i$ ganar el primer lugar].\\
only María made trap for she.\textsc{nom} win.\textsc{inf} the first place\\
\glt ‘Only María cheated in order for herself to win the first place.’
\ea\label{07:sp51a}\cmark\textit{BVA}\\
No, Daniela also cheated in order for herself to win. \\
  Daniela (λy (y also cheated in order for y to win)).\\
\ex\label{07:sp51b}\cmark\textit{Coreference}\\
No, Daniela also cheated in order for María to win.\\
 Daniela (λy (y also cheated in order for her to win))\\
(her= María)\\
\z
\end{exe}
Example (\ref{07:sp51}) with an overt PRO subject in the \textit{para}-clause can be denied either on its BV construal in (\ref{07:sp51a}) or on its coreferential construal in (\ref{07:sp51b}). In contrast, (\ref{07:sp50}) with a null PRO instead can only be denied on its BV construal in (\ref{07:sp50a}) but not on its coreferential construal in (\ref{07:sp50b}). This contrast establishes that the coreferential reading is available for overt PRO but not for null PRO. It follows, moreover, automatically under (\ref{07:AG46}\ref{07:AG46ii}): null PRO, unlike its overt counterpart, must be semantically bound, and thus obligatorily interpreted as a BV. In contrast, overt PRO can but need not be semantically bound and, as such, is free to corefer with its DP antecedent \textit{María}.

\section{Conclusion}
\label{07:sectionConclusion}
We started off by exemplifying how adjunct clauses in CS instantiate three different but systematic patterns of exception to the generalizations commonly assumed to hold of infinitival subjects (as stated in (\ref{07:StandGen})):
\begin{itemize}
    \item
    \textit{Al}-infinitives, which display the characteristic diagnostics of OC, violate the ban on overt (nominative) subjects.
    \item \textit{Para}-infinitives, which also display the characteristic diagnostics of OC, violate the ban on overt controlled subjects (leading automatically to a violation of the ban on overt subjects).
    \item
    \textit{Sin}-infinitives, which display the characteristic diagnostics of NOC, violate both the ban on overt subjects and the ban on referentially free null subjects.
 \end{itemize}


We explored a novel DP-ellipsis analysis of NOC (\textit{sin}-infinitives), providing compelling arguments in favor of ellipsis over the alternative silent pronoun (\textit{pro}) analysis (advocated by \citealt{hornstein99} for NOC). In particular, DP-ellipsis elegantly predicts that non-obligatorily controlled subjects allow sloppy readings otherwise unavailable with overt pronouns and automatically explains why overt subjects in \textit{sin}-infinitives freely alternate with referentially free null subjects.

The null subjects of \textit{para}-infinitives display all the signature properties of OC PRO. Interestingly, overt pronouns in the same position, while also displaying diagnostic properties of OC, yield dissonant results with respect to the tests for pronominal interpretation. Under the association-with-focus test (but not the ellipsis test), a NOC-like pattern emerges, with overt PRO (but not null PRO) allowing coreferential readings, alongside BV readings. We suggested that this state of affairs appears to reflect a more general property of overt anaphors, be it overt PRO or self-anaphors. We argued that the conflicting patterns of interpretation that covert vs. overt PRO yield in \textit{para}-infinitives follow from the \textit{Anaphor Generalizations} in (\ref{07:AG46}), which require null anaphors to be syntactically and semantically bound while enforcing only syntactic binding for overt anaphors.



\section*{Acknowledgements}
We would like to thank María Arche, the audiences at
LSRL50,
WOSSP16,
BLINC3,
Spadlsyn,
and two anonymous reviewers for their useful comments.
This work was partially funded by the following projects: AThEME (Advancing the European Multilingual Experience) funded by the European Seventh Framework Programme for research, technological development and demonstration grant agreement no. 613465, Région des Pays de la Loire; UV2 ANR18-FRAL0006 (ANR-DFG); PGC2018-096870-B-I00 (MICINN \& EAI); IT769-13 (Eusko Jaurlaritza); BIM ANR17-CE27-0011 (ANR).

\printbibliography[heading=subbibliography,notkeyword=this]

\end{document}
