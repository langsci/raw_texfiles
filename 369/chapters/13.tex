\documentclass[output=paper,colorlinks,citecolor=brown,draftmode]{langscibook}
\ChapterDOI{10.5281/zenodo.7525116}
\author{Carolina González\affiliation{Florida State University} and Lara Reglero\affiliation{Florida State University}}
\title{Prosodic correlates of mirative and new information focus in Spanish wh-in-situ questions}
\abstract{This paper examines the prosodic correlates of focus in two types of wh-in-situ questions in Spanish: information-seeking (INF), and echo-surprise (SUR). We hypothesize that they will have different intonational properties since the former are associated with new-information focus, while the latter are compatible with mirative focus since they express unexpectedness and surprise \citep{BadanCrocco2019}. A total of 280 sentences from a contextualized elicitation task were analyzed in Praat following SpToBI conventions. Results show that INF and SUR have similar melodic contours, involving a rise through the first pre-nuclear accent, declination, and a steep final rise on the wh-phrase. However, SUR questions have a higher nuclear peak and larger focal tonal range than INF questions. Our results show clear scaling differences in the nuclear configuration consistent with a difference between new-information and mirative focus, which can be phonologically analyzed as nuclear upstep in SUR (L+¡H*), unlike in INF (L+H*).}

% %move the following commands to the "local..." files of the master project when integrating this chapter
% \usepackage{tabularx}
% \usepackage{langsci-basic}
% \usepackage{langsci-optional}
% \usepackage{langsci-gb4e}
% \usepackage{multirow}
% \bibliography{localbibliography}
% %\newcommand{\orcid}[1]{}
% \pagenumbering{arabic}
% \setcounter{page}{277}

% \usepackage{etoolbox}
% \makeatletter
% \patchcmd{\@footnotetext}{\setcounter{fnx}{0}}{\renewcommand{\thexnumi}{\roman{xnumi}}}{}{}
% \apptocmd{\@footnotetext}{
%     \@noftnotetrue
%     \renewcommand{\thexnumi}{\arabic{xnumi}}%
% }{}{}

\IfFileExists{../localcommands.tex}{
  \addbibresource{../localbibliography.bib}
  % add all extra packages you need to load to this file

\usepackage{tabularx,multicol}
\usepackage{url}
\urlstyle{same}

\usepackage{listings}
\lstset{basicstyle=\ttfamily,tabsize=2,breaklines=true}

\usepackage{langsci-basic}
\usepackage{langsci-optional}
\usepackage{langsci-lgr}
\usepackage{langsci-osl}
% \usepackage{./langsci/styles/langsci-lgr}
% \usepackage{./langsci/styles/langsci-osl}
% \usepackage{langsci-gb4e}

\usepackage{tikz}
\usetikzlibrary{patterns,calc}
\pgfdeclarepatternformonly{south east lines}{\pgfqpoint{-0pt}{-0pt}}{\pgfqpoint{3pt}{3pt}}{\pgfqpoint{3pt}{3pt}}{
    \pgfsetlinewidth{0.6pt}
    \pgfpathmoveto{\pgfqpoint{0pt}{3pt}}
    \pgfpathlineto{\pgfqpoint{3pt}{0pt}}
    \pgfpathmoveto{\pgfqpoint{.2pt}{-.2pt}}
    \pgfpathlineto{\pgfqpoint{-.2pt}{.2pt}}
    \pgfpathmoveto{\pgfqpoint{3.2pt}{2.8pt}}
    \pgfpathlineto{\pgfqpoint{2.8pt}{3.2pt}}
    \pgfusepath{stroke}}
    
\usepackage{stmaryrd}
\usepackage{wasysym}
\usepackage{multirow}
\usepackage{caption}
\usepackage{subcaption}
\usepackage{mathrsfs}
\usepackage{qtree}

\usepackage{linguex}


  %pminos do not split footnotes
% \interfootnotelinepenalty=10000 %Footnote in Laporte chapters has to be split SN


%\DeclareIndexNameFormat{default}{%
%\nameparts{#1}%
%\usebibmacro{index:name}%
%{\index[names]}%
%{\namepartfamily}%
%{\namepartgiveni}%
% {}% L1
% {}% L2
%{\namepartprefix}% generates spurious space L3
%{\namepartsuffix}% generates spurious space L4
%}

%  {\DeclareIndexNameFormat{default}{%
%     \usebibmacro{index:name}{\index[names]}{#1}{#3}{#5}{#7}}}

%\DeclareIndexNameFormat{default}{%
%  \usebibmacro{index:name}{\sindex[nom]}{#1}{#3}{#5}{#7}}

%\DeclareIndexNameFormat{default}{%
%  \usebibmacro{index:name}{\sindex[person]}{#1}{#3}{#5}{#7}}
%\DeclareIndexNameFormat{default}{%
%\nameparts{#1} \usebibmacro{index:name}{\sindex[person]]}{\namepartfamily}{‌​\namepartgiven}{\nam‌​epartprefix}{\namepa‌​rtsuffix}}

%\newcommand{\smiley}{:)}

%\renewbibmacro*{index:name}[5]{%
%\usebibmacro{index:entry}{#1}%
%{\iffieldundef{usera}{}{\thefield{usera}\actualoperator}\mkbibindexname{#2}{#3}{#4}{#5}}}

% \newcommand{\noop}[1]{}

%remove for final
%\overfullrule=1mm

\newcommand{\tobi}[2]}}
\renewcommand{\S}[1]{\tobi{#1}{\textsc{*}}}

% this volume references
% puts: [this volume]
% already defined: \citetv
%\newcommand{\citepv}[1]{(\citeauthor{#1} \citeyear*{#1} [this volume])}
\newcommand{\citealtv}[1]{\citeauthor{#1} \citeyear*{#1} [this volume]}

%parentheses around example number
\newcommand{\pref}[1]{(\ref{#1})}

% in-text examples

\newcommand{\lnex}[1]{\textit{#1}} %target lang word
\newcommand{\lnlit}[1]{(lit.: `#1')} %literal reading
\newcommand{\lnlat}[1]{(#1)} % latinization
\newcommand{\lntrans}[1]{`#1'} %translation
\newcommand{\lnexl}[2]%
{\lnex{#1}{} \lnlat{#2}} % ex with latinization
\newcommand{\lnexlat}[3]{\lnex{#1}{} \lnlat{#2}{} \lntrans{#3}} % ex with latinization and tranl.

%ch01
\newcommand{\co}[1]{\mbox{\textbf{#1}}}

%ch09

\newcommand{\cyrbulg}[1]{\begin{otherlanguage*}{bulgarian}#1\end{otherlanguage*}}


%ch10
\newcommand{\nlp}{{\small NLP}}
\newcommand{\mwe}{{\small MWE}}
\newcommand{\rae}{{\small RAE}}
\newcommand{\lvc}{{\small LVC}}
\newcommand{\pos}{{\small P}o{\small S}}
%\newcommand{\todo}[1]{ \textcolor{red}{#1} }

%\renewcommand{\labelenumi}{\theenumi}
%\ainamefmt{{vv}{ll}{, ff}{, jj}} % fullname

\newcommand{\biberror}[1]{{\color{red}#1}}

\newcommand{\osenovaitem}{--~}
  %% hyphenation points for line breaks
%% Normally, automatic hyphenation in LaTeX is very good
%% If a word is mis-hyphenated, add it to this file
%%
%% add information to TeX file before \begin{document} with:
%% %% hyphenation points for line breaks
%% Normally, automatic hyphenation in LaTeX is very good
%% If a word is mis-hyphenated, add it to this file
%%
%% add information to TeX file before \begin{document} with:
%% %% hyphenation points for line breaks
%% Normally, automatic hyphenation in LaTeX is very good
%% If a word is mis-hyphenated, add it to this file
%%
%% add information to TeX file before \begin{document} with:
%% \include{localhyphenation}
\hyphenation{
    Beck-man
    Ngu-yen
    back-chan-nel
    back-chan-nels
    mo-not-o-nous
    ste-reo-typ-i-cal
}

\hyphenation{
    Beck-man
    Ngu-yen
    back-chan-nel
    back-chan-nels
    mo-not-o-nous
    ste-reo-typ-i-cal
}

\hyphenation{
    Beck-man
    Ngu-yen
    back-chan-nel
    back-chan-nels
    mo-not-o-nous
    ste-reo-typ-i-cal
}

  % \togglepaper[3]%%chapternumber
}{}


\shorttitlerunninghead{Prosodic correlates of mirative and new information focus in wh-in-situ}
\begin{document}
\shorttitlerunninghead{Prosodic correlates of mirative and new information focus in wh-in-situ}
\maketitle

\section{Introduction}\label{sec:13:1}
This study compares the prosodic correlates of focus in two types of Spanish wh-in-situ questions: Information-seeking (INF), and echo-surprise (SUR). While the main strategy to formulate a wh-question in Spanish involves wh-fronting (\ref{13:ex:1a}), wh-in-situ questions are also possible in some dialects, such as in North-Central Peninsular Spanish (\ref{13:ex:1b}) (\citealp{Jiménez1997,Uribe-Etxebarria2002,EtxepareUribe-Etxebarria2005,Reglero2007,RegleroTicio2013}, among others).

\ea \label{13:ex:1}
\ea \gll \label{13:ex:1a}¿\textbf{Qué}  llevó               Rosalía?\\
what  wear.\textsc{pst}.3\textsc{sg} Rosalía\\
\glt ‘What did Rosalía wear?’\\
\ex \label{13:ex:1b}
¿Rosalía llevó \textbf{qué}?
\z
\z

The pragmatic meanings of wh-in-situ questions in Spanish are varied. A sentence such as (\ref{13:ex:1b}) can be interpreted as an information-seeking (INF) question eliciting information in a neutral way \citep{Reglero2007, RegleroTicio2013}. Alternatively, (\ref{13:ex:1b}) can be interpreted as an echo question, i.e., a question requesting repetition of information (echo-repetition, henceforth REP) or conveying surprise (echo-surprise, henceforth SUR) \citep{Chernova2013, Chernova2017, RegleroTicio2013}. Regardless of the pragmatic reading, the in-situ wh-element carries the main focus of the question \citep{Horvath1986, Rochemont1986, Tuller1992, Zubizarreta1998,escandellvidal1999}.

In this study, we follow \citet{Reglero2007} and \citet{RegleroTicio2013} in considering INF questions as having new information focus; and we argue, based on \citet{BadanCrocco2019}, that SUR questions in Spanish have mirative focus, which conveys counter-expectational value. Spanish INF and SUR questions display some syntactic differences, including differences in word order. In addition, impressionistic reports and a previous small-scale study suggest some intonational differences as well \citep{gonzalez2018dime}. In the present study, we investigate the prosodic characteristics of INF and SUR within a larger set of speakers, and connect these differences to focus, taking into consideration relevant studies from other Romance languages.

\largerpage
Our study is framed within the Auto-Segmental (AM) model of intonation \citep{pierrehumbert1980,PierrehumbertBeckman1988,Ladd2008}, which views intonation as the anchoring of High (H) and Low (L) tones to metrically strong syllables and edges of phonological domains. We follow the conventions of the Spanish ToBI prosodic annotation system \citep{BeckmanMorgan2002,Estebas-VilaplanaPrieto,PrietoRoseano2010,hualde2015}. Stressed syllables bear pitch accents, indicated with *. The pitch accent on the last main stress of an utterance is the nuclear pitch; other stressed syllables bear prenuclear accents (unless deaccented). Edges of phonological domains bear boundary tones. In Spanish, boundary tones occur at the end of full intonational phrases (IPs) and intermediate phrases (ips); these are indicated with \% and -, respectively \citep{AguilarPrieto2009}. Figure 1 below provides an example of prosodic annotation for a statement with narrow focus on the direct object. The final IP boundary is low (L\%); the intermediate ip shows a steep rise (HH-). All pitch accents are rising; but while the nuclear peak is aligned with the stressed syllable (L+H*), prenuclear peaks are delayed, i.e., aligned with the post-tonic syllable (L+>H*).


\begin{figure}
    \includegraphics[scale=.15]{figures/GR_FIG1.png}
    \caption{Example of Spanish ToBI annotation. Participant 15. \textit{El niño mira a su abuelo}  ‘The child looks at his grandfather’ (narrow focus)}
    \label{13:Fig1}
\end{figure}

The rest of this paper is organized as follows. \sectref{sec:13:2} contextualizes the study in connection to focus and reviews its main syntactic and prosodic characteristics. \sectref{sec:13:3} introduces the methodology of the study. \sectref{sec:13:4} presents the results, and \sectref{sec:13:5} is the discussion. Concluding remarks are provided in \sectref{sec:13:6}.
\section{Properties of focus}\label{sec:13:2}
\subsection{Focus types}
Focus, or the information center of a sentence \citep{chomsky1971,Chomsky1976}, is expressed cross-linguistically in one or more of three ways:  prosodically, as in English; morphologically, as in Japanese; and syntactically, as in Russian (\citealp{Gutiérrez-Bravo2008} and references therein). In Spanish, focus can be expressed prosodically and syntactically (\citealp{Zubizarreta1998,Face2006,Chung2012}, among others).

Focus can be defined according to its size as broad or narrow, and according to its meaning as new information (or presentational), contrastive, or mirative \citep{DeLancey1997, Ladd2008,gussenhoven2008}. Under broad focus, the entire sentence is focused; this occurs when the whole sentence provides non-presupposed, new information, as shown in (\ref{13:ex:2}). On the other hand, under narrow focus only one sentential element is focused (\ref{13:ex:3}). The question in (\ref{13:ex:3a}) expresses the presupposition that Adriana bought something (this is the old, given information, or the sentence topic) but the value of the wh-word is unknown. The direct object in (\ref{13:ex:3b}) has narrow focus, and supplies the value for the variable bound by the wh-word.

\ea \label{13:ex:2}
\ea\label{13:ex:2a}
\gll   ¿Qué  pasó?\\
what happen.\textsc{pst}.3\textsc{sg}\\
\glt ‘What happened?’\\
\ex   \label{13:ex:2b}
\gll $[$\textsubscript{\textsc{focus}} Adriana compró un libro.$]$\\
{} Adriana  buy.\textsc{pst}.3\textsc{sg} a   book\\
\glt ‘Adriana bought a book.’  \\
\z
\z

\ea \label{13:ex:3}%Ex 3
\ea \label{13:ex:3a}
\gll  ¿Qué compró Adriana?\\
what buy.\textsc{pst}.3\textsc{sg} Adriana\\
\glt ‘What did Adriana buy?’\\
\ex  \label{13:ex:3b}
\gll Adriana compró         $[$\textsubscript{\textsc{focus}} un libro$]$.\\
Adriana  buy.\textsc{pst}.3\textsc{sg} {} a   book\\
\glt ‘Adriana bought a book.’  \\
\z
\z
Regarding meaning, new information focus corresponds to the non-presupposed part of the sentence \citep{Zubizarreta1998, chomsky1971, Chomsky1976, Jackendoff1972}, while contrastive focus negates the value assigned to a specific variable and provides a different value for it \citep{Zubizarreta1998}. On the other hand, mirative focus conveys surprise from unexpected information, has counter-expectational value, and transmits expressive and emotive attitude \citep{machuca_ayusoyRios2017,DeLancey1997,DeLancey2001,DeLancey2012,Dickinson2000,Cruschina2012, gili2015intonational, Jiménez-Fernández2015a, Jiménez-Fernández2015b,BianchiCruschina2016,BadanCrocco2019,belletti2017syntax}. The syntactic and prosodic characteristics of these focus types are reviewed next.
\subsection{Syntactic properties}
In Spanish, new information focus needs to appear as the rightmost element in the linear string to receive nuclear stress, i.e., to be assigned the main sentence prominence \citep{Zubizarreta1998,Gutiérrez-Bravo2008,López2009}. Using the question-answer test, (\ref{13:ex:4b}) is ungrammatical as an answer to (\ref{13:ex:4a}) because the new information focus \textit{un libro} ‘a book’ does not appear sentence-finally. In contrast, (\ref{13:ex:4c},\ref{13:ex:4d}) constitute valid answers since the focus appears in the rightmost position (note that in (\ref{13:ex:4d}), the pause -- indicated with \# -- effectively makes \textit{un libro} ‘a book’ rightmost in the linear string). \footnote{We follow \citegen{Zubizarreta1998} original intuitions here, but see \citet{ortegasantos2016} for a review of current experimental work that shows dialectal variation in the judgments (for example, in Argentinian Spanish, Mexican Spanish or Southern Iberian Spanish). As discussed by \citet{Jiménez-Fernández2015a}, Southern Peninsular Spanish has a specific position in the left periphery for new information focus in contrast to Standard Spanish (this includes speakers from Northern Spain and Madrid).}

\ea \label{13:ex:4}%Ex 4
\ea   \label{13:ex:4a}
\gll ¿Qué compró Adriana en la librería?\\
what buy.\textsc{pst}.3\textsc{sg}  Adriana in the bookstore\\
\glt ‘What did Adriana buy at the bookstore?’\\

\ex \gll * Adriana compró $[$\textsubscript{\textsc{focus}} un libro$]$ en la librería. \\ \label{13:ex:4b}
{}Adriana buy.\textsc{pst}.3\textsc{sg} {} a   book in the bookstore\\
\glt ‘Adriana bought a book at the bookstore.’  \\

\ex   \label{13:ex:4c}
\gll Adriana compró en la librería   $[$\textsubscript{\textsc{focus}} un libro$]$.\\
Adriana  buy.\textsc{pst}.3\textsc{sg} in the bookstore {} a book\\
\glt ‘Adriana bought a book at the bookstore.’  \\

\ex  \label{13:ex:4d}
\gll Adriana compró         $[$\textsubscript{\textsc{focus}} un libro$]$ \# en la librería.\\
Adriana  buy.\textsc{pst}.3\textsc{sg} {} a  book {} in the bookstore\\
\glt ‘Adriana bought a book at the bookstore.’  \\
\z
\z

Contrastive focus differs from new information focus in regards to word order; any element in the sentence can be contrastively focused, regardless of sentence position \citep{Zubizarreta1998}. One contextualized example is given in (\ref{13:ex:5}).\footnote{Here and throughout, capitalization is used to indicate elements with contrastive or mirative focus.} \\

\ea Contrastive statement \label{13:ex:5}%Ex 5
\ea  \label{13:ex:5a}
\gll ¿Qué compró Adriana?\\
what buy.\textsc{pst}.3\textsc{sg} Adriana\\
\glt ‘What did Adriana buy?’\\
\ex  \label{13:ex:5b}
\gll Adriana compró         $[$\textsubscript{\textsc{focus}} un LIBRO$]$ en la  librería      (no una revista).\\
Adriana buy.\textsc{pst}.3\textsc{sg} {} a book at the bookstore    not a magazine\\
\glt ‘Adriana bought a BOOK at the bookstore (not a magazine).’  \\
\z
\z

\citet{Jiménez-Fernández2015b} points to syntactic differences between contrastive and mirative focus (in the context of focus fronting). While contrastive focus can occur in an embedded sentence as a complement of a verb of saying (\ref{13:ex:6}), mirative focus is disallowed in this context (\ref{13:ex:7}) (this property was originally discussed by \citet{Cruschina2012} for Italian):\footnote{For a discussion on verb adjacency and its interaction with contrastive and mirative focus, see  \citet{Jiménez-Fernández2015b}.} \\

\ea Contrastive statement \label{13:ex:6}%Ex 6
\ea
\gll Juan va diciendo que  ha vendido la moto.\\
John go-\textsc{pres}.3\textsc{sg} say.\textsc{ger}    that have-\textsc{pres}.3\textsc{sg} sell.\textsc{ptcp}      the motorbike\\ \jambox{\citep[53]{Jiménez-Fernández2015b}}
\glt ‘John goes saying that he has sold the motorbike.’\\
\ex
\gll No, no. María dice              que el coche ha   vendido, no la    moto.\\
no   no  Mary say.\textsc{pres}.3\textsc{sg} that the car    have.\textsc{pres}.3\textsc{sg} sell.\textsc{ptcp}      not the motorbike\\ \jambox{\citep[53]{Jiménez-Fernández2015b}}
\glt ‘No, no. Mary says that he has sold the car, not the mortorbike.’  \\
\z
\z



\ea  \label{13:ex:7}%Ex 7
[??]{
\gll ¡¡No me lo puedo           creer!! ¡¡Va diciendo por ahí  que DOS BOTELLAS DE VODKA nos habíamos       bebido en la fiesta!!\\
not \textsc{cl}   it can.\textsc{pres}.1\textsc{sg} believe.\textsc{inft} go.\textsc{pres}.3\textsc{sg} say.\textsc{ger}   by there that  two bottles         of   vodka       \textsc{cl}  have.\textsc{pst}.1\textsc{pl} drink.\textsc{ptcp}   in the party\\ \jambox{\citep[53]{Jiménez-Fernández2015b}}
\glt ‘I can’t believe it! He goes saying everywhere that we had drunk TWO BOTTLES OF VODKA at the party!!’\\
}
\z


As mentioned in \sectref{sec:13:1}, in-situ wh-elements carry the main focus of a question \citep{Horvath1986,Rochemont1986,Tuller1992,Zubizarreta1998,escandellvidal1999}. \citet{Reglero2007} and \citet{RegleroTicio2013} argue that wh-phrases in INF questions have new information focus\footnote{See \citet{Uribe-Etxebarria2002} for a proposal in which in situ wh-questions in Spanish have contrastive focus. This is primarily based on a more restricted interpretation of wh-in-situ in Spanish (at least according to \citegen{Jiménez1997} intuition). Uribe-Etxebarria provides additional examples and a syntactic analysis that relates the interpretative properties of wh-in-situ in Spanish to their syntactic derivation.}  since they elicit non-presupposed information (i.e., the value of the wh-word is unknown; see (\ref{13:ex:3}), (\ref{13:ex:4})), and can also appear in out-of-the-blue contexts. One example is given in (\ref{13:ex:8}), where the question is introduced by \textit{dime una cosa} ‘tell me something’, a phrase eliciting new information.\footnote{This test is attributed to Ignacio Bosque (p. c.) \citep{RegleroTicio2013}. See also \citet{gonzalez2018dime,gonzalez_reglero2020}.}  In addition, the in situ wh-phrase needs to appear finally (\ref{13:ex:8b} -- \ref{13:ex:8d}) (see (\ref{13:ex:4}) above).\\

\ea \label{13:ex:8}%Ex 8
\ea  \label{13:ex:8a}  \gll Dime una cosa: ¿Rosalía llevó  qué?\\
tell.\textsc{imp}-\textsc{cl}.\textsc{dat}.1\textsc{sg} one thing   Rosalía wear.\textsc{pst}.3\textsc{sg}  what\\
\glt ‘Tell me something: What did Rosalía wear?’\\

\ex  \label{13:ex:8b}
\gll ¿Tú le diste el libro a quién? \\
you \textsc{cl}.\textsc{dat}.3\textsc{sg} give.\textsc{pst}.2\textsc{sg} the book to who\\
\glt ‘Who did you give the book to?’\\

\ex[??]{¿Tú le diste a quién el libro?\\} \label{13:ex:8c}


\ex  ¿Tú le diste a quién \# el libro? \label{13:ex:8d}
\z
\z

For SUR questions, \citet{RegleroTicio2013}  have argued that the wh-phrase has contrastive focus\footnote{Their claim applies to REP echo questions as well.}  since the echo wh-phrase does not need to appear finally (\ref{13:ex:9}) (see (\ref{13:ex:5}) above). In addition, SUR requires heavy contextualization, unlike INF (\ref{13:ex:10}).\\

\ea \label{13:ex:9}%Ex 9
 \gll ¿Rosalía llevó              QUÉ ayer?\\
Rosalía  wear.\textsc{pst}.3\textsc{sg} what yesterday\\
\glt ‘Rosalía wore WHAT yesterday?’\\
\z


\ea \label{13:ex:10} %Ex10
{Speaker 1:} \\
\gll Adela fue a visitar a Aristóteles.\\
Adela go.\textsc{pst}.3\textsc{sg}  to visit.\textsc{inft} \textsc{dom}  Aristotle\\
\glt ‘Adela went to visit Aristotle’\\

\sn {Speaker 2:}\\
\gll ¡No  me lo puedo creer!: ¿Adela    fue a visitar a QUIÉN?\\
\textsc{neg} \textsc{cl}.1\textsc{sg}  \textsc{cl}.\textsc{acc}.3\textsc{sg}  can.1\textsc{sg}    believe.\textsc{inft}  Adela    go.\textsc{pst}.3\textsc{sg}  to visit.\textsc{inft} \textsc{dom}  who\\
\glt ‘I can’t believe it! Adela went to visit WHO?’\\
\z

However, recent work on Italian argues that SUR in this language is associated with Mirative Focus (MirF) \citep{crocco2016,BadanFiori2017,BadanCrocco2019}. MirF is a type of focalization involving surprise and unexpectedness. For Italian in-situ questions, MirF and INF have different syntactic properties: the most obvious one is that INF needs to be fronted, unlike SUR (\ref{13:ex:5a},\ref{13:ex:5b}).\footnote{See \citet{BadanCrocco2019} for additional differences in embedded contexts (related to question availability and scope). They propose overt movement of the wh-phrase to a low focus position (MirF) in echo questions.} Unlike INF, the wh-phrase in Italian SUR is D-linked to a previous discourse. Both types of questions also show prosodic differences, as discussed in the following section.\\

\ea \label{13:ex:11}%Ex11
\ea \gll Dove vendono le mandorle?\\
where sell.\textsc{prs}.3\textsc{pl}  the almonds\\ \jambox{\citep[47]{BadanCrocco2019}}
\glt ‘Where do they sell the almonds?’\\
\ex \gll  	Le vendono DOVE le mandorle?\\
\textsc{cl}.\textsc{obj}.3\textsc{pl} sell.\textsc{prs}.3\textsc{pl}  where  the almonds\\ \jambox{\citep[47]{BadanCrocco2019}}
\glt ‘They sell (them) where the almonds?’\\
\z
\z



\subsection{Prosodic properties}
Prosodically, focused constituents tend to stand out over topics. As in many languages, in Spanish, focused elements can constitute separate intonational units \citep{Gutiérrez-Bravo2008}. A high intermediate boundary tone (H- or HH-) can occur between the old (topic) and new information (focus) (\citealp[268--270]{Hualde2014}; \citealp[369]{hualde2015}). In addition, non-focal elements tend to have reduced pitch range \citep[see for example][]{delamota1997, Face2002a}.

The realization of both prenuclear (non-final) and nuclear accents tends to differ in broad and narrow focus statements. In Madrid Spanish, pre-nuclear accents tend to have a higher pitch under narrow focus and/or be aligned with the stressed syllable, unlike under broad focus, where the peak tends to be displaced to the post-tonic \citep[][]{Face2001}. Stressed syllables are also longer under narrow focus in this dialect \citep{Face2000}. In Castilian Spanish, nuclear accents tend to have a low pitch accent (L*) under broad focus, and rising (L+H*) under narrow focus \citep{Estebas-VilaplanaPrieto}. However, in Spanish contact varieties, including in contact with Basque, prenuclear accents tend to have earlier peaks under broad focus, as well \citep{Elordieta2003,orourke2012}.

There are also prosodic differences between contrastive and new information focus. The former is characterized by higher pitch, expanded pitch range and/or earlier pitch alignment compared to the latter, at least in statements. In addition, an intermediate high or low boundary (H-, L-) can follow the contrastively focused constituent \citep{delamota1997,Face2002a,Face2002b}. Contrastive focus shows longer duration than new information focus sententially, in the focal constituent, and in its stressed syllable \citep{Chung2012}. However, sentence-finally elements in narrow focus appear to have similar pitch height and show early peak alignment, unlike in statements with broad focus, where late alignment is more frequent \citep{Domínguez2004}.

As mentioned in the previous section, wh-in-situ elements in Spanish are focused and are assigned nuclear stress since they are located at the end of the intonational phrase. The rest of the sentence is the topic since the information is presupposed. Impressionistic reports on the prosody of INF questions mention falling intonation and extra or ``marked'' stress (\citealp[63]{escandellvidal1999}; \citealp[]{Uribe-Etxebarria2002,RegleroTicio2013}). On the other hand, in situ-echo questions, particularly those conveying surprise, reportedly display (falling)-rising or sharp/strong intonation and have marked stress on the wh-phrase \citep{contreras1999, Pope1976, escandellvidal1999, Sobin2010, Chernova2013, Chernova2017}.

A preliminary investigation of wh-in-situ questions in four participants of North-Central Peninsular Spanish shows that INF have final rising intonation more often than SUR. The latter show an expanded sentential tonal range, and a substantially higher final High pitch compared to INF. On the other hand, the duration ratio of the wh-element (i.e, its duration relative to the sentence duration) is larger in INF than in SUR \citep{gonzalez2018dime}. These preliminary findings contradict the falling/falling-rising distinction previously reported for INF and SUR, but suggest that marked stress in INF is a perceptual result of increased duration of the wh-element, while sharp/strong intonation in SUR is related to expanded scaling and an elevated final pitch accent/boundary tone \citep[for stress correlates in Spanish, see][]{ortega2007, Ortega-LlebariaPrieto2011,Hualde2014}.

These preliminary results are also in line with other studies investigating intonational differences in pragmatic meaning for Spanish questions. For example, fronted wh-questions with a counter-expectational value have expanded pitch ranges compared to neutral questions. This difference usually goes hand in hand with a difference in boundary tone \citep[Argentinian Spanish:][]{gabriel2010} or nuclear configuration (\citealp[Peninsular Spanish:][]{Estebas-VilaplanaPrieto}; \citealp[][374]{hualde2015}; \citealp[Mexican Spanish:][]{delamota2010}; \citealp[Venezuelan Spanish:][]{astruc2010}).\footnote{In Ecuadorian Spanish, pitch range exclusively distinguishes between the two \citep{HuttenlauchFeldhausen2016}.} \footnote{A similar prosodic combination is also reported in Catalan and Italian \citep{gili2015intonational, prieto2015}.}  In addition, although Castilian Spanish echo-questions tend to be realized with upstepped rising nuclear accents (L+¡H*), those with a counter-expectational value tend to have a sharp final rise (HH\%) instead of a low boundary tone (L*) \citep[][]{Estebas-VilaplanaPrieto}.\footnote{In Brazilian Portuguese, neutral INF questions have falling intonation, while echo ones are rising \citep{kato2019}.}

\largerpage
Although earlier work considers that surprise echo questions have contrastive focus in Spanish, recent work on Italian suggests that mirative focus is involved since SUR questions have counter-expectational value \citep{BadanCrocco2019}. In addition to showing clear syntactic differences, SUR wh-in-situ questions in Italian are different prosodically from INF questions in several respects. First, the wh-phrase carries the main prominence of the sentence in SUR but not in INF contexts, where the main prominence falls on the verb. Second, the wh-phrase in SUR shows expanded scaling and has an upstepped rising pitch accent (L+¡H*); in comparison, INF questions have falling pitch accents, which are closely aligned with the verb. Finally, SUR questions have a high boundary tone after the wh-element and a clearly perceived disjuncture with the rest of the question. In contrast, in INF, the verb is followed by a low boundary tone, and a clear disjuncture is not typically perceived.
Assuming that INF have new information focus and SUR mirative focus, we explore the intonational properties of both question types to elucidate the prosodic characteristics of both types of focus. We examine data from 14 speakers of North-Central Peninsular Spanish, where non-fronted wh-in-situ questions can have a new information reading, in addition to echo readings. Two specific hypotheses are investigated: First, if Spanish INF and SUR have different foci, they will have distinct prosodic properties. Second, if SUR have MirF, they will differ from INF in one or more of the following: (i) intonational contour, (ii) pitch range, and/or (iii) F0 (\citealp{crocco2016,HuttenlauchFeldhausen2016,BadanFiori2017,machuca_ayusoyRios2017,BadanCrocco2019}, among others).

\section{Methodology}\label{sec:13:3}
\subsection{Participants and data collection}
Our participants are Spanish speakers from the Basque Country in northern Spain. Although bilingualism in Spanish and Basque is prevalent, and language contact with Basque influences some prosodic characteristics of Spanish in this area \citep{elordieta2020}, the impact of language contact is considered to be minimal or non-existing for this study  since Basque does not allow in-situ information or surprise echo questions \citep{EtxepareOrtiz2003, reglero2003non}.\footnote{Echo wh-questions in Basque are usually preverbal \citet{EtxepareOrtiz2003}, as shown in the example below:
\ea
\ea
\gll Zugandik  atera  dira kontu  zikin guzti horiek.\\
you.from  come  \textsc{aux} stories dirty  all     those\\
\glt‘All those dirty stories have come from you’\\

\ex
\gll Nigandik ZER    atera    dela?\\
me.from  what  come    \textsc{aux}.that\\
\glt‘(That) what has come from me?’ \\
\z
\z
\citet[][463]{EtxepareOrtiz2003} indicate that echo wh-questions with corrective/contrastive focus can appear finally with a preceding prosodic break; these are quite marked. \citet{DuguineIrurtzun2014} indicate that young Laubordin Basque speakers use an innovative strategy involving wh-in situ. None of the participants in our study come from this dialectal area.}

Data was collected in Summer 2015 in Bilbao, Spain. Participants completed two tasks: a reading task, and a controlled elicitation task. Both were facilitated via a powerpoint that included visual and auditory stimuli to provide contextual information to engage participants in the task and prompt the relevant pragmatic reading. Both tasks were designed to control the context and therefore the pragmatic reading of the stimuli. The reading task is most similar to the methodology employed in other intonational studies of Spanish, including \citet{PrietoRoseano2010} and \citet{Rao2013} and can be conceived of as involving ``scripted speech''. The controlled elicitation task, which we focus on in this paper, did not include a written script for participants to read from, and was designed to provide a more naturalistic realization of the stimuli.

The completed experiment took approximately one hour per participant. A total of 22 Spanish participants took part in the experiment; all were paid for their participation. Participants had varied degrees of Spanish-Basque bilingualism. Before the tasks, all participants completed a consent form and the Bilingual Linguistic Profile \citep[BLP;][]{birdsong2012bilingual} to obtain information on the language history, use proficiency, and attitudes towards Spanish and Basque. For this study, we report data from the elicitation data from 14 participants; all were 21--24 years old females from the province of Bizkaia.

\tabref{13:table1} provides additional participant information. Positive BLP dominance scores indicate Spanish dominance; scores close to zero indicate balanced bilingualism. Negative dominance scores indicate that participants are Basque dominant. Only three participants have negative dominance scores (P3, P7, P14); two of them are close to zero (P14, P7).\\

\begin{table}
\caption{Participant information: Procedence and BLP scores}
\label{13:table1}
 \begin{tabular}{lrrrl}
  \lsptoprule
ID & BLP Score & Spanish Score & Basque Score & Town\\
  \midrule
P15  & 168       & 199           & 31           & Santurtxi    \\
P11  & 155       & 190           & 35           & Leioa        \\
P9   & 123       & 209           & 86           & Trapagaran   \\
P22  & 85        & 161           & 76           & Galdakao     \\
P8   & 80        & 177           & 97           & Bilbao       \\
P1   & 76        & 201           & 125          & Leioa        \\
P5   & 51        & 182           & 131          & Sopelana     \\
P21  & 49        & 178           & 129          & Barakaldo    \\
P4   & 38        & 201           & 163          & Sopelana     \\
P20  & 26        & 180           & 154          & Galdakao     \\
P13  & 14        & 176           & 162          & Sopelana     \\
P14  & $-$2        & 170           & 172          & Arrankudiaga \\
P7   & $-$5        & 188           & 193          & Durango      \\
P3   & $-$40       & 159           & 199          & Gorliz       \\
  \lspbottomrule
 \end{tabular}
\end{table}

The target sentences for the elicitation task involved fronted and in-situ wh-questions, statements, and yes-no questions. Here we focus on in-situ SUR and INF questions. Contextualized examples are provided below; note that all participants completed a short practice before the tasks, and that the context and prompt were presented aurally (not in written form).

\ea %Ex12
\ea Context/Prompt: \\
\textit{Maite, Cristina, y Elena se han puesto a jugar al escondite con una amiga. Maite se ha escondido detrás de un árbol. Cristina detrás de un arbusto. Para preguntar por Elena una posibilidad sería decir: ¿y dónde se ha escondido Elena? ¿Cuál sería la otra manera de decirlo?}\\
‘Maite, Cristina and Elena are playing hide-and-seek with a friend. Maite hid behind a tree. Cristina hid behind a bush. To ask about Elena, one possibility would be to say: And where did Elena hide? What would be another way to ask this question?’\\
\ex
Expected target question:\\
\gll ¿(Y) Elena se ha escondido dónde?\\
and Elena \textsc{cl}.\textsc{refl} have.\textsc{prs}.3\textsc{sg} hide.\textsc{ptcp}  where \\
\glt ‘(And) where did Elena hide?’\\
\z
\z

\ea %Ex13
\ea SUR question: \\
\textit{Estás en la casa de una amiga y te enseña sus mascotas. Te dice: “El gato se llama Macacocogito.” Te sorprende muchísimo el extraño nombre de su gato. Hazle una pregunta para comprobar cómo se llama.}\\
‘You are at your friends’ house, and she shows you her pets. She says: “My cat’s name is Macacocogito”. You are completely surprised by the cat’s unusual name. Ask your friend a question to double-check the cat’s name.’\\
\ex
Expected target question:\\
\gll ¿El gato se        llama               C\'OMO?\\
the cat \textsc{cl}.\textsc{refl}  name.\textsc{prs}.3\textsc{sg} how \\
\glt ‘The cat’s name is WHAT?’\\
\z
\z

\subsection{Recording and coding}

Recording was conducted via a Tascam DR-05 digital recorder with built-in omni-directional microphones. Audio was recorded in 44,000 Hz in mono. 10 INF questions and 10 SUR questions were examined per participant for a total of 280 target sentences. Eight INF and six SUR questions had to be discarded because of waveform distortion and/or wh-fronting, leaving 266 sentences for the acoustic analysis.

Data was coded in Praat \citep{praat} according to Spanish ToBI conventions (\citealp{AguilarPrieto2009,FacePrieto2007} inter al.). Both authors were involved in the acoustic analysis. Disagreements, which occurred in approximately 5\% of the tokens, were resolved by consensus. The analysis focused on the following characteristics: (i) the overall melodic shape of the question, (ii) its nuclear configuration, (iii) the nuclear peak (in Hz.), and (iv) the focal tonal range (FTR), i.e., the difference between the lowest point at the beginning of the wh-phrase and its highest pitch. Pitch is reported in Hz and semitones (ST); the latter helps normalize the data and is more closely related to pitch perception. Specifically, a difference of 1.5 ST meets the perceptual threshold, i.e., it is considered to be perceivable by all speakers \citep{T’hart1981,Toledo2000,Pamies2002}. Paired two-tailed t-tests were conducted to establish whether these results are statistically significant.

\figref{13:Fig2}--\ref{13:Fig5} below provide examples of melodic contours for INF and SUR. \figref{13:Fig2} exemplifies the most frequent INF contour; it begins with an initial fall followed by a rise up to the first post-tonic syllable, which diphthongizes with the auxiliary verb to its right. Declination follows up to the beginning of the wh-question, realized with a steep final rise (L+H* HH\%). The FTR is 183 Hz, equivalent to 10.7 ST.

\begin{figure}
    \includegraphics[scale=.15]{figures/GR_FIG2.png}
    \caption{INF question. P21\_12  ‘And when has the third one gone out?’}
    \label{13:Fig2}
\end{figure}

\figref{13:Fig3} exemplifies an additional melodic pattern for INF, which starts with a slight initial fall up to the post-tonic syllable, followed by a slight rise on the verb \textit{fue}. Declination ensues, and the wh-question shows a final rise-fall (L+H* L\%). The FTR is 158 Hz, equivalent to 9.8 ST.

\begin{figure}
    \includegraphics[scale=.15]{figures/GR_FIG3.png}
    \caption{INF question. P15\_4  ‘And where did Marian go?’}
    \label{13:Fig3}
\end{figure}

Figure \ref{13:Fig4} shows a third melodic pattern for INF in our data, involving a rise up to the wh-word, followed by a final fall-rise (H+L* LH\%). The FTR is 89 Hz, equivalent to 7.4 ST.

\begin{figure}
    \includegraphics[scale=.15]{figures/GR_FIG4.png}
    \caption{INF question. P18\_13  ‘And how does Alejandra go up?’}
    \label{13:Fig4}
\end{figure}

SUR questions were realized similarly across participants. They involved an initial rise up to the first post-tonic syllable, declination up to wh-question, and a steep final rise (Figure~\ref{13:Fig5}). The nuclear configuration can be characterized as L+H* HH\%, as in Figure~\ref{13:Fig2}. The FTR is 191 Hz (11.6 ST).

\begin{figure}
    \includegraphics[scale=.15]{figures/GR_FIG5.png}
    \caption{SUR question. P3\_3  ‘The cat’s name is WHAT?’}
    \label{13:Fig5}
\end{figure}

\section{Results}\label{sec:13:4}
\subsection{Overall melodic contour}
All SUR questions in our dataset show three intonational movements: (i) a rise through the first post-tonic syllable; (ii) declination (i.e., pitch lowering) up to the wh-phrase, and (iii) a steep final rise (\figref{13:Fig5}). For INF questions, a similar pattern occurs in 85\% of cases, although an additional fall is usually present at the beginning (\figref{13:Fig2}). This fall occurs in cases where INF began with \textit{y} ‘and’, a pragmatic strategy available in INF questions to establish a transition between the previous discourse and the wh-in-situ question \citep{Jiménez1997}. Two additional melodic contours are attested for INF: one characterized by a final rise-fall (7.5\%) (\figref{13:Fig2}), and another with an overall rise up to the beginning of the wh-phrase followed by a nuclear fall-rise (7.5\%) (\figref{13:Fig4}). Most of these less frequent patterns are found in speakers 15 and 8, respectively.

\subsection{Nuclear configuration}
All SUR questions and most INF questions end in a high (HH\%) boundary tone. The main exceptions are participant 15, showing a low boundary tone (L\%) in 60\% of INF, and participant 8, with a rising (LH\%) boundary tone in 50\% of INF questions. Low or rising boundary tones are also found sporadically in participants 3, 7 and 13.

The realization of the nuclear accent is more variable. \tabref{13:table2} provides more information about the dominant nuclear configuration and its frequency per participant and type of question investigated. It can be observed that 10 of the participants analyzed (71\%) show similar nuclear pitch accents in both INF and SUR: five of them have a rising nuclear pitch accent (L+H*), and five show a low nuclear pitch accent (L*).

The four remaining participants have different nuclear pitch accents in INF and SUR. Three of the participants (P7, 8, 13) have a low or falling pitch accent (H+L*) in INF questions, and a rising pitch accent in SUR questions. Participant 15 shows a preference for a rising pitch accent in INF (L+H*), and a low pitch accent (L*) in SUR. As stated above, this participant tends to realize low or rising boundary tones in INF questions.

\begin{table}
\caption{Nuclear configurations}
\label{13:table2}
 \begin{tabular}{lrlrlr}
  \lsptoprule
ID & BLP Score & INF       & \%    & SUR       & \%    \\
  \midrule
P15         & 168       & L+H* L\%  & 60\%  & L* HH\%   & 90\%  \\
P11         & 155       & L* HH\%   & 80\%  & L* HH\%   & 60\%  \\
P9          & 123       & L* HH\%   & 60\%  & L* HH\%   & 70\%  \\
P22         & 85        & L+H* HH\% & 80\%  & L+H* HH\% & 89\%  \\
P8          & 80        & H+L* HH\% & 50\%  & L+H* HH\% & 70\%  \\
            &           & H+L* LH\% & 50\%  &           &       \\
P1          & 76        & L+H* HH\% & 100\% & L+H* HH\% & 80\%  \\
P5          & 51        & L+H* HH\% & 90\%  & L+H* HH\% & 100\% \\
P21         & 49        & L+H* HH\% & 100\% & L+H* HH\% & 60\%  \\
P4          & 49        & L* HH\%   & 100\% & L* HH\%   & 100\% \\
P20         & 38        & L* HH\%   & 55\%  & L* HH\%   & 80\%  \\
P13         & 14        & L* HH\%   & 89\%  & L+H* HH\% & 100\% \\
P14         & $-$2        & L+H* HH\% & 70\%  & L+H* HH\% & 100\% \\
P7          & $-$5        & L* HH\%   & 67\%  & L+H* HH\% & 100\% \\
P3          & $-$40       & L* HH\%   & 88\%  & L* HH\%   & 70\%\\
  \lspbottomrule
 \end{tabular}
\end{table}

There is no apparent correlation with bilingualism; the patterns showed by Basque dominant speakers P3, P7 and P14 are variable and comparable to those attested in Spanish dominant participants.

\subsection{Nuclear high}
\figref{13:Fig6} shows the values of the nuclear High for all participants in INF and SUR. Eleven participants (79\%) have a more elevated H in SUR. On average, the value of H in SUR contexts is +2.1 ST higher than in INF questions. This difference is above the perceptual threshold, suggesting that it is perceptually significant. Results from a paired two-tailed t-test indicate that this difference is statistically significant (\textit{p} = 0.0038). The examination of individual differences shows that the perceptual threshold is reached or surpassed in 8 of the participants. The remaining three participants do not follow this trend. Specifically, participants P9 and P11 have a more elevated nuclear High in INF contexts, while P5 shows a similar nuclear High in both pragmatic readings (Appendix~\ref{13:app:1:NHigh}).\\

\begin{figure}
    \includegraphics[width=\textwidth]{figures/GR_Fig6.png}
    \caption{Nuclear High in INF and SUR questions}
    \label{13:Fig6}
\end{figure}

\subsection{Focal tonal range}
\figref{13:Fig7} shows a box plot for the focal tonal range of INF and SUR questions for all participants pooled. It can be observed that the medians of INF and SUR are very different. On average, the FTR for SUR is +2.9 ST higher than for INF, well above the perceptual threshold. In addition, results from a paired two-tailed t-test indicate that this difference is statistically significant (\textit{p} < 0.001). The examination of individual differences shows that this perceptual difference holds for 11 participants. For participant P9, this difference approaches the perceptual threshold (1.4 ST.). Two participants do not follow this trend: P11, which has a higher FTR in INF, and P13, which has a similar FTR in both INF and SUR (Appendix~\ref{13:app:2:FTR}).

\begin{figure}
    \includegraphics[width=\textwidth]{figures/GR_Fig7.png}
    \caption{Focal Tonal Range in INF and SUR questions}
    \label{13:Fig7}
\end{figure}

\section{Discussion}\label{sec:13:5}
The present study set out to investigate the prosodic characteristics of two types of pragmatically different wh-in-situ questions in Spanish: those requesting new information (INF), and those expressing surprise (SUR). Both share some syntactic similarities, since the wh-in-situ phrase is sentence-final. Our analysis reveals some prosodic similarities as well: the general melodic contour tends to be similar for both in most speakers, generally comprising an initial rise, medial declination, and a steep final rise on the wh-question. In addition, a high (HH*) final boundary tone tends to be present in both question types.

Syntactically and pragmatically, INF and SUR also show some differences. INF are neutral and restricted to the rightmost position in the linear string, while SUR are counter-expectational and have a less restricted distribution. Prosodically, we find some differences as well: the nuclear High is significantly more elevated in SUR, and the focal tonal range is significantly expanded. A difference in FTR occurs in most participants, suggesting that this is the main prosodic cue distinguishing SUR from INF in this Spanish variety. We don’t observe differences according to degree of Basque/Spanish bilingualism. This is expected since, although language contact impacts the realization of some prosodic features in both languages \citep[see for example][]{Elordieta2003}, the wh-in-situ questions investigated here for Spanish are not grammatical in Basque \citep{EtxepareOrtiz2003, reglero2003non}.

The intonational properties identified in this study for Spanish SUR are comparable to those reported for Italian SUR questions \citep{BadanCrocco2019}. At first blush, unlike for Italian, the nuclear configurations of the wh-in-situ phrase in Spanish INF and SUR are similar, as in German, where the tonal contours of INF and SUR are reportedly the same \citep{ReppRosin2015}. However, we argue that Spanish INF and SUR have distinct nuclear contours: INF is most frequently realized with a rising nuclear accent (L+H*), while SUR involves upstepping (L+¡H*). The difference between these two tonal configurations is reportedly one of pitch range, as shown schematically in \figref{13:Fig8}. Upstepped rising nuclear accents are attested in Italian SUR \citep{BadanCrocco2019} and in Spanish counter-expectational questions (\citealp{AguilarPrieto2009,Estebas-VilaplanaPrieto};\citealp[][374]{hualde2015}).

\begin{figure}
    \includegraphics[scale=.95]{figures/GR_Fig8.png}
    \caption{Rising vs. upstepped rising pitch accents  \citep{AguilarPrieto2009}}
    \label{13:Fig8}
\end{figure}

The participants in our dataset have different degrees of Basque/Spanish bilingualism. We have not observed prosodic differences consistent with Spanish vs. Basque language dominance. Three participants (P5, P9, P13) show individual variation, with either an elevated nuclear peak or a higher FTR in SUR questions, but not both. Only P11 appears to be exceptional since she shows higher F0 and expanded FTR in INF than in SUR, unlike the rest of the participants. We leave open the possibility that low-statistical power and/or individual variation explains this different pattern.
\section{Conclusion}\label{sec:13:6}
This study has focused on the intonation of INF and SUR questions in Spanish. Results from an elicitation task in 14 female speakers from North-Central Peninsular Spanish show similarities in overall melodic contours and final boundary tones, but also differences in the height of the nuclear accent, the focal tonal range, and the nuclear pitch accent. We argue, following \citet{BadanCrocco2019} for Italian, that these differences are consistent with a difference between new information and mirative focus.

\hspace*{-.2pt}The analysis of intonation from the five male speakers remaining in our dataset and from the reading task will be relevant to further ascertain the patterns reported here and to inquire into possible gender differences in the intonation of wh-in-situ questions in Spanish. Future studies should investigate additional correlates of focus, including the presence of intermediate boundaries before the wh-element, wh-phrase duration and intensity, and the realization of pre-nuclear peaks (\citealp{Chung2012,Face2001,Face2002b, gryllia2016} inter alia).

We also would like to note that the investigation of SUR questions in French would be of great interest to further elucidate the prosodic properties of MirF in Romance. \citet{Glasbergen-PlasDoetjes2020} show that INF and repetition (REP) in-situ questions have similar tonal contours in French; however, REP wh-questions have extended pitch scaling and longer duration (cf. \citealp{DéprezKawahara2013,ChengRooryck2000, gryllia2016}. Based on our current understanding of in-situ questions in Italian and Spanish, we consider it extremely likely that French SUR in French will have even wider scaling than REP, and/or might involve a different tonal contour compared to REP and INF.

\largerpage[2]
\section*{Abbreviations}
\begin{tabularx}{.45\textwidth}{lQ}
1 & First person\\
2 & Second person \\
3 & Third person \\
\textsc{acc} & Accusative \\
\textsc{am} & Auto-Segmental (model) \\
\textsc{blp} & Bilingual Linguistics Profile \\
\textsc{cl} & Clitic \\
\textsc{dat} & Dative \\
\textsc{dom} & Differential object marking \\
\textsc{ftr} & Focal tonal range \\
\textsc{ger} & Gerund \\
\textsc{h} & High \\

\end{tabularx}
\begin{tabularx}{.45\textwidth}{lQ}

\textsc{imp} & Imperative \\
\textsc{inf} & Information-seeking \\
\textsc{inft} & Infinitive \\
\textsc{l} & Low\\
\textsc{MirF} & Mirative Focus \\
\textsc{neg} & Negation \\
\textsc{pl} & Plural \\
\textsc{prs} & Present \\
\textsc{pst} & Past \\
\textsc{ptcp} & Participle \\
\textsc{rep} & Echo-repetition \\
\textsc{sg} & Singular \\
\textsc{st} & Semitone \\
\textsc{sur} & Echo-surprise \\
\end{tabularx}

\section*{Acknowledgements}
We thank all participants for their time, and Eric Mart\'inez for helping design the materials for this study. Many thanks to Jessica Craft for her enthusiasm and professionalism collecting the data and preparing it for acoustic analysis, and to Erin Christopher, Tyler King and Gus O’Neil for their assistance with coding. We are also grateful to Alex Iribar at the Phonetics Lab at the University of Deusto for kindly allowing us to use their facilities; Jon Franco, who passed away in 2021, for his help recruiting participants and his support and encouragement; and Mark Amengual for assistance with the BLP. We also gratefully acknowledge the suggestions from three anonymous reviewers, and the assistance of the volume editors and Luis Avil\'es Gonz\'alez in the preparation of the final version of this manuscript. All errors are of course ours. This research was funded by an FSU COFRS grant awarded to the second author in 2014--2015.


%\newpage
\appendixsection{}\label{13:app:1:NHigh}

\begin{table}[H]
\caption{Nuclear High}
\label{13:table3}
 \begin{tabular}{l rrrrr}
  \lsptoprule
  ID & BLP&INF&SUR&\multicolumn{2}{c}{Difference} \\
  \cmidrule(lr){5-6}
  &score&&&(Hz.)&(ST)\\

  \midrule
P15&168&350&444&94&4.1\\
P11&155&383&343&$-$40&$-$1.9\\
P9&123&300&293&$-$7&$-$0.4\\
P22&85&310&349&39&2.1\\
P8&80&255&377&122&6.8\\
P1&76&355&383&28&1.3\\
P5&51&354&355&1&0.05\\
P21&49&595&424&29&1.2\\
P4&49&356&437&81&3.6\\
P20&38&304&313&9&0.5\\
P13&14&302&341&39&2.1\\
P14&$-$2&306&344&38&2\\
P7&$-$5&384&504&120&4.7\\
P3&$-$40&323&411&88&4.2\\
\hline
Average&&334 Hz&378 Hz&44 Hz&2.1 ST\\
  \lspbottomrule
 \end{tabular}
\end{table}

%\newpage
\appendixsection{}
\label{13:app:2:FTR}

\begin{table}[H]
\caption{FTR}
\label{13:table4}
 \begin{tabular}{lrrrrrr}
 %{l @{\hskip 0.75in} c| @{\hskip 0.25in} c @{\hskip 0.25in} c| @{\hskip 0.25in} c @{\hskip 0.25in} c| @{\hskip 0.25in} c@{\hskip 0.25in} c}
  \lsptoprule

 ID&BLP&
  \multicolumn{2}{c}{INF}&
  \multicolumn{2}{c}{SUR}&
  Difference\\
  \cmidrule(lr){3-4}\cmidrule(lr){5-6}
  &score&(Hz.)&(ST)&(Hz.)&(ST)&(ST)\\

  \midrule
P15&168&156&10.2&259&15.4&5.2\\
P11&155&212&14&173&12.2&$-$1.7\\
P9&123&112&8.1&124&9.5&1.4\\
P22&85&126&9&175&12.1&3.1\\
P8&80&88&7.3&180&11.2&3.9\\
P1&76&172&11.5&234&16.3&4.8\\
P5&51&152&9.7&173&11.5&1.8\\
P21&49&189&11.2&234&13.9&2.7\\
P4&49&166&10.9&254&15&4.1\\
P20&38&150&11.8&182&15&3.2\\
P13&14&159&12.9&178&12.8&$-$0.1\\
P14&$-$2&110&7.7&147&9.7&2\\
P7&$-$5&153&8.8&272&13.4&4.6\\
P3&$-$40&127&8.7&232&14.4&5.7\\
\hline
Average&&148 Hz&10.1 ST&200 Hz&13 ST&2.9ST\\
  \lspbottomrule
 \end{tabular}
\end{table}


\sloppy
\printbibliography[heading=subbibliography,notkeyword=this]


\end{document}
