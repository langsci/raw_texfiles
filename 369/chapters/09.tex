\documentclass[output=paper,colorlinks,citecolor=brown,draft,draftmode]{langscibook}
\ChapterDOI{10.5281/zenodo.7525108}
\author{Nicoletta Loccioni \affiliation{University of California Los Angeles}}
\title{A superlative challenge for a syntactic account of connectivity sentences}
\abstract{In this paper, I present two sets of data that challenge the ``question plus deletion''(Q+D) approach to connectivity. The first set of data comes from Romance data where superlative import requires relativization, whereas the second set has more generally to do with relative clauses in subject position of specificational sentences. The problem comes down to what follows. Under Q+D, a conflict emerges between the assumed syntax of the
post-copular clause and its interpretation. That is, the structural configuration required to satisfy Binding or NPI licensing cannot generate the desired (superlative) interpretation, at least not without relying on mysterious implicatures. The same problem does not arise for revisionist accounts, which maintain that variable binding does not require c-command and can therefore straightforwardly derive the correct meaning.}

%move the following commands to the "local..." files of the master project when integrating this chapter
% \usepackage{tabularx}
% \usepackage{langsci-basic}
% \usepackage{langsci-optional}
% \usepackage{langsci-gb4e}
% \bibliography{localbibliography}
% %\newcommand{\orcid}[1]{}
% \pagenumbering{arabic}
% \setcounter{page}{192}
% \usepackage{linguex}
% \usepackage{stmaryrd}
% \usepackage{lipsum}

\IfFileExists{../localcommands.tex}{
  \addbibresource{../localbibliography.bib}
  % add all extra packages you need to load to this file

\usepackage{tabularx,multicol}
\usepackage{url}
\urlstyle{same}

\usepackage{listings}
\lstset{basicstyle=\ttfamily,tabsize=2,breaklines=true}

\usepackage{langsci-basic}
\usepackage{langsci-optional}
\usepackage{langsci-lgr}
\usepackage{langsci-osl}
% \usepackage{./langsci/styles/langsci-lgr}
% \usepackage{./langsci/styles/langsci-osl}
% \usepackage{langsci-gb4e}

\usepackage{tikz}
\usetikzlibrary{patterns,calc}
\pgfdeclarepatternformonly{south east lines}{\pgfqpoint{-0pt}{-0pt}}{\pgfqpoint{3pt}{3pt}}{\pgfqpoint{3pt}{3pt}}{
    \pgfsetlinewidth{0.6pt}
    \pgfpathmoveto{\pgfqpoint{0pt}{3pt}}
    \pgfpathlineto{\pgfqpoint{3pt}{0pt}}
    \pgfpathmoveto{\pgfqpoint{.2pt}{-.2pt}}
    \pgfpathlineto{\pgfqpoint{-.2pt}{.2pt}}
    \pgfpathmoveto{\pgfqpoint{3.2pt}{2.8pt}}
    \pgfpathlineto{\pgfqpoint{2.8pt}{3.2pt}}
    \pgfusepath{stroke}}
    
\usepackage{stmaryrd}
\usepackage{wasysym}
\usepackage{multirow}
\usepackage{caption}
\usepackage{subcaption}
\usepackage{mathrsfs}
\usepackage{qtree}

\usepackage{linguex}


  %pminos do not split footnotes
% \interfootnotelinepenalty=10000 %Footnote in Laporte chapters has to be split SN


%\DeclareIndexNameFormat{default}{%
%\nameparts{#1}%
%\usebibmacro{index:name}%
%{\index[names]}%
%{\namepartfamily}%
%{\namepartgiveni}%
% {}% L1
% {}% L2
%{\namepartprefix}% generates spurious space L3
%{\namepartsuffix}% generates spurious space L4
%}

%  {\DeclareIndexNameFormat{default}{%
%     \usebibmacro{index:name}{\index[names]}{#1}{#3}{#5}{#7}}}

%\DeclareIndexNameFormat{default}{%
%  \usebibmacro{index:name}{\sindex[nom]}{#1}{#3}{#5}{#7}}

%\DeclareIndexNameFormat{default}{%
%  \usebibmacro{index:name}{\sindex[person]}{#1}{#3}{#5}{#7}}
%\DeclareIndexNameFormat{default}{%
%\nameparts{#1} \usebibmacro{index:name}{\sindex[person]]}{\namepartfamily}{‌​\namepartgiven}{\nam‌​epartprefix}{\namepa‌​rtsuffix}}

%\newcommand{\smiley}{:)}

%\renewbibmacro*{index:name}[5]{%
%\usebibmacro{index:entry}{#1}%
%{\iffieldundef{usera}{}{\thefield{usera}\actualoperator}\mkbibindexname{#2}{#3}{#4}{#5}}}

% \newcommand{\noop}[1]{}

%remove for final
%\overfullrule=1mm

\newcommand{\tobi}[2]}}
\renewcommand{\S}[1]{\tobi{#1}{\textsc{*}}}

% this volume references
% puts: [this volume]
% already defined: \citetv
%\newcommand{\citepv}[1]{(\citeauthor{#1} \citeyear*{#1} [this volume])}
\newcommand{\citealtv}[1]{\citeauthor{#1} \citeyear*{#1} [this volume]}

%parentheses around example number
\newcommand{\pref}[1]{(\ref{#1})}

% in-text examples

\newcommand{\lnex}[1]{\textit{#1}} %target lang word
\newcommand{\lnlit}[1]{(lit.: `#1')} %literal reading
\newcommand{\lnlat}[1]{(#1)} % latinization
\newcommand{\lntrans}[1]{`#1'} %translation
\newcommand{\lnexl}[2]%
{\lnex{#1}{} \lnlat{#2}} % ex with latinization
\newcommand{\lnexlat}[3]{\lnex{#1}{} \lnlat{#2}{} \lntrans{#3}} % ex with latinization and tranl.

%ch01
\newcommand{\co}[1]{\mbox{\textbf{#1}}}

%ch09

\newcommand{\cyrbulg}[1]{\begin{otherlanguage*}{bulgarian}#1\end{otherlanguage*}}


%ch10
\newcommand{\nlp}{{\small NLP}}
\newcommand{\mwe}{{\small MWE}}
\newcommand{\rae}{{\small RAE}}
\newcommand{\lvc}{{\small LVC}}
\newcommand{\pos}{{\small P}o{\small S}}
%\newcommand{\todo}[1]{ \textcolor{red}{#1} }

%\renewcommand{\labelenumi}{\theenumi}
%\ainamefmt{{vv}{ll}{, ff}{, jj}} % fullname

\newcommand{\biberror}[1]{{\color{red}#1}}

\newcommand{\osenovaitem}{--~}
  %% hyphenation points for line breaks
%% Normally, automatic hyphenation in LaTeX is very good
%% If a word is mis-hyphenated, add it to this file
%%
%% add information to TeX file before \begin{document} with:
%% %% hyphenation points for line breaks
%% Normally, automatic hyphenation in LaTeX is very good
%% If a word is mis-hyphenated, add it to this file
%%
%% add information to TeX file before \begin{document} with:
%% %% hyphenation points for line breaks
%% Normally, automatic hyphenation in LaTeX is very good
%% If a word is mis-hyphenated, add it to this file
%%
%% add information to TeX file before \begin{document} with:
%% \include{localhyphenation}
\hyphenation{
    Beck-man
    Ngu-yen
    back-chan-nel
    back-chan-nels
    mo-not-o-nous
    ste-reo-typ-i-cal
}

\hyphenation{
    Beck-man
    Ngu-yen
    back-chan-nel
    back-chan-nels
    mo-not-o-nous
    ste-reo-typ-i-cal
}

\hyphenation{
    Beck-man
    Ngu-yen
    back-chan-nel
    back-chan-nels
    mo-not-o-nous
    ste-reo-typ-i-cal
}

  % \togglepaper[3]%%chapternumber
}{}



\begin{document}
\maketitle

\section{Introduction}


\cite{higgins1973} convincingly showed that  copular sentences like \REF{predicational} are not the same as the ones in \REF{specgen}. Whereas the former clearly involve predication (the property of being really long is predicated of the subject referent in \REF{predicational}), the latter do not. Rather, \REF{specgen} seem to involve valuing of a variable introduced by the pre-copular subject. That is, the subject expression sets up a variable, and the post-copular expression \textit{My Brilliant Friend} provides the value for such variable, in a similar fashion to the question/answer pair in \REF{questionanswer}. These cases are normally referred to as \textsc{specificational} sentences, and they can consist of a  pseudo-cleft (as in \ref{pseudoclef}), or the pre-copular phrase can be a relative clause (shown in  \ref{relative}).% is a case where the subject is a relative clause .

\ea \label{predicational}\textsc{Predicational sentences}\\
The book that Betta read is really long.
\z

\ea \label{specgen}\textsc{Specificational sentences}
\ea \label{pseudoclef}What Betta read is \textit{My Brilliant Friend}. %\hfill pseudo-cleft
\ex \label{relative}The book that Betta read is \textit{My Brilliant Friend}. %headed relative clause
\z
\z

\ea \label{questionanswer}
\ea What did Betta read?
\ex  \textit{My Brilliant Friend}.
\z
\z

A well-established fact about specificational sentences like \REF{specgen} is that they exhibit connectivity effects (see \citealt{akmajian1970, higgins1973} and many others since). That is, they behave like their connected counterparts (the (b)-examples below) with respect to a variety of syntactic tests including principle A, B and C of Binding Theory, Pronominal binding, NPI licensing and opacity.  The same connectivity effects do not hold if the post-copular element is read as predicational.

\ea  Principle A
    \ea\label{anaphor}$[$The only person Narcissus$_i$ likes$]$ is himself$_i$.
    \ex  Narcissus$_i$ only likes himself$_i$.
    \z
\ex \label{principle C}Principle C
    \ea $[$The only person he$_{j/*i}$ likes$]$ is Narcissus$_i$.
    \ex  He$_j$ only likes Narcissus$_i$.
    \z
\ex \label{pronominalbind}Pronominal binding
    \ea $[$The only person every Italian$_i$ cares about$]$ is his$_i$ mother.
    \ex  Every Italian$_i$ only cares about his$_i$ mother.  \hfill
    \z
\z

The (a)-examples above are puzzling because Binding is normally taken to require syntactic c-command between the binder and the bindee. However, one can clearly see that in each of the (a)-examples the post-copular DP is not superficially c-commanded by the relevant DP. Yet,  in \REF{anaphor} the anaphor is bound by \textit{Narcissus}, in \REF{pronominalbind} co-reference between the pronoun and \textit{Narcissus} is impossible, and in
 \REF{principle C} the quantified expression binds the post-copular pronoun.

 The examples \REF{questans1}--\REF{questans3} show that question-answer pairs exhibit the same type of connectivity. Nothing overtly c-commands the anaphor in \REF{questans1} for example. Yet, Principle A is somehow satisfied.


\ea \label{questans1}
\ea  Who does Narcissus only like? \ex  Himself. \hfill Principle A
\z
\z



\ea \ea  Who does he only like? \ex  Narcissus. \hfill Principle C
\z
\z


\ea \label{questans3}\ea  Who does every Italian only care about? \ex His mother.  \hfill Pronominal binding
\z
\z

Two main lines of analyses have been proposed to account for connectivity effects: a conservative/syntactic line (see \cite{ross1972act},  \cite{dendikkenal2000pseudocleftandellipsis}, \cite{schlenker2003clausalequation}, \cite{romero2007connectivityunified, romero2018connectivity}  among others) and a revisionist/semantic one  (see \cite{jacobson1994connectivitysalt}, \cite{sharvit1999connectivity}, \cite{cecchetto2000connectivity} among others). %Let's have a look at how they analyze connectivity sentences.

Among the syntactic approaches, the one I will focus on is the \textit{Question plus deletion (Q+D)} account as developed by \citet{schlenker2003clausalequation} and \citet{romero2007connectivityunified, romero2018connectivity}. According to the proponents of this approach, the specificational sentence \REF{pseudocleft} displays the same behavior as  \REF{simplesentence} simply because at some level of representation, \REF{pseudocleft} contains a connected clause like \REF{simplesentence}, which is then partially elided, as informally shown in \REF{sottolineata}.
\ea \ea \label{pseudocleft}[What Narcissus/everybody likes] is himself$_F$
\ex \label{simplesentence}Narcissus/everybody likes himself
\z
\z

\ea  \label{sottolineata} [What Narcissus likes] is [\sout{Narcissus likes} himself]
\z

It is in the syntax of this  partially elided clause that binding is satisfied. Thus, once one takes into account the reconstructed clause, connectivity is straightforwardly accounted for. What about the interpretation? How is the meaning of a connectivity sentence derived under Q+D?

Under this account as presented by  \citet{schlenker2003clausalequation} \& \citet{romero2007connectivityunified}, the specificational subject is analyzed as a concealed question, the post-copular constituent is treated as a partially elided answer, and \textit{be} simply denotes the identity function between the two.

Specificational subjects are analyzed as concealed questions whether or not they superficially look like questions. That is, not only are  English pseudoclefts taken to denote questions, but so are other specificational subjects that  do not resemble questions at all.  This includes nominal relative clauses like \REF{conrelativa}, which will be the main focus of this paper.

\ea \label{conrelativa}The person Narcissus loves is himself.
\z


For these nominal subjects, \citet{schlenker2003clausalequation}  suggests that  \textit{the} spells out the definiteness feature of a concealed \textit{wh-word} in a similar fashion to the object of \textit{know} in \REF{know}.

\ea \label{know}John knows the capital of Italy. \hfill \citep{schlenker2003clausalequation}
\z

The relevant reading of \REF{know} does not say that John knows Rome. It asserts that John knows \textit{what the capital of Italy is}. That is, he knows the answer to the question \textit{What is the capital of Rome?} Similarly, the pre-copular element in \REF{conrelativa} is taken to have the same denotation as the question \textit{who is the person Narcissus loves?}, where -- crucially -- the definite description is read as predicational.


Here is how the meaning of \REF{conrelativa} is derived by the Q+D as it currently stands in the literature.  \citet{schlenker2003clausalequation}
adopts \citegen{groenendijk1984}  semantics for the meaning of the specificational subject. In \citegen{groenendijk1984} semantics, the denotation of a question is its exhaustive true answer in the world w. In order to implement that, \citet{romero2007connectivityunified} assumes that the specificational subject is equipped with a silent answer operator \textsc{ans}, defined in \REF{anssem}.  \textsc{ans}'s job is to turn the intension of the nominal into a question meaning.



\ea \label{anssem} $\llbracket$\textsc{ans}$\rrbracket$ = $\lambda$y$_{<s,e>}$.$\lambda$w.$\lambda$w'.y(w') = y(w)
\z



Thus, \REF{conrelativa} has the LF in \REF{conrelativaLF}.
The question denoted by the pre-copular phrase is equated to the \textit{strengthened} value of the partially elided answer. The strengthened value of the answer is equal to its normal semantic value  \textit{plus} its implicature of exhaustivity arising from focal stress. In this case, the right-hand side of the equation includes the value in \REF{normalvalue1}, as well as the implicature triggered by focal stress on  \textit{himself}, shown in \REF{implicature1}. The implicature's contribution is that all the other alternatives to \textit{himself} in \REF{whatever} are negated and -- as a results -- we interpret \REF{whatever} to mean that Narcissus only likes \textit{himself}.
\REF{conrelativa} then has the meaning in \REF{conrelativameaning}, which has the rough paraphrase in \REF{paraphrase}. %In this case...

\ea \ea \label{whatever}Narcissus likes himself$_F$
\ex \label{normalvalue1}Assertion:$\lambda$w.like(n,n,w)
\ex\label{implicature1}Implicature: $\lambda$w.$\forall$x:x$\neq$n $\rightarrow$ $\neg$like(n,x,w)
\z
\z

\ea \ea \label{conrelativaLF}$\llbracket$\textsc{ans} The person Narcissus likes is \sout{Narcissus likes} himself $\rrbracket$
\ex \label{conrelativameaning} $\lambda$w [ $\lambda$w'. $\iota$x [like(n,x,w')] = $\iota$x [like(n,x,w)] = $\lambda$w'. like(n,n,w') \& \textit{Narcissus likes nobody else in w'} ]
\z
\z

\ea \label{paraphrase} Paraphrase of \REF{conrelativaLF}\\
\glt `We are in a world w such that: the exhaustive answer to the question ``who is (the) person Narcissus like'' in w is the proposition ``that Narcissus likes himself (and nobody else)'''.

\nocite{groenendijk1984}
\z

I will now turn to the other main line of analysis, the \textit{semantic or revisionist accounts}  (\cite{jacobson1994connectivitysalt}, \cite{sharvit1999connectivity}, \cite{cecchetto2000connectivity} among others).
Revisionists do not take Binding (Scope or NPI licensing) to require a structural condition like c-command, at least not in specificational sentences. For this reason, they do not need to posit elided structure in the post-copular constituents.
Connectivity instead results from a higher-order semantics (supplemented with rule I of Reinhart (\citeyear{reinhart1983})).
\citet{sharvit1999connectivity} for example appeals to quantification over functions and analyzes the reflexive as an identity function. Under her approach, the example in \REF{conrelativa} has the LF in \REF{quantifrep} and the semantics in \REF{denotationcase}. It denotes the equation between the unique function that maps Narcissus to the person he likes and the function that maps everyone to themselves.%


\ea \ea \label{quantifrep} $\llbracket${[The person Narcissus likes] is [himself]}$\rrbracket$
\ex \label{denotationcase}$\lambda$w.$\iota$f$_{<e,e>}$ [$\exists$x.like(n,f(x),w)] = $\lambda$x.x
\z
\z

In this paper, I will show  that certain specificational subjects pose a challenge to proponents of the syntactic \textit{Q+D} account. In \sectref{secromance}, this problem is first illustrated using Romance data where superlative import requires relativization. The issue boils
down to what follows. Under a syntactic account, a conflict emerges between the required syntax of the post-copular clause and its desired interpretation. That is, the structural configuration that \textit{Q+D} requires to
satisfy Binding (or more generally to explain connectivity) cannot generate the intended superlative interpretation. A similar problem
does not arise for semantic accounts, which maintain that variable binding does not require c-command and
can, therefore, straightforwardly derive the correct meaning.
In  \sectref{secrelativization}, I then show that the challenging Romance data is part of a larger and more general problem with specificational subjects that contain relative clauses. It has to do with the fact that -- under Q+D -- the information in the head of the relative clause get lost in the reconstructed post-copular clause.

The goal of this paper is merely to highlight the existence of this challenge for a Q+A analysis and spell out what the desiderata of a promising syntactic account are. \sectref{secdesiderata} includes a short discussion of those desiderata. The proposal of a new syntactic account is left to future research.

\section{Superlatives in Romance}
\label{secromance}


In this section, I show how Romance data involving superlative phrases present a novel interesting challenge for Q+D accounts. % to raise a problem for conservative/syntactic accounts of  connectivity sentences .
The problem clearly arises in cases like \REF{exemplificat} where the pre-copular phrase is a definite relative clause that embeds a superlative, and it boils down to what follows.
The level of representation that satisfies binding does not yield superlative import and vice versa.


\ea\label{exemplificat}
\gll La persona con cui Lenuccia è più gentile è {se stessa}.\\
the person with whom Lenuccia is more kind is herself \\
\glt `The person Lenuccia is the kindest to is herself.'
\z


 Appreciating that there is a puzzle here requires some background on relative readings of superlatives and how those readings are obtained in Italian (and other Romance languages). For the purpose of this paper, discussion of one specific example will be enough. For a more detailed investigation of the phenomenon in Romance, I refer the interested reader  to \citet{loccioni2018phd}.

First, Romance lacks the morphological distinction between \textit{more} and \textit{most}. In both comparatives and superlatives, Italian makes use of the morpheme \textit{più} that I will gloss as `more' throughout the paper. In addition to this comparative morpheme, superlative import requires the presence of some definite marker. In the absence of a definite marker, the only available interpretation is a comparative one, as shown by \REF{comparative}.

\ea\label{comparative}
\gll una montagna più alta (dell' Everest)\\
a mountain more tall (than.the Everest)\\
\glt `a taller mountain (than the Everest)'
\z

\ea
\gll la montagna più alta (di tutte)\\
the mountain more tall (of all)\\
\glt `the tallest mountain (of all)\\
\z



It will not be surprising then that \REF{sententiallevel} and \REF{piulibri} can only have a comparative interpretation.


\ea\label{sententiallevel}
\gll Maria è più esigente con Lucia/ {se stessa}\\
Maria is more demanding with Lucia herself\\
\glt `Maria is \{more/*the most\} demanding with Lucia/ herself'
\z

\ea
\gll \label{piulibri}Piero ha letto più libri\\
Piero has read more books\\
\glt `Piero read more/*the most books'
\z

However, in Italian and other Iberico-Romance languages, embedding the comparative phrases above inside a definite relative clause can generate superlative import.


\ea
\gll \label{superlita}La persona con cui Maria è più esigente\\
the person with whom Maria is more demanding\\
\glt `The person with whom Maria is \{more/ the most\} demanding'
\z

\ea
\gll La persona che ha letto più libri\\
the person who has read   more books\\
\glt `The person who read more/the most books'
\z

What the contrast between \REF{sententiallevel} and \REF{superlita} shows is that relativization is necessary to get the desired superlative interpretation in these examples.\footnote{Adding
  a local definite determiner adjacent to \textit{più} instead of embedding the comparative inside a definite relative clause would not work. It would either provide the wrong interpretation or it would be ungrammatical. In the case of \REF{sententiallevel2}, the resulting sentence could only have the absolute (a)-interpretation, which is different from the one I am after in \REF{superlita}.

  \ea
  \gll \label{sententiallevel2}Maria è la più esigente con Lucia/ {se stessa}\\
  Maria is the more demanding with Lucia herself\\
  \ea[]{Maria is more demanding with Lucia/herself than anybody else is.}
  \ex[*]{Maria is more demanding with Lucia/herself than she is with anybody else.}
  \z
  \z

  Adding a definite determiner to \REF{piulibri} results in a sharply ungrammatical sentence.

  \ea[*]{
  \gll \label{piulibri2}Piero ha letto i più libri\\
  Piero has read the more books\\
  }
  \z
}


With these facts about Romance at hand, I can now go back to the problem of connectivity in specificational sentences.
The crucial data point for my argument will be one where  a relative clauses like \REF{superlita} is plugged in the pre-copular position of a specificational sentence which exhibits connectivity.


Below are the relevant examples which show connectivity effects with respect to  (i) Principle A, (ii) Principle C and (iii) Pronominal binding.\footnote{Whereas \REF{lapersona} and \REF{lapersona2} are fully acceptable, pronominal binding sentences such as \REF{pronominalbinding} are subject to a greater extent of speaker variation. I personally find them less than perfect.}



\ea
\gll \label{lapersona}$[$ La persona con cui  Maria è più esigente $]$ è {se stessa} \\
{} the person with whom Maria is more demanding {} is herself\\
\glt `The person with whom Maria is the most demanding is herself'
\z


\ea
\gll \label{lapersona2}$[$ La persona con cui \textit{pro} è più esigente $]$ è Maria \\
{} the person with whom \textit{pro} is more demanding {} is Maria\\
\glt `The person with whom he/she$_{i/*j}$ is the most demanding is Maria$_{j}$'
\z


\ea[?]{
\gll \label{pronominalbinding}$[$ La persona con cui ogni paziente è più onesto $]$ è il suo terapista\\
{} the person with whom every patient is more honest {} is the his therapist\\
\glt `The person with whom every patient is the most honest is his therapist'}
\z

Let's turn to how a \textit{Q+D} account explain these cases. Take \REF{lapersona} for example.\footnote{A
  very similar example (with virtually the same properties) can be built for Spanish:

  \ea
  \gll La persona con la que Mar\'ia es m\'as exigente es {ella misma}  \\
  the person with the which Mar\'ia is more demanding is herself\\
  \glt `The person with whom Maria is the most demanding is herself'
  \z

  French superlatives do not have the same properties as Iberico-Romance languages such as Italian and Spanish. For this reason, French does not have the exact counterpart of \REF{lapersona}. In order  to construct the relevant example, one could use what I called \textit{de}-free relatives. As far as I know they are the only French predicative constructions where a superlative interpretation can arise without determiner doubling (see \cite{loccioni2018phd} for discussion):


  \ea
  \gll Ce que Jean a de plus pr\'ecieux, c'est {lui-m\^eme}\\
  this that Jean has \textit{di} more valuable it.is  himself\\
  \glt  `The most valuable thing Jean has is himself'
  \z

}
If the post-copular phrase  is a partially elided clause, binding is easily accounted for. As shown in \REF{esempioprob}, \sout{Maria} c-commands the anaphor \textit{se stessa}.

\ea
\gll \label{esempioprob}{[ la persona con cui Maria è più esigente ]} è {[ \sout{Maria è più esigente con} se stessa  ]} \\
{[ the person with whom Maria is most demanding ]} is {[ Maria is more demanding with  herself ]}\\
\z

However, the syntactic configuration that is able to satisfy Condition A of Binding Theory fails to generate the right superlative meaning. As discussed above, superlative import in the specificational subject rests on the fact that the superlative is embedded in a definite relative clause. It is only in that environment that \textit{more esigente} (lit. ``more demanding'') can have a superlative interpretation. Once relativization is undone in the partially elided post-copular constituent, the superlative interpretation becomes unavailable, and the only possible reading for \textit{more esigente} is a comparative one. The contrast between the meaning of the pre-copular and post-copular phrases is illustrated by the translation in \REF{esempioprob}.


What a defender of the Q+D account would have say is that the correct truth-conditions result from the computation of the relevant implicature (that arises from focal stress). That is, once the implicature is factored in, the desired interpretation is obtained.
In order for it to work, they would have to maintain that in \REF{esempioprob} not only does the answer assert that Maria is more demanding with herself but also \textit{implicates} that there is nobody else she is more demanding with (than she is with herself).  \REF{mariaimplicature} shows the semantics of the assertion and the implicature, using the degree-based lexical entries of \textit{demanding with}, \textit{er} and \textit{est} provided in \REF{denotationdegree}.


\ea \label{mariaimplicature}\ea \label{statementmaria}Maria is more demanding with herself$_F$
\ex \label{normalvalue}Assertion: $\lambda$w.$\exists d$[demanding(m,m,d,w) $\wedge$ $\neg$demanding(m,g(1),d,w)]
\ex \label{implicature}Implicature: $\lambda$w.$\exists$d (demanding(m,m,d,w) \& $\forall$y$\in$ Y [y$\neq$x $\rightarrow$\\ $\neg$(demanding(m,y,d,w)])
\z
\z

\ea \label{denotationdegree}\ea $\llbracket$demanding-with$\rrbracket$ =$\lambda$w.$\lambda$d.$\lambda$x.$\lambda$y.demanding(y,x,d,w)
\ex $\llbracket$-er$\rrbracket$ = $\lambda$P$_{<d,t>}$.$\lambda$Q$_{<d,t>}$.$\exists d$[Q(\textit{d}) $\wedge$ $\neg$P(\textit{d})]
 \hfill \citep{bhatt2011reduced}
\ex $\llbracket$-est$\rrbracket$ =  $\lambda$Y$_{<e,t>}$.$\lambda$P$_{<d,et>}$.$\lambda$x$_{<e>}$.$\exists$d (P(x,d) \& $\forall$y$\in$ Y [y$\neq$x $\rightarrow$ $\neg$P(y,d)]) \\\hfill \citep{heim1999notes}
\z
\z

Examaples \REF{normalvalue} show that the normal value of the answer to the question \textit{Who is Maria the most demanding with?} is that Maria is more demanding with herself than she is with some relevant other (g(1) in \REF{normalvalue}) refers to this contextually relevant standard of comparison). The strengthened value of \REF{statementmaria}, however, is that Maria is more demanding with herself than she is with anybody else.
It is unclear how this implicature would  work. How can focal stress on the anaphor be responsible for gap between a simple \textit{than}-clause and a quantified one (more than anybody else)?


A revisionist account \textit{à la} \citet{sharvit1999connectivity} would not run into the same problem.
 \REF{lapersona} would have the LF in \REF{revis1} and would denote the equation between the unique function that maps Maria to the person she is the most demanding with and the function that maps everyone to themselves, as shown in \REF{revis2}.

 \ea \ea \label{revis1}
 $\llbracket${[The person Maria is the most demanding with] is [herself}$\rrbracket$
\ex \label{revis2}$\lambda$w.$\iota$f$_{<e,e>}$ [$\exists$d.$\exists$x.demanding(m,f(x),d,w) \& $\forall$y$\in$ Y [y$\neq$x $\rightarrow$ \\ $\neg$(demanding(m,y,d,w)]] = $\lambda$x.x
\z
\z

In the next section, I will show that this problem is not limited to relative clauses containing superlatives. Rather, it is part of a larger issue that a Q+A account runs into when dealing with relative clauses in subject position.


\section{A larger problem with relativization}
\label{secrelativization}

The problem discussed in the previous section was somewhat language-specific. It had specifically to do with the fact that in languages like Italian, relativization can be a necessary element to get superlative import in certain environments. In specificational sentences then, a tension emerged between the desired structure needed to account for connectivity and the meaning of the construction.

In this section, I want to show that a similar challenge for Q+D accounts arises when relative clauses are specificational subjects more generally. It has to do with the fact that the two clauses that are equated do not contain the same level of information. Since the right-hand side of the equation has the syntax of a connected clause, it only contains part of the pre-copular phrase meaning. In particular, the information included in the head of the relative clause is absent in the post-copular clause.\footnote{The same applies to elements in the left-periphery of the DP, such as demonstratives and ordinal numbers.}

To show that, take the English sentences in \REF{englishcases}. Under the  Q+D account as it currently stands in the literature, these sentences are analyzed as in \REF{englishcases2}. One can easily notice that despite the different meanings of the subjects, they all share the very same  connected sentence on the right-hand side, \textit{he cared about John's mother}.

\ea \label{englishcases}\ea  What he cared about is John's mother
\ex \label{oldest}The oldest woman he cared about is  John's mother
\ex \label{youngest}The youngest woman he cared about is  John's mother
\ex \label{lastsick}The last sick person he cared about is John's mother
\z
\z

\ea \label{englishcases2}
\ea  {}[What he cared about] is [\sout{he cared about} John's mother]
\ex \label{oldest2}[The oldest woman he cared about] is [\sout{he cared about} John's mother]
\ex {}[The youngest woman he cared about] is [\sout{he cared about} John's mother]
\ex  {}[The last sick person he cared about] is [\sout{he cared about} John's mother]
\z
\z

In order to make the computation work and derive the correct interpretation, the meaning gap between the specificational subject and the post-copular clause has to be filled. According to the Q+D account, this role is played by the implicature triggered by focus on \textit{John's mother}. Thus, the meaning of  `he cared about John's mother' has to be strengthened differently in each case to derive the corresponding exhaustive answers, as shown in \REF{answers}


\ea \label{answers}\ea \label{andnobodyelse}He cared about John's mother \textit{and nobody else}
\ex \label{nowomanolder}He cared about John's mother \textit{and no woman older than} that
\ex \label{nowomanyounger}He cared about John's mother and \textit{no woman younger than that}
\ex  He cared about John's mother and \textit{no other sick person after her}
\z
\z

Again, it is unclear how these implicatures work. In \REF{andnobodyelse}, focus on \textit{John's mother} would trigger the implicature that there is nobody else he cared about. However, the implicature triggered in the other examples are fairly different. In theses cases, the presupposition of the superlative requires that there are other people he cared about. But they have to fit a certain description which is provided by the head of specificational subject. \REF{oldest} for example triggers the implicature that he did not care about any woman who is older than John's mother whereas the implicature in \REF{youngest} has quite the opposite effect. The fact that this information is ``retrieved'' by an implicature seems quite undesirable. It would be much preferable if the postcopular phrase contained this information. Yet, we don't want to end up with something like \REF{repp}, where in the right-hand side we obtain the connectivity sentence we started with.

\ea \label{repp}The oldest woman he cares about is [ his mother is the oldest woman John cares about ]
\z

In the discussion of a similar case, \citet{schlenker2003clausalequation}  recognized that the precise way in which the implicature is calculated is unclear. His analysis of \REF{theworstproblem} is that  \textit{(He worries about) himself$_F$} triggers the implicature that  there is no problem that John worries about more.

\ea \label{theworstproblem}The worst problem that John worries about is himself$_F$
\z

He writes: ``How the implicature comes about is unclear (focus seems to be crucial); but once it is assumed that such an implicature exists in \REF{theworstproblem}, the mechanism developed [...] to equate the value of the pre-copular element with the strengthened value of the post-copular element will presumably yield the desired results.'' But where does the degree scale come from in this example? Why would the assertion \textit{John worries about himself} generate  the implicature that he is his worst problem?


\section{Desiderata for a syntactic account and conclusion}
\label{secdesiderata}

I am afraid that this paper is going to disappoint the reader who is looking for a solution to rescue a syntactic account of connectivity sentences. Even though I am not able to offer a solution at this point, I would like to spell out what (some of) the desiderata of a more promising syntactic account are, given what we have learned.

A good syntactic account that does not want to abandon the c-command tests needs to have a good story for what a ``connected'' elliptical structure with the right interpretation looks like.



For fragment sentences, \citet{merchant2004fragmentsandellipsis} suggests that ellipsis is preceded by an A'-movement to a clause-peripheral position.

\ea
\ea  Who did she see? John.
\ex  [FP [DP John ]i [.F' F $<$ [TP she saw t$_i$ ] $>$ ]] \hfill (adapted from \citet{merchant2004fragmentsandellipsis})
\z
\z

Given the similarities between specificational sentences and question/answer pairs, one could explore whether a similar account would work in the case of  post-copular phrases of specificational sentences. It would translate into moving [$_{DP}$ himself ]  to some focus position -- either the high left periphery position posited by  \citet{rizzi1997fine}  or the low left periphery  argued for by \citet{belletti2001inversion}.


\ea  What Narcissus likes is [ himself$_i$ [ \sout{Narcissus likes t$_i$} ] ]
\z

If the post-copular phrase is moving from a connected sentence, connectivity effects are easily accounted for. However, other issues arise. First,  Italian strongly disallows preposition stranding, which would result from such derivation.  \REF{pstrand} provides an example of it.

\ea \label{pstrand}L'unica persona con cui Maria è  esigente è [ se stessa$_i$ [ \sout{Maria è esigente con t$_i$} ] ]
\z

Second, the data I presented in \sectref{secromance} showed that the post-copular phrase should have some (form of) maximalization as part of the derivation to derive the correct superlative interpretation.
Even though topicalization has very obvious effects on the information structure, it
 is not enough to get superlative import in Italian. This is illustrated in  \REF{topicalization}, which can only have the (a)-interpretation.


\newpage
\ea \label{topicalization}CON SE STESSA Maria è più esigente.
\ea[]{  WITH HERSELF MARY is more demanding.}
\ex[*]{WITH HERSELF MARY is the most demanding.}
\z
\z



Third,  \sectref{secrelativization} showed that the derivation should have some mechanism to ``remember'' the content of the head of the relative clause in the specificational subject. This means that, even though \REF{conclusion2} accounts for binding, it cannot be the full story for \REF{conclusion1}.

\ea \ea \label{conclusion1}The oldest person John cares about is himself
\ex \label{conclusion2} [The oldest person John cares about] is [himself$_i$ \sout{John cares about t$_i$}]
\z
\z



It seems therefore unlikely that a simple movement akin to topicalization could be extended from fragment answers to post-copular phrases of specificational sentences.



Lastly, any satisfactory syntactic approaches should also make sure to avoid the circularity of obtaining another specificational sentence on the right-hand side.

\ea  The oldest person John cares about is  [ himself \sout{is the oldest person John care about} ]
\z

%\nocite{loccioni2020syntaxsuperlative}
\sloppy
\printbibliography[heading=subbibliography,notkeyword=this]

\end{document}
