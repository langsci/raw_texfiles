\documentclass[output=paper,colorlinks,citecolor=brown,draftmode]{langscibook}
\ChapterDOI{10.5281/zenodo.7525118}
\author{Ana {de Prada Pérez}\affiliation{Maynooth University}\orcid{0000-0003-4153-9638} and Nick Feroce\affiliation{Lexia Learning}\orcid{0000-0001-7615-0820}}
\title{Mechanical vs. functional processes in subject pronoun expression in Spanish second language learners}
\abstract{Subject pronoun expression in Spanish is regulated by functional factors while the role of the mechanical factor priming, or perseveration, has been a source of debate. Additionally, 3\textsc{sg} subjects have been identified as more difficult to acquire than 1\textsc{sg} in L2 Spanish. In this paper we explore the interaction between Switch Reference, Form of Previous Subject (perseveration), and Speaker group in 1\textsc{sg} and 3\textsc{sg} subjects in L2 Spanish through the oral narratives of 28 Spanish L2ers (15 higher and 13 lower proficiency) and 9 Spanish-English bilingual native speakers (NSs). Results showed differences between L2ers and NSs in rates of overt pronominal subjects and in sensitivity to Switch Reference but not in the effect of priming. We hypothesize the differences in the interactions between variables in 1\textsc{sg} vs. 3\textsc{sg} could be due to 3\textsc{sg} involving reference-tracking and perseveration only being evident in contexts where the pronoun does not signal pragmatic content.}

% %move the following commands to the "local..." files of the master project when integrating this chapter
% \usepackage{tabularx}
% \usepackage{langsci-basic}
% \usepackage{langsci-optional}
% \usepackage{langsci-gb4e}
% \bibliography{localbibliography}
% \newcommand{\orcid}[1]{}
% \pagenumbering{arabic}
% \setcounter{page}{306}

\IfFileExists{../localcommands.tex}{
  \addbibresource{../localbibliography.bib}
  % add all extra packages you need to load to this file

\usepackage{tabularx,multicol}
\usepackage{url}
\urlstyle{same}

\usepackage{listings}
\lstset{basicstyle=\ttfamily,tabsize=2,breaklines=true}

\usepackage{langsci-basic}
\usepackage{langsci-optional}
\usepackage{langsci-lgr}
\usepackage{langsci-osl}
% \usepackage{./langsci/styles/langsci-lgr}
% \usepackage{./langsci/styles/langsci-osl}
% \usepackage{langsci-gb4e}

\usepackage{tikz}
\usetikzlibrary{patterns,calc}
\pgfdeclarepatternformonly{south east lines}{\pgfqpoint{-0pt}{-0pt}}{\pgfqpoint{3pt}{3pt}}{\pgfqpoint{3pt}{3pt}}{
    \pgfsetlinewidth{0.6pt}
    \pgfpathmoveto{\pgfqpoint{0pt}{3pt}}
    \pgfpathlineto{\pgfqpoint{3pt}{0pt}}
    \pgfpathmoveto{\pgfqpoint{.2pt}{-.2pt}}
    \pgfpathlineto{\pgfqpoint{-.2pt}{.2pt}}
    \pgfpathmoveto{\pgfqpoint{3.2pt}{2.8pt}}
    \pgfpathlineto{\pgfqpoint{2.8pt}{3.2pt}}
    \pgfusepath{stroke}}
    
\usepackage{stmaryrd}
\usepackage{wasysym}
\usepackage{multirow}
\usepackage{caption}
\usepackage{subcaption}
\usepackage{mathrsfs}
\usepackage{qtree}

\usepackage{linguex}


  %pminos do not split footnotes
% \interfootnotelinepenalty=10000 %Footnote in Laporte chapters has to be split SN


%\DeclareIndexNameFormat{default}{%
%\nameparts{#1}%
%\usebibmacro{index:name}%
%{\index[names]}%
%{\namepartfamily}%
%{\namepartgiveni}%
% {}% L1
% {}% L2
%{\namepartprefix}% generates spurious space L3
%{\namepartsuffix}% generates spurious space L4
%}

%  {\DeclareIndexNameFormat{default}{%
%     \usebibmacro{index:name}{\index[names]}{#1}{#3}{#5}{#7}}}

%\DeclareIndexNameFormat{default}{%
%  \usebibmacro{index:name}{\sindex[nom]}{#1}{#3}{#5}{#7}}

%\DeclareIndexNameFormat{default}{%
%  \usebibmacro{index:name}{\sindex[person]}{#1}{#3}{#5}{#7}}
%\DeclareIndexNameFormat{default}{%
%\nameparts{#1} \usebibmacro{index:name}{\sindex[person]]}{\namepartfamily}{‌​\namepartgiven}{\nam‌​epartprefix}{\namepa‌​rtsuffix}}

%\newcommand{\smiley}{:)}

%\renewbibmacro*{index:name}[5]{%
%\usebibmacro{index:entry}{#1}%
%{\iffieldundef{usera}{}{\thefield{usera}\actualoperator}\mkbibindexname{#2}{#3}{#4}{#5}}}

% \newcommand{\noop}[1]{}

%remove for final
%\overfullrule=1mm

\newcommand{\tobi}[2]}}
\renewcommand{\S}[1]{\tobi{#1}{\textsc{*}}}

% this volume references
% puts: [this volume]
% already defined: \citetv
%\newcommand{\citepv}[1]{(\citeauthor{#1} \citeyear*{#1} [this volume])}
\newcommand{\citealtv}[1]{\citeauthor{#1} \citeyear*{#1} [this volume]}

%parentheses around example number
\newcommand{\pref}[1]{(\ref{#1})}

% in-text examples

\newcommand{\lnex}[1]{\textit{#1}} %target lang word
\newcommand{\lnlit}[1]{(lit.: `#1')} %literal reading
\newcommand{\lnlat}[1]{(#1)} % latinization
\newcommand{\lntrans}[1]{`#1'} %translation
\newcommand{\lnexl}[2]%
{\lnex{#1}{} \lnlat{#2}} % ex with latinization
\newcommand{\lnexlat}[3]{\lnex{#1}{} \lnlat{#2}{} \lntrans{#3}} % ex with latinization and tranl.

%ch01
\newcommand{\co}[1]{\mbox{\textbf{#1}}}

%ch09

\newcommand{\cyrbulg}[1]{\begin{otherlanguage*}{bulgarian}#1\end{otherlanguage*}}


%ch10
\newcommand{\nlp}{{\small NLP}}
\newcommand{\mwe}{{\small MWE}}
\newcommand{\rae}{{\small RAE}}
\newcommand{\lvc}{{\small LVC}}
\newcommand{\pos}{{\small P}o{\small S}}
%\newcommand{\todo}[1]{ \textcolor{red}{#1} }

%\renewcommand{\labelenumi}{\theenumi}
%\ainamefmt{{vv}{ll}{, ff}{, jj}} % fullname

\newcommand{\biberror}[1]{{\color{red}#1}}

\newcommand{\osenovaitem}{--~}
  %% hyphenation points for line breaks
%% Normally, automatic hyphenation in LaTeX is very good
%% If a word is mis-hyphenated, add it to this file
%%
%% add information to TeX file before \begin{document} with:
%% %% hyphenation points for line breaks
%% Normally, automatic hyphenation in LaTeX is very good
%% If a word is mis-hyphenated, add it to this file
%%
%% add information to TeX file before \begin{document} with:
%% %% hyphenation points for line breaks
%% Normally, automatic hyphenation in LaTeX is very good
%% If a word is mis-hyphenated, add it to this file
%%
%% add information to TeX file before \begin{document} with:
%% \include{localhyphenation}
\hyphenation{
    Beck-man
    Ngu-yen
    back-chan-nel
    back-chan-nels
    mo-not-o-nous
    ste-reo-typ-i-cal
}

\hyphenation{
    Beck-man
    Ngu-yen
    back-chan-nel
    back-chan-nels
    mo-not-o-nous
    ste-reo-typ-i-cal
}

\hyphenation{
    Beck-man
    Ngu-yen
    back-chan-nel
    back-chan-nels
    mo-not-o-nous
    ste-reo-typ-i-cal
}

  % \togglepaper[3]%%chapternumber
}{}


\shorttitlerunninghead{Mechanical vs. functional processes in Spanish SLA subject pronouns}
\begin{document}
\maketitle

\section{Introduction}
There is a longstanding tradition of variationist work comparing different monolingual as well as bilingual varieties with respect to subject pronoun expression (SPE). All these studies show that, while there is extant variation across varieties in terms of rates of overt pronominal subjects, little differences are found in the variables that condition the distribution. They also convincingly show that the distribution can only be explained by a combination of variables and that it is variable, except for some non-variable cases (see list of cases in  \citealt[33--36]{Cameron1997}). This implies that there are no ungrammatical or inappropriate uses in the case of specific variable uses, but there may be differences in the trends of use across speaker groups. Variationist methodologies have been widely applied to the acquisition of SPE in Spanish as a second language (L2). This paper offers further evidence of progression of acquisition of SPE in Spanish using variationist methodologies.

Several variables affect the distribution of overt and null subjects in Spanish. Among the functional factors that have been found to affect the use of an overt pronominal subject are those that favor continuity in the discourse. For instance, the variable Switch Reference, where the referent is the same or different from the previous subject, has been reported to have a large effect on the distribution of expressed vs. unexpressed pronominal subjects, as in \REF{ex:14:1}. Participant 2 used a null subject in the second clause, which has the same referent as the first clause.

\is{Cognition} %add "Cogntion" to subject index for this page

\ea\label{ex:14:1}
\gll Yo jugué fútbol durante la escuela y asistí a una escuela privada para trece años.\\
     I play.\textsc{1sg}.\textsc{past} soccer during the school and attend.\textsc{1sg}.\textsc{past} to a private school for thirteen years\\
\glt ‘I played soccer during my school years and I attended a private school for thirteen years.'
\begin{flushright}
(Participant 2, lower proficiency group)
\end{flushright}
\z
\il{Spanish} %add "Spanish" to language index for this page

Also highly ranked is the variable Grammatical Person. Several studies, particularly on monolingual varieties, report higher rates of overt pronominal subjects in first (1\textsc{sg}) than in third person singular (3\textsc{sg}), while the results are different for bilingual speakers (with the exception of  \citealt{TravisCacoullos2012}) and second language learners (L2ers), who produce more expressed subjects in 3\textsc{sg} than in 1\textsc{sg} (\citealt{GeeslinGudmestad2011, GeeslinGudmestad2016, GudmestadGeeslin2010, GudmestadGeeslin2013,
Lozano2009, Lozano2016, PradaPérezFeroce2020}), as shown in \REF{ex:14:2}. Participant 1 uses the overt pronoun with a third person singular subject in a context of same reference as the previous sentence.


\ea\label{ex:14:2}
\gll Es mayor. Ella tiene 35 años y una hija.\\
     be.\textsc{3sg}.\textsc{pres} older she have.\textsc{3sg}.\textsc{pres} 35 years and a daughter\\
\glt ‘She is older. She is 35 and has a daughter.'
\begin{flushright}
(Participant 1, higher proficiency group)
\end{flushright}
\z
\il{Spanish} %add "Spanish" to language index for this page

Additionally, previous work has included the mechanical factor of Perseveration, or priming. Previous research refers to priming or perseveration as a mechanical factor in contrast with functional factors, which are guided by meaning and communication (e.g. \citealp{TravisCacoullos2012}; however, see \citealp{Otheguy2015}). An increased use of the overt form has been reported for instances where the previous mention was also an overt form, both for native speakers (NSs) (e.g. \citealp{Cameron1994}) and L2ers (e.g. \citealp{Abreu2009}), as in \REF{ex:14:3}. Participant 5 uses an overt pronominal subject in the first clause and continues to use it with the same referent in subsequent sentences.

\ea\label{ex:14:3}
\gll Él trabaja en la oficina de un abogado pero él no es un abogado. Él trabaja con casos específicos.\\
     He work.\textsc{3sg}.\textsc{pres} in the office of a lawyer but he \textsc{neg} be.\textsc{3sg}.\textsc{pres} a lawyer. He work.\textsc{3sg}.\textsc{pres} with cases specific\\
\glt ‘He works in an attorney’s office but he is not a lawyer. He works on specific cases.'
\begin{flushright}
(Particpant 5, higher proficiency group)
\end{flushright}
\z
\il{Spanish} %add "Spanish" to language index for this page
The role of proficiency in the L2 acquisition of Spanish SPE has been previously examined in the literature. Overall, participants tend to use fewer overt subjects as proficiency increases, except for one dataset examined in \citet{GeeslinFafulas2015} and \citet{GeeslinDíaz-Campos2013}, where they reported an inverted u-shape for pronoun rates across proficiency. It has previously been proposed that L2ers at beginner levels might omit subjects, relying greatly on pragmatics and discourse context until their grammar is more developed. For intermediate and more advanced levels of proficiency, however, there is a consensus that development of L2 SPE rates moves from more overt pronominal subjects to fewer overt pronominal subjects. In this paper, we further examine the role of proficiency. We specifically examine possible differences in rates, patterns (as instantiated in the conditioning factors regulating the distribution), and their interaction with priming across different speaker groups, including two proficiency groups of L2ers. Our aim is to compare higher and lower proficiency L2ers to bilingual NSs in their use of Spanish SPE, with respect to (i) rates of overt pronominal subjects, (ii) patterns of use as instantiated in functional factors, and (iii) the effect of the mechanical factor, perseveration.

\section{Background and motivation}
\subsection{Subject expression in Spanish}
Variationist analyses of Spanish subject expression have indicated that SPE in Spanish is best explained by a combination of variables \citep{CarvalhoShin2015}, except in those cases traditionally excluded from variable rule analyses: predicates that require an expletive subject, predicates accompanying impersonal uses of the second person singular and third person plural, reverse psychological predicates, predicates in subject relative clauses, subjects with inanimate referents, and predicates in set phrases. In each case, the null pronominal subject fails to alternate with an overt counterpart.

In general terms, null subjects tend to indicate continuity in variable contexts. Thus, the factor Switch Reference, or whether the subject in the preceding sentence is the same or not, favors the use of a null subject when the previous subject has the same referent. Another relevant factor is the variable Person. Previous research indicates that Spanish speakers use more overt subjects in 1\textsc{sg} than in 3\textsc{sg}. \citet{TravisCacoullos2018} explored differences in 1\textsc{sg} and 3\textsc{sg} subjects with respect to accessibility of reference. They reported accessibility impacted 1\textsc{sg} at a shorter distance from previous mention, than 3\textsc{sg} subjects. Although there are other factors identified in previous research, Switch Reference and Person are significant and highly ranked (i.e., larger magnitude of effect) across studies and are, thus, the object of study in this paper.

In addition to functional factors, there is one mechanical factor that has received attention in the previous literature, perseveration or priming. Several previous studies using variable rule analyses have reported that pronouns lead to pronouns and null subjects to null subjects (\citealp{Abreu2012}; \citealp{Cameron1994}; \citealp{CameronFlores-Ferrán2004}; \citealp{FloresFerrán2002}; \citealp{TravisCacoullos2012}; \citealp{Travis2007}). In a rather different approach, \citet{Otheguy2015} presented cross-tabulated data from eight interviews from \citegen{OtheguyZentella2012} NYC corpus and concluded that there was no priming effect. \citet{Otheguy2015} reported the rates of overt pronominal subjects preceded by pronominal subjects vs. a different type of subject and the rates of null subjects preceded by null subjects vs. a different type of subject. In their data, null subjects tended to be preceded by null subjects (perseveration) while overt pronominal subjects tended to be preceded by a different type of subject (interspersion). Given the preponderance of null subjects in Spanish, it is likely that the analysis was affected by their frequency. In variable rule analyses, in contrast, the analyses compared the likelihood of having an overt pronominal subject vs. a null subject when the preceding subject was also pronominal vs. when it was null or lexical or any other form. Thus, although there is some discrepancy, overall, in Spanish, previous research concludes that a pronominal subject increases the likelihood of using a pronominal subject in a subsequent clause.

Lastly, it is important to note that there is an interaction between Switch Reference and Priming. \citet{CameronFlores-Ferrán2004} reported that the priming effect is larger in contexts of same reference than in contexts of switch reference. Similarly, the priming effect was larger in 1\textsc{sg} than in 3\textsc{sg} subjects in Heritage Speakers of Spanish (HS) in \citet{PradaPérez2020}.

\subsection{Subject expression in L2 Spanish}

Across studies, the variable Switch Reference has consistently been found to be a significant predictor of SPE. Specifically, NSs and L2ers of different proficiencies use more overt pronominal subjects when the previous subject has a different referent than when it is coreferential (\citealp{LinfordShin2013}; \citealp{Linford2014}). For instance, from the seven participant groups, only the two with the lowest proficiency were not sensitive to this variable in \citet{GeeslinDíaz-Campos2013} and \citet{GeeslinFafulas2015}. Additionally, sensitivity to this factor has been shown to be restricted to L2ers with higher self-reported motivation and those with at least an intermediate proficiency level \citep{Linford2014}. \citet{PradaPérezFeroce2020} examined 1\textsc{sg} and 3\textsc{sg} subjects separately. They found that higher proficiency L2ers were sensitive to Switch Reference in both persons although with a smaller effect size in 1\textsc{sg} than the control group. In contrast, the lower proficiency L2ers were sensitive to this variable only in 3\textsc{sg}. The authors proposed that acquiring 3\textsc{sg} null subjects was more difficult due to null subjects being more marked in 3\textsc{sg} \citep{Artstein1999}. Thus, both NS and L2 sensitivity to Switch Reference may differ across grammatical persons.

With respect to the variable Person, \citet{Abreu2009} reported a significant effect for the variable in all her participant groups, consisting of 10 Spanish (functionally) monolingual speakers from Puerto Rico, with 10 Spanish heritage speakers and 10 Spanish L2ers. Although it was the highest ranked variable in both the NSs and the L2ers, there were differences between the groups in the specific persons that favored the use of overt pronominal subjects more. For NSs, the hierarchy of grammatical persons in descending order was 2\textsc{sg}, 1\textsc{sg}, 3\textsc{sg}, 1\textsc{pl}, and 3\textsc{pl} subjects. For L2ers the hierarchy in descending order was 3\textsc{sg}, 1\textsc{sg}, 2\textsc{sg}, 3\textsc{pl}, and 1\textsc{pl}. \citet{LinfordShin2013} also reported an effect for person, albeit only for the higher proficiency group and \citet{GeeslinDíaz-Campos2013} and \citet{GeeslinFafulas2015} for two of the seven groups (7th semester and 4th year undergraduate students). These groups used significantly more overt pronominal subjects in 3\textsc{sg} than in 1\textsc{sg}, a trend also reported in \citet{Linford2014} and \citet{PradaPérezFeroce2020}. \citet{GeeslinGudmestad2008} also reported differences between L2ers and NSs in 3\textsc{sg} contexts and not in 1\textsc{sg} contexts, in particular with the L2ers higher use of lexical subjects in 3\textsc{sg} than NSs. \citet{Long2016} did not report Person as a significant variable (except for Level 2 and Level 4 L2ers). However, this may have been due in part to treating grammatical person and number as separate variables. Thus, it is likely that Person was not significant given differences in Number. For example, 3\textsc{sg} has been consistently found to favor overt pronominal subject while 3\textsc{pl} does not. Since \citet{Abreu2009}, \citet{Linford2014}, \citet{LinfordShin2013}, \citet{Long2016}, and \citet{GeeslinGudmestad2008} did not perform separate analyses for different persons, it is not possible to know if the differences in rates across persons also indicated differences in the variables that were significant in 1\textsc{sg} vs. 3\textsc{sg}, for instance. Within the same dataset, subsequent papers by Gudmestad and colleagues examined the distribution of subject forms in 1st and 2nd person subjects \citep{GeeslinGudmestad2016} and in 3\textsc{sg} subjects \citep{GudmestadGeeslin2013}. \citet{GeeslinGudmestad2016} compared 1st person and 2nd person subjects and concluded that NSs and L2ers were very similar in 1st person both in terms of rates of overt pronominal subject expression as well as the variables that regulated the distribution. \citet{GudmestadGeeslin2013} examined 3\textsc{sg} subjects in the same dataset. They also reported differences between the NSs and L2ers but not in the alternation between null and overt pronominal subjects but in their use of other pronominal subjects and lexical subjects (see also \citealt{LinfordGeeslin2016}). Finally, \citet{PradaPérezFeroce2020} also examined 1\textsc{sg} subjects separately from 3\textsc{sg} subjects, revealing that L2ers differed from NSs more in 3\textsc{sg} subjects in rates of use. They, however, were more similar to NSs in 3\textsc{sg} when examining the variables that regulated SPE.

With respect to the variable perseveration in L2 SPE, most studies have coded for the form of previous mention (\citealp{Abreu2009}; \citealp{GeeslinGudmestad2011}, \citeyear{GeeslinGudmestad2016}; \citealp{GudmestadGeeslin2013}; \citealp{Long2016}), that is, the form of its referent, whether it was found in the immediately preceding clause or at a longer distance, while \citet{Linford2016} and \citet{Zahler2018} coded for the immediately preceding subject. Overall, these studies report an effect of perseveration in L2 groups that parallels that of NSs, i.e., speakers used more overt pronominal subjects when the previous subject was pronominal than when it was null.

\section{The present paper}
\subsection{Research questions and hypotheses}
\largerpage
The overarching question of this paper is: What are the effects of Switch Reference, Form of Previous Subject, and Speaker group on the SPE of Spanish L2ers and NSs? And how do these factors interact?

First, we examine whether speakers are sensitive to Switch Reference. It is expected that the NSs produce more overt pronominal subjects in contexts of different than in contexts of same reference. Additionally, it is expected that learners are sensitive to this variable as proficiency increases. Thus, an interaction between Switch Reference and Speaker group is anticipated, particularly for 1\textsc{sg} subjects \citep{PradaPérezFeroce2020}.

With respect to the effect of perseveration or priming, based on previous L2 studies (e.g., \citealp{Abreu2009}), we expect the variable Form of Previous Subject to be significant in our data and to have a similar effect across speaker groups. In line with \citet{Cameron1994}, a Switch Reference by Form of Previous Subject interaction is expected in our data, where priming was only observable in coreferential subjects.

Lastly, the role of Speaker group is examined through comparisons among the three groups of speakers (higher and lower proficiency L2ers, and a NS group) with respect to rates but also interactions with Switch Reference and Form of Previous Subject. Differences across proficiency groups have been reported in previous studies in terms of rates and conditioning factors, where L2ers, particularly those at lower proficiency levels, use overt pronominal subjects more than NSs (cf. \citealp{GeeslinDíaz-Campos2013}; \citealp{GeeslinFafulas2015}), particularly in 3\textsc{sg} subjects, and are not as sensitive to functional factors as NSs \citep{PradaPérezFeroce2020}. Thus, we expect our data to show differences across the three speaker groups in rates of overt pronominal subjects and in sensitivity to Switch Reference (Switch Reference by Speaker group interaction). With respect to perseveration, however, we do not anticipate differences between the different speaker groups, as it is a mechanical factor.

\subsection{Participants}
\largerpage
A total of 28 native English speakers (19 females, age 19--24) and 9 bilingual NSs (8 females, age 18--22) in 3rd or 4th level Spanish courses were recruited from a public university in the U.S. All the NSs reported acquiring Spanish as children in their homes and were speakers of non-Caribbean dialects: Argentina (n = 2), Chile (n = 1), Colombia (n = 2), Ecuador (n = 1), El Salvador (n = 1), Nicaragua (n = 1), and Peru (n = 1). A total of six bilingual NSs were born abroad and migrated to the US at ages 5 (n = 1), 6 (n = 2), and 7 (n = 3). All the L2 participants reported learning Spanish after the age of 12 and in a classroom context. Nine of the L2 participants reported having studied abroad in either Costa Rica, Dominican Republic, Peru, or Spain (mean time abroad: 4.7 months, range: 2 weeks--30 months).

 All participants, except for one L2er, completed an independent measure of proficiency. The NSs scored higher (15--45, M = 42/50, SD = 9.9) than the L2ers (23-45, M = 34/50, SD = 6.9). The L2 participants were separated into two groups based on a median split of their proficiency scores (Median = 34): higher proficiency group, or L2H, (n = 15) and lower proficiency group, or L2L, (n = 13). The participant who did not complete the proficiency measure was assigned to the lower proficiency group based on the perception of the authors in comparison with other participants in the study. The bilingual NSs were not further divided into two groups, given the small number of participants. Additionally, five scored above 40/50 and the other four were not perceived to be less proficient by the authors in the interview. Three of these four were born in a Spanish-speaking country and migrated to the US when they were either 6 (n = 2) or 7 (n = 1). To further support our perception, we measured the number of words in a minute in the recording (minute 10 to 11, M = 126, SD = 17.9) and those four participants with scores lower than 40/50 in the proficiency measure did not appear to be less fluent than the others, as can be observed in \tabref{Table 1}.

\begin{table}
\caption{HS proficiency and fluency results}
\label{Table 1}
 \begin{tabular}{l rr}
  \lsptoprule
Participant    & Proficiency score  & Fluency (words per minute)\\
  \midrule
  HS1  &   35  &   118\\
  HS2  &   28 &   123\\
  HS3  &   42 &   141\\
  HS4  &   28 &   114\\
  HS5  &   15 &   170\\
  HS6  &   44 &   128\\
  HS7  &   46 &   103\\
  HS8  &   44 &   131\\
  HS9  &   43 &   126\\
  \lspbottomrule
 \end{tabular}
\end{table}

As for their language use, the bilingual native speakers reported using Spanish every day, most of the day (n = 2), or sporadically throughout the day (n = 2), several times per week (n = 4), or once or twice per month (n = 1). In contrast, they all reported speaking English every day for most of the day. The Spanish L2ers reported using Spanish every day, sporadically throughout the day (n = 13), several times per week (n =13), once per week (n =1), or once or twice per month (n = 1). With respect to English, they speak it every day, most of the day (n = 24) or sporadically throughout the day (n = 2)\footnote{One of the L2ers only completed the proficiency measure, another one only completed the Language Background Questionnaire (LBQ), and another one did not complete the LBQ or the proficiency measure. During the interview some of their background was discussed, but we have no comparable measure of how often they use their languages.}.

We acknowledge that the number of participants, particularly in the bilingual NS group, is rather low, and may incur in a Type II error, where some trends may not reach significance due to the low number of participants. We also recognize the difficulty that emerges to compare across studies when the comparison group is different from that used in previous studies. In order to address the comparative fallacy in L2 research \citep{Bley-Vroman1983}, we chose to compare our L2 group to another Spanish-English bilingual group they are comparable to in other ways (e.g. age, occupation, and state where they grew up).

\subsection{Materials and coding}

Participants completed a PowerPoint-guided sociolinguistic interview with the first author (a native Spanish speaker), a language background questionnaire (LBQ), and a proficiency test. In the LBQ, participants were asked to provide information on their personal history, their language history, their reported language use, and their self-reported proficiency. The Spanish proficiency test, with a total of 50 questions, consisted of a multiple-choice grammar section and a cloze test, based on the Diploma de Español como Lengua Extranjera (DELE), and is widely used in the field. The LBQ and the proficiency test were presented on the online platform Qualtrics and were completed after the interview.
The interviews lasted between 30 and 60 minutes. Participants were asked to consider the interview an informal conversation. The interview data was transcribed and coded. Only 1\textsc{sg} and 3\textsc{sg} tokens in variable contexts were included in the analysis. Variables of interest to this study were: Subject form (null or overt pronominal subject), Person (1\textsc{sg} or 3\textsc{sg}), Switch Reference (same referent or different referent), and Form of Previous Subject (null, overt pronominal, lexical). For Form of Previous Subject, we did not limit previous subjects to be eligible based on functional factors, such as coreference or eligibility (e.g., lexical subjects, unlike \citealp{Abreu2009}; \citealp{FloresFerrán2002}; \citealp{TravisCacoullos2012}; \citealp{Travis2007}) or animacy (in contrast with \citealp{Otheguy2015}). \citet{Otheguy2015} explained the motivation for a wider inclusion was to examine perseveration as a mechanical motivation. We also coded for one extra-linguistic variable: Speaker group (NSs, L2H, and L2L).

\subsection{Results}

\subsubsection{1\textsc{sg}}

Pronoun production rates for 1\textsc{sg} are presented in \figref{Figure 1} and were statistically analyzed via logistic mixed effects models using the function \emph{glmer()} in R \citep{RCoreTeam2017}. Separate analyses were done for 1\textsc{sg} and 3\textsc{sg}, given the different nature (deictic vs. referential) and behavior attested, e.g. different variables as significant \citep{PradaPérezSoler2020} and different patterns \citep{TravisCacoullos2018}. The results for the model below reveal the likelihood that a speaker would produce an overt pronoun over a null pronoun. We included all main effects and interactions between Switch Reference (Same referent, Different referent), Form of Previous Subject (Null pronoun, Overt pronoun, Lexical subject), and Speaker group (Natives, L2H, L2L), as well as random intercepts of Participant and Verb. To assess the overall contribution of the main factors and interactions, model comparisons were conducted by removing individual terms from the regression and comparing the reduced models against the maximal model. This revealed significant contributions of the main factors of Switch Reference (χ2(9) = 44.928, p < 0.001), Form of Previous Subject (χ2(12) = 25.836, p = 0.011), and Speaker group (χ2(12) = 22.378, p = 0.034), as well as the interaction between Switch Reference and Speaker group (χ2(6) = 12.979, p = 0.043). There were no significant reductions in model fit when removing the interactions between Switch Reference and Form of Previous Subject (χ2(6) = 5.160, p = 0.524), nor Form of Previous Subject and Speaker group (χ2(8) = 13.068, p = 0.110), nor the interaction between Switch Reference, Form of Previous Subject, and Speaker group (χ2(4) = 4.924, p = 0.295). This suggests that the variability in the data was primarily driven by the main factors and the interaction between Switch Reference and Speaker group. The maximal model was used to analyze statistical comparisons across conditions by releveling the reference levels for each factor. We report comparisons for each condition and group, but note that these should be taken with caution as there was no significant three-way interaction. We provide these comparisons to provide the most transparent description of the data as they raise theoretically interesting issues for future research to continue investigating, but we acknowledge that these are merely exploratory at present.

For Switch Reference, all speaker groups produced overt pronouns at a higher rate in different reference than same reference contexts, but this was modulated by the form of the previous subject. For the NSs, the effect of Switch Reference was significant when the previous subject was a null pronoun (β = 1.059, SE = 0.214, z = 4.947, p < 0.001) or a lexical subject (β = 1.378, SE = 0.616, z = 2.237, p = 0.025), but not when it was a pronoun (β = 0.771, SE = 0.602, z = 1.281, p = 0.200). For L2Hs, the effect of Switch Reference was marginally significant when the previous subject was a null pronoun (β = 0.339, SE = 0.197, z = 1.721, p = 0.085), but not when it was a pronoun (β = -0.025, SE = 0.396, z = $-$0.063, p = 0.950) or a lexical subject (β = 1.442, SE = 1.046, z = 1.379, p = 0.168). For the L2Ls, the effect of Switch Reference was significant when the previous subject was a null pronoun (β = 0.472, SE = 0.220, z = 2.148, p = 0.032) or an overt pronoun (β = 0.860, SE = 0.391, z = 2.202, p = 0.028), but not when it was a lexical subject (β = $-$0.663, SE = 0.872, z = $-$0.761, p = 0.447).

\hspace*{-2.4pt}In same reference contexts, pronoun production rates were modulated by Form of Previous Subject, based on comparisons to a previous null subject. NSs did not produce pronouns at a significantly higher rate when the previous form was an overt pronoun (β = 0.507, SE = 0.343, z = 1.477, p = 0.140) or a lexical subject (β = $-$0.292, SE = 0.576, z = $-$0.507, p = 0.612). L2Hs produced pronouns at a higher rate when the previous subject was an overt pronoun (β = 0.697, SE = 0.253, z = 2.749, p = 0.006), but not when it was a lexical subject (β = $-$0.332, SE = 1.038, z = $-$0.320, p = 0.749). L2Ls did not produce a pronoun at a higher rate when the previous form was an overt pronoun (β = $-$0.703, SE = 0.870, z = $-$0.808, p = 0.419) or a lexical subject (β = 0.703, SE = 0.871, z = 0.807, p = 0.420).

In different reference contexts, pronoun production rates were also modulated by Form of Previous Subject, based on comparisons to a previous null subject. NSs did not produce more overt pronouns when the previous form was an overt pronoun (β = 0.219, SE = 0.539, z = 0.407, p = 0.684) or a lexical subject (β = 0.027, SE = 0.290, z = 0.094, p = 0.925). L2Hs produced more overt pronouns when the previous subject was a lexical subject (β = 0.772, SE = 0.238, z = 3.241, p = 0.001), but not when it was an overt pronoun (β = 0.333, SE = 0.360, z = 0.926, p = 0.354). L2Ls did not produce more overt pronouns when the previous form was an overt pronoun (β = 0.576, SE = 0.356, z = 1.617, p = 0.106) or a lexical subject (β = $-$0.245, SE = 0.308, z = $-$0.795, p = 0.426).

Pronoun production rates were also modulated by Speaker group. In same reference contexts, the L2Ls produced overt pronouns at a marginally higher rate than the NSs when the previous subject was a lexical subject (β = 1.890, SE = 1.110, z = 1.702, p = 0.089) and marginally produced more overt pronominal subjects than the L2Hs when the previous subject was a null pronoun (β = 0.836, SE = 0.436, z = 1.915, p = 0.055). In different reference contexts, the L2Ls produced overt pronominal subjects at a higher rate than the L2Hs when the previous subject was an overt pronoun (β = 1.212, SE = 0.598, z = 2.026, p = 0.043) or a null pronoun (β = 0.969, SE = 0.466, z = 2.082, p = 0.037). Additionally, the NSs produced overt pronouns at a marginally higher rate than the L2Hs when the previous subject was a null pronoun (β = 0.849, SE = 0.497, z = 1.707, p = 0.088).


\begin{figure}
  \caption{Average percent pronoun rates for 1\textsc{sg} with 95\% confidence interval bars.}
  \label{Figure 1}
    \includegraphics[width=\textwidth]{figures/Prada_fig1.png}
\end{figure}

\subsubsection{3\textsc{sg}}

Pronoun production rates for 3\textsc{sg} are presented in \figref{Figure 2} and were analyzed using the same methods as for 1\textsc{sg}. Model comparison analyses revealed significant contributions of the main factors of Switch Reference (χ2(9) = 78.246, p < 0.001), Form of Previous Subject (χ2(12) = 26.137, p = 0.010), and Speaker group (χ2(12) = 21.945, p = 0.038) as well as the interaction between Switch Reference and Form of Previous Subject (χ2(6) = 18.147, p = 0.006). There were no significant reductions in model fit when removing the interactions between Switch Reference and Speaker group (χ2(6) = 2.617, p = 0.855), Form of Previous Subject and Speaker group (χ2(8) = 6.965, p = 0.540), nor Switch Reference, Form of Previous Subject, and Speaker group (χ2(4) = 2.563, p = 0.633). This suggests that variability in the data was primarily driven between the main factors, as well as the interaction between Switch Reference and Form of Previous Subject. Similar to 1\textsc{sg}, comparisons across individual conditions for 3\textsc{sg} were analyzed based on releveling the reference levels for each factor in the maximal model.

For Switch Reference, all speaker groups produced more overt pronominal subjects in different reference than same reference contexts, but this was modulated by the Form of the Previous Subject. For the NSs, the effect of Switch Reference was significant when the previous subject was a null pronoun (β = 1.586, SE = 0.424, z = 3.740, p < 0.001), was marginal when it was an overt pronoun (β = 1.291, SE = 0.691, z = 1.868, p = 0.062), but was not significant when it was a lexical subject (β = $-$0.855, SE = 1.178, z = $-$0.725, p = 0.468). For the L2Hs, the effect of Switch Reference was significant when the previous subject was a null pronoun (β = 1.576, SE = 0.295, z = 5.346, p < 0.001), was marginal when it was an overt pronoun (β = 0.980, SE = 0.532, z = 1.842, p = 0.066), but was not significant when it was a lexical subject (β = 0.254, SE = 0.459, z = 0.553, p = 0.580). For the L2Ls, the effect of Switch Reference was significant when the previous subject was a null pronoun (β = 1.446, SE = 0.367, z = 3.943, p < 0.001), or an overt pronoun (β = 2.115, SE = 0.813, z = 2.602, p = 0.009), but not when it was a lexical subject (β = $-$0.253, SE = 0.606, z = $-$0.418, p = 0.678).

\hspace*{-2.4pt}In same reference contexts, pronoun production rates were modulated by Form of Previous Subject, based on comparisons to a previous null subject. NSs produced significantly more overt pronominal subjects when the previous form was an overt pronoun (β = 1.170, SE = 0.595, z = 1.965, p = 0.049) but not when it was a lexical subject (β = 0.614, SE = 0.619, z = 0.992, p = 0.321). L2Hs produced overt pronominal subjects at a marginally higher rate when the previous subject was a lexical subject (β = 0.654, SE = 0.341, z = 1.918, p = 0.055), but not when it was an overt pronoun (β = 0.426, SE = 0.281, z = 1.516, p = 0.129). L2Ls did not produce more overt pronominal subjects when the previous form was an overt pronoun (β = 0.038, SE = 0.346, z = 0.111, p = 0.912) or a lexical subject (β = 0.396, SE = 0.376, z = 1.055, p = 0.291).

In different reference contexts, pronoun production rates were also modulated by Form of Previous Subject, based on comparisons to a previous null subject. NSs did not produce overt pronominal subjects at a higher rate when the previous form was an overt pronoun (β = 0.874, SE = 0.550, z = 1.589, p = 0.112) but produced overt pronominal subjects at a marginally lower rate when the previous subject was a lexical subject (β = -1.826, SE = 1.085, z = $-$1.684, p = 0.092). L2Hs did not produce more overt pronominal subjects when the previous subject was an overt pronoun (β = $-$0.169, SE = 0.546, z = $-$0.310, p = 0.757), or a lexical subject (β = $-$0.668, SE = 0.420, z = $-$1.592, p = 0.111). L2Ls did not produce more overt pronominal subjects when the previous form was an overt pronoun (β = 0.707, SE = 0.817, z = 0.865, p = 0.387) but produced overt pronominal subjects at a lower rate when the previous subject was a lexical subject (β = $-$1.303, SE = 0.598, z = $-$2.178, p = 0.029).

Pronoun production rates were also modulated by Speaker group. In same reference contexts, the L2Ls produced more overt pronominal subjects than the NSs when the previous subject was a lexical subject (β = 1.723, SE = 0.698, z = 2.468, p = 0.014) or a null pronoun (β = 1.941, SE = 0.552, z = 3.519, p < 0.001). The L2Hs also produced overt pronominal subjects at a higher rate than the NSs when the previous subject was a lexical subject (β = 1.688, SE = 0.690, z = 2.447, p = 0.014) or a null pronoun (β = 1.649, SE = 0.523, z = 3.153, p = 0.002). In different reference contexts, the L2Ls produced more overt pronominal subjects than the NSs when the previous subject was a null pronoun (β = 1.800, SE = 0.532, z = 3.386, p = 0.001), and marginally more overt pronominal subjects when the previous subject was a lexical subject (β = 2.324, SE = 1.231, z = 1.888, p = 0.059) or an overt subject pronoun (β = 1.633, SE = 0.972, z = 1.680, p = 0.093). The L2Hs produced more overt pronominal subjects than the NSs when the previous subject was a lexical subject (β = 2.797, SE = 1.162, z = 2.408, p = 0.016) or a null pronoun (β = 1.638, SE = 0.497, z = 3.295, p = 0.001).

\begin{figure}
  \caption{Average percent pronoun rates for 3\textsc{sg} with 95\% confidence interval bars.}
  \label{Figure 2}
    \includegraphics[width=\textwidth]{figures/Prada_fig2.png}
\end{figure}

The trends reported with respect to Switch Reference and Form of Previous Subject per participant group are summarized in \tabref{Table 2} and \ref{Table 3}. There was no significant three-way interaction for either 1\textsc{sg} or 3\textsc{sg}. Rather, the results for 1\textsc{sg} were driven by main effects and the interaction between Switch Reference and Group, while those for 3\textsc{sg} are driven by main effects and the interaction between Switch Reference and Form of Previous Subject.

\begin{table}
\caption{Summary of results for the effect of Switch Reference for each participant group (\textit{p}-values provided for marginally significant results)}
\label{Table 2}
\small
 \begin{tabularx}{\textwidth}{ Q  QQQQQl}
  \lsptoprule
    &\multicolumn{3}{c}{1st singular}&\multicolumn{3}{c}{3rd singular}\\
 \cmidrule(lr){2-4} \cmidrule(lr){5-7}
   Form of prev  &Null   &Overt  &Lexical    &Null   &Overt  &Lexical\\
    \midrule
     {Speaker group} &    &   &   &\\
     \midrule
  Natives & Diff> Same & n.s. & Diff> Same & Diff> Same & Diff> Same

  (\textit{p} = .062)  & n.s.\\

  L2H & Diff> Same

  (\textit{p} = .085) & n.s. & n.s. & Diff> Same & Diff> Same

  (\textit{p} = .066)  & n.s.\\

   L2L & Diff> Same & Diff> Same & n.s. & Diff> Same & Diff> Same  & n.s.\\

  \lspbottomrule
 \end{tabularx}
\end{table}

\begin{table}
\caption{Summary of results for the effect of Form of Previous Subject for each participant group (\textit{p}-values provided for marginally significant results)}
\label{Table 3}
 \begin{tabularx}{\textwidth}{ Q  QQQQQl}
  \lsptoprule
 \setlength{\tabcolsep}{18pt}
      &\multicolumn{2}{c}{1st singular}&\multicolumn{2}{c}{3rd singular}\\
 \cmidrule(lr){2-3}\cmidrule(lr){4-5}
 Switch reference    &Same ref   &Different ref  &Same ref   &Different ref\\
    \midrule
     Speaker group &    &   &   &\\
     \midrule
 Natives & n.s & n.s. & Overt > Null & Null > Lex

 (\textit{p} = .092)\\

  L2H & Overt > Null & Lex > Null & Lex > Null

  (\textit{p} = .055) &  n.s.\\

   L2L & n.s. & n.s. & n.s. & Null > Lex.\\

  \lspbottomrule
 \end{tabularx}
\end{table}

\section{Discussion}

Differences between speaker groups were explored in rates, as well as patterns of use. With respect to rates of overt pronominal subjects, different trends were found for 1\textsc{sg} and 3\textsc{sg} subjects. For 1\textsc{sg} subjects, there was evidence that the L2Hs were overshooting the target. While their rates were comparable to those of the NSs in other contexts, they used significantly fewer overt pronominal subjects in different reference contexts (particularly with a previous null subject) than the NSs. The overuse of overt pronominal subjects was only attested in the L2Ls in certain contexts in our data. For 3\textsc{sg} subjects, the rates of both learner groups differed from the NSs in similar ways. This result is in line with previous research where L2ers are found to produce more overt pronominal subjects than NSs, except for the Geeslin and Gudmestad studies with near-native speakers, where they did not find differences. The proficiency of the L2H group is probably lower than the near-natives in these studies, explaining this difference.

To answer our research question on the effects and interactions of the variables under study, we turn to each variable. The results revealed interactions between the three variables, but no three-way interaction was reported either for 1\textsc{sg} or for 3\textsc{sg} subjects.

Regarding the effect of Switch Reference, we expected that the NSs would produce more overt pronominal subjects in contexts of different reference than in contexts of same reference. Additionally, it was expected that learners would be increasingly sensitive to this variable as proficiency increased. Switch Reference was significant both in 1\textsc{sg} and in 3\textsc{sg}, where speakers tended to use more overt subjects in different than in same reference contexts. However, this production was modulated by other factors. In 1\textsc{sg}, there was a Switch Reference by Speaker group interaction. While the results revealed that this interaction was present at all levels of the variable Form of Previous Subject, not all these contexts were as informative. Importantly, it is in the context of a previous null subject where we can observe more meaningful contrasts between the groups with respect to the variable Switch Reference. The difference among the groups is reflected in effect size: Switch Reference had a large effect for the NSs, a smaller effect for the L2Ls, and a marginal effect for the L2Hs. The results for the L2Hs are surprising, as we expected L2ers to improve their sensitivity to this variable as proficiency increased. As previously attested in the literature with respect to rates of overt pronominal subjects (\citealp{GeeslinDíaz-Campos2013}; \citealp{GeeslinFafulas2015}), the path of L2 acquisition is not necessarily straight. In this case, the underuse of overt pronominal subjects in the L2H group brought about a loss of sensitivity to Switch Reference in 1\textsc{sg} subjects. This is not the case however, for 3\textsc{sg} subjects, where Switch Reference did not interact with Speaker group, suggesting that the use of overt pronominal subjects as a marker of a different referent was similar across the three groups of speakers. On the other hand, there was a significant interaction of Switch Reference with Form of Previous Subject for 3\textsc{sg} subjects. While Switch Reference was significant for all the groups when the previous subject was null, it was only significant for the L2Ls and marginally for the other two groups when the previous subject was pronominal. Thus, pronominal priming mitigated the Switch Reference effect. Additionally, Switch Reference was not significant for any of the groups when the previous subject was lexical. Since we only analyzed tokens with null or overt pronominal subjects and did not include lexical subjects, we cannot explore the hypothesis that lexical subjects might be preferred in different referent contexts after a lexical subject in the previous clause, in line with \citet[256]{Lozano2016}.

With respect to the effect of perseveration or priming, we predicted that Form of Previous Subject and its interaction with Switch Reference would be significant in our data. Additionally, as a mechanical factor, it was expected to have a similar effect across speaker groups and, thus, no interaction with Speaker group was anticipated. The results showed that Form of Previous Subject was indeed significant in our data. In 1\textsc{sg} data, there was evidence of pronominal priming. In 3\textsc{sg} subjects, there was a main effect of Form of Previous Subject but also an interaction with Switch Reference. Crucially, previous pronominal subjects favored the use of overt pronominal subjects more than previous null subjects for the NSs, which is evidence of pronominal priming, in contexts of same reference. However, this effect dissipated in contexts of different referent, a result in line with \citet{Cameron1994}. The trend to produce more overt pronominal subjects with a preceding overt pronominal subject did not reach significance in the learner groups, possibly due to the high rate of overt pronominal subjects in 3\textsc{sg} when the previous subject is null.

Lastly, for Speaker group, our results were consistent with our hypotheses. L2Ls produced significantly more overt pronominal subjects than NSs in 1\textsc{sg}. Additionally, the L2ers differed from NSs in sensitivity to Switch Reference. In 3\textsc{sg}, both learner groups used significantly more overt pronominal subjects than the NS group although their patterns of use paralleled those of NSs.

\section{Conclusion}

This paper further examined SPE in L2 Spanish in contrast to the Spanish of NSs, expanding on previous research by focusing on the interaction between variables that have previously received extant attention: Switch Reference and perseveration. The main results reported in this paper were:

\begin{itemize}
\item	Rates: The L2L group produced more overt pronominal subjects than the bilingual NS group in 1\textsc{sg} and 3\textsc{sg}. The L2H group produced more overt pronominal subjects than the bilingual NS group only in 3\textsc{sg} subjects.

\item	Patterns: L2L and L2H differed from NSs in sensitivity to Switch Reference in 1\textsc{sg} but not in 3\textsc{sg}.

\item	Perseveration: There was evidence of pronominal perseveration in our data, but it affected 1\textsc{sg} and 3\textsc{sg} subjects differently. With 1\textsc{sg} subjects, there was pronominal priming in all speaker groups. With 3\textsc{sg} subjects, pronominal priming was only evidenced in the NS group and it was modulated by Switch Reference such that the priming effect dissipated in contexts of Switch Reference.
\end{itemize}

For perseveration, we coded for the Form of Previous subject, instead of the form of previous mention, to make sure it was a mechanical factor. Contrasts of these two measures of perseveration should be performed in L2ers to be able to better compare across studies. A priming effect was reported across groups (in contrast with \citealp{Otheguy2015}). Since our coding and case inclusion largely followed \citegen{Otheguy2015} coding, we attribute this difference to a difference in analysis. While \citet{Otheguy2015} used cross tabulations, our analysis examined the probability that an overt subject was followed by another overt pronominal subject vs. a null subject. \citegen{Otheguy2015} analysis was more sensitive to the difference in overall frequency of null vs. overt pronominal subjects in the sample. Additionally, it collapsed across the category of different previous subjects rather than different subject forms (nulls, lexical, other types of subjects when the subject is overt pronominal in the second verb), which, as our data revealed, behave quite differently from each other. Our results align with \citegen{Cameron1994} data from NSs in that the priming effect is stronger in contexts of same reference than in contexts of different reference.
We argue that the examination of the interaction between mechanical and functional factors can better help us understand the acquisition path of SPE in Spanish as an L2. Our data revealed that Spanish L2ers differed from NSs in rates of overt pronominal subjects (particularly in 3\textsc{sg} subjects), as well as, in patterns of use, as exemplified by the functional variable, Switch Reference (particularly in 1\textsc{sg} subjects). Crucially, the interaction between Form of Previous Subject and Switch Reference, revealed that in contexts where those two factors may favor a different form, the effect of the other variable is obscured. For example, with different reference contexts, the effect of pronominal priming is obscured. Similarly, in contexts where there is a preceding overt pronominal subject, the effect of Switch Reference is weakened. The effect of priming, or its interaction with Switch Reference, was similar across the different groups of speakers. Overall, thus, our data indicated that differences between learners and NSs were restricted to rates and functional variables and were different in 1\textsc{sg} and 3\textsc{sg} subjects. We explain these differences based on functional factors, mainly discursive differences between 1\textsc{sg}, as a deictic person, and 3\textsc{sg}, as a referential person, and the saliency of these features in 3\textsc{sg} vs. 1\textsc{sg}. Our data is limited in scope (only including 1\textsc{sg} and 3\textsc{sg} data, only immediately preceding subject priming, and a limited number of learners as well as a NS group composed of heritage speakers) and, thus, invites further research expanding the persons included in the analysis, contexts of priming, and number and variety of participants. While the choice of the bilingual NS group may obscure comparisons with previous research, we argue that it is a welcome approach to minimize the comparative fallacy in L2 research \citep{Bley-Vroman1983}. We also presented some caveats related to our analysis and results: the low number of participants could have led to the possibility of a Type II error, the reporting of trends that did not reach significance (reporting on marginal significance) can lead to a Type I error, and reporting all pairwise comparisons even though there was no three-way interaction. These choices were made as to provide as much information as possible on the patterns attested in our data. Additionally, this paper contributes to the SLA literature by including data from a novel group of speakers and a new focus on the interaction of mechanical and functional factors and their different roles in SLA.


\section*{Acknowledgements}
The second author of this paper was supported by an NIDCD T32 training grant (5T32DC000052).

\printbibliography[heading=subbibliography,notkeyword=this]

\end{document}
