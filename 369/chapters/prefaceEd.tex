\documentclass[output=paper,draftmode]{langscibook}
\ChapterDOI{10.5281/zenodo.7525090}

\title[The digital transformation of the LSRL]{The digital transformation of the LSRL: The first 50 years of Romance linguistics in the Americas ends virtually}
\author{Barbara E. Bullock and Cinza Russi and Almeida Jacqueline Toribio    \affiliation{The University of Texas at Austin}}

\abstract{Widely considered to be the premier event in Romance linguistics worldwide, the Linguistic Symposium on Romance Languages (LSRL) offers a venue for the dissemination of the results of state-of-the-art research in linguistics as it is applied to the Romance languages. In hosting LSRL 50, we aimed to highlight innovative approaches to problems in Romance linguistics; provide a forum in which scholars of different orientations communicated with one another; showcase research that uses different types of data and methodologies; bridge linguistics with the STEM fields; promote a culture of shared tools and data; and actively involve students in conference activities. We reached our goals in the end but our path toward satisfying them was more complex than we could ever have anticipated. This chapter traces that path, surveys the many intellectual contributions that are contained within this volume, and offers our acknowledgments to those who helped us achieve our ends along the way.
}


\IfFileExists{../localcommands.tex}{
  \addbibresource{../localbibliography.bib}
  % add all extra packages you need to load to this file

\usepackage{tabularx,multicol}
\usepackage{url}
\urlstyle{same}

\usepackage{listings}
\lstset{basicstyle=\ttfamily,tabsize=2,breaklines=true}

\usepackage{langsci-basic}
\usepackage{langsci-optional}
\usepackage{langsci-lgr}
\usepackage{langsci-osl}
% \usepackage{./langsci/styles/langsci-lgr}
% \usepackage{./langsci/styles/langsci-osl}
% \usepackage{langsci-gb4e}

\usepackage{tikz}
\usetikzlibrary{patterns,calc}
\pgfdeclarepatternformonly{south east lines}{\pgfqpoint{-0pt}{-0pt}}{\pgfqpoint{3pt}{3pt}}{\pgfqpoint{3pt}{3pt}}{
    \pgfsetlinewidth{0.6pt}
    \pgfpathmoveto{\pgfqpoint{0pt}{3pt}}
    \pgfpathlineto{\pgfqpoint{3pt}{0pt}}
    \pgfpathmoveto{\pgfqpoint{.2pt}{-.2pt}}
    \pgfpathlineto{\pgfqpoint{-.2pt}{.2pt}}
    \pgfpathmoveto{\pgfqpoint{3.2pt}{2.8pt}}
    \pgfpathlineto{\pgfqpoint{2.8pt}{3.2pt}}
    \pgfusepath{stroke}}
    
\usepackage{stmaryrd}
\usepackage{wasysym}
\usepackage{multirow}
\usepackage{caption}
\usepackage{subcaption}
\usepackage{mathrsfs}
\usepackage{qtree}

\usepackage{linguex}


  %pminos do not split footnotes
% \interfootnotelinepenalty=10000 %Footnote in Laporte chapters has to be split SN


%\DeclareIndexNameFormat{default}{%
%\nameparts{#1}%
%\usebibmacro{index:name}%
%{\index[names]}%
%{\namepartfamily}%
%{\namepartgiveni}%
% {}% L1
% {}% L2
%{\namepartprefix}% generates spurious space L3
%{\namepartsuffix}% generates spurious space L4
%}

%  {\DeclareIndexNameFormat{default}{%
%     \usebibmacro{index:name}{\index[names]}{#1}{#3}{#5}{#7}}}

%\DeclareIndexNameFormat{default}{%
%  \usebibmacro{index:name}{\sindex[nom]}{#1}{#3}{#5}{#7}}

%\DeclareIndexNameFormat{default}{%
%  \usebibmacro{index:name}{\sindex[person]}{#1}{#3}{#5}{#7}}
%\DeclareIndexNameFormat{default}{%
%\nameparts{#1} \usebibmacro{index:name}{\sindex[person]]}{\namepartfamily}{‌​\namepartgiven}{\nam‌​epartprefix}{\namepa‌​rtsuffix}}

%\newcommand{\smiley}{:)}

%\renewbibmacro*{index:name}[5]{%
%\usebibmacro{index:entry}{#1}%
%{\iffieldundef{usera}{}{\thefield{usera}\actualoperator}\mkbibindexname{#2}{#3}{#4}{#5}}}

% \newcommand{\noop}[1]{}

%remove for final
%\overfullrule=1mm

\newcommand{\tobi}[2]}}
\renewcommand{\S}[1]{\tobi{#1}{\textsc{*}}}

% this volume references
% puts: [this volume]
% already defined: \citetv
%\newcommand{\citepv}[1]{(\citeauthor{#1} \citeyear*{#1} [this volume])}
\newcommand{\citealtv}[1]{\citeauthor{#1} \citeyear*{#1} [this volume]}

%parentheses around example number
\newcommand{\pref}[1]{(\ref{#1})}

% in-text examples

\newcommand{\lnex}[1]{\textit{#1}} %target lang word
\newcommand{\lnlit}[1]{(lit.: `#1')} %literal reading
\newcommand{\lnlat}[1]{(#1)} % latinization
\newcommand{\lntrans}[1]{`#1'} %translation
\newcommand{\lnexl}[2]%
{\lnex{#1}{} \lnlat{#2}} % ex with latinization
\newcommand{\lnexlat}[3]{\lnex{#1}{} \lnlat{#2}{} \lntrans{#3}} % ex with latinization and tranl.

%ch01
\newcommand{\co}[1]{\mbox{\textbf{#1}}}

%ch09

\newcommand{\cyrbulg}[1]{\begin{otherlanguage*}{bulgarian}#1\end{otherlanguage*}}


%ch10
\newcommand{\nlp}{{\small NLP}}
\newcommand{\mwe}{{\small MWE}}
\newcommand{\rae}{{\small RAE}}
\newcommand{\lvc}{{\small LVC}}
\newcommand{\pos}{{\small P}o{\small S}}
%\newcommand{\todo}[1]{ \textcolor{red}{#1} }

%\renewcommand{\labelenumi}{\theenumi}
%\ainamefmt{{vv}{ll}{, ff}{, jj}} % fullname

\newcommand{\biberror}[1]{{\color{red}#1}}

\newcommand{\osenovaitem}{--~}
  %% hyphenation points for line breaks
%% Normally, automatic hyphenation in LaTeX is very good
%% If a word is mis-hyphenated, add it to this file
%%
%% add information to TeX file before \begin{document} with:
%% %% hyphenation points for line breaks
%% Normally, automatic hyphenation in LaTeX is very good
%% If a word is mis-hyphenated, add it to this file
%%
%% add information to TeX file before \begin{document} with:
%% %% hyphenation points for line breaks
%% Normally, automatic hyphenation in LaTeX is very good
%% If a word is mis-hyphenated, add it to this file
%%
%% add information to TeX file before \begin{document} with:
%% \include{localhyphenation}
\hyphenation{
    Beck-man
    Ngu-yen
    back-chan-nel
    back-chan-nels
    mo-not-o-nous
    ste-reo-typ-i-cal
}

\hyphenation{
    Beck-man
    Ngu-yen
    back-chan-nel
    back-chan-nels
    mo-not-o-nous
    ste-reo-typ-i-cal
}

\hyphenation{
    Beck-man
    Ngu-yen
    back-chan-nel
    back-chan-nels
    mo-not-o-nous
    ste-reo-typ-i-cal
}

  % \togglepaper[3]%%chapternumber
}{}


\begin{document}
\maketitle

\noindent
\section{\#50 meets a (surmountable) obstacle}

The LSRL had been hosted, in-person, on a variety of campuses in the Americas on an uninterrupted basis since 1971. (In 2023, it is slated to be hosted in Paris, France, leaving the Americas for the first time after half a century.) Remarkably, there is no society or board behind this endeavor; instead, the tradition of LSRL is upheld from year to year by faculty and student volunteers who demonstrate their commitment to the promotion of linguistic research on Romance languages by sponsoring the conference on their campuses. LSRL 50 was to be hosted, for a record fourth time in its first half century, on the campus of the University of Texas at Austin, April 23--25, 2020. The program had been set, the travel arrangements for our plenary speakers were complete, coffee and breakfast tacos had been ordered for breaks, a poolside reception and a banquet had been planned, and the early registration period had concluded with nearly 200 attendees prepared to join us in Austin. However, on March 10, 2020, with heavy hearts, we canceled the in-person meeting in response to the perilous spread of the SARS-Covid-19 virus worldwide.

As is customary in the introduction to a volume of proceedings, we will address the scholarly content of LSRL 50 by overviewing the intellectual substance of our contributing authors. But before we do so, we present a brief history of how the conference was salvaged so that we may honor the many individuals who offered us their assistance and encouragement during such a difficult time. We also hope that our experiences in organizing an online symposium may be of benefit to future organizers who need to be prepared to transition to a digital platform should unbidden events conspire to derail their planning.  In our own case, luck intervened to help us carry through with our plans: just as we had to cancel LSRL 50, organizers from University of Massachusetts, Amherst announced on the listserv, the LINGUIST List, that the 33rd Annual CUNY Human Sentence Processing Conference would be held March 19--20, 2020, as planned, but in an online format. In early April 2020, we contacted Professor Mara Breen, the named conference organizer for CUNY 2020, for guidance and information about hosting a synchronous, online conference. In her response, she graciously detailed their strategies. Much of their practices were translated into a “How to LSRL 50!” link on our conference website that instructed attendees on (i) downloading and troubleshooting on the platform Zoom, (ii) registering to attend sessions, (iii) presenting papers and posters, and (iv) chairing sessions.

With a group of then graduate students in Romance Linguistics from our institution -- Dr. Aris Clemons, Dr. Tracey Adams, Dr. Joshua Franks, Anna Lawrence, and Luis Avilés González -- we mapped out the logistics of a new conference schedule. In May, 2020, we constructed a Qualtrics survey for our plenary and juried paper speakers and our session chairs. The survey gauged their interest in participating in the conference virtually, their preference for attending the conference on consecutive days or split over two weeks, and their preference for dates in late June or early July of 2020.  Nearly everyone responded that they would be pleased to participate and within four months of the cancellation of the physical conference, we hosted the event synchronously, online.

“LSRL 50 v2.0,” as we affectionately call our digital version, featured a novel program schedule. In order to accommodate a full conference program of juried papers, plenary talks, workshops, and a poster session, and to mitigate “Zoom fatigue,” we shortened the time allotted for the presentation of juried papers from the normal 20 minutes to 15 with the usual five minutes for discussion. And, rather than unfolding over 3 continuous days, the conference was held over five days, July 1--3, 2020 and, a week later, July 6--8, 2020, but only for three contiguous hours each day, from 17:00--20:00 GMT. This timing allowed individuals worldwide to attend the conference at reasonable morning, evening, or nighttime hours according to their respective time zones.

The program included 56 selected juried papers that were classified thematically into 15 separate sessions. The conference also included sessions for the juried posters, the keynote presentations, the two workshops, the business meeting, and the virtual “happy hour” with a trivia contest. Attendance at any session was free but individuals had to register for the session with its Zoom host, either one of us, or one of the remarkable graduate students, named above, who helped us get the conference off the ground. To register, prospective attendees clicked on the name of the Zoom host listed on the conference program that was published on our website. This automatically generated an email request for an invitation to the session. The Zoom host responded to each request with an invitation that provided the Zoom meeting ID for the session and a suggestion that the participant add the event to their calendar. The requirement to register for the conference provided us with a layer of security from “Zoom bombers” who otherwise could have interrupted sessions with malicious content. But, having experienced no outside interventions during the conference itself, we began to broadcast all the meeting IDs for the days’ sessions to the current attendees via the chat box in Zoom. This afforded them more spontaneity in choosing which session to attend and allowed them to move from one concurrent session to another if they chose to do so.

A week before the conference, all presenters and session chairs were invited to a training session with their Zoom host. The pre-raining assured us that everyone was comfortable with the platform and could access it from their locations. It also permitted speakers and chairs to interact before the conference and to ask any questions they might have about procedures before the event. That the essential participants of each session had already met and interacted with one another \textit{before} the conference took place was one of the many benefits of hosting the conference online. Unlike an in-person conference, there were no mispronounced speakers’ names and, during the conference, there was a notable congeniality between the speakers, the session chair, and the Zoom host that lent a more open, collaborative air to the discussions of the papers than that which often occurs in-person.

While the face-to-face interactions that normally occur at the LSRL were certainly missing in the online incarnation of LSRL 50, there were many benefits of the online format. The conference was free to anyone who wished to attend and accessible anywhere via an internet connection. This expanded the geographic exposure for LSRL and increased the number of attendees quite substantially. On Day 1, alone, we welcomed more than 250 individuals from every continent except Antarctica.  And no session was attended by less than 40 individuals. The technology of the Zoom platform served to enhance audience participation, too, as attendees were free to comment or pose questions of a speaker, at any time during the session, by using the chat function. And, at the conclusion of each session, the Zoom host remained in the meeting so that participants could interact with one another informally. In sum, transitioning to an online format provided a venue that fostered community among linguists worldwide and helped to decrease our sense of isolation during a difficult time.

\section{Situating the scholarly content of LSRL 50}
The origin of LSRL is rooted in generative grammar. First organized in February 1971 by Jean Casagrande and Bohdan Saciuk under the auspices of the Department of Romance Languages and Literatures and the Interdepartmental Linguistics Program at the University of Florida, the LSRL was billed as the \textit{Linguistic Symposium on Romance Languages: Application of Generative Grammar to their Description and Teaching}. The aim of the conference then, as now, was to contribute to the description of Romance languages while highlighting essential data in the evaluation, testing, and revision of theoretical proposals \citet{casagrande1972generative}. The first LSRL attracted research presentations on Romance languages from notable linguists of the era including Ronald W. Langacker, William Cressey, Albert Valdman, Sanford Schane, Maria Luisa Rivero, and Richard Kayne. The significance and impact cannot be overstated: Over the years, scholars of Romance linguistics have notably informed the direction of inquiry in general linguistics. As one example, \citegen{kayne1972subject} seminal contribution on syntax, \textit{Subject inversion in French interrogatives}, from the very first LSRL continues to accrue citations today. Distinguished linguists have continued to lay out their influential research programs in the proceedings of this venue, including Luigi Burzio, Luigi Rizzi, Margarita Suñer, An\-na  Cardinaletti, Karen Zagona, and Maria Luísa Zubizarreta in morpho-syntax and semantics, Irene Vogel, James Harris, Donca Steriade, Pilar Prieto, and José-Ignacio Hualde in phonology and phonetics, James Lantolf, Carmen Silva-Corvalán, David Birdsong, and Julia Herschensohn in bilingualism and second language acquisition, Shana Poplack, Raymond Mougeon and William Ashby in sociolinguistics, Yakov Malkiel, Dieter Wanner, and Jurgen Klausenburger in historical linguistics, among many, many others. Through the years, this state-of-the-field event has continued to attract the participation of prominent linguists and their students who have contributed to the further development of theoretical models and have helped to steer the field of linguistics in new directions. Many of them, like the current authors, have held or hold their appointments in language departments, where they serve as the point of first contact with the field of linguistics for legions of undergraduates. Collectively they have mentored generations of younger scholars who have gone on to complete their degrees in linguistics from language and from linguistic departments.

In celebrating its 50th anniversary, we envisioned a conference that would branch out from the strong, theoretical roots of LSRL and lay the groundwork for replicable research programs and data sharing practices that are necessary to move the field forward. In this, we built on and drew from successful iterations of previous meetings in emphasizing the empirical turn of Romance linguistics research, in particular LSRL 34 (University of Utah) which focused on experimental approaches, LSRL 43 (CUNY Graduate Center) with a workshop on parsed corpora, LSRL 48 (University of Delaware) which focused on bridges to other disciplines, and LSRL 49 (University of Georgia) whose theme was “Big Data.”

In the interest of documenting the history of the LSRL, we will archive the conference website, the conference Twitter feed from $@$LSRL50, and a complete database of the first 50 years of the published proceedings of the LSRL in the open access digital repository of Texas ScholarWorks of the University of Texas Libraries.

\section{The conference keynote addresses and workshops}

The keynote speakers of LSRL 50 were chosen to showcase research that employs diverse methods and different types of empirical observations applied to various Romance languages. These included two speakers from the hosting university: David Birdsong (Professor, French \& Italian) and Patience Epps (Professor, Linguistics). David Birdsong is a psycholinguist of second language acquisition and bilingualism. His recent work concerns the measurement and predictive power of language dominance in bilinguals and in the individual factors that affect ultimate attainment. He delivered a keynote entitled “Conceptualizing ultimate attainment in bilingualism and second language acquisition.”  Dr. Epps is a linguistic anthropologist known for her documentary research on Hup (indigenous language of Brazil) and on the implications of language contact on establishing linguistic typologies and pre-histories. She is a proponent of open data and the creator of public databases and archival collections of audio, text, and image media from speakers of Latin American languages. Dr. Epps spoke on the effect of contact between the colonizer languages, Portuguese and Spanish, and the Amazonian indigenous languages in a talk entitled “Multiple multilingualisms: Indigenous and Romance languages in Amazonia.”

Our external speakers were Thamar Solorio (Associate Professor, Department of Computer Science, University of Houston) and Zsuzsanna Fagyal (Associate Professor, Department of French \& Italian, University of Illinois at Urbana Champaign). Dr. Solorio specializes in the analysis of spontaneous language production, including forensic linguistics and clinical Natural Language Processing as applied to bilinguals. She analyzes written data from a variety of sources, including Twitter and speech transcripts. Bridging linguistics and STEM, Dr. Solorio delivered a keynote entitled “Enabling technology for code-switching data.” Zsuzsanna Fagyal, a sociolinguist with a linguistic focus on prosodic variation, delivered a keynote address for LSRL 50, representing phonology and phonetics.  The written version of her talk, entitled “For an integrative approach to variation and change in the French vowel system,” appears in this volume.

The first workshop of the conference, “Wrangling linguistic data with Python,” was created and facilitated by Jacqueline Serigos (Assistant Professor, Department of Modern and Classical Languages, George Washington University) with the goal of helping participants visualize and explore categorical language data, especially data that contains code-switches or borrowings. Dr. Serigos, who creates and uses large datasets in her research on Spanish, specializes in computational linguistics, language contact, semantics, and statistics. Luis Avilés-Gon\-zá\-lez (Ph.D. student in Hispanic Linguistics at the University of Texas) conducted a workshop/tutorial on LaTex using Overleaf to help prepare the conference presenters for the eventual publication of their papers in this volume. His own research is dedicated to investigating sociolinguistic variation in the use of discourse markers among Mexican migrant and heritage Spanish speakers in Southern California.
\section{An overview of the contributions of our authors}
Similar to the LSRL$@$50 event, this volume speaks to the depth and vitality of the field. The invited and refereed chapters are authored by multiple generations of scholars, from those completing postgraduate degrees to senior researchers. Their projects showcase studies in established and emergent subfields and their attendant approaches and methodologies, directed at numerous Romance languages -- Catalan, French, Italian, Picard, Portuguese, Romanian, and Spanish.

The compendium opens with the contribution by one of our invited plenary speakers. In Chapter 1, “For an integrative approach to variation and change in the French vowel system,” Zsuzsuanna Fagyal, argues that the lowering of the French nasal vowels, with respect to their oral counterparts, was motivated by French speakers’ accommodation to standardizing norms rather than by universal structural constraints on articulation. In a sweeping discussion that parallels the history of linguistics itself, Fagyal moves from the insights of linguists interested in the sociolinguistic history of French to the possibility that computational models might replicate the variation and change trajectory of nasal vowels in Romance languages.

Several of the ensuing chapters examine historical and on-going morpho-syn\-tac\-tic variation, contributing to debates surrounding the status and typologies of Romance languages as well as informing literatures and theories of micro-variation. Chapter 2, “Assessing change in a Gallo-Romance regional minority language: 1pl verbal morphology and referential restriction in Picard” by Julie Auger and Anne-José Villeneuve is an important addition to the research on minority language varieties and their preservation, especially pertaining to the consideration bestowed to regional varieties whose language or dialect status remains controversial. This study evinces the significance of adopting a comparative approach in the evaluation of language change and variation across typologically related varieties, in particular when gauging whether or not closely related varieties are actually “different languages.” The authors undertake a comparative investigation of a morphosyntactic change observed in French and Picard. Picard is an endangered dialect (stigmatized as an inferior, degraded variety) that has been at the center of substantial debate concerning its (dis)similarity with colloquial French. The change under analysis is the switch from first person plural to third person singular pronominal forms to refer to groups inclusive of the speaker, which was targeted in light of the commonly held position that French and Picard display sizeable differences at the phonological and lexical level but are very close at the morphosyntactic level. Drawing on data from an original corpus comprised of written Picard data from mid 1900s to present-day, and contemporary spoken French and Picard data, the study contrasts current spoken data from Picard-French bilinguals with present and older written data from three Picard authors born between 1931 and 1960, taking into consideration semantic properties of the referents (i.e., restricted, specific unrestricted, or general unrestricted) Auger and Villeneuve’s results uncover a different use of the two structures under analysis in French and Picard: the use of first-person plural is now marginal in colloquial French while it continues to be strong in Picard. Concerning semantic reference, in Picard, the third person singular appears to be linked primarily to general unrestricted reference but it is hardly used with restricted reference. Furthermore, it is revealed that the relative frequency of the two alternatives does not undergo notable change through time.

The resetting of the Null Subject Parameter in Brazilian Portuguese is explored by Mary A. Kato and Maria Eugenia Lammoglia Duarte in Chapter 3, “The partial loss of free inversion and of referential null subjects in Brazilian Portuguese.” The work reviews scholarship that establishes Brazilian as a Partial Null Subject Language since the middle of the 19th century; unlike European Portuguese, Brazilian Portuguese was shown to display partial optional referential subjects, as well as null expletives and free inversion with unaccusatives. Of interest for the present paper is the finding that by the 1950s, referential null subjects were lost in most contexts, though null expletives persisted. These changes were triggered by loss of rich inflectional morphology. More intriguing is the documentation of the gradual recovery of null subjects through literacy, as attested in correlation with formal instruction in elementary school. Attestation of null subjects ranged from 2.11\% for 1st grade pupils to 49.62\% among 7th/8th graders. The work is significant in its approach to parametric variation, substantiating and refining a theoretical construct with historical and contemporary sources of written and oral data.

In Chapter 4, “The antipassive as a Romance phenomenon: A case study of Italian” Karina High draws from original corpus data spanning from the 13th to the 21st century to trace the diachronic distribution of three pairs of Italian pronominal verbs and their non-pronominal, transitive counterparts (\textit{lamentar\textbf{si}/lamentare} ‘lament/complain’, \textit{ricordar\textbf{si}/ricordare} ‘remember/remind’, \textit{vantar\textbf{si}/vantare} ‘praise/boast’). She analyzes \textit{si} as a detransivizing (i.e., valency-reducing) morpheme and proposes that the pronominal verbs instantiate antipassives since they exhibit distinctive structural behavior of the antipassive: They are syntactically intransitive but semantically transitive in that they involve demotion of the logical object to non-core argument/oblique realized as a prepositional phrase headed by \textit{di}. High’s results indicate that all three reveal a high frequency from the 13th to the 15th century; from the 16th century, however, the frequency of non-pronominal transitive constructions begins to increase and they become predominant from the 17th century onward. It is suggested that the extension of \textit{si} to the antipassive construction may have been eased by the low degree of transitivity of the verbs under consideration, which all select experiencer subjects and theme (minimally or not at all affected by the action) direct objects.

Additional chapters adopt a syntactic-theoretical lens to shed new light on problems in Romance linguistics. Evidencing the special place held in Romance linguistics by constructions involving SE Irene Fernández-Serrano’s, Chapter 5, “The role of SE in Spanish agreement variation,” addresses the free alternation between agreement and non-agreement pattern in Spanish SE structures that involve inanimate postverbal subjects (e.g., \textit{se discutieron los resultados} vs. \textit{se discutió los resultados}). Based on the analysis of spoken interview data gathered from existing corpora which reveal intra-speaker variation between the two patterns, Fernández-Serrano argues that no specific (subject) properties can be identified which could be responsible for lack of agreement and offers a syntactic analysis of the asymmetry that focuses on intraspeaker free variation. In this view, the alternation between the agreeing and non-agreeing pattern resides in syntax. Following previous approaches that account for different syntactic outcomes through the timing of syntactic operations and embracing the intuition that the specific feature configuration of SE blocks person agreement with the subject, Fernández-Serrano puts forward the proposal that Spanish SE acts as a blocking element; that is, in non-agreeing SE constructions the clitic functions as an intervener that obstructs subject-verb agreement. Free alternation with the agreeing pattern, on the other hand, is accounted for in terms of the relative timing of the operations AGREE and MOVE.

Two chapters examine the distribution and interpretation of the null-subject of non-finite (controlled) adjunct clauses. Katie VanDyne’s “Object control into temporal adjuncts: the case of Spanish clitics,” Chapter 6, provides original Spanish data that defy the long-established generalization that obligatory control into temporal adjuncts is limited exclusively to subject control. More precisely, her data evidence a notable contrast between full DP objects and clitic objects in postverbal position, on the one hand, and preverbal object clitics, on the other: \textit{La policía los\textsubscript{i} está buscando despueés de PRO\textsubscript{i} robar un banco} vs.  \textit{*La policiía está buscando-los\textsubscript{i} después de PRO\textsubscript{i} robar un banco}. While the former constructions obey the proven pattern of objects being unable to exert control into an adjunct, the latter allows for object clitics to do so optionally, that is, both subject and object control may obtain in this case. Adopting \citegen{landau2015} two-tiered theory of control, VanDyne accounts for the viability of either a subject or preverbal clitic controller attested in these new data by distinguishing between two positions available for the clitic within vP: If the clitic occupies Spec vP, it can be a controller by being the closest c-commanding DP to adjunct PRO; in contrast, if the clitic moves to an outer specifier, subject control obtains because the subject is closest to the adjunct. Chapter 7, “Overt vs. null subjects in infinitival constructions in Colombian Spanish,” by Kryzzya Gómez, Maia Duguine and Hamida Demirdache, centers on the licensing of overt preverbal subjects in infinitival adjunct clauses in Colombian Spanish; the authors focus specifically on three types of adjunct clauses (introduced by \textit{a}, \textit{para}, and \textit{sin}), which demonstrate disparate patterns of exceptions to generalizations about null infinitival subjects and their interpretations. (cf., \textit{Juan\textsubscript{i} sería feliz al él\textsubscript{i/k}/PRO\textsubscript{i/*k} /José\textsubscript{k} dejar la casa; Juan\textsubscript{i} se fue para él\textsubscript{i/*j}/PRO\textsubscript{i/*}/*María\textsubscript{k} estar feliz; María\textsubscript{i} dejó de trabajar sin ella\textsubscript{i/k}/pro\textsubscript{i/k}/Rosa\textsubscript{k} decir nada}) As argued, \textit{a-}infinitives and \textit{para-}infinitives display diagnostics of obligatory control, whereas \textit{sin-}infinitives are characterized as non-obligatory control. The authors advocate for a novel DP-ellipsis analysis of \textit{sin}-infinitives and an Anaphor Generalization to account for the conflicting patterns of interpretation for null vs. overt PRO in \textit{para}-infinitives.

Chapter 8, “Oblique DOM and co-occurrence restrictions: How many types?” by Mónica Alexandrina Irimia presents a methodical examination of co-occur\-rence restrictions with oblique differential object marking (DOM) in standard and \textit{leísta} Spanish and Romanian. The author carefully surveys the data, identifying six related puzzles in the differential behavior of oblique DOM clitics vs. full DPs and the lack of systematicity of available repair strategies. For instance: Why does Spanish oblique DOM DP produce a PCC effect with an indirect object that is doubled by a dative clitic, as shown in \textit{Le enviaron (*a) todos los enfermos a la doctora?} And why does this same restriction obtain in Romanian when an oblique DOM DP binds into a dative clitic doubled indirect object? (e.g., \textit{Comisa le-a repartizat (*pe) mai mul\c{t}i\textsubscript{i} medici reziden\c{t}i unor fo\c{s}ti profesori de-ai lor\textsubscript{i}}). Analyzing the rich and complex set of data, Irimia puts forth a cogent analysis that refines antecedent accounts based on the split Agree/Case, indicating the importance of the domain in which relevant feature ([\textsc{person}]) is licensed. And Chapter 9, by Nicoletta Loccioni, “A superlative challenge for a syntactic account of connectivity sentences,” offers a careful examination of Iberico-Romance specificational sentences and contrasts such Italian,  "\textit{[L]a persona con cui Maria è più esigente è se stessa vs.  La persona con cui ogni paziente è più onesto è il suo terapista.}  Such data are of interest because they are known to exhibit connectivity effects with respect to Binding Theory; in addition, relativization in the specificational subject is required for the superlative reading of \textit{più}. In accounting for these facts, researchers have followed syntactic and semantic lines of analysis. Loccioni focuses on one syntactic account -- \textit{Question + Deletion}, put forth by \citet{schlenker2003clausalequation} and \citet{romero2007connectivityunified, romero2018connectivity} -- highlighting the challenges presented when relative clauses are specificational subjects. While Loccioni does not articulate a solution to salvage the syntactic account, the author does outline the desiderata for a solution.


Several contributions to the volume investigate the intersection of phonetics and social identity in French. In Chapter 10, “Revisiting sociophonetic competence: Variable spectral moments in phrase-final fricative epithesis for L1 \& L2 speakers of French,” Amanda Dalola and Keiko Bridwell analyze the spectral qualities of fricative epithesis, or the fricative-like noise produced by French speakers at the ends of breath groups as vowels become progressively devoiced. This has frequently been referred to in the literature as final vowel devoicing. The authors argue, however, that the production of the fricative-like element is, in fact, effortful and not the result of the kind of phrase-final devoicing that occurs with attenuating energy. The participants for their controlled reading task included two groups, 40 L1 (French) speakers and 31 advanced L2 (French$-$English) speakers, to ascertain whether they produced fricative epithesis differently as measured by center of gravity, standard deviation, kurtosis, skewness, and intensity. The results showed an interaction between speaker group and vowel for skewness with differences noted in the production of fricative epithesis following the vowel /y/ and in other measures as the vowel attenuated. The authors speculate that the /y/ vowel in French has social significance and is subject to hyperarticulation by L1 and L2 speakers alike, albeit to different degrees. Chapter 11, Hilary Walton’s, “Does social identity play a role in the L2 acquisition of French intonation? Preliminary data from Canadian French-as-a-second-language classroom learners,” examines how social identity affects the performance of individual L2 learners in two different language learning contexts. These contexts are identified as "French immersion", where anglophone learners complete a mandatory number of hours across their curriculum from Grades 1 through 12 in the French language, and “core French", where study in French is optional after Grade 4. While the results show that the immersion students self-report greater levels of in-group psychological attachment than their peers in core French, there were no statistically significant between-group differences in the linguistic dimension under study, pitch contours in non-final accentual phrases. Nonetheless, the study paves the way forward for future investigations of in-group linguistic accommodation within the context of L2 speech.


Bilingual and contact production are also at the center of several other chapters. Inspired by the lack of research on Catalan-influenced Spanish relative to the attention paid to Spanish-inflected Catalan, Annie Helms’ Chapter 12, “Sociophonetic analysis of mid front vowel production in Barcelona,” targets the contribution of age and gender to variation in the production of Spanish /e/ among Catalan–Spanish bilinguals. Her production study focuses on the production of Spanish /e/ in words that are cognate with Catalan, as cognates are argued to promote cross-linguistic interaction which, in turn, leads to the assimilation of similar phonological categories between languages. Measures of the first two formants of Spanish and Catalan /e/ productions were extracted from the productions of 17 bilingual speakers. Her findings show significant evidence of the production of a Catalan-like /e/ in Spanish, especially among young males, but no effect of cognate status. Helms explains these results in terms of an overall weakening of the Catalan front vowel contrast in Barcelona coupled with greater overall variability in the production of front vowels in Spanish and Catalan among younger speakers. In Chapter 13, “Prosodic correlates of mirative and new information focus in Spanish wh-in-situ questions,” Carolina González and Lara Reglero empirically investigate the pragmatic meaning of two types of wh in-situ questions among 22 Spanish speakers in the Basque Country of Spain.  Their participants completed a contextualized elicitation task answering prompts designed to motivate either in-situ information seeking or echo surprise questions as the response.  Their goal in gathering this data was to determine whether the prosodic correlates of the productions of their participants are compatible with mirative or with new information focus. The authors investigate the tonal shape of the question, as well as the configuration of its nuclear peak, the height of the peak in Hz and the range of the focus tone. Their findings indicate that focal tone range delineates the two types of in-situ questions with echo surprise questions showing an expanded tonal range relative to information seeking questions. They conclude that surprise questions show mirative focus.

In “Mechanical vs. functional processes in subject pronoun expression in Spanish second language learners,” Chapter 14, Ana de Prada Pérez and Nick Feroce contribute to the growing body of literature on Spanish referential pronouns, by examining their exponents among second language learners in comparison to bilingual native speakers. Data collected from sociolinguistic interviews shows that the learner groups produce more overt pronominal subjects than the native speaker group -- the low learner group in both 2\textsc{sg} and 3\textsc{sg}, and the high learner group in 3\textsc{sg} only. The variationist analysis returned differences between both learner groups and native speakers in sensitivity to Switch Reference, though in 1\textsc{sg} only. Perseveration was evidenced for 1\textsc{sg}, across all participant groups, but only the native speaker group showed the effect for 3\textsc{sg}. Finally, the interaction of perseveration and switch reference was similar across groups. The authors interpret the data as indicating that differences between learners and native speakers is restricted to rate and function of pronominal expression and that these factors operate differently over deictic and referential subjects. These results align with extant studies while contesting others; for the latter, the authors are meticulous in pursuing explanations in coding and analysis.


The final chapter considers a universal property of language -- Zipf’s law, which holds that there is a correlation between a word’s frequency and its length -- extending its scope. More specifically, Chapter 15, “Frequency and efficiency in Spanish proverbs” by Ernesto R. Gutiérrez Topete examines whether/to what extent Zipf’s law also applies to proverbs (e.g., \textit{Lo que mal comienza mal termina}; \textit{Más vale pájaro en mano que cientos volando}). The author scrutinizes 30 proverbs drawn a collection of proverbs frequently attested in the press in Tucumán, Argentina, and evaluates their occurrence in a news media corpus collected from \textit{News on the Web} (NOW) included in the online \textit{Corpus del Español} \citep{Davies}. The results of the study show a positive correlation between the length and the frequency of a proverb: Proverbs displaying lower frequency rate are more resistant to shortening than proverbs with higher frequency rate. However, some outliers are found, indicating that in addition to frequency other factors are at work. On the other hand, neither syntactic complexity nor variability appear to play a role in the proverb’s shortening rates.

\section{Acknowledgements}
We are very grateful to Mara Breen and the conference co-organizers of CUNY 2020 for leaving a detailed roadmap for hosting a conference online.

We are especially grateful to our former and current students who helped us at each stage of this adventure, many of whom we have already named. We thank Tracey Adams, Aris Clemons, Luis Avilés-González, Joshua Frank, and Anna Lawrence (profusely) for their assistance during the conference, particularly for their deft and undaunted mastery of digital environments. Other graduate students assisted in the planning phase of our “live” conference and we are grateful to them for their help: Salvatore Callesano, Victor Garre León, Tom Leslie, Amalia Merino, Marylise Rialliard, and Lamia Trifi. Special thanks go to Amanda Dalola, our tireless social media diva, for her awesome handling of the $@$LSRL50 Twitter handle.  We thank the remarkable undergraduate student, Carol Zeng, for her insightful editing of these proceedings papers. For his LaTex typesetting magic, we express our appreciation to Luis Avilés-González. Finally, we thank our conference participants for their patience, enthusiasm, and good humor during a stressful moment of history.

We express our gratitude to our keynote speakers and workshop conveners whose sessions drew the conference’s largest audiences. And we thank our chairs for their fearlessness in taking on the task of emceeing a session in the age of Zoom: Richard Meier, Randall Gess, Bruno Estigarribia, Karen Zagona, Anna Maria DiSciullo, Silvina Montrul, Adrián Riccelli, Michael Newman, Teresa Satterfield, John MacDonald, Jose Camacho, Laura Colantoni, Tim Gupton, Beth MacLeod, Julie Auger, and Bradley Hoot.

We would like to acknowledge our gratitude to those who agreed to undertake abstract reviewers for LSRL 50; they were completed with alacrity and thoughtfulness: Evangelia Adamou, Lourdes Aguilar, Gabriela Alboiu, Scott Alvord, Patricia Amaral, Mark Amengual, Raul Aranovich, Megan Armstrong, Karlos Arregi, Deborah Arteaga, Angeliki Athanasopoulos, Julie Auger, Jennifer Austin, Marc Authier, Laura Bafile, Brandon Baird, Aurora Bel, Judy Bernstein, Hélene Blondeau, Eul\`alia Bonet, Travis Bradley, Barbara Bullock, Monica Cabrera, Andrea Calabrese, José Camacho, Richard Cameron, Rebeka Campos-Astorkiza, An\-na Cardinaletti, Ana Carvalho, Isabelle Charnavel, Ioana Chitoran, J. Clancy Clements, Laura Colantoni, Sonia Colina, Marie-Hélène C\^oté, Maria Cristina Cuervo, Sonia Cyrino, Roberta D'Alessandro, Amanda Dalola, Justin Davidson, Laurent Dekydtspotter, Viviane Déprez, Anne Marie Di Sciullo, Manuel Díaz-Campos, Bryan Donaldson, Paola Giuli Dussias, Gorka Elordieta, Anna María Escobar, M. Teresa Espinal, Bruno Estigarrabia, Antonio Fábregas, Timothy Face, Zsuzsanna Fagyal, Anamaria Falaus, Raquel Fernández Fuertes, Olga Fernández Soriano, Franck Floricic, Jon Franco, Angel Gallego, Charlotte Galves, Anna Gavarró, Randall Gess, Alessandra Giorgi, Ion Giurgea, Carolina González, Grant Goodall, Alex Grosu, Tim Gupton, Julia Herschensohn, Virginia Hill, Chad Howe, José Ignacio Hualde, Haike Jacobs, Mary Kato, Carol Klee, Karen Lahousse, Manuel Leonetti, Juana Liceras, John Lipski, Conxita Lleó, Ruth Lopes, Luis López, Jonathan MacDonald, Bethany MacLeod, Rita Manzini, Fernando Martínez-Gil, Ana-Maria Martins, Diane Massam, Jaume Mateu, Eric Mathieu, Natalia Mazzaro, Egle Mocciaro, Fabio Montermini, Jean Pierre Montreuil, Silvina Montrul, Francisco Moreno Fernández, Michael Newman, Jairo Nunes, Rafael Nu\~nez-Cede\~no, Antxon Olarrea, Dan Olson, Francisco Ordó\~nez, Sandra Paoli, Diego Pescarini, Pierre Pica, Acrisio Pires, Cecilia Poletto, Pilar Prieto, Elissa Putska, Michael Ramsammy, Rajiv Rao, Lisa Reed, Lara Reglero, Peggy Renwickl, Lori Repetti, Gemma Rigau Oliver, Yves Roberge, Ian Roberts, Francesc Roca, Marcos Rohena-Madrazo, Rebecca Ronquest, Johan Rooryck, Edward Rubin, Cinzia Russi, Andrés Saab, Nuria  Sagarra, Mario Saltarelli, Liliana Sánchez, Teresa Satterfield, Leonardo Savoia, Cristina Schmitt, Sandro Sessarego, Miguel Simonet, Petra Sleeman, Jason Smith, Carmen Dobrovie-Sorin, Lauru Spinu, Dominique Sportiche, Jeffrey Steele, Jacqueline Toribio, Annie Tremblay, Mireille Tremblay, Michelle Troberg, Myriam Uribe-Etxebarria, Elena Valenzuela, Barbara Vance, Ioana Vasilescu, Julio Villa-Garcia, Anne, José Villeneuve, Irene Vogel, Lydia White, Erik Willis, Caroline Wiltshire, Malcah Yaeger-Dror, and Karen Zagona.

A special word of gratitude is owed to the colleagues who shared of their time and expertise in reviewing papers submitted for these referred proceedings: Lourdes Aguilar Cuevas, Alex Alsina, Mark Amengual, Richard Cameron, Justin Davidson, Carmen Dobrovie-Sorin, Bryan Donaldson, Antonio Fábregas, Franck Floricic, Joshua Frank, Ángel Gallego, Randall Gess, Joshua M. Griffiths, Alex Grosu, Julia Herschensohn, José Ignacio Hualde, Carol Klee, Luis López-Carretero, Bethany MacLeod, Jonathan MacDonald, Ana Maria Martins, Natalia Mazzaro, Egle Mocciaro, Jairo Nunes, Rafael Orosco, Luis Ortiz, Sandra Paoli, Rajiv Rao, Ian Roberts, Teresa Satterfield, Laura Spinu, Andrés Saab, Jacqueline Serigos, and others who wished to remain anonymous.

In closing, we wish to thank our sponsors for LSRL 50: The National Science Foundation and, from the University of Texas at Austin:  The College of Liberal Arts, The Center for European Studies, The Department of Linguistics, The Department of French and Italian, The Department of Spanish and Portuguese, and The Department of Germanic Studies.

{\sloppy\printbibliography[heading=subbibliography,notkeyword=this]}
\end{document}
