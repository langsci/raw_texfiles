\documentclass[output=paper,colorlinks,citecolor=brown]{langscibook}
\ChapterDOI{10.5281/zenodo.7525094}
\author{Julie Auger \affiliation{Université de Montréal} and Anne-José Villeneuve\affiliation{University of Alberta}}
\title[Assessing change in a Gallo-Romance regional minority language]{Assessing change in a Gallo-Romance regional minority language: 1\textsc{pl} verbal morphology and referential restriction in Picard}
\abstract{This paper examines the possible change from 1\textsc{pl} to 3\textsc{sg} forms when referring to a group that includes the speaker in Picard, a Gallo-Romance language of Northern France. Using older and contemporary Picard written data, as well as contemporary oral data, we show that, even though Picard and colloquial French use the two forms, the two languages differ. Contrary to colloquial French, where 1\textsc{pl} usage has become marginal, 1\textsc{pl} remains widely used in Picard. Our analysis of semantic reference (restricted, specific unrestricted, or general unrestricted group) indicates that 3\textsc{sg} is primarily associated with general unrestricted reference in Picard and is barely used to refer to restricted groups. Most interestingly, the relative frequency of the two variants remains stable over time. Our analysis demonstrates the importance of considering linguistic conditioning through the comparative method for assessing language change in typologically related varieties, especially when testing claims that a minority language is converging toward its dominant counterpart.}

%move the following commands to the "local..." files of the master project when integrating this chapter

% \usepackage{langsci-basic}
% \usepackage{langsci-optional}
% \usepackage{langsci-gb4e}
% \bibliography{localbibliography}
% %\newcommand{\orcid}[1]{}
% \pagenumbering{arabic}
% \setcounter{page}{36}

%language for footnotes with examples%
% \usepackage{etoolbox}

% \makeatletter
% \patchcmd{\@footnotetext}{\setcounter{fnx}{0}}{\renewcommand{\thexnumi}{\roman{xnumi}}}{}{}
% \apptocmd{\@footnotetext}{
%     \@noftnotetrue
%     \renewcommand{\thexnumi}{\arabic{xnumi}}%
% }{}{}

% \usepackage{array}

\IfFileExists{../localcommands.tex}{
  \addbibresource{../localbibliography.bib}
  % add all extra packages you need to load to this file

\usepackage{tabularx,multicol}
\usepackage{url}
\urlstyle{same}

\usepackage{listings}
\lstset{basicstyle=\ttfamily,tabsize=2,breaklines=true}

\usepackage{langsci-basic}
\usepackage{langsci-optional}
\usepackage{langsci-lgr}
\usepackage{langsci-osl}
% \usepackage{./langsci/styles/langsci-lgr}
% \usepackage{./langsci/styles/langsci-osl}
% \usepackage{langsci-gb4e}

\usepackage{tikz}
\usetikzlibrary{patterns,calc}
\pgfdeclarepatternformonly{south east lines}{\pgfqpoint{-0pt}{-0pt}}{\pgfqpoint{3pt}{3pt}}{\pgfqpoint{3pt}{3pt}}{
    \pgfsetlinewidth{0.6pt}
    \pgfpathmoveto{\pgfqpoint{0pt}{3pt}}
    \pgfpathlineto{\pgfqpoint{3pt}{0pt}}
    \pgfpathmoveto{\pgfqpoint{.2pt}{-.2pt}}
    \pgfpathlineto{\pgfqpoint{-.2pt}{.2pt}}
    \pgfpathmoveto{\pgfqpoint{3.2pt}{2.8pt}}
    \pgfpathlineto{\pgfqpoint{2.8pt}{3.2pt}}
    \pgfusepath{stroke}}
    
\usepackage{stmaryrd}
\usepackage{wasysym}
\usepackage{multirow}
\usepackage{caption}
\usepackage{subcaption}
\usepackage{mathrsfs}
\usepackage{qtree}

\usepackage{linguex}


  %pminos do not split footnotes
% \interfootnotelinepenalty=10000 %Footnote in Laporte chapters has to be split SN


%\DeclareIndexNameFormat{default}{%
%\nameparts{#1}%
%\usebibmacro{index:name}%
%{\index[names]}%
%{\namepartfamily}%
%{\namepartgiveni}%
% {}% L1
% {}% L2
%{\namepartprefix}% generates spurious space L3
%{\namepartsuffix}% generates spurious space L4
%}

%  {\DeclareIndexNameFormat{default}{%
%     \usebibmacro{index:name}{\index[names]}{#1}{#3}{#5}{#7}}}

%\DeclareIndexNameFormat{default}{%
%  \usebibmacro{index:name}{\sindex[nom]}{#1}{#3}{#5}{#7}}

%\DeclareIndexNameFormat{default}{%
%  \usebibmacro{index:name}{\sindex[person]}{#1}{#3}{#5}{#7}}
%\DeclareIndexNameFormat{default}{%
%\nameparts{#1} \usebibmacro{index:name}{\sindex[person]]}{\namepartfamily}{‌​\namepartgiven}{\nam‌​epartprefix}{\namepa‌​rtsuffix}}

%\newcommand{\smiley}{:)}

%\renewbibmacro*{index:name}[5]{%
%\usebibmacro{index:entry}{#1}%
%{\iffieldundef{usera}{}{\thefield{usera}\actualoperator}\mkbibindexname{#2}{#3}{#4}{#5}}}

% \newcommand{\noop}[1]{}

%remove for final
%\overfullrule=1mm

\newcommand{\tobi}[2]}}
\renewcommand{\S}[1]{\tobi{#1}{\textsc{*}}}

% this volume references
% puts: [this volume]
% already defined: \citetv
%\newcommand{\citepv}[1]{(\citeauthor{#1} \citeyear*{#1} [this volume])}
\newcommand{\citealtv}[1]{\citeauthor{#1} \citeyear*{#1} [this volume]}

%parentheses around example number
\newcommand{\pref}[1]{(\ref{#1})}

% in-text examples

\newcommand{\lnex}[1]{\textit{#1}} %target lang word
\newcommand{\lnlit}[1]{(lit.: `#1')} %literal reading
\newcommand{\lnlat}[1]{(#1)} % latinization
\newcommand{\lntrans}[1]{`#1'} %translation
\newcommand{\lnexl}[2]%
{\lnex{#1}{} \lnlat{#2}} % ex with latinization
\newcommand{\lnexlat}[3]{\lnex{#1}{} \lnlat{#2}{} \lntrans{#3}} % ex with latinization and tranl.

%ch01
\newcommand{\co}[1]{\mbox{\textbf{#1}}}

%ch09

\newcommand{\cyrbulg}[1]{\begin{otherlanguage*}{bulgarian}#1\end{otherlanguage*}}


%ch10
\newcommand{\nlp}{{\small NLP}}
\newcommand{\mwe}{{\small MWE}}
\newcommand{\rae}{{\small RAE}}
\newcommand{\lvc}{{\small LVC}}
\newcommand{\pos}{{\small P}o{\small S}}
%\newcommand{\todo}[1]{ \textcolor{red}{#1} }

%\renewcommand{\labelenumi}{\theenumi}
%\ainamefmt{{vv}{ll}{, ff}{, jj}} % fullname

\newcommand{\biberror}[1]{{\color{red}#1}}

\newcommand{\osenovaitem}{--~}
  %% hyphenation points for line breaks
%% Normally, automatic hyphenation in LaTeX is very good
%% If a word is mis-hyphenated, add it to this file
%%
%% add information to TeX file before \begin{document} with:
%% %% hyphenation points for line breaks
%% Normally, automatic hyphenation in LaTeX is very good
%% If a word is mis-hyphenated, add it to this file
%%
%% add information to TeX file before \begin{document} with:
%% %% hyphenation points for line breaks
%% Normally, automatic hyphenation in LaTeX is very good
%% If a word is mis-hyphenated, add it to this file
%%
%% add information to TeX file before \begin{document} with:
%% \include{localhyphenation}
\hyphenation{
    Beck-man
    Ngu-yen
    back-chan-nel
    back-chan-nels
    mo-not-o-nous
    ste-reo-typ-i-cal
}

\hyphenation{
    Beck-man
    Ngu-yen
    back-chan-nel
    back-chan-nels
    mo-not-o-nous
    ste-reo-typ-i-cal
}

\hyphenation{
    Beck-man
    Ngu-yen
    back-chan-nel
    back-chan-nels
    mo-not-o-nous
    ste-reo-typ-i-cal
}

  % \togglepaper[3]%%chapternumber
}{}

\begin{document}
\maketitle

\section{Introduction}
The debate over whether a given linguistic variety constitutes an autonomous language or a dialect of another variety is typically of little interest to formal linguists; what matters to us is that the system analyzed is coherent and that its linguistic forms are generated by the same mental grammar. However, such a question may have far-reaching consequences for endangered Romance languages, especially in the European sociopolitical context, as only languages that are “different from the official language(s) of that State” may be recognized and protected under the Charter for Regional and Minority Languages (Council of Europe 1992). Thus, while varieties like Catalan, Franco provençal and Occitan differ sufficiently from Spanish, Italian or French to unequivocally qualify for official recognition and support, regional varieties whose language-versus-dialect status is the object of debate do not benefit from the same protections. For instance, the Gallo-Romance varieties spoken in Northern France (e.g., Norman, Picard), although formally listed as “regional languages of France” in a 2013 report from the French Ministry of Culture and Communication \citep{DGLFLF2013}, continue to be perceived by many in the greater public as ``bad'', ``corrupt'', or, more neutrally, regional varieties of the national language, Continental (or Hexagonal) French\footnote{We use ``Continental French'' to refer to the variety of European French spoken in Continental France.}  \citep{eloy_laffaire_1997}. Such a perception has contributed to the stigmatization and lack of transmission of these varieties, as well as to their continued exclusion from official school curricula, even though such an inclusion is allowed, for example, by the Deixonne law and the more recent Lang initiative \citep{eloy_constitution_1997}, and more generally, to the refusal to grant them official recognition and protection at the national and European levels. In these situations, comparative sociolinguistic research, through its careful examination of variation patterns that focus on both the distribution of variants and the linguistic conditioning behind variant selection, can be of service to language policy makers. Specifically, it can serve as a tool for assessing whether the linguistic distance between two closely related varieties may be sufficient to call them “different languages” such that they can be recognized and protected under the European Charter for Regional and Minority Languages.

The status of Picard, an endangered Gallo-Romance language of Northern France, is the object of considerable debate. While scholars recognize that the two varieties’ phonology and lexicon differ considerably, \citet[137]{eloy_constitution_1997} argues that Picard’s morphosyntax does not significantly differ from that of colloquial French. Evidence against Éloy’s position is provided by detailed analyses of specific constructions. For example, \citet{burnett_what_2018} have shown that negation in Picard is realized through two different elements, \textit{point} and \textit{mie}, and that the latter serves to negate presuppositions and express emphasis. \citet{auger_two_nodate} has shown that the Vimeu variety of Picard possesses two different subject neuter clitics, \textit{a} and \textit{ch}, whose distribution depends on the type of predicate that they combine with. Thus, whereas French uses the same neuter pronoun, \textit{ce} (and its colloquial variation, \textit{ça}) with nominal and adjectival predicates (e.g., \textit{C’est mon ami} ‘it is my friend’ and \textit{C’est beau} ‘it is beautiful’), Picard uses different pronouns: \textit{Ch’est un gros férmieu} ‘it is an important farmer’ and \textit{a n’est mie bieu} ‘it is not beautiful’). For negation and neuter pronouns, the grammatical difference between Picard and (colloquial) French is clear. However, for the numerous morphosyntactic structures that are shared by the varieties, the difference is less clear. In these cases, we suspected that refusals to recognize the grammatical autonomy of Picard rely on superficial comparisons. In order to test this suspicion, we have carefully analyzed data collected from a bilingual Picard–French community of practice located in rural Picardie to determine how much Picard and French morphosyntax truly differ. This work has shown that shared morphosyntactic structures function differently in Picard and in colloquial French. This is the case, for example, for subject doubling and \textit{ne} deletion: whereas the co-occurrence of subject doubling and \textit{ne} presence is marginal in French, due to the opposite stylistic values of the two forms, this combination is the most commonly attested in Picard \citep{villeneuve_chtileu_2013}.

This paper examines first person plural verbal morphology (henceforth 1\textsc{pl}), a variable which, superficially, seems to support Éloy’s convergence claim. As we see in \tabref{02:table1}, French makes use of 1\textsc{pl} and 3\textsc{sg} indefinite pronouns to refer to a group that includes the speaker whereas Picard makes use of a homophonic pronoun that shares the same form for 1\textsc{pl} and indefinite 3\textsc{sg} reference: \textit{oz}\footnote{ \textit{Os} is also used as a 2\textsc{pl} subject pronoun. Once again, verbal morphology distinguishes 2\textsc{pl} from 3\textsc{sg}.indefinite and 1\textsc{pl}, as we can see below::
\ea
\ea
\gll os cante \\
one/we/you.PL sing\\
\glt‘one sings’\\

\ex
\gll os cant-ons \\
one/we/you.PL sing\\
\glt ‘we sing’\\

\ex
\gll os cant-eu \\
one/we/you.PL sing\\
\glt ‘you.\textsc{pl} sing’\\
\z
\z
The 1\textsc{pl} and 2\textsc{pl} pronouns result from the loss of the initial consonants in \textit{nos} and \textit{vos}, respectively. 3\textsc{sg} \textit{os} results from the denasalization of \textit{on} ‘one’ in unstressed position \citep[260--261]{hrkal_grammaire_1910}. All three pronouns are pronounced [o] before a consonant and [oz] before a vowel.}.  In the case of Picard, verbal morphology distinguishes the two persons: an \textit{–ons} ending for 1\textsc{pl} in most tenses and the absence of overt marking for 3\textsc{sg}.

\begin{table}
\caption{1\textsc{pl} and 3\textsc{sg} verbs in French and in Picard}
\label{02:table1}
 \begin{tabularx}{\linewidth}{rQQ}
 \lsptoprule
& \textbf{French} & \textbf{Picard} \\
\midrule
\textbf{1\textsc{pl} forms} & {\gll mais \textbf{nous} \textbf{all-i-ons}     au           lycée\\
                but   we    go-\textsc{pst}-1\textsc{pl} at.the.\textsc{sg} lycée\\}
        &{\gll \textbf{oz}   \textbf{all-ons} rpèrler\\
            we go-1\textsc{pl}  talk.again\\}\\
            &‘but we went to high school’&‘we are going to talk again’\\
            &&\\
\textbf{3\textsc{sg} forms} & {\gll \textbf{on}    \textbf{va}        essayer\\
                one  go.3\textsc{sg} try\\}
        &{\gll \textbf{o}     \textbf{va}        pas  revnir\\
            one  go.3\textsc{sg} not  come.back\\}\\
            &‘we are going to try’&‘we are not going to come back’\\
\lspbottomrule
\end{tabularx}

\end{table}

\hspace*{-.9pt}Previous variationist work has shown that the use of \textit{nous} has become marginal in many varieties of colloquial French (e.g., 1.6\% in Montréal, \citealt[132]{laberge_etude_1977}; 4.4\% in Picardie, \citealt[466]{coveney_vestiges_2000}; see also \citealt{king_interplay_2011}). To this day, no comparable analyses have been undertaken for Picard. Thus, in this paper, we seek to determine whether the replacement of 1\textsc{pl} by an indefinite 3\textsc{sg} pronoun is observable in Picard and to establish whether the constraints that favor the selection of person operate similarly to what has been described for colloquial French.

\section{1\textsc{pl} verbal morphology in French and Picard}
The variation between \textit{nous} -\textit{ons} and \textit{on} + 3\textsc{sg} in French has received considerable attention from linguists and sociolinguists. Because, throughout much of the history of French, 1\textsc{pl} has involved the pronoun \textit{nous} followed by a verb suffixed with -\textit{ons}, we might think that the use of \textit{on} with a 3\textsc{sg} verb form and the concomitant reduced occurrence of \textit{nous} -\textit{ons} reflects a gradual replacement of the latter form by the former. However, there are reasons to question such a scenario. Indeed, while \textit{nous} as a subject pronoun is widely attested in written documentation produced by literate speakers of French ever since Old French\footnote{We thank Barbara Vance for confirming this information.},  some scholars have raised doubts concerning its use in the speech of lower-class speakers. Citing \citet{coveney_vestiges_2000} and \citet{lodge_sociolinguistic_2004}, \citet{king_interplay_2011} invoke the widespread use of \textit{je} -\textit{ons} (cf. \ref{02:ex:1a}) forms and the rarity of \textit{nous} -\textit{ons} forms (cf. \ref{02:ex:1b}) in representations of lower-class speech from the 16th through 18th centuries. Thus, the question remains whether the near-categorical use of 3\textsc{sg} \textit{on} in Québec \citep{laberge_etude_1977}, Picardie \citep{coveney_vestiges_2000}, and Switzerland \citep{fonseca-greber_subject_2003} results from the replacement of \textit{nous} by \textit{on} or from the disappearance of the \textit{je} -\textit{ons} form. Additional support for the latter hypothesis is found in \citegen{flikeid_nest-ce_1989} analysis of 1\textsc{pl} pronouns and verbs in \textit{Atlas linguistique de la France}, which confirms the rarity of \textit{nous} -\textit{ons} forms in the northwestern parts of France, with the exception of the former Somme and Pas-de-Calais \textit{départements}, where \textit{os} -\textit{ons} forms dominate.\footnote{Our consultation of all relevant ALF maps for 1\textsc{pl} on the Symila website (\url{http://symila.univ-tlse2.fr/}) confirms \citegen{flikeid_nest-ce_1989} generalization based on 3 maps.}
\ea \label{02:ex:1}
1\textsc{pl} forms in 17th century French \jambox*{(adapted from \citealt[471]{king_interplay_2011})}

\ea\label{02:ex:1a}  \gll Moi et    le      gros Lucas, et   \textbf{je} nous \textbf{amus-i-ons}     à   bâtifoler   avec   des mottes  de  tarre\\
me  and the.M big   Lucas  and I  us     enjoy-PST-1\textsc{pl} to fool.around with some clumps of  dirt\\ \jambox*{(\textit{Don Juan}, Act II, scene 1, 1665)}
\glt   ‘Me and big fat Lucas, and we were having fun fooling around with clumps of dirt’\\
\ex\label{02:ex:1b} \gll qu’il  aille au         diable avec son mulet! ... \textbf{nous} \textbf{ir-ons}          devant les      juges\\
the he go   at.the.M devil  with  his  mule!     we      go.\textsc{fut}-1\textsc{pl} before the.\textsc{pl} judges\\ \jambox*{(\textit{Les Fourberies de Scapin}, Act I, scene 1, 1671)}
\glt ‘he can go to hell with his mule! ... we shall go before a judge’ \\

\ex\label{02:ex:1c} \gll je ne   sais   pas quand \textbf{on}  \textbf{verra}            finir   ce       galimatias\\
I  \textsc{neg} know not when  one will.see.3\textsc{sg} finish this.\textsc{m} gobbledygook\\ \jambox*{(\textit{Sganarelle}, scene 22, 1660)}
\glt ‘I don't know when we will see the end of such gobbledygook’ \\
\z
\z

The prevalence of \textit{os} –\textit{ons} forms for 1\textsc{pl} is attested since at least Middle Picard for the western parts of the Picard-speaking area and the 17th century for the southern portions of the Picard-speaking area \citep[140, 147]{flutre_moyen_1970}. Monographs from the turn of the 20th century, such as \citet{edmont_lexique_1980}, \citet{ledieu_petite_1909}, and \citet{hrkal_grammaire_1910}, provide support for the results from the \textit{ALF}. \citet{vasseur_grammaire_1996} confirms the prevalence of the \textit{os} -\textit{ons} construction in Vimeu Picard, while Picard textbooks mention only this form for 1\textsc{pl} (\cites[]{debrie_eche_1983}[86]{dawson_dictionnaire_2020}). No description of Picard mentions the use of 3\textsc{sg} \textit{os} as a competing form for 1\textsc{pl} inclusive reference.

As we have already mentioned, considerably more is known about 1\textsc{pl} variation in French than in Picard. \citet{coveney_vestiges_2000} and \citet{fonseca-greber_subject_2003} show that use of subject \textit{nous} is very infrequent in Continental and Swiss French varieties. The results compiled from other studies by \citet[501]{king_interplay_2011} reveal frequencies of \textit{nous} varying between 0.25\% and 2.6\% in Québec and Ontario French. For Acadian French, their compilation indicates variation between \textit{je} -\textit{ons} and 3\textsc{sg} \textit{on}, with no tokens of \textit{nous}. As for Louisiana French, the pattern differs based on the location investigated: for the Cajun varieties of the coastal marshes, \citet[197]{rottet_language_2001} reports the gradual loss of 3\textsc{sg} \textit{on} to the profit of disjunctive \textit{nous}-\textit{autres}, while \citet[148]{dajko_ethnic_2009} observes an overwhelming preference for 3\textsc{sg} \textit{on} in Lafourche Parish, along with very low frequencies for null pronouns, \textit{nous}-\textit{autres} \textit{on}, and \textit{nous}-\textit{autres}.

The historical and variationist analyses of the variation between \textit{nous} –\textit{ons} and 3\textsc{sg} \textit{on} also inform us on the factors that favor the two variants and, consequently, on the path taken by the grammaticalization process by which the latter replaces the former as 1\textsc{pl}. \citet{king_interplay_2011} coded for the linguistic factors that influence the grammaticalization of \textit{a gente} as a 1\textsc{pl} pronoun in Brazilian Portuguese \citep{zilles_development_2005}, namely verb tense, verb class, clause type, and referential restriction. Of the four factors considered, only the last one, referential restriction, was found to play a significant role in their French data. Given that \textit{on} has historically expressed indefinite reference, it is not surprising that its use is most strongly favored for unrestricted groups whose membership includes individuals who do not belong to a speaker’s network.

Twentieth century varieties of Modern French fail to provide appropriate data for testing the grammaticalization process whereby subject \textit{nous} gives way to \textit{on}, either because \textit{nous} is so marginal that a quantitative analysis is impossible or because the variation that persists involves different variants (\textit{je} -\textit{ons} vs. \textit{on} in Acadian French; \textit{nous} -\textit{ons} vs. \textit{on} in Continental, Swiss, and Québec French). Additionally, uncertainty remains concerning the use of \textit{nous} -\textit{ons} forms by lower-class speakers in previous centuries, which makes it difficult to evaluate the factors that have influenced the rise of 3\textsc{sg} in French. Consequently, we believe that Picard provides the perfect testing ground for gaining a better understanding of the gradual replacement of 1\textsc{pl} forms by 3\textsc{sg} ones. Indeed, the frequent use of 1\textsc{pl} subject pronoun and verbal morphology that characterizes western dialects of Picard, along with the possibility of an increase in the use of 3\textsc{sg} variants as seen in our preliminary analyses, provides the type of data that will allow us to determine the effect played by referential restriction (see below) on the choice between traditional 1\textsc{pl} \textit{oz} -\textit{ons} and innovative 3\textsc{sg} \textit{oz}.

\section{Methodology}
Our recent work assesses the degree of structural morphosyntactic convergence and divergence between French and Picard by analyzing data from the Vimeu area, located in rural Picardie, France. In continuity with our previous work, we examine three types of data: contemporary oral data for French and Picard, contemporary written Picard, and older written Picard data from the 1940s to the 1960s. Our Vimeu Picard and French contemporary oral data are extracted from sociolinguistic interviews with four Picard–French bilingual men and supplemented by Vimeu French oral data from a control group of four French monolingual men (see \citealt{villeneuve_chtileu_2013} for a detailed description); in this paper, we focus on the bilingual data described in \tabref{02:table2}.\footnote{The absence of women in our corpus stems from the gender imbalance in the number of regional minority language speakers and in their daily use of the language \citep{Pooley2003}. It is therefore difficult to find a reliable, balanced sample of female Picard speakers.}

\begin{table}
\caption{Oral Picard and French corpus, bilingual speakers’ demographic information (adapted from \citealt[119]{villeneuve_chtileu_2013})}
\label{02:table2}
 \begin{tabularx}{\linewidth}{ X X l }
 \lsptoprule
\textbf{Pseudonym} & \textbf{Year of Birth} & \textbf{Occupation} \\
\midrule
Joseph L.   &   1931    &   retired teacher \\
Gérard D. &   1945    &   factory worker, artist\\
Joël T.   &   1946    &   marketing agent, inn host\\
Thomas S.   &   1960    &   teacher \\
\lspbottomrule
\end{tabularx}
\end{table}


Because of the methodological challenge that the assessment of morphosyntactic variation in regional minority languages represents, due, for instance, to limited amounts of oral data on which to perform quantitative analyses (see \citealt[552]{auger_using_2017}), we compare our contemporary Picard oral data from bilinguals with contemporary and older written data from three Picard authors born between 1904 and 1959, as shown in \tabref{02:table3}.\footnote{Given that Picard is strongly associated with orality, it may seem somewhat ironic to seek linguistic data from written texts. However, thanks to the relatively large amount of such texts and to the fact that written Picard faithfully mirrors the spoken language \citep{auger_picard_2002, auger_picard_2003}, we are confident that this approach can help us determine whether Picard 1\textsc{pl} verbal morphology is changing and, if so, whether it is converging toward French.} Vasseur’s and Dulphy’s data come from weekly columns published in newspapers. Leclercq’s text is a novel that tells the story of a young Picard man in the 1950s. This three-way comparison allows us to assess the degree of similarity between bilinguals’ French and Picard production, measure the distance between the written and oral community norms, and assess diachronic change based on written data.

\begin{table}
\caption{Written Picard corpus over three generations of authors (adapted from \citealt[218]{auger_building_2019})}
\label{02:table3}
 \begin{tabularx}{\linewidth}{ r l l X }
 \lsptoprule
\textbf{Generation} & \textbf{Author} & \textbf{Lifespan}&\textbf{Text \& publication year} \\
\midrule
1   &   Gaston Vasseur  &   1904--1971   &   \textit{Lettes à min cousin Polyte} (1938--1971) \\
2   &   Jean Leclercq   &   1931--2021   &   \textit{Chl’autocar du Bourq-Éd-Eut} (1996) \\
3   &   Jacques Dulphy  &   1959        &   \textit{Ch’Dur et pi ch’Mo}, Tome III (2011)\\
\lspbottomrule
\end{tabularx}
\end{table}

We extracted all instances of unambiguous 1\textsc{pl} reference from our Picard and French corpora. As is customary, our data collection excluded contexts where no variation is possible, such as fixed expressions (e.g., \textit{o diroait qu’} ‘it seems like’, \textit{conme o dit} ‘as we say’). Each token was subsequently coded for the binary dependent variable, i.e., 1\textsc{pl} or 3\textsc{sg} verbal morphology, and for a variety of independent variables: verb tense, the presence or absence of an overt semantic reference expression (see \ref{02:ex:2a}--\ref{02:ex:2b}), as well as restriction and specificity of the 1\textsc{pl} semantic reference. In this paper, we follow the example of \citet{king_interplay_2011} and focus on referential restriction.

\ea \label{02:ex:2}
\ea{\label{02:ex:2a}
\gll Quoè qu’	\textbf{oz} \textbf{all-ons}  dévnir,   \textbf{mi}   \textbf{pi} 	\textbf{chol} \textbf{Dure}? \\
what that 	we go-1\textsc{pl}  become  me  and 	the.\textsc{f}   Dure\\}\jambox*{(\textit{DurMo} 418)}
\glt ‘What are we going to become, me and the Dure [my wife]?’ \\

\ex\label{02:ex:2b} \gll \textbf{Oz} 	\textbf{é-r-ons} 	eune  armée  forte    pour  pu 		avoér  la     djerre \\
we 	have.\textsc{fut}-1\textsc{pl} 	an.\textsc{f}   army    strong  for    no.longer 	have    the.\textsc{f}  war\\\jambox*{(\textit{Lettes} 1945, 165)}
\glt ‘We [implied: all French citizens] will have a strong army to no longer have war’
\z
\z

Our coding for referential restriction followed \citegen{boutet_reference_1986} ternary distinction based on restriction and specificity, as operationalized in \citet{rehner_learning_2003} and \citet[482]{king_interplay_2011}. Specifically, we distinguished between a restricted group which is specific and includes only people known to the speaker, such as members of their family (see \ref{02:ex:3a}), a specific unrestricted group of individuals, some of whom may not be known by the speaker, such as employees of a large factory or all French people (see \ref{02:ex:3b}), and a general unrestricted group -- humankind, people in general -- which includes the speaker (see \ref{02:ex:3c}).\footnote{General unrestricted references include examples that include all of humanity at a past time; e.g., \textit{Au XVI}$^e$ \textit{siècle, on mourait beaucoup plus jeune} ‘In the 16th century, one died much younger’.}  Our overall data set includes 61 tokens of unambiguous 1\textsc{pl} for which the discursive context did not allow us to reliably determine whether the group being referenced was restricted and/or specific; these were coded as ``ambiguous'' for semantic reference.\footnote{An anonymous reviewer asks how we have coded the semantic reference of examples such as \textit{Alors, on se proméne?} ‘So, one’s taking a walk?’, where \textit{on} refers to a neighbor that the speaker would pass on the sidewalk. Such examples are excluded from our analysis, as they do not meet the definition for our variable, that is, a pronoun that refers to a group of speakers that includes the speaker.}

\ea \label{02:ex:3} French
\ea\label{02:ex:3a} \gll c’est  pour ça     qu’	 \textbf{nous} \textbf{av-ons} 	 appelé  notre  fille 	  [Marie]. \\
{it is}   for    that  that we    have-1\textsc{pl} called   our     daughter Marie\\\jambox*{(Jérôme D.)}
\glt ‘this is why we [my wife and I] called our daughter Marie’

\ex\label{02:ex:3b} \gll \textbf{on} 	n’      \textbf{sait} 	  pas  pour  qui   \textbf{on}    \textbf{travaille}. \\
one 	\textsc{neg}  know.3\textsc{sg} not  for     who  one  work.3\textsc{sg}\\\jambox*{(Joël T.)}
\glt ‘we [my coworkers and I] don’t know who we work for’

\ex\label{02:ex:3c} \gll au 	        bout  d’ un 	certain temps, 	\textbf{on}   \textbf{a} 		\textbf{beau} 	    être catholique [\ldots] y 	a…  certaines     vertus   qui   prennent  le    dessus.\\
    at.the.\textsc{sg}   end   of a.\textsc{m}	some   time 	one  have.3\textsc{sg} 	beautiful  be Catholic 	 {}  there is     certain.\textsc{f}.\textsc{pl} virtues  that  take.3\textsc{pl}   the   top\\ \jambox*{(Jérôme D.)}
\glt  ‘at some point, one may be Catholic [but] some behaviours take over’ \\

\z
\z

Although \citet[482]{king_interplay_2011} “did not include reference to humankind as a whole in the unrestricted group due to the difficulty of distinguishing such utterances from indefinite reference”, the rich discursive context of our written data allows us to expand on their work by further distinguishing references to general unrestricted groups that include the speaker and all of humanity, i.e. general 1\textsc{pl} semantic reference, from 3\textsc{sg} indefinite reference. For instance, the French \textit{on} in (\ref{02:ex:4}) unambiguously refers to an indefinite 3\textsc{sg} -- the speaker was a child during the war and did not participate in the violence described -- and the Picard \textit{o} in (\ref{02:ex:5}) unambiguously excludes the speaker who is instead included in the object pronoun \textit{no}. While both studies exclude examples of this type from the variationist analysis of 1\textsc{pl}, our analysis includes utterances like (\ref{02:ex:6}), where the discursive context, which explicitly refers to the time when the letter’s author and his addressee were young, makes it clear that the unrestricted general group includes the speaker. This methodological decision allows for a more fine-grained data set on which to test the role of semantic specificity on the incursion of 3\textsc{sg} into 1\textsc{pl} domain.

\ea \label{02:ex:4} French\\
\gll \textbf{on} 	\textbf{fais-ait}	       sauter 	leur 	maison   {ou bien}  on 	les 	tu-ait.\\
one 	made-3\textsc{sg}   burst 	their.\textsc{f} 	house     or 	    one 	them 	kill-3\textsc{sg}\\ \jambox*{(Joseph L.)}
\glt ‘their houses would get bombed or they would get killed’
\z

\ea \label{02:ex:5} Picard\\
\gll J’ai 	idèe [\ldots] 	qu’  o 	  no 	prind 	  pour 	des 	  coéchons\\
{I have} 	idea [\ldots] 	that one us 	take.3\textsc{sg} for 	some.\textsc{m}  pigs\\ \jambox*{(\textit{Lettes} 1946, 152)}
\glt ‘I think [\ldots] that we are taken for pigs’
\z

\ea \label{02:ex:6} Picard\\
\gll O 	din-ouot   à 	trouos heures, t’	    in 	  souviens  -tu ? \\
we 	lunch-\textsc{ipfv} at 	three   hours    you.\textsc{refl} of-it 	  recall        you\\ \jambox*{(\textit{Lettes} 1956, 638)}
\glt ‘we used to have lunch at 3 o’clock, do you remember?’
\z

\section{Results}
Let us now turn to the results of our quantitative analysis. First, our contemporary oral data reported in \figref{02:Fig1} show a clear dominance of the innovative French-like 3\textsc{sg} form in our oral data: use of the 1\textsc{pl} form is marginal in both oral French (1.9\%, N~=~368) and oral Picard (15.1\%, N~=~338). This pattern stands in sharp contrast with our contemporary and older written data, where 3\textsc{sg} is far from generalized (54.6\%, N~=~1,304). Unsurprisingly, texts appear more conservative than spontaneous speech.

\begin{figure}
    \includegraphics[scale=.75]{figures/auger_fig 1.png}
    \caption{1\textsc{pl} in Vimeu French and Picard}
    \label{02:Fig1}
\end{figure}

\newpage
The high proportion of 3\textsc{sg} forms in interviews could be interpreted as evidence that oral Picard is converging toward French, a language where the change from 1\textsc{pl} pronoun and inflectional morphology to 3\textsc{sg} morphology is quite advanced. In fact, a similar pattern emerged from a previous analysis of verbal negation in the same oral corpus \citep{villeneuve_chtileu_2013}. However, a closer examination of linguistic factors reveals that a large proportion of the 1\textsc{pl} forms found in our oral data refers to specific restricted groups, as exemplified in (\ref{02:ex:7}) where the 1\textsc{pl} verbs refer to the participants in a specific hunting event, despite the fact that 3\textsc{sg} is also attested in these semantic contexts (see \ref{02:ex:8}); where the 1\textsc{pl} and 3\textsc{sg} verbs refer to the speaker and his fellow students.

\ea \label{02:ex:7} Picard\\
\gll \textbf{Oz}  \textbf{ons}         veillé    ein     tchot molé pi   \textbf{oz}  \textbf{ons} 	      fini     pér  nos  adoveu       \\
we  have.1\textsc{pl} stay.up one.\textsc{m} little  bit    and we have.1\textsc{pl} finish by  us   doze.off\\ \jambox*{(Joël T.,320)}
\glt ‘we stayed up a bit and we ended up dozing off’
\z

\ea \label{02:ex:8} Picard\\
\gll Mais nous, \textbf{o}   \textbf{sav-o-ème}     bien, à  l’       école   normale             que, quand \textbf{oz}   \textbf{ét-o-ème}       avec  éch’   professeur \textbf{o}     \textbf{dis-o-ait}  {« pluriel »}, mais quand \textbf{oz} \textbf{ét-o-ait}    intré       nous, \textbf{o}    \textbf{dis-ou-ot}  {« pluriél »}. \\
but    us     we knew-\textsc{pst}-1\textsc{pl} well  at the.\textsc{f} school teacher.training that  when we  were-\textsc{pst}-1\textsc{pl} with  the.\textsc{m}  professor   we  said.3\textsc{sg}   pluriel    but when we were.3\textsc{sg}  between us      we said.3\textsc{sg}    pluriél\\\jambox*{(Joseph L., 51)}
\newpage
\glt ‘But we knew well, at teacher training school that when we were with the professor we said “pluriel”, but when we were among us, we said “pluriél”'
\z

	The frequency with which the 1\textsc{pl} form is still used in written Picard can shed light on the mechanism behind similar morphological changes in Romance languages. Specifically, our 592 tokens of 1\textsc{pl} \textit{o} -\textit{ons} forms (or 45.4\% of our written data), carefully coded for referential restriction, represent a valuable data set with which to test the effect of referential restriction on 1\textsc{pl} morphology. Indeed a detailed analysis of 1\textsc{pl} semantic reference indicates that the innovative French-like 3\textsc{sg} form is still primarily associated with unrestricted general reference  in written Picard (91.9\%, N~=~678 vs 15.3\%, N~=~626 in other contexts) and is barely used to refer to restricted groups, as we can see in \figref{02:Fig2}.
\begin{figure}
    \centering
    \includegraphics[scale=.75]{figures/auger_fig2.png}
    \caption{1\textsc{pl} and semantic reference in written Picard}
    \label{02:Fig2}
\end{figure}

Although the use of the 1\textsc{pl} form remains much more frequent in written than in oral Picard, there is a possibility that its frequency may be gradually decreasing over time. One piece of data that suggests such a decrease comes from a real-time analysis of the chronicles in Vasseur’s \textit{Lettes}. Since these chronicles were published over a period of 33 years, we can compare the rate of use of 1\textsc{pl} over time for an individual author. This comparison reveals an apparent decrease in 1\textsc{pl} use across this portion of Vasseur’s lifespan, from 44.7\% in the 1940s to 35.6\% in 1960s.



\begin{table}[t]
\caption{Frequency of 1pl. per author}
\label{02:table4}
 \begin{tabularx}{\linewidth}{ X X l }
 \lsptoprule
\textbf{Author} & \textbf{1pl/total} & \textbf{\% 1pl}\\
\midrule
Vasseur & 171/547 & 31.3\% \\
Leclercq & 170/294 & 57.5\% \\
Dulphy & 195/402 & 48.5\% \\
\lspbottomrule
\end{tabularx}
\end{table}

\begin{table}[b]
\caption{Frequency of semantic reference type per author}
\label{02:table5}
 \begin{tabularx}{\linewidth}{ l Y Y Y Y Y Y r }
\lsptoprule

Author & \multicolumn{2}{c}{\textbf{Unrestricted}}& \multicolumn{2}{c}{\textbf{Unrestricted}}&\multicolumn{2}{c}{\textbf{Restricted}} & \textbf{Total}\\
 & \multicolumn{2}{c}{\textbf{general}}& \multicolumn{2}{c}{\textbf{specific}}&\multicolumn{2}{c}{\textbf{group}} &\\%double check for abbreviations with BB
 & N &\% & N & \% & N & \%& \\
\midrule
Vasseur & 362 & 66.2\% & 120 & 21.9\% & 65 & 11.9\% & 547\\
Leclercq & 120 & 40.8\% & 31 & 10.5\% & 143 & 48.6\% & 294\\
Dullphy & 196 & 40.8\% & 85 & 21.1\% & 121 & 30.1\% & 402\\
\lspbottomrule
\end{tabularx}
\end{table}


In order to test the possibility of a change in progress in the Vimeu Picard community more broadly, i.e., the gradual replacement of 1\textsc{pl} by 3\textsc{sg}, we turn to our data from three different authors who represent more distant time periods: the 1940s--the 1960s, the 1990s, and the 2000s. As we can see in \tabref{02:table4}, the overall frequencies of 1\textsc{pl} do not suggest a gradual loss of 1\textsc{pl}, as the lowest frequency is found in the older data from Vasseur and the highest occurs in Leclercq’s data. What these numbers do not tell us, however, is whether the 1\textsc{pl} and 3\textsc{sg} verbs used by the three authors have similar semantic distributions. Indeed, the greater use of 1\textsc{pl} in Leclercq’s data may be attributable, at least in part, to the fact that his novel tells the story of a young man in the 1950s, a genre that may result in a higher number of restricted references than Vasseur’s chronicles, which take the form of letters and postcards that discuss past and current events and relate them to the personal lives of their author and his addressee, or Dulphy’s chronicles, which consist of conversations on current events between two men. In order to tease out the possibility that the different rates of 1\textsc{pl} in the three texts might be due to an uneven distribution of the data across semantic references rather than to change in progress, we now break down our data for each author by semantic category. \tabref{02:table5} confirms that the distribution of semantic values differs greatly across texts, and that this difference provides a plausible explanation for the frequencies of 1\textsc{pl}. Indeed, Vasseur’s text, which features the largest frequency of 3\textsc{sg}, has by far the largest proportion of unrestricted general referents, a context known to favour the innovative 3\textsc{sg}, while the one that has the highest proportion of 1\textsc{pl}, Leclercq’s, contains the largest number of restricted group referents, a context resistant to the incursion of 3\textsc{sg} into 1\textsc{pl} domain.


We can now attempt to determine whether use of 3\textsc{sg} is really spreading over time in written Picard by breaking down our data by author and semantic reference, as shown in \figref{02:Fig3}. This nuanced breakdown reveals considerable stability over time. For general unrestricted referents, 3\textsc{sg} strongly dominates in all three authors, with an average frequency of 91.9\%. For specific unrestricted and restricted groups, 1\textsc{pl} dominates in the data from all three authors. However, signs of opposite trends separate the more recent data (1990s and 2000s) from those from the mid-20th century. Surprisingly, use of 3\textsc{sg} decreases over time for specific unrestricted referents. But, most interestingly, use of 3\textsc{sg}, which was not attested in Vasseur’s data, makes an appearance in the 1990s and 2000s data. Examples (\ref{02:ex:9}--\ref{02:ex:11}) attest to the variation between 1\textsc{pl} and 3\textsc{sg} in all three semantic contexts, namely unrestricted general (\ref{02:ex:9}), unrestricted specific (\ref{02:ex:10}), and restricted groups (\ref{02:ex:11}).

\begin{figure}[t]
    \includegraphics[width=\textwidth]{figures/auger_fig3.png}
    \caption{1\textsc{pl} and semantic reference in written Picard}
    \label{02:Fig3}
\end{figure}

\ea \label{02:ex:9} Picard\\
\ea
\gll \textbf{o} 	n’    \textbf{porr-ons} 	   pu 	      vive  su  la      terre\\
we 	\textsc{neg} can-\textsc{fut}-1\textsc{pl} anymore live  on the.\textsc{f} earth\\ \jambox*{(\textit{Lettes} 1956, 644)}
\glt    ‘we won’t be able to live on the land’
\ex
\gll \textbf{o} n’ \textbf{laiche} mie mourir parsonne \\
one \textsc{neg} let not die anybody\\ \jambox*{(\textit{Lettes} 1946, 168)}
\glt ‘we don’t let anyone die’
\z
\z

\newpage

\ea \label{02:ex:10} Picard\\
\ea
\gll o    n-n  av-ons    connu   deux d’ djerres [\ldots], o    sav-ons    ch   qu’ i  n-n   est  \\
we of-it have-1\textsc{pl} known  two  of wars 	 {}    we know-1\textsc{pl} that that it of-it is\\ \jambox*{(\textit{Lettes} 1956, 655)}
\glt ‘we have gone through two wars, we know what it is’

\ex
\gll o     s’    plaint       souvint in France qu’  oz   est  d’ trop boin [\ldots] \\
one self  complain  often    in France  that one is   of  too  good\\ \jambox*{(\textit{Lettes} 1966, 1180)}
\glt ‘we often complain in France that we’re too good’
\z
\z

\ea \label{02:ex:11}Picard\\
\ea
\gll Nous deux  mn’   honme, o    n’    é-r-o-éme	   pu 	     qu’  à  minger\\
us       two  my.\textsc{m}  man     we \textsc{neg}  have-\textsc{fut}-\textsc{ipfv}-1\textsc{pl} anymore that to eat\\ \jambox*{(\textit{Chl’autocar} 1996, 20)}
\glt ‘My husband and I, we’d only need eat’

\ex
\gll oz   est  quate chonq  camarades 	à   l’       école    insanne \\
one  is   four    five     buddies 	at  the.\textsc{f}  school  together\\ \jambox*{(\textit{Chl’autocar} 1996, 59)}
\glt ‘we’re four or five buddies in school together’
\z
\z




We close this section with a discussion of two examples drawn from newspaper chronicles that mix comments on current events and events from the personal lives of the characters that they feature and that were written and published 60 years apart. The first example (\ref{02:ex:12}), published in 1946, features four tokens of 3\textsc{sg} and one token of 1\textsc{pl}. The first instance of 3\textsc{sg} occurs in a frozen phrase (\textit{oz a bieu dire}) in which the subject has unrestricted general reference. While the last two  do not occur in frozen phrases, they also have unrestricted reference. The second token, \textit{oz étouot pététe gramint moins riches} refers to the unrestricted but specific group of people who lived in the author’s village and surrounding area. As for the only 1\textsc{pl} token, it refers specifically to the letter’s author and his addressee. This short passage illustrates that, for Gaston Vasseur, 3\textsc{sg} and 1\textsc{pl} still have distinct meanings. Published in 2006, the second example (\ref{02:ex:13}) features four tokens: three 3\textsc{sg} and one 1\textsc{pl}. The first token illustrates the exclusive reference for which use of 1\textsc{pl} is excluded. The next two tokens of 3\textsc{sg} clearly refer to the two protagonists and are coreferential with the 1\textsc{pl} token, as the last sentence, which lists the people present at the \textit{réveillon}, shows. Thus, even though use of 3\textsc{sg} for restricted reference remains infrequent in our most recent Picard data, this example provides evidence for a possible incipient change similar to the one that has taken place in French.

\ea \label{02:ex:12} Older written Picard (1946)\\
\gll Mais, \textbf{oz} \textbf{a} bieu dire, Polyte, \textbf{oz} \textbf{étouot} pététe  gramint moins riche du temps qu’ \textbf{oz} \textbf{alloémes} au djignel, \textbf{oz} \textbf{étouot} moins riche, mais \textbf{oz} \textbf{étouot} moins béte, moins  mawais d’ l’ un à l’ eute.\\
but one has beautiful say Polyte one was maybe  a-lot less rich of-the.M.SG  time that we go-IPFV-1P to-the.M.SG guignole one was less rich but one was less mean less bad of  the.SG one.M  to  the.SG  other\\ 
\glt ‘But, it is all very well, we were maybe much less rich when we used to \textit{alleu au djignel} (go door to door and ask for apples on December 24), we were less rich, but we were less stupid, less mean toward each other.’ [\textit{Lettes}, 169]
\z

\ea \label{02:ex:13} Contemporary written Picard (2000s)\\
\ea Ch'Dur\\
\gll ch’ est point pasqu’ \textbf{o} n’ o point pérlè d’ nous qu’ \textbf{o} n’ s’ a point vus. Ti point vrai, ch’Mo?\\
it is not because one NEG has not spoken of us that one NEG REFL has not seen.PL INT not true, ch’Mo\\ 
\glt ‘it’s not because they haven’t talked about us that we haven’t seen each other. Right, ch’Mo?
% Whats pi ¯\_(ツ)_/¯
\ex Ch'Mo\\
\gll Pour seur! \textbf{O} s’ a meume vu, et pi rvu. \textbf{Oz} ons meume rinvillonnè insanne. À vo moéson, qu’ a s’ a passè. Y avoait mi pi chol Molle, ti pi chol Dure, és mére, et pi Niflette no bétail dé tchiénne…\\
for sure one REFL has even seen and \textit{\textbf{ }} seen-again  we have.1PL even celebrated together. At your house that it REFL has happened. there was me and the.FEM Molle  you.SG and the.FEM Dure her mother and \textit{\textbf{ }} Niflette our animal of bitch\\
\glt For sure! We have even seen and seen each other, again and again. We have celebrated Christmas together. At your house, it was. There was me and chol Molle, you and chol Dure, her mother, Dorine, and Niflette our dog’ [\textit{Dur Mo}, 411]
\z
\z

\section{Conclusion}
The grammaticalization of pronouns and determiner phrases previously used to refer to indefinite referents into 1\textsc{pl} in French and in Brazilian Portuguese has received considerable attention from linguists. While previous studies have identified linguistic and social factors that favor this process, its analysis in contemporary French has suffered from two important limitations: the marginal use of the \textit{nous} pronoun and the uncertainty concerning the specific 1\textsc{pl} form that has undergone replacement. The Picard data from the Vimeu region that we have analyzed in this paper circumvent both limitations, as use of \textit{os} -\textit{ons} is well documented historically and this form remains solidly implanted in contemporary usage. Our diachronic analysis of written data spanning from the 1940s until the 2000s reveals a Gallo-Romance variety that remains largely unaffected by the changes that have taken place in colloquial French and in oral Picard, and where the choice between 1\textsc{pl} and 3\textsc{sg} is strongly correlated with referential restriction. While unrestricted general referents strongly favor 3\textsc{sg} and show marginal use of 1\textsc{pl}, 1\textsc{pl} remains the almost exclusive variant for specific referents but shows some signs of incipient change. Interestingly, the semantic category that would be expected to serve as a gateway for the innovative uses of 3\textsc{sg}, namely unrestricted specific referents, appears to increasingly favor 1\textsc{pl} pronouns. Analysis of a larger corpus of written data from different genres and produced by a variety of authors will be necessary in order to confirm or disconfirm the results from our preliminary analysis.

In short, our examination of this variable demonstrates the importance of carefully considering linguistic conditioning through the comparative method when assessing language change in two typologically related varieties, especially when testing popular claims that a minority language is converging toward its dominant counterpart in a bilingual community. It also shows the importance of analyzing multiple linguistic features. Indeed, the conservative character of 1\textsc{pl} in Picard mirrors what has been reported for \textit{ne} deletion, while contrasting with this variety’s innovative character with respect to subject doubling and the generalization of a single auxiliary, \textit{avoér} ‘have’ \citep{auger_using_2017, auger_building_2019, villeneuve_chtileu_2013}.

\printbibliography[heading=subbibliography,notkeyword=this]
\end{document}
