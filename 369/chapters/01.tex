\documentclass[output=paper,colorlinks,citecolor=brown]{langscibook}
\ChapterDOI{10.5281/zenodo.7525092}
\author{Zsuzsanna Fagyal\affiliation{University of Illinois Urbana-Champaign}}
\title[Integrative approach to variation and change in French nasal vowel systems]{For an integrative approach to variation and change in French nasal vowel systems}
\abstract{The complex dynamics driving the evolution of phonemic nasal vowels have long puzzled linguistic historians. This paper, focusing on French, discusses some of this rich complexity through a glimpse at internal (linguistic) and external (social) dynamics that led to the development of multiple nasal vowel systems. It suggests that parallel findings from acoustic and articulatory phonetics (production), psycholinguistics (perception), and computational modeling could be aligned to shed new light on some of the mechanisms behind well-attested historical and ongoing changes. Comparative historical and experimental data, coupled with computational modeling, could provide new ways of approaching the evolution of nasal vowel systems in Romance.}

% \usepackage{tabularx}
% \usepackage{langsci-basic}
% \usepackage{langsci-optional}
% \usepackage{langsci-gb4e}
% \bibliography{localbibliography}
% %\newcommand{\orcid}[1]{}
% \pagenumbering{arabic}
% \setcounter{page}{16}
% \newcommand{\noop}[1]{16}

%custom footer for preprints


\IfFileExists{../localcommands.tex}{
  \addbibresource{../localbibliography.bib}
  \usepackage{langsci-optional}
\usepackage{langsci-gb4e}
\usepackage{langsci-lgr}

\usepackage{listings}
\lstset{basicstyle=\ttfamily,tabsize=2,breaklines=true}

%added by author
% \usepackage{tipa}
\usepackage{multirow}
\graphicspath{{figures/}}
\usepackage{langsci-branding}

  
\newcommand{\sent}{\enumsentence}
\newcommand{\sents}{\eenumsentence}
\let\citeasnoun\citet

\renewcommand{\lsCoverTitleFont}[1]{\sffamily\addfontfeatures{Scale=MatchUppercase}\fontsize{44pt}{16mm}\selectfont #1}
  
  %% hyphenation points for line breaks
%% Normally, automatic hyphenation in LaTeX is very good
%% If a word is mis-hyphenated, add it to this file
%%
%% add information to TeX file before \begin{document} with:
%% %% hyphenation points for line breaks
%% Normally, automatic hyphenation in LaTeX is very good
%% If a word is mis-hyphenated, add it to this file
%%
%% add information to TeX file before \begin{document} with:
%% %% hyphenation points for line breaks
%% Normally, automatic hyphenation in LaTeX is very good
%% If a word is mis-hyphenated, add it to this file
%%
%% add information to TeX file before \begin{document} with:
%% \include{localhyphenation}
\hyphenation{
affri-ca-te
affri-ca-tes
an-no-tated
com-ple-ments
com-po-si-tio-na-li-ty
non-com-po-si-tio-na-li-ty
Gon-zá-lez
out-side
Ri-chárd
se-man-tics
STREU-SLE
Tie-de-mann
}
\hyphenation{
affri-ca-te
affri-ca-tes
an-no-tated
com-ple-ments
com-po-si-tio-na-li-ty
non-com-po-si-tio-na-li-ty
Gon-zá-lez
out-side
Ri-chárd
se-man-tics
STREU-SLE
Tie-de-mann
}
\hyphenation{
affri-ca-te
affri-ca-tes
an-no-tated
com-ple-ments
com-po-si-tio-na-li-ty
non-com-po-si-tio-na-li-ty
Gon-zá-lez
out-side
Ri-chárd
se-man-tics
STREU-SLE
Tie-de-mann
}
  % \togglepaper[3]%%chapternumber
}{}

\begin{document}
\maketitle

\section{Introduction}
Vowel nasalization has had a widespread impact on the phonological systems of Romance languages and, in a few cases, led to lexically distinctive nasal vowel phonemes. While the acoustic and articulatory properties of nasalization exhibit universal properties, the same cannot be said about the evolution of nasal vowel phonologies. In fact, even nasal vowel systems within the same Romance language show such degrees of variation that it would be difficult to offer generalizations of common trajectories and pathways of evolution.

And yet, for many decades, “French [has been] widely believed to provide the classic example of the nature and characteristics of vowel nasalization” \citep[][1]{Sampson1999}. In his Preface of \textit{Nasal Vowel Evolution in Romance}, Sampson states that one of his main motivations for writing his treatise was to inspire others to move beyond French as the purportedly universal key to nasalization in Romance:
\begin{quote}
/…/ it is hoped that the present work will prove of some interest to general phonologists who wish to know more of the diversity and complexity of nasal vowel evolution in Romance beyond the familiar developments found in French \citep[][v]{Sampson1999}.
\end{quote}

He argued that the persistent myth of standard French as the measuring stick of nasal vowel evolution is an artifact due to the unacknowledged bias towards a culturally prestigious Romance language. Upon closer examination, he suggested that many developments in French taken as “general guiding principles of change” have arisen from “exceptional circumstances” motivated by social factors.

In her \textit{Linguistic Change in French}, Posner also reiterates the prevailing convention of the universality of nasal vowel evolution in French. Similar to Sampson, she points to the importance of standardization and focuses her critical analysis on earlier claims of vowel height hierarchy:
\begin{quote}
    […] The lowering of nasal vowels in standard French has led some theorists to postulate that there is a universal tendency for nasality to sit better on low vowels. [But] the long-held doctrine of nasalization \textit{par étapes}, with low vowels succumbing earlier than high vowels no longer gets support from the French philological evidence. \citep[][235--236]{Posner1997}.
\end{quote}

In this paper, I argue in support of social motivations behind some of these changes in the nasal vowel system of so-called standard French. Based on a short review of historical evidence and contemporary studies of dialectal variation, I suggest that the lowering of the nasal vowel system appears to be the outcome of large-scale accommodation to emerging pronunciation norms in the variety of French selected as the language of the administration by the French kings starting from the 16th century. In light of other varieties that do not exhibit such realizations, the lowering of nasal vowels, indeed, appears exceptional rather than universal. If this accommodation hypothesis is true, then patterns of nasalization known from French, the longest-serving prestige lingua franca in Europe until the 19th century \citep[][]{Wright2004}, have been very likely reinterpreted as universal and subsequently attributed to other Romance languages.

However, the many ways in which social evaluation could have influenced the selection of norm-conforming nasal pronunciations among educated native speakers over time would be impossible to prove by controlled experiments and direct observations. Therefore, I suggest turning to indirect evidence from two dialects of French that reveal possible articulatory strategies whose acoustic effects could have been perceived to match the desired outcomes of prestigious pronunciation. In the last section of the paper, I discuss the implications of this type of ``integrative approach'' for future studies.

\section{The historical record}
The processes that led to phonemic nasalization in French are generally thought to be well-understood and indeed largely considered universal. They are typically explained in three steps:
\begin{itemize}
    \item the nasality of syllable-final nasal consonants word-finally and in closed syllables spreads to preceding nuclear vowels,
    \item through regressive assimilation, over the course of several centuries, vowels before the nasal consonant become contextually nasalized,
    \item after losing their conditioning environment, the vowels become bimoraic long vowels and phonologize to cue phonemic distinctions in the language.
\end{itemize}

The long vowel /i/ in the Latin \textit{vinum} ‘wine’, as shown in (\ref{ex:1}), first nasalizes to /vĩ/ and then, after losing its conditioning nasal consonant in coda position, lowers to a more open nasal vowel /ɛ̃/, one of the four lexically distinctive nasal phonemes attested in modern French. The Latin long vowel in the first syllable of the word \textit{plenum} ‘plenty’, shown in (\ref{ex:2}), first diphthongizes to /plẽĩn/ and, in so-called standard French, lowers and eventually becomes monophthongal (/ɛ̃/) in \textit{plein} ‘full’. Like most monosyllabic words with a word-final nasal vowel phoneme, \textit{vin} ‘wine’ and \textit{plein} ‘plenty’ are lexically distinctive and stand in phonemic oppositions, among others, with \textit{vent} /vɑ̃/, (ils) \textit{vont} /vɔ̃/, \textit{plan} /plɑ̃/, and \textit{plomb} /plɔ̃/.

\ea \label{ex:1}
V\textbf{Ĩ}NUM > /`vi:n(u)/ > /v\textbf{ĩ}:n/ > /v\textbf{ĩ}(n)/ >
/v\textbf{ĩ}/ > /v\textbf{ɛ̃}/ \textit{vin} `wine'
\z
\ea \label{ex:2}
PL\textbf{Ẽ}NUM > /pl\textbf{ẽĩ}n/ > /pl\textbf{ẽ}/ > /pl\textbf{ɛ̃}n/ > /pl\textbf{ɛ̃}/ \textit{plein} `full'
\z

As far as vowel inventories are concerned, the rich allophonic nasal system of Early Old French (\figref{fig:1}), characterized by widespread contextual nasalization, became reduced to fewer allophones and, in some dialects, to monophthongs during the Middle French period. By the end of the 16th century, nasal consonants following tautosyllabic nasal vowels sounded less prevalent (\textit{amuïssement}), which made some historians speculate on their fusion with the preceding vowel, yielding combined perceptual effects of lingual and nasal articulations \citep[][34]{Morin1994}.

\begin{figure}

%    \includegraphics[scale=1]{figures/fagyal_fig1.png}\\
\begin{tikzpicture}
\node(i) at (0,2.5) {ĩ};
\node(e) at (0,2) {ẽ};
\node(3) at (1,2) {ə̃};
\node(aea) at (1,1) {æ̃/ã};
\node(c) at (1.8,1.5) {ɔ̃};
\node(o) at (2,2) {õ};
\node(yu) at (2.3,2.5) {ỹ/ũ};
\draw (.2,1.9) -- (i.285);
\draw (.2,1.7) -- (aea.120);
\draw (aea.60) -- (1.6, 1.7);
\draw (1.7, 1.9) -- (yu.west);

\node(ei) at (4, 2) {ẽĩ};
\node(aeiai) at (5, 1) {æ̃ĩ/ãĩ};
\node(oi) at (5.5,2) {õĩ};
\node(yuui) at (5.7,2.5) {ỹĩ/ũĩ};
\draw (ei.south) -- (aeiai.150);
\draw (aeiai.135) -- (aeiai.45);
\draw (aeiai.30) -- (yuui.300);

\node(ie)[circle, draw] at (7, 2) {ĩẽ};

\node(yeue)[circle, draw] at (9,2) {ỹẽũẽ};
\end{tikzpicture}
    \caption{Early Old French (langues d’Oïl, tenth-eleventh centuries) exhibited many nasal allophones in the vowel system.}
    \label{fig:1}
\end{figure}

Although the pronunciation of nasals in Oïl varieties remained variable and, in the same time, quite specific to certain dialect areas, educated Parisian speakers were beginning to solidify their evaluations of what constituted socially desirable pronunciation in the newly forming administrative standard. Tendencies pointed to a general simplification of the nasal vowel system, albeit not without lengthy competitions and memorable debates that are well documented in the French grammarian tradition.

For instance, less than a decade after grammarians declared that the confusion between \textit{in} and \textit{en}  should be considered “a vice of provincial influence” \citep[][]{D’Aisy1685}, the merger of the two unrounded front vowels /ẽ/ and /ɛ̃/ was considered unremarkable even by authoritative commentators such as Abbé Dangeau (\citeyear[][76]{Dangeau1692})  who declared no longer hearing any difference between nasal vowels in \textit{cert\textbf{ain}} ‘certain’, \textit{dess\textbf{ein}} ‘purpose’, and \textit{div\textbf{in}} ‘divine’. The high back vowel series underwent the same evolution. 16th-century grammarians were still condemning the pronunciation of ‘ouïsites’, people who pronounced the vowel ``o'' like ``u'' in words such as \textit{chose} ‘thing’, as these speakers also tended to have a less open, u-like pronunciation of ``o'' before nasals in words such as \textit{d\textbf{u}nc} ‘so’ and \textit{m\textbf{eu}chiseu} ‘Sir’. However, towards the mid-17th century, a more open and centralized realization of the back nasal in ``oN'' letter sequences won out in nearly all lexical classes in the emerging standard language and the more closed, u-like realizations of the vowel became associated with the lower classes:

\begin{quote}
    The opposition of the learned classes and the grammarians to the pronunciation of \textit{oN} in \textit{ou} grows in the seventeenth century and denasalized \textit{o}-s pronounced \textit{ou} are denounced along with pure ‘ouïsmes’ \citep[][334]{Ruhlen1979}.
\end{quote}

\begin{figure}
%    \includegraphics[scale=1]{figures/fagyal_fig2.png}
    \begin{tikzpicture}[align=left]
    \node(vin) at (0, 4) {e.g. \textit{vin} \& \textit{certain}};
    \node(ie)[single arrow, draw, shape border rotate = 270] at (0, 3) {ĩ\\ẽ};
    \node(e) at (0, 2) {\textcolor{lsRed}{ɛ̃}};

    \node(3) at (2.5,2.7) {ə̃ \textcolor{lsRed}{(œ̃)}};
    \node(aea)[single arrow, draw, shape border rotate = 270] at (2.5,2) {æ̃/ã};
    \node(a) at (2.5,1) {\textcolor{lsRed}{ã}};

    \node(donc) at (5,4) {e.g. \textit{donc} \& \textit{monsieur}};
    \node(yuo)[single arrow, draw, shape border rotate = 270] at (5,3.1) {ỹ/ũ\\õ};
    \node(c) at (5,2) {\textcolor{lsRed}{ɔ̃}};
    \end{tikzpicture}
    \caption{Schematized depiction of the general lowering of the standard French vowel space resulting in a four-way system.}
    \label{fig:2}
\end{figure}

The resulting subsystem of nasal vowels in French selected as the future administrative standard  comprised only a few nasal vowels (\figref{fig:2}), and nearly all of them preferably pronounced as monophthongal. Occasional suggestions of a high nasal vowel (/ĩ/) continued to surface, “confined to the prefix \textit{in-}/\textit{im-} in learned words” \citep[][82]{Sampson1999}, but its use was attributed to learned styles of speaking rather than emerging phonological contrasts. The well-known four-way system (\figref{fig:2}), still the norm in conservative standard usage and orthoepic conventions today, became established and reinforced by increasingly unified typographic conventions in the following centuries.

It must be pointed out that the precise quality of these vowels is difficult to reconstruct. One reason is the nature of philological evidence that is necessarily filtered through typographic choices and spelling conventions that tend to obscure representations of variation in vowel height. When discussing the transcription of nasal vowels, for instance, \citet[67]{Hansen1998}) refers to “the graphic \textit{definition} of nasal vowel phonemes” (emphasis from me). Others note that discussions of the underlying representations of nasal vowels are always presented in terms of letter sequences (V, VN), contributing to simplified interpretations of the physiology of nasalization \citep[][236]{Posner1997}. Thus, it is likely that, starting from the early modern era, heavy orthographic bias in the representation of nasal vowels obscured some of their unique dialect-specific qualities.

In the modern era, the historically attested general lowering of the nasal vowel system of French spoken in and around Paris continued.   The tendency of the front nasal /ɛ̃/ to both lower and centralize to /ɑ̃/ even in word pairs where lexical distinctions could be jeopardized \citep[][]{Fónagy1989}, such as in examples (\ref{ex:3}) through (\ref{ex:5}) :

\ea \label{ex:3}
\ea  C’est intérieur.\\`This is interior.'\\
\ex C’est antérieur.\\ `This is prior.'\\
\z
\z

\ea \label{ex:4}
\ea C’était aux Indes.\\`\textit{This} was in India.'\\
\ex C’était aux Andes.\\`\textit{This} was in the Andes.'\\
\z
\z

\ea \label{ex:5}
\ea Quel beau teint! \\ `What a nice complexion!'\\
\ex Quel beau temps! \\ `What a nice weather!'\\
\z
\z


While such observations have been made in earlier studies, Fónagy insisted on their increasing frequency, coupled with the tendency of the open nasal /ɑ̃/ to also increasingly sound like the back nasal /ɔ̃/. His tentative conclusion based on his own impressionistic observations indicated ongoing mergers (\textit{neutralisation}) between these vowel pairs, particularly in younger speakers’ speech:
\begin{quote}
    L’expérience quotidienne indique que les phénomènes de neutralisation sont beaucoup plus fréquents dans la parole des jeunes Parisiens et Parisiennes que dans les groupes d'\^age de 50-60 ou 60-70 ans \citep[][232]{Fónagy1989}. \\‘Everyday experience indicates that the neutralization phenomena is much more frequent in the speech of young Parisians than in age groups of 50--60 and 60--70.’
\end{quote}
Although Fónagy argues that the perceptual overlap of these vowels rarely, if ever, endangers “the correct transmission of the message” \citep[][228]{Fónagy1989} thanks to the context that can disambiguate between competing lexical meanings, he does not even evoke the interpretation of ``chain shift'' as an alternative explanation to vowel merger. And yet, the idea of a ``counterclockwise push shift'' cannot be excluded in light of experimental evidence \citep[][]{Malécot1976}, showing that, following World War II, the rounded front nasal vowel /œ̃/ merged with the front nasal vowel /ɛ̃/, which could have set up a chain shift ``pushing'' /ɛ̃/ to lower and centralize, i.e., impede on the vowel space of /ɑ̃/ and, in order to preserve its distinctiveness, causing /ɑ̃/ to sound more like /õ/ in some contexts.

Exceptions to these patterns, however, are numerous in regional varieties of French in France and Quebec. \citegen{Gendron1966} classic description of the front nasal vowel /ɛ̃/ as particularly fronted and close to /ẽ/ in Quebec, and \citegen[81]{Walker1984} observations that “the phonetic realizations of the four nasalized vowels are significantly distinct from those of SF (standard French)” resonated well with historians. Posner, for instance, noted that “[lowering] has not operated in many varieties of French, especially in northern France and Canada, where the tendency \ldots is to front /ɑ̃/ to /ã/ and /ɛ̃/  to/ ẽ/” (\citealt[235]{Posner1997}).

These speculations led to some tangible hypotheses about the production of these differences in vowel quality, which will be explored in the next section.

\section{Production}
Contrary to received wisdom that nasality arises solely from the lowering of the velum, the variability of nasal vowels attested in the historical record comes from the acoustic effects of three articulatory sources:
\begin{itemize}
    \item \textit{velo-pharyngeal coupling} due to “velic” constriction when resonances from the oral and nasal cavities are combined as a result of the lowering of the velum,
    \item \textit{lingual} articulations, i.e., specific movements (fronting, raising, retracting) of the tongue body, and
    \item \textit{labial} articulations, i.e., specific movements (rounding, protrusion, etc.) of the lips.
\end{itemize}

These gestures are combined in idiosyncratic, speaker-dependent strategies that can be equally good ways of pronouncing target-like nasal vowels. When combined successfully to achieve this effect, the goal of articulatory strategies is to enhance the acoustic effects of nasalization, i.e., maintain and even reinforce the clarity and distinctiveness of each nasal phoneme.

\begin{figure}
    \includegraphics[scale=.99]{figures/fagyal_fig3.png}
    \caption{Average formant frequencies of NMF speakers participating in \citegen{Carignan2013} experiment \citep[][1209]{NicholasCarignan2019}}
    \label{fig:3}
\end{figure}

In a combined acoustic and articulatory study involving Northern Metropolitan French (NMF) and Quebec French (QF) speakers, \citet{Carignan2013} looked at velic, lingual, and labial articulatory patterns of three of the four nasal vowels in comparison with their oral counterparts. He showed that in the majority of his NMF speakers’ speech, the front nasal vowel /ɛ̃/, such as in \textit{pain} ‘bread’, had a relatively high first formant (F1) and low second formant (F2), which came to overlap with the acoustic space occupied by oral /a/. He therefore suggested alternative phonetic transcriptions for the three nasal vowels (\figref{fig:3}).

\begin{figure}
    \includegraphics[scale=.99]{figures/fagyal_fig4.png}
    \caption{Schematic representation of NMF participants’ idealized nasal vowel space based on \citet{Carignan2013} }
    \label{fig:4}
\end{figure}

Furthermore, one would expect /ɑ̃/, like in \textit{paon} ‘peacock’ to occupy an acoustic space near its oral counterpart /a/, but nasal /ɑ̃/ was realized instead with a lower F1 and F2, bringing nasal vowel /ɑ̃/ near the acoustic space occupied by oral /o/ (\figref{fig:4}). For the majority of the NMF speakers, the back nasal vowel /õ/, such as in the word \textit{pont} ‘bridge’, was also realized with a relatively low F1 and F2 compared to /o/, which brought /õ/ near the acoustic space occupied by the high back vowel /u/. These results showed acoustic and articulatory evidence that nasal vowels in this variety continue their historical path of lowering and are possibly participating in a chain-shift.

Results were more tentative on the acoustic effects of lingual articulations in the Quebec French (QF) variety, featuring speakers from the rural Saguenay region of Quebec (\figref{fig:5}). They showed that the back nasal vowel /õ/ is lowered toward the acoustic space of the open nasal /ɑ̃/, but that /ɑ̃/ was not systematically fronted and raised towards the front nasal, as a clockwise chain shift would predict (\figref{fig:6}). The dynamics of the realizations of the front nasal vowel /ɛ̃/ also went in the expected direction of a clockwise shift, but in terms of lingual articulations, the nucleus of the vowel was not different from the front oral vowel /ɛ/. Thus, a clockwise chain shift in the QF dialect could not be robustly confirmed on acoustic and articulatory grounds. These findings do lend support to those “considerable differences” between the two vowel systems, signaled by Walker (see above).

\begin{figure}

    \includegraphics[scale=.99]{figures/fagyal_fig5.png}
    \caption{Average formant frequencies of QF speakers participating in \citegen{Carignan2013} experiment \citep[][1209]{NicholasCarignan2019} }
    \label{fig:5}
\end{figure}

\begin{figure}

    \includegraphics[scale=.99]{figures/fagyal_fig6.png}
    \caption{Schematic representation of QF participants’ idealized nasal vowel space based on Carignan     (\citeyear{Carignan2013})}
    \label{fig:6}
\end{figure}

These variations in nasal vowel quality are consistent with, what is typically called, formant-frequency-related acoustic effects of nasalization on the vowel space, which means that acoustic output from nasalization is attributed to the movements of the tongue and the lips. Thus, diachronically, it is likely that as subsequent generations of native speakers perceived the effects of nasalization in the newly emerging standard variety (now NMF), they reanalyzed its sources as lingual and labial articulations and did their best to imitate what they heard. In this way, one might speculate, the lowering and centralizing acoustic effect of nasalization -- which is indeed one of the universal properties of velo-pharyngeal coupling -- became phonologized in the language over time.

Next, we will turn to the production-perception interface in two varieties of French to understand how dialect-specific acoustic variations can inform same- and cross-dialectal perception of nasal vowels. The results can help us make better predictions of listeners’ interpretations of simultaneous acoustic cues of different articulatory origins and, ultimately, their accommodations to such patterns in their own speech.

\section{Perception}
In a series of studies, \citet{Nicholas2018}, \citet{NicholasCarignan2019} and \citet{nicholas_role_review}) tackled the production-perception interface of nasal vowel realizations in two varieties of French. Some of their research questions were:
\begin{itemize}
    \item \textbf{Q1:} When presented to native listeners of French, would phonetic realizations of front (/ɛ̃/) and open (/ɑ̃/) nasals in Northern Metropolitan French and of open (/ɑ̃/) and back /ɔ̃/ nasals in Quebec French be identified with less accuracy, possibly because they impede on each other’s vowel spaces in their respective counterclockwise push (NMF) and clockwise pull (QF) shifts?
\end{itemize}

Our expected answer was ``yes, they would'' since these vowels are at the onset of each hypothesized nasal vowel shifts, respectively, and therefore should show greater formant overlap.

\begin{itemize}
    \item \textbf{Q2:} Would less familiarity with each dialect result in greater difficulty distinguishing nasal vowel contrasts cross-dialectally? And does expertise with the sounds of the language -- teaching them and researching them -- help distinguish especially difficult nasal vowel contrasts?
\end{itemize}

Again, our expected answers were ``yes'' since nasal vowels recorded in a less widely diffused rural variety of French in Quebec could be expected to be more challenging to identify cross-dialectally. Also, expertise -- defined as frequent and in-depth exposure to the sounds of the language -- could facilitate the accuracy of perception.

\begin{itemize}
    \item \textbf{Q3:} Does greater vowel duration help the accuracy of nasal vowel identifications in Quebec French where progressive increase in the degree of nasalization has been attested in previous studies?
\end{itemize}

Our hypothesis was that increased vowel duration in Quebec French should improve the accuracy of nasal vowel perception in both dialects.

\subsection{Experiment}
We used stimuli from \citegen{Carignan2013} production study targeting the two dialects: a lowered and retracted vowel space showing the three target words in Northern Metropolitan French (\figref{fig:3} and \ref{fig:4}, above), and the same target words in a raised and fronted vowel space in Quebec French (\figref{fig:5} and \ref{fig:6}, above).

Seventy listeners took part in a computerized forced-choice gating experiment constructed in E-Prime 2.0 ran by \citet{Nicholas2018}. The 19 women and 12 men who came from the Saguenay-Lac-Saint-Jean dialect area in Quebec and the 20 women and 19 men from the greater Paris area in France were divided into four age and occupational groups based on age categories taken from the 2017 Canadian Census.

The 72 target words were selected from \citegen{Carignan2013} corpus. Monosyllabic real words with a word-initial /p/ and /t/ were followed by one of the three nasal vowels. The six distractor words contained the oral counterparts of each nasal vowel. All words were nouns in order to control for morpho-syntactic category. There were three different repetitions per word per speaker in two separate sessions with a break in-between.

The target words were presented using the gating paradigm. Listeners saw the question “Which word do you hear?” written on the computer screen in French and were asked to choose the number key on the keyboard that corresponded to their answer: 1 or 2. At the first gate, they heard the first half of the vowel, at the second gate, they heard the full vowel, and at the third gate, they heard the full word that included the onset consonant. Listeners could only hear each gate once. They had to click the space bar to proceed to the next gate or to proceed to the next word pair showed on a separate screen and they were not told until after the experiment that they would be hearing two different dialects.

The order of the words on the screen and of the appearance of target words and distractors were randomized and counterbalanced across six experiment lists. We used the lme4 package in R with \textit{participant} as random intercept and \textit{age}, \textit{gender}, \textit{dialect}, \textit{target-competitor vowel pair}, and the interaction between \textit{dialect} and \textit{vowel pair} as fixed effects. To determine the significance of fixed effects, we used the mixed function from the \textit{afex} package.

\subsection{Results}
The effects of \textit{dialect}, \textit{target-competitor vowel pair}, and the interaction of \textit{dialect} and \textit{target-competitor vowel pair} on accuracy of vowel contrasts were significant. The effect of age on accuracy varied depending on the dialect and was not as robust for native dialect identification in the NMF as in the QF group. Gender was not significant.

\begin{figure}

    \includegraphics[scale=.99]{figures/fagyal_fig7.png}
    \caption{Accuracy in the perceptual identifications of nasal vowel contrasts: NMF listeners listening to NMF nasal vowel contrasts. Red arrows indicate incorrect identifications; green arrows indicate correct identifications \citep[based on][]{NicholasCarignan2019}}
    \label{fig:7}
\end{figure}

As hypothesized, in NMF stimuli heard by NMF listeners, /ɛ̃/ was often mistaken for /ɑ̃/, but the opposite was not true: /ɑ̃/ was never mistaken for /ɛ̃/ (\figref{fig:7}). Also, /ɑ̃/ was frequently misidentified as /ɔ̃/, while /ɔ̃/ was always accurately identified in contrast with /ɑ̃/.  Cross-dialectally (\figref{fig:8}), there was confusion when /ɑ̃/ was contrasted with /ɛ̃/and when /ɔ̃/ was contrasted with /ɑ̃/, while identifications in the opposite directions were much more accurate: /ɛ̃/ was identified nearly categorically when contrasted with /ɑ̃/ and /ɑ̃/was distinguished from /ɔ̃/.

\begin{figure}

    \includegraphics[scale=.99]{figures/fagyal_fig8.png}
    \caption{Accuracy in the perceptual identifications of nasal vowel contrasts: NMF listeners listening to QF nasal vowel contrasts. Red arrows indicate faulty identifications; green arrows indicate correct identifications \citep[based on][]{NicholasCarignan2019}}
    \label{fig:8}
\end{figure}

When QF listeners heard target words in their own dialect (\figref{fig:9}), they found /ɑ̃/ difficult to distinguish from /ɛ̃/ but the opposite was not true because /ɛ̃/was identified nearly categorically in contrast with
/ɑ̃/.  Similarly, supporting the interpretation of a possible clockwise shift, /ɔ̃/ was often misheard for /ɛ̃/,
but not the other way around, as /ɛ̃/ was heard as nearly categorically distinct from /ɔ̃/.

\begin{figure}

    \includegraphics[scale=.99]{figures/fagyal_fig9.png}
    \caption{Accuracy in the perceptual QF of nasal vowel contrasts: QF listeners listening to QF nasal vowel contrasts. Red arrows indicate faulty identifications; green arrows indicate correct identifications      \citep[based on][]{NicholasCarignan2019}}
    \label{fig:9}
\end{figure}

\begin{figure}

    \includegraphics[scale=.99]{figures/fagyal_fig10.png}
    \caption{Accuracy in the perceptual identifications of nasal vowel contrasts: QF listeners listening to NMF nasal vowel contrasts. Red arrows indicate faulty identifications; green arrows indicate correct identifications \citep[based on][]{NicholasCarignan2019}}
    \label{fig:10}
\end{figure}

When QF listeners heard NMF stimuli (\figref{fig:10}), vowels adjacent to each other along the peripheral tract of the vowel space, yet again, provoked confusion when contrasted with other vowels: /ɛ̃/ was often confused with /ɑ̃/ and /ɑ̃/ was often taken for /ɔ̃/. However, identification was nearly categorical when /ɑ̃/ was contrasted with /ɛ̃/, and /ɔ̃/ was contrasted with /ɑ̃/. As expected, there was more confusion cross-dialectally than within dialects, especially when NMF listeners listened to QF which, as mentioned above, might be due to unfamiliarity with the rural variety of QF used in the experiment. Also, as predicted, NMF listeners did benefit from longer stimuli in QF and showed higher accuracy. However, increased vowel durations proved less useful for QF listeners listening to NMF throughout the three gates.

In light of a forthcoming study where in-depth knowledge and professional work with the language was also considered \citep{nicholas_role_review}, we can also confirm that expertise with the language matters: NMF ``Experts'' listening to NMF input showed significantly greater accuracy in identifying nasal vowel contrasts than their ``non-Expert'' counterparts (\figref{fig:11a}) when perceiving contrasts between the front /ɛ̃/ and the open /ɑ̃/ nasals and the open /ɑ̃/ and back nasals /ɔ̃/ that show overlap in the acoustic space due to the NMF counterclockwise shift. Notice that the listeners’ difficulties, again, did not extend to any other contrasts; main difficulties in perception were limited to contrasts with the greatest variability, possibly due to ongoing change.
The same held cross-dialectally. Although both ``Expert'' and ``non-Expert'' QF participants performed below chance on stimuli from the NMF dialect (\figref{fig:11a}), ``Experts'' still performed better than ``non-Experts''. The same was true when ``Expert'' vs. ``non-Expert'' QF participants listened to QF input (\figref{fig:11b}): ‘Experts’ performed slightly better than ``non-Experts'' in the most difficult contrasts. When it comes to QF input heard by NMF ``Expert'' and ``non-Expert'' listeners, just like their Canadian counterparts hearing NMF stimuli, the ``non-Experts'' performed below 50\% accuracy. This means that they were less than 50\% sure what nasal vowel they heard in two of the most difficult vowel contrasts that are especially variable, possibly due to ongoing change.

Taken together, these results show that dialect-specific acoustic variations are challenging to perceive for all, but especially for non-native listeners of the dialect and with respect to the most variable stimuli. Prolonged exposure to the dialect, however, makes perception more reliable, even in the most variable contexts. These patterns can be predicted to shape listeners’ own interpretations and, possibly, replication of these patterns in their own speech.
\begin{figure}
  \begin{subfigure}{\textwidth}
    \includegraphics[width=\textwidth]{figures/fagyal_fig11a.png}
    %\caption{Perceptual accuracy among NMF and QF listeners listening to NMF input (top) and the same listeners listening QF input (bottom). Figure reproduced with permission from Nicholas and Fagyal (\citeyear{nicholas_role_review})}
    \caption{}
    \label{fig:11a}
    \end{subfigure}
    \begin{subfigure}{\textwidth}
    \includegraphics[width=\textwidth]{figures/fagyal_fig11b.png}
    \caption{}
    \label{fig:11b}
    \end{subfigure}
    \caption{Perceptual accuracy among NMF and QF listeners listening to NMF input (top) and the same listeners listening to QF input (bottom). Figure reproduced with permission from \citet{nicholas_role_review}}
    \label{fig:11}
\end{figure}



\section{Modeling}

How to integrate the last piece of the puzzle -- computer simulations of the actual accommodation processes -- will have to be determined in the next few decades in computational sociolinguistics. What seems clear is that traditional experimenting and testing must be integrated into broader, multi-pronged approaches to social variation and change in historical times.

Agent-based computational models of vowel shifts have been proposed in the sociolinguistic literature since the early 2000s, with the intention of simulating the collective social dynamics between populations that come into contact with each other. Applied in particular to the Northern Cities Vowel Shift, \citegen{SwarupMcCarthy2012} model, for instance, incorporated several empirically-derived rules of vowel change of the Northern Cities Shift together with psychological processes, such as representational momentum, accounting for the ways in which exemplars of various vowels are copied -- imitated -- by the simulated social agents. Sociolinguists and applied computer scientists at the University of Illinois have also investigated the role of centrally and periphery connected individuals in the diffusion of lexical innovations \citep{Fagyal2010}.

Such models could be used to simulate chain shifts in nasal vowel systems in large groups of agents that come in contact in specific ways over the course of a simulated history of events. Quite a lot is known about the long history of dialectal separation between French spoken in France and Quebec. Thanks to church records, demographic data of migrants and settlements are also abundant, which could help speculate on the social dynamics of separation and convergence between certain segments of French-speaking populations at a given time in history.

What is needed is information about the parameter space, the production and perception of the segments at play (e.g., nasal vowels) and ideas of how to model and interpret their interactions with social factors. Together, they could allow the simulation of population-level norms negotiated and adopted in the following simulation carried out for the study of \citet{Fagyal2010}: \url{https://nssac.bii.virginia.edu/~swarup/animations/degree_biased_voter_model.mov.} (The flickering red and blue dots indicate ongoing negotiations (one agent copying another one’s usage of a variable), while uniform red and blue dots signal the stabilization of a norm (the adoption of an innovation) over another.

This integrative approach is deductive in nature. It starts out with broad generalizations of possible scenarios of accommodation at the population level, whose validity is tested using the computational modeling of social dynamics underlying the multiple instances of accommodation in production and perception between individuals of a large and evolving population. If successful, they could open a new chapter in studies of vowel evolution in Romance, as well.

\section*{Acknowledgements}

I would like to thank  Christopher Carignan,  Jessica Nicholas, and  Samarth Swa\-rup for our long-term collaboration and co-authorships. Many thanks to  Joseph Roy,  Marissa Barlaz, and  Gyula Zsombok for their work as data management and analysis consultants at the University of Illinois at Urbana-Champaign. My gratitude goes to colleagues at the Illinois Phonetics and Phonology Laboratory and the Phonetics Laboratories of the \textit{Université de Québec à Chicoutimi} and \textit{l’Université de Paris III}. I am also thankful for the National Science Foundation Grant 1121780 awarded to Ryan Shosted (PI), C. Carignan, Z. Fagyal (co-PI-s) between 2011 and 2013 that helped advance research on nasal vowels in French significantly.

\printbibliography[heading=subbibliography,notkeyword=this]

\end{document}
