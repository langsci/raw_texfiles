\documentclass[output=paper,colorlinks,citecolor=brown,draftmode]{langscibook}
\ChapterDOI{10.5281/zenodo.7525112}

\author{Hilary Walton\orcid{}\affiliation{University of Toronto}}

\title{Does social identity play a role in the L2 acquisition of French intonation? Preliminary data from Canadian French-as-a-second-language classroom learners}

\abstract{This exploratory study seeks to examine the role of social identity in the acquisition of French intonation, specifically in the realization of the final pitch accent and overall shape of the pitch contour of non-final accentual phrases, among Canadian L2 learners of French. The primary objectives are to investigate the potential role of a social group-based accommodation effect in the acquisition of non-target-like speech and to identify any unique features in the French intonation contours of French immersion versus core French speakers. To such ends, two groups of six Anglophone learners of French having graduated from either a French immersion or a core French program completed a social identity questionnaire and a delayed sentence repetition task. The questionnaire results suggest that French immersion speakers have greater ingroup identification with their French program than their core French counterparts, particularly as concerns their emotional and psychological attachment to their program and peers. Due to the small sample size, differences in the French intonation contours of these learner groups were not significant and require further investigation. The results of this study expand our understanding of the role of sociological factors in the present instance social identity as a potential difference between L2 learner groups, and it is the first study to suggest a potential interaction between social identity and the production of linguistic features in an L2 context.}

%move the following commands to the "local..." files of the master project when integrating this chapter
% \usepackage{tabularx}
% \usepackage{langsci-basic}
% \usepackage{langsci-optional}
% \usepackage{langsci-gb4e}
% \bibliography{localbibliography}
% \newcommand{\orcid}[1]{}
% \pagenumbering{arabic}
% \setcounter{page}{230}

\IfFileExists{../localcommands.tex}{
  \addbibresource{../localbibliography.bib}
  % add all extra packages you need to load to this file

\usepackage{tabularx,multicol}
\usepackage{url}
\urlstyle{same}

\usepackage{listings}
\lstset{basicstyle=\ttfamily,tabsize=2,breaklines=true}

\usepackage{langsci-basic}
\usepackage{langsci-optional}
\usepackage{langsci-lgr}
\usepackage{langsci-osl}
% \usepackage{./langsci/styles/langsci-lgr}
% \usepackage{./langsci/styles/langsci-osl}
% \usepackage{langsci-gb4e}

\usepackage{tikz}
\usetikzlibrary{patterns,calc}
\pgfdeclarepatternformonly{south east lines}{\pgfqpoint{-0pt}{-0pt}}{\pgfqpoint{3pt}{3pt}}{\pgfqpoint{3pt}{3pt}}{
    \pgfsetlinewidth{0.6pt}
    \pgfpathmoveto{\pgfqpoint{0pt}{3pt}}
    \pgfpathlineto{\pgfqpoint{3pt}{0pt}}
    \pgfpathmoveto{\pgfqpoint{.2pt}{-.2pt}}
    \pgfpathlineto{\pgfqpoint{-.2pt}{.2pt}}
    \pgfpathmoveto{\pgfqpoint{3.2pt}{2.8pt}}
    \pgfpathlineto{\pgfqpoint{2.8pt}{3.2pt}}
    \pgfusepath{stroke}}
    
\usepackage{stmaryrd}
\usepackage{wasysym}
\usepackage{multirow}
\usepackage{caption}
\usepackage{subcaption}
\usepackage{mathrsfs}
\usepackage{qtree}

\usepackage{linguex}


  %pminos do not split footnotes
% \interfootnotelinepenalty=10000 %Footnote in Laporte chapters has to be split SN


%\DeclareIndexNameFormat{default}{%
%\nameparts{#1}%
%\usebibmacro{index:name}%
%{\index[names]}%
%{\namepartfamily}%
%{\namepartgiveni}%
% {}% L1
% {}% L2
%{\namepartprefix}% generates spurious space L3
%{\namepartsuffix}% generates spurious space L4
%}

%  {\DeclareIndexNameFormat{default}{%
%     \usebibmacro{index:name}{\index[names]}{#1}{#3}{#5}{#7}}}

%\DeclareIndexNameFormat{default}{%
%  \usebibmacro{index:name}{\sindex[nom]}{#1}{#3}{#5}{#7}}

%\DeclareIndexNameFormat{default}{%
%  \usebibmacro{index:name}{\sindex[person]}{#1}{#3}{#5}{#7}}
%\DeclareIndexNameFormat{default}{%
%\nameparts{#1} \usebibmacro{index:name}{\sindex[person]]}{\namepartfamily}{‌​\namepartgiven}{\nam‌​epartprefix}{\namepa‌​rtsuffix}}

%\newcommand{\smiley}{:)}

%\renewbibmacro*{index:name}[5]{%
%\usebibmacro{index:entry}{#1}%
%{\iffieldundef{usera}{}{\thefield{usera}\actualoperator}\mkbibindexname{#2}{#3}{#4}{#5}}}

% \newcommand{\noop}[1]{}

%remove for final
%\overfullrule=1mm

\newcommand{\tobi}[2]}}
\renewcommand{\S}[1]{\tobi{#1}{\textsc{*}}}

% this volume references
% puts: [this volume]
% already defined: \citetv
%\newcommand{\citepv}[1]{(\citeauthor{#1} \citeyear*{#1} [this volume])}
\newcommand{\citealtv}[1]{\citeauthor{#1} \citeyear*{#1} [this volume]}

%parentheses around example number
\newcommand{\pref}[1]{(\ref{#1})}

% in-text examples

\newcommand{\lnex}[1]{\textit{#1}} %target lang word
\newcommand{\lnlit}[1]{(lit.: `#1')} %literal reading
\newcommand{\lnlat}[1]{(#1)} % latinization
\newcommand{\lntrans}[1]{`#1'} %translation
\newcommand{\lnexl}[2]%
{\lnex{#1}{} \lnlat{#2}} % ex with latinization
\newcommand{\lnexlat}[3]{\lnex{#1}{} \lnlat{#2}{} \lntrans{#3}} % ex with latinization and tranl.

%ch01
\newcommand{\co}[1]{\mbox{\textbf{#1}}}

%ch09

\newcommand{\cyrbulg}[1]{\begin{otherlanguage*}{bulgarian}#1\end{otherlanguage*}}


%ch10
\newcommand{\nlp}{{\small NLP}}
\newcommand{\mwe}{{\small MWE}}
\newcommand{\rae}{{\small RAE}}
\newcommand{\lvc}{{\small LVC}}
\newcommand{\pos}{{\small P}o{\small S}}
%\newcommand{\todo}[1]{ \textcolor{red}{#1} }

%\renewcommand{\labelenumi}{\theenumi}
%\ainamefmt{{vv}{ll}{, ff}{, jj}} % fullname

\newcommand{\biberror}[1]{{\color{red}#1}}

\newcommand{\osenovaitem}{--~}
  %% hyphenation points for line breaks
%% Normally, automatic hyphenation in LaTeX is very good
%% If a word is mis-hyphenated, add it to this file
%%
%% add information to TeX file before \begin{document} with:
%% %% hyphenation points for line breaks
%% Normally, automatic hyphenation in LaTeX is very good
%% If a word is mis-hyphenated, add it to this file
%%
%% add information to TeX file before \begin{document} with:
%% %% hyphenation points for line breaks
%% Normally, automatic hyphenation in LaTeX is very good
%% If a word is mis-hyphenated, add it to this file
%%
%% add information to TeX file before \begin{document} with:
%% \include{localhyphenation}
\hyphenation{
    Beck-man
    Ngu-yen
    back-chan-nel
    back-chan-nels
    mo-not-o-nous
    ste-reo-typ-i-cal
}

\hyphenation{
    Beck-man
    Ngu-yen
    back-chan-nel
    back-chan-nels
    mo-not-o-nous
    ste-reo-typ-i-cal
}

\hyphenation{
    Beck-man
    Ngu-yen
    back-chan-nel
    back-chan-nels
    mo-not-o-nous
    ste-reo-typ-i-cal
}

  % \togglepaper[3]%%chapternumber
}{}



\shorttitlerunninghead{Social identity  in   L2 acquisition of French intonation}
\begin{document}
\shorttitlerunninghead{Social identity  in   L2 acquisition of French intonation}
\maketitle

\section{Introduction}
Variability in second language (L2) learning and ultimate attainment due to individual differences is an area of great interest to linguistic researchers and language instructors. Studies examine the role of learner characteristics (e.g., motivation, language aptitude, learning styles and strategies) and their interaction with contextual circumstances to explore individual engagement with the language learning process \citep{Dörnyei2005}. Among the potential individual differences contributing to inter-learner variability is social identity, defined as the strength of an individual’s cognitive and psychological attachment to a particular social ingroup \citep{Tajfel1978}. To date, the effects of social identity on linguistic behaviour have primarily been investigated in social psychology research, which has linked social identity with significant implications for group-level perceptions, attitudes and behaviours (e.g., \citealt{PerreaultBourhis1999, Reay:2010, SpearsEllemers:1999, TajfelFlament1971}). The present study explores whether the behavioural effects associated with social identity extend to speech when the social identity in question is derived from a particular language learning context. To this end, the L2 acquisition of French intonation contours by classroom learners having completed different high school French-as-a-second-language (FSL) programs is examined. Social identity was assessed using a questionnaire modified from \citet{LeachSpears2008} that served to evaluate speakers’ identification with their particular FSL program and their same-program peers, while an acoustic analysis of the French intonation contours of the learners was performed to determine any program-specific linguistic features. The production results were then analyzed in combination with the results of the social identity questionnaire to explore the interaction between the social identity derived from speakers’ FSL program and particular intonation features.


The primary objective of this paper is to introduce social identity as a new individual difference in L2 acquisition using data from first participants (n=12) in a larger study. This small sample size does limit potential findings; however, the data serves to illustrate the novel idea that individuals’ membership in a particular language learning education program may result in a specific program-based social identity that may, in turn, be linked to the L2 acquisition of speech.


The present paper is organized as follows. First, I present the construct of social identity and its potential implications for language learning (\sectref{sec:11:2.1}). Next, I characterize the language learning contexts of the populations of study, namely French immersion and core French programs, and summarize previous work targeting differences between the French speech of learners in these programs (\sectref{sec:11:2.2}). I then outline the methodology (\sectref{sec:11:3}) and present the results (\sectref{sec:11:4}). Lastly, I evaluate the hypotheses of study (\sectref{sec:11:5}) and conclude by positioning this work in the current L2 learning literature and discussing future avenues of research (\sectref{sec:11:6}).

\section{Background}
\subsection{Social identity as a predictor of linguistic behaviour}\label{sec:11:2.1}
As with other individual differences, social identity has potential implications for individuals’ language acquisition and ultimate attainment. Indeed, it may result in the acquisition of particular group-level speech patterns through a process of social group-based accommodation. In the context of the current study, social identity refers to the “part of an individual’s self-concept which derives from his [or her] knowledge of his [or her] membership [in] a social group (or groups) together with the value and emotional significance attached to that membership” \citep[63]{Tajfel1978}. A social group consists of two or more people who “identify themselves in the same way and have the same definition of who they are, what attributes they have, and how they relate to and differ from specific outgroups” \citep[251]{HoggHinkle2004}. According to Social Identity Theory (SIT; \citealt{Tajfel1978, TajfelTurner:1979}), a prominent framework in the field of social psychology, individuals self-construct any number of social identities based on common attributes with others who share a social category (e.g., players on a soccer team) which, in turn, results in satisfaction and positive self-esteem for the group and its members. As individuals amplify the positive characteristics of group members and group norms, they reduce uncertainties about themselves and their identities \citep{Hogg:2012}. This results in self-enhancement and positive self-esteem and may lead to feelings of superiority over non-members (i.e., positive distinctiveness; \citealt{Brewer1991, LeonardelliBrewer2010}). When social identities are salient, they can have implications for group-level phenomena, as they create group norms such as behaviours, beliefs, and attitudes \citep{Hogg:2018}. Such behaviours and attitudes are often established as a means of distinguishing members of one social group from those of other related social groups \citep{HoggHinkle2004}. \citet{ReicherHaslam:2010} explain that the prototypical behaviours of a social group are typically based on factors that shape the salience and expression of its particular identity and serve to differentiate a given group from opposing social groups. For example, in a high school with a rivalry between the football and basketball teams, players would form their identity based on the particular sport that they play. Although all players would (theoretically) have similar identities related to their particular school, age and athletic ability, it is the sport that would serve as the dimension of comparison between these two teams and influence the prototypical behaviours and attitudes of each group. Thus, players of each team would actively converge on attitudes and behaviour that would enhance the identity associated with their sport, while simultaneously distancing themselves from being associated with players of another team. Accordingly, it is expected that social groups based on language learning programs may be associated with specific linguistic traits. Indeed, such features may differentiate the speech of learners belonging to opposing language learning program-based social groups.


Language has long been established as an identity marker in all types of national, local, socioeconomic, educational, and occupational groupings (e.g., \citealt{Dunbar:2003, Labov1972patterns, Labov2001, MangeGeorget2009, Roberts2013, Sankoff.Fonollosa1997}). In social groupings, individuals may modify their speech due to a desire to feel a sense of belonging with the members of their social group and to maximize the distinction between themselves and members of the outgroup \citep{Ehala2018}. According to the Communication Accommodation Theory \citep{Giles1973, Giles:1991}, such modifications occur through a process of accommodation in which speakers adapt their speech to the communication style (e.g., accent, tempo, gestures, nonverbal communication) of their interlocuters either through convergence (i.e., becoming more similar) or divergence (i.e., becoming less similar). It is expected that the speech of individuals sharing a social identity would be subject to social group-based accommodation: a process in which speakers converge on the communication style of their fellow ingroup members and diverge from that of outgroup members. This would create a normative and unique within-group speech style that would, in turn, reinforce individuals’ group membership, promote the distinctiveness of their social identity \citep{Ehala2018} and strengthen their self-esteem \citep{HoggHinkle2004}.


Cases of linguistic accommodation have been reported across a wide range of speech features, gestures, and body language in L1 (first language) speech \citep{Ehala2018}, including phonetic (e.g., \citealt{Babel2012, Nielsen2011, ZellouNielsen2016}) and prosodic structures \citep{Giles:1991, LevitanHirschberg:2011}. Previous studies of social identity-based accommodation have found that the height of the low vowel /æ/ in Scottish English is significantly correlated with  speakers’ political party membership \citep{Hall-LewFriskney2017} and anti-/pro-institutional attitudes of engagement \citep{Lawson2011}, and that vocalic features of Northern Irish English are influenced by speakers’ religious identities \citep{McCafferty:1999}. Of the few studies of accommodation in classroom contexts, researchers found that all students enrolled in a French kindergarten class converged on similar usage of three nonstandard variants (word-final /ʁ/, the realization of /l/ in third person pronouns, and optional liaisons) of the most socially integrated students \citep{NardyBarbu2014}. Looking at the speech of an older group of students, \citet{Eckert1989, Eckert2008} found that American high school students converged on their use of standard and vernacular forms (e.g., “walking” versus “walkin”, marked pronunciation of word-final /t) with peers in their same social groupings (e.g., “jocks”, “burnouts”, “nerds”). As of yet, no study has investigated linguistic accommodation or the influence of social groupings on speech features in an L2 context. As such, the current study is the first to explore the role of social group-based identity as a potential factor in L2 speech.

\subsection{An overview of Canadian French-as-a-second-language programs}\label{sec:11:2.2}
In the province of Ontario, Canada, there are two primary FSL programs: French immersion and core French. These formal language learning programs are offered in English-majority communities to students of all linguistic backgrounds who rarely use French outside of class time \citep{Genesee1978}. With regard to the structure of these programs, core French students complete mandatory French language education courses between Grades 4 and 9, after which French classes are made optional until the end of Grade 12. In contrast, French immersion programs begin in either Grade 1 or Grade 2 of elementary school and students complete a minimum of 5,000 hours of French instruction in a range of academic subjects (e.g., history, geography, science) by the end of Grade 12 \citep{CanadianParentsforFrench2017}. Although the French courses differ between programs, students in French immersion and core French housed in the same school typically share the same teachers, who are often a mix of native speakers and advanced L2 learners \citep{NetelenbosRosen2016}. As is perhaps expected, due to students’ age of acquisition and linguistic environment, both Anglophone and heritage language students in these programs typically have English as their dominant language (e.g., \citealt{Birdsong2014}).


It is in these FSL contexts that students form social groups and explore their social identities. Although classroom identities are rarely researched in social psychology, it is widely accepted that school plays a critical role in the formation of a context-specific social identity as students, particularly secondary students \citep{McLeod2000}, create an image for themselves. \citet{Perry2002} explains that schools provide children and young adults with norms, practices, experiences and relationships that help them to form their identities. Furthermore, institutions also provide practical knowledge to individuals about their specific social position or category, which may in turn help them to develop a sense of self and identity \citep{MacKinnonHeise2010}. According to \citet{Jenkins:2014}, classroom social identities may be particularly salient due to the organized network of recognizable roles within school contexts. When applied to the current study, categories such as “student”, “French immersion”, and “core French” are established by the particular institutional and educational contexts. Students may, therefore, assign weight to their FSL categorization and internalize their membership in a specific program as a social identity.

\subsection{Acquisition of French intonation contours by Anglophone learners}\label{sec:11:2.3}
The intonation contours of French utterance-non-final accentual phrases (APs) in declarative sentences were selected as the target linguistic structure of study because they differ in phonetic realization from the learners’ L1 English and have proven to be problematic for Anglophone learners of French \citep{Colantoni:2014, Lepetit1989, Sunara2018}. As such, these contours are more likely to be characterized by non-target-like production and thus, differences between French immersion and core French learners are more likely to occur. Specifically, this study examines two phonetic parameters: (i) the type of the final pitch accent and (ii) the overall shape of the pitch contour.


According to \citet{JunFougeron:2000,JunFougeron2002}, APs are the basic prosodic category in French. They group together phonological words corresponding to syntactic phrases that make up the Intermediate Phrase and the Intonational Phrase (IP). French APs typically consist of 3 or 4 syllables but can contain up to a maximum of 8 syllables \citep{JunFougeron2002}. APs are marked by a maximum of one final pitch accent (i.e., a local intonation feature associated with a particular syllable) that is represented by a tone, either high (H), low (L) or a combination of these (H+L; \citealt{Jun2005}), followed by an asterisk “*” to indicate its association with a metrically strong syllable \citep{Sunara2018}. Current models of French prosody \citep{JunFougeron2002, Welby2006} characterize French APs by an obligatory rise associated with the final accented syllable (H*) which may be accompanied by a rise on the initial accented syllable, resulting in either LH* or LHiLH* tonal patterns for non-final APs. It is the final rise in fundamental frequency, along with vowel lengthening that serve as the primary characteristics of the French AP \citep{Sunara2018}. In contrast to French, English does not include APs in its prosodic hierarchy \citep{PierrehumbertHirschberg:1990} and instead assigns prominence at the word-level via the lexical stress system (e.g., \citealt{Hayes1995}) realized as higher pitch and longer vowel duration. English tends towards falling pitch accents that are associated with stressed syllables, resulting in H*L patterns within prosodic words \citep{Bullock2009}. These differences between languages result in opposing tendencies: French favors AP final prominence while English tends towards word-initial prominence \citep{Clopper2002}.


More generally, the differences in the intonation patterns of French and English can also be described with reference to the overall shape of the pitch contour. French declarative sentences are characterized by a pitch rise at the end of each non-final AP and a falling pitch on the final syllable of the final AP (e.g., \citealt{Dimperio:2012}). In contrast, both final and non-final APs in English are marked by a fall in pitch. Most important for the present study is that the end of non-final APs is signaled by a rise in pitch in French but by a fall in pitch in English.


Previous studies have found that, due to crosslinguistic influence (CLI), the English-French differences outlined above result in non-target-like realizations of French intonation among Anglophone learners. \citet{Lepetit1989} examined non-final APs in declarative sentences using a reading task and found that Anglophone learners had difficulty realizing target-like final pitch accents as well as producing target-like pitch contours for French utterances containing more than one AP. Moreover, \citet{Colantoni:2014} found that Anglophone learners of French had difficulty producing APs of the appropriate phonological size. Given that the L2 learners included in the present study were quite experienced with French, I do not expect all speakers to be influenced by the effects of CLI. Rather, I predict that CLI may influence the intonation patterns of some speakers, who will, as a result, carry over their L1 intonation patterns into their L2 to produce non-target-like realizations of French intonation patterns (e.g., falling pitch accents).

\subsection{Current study}\label{sec:11:2.4}
I propose here that the structural differences between French immersion and core French programs, including program length and degree of exposure to fellow L2 learners, promote different levels of ingroup identification for several reasons. First, because French immersion students complete the majority of their academic subjects in French, they are separated from their peers enrolled in the same school during class time throughout their education. Second, French immersion students remain in the same cohort for the entirety of their program, which creates close-knit social groups \citep{Lyster1987}. I do not expect similar groups to exist in core French programs, which typically have much larger cohorts and allow for greater mixing with students outside the program as compared to French immersion programs. Thus, French immersion students are expected to have higher levels of identification with their FSL program membership than core French students (Hypothesis 1).


In addition to potential differences in social identity, I expect to find distinct features in the L2 speech of French immersion and core French speakers. Anecdotally, the speech of French immersion learners has been referred to as an “immersion inter-language” \citep{Lyster1987} and is suggested to be unique due to production difficulties caused by L1 English influence (e.g., \citealt{Genesee1978, Lyster1987}). As concerns FSL learners’ speech in particular, \citet{Poljak:2015} found that native speakers of Canadian French were able to identify and distinguish between the spontaneous speech and sentence reading of French immersion and core French speakers in a program identification forced-choice task. These findings indicate that these learner groups have distinct (non-native) accents. Consequently, I hypothesize that there will be significant differences in the French speech of French immersion and core French speakers (Hypothesis 2), but given the lack of previous studies on these groups’ L2 intonation, I remain unable to make specific predictions regarding each group’s L2 production.


Finally, I seek to investigate the possibility that, if differences in the L2 intonation contours of French immersion and core French speakers exist, they will correspond to the social identity results (Hypothesis 3).


In summary, the present study tests the following three hypotheses. First, French immersion speakers will show greater levels of identification with their FSL program than core French speakers due to the close-knit relationships that they form with their peers in such intensive language programs (Hypothesis 1). Second, there will be measurable differences between French immersion and core French learners’ speech, specifically the phonetic realization of non-final APs (Hypothesis 2). Lastly, L2 intonation production differences between French immersion and core French speakers will be associated with the strength of their identification with their FSL program and their same-program peers (Hypothesis~3).

\section{Methodology}\label{sec:11:3}
The current study examines data from 12 female students at the University of Toronto having completed either a French immersion (\textit{mean age: 19.5 years}) or a core French (\textit{mean age: 20.3 years}) high school program. All participants attended secondary school in the Greater-Toronto-Area. It should be noted that because participants were recruited at the university level, I was not able to select students who had attended the same schools or had shared the same French teachers (see \sectref{sec:11:6} for potential implications). Due to the small sample size and participant availability, only female participants were included in this analysis to neutralize any effects of gender. All participants reported having English as their dominant language and none reported any significant time spent in a French community. Participants were tested individually in a quiet classroom at the University of Toronto while seated at a computer for the entirety of the study. As a part of a larger research project, all participants completed five experimental tasks: (i) sentence reading; (ii) passage reading; (iii) an interactive map activity; (iv) delayed sentence repetition; and (v) ingroup identity questionnaire. The data analyzed here are taken from the final two tasks.


The questionnaire was a slightly modified version of the ingroup identification questionnaire developed by \citet{LeachSpears2008} and was used to measure participants’ ingroup identification with their particular FSL program ingroup (for a full list of questions, see Appendix~\ref{app:11:a}). This questionnaire, which consists of 14 Likert scale questions assessed on a scale from 1 (\textit{Strongly disagree}) to 7 (\textit{Strongly agree}), evaluates social identity across two constructs based on the definition of social identity proposed by the SIT. The first construct, group-level self-investment, assesses individuals’ emotional and psychological connections with their ingroup (FSL program) and its members, while the second construct, group-level self-definition, reflects individuals’ perceived commonalities with their fellow ingroup members (same-program peers).


The delayed sentence repetition task (e.g., \citealt{TrofimovichBaker2006, TrofimovichBaker2007}) was used to elicit controlled speech containing natural intonation contours. For this task, 20 declarative French sentences containing three or four APs (e.g., [\textit{Aurélie}]AP [\textit{deviendra}]AP [\textit{biologiste}]AP. ‘Aurélie will become a biologist.’) were created based on a subset of sentences from \citet{MichelasD’Imperio2012}. Only sentences containing three APs were used for the present study (n=10; see Appendix~\ref{app:11:b}). All target sentences were recorded by a female native speaker of Canadian French and then, to assure that participants could not imitate the prosody of the recording, they were modified to make pitch and syllable duration parameters uniform. To create the auditory stimuli, the pitch of all target sentences was normalized to the native speaker’s mean pitch (180 Hz), and all unstressed CV and CVC syllables were normalized to the native speaker’s mean values (185 ms and 300 ms, respectively). In this task, participants were told that they would hear French sentences pronounced in a “robot voice” and that they should then repeat them aloud in their own voice as naturally as possible. Participants heard each sentence twice. The first recording of each sentence was accompanied by its written form to ensure that there were no misperceptions. Then, the orthography disappeared, and the recording was played a second time. Next, after a pause of 3 seconds, participants heard a beep and \textit{Répétez} ‘Repeat’ was displayed on the screen, prompting participants to repeat the sentence. Along with the normalized auditory stimuli, the 3 second pause served to prevent individuals from mimicking the auditory stimuli as they heard it and thus required participants to produce sentences using their phonological and phonetic grammars (e.g., \citealt{TrofimovichBaker2007}).


The intonation analysis consisted of 240 non-final APs (10 sentences × 2 non-final APs × 12 speakers). Of these productions, four were excluded due to difficulties repeating the target AP, resulting in a total of 236 APs for analysis. Using ToBI for French annotation \citep{DelaisRoussarie:2015} and PRAAT \citep{BoersmaWeenink2021}, a TextGrid containing tiers of annotation for 1) pitch contour; 2) orthography; 3) prosodic breaks (i.e., AP and IP boundaries); and 4) AP number within the utterance was created for each sentence. The author manually identified the final pitch accent of each AP as a H or a L tone and classified the overall shape of each pitch contour as either rising, falling or sustained pitch. Speakers’ production was classified as target-like if it included a H* final pitch accent and a rising overall pitch contour. The present analysis did not include any quantitative measurements, but they could be examined in a future study.

\section{Results}\label{sec:11:4}
In this section, I present the results of the ingroup identification questionnaire and the intonation analyses. As mentioned in the introduction, the data included here is taken from the initial stages of a larger project, and thus, the sample size is small. With only six participants per speaker group, this section focuses on absolute differences observed in the data; inferential statistics are not justified but will be provided as footnotes for interested readers.

\subsection{Hypothesis 1: Ingroup identification}
Hypothesis 1 predicted that French immersion students, who spend much more time with their same-program peers in an intensive French learning context, would report higher levels of ingroup identification with their FSL program than core French students, for whom there is greater diversity among their French-learning peers. \figref{Questionnaire.All} offers an in-depth view of the overall reported responses by presenting the median Likert scores of each speaker group for each question. Recall that the seven-point Likert scale ranged from “Strongly disagree” (1) to “Strongly agree” (7) with questions 1 to 10 and 11 to 14 targeting the self-investment and self-definition constructs, respectively.

\begin{figure}
    \includegraphics[width=\textwidth]{figures/walton_fig1.png}
    \caption{French immersion and core French speakers’ median Likert responses (1--7) for each questionnaire item (Questions 1 to 10: self-investment; Questions 11--14: self-definition)}
    \label{Questionnaire.All}
\end{figure}

The median Likert scores of French immersion participants are higher than those of core French participants for the majority (10) of the questions. For questions targeting the self-investment construct (1--10), French immersion participants reported levels of agreement that were greater than or equal to those of core French participants in all cases. This pattern does not hold true for the results of the self-definition construct (Questions 11--14), however. Here, French immersion participants responded with lower levels of agreement as compared to their responses to the self-investment questions, resulting in more comparable medians between groups for this construct. This was also, the only construct in which the core French participants reported higher median scores than those of the French immersion speakers for some (2) of the questions.


When analyzing the results of the self-investment construct in isolation, as shown in \figref{SI.Program}, a clear distinction between participant groups emerges. French immersion participants reported higher levels of agreement than core French participants with questions targeting group-level self-investment.\footnote{A Mann-Whitney U test determined that between-group differences were significant (\textit{p}$<$0.001).}

\begin{figure}
    \includegraphics[width=\textwidth]{figures/walton_fig2.png}
    \caption{French immersion and core French speakers’ reported level of agreement (\%) for questions targeting positive self-investment in FSL program membership}
    \label{SI.Program}
\end{figure}

The reported Likert responses for the self-definition questions are lower than those of the self-investment questions. As shown in \figref{SD.Program}, no participant selected “Strongly agree” for any of the questions targeting this construct.

\begin{figure}
    \includegraphics[width=\textwidth]{figures/walton_fig3.png}
    \caption{French immersion and core French speakers’ reported level of agreement (\%) for questions targeting positive self-definition in FSL program membership}
    \label{SD.Program}
\end{figure}

\noindent The figure shows that 25\% of core French participants reported “Agree” to such questions as compared to only 13\% of French immersion participants.\footnote{A Mann-Whitney U test determined that between-group differences were non-significant (\textit{p}$>$0.05).} However, 54\% of French immersion participants selected “Somewhat agree” for these same questions as compared to 33\% of core French participants. A clear pattern does not emerge from this subset of the data.

The results of the ingroup identification questionnaire show that French immersion speakers have higher levels of ingroup identification with their FSL program than their core French counterparts overall. This pattern is particularly clear for the results of the questionnaire’s self-investment construct, which measures individuals’ psychological attachment to their FSL ingroup and its members. Indeed, for this construct, French immersion participants reported median levels of agreement that were equal to or surpassed those of the core French participants with every questionnaire item. In contrast, the results of the self-definition construct, which measures individuals’ perceived commonalities with their same-program peers, do not present a clear distinction between the FSL groups.

\subsection{Hypothesis 2: French intonation contours}
Based on the work of \citet{Poljak:2015}, investigating the general accent of FSL learners, Hypothesis 2 proposed that there would be differences in the realization of the intonation contours of French declarative sentences between French immersion and core French speakers. In order to test this hypothesis, I examined the type of final pitch accent and the shape of the overall pitch contour of non-final APs.


\tabref{tab:1:pitch errors} presents the percentage of non-target-like pitch accents (L*) that were present in the data and organizes them by the AP’s position in the utterance.

\begin{table}
\caption{French immersion and core French final pitch accent errors (\%) by position of the AP in a sentence repetition task}
\label{tab:1:pitch errors}
 \begin{tabular}{l rrr}
  \lsptoprule
            & Initial & Medial  & Overall\\
  \midrule
  French Immersion  &   43\%  &    50\%  &    47\%\\
  Core French  &   47\% &   57\%  &    52\%\\
  \lspbottomrule
 \end{tabular}
\end{table}

\noindent As shown in \tabref{tab:1:pitch errors}, both French immersion and core French participants demonstrated great difficulty producing target-like pitch accents. In absolute terms, core French speakers were less accurate in their production of phrase-final pitch accents than French immersion speakers in both initial and medial APs with error rates of 47\% and 57\%, respectively. French immersion speakers produced non-target-like final pitch accents in 43\% and 50\% of the initial and medial APs, respectively.\footnote{A Chi-squared test revealed the between-group differences were non-significant (\textit{p}=0.51).}


I now turn to the analysis of the overall shape of the pitch contours. Three distinct pitch contours were observed: (i) target-like rising; (ii) non-target-like falling; and (iii) non-target-like sustained contours. \figref{Pitch Contour.Program} presents the percentage of each pitch contour that was observed for each FSL speaker group. Both speaker groups produced the target-like pitch contours in less than 50\% of instances (core French: 46\%; French immersion: 41\%). Both groups also showed relatively low proportions of the non-target-like falling contour (French immersion: 10\%; core French: 15\%) compared to their realizations of the non-target-like sustained contours (core French: 40\%; French immersion: 49\%).\footnote{A Chi-squared test revealed the between-group differences were non-significant(\textit{p}=0.26).}

\begin{figure}
    \includegraphics[width=\textwidth]{figures/walton_fig4.png}
    \caption{Percentage (\%) of pitch contour types for French immersion and core French speakers in a sentence repetition task}
    \label{Pitch Contour.Program}
\end{figure}

In sum, the results of the final pitch accent and pitch contour analyses show that French immersion speakers produced a small but non-significant (i) greater proportion of target-like final pitch accents than core French learners, (ii) fewer rising pitch contours than core French learners and (iii) a greater proportion of sustained pitch contours.

\subsection{Hypothesis 3: Combined ingroup identification and production results}

Hypothesis 3 predicted that if between-group production differences were observed (Hypothesis 2), then they would correspond to the results of the ingroup identification questionnaire (Hypothesis 1). Because no significant differences in the realization of non-final APs were found between French immersion and core French speakers, either in type of pitch accent or overall shape of intonation contours, this section combines the production of all speakers to investigate the relationship between the production and the questionnaire results. I first present the combined questionnaire and final pitch accent results, followed by the results of the questionnaire combined with the overall pitch contours for all participants.


\figref{SI.Pitch Accent} presents the median Likert responses for the productions of H* and L* final pitch accents in non-final APs for all participants.

\begin{figure}
    \includegraphics[scale=.6]{figures/walton_fig5.png}
    \caption{Median reported self-investment Likert responses (1-7) for H* and L* final pitch accents in a sentence repetition task for French immersion and core French participants }
    \label{SI.Pitch Accent}
\end{figure}

\noindent Speakers who produced the non-target-like L* final pitch accents reported a median score of 6 on the ingroup identification questionnaire’s self-investment construct, which is higher than the median score of 5.5 reported by speakers who produced target-like H* final pitch accents. The distributions between speakers who produced H* and L* pitch accents also differed, with greater variability in the identification scores of speakers who produced L* pitch accents.\footnote{A Mann-Whitney U test determined that between-group differences were significant (\textit{p}=0.002).}


\figref{SD.Pitch Accent} presents the median responses for the self-definition construct, for H* and L* final pitch accent productions for all speakers. The median response was 4.5 for all speakers, whether they produced a H* or a L* final pitch accent in their non-final APs.\footnote{A Mann-Whitney U test determined that between-group differences were non-significant (\textit{p}=0.65).} Accordingly, learners’ responses to the self-definition questions do not seem to interact with the production of final pitch accents.

\begin{figure}
    \includegraphics[scale=.6]{figures/walton_fig6.png}
    \caption{Median reported self-definition Likert responses (1-7) for H* and L* final pitch accents in a sentence repetition task for all participants}
    \label{SD.Pitch Accent}
\end{figure}

In sum, a comparison of the data depicted in \figref{SI.Pitch Accent} and \ref{SD.Pitch Accent} does not provide sufficient evidence to suggest that speakers’ responses to either construct of the ingroup identification questionnaire are connected to the L2 realization of pitch accents in French. Median scores between speakers having produced target-like and non-target-like pitch accents are minimal or negligible and do not allow for an extensive analysis.


I now examine the relationship between the questionnaire results and the overall shape of the pitch contours for all speakers. \figref{SI.Pitch Contour} displays the median self-investment scores for all participants based on their realization of pitch contours as either rising, falling or sustained, while \figref{SD.Pitch Contour} displays the same for the median self-definition scores.

\begin{figure}
    \includegraphics[scale=.6]{figures/walton_fig7.png}
    \caption{Median reported self-investment Likert responses (1--7) by overall pitch contour of non-final APs in a sentence repetition task for all participants}
    \label{SI.Pitch Contour}
\end{figure}

\begin{figure}
    \includegraphics[scale=.6]{figures/walton_fig8.png}
    \caption{Median reported self-definition Likert responses (1--7) by overall pitch contour of non-final APs in a sentence repetition task for all participants}
    \label{SD.Pitch Contour}
\end{figure}

As can be seen in \figref{SI.Pitch Contour}, participants who produced a non-target falling pitch contour reported a median Likert score of 5, while participants who produced a target-like rising contour reported a slightly higher median score of 5.5 and participants who produced sustained contours reported a median score of 6, the highest of the three. Participants who produced falling pitch contours reported a much wider range of responses to the self-identification questionnaire items as compared to participants who produced rising and sustained contours and who reported relatively high self-investment scores. This distribution suggests the presence of outliers, a larger sample size will help to account for individual differences in the data and allow for a more robust analysis (see \sectref{sec:11:5}).


Despite slightly lower median scores, these same patterns were present in the self-definition data (\figref{SD.Pitch Contour}): participants who produced non-target falling pitch contours had the lowest reported Likert scores and the most variability in reported responses.\footnote{Mann-Whitney U tests determined that between-group differences were statistically significant for both the self-investment (\textit{p}$<$0.001) and the self-definition (\textit{p}=0.01) constructs.}

\section{Discussion}\label{sec:11:5}

The present study evaluated three hypotheses using data from an ingroup identification questionnaire and a delayed sentence repetition task. Hypothesis 1 predicted that French immersion speakers would show greater levels of ingroup identification than core French speakers due to the close-knit relationships that they form with their same-program peers. As expected, the responses to the ingroup identification questionnaire differed between the learners in each program. French immersion participants reported higher levels of ingroup identification than their core French peers on the majority of the questionnaire items, suggesting that these speakers have a more prominent social identity associated with their FSL membership. This pattern was particularly clear for the results of the self-investment construct of the questionnaire which evaluates individuals’ emotional and psychological connection to their ingroup. This is perhaps not surprising, as the relationships with their same-program peers discussed above are more likely to encourage emotional and psychological connections with their program and their peers (self-investment) than their view of shared attributes with their peers (self-definition).


Hypothesis 2 predicted that there would be measurable differences between French immersion and core French speakers’ French intonation contours. Although both speaker groups yielded a high percentage of non-target-like realizations, the French immersion speakers were more accurate in their production of final pitch accents, while the core French speakers were more likely to produce a target-like rising pitch contour for non-final APs. However, the differences are not statistically significant. and due to the small sample size, speech differences between the speaker groups cannot be confirmed.


Lastly, Hypothesis 3 predicted that between-group speech differences would correspond to differences in ingroup identification scores between French immersion and core French learners. Because there were no verifiable between-program linguistic differences, the study combined the production of all participants to examine a potential interaction between the production and the questionnaire results. Absolute differences in participants’ self-investment scores varied in relation to their production of H* and L* final pitch accents and the overall shape of pitch contours. Lower identity scores were reported for speakers who produced non-final APs with a non-target-like falling intonation contour, while higher social identity scores were reported for speakers who produced a target-like H* final pitch accent. Despite such differences, the number of participants and the variability of the results do not allow the current study to describe a relationship between speech and ingroup identification.


A thoughtful consideration of the study presents other limitations, as well. First, as the participants were tested during their university studies, they had attended various schools and were taught by a variety of teachers. Furthermore, the current study did not include a measure of L2 proficiency, as such it is possible that any observed speech differences may stem from differential exposure to the target language. It is also possible that individual factors (e.g., motivation, language aptitude) or CLI may have influenced individual results. Such factors weaken potential generalizations that ingroup identification may influence the speech of L2 learners. An extension of the present study should include a larger sample size and more homogeneous participant groups. All participants should be enrolled in FSL programs in the same school and be taught by the same teachers at the time of testing to control for factors such as input and cohort that could potentially influence the speech of individual learners. This design would ascertain that those individuals belong to the same specific ingroup and would have a higher probability of capturing any linguistic features that may be unique to the particular cohort. Furthermore, participants should complete a validated measure of L2 proficiency to account for potential differences due to stage of acquisition. Future studies of production differences between these learner groups could examine a wide array of L2 phonetic and phonological features to increase the probability of identifying between-group speech differences.

\section{Conclusions}\label{sec:11:6}
While the findings of the present study are limited, it does present a significant contribution. Although previous studies have reported findings that social groups may promote linguistic accommodation (e.g., \citealt{Eckert1989, Eckert2008, Hall-LewFriskney2017, Lawson2011, McCafferty:1999, NardyBarbu2014}), this is the first study to investigate this question in the field of L2 speech learning. Its results provide insights into the L2 learning experience of French immersion and core French speakers using a validated measure of social identity \citep{LeachSpears2008}. It was found that L2 learners enrolled in different language learning programs reported measurable differences in their levels of identification with their particular FSL program, particularly with respect to their emotional connection with the program and their fellow ingroup members. This study also contributes to the current understanding of the previously suggested phonetico-phonological differences in the speech of FSL learners \citep{Poljak:2015}, which remain largely understudied.


This study offers preliminary insights into such topics, but due to its small sample size, further research is required to present a clear picture of the speech patterns of speakers in these programs and any potential link to individuals’ identification with their social groups. In addition to evaluating the social identity of different L2 learner groups and the interaction between ingroup identification and L2 linguistic structures, the causality between these factors should be explored to determine whether social identity, through social group-based accommodation, contributes, in part, to production differences in the speech of different learner groups. Furthermore, within-group variability in the development of normative speech patterns of a social group should be investigated to determine whether individuals who identify most strongly with their ingroup are more likely to reflect the distinct linguistic structures of their particular social group.


\newpage
\appendixsection{List of questions included in social identity questionnaire}\label{app:11:a}

Group-level self-investment
\begin{enumerate}
\item	I feel a bond with students in my French program.

\item	I feel solidarity with students in my French program.

\item	I feel committed to my French program.

\item	I am glad to be in my French program.

\item	I think that students in my French program have a lot to be proud of.

\item	It is enjoyable to be a student in my French program.

\item	Being a student in my French program feels good.

\item	I often think about the fact that I am a student in my French program.

\item	The fact that I am a student in my French program is an important part of my identity.

\item Being a student in my French program is an important part of how I see myself.
\end{enumerate}

\medskip

\noindent Group-level self-definition

\begin{enumerate}
\setcounter{enumi}{10}
\item	I have a lot in common with many students in my French program.

\item	I am similar to the average student in my French program.

\item	Students in my French program have a lot in common with each other.

\item	Students in my French program are very similar to each other.
\end{enumerate}

\appendixsection{List of 3-AP target sentences}\label{app:11:b}
\begin{enumerate}
\item  {[Florentin]AP [demandait]AP [le corrigé]AP.}

\item	{[Tes amis]AP [demandaient]AP [la permission]AP.}

\item	{[Aurélie]AP [deviendra]AP [biologiste]AP.}

\item	{[Tante Sylvie]AP [demandait]AP [le téléphone]AP.}

\item	{[Amélie]AP [demandait]AP [la vérité]AP.}

\item	{[Cette diva]AP [deviendra]AP [indépendante]AP.}

\item	{[Le contrat]AP [deviendra]AP [satisfaisant]AP.}

\item	{[Le débat]AP [deviendra]AP [intéressant]AP.}

\item	{[Sébastien]AP [adorait]AP [son professeur]AP.}

\item	{[Béatrice]AP [deviendra]AP [très fatiguée]AP.}
\end{enumerate}

\printbibliography[heading=subbibliography,notkeyword=this]

\end{document}
