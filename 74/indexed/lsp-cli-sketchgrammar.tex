\chapter{Grammar outline}
\label{sec:gramsketch}


\section{Introduction}
\label{sec:Introduction}

This chapter  provides a broad outline of the grammar and introduces those 
aspects needed to understand the formations of words and sentences found in the 
dictionary. Further, it acts as a preliminary grammar of the language, which is 
and will always be essential for future description and analysis since it sets 
forth claims to be confirmed, rejected, challenged,  or improved.   First, the 
common clause structure, the main elements of syntax and clause coordination and 
subordination are presented. Then, elements of the nominal domain  are 
introduced,  followed by the elements of the verbal domain. Finally, aspects of 
grammatical pragmatics and selected language usage phenomena are examined.  The work is 
descriptive and employs theory grounded in traditional grammar, but  influenced 
by  recent work in linguistic typology. When necessary, the relevant theoretical 
assumptions are introduced and the relevant literature provided. Recall that the 
full list of glossing tags is available on page \pageref{sec-ABB} and the 
glossing convention is discussed in Section \ref{sec:intro-outline}.

\section{Clause}
\label{sec:GRM-clause}

A clause is a  grammatical unit that can express a proposition. A clause which 
can stand as a complete utterance is an independent clause. When 
a grammatically correct clause cannot stand on its own, it is dependent on  a 
main clause.  Three sorts of speech act are presented in this section:  the statement,  the question, and 
the command. The former is by default encoded in a declarative clause (Section 
\ref{sec:GRM-decl-clause}), and the latter two are usually encoded in \isi{interrogative} 
clauses (Section \ref{sec:GRM-interr-clause}), imperative clauses  (Section 
\ref{sec:GRM-imper-clause}), and exclamative clauses (Section 
\ref{sec:GRM-excla-clause}) respectively.  Constructions are treated as 
clause-types; constructions are  formal and semantic frames which are 
conventionalized and display both compositional and non-compositional 
characteristics. In this section  the components of the common 
independent  clauses and constructions encountered are presented.  In 
Section \ref{GRM-clause-coord-subord},  clause coordination and subordination 
are introduced. Section \ref{sec:GRM-adverbial} covers the adjunct constituents
responsible for modifying a main predicate and the function of the \isi{postposition}.



\subsection{Declarative clause}
\label{sec:GRM-decl-clause}

%A declarative clause features an itnonational declinaison  
Statements may be expressed by a series of declarative clause types. The structure of most common clauses consists of  a simple predicate, one or two arguments and an optional adjunct. This structure is represented in (\ref{ex:GRM-clause-frame})
 
\ea\label{ex:GRM-clause-frame}
{\rm  {\sc s|a}  $+$ {\sc p} $\pm$ {\sc o} ($\pm$ {\sc ajc}) }
\z


\ea\label{ex:GRM-cl-fr-inst}

\ea\label{ex:GRM-cl-fr-inst-s-p}
 {\sc s}  $+$ {\sc p} 
\ex\label{ex:GRM-cl-fr-inst-s-p-o}
 {\sc a}  $+$ {\sc p} $+$ {\sc o}
\ex\label{ex:GRM-cl-fr-inst-s-p-adj}
 {\sc s}  $+$ {\sc p}  $+$ {\sc ajc} 
 \ex\label{ex:GRM-cl-fr-inst-s-p-o-adj}
 {\sc a}  $+$ {\sc p} $+$ {\sc o} $+$ {\sc ajc} 

\z
\z

The predicate ({\sc p})  is represented by a verbal syntactic constituent ({\it v}) whereas  the arguments ({\sc s, a, o}) are represented by nominal syntactic constituents   ({\it n}).  The \isi{adjunct} constituent  ({\sc ajc}) may consist of words or phrases referring to time, location, manner of action, etc.  (see Section \ref{sec:GRM-adjuncts} on adjunct types).  An argument may be seen as core or peripheral.  The core argument of an intransitive clause is realized in the subject position ({\sc s}), which precedes the predicate. 


 \ea\label{ex:GRM-core}
\begin{multicols}{2}

\ea\label{ex:GRM-core-S-A-O}{
\glll àfɪ́á díjōō.\\
 {\sc s}  {\sc p}\\
{\it n} {\it v}.{\sc foc}\\
\glt `Afia ate.'
}

\ex\label{ex:GRM-core-S-O}{
\glll àfɪ́á díjōō kɪ̀ŋkáŋ̀.\\
 {\sc s}  {\sc p} {\sc ajc}\\
{\it n} {\it v}.{\sc foc}  {\it qual}\\
\glt `Afia ate a lot.'
}


\ex\label{ex:GRM-core-A-O}{
\glll àfɪ́á dí sɪ̀ɪ̀máá rā.\\
 {\sc a}  {\sc p}  {\sc o} {}\\
{\it n} {\it v} {\it n} {\sc foc}\\
\glt `Afia ate food.'
}

\ex\label{ex:GRM-core-A-O1}{
\glll àfɪ́á dí sɪ̀ɪ̀máá  kɪ̀ŋkáŋ nà.\\
 {\sc a}  {\sc p}  {\sc o}  {\sc ajc}  {}\\
{\it n} {\it v} {\it n}  {\it qual} {\sc foc}\\
\glt `Afia ate food a lot.'
}

\z 
\end{multicols}
 \z


The core arguments of a transitive clause are realized
in the subject ({\sc a}) and object ({\sc o}), the former preceding and the
latter following the predicate in their canonical positions. These
characteristics are illustrated in 
(\ref{ex:GRM-core}).\footnote{Focus ({\sc foc}) may be 
integrated into the verb or coded in a \isi{focus} particle, among others.  Section 
\ref{sec:focus-forms} presents  the
various forms {\sc foc} can take.}



%  \end{minipage}
% \vspace*{15pt}



% Nominal
% and verbal syntactic constituents are discussed in
% Section \ref{sec:GRM-nom} and \ref{sec:GRM-verbals} respectively, whereas
% adjuncts are presented in section
% \ref{sec:GRM-adverbs}.

Grammatical relations are primarily determined by
constituent order. Thus, the subject and object functions are not
morphologically
marked,  except that the subject pronouns in {\sc s} and {\sc a} positions  can 
have  strong or  weak forms (see Section \ref{sec:GRM-personal-pronouns}). This
is extraneous to the marking of grammatical functions but pertinent to the
emphasis put on  an  event's participant. A peripheral argument  consists of a
constituent foreign to the core predication, that is, an argument which is not
part of the core participant(s) typically associated with a predicate.  As peripheral argument,  an  \isi{adjunct}  ({\sc ajc}) may be realized by a single 
word or a phrase. Reference to space, manner, and time are the typical  
denotations of peripheral arguments.  Adjuncts will be briefly discussed here;  details are offered in Sections \ref{sec:GRM-adverbial} and \ref{sec:GRM-adjuncts}.  
 
 Adjuncts are optional with respect to the main predication and can be added to both intransitive and 
transitive clauses, as shown in  (\ref{ex:vp26.12}), as well as (\ref{ex:GRM-core-S-O}) 
and (\ref{ex:GRM-core-A-O}) above (see Sections \ref{sec:SPA-blc}, 
\ref{sec:SPA-postp},  and \ref{sec:GRM-obl-phrase} for  discussions on the 
\isi{postposition}).

\ea

\ea\label{ex:vp26.12}{\rm Manner expression  in intransitive clause}\\
\gll ʊ̀ ɲʊ̃́ã́ làɣá nɪ̀.\\
   {\psg} drink {\ideo} {\postp}\\
\glt  `He drank quickly.' 
% 
\ex\label{ex:vp26.13.}{\rm  Manner expression in transitive clause}\\
\gll ʊ̀ ɲʊ̃́ã́ à nɪ́ɪ́  làɣálàɣá nɪ̀.\\
   {\psg} drink {\sc art} water  {\ideo} {\postp}\\
\glt  `He drank the water quickly.' 

\z 
 \z



A variation of the prototype  clause in (\ref{ex:GRM-clause-frame}) is a
clause containing an additional core argument.  \citet[116]{Dixo10b} calls  a
clause which contains an
additional core argument, that is,  an extended argument (i.e. {\sc e}), an
{\it extended} (intransitive or transitive) clause. The
difference between an adjunct and an additional core argument is not a clear-cut
one;   still,   the locative phrase in (\ref{ex:GRM-add-arg-e}) is treated as
  an additional core argument of the predicate {\sls bile} `put'. In Section
\ref{sec:GRM-obl-phrase}, an oblique phrase is defined as a clause constituent whose semantics is
characterized by an  affected or effected object, although realized in a
postpositional phrase. Thus, the extended argument {\sls  tìwìzéŋ nʊ̃̀ã̀  nɪ̄}   `by a main road' in (\ref{ex:GRM-add-arg-e}) should be treated as an oblique object. 

\newpage 

\ea\label{ex:GRM-add-arg-e}{{\sc a} $+$ {\sc p}  $+$  {\sc o} $+$   {\sc e}}\\
\glll ŋmɛ́ŋtɛ́l sìì à bìlè  ʊ̀  kùó  tìwìzéŋ nʊ̃̀ã̀  nɪ̄.\\
spider raise.up   {\conn} put {3.\sg.\poss}   farm road.large  {\reln}  {\postp}\\
{\sc a}  {\sc p} {} {}    {\sc o} {}    {\sc e} {}  {} \\

\glt  `Spider went to establish his farm by a main road.' [LB 003]

\z

A ditransitive clause consists of a transitive clause with an additional core argument.  In Chakali, the verb {\sls tɪɛ} `give', a predicate that conceptually implies both a Recipient (R)  and a Theme (T), forces its (right-)adjacent argument in object position to be interpreted as  beneficiary of the situation. The thing transferred (T) can never follow the verb if the beneficiary of the transfer (R) is realized. This is shown in (\ref{ex:GRM-arg-e-ditrans}).

\ea\label{ex:GRM-arg-e-ditrans}

 \ea\label{ex:GRM-arg-e-ditrans-ben-the-1}
\glll kàlá tɪ́ɛ́ àfɪ́á {à lɔ́ɔ́lɪ̀}.\\
{\sc a} {\sc p} {\sc o}$_{R}$ {\sc e}$_{T}$\\
K. give A.  {{\sc art} car}\\

\glt  `Kala gave Afia the car.' 

 \ex\label{ex:GRM-arg-e-ditrans-ben-the-2}
\glll  kàlá tɪ́ɛ́ ʊ̄  {à lɔ́ɔ́lɪ̀}.\\
{\sc a} {\sc p} {\sc o}$_{R}$ {\sc e}$_{T}$\\
K. give {\sc 3sg}  {{\sc art} car}\\
\glt  `Kala gave her  the car.' 

 \ex\label{ex:GRM-arg-e-ditrans-the-ben-1}
 * Kala tɪɛ a lɔɔlɪ Afia.
 \ex\label{ex:GRM-arg-e-ditrans-the-ben-2}
* Kala tɪɛ ʊ Afia.

\z 
 \z

The assumption is that the verb {\sls tɪɛ} `give'  is transitive and its
extended argument is always the transferred entity (i.e.
Theme) in a ditransitive clause. This is supported by the extensive use of the 
{\it manipulative serial verb construction} (see Section
\ref{sec:GRM-multi-verb-clause}), used as an alternative strategy,  in order to
express transfer of
possession  and information.



\ea\label{ex:GRM-m-svc-give}
\glll  kàlá kpá  {à lɔ́ɔ́rɪ̀ / ʊ̄} tɪ̀ɛ̀ áfɪ́á.\\
{\sc a} {\sc p} {\sc o}$_{T}$  {\sc p}  {\sc o}$_{R}$\\
K. take  {{\sc art} car / 3.\sg} give A.\\

\glt  `Kala gave  the car/it to Afia.' ({\it lit.} Kala take the car/it give
Afia.)
\z

The extended argument in sentence (\ref{ex:GRM-arg-e-ditrans-ben-the-1})  and
(\ref{ex:GRM-arg-e-ditrans-ben-the-2})  above  is the \is{Theme}Theme argument 
of the verb {\sls kpa} `take'   in the \is{serial verb construction}serial verb construction  in
(\ref{ex:GRM-m-svc-give}).   Ditransitive clauses are
very rare in the text corpus despite their grammaticality (see \ref{sec:intro-outline} for information on the text corpus). If both Recipient and Theme occur in one clause it is usually when the Recipient is pronominal.  Multi-verb clauses, which are discussed in Section \ref{sec:GRM-multi-verb-clause}, may offer better strategies for arranging arguments and predicates than ditransitive clauses as they do not overload a predication with new information. The following subsections present various clause types and
constructions which are based on the declarative clause structure introduced
above.  


\subsubsection{Identificational clause}
\label{sec:GRM-ident-cl}

An identificational clause can express generic and ordinary categorizations, or assert the identity  of two expressions. Generic categorization involves the classification of a subset to a set (e.g. Farmers are humans), whereas an ordinary categorization holds between a specific entity and a generic set  (e.g.  Wusa is a farmer). The clause can assert the identity of the referents of two specific entities, a clause type also known as equative (e.g. Wusa is the farmer). The examples in (\ref{ex:GRM-ident-cl}) illustrate the distinctions. 


\ea\label{ex:GRM-ident-cl}

\ea\label{ex:GRM-ident-gen-cat}{\rm Generic categorization}\\
\gll
 bɔ̀là jáá kɔ̀sásēl lē\\
 elephant {\ident}  bush.animal {\foc}\\
\glt `The/An elephant is a bush animal.'

\ex\label{ex:GRM-ident-ord-cat}{\rm Ordinary categorization}\\
\gll
wʊ̀sá jáá pápátá rá\\
W. {\ident} farmer {\foc}\\
\glt `Wusa is a farmer.'

\ex\label{ex:GRM-ident-tk-id}{\rm Identity}\\

\ea
\gll
wʊ̀sá jáá à tɔ́ɔ́tɪ̄ɪ̄nā\\
W. {\ident} {\sc art} landlord\\
\glt `Wusa is the landlord.'
\ex
\gll
wʊ̀sá jáá  à báàl tɪ̀ŋ ká sáŋɛ̃̄ɛ̃̄ kéŋ̀\\
W. {\ident} {\sc art} man {\sc art} {\egr} sit.{\pfv} {\dxm}\\
\glt `Wusa is the man sitting like this.'
\ex
\gll
à báàl tɪ̀ŋ kà sáŋɛ̃̄ɛ̃̄ kéŋ̀  jáá wʊ̀sá\\
 {\sc art} man {\sc art} {\egr} sit.{\pfv} {\dxm} {\ident}   W.\\
\glt `The man sitting like this is Wusa.'


%\ex\label{ex:GRM-ident-}{\it }

\z 
\z
 \z

The verb {\sls jaa}  (glossed {\ident}) always  occurs between two nominal  
expressions,  and, as shown in the last two examples in 
(\ref{ex:GRM-ident-tk-id}),  their order  does not matter, except for the 
generic categorization where the order is always [hyponym {\sls jaa} hyperonym].  So,  the sentences {\sls pápátá rá jāā wʊ̀sá} `farmer {\sc foc} 
is Wusa' and {\sls à tɔ́ɔ́tɪ̄ɪ̄nā  jāā  wʊ̀sá} `landlord {\sc foc} is Wusa'   are  as 
acceptable as in 
the 
order given in (\ref{ex:GRM-ident-ord-cat}) and the first example in 
(\ref{ex:GRM-ident-tk-id}). 


%nin na 
%re keng

\subsubsection{Existential clause} 
\label{sec:GRM-loc-cl}

One type of existential clause is the basic  locative construction, which is
described in Section \ref{sec:SPA-blc}. Its
two main characteristics are the obligatory presence of the \isi{postposition} {\sls 
nɪ},  which signals that the phrase contains the conceptual ground, and the
presence of a locative predicate or the general existential predicate {\sls 
dʊa}. An example is provided in (\ref{ex:GRM-loc-cl}).

\ea\label{ex:GRM-loc-cl}
\gll à báál dʊ́ɔ́ à dɪ̀à nɪ̄.\\
{\sc art}  man be.at {\sc art} house {\postp}\\
 \glt  `The man is at/in the house.'
\z

The existential predicate {\sls dʊa} is glossed `be at', but it is not the case
that it is only used in spatial description. For instance,  adhering to a
religion may be expressed using the existential predicate {\sls dʊa} and the
\isi{postposition} {\sls nɪ}, e.g.  {\sls ʊ̀ dʊ́á jàrɪ́ɪ́ nɪ̄} `he/she is a 
Muslim', 
even
though no space reference is involved in such an utterance. 

An existential clause is also used in order to express that something is at
hand, accessible or obtainable. The clause in (\ref{ex:GRM-avail-cl}) is called
here 
the availability construction. It slightly differs from the
locative
construction in (\ref{ex:GRM-no-avail-cl}) because of  the absence of the
\isi{postposition}
{\sls nɪ}.

\ea\label{ex:GRM-avail-vs-loc}

\ea\label{ex:GRM-avail-cl}{\rm Availability construction}\\

\gll à mòlèbíí dʊ́á dé.\\
{\sc art}  money be.at {\dem}\\
\glt  `There is money (available).'

\ex\label{ex:GRM-no-avail-cl}
\gll à mòlèbíí dʊ̄ā dé nɪ̀\\
{\sc art}  money be.at {\dem}  {\postp}\\
\glt `The money is there.'

\z 
 \z


Another use is the attribution of a property ascribed to a participant. The
example in (\ref{ex:GRM-loc-propascr}) reads literally `a sickness is at Wojo', 
i.e. a person named Wojo is sick.  In addition to the clause presented in
(\ref{ex:GRM-loc-propascr}), an ascribed property may also be conveyed in a
possessive clause (see Section \ref{sec:GRM-poss-cl}). 


\ea\label{ex:GRM-loc-propascr}
\gll gàràgá dʊ́á wòjò nɪ̄.\\
sickness be.at W. {\postp}\\
\glt  `Wojo is sick.'
\z

 The verb {\sls dʊa} is the only verb with an allolexe (i.e. a combinatorial 
\isi{variant} of a single lexeme) used only in the 
negative. Consider (\ref{ex:GRM-allolexe}).

\ea\label{ex:GRM-allolexe}

\ea\label{ex:GRM-allolexe-pos}
\gll ʊ̀  dʊ́á dɪ̀à nɪ̄.\\
{\sc 3sg} be.at house {\postp}\\
\glt  `She is in the house.'

\ex\label{ex:GRM-allolexe-neg}
\gll ʊ̀  wáá tùò dɪ̀à nɪ̀.\\
{\sc 3sg} {\neg} {\neg}.be.at house {\postp}\\
\glt  `She is not in the house.'


\ex\label{ex:GRM-allolexe-pos-out}
 \textasteriskcentered ʊ  tuo dɪa nɪ
\ex\label{ex:GRM-allolexe-neg-out}
 \textasteriskcentered ʊ  waa dʊa dɪa nɪ

\z 
 \z



\subsubsection{Possessive clause}
\label{sec:GRM-poss-cl}

A \isi{possessive} clause expresses a relation between  a
possessor and a possessed.   It consists of
the verb {\sls kpaga} `have',  and two nominal expressions acting as subject and
object; the former being the possessor (\psor) of the relation, while  the
latter being  the possessed
(\psed).

\ea\label{ex:GRM-poss-have}
\glll kàlá kpágá nã̀ɔ̃̀ rā.\\
K. have cow {\foc}\\
  {\psor} {}   {\psed} {}\\
\glt  `Kala has a cow'
\z

Example (\ref{ex:GRM-poss-have}) says that an animate alienable possession
relates  Kala (possessor) and a cow (possessed).  Since the  {\it
have-}construction does not encode animacy or alienability features,   staple
food can `have' lumps, i.e. {\sls kàpálà kpágá bīē}, and someone can `have' a
senior brother, i.e. {\sls ʊ̀ kpágá bɪ́ɛ́rɪ̀}.  Abstract possession may also be
conveyed using the same construction. In (\ref{ex:GRM-poss-have-abst}),
  shame, hunger,  thirst, and sickness are conceived as the possessors, the
possessed being the person experiencing these feelings. 



\ea\label{ex:GRM-poss-have-abst}

 \ea\label{ex:GRM-poss-have-abst-1}
\gll hɪ̃̀ɪ̃̀sáá kpāgā   à   hã́ã̀ŋ kɪ̀ŋkáŋ̀.\\
shame    have {\sc art}   woman much\\
\glt `The woman was ashamed ...' [CB 034]
\ex\label{ex:GRM-poss-have-abst-2}
\gll lʊ̀sá kpágáń̩ nà.\\
hunger have.{1.\sg} {\foc}\\
\glt `I am hungry.'
\ex\label{ex:GRM-poss-have-abst-3}
\gll  nɪ́ɪ́ɲɔ̀ksá kpágán̩ nà\\
thirst have.{1.\sg} {\foc}\\
\glt  `I am thirsty.'
\ex\label{ex:GRM-poss-have-abst-4}
\gll  gàràgá kpágán̩ nà\\
sickness have.{1.\sg} {\foc}\\
\glt  `I am sick.'

\z 
 \z

Some characteristics ascribed to animate entitites are expressed by the
relational term {\sls tɪɪna} `person characterized by, or in possession of' and thus may be expressed in an existential clause (\ref{ex:GRM-poss-owner-exist}) rather than a \isi{possessive} clause (\ref{ex:GRM-poss-owner-have}). 



\ea\label{ex:GRM-poss-owner}

 \ea\label{ex:GRM-poss-owner-exist}
\glll ʊ̀ jáá sísɪ́ámà-tɪ́ɪ́ná.\\
{3.\sg} {\ident} seriousness-owner\\
  {\psor} {}   {\psed}\\
\glt `He is serious'

 \ex\label{ex:GRM-poss-owner-have}
\gll ʊ̀ kpágá sísɪ́ámà rá.\\
{3.\sg} have {seriousness} {\foc}\\
\glt `He is serious'

\z 
 \z

\subsubsection{Non-verbal clause}
\label{sec:GRM-noverb}

As its name suggests, a non-verbal clause is a clause without verbal elements. 
Its
main function is to identify or assert the (non-)existence of 
something.  The examples in (\ref{ex:GRM-noverb-mine-2}) and (\ref{ex:GRM-noverb}) assert 
the (non-)existence
of a
referent with a single nominal expression, followed by the \isi{focus} particle in
the affirmative and the negative particle in the negative (see Section
\ref{sec:GRM-foc-neg} on \isi{focus} and \isi{negation}). 


\ea\label{ex:GRM-noverb-mine}
 \ea\label{ex:GRM-noverb-mine-1}
\gll áŋ kɪ́ŋ ká jàà kɪ́ŋ háŋ̀?\\
{\sc q.}who thing {\sc ipfv} {\ident} thing {\sc dem}\\
\glt `Whose thing is this thing.'
 \ex\label{ex:GRM-noverb-mine-2}
 \gll ŋ̀ kɪ́n nā.\\
{\sc 1sg} thing {\sc foc}\\
\glt `It is mine.'
\z 
 \z


\begin{multicols}{2}
\ea\label{ex:GRM-noverb}
 \ea\label{ex:GRM-noverb-aff-1}
\gll fʊ́n ná.\\
knife {\foc}\\
\glt `It is a shaving knife.'
 \ex\label{ex:GRM-noverb-aff-poss}
\gll ǹ̩ fʊ́n ná.\\
{\sc 1sg.poss} knife {\foc}\\
 \glt `It is my shaving knife.'
 \ex\label{ex:GRM-noverb-neg-1}
\gll fʊ́n lɛ̀ɪ́.\\
knife {\neg}\\
 \glt `It is not a shaving knife.'
 \ex\label{ex:GRM-noverb-neg-poss}
\gll ǹ̩  fʊ́n lɛ̀ɪ́.\\
{\sc 1sg.poss} knife {\neg}\\
 \glt `It is not my shaving knife.'

\z 
 \z
\end{multicols}

Correspondingly the manner deictics {\sls keŋ} and {\sls  nɪŋ} are also found in non-verbal clauses. 
For instance, {\sls kéŋ né} means `That is it!', but the same string is more often heard as {\sls 
kéŋ nȅȅ} `Is that so/it?',  i.e. constructed  as a polar question (see Section 
\ref{sec:GRM-interr-polar} on polar questions, and Section \ref{sec:GRM-adv-pro}  on  {\sls keŋ} 
and 
{\sls  nɪŋ}).

Finally, a speaker may utter {\sls mɪ́n nà} `it is me' in order to say that he or 
she must be identified by the addressee. This utterance consists solely of the third singular 
\isi{strong pronoun}, which refers to the discourse-given entity and someone whose identity will be established by the addressee, and is followed by the \isi{focus} particle (see Section \ref{sec:GRM-pronouns} on pronouns).

%see dakubu p22 The sentence is a two-argument topic-comment proposition with no
%verb.

\subsubsection{Multi-verb clause}
\label{sec:GRM-multi-verb-clause}

A \isi{multi-verb} clause is a clause containing more than one verb. The main type of
multi-verb clause is the serial verb construction (SVC), the definition of which
is still subject to contention. Let us start by stating that the SVC in Chakali
has the following properties: (i) a SVC is a sequence of verbs which act
together as a single predicate, (ii) each verb in the series could occur as a
predicate on its own, (iii)  no connectives  surface (coordination or
subordination), (iv)  tense, aspect, mood, and/or polarity are marked only once,
(v)  a verb involved in a SVC may be formally shortened,  (vi)  transitivity is
common to the series, so arguments are shared (one argument obligatorily), (vii)
the verbs in the series are not necessarily contiguous, and  (viii) the grammar
does not limit the number of verbs. These characteristics are not uncommon for 
SVCs in West Africa \citep{Amek05a}. 


Even though the construction has more than one
verb, it describes a single event and does not contain  markers of
subordination or coordination. The first sequence of verbs in
(\ref{ex:GRM-mvc-svc}) illustrates the phenomenon.



\ea\label{ex:GRM-mvc-svc}
\glll à  kɪ̀rɪ̀nsá   m̩̀   màsɪ̀   kpʊ́ àká dʊ̀gʊ̀nɪ̀ tá.\\
{\sc art}  tsetse.fly.{\pl}   {1.\sg}    beat   kill   {\conn} 
chase   let.free\\
 {} {} {}  [{\it v} {\it v}]  {} [{\it v} {\it v}]\\
\glt `I beat and killed the tsetse flies, and drove them away.' [CB 023]
\z

Together,  the verbs {\sls masɪ} `beat' and  {\sls kpʊ} `kill'  in 
(\ref{ex:GRM-mvc-svc})  constitute a single event.  The same can be said about 
the verbs {\sls dʊgʊnɪ} `chase' and {\sls ta} `let free' in the second clause
following the \isi{connective}.   If the clause following the \isi{connective}   {\sls aka}
lacks a subject,  the subject of the preceding clause shares its reference in
the two clauses   (see Section \ref{GRM-clause-coord-ka-aka} on the \isi{connective} 
{\sls aka}). The situation in (\ref{ex:GRM-mvc-svc}) is one where an SVC is separated from
another \isi{multi-verb} clause by the \isi{connective} {\sls aka},  and the three verbs 
{\sls 
masɪ},  {\sls kpʊ} and {\sls dʊgʊnɪ}  share the reference of the  nominal {\sls 
a
kɪrɪnsa} `the tsetse flies' as their Theme argument and {\sls m̩̀} as their 
Agent
argument, i.e. {\sc o} and {\sc s} respectively. The role of  the verb {\sls ta}
in the sentence depicted in  (\ref{ex:GRM-mvc-svc}) is discussed at the end of
this section.

Tense/aspect (\ref{ex:GRM-svc-tense}), mood (\ref{ex:GRM-svc-aspect}), and/or
polarity value (\ref{ex:GRM-svc-negation}) are marked only once, usually with
preverb particles. This means that they are not repeated for each verb of the  predicate. The preverb particles are discussed in Section
\ref{sec:GRM-precerv}.


\ea\label{ex:GRM-svc-preverb}
 \ea\label{ex:GRM-svc-tense}{
\gll  ǹ̩ tʃɪ́ kàá màsɪ̀   kpʊ́   à  kɪ̀rɪ̀nsá rá.\\
{1.\sg} {\cras} {\fut.\prog}   beat   kill {\sc art} tsetse.fly.{\pl} {\sc foc} \\
\glt `I will be beating and killing the tsetse flies tomorrow.'
}
 \ex\label{ex:GRM-svc-aspect}{
\gll  ǹ̩  há màsɪ̀   kpʊ́   à  kɪ̀rɪ̀nsá rá.\\
  {1.\sg}  {\mod}   beat   kill {\sc art} tsetse.fly.{\pl} {\sc foc}\\
 \glt `I am still beating  and killing the tsetse flies.'
 }
 \ex\label{ex:GRM-svc-negation}{
\gll  ǹ̩   wà másɪ́   kpʊ́   à  kɪ̀rɪ̀nsá.\\
  {1.\sg}  {\neg}   beat   kill {\sc art} tsetse.fly.{\pl}\\
\glt `I did not beat and kill the tsetse flies.'
 }

\z 
 \z
%wàá will not

SVCs must share at least one core  argument. The example 
(\ref{ex:GRM-arg-sh-objsubj}) is an instance of argument sharing: the two verbs
in the construction share the (referent of the) noun {\sls foto} `picture' and
are not contiguous. The
transitive verb {\sls tawa} `pierce' takes  {\sls foto} as its object, and similarly
{\sls laga} takes  {\sls foto} as its subject. A representation of 
object-subject
sharing (or switch sharing) appears under the free translation in
(\ref{ex:GRM-arg-sh-objsubj}).

%\footnote{The label {\it object-subject sharing}
%is borrowed from \citet[20]{Osam03}.} 

\ea\label{ex:GRM-arg-sh-objsubj}{\rm Object-subject sharing}\\
\glll  hɛ̀mbɪ́ɪ́ táwá fótò làgà dáá nɪ́.\\
nail pierce picture hang wood  {\postp}\\
{} {\it v} {}  {\it v}\\
 \glt `A picture hangs from a nail on a wooden pole.'

{\sls foto} $<x_i>$\\
{\sls tawa} $<${\sc a}$ =  y$ ,  {\sc o}$=x_i$  $> $\\
{\sls laga} $<${\sc a}$ = x_i$ , {\sc o} $= z $  $ >$\\
\z

%consistent with identifying function



Subject-subject and object-object sharing are more common than object-subject
sharing. In (\ref{ex:GRM-mvc-svc-bis-2}), the nominal expression {\sls a 
 kɪrɪnsa} is the shared object of three verbs, i.e. {\sls masɪ}, {\sls kpʊ} and {\sls dʊgʊnɪ}, and similarly the \isi{pronoun} {\sls m̩} is the shared subject for the same three verbs. However, only {\sls masɪ} and {\sls kpʊ}  make up the SVC. 

\begin{exe}

\ex\label{ex:GRM-mvc-svc-bis-2}{\rm Subject-subject and Object-object sharing}\\
\gll à  kɪ̀rɪ̀nsá   m̩̀   màsɪ̀   kpʊ́ àká dʊ̀gʊ̀nɪ̀ tá.\\
{\sc art}  tsetse.fly.{\pl}   {1.\sg}    beat   kill   {\conn} 
chase   let.free\\
\glt `I beat and killed the tsetse flies, and drove them away.'

{\sls m̩} $<x_i>$\\
{\sls kɪrɪnsa} $<y_j>$\\
{\sls masɪ} $<${\sc a}$ =  x_i$ ,  {\sc o}$=y_j$  $> $\\
%{\sls  kpʊ} $<${\sc subj}$ = x$ , {\sc obj} $= y $  $ >$\\
{\sls dʊgʊnɪ} $<${\sc a}$ = x_i$ , {\sc o} $= y_j  >$\\
 
\end{exe}

SVCs often involve two verbs, but there can be three or more verbs involved. 
Examples of three-verb and four-verb sequences are given in
(\ref{ex:GRM-mvc-3-4}). Each of the verbs involved can otherwise act alone as
main
predicate. Notice that the free translations provided do not accommodate well
the idea that
the two examples in (\ref{ex:GRM-mvc-3-4}) are conceived as single event.
In Section \ref{GRM-clause-coord-subord},  it will be shown that connectives
are usually present  when one wishes to distinguish events.


\ea\label{ex:GRM-mvc-3-4}

\ea{
\glll ʊ̀ síí kààlɪ̄ nà.\\
{3.\sg} rise go see\\
{}   {\it v}$_{1}$  {\it v}$_{2}$  {\it v}$_{3}$\\
\glt `She stood, went, and saw (it).'
}

\ex{
\glll ʊ̀ brá tùù tʃɔ́ kààlɪ̀.\\
{3.\sg} turn go.down run go\\
{} {\it v}$_{1}$  {\it v}$_{2}$  {\it v}$_{3}$  {\it v}$_{4}$\\
\glt `She returned down and ran away' (from a tree top or hill)
}

\z 
 \z

 
A  manipulative serial verb construction \citep[378]{Amek06} is a SVC
which  expresses a transfer of possession (e.g. give, bring, put)  or  
information (e.g. tell). It consists of the verb {\sls kpa} `take' and another
verb following it. The example in (\ref{ex:GRM-m-svc-give}), repeated in 
(\ref{ex:GRM-m-svc-give-1}), 
illustrates a transfer of possession. 

\newpage 
\begin{exe}

\ex\label{ex:GRM-m-svc-give-1}{\rm Manipulative serial verb construction}\\
%{\it Manipulative serial verb construction}\\
\glll  kàlá kpá  {à lɔ́ɔ́lɪ̀ / ʊ̄} tɪ̀ɛ̀ áfɪ́á.\\
K. take  {{\sc art} car / 3.\sg} give A.\\
{} {\it v} {}  {\it v} {}\\
\glt  `Kala gave the car/it to Afia.'
 
\end{exe}

Frequent collocations of the type presented in (\ref{ex:GRM-m-svc-give-1}) are {\sls kpa wa}, {\it lit.}  take come,  `bring',  {\sls kpa kaalɪ}, {\it lit.} take go, `send', {\sls kpa pɛ}, {\it lit.} take add,  `add', {\sls kpa ta}, {\it lit.} take let free, `remove', {\sls kpa bile}, {\it lit.} take put,  `put (on)'  and {\sls kpa dʊ}, {\it lit.} take put,  `put (in)'. The two verbs may or may not be contiguous;  usually the Theme argument of the  verb {\sls kpa} `take'  is found between the two verbs.


Finally, some \isi{multi-verb} clauses are not  SVCs.  There are a few verbs which bear a relation to the main predication and  contribute  aspects of the phase of execution or scope of an event.\footnote{These verbs are similar to what \citet[108]{Bonv88} calls {\it auxiliant}.} For instance, a {\it terminative}  construction describes an event coming to an end or reaching a termination, and  a {\it relinquishment} construction describes an event whose result is the release or abandonment of someone or something.  The verbs {\sls peti} `finish' and {\sls ta} `abandon' in (\ref{ex:GRM-mvc-pha-3.1}) and (\ref{ex:GRM-mvc-pha-help}), together with a non-stative predication, determine each construction. 

\ea\label{ex:GRM-mvc-phase}

\ea\label{ex:GRM-mvc-pha-3.1} {\rm Terminative construction}\\
\glll làɣálàɣá hán nɪ̀ ǹ̩ kʊ̀tɪ̀ à ʔã́ã́ pétí.\\
{\ideo} {\dem} {\postp} {1.\sg} {skin} {\sc art} bushbuck  finish\\
{} {} {} {}  {\it v} {} {} {\it v}\\
\glt `I  just finished skinning the bushbuck.'

\ex\label{ex:GRM-mvc-pha-40.3}
\gll  m̩̀ pétì à tʊ́má rá.\\
{1.\sg} finish {\sc art} work {\foc}\\
\glt `I have finished the work.'


\ex\label{ex:GRM-mvc-pha-help}{\rm Relinquishment construction}\\
\glll  kpá ǹ̩ néŋ tà.\\
take {1.\sg} hand let.free\\
 {\it v} {}  {} {\it v}\\
\glt `Let me go!'

\ex\label{ex:GRM-mvc-pha-relish} 
\gll  à bʊ̃́ʊ̃́ŋ tá ʊ̀ʊ̀ bìè rē.\\
{\sc art} goat abandon {3.\sg.\poss} child {\foc}\\
\glt `The goat abandoned its kids.'

\z 
 \z

The examples  in (\ref{ex:GRM-mvc-pha-3.1}) and (\ref{ex:GRM-mvc-pha-help}),
which may be called  {\it phasal  constructions},\footnote{The analysis of the
progressive and prospective in \ili{Ewe} and \ili{Dangme} in \citet{Amek08} influences the
way I approach and name the phenomenon.}  are treated as \isi{multi-verb} clauses
since the predication is expressed with more than one verb. Yet, they are not
SVCs because the second verb in each example only specifies aspects of the
process
of the event  and does not contribute to the main predication as verb sequences
in SVCs do. Nonetheless, these verbs can function otherwise as main predicates,
as shown in (\ref{ex:GRM-mvc-pha-40.3}) and (\ref{ex:GRM-mvc-pha-relish}).
Similarly, the verb {\sls baga} `attempt to no avail'  conveys
nonachievement, e.g. {\sls ʊ̀ búúré kísīē bàɣá} ({\it lit.} he look.for knife 
fail) 
`he looked for a knife to no avail',  and the verb {\sls na} `see' conveys confirmation 
or
verification, e.g. {\sls sʊ̀ɔ̀rɛ̀ à dɪ̀sá nā} ({\it lit.} smell soup see) `smell the
soup'. Going back to example (\ref{ex:GRM-mvc-svc-bis}) above, the verb {\sls ta}
contributes to a {\it relinquishment} \isi{multi-verb} construction, similar to
(\ref{ex:GRM-mvc-pha-help}) above, and not to a SVC. 
 
% 

\subsubsection{Basic locative construction}
\label{sec:SPA-blc}


The \is{basic locative construction} basic locative construction  of a language 
is  the prototypical  and predominant construction used to locate a figure with 
respect to a ground \citep[15]{Levi06}. In Chakali, it resembles the 
construction given in (\ref{ex:PSPV4}), although some sentences produced in 
elicitation contexts appear with the \isi{focus} particle following the 
\isi{postposition} {\sls nɪ}.  The \is{focus particle}{focus} particle  is a pragmatic 
marker which identifies for the addressee the topical subject (i.e. may be 
distinct from the grammatical subject) and does not convey locative meaning 
(Section \ref{sec:GRM-focus}). The \isi{focus} particle will be ignored in the 
discussion. The third  line  in (\ref{ex:PSPV4}) associates parts of the 
sentence with a conceptual level. On that line, one can find notions such as 
{\it figure} and {\it ground},  and \textsc{trm}, which stands for 
\is{topological relation marker} topological relation marker 
\citep[see][]{Brin12}. These are the linguistic expressions which convey  the 
spatial relationships in Chakali.  The nominal phrase {\sls a gar} `the cloth'  
functions as subject and the postpositional phrase {\sls a teebul ɲuu nɪ} `on the table' 
functions as  oblique object  of the main predicate. The last line is a free 
translation which captures  the general meaning of the situation. It is 
accompanied by a reference to the illustration which the first line 
describes.\footnote{Subscribing to the typology of locative predicates proposed in \citet{amek07b}, the illustrations of the  four stimuli created by the Language and Cognition Group at the Max Planck Institute for Psycholinguistics \citep{bowe93, amek99, meir01a, meir01b} were used in chapter 7 of \citet{brin11} to provide a description of the means by which Chakali encodes spatial
meaning. The results are compared with Gurenɛ data (\ili{Oti-Volta}) in \citet{Brin12}.}



\begin{exe}
\ex\label{ex:PSPV4}
\glll  [à gár] [ságá] [à {téébùl ɲúù} nɪ̀].\\
{\sc art} cloth be {\sc art}  {table {\reln}}  {\postp}\\
 \textit{figure} {} \textsc{trm}  {} {\textit{ground}+\textsc{trm}}  {} \\
\glt `The cloth is on the table.' [PSPV 4]
\end{exe}


In (\ref{ex:PSPV4}), the spatial relation is expressed via topological relation markers:  the main predicate {\sls saga} `be on' or `sit' and the relational nominal predicate {\sls ɲuu} `top of'. The main predicate  {\sls saga}  denotes a stative event which  localizes the figure with respect to the ground.  The relational nominal predicate {\sls ɲuu} designates the search domain and depends on the reference entity of the ground (i.e. {\sls teebul}). The \isi{postposition}  {\sls nɪ} has no other function than to signal that the oblique object is a locative phrase. The latter two topological relation markers are discussed in more detail in Sections   \ref{sec:SPA-relnoun} and \ref{sec:SPA-postp}.

\subsubsection{Comparative construction}
\label{sec:GRM-compar-ct}

A comparative construction has the semantic function of assigning a graded
position on a predicative scale to two (possibly complex) objects.
The comparative construction of inequality can be expressed with the
transitive predicate {\sls kaalɪ} `exceed, surpass', whose  two arguments are
the objects compared.\footnote{\citet{Brin05} presents a Lexical-Functional Grammar  account of the comparative construction in \ili{Ga}̃, a language also
exhibiting an  exceed-  or surpass-comparative.}  One of the arguments 
represents the
standard
against which the other is
measured and found to be unequal.  The nominal expression in subject position is
the {\it comparee}, i.e. the objective of comparison, whereas the
one in object position is the {\it standard}, i.e. the object that
serves as yardstick for comparison \citep{Stas08}. The predicative scale is verbal and is normally adjacent to  the comparee, but may
be repeated adjacent to the standard. Given that both the scale and the
transitive predicate {\sls kaalɪ} are verbs, a comparative construction is  a
 type of \isi{multi-verb} clause.  If the predicative scale is absent, as in
(\ref{ex:GRM-comp-tr-sca-abs}),  one
may still interpret the construction as a comparative one, in which case both
the
context
and the meaning of  the nominals involved  provide the property on which
the
comparison  is made. These characteristics are illustrated in
(\ref{ex:GRM-comp-tr}).


\ea\label{ex:GRM-comp-tr}{\rm Comparative transitive construction}\\
\ea\label{ex:GRM-comp-tr-sca-pres}{
\glll wʊ̀sáá zɪ́ŋá kààlɪ̀ áfɪ́á.\\
  W. grow surpass A.\\
[{\it n}]$_{comparee}$  [{\it v}]$_{scale}$ {\it v} [{\it n}]$_{standard}$\\
\glt `Wusa is taller than Afia.'
}

\ex\label{ex:GRM-comp-tr-sca-abs}{
\glll wʊ̀sá bàtʃɔ́lɪ́ káálɪ́ kàlá bàtʃɔ́lɪ́.\\
W.  running surpass K. running\\
[{\it n}  {\it n}]  {\it v} [{\it n} {\it n}]\\
\glt `Wusa's running is better/faster than Kala's running.'
}

\z 
 \z

Another way to compose a comparative construction of inequality is with the
identificational clause, as in (\ref{ex:GRM-comp-int}).  It is referred to as a comparative intransitive construction since the standard is not encoded in the grammatical object of a transitive verb, but in an oblique object following the scale.


\ea\label{ex:GRM-comp-int}{\rm Comparative intransitive construction}\\

\glll wʊ̀sá jáá nɪ́hɪ̃̀ɛ̃̂ àfɪ̀á nɪ́.\\
W.  {\ident} old A. {\postp}\\
[{\it n}]$_{comparee}$   {\it v} [{\it v}]$_{scale}$  [{\it n}]$_{standard}$  
{}\\
\glt `Wusa is older than Afia.'
\z

The same  two strategies are used to
express a superlative degree: surpassing or being superior to all others is
explicitly expressed by the \isi{pronoun} {\sls ba} `they, them'.
This is shown in (\ref{ex:GRM-super}).


\ea\label{ex:GRM-super}{\rm Superlative construction}\\

\ea
\glll wʊ̀sá zɪ́ŋá kāālɪ́ bá.\\
W. grow surpass {\sc 3pl}\\
{} {\it v} {\it v} {}\\
\glt  `Wusa is the tallest.'
\ex 
\gll wʊ̀sá jáá nɪ́hɪ̃̀ɛ̃̂ bà nɪ́.\\
W. {\ident} old {\sc 3pl} {\postp}\\
\glt  `Wusa is the oldest.'

\z 
 \z
 
 
A comparison of equality (i.e. X is same as Y) consists of a subject phrase containing both objects to be  compared joined by the  \isi{connective} {\sls (a)nɪ} followed by the scale, the verb {\sls mààsɪ̀} `equal, enough, ever' and the reciprocal word {\sls dɔ̀ŋà} `each other'  (see Section \ref{sec:GRM-recipro-reflex} on reciprocity and reflexivity). This is shown in (\ref{ex:GRM-comp-equal}).

\ea\label{ex:GRM-comp-equal}{\rm  Comparison of equality construction}\\

\gll wʊ̀sá nɪ́ àfɪ̄ā bɪ̀nsá máásé dɔ́ŋá rā.\\
W. {\conn} A.  year equal {\recp} {\foc}\\

\glt `Wusa is as old as Afia.'
\z


Finally,  the verb {\sls bɔ́} in (\ref{ex:GRM-comp-verb}) is a comparative
transitive verb which can be translated with the English comparative adjective and preposition `better than'.


\ea\label{ex:GRM-comp-verb}
\glll zàáŋ tʊ́má bɔ́ dɪ̀àrɛ̀ tɪ̀ŋ tʊ̄mā.\\
today work better.than yesterday {\sc art} work\\
{}  {} {\it v} {} {}  {}\\
\glt `Today's work is better than yesterday's work'
\z


\subsubsection{Modal clause}
\label{sec:GRM-modalclause}

A modal clause is a clause type expressing  ability, possibility,   obligation, desire, etc. The two following sections exemplify the modal clause. 

\paragraph{Ability-possibility}
\label{sec:GRM-ability-possibility}

An ability-possibility construction is a clause containing the verbal {\sls kɪ̀n} immediately 
preceding the main verb(s).  The construction conveys either the physical or mental ability of 
something or someone, or probability or possibility under some circumstances. The construction is 
more frequent in the negative, but affirming an ability or possibility is also possible using this construction. The word {\sls kɪ̀n} is glossed  {\abi} to refer to `ability'.


\ea
{\upshape Ability-Possibility construction}\\

\ea
\label{ex:GRM-modal-12.2}
\gll ʊ̀ wà kɪ́ŋ wàà.\\
{3.\sg} {\neg} {\abi} come\\
\glt  `He is not able to come.'

\ex  
\gll ɪ̀ kàá kɪ̀ŋ kààlʊ̄ʊ́.\\
 {2.\sg} {\fut} {\abi} go.{\foc}\\
\glt  `You may go.'

\ex\label{ex:GRM-modal-13.1}
\gll ǹ̩ kàá kɪ̀ŋ wàʊ̀ tʃȉȁ?\\
 {1.\sg}  {\fut} {\abi} come.{\foc} tomorrow\\
\glt `May I come tomorrow?'

\z
\z

However  the elicitation data in (\ref{ex:GRM-modal-kin-v}) shows that, unlike most preverbs  (Section \ref{sec:GRM-precerv}), {\sls kɪ̀n}   may take inflectional morphology, in this case the perfective suffix (Section \ref{sec:GRM-verb-perf-intran}).  

\ea\label{ex:GRM-modal-kin-v}
\ea 
\gll A: ʊ̀ wà kɪ́ŋ wȁȁ?\\
{} {3.\sg} {\neg} {able} come \\
\glt  `Couldn't he come?' (declarative: {\sls ʊ̀ wà kɪ́ŋ wàà.})

\ex 
\gll  B: ɛ̃̀ɛ̃́ɛ̃̀, ʊ̀ wà kɪ́njɛ̃̄ wàà\\
{}  yes {3.\sg} {\neg} able.{\sc pfv} come\\
\glt  `Yes, he couldn't come.'

\z
\z

The dubitative modality construction is a construction marked by the presence of  
{\sls abɔnɪ̃ɛ̃nɪ} in clause initial position.  The expression is transcribed into  a single word but may come from {\sls a-banɪ̃ɛ̃-nɪ}, {\it lit.} \textsc{art}-some-\textsc{postp}. It  is used when the occurrence of a 
situation  or an achievement  is in doubt (see {\sls nɪ} in Section \ref{sec:SPA-postp}).

\ea
\label{ex:GRM-modal}{\rm Dubitative construction}\\

\ea
\label{ex:GRM-modal-45.5}
\gll {àbɔ́nɪ̃́ɛ̃́nɪ́}  dʊ́ɔ́ŋ kàá wàʊ̄.\\
perhaps rain {\fut} come.{\foc}\\
\glt  `Perhaps it is going to rain.'


\ex
\label{ex:GRM-modal-45.3}
\gll  {àbɔ́nɪ̃́ɛ̃́nɪ́}   ʊ̀ dɪ̀ wááwáʊ́.\\
perhaps  {3.\sg} {\hest} come.{\pfv.\foc}\\
\glt `Perhaps he came yesterday.'

\z
\z


In some contexts, a speaker may prefer  to use a cognitive verb in a phrase like {\sls n̩ lisie} `I think (...)'  or the phrase {\sls a kʊ̃ʊ̃ n̩ na}, {\it lit.} it tires me {\sc foc},  `I wonder (...)' as an alternative to the dubitative construction. 

\paragraph{Desiderative}
\label{sec:GRM-desiderative-mood}


As an independent verb {\sls ŋma} means `say'. The same verb can also function in a construction [NP {\sls ŋma} [NP VP]]  conveying a desiderative mood,  corresponding to the English modal expression `want to'.

\ea\label{ex:dsdrtv-2}
\gll ŋ̀ ŋmá [ŋ́ káálɪ̀ dùsèè tʃɪ̄ā].\\
  {\sc 1sg} say  {\sc 1sg} go D. tomorrow\\
\glt  `I want to go to Ducie tomorrow.'
\z 

 Notice that the high tone on the  {\sc 1sg} \isi{pronoun} subject of {\sls kààlɪ̀} `go'  in (\ref{ex:dsdrtv-2})  suggests that the embedded clause is in the subjunctive mood (Section \ref{sec:GRM-subjunctive}).

 
 \subsection{Interrogative clause}
\label{sec:GRM-interr-clause}

An \isi{interrogative} clause consists either of a clause (i) with an initial
\isi{interrogative} word/phrase (Section \ref{sec:GRM-interg-pro} on pro-form 
interrogatives), or (ii) with the absence of an initial \isi{interrogative}
word but the presence of an extra-low tone at the end of the clause. The former
is called a `content' question and the latter a `polar' question. 

\subsubsection{Content question}
\label{sec:GRM-interr-content}

A content question contains an \isi{interrogative} word/phrase whose typical position is clause-initial. In (\ref{ex:GRM-inter-content}), {\sls baaŋ} `what' replaces the complement of the verb {\sls jaa}, whereas {\sls (a)aŋ} `who'  replaces the subject constituent of the clause. The inventory of \isi{interrogative} words/phrases can be found in Section \ref{sec:GRM-interg-pro}.

\ea\label{ex:GRM-inter-content}
\ea\label{ex:GRM-inter-content-what}
\gll bááŋ kàlá kàà jáà?\\
{\q}.what {3.\sg} {\ipfv} do\\
 \glt  `What is Kala doing?' 
%  \ex\label{ex:GRM-inter-content-who}

\ex\label{ex:GRM-inter-content-who}
\gll àáŋ káá wáá báŋ̄?\\
{\q}.who  {\ipfv}  come here\\
\glt  `Who is coming here?'

\z 
 \z
 
When an \isi{interrogative} word/phrase is located clause-initially,  it is found in the canonical position of the constituent replaced. In (\ref{ex:GRM-inter-content-who-rev-bear-in}), which is semantically equivalent to  (\ref{ex:GRM-inter-content-who-rev-bear-ex}),  the question word {\sls aŋ} `who' appears in the object position following the transitive verb {\sls maŋa} `beat' and is slightly lengthened. 

%double-check; super-low on 'beat' verb

\ea\label{ex:GRM-inter-content-who-rev-bear}

\begin{multicols}{2}
 \ea\label{ex:GRM-inter-content-who-rev-bear-in}
\gll  zɪ̀ɛ̀n ká màŋà àŋ́ŋ?\\
 Z.  {\sc egr} beat {\q}.who \\
 \glt  `Zien beat who?'

 \ex\label{ex:GRM-inter-content-who-rev-bear-ex}
\gll  àŋ́ zɪ̀ɛ̀n kà màŋà?\\
{\q}.who  Z. {\sc egr} beat\\
 \glt  `Who did Zien beat?'

\z 
\end{multicols}
 \z

 
\subsubsection{Polar question}
\label{sec:GRM-interr-polar}

%different from the declinaison drop of declarative

A polar question is characterized by an \isi{interrogative} intonation, consisting primarily of an extra-low tone at the end of the utterance. Additionally, \is{lengthening}lengthening  of the penultimate vocalic segment takes place. The properties differentiating an assertive clause from a polar question are illustrated in (\ref{ex:GRM-inter-polar}). The extra-low tone is represented with a double grave accent (i.e.  ̏).


\ea\label{ex:GRM-inter-polar}{\rm Assertion vs. question}\\

\begin{multicols}{2}
\ea
\gll ʊ̀ wááʊ̀.\\
{3.\sg} come.{\ipfv .\foc}\\
\glt  `He is coming.'
\ex 
\gll ʊ̀ wāāʊ̏ʊ̏.\\
{3.\sg} come.{\ipfv .\q}\\
\glt `Is he coming?'%22.1.1

\z 
\end{multicols}
 \z
 
Common to many Ghanaian languages, the  agreeing response to a  negative polar \isi{interrogative} takes into account the logical \isi{negation}, as (\ref{ex:GRM-inter-polar-neg-rep}) illustrates. 

\ea\label{ex:GRM-inter-polar-neg-rep}
\ea\label{ex:GRM-inter-polar-neg-rep-S}{\rm Speaker}\\
\gll  ɪ̀ wàà kāālɪ̏ɪ̏.\\
{2\sg} {\neg} go.{\q}\\
\glt `Aren't you going?'

\ex\label{ex:GRM-inter-polar-neg-rep-A}{\rm Addressee}\\
\gll ɛ̃̀ɛ̃́ɛ̃̀.\\
yes\\
\glt `No' ({\it lit.} Yes, I am not going)

\z 
 \z
 
A negative polar \isi{interrogative} in English usually asks about the positive proposition, i.e. with `Aren't you going?',  the speaker presupposes that the addressee is going,   while in Chakali it questions the negative proposition, i.e. with {\sls  ɪ̀ wàá káálɪ̏ɪ̏},  the speaker's belief is that the addressee is not going. That is probably why we get `yes' in Chakali and `no' in English for a corresponding negative polar \isi{interrogative}.

\subsection{Imperative clause}
\label{sec:GRM-imper-clause}

An imperative clause is clause expressing direct commands, requests, and prohibitions. It can be an exclusively addressee-oriented clause or  can include the speaker as well. This distinction, i.e. exclusive-inclusive, is rendered in  (\ref{ex:GRM-imperative-exc-inc}). In (\ref{ex:GRM-imperative-exc}) the  speaker excludes herself  from the performers of the action, i.e., only the addressee(s) is urged to perform the action,  while in  (\ref{ex:GRM-imperative-inc}) the speaker includes herself among the performers.



\ea\label{ex:GRM-imperative-exc-inc}
\ea\label{ex:GRM-imperative-exc}{\rm Exclusive}\\
\gll fùùrì à díŋ dʊ̀sɪ̀.\\
blow {\sc art} fire quench\\
\glt `Blow on this flame (to extinguish it).'

\ex\label{ex:GRM-imperative-inc}{\rm Inclusive}\\
\gll tɪ̀ɛ̀ jà mùŋ làɣàmɛ̀ kààlɪ̀ tɔ́ʊ́tɪ́ɪ́ná  pé.\\
give {1\pl} all gather go landlord end\\
\glt `Let's all go to the landlord  together.'

\z
\z

When an order is given directly to the addressee, as in 
(\ref{ex:GRM-imper-exc-var}), the clause may be introduced with the particle 
{\sls dɪ}. Some consultants believe that omitting the particle may be perceived 
as rude.  The particle {\sls dɪ} can also implicate that performing 
the action is requested by someone else than the speaker.\footnote{It is not 
known whether these `{\sls dɪ}-strategies' give rise to multiple 
interpretations.}  In addressing a command to a group, the second person \isi{plural} 
subject \isi{pronoun}  usually appears in its canonical subject position, but it may 
be absent if the speaker believes that the context allows a single 
interpretation.\footnote{If A asks `What does he want?', B may reply {\sls dɪ́  
má dɪ́ wāā} `That you ({\sc pl}) should be coming'. In this case the first 
{\sls dɪ} heads a  clause which introduces indirect speech and the second is an 
imperfective particle,  the latter being covered in Section 
\ref{sec:GRM-ipfv-part}. }


\begin{multicols}{2}
 

\ea\label{ex:GRM-imper-exc-var}
\ea\label{ex:GRM-imper-exc-var-sg}
\gll (dɪ̀)  wàà.\\
 {\comp} come\\
\glt `Come!'

\ex 
\gll dɪ̀ wáá.\\
 {\comp} come\\
 \glt `Come!' (keep coming! or follow me!)
 
\ex 
\gll máá wáà.\\
 {\sc 2pl} come\\
\glt `Come!'

\ex\label{ex:GRM-imper-exc-var-out}  
\gll dɪ́ máá wāā.\\
 {\comp}  {\sc 2pl}  come\\
\glt `Come!' (requested by someone else than speaker)
\z
\z

\end{multicols}

Example (\ref{ex:GRM-hortative-vp11.3}) expresses a wish of the speaker and no addressees are called for. Such a meaning is sometimes associated with \is{optative} optative mood. Similarly but not  identically,  an utterance like the one in (\ref{ex:GRM-hortativ-vp11.4})  assumes one or more addressees, yet the desired state of affairs is not in the control of anyone in particular, but of everyone. As in (\ref{ex:GRM-imperative-inc}), the  strategy in both cases is to use the verb {\sls tɪɛ} `give'.  

\ea\label{ex:GRM-hortative}
 
\ea\label{ex:GRM-hortative-vp11.3}{\rm Optative}\\
\gll tɪ̀ɛ̀ m̩̀ mɪ̀bʊ̀à bírgì.\\
    give {\sc 1sg.poss} life delay\\
\glt  `Let me live long!' 

\ex\label{ex:GRM-hortativ-vp11.4}{\rm Hortative}\\
\gll tɪ̀ɛ̀ à gʊ̀à píílé.\\
  give {\sc art} dance start\\
\glt  `Let the dance begin!'

\z 
 \z

A prohibitive clause consists of  a negated proposition conveying an imperative 
(or hortative) mood. It is marked by the negative particle {\sls tɪ}/{\sls te} 
`not'   (glossed {\sc neg.imp}) occurring in clause initial position.


\ea\label{GRM-neg-imp-vp15.10.}
\gll té káálíí, dʊ́ɔ́ŋ kàà wáʊ̀.\\
  {\neg.\imp} go rain {\ipfv} come.{\foc}\\
\glt  `Don't go, rain is coming.' 
\z
 
The prohibitive also involves a high front vowel  suffixed to its verb. The 
quality of the vowel, i.e. {\sls -ɪ}/{\sls -i}, is determined by the quality of 
the verbal stem.

\begin{multicols}{2}
\ea\label{ex:GRM-neg-imperative}
\ea 
\gll gó.\\
circle\\
\glt  `Move in a circle around.'
\ex 
\gll  té   góìí.\\
{\neg.\imp} circle\\
\glt  `Don't move in a circle around.'
\columnbreak
\ex  
\gll kpʊ́.\\
kill\\
\glt `Kill.'
\ex
\gll tɪ́ kpʊ́ɪ̀ɪ́.\\
{\neg.\imp} kill\\
\glt `Don't kill.'
\z 

 \z
 \end{multicols}

In addition, a distinction within the prohibitive can be made
between a prohibition (or advice) for a future situation  
(\ref{ex:GRM-neg-fut}), 
and  for an on-going situation (\ref{ex:GRM-neg-pres}). 


\begin{multicols}{2}
 
 

\ea\label{ex:GRM-neg-fut-pres}

 \ea\label{ex:GRM-neg-fut}
\gll  kʊ̀ɔ̀rɪ̀ à sɪ̀ɪ̀máà.\\
  make {\sc art} food\\
 \glt `Make the food.'
 
 
 \ex 
\gll  té kʊ́ɔ́rɪ́ sɪ̀ɪ̀máà\\
{\sc neg}  make food\\
 \glt `Do not make food.' (addressee not in the process)
 \vfill
 \columnbreak
 
  \ex \label{ex:GRM-neg-pres}
  \gll  tɪ́ɪ́ kʊ̄ɔ̄rɪ̄ɪ̄.\\
 {\sc neg.imp}  make\\
 \glt `Do not make (food).' (addressee in the process of making)
   \ex 
  \gll  tíí kʊ̄ɔ̄rɪ̀  à sɪ̀ɪ̀máà.\\
 {\sc neg.imp}  make {\sc art}  food\\
 \glt `Do not make the food.' (addressee in the process of making)
\z
\z

\end{multicols}



\subsection{Exclamative clause}
\label{sec:GRM-excla-clause}

It is generally known that the difference between a declarative and an exclamative  clause is 
that the former is meant to be informative and the latter expressive. One criterion for determining 
the class of exclamative clause is the use of  exclamatory codas  \citep[242]{alla14}, i.e.  
exclamatory words or particles such as  {\sls woo} in (\ref{ex:GRM-excla-clause}), which modify the 
illocution of the clause and are usually found clause-finally.

 \ea\label{ex:GRM-excla-clause}
\ea\label{ex:GRM-excla-clause-1}
\gll  bɛ̀lɛ̀ɛ́ tɪ́ŋ mùŋ nè kéŋ wòòò. \\
G. {\sc art} meaning {\sc foc} {\sc dxm}  {\sc interj}\\
\glt `That is the meaning of Gurumbele!' [BH 016]

\ex\label{ex:GRM-excla-clause-2}
\gll dʊ̃́ʊ̃́ ɲú kpárá rá wōōō.\\
python head double {\sc foc} {\sc interj}\\
\glt `Python is double-headed!'  [PY 074]

\z
\z


After offering a chronicle of the history of his village and the reason why it has its name, the 
speaker 
uttered (\ref{ex:GRM-excla-clause-1}) to intensify his stance in the presence of other community 
members. In  (\ref{ex:GRM-excla-clause-2}), the narrator of the folktale wants to mark the 
surprising fact that the African rock python is equipped with extraordinary visual power.



\ea\label{ex:GRM-excla-clause-3-4}
\ea\label{ex:GRM-excla-clause-3}
\gll  ʊ̀ kà báŋ wà zú dìà, ʊ̀ bàŋ ŋmá dɪ́  ɛ̃̀hɛ̃́ɛ̃̄ɛ̃̀.\\
{\sc 3sg} {\sc ipfv} just come enter room  {\sc 3sg} just say {\sc comp}  {\sc interj}  \\
\glt `When he entered  the room,  she said: ``yes!''' [PY 008]\\

\ex\label{ex:GRM-excla-clause-4}
\gll dɪ́ ʔábbā!, dɪ́  ʊ̀ʊ̀ bàmbíí nár wááwáʊ́.\\
{\sc comp}  {\sc interj} {\sc comp}  {\sc 3sg.poss} heart person come.{\sc pfv.foc}\\
\glt (Mother said) `Indeed,   her love has finally come.'  [PY 009]\\

\z
\z

In  (\ref{ex:GRM-excla-clause-3}), the  speaker is a mature girl who waited a long time and met 
many aspirants to finally encounter the right man to marry. In this context, the exclamative word  
{\sls ɛ̃hɛ̃ɛ̃}, which generally code a positive reaction (Section \ref{sec:GRM-greet}),  can be 
translated into `yes, exactly, finally' and paraphrased as  `this is the person I like'.   The 
speaker confirms that the man is the right one, with a strong emotional reaction,  allowing the 
addressee -- in this case her parents -- to know about her stance and feeling. The sequence in  
(\ref{ex:GRM-excla-clause-4})  is the reaction of her mother who confirms the daughter's reaction. 
Notice however  that the interjections {\sls ɛ̃hɛ̃}, {\sls ʔabba}, and {\sls woo} are not specific 
to Chakali: they are {\it Ghanaianism}, i.e. words found in most, if not all, languages of Ghana, and surely beyond (see  Section \ref{sec:GRM-greet} for interjections).



\subsection{Clause coordination and subordination}
\label{GRM-clause-coord-subord}
%embedded clause, subordinate clause) cannot stand alone as a sentence

A relation between two clauses is signaled with or without an overt marker,  and 
various  structures and morphemes  are used to relate clauses.  Two
relations are discussed below: coordination and subordination. 

\subsubsection{Coordination}
\label{GRM-clause-coord}


The distribution of four clausal connectives which are used in coordinating
clauses is presented: these are {\sls a}, {\sls ka}, {\sls aka} and {\sls 
dɪ}.\footnote{See \citet[143--149]{mcgi99} for an account of similar clausal
connectives in \ili{Pasaale}.}  


\paragraph{Connective {\sls a}}
\label{GRM-clause-coord-a}


The \isi{connective} {\sls a} `and'  introduces a clause without an overt subject.  When 
it occurs between two clauses, the subject of the first clause must 
cross-refer to the covert subject of the second clause  (and subsequent 
clauses). It links a sequence of closely related events carried out by the same 
agent, and the events are encoded in  verb phrases denoting temporally distinct 
events. The example in (\ref{ex:GRM-coor-vp8.1}) is  an illustration of four 
consecutive clauses introduced by the \isi{connective}  {\sls a}.   This phenomenon 
is often referred to as `clause chaining'.\footnote{The last sentence of example (\ref{ex:GRM-coor-vp8.1}) can be analysed as a coordination by clause apposition.}

\ea\label{ex:GRM-coor-vp8.1}
\gll dɪ̀àrɛ̀ tɪ̀ŋ ǹ̩ dɪ̀ káálɪ́ bɛ̀lɛ̀ɛ̀ rá, à {\ob}jàwà nàmɪ̃̀ɛ̃́{\cb}, à 
{\ob}kpá wàà dɪ̀á{\cb}, à  {\ob}wà tɪ̀ɛ̀ ǹ̩ hã́ã̀ŋ{\cb}, à {\ob}ŋmá tɪ̀ɛ̀ ǹ̩ hã́ã̀ŋ{\cb} 
dɪ́ ʊ́ʊ́ tɔ́ŋà. ʊ̀ tɔ̀ŋà jà dí.\\
yesterday {\sc art} {\sc 1sg} {\hest} {go} {G.}  {\foc} {\conn}   buy meat
{\conn}  take come home {\conn}  come give my wife  {\conn}  say give my wife
{\comp}  {\sc 3sg}   {cook} {\sc 3sg}   {cook}  {\sc 1pl} {eat}\\
\glt  `Yesterday I went to Gurumbele,  bought some meat, brought it
home to my wife, told her to cook it. She cooked and we ate.'
\z


\paragraph{Connectives {\sls ka} and {\sls aka}}
\label{GRM-clause-coord-ka-aka}

Generalizing from the examples available, for both the connectives {\sls ka} 
and 
{\sls aka} `and', either (i) the subject of the clause preceding the \isi{connective} is 
inferred in the second clause, i.e. as for  the \isi{connective}  {\sls a} in Section \ref{GRM-clause-coord-a}, or 
(ii) a different subject surfaces in the second clause. Each case is shown in 
(\ref{GRM-clause-conn-ka-1-subj}) and (\ref{GRM-clause-conn-(a)ka-2-subj}) 
respectively.   


\ea\label{GRM-clause-conn-ka-1-subj} 
\gll [ŋmɛ́ŋtɛ́l   láá nʊ̃̀ã̀  nɪ́] ká  [ŋmá dɪ́ ʊ́ʊ́  
wá  
ɲʊ̃̀ã̀ nɪ́ɪ́]\\
spider collect mouth {\postp}  {\conn} say {\comp}  
{\sc 3sg}  come   drink water\\
\glt  `(Monkey went to spider's farm to greet him.) Spider accepted
(the
greetings) and (Spider) asked him (Monkey) to come and drink water.'  [LB 011]
 \z



\ea\label{GRM-clause-conn-(a)ka-2-subj} 

\ea\label{GRM-clause-conn-ka-2-subj} 
\gll  [dɪ̀  ɪ̀    wáà    párà]  ká [kìrìmá  wà 
dʊ́mɪ́ɪ́]\\
{\conn} {\sc 2sg} {\ingr}  farm {\conn} 
tstse.fly.{\pl}  {\ingr}  bite.{\sc 2sg}\\
\glt  `When  you are doing the weeding and  tsetse flies bite
you (...)' [CB 003]

\newpage
\ex\label{GRM-clause-conn-aka-2-subj} 
\gll  [dɪ́   námùŋ tɪ́  bɪ́ wàà   jɪ́rà kɪ̀ŋkùrùgíé
ŋmɛ́ŋtɛ́l sɔ́ŋ] àká [ɪ̀ jɪ̀rà   kéŋ̀]\\
{\comp}   anyone     {\neg}   {\itr} {\ingr}  
   call   enumeration    
eight name   {\conn}   {\sc 2sg}   call  {\dxm}\\
\glt  `(The monkey said:  ``They said) that anyone should not say the
number eight and you have said the number eight''.' [LB
017] 

\z 
 \z


Secondly, the connectives  {\sls ka} and {\sls aka}  may encode a `logical' or `natural' sequence of events.   For instance, in (\ref{GRM-clause-conn-ka-1-subj}), someone traveling (or coming from the road) expects to be offered water to drink after the greetings are exchanged. The  connectives  {\sls ka} and {\sls aka} appear to suggest a causal relation between interdependent clauses. In (\ref{GRM-clause-conn-a}), it is the counting of the mounds which caused Spider to be confused, which can be seen as an unexpected outcome.  

% % % but the left-hand conjunct in the {\sls  (a)ka}-construction in 
% % % (\ref{GRM-clause-conn-a})  behaves somehow  like a subordinate clause. Once 
% % % again, the fact that the subordinate causal  clause is a VP adjunct, thus a 
% % % syntactic constituent of the matrix sentence, explains its behaviour 
% % concerning % \isi{focus} marking structures. The impossibility of ocusing the 
% second % segment of a % Justification construction through \isi{negation}  reinforces 
% its  % syntactic % peripheral status.



\ea\label{GRM-clause-conn-a} 
\gll   ʊ́ʊ́wà  ŋmɛ́ŋtɛ́l   já  kùrò àkà bùtì\\
{\sc 3sg.emph}   spider do count  {\conn}  confuse\\
\glt  `(Because) he himself (Spider) did count and he became confused'
[LB  007]
 \z


Nevertheless the connectives  {\sls ka} and {\sls aka}  can 
introduce a clause denoting an event
which is not necessarily related to the event of the previous clause. It looks
as if  the connectives in (\ref{GRM-clause-conn-ka-then}) are used to
integrate an unrelated event to  the overall situation.   

\ea\label{GRM-clause-conn-ka-then} 
\gll [nànsá sú bárá múŋ̀.] ká [dʊ̃́ʊ̃́ tɪ̀ŋ ŋmá dɪ́ kɪ̀ndɪ́gɪ́ɪ́ 
dʊ́ɔ́ à dɪ̀ā nɪ́]\\
 meat fill place all  {\conn} python {\sc art} say {\comp} something is  {\sc art}
house {\postp}\\
\glt `Meat was all over the place. Then,  Python said: ``there is something in
the room''.' [PY 069]
 \z

\ea\label{GRM-clause-conn-ka-transition} 
\gll  [à  bìpɔ̀lɪ́ɪ́  sìì     tʃɪ́ŋá]  àká   [ŋmá,  
ámɪɛ̃̀ɛ̃̀   ɪ̀      ɲɪ́ná] {...}\\
{\sc art} young.man   raise   stand {\conn} said,  {\adv}   {\sc 2sg.poss} 
father  {...}\\
\glt `The young man stood up and said:  ``So, when your father (...)''.' [CB 010]
 \z

Notice that the `standing' and `saying' events in (\ref{GRM-clause-conn-ka-transition}) are strictly transitional, but this is not the case in (\ref{GRM-clause-conn-ka-then}). The \isi{connective} {\sls ka} in (\ref{GRM-clause-conn-ka-then}) opens a sentence which marks a shift from a scene description (i.e.  `there was meat all over the place') to a character's intervention (i.e. `Python speaking').  Perceived event integration  seems to be what predicts the choice between {\sls ka} and {\sls aka}, but no firm conclusions can be drawn. 


\ea\label{GRM-ev-int-1} 

\ea\label{GRM-ev-int-1-ka} 
\gll kàlá káálɪ́ jàwá ká jàwà múrò rō.\\
K. go market {\sc conn} bought rice {\sc foc}\\
\glt  `Kala went to the market and bought rice.'

\ex\label{GRM-ev-int-1-aka} 
\gll  kàlá káálɪ́ jàwá àká pɪ̀ɛ̀sɪ̀ bùlèŋà tíísà.\\
K. go market {\sc conn} ask B. station\\
\glt  `Kala went to the market and asked for the Bulenga station.' 

\z 
 \z


The cause-consequence relation in (\ref{GRM-ev-int-1-ka}) may be seen as `tighter' than the relation between the clauses in (\ref{GRM-ev-int-1-aka}). Buying items is a stronger effect of going to the market than looking for a location; market is where buying items happens. The examples in (\ref{GRM-ev-int-1}) thus suggest that {\sls aka} connects less-integrated clauses. 
 
\ea\label{GRM-ev-int-2} 
 
\ea\label{GRM-ev-int-2-ka-1} 
\gll ʊ̀ zʊ́ʊ́ dɪ̀á ká dí sɪ̀ɪ̀máá rā.\\
 {\sc 3sg} enter house {\sc conn} eat food  {\sc foc}\\
\glt `She entered the house and ate the food.' (expected)

\ex\label{GRM-ev-int-2-aka} 
\gll  ʊ̀ zʊ́ʊ́ dɪ̀á àká vrà sɪ̀ɪ̀máá rā.\\
{\sc 3sg} enter house {\sc conn} knock food  {\sc foc}\\
\glt `She entered the house and knocked the food over.' (unexpected)

\ex\label{GRM-ev-int-2-ka} 
\gll ʊ̀ zʊ́ʊ́ dɪ̀á ká vrà sɪ̀ɪ̀máá rā.\\
{\sc 3sg} enter house {\sc conn} knock food  {\sc foc}\\
\glt `She entered the house and knocked the food over.' (knowledge of intention)

\z 
 \z

Commenting on each hypothetical situation in which (\ref{GRM-ev-int-2}) may be 
uttered, one consultant agreed 
that in (\ref{GRM-ev-int-2-ka}) the intention of the subject's referent are 
known and confirmed in the second clause, which is not the case in   
(\ref{GRM-ev-int-2-aka}). The  events expressed in the second clause in  
(\ref{GRM-ev-int-2-ka-1}) and  (\ref{GRM-ev-int-2-ka}) are perceived as more 
predictable given the first clause (and world knowledge) than the event 
expressed in the second clause  in (\ref{GRM-ev-int-2-aka}).\footnote{The 
connectives {\sls aŋ}  and {\sls ka} in \ili{Pasaale} \citep{mcgi99} offer a good 
baseline for comparison.}  




\paragraph{Connective dɪ}
\label{GRM-clause-coord-di}
The clausal \isi{connective} {\sls dɪ̀} `and' or `while'  is homophonous with a
\isi{complementizer} particle (Section \ref{GRM-clause-comp-di}), a \isi{connective} used in
conditional constructions (Section \ref{GRM-clause-subord}),   and a preverb
p\isi{article} signaling imperfective aspect (Section \ref{sec:GRM-ipfv-part}). It
connects two clauses which encode different events, yet these events must be
interpreted as occurring simultaneously.  A clause introduced by the \isi{connective}
{\sls dɪ̀} has no overt subject, instead the subject is inferred, as it has the
same referent as the subject of the preceding clause. Two examples are
provided in (\ref{GRM-clause-conn-di}). 

% Some examples of the clausal \isi{connective}
% {\sls dɪ̀} in the corpus may be argued to convey intention or purpose, e.g.
% (\ref{GRM-clause-conn-di-3}). 

\ea\label{GRM-clause-conn-di}

\ea\label{GRM-clause-conn-di-vp22.4.9.}
\gll líé ʊ̀ kààlɪ̀ dɪ̀ wá.\\
  {\q} {\sc 3sg} go {\conn} come\\
\glt  ` Where is he coming from?' ({\it lit.} where he left and  come)

\ex\label{GRM-clause-conn-di-vp47.2.9.}
\gll kpá sɪ̀ɪ̀má háŋ̀ dɪ̀ káálɪ̀.\\
 take   food {\dem}  {\conn} go\\
\glt  ` Take this food away! ({\it lit.} take this food and go)


\z 
 \z



\subsubsection{Subordination}
\label{GRM-clause-subord}

The morpheme {\sls tɪ̀ŋ} is mainly used as a  determiner in noun phrases  (see Section 
\ref{sec:GRM-np-def}).  However, there are instances where the discourse following {\sls tɪ̀ŋ} must 
be treated as subordinated and related to the noun phrase of which  {\sls tɪ̀ŋ} is part. One may 
argue that the morpheme {\sls tɪ̀ŋ} can function as a relativizer. 

\ea\label{GRM-clause-subord-rel} 
\gll  kúrò {\ob}{\ob}píé tɪ́ŋ{\cb}$_{NP}$  ʊ̀$_{i}$ kà tɔ́ à kùò nɪ́ kéŋ̀{\cb}$_{NP}$  tɪ̀ɛ̀ʊ́$_{i}$\\
 count yam.mound.{\sc pl}  {\sc art} {\sc 3sg} {\egr} cover {\sc art} farm {\postp} 
{\dxm} give.{\sc 3sg}\\
\glt  `(Spider$_{i}$ asked Buffalo to) count  for him$_{i}$ the yam mounds which
he$_{i}$ covered at the farm.' [LB 006]
 \z


In (\ref{GRM-clause-subord-rel}), the phrase {\sls ʊ̀ kà tɔ́ à kùò nɪ́ kéŋ̀} is (i) in apposition to the noun phrase {\sls píé tɪ́ŋ}, and (ii) in a subordination relation with the noun phrase {\sls píé tɪ́ŋ}. The low tone {\sls kà} frequently appears in  subordinated clauses with {\sls tɪ̀ŋ}  (see example \ref{GRM-prev-SVC-wa} in Section \ref{sec:GRM-EVC-egr-ingr}). In a conditional construction like the one in (\ref{GRM-clause-subord-if-vp46.11}), the subordinate clause is headed by the particle {\sls dɪ̀},  whereas the main clause follows the subordinate clause. 

\ea\label{GRM-clause-subord-di}
\ea\label{GRM-clause-subord-if-vp46.11}

\gll dɪ̀ ǹ̩ fɪ̀ tú kààlɪ̀ dē, bà kàá  tùgúǹ nō.\\
   {\sc conn}  {\sc 1sg} {\mod} {go.down} go {\sc dxl} {\sc 3pl.h+} {\fut}
beat.{\sc 1sg} {\foc}\\
\glt  `If I were to go down there, then they will beat me.' 

\ex\label{GRM-clause-subord-proverb}
\gll dɪ̀ ɪ̀ zíŋ wā zɪ̀ŋà,  ɪ̀ wàá kɪ̀ŋ gáálɪ́ díŋ nɪ̄.\\
  {\sc conn} {\sc 2sg}  tail {\ingr} long  {\sc 2sg} {\neg.\fut} {\abi} be.over fire {\postp}\\
\glt   `If you have a long tail, you cannot cross fire.'

 \z 
 \z
 
 Proverbs are typically  conditional constructions.  An example is given in (\ref{GRM-clause-subord-proverb}).
 
 \ea\label{GRM-clause-conces-consec}

\ea\label{GRM-clause-conces}

\gll ʊ̀ wááwáʊ́ {ànáàmùŋ} dɪ́ ʊ̀ wɪ́ɪ́ʊ̀.\\
    {\sc 3sg} come.{\pfv.\foc} {\conn} {\comp}  {\sc 3sg} sick.{\foc}\\
\glt  `He came in spite of his illness.' 


\ex\label{GRM-clause-consec-1}

\gll ǹ̩ wà kpágá sákɪ̀r, {àɲúúnɪ̀} ǹ̩ dɪ̀ válà nã̀ã̀sá.\\
{\sc 1sg} {\neg} have bicycle {\conn} {\sc 1sg}   {\ipfv} walk
leg.{\pl}\\
\glt `I don't have a bicycle, therefore I am walking.'

 \z 
 \z
 
 The subordinate clause of a concessive construction is introduced by the expression  {\sls anɪ amuŋ} [{\sls ànáàmùŋ}] ({\it lit.} and-all) `despite',  `in spite of', `although' or `even though'. A subordinate clause  which conveys a consequence or a justification of the proposition in the main clause  is introduced by the expressions {\sls a ɲuu nɪ}  [{\sls àɲúúnɪ̀}] or {\sls a wɪɛ} [{\sls àwɪ́ɛ́}] ({\it lit.} the-head-on and  the-matter)  respectively,  `therefore' or `because'. Examples are shown in (\ref{GRM-clause-conces-consec}).


\paragraph{Complementizer dɪ}
\label{GRM-clause-comp-di}

Example (\ref{GRM-clause-comp-inds}) shows that the \isi{complementizer} {\sls dɪ̀}
introduces indirect speech. 

\begin{exe}
 \ex\label{GRM-clause-comp-inds}
 \gll kùórù   bìnɪ̀hã́ã̀ŋ    ŋmá   dɪ́  ``ɛ̃̀ɛ̃́ɛ̃́ɛ̃̀''.\\
 chief   young.girl   say   {\comp} yes\\
 \glt  `The chief's daughter answered ``yes''.'  [CB 011] 
 \z

Direct speech is usually introduced by a speech verb only, e.g. {\sls ŋma (tɪɛ)} `say (give)',   {\sls tʃagalɪ} `teach, show, indicate', {\sls hẽsi} `announce', etc.  This is shown in (\ref{GRM-clause-comp-ds}) with {\sls hẽsi} `announce'.

\begin{exe}
 \ex\label{GRM-clause-comp-ds}
 \gll tɔ́ʊ́tɪ́ɪ́ná ŋmá dɪ́ bá hẽ́sí má ká pàrà kùó.\\
 landlord say  {\comp} {\sc 3pl.g}b  announce {\sc 2pl} {\egr} farm farm\\
 \glt  `The landowner says that they announced:  ``You go and work at the
farm''.' 
 \z
 
 %ba should not be high tone

In (\ref{GRM-clause-comp-2-int}),  the \isi{complementizer} {\sls dɪ̀} introduces a clause which conveys the intention of the event in the main clause. In a literal sense, the husband  {\sls lala}  `open', in the sense of waking up,  the wife in order to  have her {\sls sii} `raise up'. 
\ea\label{GRM-clause-comp-2}

 \ea\label{GRM-clause-comp-2-int}
\gll  tʃʊ̀ɔ̀sá   pɪ́sɪ̀, ʊ̀   báàl tɪ̀ŋ té lálá à hã́ã̀ŋ    dɪ̀  ʊ́   síí  dùò nɪ̀.\\
  morning scatter    {\sc 3sg.poss}   husband  {\sc art} early  wake.up 
{\sc art}  wife   {\comp}   {\sc 3sg}   raise.up asleep  {\postp}\\
 \glt  `Early in the morning her husband woke up the wife from sleep.' ({\it
lit.} that she must stand up)  [CB 030]

 \ex\label{GRM-clause-comp-2-pur}
 \gll ʊ̀ káálɪ́ (dɪ́) ʊ́ʊ́ ká ɲʊ̃̀ã̀ nɪ̄ɪ̄.\\
{\sc 3sg} go   {\comp}  {\sc 3sg}   {\egr} drink  water\\
 \glt  `He went to have a drink of water.' 
 
  \z 
 \z
 
 In (\ref{GRM-clause-comp-2-pur}) it is shown that  purpose (or intention) can be encoded when {\sls dɪ̀} introduces the goal. In the latter case, however, consultants say that the \isi{complementizer} {\sls dɪ̀} is optional.


 \paragraph{Clause apposition}
 \label{GRM-dep-comp-clause}

Example (\ref{GRM-clause-appo}) shows that a desire can be encoded by two clauses in apposition. In this example the pronominal subject of the final clause carries high tone (see Section \ref{sec:GRM-desiderative-mood} on desiderative). 
 
 
 \begin{exe}
 \ex\label{GRM-clause-appo}
 \gll jà búúrè nɪ̄ɪ̄ rā já ɲʊ̃́ã̀.\\
{\sc 1pl} want water {\foc} {\sc 1pl} drink\\
 \glt  `We want some water to drink.' 
 \z



\subsection{Adjunct adverbials and postposition}
\label{sec:GRM-adverbial}

The notion `\isi{adverbial}' is used in the sense of  `modifying a predicate', that is, adding  information to  a state of affairs. An adverbial is an expression, clause or non-clause, which is not an argument of  the main predicate and is positioned at the periphery in an adjunct constituent  ({\sc ajc}). The clause frame in (\ref{ex:GRM-clause-frame}) is repeated in (\ref{ex:GRM-clause-frame-1}). 

\begin{exe}
\ex\label{ex:GRM-clause-frame-1}
 {\sc s|a}  $+$ {\sc p} $\pm$ {\sc o} $\pm$ {\sc ajc} 
\end{exe}

Adjuncts are usually found following the core constituent(s), but may also be found at the beginning of a clause. As shown in (\ref{ex:GRM-pre-adj}), reference to time may be found at the beginning of a clause before the subject.


\ea\label{ex:GRM-pre-adj}
{{\sc ajc} $+$ {\sc s}  $+$ {\sc p}  $+$  {\sc o}}\\
\glll  {{\ob}tʃʊ̀ɔ̀sá  pɪ̀sɪ̀{\cb}}   {à bìpɔ̀lɪ́ɪ̀}  kpá {ʊ̀ páŕ}\\
 {\sc ajc}  {\sc s}  {\sc p} {\sc o}\\
{morning   scatter}   {{\sc art} young.man} take {{3.\sg.\poss} hoe}\\
  \glt  `The following day the young man took his hoe along ...' [CB 005]
\z

 In Section \ref{sec:GRM-compar-ct}, the dubitative construction was 
identified with the expression {\sls àbɔ́nɪ̃́ɛ̃́nɪ́}  `perhaps'  opening the 
clause. There are other constructions in which temporal, locative, manner, or 
tense-aspect-mood meaning is signaled by the presence of an adjunct adverbial  
initially that introduces new information.  

\ea\label{ex:GRM-phra-adv}

\ea\label{ex:GRM-phra-adv-time}{\rm Temporal}\\
\gll {\ob}tàmá fìníì{\cb} ʊ̀ fɪ̀ sʊ́wá.\\
few little {\sc 3sg} {\mod} die\\
\glt `A little longer and she would  have died.'


\ex\label{ex:GRM-phra-adv-evi}{\rm Evidential}\\
\gll {\ob}wɪ́dɪ́ɪ́ŋ ná{\cb} dɪ́ ʊ̀ náʊ́ rā.\\
truth {\foc} {\comp} {\sc 3sg} see.{\sc 3sg} {\foc}\\
\glt  `It is certain that he saw him.'

\z 
 \z



In (\ref{ex:GRM-phra-adv-time}), the phrase {\sls tama finii} `a little'  is not inherently 
temporal, but must be interpreted as such in the given context. In 
(\ref{ex:GRM-phra-adv-evi}) the verbless clause {\sls wɪdɪɪŋ na}  can be seen as 
adding an illocutionary force; it additionally signifies that the speaker has 
evidence and/or wish to convince the addressee about the proposition. In the  
next sections,  temporal and manner adverbials, then the \isi{postposition} {\sls nɪ} 
and the oblique phrase are discussed.


% 
% % They are glossed {\sc dem} (i.e. \isi{adverb} locative), {\sc dxm}  (i.e.
% % \isi{adverb} manner) and {\sc advt}  (i.e. \isi{adverb} time) respectively. Examples are
% % provided below.

\subsubsection{Temporal  adjunct}
\label{sec:GRM-temporal-adjunct}


A temporal nominal adjunct   is an expression which typically indicates when  an event occurs.

\ea\label{ex:GRM-adj-temp-adv}

\ea\label{ex:GRM-adj-temp-adv-LB5}
\gll {\ob}àwʊ̀zʊ́ʊ́rɪ̀  dɪ́gɪ́ɪ́{\cb}     kɔ̀sánã́ɔ̃́   válá \\
the.day one    buffalo walked \\
\glt `One day a buffalo walked (and greeted the spider) (...)'. [LB 005

\ex\label{ex:GRM-adj-temp-adv-CB17}
\gll {\ob}dénɪ̀{\cb},   {\ob}sáŋà   dɪ́gɪ́ɪ́{\cb}   à   hã́ã̀ŋ já   pàà à  báál   zōmō\\
thereupon time one   {\sc art}  wife  {\hab}   take.{\pl} {\sc art} husband insult.{\pl} \\
\glt `Then, during their life, it happened on one occasion that the woman
did insult  the man (...)'.  [CB 017]

\ex\label{ex:GRM-adj-temp-adv-everyday}
\gll  ǹ̩ já kààlɪ̀ ʊ̀ pé rè {\ob}{tʃɔ̀pɪ̀sɪ̀} bɪ́ɪ́-mùŋ{\cb}.\\
 {\sc 1sg} {\hab} go {\sc 3sg} end {\foc} day.break {\itr}-all\\
\glt `I visit him every day.'

\ex\label{ex:GRM-adj-temp-adv-nownow}
\gll {\ob}làɣálàɣá háǹ nɪ̄{\cb} ǹ̩ kʊ̀tɪ̀ à ʔã́ã́ pétí.\\
{\ideo}.fast {\dem} {\postp} {1.\sg} {skin} {\sc art} bushbuck  finish\\
\glt `I  just finished skinning the bushbuck.'


\z 
 \z


Some expressions tagged as temporal nominal are treated as complex, though opaque, expressions. For 
instance,  {\sls awʊzʊʊrɪ} is translated into  `that day' in (\ref{ex:GRM-adj-temp-adv-LB5}), 
but the forms {\sls wʊsa} `sun' and {\sls zʊʊ} `enter'  are perceptible. The phrase {\sls 
làɣálàɣá háǹ nɪ̄} in (\ref{ex:GRM-adj-temp-adv-nownow}) literally means `now.now this on' 
({\ideo} {\dem} {\postp}), but `only a moment ago'  is a better translation.  Similarly, {\sls 
denɪ}, analysed as the spatial demonstrative {\sls de} and the potsposition {\sls nɪ} and  
translated into English as `thereupon', `after that', `at that point', or simply `then',  is  a 
temporal nominal, but usually functions as a \isi{connective}.  It is mainly used at the beginning of a 
sentence to signal a transition  between the preceding  and the following situations; 
(\ref{ex:GRM-adj-temp-thereupon}) suggests a transition  indicating what happens `next' or 
`afterward'.


\ea\label{ex:GRM-adj-temp-thereupon}
\gll dénɪ̄   rè,     ʊ̀ʊ̀      hã́ã́ŋ   tɪ̀ŋ ŋmá   dɪ́  ``ààí, (...)''.\\
 thereupon   {\foc}  {\sc 3sg.poss}  wife  {\sc art}  say   {\comp} no {}\\
\glt `Then, the wife said: ``No, (I won't
say anything to my father)''.' [CB 036]
\z

\subsubsection{Manner adjunct}
\label{sec:GRM-manner-adv}

A manner  expression describes the way the event denoted by the verb(s) is carried out. Manner expressions tend to appear at the right periphery of an utterance. The examples in (\ref{ex:GRM-adj-mann}) illustrate the meaning and distribution of  manner expressions.

\ea\label{ex:GRM-adj-mann}

\ea\label{ex:GRM-adj-mann-carefully}
\gll dɪ̀ sã́ã́ bʊ̃̀ɛ̃̀ɪ̃̀bʊ̃̀ɛ̃̀ɪ̃̀.\\
{\comp} drive {\ideo}.carefully\\
\glt `Drive carefully.'

\ex\label{ex:GRM-adj-mann-slowly}
\gll dɪ̀ ŋmà bʊ̃̀ɛ̃̀ɪ̃̀bʊ̃̀ɛ̃̀ɪ̃̀.\\
{\comp} talk {\ideo}.slowly\\
\glt `Talk slowly.'

\ex\label{ex:GRM-adj-mann-lighly}
\gll ʊ̀ tʃɔ́jɛ̄ kààlɪ̀ félfél.\\
 {\sc 3sg} run.{\pfv} go {\ideo}.lightly\\
\glt `She ran away lightly (manner of movement, as a light weight
entity).'

\ex\label{ex:GRM-adj-mann-silently}
\gll  ǹ̩ kàà wáá dɪ̀ à   hã́ã́ŋ  sáŋà   tʃérím.\\
{\sc 1sg} {\sc ipfv} come {\sc comp} {\sc art} woman sit {\ideo}.quietly\\
\glt `When I was coming, the woman sat quietly.' 
\z 
 \z

It is common for an \is{ideophone}\isi{ideophone} to function as a manner expression 
 (Section \ref{sec:GRM-onoma}). One could argue that  all the manner expressions in 
(\ref{ex:GRM-adj-mann}) are ideophones, i.e. they display reduplicated forms and {\sls tʃerim} is 
one of a few words which ends with a bilabial nasal. The examples in 
(\ref{ex:GRM-adj-mann-ideo-adv}) show the repetition of two expressions; one is an 
\is{ideophone}\isi{ideophone}, i.e. {\sls kaŋkalaŋ} `crawl of a snake', and the other  a  reduplicated  
manner expression,  i.e. {\sls  lagalaga} `quickly' from {\sls laga} `now'.  The formal 
repetition depicts  the motion occurring with great speed  and the inceptive sense  of  {\sls 
kpà}  marks the initial stage of the activity.


\ea\label{ex:GRM-adj-mann-ideo-adv}


\ea\label{ex:GRM-adj-mann-ideo}
\gll à bààŋ kpá {kàŋkàlàŋ kàŋkàlàŋ kàŋkàlàŋ}.\\
{\conn} just take {\ideo}.rapidly\\
\glt `(She was after the python) but (he) started to crawl away like a shot.' 
(PY-137)

\ex\label{ex:GRM-adj-mann-adv}
\gll  kà bààŋ kpá {làɣàlàɣà làɣàlàɣà}.\\
{\conn} just take {\ideo}.quickly  \\
\glt `(She) started to (walk) quickly.'

\z 
 \z


The manner \isi{adverbial} {\sls kɪŋkaŋ} `abundantly',  which is composed of the \isi{classifier} {\sls kɪn} and the verb {\sls kana} `abundant',  typically quantifies or intensifies the event and always comes after the word encoding the event.  Notice in (\ref{ex:GRM-adj-mann-alot-v})  and (\ref{ex:GRM-adj-mann-alot-n})   that {\sls kɪŋkaŋ} follows a verb and a nominalized verb respectively. However, in (\ref{ex:GRM-adj-mann-alot-quant}), {\sls kɪŋkaŋ} does not function as a manner \isi{adverbial} but as a \isi{quantifier}.

\ea\label{ex:GRM-adj-mann-alot}
\ea\label{ex:GRM-adj-mann-alot-v}
\gll gbɪ̃̀ã́     ɪ̀    jáárɪ́jɛ́  kɪ́ŋkāŋ    nà
(...)\\
monkey   you  unable.{\pfv}  {\dxm} {\foc} {}\\
\glt `Monkey, you are so incompetent, (...).' [LB 016]

\ex\label{ex:GRM-adj-mann-alot-n}
\gll dúó tʃʊ̄ɔ̄ɪ̀ kɪ́ŋkāŋ wà wíré.\\
asleep lie.{\nmlz} {\dxm} {\neg} good\\
\glt `Sleeping too much is not good.'

\ex\label{ex:GRM-adj-mann-alot-quant}
\gll kùórù   kùò tɪ́ŋ   kà   kpágá kìrìnsá  kɪ́ŋkāŋ, dé rē jà kààlɪ̀\\
 chief farm {\sc art} {\rel} have tsetse.fly.{\sc pl}  {\quant}.many {\sc dxl} {\sc foc} {\sc 1pl} go\\
\glt   `The chief's farm that has  many tsetse flies, there we went.' 

\z 
 \z

\subsubsection{Oblique phrase}
\label{sec:GRM-obl-phrase}

The oblique phrase \is{oblique phrase}  is an element of a clause realized as a postpositional phrase. It usually follows the verbal predicate.  In Section \ref{sec:SPA-postp},  it is claimed that the \isi{postposition} {\sls nɪ} (i) identifies an oblique phrase, (ii) conveys that the oblique phrase contains the ground object  (Section \ref{sec:SPA-blc}), and (iii) follows its complement. While {\sls nɪ} mainly appears in sentences expressing localization, the \isi{postposition} can also be found when there is no  reference to space. 

For instance, in Section \ref{sec:GRM-manner-adv},  the \isi{connective} {\sls denɪ} (i.e. {\dem}+{\postp}) is said to signal a temporal transition and not a spatial one.  It is also analysed in adverbials and connectives: {\sls a-bɔnɪ̃ɛ̃-nɪ} `maybe, perhaps', {\sls a-ɲuu-ni} `therefore', {\sls buŋbuŋ-ni} `at first', etc. These expressions do not have a purely locative function, but are rather used as clausal adjuncts or to introduce logical conclusion (see Sections   \ref{GRM-clause-subord} and \ref{sec:GRM-adverbial}). 


\ea\label{ex:GRM-obl-obj-no-spa}

\ea\label{ex:GRM-obl-obj-no-spa-foc-2}
\gll bàáŋ ɪ̀ fɪ́ kàà sʊ́ɔ́gɪ̀ [tʃʊ̀ɔ̀sá tɪ́n nɪ̄].\\
 {\q} {\sc 2pl} {\pst} {\egr} crush  morning {\sc art} {\postp}\\
\glt  `What were you crushing this morning?' 

\ex\label{ex:GRM-obl-obj-no-spa-foc-1}
\gll ʊ̀ ɲʊ̃́ã́  [làɣálàɣá nɪ̄].\\
   {\psg} drink {\sc ideo}.fast {\postp}\\
\glt  `He drinks quickly.' 

\ex\label{ex:GRM-obl-obj-no-spa-foc}
\gll à kùórù ŋmá dɪ́ ʊ̀ bááŋ káá sīī [ǹ̩ ní] rē.\\
{\sc art} chief say {\comp} {\sc 3sg.poss} temper {\egr} raise {\sc 1sg} {\postp}
{\foc}\\
\glt  `The chief told me that he was very angry with me.' 

\z 
 \z

 The examples in (\ref{ex:GRM-obl-obj-no-spa}) illustrate some of the non-spatial uses of the oblique phrase headed by {\sls nɪ}. The \isi{postposition}'s complement is a temporal nominal phrase in (\ref{ex:GRM-obl-obj-no-spa-foc-2}), an \isi{ideophone} in  (\ref{ex:GRM-obl-obj-no-spa-foc-1}),  and a personal \isi{pronoun} in (\ref{ex:GRM-obl-obj-no-spa-foc}).

\subsubsection{Postposition {\sls nɪ} and (non-)locative adjunct}
\label{sec:SPA-postp}

The ground object in localization is found in an oblique phrase  (see Section \ref{sec:GRM-obl-phrase} for oblique phrase and \ref{sec:SPA-blc} for basic locative construction), therefore the  \isi{postposition} {\sls nɪ} is present irrespective of the locative verb involved or whether or not a \isi{relational noun} occurs. Only a  few exceptions can be found,   and they are systematically accounted for by two factors: (i) non-locative and transitive verbs  do not co-occur with {\sls nɪ}, e.g. {\sls tɔ} `cover', {\sls kpaga} `have' and {\sls su} `fill',  and (ii) some situations are described using an intransitive clause, e.g.   {\sls à bónsó tʃíégìjō} `the cup is broken'  [TRPS 26].  In describing the illustrations of the TRPS, \citet[370]{Amek06} showed that it is the  verb {\sls le}, glossed  `be at',  in \ili{Ewe} which is used in the majority of the sentences.   The translation of  \ili{Ewe} {\sls le} to Chakali would then be equivalent to {\sls dʊa {\rm NP} nɪ}.\footnote{The \ili{Ewe} verb {\sls le} may also function as predicator of qualities \citep[373]{Amek06}. In Chakali,  it was shown  in Sections \ref{sec:GRM-ident-cl} and \ref{sec:classifier} that   {\sls jaa} predicates over qualities,  not  {\sls dʊa}.}

\newpage 

\ea\label{ex:postp-corres}
 \ea {\rm [[[{\sls à dɪ̀à}]\textsubscript{\tiny NP} {\sls ɲúú}]\textsubscript{\tiny RelnP} {\sls nɪ̀}]\textsubscript{\tiny PP}  `on the roof of the house'}\\
 
 \ex  {\rm [[{\sls à dɪ̀à}]\textsubscript{\tiny NP} {\sls nɪ̀}]\textsubscript{\tiny PP}   `in/at the house'}\\
 
  \ex  {\rm [[{\sls báŋ̀}]\textsubscript{\tiny NP} {\sls nɪ̀}]\textsubscript{\tiny PP}   `here'}\\
 
  \ex  {\rm [[{\sls dé}]\textsubscript{\tiny NP} {\sls nɪ̀}]\textsubscript{\tiny PP}   `there'}\\
  
   \ex  {\rm [[{\sls ʊ̀}]\textsubscript{\tiny NP} {\sls nɪ̀}]\textsubscript{\tiny PP}    `at/on/in him/her/it'}\\
 
\z
\z


As shown  in (\ref{ex:postp-corres}), the  \isi{postposition} always follows its complement (see Section 
\ref{sec:SPA-relnoun} for relational nouns). Since there are no prepositions in the 
language, the abbreviation PP in (\ref{ex:postp-corres}) unambiguously stands for Postpositional Phrase. None of the concepts of proximity, 
contiguity, or containment is encoded in   {\sls nɪ}. The \isi{postposition} does not 
inform the addressee about any of the 
elementary topological spatial notions. It never 
selects particular figure-ground configurations, but must be present for all of 
them. 


\section{Nominal}
\label{sec:GRM-nom}


The term ``nominal"  identifies  a formal and functional  syntactic level and
lexemic level. At the syntactic level, a noun phrase is a nominal  which can
either function as core or peripheral argument.  Its composition may
vary from a single \isi{pronoun} to a noun with modifier or series of
modifiers. At the lexeme level, two categories of lexemes are assumed:
nominal and verbal. These two types correspond roughly to the semantic division
{\it entity} and {\it event}, but do not correspond to the syntactic categories
{\it noun} and {\it verb}. That is because lexemes are assumed to not be
specified for syntactic categories. The diversity  of forms and functions of
nominals is presented below. 


\subsection{Noun phrases}
\label{sec:GRM-noun-phrases}

A noun phrase (NP)  consists of a nominal head, and optionally, its dependent(s).
In this section,  the internal components of noun phrases and the roles these
components have within the noun phrase are described. First,   indefinite and
definite noun phrases are considered. Then, the elements which can be found in
the noun phrase are introduced. 

\subsubsection{Indefinite noun phrase}
\label{sec:GRM-np-indef}

Indefinite noun phrases are used when ``the speaker invites the addressee to 
construe a referent [which conforms with] the properties specified in the term'' 
\citep[184]{Dik97}.  In Chakali, a noun standing alone can  constitute a noun 
phrase (N = NP). Such a noun phrase can be interpreted as indefinite, i.e. the 
noun phrase is a non-referring expression,  or   generic,  i.e. the noun phrase 
denotes  a kind or class of entity  as opposed to an individual.  In rare cases, 
a definite noun phrase can be interpreted from a single noun, i.e. lacking  an 
\isi{article}. Each interpretation is obviously dependent on the context of the 
utterance in which the noun occurs.

\ea\label{GRM-np-type-indef}{\rm  N = NP}\\

 
\ea\label{GRM-np-indef-1}
\gll  kàlá jáwá   pɪ́ɛ́ŋ ná.\\
  Kala buy mat {\sc foc}\\
\glt  `Kala bought a mat.' 


\ex\label{GRM-np-indef-2}
\gll  dʒɛ̀tɪ̀ kɪ̀m-bɔ́n  ná.\\
  {lion.{\sc sg}} {\sc clf}-dangerous.{\sc sg} {\sc foc}\\
\glt  `A lion is dangerous.' 



\z 
 \z

In (\ref{GRM-np-type-indef}),  the noun phrase {\sls  pɪɛŋ}  describes any mat
and
is interpreted as a novelty in the addressee's knowledge of Kala, while
{\sls dʒɛtɪ} describes the entire class of lions. 


\ea\label{GRM-np-type-indef-2.1}

 \ea\label{GRM-np-type-indef-2.2}
\gll pɪ́ɛ́sɪ̀ dɪ́gɪ́ɪ́ à búkù jògùló.\\
 ask one {\sc art} book price\\
\glt  `Ask someone the price of the book.' 

 \ex\label{GRM-np-type-indef-2.3}
\gll nàdɪ́gɪ́ɪ́ búmó zʊ̀ʊ̀ ɪ̀ɪ̀ dɪ̀à háŋ̀ ká bà kpá tɪ̄ɛ̄ɪ̄.\\
 person.one precede enter {\sc 2sg} room {\sc dem} {\sc conn} {\sc 3pl.h+} take give.{\sc 2sg} \\
 \glt  `Someone was in your room before they gave it to you.' 
\z 
 \z

The examples in  
(\ref{GRM-np-type-indef-2.1})  show that noun phrases  containing the
\isi{numeral} {\sls dɪgɪɪ} `one'  may be translated as
English `a certain', `one of them', or `someone'.


\subsubsection{Definite noun phrase}
\label{sec:GRM-np-def}

Definite noun phrases are employed when ``the speaker invites the addressee to
identify a referent which he (the speaker) presumes is available to the
addressee'' \citep[184]{Dik97}.  Proper nouns are assumed to be definite on the basis that they 
are identifiable by both the speaker and the addressee. A definite noun phrase may consist of  a 
single
\isi{pronoun} (pro = NP), as
shown in (\ref{GRM-np-type-pro}).

\begin{exe}
 \ex\label{GRM-np-type-pro}{\rm pro = NP}\\
\gll  ʊ̀  sʊ́wáʊ́.\\
   {\sc 3sg} die\\
\glt  `She died.'
\z


A \isi{possessive} noun phrase is always definite. A possessive \isi{pronoun} followed by a
noun is analysed as a succession of a noun phrase and a noun. Thus,  the noun
phrase in (\ref{GRM-np-type-pro-n})  is analysed as a sequence of the noun
phrase {\sls ʊ} and the  noun  {\sls mãã} (pro + N = NP). 


\begin{exe}

 \ex\label{GRM-np-type-pro-n}{\rm pro + N = NP}\\
\gll   ʊ̀ mã̀ã̀ ŋmá dɪ́ ``ői̋''.\\
   {\sc 3sg.poss} mother say  {\sc comp} {\sc interj}\\
\glt  `Her mother said, ``Oi!''.'
\z


The treatment  of possessive noun phrase is   motivated  by the possibility of  recursion of  an 
attributive possession relation.  The complex stem noun {\sls pàbīī} (< {\sls par-bii},  
hoe-seed)  `hoe blade' is the head in the three possessive noun phrases  {\sls súgló pàbīī} 
`Suglo's hoe blade',    {\sls súgló ɲɪ̄nā pàbīī} `Suglo's father's hoe blade', and {\sls 
súgló ɲɪ̄nā bɪ́ɛ́rɪ̀ pàbīī}  `Suglo's father's brother's hoe blade'.  Notice that in these 
examples the nominal head consists of the right-most element in the noun phrase, e.g.  
[[[[súgló]$_{NP}$ [ɲɪ̄nā]$_{N}$]$_{NP}$ [bɪ́ɛ́rɪ̀]$_{N}$]$_{NP}$ [pàbīī]$_{N}$]$_{NP}$. 
Section \ref{sec:GRM-com-stem-noun} discussed complex stem nouns. 




\paragraph{Articles {\it a} and  {\it  tɪŋ}}
\label{sec:GRM-np-def-articles}

There are two articles \is{article} in Chakali:   {\sls à} (glossed {\sc art1}) and {\sls tɪ̀ŋ}  (glossed {\sc art2}).  The  \isi{article} {\sls à} is translated with the English \isi{article} {\it the}.  It must precede the head noun and cannot co-occur with the \isi{possessive} \isi{pronoun}.  In the context of (\ref{GRM-np-type-det1}), the speaker assumes that the addressee is informed about Kala's interest in buying a mat. 


 \ea\label{GRM-np-type-det1}{\rm  a + N = NP}\\
\gll  kàlá jáwá  à pɪ́ɛ́ŋ ná.\\
  Kala buy {\sc art1}  mat {\sc foc}\\
\glt  `Kala bought the mat.' 
\z


The type of mat,  its colour or the location where Kala bought the mat and so on
are not necessarily shared pieces of information between the speaker and 
addressee
in (\ref{GRM-np-type-det1}).  The only information the speaker believes they
have in common is Kala's interest in purchasing a mat. The \isi{article} {\sls à} is 
treated as a functional word which makes the noun phrase specific but not
necessarily
definite.  When a noun phrase is  specific, the speaker should have a particular
referent in mind whereas the addressee may or may not share this knowledge.


The \isi{article} {\sls tɪ̀ŋ}  (glossed {\sc art2}) can also be seen to correspond to
English {\it the},  but a preferable paraphrase would be `as referred previously' or
 `this (one)'.  The \isi{article} {\sls  tɪ̀ŋ} appears when the speaker knows that the
addressee will be able to identify the referent of the noun phrase. In that 
sense,
the referent is familiar.\footnote{In the givenness hierarchy of
\citet[278]{Gund93}, the status {\it familiar} is reached when ``the addressee
is able to uniquely identify the intended referent because he already has a
representation of it in memory.''}   When {\sls tɪ̀ŋ} follows a noun, the 
referent
must either have been mentioned previously or the speaker and addressee have an
identifiable referent in mind. Thus, compared to the examples
(\ref{GRM-np-type-indef}) and (\ref{GRM-np-type-det1}) above, a proper
interpretation of example (\ref{GRM-np-type-det2}) requires that both the
speaker and addressee have a particular mat in mind. In terms of word order, the
\isi{article}  {\sls à}  initiates the noun phrase and  the \isi{article} {\sls tɪ̀ŋ}  
belongs
near the end of the noun phrase. The \isi{article}  {\sls à}  in (\ref{GRM-np-type-det2}) is optional.
 

\begin{exe}
 \ex\label{GRM-np-type-det2}{\rm   ( {\sls a} +) N +  {\sls tɪŋ} = NP}\\
\gll  kàlá jáwá  [à pɪ́ɛ́ŋ  tɪ́ŋ]$_{NP}$ nā.\\
  Kala buy {\sc art1}  mat {\sc art2} {\sc foc}\\
\glt  `Kala bought the MAT.'
\z

Consider the slight meaning difference between
(\ref{GRM-np-type-det2-a}) and (\ref{GRM-np-type-det2-b}).


\ea\label{GRM-np-type-det2-ab}
% \vspace{-12pt}
 
  \ea\label{GRM-np-type-det2-a}
\gll ɲɪ̀nɪ̃̀ɛ̃́ ɪ̀ ɲɪ́ná kà dʊ́.\\
 {\sc q}.how {\sc 2sg.poss} father {\sc  egr} be\\
\glt  `How is your father?'

  \ex\label{GRM-np-type-det2-b}
\gll ɲɪ̀nɪ̃̀ɛ̃́ ɪ̀ ɲɪ́ná tɪ́ŋ kà dʊ́.\\
   {\sc q}.how {\sc 2sg.poss} father {\sc art2} {\sc  egr} be\\
\glt  `How is your father?'
  
 
\z 
 \z


Both sentences may be translated with `How is your father?'. However, whereas 
the sentence (\ref{GRM-np-type-det2-a}) can request  a general description
of the father (i.e.  physical description, general health, etc.), the sentence
in (\ref{GRM-np-type-det2-b}) asks for a particular aspect of the
father's condition which both the speaker and the addressee are aware of, for
instance the father's sickness. As sketched above, the \isi{article} {\sls tɪŋ}  in
(\ref{GRM-np-type-det2-b}) establishes that a particular disposition of the
father is known  by both the speaker and the addressee,  and the speaker
asks, with the question word {\sls ɲɪnɪ̃ɛ̃} `how',   for details. 

The two  articles {\sls a} and {\sls tɪŋ}  are not in complementary 
distribution. The \isi{article} {\sls tɪŋ} may occur following the head of a possessive noun phrase,
although it is not attested  following a \isi{weak pronoun}. When the articles {\sls 
a} and {\sls tɪŋ} co-occur,  language consultants could omit
the preposed {\sls a}  without affecting the interpretation of the
 proposition. 

While the  two articles  in Chakali are presented under the same heading, they are  believed to be 
of different origin.  Evidence shows that pre-nominal articles are not found in the SWG group, nor 
in \ili{Kasem}  \citep[153]{Bonv88}. Assuming that specificity and definiteness morphemes  always come 
after the noun in \ili{Grusi} languages, and that \ili{Waali} \is{Waali} and \ili{Dagaare} \is{Dagaare} make use of 
an 
 identical pre-nominal \isi{article} {\sls }, the  \isi{article} {\sls à} in Chakali is believed to be a 
contact-induced innovation. However, a preposed \isi{article} in the northwestern languages is alien to the general \ili{Oti-Volta} pattern as well. The phenomenon needs more study to see if a locus for this areal  innovation can be identified. Apparent cognates of  {\sls tɪŋ} are attested in \ili{Grusi}. For 
example   \citet[180]{Bonv88} writes that \ili{Kasem}  {\sls tɪm}  ``sert à thematiser ce qui est 
déjà connu'' (i.e. used to bring up what is already known).  Chakali {\sls tɪŋ} is discussed in 
Section \ref{GRM-clause-subord}  in relation to its role as a relativizer in subordination.


Now that the indefinite and definite noun phrases have been presented, the
subsequent sections introduce the elements which can compose  either  indefinite
or  definite noun  phrases.



% one which encodes specificity and the other
% definiteness. 

\subsection{Nouns}
\label{sec:GRM-noun}

In this section, the elements admitted in the
schematic representation (\ref{sec:GRM-noun-strcut}) are discussed.

\ea\label{sec:GRM-noun-strcut}
{\rm [[{\sc lexeme}]$_{stem}$ - [{\sc noun class}]]$_{n}$}
\z

A stem may have 
nominal or verbal lexeme status. The latter has either a state (i.e. stative) or
a event (i.e. active) meaning.  A stem can be either atomic or complex and a
noun class suffix may be overt or covert.  In a
 process which turns a lexeme into a noun-word,  the noun class provides the
syntactic category {\it noun}. 




\subsubsection{Noun classes}
\label{sec:GRM-noun-classes}

The accepted view is that ``the Gurunsi languages, and indeed all \ili{Gur} languages,
had historically a system of nominal classification which was reflected in
agreement. The third person pronominal forms and other parts of speech were at a
certain time a reflection of the nominal classification''  \citep{nade89}.
 Similar affirmations are present in \citet{Mane69b, Waa71, nade82, nade98,
Tcha07}.  In this section and in Section
\ref{sec:GRM-gender}, it is suggested that
an eroded form of this ``reflection'' is still observable in Chakali.
\citet{brin08c} claims that in Chakali inflectional class
(i.e. noun class) and agreement class (i.e. gender) should be distinguished and
analysed as separate phenomena at a synchronic level.

 The identification of noun classes is based on non-syntagmatic evidence; noun
class is a type of inflectional  affix, independent of agreement
phenomena, where the values of number
and class are exposed. In Chakali, as in all  other SWG  
languages,\footnote{\citet[136]{nade98} state that ``[i]n
\ili{Vagla} most traces
of this [noun-class system where paired singular/\isi{plural} noun affixes correlate
with concording pronouns and other items] system have been lost. The
morphological declensions of nominal pluralization have not yielded to a clear
analysis''.  Even though the authors do not attempt to allot nouns into classes,
Marjorie Crouch's field notes (1963, Ghana Institute for Linguistics, Literacy
and Bible Translation (GILLBT)) present seven classes. Nominal classifications
are proposed for other SWG languages (number of classes for each language in
parenthesis): \ili{Sisaala} of Funsi in \citet{Rowl66} (2), \ili{Sisaala}-\ili{Pasaale} in
\citet{mcgi99} (5) and Isaalo in \citet{Mora06} (4).  The number of classes is 
of
course determined by the linguist's analysis.\label{foot:noun-class}}  the
values are exposed by
suffixes: number refers to either singular or \isi{plural}, and class can be regarded
as phonological and/or semantic features encoded in the lexemes for the
selection
of the proper pair of singular and \isi{plural} suffixes. This will be considered in
Section \ref{sec:GRM-sem-ass-crit}. 



 \begin{table}
 \caption{The five most frequent noun classes \label{tab:GRM-synop-nc}}
   \centering
   \begin{Itabular}{lccccc}

 \lsptoprule
    &  {\sc cl.1} & {\sc cl.2}  & {\sc cl.3} & {\sc cl.4} & {\sc cl.5} 
\\[1ex] \midrule
{\sc sing} & -V&  \O&  \O& -V  & \O\\
{\sc plur} & -sV& -sV & -V & -V  & -nV\\
 \lspbottomrule
   \end{Itabular}
 \end{table}



One method used to identify the noun classes of a language appears in
\citet[23]{Rowl66}. The author writes that ``[t]he nouns in Sissala may be
assigned to groups on the basis of the suffixes for singular and \isi{plural}''. 
 According to this definition, there are nine noun 
classes, of which four are rare.   A synopsis is displayed in Table 
\ref{tab:GRM-synop-nc}, and each
of them is discussed below.\footnote{Some scholars treat each singular type as a class, and each \isi{plural} type as a class. In their terminology a {\it nominal declension} is a singular/\isi{plural} marker pairing, which corresponds to  a {\it noun class} in this work.}


\newpage 
\paragraph{Class 1}
\label{sec:class1}

Class 1 allows a variety of stems:  CV, CVC, CVVCV,  and CVCV are possible.
It gathers the nouns whose singular is formed by a single vowel
suffix {\it -V} and \isi{plural} by a
light syllable {\sls -sV}.


\begin{table}

\caption{Class 1 \label{tab:freq-noun-class-1}}
\centering
%\subfloat[][{\sc class 1}]{
 \begin{Itabular}{lllll}
  \lsptoprule
{\sc class} & Stem & {\sc sg} &   {\sc pl} & Gloss\\[1ex] 
\midrule
{\sc cl.1}  &   va   &  váà   &  vá{\ꜜ}sá  & dog\\
{\sc cl.1}  &  pɛn   &  pɛ̀ná   &  pɛ̀nsá  & moon\\
{\sc cl.1}  &  gun   &  gùnó   &  gùnsó  & cotton\\
{\sc cl.1}  &  tʃuom   & tʃùòmó  & tʃùònsó   & Togo hare\\
{\sc cl.1}  &  bi   &  bìé   &  bìsé  & child\\
{\sc cl.1}  &  gbieki   &gbìèkíè   &  gbìèkísé  & type of bird\\

  \lspbottomrule
 \end{Itabular} 
 %}

\end{table} 

 The quality of the vowels of the singular and \isi{plural} is
determined by
the quality of the \isi{stem vowel} and the \is{harmony rule} harmony rules in operation. The rules were
stated in Section \ref{sec:vowel-harmony} and correspond to the noun class realization rules given in 
(\ref{ex:GRM-Hrules}).


\ea\label{ex:GRM-Hrules}

\ea\label{ex:mod-front-suffix}
{\rm -(C)V$_{nc}$  $>$ [ $\beta${\sc ro},  {\sc +atr}, {\sc --hi}]  / [ $\beta${\sc 
ro}, 
{\sc +atr}] C* \_ }\\

A noun class suffix vowel becomes {\sc +atr} if preceded by a {\sc +atr}
stem vowel, and shares the same value for the
feature {\sc ro}  as the one specified on the preceding (stem) vowel. A noun
class suffix is always {\sc --hi}.

 \ex\label{ex:low-suffix}
{\rm -(C)V$_{nc}$  $>$ {\sc +lo}  / {\sc --atr} C* \_ }\\

A noun class suffix vowel becomes {\sc +lo} if the preceding stem vowel is 
either
{\sls ɪ}, {\sls ɛ}, {\sls ɔ}, {\sls ʊ} or {\sls a}.\\

\z 
 \z




 
 \paragraph{Class 2}
\label{sec:class2}

 Table \ref{tab:freq-noun-class-2} displays  nouns assigned to class 2. Typically, this class consists of nouns whose stems are CVV or CVCV. While the singular form  displays no overt suffix,  {\sls -sV} is suffixed onto the stem to form the \isi{plural}.  

 
 \begin{table}
\caption{Class 2 \label{tab:freq-noun-class-2}}
\centering

 \begin{Itabular}{lllll}
  \lsptoprule
{\sc class} & Stem & {\sc sg} &   {\sc pl} & Gloss\\[1ex] 
\midrule

{\sc cl.2}  &  daa   &  dáá   &  dààsá & tree\\
{\sc cl.2}  &  bɔla & bɔ̀là   &  bɔ̀làsá  &  elephant\\
{\sc cl.2} &  kuoru &  kùórù   &  kùòrùsó  & chief\\
{\sc cl.2} &tomo &tòmó & tòmòsó& type of tree\\

{\sc cl.2} &  bele  & bèlè  &  bèlèsé &type of bush dog\\
{\sc cl.2} & tii   &  tíì &  tísè & type of tree\\

  \lspbottomrule
 \end{Itabular} 


\end{table}



 The rules in  (\ref{ex:GRM-Hrules}) capture the majority of the
singular/\isi{plural} pairs of class 1 and 2. However, it is insufficient in some
cases, that is, there are cases which raise uncertainty in the allotment of
the pairs into one class or the other. Consider the examples in
Table \ref{tab:uncer-noun-class}.


\begin{table}
\caption{Pending class 1 or 2 \label{tab:uncer-noun-class}}
\centering
%\subfloat[][{\sc class 3}]{
 \begin{Itabular}{lllll}
  \lsptoprule
 {\sc sg} &   {\sc pl} & Gloss\\[1ex] 
\midrule
dʊ̃́ʊ̃̀  & dʊ̃́s{\ꜜ}á  &  African rock python\\
kɪ̀rɪ̀má & kɪ̀rɪ̀nsá & tsetse fly\\
lɛ́hɛ́ɛ́  & lɛ̀hɛ̀sá&  cheek\\
  tíì &  tísè & type of tree\\
  bìé   &  bìsé  & child\\

  \lspbottomrule
 \end{Itabular} 


\end{table}

Two questions are raised by looking at the data in Table \ref{tab:uncer-noun-class}: (i) What is the stem of these nouns and how are they analysed?  (ii) Is there a good reason to favour final vowel deletion instead of insertion, e.g. /kɪrɪma/ vs. /kɪrɪm/  `tsetse fly'? Addressing  the first question, consider the first pair of words of Table \ref{tab:uncer-noun-class}, i.e. {\sls dʊ̃ʊ̃}  and {\sls dʊ̃sa}. On the one hand, if {\sls dʊ̃} is treated as   the stem and  the word for `African rock python' is assigned to class 1,   the refutation of the rule in   (\ref{ex:GRM-Hrules}) must be explained, i.e. vowel suffixes are always {\sc -hi}.  On the other hand, if  the stem is  {\sls dʊ̃ʊ̃},  a deletion rule which reduces the \is{lengthening}length of the vowel, i.e. {\sls /dʊ̃ʊ̃-sa/}  $\rightarrow${\sls [dʊ̃́s{\ꜜ}á]},  must be stated. Such a decision  would assign a stem {\sls /dʊ̃ʊ̃/}   to class 2.  The decision taken here is to respect the rule in (\ref{ex:GRM-Hrules}), which is empirically supported, and assume an {\it ad hoc} deletion rule.  This deletion rule, which may be driven by general prosody or phonological structure, will not be considered here.  The word pairs in Table \ref{tab:uncer-noun-class} are assigned the following classes: `African rock python' is in class 2 and the last stem vowel is deleted in the \isi{plural}, `tsetse fly' is in class 1 and its stem is /kɪrɪm/, and   `cheek' is in class 2 and the last stem vowel is deleted in the \isi{plural}.  Finally, the final vowel of the stem /tii/ is deleted in the \isi{plural}, and a vowel is added to the stem of /bi/ in the singular.

 \paragraph{Class 3}
\label{sec:class3}

Table \ref{tab:freq-noun-class-3} shows that the  noun stems allotted to class 3 generally have a 
sonorant coda consonant in the singular, i.e. {\sls l}, {\sls r}, {\sls  ŋ}, etc.  Class 3 is analysed as 
containing nouns whose singular forms have no overt suffix and whose \isi{plural} forms   have a single 
vowel as suffix. As for class 1 and 2, the \isi{plural} vowel suffix of class 3 is determined by the 
harmony rule given in (\ref{ex:GRM-Hrules}).

\begin{table}
\caption{Class 3 \label{tab:freq-noun-class-3}}
\centering
%\subfloat[][{\sc class 3}]{
 \begin{Itabular}{lllll}
  \lsptoprule
{\sc class} & Stem & {\sc sg} &   {\sc pl} & Gloss\\[1ex] 
\midrule

{\sc cl.3}  &  nɔn &  nɔ́ŋ   &  nɔ́ná  &  fruit\\
{\sc cl.3}  &  hããn & hã́ã̀ŋ   &  hã́ã́nà  & woman\\
{\sc cl.3}  &  pʊŋ & pʊ́ŋ  &  pʊ́ŋá  & hair\\
{\sc cl.3}  &  nar &  nár   &  nárá  &  person\\
{\sc cl.3}  &  ʔol  &  ʔól &  ʔóló  & type of mouse\\
{\sc cl.3}  & butet &   bùtérː  &  bùtété & turtle\\
{\sc cl.3}  &   sel  &   sélː  & sélé  & animal\\
  \lspbottomrule
 \end{Itabular} 
 %}
 

\end{table}
 
 
 
 \paragraph{Class 4}
\label{sec:class4}

As shown in Table \ref{tab:freq-noun-class-4}, the major characteristic of class 4 is that all the stems are analysed as having a final syllable consisting of a  {\sc [+hi, -ro]} vowel.  In class 4,   a  vowel is added to the stem on both  the singular and the \isi{plural}, i.e. V]\# $>$ V]-V\#. The suffix vowel of the singular is always an exact copy of the stem vowel.  If the stem vowel is {\sc [+atr]}, the \isi{plural} suffix vowel is {\sls -e}, and if the stem vowel is  {\sc  [-atr]}, the  \isi{plural} suffix vowel  {\sls -a}. This low vowel is then raised due to the height of the stem vowel. In normal speech, one can perceive either  {\sls -a} or {\sls -ɛ} in that position.  A similar noun class is found in other SWG and Western \ili{Oti-Volta} languages (see Section \ref{sec:GRM-noun-class-recons}).
 
 \begin{table}
\caption{Class 4 \label{tab:freq-noun-class-4}}
\centering

 \begin{Itabular}{lllll}
  \lsptoprule

{\sc class} & Stem & {\sc sg} &   {\sc pl} & Gloss\\[1ex] 
\midrule

{\sc cl.4}  &  begi   &  bégíí &  bégíé  & heart\\
{\sc cl.4}  &  si   &  síí &  síé  & eye\\
{\sc cl.4}  &fili &fílíí&fílíé&bearing tray\\
{\sc cl.4}  &  bɪ   &  bɪ́ɪ́ &  bɪ́á  & stone\\
{\sc cl.4}  &  wɪ   &  wɪ́ɪ́ &  wɪ́ɛ́  & matter\\
{\sc cl.4}  &  wɪlɪ   & wɪ́lɪ́ɪ́   &  wɪ́lɪ́ɛ́  & star\\
  \lspbottomrule
 \end{Itabular}
\end{table} 



Class 4 also includes nominalized verbal lexemes.  In Section
\ref{sec:GRM-verb-act-stem},  it is observed that one way to make  a noun from a
verbal lexeme is
to suffix a  high-front vowel to the verbal stem. For instance,  the verbal lexeme  
{\sls zɪn} may be translated as  `drive', `ride' or `climb'. The suffix  -[{\sc +hi,
-ro}]  can be added to the verbal lexeme {\sls zɪn} making it nominal, i.e. {\sls kɪ́nzɪ̀nɪ́ɪ́} `horse', {\it lit.} thing-riding. 
Consequently,
the \isi{plural} of {\sls kɪ́nzɪ̀nɪ́ɪ́} `horse'  is {\sls kɪ́nzɪ̀nɪ́ɛ́}. The 
sequences 
{\sls -ie} and {\sls -ɪɛ} of class 4  often coalesce and may be  perceived as 
{\sls -ee}
and {\sls -ɛɛ}
respectively, e.g. {\sls   fɛ́rɪ́ɪ́/fɛ́rɛ́ɛ́}    ({\sc sg/pl}) `air potato'. 
 
 
 
 \paragraph{Class 5}
\label{sec:class5}


 The monosyllabic stems of class 5  can either be CVV or CVC. Class 5 consists of nouns which  form 
their singular with no overt suffix and  their \isi{plural} with the suffix {\sls -nV}. The quality of 
the 
suffix's consonant is determined by the stem and the place assimilation rules introduced in Section 
\ref{sec:focus-forms}, some of which are repeated in  (\ref{GRM-cl-5}). The vowel of the \isi{plural} 
suffix is determined by the stem vowel and the rules in (\ref{ex:GRM-Hrules}). 


\ea\label{GRM-cl-5}
{\rm Class 5 suffix -/{\sls nV}/ surfaces -[{\sls lV}] if the  coda consonant of the stem is
{\sls l} }\\
{\sc -/[nasal]V/}$_{nc}$  $>$  {\sc -/[lateral]V/}$_{nc}$  /  {\sc [lateral]} 
\_\\

\z


 
 \begin{table}
 \caption{Class 5 \label{tab:freq-noun-class-5}}
\centering
 \begin{Itabular}{lllll}
  \lsptoprule
{\sc class} & Stem & {\sc sg} &   {\sc pl} & Gloss\\[1ex] 
\midrule

{\sc cl.5}  &  zɪn &  zɪ̀ŋ́ &  zɪ́nná  &  type of bat\\
{\sc cl.5}  &hʊ̃n&hʊ̃̀ŋ́&hʊ̃́nná& farmer or hunter gear\\
{\sc cl.5}  &  kuo &  kùó   &  kùónò  & farm\\
{\sc cl.5}  &  ɲuu &  ɲúù   &  ɲúúnò  & head\\
%{\sc cl.5}  &  sũũ &  sũ̀ũ̀   &  sũ̀ũ̀nó  &  guinea fowl\\
{\sc cl.5}  &  vii & víí   &  vííné &   type of cooking pot\\
{\sc cl.5}  &din&díŋ & dínné & fire\\
{\sc cl.5}  &pel &pél & péllé & burial specialist\\

  \lspbottomrule
 \end{Itabular}
\end{table} 
 
\newpage 
 \paragraph{Nasals in noun classes' morpho-phonology}
\label{sec:gene-nasals}


 
Apart from the singular of class 4,  much of the same vocalic morpho-phonology
is found in all classes. This was reduced to the two rules in
(\ref{ex:GRM-Hrules}). Furthermore, in all the noun classes, the nasal
consonants surface differently depending on the phonological context. The rules
in  (\ref{ex:GRM-nrules}) predict the observed outputs and are derived from the
nasal assimilation rules in Section \ref{sec:internal-sandhi-nasal-place}.
 
\ea\label{ex:GRM-nrules}{\rm Possible outputs of nasals}\\

\ea\label{ex:GRM-Nrules}
 {\rm C[{\sc +nasal}]\   $>$ ŋ / \_ \# }\\
 {\rm /{\sls hããn-\O}/  $>$  [{\sls hã́ã̀ŋ}]  `female' {\sc cl.3sg}}   


\ex\label{ex:GRM-Msrules}
{\rm /m/ $>$ n / \_  C [{\sc -labial, -velar}] }\\
{\rm  /{\sls tʃuom-sV}/   $>$ [{\sls tʃùònsó}]    `Togo hares'  {\sc cl.1pl}  }

\ex\label{ex:GRM-NGsrules}
 {\rm  /ŋ/ $>$ n / \_  C [{\sc -labial, -velar}]}\\
{\rm /{\sls kɔlʊ̃ŋ-sV}/ $>$  [{\sls kɔ̀lʊ̀nsá}]   `wells'   {\sc cl.2pl} }


\z 
 \z

The rule in  (\ref{ex:GRM-Nrules})  says that  any nasal consonant occurring
word finally becomes [ŋ]. The rule in (\ref{ex:GRM-Msrules}) changes a bilabial
nasal into an alveolar when it precedes a non-labial and non-velar consonantal
segment. The rule in (\ref{ex:GRM-NGsrules}) changes a velar nasal into an
alveolar in the same environment.

 
 \paragraph{Generalization and summary}
\label{sec:gene-sum}

While the method proposed suggests that one should look for pairs of forms, the
present classification treats phonologically empty suffixes as ``exponents''. What
counts as a noun class is the paradigm determined by the  inflectional
pattern of the lexeme. The five  most frequent pairs were presented in Tables
\ref{tab:freq-noun-class-1} to \ref{tab:freq-noun-class-5} and the exponents are
gathered in  Table \ref{tab:GRM-nc-exponent}.\footnote{The percentage is based
on a list
of 978 singular/\isi{plural} pairs  (02/10/10). The five classes in
Table \ref{tab:GRM-nc-exponent} make up 88\% of the nouns which are assigned a
class in the lexicon.} 
%see all numbering/percentage as it was changed

 \begin{table}
 \caption{The five most frequent noun classes   \label{tab:GRM-nc-exponent}}
   \centering
   \begin{Itabular}{lccccc}
 \lsptoprule
    &  {\sc cl.1} & {\sc cl.2}  & {\sc cl.3} & {\sc cl.4} & {\sc cl.5} 
\\[1ex] \midrule
{\sc sing} & -V&  \O&  \O& -V  & \O\\
{\sc plur} & -sV& -sV & -V & -V  & -nV\\\midrule
    &   8\%&  32\%  &  23\% &   17\%   & 8\%\\
 \lspbottomrule
   \end{Itabular}
 \end{table}

 \newpage 
In practice the most productive and regular patterns are those recognized as noun classes. However, some words do not fit perfectly into the patterns described above but are not totally alien to genetically related languages and the reconstructions of Proto-\ili{Grusi} in \citet{Mane69a, Mane69b} and Proto-\ili{Grusi}-Kirma-Tyurama  in \citet{Mane82}.   In fact, there are more possibilities and surface  forms when the  classes  ({\sc sg/pl})  {\sls \O/\O},  {\sls \O/ta}, {\sls \O/ma} and {\sls ŋ/sV} are included in the classification. Examples are given  in Table \ref{tab:GRM-less-pro-nc}.  
 
\begin{table}
\caption{Noun classes 6, 7, 8, and 9 \label{tab:GRM-less-pro-nc}}
\centering
  \begin{Itabular}{lllll}
  \lsptoprule
{\sc class} & Stem & {\sc sg} &   {\sc pl} & Gloss\\[1ex] 
\midrule

{\sc cl.6}  & dʒɪɛnsa & dʒɪ́ɛ̀nsá & dʒɪ́ɛ̀nsá & twin\\
{\sc cl.6}  &kapʊsɪɛ &  kàpʊ́sɪ̀ɛ̀ & kàpʊ́sɪ̀ɛ̀  & kola nut\\
{\sc cl.6}  & kpibii & kpĭ̀bíì & kpĭ̀bíì  & louse\\[0.2ex] \midrule

{\sc cl.7}  & kuo & kúó &kùòtó  &roan antelope\\
{\sc cl.7}  &kie  &kìé & kìèté & half of a bird\\
{\sc cl.7}  &fɔ̃ʊ̃ &fɔ̃́ʊ̃̀& fɔ́tá & baboon\\[0.2ex] \midrule
%{\sc cl.7}  &  taa &  tàá &  tàátá  & language\\
{\sc cl.8}  & naal &náàl&nááləmà&grand-father\\
{\sc cl.8}  &ɲɪna &ɲɪ́nà&ɲɪ́námà &father\\
{\sc cl.8}  &  hɪ̃ɛ̃ŋ & hɪ̃́ɛ̃́ŋ & hɪ̃́ɛ̃́mbá &relative\\[0.2ex] \midrule

 {\sc cl.9}  &  jo   &jóŋ̀ & jósò  & slave\\
{\sc cl.9}  & zi &zíŋ̀ &zísè &tail\\
{\sc cl.9}  & ŋmɛ&ŋmɛ́ŋ̀&ŋmɛ́sà&rope\\

  \lspbottomrule
 \end{Itabular} 

\end{table} 


 The nouns in class 6 do not formally differentiate singular and \isi{plural}.   Those
in class 7 mark their \isi{plural} with the suffix {\sls -tV} and  class 8 with the 
suffix {\sls -mV}.  The singular exponent of class 7 and 8 is covert. Finally,
the nouns of class 9 have a suffix {\sls -ŋ} in the
singular and {\sls -sV} in the \isi{plural}. In Table \ref{tab:l-leasttive-class},  
the
percentage of occurence of the less productive noun classes 6, 7, 8
and 9 is given.
 
  
 \begin{table}
   \caption{Less productive  noun classes 
\label{tab:l-leasttive-class}}
   \centering
   \begin{Itabular}{lcccc}

  
 \lsptoprule
&  {\sc cl.6}   & {\sc cl.7}  &  {\sc cl.8} & {\sc cl.9}\\
[1ex] \midrule
 {\sc sing}   & \O  & \O   & \O & -N\\
{\sc plur}   & \O & -tV  & -mV & -sV\\\midrule
    &   7\%&   1.8\% &   0.9\%   & 0.8\%\\
 \lspbottomrule
  
   \end{Itabular}
 \end{table}

In addition, there are pairs which can only imperfectly be reduced  to the nine classes presented 
until now. However, the problem lies in the stem and not in the inflectional pattern, and thus  suggests suppletion rather than phonotactics. For example 
the colour terms ({\sc sg/pl}) {\sls pʊ̀mmá}/{\sls pʊ̀lʊ̀nsá} `white' and {\sls búmmó}/{\sls 
bùlùnsó} `black' do not have comparable pairs and do not fit the noun classes described above. 
One would expect *{\sls pʊmmasa} to be the \isi{plural} form for  `white' (also {\sls *tɪɪnama} for {\sls 
tɪ̀ɪ̀ná}/{\sls tʊ́mà} `owner'). Other examples are the  pairs {\sls tɪ́ɛ̀}/{\sls tɛ́sà} `foetus' 
and {\sls túò}/{\sls tósó} `bow' (see \citealt{brin15d} for an account of a similar situation 
in \ili{Waali}). Also here, one expects  the last vowel to delete in each of the \isi{plural} 
forms instead of the penultimate one. Moreover,  inconsistent class assignment across speakers, 
across villages, and even different forms (predominantly in the \isi{plural}) from the same speaker on 
different elicitation sessions do arise.
 
% 
%also tʊ̀ɔ́nɪ́ã̀ tʊ̀ɔ́nsà  type of genet
%also kùòlíè  kùòlúsò  type of tree 
% %apocope?
%  hɔ̃́ʊ̃̀
%  type of grasshopper
% \cl 2
%  hɔ̃́sà


 
\paragraph{Semantic assignment criteria}
\label{sec:GRM-sem-ass-crit}

Several authors have presented their views on  the semantic classification of
nominals.   The general idea is that there must be an underlying system which
can explain, first, why some words display identical number morphology, and
second, how these words are related in meaning. \citet[23]{Tcha07} shows that
\is{Tem}{Tem} organizes its nominals on the basis of semantic values such as humanness,
size, and countability. \citet[41]{Awed07} argues that nominal groupings  in
\ili{Kasem} should take into consideration phonological and semantic characteristics,
in addition to other more cultural factors.  \citet{Assi07}
argues at length on the shortcoming of traditional semantic rules and argues for
abandoning them. 

The semantic value of the noun class suffixes has proven difficult to
establish. It is possible that there are analogies in class assignment based on semantic criteria, but it is more likely that synchronically (i) the phonological
shape of the stem triggers the suffix type, and that (ii) some classes can be
identified as residues of former semantic assignment. Let me comment on each of 
these
points. 

First, most class 3 nouns have  a sonorant consonant in the coda position,
the stems of  class 4 nouns must have their last vowel specified for  [{\sc -hi,
-ro}] and a typical class 2 noun is either   CVV or CVCV.  These are some of
the characteristics  described for the noun classes. It seems that the
phonological
shape of the stem plays a role in class assignment and that there is no
productive class
where most of its  members are assigned to a particular semantic domain.  Using
four features of the animacy hierarchy
of  \citet{Comr89}, i.e.  \isi{human} $[${\sc
hum}$]$, animal (exclude \isi{human}) or other-animate
and insects $[${\sc anim}$]$, concrete inanimate $[${\sc conc}$]$ and abstract
(inanimate) $[${\sc abst}$]$,  \citet{brin08} shows that the noun
classes  do not encode any of these distinctions. Such
distinctions may have
been  expected given the nominal classification of other \ili{Gur} languages. For
instance in \is{Dagaare}{Dagaare}, a \is{Western Oti-Volta}{Western Oti-Volta} language in contact with
Chakali, \citet[124]{Bodo94} presents the Class 2 (V/ba) as ``unique in that it
is the only class that has exclusively [+\isi{human}] nouns in it''. From a
diachronic point of view, this could suggest that Chakali has dropped all animacy
distinctions in the noun class system while preserving one distinction in
agreement (see Section \ref{sec:GRM-gender}).


Secondly, languages related to Chakali, e.g. geographically and genetically,  
 have or had noun class systems whose classifications were based, at least 
partially, on semantic criteria \citep{nade82}. To my knowledge, the most conservative system 
today  within \is{Grusi}{Grusi} is Tem (see  ``identification sémantique'' in 
\citealt{Tcha07}). When and how the speakers of Chakali  classified nouns based on 
semantic criteria is impossible to know,  but traces can be detected in  the   
{\it less productive noun classes}, that is class 6, 7, 8, and 9 (see Table 
\ref{tab:l-leasttive-class}). Some members 
of class 6 consist of nouns with mass or abstract  denotations, e.g. rice,  
louse, struggle, profit, etc.  Recall that number has no exponent in class 6.  Class 7 also 
contains mass and abstract nouns, 
e.g. oil,  honey, water, and taboo, but also bush animals such as bushbuck, 
waterbuck, baboon, roan antelope and hartebeest. Class 7 represents 
approximately 2\% of the noun sample (see Table \ref{tab:l-leasttive-class}) and 
 mass/abstract nouns and bush animals each represent 30\% of class 7 
membership. Class 8 is likely to be the class where kinship and \isi{human} 
classification terms were assigned, as mother, father, and `owner of' are among 
remnant members of that class.  Finally, a  common trait of class 9 may be 
`elongated things', since words referring to  rope, arm, tail, and ladder are 
members. Yet, only eight nouns are assigned to class 9. Despite the arbitrary 
nature of the semantic assignment of class 9,  \citet[94]{Mane75} maintains that 
there are \is{Oti-Volta}{Oti-Volta} languages which show relics of  the Proto \ili{Oti-Volta} class 
{\sls *ŋu- *u-}, which is  itself a remnant of \is{Proto-Gur}{Proto-Gur} class 3   according to 
\citet[11]{Mieh07}, and that this class contains ``les noms du bâton, du pilon, 
du balai, de la corde, de la peau et du chemin''.  Although these nouns seem to 
denote `elongated  things',   Manessy claims that they cannot contribute to an 
hypothesis. Generally, however, the fact that members of classes 6, 7, 8, and 9 
are similarly clustered in other languages suggests that these classes are 
remnants of a more productive semantic assignment system. Beside semantic 
domains, the simple empirical fact that homonyms are allotted different classes  excludes a purely phonologically-based  assignment. There is no way a speaker can correctly pluralize the stems {\sls kuo} ({\sc cl.5})  `farm' and   {\sls kuo} ({\sc cl.9})  `type of antelope' based entirely on their (segmental) phonological shape.\footnote{I put segmental in parenthesis since  homonyms {\it with the same tonal melody} belonging to two different classes have not yet been  found. The pair {\sls pól}/{\sls pólló} ({\sc cl.5}) `water source' and {\sls pól}/{\sls póló} ({\sc cl.3}) `vein' may be treated as one example,  but their meanings could be thought of as pointing to a common \isi{etymology}.  Another is the pair  {\sls  tíì}/{\sls  tísè} ({\sc cl.2}) `type of tree' and {\sls  tíì}/{\sls  tíísè} ({\sc cl.2}) `tea', but the latter is a recent loan from English. Nevertheless, \citet{Bonv88}, \citet{Awed07} and \citet{Tcha07} provide data to support a similar claim.} It seems that apart from phonological and semantic features, combinatorial features on lexical units are necessary to account for noun class assignment.


\paragraph{Tone patterns of noun classes}
\label{sec:GRM-tone-p}


In
spite of variations,  nouns have recurrent tonal
melodies and representative examples are presented in
Table \ref{tab:GRM-tm-nc-1-5}.  The  general tendency for nouns is for 
the singular and  \isi{plural} forms to display the same tonal melody.  For
instance, a HL melody may be associated with both the singular and the \isi{plural},
e.g.  {\sls zíŋ̀}/{\sls zísè} `tail' ({\sc cl.9}) 
and   {\sls lʊ́l̀}/{\sls lʊ́là} `biological relation'  ({\sc cl.3}). These 
cases 
are tonally regular. 
Another common pattern is when a singular noun displays a H melody, but the
\isi{plural} a LH melody, e.g.  {\sls dáá}/{\sls dààsá} `tree' ({\sc cl.2}). 
While 
it
 seems that  the
\isi{plural} suffix -{\sls sV}  depresses a preceding H,  it does not do so in class 9 nouns. The 
majority 
of class 4 nouns in the data available are high tone irrespective of the number of moras and they 
are all tonally regular.  An exception is the LHL melody, of which a dozen or so pairs are 
attested, e.g.  {\sls tʃɪ̀ɪ̀rɪ́ɪ̀}/{\sls  tʃɪ̀ɪ̀rɛ́ɛ̀} `type of wasp' and  {\sls  lʊ̀gɪ́ɪ̀}/{\sls  
lʊ̀gɛ́ɛ̀} `iron'. Some cases involving
singular CVC words with moraic
coda exhibit the deletion of a low tone;  {\sls zɪ̀ŋ́}/{\sls zɪ́nná} `bat' 
({\sc 
cl.5}),   {\sls gèŕ}/{\sls gété} `lizard' ({\sc cl.3}),  and
{\sls sàĺ}/{\sls sállá} `flat roof' ({\sc cl.5})  have a LH tonal melody in 
the
singular but  H in the \isi{plural}. A downstep rule (Section
\ref{sec:tone-intonation})  predicts that a high tone preceded by a low tone is
perceived as lower than a preceding high
tone, e.g. {\sls váà}  {HL},  {\sls vá{\ꜜ}sá}  {HLH}  `dog' 
({\sc cl.1}).  

 \begin{table}
   \caption{Tonal melodies in noun classes 1--5
\label{tab:GRM-tm-nc-1-5}}
   \centering
   \begin{Itabular}{lp{1cm}lp{1cm}ll}

 \lsptoprule
{\sc class} &  Tone  melody {\sc sg}  &   Singular   &  Tone  melody  {\sc
pl} &   Plural & Gloss
\\[1ex]

\midrule

{\sc cl.1}   &   HL   & váà & HLH &   vá{\ꜜ}sá  & dog\\
  &  LH &  gùnó&  LH& gùnsó  & cotton\\
& HL & tʃíníè & HL &  tʃínísè  & type of climber\\
  & L &  dɪ̀gɪ̀nà  & LH & dɪ̀gɪ̀nsá&  ear\\[0.2ex] \midrule

{\sc cl.2}  &  H  &   síé &  LH   &
sìèsé&  face\\
&  L   &  bɔ̀là   & LH&   bɔ̀làsá  &
elephant\\
&  LH &  tòmó  &  LH &  tòmòsó  &  type of
tree\\
&  LH &  jùó  &  LH   &  jùòsó  & quarrel\\
&  HL &  kánà&  HLH   &  kánàsá  & arm ring\\[0.2ex] \midrule

{\sc cl.3} &   H &   hóg  &   H   &   hógó  &   bone\\
&  HL &  hã́ã̀ŋ&  HL  &  hã́ã́nà  & woman\\
 &  LH &  gèŕ  &  H  &  gété  & lizard\\
 &  LH &  pààtʃák  &  LH   &  pààtʃàgá & leaf\\[0.2ex] \midrule

{\sc cl.4} &  H &  síí  &  H& síé  &  appearance\\
& H & tʃɪ̃́ɪ̃́  & H & tʃɪ̃́ã́ & dawadawa seed\\
& LHL  & ààrɪ́ɪ̀& LHL &  ààrɪ́ɛ̀  & grasscutter\\[0.2ex] \midrule

{\sc cl.5}&H &  víí  &  H &  vííné & cooking  pot\\
&LH &  bèŋ́ &  H &  bénné & law\\
 &  LH &  sàĺ  &  H &  sállá &  flat roof\\
&  HL&  tʃál̀   &  LH&   tʃàllá  &  blood\\
&  HL&  pél̀ &  H&   péllé &  burial specialist\\

  
 \lspbottomrule
   \end{Itabular}
 \end{table}

\paragraph{Noun class reconstruction}
\label{sec:GRM-noun-class-recons}

The numerical labelling of the noun classes in Table \ref{tab:GRM-nc-exponent} and 
\ref{tab:l-leasttive-class} is arbitrary.  Nonetheless,  some observations  on similarities between 
the noun class system in Chakali and other SWG noun class systems can be put forward. The 
information sources are my own field notes on neighbouring languages, the reconstruction of the noun 
class suffixes of \ili{Grusi} in \citet{Mane69a, Mane69b},  and the reconstruction of noun classes in \ili{Gur} 
in \citet{Mieh07}; the latter being for the most part an update and synthesis of Manessy's work 
\citep{Mane69a, Mane69b, Mane75, Mane79, Mane82, Mane99}. Needless to say, the following statements 
are first impressions.

\newpage 
Field notes on neighbouring languages, supported with unpublished material produced by GILLBT's staff,\footnote{\label{ft:GRM-naden-donate}In 2008, Tony Naden gave me  a copy of his ongoing \ili{Vagla} and \ili{Dɛg} lexicons. I am also indebted to: Kofi Mensa (New Longoro) for \ili{Dɛg}, Modesta Kanjiti  (Bole) for \ili{Vagla} and \ili{Dɛg}, Joseph Kipo (Bole) for \ili{Vagla} and Yusseh Jamani (Bowina) for \ili{Tampulma}.}  provided relevant information on the (dis-)sim\-i\-lar\-i\-ties of Chakali with other SWG languages. As in all SWG languages, a typical \ili{Vagla} noun class is characterized by  suffixation. The most frequent \isi{plural} markers in \ili{Vagla} are {\sls -zi}, {\sls -nɪ} and {\sls -ri}. The pattern found in Chakali  class 4 is similar to the one found in \ili{Vagla}, e.g. ({\sc sg}/{\sc pl}) {\sls bàmpírí}/{\sls bàmpíré} `chest',  {\sls hūbí}/{\sls hūbé}  `bee' and   {\sls gíngímí}/{\sls gíngímé} `hill'.  In \ili{Dɛg},   the most frequent \isi{plural} markers are mid-vowel suffixes, often rounded,  and the {\sls -rV}, {\sls -nV} and  {\sls -lV} suffixes, with which the vowel harmonizes in roundness and {\sc atr} with the stem vowel. Both \ili{Vagla} and \ili{Dɛg} display miscellaneous classes which are characterized by  a simple difference in vowel quality between the last vowel of the singular and the \isi{plural}, e.g. \ili{Dɛg} {\sls dala}/{\sls dale} `cooking place'. Attested alternations  ({\sc sg}/{\sc pl}) in \ili{Vagla} are {\sls -i}/{\sls -e},  {\sls -i}/{\sls -a},  {\sls -a}/{\sls -i}, {\sls -u}/{\sls -a},  {\sls -o}/{\sls -i} and  {\sls -e}/{\sls -i},   and in \ili{Dɛg} {\sls -a}/{\sls -e}, {\sls -e}/{\sls -a}, {\sls -i}/{\sls -e}, {\sls -o}/{\sls -i} and  {\sls -i}/{\sls -a}.\footnote{These singular/\isi{plural} pairings are extracted from the \ili{Vagla} and \ili{Dɛg} lexicons (fn. \ref{ft:GRM-naden-donate}) and are not exhaustive.}  The noun classes of \ili{Tampulma} and \ili{Pasaale} correspond more to those of Chakali. \ili{Tampulma} has at least the following class suffix pairs ({\sc sg}/{\sc pl}): {\sls \O}/{\sls -V}, {\sls -i}/{\sls -e}, {\sls \O}/{\sls -nV},  {\sls  \O}/{\sls -sV}, {\sls  -V}/{\sls -sV},  {\sls -hV}/{\sls -sV} and  {\sls \O}/{\sls -tV}. \ili{Tampulma} displays similar harmony rules to those found in Chakali. Apart from the singular suffix {\sls -hV}, all the noun class suffixes in \ili{Tampulma} are manifested in Chakali.  Correspondingly, \ili{Pasaale} reveals  pairs and harmony rules similar to those of Chakali and \ili{Tampulma}.\footnote{As mentioned in footnote \ref{foot:noun-class}, the number of noun classes is determined by the linguist's analysis.  \citet[5--12]{mcgi99} is a good example of the consequence of analyzing noun classes differently. For instance,  \citet[7]{mcgi99} postulate a subclass  ({\sc sg}/{\sc pl})  {\sls -l/-lA} for word pairs like {\sls baal/baala} `man', {\sls gul/gulo} `group', {\sls miibol/miibolo} `nostril' and  {\sls mɔl/mɔlɔ} `stalk'. If these words were part of the Chakali data, they would have been allotted to class 2 ({\sls -\O/-V}), that is, I would have treated the /l/ as a coda consonant of the stem instead of a noun class suffix consonant. In addition, whereas I derive the quality of the vowel entirely from harmony rules,  \citeauthor{mcgi99} assume archiphonemes (underspecified segments), like A and E, which surface depending on harmony rules.}

It is important to keep in mind that the analysis in \citet{Mane69a, Mane69b} is based on a very limited set of SWG data,  most of the data being extracted from \citet{Bend65}. He often stresses the tentative nature of his claims and  sets forth more than one hypothesis on several occasions. Based on a comparison of word forms and meanings,  Chakali \isi{plural}  suffix of class 8 {\sls -mV} may be treated as a descendant of the Proto-\ili{Grusi} Class {\sls *B$_{1}$A} \citep[32]{Mane69b}, class 9 {\sls -ŋ} as a descendant of the Proto-\ili{Grusi} Class {\sls *NE}   \citep[37, 41]{Mane69b}, class 1 {\sls -V} as a descendant of the Proto-\ili{Grusi} Class {\sls *K$_{1}$A}  \citep[39]{Mane69b}, classes 1, 2, and 9 {\sls -sV} as descendants of the Proto-\ili{Grusi} Class  {\sls *SE}  \citep[39]{Mane69b} and class 7 {\sls -tV} as a descendant of the Proto-\ili{Grusi} Class {\sls *TE/O}  \citep[43]{Mane69b}. The vowel suffixes of class 1 and 4 may also descend from the Proto-\ili{Grusi} Class {\sls *YA} \citep[34]{Mane69b}.   In consulting \citet[7--22]{Mieh07}, Chakali's  most frequent \isi{plural} suffix  {\sls -sV}, found in class 1, 2, and 9, would seem to correspond to Proto-\ili{Gur} Class 13 *{\sls -sɪ}, the \isi{plural} suffix of class 5 {\sls -nV} to  Proto-\ili{Gur} Class 2a *{\sls -n.ba} or Proto-\ili{Gur} Class 10 *{\sls -ni}, class 7 {\sls -tV} to Proto-\ili{Gur} Class 21 *{\sls -tʊ} and class 8  {\sls -mV}  to Proto-\ili{Gur} Class 2 *{\sls -ba}. The singular suffix {\sls -ŋ} would correspond to Proto-\ili{Gur} Class 3 *{\sls -ŋʊ}.


Needless to say, these observations  deserve further investigation. Even though there is  literature to support the reconstruction of the \ili{Gur} classes, little can be done in the SWG area unless descriptions of  nominal classifications in the languages  \ili{Winyé}, \ili{Vagla}, \ili{Tampulma}, \ili{Phuie},  \ili{Dɛg}, \ili{Siti}/\ili{Kyitu},  and the dialects of \ili{Sisaala}  are made available. A synthesis of these descriptions could be compared to  ``better-documented'' nominal classfications of \ili{Grusi} languages such as \ili{Kasem} (Northern \ili{Grusi}, \citealt{Awed79, Bonv88, Awed03}),  Lyélé (Northern \ili{Grusi}, \citealt{Delp79}),  Lama  (Eastern \ili{Grusi}, \citealt{Arit87, Ours89}), Kabiyé (Eastern \ili{Grusi}, \citealt{Tcha07}),  Chala   (Eastern \ili{Grusi}, \citealt{Klei00}) and Tem (Eastern \ili{Grusi}, \citealt{Tcha72, Tcha07}), to evaluate the Proto-\ili{Grusi} noun class suffixes of \citet{Mane69b} and Proto-\ili{Gur} of \citet{Mieh07}, and to reconstruct the nominal classifications of SWG  languages.



\subsubsection{Atomic stem nouns}
\label{sec:GRM-sim-bas-noun}

The notion of stem in the present context refers to the host of a noun class suffix or the  host of a nominaliser, i.e. the element which conveys the lexical meaning and  to which affixes attach. A stem can be either irreducible or reducible morphologically: they are referred to as atomic  and complex stem respectively.  Complex stems are presented in  Section \ref{sec:GRM-com-stem-noun}.   An atomic stem is always a  nominal or a verbal lexeme.  A verbal lexeme may either be of the type ``process'' or  ``state'' (Sections \ref{sec:GRM-verb-act-stem} and \ref{sec:GRM-verb-state-stem}). Three types of nominalization formation (i.e. nominalisers) are attested: suffixation, prefixation, and reduplication.  


\largerpage[-1]
\paragraph{Nominal  stem}
\label{sec:GRM-nom-stem}

A nominal stem denotes a class of entities.   Nouns composed by the combination of  a nominal stem and a noun class affix are the most common. A nominal stem has the potential to be juxtaposed with various noun class affixes, yielding forms with different meanings. For instance, the lexeme {\sls baal} is associated with the general meaning `male'. In a context where the lexeme is used in the singular, {\sls baal} can mean either `a man' or `a husband'. Given the same context but used in the \isi{plural}, the lexeme {\sls baal} is disambiguated by the \isi{plural} suffix it takes;  {\sls báàlá} `men'  ({\sc cl.3}) and  {\sls bààlsá} `husbands'  ({\sc cl.2}).  Evidence from other \ili{Grusi} languages suggests that the situation where lexemes are found in different noun classes was certainly a   more common phenomenon than it is today \citep[126--128]{Bonv88}. This may coincide with semantically richer noun class suffixes. In addition, for many noun classes the singular forms are not overtly marked and the \isi{plural} forms are by and large less frequent. This situation makes it difficult to provide the necessary evidence which would demonstrate that nominal stems are attested with different noun classes.  

Nominal stems exist in opposition to the verbal ones. To classify a stem in such a dichotomy, the simple test carried out consists of placing the stem in several core predicative positions, i.e. positions where an argument must appear. If the sentence is perceived as grammatical and felicitous by language consultants, it cannot be nominal.  The examples in (\ref{GRM-nom-or-verb}) illustrate the procedure. It uses a frame where the predicate is in the perfective aspect and  the same predicate, as opposed to the argument,  is in \isi{focus}. The assumption is that this position  cannot be satisfied by nominal lexemes.

\ea\label{GRM-nom-or-verb}
 
\ea {\rm /{\sls di}/   `eat'  $\rightarrow$  {\sls ʊ̀  díjōó}    |{\sc 3sg} 
eat.{\sc pfv.foc}|  `he ate'}

  \ex  {\rm /{\sls kpeg}/  `hard'   $\rightarrow$ {\sls ʊ̀ kpégéó}    |{\sc 3sg} 
hard.{\sc pfv.foc}| `he is strong'}

   \ex  {\rm /{\sls sɪama}/  `red'  $\rightarrow$ *{\sls ʊ sɪamao},  but  {\sls ʊ̀ sɪ́árēó} |{\sc  3sg} red.{\sc pfv.foc}|  `it is red'} 

\ex {\rm /{\sls bi}/   `child' $\rightarrow$  *{\sls ʊ bio}} 

 
\z 
 \z


The grammatical sentences in (\ref{GRM-nom-or-verb}) show that  {\sls di} and  {\sls kpeg} are verbal,  whereas {\sls sɪama} and {\sls bi} are not. In Section \ref{sec:GRM-qualifier}, it will be shown that some colour properties change forms depending on whether they occur  in a nominal or verbal context, so `red' can be verbal but {\sls sɪama} is not.  


\largerpage[-1]
\paragraph{Verbal process stem}
\label{sec:GRM-verb-act-stem}

%Somewhere should be included a section on derivative affixes (and/or other
%types of nominaliser) 

Verbal process stems denote non-stative events. Table \ref{tab:GRM-nom-process} 
displays  two types of nominalization formation -- suffixation and 
reduplication -- involving verbal process stems,  `agent of X' and `action of 
X', where X replaces the meaning of the verbal process stem. 

\begin{table}

\centering
\caption{Examples of nominalization of verbal process stem
\label{tab:GRM-nom-process}}
 \begin{Itabular}{llll}
 \lsptoprule
Sem. value & Verb. process stem & {\sc nmlz} & Form\\
 \midrule

Agent of X &  gʊ̀ɔ̀ `dance' &  -/r/ & gʊ́ɔ́r `dancer'\\
Agent of X &  kpʊ́  `kill' &   -/r/  & kpʊ́ʊ́râ  `killer'\\
Agent of X &   búól   `sing' &  reduplication &   bùòlbúóló  `singer'\\
Agent of X &   sùmmè `help' &  reduplication &   súsúmmá 
`helper'\\[1ex]\midrule

Action of X  & gʊ̀ɔ̀ `dance' &  -/[{\sc +hi, -bk}]/ & gʊ́ɔ́ɪ́ɪ́ `dancing'\\
Action of X &  kpʊ́  `kill' &  -/[{\sc +hi, -bk}]/  & kpʊ́ɪ̀ɪ́  `killing'\\
Action of X  &  búól   `sing'  &-/[{\sc +hi, -bk}]/  & búólíí 
`singing'\\
Action of X  &   sùmmè `help'   &-/[{\sc +hi, -bk}]/  &  súmmíí  `helping'
\\
\lspbottomrule
 
 \end{Itabular} 

\end{table} 


% The first column describes in prose the meaning of each nominalization,
%the second column provides the stem, the third column provides  the
%nominalization formations and the fourth provides the  translation.

In Table \ref{tab:GRM-nom-process}, the column entitled semantic value (Sem. value)  identifies the meaning of the verbal nominalization. In such a context, `agent of X'  refers to the instigator or doer of the state of affairs denoted by  the predicate X and the nominalization is generally accomplished by the suffix {\sls -r(a)}.  However,  there are some expressions with the equivalent agentive denotation which do not suffix {\sls -r}  to the  predicate, e.g.  {\sls ʔɔra} `to sew' vs. {\sls ʔɔta} `sewer' and {\sls maŋa} `to beat' vs.  {\sls kɪŋmaŋana} `drummer'.  The singular forms  are given in the fourth column: the \isi{plural} of agent nominals  of this type, i.e. nominalized by the suffix {\sls -r}, is made by  a single vowel suffix  ({\sc cl.3}) whose surface form  depends on harmony rules.\footnote{One language consultant had a problem retrieving the \isi{plural} of some agent nouns. He often repeated the singular entry for the \isi{plural}. I interpret this as  either a situation where agent nouns do not show differences in the singular and \isi{plural} ({\sc cl. 6}), or different  {\sc sg}/{\sc pl} forms exist but he could not retrieve them. The pair {\sls kpʊra}/{\sls kpʊrəsa} `killer(s)'   is unusual.  The word {\sls  sãsaar} means `woodcarver' and not `car driver'  even though {\sls sãã} can mean both `carve' and `drive vehicle'.  People usually use  {\sls lɔ́ɔ́lɪ̀sã́ã́r}, or the English word {\sls dərávɛ̀}, which is common all over Ghana, to refer to any driver of a vehicule.} Another verbal nominalization process conveying `agent of X' is reduplication. The evidence suggests that  only the first syllable  is reduplicated.

The second nominalization process is  interpreted as `action of X' or `process of X' and consists of the suffixation of a  high front vowel to the verbal stem.\footnote{The nominalization `the process X' is often not distinguishable from `the result of a process X'.  Does `dancing'  refer to `the process of dance', `the result of the process of dance' or both?} The surface form of the vowel depends on the quality of the stem vowel and {\sc atr}-harmony  (Section \ref{sec:vowel-harmony}). Consider example (\ref{ex:vp36.1.}).

\ea\label{ex:vp36.1.}
\glll ʊ̀ píílè wáɪ́ɪ́ rá.\\
ʊ piile wa-ɪ-ɪ ra\\
  {\sc 3sg} start come-{\sc nmlz}-{\sc cl.4} {\sc foc}\\
\glt  `He begins coming.' 
\z

 The final vowels in the words referring to `the process of X' are analysed as a sequence of two vowels: first a nominaliser suffix (i.e. {\sc nmlz})  on the verbal stem,  and second,  a noun class suffix.  Such nominalized verbal stems are allotted to noun class 4;  their singular suffix is a copy of the {\sc nmlz} vowel, and their  \isi{plural} suffix is the low vowel {\sls a}, raised to a mid height, e.g. {\sls pɛrɪɪ}/{\sls pɛrɪɛ} `weaving(s)'  ($<$ {\sls pɛra} `weave', see class 4 in Section \ref{sec:class4}).

%The infinite forms are always prefixed by the low vowels. The transcription
%shows harmony in some instances but it is a topic to further investigate. 


\paragraph{Verbal state stem}
\label{sec:GRM-verb-state-stem}

Verbal state stems  denote static events. They generally function  as verbs, but they can take the role of attributive modifiers in noun phrases, referred to as  `qualifiers' in Section \ref{sec:GRM-qualifier}. In that role, their semantic value is similar to the value of adjectives in English: they denote a property  assigned to a referent.  To function as  a \isi{qualifier}, some verbal state predicates must be nominalized. As with verbal process stems,  verbal state stems are found in nouns which have been nominalized by suffixation of a  high front vowel, i.e. `the state of X'. For instance, the verbal state predicate {\sls kpeg} has a general meaning which can be translated into English as `hard' and `strong'. The expression {\sls kpégíí} in {\sls a teebul kpegii dʊa de} `The hard table is there' functions as \isi{qualifier} in the noun phrase {\sls a teebul kpegii}, {\it lit.} `the table hard'. 


\ea\label{exːGRM-v-sta-p-hard}{\rm Verbal state stem {\sls kpeg}  `hard' in
complex stem nouns}\\
 
 \ea\label{exːGRM-v-sta-p-hard-head}
{ɲúú{\ꜜ}kpég} $<$ {\rm head-hard `stubbornness'} 
%pl. ɲuukpegse

 \ex\label{exːGRM-v-sta-p-hard-arm}
{nékpég} $<$ {\rm arm-hard  `stingy'} 

 \ex\label{exːGRM-v-sta-p-hard-tree}
{dààkpég} $<$ {\rm wood-hard  `strong wood'}
 
\z 
 \z
 

 Examples  are provided in (\ref{exːGRM-v-sta-p-hard}) using  {\sls kpeg} again for the sake of illustration.  Notice that only (\ref{exːGRM-v-sta-p-hard-tree}) has a transparent and compositional meaning. Verbal state stems are mainly found in complex stem nouns (Section \ref{sec:GRM-com-stem-noun}). 

%In a nominal context, the form in X occurs in a  modifier position, i.e. ,  and
%the form in Y is found in compound formation, i.e. .

% Nevertheless there are  verbal state predicates which cannot be
% nominalized. For instance, the verbs {\sls dʊa} and  {\sls jaa} cannot
%list verb state predicate
%locative; dua
%identificational; ja

\subsubsection{Complex stem nouns}
\label{sec:GRM-com-stem-noun}

A  complex stem noun, as opposed to an atomic one,  is formed by the combination of at least two stems (XY). Either X or Y in a  XY-complex stem noun may be atomic or complex.  Nominal stems ({\sc ns}), verbal state stems ({\sc ss}) and verbal process stems ({\sc ps}), together with a single noun class  suffix (and/or other types of nominaliser) are the  elements which take part in the formation of complex stem nouns. 

\ea\label{exːGRM-cplx-stm}

  \ea\label{exːGRM-cplx-stm-NS-NS-1}%
{\sls nébíí} {\rm  `finger'}\\%
ne-bi-i    {\rm [arm-seed]}\\%
 {\sc ns + ns +  cl.3sg}
 
 \newpage 
  \ex\label{exːGRM-cplx-stm-NS-NS-2}
 {\sls pàtʃɪ́gɪ́búmmò} {\rm  `liar'}\\%
patʃɪgɪ-bummo-\O \      {\rm [stomach-black]}\\%
 {\sc ns +  ns (+  cl.1sg)}

 \ex\label{exːGRM-cplx-stm-NS-SS}
 {\sls ŋmɛ́ŋhʊ̀lɪ́ɪ̀} {\rm   `dried okro'}\\%
ŋmɛŋ-hʊl-ɪ-ɪ    {\rm [okro-dry]}\\%
 {\sc ns + ss + nmlz +  cl.4.sg}
 
 \ex\label{exːGRM-cplx-stm-PS-PS}
 {\sls jàwàdír̄} {\rm  `business person'}\\%
jawa-di-r-\O \  {\rm [buy-eat-agent]}\\%
 {\sc ps + ps + nmlz (+  cl.3sg)}
 
\z 
 \z


In (\ref{exːGRM-cplx-stm-NS-NS-1}) and (\ref{exːGRM-cplx-stm-NS-NS-2}),  all stems are nominal. In (\ref{exːGRM-cplx-stm-NS-SS}),  the verbal state stem {\sls hʊl} `dry'  follows a nominal stem,  and  in  (\ref{exːGRM-cplx-stm-PS-PS}) both stems are of the type verbal process.  In these stem appositions, it is the noun class suffix of the rightmost stem which appears. Further, stems are lexemes, as opposed to nouns or verbs.  This is readily apparent in  (\ref{exːGRM-cplx-stm-NS-NS-1}) and (\ref{exːGRM-cplx-stm-NS-NS-2}), in which the leftmost stems {\sls ne} and {\sls patʃɪgɪ} would appear as {\sls neŋ} and {\sls patʃɪgɪɪ} if they were full-fledged nouns. Thus, although complex stem nouns contain more than one stem, there is only  one noun class associated with the noun and it is always the noun class associated with the rightmost stem.  This was mentioned in Section \ref{sec:GRM-sem-ass-crit} to support the claim  that semantic criteria in noun class assignment may be  non-existent. 

If  stems are treated as lexemes, there is still a problem in accounting for the ``reduced'' form of  some lexemes when they occur in stem appositions. That is, the first stem  of a complex stem noun is often reduced to a single syllable in the case of a polysyllabic lexeme, or a monosyllabic lexeme of the type CVV is reduced to CV. For example,  {\sls lúhò}  and  {\sls lúhòsó} are respectively the singular and \isi{plural} forms for `funeral' ({\sc cl.2}).  The expectation is that when the lexeme takes part in position X of a XY complex stem noun, it should exhibit its lexemic form, i.e.   {\sls luho}. Yet, the word for `last funeral'  is {\sls lúsɪ́nnà}, {\it lit.} funeral-drink,  and not {\sls *luhosɪnna}.  Not all  lexemes get reduced in that particular environment; nevertheless, it is  more discernible for polysyllabic lexemes or monosyllabic ones built on a heavy syllable. Moreover, some lexemes are more frequent in that environment than others.

\largerpage
The relation between the stems in a complex stem noun is asymmetric.  The relation is defined in terms of what the referents of the stems and the complex noun as a whole have to do with each other.  As in a syntactic relation between a head and a modifier, one of the stems modifies while the other stem is modified. The semantic relations between the stems  are of two types: ``completive'' modification and  ``qualitative'' modification. These distinctions are discussed in  Sections \ref{sec:GRM-comp-completive} and \ref{sec:GRM-comp-quality}.


\paragraph{Completive modification}
\label{sec:GRM-comp-completive}

A completive modification in a complex stem noun XY can translate as `Y of X' of which Y is the head. For instance {\sls sììpʊ́ŋ}   `eyelash', {\it lit.} eye-hair, is a kind of hair and not a kind of eye. And {\sls ʔɪ̀lnʊ̃̀ã̀} `nipple', {\it lit.} breast-mouth, is most likely seen as a kind of orifice than as  a kind of breast.  In both cases, the noun class is suffixed to the rightmost stem, incidentally to the head of the morphological construction, i.e. {\sls sììpʊ́ŋ}/{\sls sììpʊ́ná} {\sc (cl.3)} and {\sls ʔɪ̀lnʊ̃̀ã̀}/{\sls ʔɪ̀lnʊ̃̀ã̀sá} {\sc (cl.2)}. As mentioned earlier,  either X or Y  in a complex noun XY can be complex. The word {\sls népɪ́ɛ́lpàtʃɪ́gɪ́ɪ́} `palm of the hand' is an example of two completive modifications. It consists of a complex stem {\sls nepɪɛl} `hand', which is composed of  {\sls ne} `arm' and {\sls pɪɛl} `flat', and the atomic stem {\sls patʃɪgɪ} `stomach', yielding in turn  `flat of arm' and then `inside of flat of arm'. 


\paragraph{Qualitative modification}
\label{sec:GRM-comp-quality}

A qualitative modification in a complex stem noun is the same as the  syntactic modification  noun-modifier. The difference lies in the formal status of the elements: when the relation is held at a syntactic level, the elements are words, whereas at the morphological level they are stems. As mentioned earlier,  either X or Y  in a complex noun XY can be complex. For instance, the word {\sls nebiwie} consists of the combination of {\sls ne}  `arm' ({\sc cl.9}) and {\sls bi} `seed' ({\sc cl.4}), then the combination of {\sls nebi} `finger' and {\sls wi} `small'. The noun class  of {\sls wi} `small'  is {\sc cl.1}, so the singular and \isi{plural} forms for the word `little finger' are {\sls nébíwìé} and  {\sls nébíwìsé} respectively. The first relation involved is a completive modification, i.e. `seed  of arm', while the second is a qualitative one, i.e. `small seed  of arm' or `small finger'.  A qualitative modification in a complex noun XY can translate as `X has the property Y'  of which X is the head. Therefore, unlike many languages,  it is not necessarily the head of the morphological construction which determines the type of inflection.

\begin{table}

\centering
\caption[Distinction between completive and qualitative
modification]{Distinction between completive and qualitative
modification using /daa/ `tree' or `wood'.  Abbreviations: {\sc h}= head,  {\sc
m}=
modifier, {\sc ns}= nominal stem, {\sc ss}= verbal state stems,  {\sc ps}=
verbal process stem, \label{tab:GRM-complet-and-qualit}}

\fittable{
\begin{Itabular}{llllll}
\lsptoprule
&  \multicolumn{3}{c}{Structure} & Stems &  Word\\\cline{2-4}
 & Lex. type &  Function & Semantic& &\\[1ex]\midrule
%\multirow{4}{5mm}{\begin{sideways}\parbox{15mm}{Completive}\end{sideways}}
\multirow{4}{5mm}{\begin{sideways}\parbox{20mm}{Completive}\end{sideways}}

& {\sc ns-ns} & {\sc m-h}&{\sc whole-part}&/daa/-/luto/&   dààlútó\\
& &&&`tree'-`root'&`root of tree'\\[1ex]

&{\sc ns-ss}&{\sc m-h}&{\sc whole-part}&/daa/-/pɛtɪ/ &dààpɛ́tɪ́ɪ́\\
&&&&`tree'-`end'& `bark'\\[1ex]

&{\sc ns-ns}&{\sc m-h}&{\sc whole-part}&/kpõŋkpõŋ/-/daa/&kpõ̀ŋkpṍŋdāā\\
&&&&`cassava'-`wood' &`cassava plant'\\[1ex] \midrule


 \multirow{4}{5mm}{\begin{sideways}\parbox{20mm}{Qualitative}\end{sideways}}

&{\sc ns-ns}&{\sc h-m}&{\sc thing-charac}&/daa/-/sɔta/& dààsɔ̀tá\\
&&&&`tree'-`thorn'& `type of tree'\\[1ex]

&{\sc ns-ns}&{\sc h-m}&{\sc thing-charac}&/ɲin/-/daa/& ɲíndáá\\
&&&&`tooth'-`wood'&  `horn'\\[1ex]

&{\sc ps-ns}&{\sc h-m}&{\sc purpose-thing}&/tʃaasa/-/daa/&tʃáásàdāā\\
&&&&`comb'-`wood'&`wooden comb'\\[1ex]

\lspbottomrule
\end{Itabular} 
}
\end{table} 


The examples in Table \ref{tab:GRM-complet-and-qualit} illustrate the distinction between the completive and qualitative modification. The form {\sls daa}  conveys either the meaning `tree' or `wood'. Both meanings may function as head or as modifier.  If the head stem follows its modifier, it is a completive modification, and vice-versa for the qualitative modification. A semantic relation between the stems may  be a whole-part relation, a characteristic added to define an entity or a purpose  associated with an entity. 

So far,   XY-complex stem nouns were assumed to be  endocentric compounds whose head is X in qualitative modification and the head is Y in completive modification.  However, a word such as {\sls pàtʃɪ̀gɪ̀búmmò} `liar, secretive', {\it lit.} stomach-black, suggests that some XY-complex stem nouns may  either lack a head or have more than one head. These possibilities are not ignored, but in this particular case the complex stem noun may be seen as involving  the abstract senses of {\sls patʃɪgɪɪ} and {\sls bummo}, that is  `essence' and `subtle, restrained' respectively, making {\sls patʃɪgɪbummo} a qualitative modification  which can be formulated literally as `subtle/restrained essence', i.e.   a property applicable to humans. Thus, the stem {\sls patʃɪgɪɪ} is treated as the head, and {\sls bummo} as the stem functioning as the qualitative modifier. Another example is {\sls dààdùgó}. This word consists of the stems {\sls daa} `tree' and  {\sls dugo} `infest'   and refers to a type   of insect. Unlike the analysed expressions displayed in  Table \ref{tab:GRM-complet-and-qualit}  none of the stems can be treated as the head of the expression and the meaning of the whole noun cannot be transparently  predicted from its constituent parts. This leads me to provisionally consider the expression {\sls dààdùgó}  as an exocentric compound, i.e. a complex stem noun without a head.


 
\paragraph{Compound or circumlocution}
\label{sec:GRM-comp-vs-circum}

For a few expressions,  it is hard to tell whether they are compounds, i.e. the results of  morphological operations, or circumlocutions, i.e.  the results of syntactic operations \citep[165]{Alla01}. Clear cases of circumlocution nevertheless exist. For instance,  the word {\sls kpatakpalɪ} `type of hyena'  is treated by one language consultant as {\sls kpa ta kpa lɪɪ},  {\it lit.} `take let.free take leave'.\footnote{Yet {\sls kpatakpari} is the word for `hunting trap' in \ili{Gonja} \citep{Rytz66}.}  Another example is {\sls sʊ́wàkándíkùró} `parasitic plant'. This expression refers to a type of parasitic plant lacking a  root which grows upon and survives from the nutrients provided by its  hosts. The word-level expression originates from the sentence  {\sls sʊ̀wà ká ń̩ dí kùórò}, {\it lit.}  die-and-I-eat-chief, `Die so that I can become the chief'. It is common to find names of individuals being constructed in this way: the oldest woman in Ducie is known as {\sls ǹ̩wábɪ̀pɛ̄}, {\it lit.}  {\sls n̩ wa bɪ pɛ}  `I-not-again-add'. Since two successive husbands died early,  she used to say that she will never marry again. For that reason people call her {\sls ǹ̩wábɪ̀pɛ̄}.  



\subsubsection{Derivational morphology}
\label{sec:GRM-der-morph}

A derivational morpheme is an affix which combines with a stem to form a word. The meaning it carries combines with the meaning of the stem.  By definition, a derivational morpheme is a bound affix, and  thus cannot exist on its own as a word. This property keeps apart complex stem nouns and derived nouns. Yet, the distinction between a bound affix and a lexeme is not obvious, mainly because some bound affixes were probably lexemes at a previous stage, or still are today (see the morpheme {\sls bɪ} in Section \ref{sec:NUM-repet}). 

\paragraph{Maturity and sex of animate entities}
\label{sec:GRM-der-matur}

The specification of the maturity and sex of an animate entity is accomplished in the following way: male, female, young, and adult are organized in morphemes encoding one or two distinctions. These morphemes are suffixed to the rightmost stem. To distinguish between male and female, the morphemes ({\sc sg}/{\sc pl}) {\sls wal/wala} `male'  and {\sls nɪɪ/nɪɪta} `female'  are used as (\ref{exːGRM-sex-ent}) illustrates.

\ea\label{exːGRM-sex-ent}
 
 \ea  {\sls bɔ̀là-wál-\O} /  {\sls bɔ̀là-wál-á}\\
{\rm elephant-male-{\sc sg} / elephant-male-{\sc pl}   ({\sc cl.3})}
 \ex\  {\sls bɔ̀là-nɪ́ɪ́-\O}  / {\sls bɔ̀là-nɪ̀ɪ̀-tá}\\
{\rm elephant-female-{\sc sg} / elephant-female-{\sc pl} ({\sc cl.7})}
 
 
\z 
 \z

The language employs two strategies to express the distinction between  the adult animal and its young, which is  called here `maturity'.  The first is to simply add the morpheme {\sls -bi} `child'  to the head, e.g. {\sls bɔla-bie/bɔla-bise} `young elephant(s)'. In the second strategy both the sex and maturity distinctions are conveyed by the morpheme.  This is shown in Table \ref{GRM-maturity-sex}. 

\begin{table}

\caption{Morphemes encoding maturity and sex of animate entities}
\centering
 \begin{Itabular}{lcc}
\lsptoprule
&\textsc{male}&\textsc{female}\\
\midrule
\textsc{young}& -w(a|e)lee & -lor \\
\textsc{adult} & -wal & -nɪɪ \\
\lspbottomrule
 \end{Itabular} 
\label{GRM-maturity-sex}
\end{table} 

Some examples are more opaque than others. For instance, the onset consonant of the morpheme {\sls wal/wala} `male' may surface as a bilabial plosive,  e.g. {\sls bʊ̃̀ʊ̃̀mbál} `male goat'.  One can also observe a difference in form between the word {\sls pìèsíí} `sheep',   {\sls pèmbál}  `male sheep' and  {\sls pènɪ̀ɪ́} `female sheep'. The words displayed in the first three rows of Table \ref{tab:GRM-matur-sex-ex} show the least transparent derivations.  The annotation of tone is a first impression. 

\begin{table}
\caption{Maturity and sex/gender of animals}
\centering
 \begin{Itabular}{l>{\slshape}l>{\slshape}l>{\slshape}l>{\slshape}l>{\slshape}l}
\lsptoprule
Animal & {\rm Generic} & \multicolumn{2}{c}{{\rm Adult}} 
& \multicolumn{2}{c}{{\rm Young}}\\\cline{3-4} \cline{5-6}
 && \multicolumn{1}{c}{{\rm Male}} &  \multicolumn{1}{c}{{\rm Female}} & 
\multicolumn{1}{c}{{\rm Male}} &   \multicolumn{1}{c}{{\rm Female}}\\
\midrule


fowl  &zál̀& zím{\ꜜ}bál & zápúò&
zímbéléè  & zápúwìé\\

sheep   &píésíí &pèmbál & pènɪ̀ɪ́&
pémbéléè&pélòŕ\\

goat  &bʊ̃́ʊ̃̀ŋ & bʊ̃́ʊ̃́mbál   & bʊ̃̀nɪ̀ɪ́ &  bʊ̃̀mbéléè & 
bʊ̃̀ʊ̃̀lòŕ\\

pouched rat &sàpùhĩ́ẽ̀ & sàpúwál & sàpúnɪ́ɪ̀&
 sàpúwáléè& sàpúlòŕ\\

antilope   &ʔã́ã́ &ʔã̀ã̀wál  
&ʔã̀ã̀nɪ́ɪ́&ʔã̀ã̀wéléè&ʔã̀ã̀lòŕ\\

dog   &váà  &váwāl &vánɪ̄ɪ̄&
váwáléè&válòŕ\\

cat  &dìébìé &dìèbə́wāl   
&dìèbə́nɪ̀ɪ̀&dìèbə́wáléè&dìèbə́lòr\\

cow    &nã̀ɔ̃́ &nɔ̃̀wál  
&nɔ̃̀nɪ̀ɪ́&nɔ̃̀wáléè&nɔ̃̀lòŕ\\

elephant &bɔ̀là &bɔ̀lwál & bɔ̀lə̀nɪ̀ɪ́&bɔ̀wáléè    &bɔ̀llòŕ\\

% guinea fowl &sũ̀ṹ&sũ̀wál &sũ̀nɪ̀ɪ́& - & -\\
% bush mouse &ʔól&ʔólwál &  ʔólnɪ̀ɪ́ & - & -\\
% %house mouse&dàgbòŋó&dagboŋowal  &dagboŋonɪɪ& - & -\\
% lizard  &gèŕ &géwál &génɪ̀ɪ́&  -& -\\


% To add in dictionary

\lspbottomrule
 \end{Itabular}
\label{tab:GRM-matur-sex-ex}
\end{table} 



\paragraph{Native or Inhabitant of}
\label{sec:GRM-inhabitant-of}

To express `I am from X',  where `be from X' refers  to the place where someone was born and/or the place where someone lives, the verb {\sls lɪ̀ɪ̀} is used, e.g. {\sls sɔ̀ɣlá ǹ̩ lɪ̀ɪ̀} `I am from Sawla'.  Expressions with the meaning `native of X' or  `inhabitant of X'  can be  noun words referring to this same idea, that is  `being from X'.  Table \ref{tab:inhabitant-of} shows that the suffixes {\sls -(l)ɪɪ/(l)ɛɛ/la} express the  meaning `native of X' or  `inhabitant of X'. The suffixes display vowel qualities in the singular and \isi{plural} similar to those found in noun class 4 (Section \ref{sec:class4}). 

\begin{table}
\caption{Native or Inhabitant of \label{tab:inhabitant-of}}
\centering
 \begin{Itabular}{l > {\slshape}l  > {\slshape}ll > {\slshape}l > {\slshape}l}
\lsptoprule
Location &  {\sc sg}  &  {\sc pl}   & Location & {\sc sg} &  {\sc pl} \\[1ex] \midrule
Chakali   & tʃàkálɪ́ɪ́   & tʃàkálɛ́ɛ́ &
Katua  &kàtʊ́ɔ́lɪ́ɪ́   &kàtʊ́ɔ́lɛ́ɛ́\\
Motigu   &mòtígíí  &mòtígíé &%
Tiisa &  tíísàlí  &tíísàlá\\
Ducie   &dùsélíí  &dùséléé&%
Chasia  &tʃàsɪ́lɪ́ɪ́ & tʃàsɪ́lɛ́ɛ́\\
  Bulinga  & búléŋíí & búléŋéé&%
Wa &  wáálɪ́ɪ́  &wáálà\\
Gurumbele&grʊ̀mbɛ̀lɪ́lɪ́ɪ́& grʊ̀mbɛ̀lɪ́lɛ́ɛ́&%
Tuosa  &  tʊ̀ɔ̀sálɪ́ɪ́  & tʊ̀ɔ̀sálá\\
\lspbottomrule

 \end{Itabular}
\end{table} 
 

\paragraph{Category switch}
\label{sec:GRM-der-cat-switch}

The phenomenon called `category switch' refers to a derivational process whereby two words with  related meanings and composed of the same segments change category based entirely on their tonal melody. Examples are provided in (\ref{exːGRM-der-cat-switch}).

 \ea\label{exːGRM-der-cat-switch}
 \begin{tabular}[t]{lllllll}
tʊ̀mà & ({\it v})   & {\rm `work'}  &  $\leftrightarrow$ &  tʊ́má  & ({\it n})  &  {\rm `work'}\\
gʊ̀à  & ({\it v})  &   {\rm `dance'}  & $\leftrightarrow$ &  gʊ̀á  & ({\it n})  &  {\rm `dance'}\\
jɔ̀wà &  ({\it v})  &{\rm `buy'}  & $\leftrightarrow$& jɔ̀wá&  ({\it n})   &{\rm `market'}\\
mʊ̀mà & ({\it v}) & {\rm `laugh'}&  $\leftrightarrow$ &mʊ̀má &({\it n}) &{\rm  `laughter'} \\
gòrò &({\it v}) & {\rm `circle'} & $\leftrightarrow$& góró &({\it n}) & {\rm `bent'}  
 \end{tabular}
\z


\paragraph{Agent- and event-denoting nominalizations}
\label{sec:GRM-der-agent}

Apart from their roles in complex stem nouns, it was shown in Section \ref{sec:GRM-verb-act-stem} that both verbal state and verbal process stems undergo these two nominalizations processes in order to function as atomic nouns.
The two processes are summarized in (\ref{exːGRM-der-agent}) and 
(\ref{exːGRM-der-action}). 

\ea\label{exːGRM-der-agent}{\rm Agent nominalization}\\

\ea\label{exːGRM-der-agent-suffix}
{\rm A verb stem takes the suffix -[r]  to express agent-denoting
nominalization.}\\
{\sls  sʊ̃̀ã̀sʊ́ɔ́r} / {\sls sʊ̃̀ã̀sʊ́ɔ́rá} {\rm ({\sc cl.3}) `weaver(s)'}\\
  $\leftarrow$  {\sls sʊ̃̀ã̀} ({\it v}) {\rm  `weave'}\\
{\sls  lúlíbùmmùjár} / {\sls lúlíbùmmùjárá}  {\rm ({\sc cl.3})  `healer(s)' }
\\
 $\leftarrow$ {\sls lulibummo} ({\it n}) {\rm `medicine'} + {\sls ja} ({\it v})  {\rm `do'}


 \ex\label{exːGRM-der-agent-redup}
{\rm A verb stem gets partially reduplicated  to express
agent-denoting nominalization.}\\
{\sls  súsúmmá} / {\sls súsúmmə́sá} ({\sc cl.2})   {\rm`helper(s)'}\\
 $\leftarrow${\sls sùmmè} ({\it v})  {\rm `help'}\\
{\sls  sã́sáár} / {\sls sã́sáárá} ({\sc cl.3})  {\rm  `carver(s)'}\\
 $\leftarrow$ {\sls sã̀ã̀} ({\it v})  {\rm `carve' }

\z 
 \z


\begin{exe}
 \ex\label{exːGRM-der-action}{\rm Event nominalization}\\
{\rm A verb stem takes the suffix -/[{\sc +hi, -bk}]/   to express
event-denoting
nominalization.}\\
{\sls lʊ́lɪ́ɪ́ } / {\sls  lʊ́lɪ́ɛ́} ({\sc cl.4})  {\rm `giving birth'} \\
 $\leftarrow$ {\sls  lʊla} ({\it v})  {\rm `give birth' } \\
{\sls kpégíí} / {\sls   kpégíé}  ({\sc cl.4})  {\rm  `hard'  or
`strong'} \\
  $\leftarrow$ {\sls  kpeg} ({\it v})  {\rm `hard' or `strong'} \\
\z

Some consultants prefer agent nouns ending with ({\sc sg}/{\sc pl})  {\sls -r/-rV} ({\sc cl.3}),  others prefer  {\sls -ra/-rəsV} ({\sc cl.2}).  In addition, there is another agent-denoting word formation which simply adds the word {\sls kʊɔrɪ} `make' to the noun denoting the product, e.g. {\sls nã̀ã̀tɔ̀ʊ̀kʊ́ɔ́rá} / {\sls nã̀ã̀tɔ̀ʊ̀kʊ́ɔ́rəsá} ({\sc cl.1}) `shoemaker(s)' $<$ {\sls nããtɔʊ} ({\it n}) `shoe' + {\sls kʊɔrɪ} ({\it v}) `make'.


\subsubsection{Proper nouns}
\label{sec:GRM-prop-noun}

% 
% Are these the names these people were given by their families? they look like
% nick-named if they are not actually titles.  The whole paragraph sounds
%strange. %  Either get more ethnographic information on naming or cut some of
%this out.

%Proper nouns  are  usually characterized by their  inability  to inflect for
%number and to co-occur with articles and modifiers. 
As a rule,   proper nouns have  unique referents:  they  name people, places, spirits, and so on.  So in the area where Chakali is spoken, there is only one river named {\sls gòlógòló}, only one hill named {\sls dɔ̀lbɪ́ɪ́}, one village named {\sls mòtìgú},  only one shrine named {\sls dàbàŋtʊ́lʊ́gʊ́}, etc.  Nevertheless more than one person can have the same name, and the same applies to a lesser extent to villages. For instance, {\sls sɔ̀ɣlá} `Sawla' may refer to the Chakali village situated between Tuosa and Motigu, or to a \ili{Vagla} village situated at the junction of the Bole-Wa and Damongo-Wa road. To identify the former, one must say {\sls tʃàkàlsɔ̀ɣlá} `Chakali Sawla'. 

A  Chakali person may bear two or three names: his/her father's name, the name of his/her grandfather or great-grandfather, and his own (common) name. In the case of the (great-)grandfather's name, it is a feature of the newborn or an external sign which suggests the child's name.  The common name may be changed in the course of one's life. Today, regardless of whether a  person is Muslim or not, common names are mainly of \ili{Arabic}, \ili{Hausa}, and \ili{Gonja} origin, probably due to the Islamization of the Chakali  \citep{brin15c}.


Common names among the  elders (approximately above 50 years) consist of the name of a non-Chakali village,  together with {\sls nàà} `chief'. In Tuosa, Ducie, and Gurumbele, one finds one or more Kpersi Naa, Mangwe Naa, Jayiri Naa, Wa Naa, Sing Naa,  Busa Naa, etc. The next generation (approximately below 50 years) tend to have either ``Muslim'' names or ``English-title'' names. Common Muslim names are Idrissu, Fuseini, Mohamedu, Ahmed, Mohadini, etc.  Typical ``English-title'' names are {\sls Spɛ́ntà} `inspector',  {\sls Dɔ́ktà} `doctor', {\sls Títʃà} `teacher', etc. Apart from `teacher',  which can identify actual teachers in communities in which schools are present, none of the individuals are actual teachers, doctors or inspectors. The same can be said about the older generation, none of them are/were chief of Kpersi, Mangwe, Jayiri, etc.. These villages are not Chakali villages and these individuals have no real connections with the villages used in their names. It seems that these common names were trendy  nicknames that peers  assign to each other. One consultant claims that the elders can be ranked in terms of power and influence according to their nicknames. In Chakali society, one may have two additional names, a drumming name and a Sigu name ({\sls sígù}). A drumming name is used in drummed messages sent to other villages about weddings or deaths,  while a Sigu name is a name one receives when initiated to the shrine {\sls  dààbàŋtólúgú}. 


Because of their pragmatic function,  proper nouns  are rarely observed in a \isi{plural} form, but some contexts may allow this. In (\ref{ex:GRM-propn-noun-plur}), the proper name {\sls Gbolo} takes the \isi{plural} marker {\sls -sV}.\footnote{The context of (\ref{ex:GRM-propn-noun-plur}) makes sense when one understands that the name `Gbolo' has a particular meaning.  It is understood that when a couple has a  fertility problem,  it is common to travel to the community of Mankuma and to consult their shrine. If the woman gets pregnant after the visit, they must return to Mankuma to appease the shrine. Subsequently, the child must be named `Gbolo' and automatically acquires the Red Patas monkey as  totem.}


  \begin{exe}
   \ex\label{ex:GRM-propn-noun-plur}
\gll  gbòlò-só bá-ŋmɛ̀nàá ká dʊ̀à dùsèè ní.\\
gbolo.({\sc g.}b)-{\sc pl}  {\sc g.}b-{\sc q}.many {\sc  egr} exist Ducie {\sc
postp}\\
\glt  `How many Gbolos are there in Ducie?' 
 \z

Finally, circumlocution is a common process found in names of people and dogs (e.g. the example of {\sls n̩wabɪpɛ}, {\it lit.}  {\sls n̩ wa bɪ pɛ}  `I-not-again-add', was given in Section \ref{sec:GRM-comp-vs-circum}).   A few examples of dog names are given in (\ref{ex:GRM-propn-dog-name}).

  \ea\label{ex:GRM-propn-dog-name}{\rm Dog names}\\

 \ea\label{ex:GRM-propn-dog-name-1} 
{\sls jàsáŋábʊ̃̀ɛ̃̀ɪ̀} {\rm  `Let's keep peace'}\\
ja-saŋa-bʊ̃ɛ̃ɪ    {\rm [we-sit-slowly]} 
 \ex\label{ex:GRM-propn-dog-name-2} 
{\sls ǹ̩nʊ̃̀ã́wàjàhóò}  {\rm  `I will not open my mouth again'}\\
 n̩-nʊ̃ã-wa-ja-hoo   {\rm [my-mouth-not-do-hoo]}
 \ex\label{ex:GRM-propn-dog-name-3} 
{\sls kùósòzɪ́má}   {\rm `God knows'}\\
kuoso-zɪma  {\rm  [god-know]}

\z 
 \z


% In many folktales, animals are the characters and their acts evoke
% \isi{human} beings' behaviors. There is a tendency in folktales to switch between
% using the animal name as common noun and using it as \isi{proper noun}. (example)
%1. born in village but moved from there
%2. not necesarily chief but from royal family
%3. gan naa= `more than a chief' grand father of John
%4. pure nick name

\subsubsection{Loan nouns}
\label{sec:GRM-borr-noun}

A loan noun,  or more generally a loanword, can be defined as  ``a word that at some point came into a language by transfer from another language'' \citep[58]{Hasp08}.  When a word is found in both Chakali and in another language, many loan scenarios are conceivable. However,  for some semantic domains such as   bicycle or car parts, school material, and so on, the past and present sociolinguistic situations  suggest that Chakali is the recipient language and \ili{Waali}, English, \ili{Hausa}, and  \ili{Akan} are the donor languages.  Loan scenarios differ and are harder to establish when other SWG languages are involved. It is often unfeasible to demonstrate whether the same form/meaning in two languages was inherited from a common ancestor, or  borrowed by one and subsequently passed on to other SGW languages. Moreover, it may be unwise to assume that in all cases Chakali is  the recipient language, especially for loanwords in domains which  were in the past fundamental in Chakali lifestyle,  but to a lesser degree for neighbouring ethnic groups. Thus, Chakali as a donor language can be evaluated in a wider \ili{Grusi}-Oti Volta genesis, or  at a micro-level where the influence of Chakali on \ili{Bulengi} is established. It is unlikely that Chakali borrowed from English through contact. And Ghanaian English, in Wa town and Chakali communities,  is not an effective mode of communication, at least in social spheres where the majority of  Chakali men and women interact (see discussion in Section \ref{sec:vital} and in \citealt{brin15c}).  Nonetheless, the situation is different for school children  who are exposed to Ghanaian English on a regular basis. I believe that Ghanaian English spoken by native speakers of \ili{Waali}, \ili{Dagaare}, or Chakali  is the only potential variety of English which can function as a donor language. Examples of words ultimately  from English origin are: {\sls bə̆̀lùù} `blue', {\sls ʔásɪ̀bɪ́tɪ̀} `hospital', {\sls dɔ́ktà} `doctor', {\sls bàlúù} `balloon',{\sls bɛ́lɛ́ntɪ̀} `belt',  {\sls tə̆́rádʒà} `trouser',  {\sls détì} `date', {\sls mɪ́ntɪ̀} `minute',   {\sls dʒánsè} `type of dance',  {\sls kàpɛ́ntà} `carpenter', {\sls kɔ́lpɔ̀tɛ̀} `coal pot', {\sls kɔ́tà} `quarter', {\sls lɔ́ɔ́lɪ̀} `lorry (any four-wheel vehicle)',   {\sls sákə̀r} `bicycle',  {\sls pɛ̀n} `pen', {\sls sùkúù} `school',   {\sls tʃítʃà} `teacher' and many more.  There is a recurrent falling tonal melody (i.e. HL) among the loan nouns of  ultimately English origins. Many of them,  if not all, can be found in other languages of the area \citep{sisa75, daku07}. 

\newpage 
When a word is found both in \ili{Waali} and Chakali, it is not automatically classified as borrowed from \ili{Waali}, yet it is only suspected to be non-Chakali.  Examples such {\sls dʒɪ́ɛ́rá} `sieve', {\sls dʒùmbúrò} `type of medicine', {\sls gbàgbá} `duck', {\sls kókódúró}  `ginger', {\sls kàpálà} `fufu', {\sls kã́ʊ̃́}  `mixture of sodium carbonate', {\sls nààsáárá}  (\ili{Hausa}) `Caucasian person', and  {\sls sànsánná} `prostitute' are some of the \ili{Waali}/Chakali nouns found in transcribed texts, or by chance. 

The weekdays are from \ili{Arabic} (probably via \ili{Hausa}). \ili{Vagla} and Tumulung \ili{Sisaala},  but  not \ili{Dɛg}, use similar expressions \citep[60]{nade96}: {\sls ʔàtànɪ̃́ɛ̃̀} `Monday', {\sls ʔàtàláátà} `Tuesday', {\sls ʔàlàrbá} `Wednesday',   {\sls ʔàlàmʊ́sà} `Thursday',  {\sls ʔàrɪ̀dʒímà} `Friday', {\sls ʔàsɪ́bɪ́tɪ̀} `Saturday', and {\sls ʔàlàháádì} `Sunday'.  The expressions for the lunar months seem to be borrowed from \ili{Waali}, but Dagbani and Mamprusi have similar expressions. In these \ili{Oti-Volta} languages, some of the names  correspond to important festivals, i.e. 1, 3, 7, 9, 10, and 12 below. In Chakali, only {\sls dʒɪ́mbɛ̀ntʊ́} is celebrated and  is considered the first month.\footnote{Dagbani {\sls buɣum} and \ili{Waali} {\sls  dʒɪmbɛntɪ} are both treated as first month by the speakers of these languages.} The lunar months are: {\sls dʒɪ́mbɛ̀ntʊ́} `first month (1)', {\sls sífə̀rà}  `second month (2)', {\sls dùmbá} `third month (3)', {\sls dùmbáfúlánààn} `fourth month (4)',  {\sls dùmbákókórìkó} `fifth month (5)', {\sls kpínítʃùmààŋkùná} `sixth month (6)', {\sls kpínítʃù} `seventh month (7)', {\sls ʔàndʒèlìndʒé} `eighth month (8)', {\sls sʊ́ŋkàrɛ̀} `ninth month (9)', {\sls tʃíŋsùŋù} `tenth month (10)', {\sls dùŋúmààŋkùnà} `eleventh month (11)' and {\sls dùŋú} `twelth month (12)'.  It was understood that these terms and concepts are not known by the majority, especially the younger generations.


\subsubsection{Relational nouns}
\label{sec:SPA-relnoun}


Many  languages present formal identity between body parts terms and expressions used to designate elements of space. The widely accepted view is that diachronically  spatial relational nouns  -- sometimes called spatial nominals \citep[895]{Hell07},  or adpositions  \citep[137]{Hein97} --  are ``the result of functional split'' and that ``they are derived from nouns denoting body parts or locative concepts through syntactic reanalysis'' \citep[256]{Hein84}.



Chakali relational nouns are formally identical to body part nouns although not all body part nouns have a \isi{relational noun} counterpart. For instance, whereas {\sls ɲuu} can have both  a spatial meaning, i.e. `on top of X', and  a body part one, i.e. `head',  the body part terms {\sls bembii} `heart', {\sls hog} `bone'  or {\sls fʊ̃ʊ̃} `lower back', among others,  cannot convey  spatial meanings. Table \ref{tab:relvsbody} displays the body parts found in the data which convey spatial meaning.\footnote{The body part term {\sls gàntàĺ} `back' is from the Ducie lect and corresponds to {\sls hàbʊ̀á} in the Motigu, Gurumbele, Katua, Tiisa, and Tuosa lects.}

\begin{table}
\caption[Spatial nominal relations and body part nouns]{Spatial nominal
relations and body part nouns: similar forms and different, but related,
meanings\label{tab:relvsbody}}
\centering
\begin{small}
 \begin{Qtabular}{lllll}
\lsptoprule
Projection& Spatial relation & PoS: {\it reln}  & Body parts &
 PoS: {\it n}\\\midrule
Intrinsic & &&&\\

& \textsc{top}  & {\sls ɲuu} (x,y)  & head & {\sls  ɲuu} (x)\\
&\textsc{containment} &  {\sls patʃɪgɪɪ} (x,y)  & stomach & {\sls 
patʃɪgɪɪ} (x)\\
& \textsc{side} &  {\sls  logun} (x,y)& rib & {\sls logun} (x)\\
& \textsc{mouth} &  {\sls  nʊ̃ã} (x,y)& mouth & {\sls   nʊ̃ã} (x)\\
& \textsc{base/under} &  {\sls muŋ} (x,y)& arse & {\sls  muŋ} (x)\\
& \textsc{middle} &  {\sls bambaaŋ} (x,y)& chest box & {\sls  bambaaŋ} (x)\\
\cline{1-5}

Relative  & &&&\\
& \textsc{left} &  {\sls neŋgal} (x,y) & left hand &  {\sls neŋgal} (x)\\
& \textsc{right}  & {\sls nendul} (x,y)  & right hand &{\sls nendul} (x)\\
& \textsc{back} &  {\sls gantal} (x,y)  & dorsum & {\sls gantal} (x)\\
& \textsc{front} & {\sls sʊʊ} (x,y)  & front  & {\sls sʊʊ} (x)\\

 \lspbottomrule
 \end{Qtabular}
\end{small}

\end{table} 




%Also suppose that you put the cutlass you used to cut the
% head on the head. Now you can say, 'a karintie saga a nyuu nyu ni'. In your
% given example, the kpulikpuli is on the head. Let's change it and say the
% kpulikpuli is under the head. Then we would have, 'a kpulikpuli dua a nyu mun
% gantal gantal 'behind the back'
% nua nua  'at the entrance of the
% mouth' pachigi pachigi 'inside the stomach' (Kasim)

How can we distinguish a \isi{relational noun} from a noun?  Above all,  the differentiation between relational nouns and body part nouns cannot rely solely on surface syntax criteria, precisely because the configuration of a \isi{possessive} noun phrase and a \isi{relational noun} phrase are identical. This is shown in (\ref{ex:SPA-reln-vs-bpn}). 

 
\ea\label{ex:SPA-reln-vs-bpn}
\ea\label{ex:SPA-bpn}{\rm Possessive attributive phrase}\\
 {\rm {[\textsc{n}$_{1}$-\textsc{n}$_{2}$]}$_{NP}$ where \textsc{n}$_{2}$ = body 
part,   e.g. {\sls báál  ɲúù} {\rm `a man's head'}}

\ex\label{ex:SPA-reln}{\rm Spatial nominal  phrase}\\
 {\rm {[{\sc n}$_{1}$-{\sc n}$_{2}$]}$_{NP}$ where {\sc n}$_{2}$ = spatial relation,   
e.g. {\sls téébùl  ɲúù}  {\rm  `top of the table'}}
\z
\z

Even though the two corresponding nominal structures may cause ambiguities, the interpretation is generally disclosed by the meaning of the nominal preceding the \textsc{n}$_{2}$ in  (\ref{ex:SPA-reln-vs-bpn}). The term  {\sls  ɲuu}, for instance, can only mean `top of' in a phrase in which it follows another nominal and refers to a projected location of \textsc{n}$_{1}$'s referent. In (\ref{ex:SPA-bpn}), even though {\sls ɲuu} immediately follows a nominal,  it would not normally refer to the projected location `on the top' but only to the man's head. Nevertheless, despite any attempts to identify  structural characteristics which may contribute to the disambiguation of a phrase involving a body part term,  ambiguities may still arise.


%a kpulikpuli dua a statue nyuu nyuu ni
Another aspect of body part terms is their different function in  morphological and syntactic structure. While a \isi{relational noun} is a syntactic word,  body part terms may also function as morphemes in compound nouns to express a specific part-whole relationship or a conventionalized metaphor \citep[141]{Hein97}. Whereas the distinction may be formally and semantically hard to distinguish, the number of body part terms which can be the stem in a compound noun is larger than those functioning as relational nouns. Some examples are shown in Table \ref{tab:SPA-bpt-compound}.
%fix the \isi{etymology}

\begin{table}
\caption{Body part terms in compound nouns\label{tab:SPA-bpt-compound}}
\centering
%\begin{small}

\fittable{
 \begin{Qtabular}{l>{\slshape}llp{1.6in}}
\lsptoprule
Body part term & {\rm Compound noun}  & Morph. gloss & Gloss\\\midrule


eye & tɔ́ʊ́-{\ꜜ}síí  & village-eye & village's center\\
 & kpã̀ã̀n-síí & yam-eye & yam stem\\
& nɪ̀ɪ̀-síí & water-eye & deepest area of a  river\\
 &  nã̀ã̀-síí & leg-eye & ankle bump\\

 mouth &   gɔ̀ŋ-nʊ̃̀ã́ & river-mouth & river bank\\
		&   ʔɪ̀l-nʊ̃̀ã́   & breast-mouth& nipple\\
    &   dɪ́à-nʊ̃́ã̀  & house-mouth& door\\

leg &   gɔ́n-{\ꜜ}nã́ã́   & river-leg & split of a river\\

&   dáá-{\ꜜ}nã́ã́  & tree-leg & branch\\




 head &   kùósò-ɲúù  & god-head & sky\\
    &   tìì-ɲúù  &land-head  ({\it etym})& west\\
%stomach&   \sls tɔ̀ʊ́pàtʃɪ́gɪ́ɪ́  & village-stomach & inside the village\\

 arse &  tìì-múŋ &land-arse  ({\it etym}) & east\\

neck &  vìì-báɣə̆́ná & pot-neck & neck of a container\\

testicle &   mááfà-lúró  &  gun-testicle &  gun powder container\\

penis &  mááfà-péŋ  & gun-penis &gun trigger\\

ear &   mááfà-dɪ́gɪ́ná   & gun-ear & flintlock frizzen\\

arm &  fàlá-néŋ̀ & calabash-arm & calabash stem\\
 
navel & fà-ʔúl & calabash-navel & calabash node\\

nose & píí-mɪ́ɪ́sà & yam mound-nose &  part of a yam mound\\
liver & tɔ́ʊ́-pʊ́ɔ̀l & village-liver &  important community member\\
\lspbottomrule
 \end{Qtabular}
}
%\end{small}

\end{table} 


Ignoring for the moment the structure in which they are involved, there seem to be two types of spatial interpretation accessible with body part terms. And there also seems to be a gray zone between the two.\footnote{This gray zone may receive a diachronic interpretation.  In \citet[1072]{Amek07c},  the postpositions in Sɛkpɛlé are seen as evolving ``from body part and environment terms''  and have a similar, but not identical, function as those of Chakali relational nouns. For instance, Sɛkpɛlé's  postpositions ``cannot be modified'' nor can they vary ``with respect to number marking''.} The first interpretation is the literal attribution of \isi{human} characteristics (i.e. anthropomorphic) in  reference to parts of object. In such a case, a body part term refers to a part of an object in analogy to an animate entity. For instance, a trigger of a gun (i.e. the lever that activates the firing mechanism) is  called its `penis',  to characterize its physical appearance. The second interpretation does not designate a fixed part of an object but a location projected from a part of an object. It designates a spatial environment in contact with or detached from an object \citep[44]{Hein97}. To make the distinction clear,  in the sentence `a label is glued on the neck of the bottle' the body part term `neck' designates a breakable part of the bottle, whereas in the sentence `John is standing at the back of the car' the body part term `back' does not designate any part of the car but a relative spatial location, the area behind the car. 


% 
%  In \citet[44]{Hein97}, the variety of denotations of body part terms is
% accounted for in a diachronic perspective. The  claim is sketched in
% (\ref{ex:Heine-stage}).
% 
% 
% \ea\label{ex:Heine-stage}{\rm From body part to spatial concept: A four-stage
% scenario \citep[44]{Hein97}}\\
% Stage 1: a region of the \isi{human} body\\
% Stage 2: a region of an (inanimate) object\\
% Stage 3: a region in contact with the object\\
% Stage 4:  a region detached from the object\\
% \z
% 
% 
% However, synchronically each of these stages is observable: A  Chakali 
% \isi{relational noun} is more easily interpreted as  a region in contact with or
% detached from an object, a body part term in a compound
% noun designates
% a part of an object, and a body part term used as
% a full-fledged noun is associated with a part of the \isi{human} body. 
% Nevertheless, the examples provided in Table \ref{tab:SPA-bpt-compound} show
% that the distinction is not a clear-cut one: Does the expression {\sls teebul 
% ɲuu}
% designate a spatial environment  in contact with or detached from a table or  a
% part of table? Both interpretations seem acceptable.

% % For instance, {\sls dáámúŋ}, literally `tree under', can  mean either a
% % resting area or a location for the initiation rite (neither being 
%%obligatorily
% % under a tree), and the sentence {\sls ʊ dʊa daa muŋ nɪ} can only be 
% interpreted
% %as
% % `it is under the tree', but never as `it is at the resting area' nor as  `it
% %is
% % at the location for the initiation rite'.


Relational nouns are rarely found in the \isi{plural}. Yet, on grammatical grounds, nothing prevents them from being expressed in the \isi{plural}. To describe a situation where  for every bench there is a calabash sitting on it, the sentence in (\ref{ex:reln-plu-head}) is appropriate.


\ea\label{ex:reln-plu-head}
\gll à fàlàsá ságá à kóró ɲúúnó nɪ̄.\\
{\sc art} calabash.{\pl}  sit {\sc art} bench.{\pl}  {\reln .\pl} {\postp}\\
\glt `The calabashes sit on top of the benches.' 
\z

One may argue that the `top of a bench' is a spatial environment  in contact with the bench, even a physical part of the bench, so pluralization may simply suggest that the   `top of a bench' is a word referring to an entity,  and not a locative phrase. Two pieces of evidence go against this view: first,  notice that {\sls koro} `bench'  in {\sls koro ɲuuno} is \isi{plural}.  Recall Section \ref{sec:GRM-com-stem-noun},  in which a noun class ({\sc sg/pl} marking) was argued to  appear only at the end of a word. If  `top of a bench' was a word and not a phrase, we would expect its \isi{plural} form to be  *{\sls korɲuuno}. Secondly, deciding whether or not the `top of' is indeed in contact with or detached from the bench is not conclusive. To describe a situation where several balls are under several tables, one may use the sentence in (\ref{ex:reln-plu-stomach}), in which case it cannot be argued that  under of the table is a physical part of the table.\footnote{One may argue that it is indeed a part of the table, identical to the interior space of a container.}

\ea\label{ex:reln-plu-stomach}
\gll à bɔ́lsā dʊ́á à téébùlsō pátʃɪ̀gɪ̄ɛ̄ nɪ̄.\\
{\sc art} ball.{\pl}  be.at {\sc art}   table.{\pl}  {\reln .\pl} {\postp}\\
\glt `The balls are under the tables.'
\z



% kopu suguli a teebul nyu- logun ni
% a haan dua u gantal ni 'the woman is at his back'
% The above sentences are right. But in the case of "a haana dua ba gantala ni"
%we  only pularise the subject (haana) but keep the location (gantal) singular
% because 'ba' has already indicated that they are many. So we have 'a haana dua
% ba gantal ni' NOT 'a haana dua ba gantala ni'. However, a balsa dua teebulSO
% pachigiE ni is right. Why because, we are referring to objects and not humans
%in  which case we cannot use 'ba' indicate it's a single object or many. For
% instance, we can have " a fota saga dasa nyuuni ni" (the baboons are on the
% trees) 'a zinni laga da nyu ni' (the bats are hanging on a tree).

Another aspect of relational nouns and oblique phrases in general is that they are structurally very rigid, that is, they are not easily extracted or preposed. The sentences in  (\ref{ex:reln-extract-in-1}) and  (\ref{ex:reln-extract-in-2})  are nevertheless acceptable. 

\ea\label{ex:reln-extract}

\ea\label{ex:reln-extract-in-1}
\gll  à téébùl  ɲúú nī, à fàlá sàgà.\\
 {\sc art } table {\reln} {\postp}  {\sc art } calabash sit\\
\glt `On top of the table, the calabash sits.'

 \ex\label{ex:reln-extract-in-2}
\gll  téébùl lō, à fàlá ságá ʊ̀   ɲúú nī.\\
 table {\foc} {\sc art} calabash sit {3.\sg.\poss} {\reln. top} {\postp}\\
\glt `Table, the calabash sits on top of it.' ({\it lit.} `sits on its head')

 \ex\label{ex:reln-extract-out-3}
*  teebul lo, a fala saga  ɲuu nɪ.

 \ex\label{ex:reln-extract-out-1}
 *  ʊ  ɲuu nɪ, a fala saga teebul.
 
 \ex\label{ex:reln-extract-out-2}
 * ɲuu nɪ, a fala saga teebul.

\z
\z

The sentence in  (\ref{ex:reln-extract-in-2}) is acceptable but odd. It shows that the nominal complement of the \isi{relational noun} {\sls ɲuu} can be uttered at the beginning of the sentence while the possessive \isi{pronoun} {\sls ʊ}  is located  in the complement slot of   the \isi{relational noun}, functioning as anaphora. The sentence is ungrammatical if the \isi{pronoun} is absent {\it in situ} (\ref{ex:reln-extract-out-3}),  or if the oblique phrase is preposed but the nominal {\sls teebul} stranded, whether an anaphora referring to {\sls teebul} is present  (\ref{ex:reln-extract-out-1}) or absent (\ref{ex:reln-extract-out-2}). 


We now have  evidence for treating the relational nouns as
members of a closed class of lexical items whose function is to localize the
figure to a search domain.  It is not only that body part terms acquire spatial meaning
following a noun referring to inanimate entities, but that, in diachrony, 
only a limited set of body part terms has acquired that spatial meaning,  and, in synchrony,  they form a subtype of nominal identified as \isi{relational noun}. They are 
nouns since they can pluralize, but they acquire the status of functional words
since they constitute a formal class with limited membership where each of
the members expresses spatial meaning and requires a nominal complement.

\ea\label{ex:postp-struct}
 {\rm [[[{\sls a dɪa}]$_{NP}$ {\sls ɲuu}]$_{RelP}$ {\sls nɪ}]$_{PP}$   {\rm`on the roof of the house' }}
\z

In (\ref{ex:postp-struct}), the \isi{relational noun} {\sls ɲuu} is within the complement phrase of the \isi{postposition} {\sls nɪ}.  A \isi{relational noun} phrase (RelP) consists of a head  and noun phrase complement.  We are now in a  better position to state that the complement phrase of the \isi{postposition} is a (nominal) phrase which corresponds to the conceptual ground. 

To summarize, on a diachronic basis, it is believed that the function of relational nouns  as locative adpositions originates from their purely `entity' meaning through grammaticalization \citep[44, 83]{Hein84}. The form of Chakali body part terms supports the claim.   On a synchronic basis, only  {\sls patʃɪgɪɪ} `stomach',  {\sls logun } `rib',  {\sls gantal} `dorso', {\sls muŋ}   `arse', {\sls nʊ̃ã} `mouth',  {\sls sʊʊ} `front', {\sls bambaaŋ} `chest box'  and {\sls ɲuu} `head' are relational nouns. Relational nouns are  nouns which lack the referential power of the default interpretation of  body part term  (i.e. interpreted in  isolation), and which take a complement which must obligatorily be filled by an entity capable of projecting a spatial environment.



\subsection{Pronouns and pro-forms}
\label{sec:GRM-pronouns}

 A \isi{pronoun} is a type of pro-form.  The difference between pronouns and pro-forms depends on whether they can be anaphors of nominal arguments. In this section, the personal, impersonal, demonstrative, and possessive pronouns are introduced, followed by the expressions used to convey reciprocity and reflexivity.   In Section \ref{sec:GRM-adv-pro},  the adverbial pro-forms are  introduced.

\largerpage[-1]
\subsubsection{Personal pronouns}
\label{sec:GRM-personal-pronouns}

Table \ref{tab:GRM-pers-pro} gives an overview of the personal \isi{pronoun} forms.

\begin{table}
 \caption{{weak pronoun}Weak, {strong pronoun}strong, and {emphatic pronoun}emphatic forms of personal 
pronouns\label{tab:GRM-pers-pro}}
  \centering
  \begin{Itabular}{l>{\slshape}l>{\slshape}l>{\slshape}l}
\lsptoprule 
Pronoun & {\rm Weak ({\sc wk})}   &  {\rm Strong ({\sc st})}  &  {\rm  Emphatic 
({\sc emph})} \\
Gram. function  & {\sc s|a} {\rm and} {\sc o}  &  {\sc s|a} & {\sc s|a}\\[1ex]
\midrule
{\sc 1sg} &  n̩ &   mɪ́ŋ & ń̩wà\\
{\sc 2sg}  &   ɪ& hɪ́ŋ & ɪ́ɪ́wà\\
{\sc  3sg}  &  ʊ&  wáá & ʊ́ʊ́wà\\
{\sc 1pl}  &   ja&  jáwáá & jáwà\\
{\sc 2pl} & ma &   máwáá & máwà\\
{\sc  3pl.g}a &  a  &   áwáá & áwà\\
{\sc 3pl.g}b  &   ba&   báwáá & báwà\\
 
\lspbottomrule
  \end{Itabular}
\end{table}

The weak form first person singular \isi{pronoun} is a syllabic nasal which assimilates its place feature to the following phonological material (Section \ref{sec:ext-nasal-place}).  All weak forms may be \is{lengthening}lengthened in the imperfective (Section  \ref{sec:GRM-trans-intran}). The  personal pronouns  do not encode a gender distinction in the singular but  an animacy  distinction is made between \isi{non-human} and  \isi{human} in the \isi{plural}. They are glossed {\sc 3pl.g}a and {\sc  3pl.g}b  respectively (Section \ref{sec:GRM-gender}).  The weak forms can surface either with a low or high tone; when an action  has not yet occurred or a wish is expressed the \isi{pronoun} is perceived with a high tone (Section \ref{sec:GRM-subjunctive}). Otherwise the weak forms normally have low tones.  The strong and emphatic forms are attested with the melodies with which they are associated in Table \ref{tab:GRM-pers-pro}. 


\largerpage[-1]
 \exewidth{(123)}
  \ea\label{ex:GRM-pro-WSE}  
   \ea\label{ex:GRM-pro-W}
\gll ʊ̀ dí kʊ̄ʊ̄ rā.\\
  {\sc 3sg} eat t.z. {\sc foc}\\
\glt  `She ate {\sc t.z.}.' 
   
   \ex\label{ex:GRM-pro-S}
   \gll wáá dí kʊ̄ʊ̄ (*ra).\\
  {\sc 3sg.st} eat t.z.  {\sc foc}\\
\glt  `{\sc she} ate t.z.' 
   
 \ex\label{ex:GRM-pro-E}  
   \gll ʊ́ʊ́wà dí kʊ̄ʊ̄ rā.\\
  {\sc 3sg.emph} eat t.z.  {\sc foc}\\
\glt  `{\sc it is her} who ate t.z.' 

  \ex\label{ex:GRM-pro-S-cleft}  
   \gll  wáá  m̩̀ màŋà (*ra).\\
  {\sc 3sg.st} {\sc 1sg}  beat  {\sc foc}\\
 \glt   `{\sc him}, I beat.'
  \ex\label{ex:GRM-pro-E-cleft}  
   \gll  ʊ́ʊ́wà  m̩̀ máŋʊ́ʊ́ rā.\\
  {\sc 3sg.emph}  {\sc 1sg}  beat.{\sc 3sg}   {\sc foc}\\
 \glt   `{\sc it is him} who I beat.'
 
 \ex\label{ex:GRM-pro-W-cleft}  
 * (ʊ|waa) m̩ maŋʊʊ ra.
  \z 
  \z
  
The sentences in (\ref{ex:GRM-pro-W})-(\ref{ex:GRM-pro-E}) show that while a weak or an emphatic \isi{pronoun} can co-occur with a \isi{focus} particle,  a \isi{strong pronoun} cannot. In addition, (\ref{ex:GRM-pro-W-cleft})-(\ref{ex:GRM-pro-E-cleft}) confirm that both emphatic and strong pronouns may be fronted, but weak pronouns cannot.  Both emphatic and strong pronouns typically appear  at the beginning of a sentence. An \is{emphatic pronoun}emphatic \isi{pronoun} may be coreferential with a \isi{weak pronoun} in the clause, while weak and strong pronouns may not,  as  (\ref{ex:GRM-pro-S-cleft}-\ref{ex:GRM-pro-W-cleft}) demonstrate.   The  distinction between weak and strong  is relevant when pronouns function as subject.  Their proper use is 
conditioned by the emphasis  placed on the participant(s) of the event or the 
event itself, and by the polarity of the clause in which they appear.\footnote{The purpose of such 
distinctions derives mainly from the articulation of information. \citet{Purv07} offers an analysis 
for Dagbani whereby personal \isi{pronoun} forms vary depending on their position in relation to their 
lexical host.}  In this way, strong pronouns cannot co-occur in a sentence in which another
constituent is in \isi{focus}, that is a nominal phrase flanked by the \isi{focus} marker  or   a verb ending with the assertive suffix vowel   {\sc -[+ro,  +hi]} (compare examples (\ref{ex:GRM-out-STR-FOC-buy}) and (\ref{ex:GRM-out-STR-FOC-finish}) with (\ref{ex:GRM-out-STR-FOC-buy-g}) and  (\ref{ex:GRM-out-STR-FOC-finish-g})). In addition,  in sentences where a negative operator occurs, strong pronouns are disallowed, as  (\ref{ex:GRM-out-STR-NEG-buy}) and   (\ref{ex:GRM-out-STR-NEG-finish}) show.

\ea\label{ex:GRM-weak-strong-arg} 
   \ea\label{ex:GRM-out-STR-FOC-buy-g}
\gll   mɪ́ŋ   jáwàà   kɪ̀nzɪ́nɪ́ɪ̀.\\
  {\sc 1sg.st}  buy  horse\\
\glt  `I bought a horse.' 

   \ex\label{ex:GRM-out-STR-FOC-buy}
 *mɪŋ   jawa   kɪnzɪnɪɪ ra.
   
   
%    \ex 
% \gll ǹ jáwá kɪ̀nzɪ́nɪ́ɪ́ rà.\\
%   {\sc 1sg.wk}  buy  horse  {\sc foc}\\
% \glt  `I bought a HORSE.'
%    
   \ex 
\gll ǹ̩ wà jáwá kɪ̀nzɪ́nɪ́ɪ̀.\\
 {\sc 1sg.wk} {\sc neg} buy  horse\\
\glt  `I did not buy a horse.' 
  

   \ex\label{ex:GRM-out-STR-NEG-buy}
  *mɪŋ wa  jawa   kɪnzɪnɪɪ.
  
     \ex\label{ex:GRM-out-STR-FOC-finish-g}
\gll ǹ̩   pétījó.\\
 {\sc 1sg.wk}  {terminate.{\sc pfv.foc}}\\
\glt  `I finished.' 


   \ex\label{ex:GRM-out-STR-FOC-finish}
  *mɪŋ   petijo.

   \ex 
\gll mɪ́ŋ   pétījé.\\
  {\sc  1sg.st} {terminate.{\sc pfv}}\\
\glt   `I finished.'  

   \ex\label{ex:GRM-out-STR-NEG-finish}
  *mɪŋ   wa petije.
 
   
  \z 
     
 \z
%clitizised \isi{pronoun} in objetc position here. Bring a CVVV example
  
  
 \subsubsection{Impersonal pronouns}
 \label{sec:GRM-impers-pro}

An impersonal \isi{pronoun} does  not refer to a particular person or thing. The form
{\sls a} is used as an impersonal \isi{pronoun} in some particular context.

      
\begin{exe} 
\ex\label{ex:imps-pro-sing}
\gll à mááséjó kéŋ̀.\\
  {\sc 3sg.imps}  enough.{\sc pfv.foc} {\sc dxm}\\
\glt `That's enough' or `That's it' or `Stop'
\z

Example (\ref{ex:imps-pro-sing}) is a type of impersonal construction.  It is 
characterized by its  subject position being  occupied by the \isi{pronoun} {\sls a}, which may be seen 
as 
referring to the situation,  but not to any participant: this particular example is appropriate in 
contexts involving pouring  liquids or giving food on a plate, or when people are quarrelling. In 
these hypothetical contexts, using the personal \isi{pronoun} {\sls ʊ} instead of  the  impersonal 
\isi{pronoun} 
 {\sls a} would be unacceptable.  

The language does not have a passive construction as one finds in English, for example. Nonetheless, an argument  can be demoted  by placing it in object position, here as \is{{\sc o}-clitic}{\sc o}-clitic  (see Section \ref{sec:GRM-morph-opro}). This is shown in (\ref{ex:GRM-vp23.3.}).


\ea\label{ex:GRM-vp23.3.}
\gll ká à nàmɪ̃̀ã́?  bà tíéú rò.\\
   {\sc q}.where {\sc art} meat {\sc 3pl.g}b  eat.{\sc pfv}.{\sc 3sg.o} 
{\sc foc}\\
\glt  `Where is the meat? It has been eaten.'
\z


The type of impersonal construction illustrated in  (\ref{ex:GRM-vp23.3.})  is characterized by the personal \isi{pronoun} {\sls ba} ({\sc 3pl.g}b)  in subject position. In this context,  the subject is not a known agent and the \isi{pronoun} {\sls ba} does not refer to anyone/anything in particular. Therefore,  the pair {\sls a}/{\sls ba} is treated  as the singular and \isi{plural} impersonal pronouns, only when they occur in impersonal constructions, as shown above.



\subsubsection{Demonstrative pronouns}
\label{sec:GRM-demons-pro}


In the examples (\ref{ex:GRM-demons-pro-reply-1}) to (\ref{ex:GRM-demons-pro-quest}),  the demonstrative pronouns  function as noun phrases. All the examples below were accompanied with pointing gestures when uttered.

\ea\label{ex:GRM-demons-pro-reply-1}{\rm Replies to the question: Which cloth
has she chosen?}
 
 
  \ea\label{ex:GRM-demons-pro-reply-1sg} 
 \gll háǹ nā.\\
   {\sc dem.sg} {\sc foc}\\
 \glt `It is this one' 
   
   \ex\label{ex:GRM-demons-pro-reply-1pl}
    \gll hámà rā.\\
   {\sc dem.pl} {\sc foc}\\
 \glt `It is these ones' 
 
\z 
 \z



\ea\label{ex:GRM-demons-pro-quest}{\rm The speaker asks the addressee whether 
he had moved a certain object.} 

 \gll   ɪ̀ jáá háǹ nȁ?\\
  {\sc 2sg} do {\sc dem.sg}  {\sc foc}\\
 \glt `You did THIS?' 

\z

\newpage 
\ea\label{ex:GRM-demons-pro-quest1}{\rm How the fingers cooperate
when they scoop t.z. from a bowl.}

 \gll hámàā ká zɪ̀ pɛ́jɛ̀ɛ̀ à zɪ́ já wà tììsè  háŋ̀.\\
 {\sc dem.pl} {\sc egr} then add.{\sc pfv} {\sc conn}  then do come support {\sc dem.sg}\\
 
\glt `These (two fingers) are then added, and then they come to support  this one.' 

\z



The expressions {\sls háŋ̀} ({\sc sg}) and {\sls hámà}
({\sc pl}) are employed for spatial deixis, specifically as proximal
demonstratives, corresponding to `this' and `these' respectively. The
language does not offer another set for distal demonstratives.



\subsubsection{Interrogative words}
\label{sec:GRM-interg-pro}


Interrogative constructions are of two types:  yes/no interrogatives and  pro-form interrogatives 
(see Section \ref{sec:GRM-interr-clause}). The former type, as the dichotomy suggests, requires  
a `yes' or a `no' answer.  A pro-form \isi{interrogative}  uses  an \isi{interrogative} word which identifies 
the sort of information requested. In Chakali,  some \isi{interrogative} words may be treated as 
pronouns, while others may be treated as the combination of a noun and a \isi{pronoun}.  Table 
\ref{tab:GRM-interg-pro} gives a list of \isi{interrogative} words, together with an approximate English 
translation,  the sort of information requested by each  and a link to an illustrative example of 
pro-form interrogatives.  Some examples are listed in (\ref{ex:GRM-interg-pro}), where the question 
words are marked as  {\sc q}  together with a compatible gloss.


\begin{table}
 \caption{Interrogative pronouns \label{tab:GRM-interg-pro}}
  \centering
  \begin{Itabular}{>{\slshape}llll}
\lsptoprule 
{\rm Pronoun} & Gloss  & Meaning requested & Example\\[1ex] \midrule
bàáŋ́ & what &  non-animate entity, event & \ref{ex:vp1.11.a}\\
 àŋ́ & who & animate entity & \ref{ex:vp2.5}\\
 lìé & where & location & \ref{ex:vp9.25}\\
ɲɪ̀nɪ̃̀ɛ̃́ & why/how & condition, reason& \ref{ex:vp22.4.1}\\
(ba/a)wèŋ́  & which &  entity, event & \ref{ex:vp22.4.4}\\
 (ba/a)ŋmɛ̀nà & (how) much/many & entity, event & \ref{ex:vp22.4.10}\\
 sáŋ(a)-wèŋ́ & when & time & \ref{ex:vp22.4.15}\\
\lspbottomrule
  \end{Itabular}
\end{table}


  
  \ea\label{ex:GRM-interg-pro}
  
\ea\label{ex:vp1.11.a}
\gll bàáŋ́ ɪ̀ kàà jáà?\\
{\sc q}.what  {\sc 2sg} {\sc egr} do\\
\glt `What are you doing?' 


\ex\label{ex:vp2.5}
\gll  àŋ́ ɪ̀ kà ná à tɔ́ʊ́ nɪ̄?\\
  {\sc q}.who {\sc 2sg}  {\sc egr}  see {\sc art} village {\sc postp}\\
\glt  `Whom did you see in the village?' 

%check comp here
\ex\label{ex:vp9.25}
\gll lìé nī dɪ̀ tʃʊ̀ɔ̀lɪ́ɪ́ kà dʊ̀ɔ̀?\\
  {\sc q}.where {\sc postp} {\sc comp} sleeping.room   {\sc egr} exist\\
\glt  `Where is the room for sleeping?' 



\ex\label{ex:vp22.4.1}
\gll ɲɪ̀nɪ̃̀ɛ̃́ ɪ̀ já kà jááʊ́?\\
   {\sc q}.how  {\sc 2sg} {\sc hab}   {\sc egr} do.{\sc 3sg.o}\\
\glt  `How do you do it?' 



\ex\label{ex:vp22.4.4}
\gll áwèŋ́ ɪ̀ kà kpàɣà?\\
   {\sc q}.which   {\sc 2sg}  {\sc  egr} catch\\
\glt  `Which one did you catch?' 


\ex\label{ex:vp22.4.10}
\gll àŋmɛ̀ná ɪ̀ kà kpàgàsɪ̀?\\
   {\sc q}.many {\sc 2sg}  {\sc  egr}  catch.{\sc pv}\\
\glt  `How many of them did you catch? (\isi{non-human} reference)' 

\ex\label{ex:vp22.4.15}
\gll {sáŋáwèŋ́} ɪ̀ kàà wáá?\\
    {\sc q}.when {\sc 2sg} {\sc  egr} come\\
\glt  `When are you coming?' 
  
   
  \z 
 \z

When the question word {\sls lie} `where' is  followed by the locative \isi{postposition} {\sls nɪ},  a request for a particular location is expressed. This question word can also be followed by the noun  {\sls pe} `end' in which case it should be interpreted as `where-towards' or `where-by', e.g. {\sls líé pé ɪ̀ kà válà?} `Where did you go by?'.  Another form used to request information on a location is {\sls ká(á)}. This form is neither specific to Chakali nor to location {\it per se}:  \ili{Waali} uses it for the same purpose and the form is even used to request other types  of information. For instance, {\sls káá tʊ́má?} means `how is work?' in the two languages. It might be that Chakali borrowed the form from \ili{Waali}.  It was employed consistently in an experiment which appears in \citet{brin11}. Example  (\ref{ex:GRM-vp23.3.}),  repeated in  (\ref{ex:GRM-vp23.3.-1}), illustrates the use of {\sls ká(á)} as \isi{interrogative} word.

\ea\label{ex:GRM-vp23.3.-1}
\gll ká à nàmɪ̃̀ã́?  bà tíéú rò.\\
   {\sc q}.where {\sc art} meat {\sc 3pl.b} chew.{\sc pfv}.{\sc 3sg.o} {\sc foc}\\
\glt  `Where is the meat? It has been eaten.'
\z

When they stand alone as \isi{interrogative} words, the expressions {\sls weŋ} and {\sls ŋmɛna}, roughly corresponding to English `which' and `how much/many', must be prefixed by either {\sls a-} or {\sls ba-} reflecting a distinction between \isi{non-human} and \isi{human} entities respectively (see Section \ref{sec:GRM-gender}). The expression {\sls saŋa-weŋ} in (\ref{ex:vp22.4.15}) is literally translated as `time which'.  The question word {\sls baaŋ} can be used together with {\sls wɪɪ} to correspond to English `why', i.e. {\sls bááŋ wɪ́ɪ́ ká wàà ɪ̀ dɪ̀ wíì?}  `Why are you crying?'.  The expression {\sls baaŋ wɪɪ} is equivalent to English `what matter'. 



%\begin{exe}
%\ex\label{ex:vp23.1.}
%\glll  àná ká tūgùù?.\\
 %aŋ ka tuga-u\\
 %{\sc q}.who   {\sc  egr} beat-{\sc 3sg.o}\\
%\glt  `By whom is he being beaten.'
%\z

%The question words may be followed by the \isi{focus} particle. This is shown in
%example (\ref{ex:vp23.1.}) with the question word {\sls aŋ}  `who'.




\subsubsection{Possessive pronouns}
\label{secːGRM-poss-pro}

The \isi{possessive} pronouns are displayed in Table \ref{tab:posspro}. 

\begin{table}
  \caption{Possessive pronouns \label{tab:posspro}}
  \centering
  \begin{Itabular}{l>{\slshape}l}
\lsptoprule 
Pronoun    &  {\rm Form}\\
Gram. function & {\rm   Possessive} \\[1ex] \midrule
{\sc 1sg.poss}  & n̩(ː)\\
{\sc 2sg.poss}   &   ɪ(ː)\\
{\sc  3sg.poss}   &  ʊ(ː)\\
{\sc 1pl.poss}   &   ja\\
{\sc 2pl.poss}    & ma\\
{\sc  3pl.a.poss} &  a(ː)\\
{\sc 3pl.b.poss}   &  ba\\
 
\lspbottomrule
  \end{Itabular}
\end{table}

A possessive \isi{pronoun} with a form C or V  tend to be \is{lengthening}lengthened,  although their length has no meaning. These pro-forms can function as possessor  ({\sc psor}), but never as possessed  ({\sc psed}),  in  an attributive possessive relation. This is shown in (\ref{ex:vp7.15}). 

\ea\label{ex:vp7.15}
\gll à kùórù ŋmá dɪ́ ʊ̀ʊ̀ hã́ã́ŋ tʃɔ́jɛ̄ʊ́.\\
  {\sc art} chief say {\sc comp} {\sc psor}.{3sg.poss}  {\sc psed}.wife ran.{\sc pfv.foc}\\
\glt  `The chief said that his wife ran away.' 
\z

The  weak personal pronouns and the possessive pronouns have the same forms, the differences between the two being their respective syntactic positions and  argument structures:  the \isi{weak pronoun} normally precedes a verb while the
possessive \isi{pronoun} normally precedes a noun, and the \isi{weak pronoun} is an
argument of a verbal predicate while the possessive \isi{pronoun} can only be the
possessor in a possessive attributive construction. 
%tone differences


% \ea\label{ex:vp7.15}
% \gll                
% {mɪ́n{\ꜜ} ná}\\
% {\sc 1sg.st.} {\sc foc}\\
% 
% \glt  `It is MINE.' 
% \z
% 
% 
% Phrasal possessives, as in English `it is mine, yours, etc.', are expressed with the
% strong personal \isi{pronoun}  in a verbless identificational
% construction. This is shown in (\ref{ex:vp7.15}).


\subsubsection{Reciprocity and reflexivity}
\label{sec:GRM-recipro-reflex}


Reflexive and reciprocal pronouns do not exist in Chakali.  Instead, reciprocity and reflexivity  are encoded in  the nominals {\sls dɔŋa}   and {\sls tɪntɪn}, which are glossed in the texts as {\sc recp}  and {\sc refl} respectively.   Reciprocity is illustrated in (\ref{ex:GRM-recipro}) and reflexivity in (\ref{ex:GRM-reflex}). 
%more like emphasis than reflexivity

  \ea\label{ex:GRM-recipro}
   
   
\ea\label{ex:vp24.1.}
\gll à nɪ̀báálá bálɪ̀ɛ̀ kpʊ́ dɔ́ŋá wā.\\
  {\sc art} men two kill {\sc recp}  {\sc foc}\\
\glt  `The two men killed {\sc each other}.' 

\ex\label{ex:vp24.2.}
\gll jà kàá kpʊ́ dɔ́ŋá wá.\\
   {\sc 1pl} {\sc fut} kill  {\sc recp}   {\sc foc}\\
\glt  `We will kill {\sc each other}.' 

\ex\label{ex:vp24.3.}
\gll à hàmṍwísè káá júó dɔ́ŋá rā.\\
  {\sc art} children {\sc  egr} fight {\sc recp}  {\sc foc}\\
\glt  `The children are fighting against {\sc one another}.' 
 
   
  \z 
 \z



 \ea\label{ex:GRM-reflex}
  
   
\ea\label{ex:vp25.1.}
\gll  à báál kpʊ̄ ʊ̀ tɪ̀ntɪ̀ŋ.\\
   {\sc art} man kill  {\sc 3sg.poss} {\sc refl.sg}\\
\glt  `The man killed himself.' 

\ex\label{ex:vp25.2.}
\gll jà kàá kpʊ̄ jà tɪ̀ntɪ̀nsá wá.\\
  {\sc 1pl}  {\sc fut} kill {\sc 1pl.poss}  {\sc refl.pl} {\sc foc}\\
\glt  `We shall kill {\sc ourselves}.'

\ex\label{ex:vp25.4.}
\gll à bìé kpá kísìé dʊ̄ ʊ̀ʊ̀ tɪ̀ntɪ̀ŋ dáŋɪ́ɪ́.\\
    {\sc art} child take knife put  {\sc 3sg.poss}   {\sc
refl.sg} wound\\
\glt  `The child wounded himself with his knife.' 
  
   
  \z 
 \z




\subsection{Qualifiers}
\label{sec:GRM-qualifier}

Qualifiers are treated  as part of the nominal domain. They display singular/\isi{plural} pairs, as  nouns do. Examples are presented in (\ref{ex:GRM-qual}).\footnote{Qualifiers are marked as {\it n.} in the dictionary since they are treated as nominal lexemes.}

\ea\label{ex:GRM-qual}
 
  \ea\label{ex:GRM-qual-red}
sɪ̀àmá {\rm /} sɪ̀ànsá   {\rm ({\sc cl.}1) `red'}
  \ex\label{ex:GRM-qual-bad}
 bɔ́ŋ̀ {\rm /}  bɔ́má   {\rm  ({\sc cl.}3)  `bad'}
  \ex\label{ex:GRM-qual-real}
dɪ́ɪ́ŋ {\rm /} dɪ́ɪ́má  {\rm  ({\sc cl.}3) `true, real'}

\z 
 \z


The examples in (\ref{ex:GRM-qual-agree}) are complex stem nouns  of which the \isi{qualifier}  `fat'  is 
a property of   the head `woman'  (Section \ref{sec:GRM-comp-quality}).

\ea\label{ex:GRM-qual-agree}

  \ea\label{ex:GRM-qual-agree-sg}
\gll à  hã́-pɔ́lɪ̄ɪ̀\\
{\sc art} woman-fat.{\sc cl.4.sg}\\
\glt `The fat woman' 

  \ex\label{ex:GRM-qual-agree-pl}
\gll à hã́-pɔ́lɪ̄ɛ̀\\
{\sc art} woman-fat.{\sc cl.4.pl}\\
\glt `The fat women'
 
\z 
 \z
 
 Many qualifiers are assigned to noun class 4, the reason being that qualifiers are often nominalized verbal stems (Section \ref{sec:GRM-der-agent}), e.g. {\sls pɔ́lɪ́ɪ́/pɔ́lɪ́ɛ́} ({\it qual}) `fat' $\leftarrow$ {\sls pɔ̀là}  ({\it v})  `fat (be)'.  Examples are provided in  (\ref{ex:GRM-qual-cl4}).


\ea\label{ex:GRM-qual-cl4}
 
  \ea\label{ex:GRM-qual-cl4-call}
jɪ̀rà  {\rm `call'}  $>$ jɪ́rɪ́ɪ́ {\rm   `calling'}
\ex\label{ex:GRM-qual-cl4-give-birth}
lʊ̀là {\rm  `give birth'} $>$ lʊ́lɪ́ɪ́   {\rm `giving birth'}
\ex\label{ex:GRM-qual-cl4-die}
sʊ̀wà {\rm  `die'} $>$ sʊ́wɪ́ɪ́  {\rm  `corpse'}

  
\z 
 \z


Nonetheless, the two categories, noun and \isi{qualifier}, are differentiated by the following characteristics: (i)  while a \isi{qualifier} must be semantically verbal (i.e. denoting a state or an event), a noun must not necessarily be, and (ii) while a \isi{qualifier} modifies a noun,  a  noun functions as  the nominal argument of the \isi{qualifier}. The asymmetry is reflected in (\ref{ex:GRM-qual-hot}).

\ea\label{ex:GRM-qual-hot}{\rm  /nʊm/ `hot'}
 
  \ea\label{ex:GRM-qual-hot-cmp-stem}
  \glll nɪ̀ɪ̀nʊ́ŋ ná.\\
 nɪɪ-nʊŋ na\\
  water-hot {\sc foc}\\
  \glt `It is {\sc hot water}.'

 \ex\label{ex:GRM-qual-hot-head}
  \glll  à nɪ́ɪ́ nʊ́mã́ʊ̃́.\\
 a  nɪɪ nʊma-ʊ\\
   {\sc art} water hot-{\sc pfv.foc}\\
  \glt `The water is {\sc hot}.'

 \ex\label{ex:GRM-qual-hot-qual}
  \glll  à nɪ́ɪ́ nʊ́mɪ́ɪ́ dʊ́á dé nɪ̄.\\
 [a nɪɪ nʊm-ɪ-ɪ]$_{NP}$ dʊa de nɪ \\
  {\sc art} water hot-{\nmlz}-{\sc cl.4} exist {\sc dxl} {\sc postp}\\
  \glt `The hot water is there.'
  
\z 
 \z
 
 
In (\ref{ex:GRM-qual-hot-cmp-stem}) the stem {\sls nʊm} `hot' is part of the complex stem noun {\sls nɪ̀ɪ̀nʊ́ŋ̀} `water-hot' (see Section \ref{sec:GRM-com-stem-noun}).  In this morphological configuration, a qualitative modification is  established  between the stem {\sls nʊm} and the stem {\sls nɪɪ}. In (\ref{ex:GRM-qual-hot-head}), {\sls nʊm}  functions as a verbal predicate in the intransitive clause, and the definite noun phrase {\sls a nɪɪ} `the water' occupies the argument position. In (\ref{ex:GRM-qual-hot-qual}) the stem {\sls nʊm} is nominalized and the singular of  noun class 4 is suffixed. The word {\sls nʊ́mɪ́ɪ́} may be translated as  `the result of heat'. It is treated as a \isi{qualifier} since {\sls nɪɪ} `water' is  (the head of) the argument of the predicate, and {\sls dʊa} is a predicate which needs   one core argument. Since {\sls nʊm}  can function neither as main predicate nor as head noun of the argument phrase, and since {\sls nʊm}  is understood to be a property of the entity and not of the event, then {\sls nʊm} in (\ref{ex:GRM-qual-hot-qual}) is viewed as a \isi{qualifier}.


Given the arguments put forward, one could analyse the qualifiers as adjectives. Both are  seen  categorically as nominals  and semantically as properties or states.  However, there are no lexemes in Chakali  which can be assigned the category adjective; that is, no lexeme which, in all  linguistic contexts, can be identified as categorically distinct from nouns and verbs.  Qualifiers are either derived linguistic entities or idiomatic expressions. More than one procedure is attested to construct qualifiers. In (\ref{ex:GRM-qual-types}),   some types of qualifiers are provided.

\ea\label{ex:GRM-qual-types}
 
 \ea\label{ex:GRM-qual-t0} àbúmmò  {\rm `black'}
  \ex\label{ex:GRM-qual-t1} àpʊ́lápʊ́lá {\rm  `pointed, sharp'}
  \ex\label{ex:GRM-qual-t2}  wɪ̀ɛ́zímíí  {\rm  `wise'} 

\z 
 \z

The expression {\sls bummo} `black, dark'  in (\ref{ex:GRM-qual-t0}) is treated as a nominal lexeme. When it functions as a \isi{qualifier} within a noun phrase,  the prefix vowel {\sls a-} is suffixed to the nominal stem  (see Section \ref{sec:GRM-numeral}). The type of \isi{qualifier} found in (\ref{ex:GRM-qual-t1}) is ideophonic and is used to describe perceived patterns, including colour, texture, sound, manner of motion, e.g. {\sls gã́ã́nɪ̀gã́ã́nɪ̀} `cloud state',  {\sls adʒìnèdʒìnè} `yellowish-brown',  {\sls tùfútùfú} `smooth and soft'. Reduplication characterises the form of this type of qualifiers. When a reduplicated \isi{qualifier} occurs in attributive function, i.e. following the head noun, it takes the prefix {\sls a-} as well.\footnote{Although the prefix {\sls a-} on qualifiers tends to disappear in normal speech. The prefix {\sls a-} is unacceptable in (\ref{ex:GRM-qual-t2}).} The word in (\ref{ex:GRM-qual-t2}) is segmented as [[[{\sc theme}-v]-{\sc nmlz}]-{\sc cl.4}]. The verbal stem {\sls zɪm} `know'   sees  its theme argument incorporated, i.e.  {\sls wɪɛ-zɪm} `matters-know',  a structure which is in turn nominalized by what is called  event-nominalization in Section \ref{sec:GRM-der-agent}.  

% The qualifiers in (\ref{ex:GRM-qual-types}) are presented in corresponding syntactic positions in (\ref{ex:GRM-qual-types-bar}). \ea\label{ex:GRM-qual-types-bar} \ea   {\sls  [X àbúmmò]$_{NP}$ dʊ̀à dé   tàɣàtà búmmò dʊ̄ā dē} {\rm `The black shirt  is there.'} \ex   {\sls  [X àpʊ́lápʊ́lá]$_{NP}$ dʊ̀à dé  kísínʊ̃̀ʊ̃̀ pʊ́lápʊ́lá  dʊ̄ā dē}  {\rm `The sharp  knife is there'.} \ex {\sls  [X wɪ̀ɛ́zímíí]$_{NP}$  dʊ̀à dé } {\rm  `The wise X is there'.} \z \z

There are  limitations on the number of qualifiers allowed within a noun phrase. Noun phrases with more than three qualifiers are often rejected by language consultants in elicitation sessions.  The language simply employs other strategies to stack properties. In fact noun phrases with two qualifiers are rarely found in the texts collected. The linear order of qualifiers within the noun phrase are provided in Section \ref{sec:GRM-NP-sum}.

Chakali has phrasal expressions which correspond to  monomorphemic adjectives in some other languages. These expressions have the characteristic of being metaphorical; their lexemic denotations may be seen as secondary, and phrasal  denotations as non-compositional. For instance, a speaker must say {\sls ʊ̀ kpáɣá bàmbíí}, {\it lit.}`he has heart', if he/she wishes to express `he is brave'. The word `brave' cannot be translated to {\sls bambii}, since its primary meaning is `heart',  but to {\sls kpaɣa bambii}  `to be brave'. Another way of expressing `brave person'  is {\sls bàmbìì-tɪ́ɪ́ná}, {\it lit.} `owner of heart'. Other examples  are {\sls síí-nʊ̀mà-tɪ́ɪ́ná}, {\it lit.} `eye-hot-owner', `wild, violent  person'   and {\sls síí-tɪ́ɪ́ná}, {\it lit.} `eye-owner', `stingy, greedy person'. These expressions are more frequently used as nouns in the complement position of the identificational construction, such as in {\sls ʊ̀ jáá sísɪ́ámātɪ̄ɪ̄nā}, {\it lit.} she is eye-red-owner  ({\sls si-sɪama-tɪɪna}), `she is serious'. As mentioned in Section \ref{sec:GRM-idiom},  it is often hard to establish whether an expression is idiomatic when only one of its components is used in a non-literal sense.



\subsubsection{Intensifiers}
\label{sec:GRM-intensifier}

An intensifier is a predicate modifier and appears following the word it modifies. It marks a degree and magnifies the meaning of the word it modifies.

\ea\label{ex:intens-ideo} 
\ea  ásɪ̀àmā tʃʊ̃́ɪ̃́tʃʊ̃́ɪ̃́   {\rm `very/pure  red'}
\label{ex:BCTmod-prop-red}

\ex ábúmmò jírítí {\rm `very/pure black'}
\label{ex:BCTmod-prop-black} 

\ex  ápʊ̀mmá píópíó  {\rm `very/pure  white'}
\label{ex:BCTmod-prop-white}

\ex  sʊ́ɔ́nɪ̀ júlúllú  {\rm `very cold'}
\label{ex:BCTmod-prop-cold}

\ex   nʊ̀mà kpáŋkpáŋ  {\rm `very hot'}
\label{ex:BCTmod-prop-hot}


\z
\z


The  intensifier ideophones  {\sls tʃʊ̃́ɪ̃́tʃʊ̃́ɪ̃́},  {\sls jírítí},  {\sls píópíó},  {\sls júlúllú},  and {\sls kpáŋkpáŋ} are translated into English  `very' (or `pure' in the case of colour, for instance) in (\ref{ex:intens-ideo}). They are treated together as one kind of degree predicate modifier.  Note that no other properties have been found together with a (unique and) corresponding degree modifier. For instance, if one wishes to express `very X', where X refers to a colour other than black, white, or red,   one has to employ the degree modifier {\sls pááá} `very' following the term, which is a common expression in many Ghanaian languages. 

%brindle 2017


\subsection{Quantifiers}
\label{sec:GRM-quantifier}

%complex \isi{quantifier}
%add nar as one and refer to section for other type ofwantifier sec:\isi{classifier}


Quantifiers are expressions denoting quantities and refer to the size of a referent ensemble. The words {\sls mùŋ} `all',   {\sls bánɪ̃́ɛ̃́} `some' and {\sls tàmá} `few, some' constitute the  monomorphemic quantifiers. The  former can be expanded with a  nominal prefix. For instance, in {\sls ba-muŋ} `{\sc hum}-all' and {\sls wɪ-muŋ} `{\sc abst}-all',  the prefixes identify the semantic class of the entities which the expressions quantify (see Section \ref{sec:classifier}). The form of the \isi{quantifier} {\sls bánɪ̃́ɛ̃́} `some'  is  invariable: *{\sls anɪɛ}, *{\sls abanɪɛ} and *{\sls babanɪɛ} are unacceptable words.  The same can be said for the word {\sls tàmá} `few', which stays unchanged even  when it  modifies  nouns of different semantic classes. 

The expression {\sls kɪ̀ŋkáŋ̀} `a lot, many'   is made out of the  \isi{classifier}  {\sls kɪŋ-} plus the quantitative verbal state lexeme {\sls kan}  `abundant'   (Sections  \ref{sec:classifier} and \ref{sec:GRM-verb-stative-active}, respectively). The  lexeme {\sls kan} `abundant'  is semantically verbal but turns into a \isi{quantifier} when {\sls kɪŋ-}  is prefixed to it.  Other evidence for its verbal status  is the utterance {\sls à kánã́ʊ̃́} `they are many' compared to {\sls à jáá tàmá} `they are few'.  In the former, {\sls kana} is the main verb of an intransitive perfective clause, while in the latter, {\sls tama} is the complement of the verb {\sls jaa} in an identificational construction  (Section \ref{sec:GRM-ident-cl}). Other plurimorphemic (or complex) quantifiers are based on the suffixation the morpheme {\sls  -lɛɪ} `not'. The expression {\sls wɪ-muŋ-lɛɪ} ({\it lit.} {\sc abst}-all-not) and {\sls kɪŋ-muŋ-lɛɪ} ({\it lit.} {\sc conc}-all-not)  both correspond to the English word `nothing' (Section \ref{sec:classifier} on \isi{negation}). 

\ea\label{ex:GRM-quant-int-only}
\gll àŋmɛ̀nà máŋá tʃájɛ̄ɛ́.\\
   amount only remain.{\pfv}\\
\glt `Only a few are left.'
\z

The meaning `a few' can be conveyed by  the word {\sls aŋmɛna} `how much/many', which was introduced in Section \ref{sec:GRM-interg-pro} as an \isi{interrogative} word. Example  (\ref{ex:GRM-quant-int-only}) suggest that the word {\sls aŋmɛna} can also be used in a non-\isi{interrogative} way,  co-occurring here with {\sls maŋa} `only',  in which case it is interpreted  as `amount' or `a certain number'. Another way to express `(a) few'  is to duplicate the \isi{numeral} {\sls dɪgɪɪ} `one', e.g. {\sls dɪgɪɪ-dɪgɪɪ ra} `there are just a few of them'.  The  examples in (\ref{ex:GRM-quant-mean}) show that the \isi{numeral} {\sls dɪgɪɪ} `one' can participate in the denotations of both total and partial quantities. 


\ea\label{ex:GRM-quant-mean}
 \ea {\sls mùŋ} {\rm `all' (total collective)}
 \ex  {\sls  dɪ́gɪ́ɪ́ mùŋ} {\rm `each' (total distributive)}
 \ex  {\sls  dɪ̄gɪ̄ɪ̄ dɪ́gɪ́ɪ́} {\rm `some, few' (partial distributive)}

\z 
 \z
 
The word {\sls gàlɪ̀ŋgà} `waist' or `middle'  can also carry quantification. In (\ref{ex:GRM-most}),  the word is equivalent to {\sls bàkánà} (< {\sls bar-kaŋ}, {\it lit.} part-abound),  and means `most'.


\begin{exe}
 \ex\label{ex:GRM-most}
\gll   à kpã́ã́má  gàlɪ̀ŋgà/bàkánà tʃájɛ̄ɛ́ à láʊ́ nɪ́.\\
{\sc art} yam.{\pl} most remain.{\pfv} {\sc art} farm.hut {\postp}\\
\glt  `Most of the yams remain/are left in the farm hut.'
\z

The word {\sls gba} `too' is treated as a \isi{quantifier} and restricted to appear after the subject, e.g. (\ref{ex:GRM-too-out-1})-(\ref{ex:GRM-too-out-4}). In (\ref{ex:GRM-too-pos}), the speaker  considers himself/herself  as part of a previously established set of individuals who beat their respective child. The \isi{quantifier} is additive such that  the denotation of the subject constituent is added to this previously established set.  In (\ref{ex:GRM-too-neg}), it is shown that negating the quantified expression results in an interpretation where the speaker asserts that he/she is not a member of the set of individuals who beat their child. Since generally there is only one `in \isi{focus}' constituent in a clause and that \isi{negation} and \isi{focus} cannot co-occur (see Sections \ref{sec:GRM-foc-neg} and  \ref{sec:GRM-focus}), example (\ref{ex:GRM-too}) suggests that {\sls gba} is not a \isi{focus} particle.

\ea\label{ex:GRM-too}

 \ea\label{ex:GRM-too-pos}
\gll ŋ̩̀ gbà máŋá m̩̀ bìè rē.\\
{\sc 1sg} {\quant}.too  beat {\sc 1sg.poss}  child {\foc}\\
\glt  `I beat my child too.' ({\it lit.} I too/as well/also beat my
child)

 \ex\label{ex:GRM-too-neg}
\gll ŋ̩̀ gbà lɛ̀ɪ́ máŋá  m̩̀ bìé.\\
{\sc 1sg} {\quant}.too  {\neg} beat {\sc 1sg.poss}  child\\
\glt  `I do not beat my child.' ({\it lit.} I am numbered with those known who refrain from beating their child)


 \ex\label{ex:GRM-too-out-1}   \textasteriskcentered  gba m̩  maŋa a bie re
\ex \textasteriskcentered  m̩ maŋa gba a bie re
 \ex \textasteriskcentered  m̩ maŋa  a bie gba re
 \ex\label{ex:GRM-too-out-4} \textasteriskcentered  m̩ maŋa  a bie  re gba

\z 
 \z

 
 
 \subsection{Numerals}
\label{sec:GRM-numeral}


\subsubsection{Atomic and complex numerals}
\label{sec:NUM-bas-comp}

Following \citet[263]{Gree78b}, I assume that  the simplest lexicalisation of a number is called a \isi{numeral} atom, whereas a complex \isi{numeral} is an expression in which  one can infer at least one arithmetical function.  A \isi{numeral} atom can stand alone or can be combined with another \isi{numeral}, either atomic or complex, to form a complex \isi{numeral}. Atoms are treated as  those forms which are not decomposable morpho-syntactically at a synchronic level. Table \ref{tab:numeralatoms} displays the twelve atoms of the \isi{numeral} system.


\begin{table}
  \caption{Atomic numerals from 1 to 8, 10, 20, 100, and 1000
\label{tab:numeralatoms}}
   \centering
  \begin{tabular}{llll}
\lsptoprule
Chakali &  English &  Chakali &  English\\\midrule
 dɪ́gɪ́máŋá & one &   àlʊ̀pɛ̀   &seven\\
álɪ́ɛ̀ &two   &   ŋmɛ́ŋtɛ́l &eight\\
átòrò &three &   fí &ten\\
ànáásɛ̀ &four & màtʃéó  &twenty\\
 àɲɔ̃́ &five  &  kɔ̀wá (pl.  kɔ̀sá)   & hundred(s)\\
  álòrò   &six &   tʊ́sʊ̀  (pl.  tʊ́sà) &thousand(s)\\
 
\lspbottomrule
\end{tabular}
\end{table}

The term for `one'  is expressed  as  {\sls dɪ́gɪ́máŋá},  but  {\sls dɪ́gɪ́ɪ́} alone  can also be used. In general, the meaning associated with the morpheme {\sls máŋá} is `only', e.g.  {\sls bahɪɛ̃ maŋa n̩ na} {\it old.man-only-I-saw} `I saw only an old man'. The number 8 is designated with  {\sls ŋmɛ́ŋtɛ́l}, an expression which is also used to refer to the generic term for  `spider'.  Whether they are homonyms, or whether their meanings enter into a polysemous/heterosemous relationship is unclear. Another characteristic is that the higher numerals 100 and 1000  have their own \isi{plural} form. To say a few words about some of the possible origins of higher numerals, the genesis of most of SWG higher numerals involves diffusion from non-\ili{Grusi} sources, rather than from  common SWG descents. I believe that higher numerals in the linguistic area where Chakali is spoken have two origins: one is \ili{Oti-Volta} and the other is \ili{Gonja}. The  forms for 100 and 1000  in \ili{Vagla} and \ili{Dɛg}  are similar to \ili{Gonja}'s forms with the same denotation, i.e. \ili{Gonja} {\sls  kɪ̀làfá} `100' and  {\sls  kíg͡bɪ́ŋ} `1000'.  Similar form-denotation can be found in other North \ili{Guang} languages (e.g. Krache, Kplang, Nawuri, Dwang, and Chumburung) and {\sls lafa} is found in many other \ili{Kwa} languages, as well as  non-\ili{Kwa} languages, e.g. Kabiye (Eastern \ili{Grusi})  \citep{Chan09}. Borrowing is  supported by the claim that the Vaglas and Dɛgas were where they are today before the arrival of the Gonjas (\citealt[12-13]{Good54}; \citealt[516]{Ratt32a}), and the fact that they, but mostly the Vaglas, are still in contact with the former conquerer, the Gonjas. Another \ili{Grusi} language, \ili{Tampulma}, has had more contact with Mampruli than with any other Western \ili{Oti-Volta} languages, whereas the Chakali and the \ili{Pasaale} have contact with \ili{Waali}, a language close to  Dagbani and \ili{Dagaare}, all of them classified as Western  \ili{Oti-Volta} languages. Variations of Manessy's {\it oti-volta commun} reconstructed forms {\sls *KO / *KOB}  `hundred'  and {\sls *TUS}  `thousand'  are found distributed all over Northern Ghana, cutting across genetic relationship.  It seems that the two high numerals are areal features spread by Western  \ili{Oti-Volta} languages,   and that Chakali, \ili{Pasaale}, and \ili{Tampulma} speakers may  have borrowed them from languages with which they had the most contact, i.e.   \ili{Waali}, Dagbani, \ili{Dagaare},  and Mampruli.

From the atoms,  the complex numerals are now examined. The arithmetical functions inferred are called operations. In Chakali three types of operation are found: addition, multiplication, and subtraction. An operation always has two arguments which are identified in \citet{Gree78b} as: 

\vspace{2ex}

\begin{tabular}{ll}
{ Augend:} & A value to which some other value is
added.\\
{ Addend:} & A value which is added to some other
value.\\
{ Multiplicand:} & A value to which some other
value multiplies.\\

{ Multiplier:} & A value which is multiplied to
some other value.\\

{ Subtrahend:}  & The number subtracted.\\
{ Minuend:}  & The number from which subtraction takes
place.\\
\end{tabular}

\vspace{2ex}

The \isi{numeral} {\sls   dɪ́gɪ́tūō} expresses the number 9. It is the only expression associated with subtraction.  The subtrahend is the expression {\sls dɪgɪɪ} `one'.   In {\sls   dɪ́gɪ́tūō},  the last syllable   is analysed as the operation. It may originate from the state predicate  {\sls tùó} which is translated `not exist'  or `absent from' (Section \ref{sec:GRM-loc-cl}). Thus, assuming the covert minuend 10, the \isi{numeral} expression receives the functional notation [1 {\sc absent from} 10], or 10 minus 1.  The number 9 may also be expressed as {\sls sàndòsó}  (or {\sls sandʊsə} in Tuosa and Katua). This expression is used by some individuals in Ducie, Tuosa, and Katua, all of them from the most senior generation.  One language consultant associates  {\sls sàndòsó} with the language of women, but his claim is not sustained by other language consultants. For the number 9, \citet[33]{Good54} reports {\sls saanese} from the village Katua and  \citet[117]{Ratt32b} puts {\sls sandoso} as the form for 9 in \ili{Tampulma}. 

A proper  treatment of  atomic versus  complex numerals   relies  on evidence as to whether a \isi{numeral} is synchronically  decomposable. In  that spirit,  numbers from 11 to 19 are expressed with complex numerals:  one piece of evidence, which is presented in Section \ref{sec:GRM-gender} and repeated in section \ref{sec:NUM-npstruc}, comes from the gender agreement between the head of a noun phrase and the cardinal \isi{numeral} functioning as modifier.  Table \ref{tab:numral11-19} provides the  numerals from 11 to 19 with a common structure [fi$_{10}$-d(ɪ)-X$_{1-9}$]. 


  \begin{table}
  \caption{Complex numerals from 11 to 19  \label{tab:numral11-19}}
   \centering

  \begin{tabular}{>{\slshape}ll}
\lsptoprule
 {\rm Chakali} & English\\\midrule
 fídɪ̄dɪ́gɪ́ɪ́ & eleven\\
 fídáālìɛ̀ & twelve\\
  fídáātòrò &  thirteen\\
fídànáásɛ̀ &  fourteen\\
 fídàɲɔ̃́ &  fifteen\\
 fídáālòrò  & sixteen\\
fídālʊ̄pɛ̀ &  seventeen\\
  fídɪ̀ŋmɛ́ŋtɛ́l  &  eighteen\\
  fídɪ̀dɪ́gɪ́túò &  nineteen\\

\lspbottomrule
\end{tabular}
\end{table}


The criterion employed for  the distinction between augend and addend is that an augend is serialized, that is, it is the expression which is constant in a sub-progression. This expression is called the base. In the progression from eleven to nineteen shown  in  Table  \ref{tab:numral11-19},  the augend is {\sls fi} and the addends are the expressions for one to nine. Given the above definition of a base,  the expression {\sls fi} is  the base in complex numerals  from 11 to 19. The operator for addition is {\sls dɪ} and its vowel surfaces only when the following word starts with a consonant (i.e. {\sls fídɪ̀ŋmɛ́ŋtɛ́l} `18', but {\sls fídànáásɛ̀} `14'). Table \ref{tab:numeral21-99} provides the sequences of  \isi{numeral} atoms forming the complex numerals referring to  numbers from 21 to 99. Some \isi{numeral}  forms will come after an explanation of the table.


  \begin{table}
  \caption{Complex numerals from 21 to 99  \label{tab:numeral21-99}}
  \centering

  \begin{tabular}{lll}
\lsptoprule 
Number  & Numeral & Meaning\\\midrule
 
21-29& atom {\sls anɪ} atom &  20  + (1 through 9)\\
30  &  atom  {\sls anɪ} atom  & 20  + 10\\
31-39&  atom {\sls anɪ} complex  & 20  + (11 through 19)\\
40 &  atom  atom & 20 $\times$ 2\\
41-49&   atom  atom  {\sls anɪ} atom &  20 $\times$  2  + (1 through 9)\\
50 &  atom  atom  {\sls anɪ} atom & 20 $\times$ 2 + 10\\
51-59 & atom  atom  {\sls anɪ} complex &20 $\times$ 2  + (11 through 19)\\
60 & atom  atom & 20 $\times$ 3\\
61-69 & atom  atom {\sls anɪ} atom  &20 $\times$ 3 + (1 through 9)\\
70 &  atom  atom  {\sls anɪ} atom& 20 $\times$ 3 + 10\\
71-79 &atom  atom  {\sls anɪ} complex  &20 $\times$ 3   + (11 through 19)\\
80 & atom  atom  & 20 $\times$ 4\\
81-89 & atom  atom {\sls anɪ} atom&20 $\times$ 4 + (1 through 9)\\
90 &  atom  atom  {\sls anɪ} atom&20 $\times$ 4 + 10\\
91-99 & atom  atom  {\sls anɪ} complex& 20 $\times$ 4   + (11 through 19)\\
 
\lspbottomrule
\end{tabular}
\end{table}

%The number by which another number is multiplied. In 8 X 32, the
%multiplier is 8.

\largerpage
Table \ref{tab:numeral21-99} shows us that either (i) an atom can follow another atom without any intervening particle  or (ii) the particle {\sls anɪ} can step in between two atoms, or between one atom and one complex \isi{numeral}. Case (i) is understood as a phrase which multiplies the numerical values of  two atoms. For instance, {\sls màtʃéó  ànáásɛ̀} [20 $times$ 4] results in the product `eighty'.  All \isi{numeral} phrases from 20 to 99 use {\sls matʃeo} `20'  in their formation. In case (ii),  the particle {\sls anɪ} is treated as an operator similar to the semantics of  `and' in English numerals since it adds the value of each argument, either atom or complex {\sls màtʃéó  ànáásɛ̀ ànɪ́  àlɪ̀ɛ̀} [20 $times$ 4  $+$ 2] .  The same form is also found in noun phrases expressing the union of two or more entities (see Section \ref{sec:GRM-conjunc-nom}). The vowels of {\sls anɪ} are reduced when preceded and followed by vowels. The same criterion applies for the distinction between multiplier and multiplicand: the latter  is identified on the basis of what Greenberg calls `serialization'. A  base may be   a serialized multiplicand as well since it is the constant term in the complex expressions involved in a sub-progression. The expression  {\sls matʃeo} `20' is therefore the base in complex numerals  from 21 to 99. The composition of complex numerals is summarized in Table \ref{tab:threecompo}.



\begin{table}
\caption{General structure of complex numerals  \label{tab:threecompo}}
  \centering

\begin{tabular}{lll}
\lsptoprule
  Argument & Meaning & Restriction\\
\midrule
 ($y$)   $x$   tuo  & subtraction  &$y={10}$\\
&& $x={1}$\\
   $x$ anɪ $y$ & addition  &$x>y$\\
$x$ dɪ $y$  & addition &$x={10}$\\
&& $y={1 \textrm{-}9}$\\%left sister x is 10

$x y$ & multiplication &$x=20$\\
&& $y={2,3,4}$\\%right sister y smaller than left sister x
$x y$  & multiplication  &$x={100}$\\
&& $y={2 \textrm{-}9}$\\%left sister x is 10
$x y$  & multiplication  &$x={1000}$\\
&& $y={2 \textrm{-}999, 1000}$\\%left sister x is 10
\lspbottomrule
\end{tabular}
\end{table}


As  mentioned earlier, in subtraction  the minuend $y$ is covert. The only case of subtraction is the \isi{numeral} {\sls   dɪ́gɪ́tūō} `nine'.  Both addition  and multiplication take two overt arguments  $x$ and $y$. They are presented in the first column  of Table \ref{tab:threecompo} with their surface linear order. The operator for addition {\sls dɪ} is used only  for the sum of 10 and numbers between 1 and 9. The form {\sls anɪ} is found in a variety of structures, but it restricts the right sister $y$ to be lower than the left sister $x$. In multiplication  the value of the argument $y$ depends on the value of $x$. For the numerals designating  2000 and above, the argument $x$ must be the atom {\sls tʊsʊ} `thousand' and $y$  any atom or complex \isi{numeral} between 2 and 999. There are no terms to express  `million' in Chakali. One can hear individuals at the market  using the English word `million' when referring to  currency. According to my consultants,  the expression {\sls  tʊsʊ tʊsʊ} [1000 $\cdot$ 1000] `million' was common, but became archaic even before the change of currency  in July 2007. Examples of numerals are presented in (\ref{ex:diffstrings}).

   

\ea\label{ex:diffstrings}
  \ea\label{ex:82}
\gll  màtʃéó  ànáásɛ̀ ànɪ́  àlɪ̀ɛ̀.\\
   {twenty} {four} {and} {two}\\
\glt `82'

\ex\label{ex:121}
\gll  kɔ̀wá  ànɪ́  màtʃéó  ànɪ́   dɪ́gɪ́máŋá.\\
    {hundred}  {and}  {twenty}  {and} {one}\\
\glt `121'
% 
% \ex\label{ex:232}
% \gll  kɔ̀sá  àlɪ̀ɛ̀ ànɪ̄  màtʃéó  ànɪ́  fídáālìɛ̀.\\
%     {hundreds}  {two}  {and}   {twenty}  {and}  {twelve}\\
% \glt `232'

\ex\label{ex:395}
\gll kɔ̀sá átòrò ànɪ́ màtʃéó ànáásɛ̀ ànɪ́  fídāāɲɔ̃̄.\\
 {hundreds} {three}  {and} {twenty}  {four} {and}  {fifteen}\\
\glt `395'

\ex\label{ex:501}
\gll kɔ̀sá  áɲɔ̃̄ ànɪ́  dɪ́gɪ́máŋá.\\
    {hundreds} {five}  {and}  {one}\\
\glt `501'

\ex\label{ex:1225}
\gll tʊ̀sʊ̀  ànɪ́   kɔ̀sá  álɪ̀ɛ̀   ànɪ́  màtʃéó  ànɪ́  āɲɔ̃̀.\\
    {thousand}  {and}  {hundreds}  {two}  {and}  {twenty}  {and} {five}\\
\glt `1225'

\ex\label{ex:21231}
\gll tʊ̀sà  màtʃéó   ànɪ́  dɪ́gɪ́máŋá ànɪ́  kɔ̀sá  ālɪ̀ɛ̀  ànɪ́ màtʃéó  ànɪ́     fídɪ̄dɪ́gɪ́ɪ́ \\
    {thousands}   {twenty}   {and} {one}  {and}  {hundreds}  {two} {and}   {twenty} {and} {eleven}\\
\glt `21231'


\z
\z
% 

\newpage 
In summary,  the \isi{numeral} system of Chakali is decimal (base-10) and vigesimal (base-20) and the base-20  operates throughout the formation of 20 to 99. In \citet{Comr08}, \isi{numeral} systems similar to the one described here are called \is{vigesimal-decimal}\textit{hybrid vigesimal-decimal}.


\subsubsection{Numerals as modifiers}
\label{sec:NUM-npstruc}

To a certain extent, Chakali offers a rigid word order within the noun phrase (Section \ref{sec:GRM-foc-neg}). The heading of (\ref{ex:all-num-NP}) offers an overview of  the linear order of elements in a noun phrase containing a \isi{numeral}. Tha data shows that the \isi{numeral} occurs following the head and the \isi{qualifier}(s) and precedes the demonstrative  and the \isi{quantifier}.\footnote{Note that the noun phrases in  (\ref{ex:all-num-NP}) and  (\ref{ex:GRM-np-list}) were collected in an elicitation session. They were elicited in subject position of the sentence frame {\sls X wááwáʊ́/wááwá} `X has come'.}


\ea\label{ex:all-num-NP}{\rm  \textsc{art/poss}    \textsc{head}    \textsc{qual$_\textrm{1}$}    
\textsc{ qual$_\textrm{2}$}    \textsc {num}   \textsc{quant}      \textsc{dem}      
\textsc{foc/neg}\\}

  \ea\label{ex:all-w}
\gll à nɪ̀hã́ã́n-á mùŋ wááwáʊ́.\\
\textsc{art} {woman-\textsc{pl}} \textsc{quant}.all come.{\sc prf.foc}\\
\glt `All women came.'

 \ex\label{ex:all-ten-w}
\gll à nɪ̀hã́ã́n-á fí mùŋ wááwáʊ́.\\
\textsc{art} {woman-\textsc{pl}} \textsc{num} \textsc{quant}.all come.{\sc prf.foc}\\
\glt `All ten women came.'

\ex\label{ex:all-fat-ten-w}
\gll  à nɪ̀hã́pɔ́lɛ̄ɛ̄ fí mùŋ wááwáʊ́.\\
\textsc{art} {woman-\qual} \textsc{num} \textsc{quant}.all come.{\sc prf.foc}\\
\glt `All ten fat women came.'

\ex\label{ex:all-fat-blind-two-w}
\gll ʊ̀ nɪ̀hã́ɲʊ́lʊ́má pɔ̀lɛ̄ɛ̄  bàlɪ́ɛ́ mùŋ  wááwáʊ́.\\
\textsc{poss} {woman-\qual}  {\qual} \textsc{num} 
\textsc{quant}.all  come.{\sc prf.foc}\\
\glt `Both his two fat blind wives came.'

\ex\label{ex:all-fat-ten-w-those}
\gll  à nɪ̀hã́pɔ́lɛ̄ɛ̄ fí háŋ mùŋ wááwáʊ́.\\
\textsc{art} {woman-\qual} \textsc{num}  \textsc{dem} \textsc{quant}.all
  come.{\sc prf.foc}\\
\glt `Those all ten fat women came.'


\ex\label{ex:all-fat-ten-w-n}
\gll à nɪ̀hã́pɔ́lɛ̄ɛ̄ fí mùŋ lɛ̀ɪ̄ wááwá.\\
\textsc{art} {woman-\qual} \textsc{num} \textsc{quant}.all
\textsc{neg}  come.{\sc prf}  \\
\glt `Not all ten fat women came.'

 \ex\label{ex:full-temp}
\gll à nɪ̀hã́pɔ́lɛ̄ɛ̄ fí háŋ mùŋ  lɛ̀ɪ̄ wááwá. \\
\textsc{art} {woman-\qual}  \textsc{num}   \textsc{dem}  \textsc{quant}.all
\textsc{neg}   come.{\sc prf} \\
\glt `Not all those ten fat  women came.'

\z
\z

When they appear as noun modifiers,  a limited number of numerals act as targets in gender agreement, i.e. only the forms 2-7. This grammatical phenomenon provides us with a  motivation to treat  the expressions for numbers 11-19 as complex numerals. In Section \ref{sec:GRM-gender},  Chakali is analysed as having two values for the feature gender (i.e. \textsc{g}{\sls a} or \textsc{g}{\sls b}, see also the personal pronouns in  Section \ref{sec:GRM-personal-pronouns}). The assignment is based on the humanness property and plurality of a referent.  Table \ref{tab:genders}(c) is repeated as Table \ref{tab:distagree} for convenience. 


\begin{table}
\caption{Prefix forms on the numeral modifiers  from 2 
to 7\label{tab:distagree}}
\centering
 \begin{tabular}{lcc}
\lsptoprule
&\textsc{-hum}=\textsc{g}\textit{a}&\textsc{+hum}=\textsc{g}\textit{b}\\
\midrule
\textsc{sg}&a&a\\
\textsc{pl}&a&ba\\
\lspbottomrule
 \end{tabular} 


\end{table} 

The following examples display gender agreement between the \isi{numeral} {\sls a-naasɛ} `four' and the nouns {\sls bʊ̃́ʊ̃̀nà}  `goats' in (\ref{ex:NUM-domnumA.pl}), {\sls vííné} `cooking pots' in (\ref{ex:NUM-domnumH-.pl}), {\sls tàátá} `languages' in (\ref{ex:NUM-domabst.pl}) and {\sls bìsé} `children'  in (\ref{ex:NUM-domnumH+.pl}). Again, the only numerals that agree in gender with the noun they modify are {\sls álìɛ̀} `two', {\sls átòrò}  `three', {\sls ànáásɛ̀} `four', {\sls àɲɔ̃́} `five', {\sls álòrò}  `six',  and   {\sls àlʊ̀pɛ̀} `seven' (see examples \ref{ex:NUM-ungramhum-} and \ref{ex:NUM-ungramhum+}). The data in (\ref{ex:NUM-domnumA.pl})-(\ref{ex:NUM-domnumH+.pl}) tells us that, when they function as controllers of agreement, nouns denoting \isi{non-human} animate, concrete inanimate and abstract entities  trigger the prefix form [{\sls a-}] on the modifying \isi{numeral},  whereas nouns denoting \isi{human} entities trigger the form [{\sls ba-}]. 

  \ea\label{ex:NUM-domnum}{\rm Agreement Domain: Numeral + Noun}
\ea\label{ex:NUM-domnumA.pl}
\gll   ŋ̩̀  kpágá   bʊ̃́ʊ̃́-ná {\bf à}-náásɛ̀ rā.\\
    \textsc{1sg}  {have}  {goat(\textsc{g}\textit{a})-\textsc{pl}} 
{\textsc{3pl.g}\textit{a}-four}  \textsc{foc}  \\
\glt `I have four goats.'\\


\ex\label{ex:NUM-domnumH-.pl}
\gll     ŋ̩̀  kpágá víí-né {\bf à}-náásɛ̀ rā.\\
  \textsc{1sg}  {have}  {pot(\textsc{g}\textit{a})-\textsc{pl}}  
{\textsc{3pl.g}\textit{a}-four}   \textsc{foc}  \\
\glt `I have four cooking pots.'\\


\ex\label{ex:NUM-domabst.pl}
\gll  ŋ̩̀ ŋmá  táá-tá {\bf à}-náásɛ̀ rā.\\
  \textsc{1sg}  {speak}  {language(\textsc{g}\textit{a})-\textsc{pl}}  
{\textsc{3pl.g}\textit{a}-four}   \textsc{foc}  \\
\glt `I speak four languages.'\\


\ex\label{ex:NUM-domnumH+.pl}
\gll   ŋ̩̀  kpágá bì-sé  {\bf bà}-náásɛ̀ rā.\\
  \textsc{1sg}  {have}  {child(\textsc{g}\textit{b})-\textsc{pl}}  
{\textsc{3pl.g}\textit{b}-four}  \textsc{foc}  \\
\glt `I have four children.'\\

\ex\label{ex:NUM-ungramhum-}
\gll   ŋ̩̀  kpágá víí-né   ŋmɛ́ŋtɛ́l rā /  dɪ́gɪ́tūō rō~~(*aŋmɛŋtɛl/*adɪgɪtʊʊ).\hspace*{-2cm}\\
  \textsc{1sg}  {have}  pot(\textsc{g}\textit{a})-\textsc{pl}  
 eight  \textsc{foc} ~ nine  \textsc{foc}\\
\glt `I have eight/nine cooking pots.'\\

\ex\label{ex:NUM-ungramhum+}
\gll    ŋ̩̀  kpágá bì-sé   ŋmɛ́ŋtɛ́l rā / dɪ́gɪ́tūō rō~~~(*baŋmɛŋtɛl/*badɪgɪtʊʊ).\hspace*{-2cm}\\
  \textsc{1sg}  {have}  {child(\textsc{g}\textit{b})-\textsc{pl}}  eight  \textsc{foc} ~ nine  \textsc{foc}\\
\glt `I have eight/nine children.'\\

\ex\label{ex:NUM-domnumH+.sg}
\gll  ŋ̩̀  kpágá víí-né   fídànáásɛ̀ rā.\\
  \textsc{1sg}  {have}  {pot(\textsc{g}\textit{a})-\textsc{pl}}  
{fourteen}\\
\glt `I have fourteen cooking pots.'\\


\ex\label{ex:NUM-domnumH+.sg.14}
\gll  ŋ̩̀  kpágá bì-sé    fídɪ́{\sls\bf bà}náásɛ̀ rā  (*fidanaasɛ ra).\\
    \textsc{1sg}  {have}  {child(\textsc{g}\textit{b})-\textsc{pl}}  
{fourteen}  \textsc{foc}  {}\\
\glt `I have fourteen children.'\\
\z
\z

Recall that in Table \ref{tab:numral11-19} the numbers from 11 to 19 were all
presented with the form
{\sls  fid(ɪ)X} `Xteen'. Their treatment as complex numerals makes one 
crucial
prediction: since they   have a common structure
[fi$_{10}$-d(ɪ)-[X$_{1-9}$]$_{atom}$]$_{complex}$ and not [fid(ɪ)X]$_{atom}$,
 agreement   has
access to the atoms X$_{2-7}$ within {\sls fid(ɪ)X}. This is
illustrated with the examples (\ref{ex:NUM-domnumH+.sg}) and
(\ref{ex:NUM-domnumH+.sg.14}) using the word {\sls fidanaasɛ}
`fourteen'.
These two examples show that in cases where a controller is specified for
both \textsc{g}{\sls b} and \textsc{pl}, it must trigger the form
[ba-] on X$_{2-7}$   within the expressions referring to the numbers 12-17.



\subsubsection{Enumeration}
\label{sec:NUM-enum}

Chakali has \isi{enumerative} forms. These  are numerals
with a purely sequential order characteristic and are used when one wishes
to count without 
any referential source or  to count off items one by one.

 \ea\label{ex:enum}
\gll dìèkèè, ɲɛ́wã́ã́, tòròò,  náásɛ̀,  ɲɔ̃́, lòrò, lʊ̀pɛ̀, 
ŋmɛ́ŋtɛ́l, dɪ́gɪ́tūo   (...)\\
one two three four five six seven eight nine {}\\
\glt `One,  two, three, four, five, six, seven,  eight, nine,  (...)' \\
\z

\newpage
Basically, the diffrence between the forms in (\ref{ex:enum}) and  the forms in  Table \ref{tab:numeralatoms} are: 
(i) a specific \isi{enumerative} use, 
(ii) the tendency to \is{lengthening}lengthen the last
vowel,\footnote{I also perceived \is{lengthening}lengthening  in \ili{Waali}, \ili{Dɛg} and
\ili{Vagla} for the
corresponding \isi{enumerative}
sequence.}
(iii)  the numerals  expressing two, three, four, five,
six, and seven do not usually display the agreement prefix,  and
(iv) the forms for `one'
and `two' differ to a greater extent. The rest of the \isi{enumerative} numerals, i.e. eight, nine, ten, 
etc., 
correspond entirely to those shown in Table
\ref{tab:numeralatoms}.  In (\ref{ex:monkey}), an excerpt of a folk tale
displays the
\isi{enumerative} use of numerals.

\begin{exe}
 \ex\label{ex:monkey}

\gll gbɪ̃̀ã́  píílí dìèkèè, ɲɛ́wã́ã́, tòròò,  náásɛ̀,  ɲɔ̃́, lòrò, lʊ̀pɛ̀, ànɪ́ 
háŋ̀ ŋ̩̀ kà sáŋɛ̄ɛ̄ nɪ́ŋ̀, dɪ́gɪ́tūo, fí.\\
Monkey starts one two three four five six seven {\conn}  \textsc{dem}
\textsc{1sg} \textsc{egr} sit  \textsc{dxm} nine ten\\
\glt `The monkey started to count: one, two, three, four, five, six, seven, the
one I'm sitting on, nine, ten.' [CB 013]
\end{exe}



\subsubsection{Distribution}
\label{sec:NUM-distri}

Reduplication has several functions in Chakali and example 
(\ref{ex:NUM-distri1}) shows that the meaning of
distribution is expressed by the reduplication of a \isi{numeral}.

\begin{exe}
\ex\label{ex:NUM-distri1}
 \gll  nɪ̀ɪ̀-tá álɪ̀ɛ̄-lɪ̀ɛ̄  ǹ̩  dɪ́ tɪ́ɛ́bá dɪ̀gɪ̀-dɪ̄gɪ̄ɪ̄.\\
  {water-\textsc{pl}} {two-two}   \textsc{1sg}   \textsc{hest}   {give.\sc
3pl.\textsc{g}\textit{b}} {one-one}\\
\glt  `Yesterday I gave two water bags to each individual.'\\
\end{exe}

%nɪ (\isi{human}) is good instead of ba above

In (\ref{ex:NUM-distri1}) the phrase containing the thing distributed and
its quantity opens the utterance. The recipient of the giving event is suffixed to the verb and 
is understood  as being more than one individual. From left to right, the reduplicated forms express the
quantity of things distributed and the number of recipients per things
distributed, respectively. This is how the distributive reading is
encoded in the utterance. Compare (\ref{ex:NUM-distri2a-1010}) and
(\ref{ex:NUM-distri2b-1010}) with
(\ref{ex:NUM-distri2c-10}).

 
\begin{exe}
\ex\label{ex:NUM-distri2-10}
\begin{xlist}
 
\ex\label{ex:NUM-distri2a-1010}
\gll à kùórù  zʊ́ʊ́ zágá  múŋ  nò  à làà kpã́ã́má fí-fí.\\
  \textsc{art}  {chief}  {enter}  {compound.\textsc{sg}}   {all}  \textsc{foc} 
\textsc{conn}  {collect}  {yam.\textsc{pl}}  {ten-ten}\\%
\glt  `From each house the chief takes 10 yams.'

\ex\label{ex:NUM-distri2b-1010}
\gll à  zágá  múŋ̀ tɪ́ɛ́  à  kùórù rō  kpã́ã́má fí-fí.\\
  \textsc{art} {compound} {all} {give}  \textsc{art}  {chief}  \textsc{foc}
yam.\textsc{pl}  {ten-ten}\\
\glt  `Each house gives 10 yams to the chief.'

\ex\label{ex:NUM-distri2c-10}
\gll à  zágá  múŋ̀ tɪ́ɛ́  à  kùórù rō kpã́ã́má fí.\\
  \textsc{art} {compound} {all} {give}  \textsc{art}  {chief}  \textsc{foc}
yam.\textsc{pl}  ten\\
\glt  `All the houses (the village) give 10 yams to the chief.'
\end{xlist}
\end{exe}


In (\ref{ex:NUM-distri2b-1010}) and (\ref{ex:NUM-distri2c-10}), the sources of
the giving event are kept constant. The reading in which
ten yams per house are being collected by the chief is accessible only
if the \isi{numeral} {\sls fi}  `ten' is reduplicated (i.e.  {\sls fifi}).

\ea
\ea\label{ex:NUM-door-two-two}

 \gll  tɪ̀ɛ̀  à gár  nʊ̃́ã́ zènè  à nã́ɔ̃́ná  jáà  zʊ̄ʊ̄  álɪ̀ɛ̀-lɪ̀ɛ̀.\\
  {give}   \textsc{art}  {fence}  {mouth}   {big}  \textsc{art} 
{cow.\textsc{pl}}   {do} {enter} {two-two}\\
\glt  `Make the door large enough since the cows often enter two by two.'\\


\ex\label{ex:NUM-akee-apple-three-four}

 \gll  à tíí bánɪ̃́ɛ̃́ jāā  átò-tòrò  wō, à  bánɪ̃́ɛ̃́ jāā  àná-náásɛ̄.\\
 \textsc{art}  {akee.apple}  {some} \textsc{ident}   {three-three}  \textsc{foc}   \textsc{art}   {some}   
\textsc{ident} {four-four}\\
\glt  `Akee apples (have) sometimes  three (seeds), sometimes four (seeds).'\\

\z
\z


%Distribution requires two core arguments: a recepient of the distribution and a
%thing distributed. A distributive event  differs from a giving event by the
%inherent reitatative property of the event, that is distribution involves at
%least two succesive giving action involving either the same thing or the same
%recipent. 
 

%From the data presented in this section, it is hard to establish whether
%Chakali has partial or complete reduplication. 


The reduplication of the \isi{numeral} {\sls álɪ̀ɛ̀} `two' in
(\ref{ex:NUM-door-two-two})
makes the
addressee understand that not only two cows might enter the cattle fence but a
possible sequence of  pairs. Similarly,   example 
(\ref{ex:NUM-akee-apple-three-four}) conveys a proposition which tells us that
the
fruit  {\sls tíì}  `Akee apple' (\textit{Blighia sapida}) can reveal sometimes
three
and sometimes
four seeds.


% dɪ̀ hĩ́ẽ́sì dɪ̀gɪ̀ɪ̀ dɪ́gɪ́ɪ́.

% Breathe slowly.

%(As the reference is to fruit tokens, the distributive reading is
%accessible in great part due to the precense of the \isi{quantifier} banɪa 'some'.)
%\vspace{1.1cm}


\subsubsection{Frequency}
\label{sec:NUM-repet}

When the morpheme {\sls bɪ}  (Section
\ref{sec:GRM-preverb-iteration}) is prefixed to a cardinal \isi{numeral}  stem, it
specifies the number of times an event happens. 

 
%(Mourelatos, 1981, p. 205)
%Mourelatos, Alexander (1981). "Events, processes, and states." In Syntax and
%Semantics: Tense and Aspect, edited by P. Tedeschi and A. Zaenen. New York:
%Academic Press.


\ea\label{ex:NUM-repet}
 \gll jà wíré jà kɪ́ná rá àkà vàlà gó dùsèè múŋ nàvàl bɪ́-tòrò.\\
 \textsc{1pl} undress  \textsc{1pl.poss} thing \textsc{foc}  \textsc{conn} 
walk cross Ducie  \textsc{quant}.all  circuit \textsc{itr-num}\\
\glt  `We undress then walk around Ducie three times.' 
\z

The meaning of {\sls bɪ}-{\sc num} corresponds to English `times'.  Example (\ref{ex:NUM-repet}) illustrates a  case where  the morpheme {\sls bɪ} is prefixed to the \isi{numeral} stem {\sls toro} `three' and translates into `three times'.



\subsubsection{Ordinals}
\label{sec:NUM-partitive}


Ordinal numerals are seen as those expressions conveying ranks or orders. The investigation carried out  showed that the language does not have a morphological marker or unique forms responsible for such a phenomenon. Chakali expresses ranking and order by other means.

\ea
\ea\label{ex:thirdmound}
\gll A: lìé ɪ̀ kà tá à pár?\\
 {} {where} \textsc{2sg} \textsc{egr} {leave} \textsc{art} {hoe}\\
\glt   `Where did you leave the hoe?'

\ex\label{ex:thirdmound-2}
\gll B: ǹ̩ gɪ́lá à pár rá píé ātòrò tɪ̀n  gàntàl  nɪ̄.\\
 {}  \textsc{1sg}  {leave} \textsc{art}   {hoe} \textsc{foc} 
 {yam.mound.\textsc{pl}} {three} \textsc{art}  \textsc{reln}  \textsc{postp}\\
\glt   `I left the hoe behind the third yam mound.'

\z
\z


In example (\ref{ex:thirdmound-2}),  the expression {\sls píé ātòrò tɪ̀n  gàntàl  nɪ̄} is best translated as `behind the third yam-mound' and not as `behind the three yam-mounds'. In the context of B's utterance, there is no  salient set of three mounds.  

The word  {\sls sɪnsagal} is frequently  used in combination with a \isi{numeral} to express a non-specific amount. For example  {\sls tʊ́sʊ̀ nɪ̄ sɪ́nsáɣál}  can be translated into English as  `thousand and something'. In addition,  the word {\sls sɪnsagal} can be combined with a \isi{numeral} to identify sibling ranks. In (\ref{ex:sibling})  {\sls sɪnsagal} is understood as `follower(s)'.  


\begin{exe}
\ex\label{ex:sibling}{\rm Sibling relationship}
\begin{xlist}

\ex\label{ex:sibling-a}
\gll ʊ̀ sɪ́nságál bátòrò jáá-ŋ̀.\\
   \textsc{3sg.poss} {follower} {three} \textsc{ident-1sg}\\
 \glt  `After him/her, I'm the third.'

\ex\label{ex:sibling-b}
\gll ǹ gàntàl tʊ́má jáá bàlɪ̀ɛ̀ wā.\\
 \textsc{1sg.poss} {back} {owners} \textsc{ident}  {two}  \textsc{foc}\\
\glt   `I have two siblings younger than me.' 


\ex\label{ex:sibling-c}
\gll ǹ̩ sʊ́ʊ́ tʊ́má jáá bàlɪ̀ɛ̀ wā.\\
   \textsc{1sg.poss} {front} {owners} \textsc{ident}  {two}  \textsc{foc}\\
 \glt  `I have two siblings older than me.'
\end{xlist}
\end{exe}



Further, in a situation where a speaker wishes to express the fact that he/she won a race by getting to an a priori agreed goal, a natural way of expressing this would be  {\sls n̩  jaa dɪgɪmaŋa tɪɪna},  {\it lit.} I-is-1-owner,  `I am first'. The second and third (and so on) positions can also be expressed using the same construction, e.g. {\it lit.} I-is-N-owner, `I am Nth'). However,  there are other ways to express the same proposition: any of the expressions given in (\ref{ex:race}) is appropriate in this context.

\ea\label{ex:race}{\rm Position in a race}
\ea 
\gll à bàtʃʊ́álɪ́ɪ́ nɪ̄ ǹ̩ ná àlɪ̀ɛ̀  rā\\
 \textsc{art}  {race} \textsc{postp}   \textsc{1sg}  {see} {two}   \textsc{foc}
\\
 \glt  `At the race, I arrived second.'

\ex 
\gll mɪ́ŋ díjèē\\
 \textsc{1sg.st} {eat.\textsc{pfv}}\\
 \glt  `I arrived  first.' or `I won.'

\ex 
\gll mɪ́ŋ nɪ́ té sʊ̄ʊ̄, ɪ̀ sàɣà\\
 \textsc{1sg.st} {postp} {early} {front}  \textsc{2sg} {be.on}\\
 \glt  `I arrived  first, you followed.'
\z
\z

Finally, the word {\sls búmbúŋ} is translated into the non-numeric English idiom `at first' and refers to a past state, its beginning or origin.

\ea\label{ex:seqevent}
 \gll   búmbúŋ ní ǹ̩ fɪ́ wàà nʊ̃̄ã̄ sɪ̄ŋ̀.\\
first {\postp}  {\sc 1sg} {\sc pst} {\sc neg} drink alcoholic.drink\\
\glt  `At first, I was not drinking alcoholic beverage.'
\z


% \subsection{Number verbs (TO DO)}
% \label{sec:NUM-verb}
% 
% {\sls kpá pɛ̀} `to add'
% {\sls lɛ̀sɪ́tà} [{\sls lɛsta}] `subtract'
% {\sls bóntí}  `to divide'
% X
% {\sls ja } `equal'
% `to count'

 %A large quantity; a multitude,
%determine the number or amount of; count.
% total in number or amount; add up to.
%To constitute a group or number


\subsubsection{Miscellaneous usage of number concept} 
\label{sec:NUM-misc-usage}

In the performance of some rituals or customs, the number concepts 3 and 4 are associated with male and female respectively. Let us illustrate this phenomenon with some examples. The {\sls lóbānɪ̄ɪ̄} section of Ducie has a funeral song which is performed at the death of a co-inhabitant. The song is repeated three times if the deceased is a man and four in the case of a woman. When a person is initiated to {\sls sɪ́gmàá}, a male must drink the black medicine in three successive occurrences and a female in four.  On the fifth day of the last funeral ({\sls lúsɪ́nnà}), the children of the deceased are given food in a particular way which involves offering the food and pulling  it back repeatedly: three times for a male and four for a female. The same associations number-sex (i.e. {\it three-male} and {\it four-female}) are found in \citet[68-70]{Card27} where it is reported that, among the Kasena, a woman must stay in her room three days after delivering a boy but four after delivering a girl. Also,  the umbilical cord of a boy is twisted three times around her finger after being removed, but four times in the case of a girl.


Two unusual phenomena involving numbers must be included. The first is also found in neighboring languages (\ili{Dagaare}, \ili{Waali}, Buli, and probably others). The phrase {\sls tʃɔ̀pɪ̀sɪ́ ālɪ̀ɛ̀} is used in greetings (Section \ref{sec:GRM-greet}).   It literally means `two days', yet it implies that the speaker has not met the addressee for a long period  (i.e. days, weeks or years), or an interval longer than usual interactions between co-inhabitants. In other languages, I have been informed that one can say `two months' or `two years', but in Chakali, even if someone has not seen another person for years it is appropriate to say  {\sls tʃɔ̀pɪ̀sɪ́ ālɪ̀ɛ̀}  `two days'. The second concerns the reference to the number of puppies in a litter. When a speaker wishes to express the number of puppies a bitch has delivered, then she/he must add ten to the actual number. For example,  to express that a dog has given birth to two puppies, one must say {\sls ʊ̀ lʊ́lá fídālɪ̀ɛ̀},  {\it lit.}  `She give.birth twelve'. 


\subsubsection{Currency}
\label{sec:NUM-currency}

One peculiarity of Chakali appears when numerals are used in the domain of currency. For example,  in (\ref{ex:70000}) the speaker needs to sell a grasscutter (cane-rat) for the price of seven Ghana cedis.


\ea\label{ex:70000}
\gll kɔ̀sá átòrò ànɪ̀ mátʃéó àlɪ̀ɛ̀ ànɪ́ fī.\\
 hundred.\textsc{pl} three and twenty two and ten\\
\glt `Seven new Ghana Cedis, or seventy thousand old Ghana Cedis' ({\it lit.} three hundred and fifty)\\
\z 

Accounting for the reference to seven Ghana cedis with an expression literally meaning three hundred and fifty (as was demonstrated in the previous sections) is done in two steps.  First, Chakali speakers (still) refer to the old Ghanaian currency (1967-2007), which after years of depreciation was redenominated (July 2007). Today,  one new Ghana cedi ({\W ₵}) is worth 10,000 old Ghana cedis.\footnote{The term \textit{old} and \textit{new} were especially used in the period of transition. The redenomination of July 2007 is the second in the cedis history. The cedi was introduced by Kwame N'krumah in 1965, replacing the British West African pound (2.4 cedis = 1 pound), but lasted only two years. Thus,  the first redenomination actually occured in 1967.}  Secondly, the Chakali word denoting `bag'  is  {\sls bʊ̀ɔ̀tɪ́à} ({\sc pl} {\sls  bʊ̀ɔ̀tɪ̀sá},  \textit{etym.}  {\sls bʊɔ-tɪa} `hole-give').  There is evidence that the word has at least one additional sense in the language. In (\ref{ex:price-market}) the prices of some items are presented.\footnote{The prices are those recorded at the market in Ducie in February 2008.}



\ea\label{ex:price-market}

\ea\label{ex:yamtubers}
\gll bʊ̀ɔ̀tɪ̀à màtʃéó  átòrò ànɪ́ fí dɪ̀  àɲɔ̃́.\\
bag twenty three and ten and five\\
\glt `15,000' (for  three yam tubers)




\ex\label{ex:groundnutbag}
\gll bʊ̀ɔ̀tɪ̀à tʊ́sʊ̀.\\
bag thousand\\
\glt `200,000' (for a bag of groundnuts)


\ex\label{ex:driedcassava}
\gll bʊ̀ɔ̀tɪ̀à kɔ̀sá ālɪ̀ɛ̀.\\
bag hundred two\\
\glt `40,000' (for a basin of dried cassava)


\ex\label{ex:cassavabag}
\gll bʊ̀ɔ̀tɪ̀à kɔ̀sá ŋmɛ́ŋtɛ́l.\\
bag hundred eight\\
\glt `160,000' (for a bag of dried cassava) 


\ex\label{ex:ricebowl}
\gll bʊ̀ɔ̀tɪ̀à màtʃéó   ànáásɛ̀ ànɪ́ fī.\\
bag twenty four and ten\\
\glt `18,000' (for a bowl of rice) 

\ex\label{ex:Ricebag}
\gll bʊ̀ɔ̀tɪ̀à tʊ́sʊ̀ ànɪ́ kɔ̀sá  àɲɔ̃̄.\\
bag thousand and hundred five\\
\glt `300,000' (for a bag of rice) 


\z
\z

\largerpage[-1]
In (\ref{ex:price-market}) the word {\sls bʊɔtɪa} initiates  each expression. Since the expressions refer solely to the amount of money, it is clear that the word {\it bʊɔtɪa} does not have the  meaning `bag' but that  the meaning of a \isi{numeral}, i.e. 200 can be inferred. The distinction between {\it bʊɔtɪa}$_{1}$ (=bag) and {\sls bʊɔtɪa}$_{2}$  (=200) is supported by the following observations:  On some occasions where  {\sls bʊɔtɪa} is used,  the word cannot refer to `bag' since there are no potential referents available. In the position it occupies in (\ref{ex:price-market}) {\sls bʊɔtɪa} is usually not pluralized, which is obligatory for a modified noun. Further, the word {\sls kómbòrò} `half' can modify {\sls bʊɔtɪa}$_{1}$  to mean `half a bag' (i.e. maize, groundnuts, etc), but  the expression {\sls bʊ̀ɔ̀tɪ̀à kómbòrò} cannot mean `100 cedis' in the language.\footnote{This claim was recently challenged by one of my consultants who recalls his  mother using  {\sls bʊɔtɪa komboro} to mean `100 cedis'.  Compare this with English  where one can say \textit{half a grand} to mean 500 dollars. The reason why {\sls bʊɔtɪa komboro} was originally rejected was perhaps that 100 old cedis was a very small sum  in 2008 and it was almost impossible to hear the expression. In 2009,  another informant claimed never to have heard such an expression to mean 100 old cedis.}  Going back to the form of the expression given in (\ref{ex:70000}), it was also observed that in a conversation in which the reference to money is understood, {\sls bʊɔtɪa}$_{2}$  is often not pronounced. One can use the utterance {\sls tʊ́sʊ̀}  `thousand' to refer to the price of a bag of groundnuts, that is an amount of two hundred thousand old cedis.\footnote{While a synchronic account of a sense distinction for the form {\sls bʊɔtɪa} in Chakali is introduced, a diachronic one is complicated by the reliability of oral sources and a lack of written records. The origin of a sense distinction of the form {\sls bʊɔtɪa},  and its equivalent,   is found to be widespread in West Africa.  The lexical item being discussed here is in Yoruba {\sls ʔàkpó}, Baatonum {\sls bʊɔrʊ}, \ili{Hausa} {\sls kàtàkù},  Dagbani {\sls kpaliŋa}, \ili{Dagaare} {\sls bʊɔra}, \ili{Dagaare} (Nandom dialect) {\sls vʊɔra}, \ili{Sisaala} {\sls bɔ̀tɔ́} and \ili{Waali} {\sls bʊɔra}. Whether the word is polysemous in all these languages as it is in Chakali, I do not know. \ili{Akan} and \ili{Ga}̃ had something similar but seem to have lost the reference to currency: a study of the words {\sls  bɔ̀tɔ́} and {\sls kotoko}/{\sls kɔtɔkɔ}  is needed.}  The distinguishing characteristic of {\it bʊɔtɪa}$_{1}$ is that it is a common noun and refers to `bag' and that {\it bʊɔtɪa}$_{2}$ is an atomic (and a base) \isi{numeral}. The latter is  a kind of hybrid \isi{numeral}, a blend of a measure term and a \isi{numeral} term,   which is only used in the domain of currency.
 
\subsection{Demonstratives}
\label{sec:GRM-demons}


Unlike the pronominal demonstrative which acts as a noun phrase, a demonstrative within the noun phrase modifies the head noun. The demonstratives in the noun phrase are identical to the demonstrative pronouns introduced in Section \ref{sec:GRM-demons-pro}, i.e.  {\sls haŋ}/{\sls hama}  ({\sc sg}/{\sc pl}).  


   \ea\label{ex:GRM-dem-sg}{\rm Priest talking to the shrine, holding a kola
nut above it}

\gll  má láá {\ob}kàpʊ́sɪ̀ɛ̀ háŋ̀{\cb}$_{NP}$ ká já mɔ̄sɛ̄ tɪ̀ɛ̀ wɪ́ɪ́ tɪ̀ŋ bà 
tàà búúrè.\\
{\sc 2pl} take kola.nut {\sc dem} {\sc conn} {\sc 1pl} plead give matter {\sc
art} {\sc 3pl.b} {\sc  egr} want\\
\glt   `Take this kola nut, we implore  you to give them what they desire.'

\z

Demonstrative  modifiers are mostly used in spatial deixis, but they do not
encode a proximal/distal distinction. Further, when a speaker uses {\sls haŋ}  
in a non-spatial context, he/she tends to ignore the \isi{plural} form (see example
(\ref{ex:GRM-dem-num}) below). In example (\ref{ex:GRM-dem-sg-non-spatial}), the
 demonstrative is placed before the \isi{quantifier},  which is not its canonical
position, as will be  shown in the summary examples in Section
\ref{sec:GRM-NP-sum}.\footnote{The \isi{plural} form of {\sls tɔʊ} `village' in Katua 
is {\sls tɔsɪ}. In the lect of Katua, the noun classes resemble the noun classes of
the \ili{Pasaale} dialect, especially the lect of the villages  Kuluŋ and Yaala.} 

\largerpage
   \ea\label{ex:GRM-dem-sg-non-spatial}

\gll  dɪ́ ʊ̀ nʊ̃́ʊ̃́  dɪ́ {\ob}tʃàkàlɪ̀ tɔ́sá háŋ̀ mùŋ{\cb}$_{NP}$, dɪ́ bììsáà jáá nɪ́hɪ̃̀ɛ̃̂, bánɪ̃́ɛ̃́ ká bɪ̀ ŋmá dɪ́ sɔ̀ɣlá jáá nɪ́hɪ̃̀ɛ̃̂.\\
{\sc comp} {\sc 3sg}  hear {\sc comp} Chakali villages {\sc dem} {\sc quant}.all 
{\sc comp} Biisa {\sc ident} old some {\sc egr} {\sc itr}  say {\sc comp} Sawla {\sc ident} old\\
 \glt `He hears that of all  Chakali settlements, some say that Biisa (Bisikan) is the oldest,  some also say Sawla is the oldest.' ({\it Katua, 28/03/08, Jeo Jebuni})

\z


% %How does one makes the difference then? but   we notice that by adding the
% %\isi{article} {\sls a} one can capture the meaning of the  proximal/distal
% % distinction. 

The examples in (\ref{ex:GRM-dem}) show that the typical position of  the
demonstrative is after the head noun and before the \isi{postposition}, after the
\isi{numeral},  but before the \isi{article} {\sls tɪŋ}. 


\ea\label{ex:GRM-dem} 
 
 
  \ea\label{ex:GRM-dem-n-postp} 
 \gll {\ob}tʃʊ̀ɔ̀sá háŋ̀{\cb}$_{NP}$ nɪ́ ǹ̩ǹ̩ dí kʊ̄ʊ̄ rā.\\
 morning {\sc dem} {\sc postp} {\sc 1sg} eat t.z. {\sc foc}\\
 \glt  `This morning I ate {\sc t.z.}.'

   \ex\label{ex:GRM-dem-num} 
 \gll {\ob}nárá bálɪ̀ɛ̀ háŋ̀{\cb}$_{NP}$ nā sɛ́wɪ́jɛ́ à mʊ́r.\\
person two {\sc dem}  {\sc foc} write {\sc art} story\\
\glt `{\sc these two men} wrote the story.' 

   \ex\label{ex:GRM-dem-art}
 \gll làà {\ob}mʊ́sá záál háŋ̀ tɪ̀ŋ{\cb}$_{NP}$.\\
 collect Musa fowl  {\sc dem} {\sc art}\\
 \glt `Collect  Musah's  fowl'  


 
\z 
 \z


% \subsection{}
% \label{sec:GRM-}
%gba = too


\subsection{Focus and negation}
\label{sec:GRM-foc-neg}

When the \isi{focus} is on a noun phrase, the free-standing particle {\sls ra} appears
to the right of the noun phrase (see Section \ref{sec:focus-forms} for the
various forms the \isi{focus} particle can take). The   particle {\sls lɛɪ} `not' also  appears 
free-standing
to the right of the noun phrase, but it is part of the word in the case of a complex \isi{quantifier}
(see Sections \ref{sec:GRM-quantifier}  and  \ref{sec:classifier}). Focus and
\isi{negation} particles cannot co-occur together in a single noun phrase.  


 \ea\label{ex:GRM-foc-neg}{\rm Identification repair for sets of cats shown on an illustration}\\
 \gll {\ob}à dìèbísè hámà{\cb}$_{NP}$ lɛ̀ɪ́, {\ob}hámà{\cb}$_{NP}$ rā.\\
  {\sc art} cats {\sc dem.pl} {\sc neg}  {\sc dem.pl} {\sc foc}\\
 \glt   `Not these cats, {\sc these cats}.'

\z

In  (\ref{ex:GRM-foc-neg}), {\sls lɛɪ} `not' negates the noun phrase {\sls a 
diebise hama} and {\sls ra} puts the \isi{focus} on the demonstrative \isi{pronoun} {\sls hama}, referring to a different set of cats.  Both \isi{focus} and \isi{negation} particles can be thought as having scope over the noun phrases, functioning as discourse 
particles. 


\ea\label{ex:GRM-neg-cntrst}
  \ea\label{ex:GRM-neg-cntrst-nom}

 \gll  mòlìbíí lɛ́ɪ̀  kàà tɪ́ɛ̀ nárá tʊ̀ɔ̀rà.\\
 money {\sc neg} {\sc ipfv} give people problem\\
 \glt  `It is not money that gives people problems.'

 
  \ex\label{ex:GRM-neg-cntrst-verb}
 \gll   mòlìbíí wàà tɪ́ɛ̀ nárá tʊ̀ɔ̀rà.\\
 money {\sc neg.ipfv} give people problem\\
 \glt  `Money does not give people problems.'

\z
\z

Example ({\ref{ex:GRM-neg-cntrst}}) compares similar propositions involving \isi{negation}. While 
({\ref{ex:GRM-neg-cntrst-nom}}) presupposes   it is something else than money  that gives  problems 
to people,  ({\ref{ex:GRM-neg-cntrst-verb}}) says that money does not give people problems.




\ea\label{ex:GRM-foc-lenght}
  \ea\label{ex:GRM-foc-w-lenght}
 \gll wáá/kàláá kpágá bʊ̀ɲɛ̃́.\\
 {\sc 3sg.st}/K.{\foc}  have respect\\
 \glt  `{\sc He/Kala} has respect for others'

   \ex\label{ex:GRM-foc-n-lenght} 
 \gll ʊ̀/kàlá kpágá bʊ̀ɲɛ̃́ rá.\\
{\sc 3sg}/K.  has respect {\foc}\\
\glt `He/Kala has {\sc respect for others}.' 

 \ex\label{ex:GRM-foc-w-lenght-2} 
\gll  wáá/bèléé kpágá záàl.\\
 {\sc 3sg.st}/bush.dog.{\foc}    catch fowl\\
 \glt   `{\sc It/bush dog} catches fowls.'

\ex\label{ex:GRM-foc-w-lenght-3} 
\gll  ʊ̀/bèlè kpágá záál là.\\
 {\sc 3sg}/bush.dog catch fowl {\foc}\\
 \glt  `It/bush dog catches {\sc fowls}.'

\z
\z


The \isi{focus} contrast offered in (\ref{ex:GRM-foc-lenght}) is still in need of validation:  one 
consultant insisted that if the \isi{focus} particle does not appear after the object of {\sls kpaga}, the 
subject --  in this case a \isi{pronoun} or a \isi{proper noun} -- needs to be \is{lengthening}lengthened and display high tone. 
This appears to co-relate to the distinction offered for personal \isi{pronoun} in Section 
\ref{sec:GRM-personal-pronouns}.


\subsection{Coordination of nominals}
\label{sec:GRM-coord-nom}

\subsubsection{Conjunction of nominals}
\label{sec:GRM-conjunc-nom}


The coordination of nominals is accomplished by means of the conjunction particle {\sls anɪ} (glossed {\sc conn}).  The vowels of the \isi{connective} are heavily centralized and the initial vowel is often dropped in fast speech. The particle can be weakened to [nə], or simply [n̩], when the preceding and following phonological material is vocalic.  A coordination of two indefinite noun phrases is displayed in (\ref{ex:GRM-coor-ani}). 


 \ea\label{ex:GRM-coor-ani} 
 \gll váá ànɪ́ dìèbíè káá válà.\\
dog {\sc conn} cat {\sc  egr} walk\\
 \glt  `A dog and a cat are walking.'
\z

 The coordination of a sequence of more than two nouns is given in (\ref{ex:GRM-coor-sequen}). It is possible to repeat the \isi{connective} {\sls anɪ}, but a pause between the items in a sequence is more frequently found. 

 \ea\label{ex:GRM-coor-sequen} 
 \gll  bʊ̃́ʊ̃́ŋ, váà ànɪ́ dìèbíè káá válà.\\
 goat,  dog {\sc conn} cat {\sc  egr} walk\\
 \glt   `A goat, a dog, and a cat are walking.'
\z

When a sequence of  two  modified nouns are conjoined, the head of the second
noun phrase may be omitted if it refers to the same kind of entity as
the first head noun. This is shown in (\ref{ex:GRM-coor-sequen-kind}).


 \ea\label{ex:GRM-coor-sequen-kind}
 \gll ǹ̩ kpáɣá tàɣtà zén nē ànɪ́ (tàɣtà) ábūmmò.\\
{\sc 1sg} have shirt large  {\sc foc} {\sc conn} (shirt)  black\\
 \glt   `I got a large shirt and a black shirt.'
\z

If the conjoined noun phrase is definite, the \isi{article} {\sls tɪŋ}
follows both conjuncts. This is shown in (\ref{ex:GRM-qual-conj}) where the
\isi{connective} appears between two qualifiers.

 \ea\label{ex:GRM-qual-conj}
 \gll   à kór ábúmmò ànɪ́ ápʊ̀mmá tɪ̀ŋ.\\
{\sc art}  bench black {\sc conn} white {\sc art}\\
 \glt   `the black and white chair (one particoloured chair)'
\z

When the weak personal pronouns (Section \ref{sec:GRM-personal-pronouns}) are conjoined there are limitations on the order in which they can 
appear. The disallowed sequences seem to be caused by two constraints. First, consultants usually 
disapproved of  the sequences where a singular \isi{pronoun} is placed after a \isi{plural} one. Examples are 
provided in (\ref{ex:GRM-conj-const-1}).



\ea\label{ex:GRM-conj-const-1}
% \begin{multicols}{2}
\ea\label{ex:GRM-conj-const-1-g}{\rm Acceptable}\\
{\sc 1sg  conn 2pl} $>$ /{\sls n̩ anɪ ma}/ [ǹnɪ́mā] \\
{\rm `I and you ({\sc pl})'}\\
{\sc 1sg  conn 3pl.g}a  $>$   /{\sls n̩ anɪ a}/ [ǹnánāā]\\
{\rm  `I and they ({\sc --hum})'}\\
{\sc 3sg   conn 2pl} $>$ /{\sls ʊ anɪ ma}/ [ʊ̀nɪ́mā]\\
{\rm   `she and you ({\sc pl})'}\\
{\sc 3sg conn 3pl.g}b $>$   /{\sls ʊ anɪ ba}/ [ʊ̀nɪ́bā]\\
{\rm  `she and they ({\sc +hum})'}

% \columnbreak 
% \vfill

\ex\label{ex:GRM-conj-const-1-ng}{\rm Unacceptable}\\
{\sc  2pl conn 1sg}   $>$ *{\sls /ma  anɪ n̩/}\\
{\sc 3pl.g.}a   {\sc conn 1sg}  $>$  *{\sls /a anɪ n̩/}\\
{\sc  2pl conn 3sg} $>$ */ma anɪ ʊ/\\
{\sc 3pl.g.}b  {\sc conn 3sg} $>$ *{\sls /ba anɪ ʊ/}\\

\z 
% \end{multicols}
 \z

Secondly, the first person \isi{pronoun} {\sls n̩} cannot be found after the 
conjunction, irrespective of the \isi{pronoun} preceding it. The reason may be a 
constraint on the syllabification of two successive nasals.  In 
(\ref{ex:GRM-coor-sequen-nasal}), it is shown that the vowels of the 
conjunction 
{\sls anɪ} either  drop or assimilate the quality of the following vowel. In 
addition, a segment  {\sls n} is inserted between the conjunction and the 
following \isi{pronoun}. 



 \ea\label{ex:GRM-coor-sequen-nasal}
/ʊ anɪ ʊ/   {\rm  3}{\sc sg} {\sc conn}  {\rm  3}{\sc sg} $>$ [ʊ̀nʊ́nʊ̀] {\rm  `she and she'}\\
/ʊ anɪ ɪ/   {\rm  3}{\sc sg}  {\sc conn} {\rm  2}{\sc sg} $>$ [ʊ̀nɪ́nɪ̀]  {\rm   `she and you'}\\
/n̩ anɪ ʊ/   {\rm  1}{\sc sg}  {\sc conn}  {\rm  3}{\sc sg} $>$ [ǹ̩nʊ́nʊ̀] {\rm   `I and she'}\\
/n̩ anɪ ɪ/   {\rm  1}{\sc sg}  {\sc conn}  {\rm  2}{\sc sg} $>$ [ǹ̩nɪ́nɪ̀]  {\rm `I and you'}\\
/ɪ anɪ n̩/  {\rm  2}{\sc sg} {\sc conn}  {\rm  1}{\sc sg} $>$ *[ɪn(V)nn̩]
\z


If the first person \isi{pronoun} {\sls n̩} were to follow the conjunction, there would be  (i) no vowel quality to assimilate, and (ii) three successive homorganic nasals, i.e. one from the conjunction, one inserted and one from the first person \isi{pronoun}, which would give  rise to a sequence {\sls n(V)nn̩}. As shown in Table \ref{tab:GRM-conj-pron}, these problems do not arise when the strong pronouns ({\sc st}) are used. 



\begin{table}
 
 \caption[Conjunction of pronouns]{Conjunction of pronouns;  weak
\isi{pronoun} ({\sc wk}) and   \isi{strong pronoun} ({\sc st}) \label{tab:GRM-conj-pron}}

  \centering
  \begin{Itabular}{lllll}
\lsptoprule 
 & 3.\sg \& 3.\sg & 3.\sg \& 2.\sg & 3.\sg \& 1.\sg &
2.\sg \&1.\sg\\
 \midrule

{\sc wk conn wk} &
ʊnʊnʊ & ʊnʊnɪ & \textasteriskcentered & \textasteriskcentered\\

{\sc wk conn wk} &
ʊnʊnʊ & ɪnʊnʊ &  n̩nʊnʊ &  n̩nɪnɪ\\

{\sc wk conn st} &
ʊnɪwa & ʊnɪhɪŋ &  {\sls ʊnɪmɪŋ} &  {\sls ɪnɪmɪŋ}\\

{\sc st conn wk} & 
wanʊnʊ & hɪnnʊnʊ & mɪnnʊnʊ & mɪnnɪnɪ\\

{\sc st conn st} &
wanɪwa & wanɪhɪŋ &  {\sls wanɪmɪŋ}  & mɪnnɪhɪŋ\\
\lspbottomrule
 
  \end{Itabular}
 
\end{table}


In Section \ref{sec:GRM-preverb-three-int-tense},  the temporal nominal {\sls  dɪarɛ} `yesterday' and {\sls tʃɪa} `tomorrow'  are said to have preverbs counterpart in a  three-interval tense subsystem.  The  temporal nominal  {\sls zaaŋ} (or {\sls zɪlaŋ}) expresses `today',  and   {\sls tɔmʊsʊ} can express either `the day before yesterday' or  `the day after tomorrow',   yet neither  have a corresponding preverb.   Thus {\sls  dɪare} `yesterday',  {\sls tʃɪa} `tomorrow',   and  {\sls zaaŋ} `today', which  typically function as adjunct  and can be disjunctively connected by the nominal \isi{connective} {\sls anɪ}, as in    (\ref{ex:GRM-adj-temp-adv-conn}),  are  treated as nominals.



\ea\label{ex:GRM-adj-temp-adv-conn}
\gll dɪ̀àrɛ̀,  zɪ̀láŋ ànɪ́ tʃɪ́á mūŋ jáá dɪ́gɪ́máŋá rá.\\
 yesterday  today  {\sc conn} tomorrow {\sc quant}.all  {\sc ident} one {\sc foc}\\
\glt `Yesterday, today, and tomorrow are all the same.'
\z


\paragraph{Apposition}
\label{sec:GRM-np-apposition}


\begin{exe}
 \ex\label{ex:GRM-coor-appo} 
 
 \gll kùórù bìnɪ̀hã́ã̀ŋ ŋmá tɪ̀ɛ̀ [ʊ̀ ɲɪ́ná kùórù]$_{NP}$ dɪ́ à
báàl párá   à kùó pétùù  (...)\\
chief young.girl say  give {\sc 3sg.poss} father   chief that 
{\sc art} man farm {\sc art} farm   finish.{\foc}  {}\\

 \glt  `The daughter told her father the chief that the young
man had finished weeding the farm (...)' [CB 014]
\z


There is another conjunction-type of nominal coordination. The noun phrase 
{\sls 
ʊ ɲɪna kuoru} `her father chief'  in (\ref{ex:GRM-coor-appo}) is treated as two
noun phrases in apposition. In this case, apposition is represented as [[ʊ
ɲɪna]$_{NP}$ [kuoru]$_{NP}$]]$_{NP}$.
% in which  the definite noun phrase precedes the
%indefinite one.


\subsubsection{Disjunction of nominals}
\label{sec:GRM-disjunct-nom}

In a disjunctive coordination, the language indicates a
contrast or a choice by means of a high tone and long {\sls káá}, 
equivalent
to 
English `or'. 
The \isi{connective} {\sls káá}  is  placed between  two disjuncts. The alternative questions in 
(\ref{ex:GRM-disjct}) are given as examples.

\ea\label{ex:GRM-disjct}
\ea\label{ex:GRM-disjct-1}

\gll ɪ̀ búúrè tí rē káá kɔ́fɪ̀?\\
  {\sc 2sg} want tea {\sc foc} {\sc conn} coffee\\
\glt  `Do you want tea or coffee?' 

\ex\label{ex:GRM-disjct-2}

\gll ɪ̀ búúrè tí rē káá kɔ́fɪ̀ rā ɪ̀ dɪ̀ búúrè?\\
  {\sc 2sg} want tea {\sc foc} {\sc conn} coffee {\sc foc} {\sc 2sg}
{\sc ipfv} want\\
\glt  `Do you want tea or do you want coffee?' 

\z 
 \z



This  \isi{connective}
 should not be confused with the three conjunctions used to connect verb
phrases and clauses, i.e. {\sls aka}, {\sls ka} and {\sls a} (see Section
\ref{GRM-clause-coord}).   

%It can also be used at the end
%of a yes/no \isi{interrogative} clause (see
%Section ).


\ea\label{ex:GRM-dij-vp4.5}

\gll ɪ̀ kàá tʊ̀mà tɪ̀ɛ̀ à kùórù ró zàáŋ káá tʃɪ́á?\\
  {\sc 2sg} {\sc fut} work give {\sc art} chief  {\sc foc} today or
tomorrow\\
\glt  `Will you work for the chief today or tomorrow?' 
\z



Example (\ref{ex:GRM-dij-vp4.5}) shows that the same
p\isi{article} may also occur between
 \is{temporal nominals}temporal nominals. 





\subsection{Two types of agreement}
\label{sec:GRM-agrrement}

Agreement is a phenomenon which operates
across word boundaries: it is a relation between a controller and a
target in a given syntactic domain. In \citet{Corb06} 
  agreement is defined as follow: (i) the element which determines the
  agreement is the controller, (ii) the element whose form is determined by
  agreement is the target and (iii) the syntactic environment in which
  agreement occurs is the domain. Agreement features refer to the information
which is shared in an agreement domain. Finally there may be conditions on
  agreement, that  is, there is a particular type of agreement provided certain
  other conditions apply. Chakali has two types of agreement based on animacy.
They are presented in the two subsequent sections. 

\subsubsection{The gender system}
\label{sec:GRM-gender}


Gender is identified as the grammatical encoding of an agreement class.  Chakali has four domains in which agreement in gender can be observed; antecedent-anaphor, \is{possessive} possessive-noun, numeral-noun and  quantifier-noun. The values shared reflect the humanness property of the referent, dichotomizing the lexicon of nominals into a set of lexemes $a$ (i.e. human--) and a set $b$ (i.e.  \isi{human}+), thus {\sc gender} $a$ or $b$   \citep{brin08, brin08c}.  The values for the feature {\sc gender} are presented in Table \ref{tab:genders}. 


\begin{table}
  \centering
  \caption{Gender in Chakali}
\label{tab:genders}


\subfloat[][Criteria for gender]{
 \begin{tabularx}{.8\textwidth}{lllX}
\lsptoprule 
{\sc gender} & && Criteria\\\midrule
\textit{a} &&& \textit{residuals}\\
\textit{b} &&& things that are categorized as \isi{human}\\
\lspbottomrule  \end{tabularx}
}

\subfloat[][Gender in weak  and strong third-person
pronouns]{
  \begin{tabularx}{.8\textwidth}{lXX}
\lsptoprule
Pronoun & {\sc wk}  & {\sc st}\\
Grammatical function  & {\sc s|o}  &  {\sc s}\\\midrule

{\it 3sg.}  & {\sls ʊ} & {\sls waa}\\
{\it  3pl.\textsc{g}a} & {\sls a} & {\sls awaa}\\
{\it  3pl.\textsc{g}b} &  {\sls ba} &  {\sls bawaa}\\
\lspbottomrule
  \end{tabularx}
}

\subfloat[][Agreement prefix forms]{
 \begin{tabularx}{.8\textwidth}{XXX}
\lsptoprule
&\textsc{-hum}=\textsc{g}\textit{a}&\textsc{+hum}=\textsc{g}\textit{b}\\
\midrule
{\sc sg}&{\sls a-} &{\sls a-}\\
{\sc pl}&{\sls a-} &{\sls ba-}\\
\lspbottomrule
 \end{tabularx}
}

\end{table}



  In addition to the gender values proposed in Table \ref{tab:genders}(a),  a condition constrains the controller to be \isi{plural} to observe the humanness distinction in agreement. As Tables \ref{tab:genders}(b)  and   \ref{tab:genders}(c) show, the personal pronouns in the language do not distinguish humanness in the singular but only in the \isi{plural}.
 
 The boundary separating \isi{human} from \isi{non-human} is subject to conceptual flexibility. In storytelling \isi{non-human} characters are ``humanized'', sometimes called personification, as (\ref{ex:antanaH+}) exemplifies: animals talk, are capable of thoughts and feelings, and can plan to go to funerals. If one compares the \isi{non-human} referents in example (\ref{ex:antanaH+}) and (\ref{ex:domquantH-}), the former reflects personification, while the latter does not.
 

  \ea\label{ex:antanaH+}{\rm Domain: antecedent-anaphor}\\
\gll   váá  mã̀ã̀  sʊ́wá.   ʊ̀   ŋmá   dɪ́   ʊ̀  tʃɛ̀ná  ŋmálɪ́ŋŋmɪ̃̀ɔ̃́ʊ̃̀   dɪ́   bá  káálɪ̀  ʊ̀ mã̀ã́  lúhò.\\
    dog.{\sc sg} mother.{\sc sg} {die} {he} {said} {\sc comp} {his} {friend} {bird's name} {\sc comp} {{\sc 3pl.g}{\it b}} {go} {his} {mother} {funeral}\\
\glt `The Dog's mother died. Dog asks his friend Bird ({\it Ardea purpurea}) to accompany him to his mother's funeral.'  ({\it lit}. that they should go to his mother's funeral.) 
\z

In (\ref{ex:domquant}) the \isi{quantifier} {\sls mùŋ} `all' agrees in gender with the nouns {\sls nɪ̀báálá} `men' and {\sls bɔ̀làsá} `elephants'.  The form {\sls àmùŋ} is used with \isi{non-human}, irrespective of the number value, and for \isi{human} if the referent is unique. The form  {\sls bàmùŋ} can only  appear in such a phrase if the referent is \isi{human} and the number of the referent is greater than one. In this example a contrast is being made between human-reference and animal-reference to show that it is not animacy in general but humanness which presents an opposition in the language.


\newpage 

\ea\label{ex:domquant}{\rm Domain: Quantifier + Noun}\\
\ea\label{ex:domquantH+}

\gll   nɪ̀-báál-á  \textbf{bā}-mùŋ.\\
    {person({\sc g}{\it b})-male-{\sc pl}} {\sc g}{\it b}-{\sc all}\\
\glt `all men'\\

\ex\label{ex:domquantH-}

\gll   bɔ̀là-sá  \textbf{ā}-mùŋ.\\
  {elephant({\sc g}{\it a})-{\sc pl}}  {\sc g}{\it a}-{\sc all}\\
\glt `all elephants'\\


\z 
 \z

In Section \ref{secːGRM-poss-pro}, it was shown   that the possessive pronouns have the same forms as the corresponding weak pronouns.  In (\ref{ex:domposs}),  the target pronouns agree with the covert controller, which is the possessor of the possessive kinship relation. The nouns referring to goat and \isi{human} mothers, trigger {\sc g(ender)}{\it a} and {\sc g(ender)}{\it b} respectively. In cases where the possessor is covert the proper assignment of humanness is dependent on the humanness of the possessed argument (i.e. `their child' is ambiguous in Chakali unless one can retrieve the relevant semantic  information of the possessed entity).

\ea\label{ex:domposs}{\rm Domain: Possessive (possessor) + Noun}\\

\ea\label{ex:dompossH-}

\gll  \textbf{à}   bʊ̃́ʊ̃́n-á.\\
   {\sc poss.3pl.g}{\it a} {goat.(\textsc{g}{\it a})-{\sc
pl}}\\
\glt `their goats' (possessor = goat mothers) 

\ex\label{ex:dompossH+}

\gll  \textbf{bà}   bì-sé.\\
     {\sc poss.3pl.g}{\it b}
{child.(\textsc{g}{\it b})-{\sc pl}}\\
\glt `their children' (possessor = \isi{human} mothers) 

\z 
 \z
 
 
 Example (\ref{ex:domnum}) displays agreement between the \isi{numeral} {\sls à-náásɛ̀} `four' and the nouns  {\sls bʊ̃́ʊ̃̀nà} ({\sc cl.3}) `goats',  {\sls tàátá} ({\sc cl.7}) `languages',  {\sls vííné} ({\sc cl.5}) `cooking pots' and  {\sls bìsé} ({\sc cl.1}) `children'. The  numerals that agree in gender with the noun they modify are  {\sls á-lɪ̀ɛ̀} `two', {\sls á-tòrò} `three',  {\sls à-náásɛ̀} `four',  {\sls à-ɲɔ̃́} `five', {\sls à-lòrò} `six' and  {\sls á-lʊ̀pɛ̀} `seven'. Here again, animate (other than \isi{human}), concrete (inanimate) and abstract entities on the one hand, and \isi{human} on the other hand do not trigger the same agreement pattern ({\sc anim} in (\ref{ex:domnumHA}), {\sc abst} in  (\ref{ex:domabst}),   {\sc conc} in (\ref{ex:domnumI})  vs. {\sc hum} in  (\ref{ex:domnumH+})). Clearly, as shown below, noun class membership is not reflected in agreement ({\sls tàátá} ({\sc cl.7}) `languages' triggers {\sc g}a in (\ref{ex:domabst}) and {\sls bìsé} ({\sc cl.1}) `children' triggers {\sc g}b in (\ref{ex:domnumH+})).
 
 \newpage 
\ea\label{ex:domnum}{\rm Domain: Numeral + Noun}\\
\ea\label{ex:domnumHA}

\gll  ǹ̩ǹ̩  kpáɣá  bʊ̃́ʊ̃́n-á  \textbf{à}-náásɛ̀ rā.\\
    {\sc 1sg}  {have}  {goat({\sc g}{\it a})-{\sc pl}}  {{\sc 3pl.g}{\it
a}-four} {\sc foc}\\
\glt `I have four goats.'\\

\ex\label{ex:domabst}

\gll   ǹ̩ǹ̩  ŋmá  tàà-tá \textbf{à}-náásɛ̀ rā.\\
  {\sc 1sg}  {speak}  {language({\sc g}{\it a})-{\sc pl}}   {{\sc
3pl.g}{\it a}-four} {\sc foc}\\
\glt `I speak four languages.'\\

\ex\label{ex:domnumI}

\gll  ǹ̩ǹ̩  kpáɣá  víí-né   \textbf{à}-náásɛ̀ rā.\\
  {\sc 1sg}  {have}  {cooking.pot({\sc g}{\it a})-{\sc pl}}   {{\sc 3pl.g}{\it a}-four}  {\sc foc}\\
\glt `I have four cooking pots.'\\

\ex\label{ex:domnumH+}

\gll  ǹ̩ǹ̩   kpáɣá  bì-sé  \textbf{bà}-náásɛ̀  rā.\\
  {\sc 1sg}  {have}  {child({\sc g}{\it b})-{\sc pl}}   {{\sc 3pl.g}{\it
b}-four}  {\sc foc}\\
\glt `I have four children.'\\

\z 
 \z


Example (\ref{ex:all}) shows that in a coordination construction involving the conjunction form {\sls (a)nɪ},  the targets display consistently {\sc g}{\it b} when one of the conjuncts is human-denoting.  In (\ref{ex:alla}) the noun phrase {\sls a baal} `the man' and the noun phrase {\sls ʊ  kakumuso} `his donkeys' unite to form the noun phrase acting as controller.  The noun phrase  {\sls a baal nɪ ʊ kakumuso} `the man and his donkeys' triggers {\sc g}{\it b} on targets.  Consequently, the form of the subject \isi{pronoun}, the \isi{quantifier}, the possessive \isi{pronoun} and the \isi{numeral} must expose  {\sls ba} ({\sc 3pl.}{\it b}). The rule in (\ref{ex:rule}) constrains coordinate noun phrases to trigger {\sc g}{\it b} if any of the conjuncts is specified as {\sc g}{\it b}. No test has been applied to verify whether the alignment of the conjunct noun phrases affects gender resolution.

\ea\label{ex:all}{\rm Domain: Coordinate structure with {\sls nɪ}}\\
\ea\label{ex:alla}

\gll  [à  báál   nɪ̀  ʊ̀ʊ̀  kààkúmò-sō]$_{NP}$  váláà  káálɪ̀  
tàmàlè rā.
\\
   {\sc art} {man} {\sc conn}  {\sc 3sg.poss} {donkey-{\sc pl}} {walk}  {go}
{Tamale} {\foc}\\
\glt `The man and his donkeys walked to {\sc tamale}.'\\

\ex\label{ex:Tamanaphor}

\gll  \textbf{bà}  kʊ̃́ʊ̃́wã́ʊ̃́.\\
    {{\sc 3pl}.{\sc g}{\it b}} tire.{\pfv}.{\foc}\\
\glt `They are tired.'\\

\ex\label{ex:Tamquant}

\gll   \textbf{bà}-mùŋ  nã̀ã̀sá tʃɔ́gáʊ́.\\
    {3.{\sc pl}.{\sc g}{\it b}-all} {feet.{\sc pl}} spoil.{\pfv}.{\foc}\\
\glt `They all had painful feet.'  (\textit{lit.} the feet of all.of.them)

\ex\label{ex:Tamposs}

\gll  \textbf{bà}  nã̀ã̀sá  tʃɔ́gáʊ́.\\
  {{\sc 3pl.poss.g}{\it b}}  {feet.{\sc pl}}  spoil.{\pfv}.{\foc}\\
\glt `Their feet were hurting them.'\\

\ex\label{ex:Tamnum}

\gll   \textbf{bà}  jáá \textbf{bà}-ɲɔ̃́  rā.\\
  {{\sc 3pl.g}{\it b}} {\sc ident}
{3.{\sc pl}.{\sc g}{\it
b}-five} {\sc foc}\\
\glt `They were five altogether.'\\

\ex\label{ex:rule}

{\sc resolution rule}: {\rm When unlike gender values are conjoined
(i.e. {\sc gender} {\it  a} and {\sc gender} {\it b}), the
coordinate noun phrase determines {\sc gender} {\it b} (i.e.
{\sc g}{\it a} + {\sc g}{\it a} = {\sc g}{\it a},
{\sc g}{\it a} + {\sc g}{\it b} = {\sc g}{\it b},
{\sc g}{\it b} + {\sc g}{\it a} = {\sc g}{\it b} and
{\sc g}{\it b} + {\sc g}{\it b} = {\sc g}{\it b}).}

\z 
 \z
 
Examples (\ref{ex:antanaH+}) to (\ref{ex:all}) demonstrate how one can analyse the humaness distinction as gender. The comparison between humans, animals, concrete inanimate entities and abstract entities uncovers the sort of animacy encoded in the language. Section \ref{sec:classifier}  presents a phenomenon which shows some similarity to gender agreement.

\subsubsection{The classifier system}
\label{sec:classifier}

While there is abundant  literature describing \ili{Niger-Congo} nominal classifications and agreement systems, the grammatical phenomenon  described in this section  has not received much attention.  Consider the examples in (\ref{ex:agr1}):   

\ea\label{ex:agr1}

\ea\label{ex:agrA}
\gll  dʒɛ̀tɪ̀ kɪ̀m-bɔ́n  ná.\\
  lion.{\sc sg}  {\sc anim}-dangerous.{\sc sg}  {\sc foc}\\
\glt  `A lion is {\sc dangerous}.' (generic reading) 



\ex\label{ex:agrB}
\gll dʒɛ̀tɪ̀sá kɪ̀m-bɔ́má  rá.\\
  lion.{\sc pl}  {\sc  conc;anim}-dangerous.{\sc pl} {\sc foc}\\
\glt  `The lions are {\sc dangerous}.' (individual reading) 



\ex\label{ex:agrD}
\gll   m̩̀ bɪ̀ɛ̀rəsá  nɪ̀-bɔ́má  rá.\\
{\sc poss.1sg} {brother.{\sc pl}}  {\sc hum}-dangerous.{\sc pl} {\sc foc}\\
\glt  `My brothers are {\sc dangerous}.'


\ex\label{ex:agrE}
\gll bà  jáá  nɪ̀-bɔ́má   rá.\\
{{\sc  3pl.g}{\it b}} {\sc ident} {\sc hum}-{dangerous.{\sc pl}} {\sc foc}\\
\glt  `They are {\sc dangerous}.' (\isi{human} participants) 



\ex\label{ex:agrF}
\gll   à   jáá   kɪ̀m-bɔ́má  rá.\\
{{\sc  3pl.g}{\it a}} {\sc ident}  {\sc  conc;anim}-{dangerous.{\sc pl}}
{\sc foc}\\
\glt  `They are {\sc dangerous}.' (\isi{non-human}, non-abstract participants) 



\ex\label{ex:agrG}
\gll záɪ́ɪ́   wɪ̀-bɔ́n ná.\\
 fly.{\sc nmlz} {\sc abst}-dangerous.{\sc sg}  {\sc foc}\\
\glt  `Flying is {\sc dangerous}.'\\




\ex\label{ex:agrH}
\gll à tʃígísíí wɪ̀-bɔ́má rá.\\
{\sc art}  turn.{\sc pv.nmlz}  {\sc abst}-dangerous.{\sc pl}  {\sc 
foc}
\\
\glt  `The turnings  are {\sc dangerous}.' (repetitively turning clay bowls for
drying)\\


\z 
 \z

The sentences in (\ref{ex:agr1})  are made of two successive noun phrases. The referent of the 
first 
noun phrase is an entity or a process while the second noun phrase is semantically headed by a 
state 
predicate denoting a property.  Although speakers prefer the presence of   the  identificational 
verb {\sls jaa} `to be' between the two noun phrases, its  absence is acceptable and does not 
change 
the meaning of the sentence. In these  identificational constructions,  the comment identifies the 
topic as having a certain property, i.e. being bad, dangerous, or risky. The \isi{focus} particle follows 
the second noun phrase, hence  $[$NP1 NP2 ra$]$  means `NP1 is NP2' in which salience or novelty of 
information comes from NP2. 


The form of  {\sls /bɔm/}   `bad' is determined by the number value of the 
first noun phrase. Irrespective of the animacy encoded in the referent, a  
singular noun phrase triggers the form {\sls [bɔŋ]} while a \isi{plural} triggers 
{\sls [bɔma]} (i.e. {\sc cl.3}, Section \ref{sec:class3}).  The number agreement is illustrated in 
(\ref{ex:agrA}) and (\ref{ex:agrB}).\footnote{Notice that the nominalized 
verbal lexemes in (\ref{ex:agrG}) and (\ref{ex:agrH}) each triggers a different 
form for {\sls /bɔm/}. The  form  {\sls tʃigisii}  `turning'  is analysed as a 
nominalized pluractional verb (see Section \ref{sec:GRM-PluralVerb}).}  

Properties do not appear as  freestanding words in identificational 
constructions. To say `the lion is dangerous', the grammar has to combine the 
predicate with a nominal \isi{classifier} (or dummy substantive) that will license  a noun,  i.e. {\it lit.}  
`lion is {\it thing}-dangerous',  where {\it thing} stands for the slots where animacy is 
encoded. This is represented in (\ref{ex:frame}).  


\ea\label{ex:frame} 

{\rm[[{thing}$_{animacy}$-property] {\sc foc}]}
\z 

There are three dummy substantives in  (\ref{ex:agr1}):   {\sls  nɪ-}, {\sls  wɪ-},  and  {\sls  kɪn}-.  Each of them has a fully fledged noun counterpart; it can be pluralized, precede a demonstrative, etc. Those forms are {\sls kɪn}/{\sls  kɪna} ({\sc cl.3})  `thing',  {\sls  nar}/{\sls  nara} ({\sc cl.3}) `person' and {\sls wɪɪ}/{\sls  wɪɛ} ({\sc cl.4}) `matter, palaver, problem, etc.'.  Table \ref{tab:nounclassifier} provides the three possible distinctions.  


\begin{table}

  \caption{Classifiers and Nouns   \label{tab:nounclassifier}}
  \centering
  \begin{Itabular}{lllll}
 \midrule 
 Classifier   & Animacy & Noun class  & Sing. & Plur.\\
\midrule  \midrule
   {nɪ-}/{na-} &  $[${\sc hum}$]$ & Class 3 & nár &  nárá\\
 {wɪ-} &  $[${\sc abst}$]$ & Class 4 & wɪ́ɪ́ &   wɪ́ɛ́\\
  {kɪn}-  &  $[${\sc conc;anim}$]$ & Class 3& kɪ̀n &   kɪ̀nà\\
 \midrule 
  \end{Itabular}
\end{table}

 %(see {\it forme radicale} in
%\citet[506]{Mane64})

Since there are form and sense compatibilities between the inflecting noun pairs and the forms of the expressions preceding the qualitative predicate,  a common radical form for each is identified; {\sls kɪn}- {\sc [conc;anim] } `concrete, \isi{non-human}, non-abstract',  {\sls nɪ-} {\sc [hum] } `person, \isi{human} being'  and  {\sls wɪ-} {\sc  [abst] } `non-concrete, non-person' are the three classifiers.\footnote{The  \isi{classifier} and the semantic information encoded in the head of the first noun phrase reflects one major analytical criterion for  \isi{classifier} systems \citep{Dixo86, Corb91, Grin00}.}

All the sentences in (\ref{ex:agr1}) are ungrammatical without a \isi{classifier}. The three classifiers  combine with {\sls bɔŋ}/{\sls bɔma}  to  make  proper constituents for an identificational construction. The structural setting  is the result of a combination of grammatical constraints which specify that: (i) a property in predicative function cannot stand on its own, (ii) in predicative function,  a property must be joined with a \isi{classifier}, (iii) the merging of the \isi{classifier} and the property forms a proper syntactic constituent for an identificational construction, and (iv) the form of the \isi{classifier} is dependent on the animacy encoded in the argument of a qualitative predicate. 

Finally,  classifiers are also found in the formation of the words meaning  `something' and `nothing'. Consider the examples in (\ref{ex:something}) and (\ref{ex:nothing}):

% \begin{multicols}{2}
\ea\label{ex:nothing}

\ea\label{ex:nothingH}
\gll {ná-mùŋ-lɛ̀ɪ́}\\
 {\sc hum}-all-not\\
\glt `no one'\\
\ex\label{ex:nothingC}
\gll  {wɪ́-mùŋ-lɛ̀ɪ́}\\
 {\sc abst}-all-not\\
\glt `nothing'\\
\ex\label{ex:nothingA}
\gll  {kɪ́n-mùŋ-lɛ̀ɪ́}\\
 {\sc  conc;anim}-all-not\\
\glt `nothing'\\

\z
\z

% 
\ea\label{ex:something}
 

\ea\label{ex:somethingH}
\gll {nɪ̀-dɪ́gɪ́ɪ́}\\
 {\sc hum}-one\\
\glt `someone'\\
\ex\label{ex:somethingC}
\gll  {wɪ́-dɪ́gɪ́ɪ́}\\
 {\sc abst}-one\\
\glt `something'\\
\ex\label{ex:somethingA}
\gll {kɪ̀n-dɪ́gɪ́ɪ́}\\
 {\sc conc;anim}-one\\
\glt `something'\\

\z
\z

% \end{multicols}

As with the role of classifiers in identificational constructions, here again
the classifiers narrows down the tracking of a  referent when one of those
quantifiers is used. Again, the grammar  arranges  animacies into
three categories, i.e.  {\sc abst}, {\sc conc;anim}, and {\sc hum}.  A
distinction is also made in English between {\sc hum} (i.e. someone, no one) and
 {\sc anim;conc;abst} (i.e. something, nothing), however English does not have
a distinction which captures  specifically abstract entities.

\subsection{Summary}
\label{sec:GRM-NP-sum}

The term nominal in the present context was argued to represent two separate notions. The first is  conceptual. Nominal stems denote classes of entities whereas verbal stems denote events. The second notion is  formal. A nominal stem was opposed to  a verbal stem in noun formation.  As a syntactic unit,  the nominal  constitutes an obligatory support to the main predicate and was presented above in  various forms:   as a pro-form, a single noun, or noun phrases consisting of a noun with a \isi{qualifier}(s), an \isi{article}(s), a demonstrative,  among others. The heading of (\ref{ex:GRM-np-list}) represents the order of elements in the noun phrase in Chakali.\footnote{In  (\ref{ex:GRM-np-list}) `woman' may also be interpreted as  `wife'.}


\ea\label{ex:GRM-np-list} {\small 
\textsc{art/poss} \;  \textsc{head}\;   \textsc{qual}\;  \textsc{num}  
\;  
 \textsc{quant} \;    \textsc{dem} \;   \textsc{quant} \;   \textsc{art} \;  
\textsc{foc/neg}}\\


  \ea\label{ex:GRM-pro} 
ɪ̀   wááwáʊ́ {\rm `you came'}\\{\sc head} 

  \ex\label{ex:GRM-h} 
hã́ã́ŋ wāāwāʊ̄ {\rm  `a woman came'}\\
{\sc head} 

  \ex\label{ex:GRM-ah} 
à hã́ã́ŋ  wāāwāʊ̄ {\rm  `the woman came'}\\
{\sc art1} {\sc head} 

  \ex\label{ex:GRM-aha}
  à hã́ã́ŋ tɪ̀ŋ wāāwāʊ̄  {\rm  `the woman came'}\\
{\sc art1} {\sc head} {\sc art2}

  \ex\label{ex:GRM-ph} 
ʊ̀ʊ̀ hã́ã́ŋ   wāāwāʊ̄  {\rm  `his woman came'}\\
 {\sc poss}  {\sc head}

  \ex\label{ex:GRM-pha} 
ʊ̀ʊ̀  hã́ã́ŋ tɪ̀ŋ   wāāwāʊ̄  {\rm  `his woman came'}\\
 {\sc poss} {\sc head} {\sc art2}  

 \ex\label{ex:GRM-dhq}
hámā mùŋ  wāāwāʊ̄ {\rm   `all these  came'}\\
{\sc head} {\sc quant}{\rm .all}

 \ex\label{ex:GRM-hdq}
 nɪ̀hã́ã́ná hámā mùŋ  wāāwāʊ̄ {\rm  `all these women  came'}\\
  {\sc head} {\sc dem} {\sc quant}{\rm .all} 

  \ex\label{ex:GRM-hd} 
hã́ã́ŋ háŋ̀    wāāwāʊ̄ {\rm  `this woman  came'}\\
  {\sc head} {\sc dem}  

  \ex\label{ex:GRM-hq-all} 
 nɪ̀hã́ã́ná  mùŋ wááwáʊ́ {\rm  `all women  came'}\\
{\sc head} {\sc quant}{\rm .all}

\ex\label{ex:GRM-hq-many} 
 nɪ̀hã́káná    wāāwāʊ̄  {\rm  `many women  came'}\\
{\sc head-quant}  

  \ex\label{ex:GRM-hn} 
nárá bátòrò wāāwāʊ̄ {\rm  `three persons  came'}\\
  {\sc head} {\sc num}  

  \ex\label{ex:GRM-ahqln} 
à nɪ̀hã́ã́ná pɔ́lɛ̄ɛ̀ bálɪ̀ɛ̀   wāāwāʊ̄  {\rm  `the two fat women  came'}\\
 {\sc art1} {\sc head} {\sc qual} {\sc num}  

  \ex\label{ex:GRM-ahqlnd}
  à nɪ̀hã́ã́ná bálɪ̀ɛ̀ hámà wāāwāʊ̄    {\rm  `these two women  came'}\\
 {\sc art1} {\sc head} {\sc num} {\sc dem}  

  \ex\label{ex:GRM-ahqlqln}
à nɪ̀hã́ɲʊ́lʊ́má pɔ́lɛ̄ɛ̀ bálɪ̀ɛ̀ wāāwāʊ̄    {\rm  `the two fat blind women  came'} 
\\
  {\sc art1} {\sc head}  {\sc qual} {\sc qual}  {\sc num}  

\ex\label{ex:GRM-ahqlq} 
à nɪ̀hã́pɔ́lɛ̄ɛ̀    káná   wāāwāʊ̄  {\rm  `many fat women  came'}\\
{\sc art1} {\sc head} {\sc qual} {\sc quant}{\rm .many}

 \ex\label{ex:GRM-ahqlqlq} 
à nɪ̀hã́pɔ́lɛ̄ɛ̀    ɲʊ́lʊ́ŋkáná   wāāwāʊ̄    {\rm  `many fat blind women  came'}   
\\
{\sc art1} {\sc head} {\sc qual} {\sc qual} {\sc quant}{\rm .many}  

  \ex\label{ex:GRM-ahqlqlqd} 
à nɪ̀hã́pɔ́lɛ̄ɛ̀   ɲʊ́lʊ́ŋkáná  hámà    wāāwāʊ̄ {\rm  `these many fat 
blind 
women  came'}\\
{\sc art1} {\sc head}-{\sc qual} {\sc qual}-{\sc quant}{\rm .many}  {\sc dem}


\ex\label{ex:GRM-phqlnq} 
à nɪ̀hã́pɔ́lɛ̄ɛ̀  ɲʊ́lʊ́má  fí bānɪ̃̄ɛ̃̄ wāāwāʊ̄    {\rm  `some of the ten fat 
blind women  came'}\\
{\sc art1} {\sc head}-{\sc qual} {\sc qual} {\sc num}  {\sc quant}{\rm .some} 


%  \ex\label{ex:GRM-phqlnq} 
% m̩̀m̩̀ párá áwíríjé átòrò bánɪ́ɛ́  {\rm  `some of my three good 
% hoes'}\\
% {\sc 1sg.poss} {\sc head} {\sc qual} {\sc num} {\sc quant}.some

%\ex\label{ex:GRM-}
 
\z 
 \z
 
 
 To summarize, each noun phrase in (\ref{ex:GRM-np-list}) is a grammatically and semantically acceptable noun phrase and  respects the linear order offered on the first line. They were all elicited in a frame `X came'.  Certain orders are favored, but a strict linear order, especially among the qualifiers, needs further investigation.   Notice that each noun phrase in (\ref{ex:GRM-np-list}), except for the weak personal \isi{pronoun} in (\ref{ex:GRM-pro}),  may or may not be in \isi{focus} and may or may not be definite (i.e. accompanied by the \isi{article} {\sls tɪŋ}). Also,  the slot {\sc head}  in (\ref{ex:GRM-np-list}) is not only represented in the examples by a noun or \isi{pronoun};  example (\ref{ex:GRM-dhq}) is headed by a demonstrative \isi{pronoun}. Needless to say, this list of possible distributions of nominal elements within the noun phrase is not exhaustive. Again, caution should be taken since the examples in (\ref{ex:GRM-np-list}), particularly those towards the end of the list, are the result of elicitation. Their order of appearance can only be  interpreted  as  an approximation of the noun phrase. 


\largerpage[-2]
\section{Verbal}
\label{sec:GRM-verbals}


Any expression which can take the place of  the predicate {\sc p} in  (\ref{ex:GRM-clause-frame-2}) is identified as \is{verbal}\textit{verbal}.


\begin{exe}
  \ex\label{ex:GRM-clause-frame-2} 
 {\sc ajc}  $\pm$ {\sc s|a}  $+$ {\sc p} $\pm$ {\sc o} $\pm$ {\sc ajc} 
\end{exe}

The term  can also refer to a semantic notion at the lexeme level. The language is analysed as exhibiting two types of verbal lexeme. In Section \ref{sec:GRM-der-agent},  the {\it stative} lexeme and the {\it active} lexeme were both shown to take  part in nominalization processes. The verbal stem in (\ref{ex:verb-VP})  must be instantiated with a verbal lexeme. 

\ea\label{ex:verb-VP}
{\rm [[{preverb}]\textsubscript{EVG} [[{stem}]-[{suffix}]]\textsubscript{verb}]\textsubscript{VG}}
\z


In addition, the term  can refer to the whole of the verbal constituent, including the verbal modifiers. The verbal group  \is{verbal group} (VG) illustrated in (\ref{ex:verb-VP}) consists of linguistic slots which encode   various aspects of an event  which may be realized in an utterance. A free standing verb is the minimal requirement to satisfy the role of a predicative expression. The verbal modifiers, which are called preverbs (Section \ref{sec:GRM-precerv}),  are grammatical items which specify the event according to various  semantic distinctions. They precede the  verb(s) and take part in the expanded verbal group \is{expanded verbal group} (EVG). The expanded verbal group identifies  a domain which excludes the main verb, so a  verbal group without preverbs would  be equivalent to a verb or a series of verbs (see SVC in Section \ref{sec:GRM-multi-verb-clause}).\footnote{The term and notion are inspired from analyses of the verbal system of \ili{Ga}̃ \citep{Daku70}. A verbal group is unlike the verb phrase in that it does not include its internal argument, i.e. direct object. I am aware of the obvious need to unify the descriptions of the nominal constituent and the verbal constituent.} 

While a verbal stem provides the core meaning of the predication,  a suffix may supply information on  aspect, whether or not the verbal constituent is in \isi{focus} and/or the index of participant(s) (i.e. {\sc o}-clitic, Section \ref{sec:GRM-morph-opro}).  Despite there being little \isi{focus} on tone and intonation, attention on the tonal melody of the verbal constituent is necessary since this also affects the interpretation of the event. These characteristics are presented below in a brief overview of the verbal system. 

% 
% one melodies affecting not
% only elements of the verbal constituent but elements immediately preceding or
% following it, and (iii) affixes
%without its participant(s) and other peripheral expressions.



\subsection{Verbal lexeme}
\label{sec:GRM-verb-lexeme}


\subsubsection{Syllable structure and tonal melody of the verb}
\label{sec:GRM-verb-syll-und-tone}

There is a preponderance  of open syllables of type CV and CVV, and the  common syllable sequences found among the verbs are CV, CVV, CVCV, CVCCV, CVVCV, and CVCVCV.   In the dictionary,  monosyllabic verbs make up approximately 13\% of the verbs, bisyllabic 65\%,  and trisyllabic  22\%.  All segments are attested in onset position word initially, but only {\sls m, t, s, n, r, l, g, ŋ,} and {\sls w} are found in onset position word-medially in bisyllabic verbs, and only {\sls  m, t, s, n,  l,} and {\sls g} are found  in onset position word-medially in trisyllabic verbs.   All trisyllabic,  CVVCV,   and CVCCV verbs have one of the front vowels (\{e, ɛ, i, ɪ\}) in the nucleus of their last syllable.  The data suggests that {\sc atr}-harmony is operative, but not   {\sc ro}-harmony,  in these three environments, e.g. {\sls fùòlì} `whistle'. There is no restriction on vowel quality for the monosyllabic or bisyllabic verbs and both harmonies are operative.

 Table \ref{tab:GRM-verb-tone-melody} presents  verbs which are classified based on their syllable structures and  tonal melodies.  Despite the various attested melodies, instances of low tone CV verbs, CVV verbs other than low tone,  and rising or falling CVCV, CVCCV, and CVVCV verbs are marginal. 


%\clearpage
\begin{table}[htb]
\renewcommand{\arraystretch}{0.8}
\centering
\caption{Tonal melodies on verbs  \label{tab:GRM-verb-tone-melody}}
  
\begin{Itabular}{ll>{\slshape}ll}
\lsptoprule
Syllable type &  Tonal melody  & {\rm Form} & Gloss\\[1ex] \midrule

CV  &  H & pɔ́ &plant\\
    &  L   &sɔ̀   &wash\\[0.5ex]
\midrule

CVV   &  L & pàà   &   take\\
 & H &  kíí   &  forbid\\
 & LH &wòó &  vacant (be)\\
 & HL & gbáà &  herd\\
\midrule

CVCV    &  H  & kúló  &  tilt\\
&  L & bìlè  &  put\\
  &  HL & lúlò  &  leak\\
  &  HM & pílē &  cover\\[0.5ex]
  
  
\midrule

CVCCV   & H & bóntí  &  divide\\
 & H  & kámsɪ́&  blink\\
 &L &sùmmè  &  beg\\
 &L & zèŋsì  &  limp\\[0.5ex] 
  
\midrule

 CVVCV & H& píílí & start\\
 &H& tɪ́ásɪ́ & vomit\\
 &L &kààlɪ̀  & go\\
 &L &bùòlì  &  sing\\[0.5ex] 
 
\midrule

CVCVCV  & H  & zágálɪ́ &  shake\\
   & H  & vílímí  &  spin\\
    & L & hàrɪ̀gɪ̀&  try\\
   & L  &dʊ̀gʊ̀nɪ̀  &  chase\\[0.5ex]

\lspbottomrule
\end{Itabular}   
\end{table} 
 


Typically, CV verbs  have a high melody, while CVV verbs are  a low one. The mid tone (M)  is not contrastive. Only a handful of minimal pairs can be found in the dictionary, e.g. {\sls pɔ̀} `protect' and  {\sls pɔ́} `plant'.  

\subsubsection{Verbal state and verbal process lexemes}
\label{sec:GRM-verb-stative-active}

A general distinction
between stative and non-stative events  is made: {\it verbal state} (stative
event) and {\it verbal process} (active event) 
lexemes are assumed. A verbal state lexeme can be identificational,
existential, possessive,  qualitative, quantitative, cognitive or  locative, and
refers more or less to a state or condition which is static, as opposed to
dynamic. The `copula' verbs {\sls jaa} and {\sls dʊa} (and its allolexe {\sls 
tuo})
are treated as subtypes of verbal stative lexemes since they are the only verbal
lexemes which cannot function as a main verb in  a perfective intransitive
construction (see Section \ref{sec:GRM-verb-perf-intran}). Their meaning and
distribution was introduced in the sections concerned with the identificational
construction (Section \ref{sec:GRM-ident-cl}) and existential construction
(Section \ref{sec:GRM-loc-cl}).  The possessive verb
{\sls kpaga} `have'  is treated as  a verbal state lexeme as well (see 
possessive
clause in Section   \ref{sec:GRM-poss-cl}).  A qualitative verbal state lexeme
establishes a relation between an entity and a quality. Examples are given in
(\ref{ex:GRM-v-stat-qual}).


\ea\label{ex:GRM-v-stat-qual}{\rm Qualitative verbal state lexeme}\\

 {\sls bòró}  {\rm `short'}  $>$ {\sls à dáá bóróó} {\rm  `The tree is short.'}\\
{\sls gòrò} {\rm `curved'}  $>$ {\sls à dáá góróó} {\rm  `The wood is curved.'}\\
{\sls jɔ́gɔ́sɪ́} {\rm `soft'}   $>$ {\sls   à bìé bàtɔ́ŋ jɔ́gɔ́sɪ̀jɔ̀ʊ̄} {\rm    `The 
baby's
skin is soft.'}
\z

Similarly, a quantitative verbal state lexeme  establishes a relation between an entity and a quantity. Yet, in (\ref{ex:GRM-v-stat-quant}), the subject of   {\sls maasɪ} is the impersonal \isi{pronoun} {\sls a} which refers to a situation and not an individual. The verb {\sls hɪ̃ɛ̃}  `age' or `old'  is a quantitative verbal state lexeme since it  measures  objective maturity between two individuals, i.e. {\sls mɪŋ hɪ̃ɛ̃-ɪ}, {\it lit.} {\sc 1sg.st} age-{\sc 2sg.wk}, `I am older than you'. 


\ea\label{ex:GRM-v-stat-quant}{\rm Quantitative verbal state lexeme}\\

 {\sls kánà} {\rm  `abundant'}  $>$ {\sls bà kánã́ʊ̃́} {\rm   `They are plenty 
(people).'} 
\\
{\sls mààsɪ̀} {\rm `enough'}  $>$   {\sls à máásɪ́jʊ́} {\rm    `It is sufficient.'}\\
{\sls hɪ̃̀ɛ̃̀} {\rm `age'} $>$ {\sls mɪ́ŋ hɪ̃́ɛ̃́ɪ̃̀} {\rm    `I am older than you'}
\z

Cognitive verbs such as {\sls liise} `think',  {\sls kʊ̃ʊ̃} `wonder, 
{\sls kisi} `wish', {\sls tʃii} `hate', etc.  are also treated as verbal 
state
lexemes. 

Verbal process lexemes denote non-stative events. They are often partitioned
along the
(lexical) aspectual distinctions of  \citet{Vend57}, i.e. activities, 
achievements, accomplishments. Such verbal categories did not formally emerge, 
so I am not in a position to categorize the verbal process lexemes at this 
point 
in the research (but see \citealt[51]{Bonv88} for a thorough description of a 
\ili{Grusi} verbal system), although Section \ref{sec:GRM-verb-suffix} suggests that 
there is a system of verbal derivation  that  uses verbal process lexemes which 
needs to be uncovered.  Thus, verbs which express that the participant(s) is 
actively doing something, undergoes a process, performs an action, etc. all 
fall 
within the  set of verbal process lexemes. 



\subsubsection{Complex verb}
\label{sec:GRM-complex-verb}

A complex verb is  composed of more than one verbal lexeme. For
instance, when {\sls laa} `take' and {\sls di}
`eat' are brought together in a SVC (Section \ref{sec:GRM-multi-verb-clause}),
they denote separate taking and eating event. A complex verb denotes a single 
event.

\ea\label{ex:cpx-verb-laa-di}
\ea
 \gll ǹ̩ láá kúòsò díūū.\\
{\sc 1sg} take G.  eat.{\foc}\\
\glt `I believe in God.'

\ex
 \gll  ǹ̩ láá bìé dʊ́ʊ̄.\\
{\sc 1sg} take child put.{\foc}\\
\glt `I adopted a child.'
\z 
 \z
 
 The sequences  {\sls laa}+{\sls di} `believe'  and {\sls laa}+{\sls dʊ} 
`adopt'  
are  non-compositional, and less literal. Also, unlike complex stem nouns, but 
like SVCs, the elements which compose a complex verb must not necessarily be 
contiguous,  as  (\ref{ex:cpx-verb-laa-di}) shows. Other examples, among 
others, 
 are {\sls zɪ̀mà síí}, {\it lit.} know raise, `understand',  {\sls kpá 
tā}, 
{\it lit.}  take abandon, `drop' or `stop', and {\sls gɪ̀là zɪ̀mà}, {\it 
lit.} 
allow know, `prove'.



\subsubsection{Verb forms and aspectual distinction}
\label{sec:GRM-verb-word}

The inflectional system of Chakali verbs displays  few verb
forms and is closer  to neighbor \ili{Oti-Volta} languages than, for instance,  a
`conservative' \ili{Grusi} language like \ili{Kasem} \citep[51]{Bonv88}.\footnote{Dagbani is
described as a language where the ``inflectional system  for verbs is relatively
poor''  \citep[96]{Olaw99}. It has an imperfective suffix {\sls -di}
\citep[97]{Olaw99} and  an imperative suffix {\sls -ma}/{\sls mi} 
\citep[101]{Olaw99}.
\citet[81]{Bodo97} writes that \ili{Dagaare} has four verb forms: a dictionary
form, a perfective aspectual form, a perfective intransitive aspectual form and
an imperfective aspectual form. Also for \ili{Dagaare}, \citet{Saan03}  talks about
four forms: perfective A and B, and Imperfective A  and B.}  Besides the
derivational suffixes (Section \ref{sec:GRM-deri-suff}), the verb in Chakali is
limited to two
inflectional suffixes and one assertive suffix:  (i) one signals \isi{negation} in the
negative imperative clause (i.e.  {\sls  kpʊ́} `Kill',  {\sls tíí kpʊ̄ɪ̄} 
`Don't
kill'),  (ii) another attaches to some verb stems in the perfective intransitive
only, and (iii)  the other signals assertion and puts the verbal constituent in
\isi{focus}. Since the negative imperative clause has already been presented in
Section
\ref{sec:GRM-imper-clause}, the perfective and imperfective intransitive
constructions are discussed next.  Both are recurrent clauses in data
elicitation. The former may contain both the perfective
suffix and the assertive suffix simultaneously, while the latter  displays the
 verb, with or without the assertive suffix.
 
 
%  
% \paragraph{Base form of a verb}
% \label{sec:GRM-base-verb}
% 
% %what do you do with intransitive??
% 
% The form of the verb displayed in the dictionary is called the base form.  It  
% is identified as the segmental sequence and melody which  would appear in a
% positive imperfective transitive clause (Section \ref{sec:GRM-trans-intran}). 
% 
% 
% \ea\label{ex:GRM-base-form}{\rm  Base form = Positive imperfective transitive}\\
% \gll   bàà kʊ́ɔ́rɪ̀ sɪ̀ɪ̀máá rà.\\
% 3.{\sc pl}  make food {\sc foc}\\
% \glt `They are making food.'\\\par
% $\rightarrow$ {\rm kʊɔrɪ  (HL)}
%  \z
% 
%  This sentence frame is one that does not affect the segmental sequence and 
% melody of the verb. The base form can also correspond  to a  verb elicited in 
% isolation, although consultant are generally not at ease with verbs in 
% isolation, unless they are framed in an utterance.

\paragraph{Perfective intransitive construction}
\label{sec:GRM-verb-perf-intran}

As its name suggests, a \is{perfective intransitive construction} perfective 
intransitive construction  lacks a 
grammatical
object and implies an event's completion or its 
reaching point.  In the case of \is{verbal state}verbal state,
the  \is{perfective}perfective  implies that the given state has been reached, 
or 
that the entity in subject position   satisfies the property encoded in
the \is{verbal state lexeme}verbal state lexeme. In 
(\ref{ex:GRM-intperfc-frame}),  two suffixes 
are
attached on  one verbal process stem and one  verbal state 
stem (see Section \ref{sec:nasalization-verb-suffix}
for the general phonotactics involved).\footnote{The presence of  a schwa
({\sls ə}) in a CVCəCV surface form, as in (\ref{ex:GRM-intperfc-frame-state}), 
is explained in Sections \ref{sec:epenthesis} and \ref{sec:PHO-weak-syll}.}


\ea\label{ex:GRM-intperfc-frame}{\rm Perfective intransitive construction}\\


\ea\label{ex:GRM-intperfc-frame-process}{{\it  Verbal process:} {\sc s}  $+$
{\sc p} }\\
\gll àfɪ̀á díōō.\\
A. {di-j[{\sc -lo, -hi, -ro}]-[{\sc +hi,+ro}]}\\
\glt `Afia ate.'

\ex  àfɪ̀á wá díjē.   {\rm `Afia didn't eat.'}

\ex\label{ex:GRM-intperfc-frame-state}{{\it  Verbal state:} {\sc s}  $+$ {\sc p}
}\\
\gll à dáá télèjōó.\\
{\sc art} daa  {tele-j[{\sc -lo, -hi, -ro}]-[{\sc +hi,+ro}]}\\
\glt `The stick leans.'

\ex à dáá wá tēlə̀jē. {\rm `The stick doesn't lean.'} %check

\z 
 \z

The first suffix to attach is the perfective suffix, i.e. -j[{\sc -lo, -hi, 
-ro}] or simply /jE/. Although it appears on every (positive and negative) stem 
in (\ref{ex:GRM-intperfc-frame}),  it does not surface on all verb stems. The 
information in Table \ref{tab:GRM-perf-suff} partly predicts whether or not a 
stem will surface with a suffix, and if it does, which form this suffix will 
have.


\begin{table}[htb]
 \centering
\caption{Perfective intransitive suffixes
\label{tab:GRM-perf-suff}}
\begin{Itabular}{p{2cm}p{2cm}p{2cm}}
\lsptoprule
Suffix /-jE/ & Suffix /-wA/ & No suffix\\[1ex]
\midrule

CV &  CVV & CVCV\textsuperscript{1}\\
 CVCV\textsuperscript{2} & &\\

 \lspbottomrule
\end{Itabular}
\end{table} 

Table \ref{tab:GRM-perf-suff} shows that, in a perfective intransitive
construction, a CV stem must
be suffixed with {\sls -jE} and  a CVV verb with {\sls -wA}. The examples in
(\ref{ex:GRM-jE-wA}) are negative in order to prevent the assertive
suffix from appearing (see Section \ref{sec:GRM-focus} on why \isi{negation} and the
assertive suffix cannot co-occur).


\ea\label{ex:GRM-jE-wA}


\ea{\it CV}\\
po   $>$  àfíá wá pójē  {\rm `Afia didn't divide'}\\
pɔ  $>$ àfíá wá   pɔ́jɛ̄   {\rm `Afia didn't  plant'}\\
pu  $>$ àfíá wá  pújē   {\rm `Afia didn't  cover'}\\
pʊ  $>$ àfíá wá  pʊ́jɛ̄   {\rm `Afia didn't  spit'}\\
kpe  $>$ àfíá wá  kpéjē   {\rm `Afia didn't  crack and
remove'}\\
kpa  $>$ àfíá wá  kpájɛ̄   {\rm `Afia didn't  take'  }

\ex{\it CVV}\\
tuu $>$ àfíá wá  tūūwō   {\rm `Afia didn't  go down'}\\
tie $>$  àfíá wá   tīēwō {\rm `Afia didn't chew'}\\
sii  $>$  àfíá wá  sīīwō   {\rm  `Afia didn't  raise'}\\
jʊʊ   $>$  àfíá wá  jʊ̄ʊ̄wā  {\rm  `Afia didn't  marry'}\\
tɪɛ $>$  àfíá wá tɪ̄ɛ̄wā  {\rm  `Afia didn't  give'}\\
wɪɪ $>$  àfíá wá  wɪ̄ɪ̄wā  {\rm  `Afia is not  ill'}
 

\z 
 \z

The surface form of the perfective suffix which attaches to CV stems  is 
predicted by the {\sc atr}-harmony rule of Section
\ref{sec:vowel-harmony}. Notice that  {\sc ro}-harmony does not operate
in that domain. 

\begin{Rule}\label{PHO-rule-perf-wa}{Prediction  for perfective intransitive 
-/wA/ suffix}\\
If the vowel of a CVV stem is
{\sc +atr},
the vowel of the suffix is {\sc +ro}, and if the vowel of a CVV stem is {\sc
 --atr}, the vowel of the suffix is {\sc -ro}.\\
-/wA/ $>$  $\alpha${\sc ro}$_{suffix}$  /  $\alpha${\sc atr}$_{stem}$   
\end{Rule}

The CVV stems display  harmony between the stem
vowel(s) and the suffix vowel which is easily captured by a variable feature
alpha notation, as shown in Rule (\ref{PHO-rule-perf-wa}), which  assumes that 
the segment [{\sls o}] is the
[{\sc +ro, +atr}]-counterpart of [{\sls a}]. 




Predicting  which of 
set CVCV\textsuperscript{1} or set CVCV\textsuperscript{2} in Table 
\ref{tab:GRM-perf-suff}  a
stem falls  has proven unsuccessful. Provisionally,  I suggest that a CVCV
stem must be stored with such an information. One piece of evidence
supporting this claim comes from
the minimal pair {\sls tèlè} `reach' and  {\sls télé} `lean against':  the
former displays CVCV\textsuperscript{2} (i.e. tele-jE),  whereas the latter 
displays CVCV\textsuperscript{1}
(i.e. tele-\O).  The data shows that a  CVCV stem with round vowels is less 
likely to
behave like a CVCV\textsuperscript{1} stem, yet {\sls púmó} `hatch' is a 
counter-example, i.e.
{\sls a zal wa puməje} `the fowl didn't hatch'. The CVCCV, CVVCV, and CVCVCV 
stems
have  not been investigated, but {\sls kaalɪ} `go', a common  CVVCV verb, takes
the
/-jE/ suffix.  


\paragraph{Imperfective intransitive construction}
\label{sec:GRM-verb-imperf-intran}

The imperfective  conveys the unfolding of an event, and it is often used to describe an event taking place at the moment of speech. In addition, the behavior of the egressive marker {\sls ka} (Section \ref{sec:GRM-EVC-egr-ingr}) suggest that the imperfective may be interpreted as a progressive event. As in the perfective intransitive, the assertive suffix may be found attached to the verb stem. 

\ea\label{ex:GRM-assert-suff}
{\rm [[{verb stem}]-[{\sc +hi,+ro}]]$_\textrm{verb in focus}$}
\z

Again, the constraints licensing the combination of the verb stem and the vowel features  shown in (\ref{ex:GRM-assert-suff})   are (i) none of the other constituents in the clause are in \isi{focus}, (ii) the clause does not include a \isi{negation} element, and (iii) the clause is intransitive, that is, there is no grammatical object. 

%[to do: note p. 199, + non-pronominal subject)

% , as opposed to an
% event perceived as bounded (i.e. perfective) or a hypothetical event (i.e.
% imperative)



\ea\label{ex:GRM-pos-neg-take}
\ea\label{ex:GRM-ipfv-out-pos}{\rm Positive}\\
 ʊ̀ kàá kpá  {\rm `She will take'}\\
   ʊ̀ʊ̀ kpáʊ̄ {\rm `She  is taking/takes.'}

\ex\label{ex:GRM-ipfv-out-neg}{\rm Negative}\\
 ʊ̀ wàá kpā  {\rm  `She will not take'}\\
   ʊ̀ʊ̀   wàà   kpá {\rm `She  is not taking/does not take.'}

   
   \ex\label{ex:GRM-ipfv-out-nfoc}
 \textasteriskcentered kalaa kpaʊ {\rm Kala is taking/takes.'}\\
   \ex\label{ex:GRM-ipfv-out-stpro}
 \textasteriskcentered waa kpaʊ {\rm `{\sc she} is taking/takes.'}\\
    \ex\label{ex:GRM-ipfv-out-obj}
  \textasteriskcentered ʊ kpaʊ a bɪɪ  {\rm `She  is taking/takes the 
stone.'}\\
 \ex\label{ex:GRM-ipfv-out-neg1}
   \textasteriskcentered   ʊʊ   waa   kpaʊ {\rm `She  is not taking/does 
not take.'}

\z 
 \z

In (\ref{ex:GRM-pos-neg-take}), the forms of the verb in the
intransitive imperfective take the assertive suffix to signal that the verbal
constituent is in \isi{focus}, as opposed to the nominal argument. The 
assertive suffix cannot appear
when the subject is in \isi{focus} (\ref{ex:GRM-ipfv-out-nfoc}) or when the strong
\isi{pronoun} is used as subject (\ref{ex:GRM-ipfv-out-stpro}), when a grammatical
object follows the verb  (\ref{ex:GRM-ipfv-out-obj}), or when the \isi{negation}
preverb {\sls waa} is present  (\ref{ex:GRM-ipfv-out-neg}).



\paragraph{Intransitive vs. transitive}
\label{sec:GRM-trans-intran}


Many verbs can occur in either  intransitive or transitive clauses. The subject 
of the intransitve  in (\ref{ex:GRM-clause-core-intrans}) and 
(\ref{ex:vp26.14.}) correspond to the subject of the transitive  in  
(\ref{ex:GRM-clause-core-trans}) and (\ref{ex:vp26.15.}), and the same verb is 
found with and without an object.


% \begin{multicols}{2}
\ea\label{ex:GRM-clause-core}


 \ea\label{ex:GRM-clause-core-intrans}
\gll kàlá díjōō.\\
 Kala eat.{\sc pfv.foc}\\
\glt  `Kala {\sc ate}.' 
\ex\label{ex:GRM-clause-core-trans}
\gll kàlá dí sɪ̀ɪ̀máá rā.\\
Kala eat.{\sc pfv}  food {\sc foc}\\
\glt  `Kala ate {\sc food}.' 



\ex\label{ex:vp26.14.}
\gll ʊ̀ʊ̀ búólùū.\\
    {\psg} sing.{\ipfv.\foc}\\
\glt  `He is {\sc singing}.' 

\ex\label{ex:vp26.15.}
\gll  ʊ̀ʊ̀ búólù būōl lō.\\
    {\psg}  sing.{\ipfv} song {\foc}\\
\glt  `He is singing a {\sc song}.' 

\z
 \z
%  \end{multicols}


It is possible to promote a prototypical theme argument to the subject position. However,  informants have difficulty with some nominals in the subject position of intransitive clauses.   The topic needs further investigation, although it is certainly related to a semantic anomaly.  The data in (\ref{ex:GRM-intran-theme-subj}), where the  prototypical {\sc o}(bject) is in {\sc a}-position, illustrates the problem. In order to concentrate on the activities of  `goat beating' and `tree climbing',  and turn the two clauses (\ref{ex:GRM-int-th-su-out-1}) and (\ref{ex:GRM-int-th-su-out-2}) into acceptable utterances,  the optimal solution is to use the impersonal \isi{pronoun} {\sls ba} in subject position  (see impersonal \isi{pronoun} in Section \ref{sec:GRM-impers-pro}).



\ea\label{ex:GRM-intran-theme-subj}

\ea
à bʊ̀ɔ̀ káá hírèū  {\rm `the hole is being dug'}
\ex\label{ex:GRM-int-th-su-out-1}
\textasteriskcentered a bʊ̃ʊ̃ŋ   kaa maŋãʊ̃  {\rm  `the goat is being beaten' }
$\rightarrow$ {\sls 
bàà máŋà à bʊ̃́ʊ̃́ŋ ná}
\ex\label{ex:GRM-int-th-su-out-2}
\textasteriskcentered a daa kaa zɪnãʊ̃  {\rm   `the tree is being climbed'}   
$\rightarrow$ {\sls 
bàà zɪ́ná à dáá rá}


\z 
 \z

%[to do: include image spectrogram]

Given that  the inflectional system of the verb is rather poor, and that the 
perfective
and assertive suffixes occur only in intransitive clauses,  how does one
encode a basic contrast like the one between a transitive perfective and
transitive imperfective? The paired examples in (\ref{ex:tra-pfv}) and
(\ref{ex:tra-ipfv})  illustrate 
 relevant contrasts.\nolinebreak 
 
  %[TO DO change tʃige sentence] 


% \begin{multicols}{2}
\ea\label{ex:tra-pfv}{\rm Transitive perfective}\\

  \ea\label{ex:tra-pfv-eat}
ǹ̩ dí kʊ̄ʊ̄ rā\\
{\rm  `I ate {\sc t. z.}.'} 
 \ex\label{ex:tra-pfv-plant}
ǹ̩ pɔ́ dāā rā\\
{\rm `I planted a {\sc tree}.'}
 \ex\label{ex:tra-pfv-cover}
ǹ̩ tʃígé vìì rē\\
{\rm `I turned a {\sc  pot}.'} 
 \ex\label{ex:tra-pfv-tie}
ǹ̩ lómó bʊ̃́ʊ̃́ŋ ná\\
{\rm `I tied a {\sc goat}.'}
 \ex\label{ex:tra-pfv-carry}
m̩̀ mɔ́ná díŋ né\\
{\rm `I carried {\sc fire}.'} 

\z 
 \z

\ea\label{ex:tra-ipfv}{\rm Transitive imperfective}\\

 \ea\label{ex:tra-ipfv-eat}
ǹ̩ǹ̩ dí kʊ́ʊ́ rá\\
{\rm `I am eating {\sc t.z.}.'}
 \ex\label{ex:tra-ipfv-plant}
m̩̀m̩̀ pɔ́ dáá rá\\
{\rm `I am planting a {\sc  tree}.'} 
 \ex\label{ex:tra-ipfv-cover}
ǹ̩ǹ̩ tʃígè vìì rē\\
{\rm `I am turning  a {\sc pot}.'} 

 \ex\label{ex:tra-ipfv-tie}
ǹ̩ǹ̩ lómò bʊ̃̄ʊ̃̄ŋ nā\\
{\rm `I am tying  a {\sc goat}.'} 
 \ex\label{ex:tra-ipfv-carry}
m̩̀m̩̀ mɔ́nà dīŋ nē\\
{\rm `I am carrying  {\sc fire}.'}

\z 
 \z
 
% \end{multicols}


Each pair in the verbal frames of  (\ref{ex:tra-pfv}) and (\ref{ex:tra-ipfv}) presents fairly regular patterns:  the high tone {\it versus} the falling tone on the CVCV verbs, the systematic change of the tonal melodies on the grammatical objects in the two CV-verb cases, and the \is{lengthening}length of the weak  \isi{pronoun} in the imperfective. The data suggest that it is the tonal melody, and not exclusively the one associated with the verb, which supports aspectual function in this comparison. When the verb is followed by an argument, both perfective and the imperfective are expressed with the base form of the verb.  However,  the tonal melody alone  can determine whether an utterance is to be understood as a bounded event which occurred in the past or an unbounded event unfolding at the moment of speech.



Tonal melody is crucial in the following examples as well. The examples in (\ref{GRM-pfv-inter}) are three polar questions (see Section \ref{sec:GRM-interr-polar}), one in the perfective and two in the imperfective. The two first have the same segmental content, and the last contains the egressive preverb {\sls kaa} with a rising tone indicating the future tense.  In order to signal a polar question, each has  an extra-low tone and is slightly \is{lengthening}lengthened at the end of the utterance. 

\ea\label{GRM-pfv-inter}


\ea\label{GRM-pfv-inter-pfv}
\glll {\Ttwo} {\Tnine \Tnine \Tnine} {\Tthree} {\Tthree \Tthree} {\Tone\Tone}\\
 ɪ   teŋesi  a  namɪ̃ã  raa?\\
    {\sc 2sg} {cut.{\sc pv}} {\sc art} {meat} {\sc foc}\\
\glt `Did you cut the meat (into pieces)?'\\



\ex\label{GRM-pfv-inter-impf}

\glll {\Ttwo} {\Tnine \Tnine \Tfour} {\Tthree} {\Tthree \Tthree} {\Tone\Tone}\\
ɪ   teŋesi  a  namɪ̃ã  raa?\\
    {\sc 2sg} {cut.{\sc pv}} {\sc art} {meat} {\sc foc}\\
\glt `Are you cutting the meat (into pieces)?'\\


\ex\label{GRM-pfv-inter-impf-fut}


\glll {\Ttwo} {\Tfive \Teight} {\Tseven \Tseven \Tthree} {\Tthree} {\Tthree \Tthree} {\Tone \Tone}\\
ɪ  kaa teŋesi  a  namɪ̃ã  raa?\\
    {\sc 2sg} {\sc ipfv.fut} {cut.{\sc pv}} {\sc art} {meat} {\sc foc}\\
\glt  `Will you (be) cut(ting) the meat (into pieces)?'\\
\z 
 \z

The only distinction perceived between (\ref{GRM-pfv-inter-pfv})  and (\ref{GRM-pfv-inter-impf}) is a pitch difference near the third syllable of the verb. The tonal melody associated with the verb in (\ref{GRM-pfv-inter-impf-fut}) is the same as the one in (\ref{GRM-pfv-inter-impf}).


\paragraph{Ex-situ subject imperfective particle}
\label{sec:GRM-ipfv-part}

One topic-marking strategy is to prepose a non-subject constituent to the beginning of the clause.  In  (\ref{ex:GRM-foc-top}),  the \isi{focus} particle may or may not appear after the non-subjectival topic. Notice that one effect of this topic-marking strategy is that the particle {\sls dɪ} appears between the subject and the verb when the non-subject constituent is preposed and when the clause is used to describe what is happening at the moment of speech. \nolinebreak

% \begin{multicols}{2}
\ea\label{ex:GRM-foc-top}
 \ea\label{ex:GRM-foc-top-chew-prog-1}{\rm Imperfective}\\
\gll  sɪ́gá (rá)  ʊ̀ dɪ̀  tíē.\\
 bean  ({\foc}) {3.\sg} {\ipfv} chew\\
\glt `It is {\sc beans} he is chewing.'\\


 \ex\label{ex:GRM-foc-top-chew-perf-1}{\rm Perfective}\\
\gll  sɪ́gá (rá) ʊ̀   tìè.\\
 bean  ({\foc}) {3.\sg}  chew\\
\glt `It is {\sc beans} he chewed.'\\


% \columnbreak


 \ex\label{ex:GRM-foc-top-go-prog-2}{\rm Imperfective}\\
\gll   wàà (rá) ʊ̀ dɪ̀  káálɪ̀.\\
Wa ({\foc}) {3.\sg} {\ipfv} go\\
\glt `It is to {\sc wa} that he is going.'\\


 \ex\label{ex:GRM-foc-top-go-perf-2}{\rm Perfective}\\
\gll   wàà (rá)  ʊ̀ kààlɪ̀.\\
Wa   ({\foc}) {3.\sg}  go\\
\glt `It is to {\sc wa} that he went.'\\
\z 
 \z
% \end{multicols}

The position of {\sls dɪ̀} in  (\ref{ex:GRM-foc-top-chew-prog-1}) and (\ref{ex:GRM-foc-top-go-prog-2}), that is between the subject and the verb, is generally occupied by linguistic items called  {\it preverbs},  to which the discussion turns in Section \ref{sec:GRM-precerv}.  Provisionally, the particle {\sls dɪ̀} may be treated as a preverb constrained to occur with  a preposed  non-subject constituent and an imperfective aspect.\footnote{I do not treat topicalization in this work, although the left-dislocation strategy in (\ref{ex:GRM-foc-top}) is the only one I know to exist.}


\paragraph{Subjunctive}
\label{sec:GRM-subjunctive}

In Section \ref{sec:GRM-desiderative-mood} the preverbal {\sls ŋma}  is said to  convey a 
desiderative mood,  corresponding to the English modal expression `want to',  in a construction [NP 
{\sls ŋma} [NP VP]].  The embedded clause is said to be in the subjunctive mood, which is singled 
out by its high tone on the subject \isi{pronoun} and  the non-actuality and potentiality of the event.  In the examples (\ref{ex:GRM-subj-market}) and (\ref{ex:GRM-subj-mother}) a  
\isi{subjunctive} is interpreted because it involves clauses expressing a future hypothetical time and 
realization. In all these cases, the clauses of which the high tone \isi{pronoun} is the subject seem to 
depend on and complement a more central event.

\ea
\ea\label{ex:GRM-subj-market}
\gll   ʊ̀ káálɪ̀ ʊ́ ká dí mɔ̀tìgú jáwà.\\
 {3.\sg} go  {3.\sg} {\sc ipfv} eat M. market \\
\glt `She is going to trade at the Motigu market.' ({\it lit.} eat-market,  `trade')\\

 
\ex\label{ex:GRM-subj-mother}
\gll zʊ̀ʊ̀ m̀m̀ mã̀ã̀ pé, ʊ́ kpá ǹǹ kʊ̀ʊ̀ fàlá tɪ́ɛ́ɪ́, ɪ́ kíínì.\\
 enter   {1.\sg} mother end  {3.\sg} take    {1.\sg\poss} t.z. bowl give.{\sc 2sg} {\sc 
2sg} clean.bowl\\
\glt `Go to my mother, she will give you my t.z. bowl so you can finish it.'\\

 \z
 \z
 
 In  (\ref{ex:GRM-subj-market}), according to the speaker, the trading activity is the intention of the woman and it will take place in all likelihood, and in  (\ref{ex:GRM-subj-mother}),  the speaker tells about two situations that the addressee  will most likely experience. 


% ŋ ŋma ŋ kaalɪ dusee tʃɪa
% ŋ ŋma ŋ tʃɪ  kaalɪ dusee
% ŋ tʃɪ kaalɪ dusee

\subsection{Preverb particles}
\label{sec:GRM-precerv}

\isi{Preverb} particles  encode various event-related meanings. They are part of the verbal domain  called the expanded verbal group (EVG), discussed in (\ref{sec:GRM-verbals}) and schematized in  (\ref{ex:verb-VP}). This domain  follows the subject and precedes the main verb(s) and is generally accessible  only to a limited set of linguistic items. These grammatical morphemes are not verbs, in the sense that they do not contribute to SVCs as verbs do,  but as `auxiliaries'. Still,  some of the preverbs may historically derive from verbs, and  some others may synchronically function as verbs.  Examples of the latter are the egressive particle {\sls ka} and ingressive particle {\sls wa},  which are discussed in Section \ref{sec:GRM-EVC-egr-ingr}. Nevertheless, given the data available,  it would not be incorrect to analyse some of the preverbs  as additional SVC verbs.  A preverb differs from a verb in that it exposes functional categories,  cannot inflect for the perfective or assertive suffix,  and never takes  a complement, such as a grammatical object, or cannot be modified by  an adjunct.  But again,  a first verb in a SVC and a preverb are categories which can be hard to distinguish. Structurally and functionally, many of them may be analysed as grammaticalized verbs in series. These characteristics are not special to Chakali; similar, but not identical, behavior are described for \ili{Ga}̃ and \ili{Gurene} \citep{Daku07b, Daku08}.


\subsubsection{Egressive and ingressive particles}
\label{sec:GRM-EVC-egr-ingr}


The \isi{egressive} particle {\sls ka(a)} (glossed {\sc egr})   `movement away from the deictic centre'  and   the \isi{ingressive} particle {\sls wa(a)} (glossed {\sc ingr})  `movement towards the deictic centre' are  assumed to derive from the  verbs {\sls kaalɪ} `go' and  {\sls waa} `come'.\footnote{A discussion on some aspects of grammaticalization of  `come' and `go' can be read in  \citet{Bour92}. In the literature, egressive  is also known as  {\it itive} (i.e. away from the speakers,  `thither')  and  ingressive  is  known as {\it ventive} (i.e. towards the speakers,   `hither'). }  Table \ref{tab:deict-pre-verb} shows that  {\sls kaalɪ} `go' and {\sls waa} `come',  like other verbs, change forms (and are acceptable) in these paradigms,  but {\sls ka(a)}  is  not.


\begin{table}
\centering
\caption{Deictic verbs and preverbs \label{tab:deict-pre-verb}}

\begin{Itabular}{lllll}
\lsptoprule
Verb & $\sigma$  & Aspect & Positive & Negative\\[1ex] \midrule


{\sls waa} `come' & CVV   & {\sc pfv}   &  ʊ̀ wááwáʊ́   & ʊ̀ wà wááwá\\
    &&& `she came' & `she didn't come'\\

    &   & {\sc ipfv}   &  ʊ̀ʊ̀ wááʊ̄  & ʊ̀ wà wáá\\
    &&& `she is coming' & `she is not
coming'\\[1ex] \midrule




{\sls kaalɪ} `go' & CVVCV   & {\sc pfv}   &  ʊ̀ káálɪ́jʊ́   & ʊ̀ wà 
káálɪ́jɛ́\\
    &&& `she went' & `she didn't go'\\

    &   & {\sc ipfv}   &  ʊ̀ʊ̀ káálʊ̄ʊ̄  & ʊ̀ wà káálɪ́\\
    &&& `she is going' & `she is not
going'\\[1ex] \midrule


{\sls ka(a)}  & CV   & {\sc pfv}   &  *ʊ kaʊ   & *ʊ wa kajɛ\\
  

    &   & {\sc ipfv}   &  *ʊ kaʊ  & *ʊ wa ka\\
  
\lspbottomrule


\end{Itabular}   
\end{table}


When the verbs {\sls kaalɪ} `go' and  {\sls waa} `come'
occur in a SVC,  they surface as {\sls ka} and {\sls wa} respectively. In
(\ref{GRM-prev}),  both {\sls ka} and {\sls wa} take part in  a two-verb SVC in which they are
first in the sequence.


\ea\label{GRM-prev}

\ea\label{GRM-prev-SVC-ka}
\glll gbɪ̃̀ã́     bààŋ   té   kà     sáŋá  à   píé  {(...)}\\
monkey  quickly   early   go  sit {\sc art} yam.mound.{\sc pl}   {}\\
{} [[{\it pv} {\it pv}]$_{EVG}$  {\it v} {\it v}]$_{VP}$ {} {}
{(...)}\\
\glt `Monkey quickly went and sat on the (eighth) yam mounds (...)'  [LB 012]

\ex\label{GRM-prev-SVC-wa}
\glll ŋmɛ́ŋtɛ́l   làà nʊ̃̀ã̀  nɪ́ ká  ŋmá dɪ́ ʊ́ʊ́  wá ɲʊ̃̀ã̀ nɪ́ɪ́.\\
spider collect mouth {\postp}  {\conn} say {\comp}  
{\sc 3sg}  come   drink water\\
 {} {} {} {} {}  {} {} {} {\it v} {\it v} {}\\
\glt  `(Monkey went to spider's farm to greet him.)  Spider accepted
(the
greetings) and (Spider) asked him (Monkey) to come and drink water.'  [LB 011]

 
\z 
 \z

Because they derive from deictic verbs (historically or synchronically),  the preverbs have the potential to indicate non-spatial  `event movement'  to or from a deictic centre. This phenomenon is not uncommon cross-linguistically. \citet[62]{Nico07} maintains  that when a movement verb becomes a tense marker, it may be reduced to a verbal affix and its meaning can develop ``into meaning relating temporal relations between events and reference times''. In Chakali, the  preverb {\sls ka(a)} contributes  temporal information to an expression. Consider in (\ref{exe:GRM-crack-remove-attach}) the distribution and contribution of  {\sls ka(a)} to  the clauses headed by the verbs {\sls kpe} `crack a shell and remove a seed from it' (henceforth `c\&r') and {\sls mara} `attach'.\footnote{In \ili{Gurene} (Western \ili{Oti-Volta}), it is the ingressive particle which has a similar role. The ingressive  is commonly used before the verb, and can, among other things,  express future tense \citep[see][59]{Daku07b}.}



\ea\label{exe:GRM-crack-remove-attach}
%\begin{multicols}{2}

\ea

 ʊ̀ kàá kpē  {\rm  `She will c\&r'}\\
   ʊ̀ʊ̀ kpéū  {\rm  `She  is c-\&r-ing/c-s\&r-s'}\\
   ʊ̀ kpéjòō   {\rm `She   c-\&r-ed'}\\
   kpé    {\rm  `C\&r!'}
\ex
 ʊ̀ kàá mārā   {\rm   `She will attach'}\\
   ʊ̀ʊ̀ máráʊ̄  {\rm  `She  is attaching/attaches'}\\
  ʊ̀ márɪ̀jʊ̄ {\rm `She   attached'}\\
   márá  {\rm  `Attach'}

%\end{multicols}
\z
\z

When the preverb particle {\sls kaa} is uttered with a rising pitch it situates the event in the future. The preverb particle {\sls kaa} can also be used to express that an event is ongoing at the moment of speech, which I call the present progressive.   However,  when it is used to describe what is happening now, {\sls kaa} can only appear when the subject is not a \isi{pronoun} and its tone melody differs from that of the future tense. These contrasts are given in (\ref{exe:GRM-kaa-attach}).

\ea\label{exe:GRM-kaa-attach}
 ʊ̀ kàá mārā   {\rm `She will attach'}\\
   ʊ̀ʊ̀ máráʊ̄   {\rm  `She  is attaching'}\\
wʊ̀sá kàá mārā   {\rm  `Wusa will attach'}\\
wʊ̀sá káá   máráʊ̄  {\rm  `Wusa is attaching'}\\
\textasteriskcentered  wʊ̀sá   máráʊ̄   {\rm `Wusa is
attaching'}
\z

The paradigm in  (\ref{exe:GRM-kaa-attach}) shows that when the preverb particle {\sls kaa} appears with a rising tonal melody it  expresses the future tense, but  in order to convey that a situation is ongoing at the time of speech (i.e. present progressive), the preverb particle {\sls kaa} has a high tone. Thus, it is the tonal melody on {\sls kaa}  which distinguishes between the future and the present progressive (both treated as imperfective),  plus the fact that pronouns cannot co-occur with the preverb particle {\sls kaa} in the present progressive.

\ea\label{x:GRM-tone-ipfv}
\ea\label{x:GRM-tone-ipfv-H}
\gll à bìé káá bīlīgī ʊ̀ʊ̀ nàál kɪ̀nkán nà.\\
{\sc art} child {\sc ipfv} touch {\sc poss.3sg} grand.father many  {\sc 
foc}\\
\glt `The child touches his grand-father.'

\ex\label{x:GRM-tone-ipfv-L}
\gll à bìè háŋ̀ kàà bīlīgī ʊ̀ʊ̀ nàál kɪ̀nkán nà.\\
{\sc art} child {\sc dem} {\sc ipfv} touch {\sc poss.3sg} grand.father  many   {\sc 
foc}\\
\glt `This child touches his grand-father.'

\z
\z


 In (\ref{x:GRM-tone-ipfv-L}) {\sls kaa}'s melody is shown to be affected by   the pitch  of   the  preceding  noun {\sls bie} (LH) `child' and the demonstrative {\sls haŋ} (HL) `this'.  Although little evidence is available, the preverb {\sls wa} may also be used to
express a sort of hypothetical  mood.  In  (\ref{ex:GRM-prev-wa-hypo}), the
preverb {\sls wa} should be seen as contributing a supposition, or a 
hypothetical
circumstance where
someone would be found calling the number 8. 
%[to do: check both wa and waa]

\ea\label{ex:GRM-prev-wa-hypo}

\gll ŋmɛ́ŋtɛ́l   ŋmā dɪ̄, kɔ̀sánã́ɔ̃̀,   tɔ́ʊ́tɪ̄ɪ̄nā  ŋmá dɪ́, námùŋ   wá jɪ̀rà ŋmɛ́ŋtɛ́l sɔ́ŋ, bá  kpáɣʊ́ʊ̄ wàà bá kpʊ́.\\
spider  say   {\comp}   buffalo land.owner say {\comp}  anyone   {\ingr}   call eight  name {\sc 3pl.hum+} catch.{\sc 3sg} {\foc} {\sc 3pl.hum+}  kill\\
\glt `Spider told Buffalo that landowner said anyone who calls the number 8 should be brought to him to be killed.' [LB 009]
\z


Finally, the example in (\ref{ex:GRM-verb-ta})  intends to show that some elders of Ducie and Gurumbele use  {\sls ta}  instead of {\sls ka(a)},  as a \isi{variant} of the preverb.\footnote{I gathered that  (i)   {\sls ta} is not a different preverb (\ili{Gurene} is said to have  a preverb {\sls ta}  signifying intentional action  (M. E. K. Dakubu, p.\ c.)),  and  (ii)   {\sls ta} can be heard in  Ducie and Gurumbele from people of the oldest generation, but somebody suggested to me that {\sls ta} is the common form in Motigu (Mba Zien, p.\ c.).  The distinction is  in need of further research. } 
 

\begin{exe}
   \ex\label{ex:GRM-verb-ta}{\rm Priest talking to the shrine, holding a kola
nut above it}
\gll  má láá kàpʊ́sɪ́ɛ̀ háŋ̀ ká jà mɔ́sɛ́ tɪ̀ɛ̀ wɪ́ɪ́ tɪ̀ŋ bà 
tà/kàà búúrè\\
{\sc 2pl} take kola.nut {\sc dem} {\sc conn} {\sc 1pl} plead give matter {\sc
art} {\sc 3pl.}b {\sc  egr} want\\
\glt   `Take this kola nut, we implore  you to give them what they desire.'
\z

Unfortunately, since the relation between tense, aspect, and tonal melody is not well-understood at this stage of research, the  egressive {\sls ka}   and the ingressive  {\sls wa} are  broadly glossed as {\egr} and {\ingr} respectively, but can also be associated with composite glosses such as {\ipfv .\fut} or  {\ipfv .\pres}  in cases where a distinction is clear.


\subsubsection{Negation preverb}
\label{sec:GRM-verb-neg}

%check negative concord with nobody, no one, all, nothing

%  Their
% lengths may vary depending on the speech rate, but  they are always long
% in 

There are three different particles of \isi{negation} in the language:  the forms {\sls lɛɪ} and {\sls tɪ}   were discussed in Sections \ref{sec:GRM-imper-clause} and  \ref{sec:GRM-foc-neg}  respectively.  The negative preverb particle {\sls wa(a)} precedes the verb and is used in the verbal group (in non-imperative mood). The same form is found in both  main and dependent clauses. 


\ea\label{ex:GRM-neg-pres-fut}

\ea
\gll ʊ̀  wàá pɛ̀.\\
   {\sc 3sg}  {\neg} add\\
\glt  `She will not add.'

 \ex 
\gll  ʊ̀ʊ̀ wàà pɛ́.\\
  {\sc 3sg} {\neg} add\\
\glt  `She is not adding.'


 \ex 
\gll  ʊ̀ wà pɛ́jɛ̄.\\
  {\sc 3sg} {\neg} add\\
\glt  `She didn't  add.'

\z 
 \z

 The examples in (\ref{ex:GRM-neg-pres-fut}) show that a tonal quality on the \isi{negation} particle and following verb  distinguishes between the present progressive and  the future,  as the preverb {\sls kaa} does (see example \ref{exe:GRM-kaa-attach}). The \is{lengthening}length of the \isi{negation} particle can also function as a cue.

\ea\label{ex:neg-quant-any}
\ea\label{ex:neg-quant-any-1}
\gll námùŋ wà ná-ŋ̀.\\
 {\clf}.all {\neg} see-{1.\sg}\\
\glt  `Nobody saw me.' ({\it lit.} everyone not see me) 

\ex\label{ex:neg-quant-any-2}
\gll  ǹ wà ná námùŋ.\\
  {1.\sg}  {\neg}   see  {\clf}.all\\
\glt  `I did not see anyone.' ({\it lit.} I not see everyone) 

\z 
 \z

 Example (\ref{ex:neg-quant-any}) shows that when the \isi{negation} particle {\sls wa(a)} and a \isi{quantifier} appear in the same clause the \isi{quantifier} is  in the positive. 

\ea\label{ex:GRM-neg-come}
 
  
\ea
\gll ʊ̀ wà wá dī.\\
{\sc 3sg} {\neg} come eat\\
\glt `She did not come to eat.'
\ex
\gll ʊ̀ wàá wà dí.\\
{\sc 3sg} {\neg} come eat\\
\glt `She will not come to eat.'

\z 
 \z
 
 The negative \isi{preverb}  always precedes the verb {\sls waa} `come'. Although 
length (CV or CVV) is  hard to differentiate in natural speech, the examples in 
(\ref{ex:GRM-neg-come}) suggest that the tonal melody and \is{lengthening}length  establish 
meaning differences.

Assertion and \isi{negation} seem to avoid one another and constrain the grammar  in the following way:  {\it If a clause is negated,  none of its constituents can be in \isi{focus}.} In Section \ref{sec:GRM-personal-pronouns},  it was shown that (i) \isi{negation} cannot co-occur with the strong pronouns, and (ii) \isi{negation} cannot co-occur with an argument of the predicate in \isi{focus}, i.e. with {\sls ra} or one of its \is{variant}variants having scope over the noun phrase. The third non-occurrence of \isi{negation} concerns  the \isi{assertive} form of the verb (Section \ref{sec:GRM-focus}).  Consider the forms of the verb {\sls mara} `attach' in the two paradigms in (\ref{ex:GRM-verb-neg-foc}).


   \ea\label{ex:GRM-verb-neg-foc}

   \ea\label{ex:GRM-verb-neg-foc-pos}{\rm Positive}\\

 ʊ̀ kàá mārā   {\rm  `She will attach'}\\
   ʊ̀ʊ̀ máráʊ̄  {\rm  `She  is attaching/attaches'}\\
  ʊ̀ márɪ̀jʊ̄ {\rm `She   attached'}\\

  \ex\label{ex:GRM-verb-neg-foc-neg}{\rm Negative}\\

 ʊ̀ wàá mārā  {\rm `She will not attach'}\\
   ʊ̀ʊ̀ wàà márá {\rm  `She  is  not attaching/does not attach'}\\
  ʊ̀ wà márɪ̀jɛ̄   {\rm  `She   did not attach'}\\


\z 
 \z

The paradigms in (\ref{ex:GRM-verb-neg-foc})  suggest that the \isi{negation} 
p\isi{article} 
and the assertive suffix are in complementary distribution. 

\subsubsection{Tense, aspect, and mood preverbs}
\label{sec:GRM-tam-preverbs}


\paragraph{fɪ}

The \isi{preverb} {\sls fɪ}   is identified with two different but interrelated 
meanings.  First, as (\ref{ex-preverb-fi-neut}) shows, the preverb {\sls fɪ}  
(glossed {\sc pst}) is a neutral past tense particle (i.e.  as opposed to the 
specific  {\sls dɪ} of  Section \ref{sec:GRM-preverb-three-int-tense}), and the 
event referred to in the past can no longer be in effect in the present.


\ea\label{ex-preverb-fi-neut}

\ea
\gll ʊ̀ jáá  ǹ̩ǹ̩ tʃítʃà rā.\\
  {\sc 3sg} {\ident} {\sc 3sg.poss}  teacher {\foc}\\
\glt  `He is my {\sc teacher}.' 

\ex
\gll   ʊ̀  fɪ̀ jáá  ǹ̩ǹ̩ tʃítʃà  rā.\\
  {\sc 3sg} {\pst} {\ident} {\sc 3sg.poss}  teacher {\foc}\\
\glt  `He was my {\sc teacher}.' 

\z 
 \z 

 Secondly, the preverb {\sls fɪ}   (glossed {\sc mod}) can have  deontic
meaning.  


\ea\label{ex-preverb-fi-deonc}

\ea\label{ex-preverb-fi-deonc-pos}
\gll ʊ̀ fɪ̀ɪ́ jàà  ǹ̩ǹ̩ tʃítʃà rā.\\
  {\sc 3sg}  {\mod}  {\ident} {\sc 3sg.poss}  teacher {\foc}\\
\glt  `He should have been my {\sc teacher}.' 

\ex
\gll ʊ̀ fɪ̀ wáá jàà  ǹ̩ǹ̩ tʃítʃà.\\
  {\sc 3sg} {\mod} {\neg} {\ident} {\sc 3sg.poss}  teacher\\
\glt   `He should not have been my {\sc teacher}.'  


\ex
 ʊ̀  fɪ̀ jáá  ǹ̩ǹ̩ tʃítʃà  rā  {\rm `He was my {\sc teacher}.'}
\ex
 ʊ̀  fɪ̀ wà jáá  ǹ̩ǹ̩ tʃítʃà {\rm `He was not my teacher.'}

\z 
 \z 


In (\ref{ex-preverb-fi-deonc}),  the  presence of the \isi{preverb} {\sls fɪ} still conveys  past tense, but in addition it expresses that the situation did not really occur, yet it was objectively supposed to occur or subjectively expected to occur or awaited. The \isi{lengthening} of the preverb {\sls fɪ} in the positive  is not accounted for, but I suspect it  signals the imperfective. Compare the first two sentences in (\ref{ex-preverb-fi-deonc}) with the last two  which convey the neutral past.  The positive sentence in (\ref{ex-preverb-fi-deonc-pos}) can receive  a translation along these lines:  In a desirable possible world, he was my teacher, but it is not what happened in the real world. 


\ea\label{ex-preverb-fi-pure-deonc}

\ea\label{ex:GRM-vp11.2}
\gll m̩̀m̩̀ mɪ̀bʊ̀à fɪ́  bɪ̀rgɪ̀.\\
 {\sc 1sg.poss} life {\mod}  delay\\
\glt  `May I live long!' 

\ex\label{ex:GRM-vp11.3}
\gll tɪ̀ɛ̀ m̩̀m̩̀ mɪ̀bʊ̀à bɪ́rgɪ̀.\\
   give {\sc 1sg.poss} life delay\\
\glt  `Let me live long!' 

\z 
 \z 


Finally, the preverb {\sls fɪ}  in (\ref{ex-preverb-fi-pure-deonc}) still conveys deontic \isi{modality}, where the speaker prays or asks permission for a situation. Notice, however,  that it cannot refer to a past event. The two sentences in (\ref{ex-preverb-fi-pure-deonc}) have a corresponding meaning. Example (\ref{ex:GRM-vp11.3}) is framed in an imperative clause (see \is{optative} {\it optative} in Section \ref{sec:GRM-imper-clause}). 


\paragraph{Preverb three-interval tense}
\label{sec:GRM-preverb-three-int-tense}

Chakali encodes  in  preverbs  a type of time categorization  known as \isi{three-interval tense}  \citep[366]{Fraw92}.   It is possible to express that an event occurred specifically yesterday, as opposed to earlier today and the day before yesterday, i.e. {\it hesternal tense} (glossed {\sc hest}), or specifically tomorrow, as opposed to later today and the day after tomorrow, i.e. {\it crastinal tense}  (glossed  {\sc cras}). The hesternal tense particle {\sls dɪ}/{\sls de} (glossed {\sc hest})  refers to the day preceding the speech time.  It has the temporal nominal  counterpart  {\sls dɪ̀àrɛ̀} `yesterday'.  

\ea\label{ex:vp2.11.a} 
\gll {(dɪ̀àrɛ̀ tɪ̀n)} ʊ̀ nɪ́ ʊ̀ tʃɛ̀ná dɪ́ wāāwā  {(dɪ̀àrɛ̀ tɪ̀n)}.\\
{(yesterday)} {\sc 3sg} {\sc conn} {\sc 3sg.poss} friend
{\sc hest}  come.{\sc pfv} {(yesterday)}\\
\glt  `He arrived with his friend yesterday.'
 \z



In (\ref{ex:vp2.11.a}),  the  phrase {\sls dɪare 
tɪn} `yesterday' is optional,  and  when it is used it must be expressed at the 
end or at the beginning of the clause.


\ea\label{ex:vp4.5} {\rm Will you work for the chief today or tomorrow?}\\
\gll  ǹ̩ tʃɪ́ kàá tʊ̀mà tɪ̄ɛ̄ʊ̄ rà, záàŋ,  ǹ̩ kàá hɪ̃̀ɛ̃̀sʊ̀ʊ̄.\\
 {\sc 1sg} {\sc cras}  go  work give.{\sc 3sg} {\sc foc},
today,   {\sc 1sg}  {\sc egr} rest.{\sc foc}\\
\glt  `I shall work for
him tomorrow, today,  I shall rest.' 
 \z

The crastinal tense \isi{preverb} {\sls tʃɪ} (glossed {\cras})  in  (\ref{ex:vp4.5}) functions as future particle,  but is limited to the day following the event time. In that sentence the event time referred to follows  the utterance time by one day.  The temporal nominal counterpart  of {\sls tʃɪ} is {\sls  tʃɪ̃̀ã́} `tomorrow'. As for the hesternal tense and the corresponding nominal,  the  nominal may or may not co-occur with the crastinal tense particle. 

The hesternal tense particle {\sls dɪ} is homophonous with the ({\it ex-situ
subject}) imperfective particle  {\sls dɪ} discussed in Section
\ref{sec:GRM-ipfv-part}.  In addition, the question arises as to whether the
crastinal tense  is inherently future, and if so, whether or not it can
co-occur with the future-encoding egressive preverb discussed in Section
\ref{sec:GRM-EVC-egr-ingr}. Consider their distribution and meaning in the
examples given in (\ref{ex:GRM-prev-dist}).


\ea\label{ex:GRM-prev-dist}

\ea\label{ex:GRM-prev-dist-chew-presprog}{\rm Imperfective}\\
\gll  sɪ́gá (rá)  ʊ̀ dɪ̀  tíē.\\
 bean  ({\foc}) {3.\sg} {\ipfv} chew\\
\glt `It is {\sc beans} he is chewing.'

 \ex\label{ex:GRM-prev-dist-chew-past}{\rm Perfective/Past}\\
\gll  sɪ́gá (rá) ʊ̀   tìè.\\
 bean  ({\foc}) {3.\sg}  chew\\
\glt `It is {\sc beans} he chewed.'


 \ex\label{ex:GRM-prev-dist-chew-past-hest}{\rm Hesternal past}\\
\gll  sɪ́gá (rá) ʊ̀ dɪ́ tìè.\\
 bean  ({\foc}) {3.\sg} {\hest}  chew\\
\glt `It is {\sc beans} he chewed yesterday.'


 \ex\label{ex:GRM-prev-dist-chew-past-pro}{\rm Hesternal past progressive}\\
\gll  sɪ́gá (ra) ʊ̀ dɪ́ɪ́ tīè.\\
 bean  ({\foc}) {3.\sg} {\hest}  chew\\
\glt `It is {\sc beans} he was chewing yesterday.'

 \ex\label{ex:GRM-prev-dist-chew-futprog}{\rm Future (progressive)}\\
\gll  sɪ́gá (rá) ʊ̀  kàá   tíē.\\
 bean  ({\foc}) {3.\sg} {\fut}  chew\\
\glt `It is {\sc beans} he will be chewing / will chew.'

 \ex\label{ex:GRM-foc-top-chew-crasfutprog}{\rm Crastinal future 
(progressive)}\\
\gll  sɪ́gá (rá) ʊ̀ tʃɪ́  kàá   tìè.\\
 bean  ({\foc}) {3.\sg} {\cras} {\fut}   chew\\
\glt `It is {\sc beans} he will be chewing / will chew tomorrow.'


\z 
 \z 
 
 
A specific tonal melody associated with  the sequence {\sls dɪ tie} can express 
either a present progressive, as in (\ref{ex:GRM-prev-dist-chew-presprog}),  or 
a hesternal past, as in  (\ref{ex:GRM-prev-dist-chew-past}). Lengthening\is{lengthening} the 
\isi{hesternal} past particle allows one to express the tense associated with the 
p\isi{article}, in addition to indicating  progressive 
(\ref{ex:GRM-prev-dist-chew-past-pro}). This strategy seems to correspond 
semantically  to the apparent syntactic anomaly *{\sls dɪ dɪ},  {\it lit.} {\sc 
 hest} {\sc ipfv}.  The example in (\ref{ex:GRM-foc-top-chew-crasfutprog}) 
shows that the \isi{crastinal} tense particle and the egressive particle signaling  
future  can co-occur.  Inserting the \isi{imperfective} particle {\sls  dɪ} between 
the \isi{egressive} particle and the verb in  (\ref{ex:GRM-prev-dist-chew-futprog}) 
and (\ref{ex:GRM-foc-top-chew-crasfutprog}) is  unacceptable. It is unclear 
whether these two examples must be interpreted as \isi{progressive} or not.  



\paragraph{te}
\label{sec:GRM-preverb-te}

Lacking a corresponding verb to capture its meaning, the verb {\sls te} is glossed with the English adverb `early'. Even though  it is attested as main verb,  {\sls te} can  function  as a preverb and it is indeed more common to find it in that function. 


 \ea\label{ex:GRM-prev-early}
 \ea \gll  ɪ̀ téjòō.\\
  {\sc 2sg} early.{\foc}\\
\glt  `You are early.'


\ex\label{GRM-prev-SVC-ka-1}
\glll gbɪ̃̀ã́     bààŋ   té   kà     sáŋá   à   
píé  {(...)}\\
monkey quickly   early   go  sit {\sc art}
yam.mound.{\pl}   {}\\
{} {\it pv} {\it pv}  {\it v} {\it v} {}  {} {}\\
\glt `Monkey quickly went and sat on the (eighth) yam mounds (...)'  [LB 012]
\z
\z

The main verb {\sls te} and the preverb {\sls te} are shown respectively in  (\ref{ex:GRM-prev-early}).  They contribute a relative time, one in which the event is carried out before the expected or usual time. 


\paragraph{zɪ}
\label{sec:GRM-preverb-after-then}

The \isi{preverb} {\sls zɪ} is marginal in the corpus.\footnote{There is a formally similar particle, {\sls ze}  (glossed {\sc exp}),  which is still not understood: (i) it occurs after the noun phrase, and  (ii) its meaning corresponds to `expected (by both the speaker and the addressee, or only by the speaker)'. It informs that the referent of the noun phrase was anticipated before the utterance time (or relative time) by the speaker and addressee (or only the speaker).  Consider  the following example:
\ea
\gll bà zé  wāāwāʊ̄.\\
{\sc 3pl.b} {\sc  exp} come.{\sc pfv}\\
\glt  `They (the expected people) have come.'
\z  
}

\ea
\ea\label{ex:GRM-prev-zi-1}  {\rm A father is giving a sequence of tasks to
his son}

 \glll tʊ̀mà  à  zɪ̃́ɛ̃́  mʊ̃́ã̀  ká  kà  tʊ̀mà  kùó   àká   zɪ́ kà  
tʊ̀mà à  gár\\
  {work} \textsc{art} {wall} {before}  \textsc{conn} go {work}
{farm} 
\textsc{conn} {after}  {go} {work} \textsc{art} {cattle.fence}\\
{} {} {} {} {}  {} {} {} {} {\it pv} {\it v} {\it v} {} {}\\ 
\glt  `First repair the wall, then go and farm, then repair the cattle fence.'


 \ex\label{ex:GRM-prev-zi-2}
  \glll kààlɪ̀ dɪ̀á ká zɪ́ kààlɪ̀ kùó.\\
go  house and then go farm\\
{} {} {}  {\it pv} {\it v} {}\\
 \glt `Go to the house and then go to the farm.'


\z 
 \z


There is no corresponding verb in the language.   It is used to express an 
order of events,  in such case words such as {\sls mʊ̃ã}  `before' and {\sls 
zɪ} `after' and the \isi{connective} {\sls ka/aka}  `and/then' are used, as 
(\ref{ex:GRM-prev-zi-1}) shows. However,  as (\ref{ex:GRM-prev-zi-2}) 
illustrates,  the preceding event may be presupposed, so  it is not necessarily 
uttered.







\paragraph{baaŋ}
\label{sec:GRM-preverb-baang}

 The preverb  {\sls baaŋ}  (glossed {\sc mod})  is primarily \isi{modal} and is  
usually translated into English `must', `immediately', `quickly'  or `just'. 


\ea\label{sec:GRM-prev-bg-must}
\ea\label{ex:GRM-7.17}
\gll  kùórù ŋmá dɪ́ ǹ̩ kàá bààŋ bɔ́ bʊ̃́ʊ̃́ná  fí rē.\\
 chief say {\comp} {\sc 1sg} {\fut} {\mod}   pay  goat.{\pl} ten {\sc foc}\\
\glt  `The chief says that I must pay him ten goats.' 

\ex\label{ex:GRM-14.3}
\gll  ɪ̀ɪ̀ kàá bààŋ jáʊ́ rā.\\
{\sc 2sg} {\fut} {\mod} do.{\sc 3sg} {\foc}\\
\glt  `You must do it.'

 \z 
 \z
 
First, the examples in (\ref{sec:GRM-prev-bg-must}) show that  the preverb  
{\sls baaŋ} conveys an \isi{obligation}.


\ea\label{sec:GRM-prev-bg-time}
\glll   {(...)} à kpá ʊ̀ʊ̀ néŋ à sàgà ʊ̀ʊ̀ nɪ̄ dɪ́ ʊ̀ bààŋ té 
bɛ̀rɛ̀gɪ̀ dʊ̃́ʊ̃̀\\
  {(...})  {\conn}  take {\sc 3sg.poss} arm {\conn} {be.on} {\sc
3sg}  {\postp} {\conn} {\sc 3sg} {\mod} {early} turn.into python\\
{} {} {} {} {} {} {} {} {} {} {} {\it pv} {\it pv} {\it v} {}\\

\glt  `(...) then put his hand on her  and quickly turned into
a python.' [PY 025]
\z
   
   Secondly, as illustrated in (\ref{sec:GRM-prev-bg-time}),  the preverb  
{\sls baaŋ} can express an  abrupt or
swift   manner. 

 \ea\label{ex:GRM-prev-bg-excerpt}
\ea\label{ex:FUS-mod}
\gll  ʊ̀ zɪ́má dɪ́ jà kàá ŋmá ʊ̀ʊ̀ wɪ́ɛ́ rá ʊ̀ʊ̀ bààŋ tʃùò dúò.\\
 {\sc 3sg}  know {\sc comp}  {\sc 1pl}  {\fut} talk   {\sc 3sg.poss} matter 
{\foc} {\sc 3sg} {\mod} lie sleep\\
\glt  `He knew that we would talk about him, so he quickly slept.'

\ex
\gll kàwàá bààŋ tàrɪ̀ kééééŋ ...\\
pumpkin just creep {\dxm}\\
\glt `A pumpkin just crept like that ...' 

\ex
\gll à kùò ní ʊ̀ bààŋ jírúú kéŋ néé à wà kʊ̀ʊ̀.\\
{\sc art} farm {\postp}  {\sc 3sg} {\mod} call.{\ipfv} {\sc dxm} {\foc} {\sc 
conn} {\sc ingr} tire\\
\glt `At the farm he kept calling (for someone) but got tired (gave up).'


\ex
\gll díŋ bààŋ jàà dìŋtʊ́l̀.\\
fire  just {\ident} flame\\
\glt `The fire suddenly became flame.'

 \z 
 \z

 Finally, the preverb  {\sls baaŋ} may act as a discourse particle used mainly to emphasize or intensify the action carried out, reminiscent of  the use of 
`just' 
in some English registers.  It is often translated in text as `immediately', 
`suddenly', `then',  or simply `just'. Examples are given in 
(\ref{ex:GRM-prev-bg-excerpt}).


\paragraph{bɪ}
\label{sec:GRM-preverb-iteration}

The examples in (\ref{ex:GRM-prev-bi}) show that  the preverb particle {\sls bɪ}
expresses \isi{iteration}, but also the single repetition of an event, and follows the \isi{negation} particle. 

% bɪ kuor ŋma
%  repeat
%  bɪ pɪlɪ
% start again
% start


\ea\label{ex:GRM-prev-bi}

\ea\label{ex:vp33.2.}
\gll ʊ̀ bɪ́ kʊ̀ɔ̀rɛ̀ sã̀ã̀ ʊ̀ʊ̀ dɪ̀à rá.\\
 {\sc 3sg}  {\itr} make build {\sc 3sg.poss} house {\foc}\\
\glt  `He rebuilt his hut.' 

\ex\label{ex:GRM-vp10.4}
\gll à bìtʃèlíí bɪ́ sīīú.\\
 {\sc art}  child.fall   {\itr} raise.{\foc}\\
\glt  `The fallen child gets up again.' 

\ex\label{ex:vp10.4.}
\gll ʊ̀ wà bɪ́ tùō.\\
    {3.\sg} {\neg} {\itr} be.at\\
\glt  `She is no longer here.' 

\z 
 \z 
 
 
Unlike other preverbs,  {\sls bɪ} may also appear within noun phrases to express frequency time. This is shown in (\ref{ex:GRM-vp19.2.}) (see Section \ref{sec:NUM-repet}).


\ea\label{ex:GRM-vp19.2.}
\gll  ǹ̩ jáà  káálɪ̀ ùù pé rè tʃɔ̀pɪ̀sɪ̀ bíí mùŋ.\\
{\sc 1sg} {\hab} go {\sc 3sg.poss} end {\foc}  day.break {\itr} all\\
\glt  `I do visit him every day.' 
\z 

 \paragraph{bra}
\label{sec:GRM-preverb-return}

The verb {\sls bra} ([{\sls bə̆̀rà}]) is  a motion verb which conveys a change of direction.

\ea\label{ex:GRM-verb-bra}
\ea
\gll brà à káálɪ̀.\\
return {\conn} go\\
\glt `Go back.' (Hearer coming towards speaker, speaker asks addressee to turn and go back.)

\ex
\gll brà àká tʃáʊ̀.\\
return {\conn} leave.{\sc 3sg}\\
\glt `Return and leave him.' (Speaker asks addressee to turn and go away from the person the addressee is with.)

\z 
 \z

The examples in (\ref{ex:GRM-verb-bra}) present the verb {\sls bra} in imperative clauses separated by the connectives {\sls a} and {\sls aka}. 

\ea\label{ex:vp33.1.}
\gll ʊ̀ brá tʊ̀mà à tʊ́má tɪ́ŋ kà wà wíré kéŋ̀.\\
 {\sc 3sg}  {again}  {work} {\sc art} {work}   {\sc art} {\egr} {\neg} well {\dxm}\\
\glt  `He redid the work that was badly done.'
\z


When {\sls bra} functions as a preverb, as in (\ref{ex:vp33.1.}),  it loosely keeps its motion sense and conveys in addition a sort of repetition. It differs from the morpheme {\sls bɪ} introduced in Section \ref{sec:GRM-preverb-iteration} since it does not mean that an action is done repeatedly.  Instead, the preverb {\sls bra} is associated with actions done `once more', `over again',  or `anew'.
%regloss ka

\paragraph{ja}
\label{sec:GRM-preverb-hab}

The \isi{preverb} {\sls ja(a)} (glossed {\sc hab})  indicates \isi{habitual} aspect. It 
expresses that the subject's referent is accustomed to, familiar with, or 
routinely do the action described by the predicate.


\ea\label{ex:GRM-prev-hab-do}
\gll tʃɔ̀pɪ̀sɪ̀ bɪ́-múŋ̀ ʊ̀ʊ̀ jáà jááʊ̄.\\
 day.break {\itr}-all {\sc 3sg} {\hab} do.{\sc 3sg}\\
\glt `He does it every day.'
\z 

 A variation in  \is{lengthening}length and intonation suggest  an (im)perfective aspectual 
distinction. In   (\ref{ex:GRM-prev-hab-do})  there is a  vowel sequence {\sls 
aa} pronounced with a falling intonation. Compare this with the examples in 
(\ref{ex:GRM-prev-hab}). 



% % 
% % \ea\label{ex:habitual}
% % \ea 
% % \gll  ǹ̩  já kààlɪ̀ kùó.\\
% % {\sc 1sg} {\sc hab} go farm\\
% % \glt `I do go to the farm.'
% % 
% % \ex 
% % \gll  ǹ̩  jáà káálɪ̀ kùó.\\
% % {\sc 1sg} {\sc hab} go farm\\
% % \glt `I have been going to the farm.'
% % 
% % \z 
% %  \z
% %  
 

\ea\label{ex:GRM-prev-hab}
\ea\label{ex:GRM-prev-hab-do-pfv}
\gll  kàlá já tùgòsì bísé ré.\\
 K.  {\hab}  beat.{\sc pl} child.{\sc pl} {\sc foc}\\
\glt `Kala beat children.' (He used to do it.)

\ex\label{ex:GRM-prev-hab-impv}
\gll  kàlá jáà túgósì bísé ré.\\
K.  {\hab}  beat.{\sc pl} child.{\sc pl} {\sc foc}\\
\glt `Kala beats children.' (He regularly does it.)

\z 
 \z

The aspectual distinction in (\ref{ex:GRM-prev-hab}) is reflected by the preverb's vocalic \is{lengthening}length and intonation, but also on the following verb's intonation.




\paragraph{ha}
\label{sec:GRM-preverb-yet}

The morpheme {\sls ha} (glossed {\sc mod}) is similar in meaning to the English morpheme `yet'  and is circumscribed to the expanded verbal group. The expression {\sls haalɪ}  (glossed {\sc conn}) has a similar meaning but is mainly used as a discourse \isi{connective}. It is not frequent and is ultimately of \ili{Arabic} origin, but like many other words, have been acquired via another language, in this case \ili{Hausa} \citep[157-158]{bald08}. An example is provided in (\ref{ex:yet-haali}).

\ea
\ea\label{ex:vp32.24}
\gll ʊ̀ʊ̀ háá díūū.\\
  {3.\sg}  {\mod} eat.\foc\\
\glt  `He is still eating.' 


\ex\label{ex:vp20.3.2.}
\gll ʊ̀ há wà díìjē.\\
 {3.\sg}  {\mod} {\neg} eat.{\pfv}\\
\glt  `He has not eaten yet.'


\ex\label{ex:vp21.2.1.}
\gll bà ɲíné ʊ̀ʊ̀ gɛ̀rɛ̀gá rá àká ʊ̀ʊ̀ háá wɪ̄ɪ̀.\\
 {\sc 3pl.hum+} look {\sc 3sg.poss} sickness {\foc} {\conn}  {\sc 3sg}
{\mod} ill\\
\glt  `He has been cared for to no avail; he is still ill.' 


\ex\label{ex:vp20.1.1.}
\gll ʊ̀ há  wà wāā báàŋ múŋ̀.\\
    {3.\sg} {\mod}  {\neg} come {\dem} {\quant}.all\\
\glt  `He does not come here (ever).' 

\ex\label{ex:vp20.3.1.}
\gll ʊ̀ há wà wááwá.\\
    {3.\sg} {\mod}   {\neg} come.{\pfv}\\
\glt  `He has not come yet.' 

\ex\label{ex:yet-haali}
 \gll m̀ búúré mòlèbíé bìrgì háálɪ̀ ǹ há wà nã́ã̀.\\
  {1.\sg} want money delay \textsc{conn}  {\sc 1sg}  {\sc mod}  come see.{\sc 3pl} \\
 \glt  `I struggled to get money for some time but still have not got any.' 
 
\z 
 \z
 
The morpheme {\sls ha}  is used when an event is or was anticipated and a speaker considers or considered probable the 
occurrence of the event. As  for the English `yet', it is frequently found in \isi{negative polarity}. In 
such cases  {\sls ha} indicates that the event is expected to happen and the negative 
marker {\sls wa} indicates that the event has not unfolded or happened at the referred time. In the 
cases where {\sls ha} is found in a positive polarity,  it  conveys a continuative aspect, that the event is happening at the time, similar to English `still',  as in (\ref{ex:vp32.24}) and (\ref{ex:vp21.2.1.}). 



\paragraph{tu and zɪn}
\label{sec:GRM-preverb-up-down} 

The verbs {\sls tuu} and {\sls zɪna} are motion
expressions making reference to two opposite paths. 


\ea\label{ex:GRM-verb-up-down}
\ea
\gll ǹ̩ zɪ́nà sàl lá ḿ̩ páá tʃùònò.\\
{\sc 1sg} go.up flat.roof {\foc} {\sc 1sg} take.{\pv} shea.nut.seed.{\pl}\\
\glt  `I go up on the roof to collect my shea nuts.'

\ex
\gll ǹ túú dɪ̀à rá.\\
{\sc 1sg} go.down house {\foc}\\
\glt I went down to the house.'
\z 
 \z

When they are used as main
predicate, as in example (\ref{ex:GRM-verb-up-down}),  they denote `go down' and
`go up' and  surface as {\sls tuu} and {\sls zɪna} respectively. 




\ea\label{ex:GRM-preverb-up-down}
\ea\label{ex:GRM-preverb-up}
\gll zɪ́ná tʃɔ́  à káálɪ̀.\\
   {go.up} run {\conn} go\\
\glt  `Go up,  run, and leave'  (*Run upwardly and go)

\ex\label{ex:GRM-preverb-down}
\gll tùù tʃɔ́  à káálɪ̀.\\
   {go.down} run {\conn} go\\
\glt  `Go down, run, and leave'  (*Run downwardly and go)

\z 
 \z
 
The verbal morphemes {\sls tuu} and {\sls zɪn} in 
(\ref{ex:GRM-preverb-up-down}) 
are not treated as preverbs, but first verbs in SVCs.  As explained at the beginning of  Section \ref{sec:GRM-precerv}, more criteria are required to be considered in order to categorize verbals  of that particular kind.


% % % % 6
% % % %  ́
% % % % The directional particles he (‘itive’, related to the homophonous verb 
% meaning
% % % % ‘go’ (departure from
% % % %  ́
% % % % deictic center or indexically determined location)) and va (‘ventive’, 
% related
% % % % to the homophonous verb meaning
% % % % ‘come’ (arrival at deictic center or indexically determined location)) 
% belong
% % % %to % the class of preverbs of \ili{Ewe}.
% % % % These are forms that mark functional categories such as aspect, modality, 
% and
% % % % voice on verbs. Preverbs differ
% % % % from verbs in that they do not head VPs, do not inflect for habitual 
% aspect,
% % % %and % do not take NP or PP
% % % % complements (cf. Ameka 1991, 2005a,b, Ansre 1966).
% % % 
% % % 
% % % % The particle  {\sls ja} is
% % % % polyfunctional:  when it precedes a main verb it  means  either `do'   to
% % % % emphasize the event or conveys an habitual reading, or as, in the present
% % % %case,
% % % % it links two noun phrases. The latter case is glossed in example
% % % %(\ref{ex:agrE})
% % % % and (\ref{ex:agrF}) as {\sc ident}. 
% % % 
% % % 
% % % % --Dakubu
% % % % I wonder whether what you call IPFV is an egressive particle? such a 
% p\isi{article}
% % % % derived from 'go' is quite common.  If it is incompletive / progressive 
% this
% % % % might have to do with the tone pattern



\subsection{Verbal suffixes}
\label{sec:GRM-verb-suffix}

In Section \ref{sec:GRM-verb-word}, two suffixes were introduced: the perfective intransitive suffix and the assertive suffix. It was shown that the perfective intransitive suffix surfaces either as {\sls -jE}, {\sls -wA} or {\sls  -\O} depending on  the verb stem.  The assertive suffix appears  in the imperfective and perfective  intransitive construction if  (i) none of the other constituents in the clause are in \isi{focus}, (ii) the clause does not include propositional \isi{negation}, and (iii) the clause is intransitive, that is, there is no grammatical object. Also,  as mentioned in Section \ref{sec:GRM-imper-clause},  the suffix {\sls -ɪ}/{\sls -i} appears in the negative imperative. In this section,  the incorporated \isi{object index}  ({\sc o}-clitic), the pluractional  suffix, and  other derivative suffixes whose functions are not fully understood are introduced.

\subsubsection{Incorporated object index}
\label{sec:GRM-morph-opro}


The \isi{object index}  is represented as being incorporated into the verb,  and together they form a phonological word (e.g.  {\sls wʊ̀sá tɪ́ɛ́ń nā} < {\sls wʊ̀sá tɪɛ-n̩ na}  `Wusa gave-{\sc 1sg} {\sc foc}').  For that reason the incorporated object index is referred to as the {\sc o}-clitic. Given the constraints governing the appearance of the perfective intransitive suffix and the \isi{assertive suffix}, it is obvious that the {\sc o}-clitic cannot coexist with any of them.  Table \ref{tab:object-clitic} shows that the {\sc atr}-harmony operates in the domain produced by the {\sc o}-clitic merging with a CV or CVV stem, but may or may not affect the \isi{plural} pronouns, as Tables \ref{tab:object-clitic}(b) and \ref{tab:object-clitic}(c) display.\footnote{The question mark following the third person \isi{plural} \isi{non-human} examples flags a grammatical but infelicitous example.}

\begin{table}[!htb]
\centering
\caption{Incorporated object index on  CV(V) stems\label{tab:object-clitic}}

\subfloat[tɪɛ `give']{
\begin{Itabular}{p{4cm}p{5cm}}
 wʊ̀sá tɪ́ɛ́-ń̩ nā & `Wusa gave {\sc me}'\\
wʊ̀sá tɪ́ɛ́-ɪ́ rā & `Wusa gave {\sc you}'\\
 wʊ̀sá  tɪ́ɛ́-ʊ́ rā &  `Wusa gave {\sc her}'\\
 wʊ̀sá tɪ́ɛ́-já rā &  `Wusa gave {\sc us}'\\
 wʊ̀sá tɪ́ɛ́-má rā & `Wusa gave {\sc you}'\\
 wʊ̀sá tɪ́ɛ́-á rā & `Wusa gave {\sc them}'\\
 wʊ̀sá tɪ́ɛ́-bá rā &  `Wusa gave {\sc them}'\\
\end{Itabular} 
}
\quad
\subfloat[tie `swindle']{
\begin{Itabular}{p{4cm}p{5cm}}
 wʊ̀sá tíé-ń̩ nē & `Wusa swindled {\sc me}'\\
 wʊ̀sá tíé-í rē & `Wusa swindled {\sc you}'\\
 wʊ̀sá tíé-ú rō &  `Wusa swindled  {\sc her}'\\
 wʊ̀sá tíé-já rā &  `Wusa swindled {\sc us}'\\
 wʊ̀sá tíé-má rā & `Wusa swindled {\sc you}'\\
 wʊ̀sá tíé-á rā & `Wusa swindled  {\sc them}'(?)\\
 wʊ̀sá tíé-bá rā &  `Wusa swindled  {\sc them}'\\
\end{Itabular} 
}
\quad
\subfloat[tie `swindle']{
\begin{Itabular}{p{4cm}p{5cm}}
 wʊ̀sá tíé-jé rē &  `Wusa swindled {\sc us}'\\
 wʊ̀sá tíé-mé rē & `Wusa swindled {\sc you}'\\
 wʊ̀sá tíé-é rē & `Wusa swindled  {\sc them}'(?)\\
 wʊ̀sá tíé-bé rē &  `Wusa swindled  {\sc them}'\\
\end{Itabular} 
}
\quad
\subfloat[po `divide']{
\begin{Itabular}{p{4cm}p{5cm}}
 wʊ̀sá pó-jé rē &  `Wusa divided {\sc us}'\\
 wʊ̀sá pó-mó rō & `Wusa divided {\sc you}'\\%pó-mó wō
 wʊ̀sá pó-á rā & `Wusa divided  {\sc them}'\\
 wʊ̀sá pó-bé rē &  `Wusa divided  {\sc them}'\\
\end{Itabular} 
}
\end{table}
The form of the \isi{focus} particle is determined by the preceding material (i.e. the phonological word  verb+{\sc o}-clitic) and the harmony rules introduced in Section \ref{sec:focus-forms}.  Table \ref{tab:object-clitic}(d) should be seen as displaying various renditions, i.e. with and without {\sc atr-}harmony or {\sc ro-}harmony.  I did perceive rounding throughout in conversations (i.e.  {\sls wʊ̀sá pómá rā} $>$ {\sls wʊ̀sá pómó wō} `Wusa divided you.{\sc pl}'), but I was unable to get a consultant  produce it in an elicitation session. 
 
 A CVCV stem differs from a CV or CVV stem by exhibiting vowel apocope and/or vowel coalescence.  Table \ref{tab:object-clitic-CVCV} provides paradigms for {\sls kpaga} `catch' and {\sls goro} `(go in) circle'. 

 \begin{table}[!htb]
\centering
\caption{Incorporated object index on  CVCV stems
\label{tab:object-clitic-CVCV}}

\subfloat[kpaga `catch']{
\begin{Itabular}{p{4cm}p{5cm}}
 wʊ̀sá kpáɣń̩ nā & `Wusa caught {\sc me}'\\
 wʊ̀sá kpáɣɪ́ɪ́ rā & `Wusa caught {\sc you}'\\
 wʊ̀sá kpáɣʊ́ʊ́ rā &  `Wusa caught {\sc her}'\\
 wʊ̀sá kpáɣə́já wā &  `Wusa caught {\sc us}'\\
 wʊ̀sá kpáɣə́má wā & `Wusa caught {\sc you}'\\
 wʊ̀sá kpáɣáá wā & `Wusa caught {\sc them}'\\
 wʊ̀sá kpáɣə́bá wā &  `Wusa caught {\sc them}'\\
\end{Itabular} 
}
\quad
\subfloat[goro `(go in) circle']{
\begin{Itabular}{p{4cm}p{5cm}}
wʊ̀sá górń̩ nō & `Wusa circled {\sc me}'\\
 wʊ̀sá góríí rē & `Wusa circled {\sc you}'\\
 wʊ̀sá górúú rō &  `Wusa circled {\sc her}'\\
 wʊ̀sá górə́já wā/rā &  `Wusa circled {\sc us}'\\
 wʊ̀sá górə́má wā/rā & `Wusa circled {\sc you}'\\
 wʊ̀sá góráá wā/rā & `Wusa circled {\sc them}'\\
 wʊ̀sá górə́bá wā/rā &  `Wusa circled {\sc them}'\\
\end{Itabular} 
}
\end{table}

The schwas ({\sls ə}) in {\sls kpaɣəja} and  {\sls gorəja} are perceived as 
fronted,
and the ones in {\sls kpaɣəma} and {\sls gorəma}  as rounded. Although this is
certainly due to the following consonant, they are so weak that they can only be
heard when they are carefully pronounced (see Section \ref{sec:PHO-weak-syll}). 
The paradigm in Table  
\ref{tab:object-clitic-CVCV}(b) can also be uttered in the \isi{plural} as 
{\sls górójé rē} ({\sc 1pl}),  %
{\sls górémá rā} ({\sc 2pl}), %
{\sls góráá rā} ({\sc 3pl.-h}), and %
{\sls górébá rā} ({\sc 3pl.+h}). 
 The \isi{focus} particle {\sls wa} is a
\isi{variant} of {\sls ra}. Some consultants  agree that these forms are in free 
variation,
yet the {\sls wa} form coexists only with  the \isi{plural} in the paradigms elicited.
Nonetheless, such paradigm elicitations are particularly subject to
unnaturalness.\footnote{I personally believe that the alteration is
determined by some kind of sandhi, not number. As to why {\sls wa} appears only 
in
the \isi{plural}, a scenario may be that (i) first, I install a routine by starting
with the 1.{\sc sg} {\sc me} and ending with the 3.{\sc pl} {\sc them}, (ii) in the process of eliciting, the passage
from third singular to first \isi{plural} triggers  a different verb shape, i.e.
CVCVV/CVCN  to CVCVCV, and (iii)  although formally identical to the verb forms
of the singular, the reason why {\sls wa} follows the third \isi{plural} \isi{non-human} 
could
be explained by psychological habituation.}

\subsubsection{Pluractional suffixes}
\label{sec:GRM-PluralVerb}


A pluractional verb is defined as a verb which can (i) express the repetition of an event,  (ii)   subcategorize for a \isi{plural} object and/or  \isi{plural} subject, and/or  (iii)  be marked by the pluractional suffix {\sls -sI}, a derivative suffix whose  vowel quality is always high and front and whose {\sc atr} value is determined by the stem vowel(s).\footnote{An exposition of the `\isi{plural} verbs' in \ili{Vagla} can be found in \citet{Blen03}. \citet[viii]{daku07} calls a similar morpheme `\isi{iterative}' (i.e. \ili{Gurene} {\sls -sɛ}).  Among the West African languages, it is the pluractional verbs in \ili{Hausa} which have received most attention \citep[see][]{Jose08}.}  According to (i) above, the iterativeness may affect the interpretation of the number of participants of an event. Consider the contrasts between the sentences in (\ref{ex:GRM-pv-cut}), where none of the arguments are in the \isi{plural} (i.e. contra (ii)).

\ea\label{ex:GRM-pv-cut}
  
 \ea\label{GRM-pv-cutsg}
\gll   ǹ̩  téŋé  à nàmɪ̃̀ã̀  rā.\\
    {\sc 1sg} {cut}  {\sc art} {meat} {\sc foc}\\
\glt `I cut a piece of meat (i.e.  made a cut in the flesh or cut into two
pieces).'

\ex\label{GRM-pv-cutpl}
\gll ǹ̩ téŋé-sí  à nàmɪ̃̀ã̀  rā.\\
    {\sc 1sg} {cut-{\sc pv}} {\sc art} {meat} {\sc foc}\\
\glt `I cut the meat into pieces.'

 
\z 
 \z

\newpage  
In  (\ref{GRM-pv-cutpl}),  the formal distinction on the verb `cut',  compared
to (\ref{GRM-pv-cutsg}),  causes  the event to be interpreted as one which
involves the repetition of the `same'  sub-event.  The word {\sls namɪ̃ã} 
`meat'
is allowed in both the contexts of (\ref{GRM-pv-cutsg}) and
(\ref{GRM-pv-cutpl}), although one may argue that the word {\sls namɪ̃ã} is
inherently
\isi{plural} but grammatically singular,  and that the word is appropriate in both
contexts. Despite the fact that  `meat' has indeed a \isi{plural} form, i.e. {\sls 
nansa}, it is probably the mass term denotation of {\sls namɪ̃ã} which 
makes (\ref{GRM-pv-cutpl}) acceptable. 


\ea\label{GRM-pv-turn}
  
 \ea\label{GRM-pv-turnsg}
\gll   ǹ̩  tʃígé  à  hɛ̀ná  rá.\\
  {\sc 1sg} {turn} {\sc art} {bowl.\sg} {\sc foc}\\
\glt `I turn (upside down) the bowl.'

 \ex\label{GRM-pv-turnpl1}
\gll   ǹ̩  tʃígé-sí  à  hɛ̀nsá  rá.\\
   {\sc 1sg}   {turn-{\sc pv}} {\sc art} {bowl.\pl} {\sc foc}\\
\glt `I turn (upside down) the bowls (one after the other).'


 \ex\label{GRM-pv-turnpl2}
\gll {(?)}  n̩  tʃige-si   a  hɛna  ra.\\
    {}  {\sc 1sg} {turn-{\sc pv}} {\sc art}  {bowl.\sg}  {\sc foc}\\
\glt `I turn (upside down in a repetitional fashion) the bowl.'

\z 
 \z

In (\ref{GRM-pv-turn}), however,  the grammatical object of a pluractional verb {\sls tʃigesi} `turn iteratively' or `put on face down iteratively'  must refer to individuated entities. Comparing  (\ref{GRM-pv-turnsg}) and (\ref{GRM-pv-turnpl2}) with (\ref{GRM-pv-turnpl1}),   the pluractional verb cannot coexist with a singular noun as grammatical object due to the fact that  some `turning' events are hard to conceive as affecting the same object in a repetitive fashion. However, in (\ref{GRM-pv-beat}) the `beating' can affect  one or several individuals. 

\ea\label{GRM-pv-beat}
  
 \ea\label{GRM-pv-beat.sg}
\gll   ǹ̩   túgó  à bìè  rē.\\
   {\sc 1sg}  {beat} {\sc art} {child.\sg} {\sc foc}\\
\glt `I beat the child.'

\ex\label{GRM-pv-beat.pl1}
\gll   ǹ̩ túgó-sí  à bìsé  ré.\\
   {\sc 1sg}   {beat-{\sc pv}} {\sc art} {child.\pl} {\sc foc}\\
\glt ` I beat the children.'


\ex\label{GRM-pv-beat.pl2}
\gll  ǹ̩ túgó-sí  à  bìè  rē.\\
   {\sc 1sg} {beat-{\sc pv}} {\sc art}  {child.\sg} {\sc foc} \\
\glt `I beat the child (more than once, over a short period of time).'

\z 
 \z

Whereas  (\ref{GRM-pv-beat.pl2}) has a possible interpretation, two language consultants could not assign a meaning to (\ref{GRM-pv-catchout}) below. 

\ea\label{GRM-pv-catch}
  
 \ea\label{GRM-pv-catchsg}
\gll ŋ̩̀  kpágá  à  zál  là.\\
   {\sc 1sg}   {caught} {\sc art} {chicken.\sg} {\sc foc}\\
\glt `I caught a chicken.'

 \ex\label{GRM-pv-catchpl1}
\gll ŋ̩̀ kpágá-sɪ́  à  zálɪ́ɛ́ rà.\\
    {\sc 1sg} {caught-{\sc pv}} {\sc art} {chicken.\pl} {\sc foc}\\
\glt `I caught chickens (i.e. in repeated actions).'


 \ex\label{GRM-pv-catchpl2}
\gll   ŋ̩̀  kpágá  à  zálɪ́ɛ́ rà.\\
    {\sc 1sg}  {caught} {\sc art} {chicken.\pl} {\sc foc}\\
\glt `I caught chickens (i.e. in one move).'

 \ex\label{GRM-pv-catchout}
\gll (?)  ŋ̩  kpaga-sɪ  a  zal  la.\\
  {}   {\sc 1sg}  {caught-{\sc pv}} {\sc art} {chicken.\sg} {\sc foc}\\
\glt `I caught a chicken (i.e. after unsuccessful attempts until finally succeeding with one particular chicken).'

\z 
 \z

A pluractional verb usually denotes an action, but not a state. Therefore, in (\ref{GRM-pv-catch}), the sense of {\sls kpaga}$_{1}$  is related to `catch', and not to the  possessive sense of the verbal state lexeme   {\sls kpaga}$_{2}$ `have'.\footnote{Though I like to treat {\sls dʊasɪ} as a counterexample.  The pluractional verb {\sls dʊasɪ} `be in a row'  may be  derived from the existential predicate {\sls dʊa} `be on/at/in'.  For instance, the verbs {\sls tele} `lean' and {\sls telege} `lean' are determined by the number value ({\sc sg}/{\sc  pl})  of the subject.  If more examples like these  arise, {\it pluractional} would then loose its literal signification.} Beside {\sls /-sI/}, the suffix {\sls /-gE/} may also turn a verbal process lexeme into a pluractional verb, e.g.   {\sls tɔtɪ} `pluck' $>$ {\sls  tɔrəgɛ} `pluck iteratively' and  {\sls keti} `break'  $>$ {\sls kerigi} `break iteratively'.

\ea\label{ex:GRM-kpa-paa}
  
 \ea\label{ex:GRM-kpa}
\gll kà kpá zál háŋ̀ tà.\\
go take.{\sc pl} fowl.{\sg} {\dem} let.free\\
\glt `Go and take this fowl away.'
   \ex\label{ex:GRM-paa}
\gll kà páá zálɪ́ɛ́ hámà tà.\\
go take.{\sc pl} fowl.{\pl} {\dem}.{\pl}  let.free\\
\glt `Go and take these fowls away.'
 
\z 
 \z


Finally, a \isi{pluractional} verb must not necessarily display the
suffixation pattern described above. This is confirmed by the pair  {\sls kpa}/{\sls paa} `take'  in (\ref{ex:GRM-kpa-paa}).




\subsubsection{Possible derivational suffixes}
\label{sec:GRM-deri-suff}


\citet[37]{Daku09} and \citet[69]{Bonv88} identify some \is{derivational suffix}derivational suffixes 
in 
\ili{Gurene} and \ili{Kasem} respectively, but write that their signification is hard to 
establish.  However, their descriptions indicate that  derivational suffixes 
mainly encode aspectual distinctions.

As mentioned in Section \ref{sec:GRM-verb-syll-und-tone}, about 90\% of the verbs are monosyllabic or bisyllabic, and  only the consonants {\sls m, t, s, n,  l,} and {\sls g} are found  in onset position word-medially in trisyllabic verbs. This situation could suggest that 10\% of the verbs in the current lexicon are the product of verbal derivation, and that the consonants found  in onset position word-medially in trisyllabic verbs are part of \is{derivational suffix} derivational suffixes. 


\ea
 \ea\label{ex:plur-ex}
\gll   ʊ̀ wʊ́rɪ́gɪ́ à hàɣlíbíé ré.\\
{\sc 3sg} scatter {\sc art} block.{\sc pl} {\sc foc}\\
\glt `He scattered the mud blocks.' (they were piled and packed)

 \ex 
\gll  ʊ̀ wʊ́rá à hàɣlíbíí ré.\\
 {\sc 3sg} move {\sc art} block  {\sc foc}\\
\glt `He moved the mud block.' (they are uneven, but still piled)
\z 
 \z
 
 However, apart from the \isi{pluractional} suffix discussed in Section \ref{sec:GRM-PluralVerb},  it is impossible at this stage of the research to establish a systematic mapping between the third syllable of a trisyllabic verb and a meaning.  
 
 
 %[TO DO]
\begin{table}[!htb]
\small
\centering
\caption{Possible derivational suffixes\label{tab:GRM-der-suff}}

\begin{tabular}{lllll}
\lsptoprule

 &&&{\sls -gV}&\\\midrule

wʊ̀rà {(v)}& `move, shift' & $>$ & wʊ̀rɪ̀gɪ̀ {(v)}& `scatter'\\
tàrà  {(v)}& `support' & $>$ &tàràgɛ̀ {(v)}& `pull'\\
%bɪla {(v)}& `turn repetitively' & $>$ & bɪlgɪ {(v)}& `clean'\\
brà {(v)}& `return' & $>$ & bɛ̀rɛ̀gɪ̀  {(v)}& `change direction'\\\midrule

&&&{\sls -mV} &\\\midrule

ɲàgà  {(v)} & `be sour' &$>$ & ɲàgàmɪ̀  {(v)}& `ferment'\\
víl {(n)} &`well' & $>$ &vílímí {(v)} & `whirl'\\
 mɪ̀là {(v)} & `turn round' & $>$ &mɪ̀lɪ̀mɪ̀ {(v)}& `turn'\\[1ex]\midrule

&&&{\sls -lV}&\\\midrule
 kàgà {(v)}& `choke'& $>$ & kàgàlɛ̀ {(v)} & `lie across'\\
 \lspbottomrule
\end{tabular}
\end{table}

\largerpage
The  example  provided in (\ref{ex:plur-ex}) and Table \ref{tab:GRM-der-suff} presents  some indications that {\sls m, l,} and {\sls g}, i.e. CVCV{\sls  \{m, l, g\} }V, are involved in some kinds of derivation, although the next step would be to determine their exact meaning.\footnote{The verb pair {\sls  go} `round'  and {\sls goro}  `(go in) circle'  is  manifestly a derivation as well, i.e. CV $>$ CV-rV.}


\section{Grammatical pragmatics and language usage}
\label{sec:GRM-adjuncts}

In this section are presented aspects of the grammar which do not naturally fit within the distinction \textit{clause},  \textit{verbal} or \textit{nominal} and ``which involve encoded conventions correlating between specific linguistic expressions and extra-grammatical concepts'' \citep[256]{arie10}. Sections \ref{sec:GRM-adv-pro} and \ref{sec:GRM-deic-adv} present  adverbial deixis particles, Section \ref{sec:GRM-focus} offers an overview of what has been stated on \textit{focus}, and the remaining covers selected  pieces of language usage and everyday communication. 


\subsection{Manner deictics}
\label{sec:GRM-adv-pro}

Chakali has a two-term exophoric system of \isi{manner deixis} \citep{koen12}; the  expressions  {\sls keŋ} and {\sls nɪŋ} are treated as  two manner deictics  (glossed {\sc dxm}).  Manner is a cover term since the content dimension appears to cover degree and  quality as well. Consider the examples in (\ref{ex:GRM-dxm-}).

\ea\label{ex:GRM-dxm-}
 \ea\label{ex:GRM-dxm-1}
\gll kén nè bà já jāà.\\
{\sc dxm} {\sc foc}  {\sc 3pl.h+}  {\sc hab} do  \\
\glt `That's the way to do it. (manner)'  

  \ex\label{ex:GRM-dxm-2}
\gll  hàɣlɪ́kɪ́ŋ̀ zéné mààsɪ̀ nɪ́n nà.\\
snake long equal {\sc dxm}  {\sc foc}\\
\glt `The snake was that/this big. (degree)'  

  \ex\label{ex:GRM-dxm-3}
  
  
\gll  kàlá máásɪ́ɪ́ nɪ́ŋ̀.\\
K. equal.{\sc nmlz} {\sc dxm}\\
\glt `Kala is like that. (quality) [of size]'  

  \ex\label{ex:GRM-dxm-4}
  
  
\gll kàlá dʊ́nná kéŋ̀.\\
K. type {\sc dxm}\\
\glt `Kala is like that. (quality) [of nature]'  

\z 
 \z   

The expressions  {\sls keŋ} and {\sls nɪŋ} are very frequent  and bring to mind the  English  `like this/that',  that is,  an expression which refers to something extralinguistic yet in  the context of the utterance. In that sense they can be treated as  pro-forms. Example (\ref{ex:GRM-adv-pro-keng-ning}) illustrates this point.

\ea\label{ex:GRM-adv-pro-keng-ning}

 \ea\label{ex:GRM-adv-pro-ning}
\gll bàáŋ ɲʊ̃̀ã̀sá káá sìì báŋ̀ nɪ̄ nɪ̏ŋ?\\
{\q}  smoke  {\egr} rise {\dem} {\postp} {\dxm}\\
\glt `What smoke is rising here like this?'  [PY 059]
  \ex\label{ex:GRM-adv-pro-keng}
 \gll bàáŋ káá jāā kȅŋ?\\
  {\sc q}.what {\sc egr} do  {\dxm}\\
 \glt `What is doing like that?' (Reaction to a sound coming from inside a pot)

\z 
 \z   

The meaning difference between   {\sls nɪŋ}  and {\sls 
keŋ} seem to be 
motivated by the way they  encode a sort of psychological saliency on a
\isi{proximal}/\isi{distal} dimension. This distinction needs more evidence than the one I
provide,  but consider the conversation between A and B in
(\ref{ex:GRM-adv-pro-keng-AB}). 


\ea\label{ex:GRM-adv-pro-keng-AB}

 \ea\label{ex:GRM-adv-pro-A}
\gll A: nɪ́n nā bààbá ŋmȁ?\\
 {} {\dxm} {\foc} B. say\\
\glt `Is this what Baaba said?'

  \ex\label{ex:GRM-adv-pro-B}
 \gll B: ɛ̃̀ɛ̃́ɛ̃̀ kén{\ꜜ} né ʊ̀ ŋmá.\\
 {} yes {\dxm} {\foc} {\sc 3sg} say\\
\glt `Yes, that is what he said.'
 
 
\z 
 \z   

Similarly,  the (fictional) discourse excerpt in
(\ref{ex:GRM-adv-kapok}) concerns a father (A) addressing his son (B) on the
topic of  how to ignite kapok fiber. The sentence (\ref{ex:GRM-adv-kapok-A-2})
is accompanied with a demonstration on how to strike a cutlass on a stone.


\ea\label{ex:GRM-adv-kapok}
 
 
 \ea\label{ex:GRM-adv-kapok-A-1}
\gll A: kpá kóŋ à ŋmɛ̀nà díŋ!\\
{}  take kapok {\conn} ignite fire\\
\glt `Take some kapok and start a fire!'

 \ex\label{ex:GRM-adv-kapok-B}
\gll B:  ɲɪ̀nɪ̃̀ɛ̃́ bà já kà ŋmɛ̀nà?\\
{} {\q} {\sc 3pl} do {\egr} ignite\\
\glt `How does one ignite?' 

 \ex\label{ex:GRM-adv-kapok-A-2}
\gll  A: ŋmɛ̀nà nɪ́ŋ̀!\\
{} ignite {\dxm}\\
\glt `Ignite like this!'

 \ex\label{ex:GRM-adv-kapok-A-3}
 \gll  A: tʃɪ́á dɪ̀ ɪ̀ tʃɪ́ wááwá ŋmɛ̀nà kéŋ̀.\\
{} tomorrow {\conn} \textsc{2sg} {\cras} come.{\pfv} ignite {\dxm}\\
\glt `Tomorrow when you come, ignite like that.'
 
\z 
 \z  
 
In the context of (\ref{ex:GRM-adv-kapok}), at the farm the next day, the boy (B) would tell a colleague: {\sls  kén nē bà já ŋmɛ̀nà},  {\it lit.} like.that they do ignite, `that is how one ignites'. 


\ea\label{ex:GRM-ning-prop-2}
 \gll nɪ́ŋ lɛ̀ɪ́ ʊ̀ʊ̀ dɪ̀à háŋ̀ já dʊ̀.\\
 {\dxm} {\sc neg}  {\sc 3sg.poss} house {\sc dem} {\sc hab} be\\
\glt  `This is not how his room used to be.' [PY 78]

\z


In (\ref{ex:GRM-ning-prop-2}), {\sls nɪŋ} refers to the condition of  the room,
which is not a manner  but a property of the room. 
In addition, {\sls keŋ} and 
{\sls nɪŋ} can function as  discourse particles, whose
meanings resemble   English `like' in some registers \citep{Muff02}. In
(\ref{ex:GRM-keng-like}), {\sls keŋ} is considered superfluous since it does not
contribute to the manner of  motion or the state of the
participant.\footnote{Something identical to the translation of
(\ref{ex:GRM-keng-like}) may be heard in all over the country, in both the Ghanaian languages and Ghanaian English.} 
 
 \ea\label{ex:GRM-keng-like} 
 \gll ǹ̩ káálʊ̄ʊ̄ kéŋ̀.\\
 {\sc 1sg} go.{\ipfv .\foc}  {\sc dxm}\\
 \glt `I am leaving like that.'
\z

Also, depending on the intonation associated with it, and whether or not  the \isi{focus} particle  is  present, {\sls keŋ} and {\sls nɪŋ} can function as \is{interjection} interjections used to convey comprehension or surprise. So a phrase like {\sls kén nȅȅ} could be roughly translated as `Is that so?', {\sls kén nè}   has a similar function to the English  tag-question `Isn't it?', but {\sls kéēèŋ} or {\sls kén{\ꜜ} né} could be translated as `yes, that is it'. 

Finally, \citet{mcgi99} presents  {\sls nyɛ} and {\sls ɛɛ} (\isi{variant} {\sls gɛɛ}) as demonstrative pronouns in \ili{Pasaale}, which can also modify an entire clause. The former corresponds to `this' and the latter to `that'. At this point, it is a matter of comparing the two languages and the terminology employed.  Nonetheless, in the majority of the examples provided by \citet{mcgi99}, Chakali {\sls keŋ} and {\sls nɪŋ} seem to have the same function. 


\subsection{Spatial deictics}
\label{sec:GRM-deic-adv}
 

A speaker-subjective,  two-way contrast  exists to locate entities in space. The \isi{spatial deixis} demonstrative  {\sls bááŋ̀} designates the location of the speaker, while the spatial deixis demonstrative  {\sls dé} designates  where the speaker is not located. They represent what is known as the `\isi{proximal}' and `\isi{distal}' dimensions of spatial deixis.


\ea\label{ex:vp}
\ea\label{ex:deic-adv-prox}
\gll wàà  bááŋ̀.\\
  come {\sc dxl}\\
\glt  `Come here.'

 \ex\label{ex:deic-adv-dist}
\gll ʊ̀ dʊ́á dé (nɪ̄).\\
    {\psg}  be.at  {\sc dxl}  {\postp}\\
\glt  `He is there.'

\z 
 \z

  In (\ref{ex:deic-adv-prox}) and (\ref{ex:deic-adv-dist}),  they are translated as `here' and  `there' respectively, and glossed {\sc dxl}, standing for `locative deixis'.  Notice that unlike the single demonstrative  modifier discussed in Section \ref{sec:GRM-demons},   {\sls bááŋ̀} and   {\sls dé}  do encode a proximal/distal distinction.

\subsection{Focus}
\label{sec:GRM-focus}


Since the notion of \isi{focus} has been discussed separately in connection with nominals and verbals, this section offers a basic overview of what has been stated.  \citet[326]{Dik97} writes that   ``the focal information in a linguistic expression is that information which is relatively the most important or salient in the given communicative setting''.  In Chakali, there are  several ways in which a speaker can integrate focal information, and all of them put `in \isi{focus}' a constituent.\footnote{The  terminology employed in the literature is probably the result of  complex and still obscure phenomena. For instance, for the post-verbal particle {\sls la} in \ili{Dagaare}, \citet{Bodo97} uses the term `factitive' and \is{affirmative} `affirmative' particle interchangeably, \citet{Daku05} uses `(broad- and narrow-)\isi{focus}' and glosses it either as {\sc aff} or {\sc foc}, and \citet{Saan03} uses post-verbal particle and glosses it as {\sc aff}. The latest contribution  to the  discussion is  \citet{hiro14} which uses a Lexical-Functional Grammar formalism to account for the  special distribution of  {\sls la}. In-depth accounts of \isi{focus} in \ili{Grusi} languages can only be found in \citet{blas90}, but see also \citet{mcgi99}. Anne Schwarz has worked extensively on the topic in some \ili{Gur} and \ili{Kwa} languages \citep{Schw10}.}   The first encodes focal information in a particle which  always follows a nominal, i.e. {\sls ra} and variants. Its  phonological shape is determined by the preceding phonological material (see Sections \ref{sec:focus-forms} and \ref{sec:GRM-foc-neg}). The second, which was called the \isi{assertive suffix}, takes the form of vowel features which are suffixed onto the verb  (see Sections \ref{sec:GRM-verb-perf-intran} and \ref{sec:GRM-verb-suffix}). It was claimed that  the assertive suffix surfaces only if (i) none of the other constituents in the clause are in \isi{focus}, (ii) the clause does not include propositional \isi{negation}, and (iii) the clause is intransitive. The second criterion (ii) is applicable to the particle {\sls ra} as well: thus focal information can only exist in \is{affirmative} affirmative clauses, \isi{negation} automatically prevents information from being in \isi{focus}.  In (\ref{ex:GRM-focus}),  the examples illustrate  how the  focal information is encoded when the object (\ref{ex:GRM-focus-obj}), the subject (\ref{ex:GRM-focus-subj}) and the predicate  (\ref{ex:GRM-focus-pred}) are considered the most important piece of information. 
\ea\label{ex:GRM-focus}

\newpage 
 \ea\label{ex:GRM-focus-obj}{\rm Focus on object: What has the man chewed?}\\
\gll   à báál tíē sɪ́gá rá.\\
   {\sc art} man chew bean {\foc}\\
\glt `The man chewed {\sc beans}.'

\gll  kàlá tíē sɪ́gá rá.\\
   K.   chew bean {\foc}\\
\glt `Kala chewed {\sc beans}.'

\ex\label{ex:GRM-focus-subj}{\rm Focus on subject: Who has chewed the beans?}\\
\gll   à báál là  tíē sɪ́gá.\\
    {\sc art} man {\foc} chew bean\\
\glt `{\sc the man} chewed beans.'



\gll  kàláá tíē sɪ́gá.\\
   K.   chew bean\\
\glt `{\sc kala} chewed beans.'

\ex\label{ex:GRM-focus-pred}{\rm Focus on predicate: What happened?}\\
\gll à báál tíéwóó.\\
   {\sc art} man chew.{\pfv .\foc}\\
\glt `The man {\sc chewed}.'


\z 
 \z
 
 
The \isi{focus} particle does not differentiate between  grammatical functions and some times appears to be optional.  Also,  the assertive suffix is quite rare in narratives.  \citet[94]{blas90} is the only author to my knowledge who identifies the presence  of  evidentiality --  hearsay, more precisely -- in \ili{Gur} languages. According to her the morpheme {\sls rɛ} in Sissala refers to reported or inferred information. This raise the question as to what extent the \isi{focus} particle and the assertive suffix provide evidential information.\footnote{A promising avenue to follow in the study of \isi{focus} would be the recent work of Anne Schwarz  who looks at the phenomenon from a perspective of encoding a thetic vs. categorical distinction.}

Also, a third way to encode \isi{focus} is the \is{lengthening}lengthening and emphasis of vocalic material. The issue remains far from clear and stands in need of more information.

\ea\label{ex:focus-length}

 
 \ea
\gll  à bɔ̀là tɪ̀n dí kɔ̀sá rá.\\
\textsc{art1}  elephant \textsc{art2} eat.\textsc{pfv} grass \textsc{foc}\\
\glt  `The elephant ate {\sc grass}.'

 \ex
\gll à bɔ̀là tɪ̀ńː    dí kɔ̀sá.\\
\textsc{art1} elephant \textsc{art2} eat.\textsc{pfv} grass \\
\glt  `{\sc the elephant} ate grass.'

 \ex\label{ex:focus-length-out-1}  \textasteriskcentered à bɔ̀là tɪ̀ńː  dí kɔ̀sá rá.\\
 
 \ex
\gll  kàlá káá hɪ̃́ɛ̃́rʊ̄ʊ̄.\\
K. \textsc{ipfv} voracious.\textsc{foc}.\\
\glt  `Kala is {\sc a voracious meat eater}.'

 \ex
\gll  káláá káá hɪ̃́ɛ̃́rɪ̄.\\
K.\textsc{foc} \textsc{ipfv} voracious.\\
\glt  `{\sc Kala} is a voracious meat eater.'

 \ex\label{ex:focus-length-out-2}  \textasteriskcentered káláá káá hɪ̃́ɛ̃́rʊ̄ʊ̄.\\
\z 
 \z
 
 Example (\ref{ex:focus-length}) shows that since only one constituent can be focused, the \is{lengthening} lengthening of and special intonation on   {\sls kàlá} and {\sls tɪ̀n}  which is assumed to signal \isi{focus}, together with another constituent in \isi{focus}, is ungrammatical (cf. \ref{ex:focus-length-out-1} and  \ref{ex:focus-length-out-2}). 


\subsection{Linguistic taboos}
\label{sec:GRM-ling-taboo}


A linguistic \isi{taboo} is defined here as the avoidance of certain words on certain occasions due to  misfortune associated with those words. These circumstances depend on belief; they can be widespread or marginal. The avoidance of certain words may depend on the time of the day or action carried out. For instance, not only  is sweeping  not allowed when someone eats, but uttering the word {\sls tʃãã} `broom' is also forbidden. Also, mentioning certain animal names is excluded as they may either be tabooed by someone present, due to his/her animal totem and/or its meat is forbidden,  or attract the animal's attention, i.e. the belief that the  animal may feel it is called out. The strategy is to substitute a word with another, often undertaking a \is{metonymy} metonymic strategy. 

\begin{table}[!htb]
\small
\centering
\caption{Taboo synonyms \label{ex:GRM-taboo-synonyms}}

\begin{tabular}{llll}
\lsptoprule
 Avoided word & Substitute word & Literal meaning & Gloss\\
\midrule

 {\sls bɔ̀là} & {\sls sé-zèŋ́}   & animal-big &   `elephant'\\

{\sls bɔ̀là} & {\sls néŋ-tɪ̄ɪ̄nā} &  arm|hand-owner  & (trunk>) `elephant'\\

  {\sls dʒɛ̀tɪ̀} & {\sls ɲú-zéŋ-tɪ̄ɪ̄nā} &  head-big-owner &  `lion'\\
  
  {\sls bʊ́ɔ̀mánɪ́ɪ́} & {\sls ɲú-wíé-tɪ̄ɪ̄nā} &   head-small-owner &   `leopard'\\
  
  {\sls váà} & {\sls nʊ̃̀ã̀-tɪ́ɪ́ná}  & mouth-owner &   `dog'\\
  
  {\sls kɔ́ŋ} & {\sls nɪ́ɪ́-tɪ́ɪ́ná}   & water-owner &  `cobra'\\
  
  {\sls gbɪ̃̀ã́} & {\sls néŋ-gál-tɪ̄ɪ̄nā} &  arm|hand-left-owner &   `monkey'\\
  
  {\sls hèlé} & {\sls mùŋ-zɪ́ŋ-tɪ̄ɪ̄nā} & back-big-owner &  `type of squirrel'\\
  
  {\sls tébíŋ̀} & {\sls bárà-tʃɔ́gɔ́ʊ́}  & place-spoil.{\sc pfv.foc}&  `night'\\
  
  {\sls ɲʊ́lʊ́ŋ} & {\sls ɲú-bɪ́rɪ́ŋ-tɪ́ɪ́ná} &  head-full-owner &  `blind' \\
  
%   {\sls tʃã́ã́} & {\sls kɪ̀m-pɪ̀ɪ̀g-ɪ́ɪ̀} & thing-mark-{\sc nmlz}  &  `broom'\\
  
  {\sls búmmò} & {\sls dóŋ} &   dirt  &  `black'\\
  
%   {\sls dʊ̃̀ʊ̃̀wìé} & {\sls mábíé-wāá-tèlè-púsíŋ} &   sibling-will.not-reach-meet.me & 
  
%  `type of snake'\\

\lspbottomrule
 
\end{tabular}
\end{table}


The second column of  Table (\ref{ex:GRM-taboo-synonyms}) contains expressions called \is{taboo synonyms} taboo \is{synonym} synonyms; they are substitutes to the words of the  the first column.  The substitutes  are usually \is{complex stem noun} complex stem nouns with a transparent descriptive meaning. Most of them use the stem {\sls tɪ́ɪ́ná} `owner of', e.g. {\sls néŋ-tɪ̄ɪ̄nā}, {\it lit.} arm|hand-owner.of,  `elephant',  the one with a big arm.  The stem {\sls tɪ́ɪ́ná}  `owner of'  can be characterized as a noun with an incomplete semantics which normally requires to be in an associative construction with another noun (i.e. person characterised by, \is{owner}owner of, or responsible for) and always appear following the `possessed’ stem.\footnote{Mampruli {\sls daana}, \ili{Hausa} {\sls mai}, and \ili{Arabic}  {\sls dhū} seem to correspond to the meaning of Chakali {\sls tɪɪna}.}


\subsection{Ideophones and iconic strategies} 
\is{ideophone}
\label{sec:GRM-onoma}

 Ideophones typically suggest the description of an abstract property or the manner in which an event unfolds.\footnote{See a discussion in relation to African languages in  \citet{sama01}, and a  review of the term in \citet{newm68, voel01, ding11}.}  The majority of ideophones function like  \is{qualifier}qualifiers or \is{intensifier}intensifiers (Section \ref{sec:GRM-intensifier}) or \is{adjunct adverbial}adjunct adverbials  (Section \ref{sec:GRM-adjuncts}). In Chakali ideophones tend to appear at the right periphery of the sentence and with a low tone. Examples are provided in (\ref{ex:GRM-ideo}).\footnote{The translations into English in (\ref{ex:GRM-ideo}) were not 
tested for consistency across many speakers.}
          

\ea\label{ex:GRM-ideo}

 \ea\label{ex:GRM-ideo-dxm}
\gll  à díŋ káá dīù gàlɪ̀gàlɪ̀gàlɪ̀/pèpèpè.\\
{\sc art} fire {\sc ipfv} eat  {\ideo}\\
  \glt `The fire is burning at an increasing rate.'


 \ex\label{ex:GRM-ideo-qual}
\gll à dʊ̃́ʊ̃́ síè jáá wə̀rwə̀rwə̀r.\\
{\sc art} python eye {\ident} {\ideo}\\
  \glt `The python's eyes are glittery.'

   \ex\label{ex:GRM-ideo-}
\gll à dáánɔ́ŋ márá bɪ̄jʊ̄ʊ́ lìgèlìgèlìgè.\\
{\sc art} tree.fruit well ripe.{\sc pfv}  {\ideo}\\
  \glt `The fruit is perfectly ripe.'
  
  
  \ex\label{ex:GRM-ideo-1}
\gll à sìbíé wàà márá bɪ̀ɪ̀ à dʊ́ nɪ̄ŋ wùròwùròwùrò.\\
{\sc art} beans  {\sc neg} well ripe {\sc conn} be   {\sc dxm} {\ideo}\\
  \glt `The bambara beans are not well cooked, they are still hard.'
  

\z 
 \z

An \isi{onomatopoeia} is a type of \isi{ideophone} which not only suggests the concept   it expresses with sound, but imitates  the actual sound of an entity or event.  Examples of \isi{onomatopoeia}\is{onomatopoeia} are {\sls púpù} `motorbike', {\sls tʃétʃé} `bicycle', {\sls tʃɔ̀kɔ̃́ɪ̃́ tʃɔ̀kɔ̃́ɪ̃́} `sound of a guinea fowl',  {\sls kr̀r̀r̀r̀} `sound of running',  {\sls pã̀ã̀} `sound of an eruption caused by lighting a fire',  {\sls gbàgbá}  `duck',\footnote{The word for `duck' is probably borrowed from \ili{Waali}. I was told that the bird was introduced recently. It was hard to find one in the villages visited.}   and {\sls kpókòkpókòkpókò} `sound of knocking on a clay pot'. Similarly, an iconic strategy to convey an amplified meaning or the idea of continuity is to \is{lengthening}lengthen the sound of an existing word. 


 \ea\label{ex:GRM-lenght}
   \gll  kàwāá sìì tàrɪ̀ kéééééééŋ, àkà dʊ́á  bà dɪ̀ànʊ̃́ã́ nɪ̀.\\
pumpkin rise {creep} {\dxm} {\conn} {be.at} {\sc 3sg.poss} door {\postp}\\
\glt `The pumpkin crept, crept, crept, and crept up to their door mat.' [PY 56]
\z



In (\ref{ex:GRM-lenght}) the \is{manner deictic}manner deictics {\sls keŋ} (Section \ref{sec:GRM-adv-pro}) is stretched to simulate the extention in time of the event, i.e. the pumpkin grew until it reached the door.\footnote{An equivalent meaning may be expressed in some varieties of Gh. Eng.  with the adverbial expression  {\sls ãããã}, as in `Today I worked  {\sls ãããã}, until night time.'}

Reduplication\is{reduplication} of one or two syllables is the general structural shape of \is{ideophone}ideophones and onomatopoeias. A large set of visual perception expressions can be treated as ideophonic expressions (Section \ref{sec:GRM-qualifier}), all of which are reduplicated expressions.  


%busabusa in \ili{Waali}, from \ili{Akan} ``unatteactive, dull, repulsive''
% tʃatʃara in \ili{Waali}  ``a muddy area''
% apelepele in \ili{Waali} ``clear''
% ganura ``paint/marks on sacks at market''

\ea\label{ex:BCTreduplic}{\rm Visual perception expressions  and 
non-attested stems}\\

\ea {(kɪn|a)-hɔlahɔla}	  [áhɔ̀làhɔ̀là]   *hɔla   
\ex {(kɪn|a)-ahɔhɔla}		[áhɔ̀hɔ̀là] *hɔla   
\ex {(kɪn|a)-busabusa}	[ábùsàbùsà]	 *busa  
\ex {(kɪn|a)-adʒumodʒumo}	[ádʒùmòdʒùmò] *dʒumo  
\ex {(kɪn|a)-bʊɔbʊɔna}	[ábʊ̀ɔ̀nàbʊ̀ɔ̀nà]		 *bʊɔna  
\ex {(kɪn|a)-ʔileʔile}	[áʔìlèʔìlè]	 *ʔile 

\z
\z


Assuming that \isi{reduplication} is a morphological process in which the root or 
stem is repeated (fully or partially), then it is questionable whether one can 
treat most of the naming data as reduplication. It is obvious from the examples in 
(\ref{ex:BCTreduplic}) that there is a `form-doubling' on the surface, yet such 
expressions  are not made out of attested stems (and they do not have loci in the chromatic space, 
\citealt[see][]{brin16}).  


\subsection{Interjections and formulaic language}
\label{sec:GRM-formulaic}

This section introduces some pieces of \isi{formulaic language}, which is defined as conventionalized words or phrases. It usually include greetings, idioms, proverbs,  etc. \citep{Wray05}. First, common interjections\is{interjection} are introduced in Table \ref{tab:GRM-interj},\footnote{The \isi{etymology} of {\sls ʔàmé} has not been confirmed and {\sls gááfʊ̆̀rà} is  ultimately \ili{Hausa}. The word {\sls ʃ̃ʊ̃́ɛ̃̀ɛ̃̀} is equivalent to the function  associated with the action of {\sls tʃuuse} in Chakali ({\sls tʃʊʊrɪ} in \ili{Dagaare}, {\sls tʃʊʊhɛ} in \ili{Waali}, `puf' or `paf'  in Gh.  Eng.,   $<$ English `pout'), which is a fricative sound produced by a non-pulmonic, velarized ingressive airstream mechanism, articulated with the lower lip and the upper front teeth while the lips are protruded.} then some greetings and idioms are presented. 


\begin{table}[!htb]
\small
\centering
\caption{Selected interjections \label{tab:GRM-interj}}

\begin{tabular}{>{\slshape}lp{8cm}}
\lsptoprule
{\rm Interjection} & Gloss\\[1ex] \midrule
ʔàɪ́  &  express denial or refusal\\
ʔɛ̃ɛ̃ & express affirmation\\
gááfʊ̆̀rà    &  express excuse when interrupting or disturbing  ({\it from}  \ili{Hausa})\\
tóù &  express agreement or understanding  ({\it from}  \ili{Hausa})\\
ʔàmé   & so be it  ({\it etym.}  Amen?)\\
ʔóí  &  express surprise\\
fíó  &  express strong denial or refusal\\
ʔánsà  &  1) greet hospitably, welcome, 2) accept and thank ({\it from} \ili{Gonja})\\
ʔĩ́ĩ̀ĩ́  &  express disappreciation of an action
carried out by someone else\\
ʔàwó  &  reply to greetings, a sign of appraisal of the interlocutor's
concerns ({\it from} \ili{Gonja})\\
 ʔábbà & express a  reaction to an unpalatable proposition, with disagreement and 
unexpectedness\\
ʃ̃ʊ̃́ɛ̃̀ɛ̃̀  &  express a disrespectful attitude towards what is being said and the 
one saying it \\
\lspbottomrule
\end{tabular}
\end{table}



Since they are conventionalized and idiomatic, the translations of  formulaic 
language in Table \ref{tab:GRM-interj}  are
  rough  approximations. The dictionary offers various spellings since variations are 
regularly perceived.



\subsubsection{Greetings}
\label{sec:GRM-greet}

Compulsory prior to any communicative exchange, \isi{greetings} trigger both attention and respect. When meeting with elders, one should  squat or bend forward hands-on-knees  while greeting. Praise names can be used in greetings, e.g. {\sls ɪ́tʃà} `respect to you and to your clan'. In Table \ref{tab:greetings},  typical greeting lines with some responses are provided. Note that the forms for morning and afternoon greetings are also used by the Gonjas. 


\begin{table}[!htb]
\small
\centering
\caption{Greetings\label{tab:greetings}}

\begin{tabular}{l>{\slshape}lp{7cm}}
\lsptoprule
Time & {\rm Speaker A} & Followed by either speaker A or B\\
\midrule

Morning  & ánsùmōō  & {\sls ɪ̀ sìwȍȍ} `You stood?', {\sls ɪ̄ dɪ̀ 
tʃʊ́àwʊ̏ʊ̏}   `And your lying?', {\sls ɪ̀ bàtʃʊ̀àlɪ́ɪ̀ wīrȍȍ }  `You 
sleeping place was good?'\\[1ex]

Afternoon   & ántèrēē & {\sls ɪ́ wɪ́sɪ́ tèlȅȅ}   `Has the sun reached 
you?' {\sls  ɪ́ dɪ́á} `And your house?'  {\sls ɪ̄ bìsé mūŋ} `And all 
your children?'\\[1ex]
  

Evening & ɪ́ dʊ̀ànāā &  {\sls ɪ́ dʊ̄ɔ̄n tèlȅȅ}  `Your evening 
has reached', 
{\sls ɪ́ kúó} `And your farm?'\\
\lspbottomrule
\end{tabular} 
\end{table}


The  second singular \isi{plural} {\sls ma} is added, i.e.  {\sls ánsùmōō} $\leftrightarrow$ {\sls māānsùmōō}, when there is more than one addressee or when there is  a single person but the greetings are intended to the entire house/family: thus  the number distinction {\sls ɪ}/{\sls ma} does not correspond to a politeness function. Chakali morning and afternoon \isi{greetings} resemble those of \isi{Waali} and other languages of the area. The response to various greetings such as {\sls ɪ́ dɪ́á} `(how is) your house?',  {\sls ʔánsà} `welcome, thanks' and many others is the multifunctional expression {\sls ʔàwó},  which is, among other things, a sign of appraisal of the interlocutor's concerns. The same expression is found in \isi{Gonja}, but its function is believed to be slightly  different.  I was told that the more extensive the greetings, the more respect one shows the addressee.  For instance, the elders do not appreciate the tendency of the youths to morning-greet as {\sls ã̄sūmō}, but prefer something like {\sls áánsùùmōōō}. 

Other ritualized expressions often used are: {\sls tʃɔ̄pɪ̄sɪ́ ālɪ̀ɛ̀}  {\it lit.} morning two,  `long time no see' (Section \ref{sec:NUM-misc-usage});   {\sls bámùŋ kɔ́rɛ́ɪ́}  {\it lit.}  all.\textsc{+hum} extent (unknown origin), `how are all your people?', {\sls ànɪ́ mà wʊ̀zʊ́ʊ́rɪ́ tɪ̀ŋ}, {\it lit.} and your day, used  after any bad event which happened to someone, e.g. referring to a funeral day, when the speaker has not seen the addressee since that day, among other expressions.

 

\subsubsection{Idioms}
\label{sec:GRM-idiom}

An idiom is a  composite expression which does not convey the literal  meaning 
of the composition  of its parts. Common among many African languages is a 
strategy by which  abstract nominals are expressed in idiomatic compounds. 
These compounds are made of stems whose meanings are disassociated from their 
ordinary usage.


Some examples have already been provided in Section \ref{sec:GRM-qualifier}. In Chakali, words identifying mental states and habits/behaviors are often idiomatic, e.g. {\sls síínʊ̀màtɪ́ɪ́nà} ({\sls sii-nʊma-tɪɪna}, {\it lit.} eye-hot-owner) `wild' or {\sls nʊ̃̀ã̀pʊ̀mmá} ({\sls nʊ̃ã-pʊmma}, {\it lit.} mouth-white) `unreserved'. Even though the expression {\sls síínʊ̀màtɪ́ɪ́nà} is made out of three lexical roots, it is a ``sealed'' expression and is associated with the manner in which a person behaves, i.e. a wild person. The sequence {\sls jaa nʊ̃ã dɪgɪmaŋa} in (\ref{ex:GRM-idiom-mouth}), {\it lit.} do-mouth-one,  is also treated as an idiomatic expression.

\ea\label{ex:GRM-idiom-mouth}
   \gll   bà jáá nʊ̃̀ã̀ dɪ́gɪ́máŋá à sùmmè dɔ́ŋà.\\
{\sc 3pl} do mouth one {\conn} help {\recp}\\
\glt `They should agree and help each other.'

\z

Needless to say, it is often difficult to  distinguish between an idiomatic
expression and  an expression in which only one of the  components is use in a
 non-literal sense. 
 
 

\subsection{Clicks}
\label{sec:GRM-clicks}

\citet[151]{nade89} writes that clicks\footnote{A click may be roughly defined 
as  the release of a pocket of air enclosed between two points of contact in the 
mouth. The air is rarefied by a sucking action of the tongue 
\citep[see][]{Lade93}.}  may be  heard in the \ili{Gur}-speaking area to  mean an 
affirmative `yes', or `I'm listening'.  This also occurs in the villages where I 
stayed, but I noticed that one click usually means `yes', `I understand' or `I 
agree', whereas two clicks mean the opposite. The \isi{click} is  palatal and produced 
with the lips closed.




