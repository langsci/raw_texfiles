\BackBody{\noindent Derivational affixes can be highly polysemous, producing a range of different, often related, meanings. For example, English deverbal nouns with the suffix \textit{-er} can denote \textsc{instruments} (\textit{opener}), \textsc{agents} (\textit{writer}), \textsc{locations} (\textit{diner}), or \textsc{patients} (\textit{loaner}). 
It is commonly assumed that this polysemy arises through a compositional process in which the affix interacts with the semantics of the base. 

This book presents two in-depth qualitative corpus studies of the productive relationship between the English nominalizing suffix \textit{-ment} and a semantically delimited set of verbal bases, namely change-of-state verbs and psychological verbs. 
By using frame-semantic representations, lexical rules, and inheritance hierarchies, it is shown that the derivational process is governed by an interaction of properties of the affix with the semantic elements provided by the base.
This has implications for the analysis of not only nominalization polysemy, but of the semantics of the bases as well. 
A quantitative exploration of the data set furthermore addresses the issues of gaps and ambiguity in corpus studies.}