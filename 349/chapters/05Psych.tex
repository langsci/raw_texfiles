\chapter[Psych verb bases]{Psych verb bases}
\label{ch:psy}

In this study, I investigate the semantics of nominalizations derived from psychological verbs, or \HighlightText{psych verbs}, that is, verbs which express an emotional or psychological event or state.\footnote{Most of the data which I analyze and discuss in this chapter was also the basis of a preliminary study published in \citet{Kawaletz.2015}. Since then, I have revised the data set as well as greatly modified the semantic labeling, frame formalization, and interpretation of my results.} Based on the idea that possible referents which are targeted by derivation must be available in the semantics of the base word, I will first discuss the psych base verbs in my data set (\sectref{sec:psy-input}). Psych verbs have been a widely studied topic for more than 30 years, and a number of (partially conflicting) analyses have been posited. Based on this existing work, I will propose an array of eight frame-semantic formalizations. Building on the analysis of the base semantics, I will then move on to the psych nominalizations (\sectref{sec:psy-output}). I will identify and model possible readings, and thereby also determine which of the eight frame analyses are backed up by my nominalization data. For a summary of my results, see \sectref{sec:psy-output-sum}.

The psych neologisms which form the basis of this chapter are, in alphabetical order: \textit{abashment, affrightment, annoyment, approvement, bemusement, bumfuzzlement, confoundment, convincement, disheartenment, dumbfoundment, endullment, enragement, enrapturement, musement, nonplusment, perturbment, reassurement, soothement, staggerment, upliftment, upsetment,} and \textit{worriment}.  

\section{The semantics of psych verbs} 
\label{sec:psy-input}

I will present the semantics of psych verbs in two steps:
First, in \sectref{sec:psy-input-prev}, I will describe the semantic ingredients which have been proposed in formal and informal accounts of psych verbs. These existing analyses in combination with the insights from \chapref{ch:cos} will then form the basis for the frame formalizations I propose for psych verbs in \sectref{sec:psy-input-frames}. There is much disagreement in the literature on psych verbs, and the different points of view will be represented by eight alternative frame analyses.

\subsection{Semantic ingredients} 
\label{sec:psy-input-prev}

The label \textit{psych verbs} is commonly assigned to verbs describing an emotional or psychological event or state, such as \textit{frighten} or \textit{fear}. Psych verbs have been widely discussed ever since \citeauthor{Belletti.1988}'s (\citeyear{Belletti.1988}) seminal article on the syntactic peculiarities of Italian psych verbs. Research in this vein is mostly focused on their unusual properties with regard to argument realization. For my purposes, however, a close look at their semantic properties is required. There are four topics which are relevant for a formalization of psych verbs: the distinction between different kinds of psych verbs based on semantic and syntactic criteria (\sectref{sec:psy-input-prev-SEvsOE}), the participants they take (\sectref{sec:psy-input-prev-part}), the notion of implicit causality (\sectref{sec:psy-input-prev-impl}), as well as event structure and causativity (\sectref{sec:psy-input-prev-caus}).

\subsubsection{Subclasses of psych verbs}
\label{sec:psy-input-prev-SEvsOE}

Psych verbs typically take an \textsc{experiencer} and a \textsc{stimulus} argument (see \citealt[189]{Levin.1993}). The experiencer is an animate participant who perceives or feels the event, while the stimulus is the participant evoking it. Often, English psych verbs are divided into two subclasses, depending on the syntactic realization of \textsc{experiencer} in a sentence.\footnote{In languages which are morphologically richer than English, this subdivision is traditionally based on case assignment, see e.g. \citet{Klein.2005} for German and \citet{Belletti.1988}; \citet{Varchetta.2010} for Italian.} If it is realized as the subject, they are called \HighlightText{subject experiencer psych verbs} (henceforth \HighlightText{SE psych verbs}), and if it is realized as the direct object, they are referred to as \HighlightText{object experiencer psych verbs} (henceforth \HighlightText{OE psych verbs}). Correspondingly, \textsc{stimulus} takes the other slot, respectively. Examples are given in (\ref{ex:SEvsOE}).

\begin{exe}
	\ex \label{ex:SEvsOE}
	\begin{xlist}
		\item \label{ex:SEvsOE1} SE: Charlie \textbf{fears} Jael.
		\item \label{ex:SEvsOE2} OE: Jael \textbf{frightens} Charlie.
	\end{xlist}
\end{exe}

\noindent A more fine-grained classification of psych verbs is employed by \citet{Levin.1993} and VerbNet.
Based on two criteria, \citet[188--193]{Levin.1993} distinguishes four subtypes of psych verbs (see Table \ref{tab:psych}). 
First, she follows the traditional split between SE and OE psych verbs. 
Then, she adds the dimension of transitivity, distinguishing between transitive verbs and intransitive verbs with a prepositional phrase complement. 

In my data set, most nominalizations have transitive OE psych verb bases, while three bases have been categorized as intransitive SE psych verbs (see Table \ref{tab:psych-data}). Thus, two of the Levin/VerbNet classes are represented, namely amuse verbs and marvel verbs.

\begin{table} 
    \caption[Types of psych verbs]{\label{tab:psych}Types of psych verbs according to \citet[188--193]{Levin.1993}} 
    \begin{tabular}{
        >{\RaggedRight\arraybackslash}p{0.24\textwidth}
        >{\RaggedRight\arraybackslash}p{0.33\textwidth}
        >{\RaggedRight\arraybackslash}p{0.33\textwidth}
      }
      \lsptoprule
      & SE
      & OE\\ 
      \midrule
      Transitive
      & admire verbs 
      & amuse verbs \\ 
      & \textit{The tourists admired the paintings.} & \textit{The clown amused the children.} \\ 
      \addlinespace
      Intransitive with 
      & marvel verbs 
      & appeal verbs \\
      PP complement & \textit{Megan marveled at the beauty of the Grand Canyon.}  
      & \textit{This painting appeals to Malinda.} \\
      \lspbottomrule
    \end{tabular}
\end{table}

\begin{table} 
    \caption[Properties of the psych verbs in my data set]{\label{tab:psych-data} Properties of the psych verbs in my data set, based on the classification by \citet[188--193]{Levin.1993}} 
    \begin{tabular}{
        >{\RaggedRight\arraybackslash}p{0.24\textwidth}
        >{\RaggedRight\arraybackslash}p{0.26\textwidth}
        >{\RaggedRight\arraybackslash}p{0.40\textwidth}
      }
      \lsptoprule
      & SE
      & OE \\ 
      \midrule
      Transitive 
      & admire verbs 
      & amuse verbs ($n=20$) \\ 
      && \textit{abash, affright, annoy, bemuse, bumfuzzle, confound, convince, dishearten, dumbfound, endull, enrage, enrapture, nonplus, perturb, reassure, soothe, stagger, uplift, upset, worry} \\ 
      \addlinespace
      Intransitive with 
      & marvel verbs ($n=3$) 
      & appeal verbs \\
      PP complement & \textit{approve of, muse over, worry about} &\\
      \lspbottomrule
    \end{tabular}
\end{table}

Amuse verbs denote ``the bringing about of a change in psychological or emotional state'' (\citealt[191]{Levin.1993}), while marvel verbs describe mental states (p. 193). 
Some psych verbs are found in transitive/intransitive pairs, which are then cross-listed as amuse verbs and as marvel verbs. Examples include \textit{cheer, gladden,} and \textit{thrill}, as well as, from my data set, \textit{worry}. These participate in causative alternations, as is exemplified in (\ref{ex:worry-causalt}). 

\begin{exe}
  \ex \label{ex:worry-causalt}
  \begin{xlist}
    \item Bill \textbf{worried} about the article. 
    \item The article \textbf{worried} Bill. 
  \end{xlist}
\end{exe}

\noindent \citet[191]{Levin.1993} mentions that further subdivisions based on agentivity may be sensible in the categorization of psych verbs. Agentivity here relates to the amount of control which the \textsc{stimulus} has over the event. Consider the sentences in (\ref{ex:agentive-nonagentive}) for illustration (from \citealt[11]{DiDesidero.1993}). Here, (\ref{ex:nonagentive}) is regarded as \HighlightText{non-agentive}, while (\ref{ex:agentive}) is generally interpreted as \HighlightText{agentive}.

\begin{exe}
	\ex \label{ex:agentive-nonagentive}
	\begin{xlist}
		\item \label{ex:nonagentive} The mask frightened the children. 
		\item \label{ex:agentive} The man frightened the children. 
	\end{xlist}
\end{exe}

\noindent The literature typically uses semantic tests to distinguish between agentive verbs (e.g. \textit{encourage}), non\-/agentive verbs (e.g. \textit{inspire}), and verbs which have both agentive and non-agentive senses (e.g. \textit{frighten}).
As with any such categorization, this threefold distinction is not as straightforward as it may seem.  
First of all, the contexts which are usually thought to indicate agentivity produce ambiguous results for a number of psych verbs. For example, the verb \textit{interest} works in the \textit{subjective adverb} test in (\ref{ex:agentive-test1a}), but not in the \textit{persuade} test in (\ref{ex:agentive-test1b}) (\citealt[73]{Martin.2013}).

\begin{exe}
	\ex \label{ex:agentive-test1}
	\begin{xlist}
		\ex[]{\label{ex:agentive-test1a}Paul \textbf{cleverly interested} Mattel in the toy.}
		\ex[*]{\label{ex:agentive-test1b}I \textbf{persuaded} Paul \textbf{to interest} Mattel in the toy.}
	\end{xlist}
\end{exe}

\noindent The verbs which show this mixed behavior are traditionally considered non-agentive, but since they are compatible with some agentive constructions, the author dubs them ``weakly agentive'' (p. 72).
What is clear is that semantic fuzziness is a salient problem with regard to the decision of whether a given (psych) verb is agentive, or non-agentive, or whether it can instantiate both categories.\footnote{See also \citet{Huyghe.2020} for a critical discussion of how the \textsc{agent} category is applied in semantic analyses.}

I have decided to test my psych base verbs for agentivity for two reasons. First, it is directly relevant for their frame formalization. Thus, it has been claimed that only agentive (variants of) OE psych verbs are causatives (see e.g. \citealt{DiDesidero.1993,Kailuweit.2005}). Agentive and non-agentive psych verbs would thus be represented by fundamentally different frames. 
Second, \citet{Grimshaw.1990} claims that only agentive variants of psych verbs are causatives and can therefore produce transpositional readings. This, of course, is imminently relevant for my assessment of psych nominalization readings. 

Due to the fuzziness of the labels \textit{agentive} and \textit{non-agentive} delineated above, I have decided to apply a gradable notion of agentivity. 
More precisely, I have calculated agentivity scores based on four tests, following standard diagnostics originally proposed by \citet{Lakoff.1966}:\footnote{\citet{Lakoff.1966} actually introduces diagnostics for stativity, but finds that some of his tests also indicate agentivity (p. $\mathrm{I}$-13).} 

\begin{enumerate}
\item admissibility of an imperative (\textit{Upset him!})
\item \label{ex:persuade} contexts with \textit{persuade} (\textit{She persuaded him to upset the children}.)
\item \label{ex:agent} contexts with agent-oriented adverbs 
\begin{enumerate}
\item \textit{reluctantly} (\textit{She reluctantly upset the children}.)
\item \textit{deliberately} (\textit{She deliberately upset the children.})
\end{enumerate}
\end{enumerate}

\noindent The contexts in (\ref{ex:persuade}) and (\ref{ex:agent}) were looked up in \acs{iWeb}, \acs{GloWbE}, \acs{COCA} and Google.\footnote{The queries were 〈persuaded him to V〉, 〈reluctantly Ved〉, and 〈deliberately Ved〉.} For the imperative construction, the search engine Symbolhound was used, allowing searches for special characters like exclamation points.  
Points were assigned as follows: 0 points were assigned if a given construction was not attested, 1 if it was attested, and 0.5 if it was attested only once. 
Since none of the base verbs could be found in the imperative construction, this test was supplemented by judgments of two native speakers, one of which is a trained linguist.
Here, 0 points were assigned if a sentence was judged unacceptable, 1 if it was perceived as well-formed, and 0.5 if the informant was unsure. 
Finally, all points were added together, resulting in a total possible agentivity score between 0 and 6. The results are given in Table \ref{tab:psych-agent}.

\begin{table}
		\caption{\label{tab:psych-agent}Agentivity scores of OE psych verbs} 
		\begin{tabular} {lr}
 			\lsptoprule
			Verbs & \multicolumn{1}{l}{Agentivity score} \\
			\midrule
			convince, reassure		  	& 5\phantom{.5} \\
			upset					  	& 4\phantom{.5} \\
			annoy, soothe				& 3.5 \\
			confound, enrage, uplift   	& 3\phantom{.5}             \\
			bemuse, perturb, worry		& 2\phantom{.5} \\
			dishearten, enrapture       & 1.5              \\
			bumfuzzle, dumbfound        & 1\phantom{.5}                \\
			abash, affright, stagger    & 0.5              \\
			endull, nonplus		        & 0\phantom{.5}              \\
			\lspbottomrule
		\end{tabular}
\end{table} 

However, there are two caveats to these agentivity scores: 
First, the tests which were applied do not disambiguate between different verb senses. The high score for \textit{convince}, for instance, may relate to its sense as a \textit{force verb} (\textit{convince someone to do something}). 
Second, some base verbs in my data set are rare, so the non-acceptability of contexts may be due to the informants not having a verb in their active vocabulary. An example of this is \textit{abash}, which both informants stated they only use in its adjectival form \textit{abashed}.
For these reasons, we will need to take the predictive power of these scores with a grain of salt.

That said, the agentivity scores should indicate whether an OE psych verb has a causative event structure, and whether we can thus expect its nominalization in a transpositional reading, as claimed by \citet{Grimshaw.1990}. More precisely, we can expect verbs higher up in the table (e.g. \textit{convince, upset,} or \textit{annoy}) to have a causative event structure, and to produce transpositional event nouns. Contrarily, verbs further down in the table (e.g. \textit{endull, abash,} or \textit{bumfuzzle}) should have a non-causative event structure, and their nominalizations should not be able to exhibit transpositional readings. I will come back to this issue when discussing transpositional readings of psych nouns in \sectref{sec:psy-output-survey-transOE}. 

\subsubsection{The participants of psych events}
\label{sec:psy-input-prev-part}

I have already introduced the two most frequently mentioned participants of psych verbs: \textsc{experiencer} and \textsc{stimulus}. While the \textsc{experiencer} category seems to be uncontroversial, there is some discussion about \textsc{stimulus}. Let us first have a look at subtypes of this participant. 
One point of view is that the \textsc{stimuli} in (\ref{ex:SEvsOE-rep}) (repeated from (\ref{ex:SEvsOE})) are semantically identical and merely surface in different syntactic positions (see e.g. \citealt{Belletti.1988, Grimshaw.1990, Levin.1993}):\largerpage

\begin{exe}
	\ex \label{ex:SEvsOE-rep}
	\begin{xlist}
		\item \label{ex:SEvsOE1-rep} SE: Charlie \textbf{fears} Jael.
		\item \label{ex:SEvsOE2-rep} OE: Jael \textbf{frightens} Charlie.
	\end{xlist}
\end{exe}

\noindent Other authors have identified semantic details in different instantiations of \textsc{stimulus} which are not captured by this label. 
For instance, \citet[56--57]{Pesetsky.1995} distinguishes between the subcategories \textsc{causer, target} and \textsc{subject matter} (see e.g. \citealt{Hartl.2001} for further evidence):\footnote{In (\ref{ex:pesetsky2}) I have slightly altered the original examples to streamline my account: \textit{Bill} is originally \textit{John} and \textit{article} is originally \textit{television set}.}  

\begin{exe}
	\ex \label{ex:pesetsky1}
	\begin{xlist}
		\item \label{ex:target} Bill was very angry at \textbf{the article}. (=\textsc{target})
		\item \label{ex:causer1} \textbf{The article} angered/enraged Bill. (=\textsc{causer})
	\end{xlist}
	\ex \label{ex:pesetsky2}
	\begin{xlist}
		\item \label{ex:subjectmatter} Bill worried about the \textbf{article}. (=\textsc{subject matter})
		\item \label{ex:causer2} \textbf{The article} worried Bill. (=\textsc{causer})
	\end{xlist}
\end{exe}

\noindent Let us first distinguish \textsc{causer} from the other two categories. In short, \textsc{causer} is more loosely connected to the experienced emotion than the other two. 
To motivate this distinction, Pesetsky explains that the truth conditions of the (a) sentences differ substantially from those of the (b) sentences: 
For (\ref{ex:target}) and (\ref{ex:subjectmatter}) to be true,  Bill's emotions must be directed at the article itself, or some aspect of it. For instance, he may be angry at the writing style because he expected better from his favorite columnist, and he may be worried about how the article will be received by the audience. 
For (\ref{ex:causer1}) and (\ref{ex:causer2}) to be true, on the other hand, it is not necessary that Bill be angry at/worried about the article itself. Instead, Bill may be angry at the government because the article revealed a political scandal, and he may be worried because the article reports on critical flaws in the car he is driving. In other words, in the (b) sentences, the article causes Bill to feel an emotion which is directed at something or someone else.
An additional observation concerning the distinction between \textsc{causer} and \textsc{target/subject matter} is that \textsc{causer} is always realized as the subject of the sentence, while \textsc{target} and \textsc{subject matter} are realized as the object (\citealt[56]{Pesetsky.1995}).

Now, we can tease apart \textsc{target} and \textsc{subject matter}. They are distinguished based on the presence or absence of an evaluation on the part of the experiencer. \citet[56]{Pesetsky.1995} explains that in (\ref{ex:target}), Bill must have assessed the article and concluded that he dislikes some aspect of it; the article is the \textsc{target} of emotion. A \textsc{subject matter}, on the other hand, does not require evaluation.

\begin{sloppypar}
Another terminological distinction within the \textsc{stimulus} category is often made in connection with agentivity (see \sectref{sec:psy-input-prev-SEvsOE}). Depending on the amount of control the \textsc{stimulus} has over the event, it is referred to as \textsc{causer} or \textsc{agent}, respectively. For the examples we have already seen in (\ref{ex:agentive-nonagentive}), this means that \textit{the mask} is interpreted as a causer, while \textit{the man} is usually interpreted as an agent:  
\end{sloppypar}

\begin{exe}
	\ex \label{ex:agentive-nonagentive-rep}
	\begin{xlist}
		\item \label{ex:nonagentive-rep} \textbf{The mask} frightened the children.
		\item \label{ex:agentive-rep} \textbf{The man} frightened the children.
	\end{xlist}
\end{exe}

\noindent Let us now look at what kinds of things in the world can actually be a \textsc{stimulus}. This participant is instantiated by the following three categories (my labeling; examples from \citealt[77, 190]{Levin.1993}): agentive entities (e.g. \textit{the man}), non-agentive entities (e.g. \textit{the mask}), properties (e.g. \textit{Mark's single-mindedness}), and events (e.g. \textit{the clown's antics}).

According to \citet[77]{Levin.1993}, the two NPs \textit{Mark's single-mindedness} and \textit{the clown's antics} contain a possessor (\textit{Mark}/\textit{the clown}), and an attribute or an action of this possessor (\textit{single-mindedness}/\textit{antics}). This attribute/action is what causes the psych-state. In the following example sentences I have marked the stimulus by bold print and italicized the respective possessor: 

\begin{exe}
  \ex \label{ex:Levin-poss1}
  \begin{xlist}
    \item \textbf{\ExHighlight{Mark's} single-mindedness} terrified me.   
    \item \textbf{\ExHighlight{The clown's} antics} amused the children. 
  \end{xlist}
\end{exe}
  
\noindent The combination of possessor and attribute/action can be expressed not only by a single NP, but also by two distinct constituents, namely a subject (\textit{Mark/the clown}) and a with-PP (\textit{with his single-mindedness/with his antics}):

\begin{exe}
  \ex \label{ex:Levin-poss2}
  \begin{xlist}
    \item \label{ex:Levin-poss2a}\ExHighlight{Mark} terrified me \textbf{with his single-mindedness}.
    \item \label{ex:Levin-poss2b} \ExHighlight{The clown} amused the children \textbf{with his antics}.
  \end{xlist}
\end{exe}

\noindent Besides \textsc{stimulus} and \textsc{experiencer}, VerbNet uses a third participant which is relevant in the description of my data set: \textsc{secondary result}.\footnote{In VerbNet, this role is called \textsc{result}. However, this term only refers to secondary results in the verb classes discussed in this book, and primary results have played a role in my analysis of COS nominalization. A doubling of terminology would be confusing, so that I have decided to relabel the VerbNet role.}\textsuperscript{,}\footnote{A fourth participant, \textsc{attribute}, is used in the description of admire verbs, which are not represented in my data set. It is defined as a ``[c]ircumstance that is a property of an entity or entities, as opposed to the entity itself'' (\citealt[318]{Palmer.2017}), as for example in \textit{I admired him \textbf{for his honesty}}.} 
More precisely, some amuse verbs allow the addition of a resultative construction, as exemplified in (\ref{ex:secresult}) (from \citealt[190]{Levin.1993}, my emphasis).\footnote{Notably, the name patron of the amuse verb subclass does not allow a \textsc{secondary result} participant. Instead, \citet{Levin.1993} as well as the authors of VerbNet use \textit{bore} for illustration (\textit{That movie bored me silly}). Moreover, \citet{Levin.1993} does not tag the presence of a resultative phrase as an optional property of amuse verbs, although it clearly is.} Such resultative constructions further specify the result-state which is already included in the verb's semantics. Here, \textit{bore} includes a result-state \textit{bored}, which is further specified as \textit{bored silly}.

\begin{exe}
  \ex \label{ex:secresult} That movie bored me \textbf{silly}.
\end{exe} 

\noindent I have already discussed the \textsc{secondary result} participant in some detail in the previous chapter (\sectref{sec:cos-VN-c/igeneral}), and its usage with amuse verbs does not come with any surprises. Importantly, secondary result predicates are not imminently relevant for my study: Since they are modifiers which act on the frame of a verb, their semantics cannot be accessed by an affix.



\subsubsection{Implicit causality}
\label{sec:psy-input-prev-impl}

Another central notion in the research on psych verbs is that of \HighlightText{implicit causality} (see \citealt[90--92]{Kailuweit.2005} for a summary of the origins of this concept). I discuss it here for two reasons: First, it needs to be disentangled from the \textsc{stimulus} category. Importantly, the two notions most often coincide, but they are not identical. Moreover, although both OE and SE psych verbs have been found to be implicitly causative (\citealt{Hartl.1999,Hartl.2001}), this does not necessarily mean that they also have a causative event structure (see \sectref{sec:psy-input-prev-caus}).

The basic idea behind implicit causality is that speakers have intuitions as to who is responsible\footnote{While the term \textit{cause} (and related terminology) is applied in the literature on implicit causality, in this section I am using the term \textit{responsibility} instead, in order to avoid confusion with the frame attribute \textsc{cause}.} for an event. Psych verbs are thought to linguistically behave in a way which reflects these intuitions (as does the whole range of interpersonal verbs, see \citealt{Hartl.2001}). In this context, different syntactic and semantic phenomena have been discussed (see e.g. \citealt{Hartshorne.2013,Hartshorne.2014} for an overview).
I will use speaker bias in pronoun resolution for illustration.

Studies have shown that, in ambiguous contexts such as in (\ref{ex:bias}), speakers have a bias as to who is responsible for an event, namely \textit{Mary} in (\ref{ex:bias1}), and \textit{Sally} in (\ref{ex:bias2}) (see \citealt{Hartshorne.2013,Hartshorne.2014}). 
 
\begin{exe}
	\ex \label{ex:bias}
	{\begin{xlist}
		\item \label{ex:bias1} Sally fears \textbf{Mary} because \textbf{she} is strange. 
		\item \label{ex:bias2} \textbf{Sally} frightens Mary because \textbf{she} is strange.
	\end{xlist}
	}
\end{exe}

\noindent Although this phenomenon was first observed almost 50 years ago by \citet{Garvey.1974}, it remains unclear whether implicit causality is primarily a linguistic or a cognitive phenomenon, and among proponents of either position there are many different proposals (see \citealt{Hartshorne.2014} for an overview). 

How is implicit causality related to the stimulus of an event? Most often, the two categories coincide, as in (\ref{ex:bias}). Here, speakers tend to regard the stimulus (\textit{Mary} and \textit{Sally}, respectively) as responsible for the event. It is also possible, however, to modify the context so that responsibility is assigned to the experiencer:

\begin{exe}[(123)]
	\exp{ex:bias}
	{\begin{xlist}
		\item \label{ex:bias1-mod} \textbf{Sally} fears Mary because \textbf{Sally} is strange. 
		\item \label{ex:bias2-mod} Sally frightens \textbf{Mary} because \textbf{Mary} is strange.
	\end{xlist}
	}
\end{exe}

\noindent In these examples, it is still the stimulus which makes the experiencer feel afraid, but the experiencer is ultimately considered responsible, for example because the stimulus is objectively speaking not scary. Importantly, \textit{Mary} is still the stimulus in the (a) sentence, and \textit{Sally} is still the stimulus in the (b) sentence, no matter who is considered to be ultimately responsible.

\subsubsection{Event structure and causativity}
\label{sec:psy-input-prev-caus}

In the previous chapter, we saw that event structure and causativity are central in the formalization of COS verbs. As is virtually undisputed in the existing literature, I modeled the members of this verb class as complex events with two subevents, \textsc{cause} and \textsc{effect}. With regard to the event structure of psych verbs, things are not as straightforward. The one issue that authors seem to agree about is that OE psych verbs standardly denote complex events \textendash{} and even in this respect, counter-examples have been claimed (e.g. German \textit{\"argern} `annoy,' see \citealt{Hirsch.2018}). SE psych verbs, on the other hand, are most often modeled as simple events, but there are also some accounts which interpret them as complex events. 
In the related discussion of causativity of OE psych verbs, almost any imaginable proposal can be found: Are they causatives? Non-causatives? Causatives under certain conditions? A special kind of causatives? Similar questions are asked about SE psych verbs by those who believe that they do denote complex events.	
In the following, I will only give a rough overview of the different existing proposals. Rather than evaluating their feasibility at this point, I will do so in hindsight and in view of my data. Thus, I will first suggest frames modeling the different points of view (\sectref{sec:psy-input-frames}), and then use the nominalization semantics of my data as a contribution to the discussion of which decompositions make the most sense. 
 
\subsubsubsection{States and caused states}
As I have mentioned, the most common assumption is that OE psych verbs denote causatives and are therefore complex events (see e.g. \citealt{Grimshaw.1990, Pustejovsky.1991, DiDesidero.1993, Pesetsky.1995, VanValinJr.1997, VanValin.2005, Martin.2013}; VerbNet). SE psych verbs, on the other hand, are usually thought to denote non-causative, stative, simple events (see e.g. \citealt{Grimshaw.1990,DiDesidero.1993,VanValinJr.1997,Levin.2006b}).\footnote{Note that this dichotomy may be regarded as a simplification. Causativity is a continuous property, with verbs \textendash{} or verb variants \textendash{} being located somewhere on a spectrum between non-causative and causative. See for instance \citet{Kailuweit.2005} for a more fine-grained classification of some French psych verbs.}
I will use \citeauthor{VanValin.2005}'s (\citeyear{VanValin.2005}) notation as an example of a corresponding formalization.  
The author models \textit{fear} as a state (example (\ref{ex:SE-Van})), and \textit{scare} as its causative counterpart (example (\ref{ex:OE-Van})).\footnote{The lexical entry modeled in \citet[66]{VanValin.2005} is actually \textit{be afraid}, but I assume that the SE psych verb \textit{fear} is modeled identically.}
Correspondingly, \textit{fear} is modeled with the predicate \textbf{feel$'$}, which expresses an internal experience (p. 55). It has two arguments, an experiencer (boy) and a sensation (\textbf{afraid$'$}).
In (\ref{ex:OE-Van}), \textbf{feel$'$} is embedded in the complex event structure. The causing event is an unspecified action\footnote{\citet{VanValin.2005} uses the term \textit{activity}.}, which is expressed by \textbf{do$'$}. Both subevents are connected by the operator-connective CAUSE.

	\begin{exe}
		\ex \label{ex:SE-Van} 
		\begin{xlist}
			\ex The boy feared the dog.
			\ex {[ \textbf{feel$'$} (boy, [ \textbf{afraid$'$} (dog) ] ) ]}
		\end{xlist} 
	\end{exe}
	\begin{exe}
		\ex \label{ex:OE-Van} 
		\begin{xlist}
			\ex The dog scared the boy. 
			\ex {[ \textbf{do$'$}(dog, $\emptyset$) ] CAUSE [ \textbf{feel$'$} (boy, [ \textbf{afraid$'$} ] ) ]}
		\end{xlist}
	\end{exe}

\noindent Two things are notable in examples (\ref{ex:SE-Van}) and (\ref{ex:OE-Van}).
First, \textit{dog} is not part of the second subevent in (\ref{ex:OE-Van}). This resembles the distinction between causer and subject matter (see \sectref{sec:psy-input-prev-part}): The dog causes the boy to be afraid, but it might not be the subject matter of fear (see also \citealt[38]{VanValin.2005}). 
Second, the cause is not the dog itself, but something it does. In \citet[]{Pustejovsky.1995}, this is called a ``metonymic reconstruction of the subject to an event'' (p. 209). 

Why does this first group of approaches assume that the second subevent of an OE psych verb is a state, and not a change-of-state? An explanation is given by \citet{Alexiadou.2014}. Discussing alternating SE/OE psych verb pairs such as \textit{worry/worry about}, they state that English psych verbs do not contain a change-of-state subevent because they are incompatible with \textit{in-}adverbials (p. 72):

\begin{exe}
	\ex
	\begin{xlist}
	\item John worried about the television set for/*in an hour.
	\item The television set worried John for/\textsuperscript{??}in an hour.
	\end{xlist}
\end{exe} 

\noindent They argue that existing transitive/intransitive verb pairs such as \textit{worry} and \textit{worry about} do not actually participate in the causative-inchoative alternation, and should be interpreted as idiosyncratic alternations instead (p. 54).\footnote{Further examples listed are \textit{grieve/grieve over, puzzle/puzzle over} and \textit{delight/delight in}.} The authors attribute the lack of this alternation in English psych verbs to diachronic developments (p. 75). It used to be more common, but only a small number of psych verbs with causative and inchoative variants have survived into present times, including \textit{sadden/sadden at, madden/\textsuperscript{?}madden at, weary/weary of} and \textit{thrill/\textsuperscript{?}thrill at}. From their argumentation, it can be concluded that they would model these relics from earlier stages of English with a change-of-state subevent. I will come back to this diagnostic in relation to my data set below (p. \pageref{enum:VanDiagnostics}).

\subsubsubsection{Experienced causation}
Many authors who assume that OE psych verbs are causatives distinguish them from non-psych causatives in some way (see e.g. \citealt{Asher.2000,Geuder.2000,Grimshaw.1990,Pustejovsky.1995}).
Famously, \citet{Pustejovsky.1995} distinguishes between \textit{direct} and \textit{experienced causation}.\footnote{For English, I have only found accounts which relate experienced causation to OE psych verbs. \citet{Nam.2009} proposes event templates to model the Korean SE predicates \textit{cilwuha-} `bored/boring' and \textit{komap-} `thankful' as subtypes of experienced causation.} In order to illustrate the differences between the two, I am giving his lexical conceptual paradigms (lcps) for the verbs \textit{kill} (Figure \ref{fig:lcp-kill}) and \textit{anger} (Figure \ref{fig:lcp-anger}). I will go through the differences from top to bottom. For better readability, I have marked the relevant spots green. 

An lcp consists of three levels of representation (ibid., p. 61).\largerpage
In the \HighlightText{event structure}, the event type of the lexical item is defined by stating its subevents and relating them to one another. In both types of causation, the author posits an event structure with a causing process (e\textsubscript{1}: process) which leads to a state (e\textsubscript{2}: state). What is different is the temporal relation between the subevents. For default causatives, he assumes that the process precedes the state ($<_\propto$). For experienced causation, in addition to that, the experiencing process overlaps with the resulting state ($<\circ_\propto$).\footnote{\citet{Geuder.2000} makes a slightly different point, stating that psychological causation requires concomitance of cause and effect, not precedence, while default causatives are neutral in this respect (p. 195--196).} 

\begin{figure}
		\begin{avm}
			\[
			\textbf{kill} \cr
			\textsc{eventstr} & = 
				\[ 
				\textsc{e}\textsubscript{1} = \textbf{e\textsubscript{1}:process}\cr
				\textsc{e}\textsubscript{2} = \textbf{e\textsubscript{2}:state}\cr
				\textsc{restr} = $\textcolor{lsRichGreen}{<_\propto}$ \cr
				\textsc{head} = \textbf{e}\textsubscript{1}
				\] \cr
			\textsc{argstr} & = 
				\[
				\textsc{arg1} = \@1 \[ \textcolor{lsRichGreen}{\textbf{top}}\] \cr
				\textsc{arg2} = \@2 \[ \textbf{animate\_ind} \cr \textsc{formal} = \textbf{physobj}\]
				\] \cr
			\textsc{qualia} & = 
			\[ \textbf{direct-causation\_lcp} \cr
			\textsc{formal} = \textbf{\textcolor{lsRichGreen}{dead}(e\textsubscript{2},}\@2) \cr
			\textsc{agentive} = \textbf{\textcolor{lsRichGreen}{kill}\_act(e\textsubscript{1},}\textcolor{lsRichGreen}{\@1},\@2,...)
			\]
			\] 
		\end{avm}
		\caption[Lcp for the verb \textit{kill}]{Lcp for the verb \textit{kill} (from \citealt[208]{Pustejovsky.1995})}
		\label{fig:lcp-kill}
\end{figure}

\begin{figure}
		\begin{avm}
			\[
			\textbf{anger} \cr
			\textsc{eventstr} & = 
				\[ 
				\textsc{e}\textsubscript{1} = \textbf{e\textsubscript{1}:process}\cr
				\textsc{e}\textsubscript{2} = \textbf{e\textsubscript{2}:state}\cr
				\textsc{restr} = $\textcolor{lsRichGreen}{<\circ_\propto}$ \cr
				\textsc{head} = \textbf{e}\textsubscript{1}
				\] \cr
			\textsc{argstr} &  = 
				\[
				\textsc{arg1} = \@1 \[ \textcolor{lsRichGreen}{\< \@2, \<e\textsubscript{1},\textit{t}\>\>}\] \cr
				\textsc{arg2} = \@2 \[ \textbf{animate\_ind} \cr \textsc{formal} = \textbf{physobj}\]
				\] \cr
			\textsc{qualia} & = 
			\[ \textbf{experiencer\_lcp} \cr
			\textsc{formal} = \textbf{\textcolor{lsRichGreen}{angry}(e\textsubscript{2},}\@2) \cr
			\textsc{agentive} = \textbf{\textcolor{lsRichGreen}{exp}\_act(e\textsubscript{1},}\textcolor{lsRichGreen}{\@2})
			\]
			\] 
		\end{avm}
		\caption[Lcp for the verb \textit{anger}]{Lcp for the verb \textit{anger} (from \citealt[211]{Pustejovsky.1995})}
		\label{fig:lcp-anger}
\end{figure}

In the \HighlightText{argument structure}, the verbs' arguments and their syntactic realization are specified. Here, the first argument differs. In direct causation, `top' stands for the most general possible type (typically expressed as {\sffamily{T}} in frame theory). In experienced causation, argument 1 is the event e\textsubscript{1}, in which argument 2, the experiencer, is involved. This event is an \HighlightText{experiencer process}. That is, the causing event is one in which the experiencer directly perceives something related to the stimulus (p. 210). This perception and its cognitive processing may happen subconsciously (\citealt[197]{Geuder.2000}).
If argument 1 is thus necessarily an event, contexts like \textit{Jael frightens Charlie} require metonymic reconstruction (\citealt[209]{Pustejovsky.1995}). More precisely, the stimulus \textit{Jael} is reconstructed into an event, for instance Charlie seeing Jael's scarred face. 

The third level of representation is the \HighlightText{qualia structure}, which specifies four aspects of a lexical item's meaning and relates it to other lexical items. Two of these aspects are relevant for direct and experienced causation: \textsc{formal} distinguishes the lexical item within a larger domain, and \textsc{agentive} specifies its origin.\footnote{Two aspects are not included here, namely \textsc{constitutive} (how an object and its constitutive parts are related) and \textsc{telic} (the lexical item's purpose and function).} For \textit{kill}, \citet{Pustejovsky.1995} states that the state e\textsubscript{2} of argument 2 must be `dead.' This is brought about by e\textsubscript{1}, a killing action in which both arguments are involved. For \textit{anger}, the resulting state is `angry,' and the agentive quale specifies the origin of this state as the experiencer process e\textsubscript{1}. 

\subsubsubsection{Agentives are causative}
Not all researchers believe that OE psych verbs are always causatives. More precisely, there is some disagreement regarding the influence of agentivity on causativity. Researchers like \citet[11--12]{DiDesidero.1993} or \citet{Kailuweit.2005} state that a difference in agentivity also seems to reflect a difference in event type. Thus, a sentence like \textit{The mask frightened the children} describes a reaction event with a simple event structure. A sentence like \textit{The man frightened the children}, on the other hand, presupposes that the subject has performed some action. This sentence thus has (at least) two subevents. Elsewhere, however, it has been found that both agentive and non-agentive (variants of) psych verbs are best modeled as causatives (\citealt[71]{Martin.2013}).  

Finally, it has been claimed that agentive-causative OE psych verbs differ with regard to the salience of the subevents: \citet[13, 17--18]{DiDesidero.1993} finds linguistic evidence that, depending on the verb, either the first or the second subevent is salient. The author states that, for verbs like \textit{frighten} and \textit{amuse}, the causing action is salient: 
The \textsc{agent} acts intentionally, volitionally, and with control to elicit a certain reaction \textendash{} they choose to cause it, and the verb meaning contains a high probability that the reaction is indeed going to happen. 
For verbs like \textit{amaze} and \textit{delight}, on the other hand, the reaction of the experiencer is salient. 
Here, an agent can only intend to elicit a certain reaction, but they cannot choose to. 

\subsubsubsection{Complex event structure}
So far, I have only referred to approaches which assume a complex event structure for OE psych verbs and a simple one for SE psych verbs. A rather different approach also exists, stating that both verb classes have a complex event structure. A proponent of this view is \citet{Tantos.2006}, who argues that a unified account for all psych verbs is desirable,
giving lexical evidence from Greek (where the equivalents of \textit{fear} and \textit{frighten} share the same verb stem, p. 127--128). For both SE and OE psych verbs, he assumes a causing subevent and a caused subevent, the difference being that SE psych verbs highlight the result-state while the cause fades into the background (p. 129). Formally, this is expressed by two juxtaposed rhetorical connections:\footnote{The author bases his analysis on work by \citet{Asher.2000}. In their formalization, rhetorical relations are added to lexical entries in order to tackle the connection between the discourse level and the lexical level.} \textit{result} for OE psych verbs, and \textit{explanation} for SE psych verbs.
They differ in that the two subevents are switched:\footnote{This notation has been simplified for expository purposes. In the original, the author follows the representation format used in \citet{Asher.2000}.} 
	
\begin{exe}
	\ex
	\begin{xlist}
		\item Result: e1 = Exp(e, x, y), e2 = afraid(e1, y)
		\item Explanation: e1 = afraid(e1, y), e2 = Exp(e, x, y) 
	\end{xlist}
\end{exe}	
	
\noindent Importantly, \citet{Tantos.2006} questions the assumption that psych verbs constitute a variant of causative verbs. Instead, he assumes some sort of ```weaker' notion of impact'' (p. 123) which leads up to a mental state. 

A similar approach is found in VerbNet, where the semantics of amuse verbs and of marvel verbs are modeled as follows:\footnote{See \citet{Kipper.2005} for a documentation of the semantic predicates used in VerbNet.}

\begin{exe}\sloppy\raggedright
    \ex \label{ex:VNeventamuse} The clown amused the children. \\ 
    \textsc{cause(Stimulus, E) emotional\_state(result(E), E\-motion, Experien\-cer)}
    \ex \label{ex:VNeventmarvel} 
     Megan marveled at the Grand Canyon. \\ 
    \textsc{emotional\_state(result(E), \mbox{Emotion}, Experiencer) \mbox{in\_reaction\_to(}E, Stimulus)}
\end{exe} 

\noindent The semantic decomposition in (\ref{ex:VNeventamuse}) contains two predicates, a \textsc{cause} and an \textsc{emotional state}. The \textsc{cause} involves one participant, namely a \textsc{stimulus}, and it is true at all times in the event (E). The \textsc{emotional state}, on the other hand, is true only in the consequent stage of the event (\textsc{result(E)}), and it involves an \textsc{experiencer} participant which has an \textsc{emotion}.
As with \citet{Tantos.2006}, in the event structure of marvel verbs the order of the two predicates has been swapped. In addition, the \textsc{cause} predicate is replaced by \textsc{in reaction to}.
This predicate represents an alternative type of causation alongside \textsc{cause} (which we have seen used for causative COS verbs in the previous chapter, \sectref{sec:cos-VN-c/igeneral}). \textsc{In reaction to} is used in a number of VerbNet classes such as \textit{respond verbs} (`a social interaction in reaction to a theme') or \textit{see verbs} (`a perception in reaction to a stimulus'). 

\subsubsubsection{Change-of-psych-state subevent}
\label{sec:psy-input-prev-caus-cops}
The next group of approaches models the second subevent of some OE psych verbs not as a state, but as a change-of-state. This change-of-state is then terminologically distinguished from default changes-of-state by labeling it \textit{change-of-mental-state} (e.g. \citealt{Hartshorne.2016}), \textit{change-of-psychological-state} (e.g. \citealt{Prakasam.2018}), or \textit{change-of-emotional-state} (e.g. \citealt{Wanner.1999}). In formal accounts, the semantic decomposition then looks something like in (\ref{form:cops}). In this example from \citet[68--79]{Rapp.1997}, both punctual and gradual changes-of-psych-state are accounted for, with BECOME denoting a punctual and DEV denoting a gradual change.

\begin{exe}
	\ex \label{form:cops} CAUSE (x, BECOME/DEV (PSYCH(y)))
\end{exe} 

\noindent Assuming that OE psych verbs can indeed have a change-of-psych-state subevent, how can I determine which of my base verbs do, and which do not? I have decided to use an array of seven diagnostics as proposed by \citet[35]{VanValin.2005}:

\begin{enumerate} \label{enum:VanDiagnostics}
\item Progressive (\textit{The ice is melting.})
\item Dynamic adverbs (\textit{Pat ran energetically to the park.})
\item Pace adverbs (\textit{John slowly realized his mistake.})
\item \textit{for-}PP (\textit{Mary danced for ten minutes.})
\item \textit{in-}PP (\textit{Tom drank the glass of beer in an hour.})
\item Stative modifier (\textit{the shattered window})
\item Causative paraphrase (\textit{The dog caused the boy to be afraid.})
\end{enumerate}

Technically, these tests are used to determine aktionsart, but Van Valin's aktionsart classes can be used as a proxy for the question at hand: The author models \HighlightText{causative achievements} and \HighlightText{causative accomplishments} as complex events with a change-of-state subevent, and \HighlightText{causative states} as complex events with a state subevent. Thus, if my verbs fall into one of these three classes, I have a reference point as to whether they should contain a change-of-state subevent, or not (see also \citealt{Alexiadou.2014}, who use test 5 with the same goal). 

Based on these diagnostics, my OE base verbs are indeed covered by two of Van Valin's classes, namely causative accomplishments (\textit{endull, enrage, soothe,} and \textit{uplift}) and causative states (\textit{abash, affright, annoy, bemuse, bumfuzzle, confound, convince, dishearten, dumbfound, enrapture, nonplus, perturb, reassure, stagger, upset,} and \textit{worry}). The outcome of each of the seven tests in relation to these two classes is given in Table \ref{tab:psych-Valintests}.\footnote{See \citet[33--39]{VanValin.2005} for an overview of the features used to distinguish between the twelve aktionsart classes, as well as an in-depth discussion of the applied tests.} 

\begin{table}
	\small
		\caption[\citeauthor{VanValin.2005}'s diagnostics for causative states and causative accomplishments]{\label{tab:psych-Valintests} \citeauthor{VanValin.2005}'s diagnostics for causative states and causative accomplishments. Abbreviations: acc. = accomplishment, dyn. = dynamic, prog. = progressive, stat. mod. = stative modifier.} 
		\begin{tabular} {llllllll}
 			\lsptoprule
			Class & Prog. & Dyn. & Pace & \textit{for-}PP & \textit{in-}PP & Stat. Mod. & Cause \\
			\midrule
			Causative state & Yes & Yes & No & Yes & No & Yes & Yes \\
			Causative acc. & Yes & Yes & Yes & Irrelevant & Yes & Yes & Yes\\
			\lspbottomrule
		\end{tabular}
\end{table} 

These results indicate that most of my OE psych bases should include only a state, while four base verbs should lexicalize a change-of-psych-state. I can test this with my nominalization data; we can expect to find \textsc{change-of-state} readings only for \textit{endullment, enragement, soothement,} and \textit{upliftment}. The logical structures provided by \citet[45]{VanValin.2005} are as follows, where $\alpha$ stands for any kind of event:

\begin{labeling}{causative accomplishmentXX}
	\item [causative state] $\alpha$ CAUSE [\textbf{predicate$'$} (x) or (x, y)]
	\item [causative accomplishment] $\alpha$ CAUSE [BECOME \textbf{predicate$'$} (x) or (x, y)]
\end{labeling}

\subsubsubsection{Conceptual causativity}
The next approach which I would like to present is that of \citet{Hartl.1999,Hartl.2001book,Hartl.2001}. This author makes a distinction between the conceptual level on the one hand, and the lexico-semantic, grammatical level on the other (\citealt[206]{Hartl.2001}). 
He argues that, conceptually, both OE psych verbs and SE psych verbs are implicitly causative (p. 209, see also \sectref{sec:psy-input-prev-impl}).  
Grammatically, however, most psych verbs do not display causativity, so that it is not contained in their semantic representation. Rather, the author states, OE psych verbs are best described as activities, while SE psych verbs are states with an additional \textsc{theme/stimulus} argument. \citet{Hartl.2001} finds evidence for this in the temporal homogeneity of OE psych verbs (p. 206) as well as in the agentive properties of their \textsc{stimulus} argument (p. 207). 

The author makes two concessions: First, he states that SE psych verbs can denote activities if they can be connected to a canonical perception-event (\citealt[192--193]{Hartl.1999}). Thus, German \textit{bewundern} (`admire') can be used agentively in the sense of `examine admiringly' in sentences like \textit{Peter bewundert gerade das Bildnis mit einer Lupe} (`Peter is admiring the portrait with a magnifying glass right now'). Second, some psych verbs do have grammatically causative variants and should be analyzed as (psychological) achievements (p. 193). Examples from German are \textit{erschrecken} (`give a scare') and \textit{verbl\"uffen} (`perplex,' \citealt[191]{Hartl.2001book}).

\subsubsubsection{States and actions}
It is commonly assumed that SE psych verbs denote states. However, in my data set, I suspected that the experiencer of \textit{muse over} is more active compared to those of \textit{approve of} and \textit{worry about} (see \citealt{Kawaletz.2015}). To test this, I used \citeauthor{VanValin.2005}'s (\citeyear{VanValin.2005}) diagnostics once again. 
The results indicate that, in terms of the author's aktionsart classes, \textit{muse over} is an activity, whereas \textit{approve of} and \textit{worry about} are states. The diagnostics which led to this conclusion are outlined in Table \ref{tab:psych-Valintests2}. 

\begin{table} 
		\caption[\citeauthor{VanValin.2005}'s diagnostics for states and activities]{\label{tab:psych-Valintests2} \citeauthor{VanValin.2005}'s diagnostics for states and activities. Abbreviations: dyn. = dynamic, prog. = progressive, stat. mod. = stative modifier.} 
		\begin{tabular} {llllllll}
 			\lsptoprule
			Class & Prog. & Dyn. & Pace & \textit{for-}PP & \textit{in-}PP & Stat. Mod. & Cause \\
			\midrule
			State & No & No & No & Yes & No & Yes & No \\
			Activity & Yes & Yes & Yes & Yes & No & No & No \\
			\lspbottomrule
		\end{tabular}
\end{table}   

In the corresponding formalizations, the author indicates the difference by an additional \textbf{do$'$} predicate for activities (p. 45):

\begin{labeling}{activityXX}
	\item [state] \textbf{predicate$'$} (x) or (x, y)
	\item [activity] \textbf{do$'$} (x, [\textbf{predicate$'$} (x) or (x, y)])
\end{labeling}

\noindent In my frame approach, the difference is reflected by using two different event types as defined in the type signature, namely \textit{state} and \textit{action}.

\subsubsubsection{Summary}
There are vastly different opinions with regard to the complexity and causativity of psych verbs. 
In terms of complexity, OE psych verbs are standardly regarded as denoting complex events, while SE psych verbs are usually considered to denote simple events.
Regarding causativity, OE psych verbs are often regarded as a special kind of causatives, while SE psych verbs are considered non-causatives. There is also some discussion regarding the impact of agentivity on causativity, and regarding the saliency of subevents. 
A rather different approach has been put forward as well, assuming that all psych verbs denote complex events but are not as strongly causative as default causatives. 
Furthermore, it has been claimed that all psych verbs express causatives conceptually, but that lexico-semantically they denote activities (OE psych verbs) and states (SE psych verbs). All in all, we have seen that the event structure of psych verbs is far from uncontroversial.

\subsection{Frame decomposition of psych verbs}
\label{sec:psy-input-frames}

In this section, I will model the semantics of OE and SE psych verbs in frames. 
There are two challenges in this endeavor: First, as we have seen in the previous section, the event decomposition of psych verbs is highly debatable.
Second, in the frame-semantic discourse there is no published material to build on.\footnote{At the time of writing, there is unpublished work by Rolf Kailuweit, by Sebastian L\"obner and Harald Stamm, and by Robert Van Valin Jr, which served as an inspiration for this section.} 
I will tackle both issues by translating the different approaches from the non-frame literature into eight frame variants \textendash{} five for OE psych verbs (\sectref{sec:psy-input-frames-OE}) and three for SE psych verbs (\sectref{sec:psy-input-frames-SE}). In \sectref{sec:psy-output}, I will then investigate which attributes in the formalization of the base verbs account for possible readings in their nominalizations. In other words, I will use my nominalization data to test which of the eight analyses make sense.

\subsubsection{OE psych verbs}
\label{sec:psy-input-frames-OE}

In Figure \ref{fig:psychframeoe1}, we see the first frame analysis.
Let us first look at the participants. I have chosen the standard labels \textsc{stimulus} and \textsc{experiencer}; \textsc{stimulus} can of course easily be swapped for more precise labels such as \citeauthor{Pesetsky.1995}'s (\citeyear{Pesetsky.1995}) categories \textsc{causer}\textsubscript{stim}, \textsc{target} or \textsc{subject matter} if desired (see \sectref{sec:psy-input-prev-part}).\footnote{I am using the notation \textsc{causer}\textsubscript{stim} in order to distinguish Pesetsky's category from the more general category \textsc{causer} I have used in \chapref{ch:cos}.} These two participants will figure in all frame analyses in this section. 

\begin{figure}
		\begin{avm}
			\avml
			\@0	
			\[
			\avmspan{\type{psych-state causation}}\cr
			\textsc{stimulus} & \@1 \cr
			\textsc{experiencer} & \@2 \cr
			\textsc{cause} & \@3 
				\[ 
				\textit{event} \cr 
				\att{participant} & \@1
				\] \cr
			\textsc{effect} & \@4 
				\[ 
				\textit{psych-state} \cr
				\textsc{experiencer} & \@2 
				\] \cr
			\] \cr
			{\textsc{\textsc{ref} $=$} \{\@0\} }	
			\avmr
		\end{avm}
		\caption[Frame for OE psych verbs with a caused psych-state]{Frame for OE psych verbs with a caused psych-state (e.g. Charlie being in a frightened-state because of some event Jael is involved in)}
		\label{fig:psychframeoe1}
\end{figure}

Let us turn to the eventive nodes. 
The frame is typed as a \textit{psych-state causation}. As defined in the type signature (Figure \ref{fig:signatureevent} in \sectref{sec:meth-Nsem-semcat-event}), this label is used to describe psych causation events where the second subevent is a state, with the attribute \textsc{effect} having the value \textit{psych-state}. 
To illustrate this, I will use the sentence \textit{Jael frightens Charlie}. 
In this first analysis, the second subevent is the \textit{frightened-state} that Charlie is in due to the first subevent. 
The first subevent, \textsc{cause}, is whatever Jael is involved in that frightens Charlie. This subevent is underspecified, with a type \textit{event} and a participant attribute \textsc{participant}. For example, \textit{event} may be instantiated by an action (e.g. \textit{Jael frightened Charlie by telling a scary story}), or by a state (e.g. \textit{Jael frightened Charlie with his badly scarred face}). 
The only specification is that the stimulus of the complex event (here: \textit{Jael}) must also be a participant in the first subevent (that is, also \textit{Jael}). This is indicated by co-indexation.  

The second analysis (Figure \ref{fig:psychframeoe2}) is based on \citeauthor{Pustejovsky.1995}'s (\citeyear{Pustejovsky.1995}) \textit{experienced causation}.
The first subevent is specified as \textit{perception} (a label which I find more intuitive than \textit{experiencer process}). It has two participants, a \textsc{stimulus} and an \textsc{experiencer}.  
The stimulus of the perception-event is not necessarily co-indexed with the stimulus of the complex event (\textsc{stimulus} \avmbox{1} $\veebar$ \avmbox{5}). To stay with the example I have used above, \textit{Jael frightens Charlie}, the perception-event can be Charlie seeing Jael's scarred face. In this case, the stimulus of the causing event (\textit{Jael's face}) is not co-indexed with the stimulus of the complex event (\textit{Jael}), but with something related to it, indexed with \avmbox{5}. 
This relation is captured by `$\cdots$,' which represents an attribute path of undefined length leading from \avmbox{1} to \avmbox{5}. If \avmbox{1} does have the same referent as \avmbox{5}, the length of the attribute path is 0.


\begin{figure}
		\begin{avm}
			\avml
			\@0	
			\[
			\avmspan{\textit{experienced causation}}\cr
			\textsc{stimulus} & \@1 
				\[\type{entity} \cr
				$\cdots$ & \@5
				\] \cr
			\textsc{experiencer} & \@2 \cr
			\textsc{cause} & \@3 
				\[ 
				\textit{perception} \cr
				\textsc{stimulus} & \@1 $\veebar$ \@5 \cr 
				\textsc{experiencer} & \@2
				\] \cr
			\textsc{effect} & \@4 
				\[ 
				\textit{psych-state} \cr
				\textsc{experiencer} & \@2 
				\] \cr
			\] \cr
			{\textsc{ref} $=$ \{\@0\} }	\cr 
			\avmr
		\end{avm}
		\caption[Frame for OE psych verbs with a causing perception-event]{Frame for OE psych verbs with a causing perception (e.g. Charlie being in a frightened-state because of seeing Jael's scarred face)}
		\label{fig:psychframeoe2}
\end{figure}


The third frame analysis, given in Figure \ref{fig:psychframeoe3}, models the psych causation subtype of \HighlightText{agentive psych causation}. For our example, this means that Jael is now purposefully frightening Charlie.  
The complex event has four participant att\-ri\-butes, with \textsc{instrument} and \textsc{stimulus} being co-indexed. This indicates that, whatever Jael is using to frighten Charlie (i.e. the \textsc{instrument}), is also what causes Charlie to be scared (i.e. the \textsc{stimulus}). 
This instrument/stimulus is also a participant in the first subevent; there, the corresponding attribute is unspecified in order to allow for all kinds of different scenarios. 
That is, \textsc{participant} gets specified as soon as more context is available. For example, if the first subevent is a wearing-event (\textit{Jael frightened Charlie with a scary mask}), the \textsc{participant} attribute is specified as a \textsc{theme}, and if the first subevent is a narrating-event (\textit{Jael frightened Charlie with ghost stories}), it is specified as a \textsc{topic}. 

\begin{figure}
		\begin{avm}
			\avml
			\@0	
			\[
			\avmspan{\type{agent psych-state causation}}\cr
			\att{agent} & \@1 \cr
			\att{instrument} & \@2 \cr
			\att{stimulus} & \@2 \cr
			\att{experiencer} & \@3 \cr
			\att{cause} & \@4 
			\[ 
			\type{agent-action} \cr
			\att{agent} & \@1 \cr
			\att{participant} & \@2
			\] \cr
			\textsc{effect} & \@5 
			\[ 
			\textit{psych-state} \cr
			\textsc{experiencer} & \@2 
			\] \cr
			\] \cr
			{\textsc{ref} $=$ \{\@0\} }	
			\avmr
		\end{avm}
		\caption[Frame for the agentive variant of OE psych verbs]{Frame for the agentive variant of OE psych verbs (e.g. Charlie being in a frightened-state because of something Jael does with the intention to scare Charlie)}
		\label{fig:psychframeoe3}
\end{figure}

The introduction of an \textsc{instrument} participant into the frame leads to some complications with regard to co-indexation. Stated in non-frame-theoretic terms, the assignment of participant roles to the participants in the event is problematic. Consider the assignment of participants in (\ref{ex:instr1}). Since there is no instrument participant, \textit{Jael} is assigned both the agent and the stimulus role. This is in line with VerbNet, where \textit{the clown} is tagged as \textsc{stimulus} in the sentence \textit{The clown amused the children} (see also \citealt[190]{Kailuweit.2005}).

\begin{exe}
	\ex \label{ex:instr1}
	\gll Jael frightened Charlie \\
	\textsc{ag/stim} {} \textsc{exp}  \\
\end{exe}

When an \textsc{instrument} participant is added to the context, we have to decide whether \textit{Jael} is an \textsc{agent-stimulus}, as in (\ref{ex:instr2}), or whether \textit{a scary mask} is an \textsc{instrument-stimulus}, as in (\ref{ex:instr3}).

\begin{exe}
	\ex 
	\begin{xlist}
	\item \label{ex:instr2}
	\gll Jael frightened Charlie with {a scary mask}. \\
	\textsc{ag/stim} {} \textsc{exp} {} \textsc{instr}  \\
	\item \label{ex:instr3}
	\gll Jael frightened Charlie with {a scary mask}. \\
	\textsc{ag} {} \textsc{exp} {} \textsc{instr/stim}  \\
	\end{xlist}
\end{exe}

\noindent Yet another option would be to model a metonymic reconstruction of the \textsc{instrument} or the \textsc{instrument-stimulus} to the first subevent (see \citealt[209]{Pustejovsky.1995} for a discussion of this process):\largerpage

\begin{exe}
	\ex 
	\begin{xlist}
	\item 
	\gll Jael 			frightened 	Charlie 		{by wearing a scary mask}. \\
	\textsc{ag/stim} 	{} 			\textsc{exp}	\textsc{instr}  \\
	\item 
	\gll Jael 		frightened 	Charlie 		{by wearing a scary mask}. \\
	\textsc{ag} 	{} 			\textsc{exp}	\textsc{instr/stim}  \\
	\end{xlist}
\end{exe}

\noindent Ultimately, what we are dealing with here is the conceptual question of what it is that scares Charlie: Jael, the mask, or Jael wearing the mask. For my purposes, I do not need to answer this question. What is important is that all three candidates for the \textsc{stimulus} role (\textsc{agent}/\textit{Jael}, \textsc{instrument}/\textit{a scary mask}, and \textsc{cause}/\textit{wearing a scary mask}) are represented in the base verb frame, so that I can investigate whether a shift to the respective node is possible.   

The three frame analyses presented so far share the same type of second subevent, namely a psych-state. In the fourth frame analysis, given in Figure \ref{fig:psychframeoe4}, the second subevent is a change-of-psych-state. This is also made explicit by the frame type (\textit{change-of-psych-state causation}). Note that the frame also contains a node labeled \textit{psych-state}, but it is embedded more deeply into the frame structure than in the previous analyses. 

\begin{figure}
		\begin{avm}
			\avml
			\@0	
			\[
			\avmspan{\type{change-of-psych-state causation}}\cr
			\textsc{stimulus} & \@1 \cr
			\textsc{experiencer} & \@2 \cr
			\textsc{cause} & \@3 
				\[ 
				\textit{event} \cr
				\att{participant} & \@1
				\] \cr
			\textsc{effect} & \@4 
			\[ 
			\avmspan{\textit{change-of-psych-state}} \cr
			\textsc{result-state} & \@6 
			\[ 
			\textit{psych-state} \cr
			\textsc{experiencer} & \@2\cr
			\] \cr
			\] \cr
			\] \cr
			{\textsc{ref} $=$ \{\@0\} }	
			\avmr
		\end{avm}
		\caption[Frame for OE psych verbs with a change-of-psych-state subevent]{Frame for OE psych verbs with a change-of-psych-state subevent (e.g. Charlie attaining a frightened-state because of an event Jael is involved in)}
		\label{fig:psychframeoe4}
\end{figure}

If this frame analysis is valid, it can be used to model four OE psych verbs, namely those which have a change-of-state subevent according to \citeauthor{VanValin.2005}'s (\citeyear{VanValin.2005}) diagnostics (see \sectref{sec:psy-input-prev-caus}).
Of course, the other verbs in the data set also presuppose a change-of-state of the experiencer; otherwise, Charlie would be in a perpetual state of fear. However, verbs like \textit{endull, enrage, soothe} and \textit{uplift} focus on the change, while other verbs (e.g. \textit{abash, affright, annoy}) focus on the resulting state \textendash{} and this is what is represented in the respective frame.

One last frame analysis is presented in Figure \ref{fig:psychframeoe5}. It is based on \citeauthor{Hartl.1999}'s (\citeyear{Hartl.1999,Hartl.2001book,Hartl.2001}) claim that OE psych verbs are activities with an additional participant. The frame thus models a psych-action with two participants, the traditional \textsc{stimulus} and \textsc{experiencer}. As before, the experiencer entity is in some psychological state, signified by the attribute \textsc{psych-state}. In addition, the \textsc{stimulus} is cross-indexed as an \textsc{actor} to capture which participant has the active part in the event. This is also made explicit in the frame type (\textit{stimulus psych-action} as opposed to \textit{experiencer psych-action}, see below).

\begin{figure}
		\begin{avm}
			\avml
			\@0	
			\[
			\avmspan{\textit{stimulus psych-action}}\cr
			\textsc{stimulus} & \@1 \cr
			\textsc{actor} & \@1 \cr
			\textsc{experiencer} & \@2 
				\[ 
				\textit{entity}\cr 
				\textsc{psych-state} & \@3 
				\] \cr
			\] \cr
			{\textsc{ref} $=$ \{\@0\} }	
			\avmr
		\end{avm}
		\caption[Frame for OE psych verbs as psych-actions]{Frame for OE psych verbs as psych-actions (e.g. Jael doing something and Charlie being in a frightened-state)}
		\label{fig:psychframeoe5}
\end{figure}

Note that the experiencer and their psych-state are introduced by a reversed attribute compared to what we have seen in Figures \ref{fig:psychframeoe1} to \ref{fig:psychframeoe4}. There, the psych-state has an experiencer participant, while here, an entity has a psych-state. While this may seem like a conceptual difference, it is better interpreted as a bidirectional relationship between the experiencer entity and their psych-state (see e.g. \citealt[316]{Loebner.1985}). In other words, the two attribute directions show different sides of the same coin. I do not depict bidirectional functionality within an AVM, since this introduces a lot of visual clutter while at the same time not being essential to my approach.

I would like to add a final note about the possible causativity or non-causativity of psych verbs in my data set: Three of the base verbs, \textit{endull, enrage} and \textit{enrapture}, contain the prefix \textit{en-}, which \textendash{} among other things \textendash{} produces causative readings (see e.g. \citealt[109]{Gelderen.2014}). Therefore, one of the causative analyses can be regarded as the appropriate one at least for these three types (if not for all OE psych verbs). Such a causative \textit{en}-prefixed psych nominalization is also mentioned by \citet[]{Pesetsky.1995}, who states that \textit{embitterment} is ``unique among nominalizations of causative Experiencer predicates in retaining its causative force'' (p. 321).

\subsubsection{SE psych verbs}
\label{sec:psy-input-frames-SE}

Most approaches analyze SE psych verbs as states. The corresponding frame in Figure \ref{fig:psychframese1} has an \textsc{experiencer} and a \textsc{stimulus} attribute. To make the \textsc{experiencer} the center of attention, I have switched the order of the participants in the frames for SE psych verbs compared to the frames for OE psych verbs. This is a matter of personal taste; the order of attributes is not formally significant. 


The second analysis I propose for SE psych verbs is that of an \textit{experiencer psych-action} in Figure \ref{fig:psychframese-act}. This formalization is based on my intuition that \textit{muse over} involves a rather active experiencer participant, and on the results from the diagnostics I applied to test this intuition. The frame can be seen as a counterpart to the \textit{stimulus psych-action} modeled for OE psych verbs (Figure \ref{fig:psychframeoe5}). The crucial difference is the co-indexing of the participants. Here, the experiencer takes on the more active part, so that it is co-indexed with the actor. 

\begin{figure}
	\begin{avm}
		\avml
		\@0	
			\[
			\avmspan{\textit{psych-state}}\cr
			\textsc{experiencer} & \@1 \cr
			\textsc{stimulus} & \@2
			\] \cr
			{\textsc{ref} $=$ \{\@0\}}
		\avmr
	\end{avm}
	\caption[Frame for SE psych verbs as psych-states]{Frame for SE psych verbs as psych-states (e.g. Charlie being in a frightened-state)}
	\label{fig:psychframese1}
\end{figure}

\begin{figure}
	\begin{avm}
		\avml
		\@0	
		\[
		\avmspan{\textit{experiencer psych-action}}\cr
		\textsc{experiencer} & \@1
			\[ 
			\textit{entity}\cr 
			\textsc{psych-state} & \@2 
			\] \cr
		\textsc{actor} & \@1 \cr
		\textsc{stimulus} & \@3
		\] \cr
		{\textsc{ref} $=$ \{\@0\} }	
		\avmr
	\end{avm}
	\caption[Frame for SE psych verbs as psych-activities]{Frame for SE psych verbs as psych-activities (e.g. Charlie actively musing over Jael)}
	\label{fig:psychframese-act}
\end{figure}

The third approach is to model SE psych verbs as complex events similar to the standard analysis of OE psych verbs. The frame in Figure \ref{fig:psychframese2} differs from those for OE psych verbs, however, in that the event type is not a psych causation or an experienced causation, but a \HighlightText{psych-reaction}. Borrowing the terminology from \citet{Tantos.2006}, I correspondingly do not include the attributes \textsc{cause} and \textsc{effect}, but \textsc{explanation} and \textsc{reaction}.
It should be noted that I follow VerbNet in staying on a descriptive level here;\largerpage 
the distinction between \textsc{cause/effect} and \textsc{explanation/reaction} is not formalized.{\interfootnotelinepenalty=10000\footnote{In \citeauthor{Tantos.2006}'s (\citeyear{Tantos.2006}) account, the order of the two subevents differs (see \sectref{sec:psy-input-prev-caus}), but this is because his approach is located in discourse representation theory, and not concerned with the decomposition of events per se.}} 
In order to do so and model \citeauthor{Tantos.2006}'s (\citeyear{Tantos.2006}) claim that SE psych verbs highlight the result-state, one would need to veer towards the semantics/pragmatics interface, which is outside the scope of this book.

\begin{figure}
\begin{avm}
	\avml
	\@0	
		\[
		\avmspan{\textit{psych-reaction}}\cr
		\textsc{stimulus} & \@1 \cr
		\textsc{experiencer} & \@2 \cr
		\textsc{explanation} & \@3 
			\[ 
			\textit{event} \cr
			\att{participant} & \@1
			\] \cr
		\textsc{reaction} & \@4 
			\[ 
			\textit{psych-state} \cr
			\textsc{experiencer} & \@2\cr
			\] \cr
		\] \cr
	{\textsc{ref} $=$ \{\@0\} }	
	\avmr
\end{avm}
\caption[Frame for SE psych verbs as psych-reactions]{Frame for SE psych verbs as psych-reactions (e.g. Charlie being in a frightened-state as a reaction to some event Jael is involved in)}
\label{fig:psychframese2}
\end{figure}

\subsubsection{Summary}
In this section, I have given a formal account of the psych verb bases in my data set, proposing eight frame analyses (five for OE psych verbs and three for SE psych verbs). These are typed as \textit{psych-state causation, experienced causation, agent psych-state causation, change-of-psych-state causation} and \textit{stimulus psych-action} (OE psych verbs), as well as \textit{psych-state, experiencer psych-action} and \textit{psych-reaction} (SE psych verbs). 
They differ in 
their event structure (simple vs. complex event),
the attributes they contain (e.g. \textsc{cause} vs. \textsc{explanation}),
and the target nodes these attributes are specified by (e.g. \textsc{cause:} \textit{event} vs. \textsc{cause:} \textit{perception}).
The differences between the frames account for an array of different findings and assumptions which have been proposed in the literature on psych verbs in general, as well as for properties of my base verbs specifically.

In the second part of this chapter, I will turn to the semantics of psych nouns. Using the conclusions which can be drawn from my nominalization data, I will be able to test which of the eight frames best represent the semantics of the psych verbs in my data set.

\section{The semantics of psych nouns}
\label{sec:psy-output}

In this section, I will first present which readings can be expected for psych nouns with \ment{}, based on existing literature and the frame representations of the base verbs (\sectref{sec:psy-output-predictions}). 
Next, I will turn to my results. I will first provide an informal survey of attested readings (\sectref{sec:psy-output-survey}), and then formalize my findings (\sectref{sec:psy-output-formal}). 
In the process, I will be able to determine which of the VerbNet-based frame-semantic analyses as presented in \sectref{sec:psy-input-frames} are adequate.

\subsection{Expectations regarding the semantics of psych nouns}
\label{sec:psy-output-predictions}
\subsubsection{Previous literature}
\label{sec:psy-output-lit}

In previous literature, \ment{} has been found to produce a wide range of readings: \textsc{event}, \textsc{state}, \textsc{agentive-collective}, \textsc{instrument/means}, \mbox{\textsc{patient/theme}}, \textsc{location}, \textsc{product}, and \textsc{result} (see e.g. \citealt{Gadde.1910,Marchand.1969,Bauer.2013,Lieber.2016}; see also the more comprehensive literature summary in the previous chapter, \sectref{sec:cos-output-ment}). Obviously, only a subset of these should be relevant for psych verb bases, but I am not aware of literature dealing specifically with psych nominalizations in \textit{-ment}.

Let us therefore have a look at derived psych nouns in general, starting with SE psych nouns. These are not often discussed in linguistic research. As far as I know, only one reading has been mentioned explicitly, namely the transpositional \textsc{psych-state} reading (see \citealt[119]{Grimshaw.1990}; \citealt[659]{VanValinJr.1997}). However, \textsc{stimulus} can also be regarded as a documented reading: In the \citetalias{OED}, several lexicalized derivatives of SE psych verbs are listed with this sense, for instance \textit{cheerer} (`a person who or thing which brings gladness, comfort, or solace') or \textit{thrill} (`a thrilling experience or incident').

Concerning OE psych nominalizations, the most prevalent claim is that they can only denote states or stimuli (see e.g. \citealt[72]{Pesetsky.1995}).
It has also been claimed that (some) agentive variants of OE psych verbs can produce transpositional readings (\citealt[119]{Grimshaw.1990}; \citealt{Iordachioaia.2020}). This is presumably illustrated in the examples in (\ref{ex:psy-humil}) and (\ref{ex:psy-depr}). According to Grimshaw, \textit{humiliate} has an agentive variant, while \textit{depress} is non-agentive. This leads to the following behavior, according to the author:
The (a) examples do not contain a \textsc{stimulus} argument. \textit{Humiliation} is interpreted either as a transpositional event, or as a state, while \textit{depression} can only be interpreted as a state.
The (b) examples have an agentive interpretation. This is only possible with \textit{humiliation}; for \textit{depression}, the sentence in ungrammatical.
Finally, the (c) sentences are non-agentive and ungrammatical.

\begin{exe}
	\ex \label{ex:psy-humil} 
	\begin{xlist}
		\ex[]{\label{ex:psy-humil2}The \textbf{humiliation} of the audience}
		\ex[]{\label{ex:psy-humil1}John's \textbf{humiliation} of the audience}
		\ex[*]{\label{ex:psy-humil3}The joke's \textbf{humiliation} of the audience}
	\end{xlist}
	\ex \label{ex:psy-depr}
	\begin{xlist}
		\ex[]{\label{ex:psy-depr2}The \textbf{depression} of the patients}
		\ex[*]{\label{ex:psy-depr1}John's \textbf{depression} of the patients}
		\ex[*]{\label{ex:psy-depr3}The drug's \textbf{depression} of the patients}
	\end{xlist}	
\end{exe}

\noindent In order to be able to formulate predictions for the psych noun semantics, let us come back to the agentivity scores I calculated earlier. For convenience, I am repeating Table \ref{tab:psych-agent} as Table \ref{tab:psych-agent-rep}. Based on this data, transpositional \textsc{event} readings are expected for nouns based on verbs higher up on the scale (e.g. \textit{convince,} \textit{upset}). Correspondingly, this reading should not be possible for verbs further down (e.g. \textit{nonplus}, \textit{affright}). For verbs with medium agentivity, no sensible prediction can be made. Of course, the cutoff-points are arbitrary. However, if \citeauthor{Grimshaw.1990}'s (\citeyear{Grimshaw.1990}) claim is correct, the agentivity scores should dictate a tendency.  

\begin{table}
\caption[Agentivity scores of OE psych verbs (repeated)]{Agentivity scores of OE psych verbs (repeated from Table \ref{tab:psych-agent})} \label{tab:psych-agent-rep}
		\begin{tabular} {lr}
 			\lsptoprule
			Verbs & \multicolumn{1}{l}{Agentivity score} \\
			\midrule
			convince, reassure		  	& 5\phantom{.5} \\
			upset					  	& 4\phantom{.5} \\
			annoy, soothe				& 3.5 \\
			confound, enrage, uplift   	& 3\phantom{.5}             \\
			bemuse, perturb, worry		& 2\phantom{.5} \\
			dishearten, enrapture       & 1.5              \\
			bumfuzzle, dumbfound        & 1\phantom{.5}               \\
			abash, affright, stagger    & 0.5              \\
			endull, nonplus		        & 0\phantom{.5}              \\
			\lspbottomrule
		\end{tabular}
\end{table}

A summary of the expected readings of OE and SE psych nominalizations based on the literature is given in Table \ref{tab:pred-psy-lit}. The option ``possibly'' indicates that there is some dispute regarding the reading in question.  
Some readings can theoretically be produced by \textit{-ment}, but do not figure in the discussion of psych nouns. These are tagged as ``not pertinent'' in the table. 

\begin{table}
	\caption[Expected readings of psych nouns based on the literature]{Expected readings of psych nouns based on the literature} 
	\label{tab:pred-psy-lit}
	\begin{tabular}{llll}
		\lsptoprule
		\multicolumn{3}{l}{Reading} & Prediction \\
		\midrule
		 \multicolumn{3}{l}{Eventive readings} & \\
		 & OE: & transposition		& 		possibly (agentive base verbs) \\
		 & OE: & psych-state 		&	yes \\
		 & SE: & transposed psych-state 		& yes \\
		\multicolumn{3}{l}{Participant readings} &\\
		& \multicolumn{2}{l}{agent} 				&		collective \\
		& \multicolumn{2}{l}{stimulus/instrument}& yes \\
		& \multicolumn{2}{l}{location} & not pertinent \\
		& \multicolumn{2}{l}{[−animate] patient} & not pertinent\\
		& \multicolumn{2}{l}{product} & not pertinent \\
		& \multicolumn{2}{l}{result} & not pertinent\\
		\lspbottomrule
	\end{tabular}
\end{table} 

\subsubsection{Frame-based predictions}
\label{sec:psy-output-predictions-frames}

A nominalization is only expected to be attested in a given reading if the corresponding node is represented in the nominalization's base verb frame. With regard to participants, \textsc{stimulus} and \textsc{experiencer} are shared in common among all of my proposed psych verb analyses, and one proposed analysis incorporates two additional participants, namely \textsc{agent} and \textsc{instrument}. Of these three participants, two are defined as [+animate] categories, namely \textsc{agent} and \textsc{experiencer}. In the previous chapter, I identified and modeled an \HighlightText{animacy constraint}, which states that \textit{-ment} does not derive [+animate] readings. Therefore, we can predict that only the \textsc{stimulus/instrument} participant can be shifted to.\footnote{In some frames I use the supertypes \textsc{actor} and \textsc{participant}. These are co-indexed with more informative participant roles (i.e., \textsc{stimulus}, \textsc{experiencer}, \textsc{instrument} or \textsc{agent}) and thus do not yield any additional information with regard to expected nominalization semantics.} 

\begin{table}
	\caption[Overview of conceivable eventive readings in psych nouns]{Overview of conceivable eventive readings in psych nouns. Abbreviations: c. = causation.} 
	\label{tab:eventivepsych}
	\begin{tabular}{ll>{\RaggedRight\arraybackslash}p{0.52\textwidth}}
		\lsptoprule
		\multicolumn{2}{l}{Reading} & Exemplary context\\
		\midrule
		\multicolumn{2}{l}{Transpositional readings} & \\
		& 1. psych-state c. & Charlie being in a frightened-state because of some event Jael is involved in\\
		& 2.  experienced c. & Charlie being in a frightened-state because of seeing Jael's badly scarred face\\
		& 3.  agent psych-state c. & Charlie being in a frightened-state because of something Jael does with the intention to scare Charlie\\
		& 4.  change-of-psych-state c. & Charlie attaining a frightened-state because of some event Jael is involved in\\
		& 5. stimulus psych-action & Jael doing something and Charlie being in a frightened-state\\
		& 6. psych-state & Charlie being in a frightened-state \\
		& 7. experiencer psych-action & Charlie actively musing over Jael \\
		& 8. psych-reaction & Charlie being in a frightened-state as a reaction to some event Jael is involved in\\
		\multicolumn{2}{l}{Non-transpositional readings} & \\
		& 1. cause/explanation & \\
		& \enspace \textendash{} event & Jael having a badly scarred face\\
		& \enspace \textendash{} perception-event & Charlie seeing Jael's badly scarred face\\
		& \enspace \textendash{} agent-action & Jael telling a scary story \\
		& 2. psych-state & Charlie being in a frightened-state \\
		& 3. change-of-psych-state & Charlie attaining a frightened-state \\
		\lspbottomrule
	\end{tabular}
\end{table} 

Furthermore, there are a number of conceivable eventive readings. An overview is given in Table \ref{tab:eventivepsych} alongside the exemplary contexts I used to illustrate the frames in Figures \ref{fig:psychframeoe1} to \ref{fig:psychframese2}.
Let us first look at possible transpositional readings. Representing the different analyses in the literature, there are eight different central node types in the base verb frames. In my frames, I prefer to be precise with regard to type labels so that differences between frames can be spotted right away. For the summary of possible readings, however, this approach would be confusing due to the sheer number of event types. Therefore, I will use the general label \textsc{psych causation} to cover half of these potential readings, and use the more precise labels as given in the table only when the distinction is relevant.\largerpage[-2]

In addition to these transpositional readings, I distinguish three eventive readings which could result from shifts. These are given in the lower part of \tabref{tab:eventivepsych}.
The first subevent is either introduced by the attribute \textsc{cause} or by the attribute \textsc{explanation}, depending on the analysis. In addition, I am listing three possible event types for these attributes here: The underspecified \textit{event} stands for the complete array of possible events, that is, all of its subtypes. Two of these subtypes, namely \textit{perception-event} and \textit{agent-action}, were used in my frame representations, and are therefore listed here as well.
These three instantiations of \textsc{cause/explanation} will be referred to when relevant.
Note that the reading \textsc{psych-state} is included twice in the table because it can be produced either by transposition, or by a shift, depending on the frame type.

It is obvious that distinguishing between five (or nine) potential transpositional readings in given attestations is not trivial. In essence, the decision will be made based on two indicators. The first hint will be given by the contexts in which I find my nominalizations. For example, if a nominalization should only allow agentive stimuli, its frame will be typed as an \textit{agent psych-state causation}. Second, other possible readings of the same nominalization play a role. For example, if I only find \textsc{result-state} readings, but no \textsc{change-of-psych-state} readings, this speaks in favor of an analysis without a change-of-psych-state subevent for the nominalization (and base verb) in question.
Thus, nominalization semantics allow valuable insights into the event type and causativity/non-causativity of the base verbs. My data will therefore contribute to the discussion of which (frame) analyses are to be preferred for modeling psych verbs and nouns.

\subsection{Survey of possible readings}
\label{sec:psy-output-survey}

In this section, I will first present and discuss the only attested participant reading, \textsc{stimulus} (\sectref{sec:psy-output-survey-stim}) and then the eventive ones (Sections \ref{sec:psy-output-survey-caus} to \ref{sec:psy-output-survey-resultstate}). Since the examination of psych contexts is directly related to the attestation of transpositional readings, this topic will be addressed in the corresponding sections (\ref{sec:psy-output-survey-transOE} and \ref{sec:psy-output-survey-transSE}).
Readings which have not been found attested will be treated last (\sectref{sec:psy-output-survey-unatt}), and a summary can be found in \sectref{sec:psy-output-survey-sum}.

\subsubsection{Stimulus}
\label{sec:psy-output-survey-stim}

As predicted by the animacy constraint, the only participant category which I have found attested in my nominalization data is \textsc{stimulus}, as exemplified in (\ref{ex:psy-stim}). Only the SE noun \textit{approvement} and the OE noun \textit{endullment} cannot be found in this reading, which I attribute to scarcity of data. 

\begin{exe}
	\ex \label{ex:psy-stim}
	\begin{xlist}
		\item \label{ex:psy-stim1} The Education Secretary arrived having just [...] made her first big policy declaration \textendash{} dressed up as a \textbf{reassurement} to Middle England that A-levels will be retained 
		{\small(\acs{OED} NEWS DailyMail 2005)}
		\item \label{ex:psy-stim2} [...] movies in which racial slurs towards Asians [...] (or anyone else) are used; and other `artistic' works which may be an \textbf{abashment} to a certain group of people 
		{\small(Google COMM revleft.space 2002)}
	\end{xlist}
\end{exe}

\noindent There are also attestations in which the stimulus can be cross-classified as an instrument. In these contexts, it is something used intentionally by an agent to evoke a certain psychological response. In (\ref{ex:stim/instr1}), the wordplay \textit{(a)musements} refers to items with two purposes: to be mused over by customers, and to amuse them. The example in (\ref{ex:stim/instr2}) demonstrates a typical ambiguity which can often be found in the data, namely between \textsc{stimulus} and \textsc{cause} readings. More precisely, \textit{enrapturements} in this excerpt from a book report can be anything Lily has devised to enrapture other people. In all, \textsc{stimulus/instrument} readings are not as frequent as non-instrumental \textsc{stimulus} readings. 

\begin{exe}
	\ex \label{ex:stim/instr}
	\begin{xlist}
		\item \label{ex:stim/instr1} Passage des perles Style over fifty; delights, (a)\textbf{musements} and resources for women {\small(\acs{WC} BLOG passagedesperles.blogspot.com 2014)}
		\item \label{ex:stim/instr2} the reader [...] becomes ever concerned with the ever changing, mysteriously engrossing bad girl. I was left with confounded feelings as young Lily had played out her last cancerous \textbf{enrapturements} {\small (Google COMM amazon.com 2008)}
	\end{xlist}
\end{exe}

\noindent In preceding chapter (section \sectref{sec:cos-output-survey-pat}), I showed a single example of a \mbox{[+animate]} \textsc{patient} reading, and interpreted this as an instance of coercion. In the psych data, there is a similar example, given in (\ref{ex:psy-stim-anim1}). While this is the only instance of a [+animate] \textsc{stimulus} in my data set, this reading is frequent in lexicalized psych nominalizations such as \textit{disappointment} (see its entry in the \citetalias{OED}, which contains the paraphrase `a thing which or person who disappoints'). 

\begin{exe}
	\ex \label{ex:psy-stim-anim1} 
	I am an \textbf{abashment} to myself sometimes. I have a frailty of mind that complicates simple matters into unsolvable dilemmas. {\small(\acs{GB} NONFIC SilenceScreams 2015)}
\end{exe}

\noindent Still, I suspect that this reading is not produced by derivation. Rather, we may be dealing with coercion: the [+animate] referent, which is incompatible with the lexical properties of the \textit{-ment} noun, overwrites these properties and coerces a [+animate] reading (see e.g. \citealt{Michaelis.2004} for a discussion of this process). Notably, when probing the corpora for [+animate] stimulus readings, I found that they are only attested in copula constructions.\footnote{I would like to thank Sven Kotowski, who noticed this in the first place.} These are semantically extremely flexible; the following attestations from \acs{iWeb} show that, semantically speaking, anything can go into the slot taken by \textit{abashment} above:\footnote{The corpus was searched with the query 〈is an \_nn* to himself〉, and the list in (\ref{ex:psy-stim-anim-contexts}) includes one line per returned noun type (e.g. \textit{embarrassment}). The query was formulated in the third person singular because the first person singular yielded only one result, namely \textit{an embarrassment to myself}.}

\begin{exe}
	\ex \label{ex:psy-stim-anim-contexts} 
	\begin{xlist} 
	\item \ExHighlight{Pete} is an \textbf{embarrassment} to himself.
	\item \ExHighlight{Every man} is an \textbf{evidence} to himself that he did not make himself.
	\item \ExHighlight{He} is an \textbf{object} to himself.
	\item \ExHighlight{He} is an \textbf{island} to himself.
	\item \ExHighlight{The fool} is an \textbf{enemy} to himself.
	\item \ExHighlight{Every man} is an \textbf{end} to himself. 
	\item \ExHighlight{Every obstacle [he] throws in the way} [...] is an \textbf{injury} to himself.
	\item \ExHighlight{A man} is an \textbf{honor} to himself. 
	\end{xlist}
\end{exe}

\noindent This shows that what happens in example (\ref{ex:psy-stim-anim1}) should not be attributed to \textit{-ment}, and thus does not need to be modeled in this study. Rather, it is a post-lexical shift which is coerced by the context (but see \chapref{ch:conclusion} for ideas on how this judgment could be further corroborated in future research). 

\subsubsection{Cause and explanation}
\label{sec:psy-output-survey-caus}

In the previous section, we saw that \textsc{stimulus} is a frequent reading of psych nominalizations. Its eventive counterpart, \textsc{cause}, is also possible, at least with regard to the OE subset of the data set: 

\begin{exe}
	\ex \label{ex:psy-cause}
	\begin{xlist}
		\item \label{ex:psy-cause1}  The transition has never been a threat or a \textbf{disheartenment} to me, but sometimes in my newfound yuppy life I am caught off guard when I realize that hardly anyone I know or work with has, say, been on food stamps before. 
		{\small(Google BLOG crushingkrisis.com 2007)}
		\item \label{ex:psy-cause2} Revocation of his American Express card would have been a more consequential \textbf{abashment}. {\small(\acs{TIME} November 22, 1971)}
	\end{xlist}
\end{exe}

\noindent There are only two nouns without attestations for this reading, namely \textit{reassurement} and \textit{soothement}. Again, this can be attributed to scarcity of data.

The fact that \textsc{cause} is such a frequent reading in OE psych nouns indicates that the corresponding node (\textsc{cause:} \textit{event}) must be present in the frame representation. Therefore, we can conclude that a complex event analysis is more adequate than a simple psych-action.
Furthermore, the causing events which can be found in my data represent all imaginable kinds of event. For example, \textit{transition} in (\ref{ex:psy-cause1}) refers to a change-of-state, while \textit{revocation} in (\ref{ex:psy-cause2}) is an action. Importantly, I did not find perception-events as causing events. Therefore, an adequate frame analysis should include the event which has caused the psych-state, and the fact that this event needs to be perceived by the \textsc{experiencer} can be regarded as a presupposition. Of course, presuppositions can be included in frames, but they are not relevant for modeling derivational processes. 

As regards SE psych nouns, \textit{approvement} and \textit{musement} have not been found in an \textsc{explanation} reading (recall that \textsc{explanation} in a psych-reaction frame is the counterpart to \textsc{cause} in a psych-causation frame). \textit{Worriment} can be found in contexts which allow this interpretation, as exemplified in (\ref{ex:psy-cause-worriment}), but of course this nominalization is cross-listed as an OE psych noun. Therefore, the conclusion that this reading is based on the OE psych verb \textit{worry}, and not on the SE psych verb \textit{worry about}, suggests itself. 

\begin{exe}
	\ex \label{ex:psy-cause-worriment}
	Monitor your sites for outages, errors, and other \textbf{worriments} with Stella. {\small(\acs{WC} WEB cbinsights.com 2014)}
\end{exe}

\noindent The fact that a shift to \textsc{explanation} was not attested for SE psych nouns indicates that the event-type \textsc{psych-reaction} is less likely than the other two possible event types proposed for SE verbs (\textsc{experiencer psych-action} and \textsc{psych-state}). This assumption will be confirmed by the transpositional readings of SE nouns as discussed in \sectref{sec:psy-output-survey-transSE}. 

\subsubsection{Change-of-state}
\label{sec:psy-output-survey-cos}

\textsc{Change-of-state} readings were attested for six psych nouns: \textit{confoundment, disheartenment, endullment, enragement, soothement,} and \textit{upliftment}. Each type is exemplified in (\ref{ex:psy-c-o-s}). Note that examples (\ref{ex:psy-c-o-s2}) and (\ref{ex:psy-c-o-s3}) are ambiguous. Especially in (\ref{ex:psy-c-o-s3}), \textsc{change-of-state} needs to be regarded as the less likely reading. Here, \textit{soothement} and \textit{soothing} are either co-referential, both referring to the process of becoming calm, or they refer to the stimulus (\textit{soothement}) and the change-of-state (\textit{soothing}).  

\begin{exe}\sloppy
	\ex \label{ex:psy-c-o-s}
	\begin{xlist}
	\item \label{ex:psy-c-o-s1} \textbf{Endullment} is the dulling of people's minds as a result of their nonparticipation {\small(\acs{GB} NONFIC ManagingTeamEnvironment 1998)} 
	\item \label{ex:psy-c-o-s2} In her own case, Miss Reuben said, the \textbf{enragement} began when a professor told her that it really wouldn't matter if she finished her doctoral thesis. {\small(Google MAG news.Google.com 1972)} 
	\item \label{ex:psy-c-o-s3} IrishDayDreamer \textendash TOO weird for me!! Some \textbf{soothement}, maybe? some soothing, I hope, coming up! {\small(Google COMM dance.net 2009)} 
	\item \label{ex:psy-c-o-s4} [H]alf-breeds stimulated and intensified anxieties regarding the deleterious effects of alcohol on Indians, and how drunkenness might trouble their moral \textbf{upliftment} and eventual assimilation into white society. {\small(\acs{COCA} ACAD CulturalGeog 2010)} 
	\end{xlist}
\end{exe}

\noindent This finding does not entirely correspond to my predictions. In \sectref{sec:psy-output-predictions-frames} I used aktionsart diagnostics and determined that the second subevent of only the verbs \textit{endull, enrage, soothe} and \textit{uplift} is expected to be a change-of-psych-state (\textsc{effect:} \textit{change-of-psych-state}). 
The fact that I found their nominalizations in a \textsc{change-of-state} reading is therefore expected.
However, I also found \textit{confoundment} and \textit{disheartenment}, the base verbs of which were determined to have only a psych-state instead (\textsc{effect:} \textit{psych-state}; \sectref{sec:psy-input-prev-caus}). 
This shows that aktionsart diagnostics can provide useful insights, but should be taken with a grain of salt. 

I want to point out that the absence of the \textsc{change-of-state} reading in other psych nouns does not imply that change on the part of the experiencer is absent in the real world. Of course, they are not abashed/affrighted/annoyed/... before the event, but are so during (and after). However, some psych verbs and their nominalizations cannot focus on this change. This is indicated both by the results of \citeauthor{VanValin.2005}'s (\citeyear{VanValin.2005}) and \citeauthor{Alexiadou.2014}'s (\citeyear{Alexiadou.2014}) diagnostics (see \sectref{sec:psy-output-predictions-frames}), and by the results I presented in this section. 

\subsubsection[Transposition of a complex event]{Transposition of a complex event: Psych causation/reaction}
\label{sec:psy-output-survey-transOE}

With regard to transpositional readings, it has been claimed that only agentive variants of OE psych verbs are causative, and can thus produce eventive readings, while contexts like \textit{the joke's humiliation of the audience} are ungrammatical (see \sectref{sec:psy-output-lit}; \citealt{Grimshaw.1990}). In order to test this claim, I calculated agentivity scores for the psych base verbs in my data set. Based on this data, transpositional \textsc{event} readings can be expected for nouns based on verbs higher up on the agentivity scale (e.g. \textit{convince} and \textit{upset}). Correspondingly, this reading should not be possible for verbs further down on the scale (e.g. \textit{affright, endull} and \textit{nonplus}). Based on my data, I can conclude that this is not the case since I have found transpositional readings for all OE psych nouns, as exemplified in (\ref{ex:psy-transpo-OE}).

\begin{exe}
	\ex \label{ex:psy-transpo-OE}
	\begin{xlist}
	\item \label{ex:psy-transpo-OE1} Anybody who has watched the quadrennial \textbf{abashment} of ITV in a variety of idyllic locations around the world will be familiar with this picture. {\small(Google NEWS telegraph.co.uk 2010)}
	\item \label{ex:psy-transpo-OE2} Hello... Again. Sorry for the constant \textbf{annoyment}, but I can't log on now, it just gets stuck at 100 percent, and when I even managed to log on, I couldn't even check my about page. {\small(Google COMM ourworld.com 2014)}
	\end{xlist}
\end{exe}

\noindent What is more, my data also indicates that a low agentivity score of the base verb does not, in fact, preclude agentive contexts for its nominalization. For example, both \textit{affright} and \textit{endull} have an agentivity score of 0.5, and their nominalizations can still be found in clearly agentive contexts:

\begin{exe}
	\ex 
	\begin{xlist}
		\item \label{ex:affrightment} The campaign of terrorism and \textbf{affrightment} of investors. {\small(\acs{OED} ACAD PolitSciCityNewYork 1920)}
		\item \label{ex:endullment} All of this is well-thought-out, like a military campaign of blitzkrieg \textbf{endullment}. [...] While the Indians and Chinese and all are getting smarter, we're getting enstupidated at a hell of a pace. {\small(Google BLOG unz.com 2006)}
	\end{xlist}
\end{exe}

\noindent Furthermore, if \citet{Grimshaw.1990} is correct, transposition should only be possible with an agent/stimulus participant, and not with a non-agentive stimulus. This, however, is not the case either. For example, the stimulus in (\ref{ex:enragement}) is a book:

\begin{exe}
	\ex \label{ex:enragement} And after the interview she congratulated him on \ExHighlight{the book} and its \textbf{enragement} of Trump. {\small(Twitter @brithume 2018)}
\end{exe}

\noindent What can be said is that unambiguously transpositional readings can be found much more easily in agentive contexts. In non-agentive or underspecified contexts, attestations for transpositional readings tend to be ambiguous, most frequently between a transpositional and a \textsc{stimulus} reading:

\begin{exe}
	\ex 
	\begin{xlist}
	\item \label{ex:nonplusment} 
	If our wicket-keeper avoids a \textbf{nonplusment} I believe our city can win the Planet-Wide Affair of Honour. 
	{\small(Twitter @PaulMelancon 2012)}
	\item \label{ex:dumbfoundment} The rest of the week is given over to unravelling these \textbf{dumbfoundments}, befiddlements, general bafflegab and more. Wednesday eve is the only ease. {\small(Google MAG keywordspy.com)}
	\end{xlist}
\end{exe}

\noindent In all, the verdict is clear: All OE psych nouns allow agentive contexts, and eventive readings are by no means limited to agentive stimuli, contra \citeauthor{Grimshaw.1990}'s (\citeyear{Grimshaw.1990}) claim.


\subsubsection[Transposition of a simple event]{Transposition of a simple event: State and psych-action}
\label{sec:psy-output-survey-transSE}

Transpositional readings of SE psych nouns come in two event types: \textsc{psych-state} and \textsc{experiencer psych-action}. As predicted by the results of \citeauthor{VanValin.2005}'s (\citeyear{VanValin.2005}) event diagnostics, \textit{approvement} and \textit{worriment} occur in \textsc{psych-state} readings, as exemplified in (\ref{ex:psy-transpo-state}). \textit{Musement}, on the other hand, clearly has a \textsc{psych-action} reading, as in the context in (\ref{ex:psy-transpo-action}). 

\begin{exe}
	\ex \label{ex:psy-transpo-state}
	\begin{xlist}
	\item \label{ex:psy-transpo-state1} The TETRA paper was presented at the conference by Intetics President and CEO, Boris Kontsevoi [...]. The presentation received a round of applause and \textbf{approvement} of the international technology community. {\small(Google WEB intetics.com 2019)}
	\item \label{ex:psy-transpo-state2} We can learn to let go of the agitated states of mind, such as anger, \textbf{worriment}, resentment and fear 
	{\small(Google BLOG patch.com 2013)}
	\end{xlist}
	\ex \label{ex:psy-transpo-action} When a Peircean pragmatist assesses logically an experience of free and spontaneous \textbf{musement} over ``the three Universes of experience'' [...], such a logician  will recognize [...]	an example of abductive reasoning  {\small(\acs{GB} ACAD GracingHumanExperience 2007)}
\end{exe}

\noindent Interestingly, both base verb variants, \textit{worry} and \textit{worry about}, can be found with transposed semantics in their nominalizations. 
\textit{Worriment about} is found in the expected syntactic-semantic constellation, namely with the about-PP introducing the stimulus, as in (\ref{ex:psy-worriment1}).
Parallel to that, an of-PP would be expected to be an indicator for a transpositional reading of \textit{worry}, introducing the experiencer (parallel to \textit{abashment of ITV} in example (\ref{ex:psy-transpo-OE1}) above). This construction, however, seems to only be used for \textsc{psych-state} readings, as in (\ref{ex:psy-worriment2}). In this context, the of-PP introduces the stimulus. \textit{Worriment} in a complex event reading with a full eventive structure as in the constructed example in (\ref{ex:psy-worriment3}) is not attested. 

\begin{exe}
	\ex \label{ex:psy-worriment}
	\begin{xlist}
	\ex[]{\label{ex:psy-worriment1}it shed lights on the \textbf{worriment about} the relevance of age-related deterioration in physical [...] capacities (\acs{NOW} NEWS QSWOWNew 2019)} 
	\ex[]{\label{ex:psy-worriment2}Berger's essay expresses his \textbf{worriment} of the role publicity takes in our lives. (Google  BLOG blogs.baruch.cuny.edu 2017)}
	\ex[*]{\label{ex:psy-worriment3}John's\textsubscript{EXP} constant worriment of me\textsubscript{STIM}} 
	\end{xlist}
\end{exe}

\subsubsection{Psych-state}
\label{sec:psy-output-survey-resultstate}

Unsurprisingly, OE psych nouns are very easily found denoting \textsc{psych-states}, the reading most frequently discussed in the literature on psych nominalizations (see e.g. \citealt{Pesetsky.1995}). Two examples from my data set are given in (\ref{ex:psy-resultstate}). 

\begin{exe}
	\ex \label{ex:psy-resultstate}
	\begin{xlist}
		\item \label{ex:psy-resultstate1} I know a lot of our compatriots also feel the same angst, consternation and \textbf{confoundment}. {\small(\acs{GloWbE} NEWS leadership.ng 2012)} 
		\item \label{ex:psy-resultstate2}  People get surprised by the level of \textbf{upsetment} in fandom. 
		{\small(Google COMM planetmondas.com 2014)} 
	\end{xlist}
\end{exe}

\noindent With regard to SE psych nouns, we need to distinguish between different kinds of \textsc{psych-state} readings. For \textit{approvement} and \textit{worriment}, as discussed in the previous section, this reading is transpositional.\footnote{For \textit{worriment}, a more fitting formulation would be `potentially transpositional,' since \textsc{psych-state} readings can either be transpositional (from \textit{worry about}), or non-transpositional (from \textit{worry}) \textendash{} if \textit{worriment} occurs without `about,' there is no way to know.} \textit{Musement}'s central node is a psych-action, but its frame also contains a psych-state node. The corresponding reading is therefore produced by a meaning shift:

\begin{exe}
	\ex \label{ex:psy-resultstate-muse}
 	A cock was crowing in the distance. He studied the countryside with \textbf{musement}. Here forms were gentle on the eye. {\small(\acs{GloWbE} BLOG kaganof.com 2012)} 
\end{exe}

\subsubsection{Unattested shifts}
\label{sec:psy-output-survey-unatt}

First of all, as predicted, no shifts to \textsc{experiencer} or to \textsc{agent} readings can be found in the corpora. This is in line with the finding from the previous chapter that \textit{-ment} does not allow [+animate] readings. The finding is also consistent with previous research by \citet{Melloni.2011}, who does not find \textsc{experiencer} readings for Italian \textit{-mento} nominalizations either. As was the case with the COS data, there was a single data point as an exception to this rule, namely a sole [+animate] \textsc{stimulus} reading. I have attributed this to a post-lexical shift.

Furthermore, there are a number of readings which \textit{-ment} can produce according to the literature, but which were not expected to be relevant for psych nominalizations. The corresponding shifts were not attested in the corpora. These readings are \textsc{agentive-collective, location, [−animate] patient, product,} and \textsc{result}. Since these five readings are not represented in the base verb frames, we need not concern ourselves with these further. 

Finally, the psych data exhibits some gaps, that is, readings which are expected, and which are systematically attested for a subcategory of psych nouns, but which I have not been able to find attested for single nominalizations in the data set. This issue will be discussed in detail in \chapref{ch:observations}.

\subsubsection{Summary}
\label{sec:psy-output-survey-sum}

In this section, I have been able to identify a clear range of readings which is possible for (certain subsets of) OE and SE psych verb nominalizations. Together with a thorough investigation of the contexts these readings occur in, I have also been able to draw conclusions about their event structure and causativity.

With regard to OE psych verbs and nouns, I have concluded that they are best modeled as complex causation events. 
As expected, most OE psych verbs and nouns have a \textsc{psych-state} as the second subevent, and a subset has a change-of-psych-state instead (although not all members of this subset had been predicted by aktionsart diagnostics).
For SE psych nouns, three different event types had been proposed, and I was able to show that \textit{muse over} is best modeled as a psych-action, while \textit{approve of} and \textit{worry about} denote states. 

Corresponding to these event types, there are four kinds of transpositional reading represented in my data: \textsc{psych-state causation} or \textsc{change-of-psych-state causation} for OE psych verbs, and \textsc{experiencer psych-action} or \textsc{psych-state} for SE psych verbs. Contra existing claims in the literature, agency does not play a role in the availability of transpositional eventive readings at all.

Non-transpositional eventive readings are attested as well: \textsc{Psych-state} is attested for all psych verbs, \textsc{cause} for all OE psych verbs, and \textsc{change-of-psych-state} for the expected subset of OE psych verbs.

With respect to participant readings, I only found shifts to \textsc{stimulus/instrument} attested. This was expected for two reasons: First, the other readings reported for \textit{-ment} (e.g. \textsc{location} or \textsc{product}) are not pertinent for psych base verbs. Second, \textsc{agent} and \textsc{experiencer} readings are prevented by the animacy constraint as proposed in the preceding chapter.

\subsection{Formalization of psych nominalization}
\label{sec:psy-output-formal}

Based on my findings, I will now model the nominalization of psych verbs. 
First, I will revisit the VerbNet-based frames (\sectref{sec:psy-output-formal-frames}).
Then, I will update the type signature to include all types required for the frames and the inheritance hierarchy (\sectref{sec:psy-output-formal-TS}).
I will conclude this section by proposing an inheritance hierarchy for \textit{-ment} on psych verb bases (\sectref{sec:psy-output-formal-inherit}).

\subsubsection{Frame representations}
\label{sec:psy-output-formal-frames}

The psych nouns in my data set can be modeled by means of four frame analyses, corresponding to the four event types represented in the base verbs: psych-state causation and change-of-psych-state causation for OE psych nouns, as well as psych-state and experiencer psych-action for SE psych nouns. Table \ref{tab:psych-nouns} lists the event type for each psych noun in my data set.

\begin{table}
    \caption[Event types of psych nouns in my data set]{\label{tab:psych-nouns}Event types of psych nouns in my data set}
    \begin{tabular}{
        l
        >{\RaggedRight\arraybackslash}p{0.5\textwidth}
      }
      \lsptoprule
      Event type & Nominalizations \\ 
      \midrule
      psych-state causation & abashment, affrightment, annoyment, bemusement, bumfuzzlement, convincement, dumbfoundment, enrapturement, nonplusment, perturbment, reassurement, staggerment, upsetment, worriment \\
      change-of-psych-state causation & confoundment, disheartenment, endullment, enragement, soothement, upliftment \\
      psych-state & approvement, worriment \\
      experiencer psych-action & musement \\
      \lspbottomrule
    \end{tabular} 
\end{table}

\begin{sloppypar}
The frame in Figure \ref{fig:frame-psy1} models the majority of psych nouns in my data set, namely those based on psych-state causation verbs, that is, OE psych verbs which do not contain a change-of-psych-state in their semantics. 
Possible referent nodes are \avmbox{0}, \avmbox{2}, \avmbox{4}, and \avmbox{6}. Shifts to \avmbox{1} and \avmbox{3} are prevented by the animacy constraint, which is modeled in the inheritance hierarchy. 
Two things should be noted: 
First, I am keeping the indexing constant across all frames in this section, which will facilitate comparisons. For example, \textsc{stimulus} is indexed with \avmbox{2} in all four frames.   
Second, I have included the participants \textsc{agent} and \textsc{instrument} here. In the type signature, these participants are defined as optional. In other words, the modeled contexts are not necessarily, but potentially, agentive.\footnote{The base verbs allow agentive contexts to different extents, as I have shown with the agentivity scores in \sectref{sec:psy-input-prev-SEvsOE}. The probability of an \textsc{agent} attribute could be included in the frame for a given verb (see \sectref{sec:cos-output-formal-frames} for how this could be achieved), but since agentivity did not have an effect in my data set, this would not provide any added benefit.}
\end{sloppypar}

\begin{figure}
		\begin{avm}
			\avml
			\@0	
			\[
			\avmspan{\type{psych-state causation}}\cr
			\att{agent} & \@1 \cr
			\att{instrument} & \@2 \cr
			\att{stimulus} & \@2 \cr
			\att{experiencer} & \@3 \cr
			\att{cause} & \@4 
			\[ 
			\type{event} \cr
			\att{participant} & \@2
			\] \cr
			\textsc{effect} & \@6 
			\[ 
			\textit{psych-state} \cr
			\textsc{experiencer} & \@3 
			\] \cr
			\] \cr
			{\textsc{ref} $=$ \{\@0, \@2, \@4, \@6\} }	
			\avmr
		\end{avm}
		\caption[Frame for OE psych nouns with a caused psych-state]{Frame for OE psych nouns with a caused psych-state (e.g. Charlie being in an annoyed-state because of some event Jael is involved in)}
		\label{fig:frame-psy1}
\end{figure}


The frame in Figure \ref{fig:frame-psy2} models OE psych nominalizations with bases which have a change-of-psych-state as their second subevent. Apart from the second subevent, it is identical to the previous frame. Correspondingly, \avmbox{5} is added to the set of possible referents, indicating that a shift to \textsc{change-of-psych-state} readings is possible for these nominalizations. 


\begin{figure}
		\begin{avm}
			\avml
			\@0	
						\[
						\avmspan{\type{change-of-psych-state causation}}\cr
						\att{agent} & \@1 \cr
						\att{instrument} & \@2 \cr
						\att{stimulus} & \@2 \cr
						\att{experiencer} & \@3 \cr
						\att{cause} & \@4 
						\[ 
												\type{event} \cr					
						\att{participant} & \@2
						\] \cr
						\textsc{effect} & \@5 
			\[ 
			\avmspan{\textit{change-of-psych-state}} \cr
			\textsc{result-state} & \@6 
			\[ 
			\textit{psych-state} \cr
			\textsc{experiencer} & \@3\cr
			\] \cr
			\] \cr
			\] \cr
			{\textsc{ref} $=$ \{\@0, \@2, \@4, \@5, \@6\} }	
			\avmr
		\end{avm}
		\caption[Frame for OE psych nouns with a caused change-of-psych-state]{Frame for OE psych nouns with a caused change-of-psych-state (e.g. Charlie attaining an uplifted-state because of an event Jael is involved in)}
		\label{fig:frame-psy2}
\end{figure}



Figure \ref{fig:frame-psy3} presents a formalization of SE psych nouns derived from stative base verbs. They are modeled as states with two participants, an experiencer and a stimulus.   

\begin{figure}
		\begin{avm}
			\avml
			\@6	
				\[
				\avmspan{\textit{psych-state}}\cr
				\textsc{experiencer} & \@3 \cr
				\textsc{stimulus} & \@2 
				\] \cr
				{\textsc{ref} $=$ \{\@6, \@2\}}
			\avmr
		\end{avm}
		\caption[Frame for SE psych nouns with state bases]{Frame for SE psych nouns with state bases (e.g. Charlie approving of Jael)}
		\label{fig:frame-psy3}
\end{figure}

The last frame in this section, Figure \ref{fig:frame-psy4}, is for \textit{musement}, which is based on the experiencer psych-action verb \textit{muse over}.\footnote{Since the other proposed subtype of \textit{psych-action}, \textit{stimulus psych-action}, did not turn out to be relevant as a frame type, I could technically simply call this frame \textit{psych-action}. However, I have decided to stick with the more precise label for clarity.} Here, the frame type and the co-indexation of \textsc{experiencer} with \textsc{actor} indicate that it is the experiencer of the psych-event that takes over an active role.

\begin{figure}
		\begin{avm}
			\avml
						\@0	
						\[
						\avmspan{\textit{experiencer psych-action}}\cr
						\textsc{experiencer} & \@3
							\[ 
							\textit{entity}\cr 
							\textsc{psych-state} & \@6 
							\] \cr
						\textsc{actor} & \@3 \cr
						\textsc{stimulus} & \@2
						\] \cr
			{\textsc{ref} $=$ \{\@0, \@2, \@6\} }	
			\avmr
		\end{avm}
		\caption[Frame for SE psych nouns with experiencer psych-action bases]{Frame for SE psych nouns with experiencer psych-action bases (e.g. Charlie actively musing over Jael)}
		\label{fig:frame-psy4}
\end{figure}

\subsubsection{Updated semantic categories and type signature}
\label{sec:psy-output-formal-TS}

I will now incorporate the insights gained in this analysis into the participant hierarchy and the type signature. 
In \chapref{ch:method}, I already hinted that the original figures are partly contradictory. More precisely, VerbNet's \textsc{stimulus} is categorized as a hyponym to \textsc{causer}, but it can actually be co-referential with \textsc{agent} or \textsc{instrument}. 
Moreover, \textsc{experiencer} is a grandchild to \textsc{undergoer}, which clashes with the event type \textit{experiencer psych-action}. 
I will solve both issues by allowing multiple parents in the hierarchy.\footnote{It may be more accurate to include complex subtypes of stimulus and experiencer instead, for instance \textit{instrument $\wedge$ stimulus} as a child of instrument and stimulus.} Still, the hyponymy-relations as represented in VerbNet do reflect the most frequent constellations, which is why I use dashed lines to indicate the less frequent relations. The revised participant hierarchy is given in Figure \ref{fig:participants3}. Core participants are indicated by italics.

\begin{figure}\small
		\begin{forest}
			[Participants 
				[Actor, name=act
					[\HighlightText{Agent}\\{[+intentional]}, name=ag, tier=1
					]
					[Causer\\{[−intentional]}, tier=1
						[\HighlightText{Stimulus}, name=stim, tier=2
						]
					]
				]			
				[Undergoer
					[\HighlightText{Patient}\\{[+affected]}
						[\HighlightText{Experiencer}\\{[+awareness]}, name=exp, tier=1
						]
					]
					[\HighlightText{Instrument}, name=instr, tier=1
					]
				]
				[Place
					[Goal
						[\HighlightText{Result}, tier=1
							[Product\\{[+concrete]}, tier=2
								[\HighlightText{Implicit product}
								]
							]
						]
					]
				]
			]
		\draw[dashed] (ag.south) -- (stim.north);
		\draw[dashed] (instr.south) -- (stim.north);
		\draw[dashed] (act.south) -- (exp.north);
		\end{forest}
		\caption[Revised participant categories for semantic coding]{\label{fig:participants3}Revised participant categories for semantic coding}
\end{figure}

Let us turn to the type signature in Figure \ref{fig:signature-update2}. For readability, only information which is relevant for this chapter is included, leaving out types needed to model COS nominalizations only.
The type signature includes only those eventive types that are either possible readings of psych verbs (e.g. \textit{experiencer-psych-action}), or that are needed as frame elements (e.g. \textit{change-of-psych-state}). 
The original type signature can thus be slimmed down by eliminating those types which are not required to formalize my data. To save space, I also leave out the intermediate types \textit{psych-action} and \textit{change-of-state causation}, and include their appropriateness conditions in their respective bottom-most descendant instead (\textit{exp psych-action} and \textit{change-of-psych-state causation}). Types which I have used to label the central node of my psych frames are color-coded as green.

\begin{sidewaysfigure}
			\begin{forest} for tree={font=\itshape}
			[\normalfont{\sffamily{T}}
				[event, tier=1
					[action
						\\\textsc{act} $\doteq$ \textsc{ag :} ent{,} \textsc{anim :} true
						\\ $\veebar$ \textsc{act} $\doteq$ \textsc{causer :} ent{}
						\\ $\pm$\textsc{instr :} ent{,} \textsc{anim :} false   
						\\ $\pm$\textsc{pat: } ent{}
								[\textcolor{lsRichGreen}{exp psych-action}
								\\\textsc{stim}\\
								\textsc{act} $\doteq$ \textsc{exp :} entity{,} \textsc{anim :} true
								]
					]
					[state\\\textsc{pat}
						[\textcolor{lsRichGreen}{psych-st}\\\textsc{exp}, name=psst]
					]
					[c-o-s\\\textsc{pat}\\\textsc{res-st :} state, name=cos
						[c-o-psych-st\\\textsc{exp}\\\textsc{res-state :} psych-state, name=cops]
					]
					[caus\\
						\textsc{act} \\ 
						\textsc{undergoer} \\ 
						$\pm$ \textsc{instr}\\
						\textsc{cause} : event\\ 
						\textsc{effect} : event
								[\textcolor{lsRichGreen}{c-o-psych-st caus}\\
								\textsc{stim}\\
								\textsc{exp}\\
								\textsc{effect :} c-o-psych-st, name=caps
								]
								[\textcolor{lsRichGreen}{psych-st caus}\\
								\textsc{effect}: psych-st
								]
					]
				]
					[entity
						[\normalfont{[\ldots]}]
					]
				]
			\end{forest}
		\caption[Revised type signature of eventive categories (psych subset)]{\label{fig:signature-update2} Revised type signature of eventive categories (psych subset). Optionality is indicated by $\pm$, possible central node types of psych nouns are indicated by green. Abbreviations: act = actor, ag = agent, anim = animacy, caus = causation, c-o-psych-st = change-of-psych-state, c-o-s = change-of-state, ent = entity, exp = experiencer, instr = instrument, pat = patient, psych-st = psych-state, res-st = result-state, stim = stimulus.}
\end{sidewaysfigure}

\subsubsection{Lexical rules and inheritance hierarchy}
\label{sec:psy-output-formal-inherit}

For psych nouns, I found seven distinct readings:
The eventive readings \textsc{experiencer psych-action, psych-state, change-of-psych-state causation, psych-state causation, causing-event} and \textsc{change-of-psych-state}, as well as the participant reading \textsc{stimulus}. I will use seven LFRs to model their derivation. 

In the previous chapter, I spelled out one exemplary LFR, the one producing \textsc{result-state} readings (\sectref{sec:cos-output-formal-inherit}). The corresponding reading for psych verbs is \textsc{psych-state}. For COS verbs, the LFR checked for an attribute labeled \textsc{result-state} in the base verb frame, and produced a shift to its target node if it found this attribute. In psych nominalization, the situation is more complex since the psych-state is introduced in different ways: 
The corresponding node typed \textit{psych-state} is either the central node of the frame, or it is introduced by one of three different attributes (\textsc{result-state, psych-state} or \textsc{effect}). 
The LFR I propose to model this can be seen in Figure \ref{fig:lexrule-resst2}:
The central node of the s-frame is typed as \textit{event}. 
The fact that it is some sort of psych event is ensured by the presence of a node \avmbox{x}, which is typed as \textit{psych-state}. 
The notation `$\cdots$' indicates an attribute path of unspecified length from the central node to \avmbox{x}. 
If \textit{psych-state} is itself the central node (recall that \textit{state} is a subtype of \textit{event} in my ontology), the attribute path is of length 0. That is, \avmbox{x} and \avmbox{y} are co-referential in that case.

\vfill
\begin{figure}[H]
		\begin{avm}
			\[\textit{lexeme}\cr
			\textsc{phon} & {/\@z-\textit{ment}}/\cr
			\textsc{cat} & N \cr
			\textsc{sem} & \[ s-frame & \@y 
			\]\cr
			\textsc{m-base} & 
			\[\textit{lexeme}\cr
			\textsc{phon} & \@z \cr
			\textsc{cat} & V \cr
			\textsc{sem} &
			\[ s-frame & \@y
			\[ 
			\textit{event} \cr
			$\cdots$ \@x \textit{psych-state}\cr
			\]
			\]
			\]  \cr
			\avmspan{{\textsc{ref} $=$ \{\@x\} }}
			\cr
			\]	
		\end{avm}
		\caption[Lexical rule for \textsc{psych-state} readings of \textit{-ment} on psych verbs]{Lexical rule for \textsc{psych-state} readings of \textit{-ment} on psych verbs}
		\label{fig:lexrule-resst2}
\end{figure}
\vfill\pagebreak

The seven LFRs are incorporated into the inheritance hierarchy in Figure \ref{fig:inheritance-complete-vert2} (see also \sectref{sec:fr-der-ref-inherit}).  
For better readability, I am using only four nominalizations for illustration, one for each class of psych noun: \textit{annoyment} for psych-state causation, \textit{upliftment} for change-of-psych-state causation, \textit{approvement} for psych-state, and \textit{musement} for experiencer psych-action.

\begin{figure}
	\resizebox{\linewidth}{!}{%
	\begin{tikzpicture}
	\draw (2,0) node(0){lexeme};
	\draw (2,-1) node(1){\textit{v-n-lfr}};
	\draw (4,4) node(2){\Boxed{\textsc{phon}}};
	\draw (4,-5) node(3){\Boxed{\textsc{sem}}};
	\draw (4,-6) node(4){\textsc{anim : }false};
	\draw (12,5) node(5){}; 
	\draw (6.5,5) node(55)[anchor=west]{
		\begin{avm}
		\sort{0}{\[{\textit{x-ment}}\cr
			\textsc{ph} \@1+ment\cr
			\textsc{m-base} \[\textsc{ph} \@1 \]
			\]}
		\end{avm}};
	\draw (55.east) -- (5.east);
	\draw (6.5,2.5) node(11)[anchor=west]{
		\begin{avm}
		\sort{1}{\[\textit{exp psych-act-n}\cr
			\textsc{ref} \@x\cr
			\textsc{m-base} \[ \textsc{sem} \[ \@x \type{caus-evt}  \] \]
			\]}
		\end{avm}};
	\draw (6.5,0) node(12)[anchor=west]{
		\begin{avm}
		\sort{2}{\[\textit{psych-st-n}\cr
			\textsc{ref} \@x\cr
			\textsc{m-base} \[ \textsc{sem} \[ \@x \type{psych-st} \] \]
			\]}
		\end{avm}};
	\draw (6.5,-2.5) node(13)[anchor=west]{
		\begin{avm}
		\sort{3}{\[\textit{c-o-psych-st caus-n}\cr
			\textsc{ref} \@x\cr
			\textsc{m-base} \[ \textsc{sem} \[ \@x \type{c-o-psych-st caus} \] \]
			\]}
		\end{avm}};
	\draw (6.5,-5) node(14)[anchor=west]{
		\begin{avm}
		\sort{4}{\[\textit{psych-st caus-n}\cr
			\textsc{ref} \@x\cr
			\textsc{m-base} \[ \textsc{sem} \[ \@x \type{psych-st caus} \] \]
			\]}
		\end{avm}};
	\draw (6.5,-7.5) node(15)[anchor=west]{
		\begin{avm}
		\sort{5}{\[\textit{causing-evt-n}\cr
			\textsc{ref} \@x\cr
			\textsc{m-base} \[ \textsc{sem} \[ \textsc{cause} \@x \] \]
			\]}
		\end{avm}};
	\draw (6.5,-10) node(16)[anchor=west]{
		\begin{avm}
		\sort{6}{\[\textit{c-o-psych-st-n}\cr
			\textsc{ref} \@x\cr
			\textsc{m-base} \[ \textsc{sem} \[ \textsc{effect} \@x \] \]
			\]}
		\end{avm}};
	\draw (6.5,-12.5) node(18)[anchor=west]{
		\begin{avm}
		\sort{7}{\[\textit{stimulus-n}\cr
			\textsc{ref} \@x\cr
			\textsc{m-base} \[ \textsc{sem} \[ \textsc{stimulus} \@x \] \]
			\]}
		\end{avm}};
	\draw (15,2.5) node(21){\begin{tabular}{l}
		musement
		\end{tabular}};
	\draw (15,0) node(22){\begin{tabular}{l}
		annoyment\\approvement\\musement\\upliftment
		\end{tabular}};
	\draw (15,-2.5) node(23){\begin{tabular}{l}
		upliftment
		\end{tabular}};
	\draw (15,-5) node(24){\begin{tabular}{l}
		annoyment
		\end{tabular}};
	\draw (15,-7.5) node(25){\begin{tabular}{l}
		annoyment\\upliftment
		\end{tabular}};
	\draw (15,-10) node(26){\begin{tabular}{l}
		upliftment
		\end{tabular}};
	\draw (15,-12.5) node(28){\begin{tabular}{l}
		annoyment\\\textsuperscript{?}approvement\\musement\\upliftment
		\end{tabular}};
	\draw [dotted] (0) -- (1);
	\draw (1.east) -- (2.west); 
	\draw (1.east) -- (3.west); 
	\draw (2.east) -- (55.west); 
	\draw (3.south) -- (4.north); 
	\draw (4.east) -- (11.west); 
	\draw (4.east) -- (12.west); 
	\draw (4.east) -- (13.west); 
	\draw (4.east) -- (14.west); 
	\draw (4.east) -- (15.west); 
	\draw (4.east) -- (16.west); 
	\draw (4.east) -- (18.west); 
	\draw (11.east) -- (21.west); 
	\draw (12.east) -- (22.west); 
	\draw (13.east) -- (23.west); 
	\draw (14.east) -- (24.west); 
	\draw (15.east) -- (25.west); 
	\draw (16.east) -- (26.west); 
	\draw (18.east) -- (28.west); 
	\draw (5.east) -- (21.west); 
	\draw (5.east) -- (22.west); 
	\draw (5.east) -- (23.west); 
	\draw (5.east) -- (24.west); 
	\draw (5.east) -- (25.west); 
	\draw (5.east) -- (26.west); 
	\draw (5.east) -- (28.west); 
	\end{tikzpicture}
	}
	\caption[Inheritance hierarchy of lexical rules for \textit{-ment} on psych verbs]{Inheritance hierarchy of lexical rules for \textit{-ment} on psych verbs. Abbreviations: causing-evt = causing-event, c-o-psych-st = change-of-psych-state, c-o-psych-st caus = change-of-psych-state causation, exp psych-act = experiencer psych-action, psych-st = psych-state, psych-st caus = psych-state causation, res-st = result-state.}
	\label{fig:inheritance-complete-vert2}
\end{figure}

According to this inheritance hierarchy, psych nominalization proceeds as follows.
As proposed in the previous chapter, the animacy constraint takes effect first. It precludes shifts to \textsc{agent} and \textsc{experiencer} readings from the outset. 
Next, seven LFRs come into play, producing seven distinct readings. Six of these readings are eventive, and one is a participant reading. Of the eventive readings, four are transpositional (LFRs 1 to 4). The innermost level of their AVMs does not include an attribute, but rather the type of the respective central node. Note, however, that LFR 2 also produces non-transpositional \textsc{psych-state} readings since \textit{psych-state} occurs as a non-central node in the base verbs as well. 
The bottom three LFRs check for attributes in the base verb frame, for instance \textsc{cause} in LFR 5. If the base verb has the respective attribute in its frame, the corresponding reading is produced by shifting reference to the attribute's value.
As in the COS data set, there are some gaps in the psych data. That is, I have not been able to find all nominalizations in all expected readings. In Figure \ref{fig:inheritance-complete-vert2} one such gap is represented; it is marked with a superscript question mark.

\section{Summary of Chapter 5}
\label{sec:psy-output-sum}

In this chapter, I have investigated which readings \textit{-ment} can produce on psych verb bases. The data set consisted of 23 nominalizations, more precisely, three SE psych nouns and 20 OE psych nouns. The main issue in this chapter was the variety of opinions and approaches which can be found in the existing literature on the base verbs. There is disagreement with regard to their semantics, their event structure, and their participants. Applying a range of tests, I also identified differing properties with regard to the base verbs' agentivity and aktionsart.

I addressed this variety of analyses by proposing eight frame variants. The ones for OE psych verbs differed with regard to complexity (complex vs. simple event), the first subevent (perception-event vs. underspecified event), the second subevent (psych-state vs. change-of-psych-state), and agentivity (agentive vs. non-agentive). For SE psych verbs, two frames modeled a distinction going back to aktionsart (activity vs. state), while one analyzed verbs from this subclass as complex, but non-causative events.

Based on these frames and on existing literature, I then formulated expectations with regard to the possible readings of psych nouns, and examined corpus data to verify or falsify these predictions. Regarding noun semantics, my study had the following results: I systematically found transpositional readings and, contrary to the claim made by \citet{Grimshaw.1990} and others, agentivity of the base verb did not play a role for the availability of these readings at all. I also found shifts, namely to the eventive categories \textsc{change-of-psych-state} and \textsc{psych-state}, as well as to the non-eventive category \textsc{stimulus}. A shift to a \textsc{change-of-psych-state} reading was only possible for the four base verbs which I had previously determined to have a change-of-psych-state node in their frame representation. Furthermore, I confirmed the animacy constraint as proposed in the previous chapter. For psych nouns, it rules out \textsc{agent} and \textsc{experiencer} as possible readings produced by derivation. As before, post-lexical shifts to an \textsc{agent} reading are attested, but very rare.

Based on my data, I concluded that psych nominalizations and their base verbs can be appropriately modeled by four distinct generalized frames: \textit{experiencer psych-action, psych-state, change-of-psych-state} causation, and \textit{psych-state causation}. These frames had been predicted by the results of applying \citeauthor{VanValin.2005}'s (\citeyear{VanValin.2005}) aktionsart diagnostics. I concluded my analysis by suggesting a set of seven LFRs for \textit{-ment} nominalizations with psych verb bases. These were incorporated into an inheritance hierarchy for \textit{-ment} suffixation, alongside the animacy constraint.
