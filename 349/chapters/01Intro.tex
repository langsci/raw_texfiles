\chapter[Introduction]{Introduction}\label{ch:intro}\largerpage

Research in the field of derivational semantics investigates how the internal structure of derived words is related to their meaning. One central problem in this line of research is \HighlightText{affix polysemy}: one affix being able to generate several possible readings. An oft-cited example is the English nominalizing suffix \textit{-er}. Attached to a base verb, it can exhibit the following readings (see \citealt{Bauer.2013,Lieber.2016}):\footnote{Furthermore, \textit{-er} can attach to location nouns to produce \textsc{inhabitant} readings (\textit{Londoner}), and to complex bases consisting of a number and a measure term to produce \textsc{measure} readings (\textit{20-pounder}).} 

\begin{exe}
	\ex 
	\begin{xlist}
		\item \label{ex:er-instr} \textsc{instrument}: \textit{opener}
		\item \textsc{agent}: \textit{writer}
		\item \label{ex:er-inanimpat} [−animate] \textsc{patient}: \textit{loaner}
		\item \label{ex:er-animpat} [+animate] \textsc{patient}: \textit{shooter}\footnote{See \citet[]{Lieber.2016} for an example (p. 67) and a discussion (p. 74) of this unexpected reading.}
		\item \label{ex:er-loc} \textsc{location}: \textit{diner}
	\end{xlist}
\end{exe}

\noindent
In addition, it is often the case that several readings are possible even within one and the same derivative: According to the \citeauthor{OED} (OED), an \textit{opener} can not only be an \textsc{instrument} (`An implement or device for opening tins, bottles, etc.'), but also an \textsc{agent} (`A person who opens or begins something') or a [−animate] \textsc{patient} (`A case or package that is opened by customs officials'). That is, \textit{opener} can exhibit readings (\ref{ex:er-instr}) to (\ref{ex:er-inanimpat}).
Importantly, however, a given derivative will most likely not exhibit the full range of readings that its affix can potentially produce: The \citetalias{OED} does not list \textit{opener} as `someone who has been opened' or as `the place of opening' (readings (\ref{ex:er-animpat}) and (\ref{ex:er-loc})). 


In recent years, affix polysemy has attracted considerable attention in linguistic research. 
Especially notable is the \textit{Oxford reference guide of English morphology} (\citealt{Bauer.2013}), which is the only work to date that includes a data-driven description of the whole range of semantic possibilities in English derivation. Recently, \citet{Lieber.2016} has added a theoretical vantage point to a subset of \citeauthor{Bauer.2013}'s insights, modeling English nominalizations in her Lexical Semantic Framework. 


Major studies have investigated a variety of sub-topics, only a few of which I will mention here by way of example (see \citealt{Lieber.2016} for a more comprehensive overview; see also the collection of articles in \citealt{Bauer.2015}).\footnote{By virtue of the scope of this study, I limit this overview to research dealing with the process of nominalization.} 
The semantic distinction between event and result nominalizations has figured prominently in research on morphology and syntax, especially in the generative tradition (e.g. \citealt{Alexiadou.2001,Borer.2013,Grimshaw.1990,Melloni.2007,Melloni.2011,Fradin.2011}), and has also been discussed extensively by James Pustejovsky in the field of computational lexical semantics (e.g. \citeyear{Pustejovsky.1995,Pustejovsky.1998}). 
Morphologists have also given considerable attention both to suffixes which create person nouns (such as \textit{-er} and \textit{-ee}, see e.g. \citealt{Lieber.2004,Booij.2004}), and to those giving rise to abstract interpretations (such as \textit{-ness} or \textit{-ity}, see e.g. \citealt{Trips.2009,Arndt-Lappe.2014}). Especially \textit{-er} has taken ``center stage,'' as \citet{Rainer.2014poly} phrases it.
Recently, quantitative approaches have gained momentum in the field, for example Analogical Modeling (e.g. \citealt{Arndt-Lappe.2014}) or Distributional Semantics (e.g. \citealt{Lapesa.2018,Wauquier.2020}).


Apart from this extensive research literature, derivational semantics has figured internationally as a special topic of conferences, notably the
International Morphology Meeting (Vienna 2012, see \citealt{Rainer.2014coll}), the Mediterranean Morphology Meeting (Dubrovnik 2013), and the Semantics of Derivational Morphology workshop (Düsseldorf 2014, see \citealt{Arndt-Lappe.2015}).

 Despite this intensive research, a workable model of derivational semantics is still under debate.
I see two main issues with existing analyses of affix polysemy.
First, what may be the chief desideratum in the endeavor of modeling affix polysemy: Most existing approaches put a focus on the semantic properties of the derivative, largely disregarding or leaving implicit what is contained in the base word's semantics.
This oversight has been acknowledged, for example, by \citet{Bauer.2013}, who observe that there is often a non-arbitrary relationship between the semantics of the base word and that of the derivative (p. 213), and by \citet[]{Lieber.2004}, who leaves open ``[e]xactly what the verbal body looks like'' (p. 72) in her analysis of deverbal nouns. 


That the base does play a major role for the availability of readings of its derivative can be illustrated with the example of \textit{opener}: Its three possible readings are central elements of its base verb's semantics. Thus, the action denoted by the verb \textit{open} involves someone who opens (\textsc{agent}), something that is opened (\textsc{patient}), and something that can be used to open (\textsc{instrument}).
Other central elements in the semantics of \textit{open} cannot be targeted by \textit{-er}, but are available for other derivational processes. For example, according to the \citetalias{OED}, the \textsc{action} of making open can be targeted by \textit{-ing} (\textit{opening}), and the \textsc{result} of opening can be targeted both by \textit{-ing} and by conversion (\textit{opening} and \textit{open}\textsubscript{N}; see also \citealt{Andreou.2020}). 


This example shows that the process of derivation can be considered compositional in the sense that both the base and the affix make a contribution: The base offers an array of semantic elements, and the affix may select from this array to construct the derivative's meaning. A comprehensive analysis of affix polysemy therefore requires a precise characterization of the interaction between input and output semantics. More precisely, a decompositional approach is needed to identify the elements in the base word's semantics that are potential targets for an affix. 


The second issue with the current state of research is that existing analyses either remain on a descriptive level (e.g. \citealt{Bauer.2013}), or opt for a semantically underspecified analysis (e.g. Pustejovsky's dot objects, \citealt{Pustejovsky.1998}, or Lieber's skeletal features, \citealt{Lieber.2004} et seq.).
However, the example of \textit{opener} shows that there is a great deal to be gained from an explicit semantic decomposition of the derivative. Along with the decomposition of the base that I have advocated for above, such an analysis would allow the researcher to thoroughly describe and model the contributions both of the base and of the affix. 

In order to address these issues, a semantic framework is needed in which meanings can be composed and decomposed. By giving access to the meaning components of the base, such a framework would allow the researcher to model the semantic contribution of the base in the process of derivation. Moreover, we need a framework that is both flexible enough to incorporate all possible nominalization readings, and at the same time restricted enough to preclude impossible ones. 

A powerful framework that exhibits the desired characteristics is that of \HighlightText{frames} (\citealt{Barsalou.1992a, Barsalou.1992b,Petersen.2007,Loebner.2013}). 
Frames are recursive attribute-value structures which serve to model mental representations of concepts, similar to formalisms known from frameworks such as Head-driven Phrase Structure Grammar (HPSG, \citealt{Pollard.1994}) or Lexical Functional Grammar (LFG, \citealt{Bresnan.1982}).
The core of frame theory consists in the assumption that frames are the fundamental representation of knowledge, which includes linguistic structures and processes (referred to as the \textit{Frame Hypothesis}, \citealt{Loebner.2014a,Loebner.2017}; see also \citealt{Petersen.2007}). 

In this book, I combine frame semantics with a qualitative analysis of corpus data. 
I focus on both the base and the affix, investigating how a derivational process acts on the semantics of a given base. 
Specifically, I have conducted an in-depth study of the relationship between the English nominalizing suffix \mbox{\textit{-ment}} and a semantically delimited set of verbal bases.
Since my goal is to learn how speakers of English productively and intuitively use \textit{-ment} to create new words, my data sample consists of novel derivatives. The following overarching research questions guide my analysis:

\begin{enumerate}
	\item Which readings are possible in newly formed \textit{-ment} derivatives?
	\item What are the semantic contributions of the base and of the affix? 
	\item How can this be modeled in a frame-semantic approach? 
\end{enumerate}
My research project thus tackles affix polysemy (research question 1) by investigating compositionality in derivation (research question 2). On the methodological side, I apply a framework that allows for the precise and detailed description and, ultimately, for the prediction of derivational semantics (research question 3). 
A secondary objective is to give attention to \textit{-ment}, an affix which at this point remains understudied (see \citealt[4]{Lieber.2016}).


My data set consists of 40 \textit{-ment} neologisms from the \citetalias{OED} and the Corpus of Contemporary American English (\citealt{Davies.2008}), with base verbs from two semantic classes, namely change-of-state verbs and verbs of psychological state as defined by \citet{Levin.1993} and \citet{Kipper.2008}.
For the assertion of possible \textit{-ment} readings, I take into consideration 369 attestations which were elicited from various corpora with a purposeful sampling approach. For each attestation, the \mbox{\textit{-ment}} derivative was hand-coded using common semantic categories such as \textsc{event, state, result} and \textsc{stimulus}.


Applying the Frame Hypothesis to the subject of derivational semantics, I show that the process of deverbal nominalization can be modeled by assuming separate semantic frames for the base verbs and for their derivatives. A nominalizing suffix can target a highly restricted set of elements in the frame of the base, inducing a shift of reference with regard to the resulting frame of the derivative. For example, the frame for the base verb \textit{open} describes the \textsc{action} of making open, and \textit{-er} derivation shifts reference from the verbal meaning \textsc{action} to one of the possible readings of \textit{opener}. That is, the nominalization frame now describes either the \textsc{instrument}, the \textsc{agent}, or the [−animate] \textsc{patient} of opening. Polysemous derivatives are then disambiguated in context. 
In my approach, the relationship between input and output frames is made explicit by integrating them into lexical rules (see e.g. \citealt[]{Sag.1999}) and inheritance hierarchies (see e.g. \citealt{Riehemann.1998}). 


My study relies on a two-way relationship between language data and semantic frames. The frame representations of the base verbs are devised on the basis of existing literature, offering a first lead as to which readings may be expected in the nominalizations. 
For example, the frame for \textit{open} would contain an attribute \textsc{instrument}, among others. This points to a possible \textsc{instrument} reading when \textit{open} functions as a base in a nominalization process. 
Likewise, as a by-product of this analysis, a nominalization's semantics provides evidence for (or against) the elements contained in its base verb's frame: The existence of \textit{opener} in an \textsc{instrument} reading represents evidence for the existence of the \textsc{instrument} attribute in the frame representation of the base. 

 The book is structured as follows.
In \chapref{ch:frames}, I will give an overview of the variant of frame theory applied here, introducing key terminology and concepts as well as previous approaches to derivational semantics.
In \chapref{ch:method}, I will describe the methods which I used to elicit, choose and semantically classify the data set. 
Then, I will present the results of the two studies that I have conducted for this book: Chapter \ref{ch:cos} deals with \textit{-ment} on change-of-state verb bases, and \chapref{ch:psy} addresses \textit{-ment} on psych verb bases. In each of these two chapters, I will describe and model first the semantics of the base verbs, and then the semantics of the nominalizations. 
In \chapref{ch:observations}, I will take a quantitative perspective, addressing the issues of gaps and ambiguity in my data.
In \chapref{ch:discussion}, I will discuss the insights gained in Chapters \ref{ch:cos} to \ref{ch:observations}, answering my research questions and reflecting upon methodological issues. 
Finally, in \chapref{ch:conclusion}, I will present my conclusions and outline directions for further research.
