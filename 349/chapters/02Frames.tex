\chapter{Frames}
\label{ch:frames}

Since its emergence in the 1970s, frame theory has come to be widely used in a diverse range of disciplines from artificial intelligence, human cognition, and media science, to philosophy, psychology, and linguistics. In this chapter, I will give a general overview of the relevant cognitive and linguistic literature. More specialized formalizations will be introduced en route in the analysis Chapters (\ref{ch:cos} and \ref{ch:psy}). 

First, in order to be able to place the frame approach applied here historically, I will provide a brief sketch of the origin of frames (\sectref{sec:fr-early}).\footnote{For a more detailed review of the development of the frame notion and related concepts see \citet{Ziem.2008} and \citet{Busse.2012,Busse.2017}.}
Next, in \sectref{sec:fr-dus}, I will introduce the general architecture of the type of frame theory applied here, namely \citeauthor{Barsalou.1992a}'s (\citeyear{Barsalou.1992a,Barsalou.1992b}) approach as implemented in the DFG collaborative research center 991: \textit{The Structure of Representations in Language, Cognition, and Science} (CRC 991, \citealt{Petersen.2007} et seq.; see \citealt{Loebner.2021} for an overview of different frame applications). 
Finally, I will introduce the frame-related concepts and terminology that are needed to understand how derivation can be modeled in frames (\sectref{sec:fr-der}).\footnote{Strictly speaking, the term \textit{frame} refers to a cognitive structure, while the frame graphs and matrices would more appropriately be labeled \textit{frame representations} or \textit{frame visualizations}. For convenience, however, I will also refer to all frame representations as \textit{frames}.} 

\section{Earlier frame approaches}
\label{sec:fr-early}

In this section, I will delineate how the frame approach applied in this book integrates into the history of cognitive and linguistic research. I will first sketch the origins of frames in cognitive psychology (\sectref{sec:fr-early-origin}), and then focus on Barsalou's approach (\sectref{sec:fr-early-Barsalou}). 

\subsection{The origins of frames}
\label{sec:fr-early-origin}

It has been argued that frames ultimately go back to cognitive psychology and \citeauthor{Bartlett.1932}'s (\citeyear{Bartlett.1932}) \HighlightText{schemata}, and that they are closely related to script theory, which addresses issues in artificial intelligence (among other things, see e.g. \citealt[20]{Busse.2012}). Cognitive scientist Marvin Minsky is usually regarded as the founder of cognitive frame theory (\citealt[]{Kann.2018}), while Charles Fillmore simultaneously developed the linguistic theory of frame semantics (\citealt[10]{Busse.2012}). 

In his seminal paper ``A framework for representing knowledge'' (\citealt{Minsky.1975}), Minsky posits a number of core assumptions about frames which are also relevant in linguistic frame theory (see \citealt{Kann.2018}). Crucially, he regards frames as a detailed, structured, uniform format which can model and explain cognitive processes such as thinking, language, and perception. 

Fillmorean frames (\citealt{Fillmore.1968} et seq.) prominently figure in the FrameNet project (\citealt{Fillmore.2003,Fillmore.2010}). They are case frames which are often evoked by lexical units, and they are constituted by core and non-core case roles (see \citealt{FrameNet.nd}). 
If the frame represents a verb, these roles correspond to its arguments and adjuncts. The frame ``Sleep,'' for instance, has one core role (\textsc{sleeper}) and five non-core roles (\textsc{degree, duration, manner, place} and \textsc{time}).\footnote{\url{https://framenet2.icsi.berkeley.edu/fnReports/data/frame/Sleep.xml}, accessed 19 April, 2023}
FrameNet frames are flat role structures with all participants, both core and non-core, given in the form of a list. 
Importantly, it has been argued that they are not sufficient as lexical frames because they lack other kinds of semantic or syntactic information (\citealt[18]{Loebner.2014a}).
I will come back to this issue in \sectref{sec:fr-dus}.  

The cognitive and linguistic research traditions are closely related. For instance, the work by Minsky has been recognized both by Fillmore (e.g. \citealt{Fillmore.2010}) and by Barsalou (\citeyear{Barsalou.1992a,Barsalou.1992b}), whose cognitive approach is in turn the basis for the linguistic frames developed in the CRC 991. 

\subsection{Barsalou}
\label{sec:fr-early-Barsalou}

Barsalou's frame theory aspires to be a flexible and powerful representation of conscious and unconscious knowledge, regarding frames as ``the fundamental representation of knowledge in human cognition'' (\citealt[21]{Barsalou.1992b}). Frames represent \HighlightText{concepts}, that is, bundles of information that people have stored cognitively for a given category such as \textit{bird} or \textit{color} (p. 31). The basic components of Barsalou frames are sets of \HighlightText{attributes} and \HighlightText{values} (p. 30), as well as the relationships between them (p. 40).

Attributes are the central elements of a frame. They are a special kind of concept, namely one that can be used to describe some aspect of a category member (\citealt[30]{Barsalou.1992b}). For instance, the fact that all members of the category \textit{car} have an engine can be represented by assuming an \textsc{engine} attribute in the \textit{car}-frame.\footnote{Barsalou uses caps to indicate attributes and normal font for values. I will follow the CRC 991 convention to use small caps for attributes, while italics indicate values.} 
Attributes are specified by values, which are defined as subordinate concepts of their respective attribute (p. 31). For instance, a possible value for \textsc{engine} could be \textit{four-cylinder}. The value inherits properties of its superordinate and is at the same time more specific: Among other things, \textsc{engine} passes on the property that it consumes fuel and thereby produces force, and \textit{four-cylinder} adds the information that it has four cylinders and pistons to accomplish this task. Since values represent concepts as well, they can in turn have attributes describing them, which makes frames \HighlightText{recursive} (p. 43). 

Barsalou introduces a rich inventory of possible relations between a frame's elements (see \citealt[35--39]{Barsalou.1992b} for details). 
First, a frame's attributes can be spatially, temporally or causally related (\citealt[35]{Barsalou.1992b}). To stay with the \textit{car} frame, if we assume a \textsc{driver} attribute, it will be the understanding of most people that the driver has some sort of control over the engine. Barsalou therefore assumes an invariant \textit{operates}-relation between the attributes \textsc{driver} and \textsc{engine}. 
Relations between a frame's values, on the other hand, represent logical necessities, statistical patterns and personal preferences as well as constraints introduced by physical or cultural mechanisms, or by an agent's goals (\citealt[37--39]{Barsalou.1992b}). For example, it is generally true that traveling far (\textsc{distance:} \textit{far}) requires a faster means of transportation (\textsc{speed:} \textit{fast}), while Grandma's old moped will be used for different purposes.

\section{Toolkit for frame formalization} 
\label{sec:fr-dus}
\begin{sloppypar}
In the previous section, we have seen that Barsalou frames are recursive structures which consist of attribute-value-sets that can be related in different ways. The assumption that this architecture is the fundamental representation of knowledge is the starting point for the frame theory applied here (\textit{Frame Hypothesis}, \citealt{Loebner.2014a,Loebner.2017}; see also \citealt{Petersen.2007}). 
However, in order to make Barsalou's informal frames fit as a tool for formal linguistic research, his theory has recently been enriched with a mathematically and logically sound foundation (see \citealt{Loebner.2021} for an overview). In the remainder of this book, I will use the term \textit{frame} to refer to this cognitively plausible, formally precise frame format.\footnote{To make this section more easily accessible, I have chosen to omit the formal definitions which form the basis of this framework. The interested reader is referred to \citeauthor{Petersen.2007} (\citeyear{Petersen.2007}; reprinted in \citealt{Petersen.2015}) as well as \citet{Kallmeyer.2013}.}
\end{sloppypar}

In this section, I will provide a toolkit of relevant concepts and terminology. First, I will introduce the basic elements of frame theory (\sectref{sec:fr-dus-basic}). 
Then, I will introduce the frame format I will apply in this book, namely generalized event frames (\sectref{sec:fr-dus-basic-genlex}).

\subsection{The basics}
\label{sec:fr-dus-basic}

Some basics are needed to understand frames. First, I will introduce the two formats used for representing frames in the literature, namely graphs and matrices (\sectref{sec:fr-dus-basic-graphs}). Then, I will discuss attributes (\sectref{sec:fr-dus-basic-attr}), uniqueness conditions \mbox{(\sectref{sec:fr-dus-basic-unique})}, constraints (\sectref{sec:fr-dus-basic-constr}), and type signatures (\ref{sec:fr-dus-TS}). 

\subsubsection{Graphs and attribute-value-matrices}
\label{sec:fr-dus-basic-graphs}

Frames can be visualized as graphs or as attribute-value-matrices (AVMs). Consider the representations in Figure \ref{fig:introframes} for illustration.\footnote{Note that all frames presented in this book should be interpreted as partial: I include only the level of semantic detail which is required to illustrate the issue at hand.} 
In a frame graph, the attribute-value-structure is represented by edges (also sometimes \textit{arcs}) which connect nodes. The node which is described by a given attribute is called its \HighlightText{possessor}, the node it points to as its \textit{target node}. 
The referent node is indicated by double lines. Here, the frame depicts a \textit{hit} event with two participant attributes, an \textsc{agent} and a \textsc{patient}. The values of these participants are specified by their type labels as \textit{John} and \textit{ball}, respectively. 
Attribute and value labels are usually, but not necessarily, natural language expressions. If a concept is not lexicalized in a language, a paraphrase can be applied. 
Finally, it can be useful to index nodes, as done here with simple numbering. Note that indices serve to identify nodes within one frame, and do not necessarily apply across frames.

\begin{figure}
	\begin{subfigure}[b]{0.5\linewidth}
		\centering
			\begin{tikzpicture}[->,>=stealth', main node/.style={circle, draw, minimum size=1cm}]
			
			\begin{scope}[xshift=4cm]
			\node[main node] (1) at ( 2.0, 2.0) [circle,double,draw,thick,label=90:$\textit{hit}$] {0};
			\node[main node] (2) at ( 1.0, 0.5 ) [circle,draw,label=left:$\textit{John}$] {1};
			\node[main node] (3) at ( 3.0, 0.5 ) [circle,draw,label=right:$\textit{ball}$] {2};
			\path[draw, thick]
			(1) edge[bend right=15] node[above left] {\textsc{agent}} (2) 
			(1) edge [bend left=15] node[above right] {\textsc{patient}} (3)
			;
			\end{scope}
			\end{tikzpicture}
			\caption{Frame graph}
	\end{subfigure}%
	\begin{subfigure}[b]{0.5\linewidth}
		\centering \singlespacing
			\begin{avm}
				\@0 \[ \textit{hit}\cr
				\textsc{agent} & \@1 \textit{John} \cr
				\textsc{patient} & \@2 \textit{ball} \cr
				\] 
			\end{avm}
			\caption{AVM}
	\end{subfigure}
	\caption[Two ways of depicting a frame]{Two ways of depicting a frame (\citealt[]{Plag.2018})}
	\label{fig:introframes}
\end{figure}

Frame graphs can be directly translated into AVMs similar to those used in other frameworks such as HPSG (Head-Driven Phrase Structure Grammar, see e.g. \citealt{Pollard.1994}). AVMs have the advantage that they often take up less space than their corresponding frame graph. In an AVM, the attributes and values are arranged in columns, with an attribute listed underneath its possessor. The indices are given in boxes in front of the value they refer to. 
In AVMs, indices are necessary for co-indexation, which indicates that the values of two (or more) attributes share the same referent. In frame graphs, this is expressed by two (or more) attributes pointing to one and the same node. 

\subsubsection{Attributes}
\label{sec:fr-dus-basic-attr}

Attributes, the fundamental building blocks of frames, are assumed to be \HighlightText{functional}: Each attribute assigns a unique value to its possessor (\citealt[26]{Loebner.2014a}; \citealt[153]{Petersen.2007}). 
For instance, if the possessor of a \textsc{color} attribute is a red table, the attribute can be specified by the value \textit{red} or by one of its subtypes (e.g. \textit{scarlet}). 
If the color is unknown or irrelevant, the attribute may remain underspecified, in which case the node can be left without a label, or labeled with \textit{color}, as exemplified in Figure \ref{fig:color}. The term \textit{color} can thus be used in two different ways: It can be an attribute label \textsc{color}, or a value label \textit{color} (see \citealt{Petersen.2007,Petersen.2014}).

\begin{figure}
		\centering \singlespacing
		\begin{avm}
			\@0 \[ \textit{table}\cr
			\textsc{color} & \@1 \textit{color}
			\] 
		\end{avm}
	\caption[]{Frame for \textit{table} with an underspecified \textsc{color} attribute}
	\label{fig:color}
\end{figure}

Attributes can take different kinds of values, and thus represent different kinds of relationships between possessor and target node (see \citealt{Loebner.2013,Gamerschlag.2014a} for details): The two may be mereologically related or the target node may exist independently of its possessor, the attribute may describe a property of the possessor or connect it uniquely to an event, activity or purpose.

\subsubsection{Uniqueness conditions}\label{sec:fr-dus-basic-unique}\largerpage

We have already seen that attributes are assumed to be functional, taking unique values. This requirement has been formulated as a uniqueness condition alongside two further conditions which have to be met by any formally correct frame (\citealt[307]{Loebner.2013}):

\begin{description}
	\item[Unique frame referent (UR):] There is a unique element that represents the potential referent of the frame. Every element in the frame is connected to the frame referent by a chain of attributes.
	\item[Unique attributes (UA):] For each element in the frame, an attribute is assigned no more than once.
	\item[Unique values (UV):] For each element in the frame, and each attribute applied to it, the attribute takes a unique value.
\end{description}
UR ensures that a frame is a coherent, interconnected whole (\citealt[27]{Loebner.2014a}). According to \citet[306]{Loebner.2013}, the potential referent of a frame can be generic (e.g. any member of the category \textit{table}), or it can be a specific entity (e.g. the dining table in my living room). 
UA guarantees that the frame does not contain doubled and thus redundant attributes: Since attributes take unique values, it would not be possible to have arrows with the same attribute label originating from the same possessor node while taking different values. For example, a car's wheels legs need to be introduced by distinct attributes such as \textsc{wheel}\textsubscript{1}, \textsc{wheel}\textsubscript{2}, \textsc{wheel}\textsubscript{3} and \textsc{wheel}\textsubscript{4}. Having four attributes labeled \textsc{wheel} originating from the same node would violate UA.
Note, however, that the same attribute \textit{can} occur several times in the same frame, as long as it has different possessors. Thus, if the frame should express that the wheels have different colors, each node representing a given wheel would have an attribute labeled \textsc{color} (e.g. \textsc{wheel}\textsubscript{1}: \textit{wheel}, \textsc{color:} \textit{black}).    
Finally, UV spells out that attributes need to be functional, as introduced above. 

\subsubsection{Constraints}
\label{sec:fr-dus-basic-constr}

In frame theory, constraints are used to model, for example, semantic and frame-structural restrictions, world knowledge, and logic. 
At this point, however, there is no uniform formalization. Constraints may be formulated in natural speech (``the theme of an event with a cyclic event structure is co-referential with the theme of the atoms of the event structure,'' \citealt[130]{Gamerschlag.2014b}), or formalized in a constraint schema making use of logical operators (``$e \cdot \textsc{prog} \triangleq \textit{T} \wedge e' \text{ segm } e \rightarrow e' \text{ inst } T $,'' \citealt{Balogh.inpress}). Such formal constraints are either introduced in connection with a type signature (as discussed in \sectref{sec:fr-dus-TS}), or given alongside a frame (e.g. \citealt{Kallmeyer.2012}), or simply mentioned in the running text.

Constraints can apply to a number of possible combinations of frame elements (see \citealt{Loebner.2013,Schurz.2014} for examples): possessor-attribute, attribute-target, possessor-target, attributes-attribute, or value-value. 

\subsubsection{Type signatures}
\label{sec:fr-dus-TS}

Type signatures are an essential part of frame theory because they introduce formal restrictions. Were these not included in the formalism, this could lead to non-well-typed frames such as the one in Figure \ref{fig:nonwelltyped}. There, a \textsc{taste} attribute is assigned to a \textit{sound} possessor node, although \textsc{taste} does not belong to the attribute domain of nodes of the type \textit{sound}. 

\begin{figure}
		\centering \singlespacing
		\begin{avm}
			\@0 \[ \textit{sound}\cr
			\textsc{taste} & \@1 \textit{bitter}
			\] 
		\end{avm}
 	\caption[Non-well-typed frame of a bitter-tasting sound]{Non-well-typed frame of a bitter-tasting sound (adapted from \citealt[210]{Petersen.2014})}
 	\label{fig:nonwelltyped}
\end{figure}

Type signatures are conceptualized similarly to the notion of typed feature structures as introduced in \citeauthor{Carpenter.1992} (\citeyear{Carpenter.1992}; see also \citealt[163--164]{Petersen.2007}). 
In a type signature, it is possible to formally specify the ordering of types, which attributes are appropriate for a type, and which kind of value an attribute can take (see \citealt[7]{Gamerschlag.2014a}). This is achieved by enriching a plain type hierarchy with a finite set of attributes and appropriateness conditions (see \citealt[163--164]{Petersen.2007}).\largerpage

For illustration, consider Figure \ref{fig:typebirds}, which shows a portion of a type signature relevant for modeling \textit{bird}.\footnote{$\top$ is the most general type, the \HighlightText{top type} (\citealt[153]{McGlashan.1992}).}
It fixes that the type \textit{bird} comes with two attributes \textsc{beak} and \textsc{foot}, which have two admissible, underspecified values \textit{beak} and \textit{foot}. This is expressed by an appropriateness specification introducing the conditions ``\textsc{beak}: beak'' and ``\textsc{foot}: foot'' (see also \citealt[163, 165]{Petersen.2007}). 
Appropriateness specifications have two purposes (see \citealt{Petersen.2014}). First, they declare a set of admissible attributes for a given possessor node and thus restrict the \HighlightText{attribute domain} of this node.
Second, they specify that the values of an attribute need to be of a certain type, thereby restricting the \HighlightText{attribute range}. 
Appropriateness conditions can thus be used to express constraints of the kinds \textit{possessor-attribute} and \textit{attribute-value}. 
The type signature further introduces two subtypes of \textit{bird}, namely \textit{water-bird} and \textit{land-bird}. These inherit their supertype's attributes and specify them according to the type declarations in the right part of the figure. These indicate the subtypes of \textit{beak} and \textit{foot}. Thus, the type signature specifies which values \textsc{beak} and \textsc{foot} can take.{\interfootnotelinepenalty=10000\footnote{The authors observe that it would be a more adequate representation if the subtypes \textit{round, pointed, webbed} and \textit{clawed} were introduced not as subtypes of \textit{beak} and \textit{foot}, but rather of something like \textit{shape}, for instance: ``\textit{water-bird}, \textsc{beak}: \textit{beak}, \textsc{shape}: \textit{round}'' (p. 8).}}  

\begin{figure}
	\centering
		\begin{forest} for tree={l sep = 0.9cm}
			[$\top$
			[ \textit{bird}\\{\textsc{beak:} \textit{beak}}\\{\textsc{foot:} \textit{foot}}
			[\textit{water-bird}\\{\textsc{beak:} \textit{round}}\\{\textsc{foot:} \textit{webbed}}, tier=2]
			[\textit{land-bird}\\{\textsc{beak:} \textit{pointed}}\\{\textsc{foot:} \textit{clawed}}, tier=2]
			]
			[\textit{beak}
			[\textit{round}, tier=1]
			[\textit{pointed}, tier=1]
			]
			[\textit{foot}
			[\textit{webbed}, tier=2]
			[\textit{clawed}, tier=2]
			]
			[{...}]
			]
		\end{forest}
	\caption[Example type signature]{\label{fig:typebirds}Example type signature (adapted from \citealt{Gamerschlag.2014a})}
\end{figure}    

The appropriateness specifications of \textit{water-bird} and \textit{land-bird} are furthermore in accordance with the 
 co-occurrence restrictions given in (\ref{ex:constraints}). These bi-implicational constraints (\citealt[8]{Gamerschlag.2014a}) specify how the values of \textsc{beak} and \textsc{foot} co-occur: If the beak is round, the foot is webbed (and vice versa), and if the beak is pointed, the foot is clawed (and vice versa).\footnote{This, of course, is an illustratory simplification which does not reflect ornithological reality.}
 
 \begin{exe}
 	\ex \label{ex:constraints}
\textsc{beak}: \textit{round} $\leftrightarrow$ \textsc{foot}: \textit{webbed}
 \textsc{beak}: \textit{pointed} $\leftrightarrow$ \textsc{foot}: \textit{clawed}
 \end{exe}
Type signatures have some further relevant properties which are not depicted in Figure \ref{fig:typebirds}.
First, appropriateness conditions only need to be repeated at lower levels when they are further specified; otherwise, they are inherited as is.
Second, appropriateness conditions can not only be inherited, but also introduced at lower levels. For instance, the emu is the only bird with calf muscles. Therefore, an attribute \textsc{calf muscles} would only be introduced for \textit{emu} and its subtypes.
Lastly, types can have multiple parents, which involves that they inherit both parents' appropriateness conditions (see e.g. \citealt{Kallmeyer.2013}).

Having specified in the type signature what is generally possible for \textit{bird} and its subtypes, we can now turn to the corresponding frame formalizations. In Figure \ref{fig:framebirds}, we see three frames: for \textit{bird}, for \textit{water-bird} and for \textit{land-bird}. These are well-typed because they adhere to what is specified in Figure \ref{fig:typebirds}.

\begin{figure}
	\begin{subfigure}[]{0.3\linewidth}
		\centering  \singlespacing
		\begin{avm}
			\@0 
			\[\avmspan{\textit{bird}}\cr
			\textsc{beak} & \@1 \textit{beak} \cr 
			\textsc{foot} & \@2 \textit{foot} \cr
			\] 
		\end{avm}
	\caption{\label{fig:framebirds1}\textit{bird}}
	\end{subfigure}%
	\begin{subfigure}[]{0.3\linewidth}
	\centering \singlespacing
		\begin{avm}
			\@0 
			\[\avmspan{\textit{water-bird}}\cr
			\textsc{beak} & \@1 \textit{round} \cr 
			\textsc{foot} & \@2 \textit{webbed} \cr
			\] 
		\end{avm}
	\caption{\label{fig:framebirds2}\textit{water-bird}}
\end{subfigure}%
	\begin{subfigure}[]{0.3\linewidth}
		\centering  \singlespacing
		\begin{avm}
			\@0 
			\[\avmspan{\textit{land-bird}}\cr
			\textsc{beak} & \@1 \textit{pointed} \cr 
			\textsc{foot} & \@2 \textit{clawed} \cr
			\] 
		\end{avm}
		\caption{\label{fig:framebirds3}\textit{land-bird}}
	\end{subfigure}%
	\caption[Frames for \textit{bird} and its subtypes]{\label{fig:framebirds}Frames for \textit{bird} and its subtypes (adapted from \citealt[8]{Gamerschlag.2014a})}
\end{figure}

\subsection{Generalized event frames}\label{sec:fr-dus-basic-genlex}\largerpage

In my analyses, I mostly make use of lexical event frames which represent groups of words. In this section, I will describe the general properties of the frame representations used in this book, which are generalized, lexical event frames. Any details pertaining specifically to change-of-state verbs or verbs of psychological state will be left for their respective chapters. I will first delineate the properties of a lexical frame, then clarify the distinction between instantiated and generalized frames, and finally introduce the general architecture of event frames. 

Lexical frames exhibit two crucial properties. First, they are evoked by lexical units of a language, representing concepts which exist in a speaker's mental lexicon (\citealt[208]{Petersen.2014}), and second, they model the stable, constant meaning of these lexical items (\citealt[297]{Loebner.2013}). For example, a lexical frame for \textit{apple juice} would model the meaning `juice made of apples' (\citealt[295]{Loebner.2013}). 
Correspondingly, frames are considered non-lexical either if they are not directly evoked by a lexical item (see e.g. \citealt[131]{Osswald.2014}), or if they contain meaning that goes beyond semantics in the strict sense. This includes world knowledge (i.e. cultural or personal knowledge; see \citealt[293]{Loebner.2013}) as well as encyclopedic knowledge and information provided by a specific context (\citealt[95]{Ziem.2015}).{\interfootnotelinepenalty=10000\footnote{In opposition to L\"obner's view, many cognitive linguists argue for abandoning the distinction between semantic knowledge and world knowledge, arguing that category descriptions should also include ``associative and experiential attributes'' (\citealt[95]{Ungerer.2006}).}}

In this book, the goal is to generalize over several lexemes. The frames proposed here therefore do not represent single lexical units, but rather sets of semantically similar ones.
This is not the first time that frames are used in this way. For example, \citet{Kallmeyer.2012} model directed motion verbs as in Figure \ref{fig:directedmotionframes}. In the generalized frame to the left, they use the attributes \textsc{actor} and \textsc{goal}. When applying this frame to a specific verb, it is altered to accommodate the semantics of this verb, as seen to the right. Here, \textit{walk} adds two further attributes, a \textsc{path} and a specific \textsc{manner} of motion.

\begin{figure}
\centering \singlespacing
	\begin{subfigure}[]{0.5\textwidth}\centering
		\begin{avm}
				\[\avmspan{\textit{directed-motion-action}}\cr
				\textsc{actor} & \@1 \cr 
				\textsc{goal} & \@2 
				\] 
			\end{avm}
	\end{subfigure}%
		\begin{subfigure}[]{0.5\textwidth}\centering
			\begin{avm}
				\[\avmspan{\textit{walk}}\cr
				\textsc{actor} & \@1 \cr 
				\textsc{goal} & \@2 \cr
				\textsc{path} &
				\[ \textit{path} \cr
				\textsc{startp} & \@3 \cr
				\textsc{endp} & \@4
				\] \cr
				\textsc{manner} & \textit{walking}
				\] 
			\end{avm}
	\end{subfigure}
\begin{subfigure}[]{0.5\linewidth}
		\caption{\textit{directed motion}} \label{fig:directedmotionframes-a}
\end{subfigure}%
\begin{subfigure}[]{0.5\linewidth}
		\caption{\textit{walk}} \label{fig:directedmotionframes-b}
\end{subfigure}
	\caption[Frames for general directed motion and \textit{walk} as a directed motion]{\label{fig:directedmotionframes}Frames for (a) general directed motion and (b)  \textit{walk} as a directed motion (both adapted from \citealt[43--44]{Kallmeyer.2012})}
\end{figure}

Apart from generalized frames, I will at times also include frames for single tokens of my corpus data, that is, nominalizations in specific contexts. I will refer to these as \HighlightText{instantiated frames.}

Lexical frames for events differ from those for entities such as \textit{apple juice} in that they may (or should, according to \citealt{Loebner.2014a}) include the specification of dependencies between semantic roles as well as temporal and causal characteristics. 
For the purposes of this study, temporality is not salient and will therefore be left implicit. The causality of a base verb, however, plays a central role for its nominalization's semantics, as we will see in the course of this book. 

The corresponding distinction between different kinds of events is between simple and complex ones (\citealt{Pustejovsky.1991,Pustejovsky.1995,VanValinJr.1997,RappaportHovav.1998,Levin.1999}). In this research tradition, simple events are those consisting of only one subevent, while complex events are those composed of two (causally connected) subevents. We have already seen examples of simple event frames which represent activities in Figures \ref{fig:introframes} and \ref{fig:directedmotionframes}.  
Furthermore, the simple event structure template is used for states and changes-of-state, as exemplified in Figure \ref{fig:walklovedry}. In these examples from the literature, we can see participants embedded in simple frames consisting of just one event (\textit{love-state} and \textit{dry-inchoation}; see also \textit{hit} and \textit{directed-motion-action/walk} in the figures above). Observe how the participants change with each subtype of simple event (e.g. \textsc{experiencer} and \textsc{theme} for \textit{love-state} vs. \textsc{patient} for \textit{dry-inchoation}).

Complex events, on the other hand, are decomposed as in Figure \ref{fig:cosframelaura}. There, the two subevents of the complex causation event \textit{break} are expressed by the attributes \textsc{cause} and \textsc{effect}. 
The first subevent is an unspecified activity with one participant, an \textsc{actor}. This activity causes a change-of-state, the result of which is that a patient is in a broken-state.\footnote{In the frames presented here, time is modeled implicitly by employing attributes such as \textsc{result, cause} or \textsc{effect}, which are defined as preceding or succeeding other frame elements, respectively. In studies which aspire to be semantically more explicit than is required for my purposes, it can be expedient to incorporate time overtly (see e.g. \citealt{Gamerschlag.2014b}; \citealt[109]{Loebner.2017}).}

\begin{figure}
    \centering \singlespacing
     \begin{subfigure}[b]{0.5\textwidth}
     \centering
         \begin{avm}
         			\[\textit{love-state}\cr
         			\textsc{experiencer} & \@1 \cr 
         			\textsc{theme} & \@2 
         			\] 
         		\end{avm}
         \caption{\textit{state}}
     \end{subfigure}%
     \hfill
     \begin{subfigure}[b]{0.5\textwidth}
     \centering
         \begin{avm}
         			\[\avmspan{\textit{dry-inchoation}}\cr
         			\textsc{result} & 
         			\[ \textit{dry-state} \cr
         			\textsc{patient} \, \@1 \cr
         			\] 
         			\] 
         		\end{avm}
         \caption{\textit{change-of-state}}
     \end{subfigure}
        \caption[Frames for the verbs \textit{love} and intransitive \textit{dry}]{\label{fig:walklovedry} Frames for the verbs \textit{love} (adapted from \citealt[28]{Zinova.2012}) and intransitive \textit{dry} (adapted from \citealt[140]{Osswald.2014})}
\end{figure}

\begin{figure}
	\centering \singlespacing
		\begin{avm}
			\@0	
			\[\avmspan{\textit{causation}}\cr
			\textsc{cause} & \@1 
				\[ \avmspan{\textit{activity}}\cr 
				\textsc{actor} & \@3 
				\] \cr
			\textsc{effect} & \@2 
				\[ \avmspan{\textit{change-of-state}} \cr
				\textsc{result-state} &  
					\[ \avmspan{\textit{broken-state}}\cr
					\textsc{patient} & \@4 \cr
					\] \cr
				\]
			\] 
		\end{avm}
		\caption[Frame for transitive \textit{break}]{Frame for transitive \textit{break} (adapted from \citealt[13]{Kallmeyer.2013})\label{fig:cosframelaura}}
\end{figure}

\section{Modeling derivation in frames}\label{sec:fr-der}\largerpage

In order to tackle derivation in frames, two distinct but related approaches have been put forward.   
In the first approach, derivation is modeled as a process of frame unification (briefly delineated in \sectref{sec:fr-der-uni}); in the second, as a process of a referential shift in the base word frame (\sectref{sec:fr-der-ref}). In this book, I follow the second approach.

\subsection{Unification}
\label{sec:fr-der-uni}

Frame unification is an ``information combining procedure'' (\citealt[158]{Petersen.2007}) whereby two (or more) nodes of two (or more) frames are conjoined in order to form a single frame.\footnote{Apart from derivation, unification has also been used to model the semantics of compounds and sentences (\citealt{Loebner.2013}), and of adverbial modification (\citealt{Goldschmidt.2017}).} 
For example, the frame for \textit{coffee} can be combined with the frame for \textit{cup} in order to form the frame \textit{coffee cup} (see \citealt[317--319]{Loebner.2013}).
This principle can be applied to derivation by assuming that the frame for a base merges with the frame for an affix. For instance, \citet{Zinova.2016} analyzes verbal prefixation in Russian (e.g. \textit{varit'} `to cook' > \textit{navarit'} `to cook a lot of,' p. 255) and proposes a frame for each prefix sense, which then unifies with that of the respective base verb. 

The biggest difference between the two approaches is that the unification approach assumes affix semantics, while the shift approach does not. Instead of assuming a separate frame for the affix, the shift approach looks at the affix's contribution by comparing the derived word with its base.
In terms of morphological theory, it could be said that the unification approach takes the perspective of the morpheme-based tradition, while the shift approach goes more in the direction of word-based morphology (see \citealt[179--190]{Plag.2003} for an overview of both positions). 
For many purposes, however, unification and shifts seem to be notational variants of one another. This is in line with \citet[]{Plag.2003}, who finds that morpheme- and word-based approaches are rather in a complementary (as opposed to a conflicting) relationship (p. 189), and that both would serve equally well to model affixation (p. 185). For this study, I have chosen the shift approach because it can cope better with polysemy, which is ubiquitous for \textit{-ment}.

\subsection{Referential shifts}
\label{sec:fr-der-ref}
\label{sec:fr-der-ref-der}
\label{sec:fr-der-ref-lex}
\label{sec:fr-der-ref-inherit}

\citet{Loebner.2013} first observed that some derivational processes can be modeled as referential shifts in frames. The mechanism goes back to the process of \textit{metonymy}, where an expression is used ``to refer to things that belong to the kind of objects to which the expression refers in its literal meaning'' (p. 52).
That is, the new referent is a fundamental component of the original referent's meaning. The author gives the example of a metonymical shift from the concept \textit{university} as an institution to \textit{university campus}. In frames, metonymy is modeled by shifting a frame's reference to the value of one of its attributes (see also \citealt{Schulzek.2014}; \citealt[163--165]{Terhalle.2017}; \citealt[215]{Schulzek.forthcoming}).
In this section, I will first explain how this mechanism can be applied to model the semantics of derivation. Then, I will address the morphological side of this approach, which has recently been fleshed out by means of lexical rules and inheritance hierarchies.  

The mechanism of referential shifts has been applied to model deverbal nominalization both in frames (\citealt{Loebner.2013,Schulzek.2014,Kawaletz.2015,Plag.2018,Schulzek.forthcoming}) and in other approaches (e.g. \citealt{Panther.2002}).\footnote{\citet{Kawaletz.2015} and \citet{Plag.2018} are closely connected to the project presented in this book: \citet{Kawaletz.2015} is a pilot study out of which the present analysis of psych nominalizations (\chapref{ch:psy}) has developed, and \citet{Plag.2018} is based on findings from both analyses presented here (Chapters \ref{ch:cos} and \ref{ch:psy}).} 
Figure \ref{fig:walkerframeavm} illustrates the mechanism: It shows a frame for the verb \textit{walk}, and the frames of its nominalizations \textit{walker} and \textit{walk} (in two senses: `act of walking' and `route for walking'). We see that the frames are identical in terms of their attribute-value-structure, but differ in the indication of reference \mbox{(`\textsc{ref} $~=~$ \{\dots\}')} underneath the AVM. This notation, which was introduced by \citet{Plag.2018}, uses a frame's indices to signal reference, which can be straightforwardly used to formalize referential shifts, as explained below. Thus, reference is on \avmbox{0} both for the base verb \textit{walk}\textsubscript{V} (\ref{fig:walkerframeavm1}) and its derivative \textit{walk}\textsubscript{N} in the sense `act of walking' (\ref{fig:walkerframeavm2}), while it shifts to \avmbox{1} for \textit{walker}  (\ref{fig:walkerframeavm2}) and to \avmbox{2} for \textit{walk}\textsubscript{N} in the sense `route for walking' (\ref{fig:walkerframeavm4}).\footnote{Without this specification, \avmbox{0} would standardly be assumed to indicate the referent of the frame. For this reason, explicitly indicating reference for the frames in (\ref{fig:walkerframeavm1}) and (\ref{fig:walkerframeavm3}) is not actually required.} 
The fact that the two attributes of the original reference node (\textsc{agent} and \textsc{path}) are still present in the nominalization frames reflects that the attributes still appertain to the derived concepts (\citealt[313]{Loebner.2013}).\footnote{This is often, but not necessarily the case. For example, an opaque derivative like \textit{stealth} will not have as much in common with its base \textit{steal}.} Thus, \textit{walk}\textsubscript{N} and \textit{walker} are conceptualized in their relation to the event denoted by their base verb \textit{walk}.

\begin{figure}
	\begin{subfigure}[]{0.25\linewidth}
		\centering  \singlespacing
		\begin{avm}
			\avml
			\@0 
			\[\avmspan{\textit{walk}}\cr
			\textsc{agent} & \@1 \cr 
			\textsc{path} & \@2 \cr
			\] \cr
			{\textsc{ref} $=$ \{\@0\} }
			\avmr 
		\end{avm}
	\caption{\label{fig:walkerframeavm1}\textit{walk}\textsubscript{V}}
	\end{subfigure}%
	\begin{subfigure}[]{0.25\linewidth}
	\centering \singlespacing
	\begin{avm}
		\avml
		\@0 
		\[\avmspan{\textit{walk}}\cr
		\textsc{agent} & \@1 \cr 
		\textsc{path} & \@2 \cr
		\] \cr
		{\textsc{ref} $=$ \{\@1\} }
		\avmr 
	\end{avm}
	\caption{\label{fig:walkerframeavm2}\textit{walker}}
\end{subfigure}%
	\begin{subfigure}[]{0.25\linewidth}
		\centering  \singlespacing
		\begin{avm}
			\avml
			\@0 
			\[\avmspan{\textit{walk}}\cr
			\textsc{agent} & \@1 \cr 
			\textsc{path} & \@2 \cr
			\] \cr
			{\textsc{ref} $=$ \{\@0\} }
			\avmr 
		\end{avm}
		\caption{\label{fig:walkerframeavm3}\textit{walk}\textsubscript{1}}
	\end{subfigure}%
	\begin{subfigure}[]{0.25\linewidth}
		\centering  \singlespacing
		\begin{avm}
			\avml
			\@0 
			\[\avmspan{\textit{walk}}\cr
			\textsc{agent} & \@1 \cr 
			\textsc{path} & \@2 \cr
			\] \cr
			{\textsc{ref} $=$ \{\@2\} }
			\avmr 
		\end{avm}   
		\caption{\label{fig:walkerframeavm4}\textit{walk}\textsubscript{2}}    
	\end{subfigure}%
	\caption[Frame AVMs for the verb \textit{walk} and three nominalizations]{\label{fig:walkerframeavm}Frame AVMs for the verb \textit{walk} and three nominalizations: \textit{walker}, \textit{walk}\textsubscript{1} `act of walking,' and \textit{walk}\textsubscript{2} `route for walking' (based on \citealt[312]{Loebner.2013})}
\end{figure}

What the frames in Figure \ref{fig:walkerframeavm} do not show is how we get from the base verb frame to the nominalization frames. 
\citet{Andreou.2017} and \citet{Plag.2018} introduce \HighlightText{lexical rules} (\textit{lexeme-formation rules, semantic rules}) to model this process (see also \citealt{Schulzek.forthcoming} for a related approach).\footnote{\citet{Andreou.2017} does not deal with nominalization, but with stereotype negation. The application of lexical rules is in principle the same, but the processes do not involve referential shifts. Rather, the author assumes that the affixes involved in this process modify an attribute value.} 
Lexical rules are well-established in theories such as the generative grammar theory HPSG or Lieber's Lexical Semantic Framework (\citealt{Lieber.2004} et seq.) to model, among other things, derivational morphology (\textit{derivational rules}, \citealt[194]{Sag.1999}; see also \citealt{Bredenkamp.1996}). 
If a lexical rule is provided with a lexical entry as input, it gives out another lexical entry which is systematically related to the input in terms of (morphological) form, part of speech, and meaning (\citealt[185]{Sag.1999}). 

\begin{sloppypar}
In established approaches, lexical rules have been modeled as constraints which are represented in AVMs (see \citealt{Pollard.1994,Riehemann.1998,Koenig.1999}; as well as the overview in \citealt{Mueller.2015} for approaches in the HPSG framework; see \citealt{Bonami.2016} for an overview of other constraint-based approaches to grammar).
Such AVMs can easily be integrated into a frame-based approach, as is illustrated in Figure \ref{fig:lexrulewalker}. It depicts a lexical rule for deverbal, agentive \textit{-er} derivation (such as \textit{walker} in Figure \ref{fig:walkerframeavm}) and can be read as follows:
The first three attributes describe the derivative, the fourth attribute models the base, and the last attribute indicates reference. 
The base (\textsc{m-base}) is defined as a lexeme with a phonological form \avmbox{x}, a syntactic category V, and semantics modeled as a semantic frame (\textsc{s-frame} \avmbox{y}). The s-frame is an action with an \textsc{agent} attribute, and further possible attributes (indicated by ``...''). 
The derivative's phonological form consists of the phonology of the base plus that of the suffix, and its category is N. 
The semantic frame of the derivative corresponds to that of the base, with one modification, namely that reference is now on the agent (index \avmbox{1}). 
\end{sloppypar}

\begin{figure}
	\centering \singlespacing
	\begin{avm}
		\[\textit{lexeme}\cr
		\textsc{phon} & {/\@x-\textit{er}}/\cr
		\textsc{cat} & N \cr
		\textsc{sem} & \[ \textsc{s-frame} & \@y 
		\]\cr
		\textsc{m-base} & 
			\[\textit{lexeme}\cr
			\textsc{phon} & \@x \cr
			\textsc{cat} & V \cr
			\textsc{sem} &
				\[ \textsc{s-frame} & \@y
				 	\[ 
					\textit{action} \cr
					\textsc{agent} & \@1 \cr
					... \cr 
					\]
				\]
			\]  \cr
		\avmspan{{\textsc{ref} $=$ \{\@1\} }}
		\cr
		\]	
	\end{avm}
		\caption[Lexical rule of deverbal, agentive \textit{-er} nominalization]{Lexical rule of deverbal, agentive \textit{-er} nominalization (based on \citealt{Andreou.2017,Plag.2018})}
		\label{fig:lexrulewalker}
\end{figure}

\citet{Plag.2018} integrate lexical rules into an inheritance hierarchy in order to derive individual meanings based on what the bases provide in terms of their semantic representations (for similar approaches see \citealt{Riehemann.1998,Koenig.1999,Desmets.2009,Booij.2010,Tribout.2010,Bonami.2016}). This reflects the multiplicity of meaning that is evident in many derived words.\largerpage

Figure \ref{fig:inheritancewalker} shows an inheritance hierarchy incorporating both the lexical rule in Figure \ref{fig:lexrulewalker} and a second lexical rule, which creates another possible reading of \textit{-er} derivatives, namely \textsc{instrument}. This hierarchy accounts for polysemies such as \textit{walker} in the two interpretations `person who walks' and `walking aid.'
It models deverbal nouns (\textit{v-n}) and contains two lexical formation rules (here \textit{lfr}). 
The phonology is specified to the left, while the possible readings can be found to the right. \textsc{Cat} is already covered by the hierarchy's specification as \textit{v-n-lfr}. Reference is indicated by co-indexation of the referential argument \textsc{ref} with the attribute in question, i.e. \textsc{agent} and \textsc{instr(ument)}. The existence of nominalizations in each reading is indicated under the hierarchy. More precisely, the noun \textit{walker} can be found in both readings, while \textit{teacher}, whose base \textit{teach} does not have a salient \textsc{instrument} argument, is not attested in an instrument reading. 

This kind of inheritance hierarchy is an elegant way to model derivation since it straightforwardly depicts polysemy, and it allows the integration of various verb classes at once. Furthermore, it avoids overgeneration, which is a problem in other approaches (see \citealt[{559--560}]{Plag.2018}). 

The approach in \citet{Plag.2018} was of a programmatic nature, modeling the underlying mechanism of derivational morphology based on some exemplary data. This book presents a detailed analysis of the parts of the hierarchy which are abbreviated as \textsc{sem}. 
In my analyses in Chapters \ref{ch:cos} and \ref{ch:psy}, I will use semantic frames as a representational format, and propose inheritance hierarchies based on my findings in Sections \ref{sec:cos-output-formal-inherit} and \ref{sec:psy-output-formal-inherit}. I will show how differences in the base verb frames systematically lead to different possible shifts in the nominalizations, and how this can be modeled with lexical rules and inheritance hierarchies. 

\begin{figure} \centering \singlespacing
	\resizebox{\linewidth}{!}{
		\begin{tikzpicture}
		\draw (-4,6) node(0){lexeme};
		\draw (-4,5) node(1){\textit{v-n-lfr}};
		\draw (-8,4) node(2){\Boxed{\textsc{phon}}};
		\draw (-0.5,4) node(3){\Boxed{\textsc{sem}}};
		\draw (-8,2) node(4){
			\begin{avm}
			\[{\textit{x-er}}\cr
			\textsc{ph} \@1+er\cr
			\textsc{m-base} \[\textsc{ph} \@1 \]
			\] 
			\end{avm}};
		\draw (-3,2) node(5){
			\begin{avm}
			\[\textit{agent-n}\cr
			\textsc{ref} \@x\cr
			\textsc{m-base} \[ \textsc{sem} \[ \textsc{agent} \@x \] \]
			\] 
			\end{avm}};
		\draw (2.5,2) node(6){
			\begin{avm}
			\[\textit{instr-n}\cr
			\textsc{ref} \@x\cr
			\textsc{m-base} \[ \textsc{sem} \[ \textsc{instr} \@x \] \]
			\]
			\end{avm}};
		
		\draw (-5,-1) node(8){\begin{tabular}{l}
			walker\\teacher
			\end{tabular}};
		\draw (-0.5,-1) node(9){\begin{tabular}{l}
			walker
			\end{tabular}};
		
		\draw [dotted] (0) -- (1);
		\draw (1.south) -- (2.north);
		\draw (1.south) -- (3.north);
		\draw (2.south) -- (4.north);
		\draw (3.south) -- (5.north);
		\draw (3.south) -- (6.north);
		\draw (8.north) -- (4.south);
		\draw (8.north) -- (5.south);
		\draw (9.north) -- (4.south);
		\draw (9.north) -- (6.south);
		
		\end{tikzpicture}
	}
	\caption[Partial inheritance hierarchy of lexical rules for the suffix \textit{-er}]{Partial inheritance hierarchy of lexical rules for the suffix \textit{-er} (based on \citealt{Plag.2018})}
	\label{fig:inheritancewalker}
\end{figure}
