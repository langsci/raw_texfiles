\chapter{Methodology}\label{ch:method}\largerpage

In this chapter, I will present the methods which I used to elicit, choose and semantically classify the data set. 
In \sectref{sec:meth-neo}, I will detail the procedures I applied in order to elicit the data set of \ment{} neologisms. 
Then, in \sectref{sec:meth-Vsem}, I will address the semantic classification of base verbs by means of VerbNet classes. 
In \sectref{sec:meth-revis}, I will explain how I  then trimmed the data set to contain only two base verb classes, and how I revised and expanded it with more nominalizations and attestations. 
Finally, in \sectref{sec:meth-Nsem}, I will describe the semantic classification of the nominalizations in context. 

\section{Identifying neologisms}
\label{sec:meth-neo}

I use the term \HighlightText{neologism} to refer to words which speakers actively form by following productive rules (see \citealt{Hohenhaus.2005} for a discussion of the terminological confusion around this term).
The neologisms in my data set were identified from two sources which I will present in turn: In \sectref{sec:meth-neo-OED}, I will explain the process of identifying neologisms in the Oxford English Dictionary Online (\citetalias{OED}), while in \sectref{sec:meth-neo-COCA} I will address the extraction of neologisms from the Corpus of Contemporary American English (COCA, \citealt{Davies.2008}).

\subsection{OED neologisms}
\label{sec:meth-neo-OED}
The Oxford English Dictionary Online (\citetalias{OED}), containing  600,000 words and 3.5 million quotations, is an exceptionally detailed and comprehensive dictionary of the English language. It is continuously updated with new words and usages of existing entries, giving dates of first citation for every sense in which a lemma is attested. It is, therefore, a convenient tool for the identification of neologisms.   

A list of entries containing possible neologisms was retrieved using the interface provided by the \citetalias{OED}. All nouns ending in \textit{-ment} were extracted by searching for the corresponding orthographic string 〈*ment〉 in both the \textit{Headword} and the \textit{Lemma} category, restricting the part of speech to nouns. In order to exclude a large number of lexicalized forms already at this point, only entries with first citations dating from 1900 to today were included (see, for example, \citealt{Plag.1999} for a similar procedure). 

The resulting word list was subjected to a standard revision procedure. By manually inspecting the \textit{etymology} section of their respective \citetalias{OED} entry, all types which met at least one of the following criteria were eliminated: 

\begin{enumerate}
	\item non-transparent borrowings (e.g. \textit{ravalement})
	\item words which do not contain the suffix \textit{-ment} (e.g. \textit{bioelement})
	\item results of any word-formation process other than suffixation (e.g. prefixation on a suffixed base as in \textit{disempowerment}, or blends such as \textit{edutainment}) 
	\item non-deverbal nominals (e.g. \textit{foolishment})
	\item restricted technical terms (e.g. \textit{excystment})
\end{enumerate}
The last criterion relates to those types which the editors of the \citetalias{OED} have tagged as technical terms. These types were cross-checked with other corpora. If a reasonable number of non-technical contexts could be identified, they remained in the data set; otherwise, they were deleted. 
For example, \textit{excystment} is tagged as a biological and medical term in the \citetalias{OED}, and can indeed only be found in contexts relating to biology, medicine and paleontology in the corpora.\footnote{In March 2018, all 84 Google hits returned by 〈excystment〉 were from these three fields.} It does thus not qualify as a neologism. 
\textit{Endistancement}, on the other hand, is an established technical term in theater and cinematography, but is abundantly attested also in other contexts, and was thus kept in the data set. 

After these revisions of the data, the 134 hits initially returned by the \citetalias{OED} search were reduced to the following 16 deverbal nouns: 

\begin{table}
	\begin{tabular}{lllllll}
		bemusement && endistancement && piercement && stakement \\ 
		embrittlement && motherment && reforestment && underlayment \\ 
		encirclement && munitionment && soothment && upliftment \\ 
		encoppicement && perturbment && staggerment && weldment \\ 
	\end{tabular} 
\end{table}

\noindent These \textit{-ment} derivatives were coined between 1900 and 1961. For ease of reference, I will refer to this part of the data set as the \HighlightText{\citetalias{OED} data}. 

\subsection{Hapaxes in COCA}
\label{sec:meth-neo-COCA}
In addition to the \citetalias{OED} neologisms, the data set was substantially extended by extracting very rare forms from COCA. \HighlightText{Hapax legomena} (or \HighlightText{hapaxes}, for short) are words which occur only once in a given context (e.g. the word \textit{addition} in this paragraph). The notion of the hapax legomenon is central in corpus linguistics since it can be shown that the majority of neologisms in any given corpus is contained precisely in this group of hapaxes (see \citealt [68] {Plag.2003}). For the present study, this means two things: Hapaxes can serve as a source for neologisms, and they provide a realistic indicator of a suffix's productivity. 
Note that it is of course not claimed here that every hapax is indeed a neologism. In fact, a large number of hapaxes are actually very rare or specific technical terms, archaisms, non-transparent ad hoc inventions, typing errors, or errors resulting from automatic text recognition. The size of the corpus is also a decisive factor. The larger the corpus, the higher the proportion of neologisms among the hapaxes (see \citealt{Baayen.1996}, \citealt{Baayen.2009}). This can be illustrated with a very simple example. Take this paragraph as a tiny corpus  \textendash{}  most words in it are in fact hapaxes, although of course none of them are actually a neologism. Therefore, it is necessary to consult a sufficiently large corpus in order to predict the probability of new forms with a given suffix reliably. 

Three tools were employed in this step of the process: COCA (online and on DVD), VerbNet, and Coquery.
COCA had more than 450 million words produced between 1990 and 2012 at the time of data extraction, and was thus an appropriately large corpus for the identification of hapaxes as potential neologisms.\footnote{In March 2020, COCA received a massive update, and now contains 1 billion words.} COCA is balanced with regard to year and genre of attestation, including spoken texts as well as texts written in the genres fiction, popular magazines, newspapers, and academic journals. 
The second tool, VerbNet, is a hierarchical verb lexicon of 6,088 English verbs. It is based on the classification developed in \citet{Levin.1993} and includes syntactic and semantic information.\footnote{I used VerbNet 3.2.4 (\url{https://verbs.colorado.edu/verb-index/}, accessed 19 April, 2023) to access the database. VerbNet 3.3 (\url{https://verbs.colorado.edu/verb-index/vn3.3/}, accessed 19 April, 2023), which has since been made available, incorporates a number of fundamental changes.} 
More details on VerbNet are given in \sectref{sec:meth-Vsem}.
The third tool is the corpus query software Coquery (\citealt{Kunter.2015}). In order to be able to identify a larger number of deverbal \textit{-ment} derivatives, I used it to conduct an automated search of the DVD-version of COCA (\citealt{Davies.2008b}).

Coquery probed the corpus for each listed verb in combination with the search strings 〈ment〉 and 〈ments〉, also including orthographic variants which had been added manually (e.g. \textit{soothement} and \textit{soothment}). The query returned a total of 419 types of raw data. I then extracted those with a frequency of 1 or 2 (i.e., hapaxes as well as dis legomena) for further investigation. I included also dis legomena at this point because the search results may be corrupted in various ways, concealing actual hapaxes. Take, for instance, the case of \textit{musement}: The noun is listed with a frequency of 2 in COCA, but one of the attestations is actually \textit{bemusement} with a wrongly placed space (``be musement''). Also, it occasionally happens that the very same context is listed twice. By including dis legomena, I increased the chances of avoiding these problems and thus finding a larger number of pertinent forms.

A second corpus search was conducted manually in order to identify \textit{-ment} derivatives which are formed on the basis of verbs not listed in VerbNet. 
Using the web interface provided by Brigham Young University, COCA was searched for all words with a frequency of 1 or 2 ending in either 〈ment〉 or 〈ments〉.    

The collection of attestations resulting from the two COCA searches had to be weeded heavily. In addition to applying the five criteria already listed above for the \citetalias{OED} neologisms, I excluded a large number of obvious typing errors (e.g. ``aggreement'') and a few non-English attestations (e.g. ``the French n'aiment pas la sweat''). Furthermore, without any etymological data facilitating the task as in the OED, it was necessary to examine the context of each token in order to identify unwanted borrowings (e.g. French \textit{redoublement} as a technical term in fencing) as well as spelling mistakes which happen to produce well-formed possible derivatives (e.g. the conceivable derivative \textit{agement} in ``man agement''). The OED was systematically consulted to identify whether the alleged base of each type is a legitimate base verb (e.g. \textit{upset}\textsubscript{V}, *\textit{asort}\textsubscript{V}). If both a verb and a word of a different part of speech were listed in the OED, the verb was regarded as a possible base and the nominalization remained in the data set. 

During these revision procedures, the raw corpus data was reduced to a data set of 126 types (95 from the COCA DVD and 27 additional types from COCA online), which were produced between 1990 and 2012 (the complete range of the corpus at the time). In the following, I will refer to these parts of the data set as the \textit{Coquery data} and the \textit{BYU data}.
The complete catalogue of \ment{} nominalizations, including the \citetalias{OED} data, the Coquery data and the BYU data, comprises 138 types. In the next step, these types were grouped by assessing the semantics of their base verbs. 

\section{Semantic classification of base verbs}
\label{sec:meth-Vsem}

The basis for semantic classification of the base verbs was the VerbNet lexicon (\citealt{Kipper.2005}), which is based on Beth Levin's seminal work \textit{English verb classes and alternations} (\citealt{Levin.1993}).\footnote{Many readers are probably more familiar with the somewhat similar Berkeley FrameNet project (\citealt{Fillmore.2003,Fillmore.2010}). For the present study, I preferred to use VerbNet for methodological reasons: It is limited to verbs, which allowed me to straightforwardly query COCA for potentially deverbal nominalizations with \textit{-ment}; It contains more verbs than FrameNet, which allowed me to semantically categorize a larger number of base verbs from my data set directly (see below); It makes use of rather coarse participant categories, which allowed me to generalize over verbs and verb classes \textendash{} as opposed to FrameNet's often more fine-grained, and thus less generalizable, categories; And it provides a semantic decomposition of the verbs' semantics as a helpful starting point for the recursive frame formalizations, as opposed to FrameNet's flat lists of core frame elements.}
I will first introduce my general classification process (\sectref{sec:meth-Vsem-VN}), and then discuss some issues I encountered along the way (\sectref{sec:meth-Vsem-issues}).

\subsection{VerbNet as a basis for semantic classification}
\label{sec:meth-Vsem-VN}

\citet{Levin.1993} bases her verb classification on the idea that verbs allow certain argument alternations, or \textit{diathesis alternations}, and that this behavior is grounded in verb semantics. Viewed from the opposite perspective, those verbs which allow the same alternations should also share at least some meaning components. This idea was first introduced in Fillmore's (\citeyear{Fillmore.1970}) study on the two verbs \textit{break} and \textit{hit}. He concludes that ``[s]ome facts about language [...] have been shown to be explainable within a combined syntactic-semantic component'' (p. 131). Levin applies this finding on a much larger scale, categorizing more than 3,000 verbs into 49 classes and 186 subclasses (including subclasses of subclasses). Each section provides a list of members, the diathesis alternations in which these are found, commentary on their semantic properties and further syntactic peculiarities, as well as a list of pertinent literature. 

In the VerbNet project, the Levin classes are extended and partly revised, creating both new classes and further subclasses. At the time of writing this book, 6088 verbs are captured by 101 classes and 207 subclasses (VerbNet class hierarchy\footnote{\url{https://verbs.colorado.edu/verb-index/vn/class-h.php}, accessed 19 April, 2023}).
With the additional \mbox{(sub-)}classes, it was possible to incorporate more verbs in VerbNet, and already listed verbs have been recategorized more fittingly. The class descriptions in VerbNet contain a list of members, a list of thematic roles represented in their predicate-argument structure, and a number of representative contexts (``frames'') which are tagged syntactically and semantically. These frames largely correspond to Levin's diathesis alternations.

The fact that the Coquery data was extracted from a word list based on VerbNet entails that this part of the data set was already subdivided into base verb classes. If a verb was cross-listed, the most adequate class for the attested context was chosen. 
For the \citetalias{OED} and the BYU data, the attested base verbs were manually assigned to VerbNet classes. Nine base verbs from the OED data were classified directly since they were listed in VerbNet. The remaining 40 verbs from the OED and BYU data were classified indirectly by looking up synonyms and semantically similar verbs in VerbNet. This way, possible classes were identified, and the listed alternations were tested for the verb in question. If several verb classes were possible for a given base verb, the most appropriate verb class was chosen (see \sectref{sec:meth-Vsem-issues} for discussion).
Ultimately, the 138 types were assigned to 49 classes; 23 types were cross-listed in two ($n=17$), three ($n=3$) or four ($n=3$) classes. 

\subsection{Issues with the classification of base verbs}
\label{sec:meth-Vsem-issues}
Several issues arose during the process of assigning the base verbs to Levin/VerbNet categories. These problems are due partly to general issues with the semantic categorization of corpus data, and partly to the classification system used. 

The first problem was that the meaning of some nominalizations was so unclear that a categorization of the base verb became impossible. For instance, the meaning of the BYU hapax \textit{tracement} (see (\ref{ex:tracement})), and thus its base verb, could not be reconstructed by consulting the \citetalias{OED} and Google. Such types were eliminated from the data set.\largerpage

\begin{exe}\singlespacing
	\ex \label{ex:tracement} We're always putting those things in with the long-term view of building the soil, keeping certain \HighlightText{tracement} oils in the ground for flavor and also for healthy growth {\small (\acs{COCA} SPOK PBS\_Newshour 1990)}
\end{exe}

\noindent The second issue was presented by polysemous verbs. When the nominalization in context was clearly ascribable to one base verb class, only this class was chosen. One example is \textit{staggerment}. The noun is described as meaning `great amazement, astonishment' in the \citetalias{OED}, which makes the \textit{psych verb} reading of \textit{stagger} more likely than its reading as a \textit{run verb}. In other cases, however, it was not clear which class would be the most appropriate one; the distinctions were more fine-grained, and syntactic clues which distinguish one verb sense from the other disappear in the process of nominalization. 
An example is given in (\ref{ex:approvement1}), where both \textit{approve} (an \textit{allow verb} in VerbNet) and \textit{approve of} (a \textit{marvel verb}) are conceivable bases. In these cases, all possible verb classes were registered in the database. 

\begin{exe}\singlespacing
	\ex \label{ex:approvement1} What happened is people who were looting, and thieves and hooligans, once they receive the \HighlightText{approvement} from the press, they will just draw the V sign and then continue their looting. {\small (\acs{COCA} NEWS NYTimes 2003)}
\end{exe}

\noindent The third issue is that not only the nominalizations in my data set, but also many of the base verbs are very rare. For instance, many native speakers do not believe that \textit{discolor} is actually a verb. It follows that these verbs are hard to get an intuition for. In these cases, I probed the corpora for the relevant alternations and finally classified the verbs as accurately as possible given the available information. To give one example: VerbNet lists \textit{uplift} in the class `amuse verbs,' which describes verbs describing ``the bringing about of a change in psychological or emotional state'' (\citealt[191]{Levin.1993}). This reflects one possible reading of \textit{uplift}, paraphrased in the \citetalias{OED} as `to elevate morally.' The nominalization \textit{upliftment} can, however, also be found based on the sense `to lift up to a higher level or more erect position.' Therefore, it was expedient to add \textit{uplift} to the class of \textit{remedy verbs}, a subclass of \textit{change-of-state verbs}. In some cases, this approach involved deciding against an existing classification in \citet{Levin.1993} and/or VerbNet. 

The last issue is that VerbNet, as any classification, glosses over some distinctions. A verb listed in a given verb class may not participate in one or more of the relevant alternations. Likewise, a possible alternation may not be listed in a given verb class because it has not been deemed relevant, or two verb classes may involve the same set of alternations. 
Similar problems have to be kept in mind with regard to the semantic roles. I will briefly discuss the semantic role label \textsc{extent} to illustrate the problem. The subclass \textit{verbs of calibratable change-of-state} is the only subclass of change-of-state verbs for which VerbNet lists the \textsc{extent} role (italicized in (\ref{ex:extent-increase})). Other change-of-state verbs in the data set, however, clearly also have measurable properties (e.g. \textit{decenter, worsen}). For example, they can also be found with an \textsc{extent} participant, as exemplified in (\ref{ex:extent-decenter}) and (\ref{ex:extent-progress}).  

\begin{exe}
\ex \label{ex:extent}
\begin{xlist}
	\item \label{ex:extent-increase} The price of milk \textbf{increased} by \ExHighlight{ten percent}. {\small (VerbNet)}
	\item \label{ex:extent-decenter} All surfaces from the Coordinate Break onwards are \textbf{decentered} by \ExHighlight{-5 mm}. {\small(Google WEB customers.zemax.com 2015)}
	\item \label{ex:extent-progress} Of 10 patients with NAFL who had fibrosis progression, 3 \textbf{progressed} by \ExHighlight{1 stage}, 5 by \ExHighlight{2 stages} and 2 by \ExHighlight{3 stages} {\small(Google ACAD sciencedirect.com 2014)}
\end{xlist}
\end{exe}

\noindent In such cases, I have categorized the base verbs to the best of my knowledge and into the classes they have most in common with. 

\section{Selection and revision of the data set}\label{sec:meth-revis}\largerpage

After the data set was grouped by base verb class, I selected a subset for further analysis (\sectref{sec:meth-revis-select}) and prepared it by adding more attestations (\sectref{sec:meth-revis-add}) and by making some final changes (\sectref{sec:meth-revis-final}).

\subsection{Selecting types}
\label{sec:meth-revis-select}

Of all elicited nominalizations, the nouns based on the two best represented base verb classes were chosen for analysis, namely \textit{verbs of change-of-state} (henceforth \HighlightText{COS verbs}, $n=13$) and \textit{verbs of psychological state} (henceforth \HighlightText{psych verbs}, $n=20$). Their respective nominalizations will be called \HighlightText{COS nouns} and \HighlightText{psych nouns}. As a next step, the 33 nominalizations were subjected to a final evaluation with regard to their status as neologisms. For this, I applied OED frequency bands (\citetalias{OED.key}) as an independent measure. There are eight frequency bands, which are based on recent (1970–) Google Books Ngrams data in combination with other corpora. Nominalizations which fall into frequency bands 0 to 2 were categorized as neologisms (\sectref{sec:meth-revis-select-neo}), those falling into frequency bands 3 or 4 were used as \textit{supplementary data} (\sectref{sec:meth-revis-select-sub}), and types in a frequency band of 5 or higher were eliminated from the data set. 

\subsubsection{Neologism data}
\label{sec:meth-revis-select-neo}

First, those nominalizations with a frequency band of 1 or 2 were classified as neologisms (see Table \ref{tab:neoldata set}). 
Two nominalizations are found in frequency band 1, which contains ``extremely rare words unlikely ever to appear in modern text,'' for instance \textit{abaptiston}, \textit{grithbreach} or \textit{zeagonite} (\citetalias{OED.key}). Their frequency per million words in the above-mentioned corpora and time span is given as zero.  
Next, 11 nominalizations from the data set can be found in frequency band 2, which contains words ``which are not part of normal discourse and would be unknown to most people'' (ibid.). Examples are \textit{abactinal, unwhigged} and \textit{acicularly}. Words in this frequency band occur less than 0.0099 times per million words. 

\begin{table}
	\caption{Data set of \textit{-ment} neologisms (hapaxes and OED neologisms). Subscript numbers indicate variants of polysemous base verbs.\label{tab:neoldata set}} 
		\begin{tabular}{lll}
			\lsptoprule
			Frequency band& COS nouns  & Psych nouns \\
			&  ($n=11$) &  ($n=18$) \\
			\midrule 
			1 ($n=2$) & & {soothment}\\ 
		              & & {staggerment} \\ 
			2 ($n=11$) & {congealment} & {affrightment}\\ 
			& {debauchment} & {annoyment}\\ 
			& {discolorment} & {approvement}\\ 
			& {worsenment} & {enragement}\\ 
			&& {perturbment}\\ 
			&& {worriment}\textsubscript{1}\\ 
			&& {worriment}\textsubscript{2}\\ 
			none ($n=4$) & {bedragglement} & {reassurement} \\ 
			& {befoulment} & {upsetment} \\ 
			unlisted ($n=12$) & {besmirchment} & {bumfuzzlement} \\ 
			& {decenterment} & {confoundment} \\ 
			& {embetterment} & {dumbfoundment} \\ 
			& {jugglement} & {endullment} \\ 
			& {progressment} & {enrapturement}\\ 
			&& {musement} \\
			&& {nonplusment} \\ 
			\lspbottomrule
		\end{tabular} 
\end{table}

Four derivatives are included in the \citetalias{OED} without a frequency band since they are listed under their respective base verb. These derivatives were looked up in \citet{Webster}, where none of them was listed. They were therefore also regarded as neologisms.

Finally, 12 nominalizations are not listed in the \citetalias{OED} at all. These were again looked up in \citet{Webster}, and 11 types, which did not have an entry, were categorized as neologisms. The only exception, \textit{besmirchment}, was checked in the Google Books Ngram Viewer\footnote{\url{http://storage.googleapis.com/books/ngrams/books/datasetsv2.html}, accessed 19 April, 2023}, where its highest frequency between 1970 and 2008 (the whole range of the corpus) is under $2\cdot10^{-7}\%$. Since this value is well under the benchmark for frequency band 2 (which would correspond to an average of $9.9\cdot10^{-7}\%$), \textit{besmirchment} was categorized as a neologism as well.\largerpage 

All nominalizations whose status as neologisms has been confirmed are given in Table \ref{tab:neoldata set} with their respective frequency band and noun class. The neologism data collected from hapaxes and \citetalias{OED} neologisms contains 29 types, of which 11 are COS nouns and 18 are psych nouns. Three types did not meet the criteria and were recategorized as supplementary data, as I will explain below. 

In order to be able to generalize, I decided to expand the data set again. Nominalizations with higher frequencies on the COCA DVD were added consecutively, as long as they met the neologism criteria described above. Starting with dis legomena, continuing with tris legomena, etc., I was able to add five more types to the neologism data set. They are given in Table \ref{tab:neoldata set2}. The most frequent type is \textit{dispersement} with a frequency of ten on the COCA DVD. The combined neologism data set contains 14 COS nouns and 20 psych nouns. 

\begin{table} 
	\caption{Data set of \textit{-ment} neologisms (dis legomena, tris legomena, and rare forms). Subscript numbers indicate variants of polysemous base verbs.} \label{tab:neoldata set2}
		\begin{tabular}{lll}
			\lsptoprule
			Frequency band & COS nouns  &   Psych nouns \\
			&  ($n=3$) &  ($n=2$) \\
			\midrule 
			2 ($n=4$)&increasement\textsubscript{1}  & abashment \\
			& increasement\textsubscript{2}  & disheartenment\\
			unlisted ($n=1$)&dispersement &\\
			\lspbottomrule
		\end{tabular} 
\end{table}



\subsubsection{Supplementary data}
\label{sec:meth-revis-select-sub}

Those types which were eliminated in the preceding section may not be neologisms, but they can nevertheless be regarded as unusual forms which are unknown to many native speakers, as my experience presenting them at conferences has shown. Therefore, rather than eliminating them from the study, they were moved to a second data set, which will be called \textit{supplementary data}.  
By subdividing the data set into neologism and supplementary data, it is possible to compare the neologisms with more frequent types, testing whether the latter are more restricted in their possible readings. 

In the supplementary data we find nominalizations with frequency bands 3 and 4.
Frequency band 3 contains words which ``are not commonly found in general text types like novels and newspapers, but at the same time they are not overly opaque or obscure'' (\citetalias{OED.key}). There is a spectrum between technical terms (\textit{agglutinative, argentiferous}) and very colloquial words (\textit{crackers, dirt-cheap}).
In frequency band 4 we find words which are ``recognizable to English-speakers, and are likely be [sic] used unproblematically in fiction or journalism'' (ibid.). \textit{Insectivore, egregious} and \textit{surrepticiously} are examples. 

Again, types of higher frequencies on the COCA DVD were added consecutively, as long as they were in frequency bands 3 or 4. 
The supplementary data set is given in Table \ref{tab:suppdata set}. The types' frequencies on the COCA DVD range from one (\textit{convincement}, a recategorized hapax), over several frequencies in the lower double-digit range (e.g. \textit{disbandment} with 12 attestations), to 131 (\textit{diminishment}). 

\begin{table}
\singlespacing
	\caption[Supplementary data set of \textit{-ment} nominalizations]{\label{tab:suppdata set}Supplementary data set of \textit{-ment} nominalizations. Subscript numbers indicate variants of polysemous base verbs.} 
		\begin{tabular}{lll}
			\lsptoprule
			Frequency band & COS nouns  & Psych nouns   \\
			&  ($n=7$) &  ($n=2$) \\
			\midrule 
			3 ($n=2$)&  & {bemusement}\\
                  && {convincement} \\
                  4 ($n=7$)     &{abridgement}         &     \\
                  &{diminishment}\textsubscript{1}   &                             \\
                  &{diminishment}\textsubscript{2}   &                             \\
                  &{disbandment}         &                 \\
                  &{embrittlement} &     \\
                  &{unfoldment}          &     \\
                  &{upliftment}    &     \\			
			\lspbottomrule
		\end{tabular}       
\end{table}

\subsection{Adding attestations}
\label{sec:meth-revis-add}

A general problem that arises from the chosen data collection method is ambiguity. Take, for instance, the definition of \textit{embrittlement} in the \citetalias{OED}: `The action of embrittle\textsubscript{v}, or the result of such action; loss of ductility.' Such ambiguity is problematic when investigating hapaxes, which are by definition attested only once in a given corpus. In any such unique attestation, one of two things may happen: Either, the hapax is unambiguous in the given context, making it impossible to know which further readings are conceivable. Or, the hapax is ambiguous in this context, so that it cannot be determined which meaning was intended by the speaker. Since most of the types in the data set are very rare, it is practically impossible to get the complete picture from the COCA attestations alone. A related problem occurs in the dictionary data. Although the OED aims at wide coverage, for obvious reasons it does not include every meaning variant ever attested. Since, however, it is exactly this kind of innovative, spontaneous, and fully transparent formation that is of interest for this study, it was indispensable to support the data set with further attestations. 
Therefore, a number of other corpora were probed for all types of the four semantic classes under investigation. 
The following corpora and sources contained data which I have included in the database and/or as illustrative examples in this book, in alphabetical order: 

\begin{itemize}
	\item BYU corpora
	\begin{itemize}
		\item BNC (\textit{British National Corpus,} \citealt{Davies.2004}) \acused{BNC}
		\item COCA (\textit{Corpus of Contemporary American English,} \citealt{Davies.2008}) \acused{COCA}
	    \item COHA (\textit{Corpus of Historical American English,} \citealt{Davies.2010}) \acused{COHA}
	    \item GloWbE (\textit{Corpus of Global Web-Based English}, \citealt{Davies.2013}) \acused{GloWbE}
	    \item HC (\textit{Hansard Corpus}, \citealt{Davies.2015b}) \acused{HC}
	    \item iWeb (\citealt{Davies.2018})   \acused{iWeb}
	    \item NOW (\textit{News on the Web}, \citealt{Davies.2016}) \acused{NOW}
	    \item SOAP (\textit{Corpus of American Soap Operas}, \citealt{Davies.2011}) \acused{SOAP}
	    \item TIME (\textit{TIME Magazine Corpus}, \citealt{Davies.2007}) \acused{TIME}
	    \item WIKI (\textit{The Wikipedia Corpus}, \citealt{Davies.2015}) \acused{WIKI}
	\end{itemize}
	\item Google
	\item GB (Google Books) \acused{GB}
	\item OED \acused{OED}
	\item Twitter
	\item WC (\textit{Webcorp}, \citealt{Renouf.2006}) \acused{WC}
\end{itemize}

\noindent Other corpora (including all BYU corpora available at the time) were probed as well, but either did not contain the types in question at all, or only contained readings which were already well represented in the database. 

All attestations in this study are tagged with the following information, in that order: Corpus, genre, source, year (e.g. \textit{WC NEWS articles.latimes.com 2002}). The only exception is Twitter, where mentioning the genre (that is, \textit{tweet}), would be superfluous. The following genres were distinguished:

\begin{itemize}
	\item academic literature (\acs{ACAD})
	\item comment in the comment section of a website (\acs{COMM})
	\item private blog (\acs{BLOG})
	\item fiction (\acs{FIC})
	\item newspaper or news blog (\acs{NEWS})
	\item nonfiction literature (\acs{NONFIC})
	\item popular magazine (\acs{MAG})
	\item transcript of spoken language (\acs{SPOK})
	\item website of a company or institution (\acs{WEB})
\end{itemize}

\noindent The corpora were first searched by probing for the type in question and, for each reading which was identified in the process, adding a number of representative attestations to the database. 
Crucially, the approach was qualitative and not quantitative, so that the database should be regarded as a collection of possible readings rather than a sample representing realistic ratios. 
For instance, \textit{bumfuzzlement} is mostly attested in a \textsc{result-state} reading. Once a number of clear examples with this reading had been identified, further attestations with a \textsc{result-state} were ignored. 
Then, the corpora were probed for syntactic contexts (e.g. arguments or temporal modifiers, see also \citealt{Lieber.2015}) which would evoke as yet unattested readings. For instance, 〈his bumfuzzlement of the〉 would elicit an event reading.  
The search for a given type in an unattested reading was terminated when either all attestations in the above-mentioned corpora had been examined, or when the fifth page of Google results (around 500 attestations) had been inspected, Google being the last resort with regard to data collection. 

This leads us to an important comment about the nature of the data sources. 
The rare nature of the data made it necessary to consult not only linguistic corpora but to turn to even larger sources, that is, Twitter, Google and Google Books.  
Such tools exhibit certain shortcomings in the context of serious linguistic investigation (e.g. unlimited corpus size, no data organization, no annotation, often unknown origin of the data). However, it has also been shown that they can be a convenient indicator for innovative language use (see \citealt{Diemer.2008} and the papers in \citealt{Hundt.2006}). In order to meet the requirements of academic research as well as possible, any indication that the author of a given text might not be a native speaker of English was taken as a reason to exclude this attestation. For this, the wider context was scanned for grammatical errors, awkward formulations or straightforward indicators of the country of origin.

\subsection{The final data set}
\label{sec:meth-revis-final}

These extensive corpus studies led to some final changes to the data set: 
First, \textit{jugglement} was deleted because almost no analyzable attestations could be identified.
Second, it was found that the COS noun \textit{upliftment} is frequently attested in a sense of moral elevation, which conforms with \textit{uplift}'s reading as a psych verb. It was therefore decided to cross-list \textit{upliftment} as a psych noun and a COS noun. Finally, nouns based on verbs from the COS subclass \textit{verbs of calibratable change-of-state} (e.g. \textit{increase} and \textit{diminish}) were not included in this study due to the complexity of their analysis. The modeling of scalar predicates in frames is an interesting and notorious challenge (see for instance \citealt{Gamerschlag.2014b, Zinova.2016}), but is beyond the scope of the present work.\footnote{Note that \textit{increasement} and \textit{diminishment} are still part of the data set since they are cross-listed also as `other alternating verbs of change-of-state.' I am merely disregarding their scalar properties here.} 
\tabref{tab:data set} gives an overview of the types in the final data set. In the two analyses, the distinction between neologism data and supplementary data will only be made when relevant. See the \hyperlink{Appendix}{Appendix} for a presentation of this list by source (i.e., \citetalias{OED}, Coquery, and BYU). See also the distribution of tokens in Figure \ref{fig:tokens}, \sectref{sec:meth-Nsem-coding}.

\begin{table}
	\centering 
	\caption[Final data set of \textit{-ment} nominalizations (types)]{Final data set of \textit{-ment} nominalizations (types). Subscript numbers indicate variants of polysemous base verbs.} \label{tab:data set}
	\begin{tabular}{lll}
\lsptoprule
Data set & COS nouns & Psych nouns \\
&($n=18$)&($n=23$)\\
\midrule
Neologism data & {bedragglement} & {abashment} \\
($n=32$) & {befoulment} & {affrightment} \\
& {besmirchment} & {annoyment} \\
& {congealment} & {approvement} \\
& {debauchment} & {bumfuzzlement} \\
& {decenterment} & {confoundment} \\
& {discolorment} & {disheartenment} \\
& {dispersement} & {dumbfoundment} \\
& {embetterment} & {endullment} \\
& {increasement} & {enragement} \\ 
& {progressment} & {enrapturement}\\
& {worsenment} & {musement} \\
& & {nonplusment}\\
& & {perturbment} \\
& & {reassurement} \\
& & {soothment} \\
& & {staggerment} \\
& & {upsetment}\\
& & {worriment}\textsubscript{1}\\
& & {worriment}\textsubscript{2}\\
Supplementary data & {abridgement} & {bemusement}\\
($n=9$) & {diminishment} & {convincement} \\
 & {disbandment} & {upliftment}\textsubscript{2} \\
& {embrittlement} & \\
& {unfoldment} & \\
& {upliftment}\textsubscript{1} & \\
\lspbottomrule
	\end{tabular}%
\end{table}%

\section{Semantic coding of derived nouns}
\label{sec:meth-Nsem}

The semantic coding of the derived nouns in context had three starting points: VerbNet semantic roles, previous literature, and meaning shifts predicted by the base verb frames.
In this section, I will describe which categories I applied (\sectref{sec:meth-Nsem-semcat}), how the coding proceeded (\sectref{sec:meth-Nsem-coding}), and which issues I encountered (\sectref{sec:meth-Nsem-issues}).
In the following, I will refer to the groups and subgroups of base verbs and nouns as \HighlightText{(semantic) classes} without special formatting (e.g. ``psych verbs'' and ``psych nouns''). The nominalization readings will be referred to as \HighlightText{(semantic) categories} and marked by small caps (e.g. ``\textsc{result-state}''). 

\subsection{Semantic categories}
\label{sec:meth-Nsem-semcat}

The ontology of semantic categories can be split into two major groups: participants (\sectref{sec:meth-Nsem-semcat-part}) and events (\sectref{sec:meth-Nsem-semcat-event}).  

\subsubsection{Participants} 
\label{sec:meth-Nsem-semcat-part}

The core participants of the relevant VerbNet classes are presented in Figure \ref{fig:participants} and Table \ref{tab:participant-defs}. 
Figure \ref{fig:participants} gives an overview of the hierarchical relations between the core participants and their hyperonyms, and Table \ref{tab:participant-defs} lists their definitions.\footnote{The participant hierarchy is not a type hierarchy as defined in \sectref{sec:fr-dus-basic-attr}. A type hierarchy with relational types causes a variety of problems, which I have opted to avoid.} 
The information presented here has been taken from \citet{Palmer.2017} if not indicated otherwise (see also the Unified Verb Index: References Page\footnote{\url{https://uvi.colorado.edu/references_page}, accessed 19 April, 2023}).
Later on, I will revise the listed categories according to my findings. 

\begin{figure}
	\small
		\begin{forest}
		for tree = {l=0cm}
			[Participants 
				[Actor
					[\HighlightText{Agent}\\{[+intentional]}
					]
					[Causer\\{[−intentional]}
						[\HighlightText{Stimulus}
						]
					]
				]			
				[Undergoer
					[\HighlightText{Patient}\\{[+affected]}
						[\HighlightText{Experiencer}\\{[+awareness]}
						]
					]
					[\HighlightText{Instrument}
					]
				]
				[Place
					[Goal
						[\HighlightText{Result}
						]
					]
				]
			]
		\end{forest}
		\caption[Participant categories for semantic coding]{\label{fig:participants}Participant categories for semantic coding, based on VerbNet. Relevant core participants are indicated by italics.}
\end{figure}

\begin{table}
         \caption[Definitions of participant categories and their hyperonyms]{\label{tab:participant-defs} Definitions of participant categories and their hyperonyms, adapted from \citet{Palmer.2017}. Relevant core participants are indicated by italics.}
         \begin{tabular}{l>{\RaggedRight\arraybackslash}p{0.75\textwidth}}
      \lsptoprule
      Category & Definition \\
      \midrule
      Actor & Participant that is the instigator of an event\\ 
       \HighlightText{Agent} & Actor in an event who initiates and carries out the event intentionally or consciously, and who exists independently of the event \\ 
       Causer\footnote{In VerbNet, this category is called \textit{cause}. I have renamed it in order to avoid confusion with the \textsc{cause} attribute. Coincidentally, \textsc{causer} is also the label applied in VerbNet version 3.3.} & Actor in an event (that may be animate or inanimate) that initiates the event, but that does not act with any intentionality or consciousness \\ 
      \HighlightText{Stimulus} & Causer in an event that elicits an emotional or psychological response \\ 
      Undergoer & Participant in a state or event that is not an instigator of the event or state \\
      \HighlightText{Patient} & Undergoer in an event that is usually structurally changed, for instance by experiencing a change-of-state \\
      \HighlightText{Experiencer} & Patient that is aware of the event undergone, which often involves an emotional or psychological response elicited by a stimulus \\
      \HighlightText{Instrument}\footnote{This definition of \textsc{instrument} is unusual. For most authors, an instrument is defined as an entity used by an agent to carry out an event and would therefore be subsumed among the \textsc{actor} categories rather than being defined as a subcategory of \textsc{undergoer}.} & Undergoer in an event that is manipulated by an agent, and with which an intentional act is performed \\
     Place\footnote{In VerbNet, \textsc{place} is a somewhat unusual metarole, with its daughters \textsc{location, source, goal, path,} and \textsc{value} being instantiated by both physical and abstract entities. For example, \textsc{goal} can be the physical goal of a motion action (\textit{She reached her hand \textbf{into the cookie jar}}) or a label used in a classifying action (\textit{She classified the works as \textbf{`dangerous'}}).} & The state in which an entity exists \\ 
     Goal & Place that is the end point of an action and that exists independently of the event \\ 
     \HighlightText{Result} & An outcome that comes into existence through the event \\ 
      \lspbottomrule 
     \end{tabular}
\end{table}

\subsubsection{Eventive classes} 
\label{sec:meth-Nsem-semcat-event}

In the semantic description of (deverbal) nouns, the term \textit{event} has been used with varying definitions, highlighting properties like telicity, agentivity or duration. This has resulted in contrasts between events and results (e.g. \citealt{Grimshaw.1990}), between events and states (e.g. \citealt{Filip.1999,Barque.2011}), between processes, events and states (e.g. \citealt{Ehrich.2000}), or between actions and non-agentive events (e.g. \citealt{Sil.2010}). Here, the term \HighlightText{event} has been chosen as a hyperonym for all kinds of situational, or \HighlightText{eventive}, categories (see also e.g. \citealt[84]{VanValinJr.1997}). 
Table \ref{tab:eventive-defs} gives an overview of the eventive categories that I will use for the description and frame formalization of \textit{-ment} derivatives. Note that this list should be regarded as a reference sheet; I will discuss and motivate the categories in Chapters \ref{ch:cos} and \ref{ch:psy}. 

\begin{table}
    \caption[Definitions of eventive categories]{\label{tab:eventive-defs}Definitions of eventive categories, sorted by first split in the hierarchy in Figure \ref{fig:signatureevent}. Abbreviations: c. = causation.}
    \begin{tabularx}{\textwidth}{lQ}
      \lsptoprule
      Category & Definition \\
      \midrule
      Event & Observable phenomenon taking place at a specific time and place
      \\
      Action & Actor does something (possibly with an instrument, possibly to a patient) \\ 
      Psych-action & Action that is related to an experiencer's psychological or emotional state \\ 
      Stimulus psych-action & Actor is a stimulus \\ 
      Experiencer psych-action & Actor is an experiencer \\ 
      State & State of a patient \\
      Having-form & Patient has a shape and surface \\
      Psych-state & Experiencer's psychological/emotional state \\
      Change-of-state & Patient changes in some way  \\ 
      Change-of-physical-form & Patient's shape or surface changes \\ 
      Change-of-psych-state & Experiencer changes psychologically/emotionally  \\ 
     Causation & Complex event (causing and caused subevent) \\ 
     Change-of-state c. & Caused subevent is a change-of-state \\ 
     Change-of-physical-form c. & Caused subevent is a change-of-physical-form \\ 
     Psych-state c. & Caused subevent is a psych-state \\
     Change-of-psych-state c. & Caused subevent is a change-of-psych-state \\
     Experienced c. & Causing subevent is a perception-event \\
     Agentive psych c. & Causing subevent contains an \textsc{agent} \\
      Psych-reaction & Complex psych event (\textit{explanation} and \textit{reaction} subevent) \\
     \lspbottomrule 
    \end{tabularx}
\end{table}

I have translated these eventive categories into a type signature in Figure \ref{fig:signatureevent}. The event types are there defined by their relation to each other (e.g. \textit{psych-state} as a kind of \textit{state}) as well as by appropriateness conditions which specify participants and, if applicable, subevents (see \sectref{sec:fr-dus-TS}). 

Several things should be pointed out here. 
First, these categories do not imply statements regarding the duration or telicity of the underlying concept. For instance, a change-of-state may be punctual or durative, as well as completed or ongoing. If either of those distinctions is relevant in a given context, this will be pointed out specifically.
Second, there are five relational eventive categories, namely \textsc{cause, effect, explanation, reaction} and \textsc{result-state}. Because they are relational, they are not given as types but only as attributes under the types they are introduced by. 
Third, the use of the symbol $\pm$ for optional participants should be seen as shorthand. It would be a cleaner solution to introduce distinct types for each kind of event, for instance an action with an instrument, and one without. However, this would greatly blow up the figure and require the use of unlexicalized labels, without much added value. 
Fourth, in order to reduce visual clutter, I have decided to leave out agentive psych causation types. These are implicitly depicted since the attribute \textsc{actor}, which is listed under \textit{causation}, can be instantiated by \textsc{agent}. 
Fifth, also for reasons of space, I depict hyponymy relations of participant categories separately. To spell out participant relations in a type signature, one would have to include specifications in the appropriateness conditions. For example, that \textsc{experiencer} is a hyponym of \textsc{patient} would be expressed by a notation like `\textit{psych-st}, \textsc{patient} $\doteq$ \textsc{experiencer}.' 
By lacking this kind of information, the type signature in Figure \ref{fig:signatureevent} is less explicit than it could be, but more readable. This also means that it needs to be interpreted in conjunction with Figure \ref{fig:participants} above.
Finally, it should be kept in mind that the type signature will be revised in the course of this study. In its current state, it contains those types which are required to model previous research on the investigated verb and noun classes. We will see that it contains types which are not actually relevant to describe my data, while some relevant eventive categories, participants, and constraints are missing. Parts of the type signature are even self-contradictory due to incompatibilities between the VerbNet participant hierarchy and other literature. For example, to define the \textsc{actor} as an \textsc{experiencer}, as done under the type \textit{experiencer-psych-action}, is not allowed by Figure \ref{fig:participants}. At the end of Chapters \ref{ch:cos} and \ref{ch:psy}, respectively, I will incorporate my findings and propose updates to the type signature.

\begin{sidewaysfigure}
	\centering
		\resizebox{\linewidth}{!}{	
			\begin{forest} for tree={font=\itshape,l sep=1.5cm}
			[event
				[action\\
					\textsc{actor} 
					\\ $\pm$\textsc{instr}  
					\\ $\pm$\textsc{patient}
						[\textit{perc}
						\\\textsc{stim}
						\\\textsc{exp}
						]
						[ag-act
						\\\textsc{ag}
						]
						[psych-act\\
							\textsc{stim}\\
							\textsc{exp}
								[stim psych-act\\
								\textsc{actor}$\doteq$\textsc{stim}
								]
								[exp psych-act\\
								\textsc{actor}$\doteq$\textsc{exp}
								]
						]
				]
				[state\\\textsc{patient}
					[having-form, name=hf]
					[psych-st\\\textsc{exp}, name=psst]
				]
				[c-o-s\\\textsc{patient}\\\textsc{res-st :} state, name=cos
					[c-o-phys-form\\\textsc{patient}\\\textsc{res-st :} having-form, name=copf]
					[c-o-psych-st\\\textsc{exp}\\\textsc{res-st :} psych-st, name=cops]
				]
				[caus\\
					\textsc{actor} \\ 
					\textsc{patient} \\ 
					$\pm$ \textsc{instr}\\
					\textsc{cause :} event\\ 
					\textsc{eff :} event
						[c-o-s caus
						\\\textsc{ag}
						\\\textsc{eff :} c-o-s
							[c-o-phys-form caus\\
							\textsc{eff :} c-o-phys-form, name=capf
							]
							[c-o-psych-st caus\\
							\textsc{stim}\\
							\textsc{exp}\\
							\textsc{eff :} c-o-psych-st, name=caps
							]
						]
							[psych-st caus\\
							\textsc{eff :} psych-st
								[experienced caus\\
								\textsc{cause :} perc
								]
							]
				]
				[psych-reac\\
					\textsc{stim}\\
					\textsc{exp}\\
					\textsc{expl :} event\\
					\textsc{reac :} psych-st 
				]
				]
			\end{forest}
		}
		\caption[Type signature of eventive categories]{\label{fig:signatureevent} Type signature of eventive categories. Optionality is indicated by $\pm$. Abbreviations: act = action, ag = agent, caus = causation, c-o-phys-form = change-of-physical-form, c-o-psych-st = change-of-psych-state, c-o-s = change-of-state, eff = effect, exp = experiencer, expl = explanation, instr = instrument, perc = perception, psych-st = psych-state, reac = reaction, res-st = result, stim = stimulus.}
\end{sidewaysfigure}

\subsection{The coding procedure}
\label{sec:meth-Nsem-coding}

The semantic categorization of each nominalization in context was conducted by three trained linguists.
The coding procedure was carried out in three steps. 
First, each annotator inspected the attestations on their own, assigning labels and/or paraphrases. 
As a second step, the annotators convened in person to discuss and possibly revise their decisions, as well as streamline the chosen labels. 
Finally, those attestations for which an inter-annotator agreement of at least two of three was reached were included in the database, while disputable cases were marked as such and filed for the record. 

In order to assign semantic labels to a given nominalization in context, a number of factors were considered. 
First, the context was examined for clues of various kinds: Does it include an unambiguous collocation, such as the one in (\ref{ex:enragement4-1})? Are there straightforward definitions, paraphrases or parallel constructions with the same referent, as in (\ref{ex:enragement4-2})? Does the context of the nominalization parallel one or more arguments of the base verb, as in (\ref{ex:perturbment}), pointing to a transpositional reading?

\begin{exe}
	\ex
	\begin{xlist}
	\item \label{ex:enragement4-1} Once in the \textbf{state of enragement} she will be like a fury {\small (Google BLOG tesof.com 2013)}
	\item \label{ex:enragement4-2} [T]he tonnage would be an indication of \textbf{enragement} or as we say around these parts a ``the \textbf{piss-off-edness}'' indicator. \\{\small (\ac{GloWbE} NEWS blogs.news.com.au 2012)}
	\item \label{ex:perturbment} Did you put a sound system in your car not specifically for your enjoyment but for the \textbf{perturbment of others} within three square miles? {\small (Google BLOG 2008)}
	\end{xlist}
\end{exe}

\noindent If no straightforward contextual clues could be identified, the next step was to apply substitution tests. For instance, \textit{V-ment} was considered to express the \textsc{stimulus} category when it could be substituted by a paraphrase similar to \textit{V-ing influence(s), something which V-s someone}, or \textit{things which V someone}:

\begin{exe}
	\ex \label{ex:worriment} 
	\begin{tabbing} 
		Monitor your sites for outages, errors, and other \= [\textbf{worriments}]  \kill
		Monitor your sites for outages, errors, and other \> [\textbf{worriments}]  \\
		{\small (\ac{WC} angel.co 2011)} \> [worrying things]   
	\end{tabbing}
\end{exe}

\noindent When several readings were conceivable within one attestation, all of them were regarded as valid usages of the noun. In these cases, the inter-annotator agreement had to be at least two of three for each of the assigned readings. For instance, for the example given in (\ref{ex:endullment-meth}), all three annotators agreed that \textit{endullment} can be seen as a transposition of the verb's eventive reading (`avoid working toward endulling the students') or as a result-state (`avoid working toward endulled students'). 

\begin{exe}
	\ex \label{ex:endullment-meth} Regelski, for example, writes about the need to reject ``methodolatry'' and ``taken-for-granted recipes'' and avoid working toward the ``\textbf{endullment}'' of students. {\small (\acs{COCA} ACAD MusicEduc 2005)}
\end{exe}

\noindent The final database contains 369 attestations, including 192 for COS nouns and 177 for psych nouns, and is available via \url{https://osf.io/4vwrn/}. Figure \ref{fig:tokens} shows the distribution by nominalization, from \textit{abashment} with 21 attestations down to \textit{approvement} with only 3. 
Note that, due to the purposeful sampling approach applied during data gathering, this chart does not allow for any generalizations with regard to productivity or polysemy of any given type. Thus, a higher number does not necessarily mean that a nominalization is more frequent or more polysemous than another one with a lower number.

\begin{figure}
\includegraphics[width=\textwidth]{figures/TokensAttestations.pdf}
\caption{Final data set of \textit{-ment} nominalizations (tokens)} \label{fig:tokens}
\end{figure} 

\subsection{Issues with the classification of nominalizations}\label{sec:meth-Nsem-issues}\largerpage

While classifying my nominalization data semantically, a number of methodological issues arose, both with regard to the coding procedure and to the theoretical backdrop.

The first issue relates to the choice of semantic category labels. There is a large variety of such labels to be found in the semantic literature, and more often than not, a given author introduces some distinction or semantic detail which sets their use of a semantic label apart from other uses. In order to avoid confusion, it is essential to be very transparent regarding the use and definition of semantic categories. In the present book, this is achieved by adhering to the following four principles:
First, I chose a clear starting point. For the participant categories, the VerbNet role labels were used (see \sectref{sec:meth-Nsem-semcat-part}). For the eventive categories, I have not been able to identify a classification system of the right granularity, which is why I presented my own system in \sectref{sec:meth-Nsem-semcat-event}. The individual labels were largely gathered from the existing literature. 
Second, in the two sections I just mentioned, I give a definition for each semantic category. While this may seem rather basic, it is often the case that authors use a supposedly unambiguous label without specifically defining it, which then leads to misunderstandings or misinterpretations on the part of the reader.  Third, I locate the semantic labels in the type signature in \sectref{sec:meth-Nsem-semcat-event}. Most eventive categories are non-relational, and can be included as types. The participant categories as well as some eventive ones are relational, and are therefore included as appropriateness conditions on types. This way, the relations between labels are clearly spelled out. 
Finally, in the course of this study, it will become necessary to adjust the VerbNet- and literature-based semantic classification to reflect the actual findings from my nominalization data. All such changes will be documented explicitly by updating both the list of definitions and the type signature (see Sections \ref{sec:cos-output-formal-TS} and \ref{sec:psy-output-formal-TS}). 

The second issue is that I use the same semantic categories with two distinct purposes, namely 
on the one hand for descriptions of verbs, and on the other for analyses of their contexts. This has been described as a problem by \citet{Huyghe.2020}, who work on agentivity. Specifically, they criticize approaches like VerbNet, which categorize verbs as semantically agentive or non-agentive. In reality, they state, agentivity is in most cases established by context, and thus not a matter of semantics but of pragmatics. 
In the present study, the starting point of my semantic formalization is precisely what they criticize, since I use the verbs' core participants as a proxy for their semantics. I chose this starting point because I prefer using, testing, and revising an existing categorization rather than starting from scratch. Therefore, the term \textit{starting point} really is key: I will not take the list of participants as presented above at face value, but extend and revise it during the course of this study.\largerpage 

Next, I would like to come back to the issue of granularity of semantic categories. While categorizing readings semantically, one option is to distinguish only basic categories, such as the most general distinction between \textsc{actor, undergoer, place, time} and \textsc{circumstance} in VerbNet (see \citealt[331]{Palmer.2017}). Or, semantic distinctions can be more fine-grained. On the participant level, VerbNet uses categories of medium granularity; for instance, the verb \textit{attack} is listed with \textsc{agent} and \textsc{patient}. An example of a more fine-grained approach is the Berkeley FrameNet project (\citealt{Fillmore.2003}), where the two frame elements \textsc{assailant} and \textsc{victim} are given for the lexical entry for \textit{attack}. In the present study, the goal is to generalize over semantically related but distinct verbs, so that an approach with medium granularity (i.e., VerbNet) promises the best results. 

Next, let me return to the issue of ambiguity, which frequently arises due to the polysemous nature of the nominalizations in my data set. 
As I have detailed in the previous section, I do not per se regard ambiguous attestations as a problem in this study, but deal with them systematically, regarding all possible readings as valid ones. Still, I am aware that it would be preferable to have an unambiguous attestation to show for each reading. I will take a quantitative perspective on ambiguity in my data set in \chapref{ch:observations}.

Finally, problems arise because a nominalization's semantics may be modified by context, producing a reading outside of the noun's lexical range. This post-lexical process is often referred to as \HighlightText{coercion}. Importantly, in the study of derivational semantics the readings which arise via coercion need to be distinguished from those created by affixation. To do so, I used introspection and consulted with my fellow annotators (see also \chapref{ch:conclusion}).

Keeping the issues I have just described in mind, let us now turn to the first of my two analyses, namely of \textit{-ment} on change-of-state verb bases.
