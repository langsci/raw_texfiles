\chapter{Gaps and ambiguity}
\label{ch:observations}

A critical issue in my research for this book has been the relative sparseness of neologism data, and thus the difficulty in finding attestations for readings that I assume to be possible for a given \textit{-ment} derivative. 
This has two observable effects in my data set. First of all, it exhibits gaps. That is, readings which are in principle possible for \textit{-ment} in general, or for a given derivative specifically, are not always attested. 
Second, those attestations I did find are often ambiguous between two or even more readings.  

In this chapter, I am taking a quantitative perspective on these two issues.
I will first discuss to what extent my data exhibits gaps, both from an onomasiological and from a semasiological point of view (\sectref{sec:obs-gaps}). That is, I will examine the percentage of gaps with regard to the different readings, and with regard to the different nominalizations. 
Then, I will take a closer look at the attestations in my data set, assessing how many readings and nominalizations are unambiguously attested, and for how many I have only been able to identify ambiguous attestations (\sectref{sec:obs-amb}). 

\section{Gaps}
\label{sec:obs-gaps}
 Table \ref{tab:GapsCOS} gives an overview of how comprehensively COS nouns are attested. In the leftmost column, the COS nominalizations are given in alphabetical order. The remaining columns indicate whether a given reading is attested (``att.''), or not (``gap''). For instance, \textit{abridgement}, not being attested in a \textsc{change-of-state} reading, has a gap indicated in the sixth column. For a better overview, the two rightmost columns indicate the total number and the percentage of gaps per nominalization (e.g. 1 and 14 for \textit{abridgement}), and the bottom two rows indicate the total number and the percentage of gaps per reading (e.g. 1 and 6 for \textsc{transposition}). In the bottom right corner, the total number of gaps and the overall percentage of gaps are indicated.
To facilitate the counting of gaps per reading, the three complementarily distributed readings \textsc{patient, implicit product} and \textsc{result} are collapsed into one column. 
A visual representation of the ratio between attestations and gaps is given below, in Figures \ref{fig:GapsByReadingCOS} (gaps per reading) and \ref{fig:GapsByNomCOS} (gaps per nominalization).

\begin{table}
\caption{Attested and unattested readings of COS nouns. The figures and tables in this section contain the following abbreviations: amb. = ambiguous, att. = attested, \mbox{c-o-s} = change-of-state, c-o-p-s = change-of-psych-state, evt = event, impl-prod = implicit product, instr = instrument, pat = patient, res =result, res-st = result-state, stim = stimulus, trans = transposition, unamb. = unambiguous.} 
\label{tab:GapsCOS}
\begin{tabular}{llllllllrr}
\lsptoprule
Nominalization             & \begin{sideways}Transposition\end{sideways} & \begin{sideways}Instrument\end{sideways} & \begin{sideways}Causer\end{sideways} & \begin{sideways}Causing-event\end{sideways} & \begin{sideways}Change-of-state\end{sideways} & \begin{sideways}Result-state\end{sideways} & \begin{sideways}Pat/impl-prod/res\end{sideways} & \begin{sideways}Gaps total\end{sideways} &
\begin{sideways}Gaps per cent\end{sideways}\\
\midrule
 abridgement& att.& att.& att. & att.& \textbf{gap} & att. & att. & 1 & 14 \\
bedragglement& att.& att.& att. & att.& att.& att. & att. & 0 & 0\\
 befoulment & att.& att.& att. & att.& \textbf{gap} & att. & att. & 1 & 14 \\
besmirchment & att.& att.& att. & att.& att.& att. & att. & 0 & 0 \\
 congealment& att.& att.& \textbf{gap}& \textbf{gap} & att.& att. & att. & 2 & 29 \\
debauchment& att.& att.& att. & att.& att.& att. & att. & 0 & 0 \\
 decenterment & att.& \textbf{gap} & \textbf{gap}& \textbf{gap} & att.& att. & att. & 3 & 43 \\
diminishment & att.& \textbf{gap} & att. & att.& att.& att.& att. & 1 & 14 \\
 disbandment& att.& att.& att. & att.& att.& att. & \textbf{gap}& 1 & 14 \\
discolorment & \textbf{gap} & \textbf{gap} & att. & \textbf{gap} & att.& att. & att. & 3 & 43 \\
 dispersement & att.& att.& \textbf{gap}& \textbf{gap} & att.& att. & att. & 2 & 29 \\
embetterment & att.& att.& att. & att.& att.& att.& att. & 0 & 0 \\
 embrittlement& att.& att.& att. & att.& att.& att. & att. & 0 & 0 \\
increasement & att.& att.& att. & att.& att.& att.& att. & 0 & 0 \\
 progressment & att.& att.& \textbf{gap}& att.& att.& att.& att. & 1 & 14 \\
unfoldment & att.& att.& att. & att.& att.& att. & att. & 0 & 0 \\
 upliftment & att.& \textbf{gap} & att. & att.& att.& att. & \textbf{gap}& 2 & 29 \\
worsenment & att.& \textbf{gap} & att. & \textbf{gap} & att.& att.& att. & 2 & 29 \\
\midrule
Gaps total & \multicolumn{1}{r}{1}     & \multicolumn{1}{r}{5}     & \multicolumn{1}{r}{4}      & \multicolumn{1}{r}{5}           & \multicolumn{1}{r}{2}     & \multicolumn{1}{r}{0}      & 
\multicolumn{1}{r}{2}  &
\multicolumn{1}{r}{\textbf{19}}&  -\\
Gaps per cent & \multicolumn{1}{r}{6}     & \multicolumn{1}{r}{28}     & \multicolumn{1}{r}{22}      & \multicolumn{1}{r}{28}           & \multicolumn{1}{r}{11}     & \multicolumn{1}{r}{0}      & \multicolumn{1}{r}{11}  & -
 & \multicolumn{1}{r}{\textbf{15}} \\
\lspbottomrule  
\end{tabular}
\end{table}


Let us first take an onomasiological point of view. In Figure \ref{fig:GapsByReadingCOS}, which visualizes Table \ref{tab:GapsCOS} by column, gaps are represented by light gray. We see that there are comparatively few gaps per reading, given that I have investigated neologisms. In total, as represented by the rightmost column, only 15\% of all expected combinations of readings and nominalizations are not attested.
\textsc{Result-state} readings, with no gaps at all, are best attested, closely followed by transpositional readings. The readings with the highest percentage of gaps are \textsc{instrument} and \textsc{causing event}, with five (28\%) gaps each, followed by \textsc{causer} with four gaps (22\%).

\begin{figure}[t]
	\includegraphics[width=0.89\linewidth]{figures/GapsCosReading.pdf}
	\caption{Ratios of attested and unattested COS readings}
	\label{fig:GapsByReadingCOS}
\end{figure}

There is a clear pattern in the distribution of gaps for COS nouns. The three most poorly attested readings (\textsc{instrument, causing event,} and \textsc{causer}) share the semantics of `something that causes V-ing' (henceforth \textsc{originator} readings). That \textsc{originators} are comparatively infrequent can be attributed to two possible factors. First, it may be an artifact of my sampling strategy, since these readings are harder to identify by selective corpus searches. Thus, when probing the corpora for contexts which favor a specific reading, some contexts are more helpful than others. For example, 〈``state of V-ment''〉 can be used to quickly find attestations for \textit{V-ment} in a \textsc{result-state} reading, while contexts which may indicate an \textsc{originator} reading are much less specific. Contexts that I used were, for example, the indefinite article (〈``a V-ment''〉), or the plural (〈V-ments〉). 
Second, the finding may indeed reflect reality: \textsc{Originator} readings may actually be less frequent than other \ment{} readings, which would make it less likely to find them attested in the first place. A possible explanation for the sparseness of \textsc{originator} readings could be that these are subject-oriented, for which other nominalizing suffixes are preferred (e.g. \textit{-er} and \textit{-ant}, see e.g. \citealt{Lieber.2004}).\footnote{I extend thanks to Rochelle Lieber, who pointed me to the dichotomy of subject- versus object-preferring suffixes.} Thus, it is possible that we are dealing with a partial blocking effect, with subject-oriented suffixes making \textsc{originator} readings less likely for \ment{} derivatives.

Illustrating the semasiological point of view, Figure \ref{fig:GapsByNomCOS} visualizes Table \ref{tab:GapsCOS} by row. Most nominalizations are well or even perfectly attested, with no gaps for a total of seven nouns. Some nominalizations, however, are lacking multiple readings. The highest percentage of gaps is exhibited by \textit{decenterment} and \textit{discolorment}, with three gaps each (43\%). 

\begin{figure}
	\includegraphics[width=0.89\linewidth]{figures/GapsCosNom.pdf}
	\caption{Ratios of attested and unattested COS nouns}
	\label{fig:GapsByNomCOS}
\end{figure}
	
These gaps can indicate one of two things: Either, \textit{decenterment} and \textit{discolorment} are less flexible in their range of readings, or I did not find attestations for readings which are in principle possible. I propose that the second option is more likely for three reasons:
First, the gaps for these two nominalizations seem to be related to the lack of \textsc{originator} readings discussed above. \textit{Decenterment} is not attested in either of the three \textsc{originator} readings, and \textit{discolorment} is not attested in an \textsc{instrument, causer} or transpositional reading.
Second, both nominalizations have other dominant readings, namely \textsc{implicit product} (\textit{discolorment}) and \textsc{result-state} (both). These may make the search for less frequent readings prohibitively difficult by dint of the sheer number of examples of the more common readings.\largerpage
Third, both nominalizations are extremely infrequent, compared to most other nominalizations in the COS data. 
For most of the other neologisms, the annotators were able to sift through a sizable number of hits in Google and the largest of the BYU corpora (e.g. iWeb and GloWbE), but \textit{decenterment} and \textit{discolorment} belong to the four least frequently attached COS nouns, as illustrated in Table \ref{tab:CosFrequencies} with frequencies from the NOW corpus (14.7 billion words, accessed March 15, 2022). 
Considering these three factors, it comes as no surprise that \textit{decenterment} and \textit{discolorment} exhibit comparatively many gaps.  

\begin{table}
	\caption[]{Frequencies of COS nouns in the NOW corpus} \label{tab:CosFrequencies}
	\begin{tabular}{lr}
		\lsptoprule
		Nominalization & Frequency \\
		\midrule
		bedragglement  & 0         \\
		decenterment   & 0         \\
		progressment   & 0         \\
		discolorment   & 1         \\
		befoulment     & 3         \\
		debauchment    & 4         \\
		worsenment     & 4         \\
		embetterment   & 7         \\
		besmirchment   & 17        \\
		increasement   & 18        \\
		congealment    & 29        \\
		unfoldment     & 42        \\
		dispersement   & 65        \\
		embrittlement  & 260       \\
		abridgement    & 476       \\
		diminishment   & 1209      \\
		disbandment    & 4194      \\
		upliftment     & 9289     \\
		\lspbottomrule
	\end{tabular}
\end{table}

\begin{table}
\caption[Attested and unattested readings of psych nouns]{Attested and unattested readings of psych nouns} 
\label{tab:GapsPsych}
\begin{tabular}{llllllrr}
\lsptoprule
Nominalization & \begin{sideways}Transposition\end{sideways}& \begin{sideways}Stimulus\end{sideways}& \begin{sideways}Causing-event\end{sideways}& \begin{sideways}C-o-p-s\end{sideways}& \begin{sideways}Result-state\end{sideways}& \begin{sideways}Gaps total\end{sideways}& \begin{sideways}Gaps per cent\end{sideways}\\
\midrule
 abashment      & att.  & att. & att.        &         & att. & 0 & 0 \\
affrightment   & att.  & att. & att.        &         & att. & 0 & 0  \\
 annoyment      & att.  & att. & att.        &         & att.& 0 & 0   \\
approvement    & att.  & \textbf{gap}  &         &         &      &  1 & 50  \\
 bemusement     & att.  & att. & att.        &         & att.  & 0 & 0 \\
bumfuzzlement  & att.  & att. & att.        &         & att. & 0 & 0  \\
 confoundment   & att.  & att. & att.        &   att.      & att. & 0 & 0  \\
convincement   & att.  & att. & att.        &         & att. & 0 & 0  \\
 disheartenment & att.  & att. & att.        &      att.   & att. & 0 & 0  \\
dumbfoundment  & att.  & att. & att.        &         & att. & 0 & 0  \\
 endullment     & att.  & \textbf{gap}  & att.        & att.    & att. &  1 & 20  \\
enragement     & att.  & att. & att.        & att.    & att. & 0 & 0  \\
 enrapturement  & att.  & att. & att.        &         & att. & 0 & 0  \\
musement       & att.  & att. &         &         & att. & 0 & 0  \\
 nonplusment    & att.  & att. & att.        &         & att. & 0 & 0  \\
perturbment    & att.  & \textbf{gap}  & att.        &         & att. &  1 & 25 \\
 reassurement   & att.  & att. & \textbf{gap}         &         & att. &  1 & 25 \\
soothment      & att.  & att. & \textbf{gap}         & att.    & att. &  1 & 20 \\
 staggerment    & att.  & att. & att.        &         & att. & 0 & 0  \\
upliftment     & att.  & att. & att.        & att.    & att. & 0 & 0  \\
 upsetment      & att.  & att. & att.        &         & att. & 0 & 0  \\
worriment      & att.  & att. & att.        &         & att. & 0 & 0  \\
\lspbottomrule
\end{tabular}
\end{table}


The gaps in the psych data are given in Table \ref{tab:GapsPsych}. It is constructed in parallel to Table \ref{tab:GapsCOS}, with the addition that empty cells indicate irrelevant combinations of nominalization and reading. For example, the \textsc{change-of-psych-state} reading is relevant for only six nominalizations, so that the remaining cells in this column are empty.\largerpage
Overall, we see that the number of gaps is even smaller compared to the COS data: There are five gaps, which corresponds to 6\% of the expected combinations of nominalization and reading.{\interfootnotelinepenalty=10000\footnote{\textit{Confoundment} and \textit{disheartenment} are unexpectedly attested in a \textsc{change-of-state} reading (see \sectref{sec:psy-output-survey-cos}). These two combinations are not included in this calculation.}}

Figure \ref{fig:GapsByReadingPsych} illustrates the proportion of attested and unattested readings, excluding irrelevant combinations. Only \textsc{stimulus} and \textsc{causing event} have gaps; the remaining three readings are attested for every expected psych noun. 
It is again the \textsc{originator} category that is not as well represented in the data, which supports my conclusion that \textsc{originator} readings are in principle possible, but overall less frequent.
In Figure \ref{fig:GapsByNomPsych}, we see that the percentage of gaps per nominalization is also low: Five nominalizations have one unattested reading each.

\begin{figure}
	\includegraphics[width=0.89\linewidth]{figures/GapsPsychReading.pdf}
	\caption{Ratios of attested and unattested psych readings}
	\label{fig:GapsByReadingPsych}
\end{figure}

\begin{figure}
	\includegraphics[width=0.89\linewidth]{figures/GapsPsychNom.pdf}
	\caption{Ratios of attested and unattested psych nouns}
	\label{fig:GapsByNomPsych}
\end{figure}
	
All things considered, the gaps in my data can most likely be attributed to scarcity of data in combination with a partial blocking effect of standardly sub\-ject-de\-noting suffixes, and should not be regarded as (indirect) negative evidence for a given combination of nominalization and reading being impossible. 
As can be expected, I have observed that overall the availability of data is very limited when investigating neologisms. 
For illustration, see Table \ref{tab:Frequencies}, which compares the frequencies for the COS and psych nouns in my data set with the 15 most frequent \textit{-ment} nominalizations in the NOW corpus. 
While there is clearly a wide range of frequencies within the data set, lexicalized \textit{-ment} nominalizations like \textit{government} are still thousands of times more frequent than even the most frequent neologism. It therefore comes as no surprise that their less central readings can be harder or impossible to find, even after extending the investigation to corpora in the wider sense (that is, Google and Twitter).\largerpage[-2] 

\begin{table}
	\caption[Frequencies of the investigated nominalizations compared to lexicalized \textit{-ment} nouns in NOW]{Frequencies of the investigated nominalizations compared to lexicalized \textit{-ment} nouns in NOW. Abbreviations: Freq = Frequency.} 
	\label{tab:Frequencies}
	\small
		\begin{tabular}{lrlrlr}
			\lsptoprule
COS       & Freq & Psych     & Freq & Lexicalized & Freq  \\
			\midrule
bedragglement & 0         & affrightment   & 0         & payment          & 99,880     \\
decenterment  & 0         & endullment     & 0         & commitment       & 1,007,694  \\
progressment  & 0         & perturbment    & 0         & entertainment    & 1,095,326  \\
discolorment  & 1         & soothment      & 0         & equipment        & 1,100,428  \\
befoulment    & 3         & staggerment    & 0         & movement         & 1,383,883  \\
debauchment   & 4         & bumfuzzlement  & 1         & agreement        & 1,789,040  \\
worsenment    & 4         & nonplusment    & 1         & treatment        & 1,806,375  \\
embetterment  & 7         & reassurement   & 2         & environment      & 2,009,483  \\
besmirchment  & 17        & upsetment      & 2         & advertisement    & 2,668,436  \\
increasement  & 18        & convincement   & 3         & investment       & 3,059,964  \\
congealment   & 29        & enrapturement  & 3         & management       & 3,489,531  \\
unfoldment    & 42        & annoyment      & 6         & statement        & 4,278,707  \\
dispersement  & 65        & dumbfoundment  & 6         & department       & 4,283,807  \\
embrittlement & 260       & worriment      & 14        & development      & 5,357,627  \\
abridgement   & 476       & abashment      & 20        & government       & 14,758,679 \\
diminishment  & 1,209     & approvement    & 22        &                  &            \\
disbandment   & 4,194     & musement       & 27        &                  &            \\
upliftment    & 9,289     & confoundment   & 30        &                  &            \\
&           & enragement     & 37        &                  &            \\
&           & disheartenment & 39        &                  &            \\
&           & bemusement     & 3,272     &                  &            \\
&           & upliftment     & 9,289     &                  &   \\
			\lspbottomrule
		\end{tabular}
\end{table}

Having said that, finding attestations for the more central readings of a noun (for instance \textsc{result-state} for psych nouns) proved to be surprisingly straightforward in most cases. In all, considering that finding attestations for neologisms is no trivial task, I regard the number of gaps in my data as astonishingly low.

\section{Ambiguity}
\label{sec:obs-amb}

In this section, I will assess and discuss the extent of ambiguity in the attested combinations of reading and nominalization.

Table \ref{tab:AmbiguityCOS} shows the combinations of nominalizations and readings for COS nouns (visualized in Figures \ref{fig:AmbiguityByReadingCOS} and \ref{fig:AmbiguityByNomCOS}, p. \pageref{fig:AmbiguityByReadingCOS}). A combination is marked as ``amb.'' if only ambiguous attestations are in the data set. For example, I have only found ambiguous attestations for \textit{abridgement} in an \textsc{instrument} reading. As soon as a combination is attested unambiguously at least once, this is indicated by ``unamb.'' (e.g. \textit{abridgement} in a transpositional reading). Empty cells represent gaps in the data (see \sectref{sec:obs-gaps}). The numbers in the last two columns indicate how many of the possible readings are only attested ambiguously for a given nominalization, in total numbers and in percentages. Likewise, the numbers in the last two rows indicate how many of the nominalizations are only attested ambiguously in a given reading. The total number and the overall percentage of ambiguous combinations is given in the bottom right corner. 

\begin{sloppypar}
The percentage of nominalizations for which only ambiguous attestations have been found is rather high for most readings. Only \textsc{implicit product} has a perfect score, followed by \textsc{result} with 11\% ambiguously attested nouns, and \textsc{transposition} with 18\%. The highest percentage of only ambiguously attested nouns can be seen for \textsc{instrument} (85\%), \textsc{causing event} (69\%) and \textsc{causer} (64\%). In total, 44\% of combinations are only ambiguously attested.
\end{sloppypar}

\begin{sidewaystable}[ph!]
\caption[Ambiguous and unambiguous readings of COS nouns]{Ambiguous and unambiguous readings of COS nouns} 
\label{tab:AmbiguityCOS}
%\resizebox{!}{\textwidth}{ 
%\rotatebox{90}{
\begin{footnotesize}
\begin{tabular}{llllllllllrr}
\lsptoprule
Nominalization          & Trans   & Instr   & Causer  & Causing-evt & C-o-s   & Res-st  & Pat & Impl-prod & Res & \multicolumn{1}{l}{Amb.}  & \multicolumn{1}{l}{Amb.} \\
&&&&&&&&&&\multicolumn{1}{l}{total}&\multicolumn{1}{l}{per cent}\\
\midrule
 abridgement & unamb. & \textbf{amb.} & \textbf{amb.} & \textbf{amb.} &  & \textbf{amb.} &  &  & unamb. & 4 & 67\\
bedragglement & unamb. & \textbf{amb.} & \textbf{amb.} & unamb. & \textbf{amb.} & unamb. & unamb. &  &  & 3 & 43\\
 befoulment & unamb. & \textbf{amb.} & unamb. & \textbf{amb.} &  & unamb. & \textbf{amb.} &  &  & 3 & 50\\
besmirchment & unamb. & \textbf{amb.} & \textbf{amb.} & \textbf{amb.} & unamb. & unamb. &  &  & unamb. & 3 & 43\\
 congealment & unamb. & unamb. &  &  & unamb. & unamb. & unamb. &  &  & 0 & 0\\
debauchment & unamb. & \textbf{amb.} & \textbf{amb.} & unamb. & unamb. & unamb. & \textbf{amb.} &  &  & 3 & 43\\
 decenterment & unamb. &  &  &  & unamb. & unamb. &  &  & unamb. & 0 & 0 \\
diminishment & \textbf{amb.} &  & unamb. & unamb. & unamb. & \textbf{amb.} &  &  & unamb. & 2 & 33\\
 disbandment & unamb. & \textbf{amb.} & \textbf{amb.} & \textbf{amb.} & \textbf{amb.} & unamb. &  &  &  & 4 & 67\\
discolorment &  &  & unamb. &  & \textbf{amb.} & unamb. &  & unamb. &  & 1 & 25\\
 dispersement & \textbf{amb.} & unamb. &  &  & unamb. & unamb. &  &  & unamb. & 1 & 20\\
embetterment & unamb. & \textbf{amb.} & unamb. & unamb. & unamb. & \textbf{amb.} &  &  & \textbf{amb.} & 3 & 43\\
 embrittlement & unamb. & \textbf{amb.} & \textbf{amb.} & \textbf{amb.} & unamb. & \textbf{amb.} &  & unamb. &  & 4 & 57\\
increasement & unamb. & \textbf{amb.} & \textbf{amb.} & \textbf{amb.} & unamb. & \textbf{amb.} &  &  & unamb. & 4 & 57\\
 progressment & unamb. & \textbf{amb.} &  & \textbf{amb.} & unamb. & \textbf{amb.} &  &  & unamb. & 3 & 50\\
unfoldment & unamb. & \textbf{amb.} & \textbf{amb.} & \textbf{amb.} & unamb. & unamb. & \textbf{amb.} &  &  & 4 & 57\\
 upliftment & \textbf{amb.} &  & unamb. & \textbf{amb.} & unamb. & \textbf{amb.} &  &  &  & 3 & 60\\
worsenment & unamb. &  & \textbf{amb.} &  & unamb. & \textbf{amb.} &  &  & unamb. & 2 & 40\\
\midrule
Amb. total  & \multicolumn{1}{r}{3} & \multicolumn{1}{r}{11} & \multicolumn{1}{r}{9} & \multicolumn{1}{r}{9} & \multicolumn{1}{r}{3} & \multicolumn{1}{r}{8} & \multicolumn{1}{r}{3} & \multicolumn{1}{r}{0} & \multicolumn{1}{r}{1}& \multicolumn{1}{r}{\textbf{47}} & \textendash{} \\
Amb. per cent  & \multicolumn{1}{r}{18} & \multicolumn{1}{r}{85} & \multicolumn{1}{r}{64} & \multicolumn{1}{r}{69} & \multicolumn{1}{r}{19} & \multicolumn{1}{r}{44} & \multicolumn{1}{r}{60} & \multicolumn{1}{r}{0} & \multicolumn{1}{r}{11}& \textendash{} & \multicolumn{1}{r}{\textbf{44}} \\
\lspbottomrule
\end{tabular}
\end{footnotesize}
%}
%}
\end{sidewaystable}

\begin{figure}[p]
\includegraphics[width=0.89\linewidth]{figures/AmbCosReading.pdf}
\caption{Ratios of ambiguously and unambiguously attested COS readings}
\label{fig:AmbiguityByReadingCOS}
\end{figure}

\begin{figure}[p]
\includegraphics[width=0.89\linewidth]{figures/AmbCosNom.pdf}
\caption{Ratios of ambiguously and unambiguously attested COS nouns}
\label{fig:AmbiguityByNomCOS}
\end{figure}

These findings are in line with my observations from the previous section. Again, \textsc{originator} readings are the ones which stand out negatively. This may lead to the conclusion that not only are they harder to find attested, but if they are, their contexts do not disambiguate as well as for other readings. However, this is only true on the level of granularity I have chosen in this study, that is, distinguishing between \textsc{instrument, causer} and \textsc{causing-event} in the first place. In fact, the ambiguity is largely between these three categories, and not between \textsc{originator} and other categories. In other words, it is actually not that hard to disambiguate between \textsc{originator} and other readings, but the context does then often not allow distinctions on a more fine-grained level. Quantitatively speaking, only 18\% of COS nouns (a total of 4) have no unambiguous attestations for the concatenated \textsc{originator} category. 

Let me also add a note on the fact that \textsc{implicit product} has a perfect score. One might be tempted to attribute this to \textsc{implicit product} being a dominant reading. However, only two nouns in the data set have this reading in the first place, so that generalizations of that sort are hardly feasible.

The proportion of only ambiguously attested readings per nominalization is also rather high: Most have between 40\% and 60\% ambiguously attested readings; \textit{abridgement} and \textit{disbandment} have the highest percentage, with 67\%. On the other side of the spectrum, \textit{congealment} and \textit{decenterment} have perfect scores. All things considered, I do not think that this distribution is systematic. That is, it does not seem to be the case that properties of a nominalization (e.g. frequency) correlate with its level of ambiguity. 

For psych verbs, we see a similar picture, as is summarized in Table \ref{tab:AmbiguityPsych} as well as in Figures \ref{fig:AmbiguityByReadingPsych} and \ref{fig:AmbiguityByNomPsych} (p. \pageref{fig:AmbiguityByReadingPsych}). In the previous section, we saw that the psych subset of the data set has fewer gaps, and at 42\% the extent of ambiguity is also slightly lower than in the COS data.

\begin{sidewaystable}[ph!]
\caption[Ambiguous and unambiguous readings of psych nouns]{Ambiguous and unambiguous readings of psych nouns} 
\label{tab:AmbiguityPsych}
\begin{small}
\begin{tabular}{llllllrr}
\lsptoprule
Type              & Trans   & Stim    & Causing-evt & C-o-p-s & Res-st & \multicolumn{1}{l}{Ambiguous} & \multicolumn{1}{l}{Ambiguous}\\
&&&&&&\multicolumn{1}{l}{total}& \multicolumn{1}{l}{per cent}\\
\midrule
 abashment      & unamb. & unamb. & unamb. &        & unamb. & 0 & 0  \\
affrightment   & unamb. & \textbf{amb.}   & \textbf{amb.}   &        & unamb. & 2 & 50 \\
 annoyment      & unamb. & unamb. & unamb. &        & unamb. & 0 & 0  \\
approvement    & unamb. &        &        &        &        & 0 & 0  \\
 bemusement     & \textbf{amb.}   & unamb. & unamb. &        & unamb. & 1 & 25 \\
bumfuzzlement  & unamb. & \textbf{amb.}   & \textbf{amb.}   &        & unamb. & 2 & 50 \\
 confoundment   & \textbf{amb.}   & \textbf{amb.}   & \textbf{amb.}   &        & unamb. & 3 & 75 \\
convincement   & unamb. & unamb. & \textbf{amb.}   &        & unamb. & 1 & 25 \\
 disheartenment & \textbf{amb.}   & unamb. & unamb. &        & unamb. & 1 & 25 \\
dumbfoundment  & \textbf{amb.}   & \textbf{amb.}   & \textbf{amb.}   &        & unamb. & 3 & 75 \\
 endullment     & unamb. &        & \textbf{amb.}   & unamb. & \textbf{amb.}   & 2 & 50 \\
enragement     & unamb. & \textbf{amb.}   & \textbf{amb.}   & unamb. & unamb. & 2 & 40 \\
 enrapturement  & \textbf{amb.}   & \textbf{amb.}   & \textbf{amb.}   &        & unamb. & 3 & 75 \\
musement       & unamb. & unamb. &        &        & unamb. & 0 & 0  \\
 nonplusment    & \textbf{amb.}   & \textbf{amb.}   & unamb.   &        & unamb. & 2 & 50 \\
perturbment    & unamb. &        & \textbf{amb.}   &        & unamb. & 1 & 33 \\
 reassurement   & \textbf{amb.}   & unamb. &        &        & \textbf{amb.}   & 2 & 67 \\
soothment      & \textbf{amb.}   & unamb. &        & \textbf{amb.}   & unamb. & 2 & 50 \\
 staggerment    & \textbf{amb.}   & \textbf{amb.}   & \textbf{amb.}   &        & unamb. & 3 & 75 \\
upliftment     & unamb.   & unamb. & unamb. & unamb. & unamb. & 0 & 0 \\
 upsetment      & \textbf{amb.}   & \textbf{amb.}   & \textbf{amb.}   &        & unamb. & 3 & 75 \\
worriment      & unamb. & \textbf{amb.}   & \textbf{amb.}   &        & unamb. & 2 & 50 \\
\midrule
Ambiguous total & \multicolumn{1}{r}{10} & \multicolumn{1}{r}{10} & \multicolumn{1}{r}{13} & \multicolumn{1}{r}{1} & \multicolumn{1}{r}{2} & \multicolumn{1}{r}{\textbf{35}} & \textendash{} \\
Ambiguous per cent & \multicolumn{1}{r}{45} & \multicolumn{1}{r}{45} & \multicolumn{1}{r}{60} & \multicolumn{1}{r}{25} & \multicolumn{1}{r}{10} & \textendash{} & \multicolumn{1}{r}{\textbf{42}}\\
\lspbottomrule
\end{tabular}
\end{small}
\end{sidewaystable}

\begin{figure}[p]
\includegraphics[width=0.89\linewidth]{figures/AmbPsychReading.pdf}
\caption{Ratios of ambiguously and unambiguously attested psych readings}
\label{fig:AmbiguityByReadingPsych}
\end{figure}

\begin{figure}[p]
\includegraphics[width=0.89\linewidth]{figures/AmbPsychNom.pdf}
\caption{Ratios of ambiguously and unambiguously attested psych nouns}
\label{fig:AmbiguityByNomPsych}
\end{figure}

All psych readings have a portion of only ambiguously attested nominalizations. With only 10\%, \textsc{Result-state} scores best.
The readings for which the identification of unambiguous attestations was most difficult are \textsc{causing event} (65\% only ambiguously attested nominalizations), \textsc{transposition} (50\%) and \textsc{stimulus} (45\%). Again, these results confirm my previous observation that \textsc{originator} categories exhibit the most patchy data availability. With regard to transpositional readings, it is interesting that psych nouns have a much higher percentage of nominalizations with only ambiguous attestations, namely 50\%, compared to 18\% of COS nouns. A possible explanation is the factor of agentivity as discussed in \sectref{sec:psy-output-survey-transOE}. There, I described that 
unambiguously transpositional readings for psych nouns can be found much more easily in agentive contexts. Assuming that COS nouns generally exhibit a higher degree of agentivity, it is logical that they be found more easily in unambiguous transpositional readings as well. Of course, this suspicion would have to be verified empirically.

From a semasiological point of view, the psych data is more widely distributed than was the case for the COS data: Most nominalizations fall into the range between 25\% and 75\% ambiguously attested readings (for most COS nouns, the range is between 40\% and 60\%). There are four psych nominalizations with a perfect score (\textit{abashment, annoyment, approvement} and \textit{musement}).\footnote{It should be kept in mind that \textit{approvement} has only one attested reading to begin with.} \textit{Confoundment, dumbfoundment, enrapturement, nonplusment, staggerment} and \textit{upsetment} are most ambiguous, with 75\% only ambiguously attested readings each.

The conclusion that we can draw from this overview is that, on the whole, my data tends to be quite ambiguous. Both from an onomasiological and from a semasiological point of view, there are comparatively few unambiguous attestations. 
On a more general scale, this shows us that neologism data is very often ambiguous. During data collection, it was a decided goal to identify as many unambiguous attestations as possible, and still ambiguity is widespread in my data.
Of course, this observation is very subjective. In future research, it would be interesting to compare the ambiguity of neologisms with that of lexicalized nominalizations in a sample of random attestations, or to investigate whether there are systematic patterns in the ambiguities I found. 

