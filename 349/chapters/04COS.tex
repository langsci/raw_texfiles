\chapter{Change-of-state verb bases}\label{ch:cos}

In this chapter, I will examine nominalizations derived from members of the verb class most commonly known as \HighlightText{change-of-state verbs}, henceforth abbreviated as \HighlightText{COS verbs}. 
The chapter is divided into three parts: First, I will focus on verbal semantics, proposing verb frames based on the pertinent literature (\sectref{sec:cos-input}). 
Second, I will analyze the \ment{} derivatives, which includes their frame-semantic formalization (\sectref{sec:cos-output}). 
Finally, I will consolidate my results by proposing an inheritance hierarchy for \textit{-ment} on COS verb bases (\sectref{sec:cos-summary}).

The COS nouns analyzed in this chapter are \textit{abridgement, bedragglement, befoulment, besmirchment, congealment, debauchment, decenterment, diminishment, discolorment, disbandment, dispersement, embetterment, embrittlement, increasement, progressment, unfoldment, upliftment,} and \textit{worsenment}.  

\section{The semantics of COS verbs}
\label{sec:cos-input}

In this section, I will first give a synopsis of pertinent literature, limited to issues which are relevant for the frame formalization (\sectref{sec:cos-prev}). 
Then, I will discuss the pertinent subclasses of COS verbs in \citet{Levin.1993} and VerbNet in more detail (\sectref{sec:cos-input-VN}). 
These descriptions form the basis of a preliminary frame-semantic formalization of the base verbs (\sectref{sec:cos-input-frames}). Frame-theoretical notions which go beyond the basic toolkit described in \sectref{sec:fr-dus} will be introduced en route.   

\subsection{Previous literature} 
\label{sec:cos-prev}

Analyses of COS verbs usually distinguish external from internal causation. In this section, I will first illustrate this distinction (\sectref{sec:cos-prev-extVSint}) and then address how it has been formalized (\sectref{sec:cos-prev-formal}).

\subsubsection{External vs. internal causation}
\label{sec:cos-prev-extVSint}

The class of COS verbs, as the name suggests, is comprised of verbs which express a change-of-state, often of a physical kind. COS verbs have been widely studied, especially with regard to the Causative/Inchoative Alternation (henceforth \textit{C/I Alternation}) and the question of how this alternation is related to external and internal causation. As I will elaborate below, there is some disagreement in the literature on how to conceptualize alternating verbs in terms of internal versus external causation. 
In this section I will discuss this issue since I hypothesize that the type of causation involved in the base verb may have an effect on possible nominalization readings (see \sectref{sec:cos-output-predictions} for details).

Examples (\ref{ex:caus/inch}) to (\ref{ex:*caus/inch2}) illustrate the three possible behaviors of verbs with regard to the C/I Alternation (see \citealt{Alexiadou.2015} for a recent treatment). 
First, \textit{dry} in (\ref{ex:caus/inch}) is an example of a verb which can undergo this alternation. 
Both variants of the alternation describe the same event (\textit{drying}). The difference is that the external cause, in this case an \textsc{agent} (\textit{James}), is only expressed in the transitive variant. Examples (\ref{ex:*caus/inch1}) and (\ref{ex:*caus/inch2}) illustrate verbs which do not participate in this alternation. For verbs like \textit{bloom}, the causative variant in (\ref{ex:*caus/inch1-a}) is ungrammatical, and verbs like \textit{repair} do not exhibit the inchoative variant (\ref{ex:*caus/inch2-b}).\footnote{Although \textit{bloom} is a standard example of purely inchoative verbs, some native speakers would accept causative constructions, for example in a poetic context.} I will henceforth refer to these three groups of verbs as \HighlightText{causative/inchoative (c/i) COS verbs, inchoative-only (i-only) COS verbs} and \HighlightText{causative-only (c-only) COS verbs}, respectively. In my data set there are only c/i COS verbs and c-only COS verbs.  

\begin{exe}
  \singlespacing
  \ex \label{ex:caus/inch} 
  \begin{xlist}
    \ex[]{\label{ex:caus/inch-a} James \textbf{dried} the clothes.} 
    \ex[]{The clothes \textbf{dried}.}
  \end{xlist}
  \ex \label{ex:*caus/inch1} 
  \begin{xlist}
    \ex[*]{\label{ex:*caus/inch1-a}The sun \textbf{bloomed} the bluebonnets yesterday.} 
    \ex[]{The bluebonnets \textbf{bloomed} yesterday.}
  \end{xlist}  
  \ex \label{ex:*caus/inch2} 
  \begin{xlist}
    \ex[]{\label{ex:*caus/inch2-a}Bill \textbf{repaired} the tractor.}
    \ex[*]{\label{ex:*caus/inch2-b}The tractor \textbf{repaired}.}
  \end{xlist}   
\end{exe}

\noindent This syntactic alternation is closely linked to the traditional semantic distinction between internal and external causation. The notion was first introduced by \citet{Smith.1970} with the concept of \HighlightText{control}, and further developed by \citet{Levin.1995} under the label of \HighlightText{internal} vs. \HighlightText{external causality}. Discussing c/i COS verbs, \citet{Smith.1970} states that ``external control of the change can be assumed by an agent'' (p. 101), which is reflected by the fact that they can occur in a transitive variant. At the same time, she assumes that the change-of-state happens relatively independently of an external agent since the intransitive variant is also possible for such verbs. For i-only COS verbs, the change-of-state is completely independent of any other event, and control cannot be handed over to another entity (p. 107). The change-of-state rather emerges from properties which are inherent in the verb's argument. Finally, in the case of c-only COS verbs, the change-of-state is completely dependent on the agent (p. 102).  

The distinction between internal and external causation is intuitive, and there is corpus and psycholinguistic data (production as well as perception) to back it up (\citealt{McKoon.2000}). However, it has been a matter of some debate which cognitive status the internally and externally caused variants have. COS verbs have been conceptualized in (at least) three different ways.
The first position is that they have two separate, unrelated lexical entries; one with internal and one with external causation. 
In this view, every possible argument structure of a verb would be represented by a distinct frame (as for example tested computationally by \citealt{Haugereid.2011}). Since the present study aims at a generalizing approach, it would not be feasible to create frames for every variant of every verb class in every possible alternation.
The second and third positions assume that the two templates are related. The difference between the two is which variant is given prominence.\footnote{See \citet{Levin.2011b, Levin.2011a} for a concise overview of both positions.}   
The second position is that c/i COS verbs are in principle causatives, lexicalizing a cause, and taking \textsc{causer} and \textsc{patient} arguments. Some verbs, like \textit{break}, can be used in an inchoative variant without expressing an external cause. Such uses are taken to be a special, elliptic case which is achieved by deleting an event, but our world knowledge tells us that an external cause must exist (see \citealt[93]{Levin.1995}).
In contrast, advocates of the third position (for instance \citealt{Smith.1970} as discussed above, and also \citealt{Pinker.1989}) regard c/i COS verbs as intransitives that can have a causative variant. In this view, the causative structure is derived from the inchoative one by adding an event.\footnote{An argument in favor of this position is that there are several languages in which derivation can create causative constructions. For instance, the Korean suffix \textit{-(h)ita} can causativize a verb (e.g. \textit{pota} `see' > \textit{po-ita} `make see;' see \citealt{Dixon.2000}).}

\subsubsection{{Formalization of COS verbs in the non-frame literature}}
\label{sec:cos-prev-formal}

Before turning to the frame formalization of COS verbs in \sectref{sec:cos-input-frames}, I will first present existing non-frame formalizations of both inchoative and causative COS verbs (and verb variants). \citeauthor{RappaportHovav.1998}'s (\citeyear{RappaportHovav.1998}) proposal will serve as a stand-in in order to communicate the general idea.

Causative verbs are typically regarded as complex events with two subevents, that is, a causing subevent and a caused subevent (see e.g. \citealt{Dowty.1979,Levin.1995,Pustejovsky.1991}). The causing subevent can be any kind of event, for instance a situation, an event, a process or an action (\citealt[84]{VanValinJr.1997}), but is most often an action (\citealt[104]{RappaportHovav.1998}).\footnote{\citet{RappaportHovav.1998} use the term \textit{activity} here, which is one of the Vendler/Dowty aktionsart classes of verbs. To avoid confusion, I use the roughly equivalent term \textit{action} as defined in \sectref{sec:meth-Nsem-semcat-event} throughout this book, except if specifically referring to aktionsart.} 
In COS verbs, the caused subevent is a change-of-state, as opposed to other changes such as a change of location or the emergence of an artifact, which would figure in the event structure of other complex verb classes. The inchoative variants, on the other hand, are thought to be best represented by a simple event structure consisting of only a change-of-state. 

These two types of event are formalized in the logical structures below. The structures in (\ref{form:cos-compl-a}) and (\ref{form:cos-simpl-a}) present templates, with the concrete examples of \textit{repair} (external causation), \textit{decay} (internal causation) and \textit{break} (external or internal causation) given under their respective template.
The structures in (\ref{form:cos-compl-a}), (\ref{form:cos-compl-c}) and (\ref{form:cos-simpl-a}) have been taken from \citet{RappaportHovav.1998}, the other three have been modeled analogously. In prose, the examples in (\ref{form:cos-compl}) should be read as `x acts, causing y to attain a (repaired/broken) state,' while those in (\ref{form:cos-simpl}) can be paraphrased as `x attains a (decayed/broken) state.'

\begin{small}
  \begin{exe}
    \ex \label{form:cos-compl} 
    \begin{xlist}
      \ex \label{form:cos-compl-a} [[ x ACT\textsubscript{$<\,$\textit{MANNER}$\,>$}] CAUSE [ BECOME [ y $<\,$\textit{STATE}$\,>$ ]]]  
      \ex \label{form:cos-compl-b} \textit{repair}: [[ x ACT\textsubscript{$<\,$MANNER$\,>$} ] CAUSE [ BECOME [ y \textit{$<\,$REPAIRED$\,>$} ]]]  
      \ex \label{form:cos-compl-c} \textit{break}\textsubscript{tr}: [[ x ACT\textsubscript{$<\,$MANNER$\,>$} ] CAUSE [ BECOME [ y \textit{$<\,$BROKEN$\,>$} ]]]
    \end{xlist} 
  \end{exe}
  \begin{exe} 
    \ex \label{form:cos-simpl} 
    \begin{xlist}
      \ex \label{form:cos-simpl-a} [ BECOME [ x $<\,$\textit{STATE}$\,>$ ]] 
      \ex \label{form:cos-simpl-b} \textit{decay}: [ BECOME [ x \textit{$<\,$DECAYED$\,>$} ]] 
      \ex \label{form:cos-simpl-c} \textit{break}\textsubscript{intr}: [ BECOME [ x \textit{$<\,$BROKEN$\,>$} ]]
    \end{xlist}
  \end{exe}
\end{small}

\noindent Two comments are in order: First, both \textit{repair} and \textit{break} are what \citet{RappaportHovav.1998} call \HighlightText{result verbs}, lexicalizing a result but not the manner in which this result is achieved. They model this with an underspecified MANNER constant.\footnote{An example of a manner verb is \textit{sweep}, where the manner is specified while the result is not lexicalized (the floor may still be dirty after sweeping).} 
Second, I am not aware of an approach which formalizes c/i COS verbs differently from purely causative or purely inchoative verbs. Thus, their transitive variants take template (\ref{form:cos-compl-a}), while their intransitive variants take template (\ref{form:cos-simpl-a}). This does not mean that authors like \citet[]{RappaportHovav.1998} assume a vast lexicon with multiple entries for all c/i COS verbs. Rather, they assume general rules which generate such entries in a principled way (p. 99).

With regard to these formalizations, one problem has been raised which does not only pertain to externally caused COS verbs, but to causative verbs in general: What is the nature of the first argument of the CAUSE operator, modeled by \citet[]{RappaportHovav.1998} as an action? Does it really have to be eventive, or can it also be a participant? As \citet[107]{VanValinJr.1997} observe, there is much linguistic and philosophical discussion about this issue. 
In my frame representations, I have chosen to restrict the attribute range of \textsc{cause} to eventive types, following such approaches as \citet[]{RappaportHovav.1998} and \citet[]{VanValinJr.1997}. My findings with regard to nominalization semantics (more precisely: possible referential shifts to the causing subevent) indicate that this analysis is indeed feasible (see \sectref{sec:cos-output-survey-caus}).

The notion of event structure goes hand in hand with that of participants.
Thus, the change-of-state has a \textsc{patient} participant, while the first subevent has participants in accordance with its event type \citep[835]{McKoon.2000}. For example, an agentive action has an \textsc{agent} participant. In addition, there may be optional participants, such as instruments. These are syntactically not necessary but can be expressed, for instance, in a prepositional phrase (e.g. \textit{with a hammer}).

\subsection{Levin (1993) and VerbNet} 
\label{sec:cos-input-VN}

In this section, I will give an account of COS verbs in \citet{Levin.1993} and VerbNet. After a few general remarks (\ref{sec:cos-input-VN-general}), the remainder of the section is structured by the three subclasses of COS verbs in my data (Sections \ref{sec:cos-VN-conly} to \ref{sec:cos-VN-c/irevers}), and concluded by a summary (\sectref{sec:cos-input-VN-sum}). 

\subsubsection{General remarks} 
\label{sec:cos-input-VN-general} 

\citeauthor{Levin.1993}'s (\citeyear{Levin.1993}) class of COS verbs is very heterogeneous: It includes scalar and non-scalar verbs, the change-of-state can be externally caused or inherent in the patient, it can be of a very general kind or unique to a specific kind of entity, and it can affect different kinds of attributes of the patient (such as its shape or its material integrity). 
This semantic diversity is reflected in the six subclasses of COS verbs (\citealt[240--248]{Levin.1993}). To get a general idea of this class, Table \ref{tab:cos} summarizes the semantics of the subclasses, while Table \ref{tab:cosnet} gives an overview of the respective semantic roles and restrictions in VerbNet. Note that the subclass of \textit{remedy verbs} has been added in VerbNet and does not feature in \citet{Levin.1993}.

\begin{table} []
    \caption[Types of COS verbs]{\label{tab:cos}Types of COS verbs according to \citet[240--248]{Levin.1993}} 
    \begin{tabular}{p{3.5cm}<{\raggedright} p{4.5cm}<{\raggedright} p{3cm}<{\raggedright}}
      \lsptoprule
      Verb class
      & Semantics
      & Examples \\ 
      \midrule
Break verbs
      & change in an entity's material integrity~~~
      & {break, chip, crack} \\
      \addlinespace
      Bend verbs
      & change in an entity's shape
      & {bend, crumple, fold} \\
      \addlinespace
      Cooking verbs
      & ways of food preparation
      & {cook, steam, barbecue} \\
      \addlinespace
      Other alternating verbs of change-of-state
      & externally caused change of an entity's (physical) state
      & {enlarge, blacken, acidify} \\
      \addlinespace
      Verbs of entity-specific change-of-state 
      & change in an entity's state which is particular, and often inherent, to this entity
      & {blossom, rust, swell} \\
      \addlinespace
      Verbs of calibratable changes-of-state
      & positive or negative change of an entity along a scale
      & {decline, plunge, rise} \\      
      \lspbottomrule
    \end{tabular}
\end{table}

\begin{table} []
    \caption[Semantic roles and selectional restrictions for subclasses of COS verbs in VerbNet]{\label{tab:cosnet} Semantic roles and selectional restrictions for subclasses of COS verbs in VerbNet. Abbreviations: 
      + = core,
      \textendash{} = non-core or not allowed,
      A = animate, 
      C = concrete, 
      IC = intentional control, 
      S = solid, 
      SC = scalar,
      sec. = secondary.} 
    \begin{tabular}{lcccccc}
      \lsptoprule
      & \begin{sideways} Agent \end{sideways}
      & \begin{sideways} Patient \end{sideways}
      & \begin{sideways} Instrument \end{sideways}
      & \begin{sideways} Sec. result\footnote{In VerbNet, this role is called \textsc{result}.} \end{sideways}
      & \begin{sideways} Attribute \end{sideways}
      & \begin{sideways} Extent \end{sideways}
      \\ 
      \midrule
      Break verbs 
            & IC
            & S
            & S
            & +
            & \textendash{}
            & \textendash{} \\ 
            
            Bend verbs
            & IC
            & S
            & S 
            & +
            & \textendash{}
            & \textendash{} \\
            
            Cooking verbs 
            & A
            & C
            & S
            & +
            & \textendash{}
            & \textendash{} \\
            
            Other alternating verbs of change-of-state 
            & IC
            & + 
            & +
            & +
            & \textendash{}
            & \textendash{} \\
            
            Verbs of entity-specific change-of-state
            & \textendash{}
            & C
            & \textendash{}
            & \textendash{}
            & \textendash{}
            & \textendash{} \\
            
            Calibratable verbs of change-of-state 
            & \textendash{}
            & + 
            & \textendash{}
            & \textendash{}
            & SC
            & + \\
            
            Remedy verbs
            & IC
            & + 
            & +
            & \textendash{}
            & \textendash{}
            & \textendash{} \\
      \lspbottomrule
    \end{tabular}
\end{table}

I will now present the relevant subclasses of COS verbs in more detail. This overview serves three purposes. Firstly, it covers a methodological aspect in justifying why the base verbs occurring in my data set have been assigned to their respective subclass. 
Secondly, it complements Tables \ref{tab:cos} and \ref{tab:cosnet} by reviewing relevant alternations as well as differences between the subclasses, and by illustrating the respective semantic roles with example sentences. I will also motivate where and why I am adjusting some of VerbNet's terminology and assumptions. 
Finally, the informal presentation of the subclasses' semantics will pave the way for the frame-semantic formalization. The relevant subclasses are remedy verbs, other alternating verbs of change-of-state, and bend verbs. In the following, I will substitute these unintuitive labels by more descriptive ones which reflect their members' semantics: \HighlightText{causative-only COS verbs, causative/inchoative general COS verbs,} and \HighlightText{causative/inchoative reversible COS verbs} (abbreviated as c-only COS verbs, c/i general COS verbs, and c/i reversible COS verbs, respectively). An overview of the label matching is given in Table \ref{tab:labels}.

\begin{table}
    \caption[Overview of the labels used for COS verb subclasses]{\label{tab:labels}Overview of the labels used for COS verb subclasses}
    \begin{tabular}{
        >{\RaggedRight\arraybackslash}p{0.30\textwidth}
        >{\RaggedRight\arraybackslash}p{0.30\textwidth}
        >{\RaggedRight\arraybackslash}p{0.30\textwidth}
      }
      \lsptoprule
      Levin (1993)/VerbNet & Adjusted label & Abbreviated label \\ 
      \midrule
      Remedy verbs & causative-only change-of-state verbs & c-only COS verbs \\
      Other alternating verbs of change-of-state & causative/inchoative general change-of-state verbs & c/i general COS verbs \\
      Bend verbs & causative/inchoative reversible change-of-state verbs & c/i reversible COS verbs \\
      \lspbottomrule
    \end{tabular} 
\end{table}

\subsubsection{{Causative-only COS verbs}}\label{sec:cos-VN-conly}\largerpage

As has been mentioned above, c-only COS verbs are a supplement of the VerbNet classification (there \textit{remedy verbs}). It is obvious that the class was added in order to accommodate COS verbs without an inchoative variant (see \sectref{sec:cos-prev}), which had not had a place in \citet{Levin.1993}. Although VerbNet does not provide spelled-out definitions of verb classes, the verbs listed in the c-only COS subclass (e.g. \textit{disinfect, disorganize, transplant}) are clearly externally caused changes-of-state.

I classified seven types in my data set as c-only COS verbs: \textit{abridge, bedraggle, befoul, besmirch, debauch, embetter} and \textit{uplift}.{\interfootnotelinepenalty=10000\footnote{\textit{Uplift} will later recur as a psych verb relating to stimulating somebody morally. As a COS verb it often refers to social or economic stimulation.}}

C-only COS verbs have three semantic roles: \textsc{agent [+int\_control], patient}, and \textsc{instrument}.
An important observation can be made here: The roles listed in VerbNet are core roles, which means that they are frequently found in the respective verbs' participant structure. In reality, however, there is a continuum between verbs which always require a certain role (expressed or implied), and those that do not necessarily do so. For instance, \textit{christianize} and \textit{mineralize} are both c-only COS verbs, but \textit{christianize} (presumably) always takes a volitional \textsc{agent} as an argument, while \textit{mineralize} can also take inanimate \textsc{causers} such as chemical elements. In other words, \textit{mineralize} does not actually require an \textsc{agent} with [+int\_control], while \textit{christianize} does.

\subsubsection{Causative/inchoative general COS verbs}
\label{sec:cos-VN-c/igeneral}

Most COS verbs in my data set are c/i general COS verbs ($n=9$): \textit{congeal, decenter, diminish, disband, discolor, disperse, increase}, \textit{progress} and \textit{worsen}.

Verbs in this subclass lexicalize ``externally caused change[s] of an entity's [...] state'' (\citealt[246]{Levin.1993}). The effected change is often, but not necessarily, of a physical kind (e.g. \textit{liquefy, freeze}; \textit{hasten, deteriorate}). Many c/i general COS verbs are de-adjectival (e.g. \textit{americanize, blacken}), and all members prominently participate both in the C/I Alternation and in the Instrument Subject Alternation, given in (\ref{ex:ISA}) (from \citealt[245]{Levin.1993}). 

\begin{exe}
  \ex \label{ex:ISA}
  \ea Bill \textbf{dried} the clothes with a hairdryer. 
  \ex The hairdryer \textbf{dried} the clothes.
  \z
\end{exe}

\noindent As regards semantic roles, c/i general COS verbs are almost identical to c-only COS verbs. All of them occur with \textsc{agent [+int\_control], patient,} and \textsc{instrument} roles, and a subgroup furthermore exhibits the semantic role of \textsc{secondary result}. This participant is of a different status than the other three since it only surfaces syntactically as a secondary predicate in this verb class.\footnote{All semantic roles in VerbNet are called \textit{participants}, no matter their syntactic realization.} For illustration, consider example (\ref{ex:result}). Since the verb \textit{smooth} is a result verb (see \sectref{sec:cos-prev-formal}), it lexicalizes a result-state: After the event, the extensions will be smooth (or at least smoother than before). It has been observed that, if a resultative construction is added to such a verb, the lexicalized result is further specified. Thus, the sheets will not only be smooth, but, more precisely, flat (see \citealt[50]{Levin.1995}; \citealt[7]{Levin.2013}). 

\begin{exe}
  \ex \label{ex:result} Diane applies half a head, which is 12 wefts taped around the sides and back of my head [...]. That night in bed I have to keep \textbf{smoothing} them \textbf{flat} so they don't pull. {\small(\acs{NOW} MAG VogueAustralia 2013)}
\end{exe}

\noindent The difference between lexicalized and secondary results is reflected in the VerbNet frames as follows (my emphasis, \textsc{patient} is abbreviated as \textsc{pat}):\footnote{Compare \citet{Jackendoff.1990}, who represents the fact that a participant is completely incorporated into the verb meaning by leaving it unindexed in the verb's LCS (e.g. p. 164).} 

\begin{exe}
  \ex \label{ex:VNdry}
  \label{ex:VNdry1} Example: The clothes \textbf{dried}. \\ Syntax: \textsc{Pat V} \\ Semantics: \textsc{state(\textbf{result}(E), Endstate, Pat)}
  \ex \label{ex:VNdry2} Example: The clothes \textbf{dried} \ExHighlight{wrinkled}. \\ Syntax: \textsc{Pat} V \textsc{\ExHighlight{Result}} \\ Semantics: \textsc{state(\textbf{result}(E), Endstate, Pat) Pred(\ExHighlight{result}(E), Pat)}
\end{exe}

\noindent In (\ref{ex:VNdry1}), there is only a \textit{dry}-state, which is lexicalized in the verb and thus represented on the semantic level of the representation. 
In (\ref{ex:VNdry2}), there is an additional \textit{wrinkled}-state, which is represented on both the syntactic and the semantic level.
In order to avoid confusion, I will call only primary results \textsc{result}, while the label for secondary results will correspondingly be \textsc{secondary result}.

In the c/i general COS subset of my data, only \textit{congeal} and \textit{discolor} seem to allow resultative secondary constructions. This can be concluded from corpus data as well as native speaker judgments.\footnote{An informant pointed out that \textit{increase} can be found in the result-like construction \textit{increase abundant}. However, the grammatical status of \textit{abundant} is unclear, and other informants have rated the construction as archaic.} 

\subsubsection{{Causative/inchoative reversible COS verbs}}
\label{sec:cos-VN-c/irevers}

The smallest subclass of COS verbs in my data set is that of c/i reversible COS verbs ($n=2$). It is represented by \textit{embrittle} and \textit{unfold}.

C/i reversible COS verbs refer to reversible changes in the shape of an entity (\citealt[243]{Levin.1993}). This seems to include surface structure, since verbs like \textit{wrinkle} and \textit{crinkle} can also be found in this class.
They are semantically similar to c/i general COS verbs in that both subclasses share the same participant structure, differing only in the requirement [+solid] on the \textsc{patient} and \textsc{instrument} roles of c/i reversible COS verbs. 
Furthermore, both subclasses participate in the C/I Alternation, the Middle Alternation, and the Instrument Subject Alternation. What distinguishes them syntactically is that c/i reversible COS verbs are found in various additional resultative constructions which are not possible with c/i general COS verbs, as can be seen for example in (\ref{ex:bendres}). 

\begin{exe}
  \ex \label{ex:bendres} 
  \ea[]{Tony \textbf{folded} the flaps open with his feet.} 
  \ex[*]{Bill \textbf{dried} the clothes wrinkled with a hairdryer.}
  \z
\end{exe}

\noindent Assigning verbs to this class is problematic since the subclasses of c/i reversible COS verbs and break-COS verbs are syntactically identical and semantically very similar: Both participate in the same alternations given in \citet{Levin.1993} and VerbNet, and their semantics differs only in the aspect of reversibility, with break-COS verbs denoting irreversible changes-of-state (p. 242). 
Especially for \textit{embrittle}, assignment to this subclass should be taken with a grain of salt: The reversibility of brittleness depends on the patient, so that \textit{embrittle} could be considered both a break-COS verb and a c/i reversible COS verb. It is in fact often a matter of personal opinion whether a given change-of-state is reversible or not, and thus whether a verb is a c/i reversible COS verb or a break-COS verb.

\subsubsection{{Summary}}
\label{sec:cos-input-VN-sum}

The 18 base verbs in the COS data set can be assigned to three subclasses. These differ but also partly overlap with regard to their semantics, syntactic behavior (i.e. alternations), and participants. The similarities and differences are summarized in Table \ref{tab:cos-overview}. 

\begin{table}
  \centering 
    \caption[Properties of the COS base verbs in my data set]{\label{tab:cos-overview}Properties of the COS base verbs in my data set. Abbreviations: alt. = alternation, ext. = external, int. = internal, int\_control = intentional control, sec. = secondary.}
    \begin{tabular}{lllll}
      \lsptoprule
      && C-only COS & C/i general COS & C/i reversible COS \\ 
      && ($n=7$) & ($n=9$) & ($n=2$)\\
      \midrule
       \multicolumn{2}{l}{Event properties}  &&& \\
       & C/I Alt. & causative & alternating & alternating \\ 
       & Causation & ext. & ext. or int. & ext. or int. \\ 
       &  Type of change & state & state & physical form \\
      \multicolumn{2}{l}{Participants}&&&\\
       & Agent &int\_control&int\_control&int\_control\\
      & Patient &+&+&solid\\
       & Instrument &+&+&solid\\
      & Sec. result &−&−/+*&+\\
 \multicolumn{2}{l}{Verbs}  & {abridge}  & {congeal}*   & {embrittle}   \\
      && {bedraggle}   & {decenter}   & {unfold}  \\
      && {befoul}    & {diminish}   & {}   \\
      && {besmirch}    & {disband}  & {}   \\
      && {debauch}   & {discolor}*  & {}   \\
      && {embetter}    & {disperse}   & {}  \\
      && {uplift}    & {increase}   & {}   \\
      && {}    & {progress}   & {}    \\
      && {}    & {worsen}   & {}   \\      
      \lspbottomrule 
    \end{tabular} 
\end{table}


Note that a ``−'' in Table \ref{tab:cos-overview} (and all similar tables) does not necessarily imply that a given participant cannot occur in the context of the verbs in this subclass. This is because the participants listed in VerbNet are those \textit{typically} occurring with a given verb, while others may also be possible. For instance, some c/i general COS verbs do allow for constructions with \textsc{extent}, italicized in (\ref{ex:ext}) below. In the next section, I will propose how to model these properties in frames. 

\begin{exe}
  \ex \label{ex:ext} NASS itself reports that in 2005--2006, students' average reading scores \textbf{progressed} \ExHighlight{by 1.5 grade levels} {\small(\acs{COCA} ACAD EducationWeek 2007)}
\end{exe}

\subsection{Frame decomposition of COS verbs}
\label{sec:cos-input-frames}

In the previous section, we saw that the semantics of COS verbs are highly intricate. 
In this section, I will model these intricacies in frames. 
The section is organized by type of causation: I will start with those verbs which are exclusively externally caused (c-only COS verbs, \sectref{sec:cos-input-frames-conly}), and then extend the formalization to those verbs which can be either externally or internally caused (c/i general COS verbs, \sectref{sec:cos-input-frames-c/igeneral}, and c/i reversible COS verbs, \sectref{sec:cos-input-frames-c/irevers}). I will later refer back to the frames developed in this section as \HighlightText{VerbNet-based frames} since they represent subclasses of COS verbs in VerbNet,  and the participants correspond to VerbNet semantic roles (see \sectref{sec:cos-input-VN}).\footnote{Semantic roles are correlate attributes for events (\citealt[4]{Loebner.toappear}).}
The event decomposition is geared to the event frames as introduced in \sectref{sec:fr-dus}; further frame-related literature will be introduced as needed. Further below, the resulting formalizations will then be put to the test with my data set (see \sectref{sec:cos-output}). 

\subsubsection{Causative-only COS verbs} 
\label{sec:cos-input-frames-conly}

C-only COS verbs are modeled in causation frames, which are frequently discussed in the frame literature. We have already seen such a frame for the causative variant of the c/i verb \textit{break} in \sectref{sec:fr-dus}. Verbs which only have a causative variant are modeled with the same event structure template. For example, \citet[50]{Kallmeyer.2012} model the caused motion verb \textit{throw} as a complex causation-event with a \textsc{cause} and an \textsc{effect} subevent. 
I can therefore use the event structure template for causative (variants of) verbs provided in the literature, and adjust it to match the semantics of c-only COS verbs according to VerbNet.
Figure \ref{fig:remframe} depicts a first attempt at modeling the class of c-only COS verbs accordingly. 
The frame is a generalized lexical frame as motivated in \sectref{sec:fr-der-ref-lex}. 

\begin{figure}
    \begin{avm}
      \avml
      \@0 
      \[
      \avmspan{\textit{change-of-state causation}}\cr
      \textsc{agent} & \@1 \cr
      \textsc{patient} & \@2 \cr
      \textsc{instrument} & \@3 \cr
      \textsc{cause} & \@4 
      \[ 
      \textit{agent-action}\cr 
      \textsc{agent} & \@1 \cr
      \textsc{patient} & \@2 \cr
      \textsc{instrument} & \@3 \cr
      \] \cr
      \textsc{effect} & \@5 
      \[ 
      \textit{change-of-state} \cr
      \textsc{patient} & \@2 \cr
      \textsc{result-state} & \@6 
      \[ 
      \textit{state} \cr
      \textsc{patient} & \@2\cr
      \] \cr
      \] \cr
      \] \cr
      {\textsc{ref} $=$ \{\@0\} }
      \avmr
    \end{avm}
    \caption{Frame for c-only COS verbs, e.g. \textit{bedraggle}\label{fig:remframe}}
\end{figure}

At the top of the AVM is the frame type, \textit{change-of-state causation}. As defined in the type signature, this event type has a caused subevent \textit{change-of-state}.\largerpage
Next come the participants of the complex event. These three attributes are the semantic roles given for c-only COS verbs in VerbNet. They are followed by the event decomposition. The first subevent, \textsc{cause}, is an agentive action, which accounts for the fact that VerbNet only lists an \textsc{agent} as a causing entity (as opposed to other subtypes of \textsc{actor}).  
The first subevent also has an \textsc{instrument} and a \textsc{patient} participant.{\interfootnotelinepenalty=10000\footnote{For my purposes, it is irrelevant whether or not a participant is obligatory or just frequent; I am only interested in its status as a core role of a given verb class. Optionality of a participant in a given event type is fixed in the type signature (see Figure \ref{fig:signatureevent}).}}
The second subevent, \textsc{effect}, is a change-of-state which at some point reaches a result-state, the only participant of which is the \textsc{patient} of the complex event.\footnote{In previous publications, I included an \textsc{initial state} in the second subevent. Now, I think that the initial state is not in fact part of the base verb semantics (i.e., lexical knowledge), but is only presupposed (i.e., world knowledge). This is in line with other frame-semantic approaches to COS verbs, such as \citet{Kallmeyer.2013} or \citet{Osswald.2014}, and also with other formal approaches such as \citeauthor{RappaportHovav.1998}'s (\citeyear{RappaportHovav.1998}) modeling of result verbs.}
The subevents are unspecified with regard to direct versus indirect causation as well as with regard to punctual versus durative change. 

Below the semantic decomposition, the referent of the frame is specified by a mathematical set (see \sectref{sec:fr-der-ref-der}). Strictly speaking, reference in an AVM would standardly be assumed to be on \avmbox{0} anyway, but it is included here for the sake of explicitness and comparability to later frames.

\subsubsection{Causative/inchoative general COS verbs}\label{sec:cos-input-frames-c/igeneral}\largerpage
The first subclass of COS verbs which allows for the C/I Alternation is that of c/i general COS verbs. 
In the frame literature, verbs participating in this alternation are represented by separate frames, a complex event frame for the causative variant and a simple event frame for the inchoative variant (e.g. \citealt{Osswald.2014,Seyffarth.2018}). These are related to each other in that the inchoative frame is embedded in the causative frame.\footnote{A joint, underspecified frame could cover the shared semantics of both variants (\citealt{Seyffarth.2018}). This frame would, however, not model event structure, which is required for this study.}
The causative frame of a c/i general COS verb does not, however, differ from the frame of a c-only COS verb. 

I propose two options for representing c/i general COS verbs, which can then be tested with my nominalization data. The two options reflect two positions in the literature (see \sectref{sec:cos-prev}), namely that the inchoation can either depend on a \textsc{cause} or happen independently of it. In this line of thinking, the approach assuming embedded frames would reflect that inchoation can indeed happen independently since the inchoation frame can exist on its own. The corresponding frame for c/i general COS verbs is given in Figure \ref{fig:cosframeembed}. 

\begin{figure}
% \small
% 	    \resizebox{ \linewidth}{!}{ 
% 	\begin{minipage}[]{0.55\textwidth}
% 		\centering \singlespacing
			\begin{avm}
				\avml
				      \@0 
				      \[
				      \avmspan{\textit{change-of-state causation}}\cr
				      \textsc{agent} & \@1 \cr
				      \textsc{pat} & \@2 \cr
				      \textsc{instr} & \@3 \cr
				      \textsc{cause} & \@4 
				      \[ 
				      \textit{agent-action}\cr 
				      \textsc{agent} & \@1 \cr
				      \textsc{pat} & \@2 \cr
				      \textsc{instr} & \@3 \cr
				      \] \cr
				      \textsc{effect} & \@5 
				      \[ 
				      \avmspan{\textit{change-of-state}} \cr
				      \textsc{pat} & \@2 \cr
				      \textsc{res-st} & \@6 
				      \[ 
				      \textit{state} \cr
				      \textsc{pat} & \@2\cr
				      \] \cr
				      \] \cr
				      \] \cr
				      {\textsc{ref} $=$ \{\@0\} }
				      \avmr
% 			\end{avm}
% 	\end{minipage}
% 	\begin{minipage}[]{0.05\textwidth}
  $\quad \longleftrightarrow \quad$ 
% 	\end{minipage}
% 	\begin{minipage}[]{0.4\textwidth}
% 		\centering \singlespacing
% 			\begin{avm}
				\avml
				\@5
				      \[ \avmspan{\textit{change-of-state}} \cr
				      \textsc{pat} & \@2 \cr
				      \textsc{res-st} & \@6  \[ \textit{state}\cr
				      \textsc{pat} & \@2\cr
				      \]
				      \] \cr
				      {\textsc{ref} $=$ \{\@5\} }
				      \avmr 
			\end{avm}
% 	\end{minipage}
	\caption[Frame for c/i general COS verbs (embedded subevent)]{Frame for c/i general COS verbs (e.g. \textit{congeal}; embedded subevent), inspired by \citet{Osswald.2014}. Abbreviations: pat = patient, instr = instrument, res-st = result-state.} \label{fig:cosframeembed}
% }
\end{figure}

This figure shows a complex event frame to the left, and an independently existing change-of-state frame to the right. With the shorthand of the two-headed arrow between them, I intend to express two things. First, it indicates that these two frames are connected. This connection is also expressed by co-indexation of the \textit{change-of-state} node (\avmbox{5}), which shows that the frame to the right is embedded in the frame to the left. However, since frame indices are usually not shared across frames, the two-headed arrow makes the connection explicit. 
Second, I am not making a statement as to which variant is primary, as, for example, a one-headed arrow would imply. This question cannot be answered using my data, so I will not join in on its discussion.   

The second approach builds on the idea to indicate reference as a mathematical set within the frame. 
As can be seen in Figure \ref{fig:othercosframecompl}, the possible referents can be identified as either the complex event (\avmbox{0}), or the change-of-state (\avmbox{5}). This represents the position in the literature that the inchoation event cannot happen independently; the \textit{change-of-state} \avmbox{5} is always linked to the causing \textit{agent-action} \avmbox{4} by way of the complex event \avmbox{0}.\largerpage[2]

\begin{figure}[htb] 
  \centering \singlespacing
    \begin{avm}
      \avml
      \@0 
      \[
      \avmspan{\textit{change-of-state causation}}\cr
      \textsc{agent} & \@1 \cr
      \textsc{patient} & \@2 \cr
      \textsc{instrument} & \@3 \cr
      \textsc{cause} & \@4 
      \[ 
      \textit{agent-action}\cr 
      \textsc{agent} & \@1 \cr
      \textsc{patient} & \@2 \cr
      \textsc{instrument} & \@3 
      \] \cr
      \textsc{effect} & \@5 
      \[ 
      \textit{change-of-state} \cr
      \textsc{patient} & \@2 \cr
      \textsc{result-state} & \@6 
      \[ 
      \textit{state} \cr
      \textsc{patient} & \@2\cr
      \] \cr
      \] \cr
      \] \cr
      {\textsc{ref} $=$ \{\@0, \@5\} }
      \avmr
    \end{avm}
  \caption[Frame for c/i general COS verbs (single frame)]{Frame for c/i general COS verbs (e.g. \textit{congeal}; single frame)}
  \label{fig:othercosframecompl}
\end{figure} 

Note that the frames in Figures \ref{fig:cosframeembed} and \ref{fig:othercosframecompl} do not reflect that some c/i general COS verbs allow a secondary result of the type \textit{The clothes dried wrinkled} in their argument structure. In VerbNet, this is reflected by the semantic role \textsc{result}. 
Modeling a secondary result-state in frames is rather complex since this construction is at the syntax-semantics interface.
The mechanism which is best suited to model contexts like this is that of frame modification. For example, \citet{Balogh.inpress} use a combination of frames and Role and Reference Grammar (RRG; \citealt{VanValin.2005}) to model the interaction between semantics and syntax in the resultative modification of verbal particles in Hungarian.\footnote{See also \citet{Osswald.toappear} for a frame-syntactic approach to English adjectival resultative constructions (\textit{kick open, wipe clean}), and \citet{Petersen.2014} for a frame model of depictive secondary predicates.}
I will not go into the details of their approach here, since secondary result predicates are not imminently relevant for the semantic analysis of derivation: Being modifiers which act on the frame of certain base verbs, their semantics cannot be accessed by an affix. 
Further research on the syntax-semantics interface will be necessary to determine why only \textit{congeal} and \textit{discolor} allow for secondary resultative modification while the other verbs in the data set do not.\footnote{I have observed that the nominalizations can be found in the same constructions as their base verbs, semantically speaking (e.g. \textit{yellow discolorment, solid congealment, white congealment}). It can therefore be assumed that the approach put forward by \citet{Balogh.inpress} for verbs can rather easily be adapted to model the modification of nominalizations.} 

\subsubsection{Causative/inchoative reversible COS verbs}\label{sec:cos-input-frames-c/irevers}

C/i reversible COS verbs have the same participant structure as those c/i general COS verbs with a secondary result-state, and they participate in the same alternations, including the C/I Alternation. Therefore, the two frame options for c/i general COS verbs also hold for c/i reversible COS verbs, with a few specifications (see Figures \ref{fig:bendframe-embedded} and \ref{fig:bendframe-single}).{\interfootnotelinepenalty=10000\footnote{C/i reversible COS verbs participate in a larger number of resultative constructions compared to c/i general COS verbs. These, however, are syntactic details which do not concern us here, as I have justified in the previous section.}} 

\begin{figure}
% 	\begin{minipage}{0.575\textwidth}\setstretch{0.95}
  \resizebox{\textwidth}{!}{\begin{avm}
     \avml
      \@0 
      \[
      \avmspan{\textit{change-of-physical-form causation}}\cr
      \textsc{agent} & \@1 \cr
      \textsc{pat} & \@2 
      \[
      \textit{entity} \cr
      {\textsc{physical-state}} & {\textit{solid}}
      \]
      \cr
      \textsc{instr} & \@3 
      \[
      \textit{entity} \cr
      {\textsc{physical-state}} & {\textit{solid}}
      \]
      \cr
      \textsc{cause} & \@4 
      \[ 
      \textit{agentive action}\cr 
      \textsc{agent} & \@1 \cr
      \textsc{pat} & \@2 \cr
      \textsc{instr} & \@3 \cr
      \] \cr
      \textsc{effect} & \@5 
      \[ 
      \avmspan{\textit{change-of-physical-form}} \cr
      \textsc{pat} & \@2 \cr
      \textsc{res-st} & \@6 
      \[ 
      \avmspan{\textit{having-form}} \cr
      \textsc{pat} & \@2\cr
      \] \cr
      \] \cr
      \] \cr
      {\textsc{ref} $=$ \{\@0\} }
      \avmr
% 			\end{avm}
% 	\end{minipage}\begin{minipage}[]{0.425\textwidth}\setstretch{0.95}\begin{avm}
	\avml  
	$\quad \longleftrightarrow \quad$ \@5
      \[ \avmspan{\textit{change-of-physical-form}} \cr
      \textsc{pat} & \@2 \cr
      \textsc{res-st} & \@6  \[ \avmspan{\textit{having-form}}\cr
      \textsc{pat} & \@2\cr
      \]
      \] \cr
      \hspace{1.5cm}{\textsc{ref} $=$ \{\@5\} }
      \avmr
	\end{avm}}
% 	\end{minipage}%
	\caption[Frame for c/i reversible COS verbs (embedded subevent)]{Frame for c/i reversible COS verbs (e.g. \textit{embrittle}; embedded subevent). Abbreviations: pat = patient, instr = instrument, res-st = result-state.} \label{fig:bendframe-embedded}
\end{figure}

\begin{figure} \singlespacing
  \centering
    \begin{avm}
      \avml  
      \@0  
      \[
      \avmspan{\textit{change-of-physical-form causation}}\cr
      \textsc{agent} &\@1 \cr
      \textsc{patient} & \@2 
      \[
      \textit{entity} \cr
      {\textsc{physical-state}} & {\textit{solid}}
      \]
      \cr
      \textsc{instrument} & \@3 
      \[
      \textit{entity} \cr
      {\textsc{physical-state}} & {\textit{solid}}
      \] \cr
      \textsc{cause} & \@4 
      \[ 
      \textit{agent-action}\cr 
      \textsc{agent} & \@1 \cr
      \textsc{patient} & \@2 \cr
      \textsc{instrument} & \@3 \cr
      \] \cr
      \textsc{effect} & \@5 
      \[ 
      \avmspan{{\textit{change-of-physical-form}}} \cr
      \textsc{patient} & \@2 \cr
      {\textsc{result-state}} & \@6 
      \[ {\textit{having-form}} \cr
      \textsc{patient} \; \@2
      \] \cr
      \]\cr
      \] 
      \cr
      {\textsc{ref} $=$ \{\@0, \@5 \} } \cr
      \avmr 
    \end{avm}
    \caption[Frame for c/i reversible COS verbs (single frame)]{Frame for c/i reversible COS verbs (e.g. \textit{embrittle}; single frame)}
    \label{fig:bendframe-single}
\end{figure} 

There are two differences between these frames as compared to the ones in the previous section.\largerpage
First, c/i reversible COS verbs describe reversible changes. In VerbNet, this is reflected by the semantic primitive \textsc{physical\_form}. Irreversible change, on the other hand, is indicated by an additional primitive \textsc{de\-gra\-da\-tion\_ma\-te\-ri\-al\_in\-te\-gri\-ty} (as, for example, in the description of break-COS verbs). \textsc{Physical\_form} is applied in lieu of the \textsc{state} primitive, which is present in the semantic description of c-only COS and c/i general COS verbs, and which I have so far translated into frames as a \textit{change-of-state} subevent. Correspondingly, the frames which I propose for c/i reversible COS verbs do not include a \textit{change-of-state}, but rather its subtype \textit{change-of-physical-form}, including the corresponding attribute-value combination \textsc{result-state} : \textit{having-form}. 

\begin{sloppypar}
Second, VerbNet introduces a selectional restriction [+solid] on both the \textsc{patient} and the \textsc{instrument} roles. 
This can straightforwardly be modeled in frames by introducing a \textsc{physical-state} attribute and specifying it with the value \textit{solid}. 
\end{sloppypar}

\section{The semantics of COS nouns}
\label{sec:cos-output}

In this section, I will first summarize the literature dealing with \textit{-ment}'s general semantics (\sectref{sec:cos-output-ment}). Based on this information and the preceding discussion of COS verb bases, I will then describe what can be expected for the semantics of COS nouns (\sectref{sec:cos-output-predictions}). 
I will then present an informal survey of readings attested in COS nominalizations, relating these findings to the literature (\sectref{sec:cos-output-survey}). 
Finally, I will proceed to the formalization of COS nominalization (\sectref{sec:cos-output-formal}). Based on my findings regarding \ment{} derivatives, I will be able to revise the VerbNet-based frame-semantic analyses as presented in \sectref{sec:cos-input-frames}.

\subsection{The semantics of \textit{-ment}}
\label{sec:cos-output-ment}

There are a number of (traditional) accounts describing the semantics of \textit{-ment} in some detail. Table \ref{tab:mentsemlit} gives an inventory of {five} representatives, including the semantic categories used by these authors to describe \textit{-ment} derivatives. From these accounts, it is obvious that \textit{-ment} is (and has been) a versatile suffix which can produce a whole range of readings, including both eventive categories (e.g. \textsc{action, state, event}) and non-eventive ones (e.g. \textsc{means, patient, product}).

\begin{table} 
  \caption{\label{tab:mentsemlit}Possible readings of \textit{-ment} derivatives as stated in the literature}
      \begin{tabular}{lll}
        \lsptoprule
        Author        & Category      & Example \\ 
        \midrule
\citet{Gadde.1910}  & action      & {repayment} \\
        & state, condition  & {astonishment} \\
         & means       & {embarrassment} \\
        & result (product)  & {entrenchment}\\
        & concretes     & {advertisement} \\ 
        \citet{Marchand.1969}   & act, fact, instance         & {enthronement} \\
        & something concrete or material    & {equipment} \\
        & state               & {amazement} \\
        & place               & {settlement} \\
        \citet{Bauer.2013}  & eventive  &{ceasement} \\
        & state   & {contentment} \\
        & result  & {improvement} \\
        & product & {pavement} \\
        & instrument/means\footnote{Note that the terms \textit{instrument} and \textit{means} are used inconsistently in \citet{Bauer.2013}. In chapter 10, both are used synonymously. Later, a distinction is made, with an instrument such as \textit{beeper} being defined as being directly involved, while a means such as \textit{stroller} is something that permits one to perform an action (see p. 241). It is for this reason that Table 11.1 in \citet[231]{Bauer.2013} lists \textsc{means} as a secondary reading for \textit{-ment}, and excludes \textsc{instrument}.} & {refreshment} \\
        & patient/theme & {investment} \\
        & location & {establishment} \\
        \citet{Lieber.2016} & event/state &{appointment} \\
        & result  & {impeachment} \\
        & instrument/means & {adornment} \\
        & [−animate] patient & {investment} \\
        & location & {development} \\
        \citet{Lloyd.2011}  & collective    & {vesselment} \\
        (Middle English)  & agent       & {parlement} \\
        & action: fact    & {attainement} \\
        & action: quality & {jugement}\\
        & object      & {wonderment} \\
        & result/state    & {consentement} \\
        & action: instance  & {confermement}\\
        & instrument    & {encumberment}\\               
        \lspbottomrule
      \end{tabular} 
  \end{table}

The first listed account, \citet[77--85]{Gadde.1910},  analyzes dictionary data and finds that English \textit{-ment} derivatives most often denote abstract ideas such as \textsc{actions} or \textsc{states}, but can also be found denoting \textsc{means} or \textsc{results} of an action (which he calls \textit{quasi-concretes}). Least productively, \textit{-ment} nominalizations may express concrete notions such as `something written.'

\citeauthor{Marchand.1969}'s (\citeyear[332]{Marchand.1969}) findings are similar: He reports ``the meanings which are usual with deverbal substantives,'' with \textsc{state} readings being the most productive. These are formed mainly on the basis of verbs which denote mental or emotional states (ibid.). In the present study, these formations will be dealt with in \chapref{ch:psy} under the label of \HighlightText{psych verbs}. The author furthermore mentions a small group of \textsc{place} nouns in \textit{-ment}. 

A recent corpus study investigating \textit{-ment} neologisms can be found in \citet{Bauer.2013}. In their data extracted from \acs{COCA} and the \acs{BNC}, the authors find a range of readings which is similar to the ones sketched for established formations by \citet{Gadde.1910} and \citet{Marchand.1969}. The authors state that the suffix primarily forms \textsc{event} and \textsc{state} readings, with a secondary function of forming a number of participant nouns. Within this second group, the \textsc{result} reading is ``by far the most frequent one exhibited'' and may thus be considered the ``default non-eventive interpretation'' (p. 212).

\citet{Lieber.2016} continues some of the work done in \citet{Bauer.2013}. The author introduces a new category [−animate] \textsc{patient} (as opposed to [+animate] \textsc{patient}), which subsumes both the inanimate members of the traditional \textsc{patient} category as well as \citeauthor{Bauer.2013}'s \textsc{product} (\citealt[19]{Lieber.2016}). \textsc{Event, state} and \textsc{result} readings are predominant readings found with \textit{-ment} (p. 60).

Finally, I have included \citet{Lloyd.2011}, who treats the semantics of \textit{-ment} in Middle English. While most categories found in Middle English neologisms correspond to those described for contemporary English, we find two additional ones. The two denominal formations \textit{vesselment} and \textit{utensilment} signify `group of Ns,' representing the category of \textsc{collective} (p. 40). The author's second category contains two deverbal formations and is therefore more interesting in the context of my study: \textit{parlement} and \textit{governement} were both coined in the 14\textsuperscript{th} century and have since kept their senses of `executive body' and `body of people who govern' (p. 40, p. 46). While \citet{Lloyd.2011} summarizes the two under the label \textsc{agent}, something like \citeauthor{Melloni.2011}'s (\citeyear{Melloni.2011}) label \textsc{agentive-collective} seems to be more appropriate. She finds that Italian nominalizations like \textit{amministrazione} `administration,' and \textit{redazione} `editorial staff' lack an agentive value and thus cannot denote agents, with the exception of \textsc{agentive-collective} readings. These are defined as ``groups of people agentively and volitionally involved in the performance of a certain activity'' (p. 121). In English, this reading is very rare, and the standard suffixes for this purpose are not \textit{-ment}, but \textit{-age} and \textit{-ery} (see \citealt[148 ff.]{Lieber.2004}). Those \textsc{agentive-collective} derivatives which do exist with \textit{-ment} (\textit{government, management, parliament}, and the now opaque \textit{department}, as well as \textit{scholarment}  \textendash{}  a nonce word coined by James Joyce, according to the \citetalias{OED}), also lexicalize habituality, or even profession.

Apart from \textit{-ment's} ability to form various readings on different kinds of bases, the polysemy of its individual formations has also been recognized. Thus, \citet[332]{Marchand.1969} states that ``[m]any words join several sense groups,'' and \citet{Gadde.1910} finds a systematic sense extension from \textsc{means} and \textsc{product} to \textsc{action} \mbox{(p. 80)}. This interplay between polysemy and systematicity is one of the starting points of this book. 

\subsection{Expectations regarding the semantics of COS nouns}
\label{sec:cos-output-predictions}

Based on the existing literature on \textit{-ment} as well as the COS verb frames, what can we expect with regard to the semantics of COS nouns?
In this section, I will first give an overview of expected possible and impossible readings (\sectref{sec:cos-output-predictions-read}). 
Then, I will elaborate on the contexts in which we can expect the different COS nouns, based on the type of causation denoted by the respective base verb (\sectref{sec:cos-output-predictions-cont}).

The expectations formulated in this section serve two main purposes.
Firstly, they provide a structure for later discussion, making the rather complex analysis more accessible.  
Secondly, they offer reference points for testing the validity of the frames proposed above. If an expectation is not met by the data, the frames have to be adjusted accordingly. This interplay between frames and language data will allow for an appropriate and useful formalization.     

\subsubsection{{(Im-)possible readings}}
\label{sec:cos-output-predictions-read}

A summary of expected possible and impossible readings is given in Table \ref{tab:predictions}. It is sorted by category, more precisely by type of node (event vs. participant), and alphabetically.\footnote{Since there is no unified terminology for these categories in the semantic literature, I have taken the labels from two of the most detailed analyses, namely \citet{Bauer.2013} and \citet{Lieber.2016}.}
Since the starting point of this book is the assumption that derivational processes can target nodes in the base verb frame, very generally we can expect that morphological processes can target any node which fulfills the frame-theoretical requirements for shifts (see \sectref{sec:fr-der-ref-der}). Therefore, only the presence or absence of nodes in the base verb frames predict the (im-)possibility of shifts at this point (second column).\footnote{Obviously, there are restrictions related to a given derivational process: Readings will be more or less likely or frequent, and some readings will not be attested at all. The frame formalizations as given above, however, do not yet point to any such tendencies.}
Obviously, the literature is more differentiated and realistic. Some readings are expected (e.g. \textsc{result-state}), while some are expected under certain conditions (e.g. [−animate] \textsc{patient}, but not [+animate] \textsc{patient}). Others are not expected (e.g. \textsc{behavior}), while one reading (\textsc{cause}) has not been addressed in the literature as of yet.\footnote{The list of unexpected readings is largely based on \citeauthor{Lieber.2016}'s (\citeyear{Lieber.2016}) Table 4.1. Further categories are irrelevant here because they are very affix-specific and/or do not take verbal bases: \textsc{abstract} (\textit{happiness}), \textsc{inhabitant/language} (\textit{New Yorker}), and \textsc{belief} (\textit{atheism}).}  

\begin{table}
  \caption[Expected and unexpected shifts in COS verb frames]{Expected and unexpected shifts in COS verb frames. Abbreviations: lit. = literature, n.a. = not available.} 
  \label{tab:predictions}
  \begin{tabular}{llll}
    \lsptoprule
    \multicolumn{2}{l}{Reading} & Predicted by frame & Predicted by lit. \\
    \midrule
    \multicolumn{2}{l}{Eventive readings} &&\\
    &change-of-state causation&yes&yes\\
    &cause&yes&n.a.\\
    &change-of-state&yes&yes\\
    &result-state&yes&yes\\
    \multicolumn{2}{l}{Participant readings}&&\\
    &adherent/follower&no&no\\
    &agent&yes&collective\\
    &behavior&no&no\\
    &experiencer&no&no\\
    &instrument&yes&yes\\
    &location&no&yes\\
    &measure&no&no\\
    &path&no&no\\
    &patient&yes&inanimate\\
    &product&no&yes\\
    &result&no&yes\\
    \lspbottomrule
  \end{tabular}
\end{table} 

Whether or not these expectations are met by the data has different consequences for the frames proposed above. 
If I fail to find referential shifts to some of the nodes, this needs to be explained and modeled accordingly. 
If, on the other hand, I find shifts to readings which are not modeled in the base verb frames, there are two options: Either the frame is incomplete, so that the respective nodes need to be introduced into the base verb semantics, or it would have to be argued that the nodes are somehow introduced into the frame during the derivation process by a unification mechanism (initiated by the suffix, or by the context). 

\subsubsection{Contexts}\label{sec:cos-output-predictions-cont}\largerpage

I will now turn to the effects that the base verb's type of causation may have on its derived noun.
More precisely, I hypothesize that the contexts that COS nouns occur in depend partly on whether their base verb denotes internal or external causation (see \sectref{sec:cos-prev}). This is also related to the question of which frame-semantic representation is the best fit for c/i COS verbs (see \sectref{sec:cos-input-frames}).

As discussed in \sectref{sec:cos-prev}, \citet{Smith.1970} claims that the change-of-state included in the semantics of c-only COS verbs is completely dependent on some external entity. The change-of-state in the semantics of c/i COS verbs can also be brought about by an external causer, but is at the same time thought to be relatively independent of one. These two assumptions can be tested by examining the contexts of c-only COS nouns and c/i COS nouns in \textsc{change-of-state} readings. 
First, for c-only COS nouns in a \textsc{change-of-state} reading, an external entity should always be expressed in context, or at least implied. This is modeled by the complex event structure, which by definition includes an external cause.
Second, for c/i COS nouns in a \textsc{change-of-state} reading, it could be possible to leave out external causes in context if the inchoative base exists independently of its causative counterpart. 
This, in turn, needs to be reflected in the frame formalization. 
If the cause is always expressed or implied in context, a complex event frame with reference on the change-of-state node should be used both for c/i COS nouns in a \textsc{change-of-state} reading and for the inchoative variant of c/i COS verbs.
In this case, c/i COS nouns (i.e. c/i general COS and c/i reversible COS nouns) would behave the same as c-only COS nouns in this respect. The \textsc{change-of-state} reading comes about by a shift from the complex event node in the verb to the change-of-state subevent node in the noun.
If, however, we do not find expressed or implied causes in the contexts of c/i COS nouns, the most adequate representation for their \textsc{change-of-state} reading, and for the inchoative variant of c/i COS verbs, is a simple event frame with reference on the central node. In this case, the \textsc{change-of-state} reading comes about by transposition from the inchoative variant of the verb.
To summarize: 

\begin{enumerate}\sloppy
  \item C-only COS verb bases: In \textsc{change-of-state} readings of c-only COS nouns, external causes are either expressed in the context, or implied.
  \item C/i COS verb bases: In \textsc{change-of-state} readings of c/i general COS nouns and c/i reversible COS nouns, ... 
  \begin{enumerate}
    \item ...external causes are either expressed in the context, or implied; or,
    \item ...external causes are not expressed in the context, nor implied.
  \end{enumerate}
\end{enumerate} 

\subsection{Survey of possible readings}
\label{sec:cos-output-survey}

In this section, I will first present and discuss participant readings (Sections \ref{sec:cos-output-survey-pat} to \ref{sec:cos-output-survey-instrument}) and then move on to eventive categories (Sections \ref{sec:cos-output-survey-trans} to \ref{sec:cos-output-survey-caus}). Shifts which have not been found attested will be treated last (\sectref{sec:cos-output-survey-unatt}), and a summary can be found in \sectref{sec:cos-output-survey-sum}.

\subsubsection{{Inanimate patient}}
\label{sec:cos-output-survey-pat}

Shifts to \textsc{patient} readings can easily be found for five nominalizations, as exemplified in (\ref{ex:cos.ro}). The remaining nominalizations are attested as \textsc{results} or \textsc{products} instead. I will discuss this distribution in the respective sections below. 

\begin{exe}
  \ex \label{ex:cos.ro}
  \begin{xlist}
    \item \label{ex:cos.ro1} I set down the scrap of doll's dress, a \textbf{bedragglement} of loose lace hem {\small(\acs{COCA} FIC Bk:MournersBench 1999)}
    \item \label{ex:cos.ro2} ``When yoga was in its womb in India, it was safe and protected, but as it ventures into the harsh world, it is in danger of disintegrating,'' [...]  Gerson refers to most of the newer yoga classes as ``\textbf{debauchment}.'' Yoga purists such as Gerson are calling for a return to teaching yoga in its original form {\small(\acs{WC} NEWS articles.latimes.com 2002)}
  \end{xlist}
\end{exe}

\noindent Note that \textit{debauchment} is ambiguous here: It can be interpreted as an \textsc{instrument} (new yoga classes debauch the tradition) or as a \textsc{patient} (yoga is something that has been debauched by new classes).
This is a typical case of insufficient context. It is obvious that the speaker only intended one of the two readings, but without an unambiguous cue (such as \textit{because they debauch the tradition} indicating an \textsc{instrument} reading) it is impossible to know. 

\citet{Lieber.2016} found that \textit{-ment} can only produce [−animate] \textsc{patient} readings, which is why I did not expect to find any [+animate] \textsc{patient} readings. I can corroborate her claim, with very few exceptions like this one: 

\begin{exe}
  \ex \label{ex:cos.ro3} A short, squallid [sic] woman with a face like a toad wearing a horrible pink cardigan pattered past, turning up her nose in distaste at this \textbf{befoulment} sitting on her ministry bench. {\small(\acs{WC} FIC fanfiction.net 2007)}
\end{exe}

\noindent In this attestation, which is ambiguous between a \textsc{patient} and a \textsc{causer} reading (see \sectref{sec:cos-output-survey-instrument}), the \textit{befoulment} is a person. This, however, is a typical example of a post-lexical shift coerced by context. Lexicalized examples can easily be found in the corpora (see also \sectref{sec:psy-output-survey-stim}; \citealt{Loebner.2008}): 

\begin{exe}
	\ex You are such a \textbf{comfort} to me, dear. {\small(\acs{COCA} FIC Bk:BasketBrigade 2016)}
\end{exe}

\noindent The limitation to [−animate] \textsc{patient} readings does not go back to the base verb semantics, since animate patient participants are possible for most base verbs in the data set. For example, something may \textit{bedraggle the most noble among us} (\acs{GB}) or \textit{befoul one family after another} (\acs{iWeb}). Some verbs only allow collective [+animate] patients (\textit{disperse the Germans, \acs{iWeb}; disband the committee}, \acs{COCA}). Only a few do not allow for animate patients at all (\textit{abridge, embrittle}). 
It can thus be concluded that the restriction to [−animate] \textsc{patient} readings in the \mbox{\textit{-ment}} nominalizations must originate from preferences of the suffix. 

\subsubsection{Result} 
\label{sec:cos-output-survey-result}
The next reading, attested for nine types in my data, is that of \textsc{result}:

\begin{exe}
  \ex 
  \begin{xlist}
  \item \label{ex:cos.res1} The performance \textbf{increasement} and the darker interface are looking gorgeous! \#photoshop \#cs6 bit.ly/xzlsAV {\small(Twitter @maxlewe 2012)}
  \item \label{ex:cos.res2} No one could say that he was going bald -- he is grateful for that -- but his hair has lost its luxuriance and, once jet black, is gray now, and he keeps it cropped short to conceal the \textbf{diminishment} 
  {\small(\acs{COCA} FIC Bk:LoveMyYouth 2011)}
  \item \label{ex:cos.res3} In 2010 only, 2627 cases of morbidity were reported followed by an \textbf{increasement} of 9\% in 2011 with 2865 reported cases. \\ 
  {\small(Google ACAD ijergs.org 2014)}
  \end{xlist}
\end{exe}


\noindent This finding is especially interesting because, according to VerbNet, the only context in which the \textsc{result} role surfaces for the base verbs in question is in secondary predication (as in \textit{strangle to death}). Therefore, as justified above (\sectref{sec:cos-input-frames-c/igeneral}), \textsc{result} has so far not been included in the base verb frames  \textendash{}  a decision which needs to be revised in the final frame formalization.

\textsc{result} and \textsc{patient} readings are found in complementary distribution in my data: If a type can denote a patient, it will not denote an implicit product, and vice versa. Thus, \textit{bedragglement, befoulment, congealment, debauchment} and \textit{unfoldment} are only attested in \textsc{patient} readings, while \textit{abridgement, besmirchment, decenterment, diminishment, dispersement, embetterment, increasement, progressment} and \textit{worsenment} are only attested as \textsc{results}.
I will come back to this distribution in the next section.

\subsubsection{{Implicit product}}

Following the VerbNet definition (see \citealt[319]{Palmer.2017}), a product is a ``[r]esult that is a concrete object.'' While this semantic role is not listed for COS verbs, I have found \textsc{product} readings for two nominalizations: \textit{discolorment} and \textit{embrittlement}.\footnote{\textsc{Products} have also been called \textsc{result objects} (e.g. \citealt{Schulzek.forthcoming}), or \textsc{effected objects},  in opposition to \textsc{affected objects}, which correspond to \textsc{patients} (see e.g. \citealt[343]{Motsch.1999}; \citealt{Hopper.1985}).} Consider the following examples for illustration:

\begin{exe}
  \ex 
  \begin{xlist}
  \item \label{ex:cos.ip1}  Interior is generally very well kept, just some \textbf{discolorment} \ExHighlight{on the steering wheel} {\small(Google COMM sfbay.craigslist.org 2017)}
  \item \label{ex:cos.ip2} After 8 weeks of hydrolytic degradation, the nonwoven fabric was broken. There is an obvious \textbf{embrittlement} and cracking \ExHighlight{on the nonwoven fabric} (Figure 6.5b). {\small(\acs{GB} ACAD Cellulose Based Composites 2014)}
  \end{xlist}
\end{exe}

\noindent The italicized parts in these attestations illustrate that the \textsc{product} nominalizations in my data set denote what I will call \HighlightText{implicit products}. In such contexts, two entities are involved in the event: 
A patient is affected, and a product, which is inherently related to the patient, is created. 
As \citet[264]{Osswald.2005}  observes: ``Injuries and damages are objects on a par with stains and holes  \textendash{}  dependent on the object they are attached to,'' stating further that ``something is implicitly `created.'\thinspace''
For example, an embroidery (\textsc{implicit product}) is created while embroidering a pillow (\textsc{patient}).  
Implicit products can be distinguished from \HighlightText{explicit products} such as \textit{building}. Here, the action denoted by the base verb \textit{build} also produces something, but it does not affect a patient in the process. 
I need the distinction between implicit and explicit products for two reasons: First, the two categories are modeled differently in frames, and second, only implicit products are attested in my data.

Importantly, the implicit products created by discoloring and embrittling, respectively, do not surface syntactically. Therefore, like the primary \textsc{result} reading discussed in the preceding section, \textsc{implicit product} is not listed as a participant in VerbNet, and is therefore not included in the VerbNet-based frames. 

It can be a bit tricky to distinguish \textsc{implicit product} from \textsc{patient} nominalizations in corpus data. 
Disentangling the participant structure of a given attestation helps in this case. 
For example, compare the \textsc{patient} attestation in (\ref{ex:cos.ro1}) to the \textsc{implicit product} attestation in (\ref{ex:cos.ip1}).
Both examples have a \textsc{patient}, but a product is created only in the second example. In (\ref{ex:cos.ro1}), \textit{bedragglement} is co-referential with \textit{the scrap of doll's dress} (the \textsc{patient}), while in (\ref{ex:cos.ip1}), the \textit{discolorment} (\textsc{product}) is on \textit{the steering wheel} (\textsc{patient}). 

\begin{table}[b]
  \centering 
  \caption[Distribution of \textsc{patient, implicit product} and \textsc{result} readings]{Distribution of \textsc{patient, implicit product} and \textsc{result} readings across COS noun subclasses, rounded to the nearest percent. Abbreviations: impl = implicit, n.a. = not attested.} 
  \label{tab:patprodres}
  \begin{tabular}{lrrrr}
    \lsptoprule 
    Noun class  & \multicolumn{1}{l}{\textsc{Patient}} & \multicolumn{1}{l}{\textsc{Impl. product}} & \multicolumn{1}{l}{\textsc{Result}} & \multicolumn{1}{l}{n.a.} \\
    \midrule
    c-only COS             & 43\%    & 0\%    & 43\%      & 14\%         \\
    c/i general COS              & 11\%    & 11\%    & 67\%      & 11\%         \\
    c/i reversible COS               & 50\%    & 50\%    & 0\%       & 0\%          \\
    \lspbottomrule    
  \end{tabular}
\end{table}

Let us now return to the distribution of \textsc{patient} and \textsc{result} categories. I have mentioned in the preceding section that these are in complementary distribution in my data, and the same observation can be made for \textsc{implicit product}  \textendash{}  a subtype of \textsc{result}: \textit{discolorment} and \textit{embrittlement} are only attested in \textsc{implicit product} readings. 
This pattern is not predicted by the VerbNet-based frames since it is not systematically related to the VerbNet classes. In other words, introducing the categories \textsc{result} and \textsc{product} into the mix creates subdivisions of the data which do not correspond to VerbNet classes. 
An overview is given in Table \ref{tab:patprodres}. 
For example, 50\% of c/i reversible COS verbs are attested in a \textsc{patient} reading, while the other 50\% can produce an \textsc{implicit product} reading. However, \textsc{patient} readings were also found both for 43\% of c-only COS nouns and for 11\% of c/i general COS nouns.

Based on this distribution, we can formulate the following pattern: All events denoted by the base verbs in my data set affect a \textsc{patient} participant. If, additionally, an \textsc{implicit product} or a \textsc{result} is created in the process, the \textit{-ment} nominalization will refer to this entity, respectively, and not to the \textsc{patient}. 

\subsubsection{{Instrument and inanimate causer}}
\label{sec:cos-output-survey-instrument}

As expected, I have found shifts to \textsc{instrument} readings in all COS subclasses.
An example is given in (\ref{ex:cosinstr}), where \textit{congealment} clearly refers to something that is used by an agent to congeal the blood.

\begin{exe}
  \ex \label{ex:cosinstr} Minimal bleeding and I didn't have to have any guaze[sic]/tissue in my mouth at all to try and stop it? I'm thinking that they must have used a \textbf{congealment} or something to make it clot while I was under or something? {\small(\acs{GloWbE} COMM forums.whirlpool.net.au 2010)}
\end{exe}

\noindent Clear \textsc{instrument} readings were attested for eight types in the data set: \textit{befoulment, besmirchment, congealment, dispersement, embetterment, embrittlement, increasement} and \textit{progressment}. For ten types, I found readings which are more appropriately categorized as \textsc{causers}. A causer is a participant that ``initiates the event, but that does not act with any intentionality or consciousness'' (\citealt[317]{Palmer.2017}). It is conceptually similar to an instrument, but it is not manipulated by an agent and is therefore a subtype of \textsc{actor} rather than of \textsc{undergoer} in VerbNet. \textsc{Causer} has already been introduced as a hyperonym of the \textsc{stimulus} category, which will be needed for the formalization of psych verbs and nouns. 

\textsc{Causers} in my data include a variety of inanimate entities, for example \textit{the phial} in (\ref{ex:coscauser1}), \textit{Seifer's blood} in (\ref{ex:coscauser2}), or \textit{an approach} in (\ref{ex:coscauser3}). Inconclusive attestations in which the nominalization can be paraphrased as `something which causes V-ing' have been categorized as ambiguous between a \textsc{causer} and an \textsc{instrument} reading, since it can neither be determined nor ruled out that an intentional agent is involved. An example is given in (\ref{ex:cos.instrcauser}). 

\begin{exe}
  \ex \label{ex:coscauser}
  \begin{xlist}
    \item \label{ex:coscauser1} You see, almost directly after sipping the potion, I noticed the \textbf{befoulment} on Severus's otherwise orderly working area. Yes... the phial in which rested the forbidden love-potion. {\small(Google FIC fanfiction.net 2006)}
    \item \label{ex:coscauser2} I stood, looking at the rip that ran through the back of the black material, surrounded in \textbf{discolorment} I classified as Seifer's blood. {\small(Google FIC fanfiction.net 2001)}
    \ex \label{ex:coscauser3} Besides, such an approach is seen not as a \textbf{diminishment} of public health but rather as a net zero-sum game in which the source of funding gets changed but not the overall funding level. {\small(\acs{COCA} ACAD EnvironHealth 2002)}
    \ex \label{ex:cos.instrcauser} @OfficialMCPB They've been a great view \textbf{increasement} for me, They got me the views and I got accepted to The Game Station Network :)
    {\small(Twitter @ShotbowNetwork 2012)}
  \end{xlist}
\end{exe}

\noindent The distribution between \textsc{instrument} and \textsc{causer} readings is partly complementary. That is, most nominalizations have been found attested as either one or the other (e.g. \textit{congealment} as an \textsc{instrument}, \textit{discolorment} as a \textsc{causer}), but some types can also denote both instruments and causers (e.g. \textit{befoulment}). 
Attempts to explain this pattern are speculations at this point, but it seems likely that it is to a large extent due to a frequency effect of the base verb participants. Thus, speakers may tend to use verbs like \textit{congeal} more often with an agent and an instrument participant, as in (\ref{ex:instrcauscont1}), and verbs like \textit{discolor} more often with a causer, as in (\ref{ex:instrcauscont2}), while the distribution of \textit{befoul} may be rather balanced, as in (\ref{ex:instrcauscont3}) and (\ref{ex:instrcauscont4}). 
Spot checks in the corpora verify that there are definitely tendencies in this direction, but a robust quantitative analysis will have to be left for future research. 

\begin{exe}
  \ex \label{ex:instrcauscontexts}
  \begin{xlist}
    \item \label{ex:instrcauscont1} Baking powder is often times [sic] used in sauces (like this one) to help \textbf{congeal} the ingredients together. {\small(\acs{iWeb} COMM acozykitchen.com 2015)}
    \item \label{ex:instrcauscont2} Over time, food, beverages and other substances will form another layer on top of the enamel called a pellicle film, which is a buildup of foreign materials that can \textbf{discolor} the surface of teeth and lead to stains.  {\small(\acs{iWeb} WEB fsastore.com 2020)}
    \item \label{ex:instrcauscont3} This [...] removes the necessity of paying \$8000.00 to U-haul and \textbf{befouling} the air with diesel. {\small(\acs{iWeb} COMM thetinylife.com 2012)}
    \item \label{ex:instrcauscont4} [...] a flock of up to 30 turkey vultures spend the night \textbf{befouling} an upper crust roof befeore [sic] catching the morning thermals to cruise the Huron river for carrion.... {\small(\acs{iWeb} COMM annarbor.com 2011)}
  \end{xlist}
\end{exe}

\noindent Two types have been found in neither an \textsc{instrument} nor a \textsc{causer} reading (\textit{decenterment} and \textit{disbandment}). This also seems to be a frequency effect. \textit{Decenter} can rarely be found with a causer instead of an agent, 
instruments are also highly infrequent. 
Similarly, \textit{disband} seems to always denote an intentional event, allowing no causers at all, while contexts with an instrument are possible but very rare.

Finally, a note on the selectional restrictions of the \textsc{instrument} and \textsc{causer} categories is in order.
First, [+solid] is given in VerbNet for instrument participants of c/i reversible COS verbs. This restriction indeed applies to most attestations, with some exceptions such as the one in (\ref{ex:cosinstrliq}).

\begin{exe}
  \ex \label{ex:cosinstrliq}  Supercaustics, also called supercorrosives, are powerful acids that quickly eat through polymers (rubber and plastic) and metals. They include acids known as \textbf{embrittlements} which weaken metals, and ones which dissolve polymers called depolymers. {\small(\acs{WC} BLOG newworldwar.org 2011)}
\end{exe}   

\noindent Second, all identifiable referents both of \textsc{instrument} and of \textsc{causer} nominalizations are inanimate. This will play a role in the modeling of (im-)possible readings.  

\subsubsection{{Transposition of a complex event}}
\label{sec:cos-output-survey-trans}


\HighlightText{Transposition} is generally regarded as a change of syntactic category of a word without a change in the semantics (though see \citealt{Lieber.2015} for criticism of this notion). In this study, this means that the complex event denoted by a given base verb is also a possible reading of its nominalization. 
As has also been observed by \citet[207]{Bauer.2013}, such readings are most easily identifiable when the complete argument structure of the verb is present, as exemplified with a constructed sentence in (\ref{ex:costranspo1}) and a corresponding attestation in (\ref{ex:costranspo2}). Transpositions of complex events were identified for all types in the data set.

\begin{exe}
  \ex \label{ex:costranspo} 
  \begin{xlist}
    \ex \label{ex:costranspo1} Hydrides \textbf{embrittle} the cladding.
    \ex \label{ex:costranspo2} Hydrides then form and can limit the fuel lifetime due to their \textbf{embrittlement} of the cladding. {\small(Google WEB imperial.ac.uk 2014)}
  \end{xlist}
\end{exe}

\noindent The complex event structure demands an external cause which, I have argued, can be expressed in the context of the nominalization, or merely implied. Indeed, in most examples, identifying a cause is as straightforward as in (\ref{ex:costranspo2}). An example of an implied cause is given in (\ref{ex:costranspo4}): World knowledge dictates that rural areas don't just uplift themselves; they need someone to act or something to happen to be uplifted. 

\begin{exe}
  \ex \label{ex:costranspo4} Nor have hopes and expectations for the \textbf{upliftment} of the rural areas been aroused as the Busia group managed to do in the late 1960s. {\small(\acs{COCA} ACAD AfricaToday 1991)}
\end{exe}

\subsubsection{{Change-of-state}}
In parallel to the transposition of a complex event, nominalizations can denote a transposed simple event if a given base verb has a simple event structure.
There is no nominalization based on an inchoative verb in the data set, so I am using the calibratable-COS noun \textit{dwindlement}, which is not in the data set, for illustration: 

\begin{exe}
  \ex \label{ex:cos.dwindlement} Gnathal, it's time for your \textbf{dwindlement} into Civil Death. {\small(Google FIC Cell-U.R.-Tales 2009)}
\end{exe}

\noindent In the data set, \textsc{change-of-state} readings were attested both for nominalizations of c/i COS verbs, and for those of c-only COS verbs. I will discuss both in turn.

The examples in (\ref{ex:cos-cos}) show \textsc{change-of-state} readings of the c/i COS nouns \textit{embrittlement} and \textit{increasement}. They also illustrate the strategies which the group of annotators applied to identify such readings. First, contextual clues such as \textit{transition} in (\ref{ex:cos-cos1}) can indicate that we are dealing with a \textsc{change-of-state} reading. Second, as exemplified in (\ref{ex:cos-cos2}), the causing subevent can be spelled out in relation to the caused subevent. Here, the causing subevent is \textit{the protocol is used as a sub-protocol}, with the caused subevent \textit{the rapid increasement of network traffic and computational complexity}. This second strategy also exemplifies a simple yet effective advantage of frames: It is often instructive to draw a frame structure and fill the nodes with the elements of the given attestation. This simple procedure has proven to facilitate the task of categorizing, as well as of explaining a given categorization. 

\begin{exe}
  \ex \label{ex:cos-cos} 
  \begin{xlist}
    \ex \label{ex:cos-cos1} Biodegradation is characterized by \textbf{embrittlement}, or the \ExHighlight{transition} of plastic from solid pieces into tiny particulates {\small(\acs{COCA} ACAD IntlAffairs 2005)}
    \ex \label{ex:cos-cos2} The biggest problem for the oblivious transfer protocol is the rapid \textbf{increasement} of network traffic and computational complexity if \ExHighlight{the protocol is used as a sub-protocol} {\small(Google ACAD springer.com 2007)}
  \end{xlist}
\end{exe}

\noindent The finding that \textsc{change-of-state} readings are possible for c/i COS nouns contributes to a topical controversy: According to what \citet[587]{Borer.2013} has called the \HighlightText{Transitivity Effect} in compounding, the head of a synthetic compound based on a c/i COS verb should only get a complex event reading, including either an explicit or at least an implied external argument. This alleged effect can straightforwardly be translated to the subject of derivation as well. \citet{Lieber.2016}, however, finds evidence counter to this claim in corpus data: She identifies both synthetic compounds (\textit{water-boiling}, p. 158; \textit{glacier melting}, p. 156) and derivatives (\textit{Hezbollah's expansion}, p. 49) with clear inchoative interpretations in context. My data corroborates her counter-evidence.

We can regard such \textsc{change-of-state} readings as either shifts to part of the complex verb frame, or as transpositions of the simple event variant of the base verb. At this point, it seems that both options may exist in parallel. Consider again example (\ref{ex:cos-cos2}). Here, the complex event structure is spelled out in the attestation: The first subevent is \textit{the protocol is used as a sub-protocol}, and the second subevent is \textit{the rapid increasement of network traffic}.  
This points to the conclusion that we are dealing with a shift in a complex event frame, since the cause is still part of the representation in a \textsc{change-of-state} reading of \textit{increasement}. In examples such as (\ref{ex:cos-cos1}), on the other hand, it is a question of world knowledge, or even expert knowledge, whether a given change can occur on its own, or whether there has to be a cause. Finally, there are attestations with c/i COS nouns which are clearly transpositions of a simple event, as exemplified in (\ref{ex:cos-trcos}). Here, an external cause is neither expressed nor implied. 

\begin{exe}
  \ex \label{ex:cos-trcos} Because of the baby boomers and their rapid \textbf{progressment} to an older age, it is natural for them to start taking more medications {\small(Google COMM sectalk.com 2012)}
\end{exe}

\noindent Thus, as regards c/i COS nouns, both zooming into the change-of-state subevent and transposing the semantics of a simple base verb variant are valid mechanisms. The cause is most often, but not necessarily, expressed or implied. 

That zooming into the change-of-state subevent is a valid, systematic mechanism is further corroborated by nouns with transitive base verbs, i.e. c-only COS nouns. 
Only one of the seven c-only COS nouns in the data set, \textit{befoulment}, has not been found in a \textsc{change-of-state} reading. Consider the following examples for illustration:

\begin{exe}
  \ex \label{ex:cos-cosnonalt} 
  \begin{xlist}
    \ex \label{ex:cos-cosnonalt1} When our citizens feel a need to \ExHighlight{change their constitution} for their \textbf{embetterment}  \textendash{}  it will be because they want it because they changed. {\small(Twitter @HaneenKnown 2014)}
    \ex \label{ex:cos-cosnonalt2} It has been contemplated in the present treatise that expansion of the earth has taken place due to gradual \textbf{upliftment} of the semi-fluid mantle, \ExHighlight{in response to tidal bulge of that medium}  
    {\small(\acs{GB} ACAD Earth: The Planet Extraordinary 2007)}
  \end{xlist}
\end{exe} 

\noindent In both attestations given in (\ref{ex:cos-cosnonalt}), the nominalization can exhibit a \textsc{change-of-state} reading, and in both cases, the causing subevent is spelled out (italicized in the examples). 
Thus, we can conclude that speakers are using the nominalizations to zoom in on part of the complex event (see also \sectref{sec:fr-der-ref-der}). 

In \sectref{sec:cos-output-predictions}, I have hypothesized that the cause needs to be expressed or implied in c-only COS nouns. However, there is one attestation which can exhibit an isolated \textsc{change-of-state} reading, indicated by the parallel phrase \textit{things are getting better} (see \ref{ex:cos-cosnonalt3}).\footnote{It has been pointed out to me that, being the first part of a compound, \textit{embetterment} may have an implicit agent which is not necessarily expressed in context (Lieber p.c., 11.11.2019). The context is therefore ambiguous: There may or there may not be an implied cause.}

\begin{exe}
  \ex \label{ex:cos-cosnonalt3} Actually, if anyone has read this far down, it's my opinion that \ExHighlight{things are getting better} all the time: just not as fast as we would like or in the way that we expect, and the \textbf{embetterment} process also riles up the crazies {\small(Google COMM unfogged.com 2009)}
\end{exe} 

\noindent The most likely explanations for this attestation are variation of the base verb, or analogy. 
While intransitive constructions with \textit{embetter} cannot be found on Google (or in the smaller corpora), there is a recent trend to create reflexive constructions with it (e.g. \textit{service seems to have embettered itself}, Google COMM yelp.ca 2012). Therefore, a \textsc{change-of-state} reading without a cause could be based on the reflexive variant of \textit{embetter}.
Second, the attestation may have been created in analogy with the semantically similar, lexicalized nominalizations \textit{betterment} and \textit{improvement}, both of which have c/i COS base verbs.
All things considered, I have decided that this attestation does not challenge the assumption that c-only COS verbs need an expressed or implied cause.

With regard to \textsc{change-of-state} readings of COS nouns, we can thus summarize that they are possible for all COS subclasses. For c/i COS nouns, two mechanisms are possible, namely both transposition of the simple event variant and shifting/zooming, while those based on c-only COS verbs can only shift/zoom. 

\subsubsection{{Result-state}}
\textsc{Result-state} readings were easily found for 13 types in the data set, as exemplified by the attestations in (\ref{ex:cos-rs}). 

\begin{exe}
  \ex \label{ex:cos-rs}  
  \begin{xlist} 
    \item I encounter the dates  \textendash{}  a dozen sellers offering them in different states of \textbf{congealment}, from the deep-brown gooey \textit{Khejur} oozing syrup to bone-dry \textit{Khormas} and orange-yellow unripe dates.\\\hbox{}\hfill\hbox{\small(\acs{GloWbE} BLOG backtobangladesh.blogspot.de 2010)}
    \item They seemed as eager to see if I was as wasted away as rumour had it as anything else, sizing me up as if to say my state of \textbf{bedragglement} was scandalous {\small(\acs{GB} FIC Bk:ColonyUnrequitedDreams 2000)}
  \end{xlist}
\end{exe}

\noindent In many attestations it was hard to decide among annotators whether we are dealing with a \textsc{result-state} or some other sort of abstract result which is not a state. Consider the examples in (\ref{ex:cos-rs?}). Is the skin in a state of having been discolored, or is \textit{discolorment} an abstract \textsc{result}? Similarly, is the \textit{dispersement} the state of the balls lying scattered on the ground? 

\begin{exe}
  \ex \label{ex:cos-rs?}  
  \begin{xlist} 
    \item I read that permanent \textbf{discolorment} of the skin can happen from using the cream. {\small(Google COMM askapatient.com 2007)}
    \item I got a large and a small bucket (about 100 balls) because I wanted to just hit a ton with my wedges and dial their distances in better and work on my iron and driver swing. [...] Now the \textbf{dispersement} is something I need to work on but thats more of my aim and alignment than anything else. {\small(Google  COMM thesandtrap.com 2010)}
  \end{xlist}
\end{exe}

\begin{sloppypar}
\noindent In this respect, one group of nominalizations, namely those which denote a change which is typically or often measured on a scale, is notoriously unclear. For \textit{diminishment, embetterment, increasement, progressment} and \textit{worsenment}, only attestations like (\ref{ex:cos.res1rep}) and (\ref{ex:cos.res2rep}) (repeated from (\ref{ex:cos.res1}) and (\ref{ex:cos.res2})) have been found. 
While discussed in \sectref{sec:cos-output-survey-result} under the label \textsc{result}, these attestations (and others in the data set) actually tend to be ambiguous, depending on the informant. A \textsc{result-state} reading is not available for all speakers, and it is never regarded as the most likely option in a given context. Why prominently scalar base verbs present such a fuzzy picture with regard to \textsc{result-state} readings in their nominalizations will have to be the subject of further research, since I have decided to omit the modeling of scalarity in this study (see also \sectref{sec:meth-revis-final}).
\end{sloppypar}

\begin{exe}
  \ex 
  \begin{xlist}
  \item \label{ex:cos.res1rep} The performance \textbf{increasement} and the darker interface are looking gorgeous! 
  {\small(Twitter @maxlewe 2012)}
  \item \label{ex:cos.res2rep} No one could say that he was going bald [...] but his hair has lost its luxuriance and, once jet black, is gray now, and he keeps it cropped short to conceal the \textbf{diminishment} {\small(\acs{COCA} FIC Bk:LoveMyYouth 2011)}
  \end{xlist}
\end{exe}

\noindent Interestingly, prominently state-forming suffixes such as \textit{-ness} can coerce these scalar base verbs into less ambiguous \textsc{result-state} readings:

\begin{exe}
  \ex \label{ex:ness-rs} Who would like to bet that Dell gives up on this experiment pretty quickly  \textendash{}  I know I won't be suprised [sic] if \textbf{embetterness} is quickly replaced by embitteredness on Ubuntu's part. {\small(Google COMM mattcutts.com 2007)}
\end{exe}

\noindent This example shows that \textsc{result-state} readings are possible, given a suffix for which this reading is prevalent. We may thus be dealing with a partial blocking effect, where speakers prefer specialized suffixes like \textit{-ness} over the more diversified \textit{-ment}.

To sum up, 13 \textit{-ment} neologisms in the data set readily produce \textsc{result-state} readings, while five prominently scalar types produce \textsc{result-state} readings only in very ambiguous contexts, and only for some speakers. 
The \textsc{state-}forming suffix \textit{-ness} is more successful in turning the five base verbs in question into \textsc{result-state} nouns. Therefore, it can be concluded that prominently scalar base verbs disprefer being shifted to a \textsc{result-state} reading by \textit{-ment}. Due to the complexity of modeling scalar concepts in frames, clarifying the reasons for this observation will be left for future research. 

\subsubsection{{Cause}}
\label{sec:cos-output-survey-caus}

In the literature, the causing subevent is claimed to be most often an action, while other types of event are also possible (see \sectref{sec:cos-prev-formal}). 
In the nominalizations' semantics, we find the same situation: Most of the time, the denoted causing event is an \textsc{action}, as exemplified in (\ref{ex:cos-causev1}), where the cause is an agentive action. However, it can also be another kind of event, such as the \textsc{causing state} reading in (\ref{ex:cos-causev2}) and the non-agentive \textsc{action} reading in (\ref{ex:cos-causev3}). 

\begin{exe}
  \ex \label{ex:cos-causev} 
  \begin{xlist}
    \ex \label{ex:cos-causev1} The puke-related world-\textbf{embetterments} in this thread are blowing my mind. When people come to my house, I pretty much always offer them a cup of tea. 
    {\small(Google COMM theppk.com 2011)}
    \ex \label{ex:cos-causev2} The staff waited on the Blundens, devoted, and prescient, too, anticipating their desires. When they skipped meals, the waiters would say, ``We missed you last night,'' as though their absence were a \textbf{diminishment}. {\small(\acs{COCA} FIC NewYorker 2006)}
    \ex \label{ex:cos-causev3} The white of the breast feathers was turned a dull gray by the rain's \textbf{bedragglement} {\small(Google FIC forgottenbooks.com 1922)}
  \end{xlist}
\end{exe}  

\noindent It should be noted that finding attestation for the \textsc{cause} reading was not easy: Of the 18 types in the data set, ten have this reading attested, and only three attestations are not ambiguous. The first intuition may be to assume that the base verbs of the eight unattested types are not complex events after all, that is, they might not have a causing subevent. This was checked in the corpora, where all base verbs in question were found to be more or less frequently attested in a context with a causing subevent.\footnote{\acs{iWeb} was probed with the query 〈V* * \_nn* by〉 and the results were manually checked. For \textit{congeal}, no results were found, but a Google search for 〈``congealed the * by''〉 yielded the desired constructions.} Therefore, it can be concluded that we are dealing with a gap due to scarcity of data, and that \textsc{cause} is a rare but possible reading of COS nouns (see also \chapref{ch:observations}).

\subsubsection{Unattested shifts}
\label{sec:cos-output-survey-unatt}

In the following, I will discuss the shifts that are systematically not attested in my data. These patterns can be explained by two factors: Preferences of \textit{-ment}, and properties of the base verb classes. 
The concrete changes necessary to  reflect these factors in my frames will be elaborated on in the frame analysis of COS nouns in \sectref{sec:cos-output-formal}. 
In addition, some shifts were expected and are systematically attested, with the exception of single nominalizations in the data set. This issue of gaps in the data will be discussed in detail in \chapref{ch:observations}.

There are seven readings which are commonly produced by derivation, but which are not possible in \textit{-ment} derivatives (see \sectref{sec:cos-output-predictions}). I have not found these in my data either. 
Five of these readings are not represented in the VerbNet-based frames, namely \textsc{path, adherent/follower, behavior, measure} and \textsc{experiencer}.
These need not be addressed further since no shifts are predicted by the base verb frames in the first place.
Two of the seven readings not predicted in the literature, however, are represented in the VerbNet-based frames, namely \textsc{agent} and [+animate] \textsc{patient}. Therefore, the formalization needs to represent that \textit{-ment} does not allow the corresponding shifts. For \textsc{agent}, this can be straightforwardly done by not including the corresponding indices in the set of possible referents below the frame. To prevent shifts to [+animate] \textsc{patients}, a constraint is required, since shifts to the \textsc{patient} node are allowed as long as the \textsc{patient} is inanimate. A similar conclusion has been drawn by \citet[115, 237]{Melloni.2011}, who investigates Italian nominalizations in \textit{-mento}: A shift to the  \textendash{}  prototypically [+animate]  \textendash{}  \textsc{experiencer} reading is not possible since the referent has to be [−animate] and [−sentient]. Instead, Italian makes use of its present participle suffix to express sentient categories such as \textsc{agent} and \textsc{experiencer}.

The second factor governing possible readings are the properties of the base verbs. More precisely, if a base verb frame does not include a given participant or subevent, no shift to this participant or subevent is expected. I have already mentioned the participants \textsc{path, adherent/follower, behavior, measure} and \textsc{experiencer}, which are also not predicted by the existing literature. 
In addition, the established literature gives \textsc{location} and \textsc{agentive-collective} as possible readings for \textit{-ment} derivatives. However, there are no corresponding nodes in the COS verb frames, and, as expected, no \textsc{location} and \textsc{agentive-collective} readings were attested.\footnote{Groups of people can of course be the \textsc{agents} of a COS event, and in the case of \textit{disband} even prominently so. However, I have argued above that the \textsc{agentive-collective} category as attested in \textit{-ment} derivatives like \textit{government} also requires this group to act habitually and/or professionally. This is not the case for the \textsc{agent} participants of the COS verbs in the data set.} 

\pagebreak
\subsubsection{Summary}
\label{sec:cos-output-survey-sum}

The findings presented in this chapter confirm some of the expectations discussed in \sectref{sec:cos-output-predictions}, while disputing others.
I will first summarize these findings before formalizing COS nominalization in \sectref{sec:cos-output-formal}. Where the VerbNet-based frames evoke incorrect predictions, the formalization will be revised accordingly.  
For a rough overview of my findings at a glance, I am including Table \ref{tab:predictions-results}, which is an updated version of Table \ref{tab:predictions}. Changes in this modified table as compared to the original one are highlighted by italics.

\begin{table}
  \caption[Expected shifts in COS frames and results of corpus study]{Expected shifts in COS frames and results of corpus study. Findings which require changes in the VerbNet-based frames are marked by italics. Abbreviations: c. = causation, lit. = literature, n.a. = not available.} 
  \centering 
  \label{tab:predictions-results}
  \begin{tabular}{ll>{\RaggedRight\arraybackslash}p{0.24\textwidth}l}
    \lsptoprule
    \multicolumn{2}{l}{Reading} & Expected shift (frame/lit.) &  Findings \\
    \midrule
    \multicolumn{2}{l}{Eventive readings} &&\\
    &change-of-state c. & yes/yes & yes \\
    &change-of-state& yes/yes & yes \\
    &result-state & yes/yes & \HighlightText{yes} \\
    \multicolumn{2}{l}{Participant readings} &&\\
    &adherent & no/no & no \\
    &agent & yes/only collective & \HighlightText{no} \\
    &behavior & no/no & no \\
    &cause & yes/n.a. & yes \\
    &experiencer& no/no & no \\   
    &instrument & yes/yes & \HighlightText{instrument or causer} \\
    &location & no/yes & no \\
    &measure & no/no & no \\
    &path & no/no & no \\
    &patient & yes/only inanimate & \HighlightText{inanimate (subset)} \\
    &product& no/yes & \HighlightText{implicit product (subset)} \\    
    &result & no/yes & \HighlightText{yes (subset)} \\
    \lspbottomrule
  \end{tabular}
\end{table}   

As in previous literature, I systematically found \textsc{event, result-state, result, product, instrument} and [−animate] \textsc{patient} readings. 
Of these, \textsc{result} and \textsc{product} are not part of the VerbNet-based frames. 
Shifts to \textsc{location} and \textsc{agentive-collective} are discussed in the literature but are not represented in my data. This indicates that it was correct to assume that these attributes are not part of the base verb semantics to begin with.

\begin{sloppypar}
In this study, I have identified some semantic details which have, to my knowledge, not been addressed in the literature so far. 
First, the eventive categories can be differentiated further than has been done in the literature to date. As suggested by the frame structure, I found \textsc{change-of-state causation} and \textsc{change-of-state} readings as well as different kinds of \textsc{causes} such as \textsc{actions} and \textsc{states}. 
Second, as to the \textsc{product} category, I have argued that a distinction between \textsc{implicit products} and \textsc{explicit products} is sensible. Only \textsc{implicit products} are attested in my \textit{-ment} nominalizations.
Finally, I have found that the category \textsc{causer} is needed in addition to \textsc{instrument} in order to account for all attested instrument-like readings. 
\end{sloppypar}

Regarding the assumed impossible readings, my nominalizations behave as expected: I did not find the readings [+animate] \textsc{patient, agent, experiencer, path, adherent/follower, behavior,} or \textsc{measure}. \textsc{Experiencer, path, adherent/follower, behavior} and \textsc{measure} are not part of the frame formalizations to begin with, while the impossibility of shifts to \textsc{agent} and [+animate] \textsc{patient} will be formalized below.\footnote{\textsc{Measure}, or a related category such as VerbNet's \textsc{extent} role, will figure in future formalizations of scalar base verbs.}

I have also observed distributions of possible readings which are not predicted by the VerbNet-based frames. 
The first observation which I have discussed is the complementary distribution of \textsc{patient, implicit product} and \textsc{result} readings. The pattern can be summarized by two principles: 
First, \textsc{implicit product} and \textsc{result} can be shifted to if the base verb has the corresponding node, respectively.
Second, if a base verb has neither an \textsc{implicit product} nor a \textsc{result} participant, the \textsc{patient} can be shifted to.
The second observation is that we can find shifts either to \textsc{instrument} or to \textsc{causer}, to both, or to neither of the two, depending on the base verb.  

Apart from the participants of COS-events, I also investigated their event structure. Three findings are central in this respect:
First, I have found that external causes are either expressed or implied in the context of \textsc{change-of-state causation} readings of COS nouns, which confirms the complex event structure applied in the VerbNet-based frames.
Second, as claimed in the literature, the first subevent is most often an action, but can also be any other type of event. This can be concluded both from the subtypes of \textsc{cause} readings in which I found the nominalizations, and from the contexts in which they are attested.
Finally, I looked at the contexts of \textsc{change-of-state} readings and found that whether external causes are present depends on the base verb: For c-only COS verbs, external causes are expressed or implied. For c/i COS verbs, an external cause is most often, but not always, expressed or implied. I have argued that this reflects two different mechanisms: If an external cause is present, we are dealing with a complex event, the reference of which has been shifted to the frame's \textit{change-of-state} node. If no external cause is present, we are dealing with a transposition of the base verb's simple event variant.  

It should always be kept in mind that these findings relate to those readings which are systematically attested, or systematically unattested. However, this does not mean that there are no exceptions: The annotators have not been able to identify every single reading for every type in the data set. All cases of unattested readings which I have not explicitly discussed in this section can be considered gaps in the data. For example, no \textsc{change-of-state} reading was found for \textit{befoulment} and \textit{abridgement}, but it was attested for all other types, and I have not been able to identify a possible motivation behind this other than scarcity of data. 
In \chapref{ch:observations}, I will take a quantitative perspective on gaps in my data set.

On the whole, my findings regarding possible and impossible nominalization semantics show that the VerbNet-based frames do not suffice to predict all possible and impossible readings. In the next section, I will address all necessary modifications to the frame formalization. 

\subsection{Formalization of COS nominalization}
\label{sec:cos-output-formal}

I will now address how the VerbNet-based frames need to be modified in order to reflect the findings with regard to nominalization semantics. 
First, in \sectref{sec:cos-output-formal-frames}, I will introduce a notation for indicating possible readings, revise the participants which have so far been based on the semantic roles given in VerbNet, revisit the event structure, address the issue of representing probabilities in frames, and finally I will tidy up all loose ends by proposing frame formalizations for COS nouns.
Then, as part of the inheritance hierarchy for \textit{-ment}, I will propose an animacy constraint and model the interplay of \textsc{patient, implicit product} and \textsc{result} readings (\sectref{sec:cos-output-formal-constr}).
As a next step, I will update the participant hierarchy and the type signature to include all labels and types required for the frame formalization and the inheritance hierarchy (\sectref{sec:cos-output-formal-TS}).
I conclude this section by proposing a set of nine lexeme formation rules (LFRs) for \textit{-ment} on COS verb bases, embedded in an inheritance hierarchy (\sectref{sec:cos-output-formal-inherit}).

\subsubsection{Frame representations}
\label{sec:cos-output-formal-frames}

\subsubsubsection{Indicating possible readings}
Possible nominalization readings will again be indicated by means of a mathematical set of indices under the AVM. 
As a next step, the reference sets of all nominalization frames can be compared and then translated into an inheritance hierarchy for \textit{-ment}. The complete inheritance hierarchy will be given in the chapter summary in \sectref{sec:cos-summary}. 

\subsubsubsection{Revision of participants}

The participant which can be added most straightforwardly is \textsc{causer}. 
In the participant hierarchy, \textsc{agent} and \textsc{causer} are the only hyponyms of \textsc{actor}, with \textsc{agent} being  [+intentional] and \textsc{causer} being  \mbox{[−intentional]}.  
Therefore, by simply including \textsc{actor}, both \textsc{agent} and \textsc{causer} are covered as possible participants (see e.g. Figure \ref{fig:cosframe-firstsubevent-general} below).

The second participant which needs to be added is \textsc{implicit product}. As argued above, this kind of product is something intrinsically connected to the patient of some verbs. I will use an embroidering-event as an example to illustrate how this can be modeled in frames.
In Figure \ref{fig:implprodframe}, the patient, \textit{pillow}, has an attribute \textsc{on-region}, which represents its surface. In other words, all points located on the pillow are contained in its on-region (see also \citealt[38]{Kallmeyer.2013}).\footnote{My data includes only implicit products which require \textsc{on-region}, like \textit{embroidery}. For modeling implicit products like \textit{hole}, \textsc{in-region} is needed instead.} 
At the same time, the implicit product \textit{embroidery} has an attribute \textsc{region}, which represents the space it occupies. The fact that the embroidery is located on the pillow is expressed by a mereological part-of relation between the values of \textsc{region} and \textsc{on-region}.
This kind of relation between two nodes is not functional, and is therefore not modeled as an attribute. The notation used here, `\textit{part-of} (\avmbox{4}, \avmbox{3}),' was introduced by \citet[35--36]{Kallmeyer.2013}. 

\begin{figure} \singlespacing
  \centering
    \begin{avm}
      \avml  
      \@0  
      \[
      \avmspan{\textit{embroidering-event}}\cr
      \textsc{patient} &\@1 
      \[
      \textit{pillow} \cr
      \textsc{on-region} & \@3 \cr 
      \] \cr
      \textsc{product} & \@2 
      \[
      \avmspan{\textit{embroidery}} \cr
      \textsc{region} & \@4       
      \] \cr
      \] 
      \cr
      {\textsc{ref} $=$ \{\@2\} } \cr
      {\textit{part-of}(\@4, \@3)}  
      \cr
      \avmr 
    \end{avm}
    \caption{Frame for the implicit product \textit{embroidery}}
    \label{fig:implprodframe}
\end{figure} 

In this frame, \textsc{product} is attached directly to the central node \textit{embroidering-event}. This is not satisfying in terms of event decomposition. One way to include \textsc{product} as well as its hyperonym \textsc{result} more appropriately in a complex event frame is to introduce it into the caused subevent. This presents us with the problem that, in the VerbNet-based frames, the caused subevent is a change-of-state, and not a come-into-being event. However, the fact that we do find \textsc{product} and \textsc{result} readings in some nominalizations suggests that such an event is present in the corresponding base verb semantics. I propose to tackle this issue by introducing a complex type `\textit{change-of-state} $\wedge$ \textit{come-into-being}.' This type expresses that there is a change-of-state with an accompanying come-into-being event. Formally, it is a subtype with two parents: \textit{change-of-state} and \textit{come-into-being}. 
Figure \ref{fig:resultframe} integrates the complex type into the \textit{embroidery}-frame from Figure \ref{fig:implprodframe}. We see that the patient \textit{pillow} undergoes a change-of-state which results in an embroidered-state, and at the same time, the embroidery comes into being. With regard to my data set, I propose this kind of frame for \textit{discolor}, \textit{embrittle}, and their respective nominalizations. In addition, \textsc{product} can be exchanged for its hyperonym \textsc{result} when modeling the nominalizations with a possible \textsc{result} reading, and their bases.

Whether speakers conceptualize such a combined event primarily as one of change-of-state or of come-into-being, or whether both are equal, is surely a matter of debate. What is clear from my data at this point is that different contexts can focus on one or the other, and that there is variation between different base verbs. More precisely, a spot check in \acs{iWeb} shows that, for \textit{discolor}, the product is frequently made explicit in the wider context, while for \textit{embrittle} it is largely left implicit. For \textit{embroider}, the product is in all likelihood more cognitively salient in speakers' representations since it can be found easily and frequently in the participant structure of the verb.

\subsubsubsection{Event structure}
Let us now look at the event structure of COS verbs and nouns. 
I have shown that, as expected, the most adequate frame structure for c-only COS verbs is that of a complex event, and that the \textsc{change-of-state} reading comes about by a shift to the change-of-state node. For c/i COS verbs and their nominalizations, the data showed a mixed picture: Contexts with an expressed or implied cause greatly outnumber those without, but the latter are definitely possible. I have therefore concluded that both suggested frame structures seem to be viable. 
As a reminder, the two patterns are given in Figure \ref{fig:othercosverbframereminder}. 

\begin{figure}
    \begin{avm}
      \avml  
      \@0  
      \[
      \avmspan{\textit{embroidering-event}}\cr
      \att{agent} & \@1 \cr
      \att{patient} &\@2 
      \[
      \textit{pillow} \cr
      \att{on-region} & \@6 \cr 
      \] \cr
      \att{product} & \@3 
      \[
      \avmspan{\textit{embroidery}} \cr
      \att{region} & \@7        
      \] \cr
      \att{cause} & \@4  \cr
      \att{effect} & \@5
      \[\avmspan{\type{change-of-state} $\wedge$ \type{come-into-being}}\cr
      \att{patient} & \@2 \cr
      \att{product} & \@3 \cr
      \att{result-state} & \@8 
      \[\avmspan{\type{embroidered-state}} \cr
      \att{patient} & \@2
      \] 
      \]      
      \] \cr
      {\textit{part-of}(\@7, \@6)}  
      \avmr 
    \end{avm}
    \caption{Complex frame for an \textit{embroidering-event}}
    \label{fig:resultframe}
\end{figure}

\begin{figure}
\centering
  \begin{subfigure}[b]{0.4\linewidth}
    \centering \singlespacing
      \begin{avm}
            \avml
            \@0 
            \[
            \avmspan{\textit{c-o-s causation}}\cr
            \textsc{cause} & \@1 
            \cr
            \textsc{effect} & \@2 
            \textit{c-o-s} 
            \] \cr
            {\textsc{ref} $=$ \{\@0, \@2\}}
            \avmr
          \end{avm}
      \caption{Single frame}
  \end{subfigure}%
  \begin{subfigure}[b]{0.6\linewidth}
    \centering \singlespacing
    \begin{avm}
      \avml
      \@0 
      \[
      \avmspan{\textit{c-o-s causation}}\cr
      \textsc{cause} & \@1 
      \cr
      \textsc{effect} & \@2 
      \textit{c-o-s}
      \] \cr 
      {\textsc{ref} $=$ \{\@0\}}
      \avmr 
      \avml
      \cr 
      \ $\longleftrightarrow$ \@2
      \[ \avmspan{\textit{c-o-s}}
      \cr
      \]
      \cr
      \hspace{0.9cm}{\textsc{ref} $=$ \{\@2\} }
      \avmr 
    \end{avm}
      \caption{Embedded subevent}
  \end{subfigure}
  \caption[Frame variants for c/i COS nouns]{Frame variants for c/i COS nouns (e.g. \textit{congealment})}
  \label{fig:othercosverbframereminder}
\end{figure}

Next, let us address the first subevent, \textsc{cause}. In the VerbNet-based frames, \textsc{cause} is typed with \textit{action}, but it is more useful for my purposes to swap the \textit{action} type for the more general \textit{event}, which can then be specified in a given context. Since the participants of the first subevent change substantially depending on the different possible types of event (e.g. \textit{state} versus \textit{action}), they are best left underspecified in the generalized frame, as depicted in Figure \ref{fig:cosframe-firstsubevent-general}. The only specification is that the actor of the causation event also has to be a participant of the causing event. Therefore, \textsc{actor} is co-indexed with \textsc{participant}. 

\begin{figure}
    \begin{avm}
      \avml
      \@0 
      \[
      \avmspan{\textit{change-of-state causation}}\cr
      \textsc{actor} & \@1 \cr
      \textsc{patient} & \@2 \cr
      \textsc{instrument} & \@3 \cr 
      \textsc{cause} & \@4  \[\textit{event} \cr
      \avmspan{\att{participant} \@1} 
      \]
      \cr
      \textsc{effect} & \@5 
      \[ 
      \avmspan{\textit{change-of-state}} \cr 
      \textsc{patient} & \@2 \cr
      \att{result-state} & \@6
      \] 
      \] \cr
      {\textsc{ref} $=$ \{\@0\} }
      \avmr
    \end{avm}
    \caption{Change-of-state causation with an underspecified first subevent}
    \label{fig:cosframe-firstsubevent-general}
\end{figure}

In order to illustrate how fundamentally the first subevent changes over different contexts, I am also including two instantiated frames here. In Figure \ref{fig:cosframe-firstsubevent-instantiated}, the participant structures of the contexts in (\ref{ex:cos-specified1}) and (\ref{ex:cos-specified2}) are modeled.

\begin{exe}
  \ex \label{ex:cos-specified1} The white of the breast feathers was turned a dull gray by the rain's \textbf{bedragglement} {\small(Google FIC forgottenbooks.com 1922)}
  \ex \label{ex:cos-specified2} Well, David Glasner is on fire, another post! I guess a real economist (such as Glasner), after having \textbf{befouled} himself by reading a Wall Street Journal editorial, has to ``take a cleansing tonic'' in the form of reading Hayek. {\small(Google COMM uneasymoney.com 2011)}
\end{exe}

\noindent In frame (a), \textit{rain} as the causer of the bedraggle-event is co-indexed with the theme of the causing subevent since it is the rain's raining which causes the breast feathers to become bedraggled.{\interfootnotelinepenalty=10000\footnote{The semantics of \textit{bedraggle} would standardly call for something like \textit{soggy} as a result-state, but the accompanying change of color is focused on here. Note also that reference is on \avmbox{0}: The frame depicts a bedraggle event, and not the reading of \textit{bedragglement} in this context, which has been classified as a \textsc{cause}.}} 
In frame (b), the causer of the causation event does something to himself by conducting the action in the first subevent. Therefore, \textsc{causer, agent} and \textsc{patient} are co-indexed here.

\begin{figure}
  \begin{subfigure}[b]{0.5\linewidth}
      \begin{avm}
        \avml
        \@0 
        \[
        \avmspan{\textit{bedraggle-event}}\cr
        \textsc{causer} &\@1 \textit{rain} \cr
        \textsc{pat} & \@2 \textit{breast feathers}\cr
        \textsc{cause} & \@3 
        \[ 
        \avmspan{\textit{rain-event}} \cr 
        \textsc{theme} & \@1 \textit{}
        \] \cr
        \textsc{effect} & \@4 
        \[ 
        \avmspan{\textit{change-of-color}} \cr 
        \textsc{pat} & \@2 \cr
        \textsc{ini-st} & \@5 \textit{white} \cr
        \textsc{res-st} & \@6 \textit{gray}
        \] 
        \] \cr
        \avmr
      \end{avm}
      \caption{\textit{Bedraggle-event}, example (\ref{ex:cos-specified1})}
  \end{subfigure}%
  \begin{subfigure}[b]{0.5\linewidth}
      \begin{avm}
        \avml
        \@0 
        \[
        \avmspan{\textit{befoul-event}}\cr
        \textsc{causer} &\@1 \type{David Glasner} \cr
        \textsc{pat} & \@1 \cr
        \textsc{cause} & \@2 
        \[ 
        \avmspan{\textit{read-action}} \cr 
        \att{agent} & \@1 \cr 
        \textsc{theme} & \@4 \type{editorial} 
        \] \cr
        \textsc{effect} & \@3 
        \[ 
        \avmspan{\textit{change-of-state}} \cr 
        \textsc{pat} & \@1 \cr
        \textsc{res-st} & \@5 \textit{befouled}
        \] 
        \] \cr
        \avmr
      \end{avm}
      \caption{\textit{Befoul-event}, example (\ref{ex:cos-specified2})}
  \end{subfigure}
  \caption[Instantiated causation events]{Instantiated causation events. Abbreviations: ini-st = initial state, pat = patient, res-st = result-state.}
  \label{fig:cosframe-firstsubevent-instantiated}
\end{figure}

\subsubsubsection{Representing gradient phenomena in frames}
In the discussion of \textsc{instrument} and \textsc{causer} readings in \sectref{sec:cos-output-survey-instrument}, I have stated that their distribution seems to be related to the frequency with which \textsc{instrument} and \textsc{causer} participants are attested with a given base verb. Representing such a gradient phenomenon in frames poses a challenge. Here, I will address the instrument/causer distribution by means of \HighlightText{stochastic frames}, that is, frames which include information about probabilities. 
Testing my hypothesis by gathering co-occurrence data and then modeling the results in stochastic frames would exceed the scope of this book, which is why I have decided to merely sketch the literature, illustrate the issue in a tentative frame, and leave the rest for future research.
Since dealing with gradience is a fundamental problem in formal approaches, I will return to the issue in the general discussion of this book (\chapref{ch:discussion}).

In research based on Barsalou frames, stochastic frames have recently piqued some researchers' interest (see \citealt{Schuster.inpress} for a discussion of exemplary applications). Since the notion is comparatively new, we do not yet find a unified approach, but rather different starting points. The only available approach which is formally fleshed out is that of \citet{Schuster.inpress}, who
assume that attributes can take probability distributions as values. 
An example is given in Figure \ref{fig:birdframe}, where a frame for \textit{bird} is enriched with probabilistic information. Here, we see that \textit{bird} has two relevant attributes, \textsc{main locomotion} and \textsc{foot structure}. Birds can have three possible types of locomotion, \textit{swim, fly} or \textit{walk}, with \textit{fly} being the most likely. Likewise, a bird's foot structure can be either \textit{clawed} or \textit{webbed}. Note that the probabilities of one cluster of values add up to 1.

\begin{figure}
  \centering \singlespacing
    \begin{avm}
      \@0 
      \[
      \avmspan{\textit{bird}}\cr
      \textsc{main locomotion} & \@1 & \textit{swim} & 0.15 \cr
      & & \textit{fly} & 0.75 \cr
      && \textit{walk} & 0.1 \cr
      \textsc{foot structure} & \@2 & \textit{clawed} & 0.8 \cr
      & & \textit{webbed} & 0.2 \cr
       \] 
    \end{avm}
    \caption[Representing probabilistic information in frames]{Representing probabilistic information in frames (adapted from \citealt[4]{Schuster.inpress})}
    \label{fig:birdframe}
\end{figure}

Importantly, the authors mention that not only values, but also attributes can have a probability (p. 2). Since I want to model how likely \textsc{instrument} and \textsc{causer} participants are with a given base verb, this is what I need. Thus, the frequency with which COS verbs are attested with \textsc{instrument} and \textsc{causer} can be translated into probabilities, and cutoff points can be identified empirically and then defined in the inheritance hierarchy. This is illustrated with devised numbers in the tentative frames and inheritance hierarchy in Figures \ref{fig:frame-weighted} and \ref{fig:inheritance-probability}, respectively.
In the three depicted frames, we find probabilities for \textsc{instrument} and \textsc{causer} participants for the base verbs \textit{abridge} (equal distribution), \textit{congeal} (high/low) and \textit{discolor} (low/high). 
With a cutoff point of 0.5 defined in the inheritance hierarchy ($\ge 0.5$), we correspondingly find shifts to both \textsc{instrument} and \textsc{causer} for \textit{abridge}, to \textsc{instrument} for \textit{congeal}, and to \textsc{causer} for \textit{discolor}. 

\begin{figure}
	\begin{subfigure}{0.3\linewidth}
      \begin{avm}
        \avml
        \@0 
        \[
        \avmspan{\textit{abridge-event}}\cr
        \textsc{instr}    & 0.5 &\@1 \cr
        \textsc{causer}   & 0.5 &\@2 \cr
        \] \cr
        {\textsc{ref} $=$ \{\@0\} }
        \avmr
      \end{avm}
	\caption{\label{fig:frame-weighted1}\textit{abridge}}
	\end{subfigure}%
	\begin{subfigure}{0.3\linewidth}
     \begin{avm}
        \avml
        \@0 
        \[
        \avmspan{\textit{congeal-event}}\cr
        \textsc{instr}    & 0.6 &\@1 \cr
        \textsc{causer}   & 0.4 &\@2 \cr
        \] \cr
        {\textsc{ref} $=$ \{\@0\} }
        \avmr
      \end{avm}
	\caption{\label{fig:frame-weighted2}\textit{congeal}}
\end{subfigure}%
	\begin{subfigure}{0.3\linewidth}
      \begin{avm}
        \avml
        \@0 
        \[
        \avmspan{\textit{discolor-event}}\cr
        \textsc{instr}    & 0.2 &\@1 \cr
        \textsc{causer}   & 0.8 &\@2 \cr
        \] \cr
        {\textsc{ref} $=$ \{\@0\} }
    \avmr
      \end{avm}
		\caption{\label{fig:frame-weighted3}\textit{discolor}}
	\end{subfigure}%
  \caption[Weighted participant attributes in tentative frames]{\label{fig:frame-weighted}Weighted participant attributes in tentative frames. Abbreviations: instr = instrument.}
\end{figure}

\begin{figure}
  \footnotesize \centering \singlespacing
  \begin{tikzpicture}
  \draw (-6,6) node(0){lexeme};
  \draw (-6,5) node(1){\textit{v-n-lfr}};
  \draw (-8,4) node(2){\Boxed{\textsc{phon}}};
  \draw (-2.5,4) node(3){\Boxed{\textsc{sem}}};
  \draw (-8,2) node(4){
    \begin{avm}
    \[{\textit{x-ment}}\cr
    \textsc{ph} \@1+ment\cr
    \textsc{m-base} \[\textsc{ph} \@1 \]
    \] 
    \end{avm}};
  \draw (-4.5,2) node(6){
    \begin{avm}
    \[\textit{instr-n}\cr
    \textsc{ref} \@x\cr
    \textsc{m-base} \[ \textsc{sem} \[ \textsc{instr} $\ge$ 0.5 \@x \] \]
    \]
    \end{avm}};
  \draw (0,2) node(5){
    \begin{avm}
    \[\textit{causer-n}\cr
    \textsc{ref} \@x\cr
    \textsc{m-base} \[ \textsc{sem} \[ \textsc{causer} $\ge$ 0.5 \@x \] \]
    \] 
    \end{avm}};
  \draw (-7,0.0) node(9){\begin{tabular}{l}
    abridgement\\congealment
    \end{tabular}};
  \draw (-2.5,0.0) node(8){\begin{tabular}{l}
    abridgement\\discolorment
    \end{tabular}};
  
  \draw [dotted] (0) -- (1);
  \draw (1.south) -- (2.north);
  \draw (1.south) -- (3.north);
  \draw (3.south) -- (6.north);
  \draw (3.south) -- (5.north);
  \draw (2.south) -- (4.north);
  \draw (8.north) -- (4.south);
  \draw (8.north) -- (5.south);
  \draw (9.north) -- (4.south);
  \draw (9.north) -- (6.south);
  \end{tikzpicture}
  \caption[Introducing probabilities into the inheritance hierarchy]{Introducing probabilities into the inheritance hierarchy of lexical rules for the suffix \textit{-ment} (partial hierarchy). Abbreviations: instr = instrument.}
  \label{fig:inheritance-probability}
\end{figure}

However, obtaining feasible probabilities is not trivial (see e.g. \citealt[201]{Petersen.2014}). 
Fundamental problems arise mainly with regard to gathering reliable data and to capturing the interdependence of different participants in absolute or relative probabilities (e.g. instruments depend on agents, and causers are mutually exclusive with agents). 

Therefore, although stochastic frames are a promising approach to tackling gradient linguistic phenomena, I will not use them in this study. Rather, I will translate the assumed gradient distribution into a binary decision: An attribute is either present, or not. To stay with the three verbs used for illustration in \figref{fig:frame-weighted}: I am assuming that the frame for \textit{congeal} has an \textsc{instrument} attribute, the frame for \textit{discolor} has a \textsc{causer}, and the frame for \textit{abridge} has both. \figref{fig:frame-nonweighted} depicts how \figref{fig:frame-weighted} can be modified accordingly.

\begin{figure}
	\begin{subfigure}[b]{0.3\linewidth}
		\centering  \singlespacing
      \begin{avm}
        \avml
        \@0 
        \[
        \avmspan{\textit{abridge-event}}\cr
        \textsc{actor}    & \@1 \cr
        \textsc{patient}  & \@2 \cr
        \textsc{instr}    & \@3 \cr
        \textsc{causer}   & \@4 \cr
        \textsc{cause}    & \@5 
        \cr
        \textsc{effect}   & \@6 
        \] \cr
        {\textsc{ref} $=$ \{\@0\} }
        \avmr
      \end{avm}
	\caption{\label{fig:frame-nonweighted1}\textit{abridge}}
	\end{subfigure}%
	\begin{subfigure}[b]{0.3\linewidth}
	\centering \singlespacing
     \begin{avm}
        \avml
        \@0 
        \[
        \avmspan{\textit{congeal-event}}\cr
        \textsc{actor}    & \@1 \cr
        \textsc{patient}  & \@2 \cr
        \textsc{instr}    & \@3 \cr
        \textsc{cause}    & \@5 
        \cr
        \textsc{effect}   & \@6 
        \] \cr
        {\textsc{ref} $=$ \{\@0\} }
        \avmr
      \end{avm}
	\caption{\label{fig:nonweighted2}\textit{congeal}}
\end{subfigure}%
	\begin{subfigure}[b]{0.3\linewidth}
		\centering  \singlespacing
      \begin{avm}
        \avml
        \@0 
        \[
        \avmspan{\textit{discolor-event}}\cr
        \textsc{actor}    & \@1 \cr
        \textsc{patient}  & \@2 \cr
        \textsc{causer}   & \@4 \cr
        \textsc{cause}    & \@5 
        \cr
        \textsc{effect}   & \@6 
        \] \cr
        {\textsc{ref} $=$ \{\@0\} }
    \avmr
      \end{avm}
		\caption{\label{fig:nonweighted3}\textit{discolor}}
	\end{subfigure}%
  \caption[Revised frames for \textit{abridge, congeal,} and \textit{discolor}]{\label{fig:frame-nonweighted} Revised frames for \textit{abridge, congeal,} and \textit{discolor}. Abbreviations: instr = instrument.}
\end{figure}

\subsubsubsection{COS-noun frames}
The frames as introduced in \sectref{sec:cos-input-frames} have to be revised from the ground up. Looking at the subclasses of COS verbs, it becomes obvious that the VerbNet-based frames do not predict the patterns which I found in my nominalizations:
The three subclasses are distinguished by type of causation, type of change, selectional restrictions of the \textsc{patient} and \textsc{instrument} participants, and the presence or absence of a secondary result participant. My findings with regard to possible nominalization readings, however, show that relevant differences are the presence or absence of \textsc{instrument, causer, implicit product} and (primary) \textsc{result}. 
I will now first present a frame which generalizes over all COS verbs in my data set, and then discuss three exemplary COS noun frames addressing the patterns I found in my data. For ease of comparison, the indices are kept constant over all remaining frames in this section.\largerpage

The generalized COS verb frame in \figref{fig:cosnounframe} incorporates all formal decisions I have discussed in this section: 
First, the causing entity is an actor and can thus be instantiated either as an agent or as a causer. 
Second, I added the participant attributes \textsc{result} and \textsc{implicit product} (the latter being shorthand for the frame notation introduced in \figref{fig:implprodframe}, \sectref{sec:cos-output-formal-frames}). 
These two attributes, along with \textsc{instrument}, are optional, as defined in the type signature. 
Third, the first subevent is typed with the most general eventive type \textit{event} and the most general participant \textsc{participant}.
Fourth, the second subevent is typed as \textit{change-of-state}; it can be specified as its subtype \textit{change-of-state} $\wedge$ \textit{come-into-being} if required.

\begin{figure} 
  \centering \setstretch{0,95} 
    \begin{avm}
      \avml
      \@0 
      \[
      \avmspan{\type{change-of-state causation}}\cr
      \textsc{actor} & \@1 \cr
      \att{patient} & \@2 \cr
      \att{instrument} & \@3 \cr
      \att{result} & \@4 \cr
      \att{implicit product} & \@5 \cr
      \att{cause} &   \@6 \[\type{event} \cr
      \att{participant} & \@1 \cr
      \]\cr
      \att{effect} & \@7 
      \[ 
      \type{change-of-state} \cr
      \att{patient} & \@2 \cr
      \att{result-state} & \@8 
      \[ 
      \type{state} \cr
      \att{patient} & \@2\cr
      \] \cr
      \] \cr
      \] \cr
      {\textsc{ref} $=$ \{\@0 [\@7]\}}
      \avmr
    \end{avm}
  \caption{Generalized lexical frame for COS verbs}
  \label{fig:cosnounframe}
\end{figure}

Two things should be noted in this generalized frame with regard to optionality. 
First, in the set indicating reference I have included \avmbox{7} in square brackets, representing that not all base verbs have an inchoative variant. More accurately, these should be represented by a second,  almost identical frame with a reference set of \avmbox{1}. 
Second, \textsc{instrument}, \textsc{implicit product} and \textsc{result} are not relevant for all types of causation event. This is modeled in the type signature (see \figref{fig:signature-update} in \sectref{sec:cos-output-formal-TS}).

Let us move on to more specific semantics. The types in my data set can be organized into ten groups of verbs and corresponding nouns, based on the distribution of \textsc{instrument} versus \textsc{causer} readings and of \textsc{patient} versus \textsc{implicit product} versus \textsc{result} readings in the nominalizations. Of these ten groups, six have only one member, as shown in Table \ref{tab:overview-distributions}. Since I do not think that showing highly similar frames for all ten groupings has much added value over showing just a few representative frames, I am opting for the latter. 
\begin{table}
  \centering 
    \caption[Groupings of \textsc{instrument/causer} and \textsc{implicit product/patient/result} distributions]{\label{tab:overview-distributions}Groupings of \textsc{instrument/causer} and \textsc{patient/implicit product/result} distributions. Abbreviations: impl. prod. = implicit product.}
    \begin{tabular}{p{2,7cm}<{\raggedright} p{4cm}<{\raggedright} p{4,5cm}<{\raggedright}}
      \lsptoprule
      Instrument/causer & Patient/impl. prod./result &  Nominalizations \\
      \midrule
       not attested & not attested & disbandment \\
      not attested & result & decenterment  \\
       causer & not attested & upliftment  \\
      causer & result & diminishment, increasement,   \\
      && worsenment\\
       causer & implicit product & discolorment  \\
      instrument & result & dispersement, progressment  \\
       instrument & patient  & congealment \\
      both & implicit product & embrittlement \\
       both & patient  & bedragglement, befoulment,    \\
       &&  debauchment, unfoldment \\
      both & result & abridgement, besmirchment,  \\
      && embetterment\\
      \lspbottomrule
    \end{tabular}
\end{table}

The frames in Figures \ref{fig:nounframe-instrpat} to \ref{fig:nounframe-causimpl} represent three exemplary patterns, namely nominalizations which have \textsc{instrument} and \textsc{patient} readings (\textit{congealment}), those which have \textsc{instrument, causer} and \textsc{result} readings (\textit{abridgement, besmirchment, embetterment}), and those which have \textsc{causer} and \textsc{implicit product} readings (\textit{discolorment}).
As can be seen in \figref{fig:nounframe-instrpat}, instrument/patient-COS nouns have an \textsc{agent}, and no \textsc{result} or \textsc{implicit product}. Apart from the underspecified first subevent and the possible referents, the frame is identical to the VerbNet-based frames.

\begin{figure} 
  \centering \setstretch{0.9}
    \begin{avm}
      \avml
      \@0 
      \[
      \avmspan{\textit{change-of-state causation}}\cr
      \textsc{agent} & \@1 \cr
      \textsc{patient} & \@2 \cr
      \textsc{instrument} & \@3 \cr
      \textsc{cause} & \@6 \[\textit{event}\cr \att{participant} & \@1 \] \cr
      \textsc{effect} & \@7 
      \[ 
      \textit{change-of-state} \cr
      \textsc{patient} & \@2 \cr
      \textsc{result-state} & \@8 
      \[ 
      \textit{state} \cr
      \textsc{patient} & \@2\cr
      \] \cr
      \] \cr
      \] \cr
      {\textsc{ref} $=$ \{\@0, \@2, \@3, \@6, \@7, \@8 \}}
      \avmr
    \end{avm}
  \caption[Frame for instrument/patient-COS nouns]{Frame for instrument/patient-COS nouns (e.g. \textit{congealment})}
  \label{fig:nounframe-instrpat}
\end{figure} 

In \figref{fig:nounframe-instrcausres}, which shows the formalization for instrument/causer/result-COS nouns, a \textsc{result} attribute is added to the frame, and the \textsc{agent} attribute is replaced by its hyperonym \textsc{actor}, allowing for instantiation by either of its subtypes. 
If \textsc{actor} is instantiated by \textsc{agent}, a shift to \avmbox{1} is precluded by the animacy constraint in the inheritance hierarchy. If, however, it is instantiated by \att{causer}, a shift to \avmbox{1} is possible. In addition, the inheritance hierarchy prevents a shift to \avmbox{2}, that is, a \textsc{patient} reading.

\begin{figure}
    \begin{avm}
      \avml
      \@0 
      \[
      \avmspan{\type{change-of-state causation $\wedge$ come-into-being causation}}\cr
      \textsc{actor} & \@1 \cr
      \att{patient} & \@2 \cr
      \att{instrument} & \@3 \cr
      \att{result} & \@4 \cr
      \att{cause} & \@6 \[\textit{event}\cr \att{participant} \@1 \] \cr
      \att{effect} & \@7 
      \[ 
      \avmspan{\type{change-of-state} $\wedge$ \type{come-into-being}}\cr
      \att{patient} & \@2 \cr
      \att{result} & \@4 \cr
      \att{result-state} & \@8 
      \[ 
      \type{state} \cr
      \att{patient} & \@2\cr
      \] \cr
      \] \cr
      \] \cr
      {\textsc{ref} $=$ \{\@0, \@1, \@3, \@4, \@6, \@7, \@8\} }
      \avmr
    \end{avm}
    \caption[Frame for instrument/causer/result-COS nouns]{Frame for instrument/causer/result-COS nouns (e.g. \textit{abridgement, besmirchment, embetterment})}
    \label{fig:nounframe-instrcausres}
\end{figure}

Finally, \figref{fig:nounframe-causimpl} models causer/implicit-product-COS nouns. The  \textsc{result} attribute is now replaced by its subtype \textsc{product}, and further attributes and specifications that are required to model the implicit product participant are added; That is, \textsc{region} is attached to the \textsc{product} entity, \textsc{on-region} is attached to the \textsc{patient} entity, and both are related to one another by \textit{part-of}(\avmbox{10},\avmbox{9}).


\begin{figure} 
    \begin{avm}
      \avml
      \@0 
      \[
      \avmspan{\textit{change-of-state causation $\wedge$ come-into-being causation}}\cr
      \textsc{actor} & \@1 \cr
      \textsc{patient} & \@2 
      \[\type{entity}\cr 
      \att{on-region} & \@9
      \] \cr
      \textsc{instrument} & \@3 \cr
       \textsc{product} & \@5 
      \[\type{entity}\cr 
      \att{region} & \@{10}
      \] \cr
      \textsc{cause} & \@6 \[\textit{event}\cr \att{participant} \@1 \] \cr
      \textsc{effect} & \@7 
      \[ 
      \avmspan{\type{change-of-state} $\wedge$ \type{come-into-being\textsubscript{ip}}}\cr
      \textsc{patient} & \@2 \cr
      \att{product} & \@5 \cr
      \textsc{result-state} & \@8 
      \[ 
      \textit{state} \cr
      \textsc{patient} & \@2\cr
      \] \cr
      \] \cr
      \] \cr
      {\textsc{ref} $=$ \{\@0, \@1, \@5, \@6, \@7, \@8\} }\:; 
      {\textit{part-of}(\@{10}, \@9)}\cr
      \avmr
    \end{avm}
  \caption[Frame for causer/implicit-product-COS nouns]{Frame for causer/implicit-product-COS nouns (e.g. \textit{discolorment})}
  \label{fig:nounframe-causimpl}
\end{figure} 

\subsubsection{Constraints}
\label{sec:cos-output-formal-constr}

\subsubsubsection{Animacy constraint}
I have argued that some of the impossible readings (more precisely, shifts to \textsc{agent} and to [+animate] \textsc{patient}) are prevented because \mbox{\textit{-ment}} systematically does not produce animate readings. I propose to formalize this with an animacy constraint which posits that, with regard to non-eventive readings, reference can only be on inanimate event participants. 
In the framework applied here, the easiest way to model the constraint is by directly incorporating it into the inheritance hierarchy instead of specifying it for each nominalization frame. Specifically, it can be introduced as a supertype of the LFRs
(see \figref{fig:inheritance-animacy}). The notation ``\textsc{animacy :} \textit{false}'' allows shifts to inanimate participants (e.g. [−animate] patients) as well as shifts to eventive nodes, for which animacy is not a relevant parameter in the first place. 
This only works, however, if information about animacy is included in the type signature. It will be updated accordingly in \sectref{sec:cos-output-formal-TS}. 

\begin{figure}
  \resizebox{\linewidth}{!}{%
  \begin{tikzpicture}
  \draw (-4,6) node(0){lexeme};
  \draw (-4,5) node(1){\textit{v-n-lfr}};
  \draw (-8,4) node(2){\Boxed{\textsc{phon}}};
  \draw (-0.5,4) node(3){\Boxed{\textsc{sem}}};
  \draw (-0.5,3) node(10){\textsc{animacy : }false};
  \draw (-8,1) node(4){
    \begin{avm}
    \[{\textit{x-ment}}\cr
    \textsc{ph} \@1+ment\cr
    \textsc{m-base} \[\textsc{ph} \@1 \]
    \] 
    \end{avm}};
  \draw (-3,1) node(6){
    \begin{avm}
    \[\textit{cause-n}\cr
    \textsc{ref} \@x\cr
    \textsc{m-base} \[ \textsc{sem} \[ \textsc{cause} \@x \] \]
    \]
    \end{avm}};
  \draw (2.5,1) node(5){
    \begin{avm}
    \[\textit{patient-n}\cr
    \textsc{ref} \@x\cr
    \textsc{m-base} \[ \textsc{sem} \[ \textsc{patient} \@x \] \]
    \] 
    \end{avm}};
  \draw (-5,-1.5) node(9){\begin{tabular}{l}
    bedragglement\\besmirchment
    \end{tabular}};
  \draw (-0.5,-1.5) node(8){\begin{tabular}{l}
    bedragglement
    \end{tabular}};
  
  \draw [dotted] (0) -- (1);
  \draw (1.south) -- (2.north);
  \draw (1.south) -- (3.north);
  \draw (2.south) -- (4.north);
  \draw (10.south) -- (5.north); 
  \draw (10.south) -- (6.north); 
  \draw (3.south) -- (10.north); 
  \draw (8.north) -- (4.south);
  \draw (8.north) -- (5.south);
  \draw (9.north) -- (4.south);
  \draw (9.north) -- (6.south);
  \end{tikzpicture}
  \caption[Introducing an animacy constraint into the inheritance hierarchy]{Introducing an animacy constraint into the inheritance hierarchy (partial hierarchy)}
  \label{fig:inheritance-animacy}
}
\end{figure}

Two issues need to be addressed with regard to the animacy constraint. 
First, the constraint as formulated in \figref{fig:inheritance-animacy} does not capture the fact that \textsc{agen\-tive-col\-lec\-tive} used to be a productive reading of \textit{-ment} derivatives. Should this be desired, for instance in a diachronic study, the constraint can be modified accordingly. 
Second, the constraint only relates to referential shifts brought about by derivation. It does not preclude post-lexical phenomena, more precisely, that the context may coerce the \textit{-ment} derivative into an animate reading.\footnote{One attempt to model coercion in frames can be found in \citet{Babonnaud.2016}, where frames are combined with LTAG and Hybrid Logic.}
 
\subsubsubsection{Patient, implicit product, and result}
Next, we need to model that shifts to a \textsc{patient} reading are only possible if neither a \textsc{result} nor an \textsc{implicit product} participant are present in the frame. Like the animacy constraint, this constraint is specified in the inheritance hierarchy, but since only one reading is affected, I do not introduce a supertype but rather include the information directly in the AVM in question, namely the AVM representing \textsc{patient} nouns (the bottom AVM in \figref{fig:inheritance-depend}).\footnote{To save space, I have flipped the usual depiction of an inheritance hierarchy by 90$^{\circ}$, to be read from left to right instead of from top to bottom.} 
Here, I use the logical negation symbol from Boolean algebra, `$\neg$,' as a negation operator on the attribute \textsc{result} (see also \citealt{Andreou.2017}, who uses this operator to negate values in a frame). The notation `$\neg$\textsc{result}' thus indicates that a shift to \textsc{patient} is possible if the frame contains only attributes that are incompatible with \textsc{result} and its subtypes. 
Shifts to \textsc{result} and \textsc{implicit product}, on the other hand, are possible as soon as the corresponding attributes are present in the base verb frame, which is why the corresponding AVMs for \textsc{implicit product} and \textsc{result} nouns can straightforwardly be included in the inheritance hierarchy. 

\begin{figure}
	\resizebox{\linewidth}{!}{%
	\begin{tikzpicture}
	\draw (2,3) node(0){lexeme};
	\draw (2,2) node(1){\textit{v-n-lfr}};
	\draw (4,4) node(2){\Boxed{\textsc{phon}}};
	\draw (3.9,0) node(3){\Boxed{\textsc{sem}}};
	\draw (10,5) node(5){}; 
	\draw (5.5,5) node(55)[anchor=west]{
		\begin{avm}
		\[{\textit{x-ment}}\cr
			\textsc{ph} \@1+ment\cr
			\textsc{m-base} \[\textsc{ph} \@1 \]
			\]
		\end{avm}};
	\draw (55.east) -- (5.east); 
	\draw (5.5,2.5) node(11)[anchor=west]{
		    \begin{avm}
		    \[\textit{result-n}\cr
		    \textsc{ref} \@x\cr
		    \textsc{m-base} \[ \textsc{sem} \[ \textsc{result} \@x \] \]
		    \]
		    \end{avm}};
	\draw (5.5,0) node(12)[anchor=west]{
		\begin{avm}
		    \[\textit{impl-prod-n}\cr
		    \textsc{ref} \@x\cr
		    \textsc{m-base} \[ \textsc{sem} \[ \textsc{impl-prod} \@x \] \]
		    \]
		    \end{avm}};
	\draw (5.5,-2.75) node(13)[anchor=west]{
		\begin{avm}
		    \[\textit{patient-n}\cr
		    \textsc{ref} \@x\cr
		    \textsc{m-base} \[ \textsc{sem} \[ \textsc{patient} \@x \cr $\neg$\textsc{result} \] \]
		    \] 
		    \end{avm}};
	\draw (13,2.5) node(21){\begin{tabular}{l}
		decenterment
		\end{tabular}};
	\draw (13,0) node(22){\begin{tabular}{l}
		discolorment
		\end{tabular}};
	\draw (13,-2.5) node(23){\begin{tabular}{l}
		bedragglement
		\end{tabular}};
	\draw [dotted] (0) -- (1);
	\draw (1.east) -- (2.west); 
	\draw (1.east) -- (3.west); 
	\draw (2.east) -- (55.west); 
	\draw (3.east) -- (11.west); 
	\draw (3.east) -- (12.west); 
	\draw (3.east) -- (13.west); 
	\draw (11.east) -- (21.west); 
	\draw (12.east) -- (22.west); 
	\draw (13.east) -- (23.west); 
	\draw (5.east) -- (21.west); 
	\draw (5.east) -- (22.west); 
	\draw (5.east) -- (23.west); 
	\end{tikzpicture}
	\caption[Introducing the interplay of \textsc{patient, implicit product} and \textsc{result} into the inheritance hierarchy]{Introducing the interplay of \textsc{patient, implicit product} and \textsc{result} into the inheritance hierarchy (partial hierarchy). Abbreviations: impl-prod = implicit product.}
	\label{fig:inheritance-depend}
}
\end{figure}

\subsubsection{Updated semantic categories and type signature}
\label{sec:cos-output-formal-TS}

Let us now revisit the semantic categories and the type signature introduced in \sectref{sec:meth-Nsem-semcat}. First, the \textsc{implicit product} category needs to be added. Then, I will turn to the attributes needed to model it, namely \textsc{region} and \textsc{on-region}. I will also include information about animacy in the type signature, so that the animacy constraint modeled in \figref{fig:inheritance-animacy} can take effect. The section is concluded by an updated type signature in \figref{fig:signature-update}.

In Table \ref{tab:participant-defs-updated} and \figref{fig:participants2}, \textsc{implicit product} is added to the list of semantic role definitions and to the participant hierarchy, respectively.\footnote{Note that \textsc{product} already exists in VerbNet as a [+concrete] subtype of \textsc{result}; I had not included it because in VerbNet it is not a core role for the verb classes under investigation.} For brevity, I am only including the definitions for the relevant branch of semantic categories, i.e., \textsc{place} and its hyponyms. 

\begin{table}[p]
    \caption[Introducing \textsc{implicit product} into the list of participants]{\label{tab:participant-defs-updated} Introducing \textsc{implicit product} into the list of participants. Revised section of Table \ref{tab:participant-defs}. Relevant core participants are indicated by italics.}
    \begin{tabular}{l>{\RaggedRight\arraybackslash}p{0.7\textwidth}}
      \lsptoprule
      Category & Definition \\
      \midrule
       Place & The state in which an entity exists \\ 
       Goal & Place that is the end point of an action and that exists independently of the event \\ 
       \HighlightText{Result} & An outcome that comes into existence through the event \\ 
       Product & Result that is a concrete object \\
       \HighlightText{Implicit product} & A product that is inherently related to the patient \\
      \lspbottomrule 
    \end{tabular} 
\end{table}

\begin{figure}[p]
    \begin{forest} 
      [Participants 
        [Actor
          [\HighlightText{Agent}\\{[+intentional]}
          ]
          [Causer\\{[−intentional]}
            [\HighlightText{Stimulus}
            ]
          ]
        ]     
        [Undergoer
          [\HighlightText{Patient}\\{[+affected]}
            [\HighlightText{Experiencer}\\{[+awareness]}
            ]
          ]
          [\HighlightText{Instrument}
          ]
        ]
        [Place
          [Goal
            [\HighlightText{Result}
              [\textcolor{lsRichGreen}{Product}\\\textcolor{lsRichGreen}{{[+concrete]}}
                [\textcolor{lsRichGreen}{\HighlightText{Implicit product}}
                ]
              ]
            ]
          ]
        ]
      ]
    \end{forest}
    \caption[Participant categories for semantic coding]{\label{fig:participants2}Participant categories for semantic coding (revised version of Figure \ref{fig:participants}; changes are indicated in green).}
\end{figure}

\begin{figure}
    \begin{forest} for tree={align=center}
      [\textsf{T}
        [\textit{event}
          [\textit{change-of-state}\\\textsc{patient :} \textit{entity}]
        ]
          [\textit{entity}\\\textsc{animacy :} \textit{true} $\veebar$ \textit{false}
          [\textit{abstract entity}]
            [\textit{concrete entity}\\\textsc{region :} \textit{enclosed volume}\\\textsc{on-region :} \textit{surface}
            [\textit{person}\\\textsc{animacy :} \textit{true}]
            [\textit{object}\\\textsc{animacy :} \textit{false}]
            [\textit{place}\\\textsc{animacy :} \textit{false}]
            ]
          ]
      ]  
    \end{forest}
    \caption[Introducing \textsc{region, on-region} and \textsc{animacy} into the type signature]{\label{fig:signature-region} Introducing \textsc{region, on-region} and \textsc{animacy} into the type signature (tentative)}
\end{figure}

So far, I have only depicted eventive types in the type signature. I will now add non-eventive types in order to introduce the attributes \textsc{region, on-region} and \textsc{animacy}. 
They describe properties of an \HighlightText{entity} with physical existence, namely the space it occupies (\textsc{region}), its surface (\textsc{on-region}), and whether it is animate or not (\textsc{animacy}). Therefore, they are not included in the type signature as independent types, but as constraints on entity types.
Creating a type signature of entity types is extremely complex, opening up all sorts of problems and demanding meaningful decisions which are unnecessary in this study. Therefore, I am only giving a tentative signature in Figure \ref{fig:signature-region}, without any claim for it to be the absolute truth.
For my purposes, the most straightforward solution is to split \textit{entity} into the two types \textit{abstract entity} and \textit{concrete entity}. \textit{Concrete entity} then introduces the attributes \textsc{region} and \textsc{on-region}.\footnote{Alternatively, \textsc{region} and \textsc{on-region} can be introduced directly under {\sffamily{T}} alongside a bi-implicational constraint `\textsc{concreteness :} \textit{true} $\leftrightarrow$ \textsc{region} $\wedge$ \textsc{on-region}' (see \citealt[8]{Gamerschlag.2014a}), specifying that all concrete entities have the attributes \textsc{region} and \textsc{on-region}.} As value labels I use the geometrical terms \textit{enclosed volume} and \textit{surface}.

\textsc{Animacy} is introduced under \textit{entity}, allowing both concrete and abstract entities to be animate (e.g. \textit{bird} and \textit{flock of birds}).
The value of \textsc{animacy} is specified further down in the type signature  \textendash{}  here exemplarily by \textit{person, object} and \textit{place}. 
How does the animacy constraint as modeled in Figure \ref{fig:inheritance-animacy} operate during, say, the creation of a \textsc{patient} reading? In the type signature, the \textsc{patient} attribute is introduced by \textit{change-of-state}. Its value can be any entity. If the value is specified as a \textit{person}, a \textsc{patient} reading is blocked for \textit{-ment}, since \textit{person} is specified as [+animate] in the type signature (\textsc{animacy:} \textit{true}). If, however, the value of \textsc{patient} is specified as an \textit{object}, the animacy constraint allows a \textsc{patient} reading.

Let us now revisit the type signature. In Figure \ref{fig:signature-update}, I am only including information which is relevant for this chapter, leaving out types specific to psych nominalization. We find the following changes (from left to right): 
First, the values which can be taken by the participant attributes now contain information about animacy (color-coded in the figure as green). For instance, agents are [+animate] entities (\textsc{agent :} \textit{entity}, \textsc{animacy :} \textit{true}), instruments are inanimate (\textsc{instrument :} \textit{entity}, \textsc{animacy :} \textit{false}), and patients are underspecified (\textsc{patient :} \textit{entity}). For the sake of space, these specifications are only spelled out in the leftmost appearance of a given attribute.
Second, I have introduced six new event types (color-coded as purple) to account for \textsc{result} and \textsc{implicit product} readings: The event type \textit{come-into-being} as well as its daughter \textit{come-into-being}\textsubscript{ip}, the corresponding causation event types \textit{come-into-being causation} and \textit{come-into-being\textsubscript{ip} causation}, and the multi-parent event types \textit{change-of-state} $\wedge$ \textit{come-into-being} and \textit{change-of-state} $\wedge$ \textit{come-into-being}\textsubscript{ip}.\footnote{For reasons of space, I am leaving out some event types: the causation events which have the multi-parent event types as second subevent, and all event types which would be needed to model \textsc{product} (as a daughter to \textsc{result} and a parent to \textsc{implicit product}).} 
Third, I have revised three participants (color-coded as violet): \textsc{result} and \textsc{implicit product} (as shorthand for the notation introduced in Figure \ref{fig:implprodframe}) were added, and \textsc{actor} replaces \textsc{agent} in the type \textit{change-of-state causation}.\footnote{Simply deleting the \textsc{agent} appropriateness condition would have sufficed here, but I am including \textsc{actor} to visualize the change.}
Finally, I have added \textit{entity} as a sister node to \textit{event} (color-coded as orange). For the sake of space, the subtypes of \textit{entity} as introduced in the tentative type signature in Figure \ref{fig:signature-region} are not repeated here.

\begin{sidewaysfigure}
    \resizebox{\linewidth}{!}{  
      \begin{forest} for tree={l sep=1cm}
        [\normalfont{\sffamily{T}}
        [\textit{event}
        [\textit{action}
          \\\textsc{act} $\doteq$ \textsc{ag :} \textcolor{lsRichGreen}{\textit{ent}{,} \textsc{anim :} \textit{true}}
          \\ $\veebar$ \textsc{act} $\doteq$ \textsc{causer :} \textcolor{lsRichGreen}{\textit{ent}{}}
          \\ $\pm$\textsc{instr :} \textcolor{lsRichGreen}{\textit{ent}{,} \textsc{anim :} \textit{false}   }
          \\ $\pm$\textsc{pat: } \textcolor{lsRichGreen}{\textit{ent}{}}
        ]
        [\textit{state}\\\textsc{pat}
        [\textit{having-form}]
        ]
        [\textit{c-o-s}\\\textsc{pat}\\\textsc{res-st} : \textit{state}, name=cos
        [\textit{c-o-phys-form}\\\textsc{pat}\\\textsc{res-st :} \textit{having-form}]
        ]
          [\textcolor{purple}{\textit{come-into-being}}\\\textsc{result :} \textcolor{lsRichGreen}{\textit{entity}{,} \textsc{anim :} \textit{false}}
            [\textcolor{purple}{\textit{c-o-s} $\wedge$ \textit{c-i-b}}, name=coscib
            ]
            [\textcolor{purple}{\textit{come-into-being}\textsubscript{ip}}\\\textcolor{violet}{\textsc{impl-prod :}} \textcolor{lsRichGreen}{\textit{concr-ent}{,} \textsc{anim :} \textit{false}}
            [\textcolor{purple}{\textit{c-o-s} $\wedge$ \textit{c-i-b}\textsubscript{ip}}, name=coscibp
            ]
            ]
          ]
        [\textit{caus}
          \\\textsc{actor} 
          \\ \textsc{pat} 
          \\ $\pm$\textsc{instr} 
          \\ $\pm$\textcolor{violet}{\textsc{result}}
          \\ \textsc{cause :} \textit{event}
          \\ \textsc{effect :} \textit{event}
          [\textit{c-o-s caus}
          \\\textcolor{violet}{\textsc{actor}}
          \\\textsc{effect :} \textit{c-o-s}
            [\textit{c-o-phys-form caus}
            \\\textsc{effect :} \textit{c-o-phys-form}
            ]
          ]
          [\textcolor{purple}{\textit{c-i-b caus}}
          \\\textsc{effect :} \textit{c-i-b}
            [\textcolor{purple}{\textit{c-i-b}\textsubscript{ip} caus}
            \\\textsc{effect :} \textit{c-i-b}\textsubscript{ip}
            ]
          ]
        ]
        ]
        [\textcolor{lsMidOrange}{\textit{entity}}
        [\normalfont{[...]}]
        ]
        ]
    \draw (cos.south) -- (coscib.north);
    \draw (coscib.south) -- (coscibp.north);  
    \end{forest}
    }
    \caption[Revised type signature of eventive categories (COS subset)]{\label{fig:signature-update} Revised type signature of eventive categories (COS subset). Optionality is indicated by $\pm$, changes compared to the original type signature are indicated by color. Abbreviations: act = actor, ag = agent, anim = animacy, caus = causation, concr-ent = concrete entity, c-i-b = come-into-being, c-o-phys-form = change-of-physical-form, c-o-psych-st = change-of-psych-state, c-o-s = change-of-state, ent = entity, impl-prod = implicit product, instr = instrument, ip = implicit product, pat = patient, res-st = result-state.}
\end{sidewaysfigure}

\subsubsection{Lexical rules and inheritance hierarchy}
\label{sec:cos-output-formal-inherit}

The aim of this study was to establish a set of lexeme formation rules (LFRs) for \mbox{\textit{-ment}} on COS verb bases. I identified nine such rules, producing nine distinct readings: \textsc{change-of-state causation, causing event, change-of-state, result-state, result, implicit product, instrument, patient,} and \textsc{causer}. Here, I am only spelling out the lexical rule for \textsc{result-state} readings. The other eight LFRs differ only in the attribute label given for the base verb semantics (label \textsc{causer} for \textsc{causer} readings, and so forth.) 

The LFR in Figure \ref{fig:lexrule-resst} creates a noun with the phonology /\avmbox{z}-ment/ and a semantics which is specified by a frame. 
The input which is fed into the LFR is a verb with the phonology /\avmbox{z}/. Its semantics is specified as \textit{change-of-state causation} with, among others, a \textsc{result-state} attribute.
The last line tells us that reference is on the node indexed `\avmbox{x}' so that the depicted LFR produces a \textsc{result-state} reading.

\begin{figure}
    \begin{avm}
      \[\textit{lexeme}\cr
      \textsc{phon} & {/\@z-\textit{ment}}/\cr
      \textsc{cat} & N \cr
      \textsc{sem} & \[ s-frame & \@y 
      \]\cr
      \textsc{m-base} & 
      \[\textit{lexeme}\cr
      \textsc{phon} & \@z \cr
      \textsc{cat} & V \cr
      \textsc{sem} &
      \[ s-frame & \@y
      \[ 
      \textit{change-of-state causation} \cr
      \textsc{result-state} & \@x \cr
      ... \cr 
      \]
      \]
      \]  \cr
      \avmspan{{\textsc{ref} $=$ \{\@x\} }}
      \cr
      \]  
    \end{avm}
    \caption{Lexical rule for \textsc{result-state} readings of \textit{-ment} on COS verbs}
    \label{fig:lexrule-resst}
\end{figure}

\begin{figure}
  \resizebox{!}{\textheight}{%
  \begin{tikzpicture}
  \draw (2,0) node(0){lexeme};
  \draw (2,-1) node(1){\textit{v-n-lfr}};
  \draw (4,4) node(2){\Boxed{\textsc{phon}}};
  \draw (4,-5) node(3){\Boxed{\textsc{sem}}};
  \draw (4,-6) node(4){\textsc{anim : }false};
  \draw (12,5) node(5){}; 
  \draw (6.5,5) node(55)[anchor=west]{
    \begin{avm}
    \sort{0}{\[{\textit{x-ment}}\cr
      \textsc{ph} \@1+ment\cr
      \textsc{m-base} \[\textsc{ph} \@1 \]
      \]}
    \end{avm}};
  \draw (55.east) -- (5.east); 
  \draw (6.5,2.5) node(11)[anchor=west]{
    \begin{avm}
    \sort{1}{\[\textit{c-o-s-causation-n}\cr
      \textsc{ref} \@x\cr
      \textsc{m-base} \[ \textsc{sem} \[ \@x \type{c-o-s caus}  \] \]
      \]}
    \end{avm}};
  \draw (6.5,0) node(12)[anchor=west]{
    \begin{avm}
    \sort{2}{\[\textit{causing-evt-n}\cr
      \textsc{ref} \@x\cr
      \textsc{m-base} \[ \textsc{sem} \[ \textsc{cause} \@x \] \]
      \]}
    \end{avm}};
  \draw (6.5,-2.5) node(13)[anchor=west]{
    \begin{avm}
    \sort{3}{\[\textit{change-of-state-n}\cr
      \textsc{ref} \@x\cr
      \textsc{m-base} \[ \textsc{sem} \[ \textsc{effect} \@x \] \]
      \]}
    \end{avm}};
  \draw (6.5,-5) node(14)[anchor=west]{
    \begin{avm}
    \sort{4}{\[\textit{result-state-n}\cr
      \textsc{ref} \@x\cr
      \textsc{m-base} \[ \textsc{sem} \[ \textsc{res-st} \@x \] \]
      \]}
    \end{avm}};
  \draw (6.5,-7.5) node(15)[anchor=west]{
    \begin{avm}
    \sort{5}{\[\textit{result-n}\cr
      \textsc{ref} \@x\cr
      \textsc{m-base} \[ \textsc{sem} \[ \textsc{result} \@x \] \]
      \]}
    \end{avm}};
  \draw (6.5,-10) node(16)[anchor=west]{
    \begin{avm}
    \sort{6}{\[\textit{implicit-product-n}\cr
      \textsc{ref} \@x\cr
      \textsc{m-base} \[ \textsc{sem} \[ \textsc{impl-prod} \@x \] \]
      \]}
    \end{avm}};
  \draw (6.5,-12.5) node(17)[anchor=west]{
    \begin{avm}
    \sort{7}{\[\textit{instrument-n}\cr
      \textsc{ref} \@x\cr
      \textsc{m-base} \[ \textsc{sem} \[ \textsc{instrument} \@x \] \]
      \]}
    \end{avm}};
  \draw (6.5,-15) node(18)[anchor=west]{
    \begin{avm}
    \sort{8}{\[\textit{patient-n}\cr
      \textsc{ref} \@x\cr
      \textsc{m-base} \[ \textsc{sem} \[ \textsc{patient} \@x \cr $\neg$ \textsc{result} \] \]
      \]}
    \end{avm}};
  \draw (6.5,-17.5) node(19)[anchor=west]{
    \begin{avm}
    \sort{9}{\[\textit{causer-n}\cr
      \textsc{ref} \@x\cr
      \textsc{m-base} \[ \textsc{sem} \[ \textsc{causer} \@x \] \]
      \]}
    \end{avm}};
  \draw (15,2.5) node(21){\begin{tabular}{l}
    abridgement\\besmirchment\\congealment\\\textsuperscript{?}discolorment\\embetterment
    \end{tabular}};
  \draw (15,0) node(22){\begin{tabular}{l}
    abridgement\\besmirchment\\\textsuperscript{?}congealment\\\textsuperscript{?}discolorment\\embetterment
    \end{tabular}};
  \draw (15,-2.5) node(23){\begin{tabular}{l}
    \textsuperscript{?}abridgement\\besmirchment\\congealment\\discolorment\\embetterment
    \end{tabular}};
  \draw (15,-5) node(24){\begin{tabular}{l}
    abridgement\\besmirchment\\congealment\\discolorment\\\textsuperscript{?}embetterment
    \end{tabular}};
  \draw (15,-7.5) node(25){\begin{tabular}{l}
    abridgement\\besmirchment\\embetterment
    \end{tabular}};
  \draw (15,-10) node(26){\begin{tabular}{l}
    discolorment
    \end{tabular}};
  \draw (15,-12.5) node(27){\begin{tabular}{l}
    abridgement\\besmirchment\\congealment\\embetterment
    \end{tabular}};
 \draw (15,-15) node(28){\begin{tabular}{l} 
   congealment
   \end{tabular}};
 \draw (15,-17.5) node(29){\begin{tabular}{l}
   abridgement\\besmirchment\\discolorment\\embetterment 
   \end{tabular}};
  \draw [dotted] (0) -- (1);
  \draw (1.east) -- (2.west); 
  \draw (1.east) -- (3.west); 
  \draw (2.east) -- (55.west); 
  \draw (3.south) -- (4.north); 
  \draw (4.east) -- (11.west); 
  \draw (4.east) -- (12.west); 
  \draw (4.east) -- (13.west); 
  \draw (4.east) -- (14.west); 
  \draw (4.east) -- (15.west); 
  \draw (4.east) -- (16.west); 
  \draw (4.east) -- (17.west); 
  \draw (4.east) -- (18.west); 
  \draw (4.east) -- (19.west); 
  \draw (11.east) -- (21.west); 
  \draw (12.east) -- (22.west); 
  \draw (13.east) -- (23.west); 
  \draw (14.east) -- (24.west); 
  \draw (15.east) -- (25.west); 
  \draw (16.east) -- (26.west); 
  \draw (17.east) -- (27.west); 
  \draw (18.east) -- (28.west); 
  \draw (19.east) -- (29.west); 
  \draw (5.east) -- (21.west); 
  \draw (5.east) -- (22.west); 
  \draw (5.east) -- (23.west); 
  \draw (5.east) -- (24.west); 
  \draw (5.east) -- (25.west); 
  \draw (5.east) -- (26.west); 
  \draw (5.east) -- (27.west); 
  \draw (5.east) -- (28.west); 
  \draw (5.east) -- (29.west); 
  \end{tikzpicture}
  } 
  \caption[Inheritance hierarchy of lexical rules for \textit{-ment} on COS verbs]{Inheritance hierarchy of lexical rules for \textit{-ment} on COS verbs. Abbreviations: caus = causation, impl-prod = implicit product.}
  \label{fig:inheritance-complete-vert}
\end{figure}

In the inheritance hierarchy in Figure \ref{fig:inheritance-complete-vert}, all my findings with regard to possible nominalization readings come together. 
The figure incorporates (abbreviated versions of) all nine LFRs and thus allows the derivation of the nine readings which I have found attested in my data. For ease of reference, I have numbered each LFR in the bottom left corner.  
The inheritance hierarchy also contains the constraints as introduced in Figures \ref{fig:inheritance-animacy} and \ref{fig:inheritance-depend} (``\textsc{anim :} false'' as a supertype and ``$\neg$ \textsc{result}'' within the \textit{patient-n}-AVM). 
Since there is not enough space to include all 18 nominalizations in the figure, I am again calling upon the types which I used for illustration in Figures \ref{fig:nounframe-instrpat}, \ref{fig:nounframe-instrcausres} and \ref{fig:nounframe-causimpl} above: \textit{abridgement, besmirchment, congealment, discolorment} and \textit{embetterment}.

As described in detail in \sectref{sec:fr-der-ref-inherit}, the mechanism depicted in the figure works on two levels: It accesses the phonology (\avmbox{\textsc{phon}}) and the semantics (\avmbox{\textsc{sem}}) of a morphological base (\textsc{m-base}) and outputs a nominalization with certain properties. 
On the phonological level, the nominalization takes the shape of the base plus \textit{-ment}, as modeled in the bracket notation marked 0. 
On the semantic level, the animacy constraint takes effect first. It filters out animate readings (e.g. \textsc{agent}) as well as animate variants of readings (e.g. [+animate] \textsc{patient}).

Then, nine LFRs, one for each reading, become operative. The first possible reading is \textsc{change-of-state causation}, as per LFR 1. This LFR checks for a node labeled \textit{change-of-state causation}. Since this is the central node of all investigated base verb frames, all nominalizations can theoretically produce it. In other words, LFR 1 models the mechanism of transposition. 
The other eight LFRs check for attributes in the base verb frame, for instance \textsc{cause} (LFR 2) or \textsc{patient} (LFR 8). A reading is possible if the base verb has the corresponding attribute in its frame. For example, all five types can produce a \textsc{result-state} reading (LFR 4), but only \textit{discolorment} can produce an \textsc{implicit product} reading (LFR 6). In LFR 8 we see the second constraint, which prevents shifts to a \textsc{patient} reading if \textsc{result} is in the base frame. Out of the five base verbs represented here, only \textit{congeal} has neither a \textsc{result} attribute nor its subtype \textsc{implicit product}. Therefore, \textit{congealment} is the only nominalization which is expected to have a \textsc{patient} reading.

Finally, let me briefly address the issue of gaps in the data. Some types in Figure \ref{fig:inheritance-complete-vert} are marked with a superscript question mark, for instance \textit{discolorment} next to LFR 1. These are the types which I expected to find in a given reading, but did not. There are a number of conceivable reasons for and implications of such gaps, which I will discuss in detail in \chapref{ch:observations}. 
For now, let me just say that we are most likely dealing with mere data issues, and not with negative evidence.  

\section{Summary of Chapter 4} \label{sec:cos-summary} 


In this chapter, I have investigated the readings which can be produced by the suffix \textit{-ment} on COS verbs. The data set contained 18 nominalizations which I have assigned to three subclasses of COS verbs in VerbNet: c-only COS verbs, c/i general COS verbs and c/i reversible COS verbs. In VerbNet, all three share the same set of possible syntactic arguments, or participants, namely \textsc{agent, patient} and \textsc{instrument}. They are distinguished by their participation in the Caus/Inch-Alternation, by the type of change they lexicalize, by selectional restrictions on the \textsc{patient} and \textsc{instrument} participant, and by the possibility of expressing a secondary result in their participant structure. 

As a next step, I devised what I called \textit{VerbNet-based frames} by combining the information from VerbNet with the event structure templates commonly applied in frame semantics to model verbs. For this, I included the participants as attributes, and translated VerbNet's selectional restrictions into type specifications of these attributes. 
C-only COS verbs were modeled with a causative event structure template. With regard to c/i COS verbs, I hypothesized that their nominalizations could shed some light on which frame is more appropriate: a complex and a simple event frame, one for each variant, or an integrated complex event frame with two possible referents.
Furthermore, I decided to leave out the modeling of the secondary result for two interrelated reasons: I did not want to extend my study of morpho-semantics to the syntax-semantics interface, and I did not expect the different syntactic realizations of a secondary result to have any effect on the semantics of the derivatives in the first place.

Based on the VerbNet-based frames and on existing literature, I then formulated expectations about the possible readings and contexts of COS nouns, and examined the corpus attestations to verify or falsify these expectations. 
  Regarding possible readings, my study had the following results: 
  I systematically found shifts to the eventive categories \textsc{change-of-state causation, causing event, change-of-state} and \textsc{result-state}, as well as to the non-eventive categories \textsc{result, implicit product, instrument, patient} and \textsc{causer}. 
  Furthermore, I identified three constraints which help explain my data: 
  First, animate readings are not produced by \textit{-ment}. This affects \textsc{agent}, which is ruled out as a possible reading, as well as  \textsc{patient} and \textsc{causer}, which are restricted to \mbox{[−animate]} instantiations. Post-lexical shifts are possible, but rare.
  Second, I found a complementary distribution of \textsc{patient, implicit product} and \textsc{result} readings. I captured this distribution by formulating two principles: If a base verb has an \textsc{implicit product} or a \textsc{result} participant, this participant can be shifted to, respectively, and if a base verb has no \textsc{implicit product} or \textsc{result} participant, the \textsc{patient} can be shifted to. 
  Third, I looked at the distribution of possible readings denoting a causing entity; that is, I looked for systematic distributions of \textsc{instrument} versus \textsc{causer}. I concluded that the constellation we find (\textsc{instrument}, or \textsc{causer}, or both, or neither) depends on the base verb.


Regarding the contexts in which the nominalizations are attested, I demonstrated that external causes are either expressed or implied in the contexts of \textsc{change-of-state causation} readings. This finding supports the conventional approach of representing causation events as complex events. 
In the contexts of \mbox{\textsc{change-of-state}} readings, a mixed picture emerged: 
For c-only COS nouns, as expected, external causes are expressed or implied. 
For c/i COS nouns, external causes are most often expressed or implied, but some attestations without an external cause can be found. 
Therefore, I concluded that c/i COS nouns are best modeled with a complex event structure covering both variants, but that the inchoative variant may also be modeled with a simple event frame under certain conditions. Unfortunately, what exactly these conditions are has yet to be uncovered.

Based on these findings, I revised the VerbNet-based frames from the ground up. 
Since they only included syntactic arguments  \textendash{}  which are thought to be the core participants of a verb  \textendash{}  I added the semantic arguments \textsc{implicit product} and \textsc{result}. 
Additionally, I also modified existing frame elements. 
Both the \textsc{agent} attribute and the type of the first subevent, \textit{action}, proved to be too restrictive. They were replaced by \textsc{participant} and \textit{event}, respectively.
I also discussed the possibility of representing gradient phenomena by means of stochastic frames. Unfortunately, the scope of this book only allowed for a tentative exploration of the topic, rather than a full-fledged quantitative analysis.
The patterns which I identified (\textsc{patient/implicit product/result} and \textsc{instrument/causer}) produce ten classes of nouns and base verbs, three of which I modeled by way of example. The type signature was also updated accordingly.

I concluded my analysis by suggesting LFRs for \textit{-ment} nominalizations with COS verb bases. One LFR, producing \textsc{result-state} readings, was given by way of example, and all nine proposed LFRs were incorporated into the inheritance hierarchy for \textit{-ment} suffixation. 
The inheritance hierarchy also contains two constraints: A superimposed animacy constraint prevents shifts to \textsc{agent}, to [+animate] \textsc{patient} and to [+animate] \textsc{causer}, and a specification within the \textit{patient-n}-LFR handles the \textsc{patient/implicit product/result} distribution. 
