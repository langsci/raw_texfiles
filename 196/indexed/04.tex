\chapter{Dealing with {post-editing} and  {machine translation} – five perspectives}
\label{sec:4}
\is{post-editing} \is{machine translation}

PE and MT have influenced the field of (human) translation in many aspects. However, research and practice do not always go hand in hand. The present chapter will introduce five fields of translation – theoretical translation studies, empirical translation process research, translation practice, the professional translation community, and didactics – and analyse their work with and on MT and PE to give a comprehensive overview of the opinions and approaches in translation research and practice. The early years of MT and PE will not be considered as the processes (also concerning translation from scratch) were too different from today's standards, e.g. using a typewriter instead of a computer. However, the PAHO, for example, established MT and PE in the late 1970s, early 1980s and still uses both today. Hence, this organisation is the example with the longest history in PE and sets the historical starting point of the portrait.


\section{Post-editing and machine translation in (theoretical) translation studies}
\label{sec:4:1}

Although hardly anyone who professionally deals with MT systems (including users, computer linguists, and translators\slash translation scientists) maintains that these systems are capable of fully replacing human translators at the moment or any time soon, translators are still very suspicious towards MT. This section will present (in excerpts) how MT is received in (theoretical) translation studies and how the arising new changes and challenges are assessed.



In her unpublished dissertation, %\label{ref:ZOTEROITEMCSLCITATIONcitationID89bNXd37propertiesformattedCitationKaiserCooke1993plainCitationKaiserCooke1993citationItemsid222urishttpzoteroorgusers1255332itemsN458885Kurihttpzoteroorgusers1255332itemsN458885KitemDataid222typearticlejournaltitleMachineTranslationandthehumanfactorKnowledgeanddecisionmakinginthetranslationprocesscontainertitleUnpublishedPhDdissertationUniversityofViennaauthorfamilyKaiserCookegivenMichleissueddateparts1993schemahttpsgithubcomcitationstylelanguageschemarawmastercslcitationjsonRNDkAkoNMzAra}
\citet{Kaiser-cooke1993} wanted to analyse the interplay of the different types of knowledge that are important in the translation process – “what translators do and how they do it or, in other words, what they know and how they know it” – to finally provide MT researchers with information on what to take into account and what is important for the translation process. She argues that translation is more than a mere transferral of words from a source language into a \isi{target language} and more than just a transferral of meaning. A translator needs knowledge of the \isi{target language} and culture and must know what is to be reproduced and in which manner in the \isi{target language}. Further, a translator needs to have domain knowledge and problem solving skills. All these characteristics cannot be replicated by a machine. Translation is a constant decision making process in which problems may occur. For each decision that a human or a machine has to make while translating, there might be a few correct choices, but there are many more incorrect choices, which is why MT is so difficult to implement. A human translator usually knows from (translation) experience as well as linguistic and cultural knowledge, how to make the correct decision or which choices are acceptable. The machine does not (ibid.: 172-188). One important question MT and human translation have in common, but from very different angles, is how much the machine\slash the translator needs to know in order to translate properly. Especially in domain-specific translation, not every translator can be as competent as a domain expert, but (s)he has to know enough to ensure a complete and correct target text and delivery of the same message to experts in both cultures. (cf. ibid.: 161-162) Kaiser-Cooke concludes that translators need comparative language knowledge and cultural knowledge as well as translation expertise, which MT cannot offer, because “as a 'hard' discipline, [MT] necessarily sees itself constrained to work on the basis of an objective, concrete world and is thus forced to operate with concrete, quantifiable data.” (ibid.: 219) Furthermore, Kaiser-Cooke argues that ful\-ly-au\-to\-mat\-ic MT might become possible, if computers learnt how to deal with the individuality of each translation situation, how to present a situation from two different angles, and how to simulate human processing skills such as abstraction and extrapolation. If this was given, the only thing missing to enable MT would be comparative and cultural knowledge for the respective languages. She concludes that MT has a future if computer scientists start to become more acquainted with the human translation process and translation scholars are willing to deliver a theoretical basis for the translation process. These theories need to be rooted in translation practice and require validation from experiments that include knowledge of other disciplines like artificial intelligence, linguistics, psychology, cultural studies etc.\footnote{During that time, translation process research was a very new field and the explorative first think-aloud studies had just been published. Today, the field is thriving and we are getting more and more insights on what is going on in the translator's mind.} (cf. ibid.: 220-225)

\largerpage

\begin{quote}
Machine translation and translation studies could both benefit from mutual recognition of the other's problems and achievements and by furthering the interdisciplinary cooperation necessary to unravel all the complexities of translating as a highly skilled specialist activity. (ibid.: 230)
\end{quote}


%\label{ref:ZOTEROITEMCSLCITATIONcitationIDy3837Z1dpropertiesformattedCitationCronin2003plainCitationCronin2003citationItemsid141urishttpzoteroorgusers1255332itemsX5HB89Z6urihttpzoteroorgusers1255332itemsX5HB89Z6itemDataid141typebooktitleTranslationandglobalizationpublisherRoutledgepublisherplaceLondonNewYorkeventplaceLondonNewYorkauthorfamilyCroningivenMichaelissueddateparts2003schemahttpsgithubcomcitationstylelanguageschemarawmastercslcitationjsonRNDUo7vxKCule}
\citet[111--119]{Cronin2003} exemplifies the arbitrary world in which MT exists: while some translators refuse the use of MT and summon the extinction of translation professionals, the need for translation increases exponentially. Every day, multilingual websites of large companies change their web-content, which has to be localised accordingly. Similarly, multinational companies, e.g. Caterpillar, may produce hundreds of pages of written content every day that need to be distributed to all plants all over the world. There are not enough human translators available to cope with these amounts of text. Hence, technology and automation are necessary to handle this demand. Nonetheless, many translators speak ill of CAT-tools and MT:


\begin{quote}
Although an understandable reaction to cyberhype, the endlessly recycled translation howlers from failed MT projects and the derisive dismissal of ‘pocket translation’ and free MT services on the Web are unhelpful both because they misinterpret the history or achievements of CAT and MT […] and, more seriously, perhaps, because they blind translators and many of those who write about translation to the close connection between translation and the new economy in the global age. […] Translation, like every other sector of human activity, is affected by economic and technical developments and so the move towards automation […] cannot simply be rejected as a malevolent action of technocratic Philistines intent on the dumbing-down of culture. (ibid: 113)
\end{quote}


Cronin states that the translation profession not only enables a digitalising and technologising world through the texts they translate, but that the profession is also shaped by these technologies, turning translators into “\textit{translational cyborgs}” (ibid.: 112). There should be no expectation for MT to automatically translate e.g. literary classics, but it should rather be seen as a tool that can help accelerate the translation process; or MT can be considered a tool that undertakes simple gisting tasks. This also means that MT systems cannot simply replace the human translators (and interpreters) overnight.



Similarly, %\label{ref:ZOTEROITEMCSLCITATIONcitationIDMAR1xuJ8propertiesformattedCitationHeller2012plainCitationHeller2012citationItemsid122urishttpzoteroorgusers1255332items97QRJNV7urihttpzoteroorgusers1255332items97QRJNV7itemDataid122typebooktitleTranslationswissenschaftlicheBegriffsbildungunddasProblemderperformativenUnaufflligkeitvonTranslationpublisherFrankTimmeGmbHpublisherplaceBerlinvolume51eventplaceBerlinauthorfamilyHellergivenLaviniaissueddateparts2012schemahttpsgithubcomcitationstylelanguageschemarawmastercslcitationjsonRNDpmQFKjOp8d}
\citet[280--281]{Heller2012} points out that translations and human translation processes have not yet been replaced by other acts or processes in this globalised world, and are therefore socially and (inter-)culturally highly relevant. However, she concedes that MT might replace human translation at least to a certain extent in the future. To her, the fact that MT has attracted so much attention in recent years only proves that the need for (fast) translations is currently even higher than it used to be and that this demand can hardly be handled exclusively by human translators.



%\label{ref:ZOTEROITEMCSLCITATIONcitationID6s7rpYrepropertiesformattedCitationPym2013plainCitationPym2013citationItemsid206urishttpzoteroorgusers1255332itemsG8UASJ79urihttpzoteroorgusers1255332itemsG8UASJ79itemDataid206typearticlejournaltitleTranslationSkillsetsinaMachinetranslationAgecontainertitleMetaJournaldestraducteursMetaTranslatorsJournalpage487503volume58issue3authorfamilyPymgivenAnthonyissueddateparts2013schemahttpsgithubcomcitationstylelanguageschemarawmastercslcitationjsonRNDO3VH0V3QIL}
\citet{Pym2013} also acknowledges the influence of technology and MT on the field of translation. He suspects that the combination of MT and \isi{translation memory} systems will at some point replace full human translation in many aspects of translation and consequently the translator will be required to have different skills. Furthermore, the spread of online MT systems might also change the social aspect of translation. When using MT\slash TM, the translator must be able to evaluate what output can be trusted. After assessing different aspects of translation – language, area, and intercultural knowledge - %\label{ref:ZOTEROITEMCSLCITATIONcitationIDxI7socbQpropertiesformattedCitationPym2013plainCitationPym2013citationItemsid206urishttpzoteroorgusers1255332itemsG8UASJ79urihttpzoteroorgusers1255332itemsG8UASJ79itemDataid206typearticlejournaltitleTranslationSkillsetsinaMachinetranslationAgecontainertitleMetaJournaldestraducteursMetaTranslatorsJournalpage487503volume58issue3authorfamilyPymgivenAnthonyissueddateparts2013schemahttpsgithubcomcitationstylelanguageschemarawmastercslcitationjsonRND0ZbSd3JDWv}
\citet[491]{Pym2013} concludes that “[t]he active and intelligent use of TM/ MT should eventually bring significant changes to the nature and balance of all other components, and thus to the professional profile of the person we are still calling a translator” and therefore, he adds, some doubts about traditional terminology like \textit{translator} or \textit{source text} and traditional translation models. While translators used to have to apply their skills to identifying and generating possible solutions for problems in translation from scratch, he assumes a shift towards selecting between available solutions when technologies are involved. Accordingly, new strategies for translation didactics have to be developed (see further discussions in \sectref{sec:4:5}).



While Kaiser-Cooke and others in the last decades were rather open-minded towards MT as early as in the early 1990s, this is not always a commonly shared state of mind. Even in the second edition of his book “Übersetzung und Linguistik”\footnote{“Translation and Linguistics”} that was published in 2013, %\label{ref:ZOTEROITEMCSLCITATIONcitationIDXE4pmV4xpropertiesformattedCitationAlbrecht2013plainCitationAlbrecht2013citationItemsid252urishttpzoteroorgusers1255332itemsVWXA4H73urihttpzoteroorgusers1255332itemsVWXA4H73itemDataid252typebooktitlebersetzungundLinguistikcollectiontitleGrundlagenderbersetzungsforschungcollectionnumberBd2publisherNarrpublisherplaceTbingennumberofpages312edition2berarbAuflsourceGemeinsamerBibliotheksverbundISBNeventplaceTbingenISBN9783823367932languagegerauthorfamilyAlbrechtgivenJrnissueddateparts2013schemahttpsgithubcomcitationstylelanguageschemarawmastercslcitationjsonRNDMh6dkHyJNk}
\citet[76]{Albrecht2013} speaks disparagingly about MT systems:


\begin{quote}
My experiences with these so called 'translation systems' give me no reason to go into detail. [...] With regard to all the useful goals that need to be tackled by computer linguists and computer scientists in the field of computer-assisted translation, the development of full-automatic translation systems seems to be intellectual dalliance, at least for the practising translator\footnote{„Meine Erfahrungen mit sogenannten 'Übersetzungssystemen' lassen mich davon absehen, auf diese Hilfsmittel einzugehen. [...] Angesichts der vielen sinnvollen Aufgaben, die im Bereich der computergestützten Übersetzung auf Computerlinguisten und Informatiker warten, erscheinen die ehrgeizigen Versuche, vollautomatische Übersetzungssysteme zu entwickeln, zumindest dem Praktiker als intellektuelle Spielerei.“} (ibid. [translated by J.N.])
\end{quote}

\largerpage
One problem with the contemptuous opinions of MT is that this attitude is communicated to student translators who read these types of textbooks or attend lectures in which MT systems are criticised instead of dealing with the topic reasonably. Instead of looking down on automated systems, students should, in my opinion, learn what these systems are capable and (especially) what they are not capable of, so that they can reason with clients. The spread of MT and PE seems threatening to some (professional) translators who fear for their job instead of recognising the opportunity. This fear often results from unfamiliarity with the technology and its advantages and disadvantages. Hence, it cannot be helpful to either endorse these fears or spread unjustified or uneducated personal opinions.



Similarly as %\label{ref:ZOTEROITEMCSLCITATIONcitationIDjlIux5kJpropertiesformattedCitationKaiserCooke1993plainCitationKaiserCooke1993citationItemsid222urishttpzoteroorgusers1255332itemsN458885Kurihttpzoteroorgusers1255332itemsN458885KitemDataid222typearticlejournaltitleMachineTranslationandthehumanfactorKnowledgeanddecisionmakinginthetranslationprocesscontainertitleUnpublishedPhDdissertationUniversityofViennaauthorfamilyKaiserCookegivenMichleissueddateparts1993schemahttpsgithubcomcitationstylelanguageschemarawmastercslcitationjsonRND3TR43PyqTR}
\citet{Kaiser-cooke1993} reported in her unpublished dissertation, %\label{ref:ZOTEROITEMCSLCITATIONcitationIDH3vYzlKUpropertiesformattedCitationrtfuc0u268ulo2014plainCitationulo2014citationItemsid175urishttpzoteroorgusers1255332itemsES4D5SK4urihttpzoteroorgusers1255332itemsES4D5SK4itemDataid175typepaperconferencetitleFromTranslationMachineTheorytoMachineTranslationTheorysomeinitialconsiderationscontainertitleTheFutureofInformationSciencepage3138eventINFuture2013InformationGovernanceauthorfamilyulogivenOliverissueddateparts2014schemahttpsgithubcomcitationstylelanguageschemarawmastercslcitationjsonRNDdGMBUxKkUW}
\citet{Culo2014} points out that both MT developers as well as translation scientists need to learn from each other. MT could integrate insights yielded by translation studies to improve MT systems e.g. in regard to the linguistic behaviour of text types, domain convention, or register, while translation studies could acknowledge MT as a form of documentary translation, which would probably enhance the acceptance of MT in the field in the long run.



%\label{ref:ZOTEROITEMCSLCITATIONcitationID8PkvquZYpropertiesformattedCitationRozmyslowicz2014plainCitationRozmyslowicz2014citationItemsid212urishttpzoteroorgusers1255332itemsPI2URSVRurihttpzoteroorgusers1255332itemsPI2URSVRitemDataid212typearticlejournaltitleMachineTranslationAProblemforTranslationTheorycontainertitleNewVoicesinTranslationStudiespage145163issue11authorfamilyRozmyslowiczgivenTomaszissueddateparts2014schemahttpsgithubcomcitationstylelanguageschemarawmastercslcitationjsonRNDxHXw4CM87x}
\citet{Rozmyslowicz2014} analyses the need for translation theory to deal with MT. He argues that the basic assumptions underlying \textit{translation} have to be revised in order to include MT in translation theory. MT has become part of everyday translation and communication – not only for professional translators, but especially in the everyday life of laypersons. In recent translation theories, the term \textit{culture} has become indispensable to define \textit{translation,} to disengage \textit{translation} from its purely linguistic history, while \textit{culture} is often used in a vague manner. It is, however, often not acknowledged that defective communication still initialises communication. Rozmyslowicz hence suggests to assume \textit{understanding} as the initial point of \textit{communication}.


\begin{quote}
[T]he degree of technological perfection or imperfection in computers is theoretically irrelevant. What \textit{is} relevant is that communication is initiated and maintained without necessarily presupposing another conscious being as the direct source of an utterance, and the same holds true, by extension, for translation – at least since the advent of \isi{machine translation}. Whether and to what extent the ‘defects’ of machine-generated translations become a communicative problem is an empirical question and cannot be decided by theoreticians, [...] for as long as no one “protests”, we have no reason to assume that translation has failed \citep[101]{vermeer1978rahmen}. %\label{ref:ZOTEROITEMCSLCITATIONcitationIDkGBtReBRpropertiesformattedCitationRozmyslowicz2014plainCitationRozmyslowicz2014citationItemsid212urishttpzoteroorgusers1255332itemsPI2URSVRurihttpzoteroorgusers1255332itemsPI2URSVRitemDataid212typearticlejournaltitleMachineTranslationAProblemforTranslationTheorycontainertitleNewVoicesinTranslationStudiespage145163issue11authorfamilyRozmyslowiczgivenTomaszissueddateparts2014schemahttpsgithubcomcitationstylelanguageschemarawmastercslcitationjsonRNDvTgDZopZGZ}
(ibid.: 158, emphasis in original text)
\end{quote}

\newpage 
In conclusion, %\label{ref:ZOTEROITEMCSLCITATIONcitationIDRpzep5PxpropertiesformattedCitationRozmyslowicz2014plainCitationRozmyslowicz2014citationItemsid212urishttpzoteroorgusers1255332itemsPI2URSVRurihttpzoteroorgusers1255332itemsPI2URSVRitemDataid212typearticlejournaltitleMachineTranslationAProblemforTranslationTheorycontainertitleNewVoicesinTranslationStudiespage145163issue11authorfamilyRozmyslowiczgivenTomaszissueddateparts2014schemahttpsgithubcomcitationstylelanguageschemarawmastercslcitationjsonRNDk6miOV2rAs}
Rozmyslowicz argues, similar to %\label{ref:ZOTEROITEMCSLCITATIONcitationIDnZhSoKbOpropertiesformattedCitationPym2013plainCitationPym2013citationItemsid206urishttpzoteroorgusers1255332itemsG8UASJ79urihttpzoteroorgusers1255332itemsG8UASJ79itemDataid206typearticlejournaltitleTranslationSkillsetsinaMachinetranslationAgecontainertitleMetaJournaldestraducteursMetaTranslatorsJournalpage487503volume58issue3authorfamilyPymgivenAnthonyissueddateparts2013schemahttpsgithubcomcitationstylelanguageschemarawmastercslcitationjsonRNDICiswYcgda}
\citet{Pym2013}, that concepts such as \textit{agency}, \textit{translation}, \textit{culture}, or \textit{communication} have to be revised to theoretically embed MT into translation studies.



In his book on training translators and interpreters, %\label{ref:ZOTEROITEMCSLCITATIONcitationID4sXFEtrCpropertiesformattedCitationOrlando2016plainCitationOrlando2016citationItemsid87urishttpzoteroorgusers1255332itemsXMT9SC72urihttpzoteroorgusers1255332itemsXMT9SC72itemDataid87typebooktitleTraining21stcenturytranslatorsandinterpretersAtthecrossroadsofpracticeresearchandpedagogypublisherFrankTimmeGmbHpublisherplaceBerlinvolume21eventplaceBerlinauthorfamilyOrlandogivenMarcissueddateparts2016schemahttpsgithubcomcitationstylelanguageschemarawmastercslcitationjsonRNDsLLCYXuh3F}
\citet{Orlando2016} acknowledges that technology, MT, and PE have become part of the industry and that translators and interpreters are expected to deal with these technologies. However, the book does not indicate how to integrate these technologies in training. Some publications have already devoted ideas on integrating MT into translation didactics, which will be introduced in \sectref{sec:4:5}.



In summary, most translation scholars introduced in this chapter agree that MT technology has arrived in the translators' work environment, but also in the everyday life of laypersons. Some recognise that MT (in combination with other translation tools) will partly replace full human translations, which is also necessary, because the need for translations is growing continuously.


\section{Post-editing and machine translation in translation process research}
\label{sec:4:2}

A rather different approach to PE (and consequently MT) can be found in (empirical) translation process research. This research area has already identified the practical need for PE and has included the task in numerous studies. The following overview of such studies is not intended to be exhaustive, but to highlight some ideas on research interests in the field.



The first translation process study on PE was published by %\label{ref:ZOTEROITEMCSLCITATIONcitationIDU40ggZimpropertiesformattedCitationKrings2001plainCitationKrings2001dontUpdatetruecitationItemsid228urishttpzoteroorgusers1255332items4D665XEKurihttpzoteroorgusers1255332items4D665XEKitemDataid228typebooktitleRepairingtextsempiricalinvestigationsofmachinetranslationposteditingprocessespublisherKentStateUniversityPresspublisherplaceKentOhionumberofpages635sourceLibraryofCongressISBNeventplaceKentOhioISBN9780873386715callnumberP309K75132001shortTitleRepairingtextslanguageengauthorfamilyKringsgivenHansPeditorfamilyKobygivenGeoffreySissueddateparts2001schemahttpsgithubcomcitationstylelanguageschemarawmastercslcitationjsonRND958gEslEX5}
\citet[this summary refers to the \ili{English} version, the first edition in \ili{German}, however, was published in 1997]{Krings2001}. This think-aloud study dealt with technical texts that described simple every-day appliances in \ili{English}, \ili{French}, and \ili{German}. The texts were automatically translated by the SYSTRAN system used by the European Community (\ili{English} and \ili{French} into \ili{German}), and the METAL system at the Institute of Applied Linguistics in Hildesheim (\ili{German} into \ili{English}). Three types of data were collected: for PE with and without the source text and for translation from scratch. As think-aloud is a quite intrusive method, three control data sets were recorded: two sets with other verbal data, i.e. retrospective commentary and dialogue protocols, and one without verbal data. The study was conducted with pen and paper, except for one control set. (cf. ibid.: 186-195) Taking all languages, translation modes, methods, tools, and participants into account (the participants in one experiment were professionals), 13 experiments were prepared (for an exact list see Table 5.4 in ibid.: 194). Altogether, 52 subjects took part in 48 sessions. The participants were enrolled in the technical translation studies programme in Hildesheim and were picked from a pool of volunteers. The experimental session took about 2.5 hours and the participants received monetary compensation. Further, they had to fill out a questionnaire that gathered basic information on their course of studies before the experiment. Some dictionaries (mono- and bilingual) as well as one encyclopedia on sciences and technology were provided for the participants. (cf. ibid.: 195-204) To put the analysis of the TAPs into perspective, MT output was ranked on the sentence level – on a scale from 1 to 5, one meaning that the quality is poor, five meaning the quality is good. These results were then compared with the monolingually post-edited texts, which were given 0.81 to 1.57 points more than the MT output. (cf. ibid.: 253-258) Then, attention was shifted to parameters concerning time, verbalisation, and final product. Time-related parameters were processing time, processing speed, and relative PE effort. The processing speed was a little higher for PE without source text and higher for PE without thinking aloud. In addition, experienced post-editors were a bit slower and the quality of the MT output seemed to show a negative correlation with processing speed (the better the output the less time was needed). Relative PE effort as the relation of translation speed to PE speed\footnote{A quotient of one would mean that PE was as fast as translation from scratch, a quotient under one would mean that PE was faster and over one that PE was slower than translation from scratch.} showed that PE was 7-20\% faster. (cf. ibid.: 276-286) Further, verbalisation effort and relative PE effort in relation to verbalisation effort were measured as well as the similarity of the MT output and the post-edited text. For the latter, Krings (cf. ibid.: 300-301) found that the final texts were 36.9\% similar to the MT output, ranging from 24.2\% to 44.6\% between the texts. In the next step, the recorded processes were categorised, which demonstrated that all tasks showed a comparable basic structure, involving seven distinct processes and various sub-processes. Most of these processes were target text related (about two thirds) and the most time was spent on the process of text production. Research in the reference books was chosen more often in TfS than in PE and in PE high quality MT output requires much less research than low and medium quality output. Interestingly, PE (with the source text) demanded more source text related processes than TfS, and PE effort was higher on medium quality MT than on low level MT.
\largerpage
Finally, text production and text evaluation processes were independent of MT quality. Krings (cf. ibid.: 318-320) further summarises that – considering changes in attention focus – less cognitive effort is necessary for PE than for TfS, but in general “\isi{post-editing}, seen as a process, led not to less, but rather tended towards more cognitive effort” (ibid: 534).


\newpage 
In her dissertation, %\label{ref:ZOTEROITEMCSLCITATIONcitationIDkZTcw3LvpropertiesformattedCitationrtfOuc0u8217Brien2006plainCitationOBrien2006citationItemsid214urishttpzoteroorgusers1255332itemsAE7QW2HGurihttpzoteroorgusers1255332itemsAE7QW2HGitemDataid214typethesistitleMachinetranslatabilityandposteditingeffortAnempiricalstudyusingTranslogandChoiceNetworkAnalysispublisherDublinCityUniversityauthorfamilyOBriengivenSharonissueddateparts2006schemahttpsgithubcomcitationstylelanguageschemarawmastercslcitationjsonRND5h9yt0l7yn}
\citet{OBrien2006}
analysed the impact of negative translatability indicators (NTIs) on PE effort. She used controlled language checkers to locate the NTI instances. Her participants were twelve professional translators who worked at IBM and had to fill out a questionnaire in advance to check whether they were suitable for the study. Nine of them had to post-edit the MT output, while three translated the texts from scratch as a baseline. The text was from the IT domain and contained 1777 words. Additionally, different kinds of passages were chosen that contained typical characteristics of IT texts such as descriptive and instructive passages, lists, abbreviations, menu names, etc. O’Brien used \isi{keylogging software} (\isi{Translog}) and Choice Network Analysis for her analysis. First, she analysed the temporal effort during PE. In general, PE was faster than human translation. Her results show that NTIs significantly extend the PE duration of each segment. Most segments caused a lower Relative Post Editing Effort (RPE) than translation effort. However, not all NTIs have the same effect. Some seem to be more demanding than others. The analysis of technical effort showed that segments with few NTIs needed significantly fewer insertions and deletions than those with many. Finally, the Choice Network Analysis showed again that some NTIs influence PE effort more than others.



The study by %\label{ref:ZOTEROITEMCSLCITATIONcitationID7231PWj4propertiesformattedCitationArenas2008plainCitationArenas2008citationItemsid157urishttpzoteroorgusers1255332itemsNR4JFAWHurihttpzoteroorgusers1255332itemsNR4JFAWHitemDataid157typearticlejournaltitleProductivityandqualityintheposteditingofoutputsfromtranslationmemoriesandmachinetranslationcontainertitleTheInternationalJournalofLocalisationpage1121volume7issue1authorfamilyArenasgivenAnaGuerberofissueddateparts2008schemahttpsgithubcomcitationstylelanguageschemarawmastercslcitationjsonRNDdrvb6BFkVQ}
\citet{Arenas2008} compares the productivity and final quality of fuzzy matches and post-edited segments. Nine participants, all of them professional translators, had to translate segments from scratch, edit fuzzy matches (different degrees of agreement), and post-edit MT output without knowing the origin of the pre-translation. The job was performed in a special web-based PE tool that records the editing\slash translation time and was fed with a trained MT system, TM entries, and a terminology list. Further, the participants had to fill out a questionnaire dealing with their experience concerning localisation, tools, domains, and PE. Three hypotheses were investigated: First, Arenas assumed that \isi{post-editing} MT output would take as long as editing a fuzzy match segment with 80--90\% agreement. Further, it was hypothesised that the quality of an edited fuzzy match and post-edited MT output is equal. Finally, it was assumed that participants with greater technical knowledge would be more productive as the texts were taken from the localisation industry. The analysis showed that, on average, the participants were faster when they post-edited MT output (25\% faster than from scratch), and when they edited fuzzy matches (11\% faster than from scratch). They were slowest when they translated from scratch. (cf. ibid.: 14) Interestingly, most mistakes were found in the final text segments when they originated from TM output. The fewest mistakes were made in translations from scratch, except for two participants who made fewer mistakes when \isi{post-editing} MT, and one who made the same number of mistakes in both segment types. We have to keep in mind, though, that the translators could not go back once they marked the segment as done and could not review their translations. The total number of errors in segments translated from scratch and post-edited segments was quite similar (27 and 34, respectively), while the distance to the TM segments was much higher (64 mistakes in total). The error type that occurred most often is “accuracy”. (cf. ibid.: 15-16) %\label{ref:ZOTEROITEMCSLCITATIONcitationID2leUqfVHpropertiesformattedCitationArenas2008plainCitationArenas2008citationItemsid157urishttpzoteroorgusers1255332itemsNR4JFAWHurihttpzoteroorgusers1255332itemsNR4JFAWHitemDataid157typearticlejournaltitleProductivityandqualityintheposteditingofoutputsfromtranslationmemoriesandmachinetranslationcontainertitleTheInternationalJournalofLocalisationpage1121volume7issue1authorfamilyArenasgivenAnaGuerberofissueddateparts2008schemahttpsgithubcomcitationstylelanguageschemarawmastercslcitationjsonRNDqAAcoO6V8Z}
Arenas (ibid.) argues that TM segments contain more errors because they are more fluent, as they originate from other manual translations and which makes it harder to detect mistakes. In the next step, a penalty according to the number of mistakes the participant made in the segment type was added to the processing speed. The new productivity gain calculations, including the penalty, showed that six out of eight participants were still faster when they post-edited MT output instead of translating the segments from scratch. However, only three were faster when editing fuzzy matches (cf. ibid: 17). In general, editing MT output was still 25\% faster, while editing fuzzy matches was 3\% slower. Finally, experience seemed to have a positive influence on the processing speed, but not on error rates. (cf. ibid.: 18-19)



In a pilot study, %\label{ref:ZOTEROITEMCSLCITATIONcitationIDya0ZbgB0propertiesformattedCitationCarletal2011plainCitationCarletal2011citationItemsid20urishttpzoteroorgusers1255332itemsMPWXCXVRurihttpzoteroorgusers1255332itemsMPWXCXVRitemDataid20typechaptertitleTheprocessofposteditingapilotstudycontainertitleCopenhagenStudiesinLanguage41publisherplaceCopenhagenDenmarkpage131142eventplaceCopenhagenDenmarkauthorfamilyCarlgivenMichaelfamilyDragstedgivenBarbarafamilyElminggivenJacobfamilyHardtgivenDanielfamilyJakobsengivenArntLykkeissueddateparts2011schemahttpsgithubcomcitationstylelanguageschemarawmastercslcitationjsonRNDLOmgAVckLN}
\citet{CarlEtAl2011} compared the PE behaviour of seven post-editors to the behaviour of 24 translators who translated from scratch and the quality of the output of these sessions. For the latter, four evaluators were presented with one source sentence and four final versions – two post-edited sentences and two human translated sentences. The evaluators had to rank the four final versions from best to worst – ties were allowed – without knowing whether the sentences were post-edited or translated from scratch. Interestingly, the post-edited sentences achieved an altogether better rating than the human translations. However, some sentences were presented to the evaluators twice and did not necessarily get the same ranking, which indicates that the assessment may not be completely reliable. (cf. ibid: 133-136) The editing distance did not correlate with the score of the post-edited sentence, which shows that more editing does not necessarily improve the quality. The PE sessions took on average only slightly less time (7 min 35 s for PE per text vs. 7 min 52 s for TfS per text), which is quite surprising, but the post-editors were not experienced in PE and CAT tools, while the translators were experienced. The eyetracking analysis showed that more \isi{gaze time} was spent on the screen for PE than for TfS. There might be two reasons for this: First, the translators might have spent more time looking at the keyboard as they initially had to produce text and second, they might have had to spend more time thinking about a translation solution and they did not have to look at the screen while they think. The eyetracking data further showed that in PE, more time was spent processing the target text (\isi{total gaze time} and fixation count were significantly higher on the target text), while source and target text were looked at equally long in translation from scratch. Moreover, the fixations on the source text are significantly longer than on the target text in the translation from scratch task. (cf. ibid.: 138-140) %\label{ref:ZOTEROITEMCSLCITATIONcitationIDywpzBWggpropertiesformattedCitationCarletal2011plainCitationCarletal2011citationItemsid20urishttpzoteroorgusers1255332itemsMPWXCXVRurihttpzoteroorgusers1255332itemsMPWXCXVRitemDataid20typechaptertitleTheprocessofposteditingapilotstudycontainertitleCopenhagenStudiesinLanguage41publisherplaceCopenhagenDenmarkpage131142eventplaceCopenhagenDenmarkauthorfamilyCarlgivenMichaelfamilyDragstedgivenBarbarafamilyElminggivenJacobfamilyHardtgivenDanielfamilyJakobsengivenArntLykkeissueddateparts2011schemahttpsgithubcomcitationstylelanguageschemarawmastercslcitationjsonRNDU3lBJtmi7H}
Carl et al. (ibid.: 140) explain that


\begin{quote}
[m]anual translation seems to imply a deeper understanding of the ST, requiring more effort and thus longer fixations, whereas in \isi{post-editing}, the ST is consulted frequently but briefly in order to check that the SMT output is an accurate and\slash or adequate reproduction of the ST.
\end{quote}


%\label{ref:ZOTEROITEMCSLCITATIONcitationIDaaBMVWOBpropertiesformattedCitationDeAlmeida2013plainCitationDeAlmeida2013citationItemsid221urishttpzoteroorgusers1255332itemsKKNP5KHKurihttpzoteroorgusers1255332itemsKKNP5KHKitemDataid221typethesistitleTranslatingtheposteditoraninvestigationofposteditingchangesandcorrelationswithprofessionalexperienceacrosstwoRomancelanguagespublisherDublinCityUniversityURLhttpdorasdcuie177321THESISGdeAlmeidapdfauthorfamilyDeAlmeidagivenGiselleissueddateparts2013schemahttpsgithubcomcitationstylelanguageschemarawmastercslcitationjsonRNDTYyAlWRqgX}
\citeauthor{De_almeida2013} %The citation is not in the list!
investigates two main questions in her dissertation from 2003. First, whether translation experience influences the PE performance and second, whether similar languages evoke similar PE behaviour. The study further explores typical difficulties in PE and introduces strategies to cope with these difficulties. Finally, the insights of the study can be used to improve MT systems and develop new MT-related translation tools. A total of 20 translators participated in the study – 10 translating into \ili{Brazilian Portuguese} and 10 into \ili{French} (all of them native speakers of the respective language). The participants were either professionals or students. Some had PE experience, some did not. In the recruiting phase, participants had to complete a short survey that was concerned with translation and PE experience, as well as academic education. An additional questionnaire had to be filled out right before the experiments and dealt with the participants' attitude towards MT and PE. Further, the PE sessions were recorded with a screen recording and \isi{keylogging software}. All data were combined with the final PE products to analyse the process. The texts were from the field of IT and contained 1008 words (74 segments). The participants could use the Internet for research if they wanted. The time for the sessions was limited to two hours and the participants were paid for their work (cf. ibid.: 73-77). In addition, the workbench in which the participants had to post-edit was similar to SDL Trados and PE instructions were provided that explained the task, defined PE, outlined the expected quality and listed which changes needed to be carried out (cf. ibid.: 108-116). The PE products were classified according to the following schema (cf. ibid.: 95):


\begin{itemize}
\sloppy
\item master categories:\\ essential changes, preferential changes, essential changes not implemented, introduced errors
\fussy
\item subclasses of master categories:\\ accuracy, consistency, country, format, language, mistranslation, style, lexical choice (with further subcategories for accuracy, country, and language)
\item to examine how former translation and PE experience as well as attitude towards MT influence the PE performance, the following items were analysed:
\item number of corrections
\item type of corrections (according to schema introduced above)
\item total time
\item switches between keyboard and mouse (and time spent per input method)
\item amount of conducted online research
\item items researched
\item existence of final revision
\end{itemize}

Further, the data of the two languages were compared to test whether similar strategies were used in both languages. The analysis part of the study presents correlations between former experience with and attitude toward the categories introduced above, but unfortunately does not report the p-value of the correlations, so the reader does not know whether these results are statistically significant. She concludes, however, that previous translation and PE experience do not influence the PE performance. Translation and PE experience does not seem to influence PE time. Further, translation experience does not influence the decision whether or not to revise the text. Inexperienced translators seem to have researched more online than experienced translators. The participants who performed best overall had translation and PE experience and conducted little to no research. Similar changes were made in both languages indicating similar PE behaviour in related languages. (cf. ibid.: 199-201)



The cognitive demand and cognitive effort in PE are topics of major interest in PE research, because they are, aside from PE productivity, the most important indicators of whether or not PE is more effective than TfS. %\label{ref:ZOTEROITEMCSLCITATIONcitationIDi5s1xDXlpropertiesformattedCitationLacruzDenkowskiandLavie2014plainCitationLacruzDenkowskiandLavie2014citationItemsid202urishttpzoteroorgusers1255332items4QGM2U2Zurihttpzoteroorgusers1255332items4QGM2U2ZitemDataid202typepaperconferencetitleCognitiveDemandandCognitiveEffortinPostEditingcontainertitleProceedingsoftheThirdWorkshoponPostEditingTechnologyandPracticepublisherAMTApage7384authorfamilyLacruzgivenIsabelfamilyDenkowskigivenMichaelfamilyLaviegivenAlonissueddateparts2014schemahttpsgithubcomcitationstylelanguageschemarawmastercslcitationjsonRNDtL01KX9fZ5}
\citet{LacruzEtAl2014} wanted to find an expressive measure for cognitive demand in their study, i.e. the demand established by the MT output rather than the cognitive effort that is actually required by the individual post-editor. They also wanted to investigate which pause ratio is usable for cognitive effort by correlating the pause to word ratio (PWR\footnote{PWR = number of pauses / number of words}) with the different pause thresholds; how the MT quality (measured in HTER\footnote{HTER = number of required edits / number of reference word  refers to the least number of necessary changes for PE; most post-editors, however, do not chose the easiest way; low HTER equals high MT quality}) influence PWR; and how MT quality ratings correlate with PWR. Finally, they hypothesise that the type of error in the MT output influences cognitive demand. (cf. ibid.: 75-79) Five participants with \ili{English} as their first language and \ili{Spanish} as their second took part in the study. They were enrolled in a Master's Programme in \ili{Spanish} translation and had all passed a course on PE and TM systems. The four \ili{Spanish} source texts could be considered texts written in general language – excerpts of TED talks\footnote{"Source texts were extracts of \ili{Spanish} language transcripts of TED talks on matters of general interest with little technical language." (ibid.: 79)}. Each text was automatically translated by two adaptive MT systems, i.e. the PE changes of one segment influenced the automatic translation of the next segment. Each participant translated each text once, i.e. two texts per MT system. The training sessions contained ten segments, while the remaining three experimental texts contained 30 segments each. They used an online PE tool that simultaneously logged key strokes. The participants could work from home. They were also requested to rate the MT quality on a scale from one (“gibberish”) to five (“very good”). The final texts were independently rated by two experienced translators (cf. ibid.: 79-80). The results of the study show that a pause threshold of 300ms seems to be very reasonable; that an increasing HTER score (decreasing MT quality) has a strong positive and significant correlation with PWR; that a low human rating of the MT output increases PWR accordingly (strong negative correlation); and that transfer errors – errors where the translator has to consult the source text – generate more cognitive demand than mechanical errors – errors that can be fixed without consulting the source text. (cf. ibid.: 80-82)



%\label{ref:ZOTEROITEMCSLCITATIONcitationID70HgLpo4propertiesformattedCitationMoorkensetal2015plainCitationMoorkensetal2015citationItemsid201urishttpzoteroorgusers1255332itemsV5JNX6XEurihttpzoteroorgusers1255332itemsV5JNX6XEitemDataid201typearticlejournaltitleCorrelationsofperceivedposteditingeffortwithmeasurementsofactualeffortcontainertitleMachineTranslationpage267284volume29issue34authorfamilyMoorkensgivenJossfamilyOBriengivenSharonfamilySilvagivenIgorALnondroppingparticledafamilyLimaFonsecagivenNormaBnondroppingparticledefamilyAlvesgivenFabioissueddateparts2015schemahttpsgithubcomcitationstylelanguageschemarawmastercslcitationjsonRNDV8Sk7waznJ}
In their study, \citet{MoorkensEtAl2015} tested how perceived PE effort actually correlates with real PE effort, including temporal, technical, and cognitive aspects (the latter was measured with eyetracking data). Three stages and two groups were necessary for this study. Group 1 involved six professional translators who first had to rate two texts on how much effort it would take to post-edit the single segments (Stage 1). A few weeks later, the same participants had to post-edit the same texts (Stage 2); only four participants completed this task entirely so that the estimated and the actual PE effort could only be compared for these four participants. Finally, students with little PE experience post-edited the texts and their temporal and technical effort was also recorded (Stage 3). Additionally, they received an indicator for half the segments showing how high the PE effort had been estimated for the segment in Stage 1 (green – low effort, yellow – medium effort, red – high effort) to measure whether this has an influence on PE performance. The texts were Wikipedia excerpts written in general language and were post-edited from \ili{English} into \ili{Brazilian Portuguese} in an online PE tool. (cf. ibid.: 267--273) The correlation of the individual ratings of the segments' estimated PE effort and the average group ratings were significant but not very strong ($r=.373$), which already indicates that perceived PE effort is not really reliable. Similarly, the rating of the estimated PE effort correlated significantly but not strongly ($r=.492$) with the time the participants needed to post-edit the segments (cf. ibid.: 274--276). When measuring cognitive effort through eyetracking, it could be observed that the eyetracking data (total numbers of fixation and mean \isi{fixation duration}) increased when the segments were rated badly; however, there was only a significant difference between green and yellow segments and green and red segments, but not between yellow and red segments. Furthermore, manual ratings and technical effort correlated significantly and strongly ($r=.652$), while temporal and technical effort correlated significantly and moderately ($r=.524$). (cf. ibid.: 276--278). The significant correlations between eyetracking data and time propose “that some segments presented time-consuming PE problems that required a related measure of cognitive effort without requiring a related amount of edits to the text.” (ibid.: 278) Time spent on PE was very similar for the professional and the student groups, but students edited more than professionals. Further, the indication of the segments' ratings did not noticeably influence PE time or technical effort. (cf. ibid.: 278-281) In summary, the study showed on the one hand that the ratings of perceived effort predict the actual effort, but not as strongly and confidently as might be expected and on the other hand that confidence scores might not be beneficial at all. (cf. ibid.: 281--282)



To conclude, the main task in empirical PE research seems to be – as predicted by %\label{ref:ZOTEROITEMCSLCITATIONcitationIDQ52qUKfkpropertiesformattedCitationAllen2003plainCitationAllen2003citationItemsid160urishttpzoteroorgusers1255332itemsWFGP5FUKurihttpzoteroorgusers1255332itemsWFGP5FUKitemDataid160typechaptertitlePosteditingcontainertitleComputersandTranslationAtranslatorsguidepublisherJohnBenjaminsTranslationLibrarypublisherplaceAmsterdamPhiladelphiapage297318volume35eventplaceAmsterdamPhiladelphiaauthorfamilyAllengivenJeffreyeditorfamilySomersgivenHaroldissueddateparts2003schemahttpsgithubcomcitationstylelanguageschemarawmastercslcitationjsonRNDmzGxq6MDzP}
\citet[298]{Allen2003}, see \chapref{sec:3} – to prove that PE is more efficient than translation from scratch with or without the help of TM systems in regard to temporal, technical, and cognitive effort. More studies on PE will be presented in \sectref{sec:7:4} in regard to the data set and in \sectref{sec:9:1} with regard to research efforts in PE.


\section{Post-editing and machine translation applications in practice}
\label{sec:4:3}

%\label{ref:ZOTEROITEMCSLCITATIONcitationID2zi8UKFjpropertiesformattedCitationKrings2001plainCitationKrings2001dontUpdatetruecitationItemsid228urishttpzoteroorgusers1255332items4D665XEKurihttpzoteroorgusers1255332items4D665XEKitemDataid228typebooktitleRepairingtextsempiricalinvestigationsofmachinetranslationposteditingprocessespublisherKentStateUniversityPresspublisherplaceKentOhionumberofpages635sourceLibraryofCongressISBNeventplaceKentOhioISBN9780873386715callnumberP309K75132001shortTitleRepairingtextslanguageengauthorfamilyKringsgivenHansPeditorfamilyKobygivenGeoffreySissueddateparts2001schemahttpsgithubcomcitationstylelanguageschemarawmastercslcitationjsonRNDAV4IIJ45rR}
\citet[558, emphasis in original text]{Krings2001} concludes his extensive think-aloud study on PE with the following statement on the practical use of MT: “As long as \textit{fully automated high-quality translation} remains an unreached future prospect […] how the machine can support the translator will in practice remain the true measure, and not the machine itself.” Accordingly, some organisations and companies will be introduced in this chapter that apply MT and\slash or PE in their everyday business or conduct research in (one of) the respective fields. The aim is to show that PE is not a job that might emerge for translators some day, but has already been established in some companies\slash organisations for years. This is intended as an excerpt of PE practice and not a complete elaboration, because on the one hand some examples suffice to paint the picture and on the other hand many companies and organisations do not publish much information about their processes and best practices (as also mentioned in %\label{ref:ZOTEROITEMCSLCITATIONcitationIDqaKnP4D7propertiesformattedCitationAllen2003plainCitationAllen2003citationItemsid160urishttpzoteroorgusers1255332itemsWFGP5FUKurihttpzoteroorgusers1255332itemsWFGP5FUKitemDataid160typechaptertitlePosteditingcontainertitleComputersandTranslationAtranslatorsguidepublisherJohnBenjaminsTranslationLibrarypublisherplaceAmsterdamPhiladelphiapage297318volume35eventplaceAmsterdamPhiladelphiaauthorfamilyAllengivenJeffreyeditorfamilySomersgivenHaroldissueddateparts2003schemahttpsgithubcomcitationstylelanguageschemarawmastercslcitationjsonRNDOYyrOpHxDe}
\citealt{Allen2003}).


\subsection{Pan American Health Organization (PAHO)}
\label{sec:4:3:1}

The Pan American Health Organization is a regional sub-organisation of the World Health Organization for the American continents and therefore part of the United Nations Organization. It was founded in 1902 after a yellow fever epidemic spread in parts of South America in 1870, which even reached the United States eight years later. The expansion of sea transportation enabled the international spread of diseases and a control instance between different countries became necessary. Altogether, PAHO has 35 member states and four associated members. The main goal of PAHO is to improve and maintain people’s health. Further objectives are to ensure technical cooperation between the member states to fight diseases and their causes, improve the health systems, and act in emergency situations. Everyone in the member states should be able to access the medical care that (s)he needs. (www.paho.org\footnote{\url{http://www.paho.org/hq/index.php?option=com_content& view=article&id=91&Itemid=220&lang=en}})

The PE service at PAHO might be the oldest and one of the best examples of PE in practice (%\label{ref:ZOTEROITEMCSLCITATIONcitationIDtPaPMftIpropertiesformattedCitationAymerich2004plainCitationAymerich2004citationItemsid54urishttpzoteroorgusers1255332itemsMJX7II5Aurihttpzoteroorgusers1255332itemsMJX7II5AitemDataid54typepaperconferencetitleUsingMachineTranslationforfastinexpensiveandaccuratehealthinformationassimilationanddisseminationExperiencesatthePanAmericanHealthOrganizationcontainertitleATMA2004SixthbiennalconferenceoftheAssociationforMachineTranslationintheAmericaspublisherplaceWaschingtonpage19eventplaceWaschingtonauthorfamilyAymerichgivenJuliaissueddateparts2004schemahttpsgithubcomcitationstylelanguageschemarawmastercslcitationjsonRNDIGoVWQB4qM}
\citealt{Aymerich2004}, %\label{ref:ZOTEROITEMCSLCITATIONcitationIDN1NbvnaMpropertiesformattedCitationAymerich2005plainCitationAymerich2005citationItemsid1146urishttpzoteroorggroups3587itemsUD5I4T8Surihttpzoteroorggroups3587itemsUD5I4T8SitemDataid1146typepaperconferencetitleUsingMachineTranslationforFastInexpensiveandAccurateHealthInformationAssimilationandDisseminationExperiencesatthePanAmericanHealthOrganizationcontainertitleProceedingsofthe9thWorldCongressonHealthInformationandLibrariespublisherplaceSalvadorBahiaBrazilevent9thWorldCongressonHealthInformationandLibrarieseventplaceSalvadorBahiaBrazilURLhttpwwwpahoorgenglishamgsptr2005ICML9AymerichpdfauthorfamilyAymerichgivenJuliaissueddateparts2005schemahttpsgithubcomcitationstylelanguageschemarawmastercslcitationjsonRNDOM3k8dp5hq}
\citealt{Aymerich2005}). The success of PE is based on the organisation's own rule-based MT system (named PAHOMTS) that was established in 1980. At that time, the first language combination that was established was from \ili{Spanish} into \ili{English}. Nowadays, PAHOMTS includes engines for all language combinations between \ili{English}, \ili{Spanish} and \ili{Portuguese}. A special characteristic of PAHOMTS is that the engine was not only developed solely by computational linguists, but is also improved by translators who give feedback on the MT output so that dictionaries and algorithms can be adapted accordingly. PAHOMTS runs on Windows (Windows Vista – Windows 10), has a trilingual user interface as well as trilingual online support, and each MT dictionary contains over 150,000 words, phrases and rules. The latest version (4.12) was released in December 2015. (cf. www.paho.org\footnote{\url{www1.paho.org/english/am/gsp/tr/machine_trans.htm} last accessed 28 July 2018.})

The PE activities began with the establishment of PAHOMTS, which has been trained with PAHO documents for decades now. Therefore, the MT output is of very good quality in the PAHO contexts and is only post-edited if the document is intended for publication and not only for gisting purposes. Furthermore, the long-lasting use and the well-tried functionality of PAHOMTS explains why PAHO still uses a rule-based MT system, because it has been customised so well that newer approaches would not improve the MT output. All in all, MT is used to prepare 90 percent of the documents. The translators rarely use the MT in combination with \isi{translation memory} systems (only for five percent of all translation jobs) and when they do, it is only for financial reports as well as governing body documents due to the repetitiveness of these text types. When a translator has to post-edit a text, (s)he gets the source document, background texts if available, the unedited translation file, a side-by-side file, and the list with the words that had no dictionary entry. The MT systems are not only specialised on medical texts but can also be used for manuals, reports, scientific articles, etc. Finally, the source texts neither undergo pre-editing processes nor are they written in controlled language, although an assistant revises the source text according to the general guidelines, e.g. spell check or formatting (cf. %\label{ref:ZOTEROITEMCSLCITATIONcitationIDVkWCyBr4propertiesformattedCitationAymerich2004plainCitationAymerich2004citationItemsid54urishttpzoteroorgusers1255332itemsMJX7II5Aurihttpzoteroorgusers1255332itemsMJX7II5AitemDataid54typepaperconferencetitleUsingMachineTranslationforfastinexpensiveandaccuratehealthinformationassimilationanddisseminationExperiencesatthePanAmericanHealthOrganizationcontainertitleATMA2004SixthbiennalconferenceoftheAssociationforMachineTranslationintheAmericaspublisherplaceWaschingtonpage19eventplaceWaschingtonauthorfamilyAymerichgivenJuliaissueddateparts2004schemahttpsgithubcomcitationstylelanguageschemarawmastercslcitationjsonRNDywfWLu0avm}
\citealt{Aymerich2004}).

PAHOMTS has processed over 88 million words since 1980. Thanks to \isi{post-editing}, the increase in productivity is estimated at 30--50\%. In addition, licenses for PAHOMTS can be bought by educational institutions, international and intergovernmental organisations, government agencies and NGOs, but not by private persons or businesses (cf. www.paho.org\footnote{\url{www1.paho.org/english/am/gsp/tr/machine_trans.htm} last accessed 28 July 2018.}).

\subsection{European Commission (EC)}
\label{sec:4:3:2}

As was already mentioned in \sectref{sec:2:1} and \sectref{sec:3:1}, the European Commission also started to approach MT relatively early. With 24 official languages today, and policies that specify that every official document needs to be available for every citizen in the official language(s) of the country (s)he lives in, the need for MT in the European Union is obvious. SYSTRAN was already established in the European Commission (EC) in 1976. However, it only became widely used when e-mails became a reliable source of communication for the different departments in the early 1990s and after the EUROTRA project (see \sectref{sec:3:1}) did not deliver a working system. The use of MT increased to 260,000 pages per year in 1998 (cf. %\label{ref:ZOTEROITEMCSLCITATIONcitationID84P1LfkJpropertiesformattedCitationSenez1998plainCitationSenez1998citationItemsid52urishttpzoteroorgusers1255332itemsP5EHJJUQurihttpzoteroorgusers1255332itemsP5EHJJUQitemDataid52typepaperconferencetitlePosteditingserviceformachinetranslationusersattheEuropeanCommissioncontainertitleTranslatingandtheComputer20ProceedingsfromAslibconferencepublisherplaceLondonpage16eventplaceLondonauthorfamilySenezgivenDorothyissueddateparts19981112schemahttpsgithubcomcitationstylelanguageschemarawmastercslcitationjsonRND12xYBdrcwR}
\citealt{Senez1998}: 1).


In 1994, the PER-Service (PER~=~post-édition rapide\slash rapid \isi{post-editing}) was established at the EC. At the beginning, a small group of freelancers volunteered to handle the PE tasks. They were not trained in the task, but gained experience in this new area through practice. MT and the PER-Service were only used when necessary, e.g. when deadlines did not allow for human translation, because there was only enough time available to make a few changes in the MT output. The customers of the PER-Service had to complete questionnaires concerning general satisfaction with the service as well as feedback on the terminology so that new terminology could be fed into SYSTRAN and changes could be communicated to the translators. The use of the service increased by 20--50\% per year between 1994 and 1998 and one post-edited page was about half the price of a human translated page (cf. ibid.: 2-3).



The first job vacancy for PE was advertised in 1998 when the EC was looking for a post-editor for the languages \ili{German}, \ili{English}, and \ili{French}. In general, the EC distinguished between c\textit{orrecting} MT output, which meant that the MT output was used as the first draft of the translation and was then edited into a full translation, and \textit{post-editing} MT output, for which the final output did not need to be perfect and that was only the chosen approach if there was not enough time and the target text was not intended for publication. The main goal of post-edited documents was to reliably deliver the information and content; style was not important. The final product was shaped by the urgency of the task and therefore perfection was not the main objective (cf. ibid.: 3-6).



Different environments were developed in the context of MT. The \textit{Machine Translation Help Desk} was introduced to enable communication between developers and users of SYSTRAN. Further, the POETRY interface was created and allowed the customer to choose what was supposed to happen with the document: it could either be translated by humans, summarised in writing or orally, proof-read, or post-edited (cf. ibid.: 2-5).



While the early MT system was able to translate \ili{German}, \ili{English}, and \ili{French}, things were about to change for MT when it was decided to add the rest of the official languages as well. The goal of incorporating all languages combinations in an MT system finally seemed impossible after the latest EU expansions in 2004 and 2007. The MTS was ruled-based and it would have taken far too long to develop new MT engines for all languages, language pairs, and language directions. At that time, \isi{translation memory} systems were considered to be much more effective than MT because less money needed to be invested to make the systems efficient and they worked equally well for all language pairs. Hence, financing for the development of MT ended. However, \isi{translation memory} systems were neither the perfect nor the final answer because they could only repeat what had been translated before (cf. %\label{ref:ZOTEROITEMCSLCITATIONcitationIDKziV8WXfpropertiesformattedCitationBonet2013plainCitationBonet2013citationItemsid42urishttpzoteroorgusers1255332items75DVDBCQurihttpzoteroorgusers1255332items75DVDBCQitemDataid42typearticlejournaltitleNorageagainstthemachinecontainertitleLanguagesandtranslationpage45issue6authorfamilyBonetgivenJosepissueddateparts2013schemahttpsgithubcomcitationstylelanguageschemarawmastercslcitationjsonRNDROAEa492J0}
\citealt{Bonet2013}: 4-5).



A superior solution for the EC's requirements was found in statistic MT and resulted in the launching of a new MT project in 2010. The aims of this project were the following: The rule-based MTS was replaced with a statistical one; the MTS was to be used by every member of the EC; communication was to become faster; the judgement of whether or not a text required translation was to become easier; and experts were to be able to communicate their knowledge no matter how well they knew the language. (ec.europa.eu\footnote{\url{http://ec.europa.eu/isa/actions/02-interoperability-architecture/2-8action_en.htm},\\ last accessed on 7\textsuperscript{th}  November 2016})



The MT system called \textit{MT@EC} is available free of charge to the staff working for an EU body or agency as well as for all public administrations of any EU country, Norway, and Iceland. Furthermore, interested individuals can download \isi{translation memory} entries for free. Documents in eleven different formats and text snippets can be automatically translated within seconds or a few minutes – depending on the length of the documents – in all language combinations of the 24 official EU languages. The output retains the original format and indicates the expected quality. The website specifically states that the MT output is raw translation data and that a “skilled professional translator” must revise the text if a high-quality translation is required (www.ec.europa.eu\footnote{\url{http://ec.europa.eu/dgs/translation/translationresources/machine_translation/index_en.htm}, last accessed 7\textsuperscript{th} November 2016}).


\subsection{Ford}
\label{sec:4:3:3}

The Ford Motor Company was founded by Henry Ford in 1903. The current headquarters are in Darborn, Michigan, USA. In 2015, the company employed about 199,000 people in 67 plants worldwide. %\label{ref:ZOTEROITEMCSLCITATIONcitationIDsYDxuSRUpropertiesformattedCitationFORD2016plainCitationFORD2016citationItemsid198urishttpzoteroorgusers1255332itemsSW25JNBKurihttpzoteroorgusers1255332itemsSW25JNBKitemDataid198typereporttitleFordMotorCompany2015AnnualReportpublisherFordMotorCompanyURLhttpcorporatefordcomcontentdamcorporateeninvestorsreportsandfilingsAnnual20Reports2015AnnualReportpdfauthorfamilyFORDgivenissueddateparts2016schemahttpsgithubcomcitationstylelanguageschemarawmastercslcitationjsonRNDl7FK5UkRgl}
\citet[1]{Ford2016} Years before Ford started to use MT systems in 1998, they had already established a controlled language. In 1990, Ford established the \textit{Standard Language} at \textit{Ford Body \& Assembly Operations} in the USA. Standard Language is a Ford-specific, restricted version of \ili{English} that focuses on vehicle assembly processes. It is only used in unpublished documents, but is used by the staff in Ford plants around the world. Further, an AI-system uses process sheets that are composed in Standard Language to generate work assembly instructions. First introduced in the USA, Standard Language has spread into Ford plants in Europe, South America, and Asia, too. (cf. %\label{ref:ZOTEROITEMCSLCITATIONcitationIDjbk6h488propertiesformattedCitationRychtyckyj2006plainCitationRychtyckyj2006citationItemsid41urishttpzoteroorgusers1255332itemsE8WUJPDPurihttpzoteroorgusers1255332itemsE8WUJPDPitemDataid41typepaperconferencetitleStandardlanguageatFordMotorsacasestudyincontrolledlanguagedevelopmentanddeploymentcontainertitleCLAW20065thInternationalWorkshoponControlledLanguageApplicationspublisherplaceCambridgeMAUSAeventCLAW20065thInternationalWorkshoponControlledLanguageApplicationseventplaceCambridgeMAUSAauthorfamilyRychtyckyjgivenNestorissueddateparts2006612schemahttpsgithubcomcitationstylelanguageschemarawmastercslcitationjsonRNDhFV33yio1g}
\citealt{Rychtyckyj2006}: 1)



SYSTRAN was introduced for MT in 1998. Many challenges arose because the MT system needed to be adapted to the controlled language, called Standard Language. Standard Language uses, e.g., unconventional or non-existing grammar rules to specify information on time and motion. These rules not only need equivalents in the target languages but are also unknown to the MT system. However, the controlled language was also adapted to the MT system in some aspects, e.g., the system now adds articles (which are optional in Standard Language and are often left out to save time) to words\slash phrases when parsing. This improves the quality of the MT output. (cf. ibid: 6-8). The MT system is capable of translating \ili{English} into \ili{German}, \ili{Dutch}, \ili{Spanish}, and \ili{Portuguese}. In addition to the texts in Standard Language, the MT system also has to translate comments that the authors added to the instructions, which are in natural language and hence more difficult to translate for the MT system that was adapted to Standard Language. Therefore, an additional component was added to the system that converts the natural language into a more MT-friendly language before the translation is performed. In general, the translation of the process sheets does not require human intervention, but the employees at the assembly plants can correct the translation manually in the online system if they think it is necessary. When the glossary is updated, the process sheets are re-translated so that users can benefit from the changes %\label{ref:ZOTEROITEMCSLCITATIONcitationID90eEyeHVpropertiesformattedCitationRychtyckyj2007plainCitationRychtyckyj2007citationItemsid231urishttpzoteroorgusers1255332itemsGCIK8RM2urihttpzoteroorgusers1255332itemsGCIK8RM2itemDataid231typearticlejournaltitleMachinetranslationformanufacturingacasestudyatFordMotorCompanycontainertitleAIMagazinepage17281735volume28issue3authorfamilyRychtyckyjgivenNestorissueddateparts2007schemahttpsgithubcomcitationstylelanguageschemarawmastercslcitationjsonRNDjSfWFm1SkN}
(cf. \citealt{Rychtyckyj2007}).


\subsection{DARPA}
\label{sec:4:3:4}

The first wave of MT financing and development can be traced to the early years of the Cold War. And even today, MT is still important in the military sector. DARPA (\textit{Defence Advanced Research Projects Agency}) is a US agency that was established in 1957 with the launch of Sputnik and is part of the US Defence Department. The mission of DARPA is to finance new technologies for national security. It employs 220 people who oversee about 250 research and development programs.\footnote{\url{http://www.darpa.mil/about-us/about-darpa}, last accessed 8 November 2016.} Some of these projects focus on MT application because, on the one hand, information from news or blogs in other languages needs to be accessed very quickly and, on the other, soldiers need technology to help them communicating with civilians. (cf. %\label{ref:ZOTEROITEMCSLCITATIONcitationID79R6LvjQpropertiesformattedCitationDARPA2008plainCitationDARPA2008citationItemsid36urishttpzoteroorgusers1255332itemsT8WQJ9B3urihttpzoteroorgusers1255332itemsT8WQJ9B3itemDataid36typechaptertitleTranslationtechnologyBreakingtheLanguageBarriercontainertitleDARPA50YearsofBridgingtheGappage98101authorfamilyDARPAgivenissueddateparts2008schemahttpsgithubcomcitationstylelanguageschemarawmastercslcitationjsonRNDKB0cYhTOlH}
\citealt[98]{Darpa2008}) In the following, some projects will be introduced briefly. Although they do not include PE tasks, these projects present important examples of how raw MT output can be used.



The first project presented is the GALE (\textit{Global Autonomous Language Exploitation}) programme. This programme concentrates on developing MT systems for \ili{Chinese} (Mandarin) and \ili{Arabic} (Modern \ili{Arabic} Standard Language) into \ili{English} to monitor news, web pages, and TV reports in real time. The system is also supposed to convert audio data into written text first if necessary. The ultimate goal was envisioned as automatically produced, live subtitles for news broadcasts and other TV shows. The previous system, \textit{eTAP,} was only 35--55\% accurate but still significantly reduced manual labour because it helped make decisions about whether a document needed to be translated. The result was that only 5\% of the documents were considered important enough for translation. The main GALE objective, however, was to raise accuracy up to 95\% for formal texts (slightly lower for informal texts) and to 90\% for (controlled) speech.\footnote{Reports on the final accuracy could not be found, which might suggest that the goals were not met.} The MT component is hybrid and consists of rule-based and statistical components. %\label{ref:ZOTEROITEMCSLCITATIONcitationIDVb9ELRC4propertiesformattedCitationDARPA2008plainCitationDARPA2008citationItemsid36urishttpzoteroorgusers1255332itemsT8WQJ9B3urihttpzoteroorgusers1255332itemsT8WQJ9B3itemDataid36typechaptertitleTranslationtechnologyBreakingtheLanguageBarriercontainertitleDARPA50YearsofBridgingtheGappage98101authorfamilyDARPAgivenissueddateparts2008schemahttpsgithubcomcitationstylelanguageschemarawmastercslcitationjsonRNDwi7Xu1EZNR}
(cf. ibid.: 98-100)



Another noteworthy project is the \isi{TRANSTAC} (\textit{Spoken Language Communication and Translation System for Tactical Use}) programme. This project focused on developing a bidirectional translator for spoken language to enable communication between soldiers and locals outside the USA. The main difference to GALE was its aim to capture spoken language, which is not as controlled as language on TV or in other media. Civilians speak in dialects and may have different pronunciation habits, which makes speech recognition and MT much more complicated. Further, the device should be mobile and hence has to be wearable. In 2001, a forerunner device was developed that could translate several hundred pre-defined spoken phrases into \ili{Arabic}, Pashto, and other languages. Ideally, TRANSTAC, should be able to use a lexicon with tens of thousands of entries as well as specialise “tactically relevant questions and answers” (ibid.: 101). After interviewing soldiers and marines about necessary phrases, native speakers of all languages involved (initially, \ili{English} and Iraqi \ili{Arabic}) were asked to record different interactions in a studio. The recordings and transcripts were used to train and build the machine. The system could handle 25 questions and answers in ten minutes in 2007 (cf. ibid.: 100-101). In the end, the system reached an accuracy of 80\% but did not gain much acceptance from the potential users. (cf. www.slate.com\footnote{\url{http://www.slate.com/articles/technology/future_tense/2012/05/darpa_s_transtac_bolt_and_other_machine_translation_programs_search_for_meaning_.html}, last accessed 8 November 2016.})



The task of the MADCAT (\textit{Multilingual Automatic Document Classification, Analysis and Translation)} programme is to translate foreign language text images into \ili{English}. The technologies are able to analyse, classify, and segment the image, determine the script and the text, produce transcripts in the source language, and finally produce an accurate translation into \ili{English}.\footnote{\url{http://www.darpa.mil/program/multilingual-automatic-document-classification-analysis-and-translation}, last accessed 8 November 2016})



The last project that will be presented here is the LORELEI (\textit{Low Resource Languages for Emergent Incidents}) project, which targets languages with low resources. The aim is to develop “partial or fully automated speech recognition and\slash or \isi{machine translation}” within 24 hours after a new language is needed, e.g. in emergency situations. The goal is not to develop a full working system, but to identify parts of the information in the respective language like names, places, topics, events, etc. (cf. http://www.darpa.mil\footnote{\url{http://www.darpa.mil/program/low-resource-languages-for-emergent-incidents}, last accessed 8 November 2016})


\section[Post-editing and machine translation in the professional translation community]{Post-editing and {machine translation} in the professional translation community\sectionmark{PE and MT in the professional translation community}}\sectionmark{PE and MT in the professional translation community}
\label{sec:4:4}

In this chapter, we will focus on MT in professional translation communities. The BDÜ\footnote{Bundesverband für Dolmetscher und Übersetzer – Federal Association for Interpreters and Translators} is one of the leading \ili{German} professional associations for interpreters and translators with more than 7500 members, and will be used as an example for the professional communities. The BDÜ has recognised the need to talk about MT and PE in recent years and published a number of articles, co-hosted a conference, and offered training in and on the topics. These will be briefly presented in the following.



First of all, the respective publications will be discussed. The internal magazine of the BDÜ is called MDÜ\footnote{Fachzeitschrift für Dolmetscher und Übersetzer – Professional Journal for Interpreters and Translators} and is published once every quarter. Two issues have been (partially) concerned with the topics of MT and PE in recent years. First, the final issue of 2012 called “The Future of Translation and Interpretation” devoted two articles to the topics. The first one by %\label{ref:ZOTEROITEMCSLCITATIONcitationIDNx3YVSWHpropertiesformattedCitationReinkeandSeewaldHeeg2012plainCitationReinkeandSeewaldHeeg2012citationItemsid219urishttpzoteroorgusers1255332itemsG699QPVFurihttpzoteroorgusers1255332itemsG699QPVFitemDataid219typearticlemagazinetitleDenTigerreitencontainertitleMDpage1014volume58issue4authorfamilyReinkegivenUwefamilySeewaldHeeggivenUtaissueddateparts2012schemahttpsgithubcomcitationstylelanguageschemarawmastercslcitationjsonRNDgBfN3Of4zJ}
\citet{ReinkeSeewald-Heeg2012} evaluates whether MT will be able to replace human translators. In the first part of the article, Reinke argues that context is important for comprehension and that natural language is very vague and hence is very problematic for machine processing. After very briefly introducing rule-based, statistical and hybrid MT approaches, he discusses useful applications of MT. They assess that use for professional translation is very restricted and that most texts would require PE, which, he muses, would be much more time consuming for many texts than human translation\footnote{An opinion which is not shared by me nor by most empirical studies on \isi{post-editing}.}. However, he acknowledges that the combined use of MT and \isi{TMS} can increase productivity by up to 40\% and sees the main use of MT in private translation for information gathering. He summarises that full automatic translation will not become a reality in the near future. Seewald-Heeg elaborates on the potential of combining MT and TM technologies. The second article by %\label{ref:ZOTEROITEMCSLCITATIONcitationIDk34Z0Zi3propertiesformattedCitationElsen2012plainCitationElsen2012citationItemsid55urishttpzoteroorgusers1255332items4Z4NSI7Murihttpzoteroorgusers1255332items4Z4NSI7MitemDataid55typearticlemagazinetitlePosteditingSchreckgespenstoderPerspektivecontainertitleMDpage1621volume58issue4authorfamilyElsengivenHaraldissueddateparts2012schemahttpsgithubcomcitationstylelanguageschemarawmastercslcitationjsonRND9PMBIs2SeE}
\citet{Elsen2012} deals with PE. He first generally defines what PE is and how it can be cost effective. Next, he explains how PE works in the TM environment and what differentiates PE from human translation. A sensible use of MT is only possible if the output is post-edited and the quality of the MT output is reasonable. He concludes that PE needs to be learned and that good post-editors will develop their skills with training, experiences, and good self-assessments. Finally, he adds that a sceptical opinion towards MT systems may even be good for post-editors – a contrary opinion to many academic writers and studies. These two articles are written for a target audience that knows a great deal about translation but only little about MT. While both cases only provide a rudimentary presentation of the different approaches to MT, the knowledge about TM systems is taken as a given. Further, the titles\footnote{“Den Tiger reiten” which translates literally as “ride the tiger” and figuratively roughly as “tame the beast”, and “Postediting – Schreckgespenst oder Perspektive?” which means “Postediting – Ghoul or Perspective?”.} of the articles already suggest that the target audience is sceptical towards MT, but that MT is nothing to be afraid of.



The second issue is a special issue on MT and contains five articles on the topic. The first article collection by %\label{ref:ZOTEROITEMCSLCITATIONcitationIDWsRSbAmrpropertiesformattedCitationKellerDutzandHofmann2016plainCitationKellerDutzandHofmann2016citationItemsid218urishttpzoteroorgusers1255332itemsZ8F73B6Wurihttpzoteroorgusers1255332itemsZ8F73B6WitemDataid218typearticlejournaltitleDreiModelleimFokuscontainertitleMDpage1014volume2016issue1authorfamilyKellergivenNicolefamilyDutzgivenPetrafamilyHofmanngivenNadriaissueddateparts2016schemahttpsgithubcomcitationstylelanguageschemarawmastercslcitationjsonRNDIS8KYzkMoN}
\citet{KellerEtAl2016} deals with technical aspects of the integration of MT in the TM systems \textit{Across v6.3}, \textit{SDL Trados} \textit{Studio 2015}, and \textit{STAR}. The next article by %\label{ref:ZOTEROITEMCSLCITATIONcitationIDkocxEjDhpropertiesformattedCitationrtfRuc0u252thHungerandAltmann2016plainCitationRthHungerandAltmann2016citationItemsid217urishttpzoteroorgusers1255332itemsUN47IC59urihttpzoteroorgusers1255332itemsUN47IC59itemDataid217typearticlejournaltitleErfahrungsberichtecontainertitleMDpage2427volume2016issue1authorfamilyRthgivenLisafamilyHungergivenAnnettefamilyAltmanngivenManfredissueddateparts2016schemahttpsgithubcomcitationstylelanguageschemarawmastercslcitationjsonRNDpTxDnCkS8T}
\citet{RuthEtAl2016} presents two reports with practical experiences of MT in real life professional translation. Rüth reports on the positive experiences she and her colleagues have had with the use of MT suggestions in the TM tool on a word\slash phrase basis rather than on a segment basis. If the automatically suggested word\slash phrase is reasonable, the translator can approve the suggested translation and continue with the rest of the segment; if it is not, the translator can ignore the suggestion and translates the word\slash phrase from scratch. Hunger and Altmann, on the other hand, present one good and one bad example of client behaviour. One client insisted on the use of the MT output and only wanted to pay the price of a fuzzy segment for the MT segments (with partly poor quality); an unreasonable amount for MT segments that needed major changes. Another client judged the MT output as suggestions and paid as much for MT segments as if they were translated from scratch – the translators were free to choose whether or not they wanted to use the MT output. %\label{ref:ZOTEROITEMCSLCITATIONcitationIDeMz3spZUpropertiesformattedCitationMuegge2016plainCitationMuegge2016citationItemsid216urishttpzoteroorgusers1255332itemsZBCWCPUWurihttpzoteroorgusers1255332itemsZBCWCPUWitemDataid216typearticlejournaltitleDoityourselfMcontainertitleMDpage1923volume2016issue1authorfamilyMueggegivenUweissueddateparts2016schemahttpsgithubcomcitationstylelanguageschemarawmastercslcitationjsonRNDQHbwbDfdVF}
\citet{Muegge2016} explains in his article how well-trained MT systems can become available for small- and medium-sized companies. After explaining statistical MT, traditional approaches to MT training, and the functionality of cloud-based MT, he advises the reader to maintain the TM data before feeding it to the MT system, to invest into (human) training courses, and to keep expectations realistic. %\label{ref:ZOTEROITEMCSLCITATIONcitationIDMkZXjHjgpropertiesformattedCitationNitzke2016korrekturplainCitationNitzke2016korrekturdontUpdatetruecitationItemsid226urishttpzoteroorgusers1255332itemsTKT6J8G7urihttpzoteroorgusers1255332itemsTKT6J8G7itemDataid226typearticlejournaltitleAuchnurKorrekturlesenPostEditingcontainertitleMDpage2427volume2016issue1authorfamilyNitzkegivenJeanissueddateparts2016schemahttpsgithubcomcitationstylelanguageschemarawmastercslcitationjsonRNDbNoi2itSPD}
\citet{Nitzke2016korrektur} explores the differences between PE and proof-reading, special characteristics of the PE task regarding the main occurring error types those that hardly occur in MT output, different PE requirements (light vs. full PE). She concludes that PE is not comparable to traditional proof-reading, but rather a special form of translation and that the translator\slash post-editor might have to advise the clients if they cannot entirely judge what \isi{post-editing} MT output means. The last article on PE and MT in this issue by %\label{ref:ZOTEROITEMCSLCITATIONcitationIDJXnTPdmApropertiesformattedCitationEbling2016plainCitationEbling2016citationItemsid215urishttpzoteroorgusers1255332items6VANJJGUurihttpzoteroorgusers1255332items6VANJJGUitemDataid215typearticlejournaltitleHierliegtmaschinellebersetzungaufderHandcontainertitleMDpage2831volume2016issue1authorfamilyEblinggivenSarahissueddateparts2016schemahttpsgithubcomcitationstylelanguageschemarawmastercslcitationjsonRNDyO4phEgKyQ}
\citet{Ebling2016} deals with the automatic processing of natural language into sign language and vice versa. The technology could be useful in everyday situations when a sign language interpreter is not available.



According to the content and the details of the articles, we can observe a change in the reception of MT and PE. While the articles in the 2012 issue provide more of an overview and suggest that the target audience might be uninformed and insecure about the topics, the articles in the 2016 issue are much more specialised and show that PE and MT have arrived in the everyday work environment of professional translators. This attitude was maintained by the BDÜ, which published a Best Practice guide %\label{ref:ZOTEROITEMCSLCITATIONcitationID68K8na1zpropertiesformattedCitationOttmann2017plainCitationOttmann2017citationItemsid170urishttpzoteroorgusers1255332itemsCRV7UMMTurihttpzoteroorgusers1255332itemsCRV7UMMTitemDataid170typebooktitleBestPracticesbersetzenundDolmetschenEinNachschlagewerkausderPraxisfrSprachmittlerundAuftraggebersourceOpenWorldCatISBN9783938430859noteOCLC968251708shortTitleBestPracticesbersetzenundDolmetschenlanguageGermanauthorfamilyOttmanngivenAngelikaissueddateparts2017schemahttpsgithubcomcitationstylelanguageschemarawmastercslcitationjsonRNDySpvMxdAkH}
\citep{Ottmann2017} for professional translators and interpreters covering all topics relevant for the market, including a whole chapter on PE.


\largerpage
The insecurity of the translation sector, which was already expressed in the earlier publications of the BDÜ, might explain the following publication (also available online\footnote{\url{http://www.bdue.de/uploads/media/2796_BDUe__Pressedossier_MenschMaschine_10.2012.pdf} last accessed 6 July 2017} since 2012); a rather bad example of information about MT published by the %\label{ref:ZOTEROITEMCSLCITATIONcitationIDC1DLlHoipropertiesformattedCitationrtfBDuc0u2202012plainCitationBD2012citationItemsid150urishttpzoteroorgusers1255332itemsUHRG8PWGurihttpzoteroorgusers1255332itemsUHRG8PWGitemDataid150typereporttitleMenschMaschineErgebnissederBDUntersuchungzurQualittderbersetzungendurchGoogleTranslatepublisherBDpublisherplaceBerlinpage44genrePressedossiereventplaceBerlinURLhttpwwwbduedeuploadsmedia2796BDUePressedossierMenschMaschine102012pdfauthorfamilyBDgivenissueddateparts201210schemahttpsgithubcomcitationstylelanguageschemarawmastercslcitationjsonRNDU6X7P34krD}
\citet{Bdu2012}. The article evaluated the use of Google Translate, back then a statistical MT system. The study deemed the programme a great online source for private communication, e.g., for holiday preparation or as an aid while on holiday but insinuated that the free programme was not suitable and reached its limits very quickly in business communication. The BDÜ article concludes that it can be embarrassing and bad for business to send error-laden e-mails or, even worse, run badly translated websites. (cf. ibid.: 9) Although the latter points are very true, the study itself and the way it was conducted have to be treated very critically. Professional translators were asked to evaluate MT output for common language texts (newspaper articles about politics and menus\slash recipes), a part of a manual for a technical gadget, general terms and conditions of an online shop, and a business e-mail. Although different domains were covered, only one translator evaluated each text per language combination (\ili{German} into \ili{English}, \ili{Spanish}, \ili{Polish}, and \ili{Chinese}) and each domain was represented by only one text. Further, the texts created by the MT system were evaluated using a pointing system that is equal to the \ili{German} grading system (1 to 6, with 1 being the best grade and 6 the worst) in the following categories: correct content, grammar, spelling, idiomacy, and overall satisfaction with the text. The grades for the MT output texts were very bad for most text types – except in the category spelling. However, this way of grading the texts is very subjective and does not represent what is actually important for MT output. From a research point of view, the question of interest should rather be: When the MT output is used in a professional environment, how much effort does it take to turn it into a reasonable target? In the BDÜ study, PE was also acknowledged as a necessary step to achieving a meaning target text (cf. ibid.), but was not explained or referred to in detail. The advantage of the study is that it shows translators and potential translation vendors that Google Translate is not almighty and that it cannot work without the help of a human translator. However, automatic translation is a rapidly developing branch and it has to be acknowledged how far the field has developed and that the systems can work quite adequately. Further, it should have been made more explicit that free online MT systems do not represent the best MT systems have to offer. Systems that are trained for one text domain and for one company will achieve much better results – assuming that they are used for the texts they were trained for: A system that was trained for manuals of household appliances will probably not produce good translations for medical package inserts.


\largerpage
Finally, the BDÜ has hosted three conferences so far that dealt with the professional fields of translation and interpreting. The first two were called “Übersetzen in die Zukunft” (“translating into the future”) and covered current developments in the field. The first conference, in 2009, included one presentation about PE and two presentation on MT (%\label{ref:ZOTEROITEMCSLCITATIONcitationIDGV2NvCFEpropertiesformattedCitationBauretal2009plainCitationBauretal2009dontUpdatetruenoteIndex0citationItemsid213urishttpzoteroorgusers1255332items4339JZDMurihttpzoteroorgusers1255332items4339JZDMitemDataid213typebooktitlebersetzenindieZukunftHerausforderungenderGlobalisierungfrbersetzerundDolmetschercollectiontitleSchriftendesBDcollectionnumber32publisherBDeditorfamilyBaurgivenWfamilyKalinagivenSfamilyMayergivenFfamilyWitzelJgivenissueddateparts2009schemahttpsgithubcomcitationstylelanguageschemarawmastercslcitationjsonRNDxhhiDVqV5e}
\citealt{BaurEtAl2009}).\footnote{Unfortunately, the conference programme is not available online anymore. Hence, the information about the programme was taken from the conference proceedings.} At the second conference in 2012, two posters and four presentations dealt with MT, while no presentation directly focused on PE\footnote{BDÜ. „Übersetzen in die Zukunft“. Online programme. \url{http://uebersetzen-in-die-zukunft.de/util/download.php?art=konf12_dl & dokument=2754}. last accessed 10 October 2016 (11:09).}. The third conference was held in 2014, in cooperation with the \textit{International Federation of Translators (FIT) World Congress}, and was subsumed under the heading “Man vs. Machine? The Future of Translators, Interpreters, and Terminologists”. As the main topic of the conference was MT, over 20 presentations dealt with the topic, six presentations focused on PE, and another four posters were presented on the topics. Further, the panel discussion's topic was “Machine Translation – Blessing, Curse, or Something In Between?”. Conclusively, the topical focus of the conference also reflects that the topics MT and PE have become more important in recent years and are now taken seriously by the community.\footnote{This is only one of many examples. The IATIS conference, for example, hosted a conference on ‘Innovation Paths in Translation and Intercultural Studies’ that hosted 12 presentations on PE as opposed to one presentation three years earlier.}



Taking the work of the BDÜ as a mirror of the \ili{German} translation and interpreting market, the publications and conferences presented in this chapter show that MT and PE reached the \ili{German} job market once and for all during the last five years. The technology is no longer ignored and it is not only international organisations and businesses that employ a few post-editors or freelancers for PE; it is now a feature of the entire profession. Although many professional translators still seem anxious about or unmotivated by MT technology, many seem to be accepting that it is part of the professional field now, which is reflected in the aforementioned best practice guidelines and a publication that deals exclusively with MT and PE (\citealt{Porsiel2017}).



Of course, other translation associations also consider and discuss the topics of MT and PE, either in articles in their magazines or in articles, white-papers, etc. on their websites. The American Translation Association (ATA), for example, published two articles about PE in the last volume of their Chronicle in 2015. The first article by %\label{ref:ZOTEROITEMCSLCITATIONcitationIDOmzFRopUpropertiesformattedCitationCassemiro2015plainCitationCassemiro2015citationItemsid172urishttpzoteroorgusers1255332itemsPEPJAE7Furihttpzoteroorgusers1255332itemsPEPJAE7FitemDataid172typearticlejournaltitlePEMTYourselfcontainertitleTheATAChroniclepage1315issue9authorfamilyCassemirogivenWilliamissueddateparts2015schemahttpsgithubcomcitationstylelanguageschemarawmastercslcitationjsonRNDHpfU1K7DDj}
\citet{Cassemiro2015} describes how a self-trained, rule-based MT system can be used as a tool for translations from scratch. Further, he emphasises that translators should not be afraid of PE and MT and should not fight it. Instead, they should embrace the new technology and use it to their advantage in order to meet the current needs of the market. In the second article, %\label{ref:ZOTEROITEMCSLCITATIONcitationIDPGV3Zml5propertiesformattedCitationGreen2015plainCitationGreen2015citationItemsid171urishttpzoteroorgusers1255332itemsUUAI3PJVurihttpzoteroorgusers1255332itemsUUAI3PJVitemDataid171typearticlejournaltitleBeyondPostEditingAdvancesinInteractiveTranslationEnvironmentscontainertitleTheATAChroniclepage1922issue9authorfamilyGreengivenSpenceissueddateparts2015schemahttpsgithubcomcitationstylelanguageschemarawmastercslcitationjsonRNDlKolnR3sM6}
\citet{Green2015} argues that PE might be such an ill-received task, because the MT systems do not learn from their mistakes and translators have to correct the same errors over and over again. After presenting a brief overview of the history of PE, he, therefore, introduces three interactive PE systems that learn from the changes made by a post-editor. Both articles allude to the translators' resistance to cope with PE and MT. While Cassemiro, on the one hand, encourages the translators to be open towards the new technologies, Green rather acknowledges the negative attitudes of the translators. At the end of his article, however, Green encourages translators to give those more interactive solutions another chance.


\section{Post-editing training}
\label{sec:4:5}

Although the field is thriving, little has been published solely on PE training yet. While many empirical studies conclude with implications on what needs to be integrated in PE training or that PE training is necessary to educate professional post-editors, only few publications focused on how to design PE training. The available publications on the topic will be introduced in the following chapter.



Probably the first publication to focus on PE training was written by %\label{ref:ZOTEROITEMCSLCITATIONcitationIDyXGdfj5VpropertiesformattedCitationrtfOuc0u8217Brien2002plainCitationOBrien2002citationItemsid1139urishttpzoteroorggroups3587itemsA4J5F4GMurihttpzoteroorggroups3587itemsA4J5F4GMitemDataid1139typepaperconferencetitleTeachingposteditingaproposalforcoursecontentcontainertitleSixthEAMTWorkshoppublisherplaceManchesterUKpage99106eventSixthEAMTWorkshopeventplaceManchesterUKURLhttpwwwmtarchiveinfoEAMT2002OBrienpdfauthorfamilyOBriengivenSharonissueddateparts2002season1511accesseddateparts20121123schemahttpsgithubcomcitationstylelanguageschemarawmastercslcitationjsonRNDyEzz9eUAIj}
\citet{OBrien2002}. First, she explains why PE training would be necessary as an addition to regular translation training. There is still a growing demand for translations, PE skills are probably acquired gradually and are different from translation skills, and translators who are familiar with MT and PE will probably be less hostile towards the topics, which in turn is necessary for successful PE. %\label{ref:ZOTEROITEMCSLCITATIONcitationIDsR1pZ0DwpropertiesformattedCitationrtfOuc0u8217Brien2002plainCitationOBrien2002citationItemsid1139urishttpzoteroorggroups3587itemsA4J5F4GMurihttpzoteroorggroups3587itemsA4J5F4GMitemDataid1139typepaperconferencetitleTeachingposteditingaproposalforcoursecontentcontainertitleSixthEAMTWorkshoppublisherplaceManchesterUKpage99106eventSixthEAMTWorkshopeventplaceManchesterUKURLhttpwwwmtarchiveinfoEAMT2002OBrienpdfauthorfamilyOBriengivenSharonissueddateparts2002season1511accesseddateparts20121123schemahttpsgithubcomcitationstylelanguageschemarawmastercslcitationjsonRNDgVfueiCmXu}
\citet{OBrien2002} further argues that although some characteristics of PE and MT are contrary to human translation, translators should be the ones trained for PE. However, this training should be optional for students seeking a translation degree. Next, she argues that there are certain skills that well-trained post-editors need in addition to the skills well-trained translators have, like the ability to use macros and code dictionaries for the MT system, knowledge of MT and a positive attitude towards MT, the ability to use terminology management systems (a skill many translators already acquire when they learn how to use TM systems) and solid text linguistic skills, knowledge about pre-editing and controlled languages, and at least some basic programming skills. Finally, she proposes a PE module that could be integrated into translator training, which would best be offered in a late undergraduate stadium (B.A. degree) or even only in postgraduate training (M.A. degree). One half (approximately) of this module would focus on theoretical issues, including an introduction to PE, MT, and controlled languages, basic programming skills, and higher terminology management as well as text linguistic skills. The other half of the module would include practical exercises for preferably all language combinations the individual student studies and with different MT systems, as well as combining MT output with TM tools, using different PE guidelines, and using terminology management tools. O’Brien additionally introduces some ideas on how practical experiences in controlled authoring tools, corpus analysis tools, and programming could be integrated.



%\label{ref:ZOTEROITEMCSLCITATIONcitationIDqV9dqnoIpropertiesformattedCitationBelam2003plainCitationBelam2003citationItemsid209urishttpzoteroorgusers1255332items55ZBS2PUurihttpzoteroorgusers1255332items55ZBS2PUitemDataid209typepaperconferencetitleBuyinguptofallingdownAdeductiveapproachtoteachingposteditingcontainertitleMTSummitIXWorkshoponTeachingTranslationTechnologiesandToolsT4ThirdWorkshoponTeachingMachineTranslationpublisherCiteseerpage110authorfamilyBelamgivenJudithissueddateparts2003schemahttpsgithubcomcitationstylelanguageschemarawmastercslcitationjsonRNDDdX4kq1J1Y}
\citet{Belam2003} introduces a workshop on PE guidelines that was held in the scope of a machine-assisted translation course. The students were in the last year of their undergraduate programme in Modern Languages. One lecture of this course was on basic PE knowledge and required the students to submit one practical assignment, in which they post-edited a text and commented on their procedure. In the scope of this assignment, students started to demand more precise PE guidelines, which was the starting point of the discussion workshop, where the focus was also on different PE types (rapid, minimal and full as suggested by %\label{ref:ZOTEROITEMCSLCITATIONcitationIDdhPnjnKfpropertiesformattedCitationAllen2003plainCitationAllen2003citationItemsid160urishttpzoteroorgusers1255332itemsWFGP5FUKurihttpzoteroorgusers1255332itemsWFGP5FUKitemDataid160typechaptertitlePosteditingcontainertitleComputersandTranslationAtranslatorsguidepublisherJohnBenjaminsTranslationLibrarypublisherplaceAmsterdamPhiladelphiapage297318volume35eventplaceAmsterdamPhiladelphiaauthorfamilyAllengivenJeffreyeditorfamilySomersgivenHaroldissueddateparts2003schemahttpsgithubcomcitationstylelanguageschemarawmastercslcitationjsonRNDao8nEmgH8r}
\citet{Allen2003}). In the workshop, the students were given a text to post-edit in groups and then asked to develop PE guidelines. The most obvious rules were defined immediately and without much discussion, such as “Correct any word which had not been translated.” Guidelines for less obvious errors in the MT output, however, were much harder to define and error categories had to be summarised in one guideline. Similarly, it became more difficult to decide which rule needed to be applied to which PE type (cf. ibid.: 2-3). Belam (cf. ibid.: 3-4) reports that students were much better at formulating the guidelines and matching these to the PE type as soon as they started to construct a scenario for the PE job. In the end, two guidelines – one for rapid PE and one for minimal PE (full PE was abandoned at the beginning of the workshop, because this would require the same quality standards as human translation) – were developed with three to four dos and don'ts (cf. ibid.: 7).



%\label{ref:ZOTEROITEMCSLCITATIONcitationIDbCK78cMSpropertiesformattedCitationDepraetere2010plainCitationDepraetere2010citationItemsid137urishttpzoteroorgusers1255332itemsGB62GGZQurihttpzoteroorgusers1255332itemsGB62GGZQitemDataid137typepaperconferencetitleWhatcountsasusefuladviceinauniversityposteditingtrainingcontextReportonacasestudycontainertitleEAMT2010proceedingsofthe14thannualconferenceoftheEuropeanassociationformachinetranslationpublisherplaceSaintRaphalFranceeventplaceSaintRaphalFranceauthorfamilyDepraeteregivenIlseissueddateparts2010schemahttpsgithubcomcitationstylelanguageschemarawmastercslcitationjsonRNDlLygb40jYu}
\citet[]{Depraetere2010} analyses a corpus of ten post-edited texts in her study. She asked ten students to post-edit a text from \ili{English} into \ili{French} consisting of 2230 words (110 segments) of which half were pre-translated by a customised rule-based MT system and the other half by a customised statistical MT system\footnote{Unfortunately, the author makes no comments of the quality or the differences of those to MT systems.}. The participants, who were all \ili{French} native speakers and were receiving training to become translators, post-edited in a web-based online tool. The aim of this study was to determine what problems occur in texts post-edited by translation students who are not trained in PE, to assess what students should be taught in PE classes and what they intuitively deem necessary for correction. Hence, they only received few PE instructions and a few examples of necessary and unnecessary PE corrections. Unfortunately, the results of this study are not quantified and only observations are reported. MT translations are usually very literal translations; nonetheless, Depraetere observes that students did not change the phrasing of the texts as long as it was not incorrect, which means that the students did not change anything just to improve the flow of the translation, even though there may have been a more idiomatic solution. Hence, she concludes that it is not necessary to over-emphasise in classes that style should not be considered in PE. The same applies to terminology – the MT output is accepted as long as it is not wrong, even if a better solution exists. The students were slightly careless when it came to formatting issues and capitalisation. Depraetere claims that the most striking observation was that students were often too careless towards improving the MT output, as they missed numerous mistakes made by the MT system. Some students did not even realise that some source text units were untranslated. No clear strategy could be detected concerning grammatically incorrect verbs, which would therefore need to be addressed in a PE class. In her conclusions, Depraetere summarises that trainees need to be confronted with typical MT errors so that they do not blindly rely on the MT output. Further, the need for consistency and formal accuracy needs to be highlighted.\footnote{This is something we also observed and reported in our studies. In the data set at hand, e.g., some participants did not change the translation of nurse as \textit{Krankenschwester} (female nurse, as suggested by the MT system) into \textit{Krankenpfleger} (male nurse) in the PE task %\label{ref:ZOTEROITEMCSLCITATIONcitationIDZjdtk3LypropertiesformattedCitationrtfuc0u268uloetal2014plainCitationuloetal2014citationItemsid140urishttpzoteroorgusers1255332items7V2QCKK7urihttpzoteroorgusers1255332items7V2QCKK7itemDataid140typechaptertitleTheInfluenceofPostEditingonTranslationStrategiescontainertitlePostEditingofMachineTranslationProcessesandApplicationspublisherCambridgeScholarsPublishingpublisherplaceNewcastleuponTynepage200218eventplaceNewcastleuponTyneauthorfamilyulogivenOliverfamilyGutermuthgivenSilkefamilyHansenSchirragivenSilviafamilyNitzkegivenJeaneditorfamilyWintherBallinggivenLaurafamilyCarlgivenMichaelissueddateparts2014schemahttpsgithubcomcitationstylelanguageschemarawmastercslcitationjsonRNDNqzgeybjQR}
(cf. \citet{CuloEtAl2014}). Further, we observed inconsistencies in terminology in the PE task in a study with domain-specific texts. These inconsistencies were introduced by the MT output and were often not eliminated by the participants (cf. %\label{ref:ZOTEROITEMCSLCITATIONcitationIDdHSTD7QmpropertiesformattedCitationCuloandNitzke2016korrekturplainCitationCuloandNitzke2016korrekturdontUpdatetruecitationItemsid69urishttpzoteroorgusers1255332itemsGHJ5CU2Zurihttpzoteroorgusers1255332itemsGHJ5CU2ZitemDataid69typearticlejournaltitlePatternsofTerminologicalVariationinPosteditingandofCognateUseinMachineTranslationinContrasttoHumanTranslationcontainertitleBalticJournalofModernComputingpage106114volume4issue2authorfamilyulogivenOliverfamilyNitzkegivenJeanissueddateparts2016schemahttpsgithubcomcitationstylelanguageschemarawmastercslcitationjsonRNDOyOC6tk91W}
\citealt{CuloNitzke2016}).} She further states that PE trainers have to keep in mind that students might not produce perfect translations, because they have less experience than professionals. Therefore, it might be easier for students to accept imperfections. All in all, this study gave valuable insights into what might be necessary to prioritise in PE training. However, the participant number was low and there might be other issues in other language combinations. Hence, more data need to be collected in order to paint a clearer picture.



As discussed in a previous chapter, %\label{ref:ZOTEROITEMCSLCITATIONcitationIDd3tXEeYfpropertiesformattedCitationPym2013plainCitationPym2013citationItemsid206urishttpzoteroorgusers1255332itemsG8UASJ79urihttpzoteroorgusers1255332itemsG8UASJ79itemDataid206typearticlejournaltitleTranslationSkillsetsinaMachinetranslationAgecontainertitleMetaJournaldestraducteursMetaTranslatorsJournalpage487503volume58issue3authorfamilyPymgivenAnthonyissueddateparts2013schemahttpsgithubcomcitationstylelanguageschemarawmastercslcitationjsonRNDhxz1vWmWLx}
\citet[494--497]{Pym2013} also discusses teaching technology (including MT) to translators. First he points out that it is most important to learn how to learn to use new tools, because most will be outdated within a few years. Further, it seems important that (future) translators learn to assess which data can be trusted and which cannot be trusted. It is dangerous to blindly trust MT output, while disregarding and overanalysing all MT output does not contribute to the original purpose of MT, namely increasing productivity. Additionally, the overall text as the greatest macro-unit has to be kept in mind and special revision strategies should be developed. %\label{ref:ZOTEROITEMCSLCITATIONcitationIDsbRbBJQRpropertiesformattedCitationPym2013plainCitationPym2013citationItemsid206urishttpzoteroorgusers1255332itemsG8UASJ79urihttpzoteroorgusers1255332itemsG8UASJ79itemDataid206typearticlejournaltitleTranslationSkillsetsinaMachinetranslationAgecontainertitleMetaJournaldestraducteursMetaTranslatorsJournalpage487503volume58issue3authorfamilyPymgivenAnthonyissueddateparts2013schemahttpsgithubcomcitationstylelanguageschemarawmastercslcitationjsonRND6vOOIARUz8}
Pym (cf. ibid.: 497--499) further advises that the technologies should be used as often as possible during training, that the classrooms need to be sufficiently equipped, that it might be helpful to work in pairs or groups (to assess and reflect on their own translation processes) and that working with field experts would be very valuable.



Another study on PE guidelines is presented by %\label{ref:ZOTEROITEMCSLCITATIONcitationID8AzAj1xmpropertiesformattedCitationFlanaganandChristensen2014plainCitationFlanaganandChristensen2014citationItemsid207urishttpzoteroorgusers1255332itemsBU6MTMQ9urihttpzoteroorgusers1255332itemsBU6MTMQ9itemDataid207typearticlejournaltitleTestingposteditingguidelineshowtranslationtraineesinterpretthemandhowtotailorthemfortranslatortrainingpurposescontainertitleTheInterpreterandTranslatorTrainerpage257275volume8issue2authorfamilyFlanagangivenMarianfamilyChristensengivenTinaPaulsenissueddateparts2014schemahttpsgithubcomcitationstylelanguageschemarawmastercslcitationjsonRNDS5UYpOCerr}
\citet{FlanaganChristensen2014}. They asked three MA students to retrospectively interpret the PE guidelines for publishable quality developed by TAUS and CNGL\footnote{Translation Automation User Society (TAUS) and Centre for Global Intelligent Content (CNGL)}, which the students had to use for their final assessment. The aim of the study was to see whether the guidelines were straightforward as well as easy to understand and apply. The module, in which this final assessment was included, was a Case Module, which the students could choose in the third semester of their Master's degree instead of a work placement. This module consisted of two workshops: The first introduced MT and PE, the second was a hands-on session, in which the students learnt to use the technology. Further, they had to complete two assessments. The first was for training purposes - students had to post-edit according to guidelines that aimed at an output that was good enough to understand the content. The second assessment, which this study focuses on, was the final exam that was graded. They had to post-edit a medical text for publishable quality and write a ten-page reflective report. Three weeks later, the students were asked to come for a retrospective interview on the guidelines. This interview and the post-edited texts were taken into consideration in this study. (cf. ibid.: 261-262) The findings showed that the students had problems interpreting the guidelines which was on the one hand due to little PE experience in general, but also caused by the wording in the guidelines themselves. A guideline was considered problematic, when a) at least two students did not adhere to the guideline or ignored it, b) at least two students misinterpreted the guideline, or c) one of both (cf. ibid.: 263-264). After the analysis of the interview and the PE products, the introductory part and all guidelines except two were classified as problematic. Hence, Flanagan and Christensen adjusted the order and the wording of the guidelines accordingly so that these became easier to understand and apply.



%\label{ref:ZOTEROITEMCSLCITATIONcitationIDCsnAFAgMpropertiesformattedCitationKennyandDoherty2014plainCitationKennyandDoherty2014citationItemsid211urishttpzoteroorgusers1255332items5P4DN4EDurihttpzoteroorgusers1255332items5P4DN4EDitemDataid211typearticlejournaltitleStatisticalmachinetranslationinthetranslationcurriculumovercomingobstaclesandempoweringtranslatorscontainertitleTheInterpreterandTranslatorTrainerpage276294volume8issue2authorfamilyKennygivenDorothyfamilyDohertygivenStephenissueddateparts2014schemahttpsgithubcomcitationstylelanguageschemarawmastercslcitationjsonRNDK12pEOqBuc}
\citet{KennyDoherty2014} describe in their article how statistical MT should be integrated in translation training. Translation technology has changed the translation profession and professional translators need to decide which translation technology they want\slash need to apply. Accordingly, translation trainees need to learn to handle these technologies to make educated decisions, when and whether to use which translation tool. Surveys on the translation market do not agree on the importance of MT and PE (and most studies do not take freelance translators into consideration). Hence, %\label{ref:ZOTEROITEMCSLCITATIONcitationIDX43mmihPpropertiesformattedCitationKennyandDoherty2014plainCitationKennyandDoherty2014citationItemsid211urishttpzoteroorgusers1255332items5P4DN4EDurihttpzoteroorgusers1255332items5P4DN4EDitemDataid211typearticlejournaltitleStatisticalmachinetranslationinthetranslationcurriculumovercomingobstaclesandempoweringtranslatorscontainertitleTheInterpreterandTranslatorTrainerpage276294volume8issue2authorfamilyKennygivenDorothyfamilyDohertygivenStephenissueddateparts2014schemahttpsgithubcomcitationstylelanguageschemarawmastercslcitationjsonRNDGP50XErioj}
Kenny and Doherty (ibid: 286) conclude in regard to translation training that “there is a growing demand for \isi{post-editing} services, but that it may not be wise for those who are about to graduate to focus on \isi{post-editing} at the expense of other 'traditional' translation skills.” Further, they discourage post-editors from using free and online-based MT systems, because they have many disadvantages especially concerning data protection and security. On the other hand, Do-It-Yourself statistical MT systems like Moses are very hard to implement for people with little experience in computer science, which applies to many translators and translation students. Therefore, they suggest the use of cloud-based statistical MT. The user can use his\slash her own monolingual and bilingual data (sometimes as an addition to the data of the service provider) to train the MT system and the software does not have to be installed locally. Further, the user can go through all stages of the MT cycle, which is especially interesting for student training purposes: he\slash she has to upload the data, train and test the engine, intervene to improve the MT quality, retrain and deploy the system in the end. The user interface is usually easy to handle, developers can interact with reviewers and testers, and the systems can be kept private, shared with others, or can be published for everyone to use. (cf. ibid.: 287-290)



In the accompanying paper, %\label{ref:ZOTEROITEMCSLCITATIONcitationIDxb8YUrQepropertiesformattedCitationDohertyandKenny2014plainCitationDohertyandKenny2014citationItemsid210urishttpzoteroorgusers1255332itemsJZMSC6GNurihttpzoteroorgusers1255332itemsJZMSC6GNitemDataid210typearticlejournaltitleThedesignandevaluationofaStatisticalMachineTranslationsyllabusfortranslationstudentscontainertitleTheInterpreterandTranslatorTrainerpage295315volume8issue2authorfamilyDohertygivenStephenfamilyKennygivenDorothyissueddateparts2014schemahttpsgithubcomcitationstylelanguageschemarawmastercslcitationjsonRNDKrUchhHd9g}
\citet{DohertyKenny2014} provide more information on the syllabus they integrated in the curriculum at Dublin City University. Half of the module on translation technologies focuses on training how to implement and use statistical MT. The content was delivered partly as lectures and partly as hands-on sessions in labs and included the following topics (ibid.: 299-300):


\begin{itemize}
\item brief history of MT
\item rule-based MT (basic architectures, linguistic problems, etc.)
\item statistic MT (basic architecture, alignment, n-gram processing, models)
\item MT evaluation (human and automatic)
\item pre- and post-processing (controlled languages, \isi{post-editing})
\item professional issues with MT like ethics, payment, etc.
\end{itemize}

The knowledge of the students was tested in an assignment that was worth 60\% of the module grade. The students had to create and evaluate a statistical MT system by themselves with the skills and knowledge they acquired in the lectures and lab sessions. They had to find training data, train an engine on a cloud-based platform, test the engine with texts from the same domain as the training data and with another domain, evaluate the output, consider ways of improving the output (e.g. more training data or the use of controlled language), use those potential improvements, retest the engine, and evaluate the output. The students had to describe how they proceeded and critically assessed their processes (cf. ibid.: 300-301). To evaluate the course outline, %\label{ref:ZOTEROITEMCSLCITATIONcitationIDAUsdp68XpropertiesformattedCitationDohertyandKenny2014plainCitationDohertyandKenny2014citationItemsid210urishttpzoteroorgusers1255332itemsJZMSC6GNurihttpzoteroorgusers1255332itemsJZMSC6GNitemDataid210typearticlejournaltitleThedesignandevaluationofaStatisticalMachineTranslationsyllabusfortranslationstudentscontainertitleTheInterpreterandTranslatorTrainerpage295315volume8issue2authorfamilyDohertygivenStephenfamilyKennygivenDorothyissueddateparts2014schemahttpsgithubcomcitationstylelanguageschemarawmastercslcitationjsonRND2OZJhltvRT}
Doherty and Kenny (cf. ibid.: 305-307) used a ten-item self-efficacy questionnaire that the students had to complete at the beginning and at the end of the course. In total, 29 students participated and the questionnaire proved that self-efficacy of the students increased significantly during the course. In their written assessment, many students reported on technical problems and evaluation issues. However, the students completed the task successfully. (cf. ibid.: 307-310)



All in all, little research has been published on PE training. %\label{ref:ZOTEROITEMCSLCITATIONcitationIDVGw1KnVJpropertiesformattedCitationrtfOuc0u8217Brien2002plainCitationOBrien2002citationItemsid1139urishttpzoteroorggroups3587itemsA4J5F4GMurihttpzoteroorggroups3587itemsA4J5F4GMitemDataid1139typepaperconferencetitleTeachingposteditingaproposalforcoursecontentcontainertitleSixthEAMTWorkshoppublisherplaceManchesterUKpage99106eventSixthEAMTWorkshopeventplaceManchesterUKURLhttpwwwmtarchiveinfoEAMT2002OBrienpdfauthorfamilyOBriengivenSharonissueddateparts2002season1511accesseddateparts20121123schemahttpsgithubcomcitationstylelanguageschemarawmastercslcitationjsonRNDBGOsp5zk8w}
\citegen{OBrien2002} early study presents a reasonable outline on how to design a PE module, but detailed course contents are not provided. While some theoretical thoughts and some results from final exams in PE courses were mentioned, the \citet{DohertyKenny2014} course outline on integrating statistical MT is very detailed, seems very reasonable and could be adapted easily at other universities. However, none of the above mentioned publications describes to a full extent how PE as a translation task should be taught. As we have seen in \sectref{sec:4:2}, numerous process research studies focus on PE. These findings need to be included in PE training. Hence, PE training is a topic that still needs to be addressed more thoroughly in the future so that process research results can be used in training and trained post-editors can be used in process studies.



In summary, both MT and PE are rapidly developing fields. While ful\-ly-au\-to\-mat\-ic MT has been an unfulfilled dream for many decades now, PE has only recently found its way into professional translation practice (although, some counterexamples show that it has been around longer than generally appreciated) and translation science. The developments and the attention given to MT and PE in translation in the last five to ten years indicate that MT and PE has come to stay.


