\chapter{Problem solving in psychology and translation studies}
\label{sec:5}

This study will not only focus on the PE task itself, but in particular on \isi{problem solving behaviour} during the PE task compared to the TfS task. The aim of this chapter is to define the terms \textit{\isi{problem}} and \textit{\isi{problem solving}} in regard to TfS and PE. To this end, the literature on problem solving in translation was analysed. However, it quickly became obvious that, depending on the instance, the term \textit{problem} is used in different ways and maybe sometimes too carelessly in translation studies. Therefore, the terms \textit{problem} and \textit{problem solving} will first be approached from the perspective of psychology, in which problem solving is a thriving field. Then, the approaches in translation studies will be introduced and the insights from both fields will subsequently be combined to define problem solving in TfS and PE. First, however, a general introduction will be provided and the chapter will be clearly outlined.



During our life, we are forced to deal with problems on a daily basis. Although, over time, we familiarise ourselves with the problem solving strategies that we need for everyday problems, new problems regularly arise. We might not consciously realise that we are dealing with a problem each time we encounter one: e.g. “How do I get to work?”, which includes decisions on questions like “What means of transportation do I take? Which is the right way? When do I have to leave so that I do not arrive too late?”. Once we find a solution to the problem, we can use this solution again and the situation no longer poses a problem for us every morning -- as long as the basic situation does not change: if we have to take a bus, because our bicycle is broken, a new problematic situation arises. Additionally, problems may occur although we have been in the same situation before (“I'm hungry on my lunch break, but I forgot my lunch at home. Where can I get something to eat now?”); and some problems may seem insurmountable (“I lost my job after fifteen years and all my applications for a new one are failing. Will I ever find work again?”).



Due to its ubiquitous nature, problem solving is not only an issue in our everyday life, but also in many social and political settings, as well as almost every scientific discipline. Problem solving methods and strategies need to be shared so that science can evolve and not every individual has to overcome the same problems over and over again. But the processes underlying problem solving itself are of central interest in many fields: In mathematics, statistics, and physics, it is a basic feature of the discipline with which to learn the required strategies to manually, or with the help of software, calculate arising numeric problems (e.~g. %\label{ref:ZOTEROITEMCSLCITATIONcitationIDfpLJxnzTpropertiesformattedCitationEngel1998plainCitationEngel1998citationItemsid135urishttpzoteroorgusers1255332itemsTK68H4B3urihttpzoteroorgusers1255332itemsTK68H4B3itemDataid135typebooktitleProblemsolvingstrategiespublisherSpringerpublisherplaceNewYorksourceOpenWorldCateventplaceNewYorkISBN0387982191languageEnglishauthorfamilyEngelgivenArthurissueddateparts1998schemahttpsgithubcomcitationstylelanguageschemarawmastercslcitationjsonRNDDASGsw5UNJ}
\citealt{Engel1998} "Problem solving strategies", %\label{ref:ZOTEROITEMCSLCITATIONcitationIDVe1FiRZCpropertiesformattedCitationKamal2010plainCitationKamal2010citationItemsid27urishttpzoteroorgusers1255332itemsNFU76BREurihttpzoteroorgusers1255332itemsNFU76BREitemDataid27typebooktitle1000solvedproblemsinmodernphysicspublisherSpringerpublisherplaceHeidelbergNewYorksourceOpenWorldCateventplaceHeidelbergNewYorkISBN9783642043321languageEnglishauthorfamilyKamalgivenAhmadAissueddateparts2010schemahttpsgithubcomcitationstylelanguageschemarawmastercslcitationjsonRNDCRBmRsJma0}
\citealt{Kamal2010} "1000 solved problems in modern physics" and %\label{ref:ZOTEROITEMCSLCITATIONcitationIDQFGY2gPRpropertiesformattedCitationQuirkQuirkandHorton2013plainCitationQuirkQuirkandHorton2013citationItemsid28urishttpzoteroorgusers1255332itemsF9Q8NFU6urihttpzoteroorgusers1255332itemsF9Q8NFU6itemDataid28typebooktitleExcel2010forphysicalsciencesstatisticsaguidetosolvingpracticalproblemspublisherSpringerpublisherplaceChamNewYorksourceOpenWorldCateventplaceChamNewYorkURLhttpdxdoiorg1010079783319006307ISBN9783319006307languageEnglishauthorfamilyQuirkgivenThomasJosephfamilyQuirkgivenMeghanfamilyHortongivenHowardissueddateparts2013accesseddateparts2014121schemahttpsgithubcomcitationstylelanguageschemarawmastercslcitationjsonRNDgRWmrbdOXK}
\citealt{QuirkEtAl2013} "Excel 2010 for physical sciences statistics; a guide to solving practical problems"\footnote{At this point, the titles of the books are specified in the running text to emphasise their problem solving content, although this differs from the standard citation method.}). Computer science frequently deals with problem solving and much literature has been published in special areas like artificial intelligence (e.g. %\label{ref:ZOTEROITEMCSLCITATIONcitationID5sukoOxLpropertiesformattedCitationZhangandZhang2004plainCitationZhangandZhang2004citationItemsid25urishttpzoteroorgusers1255332itemsKDDMB6C2urihttpzoteroorgusers1255332itemsKDDMB6C2itemDataid25typebooktitleAgentbasedhybridintelligentsystemsanagentbasedfromeworkforcomplexproblemsolvingpublisherSpringerpublisherplaceBerlinNewYorksourceOpenWorldCateventplaceBerlinNewYorkISBN3540209085shortTitleAgentbasedhybridintelligentsystemslanguageEnglishauthorfamilyZhanggivenZilifamilyZhanggivenChengqiissueddateparts2004schemahttpsgithubcomcitationstylelanguageschemarawmastercslcitationjsonRNDdMoW0DpqJ7}
\citealt{ZhangZhang2004} "Agent-based hybrid intelligent systems: An agent-based framework for complex problem solving") and on how to solve problems in or with the help of programming languages (e.g. %\label{ref:ZOTEROITEMCSLCITATIONcitationIDEUbVC3bYpropertiesformattedCitationHanlyKoffmanandTahiliani2013plainCitationHanlyKoffmanandTahiliani2013citationItemsid26urishttpzoteroorgusers1255332itemsF9GBRSMIurihttpzoteroorgusers1255332itemsF9GBRSMIitemDataid26typebooktitleProblemsolvingandprogramdesigninCpublisherPearsonEducationpublisherplaceHarlowsourceOpenWorldCateventplaceHarlowISBN9780273774198languageEnglishauthorfamilyHanlygivenJeriRfamilyKoffmangivenElliotBfamilyTahilianigivenMohitPissueddateparts2013schemahttpsgithubcomcitationstylelanguageschemarawmastercslcitationjsonRNDTX5MsBV2Ni}
\citealt{HanlyEtAl2013} “Problem solving and program design in C” or %\label{ref:ZOTEROITEMCSLCITATIONcitationIDLar8zr9TpropertiesformattedCitationSavitch2012plainCitationSavitch2012dontUpdatetruecitationItemsid80urishttpzoteroorgusers1255332itemsEND4328Hurihttpzoteroorgusers1255332itemsEND4328HitemDataid80typebooktitleJavaanintroductiontoproblemsolvingprogrammingpublisherPearsonpublisherplaceLondonsourceOpenWorldCateventplaceLondonISBN9780273751427shortTitleJavalanguageEnglishauthorfamilySavitchgivenWalterJfamilyCarranogivenFrankMissueddateparts2012schemahttpsgithubcomcitationstylelanguageschemarawmastercslcitationjsonRNDcAyv4cKFi0}
\citealt{SavitchCarrano2012} “Java: An introduction to problem solving \& programming”). Specific models and methods have been developed to help solve problems in engineering (e.g. %\label{ref:ZOTEROITEMCSLCITATIONcitationIDhMnpq6kvpropertiesformattedCitationGadd2011plainCitationGadd2011citationItemsid23urishttpzoteroorgusers1255332itemsIQXKMB56urihttpzoteroorgusers1255332itemsIQXKMB56itemDataid23typebooktitleTRIZforengineersenablinginventiveproblemsolvingpublisherWileypublisherplaceChichesterWestSussexUKHobokenNJsourceOpenWorldCateventplaceChichesterWestSussexUKHobokenNJURLhttpsiteebrarycomid10510618ISBN9780470684320languageEnglishauthorfamilyGaddgivenKarenissueddateparts2011accesseddateparts2014121schemahttpsgithubcomcitationstylelanguageschemarawmastercslcitationjsonRNDaAkrpQ4QwE}
\citealt{Gomez-perez2010} “TRIZ for engineers; Enabling inventive problem solving”) and economics (e.g. %\label{ref:ZOTEROITEMCSLCITATIONcitationID3VAAPC20propertiesformattedCitationrtfGuc0u243mezPuc0u233rez2010plainCitationGmezPrez2010citationItemsid125urishttpzoteroorgusers1255332itemsXN6U4DFZurihttpzoteroorgusers1255332itemsXN6U4DFZitemDataid125typebooktitleAcquisitionandunderstandingofprocessknowledgeusingproblemsolvingmethodspublisherIOSPressAKApublisherplaceAmsterdamHeidelbergGermanysourceOpenWorldCateventplaceAmsterdamHeidelbergGermanyISBN1607506009languageEnglishauthorfamilyGmezPrezgivenJosManuelissueddateparts2010schemahttpsgithubcomcitationstylelanguageschemarawmastercslcitationjsonRNDg60UhGsTZS}
\citealt{Gomez-perez2010}“Acquisition and understanding of process knowledge using problem solving methods”). Medicine deals with strategies to diagnose illnesses accurately and rapidly (e.g. %\label{ref:ZOTEROITEMCSLCITATIONcitationIDjfuwjr5IpropertiesformattedCitationAghamohammadiandRezaei2012plainCitationAghamohammadiandRezaei2012citationItemsid30urishttpzoteroorgusers1255332itemsF5ZD8SHTurihttpzoteroorgusers1255332itemsF5ZD8SHTitemDataid30typebooktitleClinicalcasesinprimaryimmunodeficiencydiseasesaproblemsolvingapproachsourceOpenWorldCatISBN9783642317842shortTitleClinicalcasesinprimaryimmunodeficiencydiseaseslanguageEnglishauthorfamilyAghamohammadigivenAsgharfamilyRezaeigivenNimaissueddateparts2012schemahttpsgithubcomcitationstylelanguageschemarawmastercslcitationjsonRNDHlOW39zYPK}
\citealt{AghamohammadiRezaei2012} “Clinical cases in primary immunodeficiency diseases: A problem-solving approach”) and pharmacy tries to use nature to find cures to illnesses (e.g. %\label{ref:ZOTEROITEMCSLCITATIONcitationID1jM2BT7XpropertiesformattedCitationMehlhorn2011plainCitationMehlhorn2011citationItemsid22urishttpzoteroorgusers1255332itemsB3N5DQT4urihttpzoteroorgusers1255332itemsB3N5DQT4itemDataid22typebooktitleNaturehelpshowplantsandotherorganismscontributetosolvehealthproblemspublisherSpringerpublisherplaceHeidelbergNewYorksourceOpenWorldCateventplaceHeidelbergNewYorkURLhttppubliceblibcomEBLPublicPublicViewdoptiID763732ISBN9783642193828languageEnglishauthorfamilyMehlhorngivenHeinzissueddateparts2011accesseddateparts2014121schemahttpsgithubcomcitationstylelanguageschemarawmastercslcitationjsonRNDsZwy8pmarK}
\citealt{Mehlhorn2011} “Nature helps - How plants and other organisms contribute to solve health problems”).



In the following, two fields will be analysed to determine a theoretical framework: problem solving in psychology (\sectref{sec:5:2}) and in translation studies (\sectref{sec:5:3}). Psychology was selected for analysis because it, just like translation, does not belong to the hard sciences but deals with the concept of problems and problem solving also on a theoretical basis. Finally, psychology does not only apply the concepts, but also deals with their very nature. Problem solving is a much discussed topic in psychology. Hence, it is not the scope of this chapter to describe every detail and approach in the field, but only a selection will be introduced. This selection concentrates in particular on notions that can be related to translation studies. The psychological theories and findings will be used to evaluate and extend the work on problem solving in translation studies (\sectref{sec:5:4}). But first, we will attempt to define the term \textit{problem} and differentiate between \textit{problem solving} and \textit{decision making} – two terms that are often used synonymously.


\section[Defining the term \textit{problem} and differentiating between \textit{problem solving} and \textit{decision making}]{Defining the term \textit{problem} and differentiating between \textit{problem solving} and \textit{decision making}\sectionmark{Defining the term {\upshape problem}}}\sectionmark{Defining the term {\upshape problem}}
\label{sec:5:1}

%\label{ref:ZOTEROITEMCSLCITATIONcitationIDLj4Yd42ypropertiesformattedCitationrtfDuc0u246rner1987plainCitationDrner1987citationItemsid247urishttpzoteroorgusers1255332items2V6UZUD9urihttpzoteroorgusers1255332items2V6UZUD9itemDataid247typebooktitleProblemlsenalsInformationsverarbeitungcollectiontitleKohlhammerStandardsPsychologieBasisbcherundStudientextepublisherKohlhammerpublisherplaceStuttgartnumberofpages151edition3AuflsourceGemeinsamerBibliotheksverbundISBNeventplaceStuttgartISBN3170097113languagegerauthorfamilyDrnergivenDietrichissueddateparts1987schemahttpsgithubcomcitationstylelanguageschemarawmastercslcitationjsonRNDaAwyaNIUBn}
\citet[10]{Dorner1987} describes a \textit{\isi{problem}} as a state which is not desirable for the individual, but that the individual does not have the means to change at that moment. Three basic components characterise a problem: First, there is an undesired initial state. Then, there is a desired final state. And finally, there is a hurdle between these two states that prevents the transformation from the initial state to the desired final state for the time being. Further, Dörner (ibid.: 10-11) differentiates between \textit{problems} and \textit{\isi{tasks}}. \textit{Tasks} are mental challenges which the individual knows how to solve. This means that a task lacks the third property of a problem, namely the hurdle, to overcome the initial state. While it might be a problem for a third grader or for someone who has never actively learned to calculate to divide 625 by 25, it becomes a simple task only for people who are experienced in division (this might not even be a task for people who had to learn square numbers by heart at some point in their life, but something they can recall from memory).



%\label{ref:ZOTEROITEMCSLCITATIONcitationIDLkho5zIKpropertiesformattedCitationJonassen2000plainCitationJonassen2000citationItemsid243urishttpzoteroorgusers1255332items5E7DVHUGurihttpzoteroorgusers1255332items5E7DVHUGitemDataid243typearticlejournaltitleTowardadesigntheoryofproblemsolvingcontainertitleEducationaltechnologyresearchanddevelopmentpage6385volume48issue4authorfamilyJonassengivenDavidHissueddateparts2000schemahttpsgithubcomcitationstylelanguageschemarawmastercslcitationjsonRNDLVz5avxcVM}
\citet[65]{Jonassen2000} also argues that problems consist of more characteristics than initial and desired state. While he agrees that there needs to be a gap between the current state and the desired (unknown) target state, he adds that a social, intellectual or cultural motivation is required to bridge the gap. “Finding the unknown is the process of problem solving” (ibid.). If nobody has the desire to bridge the gap, problem solving is not necessary.



The concepts \textit{\isi{problem solving}} and \textit{\isi{decision making}} are often closely related and are regularly mentioned in the same breath. %\label{ref:ZOTEROITEMCSLCITATIONcitationIDHMDPnymbpropertiesformattedCitationStrohschneider2006plainCitationStrohschneider2006citationItemsid74urishttpzoteroorgusers1255332items96AEPN3Jurihttpzoteroorgusers1255332items96AEPN3JitemDataid74typechaptertitleKulturelleUnterschiedebeimProblemlsencontainertitleDenkenundProblemlsenpublisherHogrefepublisherplaceGttingenaopage547615volumeEnzyklopdiederPsychologieThemenbereichCTheorieundForschungSer2KognitioneventplaceGttingenaoauthorfamilyStrohschneidergivenStefaneditorfamilyFunkegivenJoachimissueddateparts2006schemahttpsgithubcomcitationstylelanguageschemarawmastercslcitationjsonRNDoH3JeRMc7b}
\citet[577]{Strohschneider2006} states that problem solving and decision making are often regarded as synonyms, but that in his opinion, decisions are only one (often central) measure (or a collection of measures) in the problem solving process. The difference between the two terms does not seem to be immediately obvious. Therefore, the focus will be on defining \textit{problem solving} and \textit{decision making} and deciding which term is more appropriate for further discussion of the translation process.



In \textit{The Dictionary of Psychology}, the term \textit{decision making} is defined as the “[a]bility to make independent and intelligent choices, a process which counsellors seek to enhance” %\label{ref:ZOTEROITEMCSLCITATIONcitationIDzir1FKI1propertiesformattedCitationCorsini2002plainCitationCorsini2002citationItemsid250urishttpzoteroorgusers1255332items6ZJTD93Iurihttpzoteroorgusers1255332items6ZJTD93IitemDataid250typebooktitleThedictionaryofpsychologypublisherBrunnerRoutledgepublisherplaceNewYorknumberofpages1156sourceGemeinsamerBibliotheksverbundISBNeventplaceNewYorkISBN1583913289languageengauthorfamilyCorsinigivenRaymondJissueddateparts2002schemahttpsgithubcomcitationstylelanguageschemarawmastercslcitationjsonRND8bljlWhDR2}
\citep[253]{Corsini2002}, while \textit{problem solving} is described as “[p]rocedures, overt or covert, in the solutions of problems” (ibid.: 762) with different references to other problem solving categories. These short definitions highlight that decision making is one activity in the human mind, whereas problem solving is a more complex pattern. 



The Oxford \textit{Dictionary of Psychology} provides more detailed definitions for the two terms with an increased focus on the field of psychology. \textit{Decision making} is defined as


\begin{quote}
[t]he act or process of choosing a preferred option or course of action from a set of alternatives. It precedes and underpins almost all deliberate or voluntary behaviour. Three major classes of theories have guided research into decision making: normative, descriptive (or positive), and prescriptive theories (%\label{ref:ZOTEROITEMCSLCITATIONcitationIDZ8njzEmNpropertiesformattedCitationColman2009plainCitationColman2009citationItemsid249urishttpzoteroorgusers1255332itemsMINIWAK3urihttpzoteroorgusers1255332itemsMINIWAK3itemDataid249typebooktitleAdictionaryofpsychologycollectiontitleOxfordpaperbackreferencepublisherOxfordUniversityPresspublisherplaceOxfordNewYorknumberofpages882edition3rdedsourceLibraryofCongressISBNeventplaceOxfordNewYorkISBN9780199534067callnumberBF31C652009authorfamilyColmangivenAndrewMissueddateparts2009schemahttpsgithubcomcitationstylelanguageschemarawmastercslcitationjsonRNDLL4Rm0Bjp1}
\citealt{Colman2009}: 217).
\end{quote}


while \textit{problem solving} is described as


\begin{quote}
[c]ognitive processing directed at finding solutions to well-defined problems, such as the Tower of Hanoi, Wason selection task, or a water-jar problem, by performing a sequence of operations. Problem solving by means of logic or logical analysis is usually called reasoning.\footnote{In the following chapters, it will become obvious that problem solving does not only apply to well-defined but also to ill-defined problems (both terms will be described in more detail in \sectref{sec:5:3}). However, the definition fits its purposes for these initial considerations} (ibid.: 693).
\end{quote}


The latter definition signalises that logical problem solving is called \textit{\isi{reasoning}}. \textit{Reasoning}, in turn, is defined as the “[c]ognitive processing directed at finding solutions to problems by applying formal rules of logics or some other rational procedure” (ibid.: 620). If one attends translation classes at an undergraduate level, one often hears that a translation solution was selected, because “it sounds fitting” or sometimes, that something was disregarded, because the person “sensed it was not correct due to a feeling for the language”. However, these seemingly intuitive arguments decline with growing experience and knowledge about translation and language, because professional translators know the rules of language and translation. They know about grammar, registers, text type and domain conventions, etc. They can, hence, tackle a problem through reasoning.



Returning to the difference between decision making and problem solving: The main difference that becomes clear in these definitions is that decision making is often a one-step operation while problem solving embodies more than one operation. This point is also featured in %\label{ref:ZOTEROITEMCSLCITATIONcitationIDFntenMylpropertiesformattedCitationKoppenjanandKlijn2004plainCitationKoppenjanandKlijn2004citationItemsid251urishttpzoteroorgusers1255332items887Z4SXWurihttpzoteroorgusers1255332items887Z4SXWitemDataid251typebooktitleManaginguncertaintiesinnetworksanetworkapproachtoproblemsolvinganddecisionmakingpublisherRoutledgepublisherplaceLondonNewYorknumberofpages289sourceLibraryofCongressISBNeventplaceLondonNewYorkISBN0415369401callnumberHM291K6662004shortTitleManaginguncertaintiesinnetworksauthorfamilyKoppenjangivenJohannesFranciscusMariafamilyKlijngivenErikHansissueddateparts2004schemahttpsgithubcomcitationstylelanguageschemarawmastercslcitationjsonRNDevhZAIALIj}
\citet{KoppenjanKlijn2004}, who begin their book on problem solving and decision making (both terms are used in the subtitle of the book) for the management of uncertainties in networks with a short “[e]xample of wicked problems: the greenhouse effect” (ibid.: 2). Although the example does not aim to explain the difference between problem solving and decision making, it indirectly points out an interesting fact: In such complex problems, it is possible and necessary to make a lot of decisions to improve the situation. However, it takes a while (if it is possible at all) to solve the problem of the “greenhouse effect” – it does not matter on which political level which decision is made; solving the problem (a) takes time and (b) requires thousands of people to participate, accept and adhere to the decision. Conclusively, a decision does not necessarily solve a problem, but is one part of the problem solving process. The world is filled with complex problems that force the decision makers to form networks. Similarly, complex translation jobs are seldom handled by a single person, but include project manager(s), numerous translators, proof-readers and, potentially, even more people.



Similarly, %\label{ref:ZOTEROITEMCSLCITATIONcitationIDJSCEvATopropertiesformattedCitationJonassen2000plainCitationJonassen2000citationItemsid243urishttpzoteroorgusers1255332items5E7DVHUGurihttpzoteroorgusers1255332items5E7DVHUGitemDataid243typearticlejournaltitleTowardadesigntheoryofproblemsolvingcontainertitleEducationaltechnologyresearchanddevelopmentpage6385volume48issue4authorfamilyJonassengivenDavidHissueddateparts2000schemahttpsgithubcomcitationstylelanguageschemarawmastercslcitationjsonRNDko93lriDRx}
\citet{Jonassen2000} integrates decision making problems into his scale for the degree of complexity of problems. This scale starts with very \isi{well-defined problems} and ends with very \isi{ill-defined} \isi{complex problems} (e.~g. “Should I move in order to take another job” (ibid.: 76) – see further information in the next subchapter) – and simple decision making problems are more complex than logical problems or algorithms problems and often include more factors to be considered. Furthermore, what appear to be simple cases of decision making with only one answer (“Should abortion be banned?” – the answer to the decision is either yes or no), are sometimes categorised as dilemmas and are the most difficult problems to solve. If a decision is made, it does not solve the personal dilemma of whether abortion is legal or not and to what degree. Further, parts of the population to whom the law applies will not be satisfied with the decision.



However, there are theories in decision making that deal with complex decision making situations. As an example, one of these deals with \textit{\isi{phased decision strategies}} that suggest that, in complex decision making situations, not only one decision making rule is used but rather different rules are applied successively or even randomly %\label{ref:ZOTEROITEMCSLCITATIONcitationIDZDvVDZUzpropertiesformattedCitationHelmutJungermannPfisterandFischer2010plainCitationHelmutJungermannPfisterandFischer2010citationItemsid109urishttpzoteroorgusers1255332itemsNPNAU6BFurihttpzoteroorgusers1255332itemsNPNAU6BFitemDataid109typebooktitleDiePsychologiederEntscheidungeineEinfhrungpublisherSpektrumAkadVerlpublisherplaceHeidelbergnumberofpages481edition3korrAuflsourceGemeinsamerBibliotheksverbundISBNeventplaceHeidelbergISBN9783827423863shortTitleDiePsychologiederEntscheidunglanguagegerauthorfamilyJungermanngivenHelmutfamilyPfistergivenHansRdigerfamilyFischergivenKatrinissueddateparts2010schemahttpsgithubcomcitationstylelanguageschemarawmastercslcitationjsonRNDSXtdmHfBEs}
(\citealt{JungermannEtAl2010}). Therefore, decision making is also considered a multiple-phase activity.



%\label{ref:ZOTEROITEMCSLCITATIONcitationIDxgBNEHexpropertiesformattedCitationWilss1994plainCitationWilss1994citationItemsid236urishttpzoteroorgusers1255332itemsKDSR29MCurihttpzoteroorgusers1255332itemsKDSR29MCitemDataid236typearticlejournaltitleAFrameworkforDecisionmakinginTranslationcontainertitleTargetpage131150volume6issue2authorfamilyWilssgivenWolframissueddateparts1994schemahttpsgithubcomcitationstylelanguageschemarawmastercslcitationjsonRNDl6vPmZGfBK}
\citet{Wilss1994} is one of a few in translation who discusses the difference between problem solving and decision making, which to him are “not identical”, but “occasionally equated with each other since the boundary between the two cannot always be clearly drawn” (ibid.: 132). He argues that problem solving is the wider concept, that decision making is part of problem solving, and that decision making processes only start when all factors and criteria for the decision have been defined. However, both activities are essential in the translation process.



The terms \textit{problem solving} and \textit{decision making} are not used consistently. However, a tendency seems to be that decision making is used for one step operations, where one out of two or many options needs to be chosen, while problem solving often includes more steps. Further, problem solving involves a hurdle between initial and desired state to make the situation problematic, which is not necessary in the case of decision making. Hence, the term \textit{problem solving} is more suited to translation processes than the term \textit{decision making}, because we have an initial state (the source text) and a desired final state (the target text) and do not immediately know how to get there (hurdle) – an extensive discussion will follow in \sectref{sec:5:3} and \sectref{sec:5:4}. Decisions have to be made for single translation items, as many possible translation equivalents exist in the \isi{target language}. The translator can make these decisions consciously or subconsciously. However, the choice is not always obvious to the translator. Hence, a hurdle exists between source and target text, and the translator then has to solve a translation problem. How can we decide when and why a \isi{translation unit} is a decision or a problem? And is translation generally rather a decision making or a problem solving activity? Can a clear line be drawn in a construct as complex as translation? The following analysis will attempt to shed some light on these questions.


\section{Problem solving in psychology}
\label{sec:5:2}

Problem solving is an important sub-field in psychology. As %\label{ref:ZOTEROITEMCSLCITATIONcitationIDMK3o8kxApropertiesformattedCitationFunke2005plainCitationFunke2005dontUpdatetruecitationItemsid238urishttpzoteroorgusers1255332itemsQPPBCTH6urihttpzoteroorgusers1255332itemsQPPBCTH6itemDataid238typechaptertitleDenkenundProblemlsenVorwortundEinleitungcontainertitleJFunkeHgDenkenundProblemlsenpublisherHogrefepublisherplaceGttingenaopageXVIIXXVIIIvolumeEnzyklopdiederPsychologieThemenbereichCTheorieundForschungSer2KognitioneventplaceGttingenaoauthorfamilyFunkegivenJoachimissueddateparts2006schemahttpsgithubcomcitationstylelanguageschemarawmastercslcitationjsonRNDcG7kAdT0lM}
\citet[XXI]{Funke2006b} describes, problem solving is viewed as a part of the thinking process. Thinking is considered a higher cognitive function that takes advantage of simpler cognitive functions like perception, learning, and memory. Further, thinking has different appearances: In logical deductions, the human mind makes deductive judgements; conclusions for future events are drawn when judging probabilities; thinking with problem solving in mind helps to fill gaps for planned actions; and creative thinking creates new and helpful connections between what is already known.



Before we turn to problem solving, the general connection of \isi{thinking and language} will be briefly described according to %\label{ref:ZOTEROITEMCSLCITATIONcitationIDGLzhxjHxpropertiesformattedCitationrtfDuc0u246rner2006plainCitationDrner2006citationItemsid136urishttpzoteroorgusers1255332itemsICSAD2I5urihttpzoteroorgusers1255332itemsICSAD2I5itemDataid136typechaptertitleSpracheundDenkencontainertitleDenkenundProblemlsenpublisherHogrefepublisherplaceGttingenaopage617643volumeEnzyklopdiederPsychologieThemenbereichCTheorieundForschungSer2KognitioneventplaceGttingenaoauthorfamilyDrnergivenDietricheditorfamilyFunkegivenJoachimissueddateparts2006schemahttpsgithubcomcitationstylelanguageschemarawmastercslcitationjsonRNDinsaOPTU0Q}
\citet{Dorner2006}. While some philosophers like Plato, Aristotle, and Wilhelm von Humboldt have stated that thinking and language are clearly the same – thoughts are expressions of inner speech – others such as the scientists Faraday and Einstein strongly disagree with this assumption. In their opinion, speech is only a means to transmit information and even interferes with the thinking process (which can be demonstrated, e.g., when using think-aloud protocols\footnote{Find more information on think-aloud protocols in \sectref{sec:7:1:1}} in psychological studies: some participants think in a more structured way when they have to express their thoughts aloud, while others are delayed in their thinking process). These positions could not be further apart. However, a combination of both seemingly contradictory positions may hold the truth: Thinking is not possible without language, and thinking has nothing to do with language (cf. ibid.: 619-621). Complicated thinking processes are probably not possible without language, but even simple, non-language phases of thinking, e.g. during sleep, have their origin in memories of language-based thinking phases (cf. ibid.: 640). When we go to the bus stop closest to our home, perform other routine operations, or operations that are similar to what we have done before, we do not have to verbalise those thoughts in our head and still make it to the bus stop safely (cf. ibid.: 635-636), which is an argument for thinking without language.\footnote{However, in my opinion, this cannot be categorised as thinking, because it is an automated action during which we can verbalise other thoughts in our head. I would argue that if anything really new happens to us – nothing similar has ever happened before – we would verbally think about it first (setting aside reflexes that might intervene in the situation).}



Thinking can sometimes be categorised as a problem solving activity, considering that one has to find a path between a starting point and a final point. However, if the path has been created once before, it is not problem solving, it is solely remembering. \isi{Memories}, however, are part of the thinking process as well. Further, thinking also creates opinions and ideologies which, on the other hand, influence our \isi{problem solving behaviour} \citet[621--623]{Dorner2006}. So, we can conclude that not all thinking is problem solving, but all problem solving is thinking. If we remember the solution to a problem, because we encountered the problem before, there is no hurdle between the present state and the desired state. We need to think to get to the desired state, but we do not have to solve a problem (again).



If we assume that thinking is most often connected with language, problem solving has to be connected with language, too. Language is not fixed. Many words have numerous lexical meanings which we can apply accordingly in everyday language. Depending on education, profession and interests, our lexicon is specialised in different fields and every person has an individual lexicon. A gardener may be able to differentiate between various apple trees; this does not make him a different thinker, but rather a more informed person from whom we can learn. Misunderstandings are part of our everyday life and we learn through experience that we have to adapt our speech (and texts) according to who we are talking to, e.~g. other experts or laypersons. Hence, our way of thinking is not solely determined by the language we grow up with and which we develop over time – a gardener might not solve a mathematical problem much differently than a dog breeder – but there are tendencies (\citealt[627--628]{Dorner2006}). Accordingly, problems are perceived differently depending on our life experiences, which will be discussed in more detail later in this chapter, but also depending on the vocabulary we developed during our lives and the semantic connections we have with these words.



Coming back to the theory of problem solving, problems are basically categorised as well-structured (or \isi{well-defined}) and ill-structured (or \isi{ill-defined}) problems. Well-structured problems “require a finite number of concepts, rules, and principles being studied to a constrained problem situation” (%\label{ref:ZOTEROITEMCSLCITATIONcitationID04Js42eNpropertiesformattedCitationJonassen2000plainCitationJonassen2000citationItemsid243urishttpzoteroorgusers1255332items5E7DVHUGurihttpzoteroorgusers1255332items5E7DVHUGitemDataid243typearticlejournaltitleTowardadesigntheoryofproblemsolvingcontainertitleEducationaltechnologyresearchanddevelopmentpage6385volume48issue4authorfamilyJonassengivenDavidHissueddateparts2000schemahttpsgithubcomcitationstylelanguageschemarawmastercslcitationjsonRNDH5U4ZNve7L}
\citealt{Jonassen2000}: 67). These problems are also known as transformation problems and are often encountered, for example, in school and university environments to check whether students have studied the subject and have familiarised themselves with the subject's contents and strategies. All elements of the problem are presented to the problem solver in the initial state. The operators required to arrive at the solution are known (or should be known) to the problem solver, so (s)he “only” has to apply rules and principles which (s)he has previously learned in advance\footnote{Many pupils, who are not very strong in mathematics, will probably agree that it is not as easy as it might seem to apply mathematical rules and principles to algorithmic problems.}. Finally, the desired target state is sometimes even known as well (e. g. in mathematical text problems).



Unfortunately, well-structured problems are hardly encountered in real-life situations. Problems do not usually possess a predictable or concurrent solution in private or work situations. The steps to solve the problems were not specifically learned in advance; experience from different domains is required – more experience in (one of) the crucial domains the problem is situated in will probably help to solve ill-defined problems more easily – and personal opinions and judgements are often necessary. Furthermore, different solutions, approaches or even no solution at all are possible outcomes when trying to solve an ill-structured problem. Accordingly, the solution of the problem cannot be assessed as simply correct or incorrect (cf. ibid.: 67). While “[w]ell-structured problems focus on correct, efficient solutions, [...] ill-structured problems focus more on decision articulation and argumentation” (ibid.: 73). However, this is only a preliminary division and further categorisations are necessary to embrace the variety and complexity of problem categorisation (cf. ibid.: 64).


\largerpage[-1]
While much research focused on well-defined problems in the early days of problem solving research, studies on \isi{complex problem solving} have their origins in the late 1960s, early 1970s and deal with ill-defined problems %\label{ref:ZOTEROITEMCSLCITATIONcitationIDesT1Xso6propertiesformattedCitationFunke2006bplainCitationFunke2006bcitationItemsid126urishttpzoteroorgusers1255332itemsZMSXPPGAurihttpzoteroorgusers1255332itemsZMSXPPGAitemDataid126typechaptertitleKomplexesProblemlsencontainertitleDenkenundProblemlsencollectiontitleEnzyklopdiederPsychologieThemenbereichCTheorieundForschungSer2KognitionpublisherHogrefepublisherplaceGttingenaopage373443eventplaceGttingenaoauthorfamilyFunkegivenJoachimeditorfamilyFunkegivenJoachimissueddateparts2006schemahttpsgithubcomcitationstylelanguageschemarawmastercslcitationjsonRNDvkDFgoM8SY}
(\citealt[376]{Funke2006}). According to Funke (cf. ibid.: 379-380), a problem becomes a complex problem when it fulfils the following five characteristics:


\begin{itemize}
\item \textit{complexity –} numerous variables are involved
\item \textit{interconnectedness} – the variables are connected
\item \textit{dynamism} – the problem changes over time
\item \textit{non-transparency} – not all the information necessary to solve the problem is available to the problem solver
\item \textit{multiple aims} – more than one criterion needs to be optimised
\end{itemize}

Not all of these characteristics are unique for complex problems – some apply to simple problems as well. It is self-evident that problems can be differentiated with regard to the difficulty of the problem. However, it is less evident to decide what makes a problem difficult. A term which is often referred to in the context is \textit{complexity}.\footnote{The terminology in this discourse has room for improvement, e.g. it is difficult to claim that one characteristic of a complex problem is its complexity. However, it would go beyond the scope of this dissertation to make adjustments.} The more variables need to be accounted for in a problem, the more difficult it becomes. However, another aspect that needs to be taken into account is how these variables are \textit{connected}. Fifty interconnected variables might form a more difficult problem than 100 unconnected variables, because interconnected variables influence each other. \textit{Dynamic} problems change while they are being solved. They are not static tasks, but processes that need to be steered into the right directions so that the initial situation improves. In the problem solving scenario called “fire fighting”, the participants are asked to extinguish a burning wall in a computer game. However, it might be possible for the wall to start to burn in another area. Hence, the fire cannot be fought sequentially and the problem changes over time. Further, a problem becomes more difficult when it is \textit{not transparent}, which means that not all information necessary to solve the problem is available to the problem solver. Therefore, decisions are made with uncertainty in these situations. Complex problems often pursue not only one purpose, but many (\textit{multiple aims}). Consequently, the evaluation process is also more complex – the solution cannot be judged as correct or incorrect (cf. %\label{ref:ZOTEROITEMCSLCITATIONcitationIDuTllDQbBpropertiesformattedCitationFunke2006bplainCitationFunke2006bcitationItemsid126urishttpzoteroorgusers1255332itemsZMSXPPGAurihttpzoteroorgusers1255332itemsZMSXPPGAitemDataid126typechaptertitleKomplexesProblemlsencontainertitleDenkenundProblemlsencollectiontitleEnzyklopdiederPsychologieThemenbereichCTheorieundForschungSer2KognitionpublisherHogrefepublisherplaceGttingenaopage373443eventplaceGttingenaoauthorfamilyFunkegivenJoachimeditorfamilyFunkegivenJoachimissueddateparts2006schemahttpsgithubcomcitationstylelanguageschemarawmastercslcitationjsonRNDasOc6TVlUN}
\citealt[399--410]{Funke2006}).



In another attempt to categorise problem types, %\label{ref:ZOTEROITEMCSLCITATIONcitationIDUf509ymfpropertiesformattedCitationJonassen2000plainCitationJonassen2000citationItemsid243urishttpzoteroorgusers1255332items5E7DVHUGurihttpzoteroorgusers1255332items5E7DVHUGitemDataid243typearticlejournaltitleTowardadesigntheoryofproblemsolvingcontainertitleEducationaltechnologyresearchanddevelopmentpage6385volume48issue4authorfamilyJonassengivenDavidHissueddateparts2000schemahttpsgithubcomcitationstylelanguageschemarawmastercslcitationjsonRNDRQo48ZAfux}
\citet{Jonassen2000} describes elev\-en different types of problems, which were created based on 100 problem scenarios, starting with very \isi{well-structured problems} that result in correct or incorrect solutions (which can be evaluated easily) and ending with the most \isi{ill-structured problem} types, which have no single, exact solution and the solutions are difficult to evaluate:


\begin{description}
\item[\isi{logical problems}:] abstract tests of reasoning, like matchstick puzzles, the Tower of Hanoi puzzle or a Rubic's cube®; they are usually not embedded in any authentic context and are therefore abstract and hardly transferable
\item[\isi{algorithmic problems}] (like multiplying or statistical testing)
\item[\isi{story problems}:] algorithmic problems presented in a story; the variables and the mathematical operator have to be selected by the problem solver, e.~g. how long does it take Lorry A to overtake Lorry B
\item[\isi{rule-using problems}:] problems with correct answers but different possible approaches to solutions; can be of different complexity, such as expanding a recipe for more people, finding information with a search engine, or card and board games
\item[\isi{decision making problems}:] select one option from many alternatives with different consequences; can vary a lot in terms of complexity and may include risk and uncertainty
\item[\isi{troubleshooting problems}:] among the most common everyday problems, e.~g. mechanics who fix broken cars; require the problem solver to have different skills and knowledge
\item[\isi{diagnosis-solution problems}:] any kind of medical diagnosis and treatment proposal
\item[\isi{strategic performance}:] “involves real-time, complex and integrated activity structures, where the performers use a number of tactics to meet a more complex and ill-structured strategy while maintaining situational awareness” (ibid.: 79); e.~g. flying an aeroplane, arguing in front of a judge, playing professional sports
\item[\isi{situated case-policy problems}:] real-life, job-related problems where analysing situated, complex case problems is essential for the work; the goals cannot be strictly defined, little is known about how to approach the problem, there is no overall agreement on what a good solution needs to include; e.~g. international relations problems or business problems
\item[\isi{design problems}:] creating a product or system; there are only vague requirements on the output, no predefined approach to the solution, and general and domain-specific knowledge needs to be included; usually do not have clear standards on evaluation; e.~g. writing a poem, designing a bird table / bridge / vehicle that flies, developing a curriculum for a university
\item[\isi{dilemmas}:] personal, social, and ethical dilemmas; often appear as decision making, but are the most ill-structured, because there are no solutions that satisfies everybody, compromises are necessary, often involves a large group of people; e.~g. should healthcare be regulated privately or by the government, resolving the Middle East conflict
\end{description}

The ranking is not related to how difficult it is for the individual to solve problems of the different groups. Some logical problems like matchstick puzzles might be unsolvable to one person, while (s)he has no difficulty in design problems such as writing a poem on a specific topic. In addition to the type of problem and its complexity, domain specificity and problem representation influence the problem solving activity. Further, individual differences of the problem solvers also affect the problem solving activity. This includes, amongst others, the individual's familiarity with the problem type, his\slash her domain-specific knowledge, the cognitive ways (s)he processes information (cognitive controls), his\slash her reflection on information and the problem (metacognition), his\slash her epistemological beliefs, and his\slash her attitude on and motivation for the problem (cf. %\label{ref:ZOTEROITEMCSLCITATIONcitationIDRNPVJVFOpropertiesformattedCitationJonassen2000plainCitationJonassen2000citationItemsid243urishttpzoteroorgusers1255332items5E7DVHUGurihttpzoteroorgusers1255332items5E7DVHUGitemDataid243typearticlejournaltitleTowardadesigntheoryofproblemsolvingcontainertitleEducationaltechnologyresearchanddevelopmentpage6385volume48issue4authorfamilyJonassengivenDavidHissueddateparts2000schemahttpsgithubcomcitationstylelanguageschemarawmastercslcitationjsonRNDKrFha3LiOm}
\citealt{Jonassen2000}: 67-72).



How can problems be solved? The simplest way to solve problems is via \isi{trial-and-error}, which might be sufficient to solve simple problems but is not very efficient. For more complicated problems, it is necessary to plan internally\slash mentally. Different operators might be necessary to find the solution of the problem. Sometimes these operators are known, but it is not evident how to combine them, and sometimes these operators are unknown. If both the operators and the combination of the operators are familiar, it is not a \isi{problem solving activity}, the individual simply has to solve a task (cf. %\label{ref:ZOTEROITEMCSLCITATIONcitationIDdLP3SoPrpropertiesformattedCitationrtfDuc0u246rner2006plainCitationDrner2006citationItemsid136urishttpzoteroorgusers1255332itemsICSAD2I5urihttpzoteroorgusers1255332itemsICSAD2I5itemDataid136typechaptertitleSpracheundDenkencontainertitleDenkenundProblemlsenpublisherHogrefepublisherplaceGttingenaopage617643volumeEnzyklopdiederPsychologieThemenbereichCTheorieundForschungSer2KognitioneventplaceGttingenaoauthorfamilyDrnergivenDietricheditorfamilyFunkegivenJoachimissueddateparts2006schemahttpsgithubcomcitationstylelanguageschemarawmastercslcitationjsonRNDcJHy9REPdB}
\citealt{Dorner2006}: 623-624). 



%\label{ref:ZOTEROITEMCSLCITATIONcitationIDVozxiuQwpropertiesformattedCitationPretzNaplesandSternberg2003plainCitationPretzNaplesandSternberg2003citationItemsid84urishttpzoteroorgusers1255332itemsU63243KXurihttpzoteroorgusers1255332itemsU63243KXitemDataid84typechaptertitleRecognizingdefiningandrepresentingproblemscontainertitleThepsychologyofproblemsolvingpublisherCambridgeScholarsPublishingpublisherplaceCambridgepage330volume30eventplaceCambridgeauthorfamilyPretzgivenJeanEfamilyNaplesgivenAdamJfamilySternberggivenRobertJissueddateparts2003schemahttpsgithubcomcitationstylelanguageschemarawmastercslcitationjsonRNDOIUKMjmFUR}
\citet[3--4]{PretzEtAl2003} suggest that problem solving activities can be considered a cycle with the following seven steps:


\begin{quote}
  1. Recognize or identify the problem.
\end{quote}

\begin{quote}
  2. Define and represent the problem mentally.
\end{quote}

\begin{quote}
  3. Develop a solution strategy.
\end{quote}

\begin{quote}
  4. Organize the knowledge about the problem.
\end{quote}

\begin{quote}
  5. Allocate mental and physical resources for solving the problem.
\end{quote}

\begin{quote}
  6. Monitor the progress toward the goal.
\end{quote}

\begin{quote}
  7. Evaluate the solution for accuracy.
\end{quote}


These steps do not have to be executed in the given order and not all steps are always necessary – a successful problem solver is flexible and can adjust the cycle to his\slash her needs. These steps are considered a cycle, because in complex problem solving situations, the solution to a problem might lead to a new problem. Hence, the solving process has to restart for the new problem and the single steps have to be executed again. It also seems plausible that complex problems can be divided into smaller problem units that will be solved individually in this cycle or parts of the cycle. For example, “defining the problem mentally” in step two could also include “define subordinate problem units”. The following steps would then be implemented first for the individual subordinate units and in the end for the whole problem. The last step may be expanded to “evaluate the solution for accuracy for the problem unit and the whole unit”.



We have already connected problem solving with individual traits, amongst others the problem solver's familiarity with the problem type and his\slash her domain knowledge. %\label{ref:ZOTEROITEMCSLCITATIONcitationIDl9n6iMtypropertiesformattedCitationEricsson2003plainCitationEricsson2003citationItemsid134urishttpzoteroorgusers1255332itemsTWGZGJG8urihttpzoteroorgusers1255332itemsTWGZGJG8itemDataid134typechaptertitleTheacquisitionofexpertperformanceasproblemsolvingcontainertitleThepsychologyofproblemsolvingpublisherCambridgeUniversityPresspublisherplaceCambridgeaopage3183eventplaceCambridgeaoauthorfamilyEricssongivenAeditorfamilyDavidsongivenJanetEfamilySternberggivenRobertJissueddateparts2003schemahttpsgithubcomcitationstylelanguageschemarawmastercslcitationjsonRNDJ6oHpSJcLZ}
\citet{EricssonSternberg2003} links \isi{problem solving with expert performances} and considers problem solving a major contribution to acquiring expert knowledge. Even the most talented individuals have to learn the tasks they seek to become experts in and have to enhance their knowledge. “Different levels of mastery present the learner with different kinds of problems that must be solved for the skill to develop further” (ibid.: 31). At the beginning of the learning process, every individual can only successfully conduct the simplest tasks, activities, and challenges. The knowledge base is developed with the help of instructions and training, and reinforced by experience and exercises. If a person wants to solve a task that is too difficult for him\slash her, because his\slash her selection of methods and skills is not sufficient, (s)he cannot solve the task. On the other hand, if the person only encounters the same problems\slash exercises over and over again, (s)he does not become more skilled in the field. Only new problems and tasks challenge the person and help expand the knowledge of the individual. The problems that were initially impossible to handle, become easier and less problematic with increasing expertise. Problems have to be solved to increase the level of expertise as they broaden “cognitive mechanisms, representations, and knowledge” (ibid.: 32). Expertise is not only characterised by acquired knowledge, but also by different reactions to problem situations. Novices might not even be able to create one solution to the problem, while experts come up with different approaches and choose the most efficient. But it is not only speed and capacity, it is also “complex, highly specialised mechanisms" that make experts superior “in representative domain-specific tasks […] such as planning, anticipation, and reasoning” (ibid.: 62-63).



Everyday problem solving is often text and comprehension related %\label{ref:ZOTEROITEMCSLCITATIONcitationIDfwv9BxPMpropertiesformattedCitationWhittenandGraesser2003plainCitationWhittenandGraesser2003citationItemsid72urishttpzoteroorgusers1255332items7T7C4VIVurihttpzoteroorgusers1255332items7T7C4VIVitemDataid72typechaptertitleComprehensionoftextinproblemsolvingcontainertitleThepsychologyofproblemsolvingpublisherCambridgeUniversityPresspublisherplaceCambridgeaopage207229eventplaceCambridgeaoauthorfamilyWhittengivenShannonfamilyGraessergivenArthurCeditorfamilyDavidsongivenJanetEfamilySternberggivenRobertJissueddateparts2003schemahttpsgithubcomcitationstylelanguageschemarawmastercslcitationjsonRNDQSfZuy1axD}
(cf. \citealt{WhittenGraesser2003}) – as soon as kitchen appliances, software, electronic items or even our means of transportation do not work properly and we do not know how to fix them, we consult a manual, the Internet or any other source of information (if we do not have a human instructor who can teach us). We have to understand the instructions and learn how to fix the problem. If the learning process is unsuccessful, we have to consult an expert for help, which often means additional costs and waiting for a certain length of time until we can use the item again. Hence, the learn-and-fix solution should be more desirable. Whether a text can be understood and transferred to the problem is related to the text's cognitive representation, which is basically dependent on two property classes: \isi{human factors} (e. g. reader's domain knowledge and reading skills) and \isi{text factors} (e.~g. organisation of the text). %\label{ref:ZOTEROITEMCSLCITATIONcitationIDE25PigFLpropertiesformattedCitationWhittenandGraesser2003plainCitationWhittenandGraesser2003citationItemsid72urishttpzoteroorgusers1255332items7T7C4VIVurihttpzoteroorgusers1255332items7T7C4VIVitemDataid72typechaptertitleComprehensionoftextinproblemsolvingcontainertitleThepsychologyofproblemsolvingpublisherCambridgeUniversityPresspublisherplaceCambridgeaopage207229eventplaceCambridgeaoauthorfamilyWhittengivenShannonfamilyGraessergivenArthurCeditorfamilyDavidsongivenJanetEfamilySternberggivenRobertJissueddateparts2003schemahttpsgithubcomcitationstylelanguageschemarawmastercslcitationjsonRNDicnfCqIc49}
\citet[215]{WhittenGraesser2003} also remark that most discourse psychologists agree that the reader's general knowledge has a huge influence on text comprehension. The more familiar the individual is with the problem or the domain the problem is located in, the easier it is for him\slash her to understand and apply the instructions in the written text. Further, it might be possible for the individual to recognise mistakes in the text and overcome them, which is impossible for individuals who have no prior knowledge of the problem\slash domain. Nonetheless, understanding the written text is still part of the problem solving activity: If no instructions were needed, the text would not be consulted at all. On the other hand, if the potential problem solver does not even understand the original problem, the best instructions will probably not help in solving the problem.



Another final aspect of problem solving that will be introduced is \textit{\isi{full insight problem solving}}. Although the term is used differently within the field, it usually describes the phenomenon where a solution to a problem seems to be found by accident rather than by consciously applying strategies. The problem solving process does not deliver any solutions; the solution only comes to mind all of a sudden when the problem is not thought about actively (cf. %\label{ref:ZOTEROITEMCSLCITATIONcitationIDNHkWkpbLpropertiesformattedCitationrtfKnoblichanduc0u214llinger2006plainCitationKnoblichandllinger2006citationItemsid240urishttpzoteroorgusers1255332items7ARRJ75Surihttpzoteroorgusers1255332items7ARRJ75SitemDataid240typechaptertitleEinsichtundUmstrukturierungbeimProblemlsencontainertitleDenkenundProblemlsencollectiontitleEnzyklopdiederPsychologieThemenbereichCTheorieundForschungSer2KognitionpublisherHogrefepage183authorfamilyKnoblichgivenGntherfamilyllingergivenMichaelissueddateparts2006schemahttpsgithubcomcitationstylelanguageschemarawmastercslcitationjsonRNDgdAyVUg7KG}
\citealt{KnoblichÖllinger2006}). Different anecdotes about scientific puzzles are known that were supposed to be solved by insight problem solving. Although it is thought to be a myth, the story of Archimedes is the most famous: The local tyrant hired Archimedes to figure out whether the goldsmith betrayed him and replaced some parts of the golden crown with silver. After struggling to find a way to prove whether the crown was pure gold or not, Archimedes found the solution to the puzzle by accident in the public bath. He observed that the more of his body was immersed in the water, the more water was displaced and that the amount of replaced water exactly equalled his body volume. Hence, the volume of the crown could be measured by putting it into water, and then the weight of the crown could be compared with the weight of an equal amount of gold. If the crown weighed less, the goldsmith did not make the crown from solid gold exclusively. Jubilant about his discovery, Archimedes exclaimed “Eureka! Eureka!” (“I've found it! I've found it!”) and ran home naked. Therefore, this so called “Aha!”-effect is also referred to as the “Eureka effect”. Other scientific breakthroughs are said to have originated in similar out-of-context situations like Einstein's theory of relativity and Newton's discovery of gravity – an apple fell on his head when he was sitting beneath an apple tree (cf. %\label{ref:ZOTEROITEMCSLCITATIONcitationID3imtdiVhpropertiesformattedCitationBiello2006plainCitationBiello2006citationItemsid149urishttpzoteroorgusers1255332itemsAMG2G67Qurihttpzoteroorgusers1255332itemsAMG2G67QitemDataid149typearticlejournaltitleFactorFictionArchimedesCoinedtheTermEurekaintheBathcontainertitleScientificAmericanvolume8URLhttpswwwscientificamericancomarticlefactorfictionarchimedeauthorfamilyBiellogivenDavidissueddateparts2006schemahttpsgithubcomcitationstylelanguageschemarawmastercslcitationjsonRNDiAtJ6Q9KiH}
\citealt{Biello2006}).



However, insight problem solving is more the exception than the rule. Problems are more often solved step by step (\textit{\isi{partial insight problem solving}}), according to rules and strategies, as mentioned before. When the problem seems to be unsolvable, it needs to be placed in another problem space, which means that the problem needs to be represented in another way. In this new space, the approach to solve the problem can be very easy, because part of the thinking process has been previously performed. This can lead to a very fast, full insight problem solving situation, or the approach can still be difficult and numerous steps may still be necessary to solve the problem (cf. %\label{ref:ZOTEROITEMCSLCITATIONcitationIDX5c8IObVpropertiesformattedCitationrtfKnoblichanduc0u214llinger2006plainCitationKnoblichandllinger2006citationItemsid240urishttpzoteroorgusers1255332items7ARRJ75Surihttpzoteroorgusers1255332items7ARRJ75SitemDataid240typechaptertitleEinsichtundUmstrukturierungbeimProblemlsencontainertitleDenkenundProblemlsencollectiontitleEnzyklopdiederPsychologieThemenbereichCTheorieundForschungSer2KognitionpublisherHogrefepage183authorfamilyKnoblichgivenGntherfamilyllingergivenMichaelissueddateparts2006schemahttpsgithubcomcitationstylelanguageschemarawmastercslcitationjsonRNDnRMOTkNVcH}
\citealt{KnoblichÖllinger2006}).



In the decision making context, %\label{ref:ZOTEROITEMCSLCITATIONcitationIDys0rK5B6propertiesformattedCitationHelmutJungermannPfisterandFischer2010plainCitationHelmutJungermannPfisterandFischer2010citationItemsid109urishttpzoteroorgusers1255332itemsNPNAU6BFurihttpzoteroorgusers1255332itemsNPNAU6BFitemDataid109typebooktitleDiePsychologiederEntscheidungeineEinfhrungpublisherSpektrumAkadVerlpublisherplaceHeidelbergnumberofpages481edition3korrAuflsourceGemeinsamerBibliotheksverbundISBNeventplaceHeidelbergISBN9783827423863shortTitleDiePsychologiederEntscheidunglanguagegerauthorfamilyJungermanngivenHelmutfamilyPfistergivenHansRdigerfamilyFischergivenKatrinissueddateparts2010schemahttpsgithubcomcitationstylelanguageschemarawmastercslcitationjsonRNDhN4tnKrfIC}
\citet{JungermannEtAl2010} discuss the selection of rules in decision making tasks. First, they point out that the literature does not use the terms \textit{\isi{rule}} and \textit{\isi{strategy}} consistently. However, they define a \textit{rule} as the way information is processed and the decision maker chooses between different options, while a \textit{strategy} describes the way a decision maker chooses between these rules. They further specify two characteristics that affect the decision making\footnote{In my perspective, this would rather apply to problem solving tasks than decision making tasks.} task: the \textit{\isi{complexity of the problem}} and the \textit{types of \isi{information supply}}. \textit{Complexity} can be defined in different manners (as discussed above), but Jungermann et al. consider the following the most important in their context: amount of options, amount of features of the single options, and similarity of the options and time pressure. \textit{Information supply} refers to the way in which the information necessary for making a decision is present or presented (e.g. in an experiment), which naturally influences the problem solving activity.



In this section, approaches of psychology regarding problem solving were presented. First, we learned that problem solving is connected with thinking and discussed how thinking is related to language. Further, problems can be categorised as well- and ill-defined or as less or more complex. %\label{ref:ZOTEROITEMCSLCITATIONcitationIDLuDennpYpropertiesformattedCitationJonassen2000plainCitationJonassen2000citationItemsid243urishttpzoteroorgusers1255332items5E7DVHUGurihttpzoteroorgusers1255332items5E7DVHUGitemDataid243typearticlejournaltitleTowardadesigntheoryofproblemsolvingcontainertitleEducationaltechnologyresearchanddevelopmentpage6385volume48issue4authorfamilyJonassengivenDavidHissueddateparts2000schemahttpsgithubcomcitationstylelanguageschemarawmastercslcitationjsonRNDBcdKLK8VMp}
\citet{Jonassen2000} introduced eleven more fine-grained categories for problems sorted by the structure of the problem, starting with well-defined problems and ending with the most ill-defined category. Moreover, problem solving can be performed in a cycle that starts with identifying the problem and ends with assessing the problem. They cycle may have to be repeated, for example when a new problem results from the old, or when individual subordinate problems are processed one after the other. Another aspect is the level of expertise the problem solver has in the respective domain. Finally, we learned about the text representation aspect of problem solving and the phenomenon of insight problem solving. In the next chapter, we will address problem solving in translation and more specifically how it has been discussed in translation studies so far.


\section{Problem solving in translation studies}
\label{sec:5:3}

Some thoughts and ideas on problem solving have been posited in translation studies as well and will be discussed in the following chapter. In contrast to psychology, there are no published overviews on \isi{translation and problem solving} that merely focus on theoretical aspects, but single studies exist that address the topic empirically or in which problem solving was part of a broader theoretical framework. This chapter aims to introduce the most important of these studies, but no claim to completeness is raised. The methodology of the selected empirical and process-related studies will be described briefly for reasons of completeness and for later analogies to the own methodology. As a side note, some of the translation scientists consider and cite thoughts and ideas from psychology literature as well, which will be mentioned if relevant. Nonetheless, this chapter will only introduce the work in translation studies on problem solving, while \sectref{sec:5:4} will discuss whether these considerations were sufficient from the psychology perspective and will expand or adapt them if necessary. Accordingly, I do not agree with all statements that are presented in this chapter – it shall merely describe the state of the art of problem solving in translation studies.



%\label{ref:ZOTEROITEMCSLCITATIONcitationIDlD1HkVf4propertiesformattedCitationrtfLevuc0u79231968plainCitationLev1968citationItemsid98urishttpzoteroorgusers1255332itemsP86W2B9Purihttpzoteroorgusers1255332itemsP86W2B9PitemDataid98typechaptertitleTranslationasadecisionprocesscontainertitleTheTranslationStudiesReaderpublisherRoutledgepublisherplaceLondonNewYorkpage148159eventplaceLondonNewYorkauthorfamilyLevgivenJieditorfamilyVenutigivenLawrenceissueddateparts1968season2000schemahttpsgithubcomcitationstylelanguageschemarawmastercslcitationjsonRNDnzG4QHv38Z}
\citet{Levy1968} was the first to describe translation as a \isi{decision making process}.\footnote{\sectref{sec:5:1} discusse the differences between problem solving and decision making.} The translator may have different equivalents of a source item for the target text, but (s)he has to decide on one of them. Hence, as soon as one decision is made, the rest of the text has to be interpreted in favour of this decision. Translators' decisions can be motivated or unmotivated, necessary or unnecessary. The methods of defining decision making problems introduced in the paper should be seen as a starting point to develop a generative model of translation. %\label{ref:ZOTEROITEMCSLCITATIONcitationIDYax4J235propertiesformattedCitationReiss1981plainCitationReiss1981citationItemsid237urishttpzoteroorgusers1255332itemsCQZS5TNEurihttpzoteroorgusers1255332itemsCQZS5TNEitemDataid237typearticlejournaltitleTypekindandindividualityoftextdecisionmakingintranslationcontainertitlePoeticsTodaypage121131volume2issue4authorfamilyReissgivenKatharinaissueddateparts1981schemahttpsgithubcomcitationstylelanguageschemarawmastercslcitationjsonRNDQ4oWqqiKHk}
\citet{Reiss1981} and %\label{ref:ZOTEROITEMCSLCITATIONcitationIDgZTJataupropertiesformattedCitationrtfKuuc0u223maul1986plainCitationKumaul1986citationItemsid101urishttpzoteroorgusers1255332items56I773VKurihttpzoteroorgusers1255332items56I773VKitemDataid101typechaptertitlebersetzenalsEntscheidungsprozeDieRollederFehleranalyseinderbersetzungsdidaktikcontainertitlebersetzungswissenschaftEineNeuorientierungpublisherFranckepublisherplaceTbingenBaselpage206229eventplaceTbingenBaselauthorfamilyKumaulgivenPauleditorfamilySnellHornbygivenMaryissueddateparts1986schemahttpsgithubcomcitationstylelanguageschemarawmastercslcitationjsonRNDIxVa4AVg84}
\citet{Kusmaul1986} referred to translation as a decision making process as well – the former in regard to decisions regarding text types in literary translations, the latter in regard to translation mistakes analyses and translation didactics.



As I argue that translation is not only a decision making task, but occasionally also a problem solving activity, the studies on problem solving in translation will be discussed more extensively in the following. First of all, I will discuss terminology issues concerning the difference between \isi{translation \textit{difficulties}} and \isi{translation \textit{problems}}. First raised by %\label{ref:ZOTEROITEMCSLCITATIONcitationIDi4DAxHjnpropertiesformattedCitationNord1987plainCitationNord1987dontUpdatetruecitationItemsid246urishttpzoteroorgusers1255332itemsMMBEBR62urihttpzoteroorgusers1255332itemsMMBEBR62itemDataid246typearticlejournaltitlebersetzungsproblemebersetzungsschwierigkeitenWasindenKpfenvonbersetzernvorgehensolltecontainertitleMitteilungsblattfrDolmetscherundbersetzerpage58volume2issue1987authorfamilyNordgivenChristianeissueddateparts1987schemahttpsgithubcomcitationstylelanguageschemarawmastercslcitationjsonRNDEnViRJiThj}
\citet{Nord1987}, she categorises translation difficulties as learner-dependent and translation problems as learner-independent. Translation difficulties are components in the text that the translator struggles with because (s)he does not know a lexical, syntactic, or grammatical element in the source language, does not yet know how to solve the particular translation problem, lacks domain-specific competences, etc. Translation problems on the other hand may result from the source text, the translation skopos, differences between source and target culture or gaps in the involved languages. Translation as a teachable and learnable task should not be taught only by doing, but 


\begin{quote}
the attention of the translator should be directed on the one hand to the (cognitive graspable and solvable) translation problems. On the other hand, he has to learn to recognise his subjective translation difficulties and to apply suitable methods to overcome these\footnote{Original phrasing: “daß die Aufmerksamkeit des Übersetzers zum einen auf die (kognitiv erfaßbaren und lösbaren) Übersetzungsprobleme gelenkt wird und daß er zweitens lernt, seine subjektiven Übersetzungsschwierigkeiten zu erkennen und geeignete Methoden zu ihrer Überwindung einzusetzen”} (ibid.: 5, translated J.N.).
\end{quote}


%\label{ref:ZOTEROITEMCSLCITATIONcitationID9ninFeFKpropertiesformattedCitationKrings1986cplainCitationKrings1986cdontUpdatetruecitationItemsid103urishttpzoteroorgusers1255332itemsVMUMBHRNurihttpzoteroorgusers1255332itemsVMUMBHRNitemDataid103typebooktitleWasindenKpfenvonbersetzernvorgehtEineempirischeUntersuchungzurStrukturdesbersetzungsprozessesanfortgeschrittenenFranzsischlernernpublisherGunterNarrVerlagpublisherplaceTbingenvolume291eventplaceTbingenauthorfamilyKringsgivenHansPissueddateparts1986schemahttpsgithubcomcitationstylelanguageschemarawmastercslcitationjsonRNDpW3LSxYnA0}
\citet{Krings1986}
published an extensive study on the processes during translation based on think-aloud protocols of eight language learners who translated from their foreign language into their native language (four participants) and vice versa. The analysis of the \isi{think-aloud protocols} was based on the identification of translation problems. As professionals can hardly constitute a homogeneous group because of their different experiences and as it is assumed that with increasing experience certain processes become automatised and hence will not be verbalised in the think-aloud protocols, professional translators were not taken into consideration for this study (cf. ibid.: 51-52).\footnote{Automation of translation processes was also considered the reason why about 90\% of all verbalisations were related to translation problems (\citet[118]{Krings1986b}).} Krings' motivation was to explore the translation process in a structured, psychological manner with empirical data to develop a theoretical model of the translation process, as translation studies relied mostly on theoretical assumptions and product data (cf. ibid. 10-11). Altogether, \citet[484--499]{Krings1986} identified 117 features of the translation process, of which most could not have been found with a simple analysis of the translation product. He specified a model with primary and secondary problem indicators that can be specified in the translation process, when the translators are asked to think aloud\footnote{The translations were produced manually without any electronic aids in this study.} (cf. ibid.: 120-143):


\begin{itemize}
\item \isi{primary indicators}


\begin{itemize}
\item explicit or implicit problem identification by the translator
\item use of aids (dictionaries and alike)
\item gaps in the target text
\end{itemize}
\item \isi{secondary indicators}


\begin{itemize}
\item many equivalent translation choices
\item changes in the translated text
\item underlining of source text items
\item negative judgement of the translation by the translator (the translator is unsatisfied with the translation)
\item not enough attention to the function of the target text
\item unfilled pauses
\item paralingual indicators, like sighing, groaning, or laughter
\item primary equivalent associations
\end{itemize}
\end{itemize}

It is assumed that either one primary or two (or more) secondary indicators imply a problem. Events in the think-aloud protocols like reading the source\slash target text out loud or comments during the production of the translation were not considered problem indicators, because they have other functions like attention control or justification of translation choice. Translation problems are further categorised as comprehension (the source item is problematic), reproduction (the transfer into the \isi{target language} is problematic) or comprehension-reproduction (both are problematic) problems, depending on the level in the translation process at which the problem arises. These problems arise either from language deficits or from translation problems, which depend on whether the problem results from problems in the mother tongue\footnote{Problems in the mother tongue might not occur due to deficits in the participant's language knowledge, but due to deficits in the source text.} or second language, or whether they result from a transfer problem, i.~e. the problem is not merely linguistic (cf. ibid.: 144-171). Krings (ibid.: 175, translated J.N.) defines translation strategies as “potentially conscious plans of a translator to solve a specific translation problem in the scope of a specific translation task”\footnote{Original phrasing: “potentiell bewußte Pläne eines Übersetzers zur Lösung konkreter Übersetzungsprobleme im Rahmen einer konkreten Übersetzungsaufgabe.”} at two strategic levels: the macro- and the micro-strategic level. The study concludes with two extensive models. Both start with the question of whether a translation problem occurs or not. If a translation problem occured, different problem solving strategies were found for the translation from the foreign language on the one hand – including equivalent finding strategies, evaluation strategies, retrieval strategies, and reduction strategies – and into the foreign language on the other – including equivalent finding strategies, evaluation strategies, and decision making strategies (cf. ibid. 480-482). These models are summarised in one translation process model in another publication (%\label{ref:ZOTEROITEMCSLCITATIONcitationIDFfGp2oAZpropertiesformattedCitationKrings1986aplainCitationKrings1986adontUpdatetruecitationItemsid104urishttpzoteroorgusers1255332items4PCIW24Nurihttpzoteroorgusers1255332items4PCIW24NitemDataid104typechaptertitleTranslationproblemsandtranslationstrategiesofadvancedGermanlearnersofFrenchL2containertitleInterlingualandinterculturalcommunicationpublisherGunterNarrVerlagpublisherplaceTbingenpage263276eventplaceTbingenauthorfamilyKringsgivenHansPeditorfamilyHousegivenJulianefamilyBlumKulkagivenShoshanaissueddateparts1986schemahttpsgithubcomcitationstylelanguageschemarawmastercslcitationjsonRNDOWlS7SIa61}
\citealt{Krings1986}%Krings 1986a
; cf. \figref{fig:key:5:1}), which also has its starting point in the question whether a translation problem occurred.


\begin{figure}
\includegraphics[width=.75\textwidth]{figures/Figure_5_4_1.png}
\caption{\citet[269]{Krings1986} tentative model of the translation process}
\label{fig:key:5:1}
\end{figure}

  


In her article, %\label{ref:ZOTEROTEMPRNDSLhAKiausy}
\citet{Kaiser-cooke1994} combines expertise, knowledge and problem solving. Due to differences in available knowledge, knowledge processing and other ways of recognising problem representations, novices and experts behave differently in problem solving in general, which can also be transferred to problem solving in translation. Further, she states that experts do not have to reflect on problems over and over again, but the path to the solved problem is shortened with increasing expertise until it is routinised. Hence, the procedure is automatised and the \isi{cognitive load} decreases. As translation fits all criteria, it can be considered an expert task and hence a problem solving task, which can be seen in the inability of novices and laypersons to translate and judge a source text in terms of its difficulty. She concludes that “not only […] all translations are problem solving activities but all are difficult […], although some are, of course, more difficult than others” (ibid.: 137). 



%\label{ref:ZOTEROITEMCSLCITATIONcitationID6X13fIDxpropertiesformattedCitationWilss1994plainCitationWilss1994citationItemsid236urishttpzoteroorgusers1255332itemsKDSR29MCurihttpzoteroorgusers1255332itemsKDSR29MCitemDataid236typearticlejournaltitleAFrameworkforDecisionmakinginTranslationcontainertitleTargetpage131150volume6issue2authorfamilyWilssgivenWolframissueddateparts1994schemahttpsgithubcomcitationstylelanguageschemarawmastercslcitationjsonRNDiL2dMXPQ4W}
\citet{Wilss1996} argues that, as translation items usually have more than one possible \isi{target language} representation, the translator has to decide which to choose; this choice is determined by various characteristics of the individual translator as well as environmental influences. The translator needs declarative and procedural knowledge, which (s)he has to apply to \isi{macrocontextual problems} – that apply to the whole text, including factors like overall content, communicative purpose and intended readership – as well as \isi{microcontextual problems} – “includ[ing], amongst others, singular (episodic) phenomena of the text-to-be-translated” (ibid.: 135). General problem solving strategies can hardly be applied in the translation process, as translation problems can seldom be generalised and it is further possible for the translator to “schematically reduce translation problems to a sequence of standardly operative moves guaranteeing translational success” (ibid.: 136). Instead, the translator needs to learn problem solving strategies according to different domains and text types to create informed, professional translations. Further, Wilss points out that the \isi{problem solving strategies} might differ a lot between novices and expert translators, as they are learnt and evolve with increasing professionalisation. Finally, he divides the problem solving activity into six stages (ibid.: 145): problem identification, problem clarification (description), information collection, considerations on how to proceed, the choice of a solution, and the evaluation of translation result. %\label{ref:ZOTEROITEMCSLCITATIONcitationIDIISyTRIBpropertiesformattedCitationWilss1996plainCitationWilss1996citationItemsid29urishttpzoteroorgusers1255332itemsVB42BXHSurihttpzoteroorgusers1255332itemsVB42BXHSitemDataid29typebooktitleKnowledgeandskillsintranslatorbehaviorpublisherBenjaminspublisherplaceAmsterdamuasourceOpenWorldCateventplaceAmsterdamuaISBN1556196962languageEnglishauthorfamilyWilssgivenWolframissueddateparts1996schemahttpsgithubcomcitationstylelanguageschemarawmastercslcitationjsonRNDQZNpEYyJgm}
\citet[47--48]{Wilss1996} later agrees with \citet[136]{Kaiser-cooke1994} that “all translations are problem-solving activities” (and \citealt{Risku1998} agrees with this opinion, too), but reduces the range of problem solving activities to translation problems and therefore also implies that not every translation activity is problem solving. This idea is continued later in his argumentation:


\begin{quote}
Whereas translation method always requires problem-solving activities, the essential feature of translation techniques [...] is the subconscious, so to speak “self-monitoring” reproduction of specific, interlingually standardized text segments on the basis of functional one-to-one correspondence (with or without formal one-to-one correspondence). \citep[155--156]{Wilss1996}
\end{quote}


Further, he argues that translation science has not described problem solving systematically yet, but is aware of it and “has had, and still does have, great trouble in defining a suitable and reliable conceptual framework for problem-solving” (ibid.: 47). Additionally, he suggests that the field should explore problem solving in longitudinal studies on translation students, because there might not be “a straight-line, continuous growth from less to more competence in problem-solving” (ibid.: 48). Such investigations would provide insights into problem solving development and how this could be integrated into translation teaching.



A rather extensive theoretical approach on problem solving is offered by %\label{ref:ZOTEROITEMCSLCITATIONcitationIDRLGkmtUwpropertiesformattedCitationRisku1998plainCitationRisku1998citationItemsid81urishttpzoteroorgusers1255332items5AWQUSUMurihttpzoteroorgusers1255332items5AWQUSUMitemDataid81typebooktitleTranslatorischeKompetenzpublisherStauffenburgpublisherplaceTbingeneventplaceTbingenauthorfamilyRiskugivenHannaissueddateparts1998schemahttpsgithubcomcitationstylelanguageschemarawmastercslcitationjsonRNDBdSQ0JRatj}
\citet{Risku1998} in her discussion on translation expertise, in which she also considers literature from psychology. She, like \citet{Dorner1987}, considers the hurdle as a characteristic of problems, but in a different way: In her understanding, every act of thinking in general is a problem, because building a representation in the mind already requires overcoming a hurdle – only reflexes occur automatically and do not create something new %\label{ref:ZOTEROITEMCSLCITATIONcitationIDwXxRxvtjpropertiesformattedCitationRisku1998plainCitationRisku1998citationItemsid81urishttpzoteroorgusers1255332items5AWQUSUMurihttpzoteroorgusers1255332items5AWQUSUMitemDataid81typebooktitleTranslatorischeKompetenzpublisherStauffenburgpublisherplaceTbingeneventplaceTbingenauthorfamilyRiskugivenHannaissueddateparts1998schemahttpsgithubcomcitationstylelanguageschemarawmastercslcitationjsonRNDsRhQdJR9FS}
(cf. \citealt{Risku1998}: 50). Hence, translation is always a problem, never a task for Risku:


\begin{quote}
Translation, however, can never be a task in this sense.\footnote{Referring to Dörner's differentiation between problem and task.} The translator would need to be privy to source and target situation, the intentions of the client, the own role in the framework of action, would need to have already developed an individual production strategy suitable for the target communication, and would have finished the decision making process in order to posses all knowledge for the translation job right at the beginning. The translation would already need to be completed.\footnote{Original phrasing: “Übersetzen kann aber nie eine Aufgabe in diesem Sinne sein. Um bei der Auftragssituation das gesamte zum Übersetzen nötige Wissen 'bereit' zu haben, müßte der Übersetzende die Ausgangs- und Zielsituation, die Intentionen des Auftraggebers und die eigene Rolle in diesem Handlungsrahmen kennen, eine eigene, zielkommunikationsadäquate Produktionsstrategie entwickelt und den Entscheidungsprozess durchlaufen haben. Die Übersetzung müsste also bereits buchstäblich 'in der Tasche' sein."} %\label{ref:ZOTEROITEMCSLCITATIONcitationIDewUuy8OOpropertiesformattedCitationRisku1998plainCitationRisku1998citationItemsid81urishttpzoteroorgusers1255332items5AWQUSUMurihttpzoteroorgusers1255332items5AWQUSUMitemDataid81typebooktitleTranslatorischeKompetenzpublisherStauffenburgpublisherplaceTbingeneventplaceTbingenauthorfamilyRiskugivenHannaissueddateparts1998schemahttpsgithubcomcitationstylelanguageschemarawmastercslcitationjsonRNDxTnls1rl3p}
(ibid.: 226-227)
\end{quote}


Conclusively, according to Risku there is no difference between thinking and problem solving.\footnote{As we have seen at the beginning of \sectref{sec:5:2}, approaches in psychology and philosophy do not necessarily agree and neither do I. Find the discussion of the different thoughts and opinions in \sectref{sec:5:4}.} However, problem solving shall remain a concept in research as it emphasises the connection between internal and external activity and helps create models that represent cognitive activity in action situations (cf. ibid.: 50-51). \citegen{Funke2006b} concept of \isi{\textit{complex problem solving}} is considered especially appropriate by Risku for translation studies as it defines problem making situations with many dependencies and components; complex problems require plans for whole chains of courses of action. Hence, the ability to solve complex problems and the way these are solved are indicators of the level of expertise \citet[89]{Risku1998}. Experts can combine learned methods with their own experience, which makes these methods more appropriate for certain (translation-related) communication and problem types, and enables the experts to cooperate with all involved people (cf. ibid: 105) – the more connected knowledge exists, the more usable it is in problem solving situations (cf. ibid: 110). While novices tend to approach each single problem with formerly learned micro rules, experts let themselves be guided by communicative macro strategies and approach single problems more slowly, because not as many single problems evolve when the focus is on the macro level (cf. ibid.: 220). Risku's (cf. ibid.: 117) cognitive procedural approach assumes that the cognitive reality of the problem solver is characterised by four elements which all influence each other:


\begin{itemize}
\item the problem solver him\slash herself with his\slash her cognitive characteristics
\item the situation as the socio-cultural position and role of the expert
\item the aim of the translation from the translator's perspective (macro strategy)
\item the system that needs to be recognised and controlled (i.e. the translation purpose and the target communication with its references to the source communication)
\end{itemize}

An underestimated part of problem solving is the recognition of problems contained in a source text, which requires a great deal of expertise. As soon as the problem is identified, four \isi{problem solving strategies} that influence each other can be determined for the translation process (based on Dörner's solution requirements for complex problems): integrating information, composing a macro strategy, planning actions, and planning action schemes \& making decisions. The translation situation offers the guiding principles that tell the translator what to do and suggest the macro strategy for the individual translation job. Only the action plans and decisions tell the translator how to act (cf. ibid.: 136-139).



%\label{ref:ZOTEROITEMCSLCITATIONcitationIDjSMcishlpropertiesformattedCitationrtfOuc0u8217Brien2006plainCitationOBrien2006citationItemsid214urishttpzoteroorgusers1255332itemsAE7QW2HGurihttpzoteroorgusers1255332itemsAE7QW2HGitemDataid214typethesistitleMachinetranslatabilityandposteditingeffortAnempiricalstudyusingTranslogandChoiceNetworkAnalysispublisherDublinCityUniversityauthorfamilyOBriengivenSharonissueddateparts2006schemahttpsgithubcomcitationstylelanguageschemarawmastercslcitationjsonRNDY1ADM14FTv}
\citet{OBrien2006}
presents an interesting approach to problem identification in PE – one of the few studies that deal with problem solving in PE – in her dissertation (find more details on and findings of the study in \sectref{sec:4:2}). In the study, the source texts were examined by two controlled language checker systems that highlighted the parts of the source segments that did not abide the rules of the controlled language, so-called \isi{negative translatability indicators} (NTIs). Controlled languages are natural languages that are restrained in certain aspects to make texts easier to read for a broader audience. Consequently, it is assumed that texts written in a controlled language are easier to translate for both human translators and MT systems (for more information on controlled languages see \sectref{sec:3:2}). The NTIs include source text characteristics such as ungrammatical constructions, misspellings, or disrupted syntactic structures which are obviously source text defects but also regular parts of speech, which might be harder to process e.g., abbreviations, gerunds, slang, ellipses, etc. This is one of very few empirical studies that bases translation\slash PE problem identification not on the translator’s behaviour during the experiments, but first identifies potential problems and then tests whether or not these then influence the PE effort.



%\label{ref:ZOTEROITEMCSLCITATIONcitationIDHbBDQPxfpropertiesformattedCitationKubiak2009plainCitationKubiak2009dontUpdatetruecitationItemsid102urishttpzoteroorgusers1255332itemsKWMUG2RKurihttpzoteroorgusers1255332itemsKWMUG2RKitemDataid102typethesistitleUbersetzeralsProblemloserEinequalitativeStudiezumProblemloseverhaltenvonsemiprofessionellenUbersetzernpublisherWydawnictwoNaukoweUAMissueddateparts2009schemahttpsgithubcomcitationstylelanguageschemarawmastercslcitationjsonRNDXReZbNiDds}
\citet{Kubiak2009} presents in his PhD thesis a study using think-aloud protocols on problem solving in semi-professional translators. Eight participants were asked to translate a newspaper text from \ili{German} into \ili{Polish} and vocalise every thought and emotion that came to their mind without any omissions. The participants were grouped according to translation direction. Kubiak does not differentiate his participants into natives and non-natives, because the non-natives had a very heterogeneous \ili{Polish}-speaking family background, but he differentiated participants who were educated in \ili{Polish} (four participants) or in \ili{German} (four participants).\footnote{The studies were conducted at the Adam-Mickiewicz University in Poznań (\ili{Polish} group) and the University of Vienna (\ili{German} group).} He analysed the \isi{problem solving behaviour} of his participants and compared both groups to identify the differences in both groups and the preferred strategies. Another research goal was to uncover which deficits can be observed in the translation processes and where low-quality translations and mistranslations originate. In his theoretical framework on problem solving, Kubiak bases his considerations mainly on %\label{ref:ZOTEROITEMCSLCITATIONcitationIDvMcdIIbypropertiesformattedCitationRisku1998plainCitationRisku1998citationItemsid81urishttpzoteroorgusers1255332items5AWQUSUMurihttpzoteroorgusers1255332items5AWQUSUMitemDataid81typebooktitleTranslatorischeKompetenzpublisherStauffenburgpublisherplaceTbingeneventplaceTbingenauthorfamilyRiskugivenHannaissueddateparts1998schemahttpsgithubcomcitationstylelanguageschemarawmastercslcitationjsonRNDjmzYmhoyfg}
\citegen{Risku1998} as well as %\label{ref:ZOTEROITEMCSLCITATIONcitationIDBjDcrvGMpropertiesformattedCitationKrings1986cplainCitationKrings1986cdontUpdatetruecitationItemsid103urishttpzoteroorgusers1255332itemsVMUMBHRNurihttpzoteroorgusers1255332itemsVMUMBHRNitemDataid103typebooktitleWasindenKpfenvonbersetzernvorgehtEineempirischeUntersuchungzurStrukturdesbersetzungsprozessesanfortgeschrittenenFranzsischlernernpublisherGunterNarrVerlagpublisherplaceTbingenvolume291eventplaceTbingenauthorfamilyKringsgivenHansPissueddateparts1986schemahttpsgithubcomcitationstylelanguageschemarawmastercslcitationjsonRNDCHosvVmFUM}
\citegen{Krings1986} observations from the translation perspective and \citegen{Dorner1987} work from the psychological perspective. Hence, he agrees that translation can be categorised as complex problem solving. Factors that influence the composition of the problem space – the mental representation of the problems – are (%\label{ref:ZOTEROITEMCSLCITATIONcitationIDkCInRQr3propertiesformattedCitationKubiak2009plainCitationKubiak2009dontUpdatetruecitationItemsid102urishttpzoteroorgusers1255332itemsKWMUG2RKurihttpzoteroorgusers1255332itemsKWMUG2RKitemDataid102typethesistitleUbersetzeralsProblemloserEinequalitativeStudiezumProblemloseverhaltenvonsemiprofessionellenUbersetzernpublisherWydawnictwoNaukoweUAMissueddateparts2009schemahttpsgithubcomcitationstylelanguageschemarawmastercslcitationjsonRNDGPCuQwxlEo}
\citealt{Kubiak2009}: 46-85):


\begin{itemize}
\item individual factors: translation knowledge\slash competence (linguistic, cultural, domain, and tool knowledge), memory, creativity, and further factors (e.~g. emotions, motivation, etc.)
\item translation skopos
\item environmental factors
\end{itemize}

\citet[96]{Kubiak2009} defines translation problems as subjective difficulties\footnote{As opposed to \citet{Nord1987}.} that a translator with certain knowledge in a certain translation situation has to overcome to produce a target text that fulfils the translation skopos. In contrast, problem solving strategies are “potentially conscious behavioural patterns of a translator to solve emerging translation problems which have to be seen as a transcultural translation task within the scope of a certain translation task”\footnote{Original phrasing: “Unter Problemlösungsstrategien sind potenziell bewusste Verhaltensmuster eines Übersetzers zur Lösung emergierender Übersetzungsprobleme im Rahmen einer bestimmten Translationsaufgabe als transkultureller Kommunikationsaufgabe zu verstehen.”} (ibid.: 99, translated J.N.; The similarity to \citeapo{Krings1986b} definition of translation strategies, mentioned above, cannot be denied). The think-aloud protocols were transcribed and analysed qualitatively according to \citeapo{Krings1986} primary and secondary indicators for problems. Kubiak replicates the study of Krings with other participant groups and combines the thoughts on problem solving of Krings and Risku. However, the study does not really contribute anything new to the field and, hence, is only described for the sake of completeness here.



%\label{ref:ZOTEROITEMCSLCITATIONcitationID2RmbufnGpropertiesformattedCitationPrassl2010plainCitationPrassl2010citationItemsid225urishttpzoteroorgusers1255332itemsNQ5FRI5Purihttpzoteroorgusers1255332itemsNQ5FRI5PitemDataid225typearticlejournaltitleTranslatorsdecisionmakingprocessesinresearchandknowledgeintegrationcontainertitleNewapproachesintranslationprocessresearchpage5782authorfamilyPrasslgivenFriederikeissueddateparts2010schemahttpsgithubcomcitationstylelanguageschemarawmastercslcitationjsonRNDUbdw9PB7Jg}
\citet{Prassl2010} focuses on different decision making processes that could be identified in the think-aloud protocols in her study. She bases her categorisation on %\label{ref:ZOTEROITEMCSLCITATIONcitationIDhnG1wdqlpropertiesformattedCitationHJungermannPFISTERandFischer2005plainCitationHJungermannPFISTERandFischer2005dontUpdatetruecitationItemsid110urishttpzoteroorgusers1255332itemsWJSF2SEWurihttpzoteroorgusers1255332itemsWJSF2SEWitemDataid110typebooktitleDiePsychologiederEntscheidungEineEinfhrungpublisherSpektrumAkademischerVerlagpublisherplaceHeidelbergedition2eventplaceHeidelbergauthorfamilyJungermanngivenHfamilyPfistergivenHansRdigerfamilyFischergivenKissueddateparts2005schemahttpsgithubcomcitationstylelanguageschemarawmastercslcitationjsonRNDRViUyzBO5G}
\citet{JungermannEtAl2005} classification and adapts it to the translation process, resulting in four decision making types: routinised, stereotype, reflected, and constructed decisions. \isi{\textit{Routinised} decisions} happen unconsciously and the options are evaluated automatically; similar patterns have been handled before, the choice for the option is routinised, the cognitive effort is very small. In \textit{sterotype} decisions, options are perceived unconsciously as well, but a minor uncontrolled evaluation takes place. A translator might, for example, first decide on one option, but change the translation immediately in favour of another option. The cognitive effort is still very low. Part of the option occurs automatically in \isi{\textit{reflected decisions}} as well, but these options are not satisfying for the goal, which usually has to be defined first, and new options have to be generated. Experts show different patterns and behaviour in dealing with reflected decisions than novices. Experts rather look for satisficing (satisfying and sufficient) than optimising strategies, because optimising strategies might lead to a never-ending process (a phenomenon well known to translators). The decision is not made right away, but might be postponed until enough evidence for the right decision has been gathered. A high degree of cognitive effort is necessary. \isi{\textit{Constructed decisions}} have to be made when the translation goal is not clear and the translator might not understand the \isi{translation unit} due to lacking linguistic or world knowledge. Internal and external knowledge has to be consulted to generate options, because newly acquired knowledge is necessary for the decision. The decider cannot rely on previous experience and habits. The \isi{cognitive load} is very high. If not enough information can be gathered due to lacking time and\slash or external resources, the translator might end up having to guess a solution (cf. %\label{ref:ZOTEROITEMCSLCITATIONcitationIDyWlFwJNrpropertiesformattedCitationPrassl2010plainCitationPrassl2010citationItemsid225urishttpzoteroorgusers1255332itemsNQ5FRI5Purihttpzoteroorgusers1255332itemsNQ5FRI5PitemDataid225typearticlejournaltitleTranslatorsdecisionmakingprocessesinresearchandknowledgeintegrationcontainertitleNewapproachesintranslationprocessresearchpage5782authorfamilyPrasslgivenFriederikeissueddateparts2010schemahttpsgithubcomcitationstylelanguageschemarawmastercslcitationjsonRND76omA8MvDg}
\citealt{Prassl2010}: 61-65). The data that were analysed in Prassl's (ibid.) study were part of the TransComp research project.\footnote{A longitudinal translation process study; for an overview and detailed information see e.g. %\label{ref:ZOTEROITEMCSLCITATIONcitationIDQG2Rg7OYpropertiesformattedCitationrtfGuc0u246pferichBayerHohenwarterandStigler2008plainCitationGpferichBayerHohenwarterandStigler2008citationItemsid224urishttpzoteroorgusers1255332itemsC6RXMJ8Kurihttpzoteroorgusers1255332itemsC6RXMJ8KitemDataid224typebooktitleTransCompTheDevelopmentofTranslationCompetenceCorpusandAssetManagementSystemfortheLongitudinalStudyTransComppublisherGrazUniversityofGrazhttpgamsunigrazatcontainertclastaccessedAugust312009authorfamilyGpferichgivenSusannefamilyBayerHohenwartergivenGerritfamilyStiglergivenHubertissueddateparts2008schemahttpsgithubcomcitationstylelanguageschemarawmastercslcitationjsonRNDdyh5nz8kIH}
\citealt{GopferichEtAl2008}.} The decision making processes of 12 BA students at the beginning of their studies were compared to those of ten professional translators by looking at the translation process data (screen recording, keylogging and think aloud protocols) of five difficult source text phrases that occurred in the texts. Her analysis shows that professional translators made correct decisions more often than novices, although only slightly more than half of the decisions (13 of 25) of the professionals could be considered correct – and only two of 25 decisions of the novices were correct. Further, professionals used routinised decisions more often than novices, which was expected, and the success rate of reflected decisions was much higher for professionals than for novices.



In another think aloud study, %\label{ref:ZOTEROITEMCSLCITATIONcitationIDnN5KouQIpropertiesformattedCitationAngelone2010plainCitationAngelone2010citationItemsid158urishttpzoteroorgusers1255332itemsBD4NRZGMurihttpzoteroorgusers1255332itemsBD4NRZGMitemDataid158typechaptertitleUncertaintyuncertaintymanagementandmetacognitiveproblemsolvinginthetranslationtaskcontainertitleTranslationandcognitioncollectiontitleAmericanTranslatorsAssociationScholarlyMonographSeriescollectionnumber15publisherJohnBenjaminsPublishingCompanypublisherplaceAmsterdamPhiladelphiapage1740eventplaceAmsterdamPhiladelphiaauthorfamilyAngelonegivenErikeditorfamilyShrevegivenGregoryMfamilyAngelonegivenErikissueddateparts2010schemahttpsgithubcomcitationstylelanguageschemarawmastercslcitationjsonRNDDzP4uUYzkq}
\citet{Angelone2010} combined uncertainty management and problem solving. As a basic definition, translation is seen as a “higher order cognitive task, like reading and writing, but with a very significant problem solving component concerned with mediation between languages” which makes the translation task a “chain of decision making activities relying on multiple, interconnected sequences of \isi{problem solving behaviour}” (ibid.: 17). Uncertainty is specified “as a cognitive state of indecision” and can be recognised by “an observable interruption in the natural flow of translation” (ibid.: 18), which helps to identify the problem. The problem solving activity is divided into three elements which are also the three elements that constitute the optimal \isi{problem solving bundle}: problem recognition, solution proposal, and solution evaluation (cf. ibid.: 20). It is assumed that problem solving bundles are used to manage uncertainty. One professional and three student translators were asked to translate a 50-word excerpt from a travel guide in this study to identify when, where, and how problem solving bundles are used, how the metacognitive activity varies between professionals and semi-professionals and when the metacognitive activities are associated with uncertainty management. Screen recording and think-aloud protocols were applied, analysed, and triangulated. The results suggest that professional translators have a greater capacity to recognise problems than novices.



As we have seen, translation studies uses the terms \textit{decision making} and \textit{problem solving} for the analysis of the translation process. However, there seems to be no consensus on how these terms are used and whether translation can be described as one and\slash or the other. These desiderata will be tackled in the following section.


\section[Modeling the concept of problem solving in translation studies by adding psychological approaches]{Modeling the concept of problem solving in translation studies by adding psychological approaches\sectionmark{Modeling the concept of problem solving in translation studies}}\sectionmark{Modeling the concept of problem solving in translation studies}
\label{sec:5:4}

\largerpage
In this section, the views and ideas of psychology on problem solving will be applied to the translation process and there will be an examination of what has been discussed about problem solving in translation studies in recent approaches. Let us first discuss the terminology. Truly, a translator has the choice of many different expressions, structures, styles, etc. and translation can accordingly be a decision making activity. However, often it is more than the simple selection of one of a variety of expressions, because (a variety of) target text expressions or units are not always immediately available for every source text unit in the mind of a translator. \citegen{Dorner1987} basic definition of problems suggests that translation processes include both problems and tasks. When we apply Dörner's definitions to the very foundation of translation, it becomes obvious that translation in its basic form is the task. The source text is the (undesired) initial state that has to be transformed into the target text – the desired final state. As soon as the translator has to deal with a \isi{hurdle} that prevents him\slash her from transferring the source text into the target text, the translator faces a problem and needs a plan to solve this problem. Hence, the definitions and differentiations in psychology and some publications in translation studies make it obvious that translation will be considered a problem solving activity in the following if it exceeds the conditions of a simple task – because it is a more complex activity than the simple selection between different choices, although translation was also referred to as a decision making process, especially in earlier considerations. A problem further requires social, intellectual, or cultural motivation to arrive at the unknown target state, as mentioned in %\label{ref:ZOTEROITEMCSLCITATIONcitationIDCjjUDam0propertiesformattedCitationJonassen2000plainCitationJonassen2000citationItemsid243urishttpzoteroorgusers1255332items5E7DVHUGurihttpzoteroorgusers1255332items5E7DVHUGitemDataid243typearticlejournaltitleTowardadesigntheoryofproblemsolvingcontainertitleEducationaltechnologyresearchanddevelopmentpage6385volume48issue4authorfamilyJonassengivenDavidHissueddateparts2000schemahttpsgithubcomcitationstylelanguageschemarawmastercslcitationjsonRNDQGv3Nrfr1r}
\citet{Jonassen2000}. This clearly applies to translation situations. If nobody wanted a translation of the source text in the \isi{target language}, there would be no translation job. However, the characteristics of a problem do not apply to every translation instance. Although %\label{ref:ZOTEROITEMCSLCITATIONcitationIDmTgDp0CGpropertiesformattedCitationKaiserCooke1994plainCitationKaiserCooke1994citationItemsid108urishttpzoteroorgusers1255332itemsIU4P2FXUurihttpzoteroorgusers1255332itemsIU4P2FXUitemDataid108typechaptertitleTranslatorialExpertiseACrossCulturalPhenomenonfromanInterdisciplinaryPerspectivecontainertitleTranslationStudiesAnInterdisciplinepublisherJohnBenjaminsPublishingCompanypublisherplaceAmsterdamPhiladelphiapage135139eventplaceAmsterdamPhiladelphiaauthorfamilyKaiserCookegivenMichleeditorfamilySnellHornbygivenMaryfamilyPchhackergivenFranzfamilyKaindlgivenKlausissueddateparts1994schemahttpsgithubcomcitationstylelanguageschemarawmastercslcitationjsonRNDIiUnc5ZExV}
\citet{Kaiser-cooke1994} and %\label{ref:ZOTEROITEMCSLCITATIONcitationIDrjcd31kGpropertiesformattedCitationRisku1998plainCitationRisku1998citationItemsid81urishttpzoteroorgusers1255332items5AWQUSUMurihttpzoteroorgusers1255332items5AWQUSUMitemDataid81typebooktitleTranslatorischeKompetenzpublisherStauffenburgpublisherplaceTbingeneventplaceTbingenauthorfamilyRiskugivenHannaissueddateparts1998schemahttpsgithubcomcitationstylelanguageschemarawmastercslcitationjsonRNDq2299lz7QY}
\citet{Risku1998} express as their basic assumptions that all translation activities are problem solving instances, which could be argued even from psychology's perspective on problem solving, their considerations on problem solving contradict the definitions of %\label{ref:ZOTEROITEMCSLCITATIONcitationIDQOMmdOeOpropertiesformattedCitationrtfDuc0u246rner1987plainCitationDrner1987citationItemsid247urishttpzoteroorgusers1255332items2V6UZUD9urihttpzoteroorgusers1255332items2V6UZUD9itemDataid247typebooktitleProblemlsenalsInformationsverarbeitungcollectiontitleKohlhammerStandardsPsychologieBasisbcherundStudientextepublisherKohlhammerpublisherplaceStuttgartnumberofpages151edition3AuflsourceGemeinsamerBibliotheksverbundISBNeventplaceStuttgartISBN3170097113languagegerauthorfamilyDrnergivenDietrichissueddateparts1987schemahttpsgithubcomcitationstylelanguageschemarawmastercslcitationjsonRNDv49crNUMxu}
\citet{Dorner1987}. A problem needs to have a hurdle between initial state and solution to qualify as problem and not merely a task. If the problem can be solved and the problem solver does not have to overcome a hurdle, it is not a problem, it is a task. Hence, with a growing level of expertise, the translation activity converts from a problem solving activity to a task solving activity (although some problems might still arise, no matter how experienced a translator is), especially when the translator works in his\slash her standardised working environment (well-known text domain, text type, terminology, tool, client, etc.). It is well documented that translators need to specialise to become good and efficient translators\footnote{If the job market allows the translator to specialise. This is possible for language combinations with large translation volumes like \ili{English}-\ili{German}. However, combinations of very small languages might require all-round translators instead.} (e.g. %\label{ref:ZOTEROITEMCSLCITATIONcitationIDmN4SGgaSpropertiesformattedCitationSchmitt2003aplainCitationSchmitt2003acitationItemsid78urishttpzoteroorgusers1255332itemsC5QBMRTCurihttpzoteroorgusers1255332itemsC5QBMRTCitemDataid78typechaptertitleBerufsbildcontainertitleHandbuchTranslationpublisherStauffenburgpublisherplaceTbingenpage15edition2eventplaceTbingenauthorfamilySchmittgivenPeterAeditorfamilySnellHornbygivenMaryfamilyHniggivenHansGfamilyKumaulgivenPaulfamilySchmittgivenPeterAissueddateparts2003schemahttpsgithubcomcitationstylelanguageschemarawmastercslcitationjsonRNDWI1p9FLIXe}
\citealt{Schmitt2003berufsbild,Schmitt2003markt}, 
%\label{ref:ZOTEROITEMCSLCITATIONcitationIDiNYFMEUYpropertiesformattedCitationrtfHommerichandReiuc0u2232011plainCitationHommerichandRei2011citationItemsid6urishttpzoteroorgusers1255332items536WSTZXurihttpzoteroorgusers1255332items536WSTZXitemDataid6typearticletitleErgebnissederBDMitgliederbefragungauthorfamilyHommerichgivenChristophfamilyReigivenNicoleissueddateparts20114schemahttpsgithubcomcitationstylelanguageschemarawmastercslcitationjsonRNDUrlM0yz16T}
\citealt{HommerichReiß2011}) and as I do not assume the existence of a general problem solver,\footnote{The notion of a general problem solver goes back to %\label{ref:ZOTEROITEMCSLCITATIONcitationIDY3RToJRHpropertiesformattedCitationNewellandSimon1972plainCitationNewellandSimon1972citationItemsid234urishttpzoteroorgusers1255332items3N52C7H8urihttpzoteroorgusers1255332items3N52C7H8itemDataid234typebooktitleHumanproblemsolvingcollectionnumber9publisherPrenticeHallpublisherplaceEnglewoodCliffsNJvolume104eventplaceEnglewoodCliffsNJauthorfamilyNewellgivenAllenfamilySimongivenHerbertAlexanderissueddateparts1972schemahttpsgithubcomcitationstylelanguageschemarawmastercslcitationjsonRNDENiB5FKMYj}
\citet{Newell1972} and was very influential in the field of artificial intelligence. Their computer system was capable of solving a variety of well-defined problems, although not all of them in the same manner as a human would have performed it. They computed their simulation according to think-aloud protocols of humans who had to solve well-defined problems (for a brief introduction see e.g. %\label{ref:ZOTEROITEMCSLCITATIONcitationIDu4ls0cE1propertiesformattedCitationEysenck2004plainCitationEysenck2004citationItemsid233urishttpzoteroorgusers1255332itemsHS3I6PUPurihttpzoteroorgusers1255332itemsHS3I6PUPitemDataid233typebooktitlePsychologyAninternationalperspectivepublisherTaylorFrancispublisherplaceHoveNewYorkeventplaceHoveNewYorkauthorfamilyEysenckgivenMichaelWissueddateparts2004schemahttpsgithubcomcitationstylelanguageschemarawmastercslcitationjsonRND2Lj0dVPHr0}
\citealt[341-342]{Eysenck2004}).} I do not assume the existence of a general translation solver (neither human nor machine-made) or general problem solving strategies as mentioned by %\label{ref:ZOTEROITEMCSLCITATIONcitationIDPujSlyKYpropertiesformattedCitationWilss1994plainCitationWilss1994citationItemsid236urishttpzoteroorgusers1255332itemsKDSR29MCurihttpzoteroorgusers1255332itemsKDSR29MCitemDataid236typearticlejournaltitleAFrameworkforDecisionmakinginTranslationcontainertitleTargetpage131150volume6issue2authorfamilyWilssgivenWolframissueddateparts1994schemahttpsgithubcomcitationstylelanguageschemarawmastercslcitationjsonRNDdVnIS4M2Kf}
\citet{Wilss1994}. %\label{ref:ZOTEROITEMCSLCITATIONcitationIDVT3uJca7propertiesformattedCitationKrings1986cplainCitationKrings1986cdontUpdatetruecitationItemsid103urishttpzoteroorgusers1255332itemsVMUMBHRNurihttpzoteroorgusers1255332itemsVMUMBHRNitemDataid103typebooktitleWasindenKpfenvonbersetzernvorgehtEineempirischeUntersuchungzurStrukturdesbersetzungsprozessesanfortgeschrittenenFranzsischlernernpublisherGunterNarrVerlagpublisherplaceTbingenvolume291eventplaceTbingenauthorfamilyKringsgivenHansPissueddateparts1986schemahttpsgithubcomcitationstylelanguageschemarawmastercslcitationjsonRNDdP41Lcnj2x}
\citet{Krings1986} %Krings 1986b
does not view every translation action as a problem solving activity either. Rather, he defines different problem indicators from the think-aloud protocols, which indicate that only certain instances cause problems.



In line with these assumptions, both \isi{translation difficulties and problems} as defined by %\label{ref:ZOTEROITEMCSLCITATIONcitationIDT6ZCcFPRpropertiesformattedCitationCNord1987plainCitationCNord1987citationItemsid246urishttpzoteroorgusers1255332itemsMMBEBR62urihttpzoteroorgusers1255332itemsMMBEBR62itemDataid246typearticlejournaltitlebersetzungsproblemebersetzungsschwierigkeitenWasindenKpfenvonbersetzernvorgehensolltecontainertitleMitteilungsblattfrDolmetscherundbersetzerpage58volume2issue1987authorfamilyNordgivenChristianeissueddateparts1987schemahttpsgithubcomcitationstylelanguageschemarawmastercslcitationjsonRNDJrCuNL407E}
\citet{Nord1987} will be analysed and summarised as translation problems in the study at hand. The distinction by %\label{ref:ZOTEROITEMCSLCITATIONcitationIDAglu49jlpropertiesformattedCitationNord1987plainCitationNord1987dontUpdatetruecitationItemsid246urishttpzoteroorgusers1255332itemsMMBEBR62urihttpzoteroorgusers1255332itemsMMBEBR62itemDataid246typearticlejournaltitlebersetzungsproblemebersetzungsschwierigkeitenWasindenKpfenvonbersetzernvorgehensolltecontainertitleMitteilungsblattfrDolmetscherundbersetzerpage58volume2issue1987authorfamilyNordgivenChristianeissueddateparts1987schemahttpsgithubcomcitationstylelanguageschemarawmastercslcitationjsonRND8zB1vzaKw1}
Nord (cf. ibid.: 7) is reasonable and insightful, but she also claims that only the “ideal” translator does not have to cope with translation difficulties, while the “real” translator always has to struggle with difficulties, even with a lot of experience (although they are supposed to decrease with growing translation experience). Therefore, she argues that it is part of the translator's competence to know how to deal with these difficulties. As the study comprises professional and semi-professional translators, I am not only interested in Nord's translation problems alone but also in what she defines as difficulties and how translators deal with both groups. Additionally, all participants have a certain amount of experience (even the semi-professional translators; see \sectref{sec:8:1}) which rules out beginner's mistakes. As the difference between a translation difficulty and a translation problem according to \citet{Nord1987} is that the first is an individual phenomenon (one translator has a difficulty with a source text unit, while the next translator does not have any difficulty with the source text unit), while the second applies to a text unit that is problematic in itself, independent of the individual, it is almost impossible to differentiate between difficulty and problem in mere process data. Further, the definition of \textit{problems} in psychology would include translation problems as well as translation difficulties, too, because there is a hurdle between the source text and the target text for the individual, which (s)he has to overcome.



We have to keep in mind that \isi{translation problems sometimes} apply to small text units (micro structure) and sometimes to larger chunks or the whole text (macro structure). Still, the problems that are focused on in psychology are often more broad and time-consuming than single problems in translation. Even well-structured problems might take longer to solve than most problems in translations; well-structured problems do not need to be easy for a person who has never encountered the problem before. And some very ill-structured, \isi{complex problems} such as ending the Middle-East conflict or stopping global warming might takes years to solve, if possible at all. Hence, we apply rules and assumptions that were defined for broader contexts and situations to smaller units in the translation context. In general, however, translation problems can seldom be categorised as well-defined, but most often as ill-defined. One argument is that the desired final state of the problem is never known to the translator (however, sometimes the final state is already known in \isi{well-defined problems}) and another is that the means that help the translators to arrive at a solution are sometimes also unknown to the translator. If a translator, e.g., simply does not know a lexical unit, (s)he knows that (s)he has to consult a dictionary or glossary. However, if a source text unit applies to a cultural or linguistic feature that does not exist in the \isi{target language}\slash culture, the procedure of finding a solution is not that obvious. Accordingly, what might be ill-defined problems for novice translators, can be well-defined problems for professional translators because the translator familiarises himself\slash herself with more and more operators and develops more and more strategies to overcome hurdles with advancing training and experience.



To extend this context, \isi{experience} and growing expertise change the translation process, and that which is considered a problem can become a simple task, as discussed by %\label{ref:ZOTEROITEMCSLCITATIONcitationIDTvefn9k7propertiesformattedCitationEricsson2003plainCitationEricsson2003citationItemsid134urishttpzoteroorgusers1255332itemsTWGZGJG8urihttpzoteroorgusers1255332itemsTWGZGJG8itemDataid134typechaptertitleTheacquisitionofexpertperformanceasproblemsolvingcontainertitleThepsychologyofproblemsolvingpublisherCambridgeUniversityPresspublisherplaceCambridgeaopage3183eventplaceCambridgeaoauthorfamilyEricssongivenAeditorfamilyDavidsongivenJanetEfamilySternberggivenRobertJissueddateparts2003schemahttpsgithubcomcitationstylelanguageschemarawmastercslcitationjsonRNDEVpEZXTWMN}
\citet{EricssonSternberg2003} for problem solving in general. While some difficult grammatical structures are problematic at first, because the inexperienced translator does not have any plans or strategies to solve this problem yet, an experienced translator might have encountered this structure various times and therefore only has to solve a task, not a problem, as there is no hurdle between source text and target text. Translation students might encounter numerous problems at the beginning of their studies as they still can be considered laypersons. However, solutions are found and strategies are learned for some problems with growing experience that can be reused in future translations. Therefore, there is no hurdle between the source and the target text (any more), which would classify the translation as a task rather then a problem solving activity. I suggest that many source texts contain translation problems, that problems can be encountered in addressing the target audience, fulfilling the translation skopos\slash brief, or that problems can be caused by time pressure, missing information, cultural or linguistic differences, or (non-)usage of translation technologies – even for the most well-trained and most experienced translator. Nonetheless, the majority of translation activities become routinised (and automatised) over time and are therefore not problematic (any longer). This is also verified in translation studies as it is assumed that laypersons and translation beginners consider different aspects of the translation as problematic and use different problem solving strategies than experts. Beginners may rely more on the source text and on micro structural strategies, while experts are more influenced by their communicative macro strategies with smaller problems fading from the spotlight (cf. %\label{ref:ZOTEROITEMCSLCITATIONcitationID52Iz1nKjpropertiesformattedCitationRisku1998plainCitationRisku1998citationItemsid81urishttpzoteroorgusers1255332items5AWQUSUMurihttpzoteroorgusers1255332items5AWQUSUMitemDataid81typebooktitleTranslatorischeKompetenzpublisherStauffenburgpublisherplaceTbingeneventplaceTbingenauthorfamilyRiskugivenHannaissueddateparts1998schemahttpsgithubcomcitationstylelanguageschemarawmastercslcitationjsonRNDPH8Sq2xzop}
\citealt{Risku1998}: 220 – an assumption that was confirmed for semi-professionals in %\label{ref:ZOTEROITEMCSLCITATIONcitationIDyuwDBvUGpropertiesformattedCitationKubiak2009plainCitationKubiak2009dontUpdatetruecitationItemsid102urishttpzoteroorgusers1255332itemsKWMUG2RKurihttpzoteroorgusers1255332itemsKWMUG2RKitemDataid102typethesistitleUbersetzeralsProblemloserEinequalitativeStudiezumProblemloseverhaltenvonsemiprofessionellenUbersetzernpublisherWydawnictwoNaukoweUAMissueddateparts2009schemahttpsgithubcomcitationstylelanguageschemarawmastercslcitationjsonRNDAcGjEVu3Xs}
Kubiak's 2009 analysis of his data). This acknowledged change in problem solving strategies also implies that an expert translator has to deal with different (and presumably fewer) problems than beginners. The assumption by %\label{ref:ZOTEROITEMCSLCITATIONcitationIDDFOlV96UpropertiesformattedCitationrtfJuc0u228uc0u228skeluc0u228inenandTirkkonenCondit1991plainCitationJskelinenandTirkkonenCondit1991citationItemsid116urishttpzoteroorgusers1255332itemsX9XQCN27urihttpzoteroorgusers1255332itemsX9XQCN27itemDataid116typechaptertitleAutomatisedprocessesinprofessionalvsnonprofessionaltranslationAthinkaloudprotocolstudycontainertitleEmpiricalresearchintranslationandinterculturalstudiespublisherGunterNarrpublisherplaceTbingenpage89109eventplaceTbingenauthorfamilyJskelinengivenRiittafamilyTirkkonenConditgivenSonjaissueddateparts1991schemahttpsgithubcomcitationstylelanguageschemarawmastercslcitationjsonRND6kCb1Kx5nO}
\citet{JaaskelainenTirkkonen-Condit1991} that certain activities in the translation process become automatised with increasing translation experience strengthens the premise that translators' problem perception changes with growing expertise (an assumption also posited by %\label{ref:ZOTEROITEMCSLCITATIONcitationIDMJpissCapropertiesformattedCitationWilss1981plainCitationWilss1981citationItemsid235urishttpzoteroorgusers1255332itemsGPB6WMCAurihttpzoteroorgusers1255332itemsGPB6WMCAitemDataid235typearticlejournaltitleHandlungstheoretischeAspektedesbersetzungsprozessescontainertitleEuropischeMehrsprachigkeitFestschriftzumpage455468volume70authorfamilyWilssgivenWolframissueddateparts1981schemahttpsgithubcomcitationstylelanguageschemarawmastercslcitationjsonRND3hi3ZFvVpx}
\citealt{Wilss1981}, %\label{ref:ZOTEROITEMCSLCITATIONcitationIDDheoduOcpropertiesformattedCitationHansPKrings1986bplainCitationHansPKrings1986bcitationItemsid103urishttpzoteroorgusers1255332itemsVMUMBHRNurihttpzoteroorgusers1255332itemsVMUMBHRNitemDataid103typebooktitleWasindenKpfenvonbersetzernvorgehtEineempirischeUntersuchungzurStrukturdesbersetzungsprozessesanfortgeschrittenenFranzsischlernernpublisherGunterNarrVerlagpublisherplaceTbingenvolume291eventplaceTbingenauthorfamilyKringsgivenHansPissueddateparts1986schemahttpsgithubcomcitationstylelanguageschemarawmastercslcitationjsonRNDXZG5IdVB3s}
\citealt{Krings1986}
%\label{ref:ZOTEROITEMCSLCITATIONcitationID9fzAntFJpropertiesformattedCitationKrings1986bplainCitationKrings1986bdontUpdatetruecitationItemsid1042urishttpzoteroorggroups3587itemsQFD3HVEVurihttpzoteroorggroups3587itemsQFD3HVEVitemDataid1042typebooktitleWasindenKpfenvonbersetzernvorgehtcollectiontitleTbingerBeitrgezurLinguistikcollectionnumber291publisherNarrpublisherplaceTbingennumberofpages570eventplaceTbingenISBN9783878082910languagedeauthorfamilyKringsgivenHansPissueddateparts1986schemahttpsgithubcomcitationstylelanguageschemarawmastercslcitationjsonRNDjfVeYxuE6e}
 – who excluded professional translators from his study, because the group would have been too heterogeneous and automatised translation processes cannot be verbalised in think-aloud protocols – and in later studies). The purpose of their study was to show that these automatised processes exist for experts and cannot be verbalised during think-aloud and hence the protocols differ naturally between novice and expert translators. The second goal was to examine whether automation also takes place during the translation task itself, which turned out to be true. When processes become automatised with growing experience, it cannot be assumed that all translation processes are problem-driven and that problem solving is ubiquitous in translation processes. Only translation units that are not automatised require a high \isi{cognitive load} and hence can be considered problematic in the translation process. %\label{ref:ZOTEROITEMCSLCITATIONcitationIDyTxnszcBpropertiesformattedCitationrtfJuc0u228uc0u228skeluc0u228inenandTirkkonenCondit1991plainCitationJskelinenandTirkkonenCondit1991citationItemsid116urishttpzoteroorgusers1255332itemsX9XQCN27urihttpzoteroorgusers1255332itemsX9XQCN27itemDataid116typechaptertitleAutomatisedprocessesinprofessionalvsnonprofessionaltranslationAthinkaloudprotocolstudycontainertitleEmpiricalresearchintranslationandinterculturalstudiespublisherGunterNarrpublisherplaceTbingenpage89109eventplaceTbingenauthorfamilyJskelinengivenRiittafamilyTirkkonenConditgivenSonjaissueddateparts1991schemahttpsgithubcomcitationstylelanguageschemarawmastercslcitationjsonRNDkQ1xbOgtbi}
\citet[106]{JaaskelainenTirkkonen-Condit1991} conclude that “[w]hile some decisions become non-conscious, or 'automatic', the translator becomes sensitised to new aspects of the task which require conscious decision-making.” Further, as %\label{ref:ZOTEROITEMCSLCITATIONcitationID9dm3TS7xpropertiesformattedCitationKaiserCooke1993plainCitationKaiserCooke1993citationItemsid222urishttpzoteroorgusers1255332itemsN458885Kurihttpzoteroorgusers1255332itemsN458885KitemDataid222typearticlejournaltitleMachineTranslationandthehumanfactorKnowledgeanddecisionmakinginthetranslationprocesscontainertitleUnpublishedPhDdissertationUniversityofViennaauthorfamilyKaiserCookegivenMichleissueddateparts1993schemahttpsgithubcomcitationstylelanguageschemarawmastercslcitationjsonRNDheloHJs2gX}
\citet[187]{Kaiser-cooke1993} (emphasis in original text) puts it in her dissertation: “Decisions which seem trivial because they are taken 'automatically' by humans are still decisions.” The statements confirm the assumption that translation is a constant decision making process, while problems only occur, when a hurdle between source and target text exists.\footnote{This contradicts %\label{ref:ZOTEROITEMCSLCITATIONcitationIDTXgyciimpropertiesformattedCitationKaiserCooke1994plainCitationKaiserCooke1994citationItemsid108urishttpzoteroorgusers1255332itemsIU4P2FXUurihttpzoteroorgusers1255332itemsIU4P2FXUitemDataid108typechaptertitleTranslatorialExpertiseACrossCulturalPhenomenonfromanInterdisciplinaryPerspectivecontainertitleTranslationStudiesAnInterdisciplinepublisherJohnBenjaminsPublishingCompanypublisherplaceAmsterdamPhiladelphiapage135139eventplaceAmsterdamPhiladelphiaauthorfamilyKaiserCookegivenMichleeditorfamilySnellHornbygivenMaryfamilyPchhackergivenFranzfamilyKaindlgivenKlausissueddateparts1994schemahttpsgithubcomcitationstylelanguageschemarawmastercslcitationjsonRNDjO6RxDLiaK}
\citegen{Kaiser-cooke1994} statement that all translations are problem solving activities (as mentioned above).} On the other hand, she argues (ibid: 217-218) that with growing experience, translators formulate problem prototypes which can be applied to new problems, but each problem is at least slightly different from another problem and hence “new problem-solving strategies, i.e. decisions which have not been taken before” (218) are required. Here again, it becomes obvious that the differentiation between problem solving and decision making is often difficult and not applied strictly.


\largerpage
%\label{ref:ZOTEROITEMCSLCITATIONcitationIDJScloqRtpropertiesformattedCitationPrassl2010plainCitationPrassl2010citationItemsid225urishttpzoteroorgusers1255332itemsNQ5FRI5Purihttpzoteroorgusers1255332itemsNQ5FRI5PitemDataid225typearticlejournaltitleTranslatorsdecisionmakingprocessesinresearchandknowledgeintegrationcontainertitleNewapproachesintranslationprocessresearchpage5782authorfamilyPrasslgivenFriederikeissueddateparts2010schemahttpsgithubcomcitationstylelanguageschemarawmastercslcitationjsonRNDOiBuj32x1u}
\citegen{Prassl2010} categorisation of \isi{decision making processes} in translation supports the assumption of the automation of translation choices. While routinised and stereotype decisions are made unconsciously, and hence cannot be seen as problem solving processes, because there is no hurdle between source and target text for the translator, reflected and constructed decisions require a high amount of conscious thinking and cognitive effort. Therefore, these decision making processes can be categorised as problem solving activities. However, as Prassl mentions “guessing” as a possible solution strategy for constructed decisions, this would not qualify as a problem solving activity, because the hurdle between source and target text might still exist. Here again it becomes clear that a strict differentiation between problem solving and decision making is not available, neither in psychology nor in translation literature.


\largerpage
To extend on the points mentioned above, translation problems can often be categorised as \isi{ill-defined problems} rather than well-defined, because the finale state is, as mentioned above, unknown and the operators to arrive at the final state are often unknown. Further, the assessment of translations is especially difficult because only few characteristics can be judged as strictly correct or incorrect (e.~g. spelling or grammar) and personal opinions or judgements (preference of one translation option over another) are often necessary. Most of the problem solving studies in psychology (as mentioned in %\label{ref:ZOTEROITEMCSLCITATIONcitationIDYW2tDP77propertiesformattedCitationFunke2006bplainCitationFunke2006bcitationItemsid126urishttpzoteroorgusers1255332itemsZMSXPPGAurihttpzoteroorgusers1255332itemsZMSXPPGAitemDataid126typechaptertitleKomplexesProblemlsencontainertitleDenkenundProblemlsencollectiontitleEnzyklopdiederPsychologieThemenbereichCTheorieundForschungSer2KognitionpublisherHogrefepublisherplaceGttingenaopage373443eventplaceGttingenaoauthorfamilyFunkegivenJoachimeditorfamilyFunkegivenJoachimissueddateparts2006schemahttpsgithubcomcitationstylelanguageschemarawmastercslcitationjsonRND0mGc3YiSgC}
\citealt{Funke2006b} and %\label{ref:ZOTEROITEMCSLCITATIONcitationIDERSGR54qpropertiesformattedCitationPretzNaplesandSternberg2003plainCitationPretzNaplesandSternberg2003citationItemsid84urishttpzoteroorgusers1255332itemsU63243KXurihttpzoteroorgusers1255332itemsU63243KXitemDataid84typechaptertitleRecognizingdefiningandrepresentingproblemscontainertitleThepsychologyofproblemsolvingpublisherCambridgeScholarsPublishingpublisherplaceCambridgepage330volume30eventplaceCambridgeauthorfamilyPretzgivenJeanEfamilyNaplesgivenAdamJfamilySternberggivenRobertJissueddateparts2003schemahttpsgithubcomcitationstylelanguageschemarawmastercslcitationjsonRND4Mu3CFxG1S}
\citealt{PretzEtAl2003}) focused on well-defined problem solving, because they are much easier to control and evaluate. Accordingly, it is also complicated to study problem solving activities in translation because translation problems can usually not be characterised as well-defined (which will be discussed in detail in the following paragraphs). Individual differences influence the perception of what is problematic and solutions to problems can – as mentioned above – hardly be evaluated as either correct or incorrect. %\label{ref:ZOTEROITEMCSLCITATIONcitationID4leAwucVpropertiesformattedCitationPretzNaplesandSternberg2003plainCitationPretzNaplesandSternberg2003citationItemsid84urishttpzoteroorgusers1255332itemsU63243KXurihttpzoteroorgusers1255332itemsU63243KXitemDataid84typechaptertitleRecognizingdefiningandrepresentingproblemscontainertitleThepsychologyofproblemsolvingpublisherCambridgeScholarsPublishingpublisherplaceCambridgepage330volume30eventplaceCambridgeauthorfamilyPretzgivenJeanEfamilyNaplesgivenAdamJfamilySternberggivenRobertJissueddateparts2003schemahttpsgithubcomcitationstylelanguageschemarawmastercslcitationjsonRNDNaIVa82YmJ}
\citet[26]{PretzEtAl2003} conclude for problem solving in general that which also applies to problem solving in translation: Problem solving activities are influenced by the knowledge the problem solver gained in earlier experiences. Individual cognitive abilities, personality, and social background may explain why some people are better at solving problems than others.


\citet[129]{Risku1998} argues that all five criteria that \citet{Funke2006b} states for complex problem solving apply to the translation process as well. Her arguments for those criteria, which I do not consider suitable for translation (dynamism and non-transparency), are, however, not very convincing to me:


\begin{quote}
[…] the dynamism of the communication situations requires fast or “multi-compatible” decisions; the non-transparency of texts and situations requires further information gathering; conflicts between multiple aims have to be considered like domain expertise vs. comprehensibility [...]\footnote{original phrasing: “[...] die Eigendynamik der Kommunikationssituationen verlangt rasche bzw. 'mehrfach kompatible' Entscheidungen; die Intransparenz (Unbestimmtheit) der Texte und Situationen erfordert weitere Informationsbeschaffung; Konflikte zwischen verschiedenen Zielen wie Fachlichkeit und Verständlichkeit müssen abgewogen werden [...]"}
\end{quote}


In my opinion, %\label{ref:ZOTEROITEMCSLCITATIONcitationIDw9AYRsOJpropertiesformattedCitationFunke2006bplainCitationFunke2006bcitationItemsid126urishttpzoteroorgusers1255332itemsZMSXPPGAurihttpzoteroorgusers1255332itemsZMSXPPGAitemDataid126typechaptertitleKomplexesProblemlsencontainertitleDenkenundProblemlsencollectiontitleEnzyklopdiederPsychologieThemenbereichCTheorieundForschungSer2KognitionpublisherHogrefepublisherplaceGttingenaopage373443eventplaceGttingenaoauthorfamilyFunkegivenJoachimeditorfamilyFunkegivenJoachimissueddateparts2006schemahttpsgithubcomcitationstylelanguageschemarawmastercslcitationjsonRNDkLtzdc64Fq}
\citegen{Funke2006b} suggested five criteria for \isi{complex problem solving} (complexity, interconnectedness, dynamism, non-transparency, multiple aims) are not all met by translation as a problem solving activity, which is why I do not agree with %\label{ref:ZOTEROITEMCSLCITATIONcitationID9LjX7m4IpropertiesformattedCitationRisku1998plainCitationRisku1998citationItemsid81urishttpzoteroorgusers1255332items5AWQUSUMurihttpzoteroorgusers1255332items5AWQUSUMitemDataid81typebooktitleTranslatorischeKompetenzpublisherStauffenburgpublisherplaceTbingeneventplaceTbingenauthorfamilyRiskugivenHannaissueddateparts1998schemahttpsgithubcomcitationstylelanguageschemarawmastercslcitationjsonRNDni8RYpwsSA}
\citegen{Risku1998} assumption, shared by %\label{ref:ZOTEROITEMCSLCITATIONcitationIDb9ZcMIlOpropertiesformattedCitationKubiak2009plainCitationKubiak2009dontUpdatetruecitationItemsid102urishttpzoteroorgusers1255332itemsKWMUG2RKurihttpzoteroorgusers1255332itemsKWMUG2RKitemDataid102typethesistitleUbersetzeralsProblemloserEinequalitativeStudiezumProblemloseverhaltenvonsemiprofessionellenUbersetzernpublisherWydawnictwoNaukoweUAMissueddateparts2009schemahttpsgithubcomcitationstylelanguageschemarawmastercslcitationjsonRNDlsqHk9dO9S}
\citet{Kubiak2009}, that translation processes are complex problem solving situations. While many interconnected variables influence the translation process (e.~g. experience, time-pressure, text domain, translation skopos, etc. – hence the translation process is complex), translations are usually not very dynamic (translation jobs usually stay the same for the duration of the job, even if the source text might be changed by the client when the translation is already in progress) and quite transparent (if the skopos of the translation is defined and the whole source text is available). The last characteristic of complex problem solving (multiple aims) is hard to judge: On the one hand, a translation job often has only one aim\slash purpose, while the evaluation of a translation is usually still versatile and subjective, with no correct answer.



Further, %\label{ref:ZOTEROITEMCSLCITATIONcitationID5mu7Lz9tpropertiesformattedCitationWenkeandFrensch2003plainCitationWenkeandFrensch2003citationItemsid241urishttpzoteroorgusers1255332items2JBGKJ8Murihttpzoteroorgusers1255332items2JBGKJ8MitemDataid241typearticlejournaltitleIssuccessorfailureatsolvingcomplexproblemsrelatedtointellectualabilitycontainertitleThepsychologyofproblemsolvingpage87126authorfamilyWenkegivenDoritfamilyFrenschgivenPeterAissueddateparts2003schemahttpsgithubcomcitationstylelanguageschemarawmastercslcitationjsonRNDoR9kLUfEDx}
\citet{WenkeFrensch2003} differentiate between ill-defined (or “ill-stat\-ed”) problems and complex problem solving. Complex problems are ill-defined. However, they have additional features in Dörner's, Funke's, and Frensch's understanding of complex problem solving, like a dynamically changing problem situation or unknown exact properties of given state, final state, and hurdles. This leads to the conclusion that problems in translation situations are ill-defined, but not entirely complex (as categorised by %\label{ref:ZOTEROITEMCSLCITATIONcitationIDcmeSPGF0propertiesformattedCitationFunke2006bplainCitationFunke2006bcitationItemsid126urishttpzoteroorgusers1255332itemsZMSXPPGAurihttpzoteroorgusers1255332itemsZMSXPPGAitemDataid126typechaptertitleKomplexesProblemlsencontainertitleDenkenundProblemlsencollectiontitleEnzyklopdiederPsychologieThemenbereichCTheorieundForschungSer2KognitionpublisherHogrefepublisherplaceGttingenaopage373443eventplaceGttingenaoauthorfamilyFunkegivenJoachimeditorfamilyFunkegivenJoachimissueddateparts2006schemahttpsgithubcomcitationstylelanguageschemarawmastercslcitationjsonRNDi9sIEDBJWo}
\citealt{Funke2006b}), because usually neither the problem situation changes during the solving process, nor are the exact properties of the given state unknown.



According to %\label{ref:ZOTEROITEMCSLCITATIONcitationIDOPteIuZlpropertiesformattedCitationJonassen2000plainCitationJonassen2000citationItemsid243urishttpzoteroorgusers1255332items5E7DVHUGurihttpzoteroorgusers1255332items5E7DVHUGitemDataid243typearticlejournaltitleTowardadesigntheoryofproblemsolvingcontainertitleEducationaltechnologyresearchanddevelopmentpage6385volume48issue4authorfamilyJonassengivenDavidHissueddateparts2000schemahttpsgithubcomcitationstylelanguageschemarawmastercslcitationjsonRNDBdWh3qv1RJ}
\citegen{Jonassen2000} sorted list on well- and ill-structured problems,  translations might be characterised as either strategic performance problems – as it is applied to expert tasks in which a certain degree of professionalism and know-how is required to cope with the problem – or design problems – which would put the focus more on the creation of creative text with a vague solution outline and an unknown approach to the solution. Considering that translation entails characteristics of both groups, it can surely be classified as an ill-structured problem, because both problem groups are more often categorised as ill-structured than well-structured.\footnote{Problems that occur in PE might even be categorised as troubleshooting problems, because the dysfunctional MT output needs to be fixed by a professional, i.e. the translator.} However, as Jonassen only assessed and categorised a selection of possible problems, these categories might not be well suited for translation problems in general and another category might be appropriate.


\largerpage
As %\label{ref:ZOTEROITEMCSLCITATIONcitationIDS65aah38propertiesformattedCitationrtfDuc0u246rner2006plainCitationDrner2006citationItemsid136urishttpzoteroorgusers1255332itemsICSAD2I5urihttpzoteroorgusers1255332itemsICSAD2I5itemDataid136typechaptertitleSpracheundDenkencontainertitleDenkenundProblemlsenpublisherHogrefepublisherplaceGttingenaopage617643volumeEnzyklopdiederPsychologieThemenbereichCTheorieundForschungSer2KognitioneventplaceGttingenaoauthorfamilyDrnergivenDietricheditorfamilyFunkegivenJoachimissueddateparts2006schemahttpsgithubcomcitationstylelanguageschemarawmastercslcitationjsonRND8DyfZnridU}
\citet{Dorner2006} already pointed out, the trial-and-error strategy to solve a problem is for humans neither very efficient in everyday situations nor for translation in particular. If a translator is confronted with a hurdle between source and target text, (s)he has to plan how to solve the problem rather than trying to find a solution by accident. The \isi{seven-step-cycle} on problem solving by %\label{ref:ZOTEROITEMCSLCITATIONcitationIDlucUI7BApropertiesformattedCitationPretzNaplesandSternberg2003plainCitationPretzNaplesandSternberg2003citationItemsid84urishttpzoteroorgusers1255332itemsU63243KXurihttpzoteroorgusers1255332itemsU63243KXitemDataid84typechaptertitleRecognizingdefiningandrepresentingproblemscontainertitleThepsychologyofproblemsolvingpublisherCambridgeScholarsPublishingpublisherplaceCambridgepage330volume30eventplaceCambridgeauthorfamilyPretzgivenJeanEfamilyNaplesgivenAdamJfamilySternberggivenRobertJissueddateparts2003schemahttpsgithubcomcitationstylelanguageschemarawmastercslcitationjsonRNDApiF8Gvw2k}
\citet{PretzEtAl2003} can be easily transferred to problem solving processes in translation. For the translation purpose, it would be reasonable to swap step three (develop solution strategy) and step four (organise knowledge about problem), as is done in the following. Further, I will include “The problem gets solved” as the second-to-last step. Several similarities to %\label{ref:ZOTEROITEMCSLCITATIONcitationIDL8VSDBu7propertiesformattedCitationKrings1986aplainCitationKrings1986adontUpdatetruecitationItemsid104urishttpzoteroorgusers1255332items4PCIW24Nurihttpzoteroorgusers1255332items4PCIW24NitemDataid104typechaptertitleTranslationproblemsandtranslationstrategiesofadvancedGermanlearnersofFrenchL2containertitleInterlingualandinterculturalcommunicationpublisherGunterNarrVerlagpublisherplaceTbingenpage263276eventplaceTbingenauthorfamilyKringsgivenHansPeditorfamilyHousegivenJulianefamilyBlumKulkagivenShoshanaissueddateparts1986schemahttpsgithubcomcitationstylelanguageschemarawmastercslcitationjsonRNDrYaWJU8ecy}
\citet{Krings1986b} %Krings 1986a
model (cf. \figref{fig:key:5:1}) can be found, which will be briefly highlighted in italics\footnote{\citeapo{Wilss1994} six steps for problem solving, namely problem identification, problem clarification (description), information collection, considerations on how to proceed, the choice of a solution, and the evaluation of translation result, can be recognised in this list, too.}:


\begin{itemize}
\item A translator has to realise that (s)he has a problem with a \isi{translation unit} (recognise and identify problem); \textit{problem? → yes}
\item Then, (s)he has to figure out where exactly the problem in this \isi{translation unit} is, e.~g. lexical, syntactic, macro structural, etc. (define and represent problem mentally); \textit{identification of problem}
\item Have similar problems occurred in earlier translations? (organise knowledge about problem); not mentioned in Krings' model
\item Next, the translator has to decide what strategy can be applied to solve the problem, e.~g. look up words in a dictionary, read parallel texts, restructure the sentence\slash phrase (develop solution strategy); \textit{what type of problem? →} either \textit{comprehension problem} followed by \textit{comprehension strategies} or \textit{problem of rendering} followed by \textit{retrieval strategies}
\item The translator applies the strategy (allocate mental and physical resources); \textit{potential equivalent found?}
\item Now, (s)he has to evaluate whether the strategy will solve the problem (monitor progress); \textit{monitoring strategies → one adequate equivalent} or \textit{competing equivalents} or \textit{no adequate equivalent}
\item The translator solves the problem (not mentioned by Pretz et. al.); \textit{\isi{target language} text}
\item Finally, the translator has to evaluate whether the translation is suitable for the translation purpose, text type and domain as well as the style guide references – and further influences (evaluate solution); not mentioned in Krings' model
\end{itemize}

These steps do not necessarily occur consciously or might be skipped automatically. For example, if the translator does not know one single term in a domain-specific list, the translator may not have to define the problem (step 2) as it is obvious and s(he) knows the exact strategy to solve the problem (step 3), and may continue with step 4 – looking up the word in a dictionary, a job-related terminology list, or via the concordance search in a \isi{translation memory} system, etc. Further, in this simple translation problem, step 5 and 7 fuse, because while (s)he is consulting a dictionary, (s)he has to evaluate whether the suggestions in the dictionary fit the purpose etc. of the text. Meanwhile, the translator can evaluate whether the translation strategy actually solves the problem: If the dictionary does not contain an acceptable translation, another strategy might be useful, like consulting parallel texts or another dictionary. As %\label{ref:ZOTEROITEMCSLCITATIONcitationID7O8pSZzdpropertiesformattedCitationPretzNaplesandSternberg2003plainCitationPretzNaplesandSternberg2003citationItemsid84urishttpzoteroorgusers1255332itemsU63243KXurihttpzoteroorgusers1255332itemsU63243KXitemDataid84typechaptertitleRecognizingdefiningandrepresentingproblemscontainertitleThepsychologyofproblemsolvingpublisherCambridgeScholarsPublishingpublisherplaceCambridgepage330volume30eventplaceCambridgeauthorfamilyPretzgivenJeanEfamilyNaplesgivenAdamJfamilySternberggivenRobertJissueddateparts2003schemahttpsgithubcomcitationstylelanguageschemarawmastercslcitationjsonRNDvr3EpcuQbX}
\citet{PretzEtAl2003} have also pointed out, the solution of one translation problem might lead to another problem, e.g. the solution of a lexical problem might lead to a collocation problem in the already existing translation. The cycle property is not represented in Krings' model. %\label{ref:ZOTEROITEMCSLCITATIONcitationIDYq1od9IcpropertiesformattedCitationAngelone2010plainCitationAngelone2010citationItemsid158urishttpzoteroorgusers1255332itemsBD4NRZGMurihttpzoteroorgusers1255332itemsBD4NRZGMitemDataid158typechaptertitleUncertaintyuncertaintymanagementandmetacognitiveproblemsolvinginthetranslationtaskcontainertitleTranslationandcognitioncollectiontitleAmericanTranslatorsAssociationScholarlyMonographSeriescollectionnumber15publisherJohnBenjaminsPublishingCompanypublisherplaceAmsterdamPhiladelphiapage1740eventplaceAmsterdamPhiladelphiaauthorfamilyAngelonegivenErikeditorfamilyShrevegivenGregoryMfamilyAngelonegivenErikissueddateparts2010schemahttpsgithubcomcitationstylelanguageschemarawmastercslcitationjsonRND1eCBKRG5rB}
\citegen{Angelone2010} optimal problem solving bundle, containing problem recognition, solution proposal, and solution evaluation, can be found in the cycle as well. In \figref{fig:key:5:2}, we can see the combined models (\citealt{PretzEtAl2003} \& \citealt{Krings1986b}) for problem solving in one cycle.


\begin{figure}
\includegraphics[width=\textwidth]{figures/Figure_5_4_1.png}
\caption{Problem solving cycle for translation}
\label{fig:key:5:2}
\end{figure}

  

The \isi{problem solving cycle} in \figref{fig:key:5:2} applies both for TfS as well as for PE. However, recognising and identifying the problem are supposedly different in both tasks, because the problems that occur supposedly vary to a high degree (see \ref{sec:6}).



In %\label{ref:ZOTEROITEMCSLCITATIONcitationIDdy1iNaUcpropertiesformattedCitationWhittenandGraesser2003plainCitationWhittenandGraesser2003citationItemsid72urishttpzoteroorgusers1255332items7T7C4VIVurihttpzoteroorgusers1255332items7T7C4VIVitemDataid72typechaptertitleComprehensionoftextinproblemsolvingcontainertitleThepsychologyofproblemsolvingpublisherCambridgeUniversityPresspublisherplaceCambridgeaopage207229eventplaceCambridgeaoauthorfamilyWhittengivenShannonfamilyGraessergivenArthurCeditorfamilyDavidsongivenJanetEfamilySternberggivenRobertJissueddateparts2003schemahttpsgithubcomcitationstylelanguageschemarawmastercslcitationjsonRNDFzpORqPmCe}
\citet{WhittenGraesser2003} observations on the \isi{interconnectedness of language or reading comprehension and problem solving}, the translator is not only a problem solver during translation, but also enables problem solving for the recipient of his\slash her translation. Depending on the text type, the translator has to translate and the need to solve a problem might be a big motivation for the target text reader (most obviously in manuals, but contracts or advertisements might also be read to solve a problem). The translator has to ensure that the target text is understandable in the target culture, that the approaches to problem solution are represented in a way the target audience can understand it (see in this regard %\label{ref:ZOTEROITEMCSLCITATIONcitationIDx2edp4vtpropertiesformattedCitationBaker1996plainCitationBaker1996citationItemsid154urishttpzoteroorgusers1255332itemsQWAZ8BNWurihttpzoteroorgusers1255332itemsQWAZ8BNWitemDataid154typechaptertitleCorpusbasedtranslationstudiesThechallengesthatlieaheadcontainertitleTerminologyLSPandTranslationStudiesinLanguageEngineeringinHonourofJuanCSagerpublisherJohnBenjaminsPublishingCompanypublisherplaceAmsterdamPhiladelphiapage175186volume18eventplaceAmsterdamPhiladelphiaauthorfamilyBakergivenMonaeditorfamilySomersgivenHaroldissueddateparts1996schemahttpsgithubcomcitationstylelanguageschemarawmastercslcitationjsonRNDYoVQoi7y0C}
the observations of \citealt{Baker1996} on explicitation and simplification). When troubleshooting a broken electrical appliance, a British manual might suggest to check whether or not the wall socket is turned on. However, \ili{German} wall sockets cannot be turned on and off. Hence, this troubleshooting suggestion would be unnecessary or even confusing in a \ili{German} manual. In an even more general consideration, the translator solves the language hurdle that exists between the reader\slash recipient and the source text by delivering an understandable target text so that reading the text is only a task for the recipient, not a problem.



A few studies have been conducted in psychology that place cultural differences in the focus of \isi{problem solving behaviour} (an overview can be found in %\label{ref:ZOTEROITEMCSLCITATIONcitationIDT2cpXRsmpropertiesformattedCitationStrohschneider2006plainCitationStrohschneider2006citationItemsid74urishttpzoteroorgusers1255332items96AEPN3Jurihttpzoteroorgusers1255332items96AEPN3JitemDataid74typechaptertitleKulturelleUnterschiedebeimProblemlsencontainertitleDenkenundProblemlsenpublisherHogrefepublisherplaceGttingenaopage547615volumeEnzyklopdiederPsychologieThemenbereichCTheorieundForschungSer2KognitioneventplaceGttingenaoauthorfamilyStrohschneidergivenStefaneditorfamilyFunkegivenJoachimissueddateparts2006schemahttpsgithubcomcitationstylelanguageschemarawmastercslcitationjsonRNDfSxIT0NGxl}
\citealt{Strohschneider2006}). It is likely that cultural differences have an influence on \isi{problem solving behaviour} in translation, as well. This might be highly effected by the teaching methods at universities in the different regions. As the study at hand only includes \ili{German} natives (find details in \sectref{sec:7:2}) who translated from \ili{English} into \ili{German} and were most likely all educated in Germany (at least mainly), this factor on \isi{problem solving behaviour} is not taken into consideration in this study. However, cultural differences in \isi{problem solving behaviour} could be a promising research area for future studies in translation process studies.



Usually, solving a translation problem is a one- or multi-step operation (partial insight). However, full insight into a problem is not uncommon. Every translator has probably experienced dissatisfaction with a target text representation of part of the source text, but could not come up with something more suitable. However, a great solution for this problem came to mind much later in the translation, during a break or when even not translating at all (see also %\label{ref:ZOTEROITEMCSLCITATIONcitationIDVbNaqi8XpropertiesformattedCitationRisku1998plainCitationRisku1998citationItemsid81urishttpzoteroorgusers1255332items5AWQUSUMurihttpzoteroorgusers1255332items5AWQUSUMitemDataid81typebooktitleTranslatorischeKompetenzpublisherStauffenburgpublisherplaceTbingeneventplaceTbingenauthorfamilyRiskugivenHannaissueddateparts1998schemahttpsgithubcomcitationstylelanguageschemarawmastercslcitationjsonRNDBUAz1Mcr9b}
\citealt{Risku1998}: 204-205).



Translation processes consist of tasks and problems. How can these problems be identified? Studies have mainly used think aloud protocols to identify problem units in the translation from scratch task. This method, however, is very subjective and dependent on the participants' willingness to verbalise their thoughts.\footnote{The different methods will be introduced in \sectref{sec:7}.} However, is there a more neutral way or a more objective method to identifying translation problems? And do the same problems exist when the translators have to post-edit texts? The following chapter will formulate the research questions. Afterwards, I will focus on an empirical analysis of translation process data to tackle the research questions.


