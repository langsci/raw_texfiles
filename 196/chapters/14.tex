\chapter{Final remarks and future research}
\label{sec:14}

As most studies do, this study raises more questions than it answers. Can similar patterns be found in larger data sets for lexical and syntactic problem solving? Further, it needs to be tested if the same predictors are influential for the single \isi{PoS} classes in another and\slash or a larger data set. Also, it seems plausible that the results change and the predictors need to be adjusted for languages for special purposes. {What are the patterns in other languages? Are some predictors universal to a \isi{PoS} class independent of the language? The participants of this study had a rather negative attitude towards MT\slash PE and were dissatisfied with the PE\slash MPE tasks. However, does the negative attitude towards MT\slash PE influence the subconscious processes\slash the behaviour during the task? And is this attitude towards MT\slash PE changing at the moment? Are translators starting to accept that the occupational field for translators is changing and that recent developments concerning translation technologies can be seen as an opportunity rather than a threat?}



{As described in \sectref{sec:7:9}, the study design and execution had some flaws – some of them were inevitable, like the fact that the study was not conducted in the common working environment of the participants, and some of them could have been avoidable, like the suboptimal phrasing of some the questions in the questionnaire (see \sectref{sec:8:3}). These insights will be considered in the next study so that, on the one hand, mistakes are not repeated and, on the other hand, some issues will be weighted again, like ecological validity vs. feasibility.}



{The study at hand has shown a way to define translation problems, both theoretically and empirically. However, the analysis I conducted can only be considered a starting point for the interesting field of analysing and determining translation problems both in translation from scratch and in \isi{post-editing}. The results and approaches of this study need to be verified and expanded and\slash or falsified and improved. The same applies for the analysis considering \isi{translation competence}.}



{In conclusion, a lot of interesting research is still ahead of us and with the  developing electronic aids for translators and other new advances, the whole field keeps in motion. This study had set the course to identifying translation problems (in contrast to translation decisions and other related behaviour) in process data, but there might be considerations to improve this identification methods. Further, I still} {strongly believe that problem perception between professionals and semi-professionals varies a lot, but we have not found the right measurement to identify \isi{translation competence} yet. Mere education and experience do not seem to be enough to measure professionalism. Hence, the differences between the \isi{problem solving behaviour} of individual translator might be easier to detect if we find an improved way to mirror professionalism.}
