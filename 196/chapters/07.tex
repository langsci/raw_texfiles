\chapter{The data set}
\label{sec:7}

This chapter will introduce the data set which will be used to contrast problem solving in PE and TfS. In the first part of this chapter, the methods of translation process research will be outlined with a special focus on those used in the study. Further, the focus will be on the importance of \isi{data triangulation} and selecting participants. In the next section, the particular characteristics of the data set will be presented including the tools used to record the data, followed by a brief overview of other studies that previously used the data set to examine their research questions. Afterwards, the first general analyses will be presented, which will be useful to understand the study on a large scale. These analyses include the \isi{session's durations} (\sectref{sec:7:5}), \isi{complexity levels} of the texts (\sectref{sec:7:6}), general \isi{keystroke effort} in the different tasks (\sectref{sec:7:7}), and an \isi{error analysis} (\sectref{sec:7:8}). As the data were collected in a major attempt to gather comparable data in different languages, the process has some flaws which will be addressed in the final section (\sectref{sec:7:9}).


\section{A short introduction to methods in translation process research}
\label{sec:7:1}

{Different methods have been established to analyse \isi{translation processes} in the field (for an overview see} %\label{ref:ZOTEROITEMCSLCITATIONcitationIDc6lnbpdupropertiesformattedCitationrtfGuc0u246pferich2008plainCitationGpferich2008citationItemsid1487urishttpzoteroorggroups3587items8SGPKEZ2urihttpzoteroorggroups3587items8SGPKEZ2itemDataid1487typebooktitleTranslationsprozessforschungStandMethodenPerspektivencollectiontitleTranslationswissenschaftpublisherNarrpublisherplaceTbingenvolume4eventplaceTbingenISBN9783823364399languagegerauthorfamilyGpferichgivenSusanneissueddateparts2008schemahttpsgithubcomcitationstylelanguageschemarawmastercslcitationjsonRNDgYAR97UlL2}
{\citealt{Gopferich2008}}). Naturally, the analysis of the source text and the translation product is still very important for investigating the translation process. They are necessary to outline what might be happening in the translator's mind – without the source text, we do not know what structures we want to focus on when analysing the translation process; without the translation product, we cannot assess whether or not the translation was successful. To analyse the translation process, however, the following methods have been used most often: think-aloud protocols, questionnaires, keylogging, eyetracking, and neuroscientific methods like EEG or fMRI (find more methods and a categorisation of online and offline methods in %\label{ref:ZOTEROITEMCSLCITATIONcitationIDzq87qK6lpropertiesformattedCitationHansPeterKrings2005plainCitationHansPeterKrings2005citationItemsid245urishttpzoteroorgusers1255332itemsEEX7687Vurihttpzoteroorgusers1255332itemsEEX7687VitemDataid245typearticlejournaltitleWegeinsLabyrinthFragestellungenundMethodenderbersetzungsprozessforschungimberblickcontainertitleMetaJournaldestraducteursMetaTranslatorsJournalpage342358volume50issue2authorfamilyKringsgivenHansPeterissueddateparts2005schemahttpsgithubcomcitationstylelanguageschemarawmastercslcitationjsonRNDWA3unQao5n}
\citealt{Krings2005}). Three of these methods were selected for the study at hand, namely questionnaires, keylogging, and eyetracking methods, which will be explained in detail in the following. Think-aloud protocols and the neuroscientific methods EEG and fMRI will be outlined only briefly for the sake of completeness and because they are powerful tools, too. To reconstruct translation processes, it is necessary to combine different methods to benefit from the advantages of each method. This data \isi{triangulation} can help provide interested parties with a better idea of what is happening in the translator's black box.\footnote{This term refers to an electronic device, whose internal mechanisms and functionality is unknown to the user. In a broader, non-literal sense, it is applied to all systems, objects, or concepts, the inner working processes of which are unknown to the user (cf. \url{http://www.merriam-webster.com/dictionary/black box}, last accessed 21 November 2016).}


\subsection{Think-aloud protocols}
\label{sec:7:1:1}

{\isi{Think-aloud protocols} (TAPs) are used to record and analyse thoughts during the translation process. The translators are asked to verbalise their thoughts while translating. This can be done either directly during the translation or retrospectively; the first variant, however, is more common and has been conducted more often. The transcription of these verbalisations are called think-aloud protocols. Using TAPs has different advantages and disadvantages. Amongst others, one major disadvantage of immediate verbalisation is that studies have shown that verbalisation changes the thought process (cf.} %\label{ref:ZOTEROITEMCSLCITATIONcitationIDuyBaMGsApropertiesformattedCitationJakobsen2003plainCitationJakobsen2003citationItemsid16urishttpzoteroorgusers1255332itemsMFUTXCFUurihttpzoteroorgusers1255332itemsMFUTXCFUitemDataid16typechaptertitleEffectsofThinkAloudonTranslationSpeedRevisionandSegmentationcontainertitleTriangulatingTranslationPerspectivesinProcessOrientedResearchpublisherJohnBenjaminspublisherplaceAmsterdamPhiladelphiapage6995eventplaceAmsterdamPhiladelphiaauthorfamilyJakobsengivenArntLykkeeditorfamilyAlvesgivenFabioissueddateparts2003schemahttpsgithubcomcitationstylelanguageschemarawmastercslcitationjsonRNDLBAjooRl66}
{\citealt{Jakobsen2003}}){ and therefore may also change the translation process. TAPs that are produced retrospectively do not change the translation process, because the translator is asked about specific translation units only after the whole translation was produced. The problem is, however, that a lot of valuable thoughts might get lost between translation production and verbalisation. Here, it is helpful to use a screen-recording of the session – maybe with eyetracking data – to help the participant remember the passages of interest. Further, only thoughts can be uttered that are conscious, and a high \isi{cognitive load} during the task might prevent the participants from verbalising their thoughts. However, it is one of the few methods that can actually reproduce what is happening inside the heads of the participants, even if not completely} %\label{ref:ZOTEROITEMCSLCITATIONcitationIDygGNKUKPpropertiesformattedCitationrtfJuc0u228uc0u228skeluc0u228inen2010plainCitationJskelinen2010citationItemsid115urishttpzoteroorgusers1255332itemsKSHNFQ36urihttpzoteroorgusers1255332itemsKSHNFQ36itemDataid115typechaptertitleThinkaloudprotocolcontainertitleHandbookoftranslationstudiespublisherJohnBenjaminsPublishingCompanypublisherplaceAmsterdamPhiladelphiapage371374volume1eventplaceAmsterdamPhiladelphiaauthorfamilyJskelinengivenRiittaeditorfamilyGambiergivenYvesfamilyDoorslaergivenLucnondroppingparticlevanissueddateparts2010schemahttpsgithubcomcitationstylelanguageschemarawmastercslcitationjsonRNDae4oxdiyld}
{\citep{Jaaskelainen2010}.}{ TAPs have their origins in cognitive psychology and are, despite criticism, still a standard research method in research on problem solving in psychology (cf.} %\label{ref:ZOTEROITEMCSLCITATIONcitationIDVEPDNcEvpropertiesformattedCitationrtfKnoblichanduc0u214llinger2006plainCitationKnoblichandllinger2006citationItemsid240urishttpzoteroorgusers1255332items7ARRJ75Surihttpzoteroorgusers1255332items7ARRJ75SitemDataid240typechaptertitleEinsichtundUmstrukturierungbeimProblemlsencontainertitleDenkenundProblemlsencollectiontitleEnzyklopdiederPsychologieThemenbereichCTheorieundForschungSer2KognitionpublisherHogrefepage183authorfamilyKnoblichgivenGntherfamilyllingergivenMichaelissueddateparts2006schemahttpsgithubcomcitationstylelanguageschemarawmastercslcitationjsonRNDFByCmsewjj}
\citealt[5]{KnoblichÖllinger2006}). J. B. Watson was the first to use think-aloud protocols in psychology as early as 1920 %\label{ref:ZOTEROITEMCSLCITATIONcitationIDmZEZheptpropertiesformattedCitationEricsson2003plainCitationEricsson2003citationItemsid134urishttpzoteroorgusers1255332itemsTWGZGJG8urihttpzoteroorgusers1255332itemsTWGZGJG8itemDataid134typechaptertitleTheacquisitionofexpertperformanceasproblemsolvingcontainertitleThepsychologyofproblemsolvingpublisherCambridgeUniversityPresspublisherplaceCambridgeaopage3183eventplaceCambridgeaoauthorfamilyEricssongivenAeditorfamilyDavidsongivenJanetEfamilySternberggivenRobertJissueddateparts2003schemahttpsgithubcomcitationstylelanguageschemarawmastercslcitationjsonRNDYHmiheEFyz}
{(cf. \citealt[37]{EricssonSternberg2003}).}{ Nonetheless, the first TAP studies in translation science were reported only in the 1980s (e.g.} %\label{ref:ZOTEROITEMCSLCITATIONcitationID1xtyoUcYpropertiesformattedCitationKrings1986plainCitationKrings1986dontUpdatetruecitationItemsid1709urishttpzoteroorggroups3587itemsX92T2ZTGurihttpzoteroorggroups3587itemsX92T2ZTGitemDataid1709typebooktitleWasindenKpfenvonbersetzernvorgehtEineempirischeUntersuchungzurStrukturdesbersetzungsprozessesanfortgeschrittenenFranzsischlernernpublisherGunterNarrVerlagpublisherplaceTbingeneventplaceTbingennoteZuglBochumUnivDiss198586authorfamilyKringsgivenHansPeterissueddateparts1986schemahttpsgithubcomcitationstylelanguageschemarawmastercslcitationjsonRNDEJTFaO5uAb}
{\citealt{Krings1986}}%Krings 1986a
{)}.



{The disadvantages outweighed the advantages for the analysis in this dissertation. Although think-aloud data might have been very helpful to analyse problem solving in translation, the experiment was not designed to focus on one research question} {only, but was supposed to collect a basis for various research questions.}\footnote{{Details on the experiments that have been conducted with the data set can be found in \sectref{sec:7:2}.}}{ As the} {experiment setting was also replicated for other language combinations, retrospective think-aloud data were not gathered either. Further, the aim of the study is to uncover not only conscious but also unconscious translation problems. Therefore, the data collected should be as close to the natural translation process as possible. Another aspect is that TAPs for six texts would have been cognitively very demanding for the participants and would have produced a huge amount of data that would need transcription.} Finally, TAPs have been neglected more and more in recent years in translation studies in general because they affect the translation flow too much.

\largerpage
\subsection{\rmfamily  Questionnaires}
\label{sec:7:1:2}

\isi{Questionnaires} are usually distributed in a written form – either on paper or electronically – and contain a set of questions. These questions can be open – the participant decides what to answer and in which detail – or closed – the participant can choose from a set of answers. \isi{Mixed questions} contain a number of possible answers. The participant, however, has the possibility to add his/her own answers. \isi{Closed questions} are much easier to assess which saves time and money. Open questions, however, deliver more extensive and more differentiated data. %\label{ref:ZOTEROITEMCSLCITATIONcitationIDcueUELAIpropertiesformattedCitationrtfKluc0u246cknerandFriedrichs2014plainCitationKlcknerandFriedrichs2014citationItemsid106urishttpzoteroorgusers1255332items9FHPD83Surihttpzoteroorgusers1255332items9FHPD83SitemDataid106typechaptertitleGesamtgestaltungdesFragebogenscontainertitleHandbuchMethodenderempirischenSozialforschungpublisherSpringerpublisherplaceWiesbadenpage675685eventplaceWiesbadenauthorfamilyKlcknergivenJenniferfamilyFriedrichsgivenJrgeneditorfamilyBaurgivenNinafamilyBlasiusgivenJrgissueddateparts2014schemahttpsgithubcomcitationstylelanguageschemarawmastercslcitationjsonRNDylhnSBCAxU}
(\citealt{KlocknerFriedrichs2014}) Compared to interviews, questionnaires are easier to distribute and participants might be more willing to fill out a questionnaire whenever they have time than to schedule a date with the interviewer. Further, questionnaires are more discrete and more anonymous. On the other hand, the participants must be able to read and write, which excludes a few potential target groups (but this should usually not be a problem in translation process studies). Written answers will probably be shorter and less detailed than answers in an interview. Finally, a questionnaire does not allow for personal contact so that participants are not able to ask questions and the examiner cannot get an impression of the person.\footnote{This is not necessarily a problem. Like in the study at hand, short questionnaires might be handed out and completed in the presence of the examiner, because they are only a part of the study and not the study itself.} (cf. %\label{ref:ZOTEROITEMCSLCITATIONcitationIDuQWELo4UpropertiesformattedCitationrtfDuc0u246ringandBortz2014plainCitationDringandBortz2014citationItemsid220urishttpzoteroorgusers1255332itemsKJHAHZW5urihttpzoteroorgusers1255332itemsKJHAHZW5itemDataid220typebooktitleForschungsmethodenundEvaluationpublisherSpringerpublisherplaceBerlinHeidelbergedition5eventplaceBerlinHeidelbergauthorfamilyDringgivenNicolafamilyBortzgivenJrgenissueddateparts2014schemahttpsgithubcomcitationstylelanguageschemarawmastercslcitationjsonRNDDwT3UL56Wk}
\citealt[398--399]{DoringBortz2014})



According to the required information, five \isi{types of questions} are differentiated in social sciences (cf. %\label{ref:ZOTEROITEMCSLCITATIONcitationIDVDsdhWLJpropertiesformattedCitationReinecke2014plainCitationReinecke2014citationItemsid83urishttpzoteroorgusers1255332items4BJ5VSI6urihttpzoteroorgusers1255332items4BJ5VSI6itemDataid83typechaptertitleGrundlagenderstandardisiertenBefragungcontainertitleHandbuchMethodenderempirischenSozialforschungpublisherSpringerpublisherplaceBerlinpage601617eventplaceBerlinauthorfamilyReineckegivenJosteditorfamilyBaurgivenNinafamilyBlasiusgivenJrgissueddateparts2014schemahttpsgithubcomcitationstylelanguageschemarawmastercslcitationjsonRNDScnMF5AJHQ}
\citealt[604--608]{ReineckeBlasius2014}), namely questions that ask for


\begin{itemize}
\item opinions – usually the participant has to agree or disagree with a statement according to a Likert scale, buttons, or a visual analogue scale, which usually has either five or seven response options between “I strongly agree” and “I strongly disagree”.
\item facts and knowledge – multiple choice questions with (usually) one correct answer, e.g. when asking for the participant's knowledge of history.
\item incidents, attentions, and behaviour – relate to behaviour of the participant in the past; often with yes-/no-questions, open questions or bipolar scales (“very unlikely” vs. “very likely”).
\item social-statistical characteristics – questions about age, gender, education, marital status, income, etc. 
\item and network questions – concerning the social behaviour of the participant.
\end{itemize}

The requirements for a high quality questionnaire include that the participant is able to answer the questions, and that it is objective, valid, and reliable. However, even a very sophisticated questionnaire can cause non-responses and different response qualities due to the individual characteristics of the participants. Furthermore, participants tend to follow certain response strategies, e.g. they prefer extreme categories or medium categories, they prefer the first or the last response choice (also called primacy or recency effect), or they answer according to social conventions (cf. ibid.: 612-613).



In general, a question might be hard to understand for the participant on a semantic and pragmatic level. Semantic difficulties might arise if the questions contain unknown terms, terms that are ambiguous, terms that can be interpreted differently by individuals or by groups, or if the questions are worded vaguely or are too difficult. Pragmatic problems arise when the question does not reveal what the researcher actually wants to know. Hence, terms should be used in the questions that are simple, distinct, and can only be interpreted in one way. If some of the used terms might be unknown to some participants, they have to be explained. Further, long and complex, or hypothetical questions as well as two stimuli or double negatives should be avoided. Researchers should avoid using imputing or suggestive questions. In general, the participant should be able to identify the required information. If necessary, it should be clear to which time period the questions refer. If the participants can choose from responses, these need to be disjointed and exhaustive. Finally, it is necessary that the context of the question does not influence the answer to the question. (cf. %\label{ref:ZOTEROITEMCSLCITATIONcitationIDMR3G5sWypropertiesformattedCitationPorst2014plainCitationPorst2014citationItemsid85urishttpzoteroorgusers1255332itemsR3FXF6DQurihttpzoteroorgusers1255332itemsR3FXF6DQitemDataid85typechaptertitleFrageformulierungcontainertitleHandbuchMethodenderempirischenSozialforschungpublisherSpringerpublisherplaceWiesbadenpage687699eventplaceWiesbadenauthorfamilyPorstgivenRolfeditorfamilyBaurgivenNinafamilyBlasiusgivenJrgissueddateparts2014schemahttpsgithubcomcitationstylelanguageschemarawmastercslcitationjsonRND271Vr9cefD}
\citealt{Porst2014}: 688-697) Similarly, the possible responses need to be sophisticated as well, if the questions are not open. For easy numeric questions, it might be reasonable to choose open answers rather than categories, because they are easy to assess even as open questions and the participants are not biased and have to think harder about their answer. The number of possible answers – especially on scales – can differ, too. More response categories give the participant more options and allow more nuances. However, they also increase the cognitive effort required to answer the question. Recent studies confirmed the rule of thumb that a choice of between five and nine responses is legitimate. Fewer response options are paradoxically less reliable and more often do not bring any advantages. Additionally, researchers have to consider the educational background of the participant group and the mode in which the questions are presented. Another aspect is whether to present an even or uneven number of responses. The advantage (and disadvantage) of even numbers is that the participant has to decide on a tendency. Generally, uneven scales are recommended. Sometimes it might be sensible to include the category “I don't know”. It might not be necessary to label all possibilities of the scale, e.g. on a seven-item scale the only labels could be “I strongly disagree”, “I do not agree nor disagree”, and “I strongly agree”. However, it is better to formulate all possibilities. Finally, the responses might start with the positive answers first and change to starting with the negative answers after a couple of questions to avoid response patterns. (cf. %\label{ref:ZOTEROITEMCSLCITATIONcitationIDveeqCLjRpropertiesformattedCitationFranzen2014plainCitationFranzen2014noteIndex0citationItemsid129urishttpzoteroorgusers1255332itemsUVPXC2M3urihttpzoteroorgusers1255332itemsUVPXC2M3itemDataid129typechaptertitleAntwortskaleninstandardisiertenBefragungencontainertitleHandbuchMethodenderempirischenSozialforschungpublisherSpringerpublisherplaceWiesbadenpage701711eventplaceWiesbadenauthorfamilyFranzengivenAxeleditorfamilyBaurgivenNinafamilyBlasiusgivenJrgissueddateparts2014schemahttpsgithubcomcitationstylelanguageschemarawmastercslcitationjsonRNDa8ORmxpPSo}
\citealt{Franzen2014}: 703-709)



Depending on length and topic of the questionnaire, the order of the questions can be of utmost importance. The questionnaire should start with easy questions to motivate the participants and end with easy questions so they do not give up at the end. The questionnaire should not start with social-statistical questions, which are essential to most studies, because they might bore the participant and make him\slash her quit before any relevant questions have been answered. The most important questions should be placed in the middle of the questionnaire so that they are answered even if the participants do not make it to the end of the questionnaire. Very sensitive and awkward questions should be asked towards the end, because in an early stage they might cause the participant to quit the questionnaire and might influence the response behaviour for the following questions. Finally, questions should be ordered in content blocks to avoid confusing the participants. (cf. %\label{ref:ZOTEROITEMCSLCITATIONcitationIDMkJjCB8qpropertiesformattedCitationrtfKluc0u246cknerandFriedrichs2014plainCitationKlcknerandFriedrichs2014citationItemsid106urishttpzoteroorgusers1255332items9FHPD83Surihttpzoteroorgusers1255332items9FHPD83SitemDataid106typechaptertitleGesamtgestaltungdesFragebogenscontainertitleHandbuchMethodenderempirischenSozialforschungpublisherSpringerpublisherplaceWiesbadenpage675685eventplaceWiesbadenauthorfamilyKlcknergivenJenniferfamilyFriedrichsgivenJrgeneditorfamilyBaurgivenNinafamilyBlasiusgivenJrgissueddateparts2014schemahttpsgithubcomcitationstylelanguageschemarawmastercslcitationjsonRNDxsr7zMRQsG}
\citealt{KlocknerFriedrichs2014})


\largerpage
For the study at hand, two short questionnaires were designed. One was distributed before the actual tasks, retrieving meta data from the participants and asking questions about experiences and first opinions, while the retrospective questionnaire dealt mainly with assessing personal performance and assessing the MT output. A detailed analysis of the questionnaires can be found in \chapref{sec:8}.


\subsection{Keylogging}
\label{sec:7:1:3}

\isi{Keylogging} software allows the researcher to analyse the text production process and the associated mental processes. All key (and mouse) activities are recorded during text production, including \isi{typing processes}, special key combinations and deleting activities. Further, \isi{pauses} are recorded, which indicate reading processes and the segmentation of the text, which is done subconsciously by the participant and might highlight text passages that require high cognitive effort. {However, one has to keep in mind that the data only allows us to speculate on mental processes of participants engaged in text production. Other methods, especially eyetracking, can help to interpret the participants' behaviour, e.g. during pauses (cf.} %\label{ref:ZOTEROITEMCSLCITATIONcitationID6yz29RUOpropertiesformattedCitationJakobsen2011plainCitationJakobsen2011citationItemsid111urishttpzoteroorgusers1255332itemsXXFWDVVWurihttpzoteroorgusers1255332itemsXXFWDVVWitemDataid111typechaptertitleTrackingtranslatorskeystrokesandeyemovementswithTranslogcontainertitleMethodsandStrategiesofProcessResearchpublisherJohnBenjaminsPublishingCompanypublisherplaceAmsterdamPhiladelphiapage3755eventplaceAmsterdamPhiladelphiaauthorfamilyJakobsengivenArntLykkeeditorfamilyAlvstadgivenCeciliafamilyHildgivenAdelinafamilyTiseliusgivenElisabetissueddateparts2011schemahttpsgithubcomcitationstylelanguageschemarawmastercslcitationjsonRND8LyWAJT6ad}
{\citealt{Jakobsen2011}}%\label{ref:ZOTEROITEMCSLCITATIONcitationID9kUvOP8mpropertiesformattedCitationJakobsen2011plainCitationJakobsen2011citationItemsid111urishttpzoteroorgusers1255332itemsXXFWDVVWurihttpzoteroorgusers1255332itemsXXFWDVVWitemDataid111typechaptertitleTrackingtranslatorskeystrokesandeyemovementswithTranslogcontainertitleMethodsandStrategiesofProcessResearchpublisherJohnBenjaminsPublishingCompanypublisherplaceAmsterdamPhiladelphiapage3755eventplaceAmsterdamPhiladelphiaauthorfamilyJakobsengivenArntLykkeeditorfamilyAlvstadgivenCeciliafamilyHildgivenAdelinafamilyTiseliusgivenElisabetissueddateparts2011schemahttpsgithubcomcitationstylelanguageschemarawmastercslcitationjsonRNDjLxduriIzZ}
{: 37-38).} Recording keylogging data with a keylogger is unobtrusive, because the programme runs in the background and, hence, the recording process is not noticeable (cf. %\label{ref:ZOTEROITEMCSLCITATIONcitationIDAUU9RakWpropertiesformattedCitationCarl2012plainCitationCarl2012citationItemsid196urishttpzoteroorgusers1255332itemsR5RP4KN4urihttpzoteroorgusers1255332itemsR5RP4KN4itemDataid196typepaperconferencetitleTranslogIIaProgramforRecordingUserActivityDataforEmpiricalReadingandWritingResearchcontainertitleLRECpage41084112authorfamilyCarlgivenMichaelissueddateparts2012schemahttpsgithubcomcitationstylelanguageschemarawmastercslcitationjsonRND5hIphBTtc7}
\citealt{Carl2012}: 4108). The keylogging recordings, however, only provide information about filled times, i.e. when a key was pressed, and about unfilled times, i.e. when no key was pressed. Therefore, the experimenter has to interpret what this filled and unfilled times mean in terms of the cognitive writing process (cf. %\label{ref:ZOTEROITEMCSLCITATIONcitationIDwZqOZolkpropertiesformattedCitationBaaijenGalbraithanddeGlopper2012plainCitationBaaijenGalbraithanddeGlopper2012citationItemsid156urishttpzoteroorgusers1255332items2GDB93KDurihttpzoteroorgusers1255332items2GDB93KDitemDataid156typearticlejournaltitleKeystrokeanalysisReflectionsonproceduresandmeasurescontainertitleWrittenCommunicationpage246277volume29issue3authorfamilyBaaijengivenVeerleMfamilyGalbraithgivenDavidfamilyGloppergivenKeesnondroppingparticledeissueddateparts2012schemahttpsgithubcomcitationstylelanguageschemarawmastercslcitationjsonRNDU6YFm26PVh}
\citet[246--247]{BaaijenEtAl2012}).



Keystroke logging is not only used to record translation processes but all kinds of \isi{writing processes}, such as observing cognitive processes while writing in general, writing strategies in professional and creative writing, writing progress in children, first and second language acquisition, studies of writing difficulties or professionalism, as well as in educational environments. The main focus is often on pauses and revisions in these studies, because they are considered a clear indicator of cognitive effort and of discrepancies respectively, which also indicate problems in the writing process. %\label{ref:ZOTEROITEMCSLCITATIONcitationIDr61eMRBIpropertiesformattedCitationLeijtenandVanWaes2013plainCitationLeijtenandVanWaes2013citationItemsid195urishttpzoteroorgusers1255332items2NKTNR86urihttpzoteroorgusers1255332items2NKTNR86itemDataid195typearticlejournaltitleKeystrokelogginginwritingresearchusingInputlogtoanalyzeandvisualizewritingprocessescontainertitleWrittenCommunicationpage358392volume30issue3authorfamilyLeijtengivenMarillefamilyVanWaesgivenLuukissueddateparts2013schemahttpsgithubcomcitationstylelanguageschemarawmastercslcitationjsonRNDlPaqbNv0eR}
(cf. \citealt[360--361]{LeijtenVanWaes2013}) Writing process observations can further be divided into direct and indirect as well as synchronous and asynchronous methods. Keystroke logging can be characterised as indirect and synchronous, while think-aloud protocols would be considered indirect and synchronous and retrospective protocols indirect and asynchronous. An example for indirect and asynchronous methods would be the analysis of the produced text. (cf. ibid.: 361)



{While the participants usually are informed about the recording process in scientific studies, \isi{keylogging software} is also used to record keystrokes of computer and Internet users secretly, e.g. employers might use \isi{keylogging software} to supervise whether the employees only use the computer\slash the Internet for job-related task, or \isi{keylogging software} might be used by attackers in malware to steal passwords for e-mail or e-commerce accounts, etc. Although security software can usually cope with hidden keyloggers, it is very hard to detect this kind of malware, because it sends information via software which is very similar to the software of e-mail services and, hence, becomes almost undetectable. (cf.} %\label{ref:ZOTEROITEMCSLCITATIONcitationID0iqaxWDmpropertiesformattedCitationDavarpanahJazietal2014plainCitationDavarpanahJazietal2014citationItemsid197urishttpzoteroorgusers1255332itemsVA4HZTAEurihttpzoteroorgusers1255332itemsVA4HZTAEitemDataid197typearticlejournaltitleAnIntroductiontoUndetectableKeyloggerswithExperimentalTestingcontainertitleInternationalJournalofComputerCommunicationsandNetworksIJCCNpage15volume4issue3authorfamilyDavarpanahJazigivenMohammadfamilyCiobotarugivenAnaMariafamilyBaratigivenElahehfamilyDadkhahgivenMehdiissueddateparts2014schemahttpsgithubcomcitationstylelanguageschemarawmastercslcitationjsonRNDoobNnZMU9v}
{\citealt{Davarpanah_JaziEtAl2014})}



{Different keylogging parameters will be used to assess the cognitive effort and to predict problem solving activity in the study at hand. The individual parameters will be introduced when they first become relevant in Chapters \ref{sec:9} to \ref{sec:12}}


\subsection{{ Eyetracking}}
\label{sec:7:1:4}

{The term} {\isi{\textit{eyetracking}}}{ refers to the methodology with which human \isi{eye movement} can be recorded and assessed. The light enters the eye through the} {\isi{\textit{pupil}}}{ in human vision, which is smaller when it is bright and bigger when there is only little light.}\footnote{The pupil itself, however, does not get larger or smaller as it is a hole in the Iris that lets light pass through to the retina. The muscles \isi{\textit{sphincter pupillae}} and \isi{\textit{dilator pupillae}} are responsible for the dilation and the diminution of the pupil. (cf. %\label{ref:ZOTEROITEMCSLCITATIONcitationIDaZImWMP8propertiesformattedCitationSnellandLemp2013plainCitationSnellandLemp2013citationItemsid75urishttpzoteroorgusers1255332items8BISFH8Curihttpzoteroorgusers1255332items8BISFH8CitemDataid75typebooktitleClinicalanatomyoftheeyepublisherBlackwellSciencepublisherplaceOxfordeventplaceOxfordauthorfamilySnellgivenRichardSfamilyLempgivenMichaelAissueddateparts2013schemahttpsgithubcomcitationstylelanguageschemarawmastercslcitationjsonRNDuOWPCIw3v7}
\citealt{SnellLemp2013}: n. p.)}{ The images are turned upside down and projected onto the} {\isi{\textit{retina}}}{ – the back of the eyeball. The retina consists of light-sensitive cells, called} {\isi{\textit{cones}}}{ and} {\isi{\textit{rods}}}{. Cones are responsible for colour vision, while rods are very sensitive to light and enable vision even in dim environments. The} {\isi{\textit{fovea}}}{ is a very small spot – about two percent of our visual field – at the back of our eyeball where cones are excessively over-presented. Hence, this is the spot where we can see the most clearly. Accordingly, we have to move our eyes so that the information we want to concentrate on is projected onto the fovea. The information gathered via the fovea is prioritised during processing due to a magnification factor. When recording eye movements, the pupil and the} {\isi{\textit{cornea}}}{ are very important. The cornea covers most of the eye and reflects light. Light is reflected by the cornea and the lens as well, but the reflection of the cornea is the brightest reflection. Three pairs of} {\textit{muscles}}{ regulate human \isi{eye movement}, which conduct the three-dimensional coordination of the eyeballs in the human head. The most important measurement in eyetracking does not report \isi{eye movement}, but} {rather at which point the eye lingers and focuses. This point is called} {\isi{\textit{fixation}}}{ and can last from a few milliseconds to a couple of seconds (}%\label{ref:ZOTEROITEMCSLCITATIONcitationIDt7vkkxl3propertiesformattedCitationHolmqvistetal2011plainCitationHolmqvistetal2011citationItemsid191urishttpzoteroorgusers1255332itemsJ5E7FB37urihttpzoteroorgusers1255332itemsJ5E7FB37itemDataid191typebooktitleEyetrackingacomprehensiveguidetomethodsandmeasurespublisherOxfordUniversityPresspublisherplaceOxfordNewYorknumberofpages537sourceLibraryofCongressISBNeventplaceOxfordNewYorkISBN9780199697083callnumberQA769H85E9732011noteOCLCocn741340045shortTitleEyetrackingauthorfamilyHolmqvistgivenKennethfamilyNystrmgivenMarcusfamilyAnderssongivenRichardfamilyDewhurstgivenRichardfamilyJarodzkagivenHalszkafamilyVandeWeijergivenJoostissueddateparts2011schemahttpsgithubcomcitationstylelanguageschemarawmastercslcitationjsonRNDdh7AdocE1q}
{\citealt{HolmqvistEtAl2011}}:{ 21-23).}{ The basic assumption is that the human pays attention to the point (s)he fixates on (a concept also known as the \isi{eye-mind assumption} introduced in a reading study by} %\label{ref:ZOTEROITEMCSLCITATIONcitationID9VPo205TpropertiesformattedCitationJustandCarpenter1980plainCitationJustandCarpenter1980citationItemsid192urishttpzoteroorgusers1255332itemsC436SB27urihttpzoteroorgusers1255332itemsC436SB27itemDataid192typearticlejournaltitleAtheoryofreadingfromeyefixationstocomprehensioncontainertitlePsychologicalreviewpage329volume87issue4authorfamilyJustgivenMarcelAfamilyCarpentergivenPatriciaAissueddateparts1980schemahttpsgithubcomcitationstylelanguageschemarawmastercslcitationjsonRNDf6eEd6sxtn}
\citet{JustCarpenter1980}),{ although this is not always the case. The eye, however, is not completely still when it fixates a point. There are also} {\isi{\textit{tremors}}, whose exact role is unknown, {\isi{\textit{drifts}}}{, which move the eye away form the fixated points, and} {\isi{\textit{microsaccades}}}{, which bring the eyes back to the fixated point. The eye's movement from one fixation to the next fixation is called} {\isi{\textit{saccade}}}{. Saccades are the fastest movements the body can make and they last only 30 to 80 milliseconds. Further, it is assumed that the human is blind during this movement. The eye does not always take the shortest way to the next fixation and does not always hit the correct position and hence has to reposition itself before fixating. This small repositioning movement is often called} {\isi{\textit{glissade}}}{. If our eyes follow a moving object, it makes a completely different movement, which is also controlled by a different part of the brain, the so-called} {\isi{\textit{smooth pursuit}}}{. “Smooth pursuit requires something to follow, while saccades can be made on a white wall or even in the dark, with no stimuli at all.” (}%\label{ref:ZOTEROITEMCSLCITATIONcitationIDvkOqwDX1propertiesformattedCitationHolmqvistetal2011plainCitationHolmqvistetal2011citationItemsid191urishttpzoteroorgusers1255332itemsJ5E7FB37urihttpzoteroorgusers1255332itemsJ5E7FB37itemDataid191typebooktitleEyetrackingacomprehensiveguidetomethodsandmeasurespublisherOxfordUniversityPresspublisherplaceOxfordNewYorknumberofpages537sourceLibraryofCongressISBNeventplaceOxfordNewYorkISBN9780199697083callnumberQA769H85E9732011noteOCLCocn741340045shortTitleEyetrackingauthorfamilyHolmqvistgivenKennethfamilyNystrmgivenMarcusfamilyAnderssongivenRichardfamilyDewhurstgivenRichardfamilyJarodzkagivenHalszkafamilyVandeWeijergivenJoostissueddateparts2011schemahttpsgithubcomcitationstylelanguageschemarawmastercslcitationjsonRNDuV1tPL38QT}
{\citealt{HolmqvistEtAl2011}: 23)} Eye movements are typically measured in \textit{visual degrees} or \textit{minutes} instead of mm on the screen. While the eyes often move in relation to each other, some eye movements also work in the opposite direction (\isi{\textit{vergence} eye movement}). Furthermore, most people have two equally weighted eyes, but there are also many people who have a dominant and a more passive eye. (cf. ibid.: 21-24)



{\isi{Eyetracking} is a very promising method for studying translation processes, because it records where and how long the eyes are fixating and accordingly what the participant is concentrating on – valuable insights into the process that cannot be recorded by other methods. The \isi{eye movement} data give us promising and important hints about what is going on in the translators mind, although we can only interpret the data and cannot be entirely sure what is happening in the translators' black box. This research method, however, brings not only advantages, but also challenges to translation process research. The kind of eyetracking system, for example, has to be considered. A remote eyetracker is considered ecologically more valid, because the participant can move comparably freely in front of the computer screen, while head-mounted eyetrackers and eyetrackers with chin or head rests produce more accurate data. \isi{Eyetracking} glasses, on the other hand, liberate the participants from the screen.} {In contrast to questionnaires, think-aloud protocols, and keylogging technology, which cause only very low costs if the experimenter has a PC or laptop, a recording device and\slash or Internet available (if necessary at all), eyetracking requires expensive hard- and software. Further, the eyetracking experiments must be conducted in a comparable environment, e.g. similar lighting conditions (cf.} %\label{ref:ZOTEROITEMCSLCITATIONcitationIDpMn8KNjLpropertiesformattedCitationrtfOuc0u8217Brien2009plainCitationOBrien2009citationItemsid3041urishttpzoteroorgusers1255332items9WCPJ74Xurihttpzoteroorgusers1255332items9WCPJ74XitemDataid3041typearticlejournaltitleEyetrackingintranslationprocessresearchmethodologicalchallengesandsolutionscontainertitleMethodologyTechnologyandInnovationinTranslationProcessResearchpage251266volume38authorfamilyOBriengivenSharonissueddateparts2009schemahttpsgithubcomcitationstylelanguageschemarawmastercslcitationjsonRNDO79Ytd3vnq}
{\citealt[251--254]{OBrien2009};}{ on detailed information on requirements for a suitable \isi{eyetracking lab} see} %\label{ref:ZOTEROITEMCSLCITATIONcitationID0O74xe1ipropertiesformattedCitationrtfRuc0u246sener2016plainCitationRsener2016citationItemsid184urishttpzoteroorgusers1255332items9ZC424RGurihttpzoteroorgusers1255332items9ZC424RGitemDataid184typechaptertitleEyetrackingandbeyondThedosanddontsofcreatingacontemporaryusabilitylabcontainertitleEyetrackingandAppliedLinguisticspublisherLanguageSciencePresspublisherplaceBerlinpage143162eventplaceBerlinauthorfamilyRsenergivenChristopheditorfamilyHansenSchirragivenSilviafamilyGruzcagivenSamborissueddateparts2016schemahttpsgithubcomcitationstylelanguageschemarawmastercslcitationjsonRNDoszFAon8jt}
\citet[251--254]{Rosener2016}). The texts should be no longer than about 300 words in eyetracking studies so that scrolling does not become necessary, because this makes the data harder to assess. Similarly, the texts should be in a font size of 16 or 18 and at least 1.5 line spacing so that the eyetracking data can be mapped correctly. (cf. %\label{ref:ZOTEROITEMCSLCITATIONcitationID9MhyOeeOpropertiesformattedCitationrtfOuc0u8217Brien2009plainCitationOBrien2009citationItemsid3041urishttpzoteroorgusers1255332items9WCPJ74Xurihttpzoteroorgusers1255332items9WCPJ74XitemDataid3041typearticlejournaltitleEyetrackingintranslationprocessresearchmethodologicalchallengesandsolutionscontainertitleMethodologyTechnologyandInnovationinTranslationProcessResearchpage251266volume38authorfamilyOBriengivenSharonissueddateparts2009schemahttpsgithubcomcitationstylelanguageschemarawmastercslcitationjsonRNDYFMszOfV1L}
{\citealt[261--262]{OBrien2009})}



{\isi{Eyetracking} is not only relevant for translation process studies – indeed it is a rather small research area that uses eyetracking – but also in reading and writing research, psycholinguistics, neuropsychology, cognitive psychology, usability testing, research in sports, advertising, marketing, product placements, medical appliances, human-machine interactions, computer science, etc.} %\label{ref:ZOTEROITEMCSLCITATIONcitationIDCu0dbWP3propertiesformattedCitationDuchowski2003plainCitationDuchowski2003citationItemsid193urishttpzoteroorgusers1255332items742EKQDEurihttpzoteroorgusers1255332items742EKQDEitemDataid193typebooktitleEyetrackingmethodologytheoryandpracticepublisherSpringerpublisherplaceNewYorknumberofpages251sourceLibraryofCongressISBNeventplaceNewYorkISBN9781852336660callnumberQA769H85D832003shortTitleEyetrackingmethodologyauthorfamilyDuchowskigivenAndrewTissueddateparts2003schemahttpsgithubcomcitationstylelanguageschemarawmastercslcitationjsonRNDJaSG3t5FBD}
{(for more information on eyetracking applications see e.g. \citealt{Duchowski2003}: 131-226)} The analysis in the upcoming chapters will focus on fixation durations and fixation counts, however the concrete parameters will be introduced, when they appear first in \sectref{sec:9} to \sectref{sec:12}.

\subsection{{ Neuroscientific methods}}
\label{sec:7:1:5}

{Translation studies have only in recent years slowly begun to use neuroscientific methods to ultimately tackle the problem of what is going on in the black box while translating. In this chapter, I will briefly present the functionalities of} {\isi{\textit{EEG}}}{ and} {\isi{\textit{fMRI}}}{ methods. As these methods are used in very controlled experiments and cannot (yet) be used in authentic translation tasks, none of these methods were used in the study at hand.}



{\isi{\textit{Electroencephalography}}}{ (EEG) is used to record electrical activity in the brain. To measure these activities, high conductance electrodes are put on certain locations of the human skull. Depending on what needs to be measured and how precise the recordings need to be, 16 to 256 electrodes are distributed on the head. The electrodes are usually placed on the participant's head with the help of a cap, and a special gel or another liquid is used to increase the conductivity between the electrodes and the skull. The EEG is either recorded in reference to one common} {passive electrode – monopolar (referential) recordings – or between different pairs of electrodes – bipolar recordings. EEG signals are recorded with a sampling rate of at least 100Hz, but 500Hz or higher are standard today. The electrodes are named according to the brain area they are located on (letters) and on which brain sites (odd numbers are on the left, even on the right), e.g. electrode F3 is on the frontal lobe on the left-hand side. The electrical activity in the brain produces oscillations which have been assigned to functions, and the pathology of the brain. However, not only oscillations but also typical patterns can be studied with the help of the EEG signals. \isi{Artefacts} in the EEG signals can be produced by external influences like head movement, blinking, or other muscular activities. These artefacts can hardly be avoided and have to be eleminated from the recordings, if possible, before the signals are analysed and interpreted (cf.} %\label{ref:ZOTEROITEMCSLCITATIONcitationID1kIfBD8upropertiesformattedCitationFreemanandQuianQuiroga2013plainCitationFreemanandQuianQuiroga2013citationItemsid189urishttpzoteroorgusers1255332itemsGQ997RJHurihttpzoteroorgusers1255332itemsGQ997RJHitemDataid189typebooktitleImagingbrainfunctionwithEEGadvancedtemporalandspatialanalysisofelectroencephalographicsignalspublisherSpringerpublisherplaceNewYorknumberofpages248sourceLibraryofCongressISBNeventplaceNewYorkISBN9781461449836callnumberQP3765F742013noteOCLCocn835959758shortTitleImagingbrainfunctionwithEEGauthorfamilyFreemangivenWalterJfamilyQuianQuirogagivenRodrigoissueddateparts2013schemahttpsgithubcomcitationstylelanguageschemarawmastercslcitationjsonRNDwLaEW3m7XY}
{\citealt[5--6]{freeman2012imaging}).} One \isi{event-related brain potential} (\isi{ERP}) that can be measured with the EEG is the \isi{N400}. This amplitude peaks negatively after about 400ms after a stimuli was presented. The N400 is widely acknowledged as a measure for semantic processing. If, for example, a sentence is presented including one stimuli that is semantically nonsensical, the negative amplitude will be greatly visible 400ms after the nonsensical stimuli was presented. %\label{ref:ZOTEROITEMCSLCITATIONcitationIDnHksUjiGpropertiesformattedCitationKutasandFedermeier2011plainCitationKutasandFedermeier2011citationItemsid188urishttpzoteroorgusers1255332itemsWQVW8FN6urihttpzoteroorgusers1255332itemsWQVW8FN6itemDataid188typearticlejournaltitleThirtyyearsandcountingFindingmeaningintheN400componentoftheeventrelatedbrainpotentialERPcontainertitleAnnualreviewofpsychologypage621643volume62authorfamilyKutasgivenMartafamilyFedermeiergivenKaraDissueddateparts2011schemahttpsgithubcomcitationstylelanguageschemarawmastercslcitationjsonRNDdb2bcl8INh}
(cf. \citealt{KutasFedermeier2011}) Translation studies has used EEG, for example to investigate priming, monitoring and exhibiton\footnote{Which are being examined by Katharina Oster in her PhD thesis in Germersheim, Uni Mainz (work in progress)}; conceptualisation %\label{ref:ZOTEROITEMCSLCITATIONcitationIDVwjeGZybpropertiesformattedCitationGrabneretal2007plainCitationGrabneretal2007citationItemsid167urishttpzoteroorgusers1255332items48N5HQD5urihttpzoteroorgusers1255332items48N5HQD5itemDataid167typearticlejournaltitleEventrelatedEEGthetaandalphabandoscillatoryresponsesduringlanguagetranslationcontainertitleBrainresearchbulletinpage5765volume72issue1authorfamilyGrabnergivenRolandHfamilyBrunnergivenClemensfamilyLeebgivenRobertfamilyNeupergivenChristafamilyPfurtschellergivenGertissueddateparts2007schemahttpsgithubcomcitationstylelanguageschemarawmastercslcitationjsonRNDt7G0iwnvH6}
(e.g. \citealt{GrabnerEtAl2007}); or expectancy violations (e.g. %\label{ref:ZOTEROITEMCSLCITATIONcitationIDziirbB3ppropertiesformattedCitationElmerMeyerandJancke2010plainCitationElmerMeyerandJancke2010citationItemsid166urishttpzoteroorgusers1255332itemsWAMJHKIDurihttpzoteroorgusers1255332itemsWAMJHKIDitemDataid166typearticlejournaltitleSimultaneousinterpretersasamodelforneuronaladaptationinthedomainoflanguageprocessingcontainertitleBrainresearchpage147156volume1317authorfamilyElmergivenStefanfamilyMeyergivenMartinfamilyJanckegivenLutzissueddateparts2010schemahttpsgithubcomcitationstylelanguageschemarawmastercslcitationjsonRNDmU6e1o8Gf5}
\citealt{ElmerEtAl2010}). Details on the studies can be found in the overview article on EEG and translation published by %\label{ref:ZOTEROITEMCSLCITATIONcitationIDWuMxaUcLpropertiesformattedCitationHansenSchirra2017plainCitationHansenSchirra2017citationItemsid168urishttpzoteroorgusers1255332items3ACQANXAurihttpzoteroorgusers1255332items3ACQANXAitemDataid168typechaptertitleEEGandUniversalLanguageProcessinginTranslationcontainertitleTheHandbookofTranslationandCognitionpublisherWileyBlackwellpublisherplaceMaldenMAOxfordEnglandpage232247eventplaceMaldenMAOxfordEnglandauthorfamilyHansenSchirragivenSilviaeditorfamilySchwietergivenJohnWfamilyFerreiragivenAlineissueddateparts2017schemahttpsgithubcomcitationstylelanguageschemarawmastercslcitationjsonRNDHMrCheGWYc}
\citet{Hansen-Schirra2017}.



Although the field of \isi{\textit{functional magnetic resonance imaging}} (fMRI) is still very young, the discoveries made so far are tremendous. The applications of fMRI include all areas of brain imaging and have become a very important tool for neuroscience research, both in clinical as well as cognitive research. %\label{ref:ZOTEROITEMCSLCITATIONcitationIDPYIpDefXpropertiesformattedCitationFaroandMohamed2006plainCitationFaroandMohamed2006citationItemsid132urishttpzoteroorgusers1255332itemsQUWMU7BHurihttpzoteroorgusers1255332itemsQUWMU7BHitemDataid132typebooktitleFunctionalMRIbasicprinciplesandclinicalapplicationspublisherSpringerScienceBusinessMediapublisherplaceNewYorkeventplaceNewYorkauthorfamilyFarogivenScottHfamilyMohamedgivenFerozeBissueddateparts2006schemahttpsgithubcomcitationstylelanguageschemarawmastercslcitationjsonRNDS10xKf6Vt3}
(cf. \citealt{FaroMohamed2006}: v-vi) The basic idea in brain imaging techniques goes back to first experiments in 1980 that assumed that the regional cerebral blood flow could mirror the activities of neurons. In 1990, it was documented “that functional brain mapping is possible by using the venous blood oxygen level-dependent (BOLD) magnetic resonance imaging (MRI) contrast” (%\label{ref:ZOTEROITEMCSLCITATIONcitationIDw6qXcqVGpropertiesformattedCitationKimandBandettini2006plainCitationKimandBandettini2006citationItemsid107urishttpzoteroorgusers1255332items3NDSRDPGurihttpzoteroorgusers1255332items3NDSRDPGitemDataid107typechaptertitlePrinciplesoffunctionalMRIcontainertitleFunctionalMRIBasicPrinciplesandClinicalApplicationspublisherSpringerpublisherplaceNewYorkpage322eventplaceNewYorkauthorfamilyKimgivenSeongGifamilyBandettinigivenPeterAeditorfamilyFarogivenScottHfamilyMohamedgivenFerozeBissueddateparts2006schemahttpsgithubcomcitationstylelanguageschemarawmastercslcitationjsonRNDWbXmIjYD6u}
\citealt{KimBandettini2006}: 3). The \isi{BOLD method} depends on the level of deoxyhemoglobin, which can be seen in the signal intensity of magnetic resonance images when the level changes, and can hence be used for human brain imaging. fMRI can be used to study diverse brain functions like vision, language, motor abilities, and cognition (cf. ibid.). Language was one of the first functions that was ascribed to a particular area in the brain and has been the subject of research for over 100 years now. Compared to other brain imaging methods, fMRI is especially useful to show language areas in the brain, because it is non-invasive and produces images of good quality and with good localisation (amongst other benefits). It is rather difficult to assign brain regions to single language processes like phonetic, semantic, or syntactic processes, because they often work together. Carefully selected research designs with contrasting conditions can, however, help in tackling these problems %\label{ref:ZOTEROITEMCSLCITATIONcitationIDpTjhrYaQpropertiesformattedCitationBinder2006plainCitationBinder2006citationItemsid148urishttpzoteroorgusers1255332items576JP956urihttpzoteroorgusers1255332items576JP956itemDataid148typechaptertitlefMRIoflanguagesystemsmethodsandapplicationscontainertitleFunctionalNeuroradiologypublisherSpringerpublisherplaceNewYorkpage245277eventplaceNewYorkauthorfamilyBindergivenJeffreyReditorfamilyFarogivenScottHfamilyMohamedgivenFerozeBfamilyLawgivenMengfamilyUlmergivenJohnTissueddateparts2006schemahttpsgithubcomcitationstylelanguageschemarawmastercslcitationjsonRND6lAKKlSeAe}
(cf. \citealt{Binder2006}: 245-248).



{These methods for analysing the brain functions open doors to find out what is happening in the human mind. Although these methods are mainly used in a medical and diagnostic context, they also help us to understand the human \isi{black box}. Especially concerning translation research, most of the work is still ahead of us, as these methods have hardly been used, yet, and it appears difficult to test translation concepts with these methods. However, some work has already been done, e.g. the above mentioned studies using EEG, or} %\label{ref:ZOTEROITEMCSLCITATIONcitationIDTRJN7rhEpropertiesformattedCitationAhrensetal2010plainCitationAhrensetal2010citationItemsid161urishttpzoteroorgusers1255332itemsMTKX5PSJurihttpzoteroorgusers1255332itemsMTKX5PSJitemDataid161typechaptertitlefMRIforexploringsimultaneousinterpretingcontainertitleWhytranslationstudiesmatterspublisherJohnBenjaminsPublishingCompanypublisherplaceAmsterdamPhiladelphiapage237249volume88eventplaceAmsterdamPhiladelphiaauthorfamilyAhrensgivenBarbarafamilyKalderongivenElizafamilyKrickgivenChristophMfamilyReithgivenWolfgangeditorfamilyGilegivenDanielfamilyHansengivenGydefamilyPokorngivenNikeissueddateparts2010schemahttpsgithubcomcitationstylelanguageschemarawmastercslcitationjsonRNDyB1oB1B4s6}
{\citet{AhrensEtAl2010}}{ or} %\label{ref:ZOTEROITEMCSLCITATIONcitationIDufGwhZ6XpropertiesformattedCitationFranceschiniZappatoreandNitsch2003plainCitationFranceschiniZappatoreandNitsch2003citationItemsid130urishttpzoteroorgusers1255332itemsTZJ9CK9Hurihttpzoteroorgusers1255332itemsTZJ9CK9HitemDataid130typechaptertitleLexiconinthebrainWhatneurobiologyhastosayaboutlanguagescontainertitleThemultilinguallexiconpublisherKluwerAcademicPublisherspublisherplaceNewYorkaopage153166eventplaceNewYorkaoauthorfamilyFranceschinigivenRitafamilyZappatoregivenDanielafamilyNitschgivenCordulaeditorfamilyCenozgivenJasonefamilyHufeisengivenBrittafamilyJessnergivenUlrikeissueddateparts2003schemahttpsgithubcomcitationstylelanguageschemarawmastercslcitationjsonRNDRvfJNBV9ih}
\citet{FranceschiniEtAl2003}{ using fMRI. A great deal of exciting research can be expected in the next years, which might show if and in which aspects the translators' brains work differently to other bilinguals.}


\subsection{{ Data triangulation and choice of participants}}
\label{sec:7:1:6}

{\isi{Triangulation} in research means linking two or more sources of data, researchers, methodological approaches, theoretical ideas or analytical designs so that the different advantages of each methods can be exploited} %\label{ref:ZOTEROITEMCSLCITATIONcitationID7AwLkCdipropertiesformattedCitationThurmond2001plainCitationThurmond2001citationItemsid190urishttpzoteroorgusers1255332itemsMKUFM4F6urihttpzoteroorgusers1255332itemsMKUFM4F6itemDataid190typearticlejournaltitleThepointoftriangulationcontainertitleJournalofnursingscholarshippage253258volume33issue3authorfamilyThurmondgivenVeronicaAissueddateparts2001schemahttpsgithubcomcitationstylelanguageschemarawmastercslcitationjsonRNDHH3kOnU7Sg}
{(cf. \citealt{Thurmond2001}: 253-257).}{ As a result, \isi{data triangulation} has become more and more important in translation studies, too (e.g.} %\label{ref:ZOTEROITEMCSLCITATIONcitationIDbyRdlGbxpropertiesformattedCitationAlves2003plainCitationAlves2003citationItemsid169urishttpzoteroorgusers1255332itemsFAJCQQPFurihttpzoteroorgusers1255332itemsFAJCQQPFitemDataid169typebooktitleTriangulatingtranslationperspectivesinprocessorientedresearchpublisherJohnBenjaminsPublishingpublisherplaceAmsterdamPhiladelphiavolume45eventplaceAmsterdamPhiladelphiaauthorfamilyAlvesgivenFabioissueddateparts2003schemahttpsgithubcomcitationstylelanguageschemarawmastercslcitationjsonRNDlr6SZoVBwr}
{\citealt{Alves2003})}{. Triangulation has the advantages that the research data become more reliable, inventive approaches are developed to comprehend and interpret research hypotheses, existing theories might be challenged or confirmed, and a phenomenon can be better understood. Every \isi{triangulation} approach on its own has individual advantages ad disadvantages, whether it is data, researcher, methodological, theory, or analysis \isi{triangulation}. In general, these include}


\begin{quote}
{[an]}{ increased} amount of time needed in comparison to single strategies, [...] difficult[ies] of dealing with the vast amount of data, [...] potential disharmony based on investigator biases, [...] conflicts because of theoretical frameworks, and [...] lack of understanding about why \isi{triangulation} strategies were used. (ibid.: 256)
\end{quote}


{Conclusively, \isi{triangulation} is valuable to gather findings from different perspectives that are complete and confirm each other, and to strengthen the findings. The researchers, however, must be able to explain why they used the \isi{triangulation} method and why it was necessary. (cf. ibid.: 257) Many studies have adopted this approach in translation process research in recent years, especially concerning data \isi{triangulation}. The danger, however, is that, amongst other possible problems, the amount of data becomes overwhelming. Some solutions on how to deal with large data sets might be to work in research teams so that either the same research topic is analysed together or that different researchers analyse various hypotheses on the same data (}%\label{ref:ZOTEROITEMCSLCITATIONcitationIDYE9a2spspropertiesformattedCitationrtfOuc0u8217Brien2009plainCitationOBrien2009citationItemsid3041urishttpzoteroorgusers1255332items9WCPJ74Xurihttpzoteroorgusers1255332items9WCPJ74XitemDataid3041typearticlejournaltitleEyetrackingintranslationprocessresearchmethodologicalchallengesandsolutionscontainertitleMethodologyTechnologyandInnovationinTranslationProcessResearchpage251266volume38authorfamilyOBriengivenSharonissueddateparts2009schemahttpsgithubcomcitationstylelanguageschemarawmastercslcitationjsonRNDaquzVCm0ds}
{\citealt[260-261, 264]{OBrien2009})}{.}



{No matter which methodologies are chosen for the individual study, they all have in common that the \isi{participants} for the study must be chosen carefully. Although professional translators are often considered more valuable as they have practical experience, they are harder to acquire for a study and often expect financial compensation as they miss (part of) their work day. Students on the other hand are easier to acquire as eyetracking studies in translation process research are usually conducted at a university and the study might even be credited in classes. Further, the participants may complete some studies, like questionnaire studies, at home, while other studies, especially those that require special equipment like most eyetracking-, EEG-, or fMRI-studies can only be realised in special labs. In addition, one question is whether the participants will commit to the task with equal enthusiasm when they invest their free time in participating in studies as opposed to when they are paid or rewarded in a different manner for the task. Additionally, limited funding might restrict the number of participants that can be recruited for the study. However, it is doubtful whether small participant numbers, e.g. twelve participants or even less, can actually return generalisable results. Nonetheless, these studies are valuable to build hypotheses for larger studies. Finally, the professionalism of the participants has to be addressed. Not every translator with a degree in translation is equally capable of all tasks, e.g. domain and text types might play a role or experience with CAT-tools depending on the kind of study. Some issues may also occur which disqualify the} {participant for the study, e.g. his\slash her typing skills, the language competence, the ability to follow instructions, or if the participant feels intimidated, judged, or pressured during the session (cf.} %\label{ref:ZOTEROITEMCSLCITATIONcitationIDzX8dI9h5propertiesformattedCitationrtfOuc0u8217Brien2009plainCitationOBrien2009citationItemsid3041urishttpzoteroorgusers1255332items9WCPJ74Xurihttpzoteroorgusers1255332items9WCPJ74XitemDataid3041typearticlejournaltitleEyetrackingintranslationprocessresearchmethodologicalchallengesandsolutionscontainertitleMethodologyTechnologyandInnovationinTranslationProcessResearchpage251266volume38authorfamilyOBriengivenSharonissueddateparts2009schemahttpsgithubcomcitationstylelanguageschemarawmastercslcitationjsonRNDkivtUl47Rq}
{\citealt[254-259, 262]{OBrien2009}}{).}

\section[General information on the data set, post-editing guidelines, and setup of the experiment]{General information on the data set, post-editing guidelines, and setup of the experiment\sectionmark{General information}}\sectionmark{General information}
\label{sec:7:2}

{The study was conducted at the University of Mainz}{, F}{aculty of Translation Studies, Linguistics and Cultural Studies}{} {in Germersheim by a team of the \ili{English} Linguistics and Translation Studies in 2012 on behalf of the \isi{Center for Research and Innovation in Translation and Translation Technology} (\isi{CRITT}), Copenhagen Business School, Denmark. The experiments became part of the \isi{CRITT TPR database} that collects translation process data for different tasks and in different languages (find more information on the database later in this chapter). In total, 24 \isi{participants} took part in the study, twelve of them professional translators (university degree and at least some professional work experience), and twelve semi-professional translators (students of the university with only little professional work experience) – see detailed information in \sectref{sec:8:1}.}



{Four newspaper articles and two sociology-related texts with different \isi{complexity levels} had to be processed – all \ili{English} to \ili{German}. The length of the texts varies between 100 and 148 words. Text~1 (148 words) deals with a former hospital nurse who killed four of his patients. Text~2 (139 words) covers the increasing prices in Great Britain that are not in balance to salary increases. Steven Spielberg's refusal to be part of the Olympics in China to protest against \ili{Chinese} politics is the topic of Text~3 (132 words). These first three texts of this study were also part of} %\label{ref:ZOTEROITEMCSLCITATIONcitationID8NB6KxECpropertiesformattedCitationHvelplund2011plainCitationHvelplund2011citationItemsid164urishttpzoteroorgusers1255332itemsA5SKCFC4urihttpzoteroorgusers1255332itemsA5SKCFC4itemDataid164typebooktitleAllocationofcognitiveresourcesintranslationAneyetrackingandkeyloggingstudypublisherCopenhagenBusinessSchoolPhDSeriespublisherplaceCopenhagenDenmarkeventplaceCopenhagenDenmarkauthorfamilyHvelplundgivenKristianTangsgaardissueddateparts2011schemahttpsgithubcomcitationstylelanguageschemarawmastercslcitationjsonRNDVTv8Uf0VVX}
{\citegen{Hvelplund2011}}{ PhD thesis and three texts were added for this study. Text~4 (100 words) reports on the necessity that developing countries need to be supported in environmental issues. Text~5 (121 words) informs about the origins of the field of sociology. And finally, Text~6 (112 words) describes hunter-gatherer societies.}



{The participants}{ were asked to translate two text from scratch (TfS), bilingually post-edit (PE) two machine translated texts and monolingually post-edit (MPE) two machine translated texts. There were no time restrictions and the participants could use the Internet freely as a \isi{research} tool. Before and after the processing task, they had to complete \isi{questionnaires} that dealt with general informations about the participant, his\slash her attitude towards MT, and a self-estimation (see \chapref{sec:8}). The} {texts were distributed in a manner that every text was translated eight times from scratch, bilingually post-edited eight times, and monolingually post-edited eight times, but no participant worked with the same text sequence (cf. \tabref{tab:key:7:1}).}

\begin{table}
% \resizebox{\textwidth}{!}{%
\begin{tabular}{l*{3}{l@{~~}l}}
\lsptoprule
 Participant & \multicolumn{2}{c}{ TfS} & \multicolumn{2}{c}{ PE} & \multicolumn{2}{c}{ MPE}\\
 \midrule
 P01 & Text~1 & Text~2 & Text~3 & Text~4 & Text~5 & Text~6\\
 P02 & Text~3 & Text~4 & Text~5 & Text~6 & Text~1 & Text~2\\
 P03 & Text~5 & Text~6 & Text~1 & Text~2 & Text~3 & Text~4\\
 P04 & Text~2 & Text~1 & Text~4 & Text~3 & Text~6 & Text~5\\
 P05 & Text~4 & Text~3 & Text~6 & Text~5 & Text~2 & Text~1\\
 P06 & Text~6 & Text~5 & Text~2 & Text~1 & Text~4 & Text~3\\
 P07 & Text~1 & Text~3 & Text~2 & Text~4 & Text~5 & Text~6\\
 P08 & Text~3 & Text~5 & Text~4 & Text~6 & Text~1 & Text~2\\
\multicolumn{7}{c}{ etc.}\\
\lspbottomrule
\end{tabular}%}
\caption{Distribution of the texts exemplified on the first eight participants\label{tab:key:7:1}}
\end{table}


There were no time restrictions for the tasks and the participants were given the following \isi{guidelines for the PE task} (see also %\label{ref:ZOTEROITEMCSLCITATIONcitationIDpcHJ5F1ppropertiesformattedCitationCarlGutermuthandHansenSchirra2014plainCitationCarlGutermuthandHansenSchirra2014citationItemsid165urishttpzoteroorgusers1255332items2ME62S8Turihttpzoteroorgusers1255332items2ME62S8TitemDataid165typechaptertitlePostEditingMachineTranslationaUsabilityTestforProfessionalTranslationSettingscontainertitlePsycholinguisticandcognitiveinquiriesintranslationandinterpretationstudiespublisherCambridgeScholarsPublishingpublisherplaceNewcastleuponTynepage145174eventplaceNewcastleuponTyneauthorfamilyCarlgivenMichaelfamilyGutermuthgivenSilkefamilyHansenSchirragivenSilviaeditorfamilyFerreiragivenAlinefamilySchwietergivenJohnWissueddateparts2014schemahttpsgithubcomcitationstylelanguageschemarawmastercslcitationjsonRNDF67H4LSciq}
\citealt[153]{CarlSchwieter2014}):


\begin{itemize}
\item Retain as much raw translation as possible.
\item Don’t hesitate too long over a problem.
\item Don’t worry if style is repetitive.
\item Don’t embark on time-consuming research.
\item Make changes only where absolutely necessary: correct words or phrases that are nonsensical, wrong, and if there’s enough time left, ambiguous.
\end{itemize}

{The tasks were conducted in} {\textit{\isi{Translog II}}}{, a program used to record keystrokes, mouse activities and gaze data with the help of the} {\isi{\textit{Tobii TX300}}}{ eyetracker, which also recorded the sessions, keystrokes, mouse activities and gaze data in Tobii Studio. The eyetracking and keylogging data were combined in \isi{Translog II} via word alignment. The aligned keylogging and eyetracking data are available in the CRITT TPR database. The data of Version 1.6 of the database were used for the analysis in the thesis at hand, if not stated differently. The \isi{keylogging software}, the eyetracking system, and the database will be described in detail in the following.}



\isi{Translog} was first developed in 1995 with the primary goal of adding “hard information” to think-aloud protocol studies that were conducted frequently in the early days. The software is designed to log translation processes rather than mere writing processes. Further, the recordings of the sessions could be presented to the participants after the experiments for retrospective think-aloud interviews. The programme had three main functions: it could display the source text, it could record all key activities, and the recorded data could be displayed dynamically as well as linearly. The first \isi{Translog} version for Windows was released towards the end of 1999 and called Translog2000 (cf. %\label{ref:ZOTEROITEMCSLCITATIONcitationIDyfcrst4ZpropertiesformattedCitationJakobsen2011plainCitationJakobsen2011citationItemsid111urishttpzoteroorgusers1255332itemsXXFWDVVWurihttpzoteroorgusers1255332itemsXXFWDVVWitemDataid111typechaptertitleTrackingtranslatorskeystrokesandeyemovementswithTranslogcontainertitleMethodsandStrategiesofProcessResearchpublisherJohnBenjaminsPublishingCompanypublisherplaceAmsterdamPhiladelphiapage3755eventplaceAmsterdamPhiladelphiaauthorfamilyJakobsengivenArntLykkeeditorfamilyAlvstadgivenCeciliafamilyHildgivenAdelinafamilyTiseliusgivenElisabetissueddateparts2011schemahttpsgithubcomcitationstylelanguageschemarawmastercslcitationjsonRNDgoSV2muHxk}
\citealt{Jakobsen2011}: 38-39). Within the scope of the Eye-to-IT project, \isi{Translog} was rewritten to supplement eyetracking with the keylogging data. This combination of methods was first available in Translog2006. However, an external eyetracking device is still essential to record the eyetracking data. The data are combined via a gaze-to-word mapping (GWM) application which was developed in Tampere. Further, the data were no longer stored as binary code, but as open XML code (cf. ibid.: 42-43). However, the transmission between the eyetracker and the GWM programme was not flexible enough, so a new version of \isi{Translog}, \textit{\isi{Translog} II}, was developed, which directly communicates with the eyetracking hardware. The eyetracking and keylogging data can be mapped automatically via external software. If noise is in the data, the mappings can be improved manually (cf. %\label{ref:ZOTEROITEMCSLCITATIONcitationIDLSfzkVxRpropertiesformattedCitationCarl2012plainCitationCarl2012citationItemsid196urishttpzoteroorgusers1255332itemsR5RP4KN4urihttpzoteroorgusers1255332itemsR5RP4KN4itemDataid196typepaperconferencetitleTranslogIIaProgramforRecordingUserActivityDataforEmpiricalReadingandWritingResearchcontainertitleLRECpage41084112authorfamilyCarlgivenMichaelissueddateparts2012schemahttpsgithubcomcitationstylelanguageschemarawmastercslcitationjsonRNDZFrHAraIX1}
\citealt{Carl2012}: 4108-4109). In \isi{Translog}~II, projects with various properties can be created, run, and recorded. Further, the log files can be replayed in real time, analysed according to event statistics, and presented as linear presentations of the user activities as well as plots of the pauses. Two programmes are contained in the software. First, the experimenter can create, replay and analyse the projects in \textit{\isi{Translog} Supervisor}, while the experiments are conducted in \textit{\isi{Translog} User} (cf. ibid.: 4109). \isi{Translog} II and other auxiliary tools as well as publications and instructions can be downloaded for free on the CRITT's website.\footnote{\url{https://sites.google.com/site/centretranslationinnovation/translog-ii}, last accessed 18 November 2016}


The \textit{Tobii TX300} is a remote eyetracker that records gaze data with a sampling rate of 300Hz. These raw gaze data include a time stamp, the eye position, the gaze point, the pupil diameter, and a validity code (indicating the confidence of correctly identifying which is the left and the right eye) for each eye. Due to the large head movement box, the participants can move their heads relatively freely in front of the screen (in an operating distance of 50-80~cm) and no chin rest or alike is necessary. All hardware is integrated in one eyetracking unit that looks like an ordinary screen so the participants can work in an almost natural environment. The single components are a screen unit including a web cam, an eyetracking unit, and a digital angle gauge (cf. %\label{ref:ZOTEROITEMCSLCITATIONcitationIDqogd8dSepropertiesformattedCitationtobii2016plainCitationtobii2016dontUpdatetruecitationItemsid194urishttpzoteroorgusers1255332items6KT5N2WGurihttpzoteroorgusers1255332items6KT5N2WGitemDataid194typearticletitleTobiiTX300EyeTrackerForresearchofoculomotorfunctionsandnaturalhumanbehaviorURLhttpwwwtobiiprocomsiteassetstobiiprobrochurestobiiprotx300brochurepdfauthorfamilytobiigivenaccesseddateparts20161120schemahttpsgithubcomcitationstylelanguageschemarawmastercslcitationjsonRND9YrFJzmuUC}
\citealt{Tobii????}). The software \textit{Tobii Studio} can be used to set up and conduct the experiments (including an automatic calibrating system). Further, the experimenter can replay the screen recordings, track the key and mouse activities and assess the eyetracking data, which can either be downloaded as raw data or be automatically pre-interpreted by the software. However, Tobii Studio will not be presented in detail, because the experiment was conducted mainly via \textit{\isi{Translog II}} and most of the evaluation was done with the data of the CRITT TPR database (except for the analysis of the screen recording data).



As mentioned above, the data which are used for the study at hand is part of the \isi{CRITT TPR database} (cf. \href{https://sites.google.com/}{https://sites.google.com}\footnote{\url{https://sites.google.com/site/centretranslationinnovation/tpr-db}, last accessed 20\textsuperscript{th} November 2016}). The database contains process data of various studies conducted to explore the translation process. The data set at hand is part of a larger multilingual collection in which students and professionals worked with the same six texts, but translated and post-edited them in \isi{Translog} II into different languages: \ili{Spanish}, \ili{Japanese}, \ili{Danish}, \ili{Chinese}, \ili{Hindi}, and \ili{German}. Sometimes only a subset of the tasks or texts was recorded. Additionally, one study dealt with copying the texts. The database further contains eleven studies that were conducted in \isi{CASMACAT}, a CAT tool especially designed for PE.\footnote{\url{http://www.casmacat.eu/}, last accessed 20 November 2016} Finally, the database also includes 13 individual, unrelated studies that were all conducted with \isi{Translog}, too. So, the database consists of 1562 sessions with seven source languages and nine target languages, 15 different tasks, 620,210 source text tokens and 657,948 target text tokens. The tables that are provided for each data set in the database already present a great deal of information on keylogging and eyetracking data, most of which are already aligned on a word and sentence level, which simplifies the evaluation. While some data are “pure” eyetracking and keylogging data, like \textit{Del} that presents the number of deleted tokens or \textit{FixS} that informs on the number of fixations on the source text unit, other parameters are provided which present processed data, like \textit{Nunit} that reports on the number of micro units or \textit{Nedit} that informs about the number of times the segment has been edited. Finally, there is also additional information on the source and target text units like \textit{PoS}, which presents the part of speech of the token (cf. %\label{ref:ZOTEROITEMCSLCITATIONcitationID5BN07BnrpropertiesformattedCitationCarlandSchaeffer2013plainCitationCarlandSchaeffer2013citationItemsid144urishttpzoteroorgusers1255332itemsP4KVCC89urihttpzoteroorgusers1255332itemsP4KVCC89itemDataid144typearticlejournaltitleTheCRITTTranslationProcessResearchDatabaseV14URLhttpbridgecbsdkresourcestprdbTPRDB14pdfauthorfamilyCarlgivenMichaelfamilySchaeffergivenMoritzJissueddateparts2013accesseddateparts2014610schemahttpsgithubcomcitationstylelanguageschemarawmastercslcitationjsonRNDeZLwnPnP1F}
\citealt{CarlSchaeffer2013} or %\label{ref:ZOTEROITEMCSLCITATIONcitationIDRbwHtHIEpropertiesformattedCitationCarlSchaefferandBangalore2016plainCitationCarlSchaefferandBangalore2016citationItemsid145urishttpzoteroorgusers1255332itemsFRU2TXJ9urihttpzoteroorgusers1255332itemsFRU2TXJ9itemDataid145typechaptertitleTheCRITTTranslationProcessResearchDatabasecontainertitleNewDirectionsinEmpiricalTranslationProcessResearchpublisherSpringerpublisherplaceHeidelbergNewYorkaopage1354eventplaceHeidelbergNewYorkaoauthorfamilyCarlgivenMichaelfamilySchaeffergivenMoritzfamilyBangaloregivenSrinivasissueddateparts2016schemahttpsgithubcomcitationstylelanguageschemarawmastercslcitationjsonRND4q8o4wZK7v}
\citealt{CarlEtAl2016critt}, the papers also provides a detailed description of the parameter contained in the database).


\section{Placing the research hypotheses and methods into the field of translation process research}
\label{sec:7:3}

{The exploration of the \isi{translation process} seems like a bottomless pit because so many aspects of the process can be considered with various methods. Krings published an article in 2005, in which he attempts to model the countless aspects and methods of translation process research. These models will be used in the following to place the research hypotheses and methods into the field.} In his factor model, %\label{ref:ZOTEROITEMCSLCITATIONcitationIDxPIo4oxIpropertiesformattedCitationKrings2005plainCitationKrings2005dontUpdatetruecitationItemsid245urishttpzoteroorgusers1255332itemsEEX7687Vurihttpzoteroorgusers1255332itemsEEX7687VitemDataid245typearticlejournaltitleWegeinsLabyrinthFragestellungenundMethodenderbersetzungsprozessforschungimberblickcontainertitleMetaJournaldestraducteursMetaTranslatorsJournalpage342358volume50issue2authorfamilyKringsgivenHansPeterissueddateparts2005schemahttpsgithubcomcitationstylelanguageschemarawmastercslcitationjsonRNDZ2vaLGfD8J}
\citet[344--347]{Krings2005} summarises the single factors that influence the translation process in three main bundles: \isi{task-related factors}, \isi{translator-related factors}, and \isi{work environment-related factors}. The individual factors that belong to these bundles are judged by Krings to be the most influential factors on the translation process, however he does not claim completeness.



The first bundle refers to the factors that influence the process due to the differences in the translation tasks and include factors like different source and target languages, different translation assignments, different text types and text domains, the language direction of the translation (from or into the native language or between two foreign languages), and differences and similarities to neighbouring tasks like interpreting and PE. These factors are controlled by the experiment settings. Every translator works with the same texts, has to do the same tasks, and works with the same language combination.



The second bundle approaches the individual differences of translators like experience, language proficiency, domain knowledge, or individual strategical preferences. These factors are, on the one hand, retrieved in the questionnaires, and on the other hand, statistical methods are used in the analysis to figure out whether the findings in the sample of 24 individual translators could indicate assumptions for the total population.



Finally, the third bundle of factors includes the influence of technical aids or MT, the availability of research aids, and general factors like available time or possible contact to colleagues, etc. These factors are also predefined and controlled by the experiment settings. All participants had to work on the same computer, in the same editor, with the same MT output. They could use the Internet for research and could take as much time as they needed. Although this constricted setup prevents an all natural work environment, it still has the major advantage that all participants have to work under the same conditions.


\section{Previous research with the data set}
\label{sec:7:4}

The data set used in this dissertation is highly relevant for the designated research purposes. In the following, different studies will be introduced that were published on the same data set but with different research hypotheses. This is not intended to be exhaustive, but rather to provide some insights on the diverse use of the data. First, studies are presented that dealt only with the \ili{English}-\ili{German} data set, then three studies are introduced that investigated multilingual relationships using the \isi{CRITT TPR database}, including the \ili{English}-\ili{German} data set.



{In a pilot study by} %\label{ref:ZOTEROITEMCSLCITATIONcitationIDrE5sF7w5propertiesformattedCitationrtfuc0u268uloetal2014plainCitationuloetal2014citationItemsid140urishttpzoteroorgusers1255332items7V2QCKK7urihttpzoteroorgusers1255332items7V2QCKK7itemDataid140typechaptertitleTheInfluenceofPostEditingonTranslationStrategiescontainertitlePostEditingofMachineTranslationProcessesandApplicationspublisherCambridgeScholarsPublishingpublisherplaceNewcastleuponTynepage200218eventplaceNewcastleuponTyneauthorfamilyulogivenOliverfamilyGutermuthgivenSilkefamilyHansenSchirragivenSilviafamilyNitzkegivenJeaneditorfamilyWintherBallinggivenLaurafamilyCarlgivenMichaelissueddateparts2014schemahttpsgithubcomcitationstylelanguageschemarawmastercslcitationjsonRNDtGZZCFoOXO}
\citet{CuloEtAl2014},{ the authors concentrated not only on processes but mainly on the TfS and PE products. Therefore, different linguistic properties and their realisation in the TfS and (monolingual) PE tasks were analysed. The main focus was on} the {changes of \isi{translation strategies} in the different tasks. PE should not only accelerate the translation process, but should produce an intelligible text, though questions of style and idiomacy may be secondary.}



{In addition to inconsistent translations and atypical syntax, unidiomatic translations were one issue for discussion. In Text 3, the translators had to deal with the phrase} {\textit{In a gesture […]}}{. While the translators naturally chose an idiomatic translation in the TfS task, e.g.} {\textit{Mit einer Geste}}{ or} {\textit{Als Zeichen}}{, the MT system translated the phrase literally with} {\textit{In einer Geste}}{ which is unidiomatic in \ili{German}. Five out of seven post-editors kept this translation and did not change it into an idiomatic expression in the bilingual PE task. This cannot be counted as an error given that the task was for post-editors to retain as much of the unedited \isi{machine translation} as long as the final target sentence was understandable to a \ili{German} native speaker. However, in the MPE task, more participants were inclined to change the unidiomatic version into an idiomatic one. This indicates that the PE process shifts priorities, maybe due to the fact that the translators are working with two texts in parallel instead of just one text.}



The last analysed example in the paper highlights that \isi{MPE} can be very problematic due to the missing source, as content mistakes might remain unnoticed. Although the other examples might suggest that the quality of the monolingually post-edited texts is better than in the post-edited texts, severe content mistakes only occurred in MPE (further information on content mistakes in MPE can be found in %\label{ref:ZOTEROITEMCSLCITATIONcitationIDp9MCSwRlpropertiesformattedCitationNitzke2016monoplainCitationNitzke2016monodontUpdatetruecitationItemsid183urishttpzoteroorgusers1255332itemsUPJZPQ8Rurihttpzoteroorgusers1255332itemsUPJZPQ8RitemDataid183typechaptertitleMonolingualposteditingAnexploratorystudyonresearchbehaviourandtargettextqualitycontainertitleEyetrackingandAppliedLinguisticspublisherLanguageSciencePresspublisherplaceBerlinpage83108eventplaceBerlinauthorfamilyNitzkegivenJeaneditorfamilyHansenSchirragivenSilviafamilyGruzcagivenSamborissueddateparts2016schemahttpsgithubcomcitationstylelanguageschemarawmastercslcitationjsonRNDcFlBJys66G}
\citet{Nitzke2016mono} – the study will also be described briefly further below.)



{In the final analysis, \isi{fixation counts} were compared between TfS and PE in Text 3 for finite and non-finite clauses. While the fixation counts were equal for both clause types in TfS, fewer fixations were counted in the PE task and the results were not balanced: there were more fixation counts on finite clauses than on infinite. Therefore, the hypothesis that non-finite clauses cause errors in the MT output and interference effects was not confirmed nor that they cause longer processing times in PE.}



The study by %\label{ref:ZOTEROITEMCSLCITATIONcitationIDA6DLVOtqpropertiesformattedCitationCarlGutermuthandHansenSchirra2014plainCitationCarlGutermuthandHansenSchirra2014citationItemsid165urishttpzoteroorgusers1255332items2ME62S8Turihttpzoteroorgusers1255332items2ME62S8TitemDataid165typechaptertitlePostEditingMachineTranslationaUsabilityTestforProfessionalTranslationSettingscontainertitlePsycholinguisticandcognitiveinquiriesintranslationandinterpretationstudiespublisherCambridgeScholarsPublishingpublisherplaceNewcastleuponTynepage145174eventplaceNewcastleuponTyneauthorfamilyCarlgivenMichaelfamilyGutermuthgivenSilkefamilyHansenSchirragivenSilviaeditorfamilyFerreiragivenAlinefamilySchwietergivenJohnWissueddateparts2014schemahttpsgithubcomcitationstylelanguageschemarawmastercslcitationjsonRNDAKiLaKgpM7}
\citet{CarlSchwieter2014} first discusses the motivation for PE in the translation business. Afterwards, different research areas for PE are introduced: different PE types that appeal to different target text functions, text types and their suitability for MT\slash PE, users' needs, PE effort, PE as a MT quality evaluation method, technical aspects, training of post-editors, and the changing role of translators. Further, they point out key aspects of CASMACAT, MateCat and \isi{Translog~II}, and provide a detailed description of the \ili{English}-\ili{German} data set.



{In addition to some analyses of the \isi{questionnaires} (see full discussion in \chapref{sec:8}), one chapter of the study is dedicated to the evaluation of the unconscious reading and writing data of the participants. The average time a participant needed to translate a word reveals that most participants needed more time for translation from scratch; only one participant required more time for the PE task and three for the MPE task. No translator was the fastest in TfS (see more details on session durations in). Further, different PE styles are discussed and visualised: In both PE and TfS, different production phases can be separated – “an (optional) orientation phase, a drafting (or \isi{post-editing}) phase in which the actual translation is produced (or post-edited) and an optional final revision.” (ibid: 159). Accordingly, different PE patterns can be identified, e.~g. during drafting, some post-editors first read the source text (ST) and then check whether the MT reproduces the information from the ST, while other post-editors read the MT output and only refer to the ST when they come across words\slash passages that seem unlikely or problematic. In the subchapter on PE strategies, a relation between text complexity and eyetracking data is indicated: “}{\isi{Fixation duration} as well as fixation counts clearly show that the values increase in dependence of the complexity during translation while source text complexity does} {not seem to have such a strong impact on the \isi{post-editing task}.” (ibid: 164) The keylogging data are also investigated for the PE task of Text~3. The eight participants} {who post-edited this text differ a lot in their editing activity. Therefore, a more sensitive value for inefficiency is introduced (}{\textit{InEff}}{) which evaluates the amount of editing activity – a high \isi{InEff} score indicates a lot of activity, which is less efficient, and vice versa. Despite individual PE behaviour, some phrases reach a higher inefficiency score than others, which indicates that either the ST phrases are very complex and difficult to translate or the MT output is hard to adjust.}



{In her study,}%\label{ref:ZOTEROITEMCSLCITATIONcitationID2gUJBm9spropertiesformattedCitationNitzke2016monoplainCitationNitzke2016monodontUpdatetruecitationItemsid183urishttpzoteroorgusers1255332itemsUPJZPQ8Rurihttpzoteroorgusers1255332itemsUPJZPQ8RitemDataid183typechaptertitleMonolingualposteditingAnexploratorystudyonresearchbehaviourandtargettextqualitycontainertitleEyetrackingandAppliedLinguisticspublisherLanguageSciencePresspublisherplaceBerlinpage83108eventplaceBerlinauthorfamilyNitzkegivenJeaneditorfamilyHansenSchirragivenSilviafamilyGruzcagivenSamborissueddateparts2016schemahttpsgithubcomcitationstylelanguageschemarawmastercslcitationjsonRND2iRGGGyTuF}
{ \citet{Nitzke2016mono}} explores research behaviour and target text quality of \isi{monolingual post-edited texts} through product analysis as well as screen recording, keylogging and eyetracking data. Text quality aspects were divided into superficial mistakes and content mistakes. The data show that superficial mistakes are made almost as often in TfS, PE, and MPE with no significant correlation to the experience of the participants. Content mistakes {(mean per session and participant: 2.23, sd: 1.18)}{, however, occur much more often in MPE than in PE} {(mean: 0.30, sd: 0.51)}{ and TfS} {(mean: 0.61,} sd: 0.83). The screen-recordings revealed that fewer words and phrases are researched in MPE than in the other two tasks, but when a word\slash phrase is researched, it usually requires more steps to find a solution than in the other tasks. Further, no significant correlation between research behaviour and experience of the participants was found and sources were used slightly differently. The keylogging data indicate longer production times when the word\slash phrase was researched, but shorter production times for monolingual PE than in the other tasks. The eyetracking data (\isi{total gaze duration} on the target text), however, do not suggest differences in the tasks (see also \chapref{sec:9} on research behaviour). Unfortunately, the data set is not big enough for MPE to conduct statistical tests on these primary results.



So far, the three discussed studies have dealt exclusively with the \ili{English}-\ili{German} data set. However, the data set was also used for multilingual studies, which made use of the different subsets in the CRITT TPR database. Some of these will be summarised in the following.



%\label{ref:ZOTEROITEMCSLCITATIONcitationIDYppV1kP5propertiesformattedCitationWintherBallingandCarl2014plainCitationWintherBallingandCarl2014citationItemsid71urishttpzoteroorgusers1255332itemsC2AENCP3urihttpzoteroorgusers1255332itemsC2AENCP3itemDataid71typechaptertitleProductionTimeAcrossLanguagesandTasksALargeScaleAnalysisUsingtheCRITTTranslationProcessDatabasecontainertitlePsycholinguisticandcognitiveinquiriesintranslationandinterpretationstudiespublisherCambridgeScholarsPublishingpublisherplaceNewcastleuponTyneeventplaceNewcastleuponTyneauthorfamilyWintherBallinggivenLaurafamilyCarlgivenMichaelissueddateparts2014schemahttpsgithubcomcitationstylelanguageschemarawmastercslcitationjsonRNDRK2rJfP2ll}
\citet{Winther_balling2014}{ examine which parameters have an effect on the time that is needed to translate from scratch or post-edit a target text equivalent of the} {corresponding source text unit. For their analysis, they used keylogging and eyetracking data of 65 translators translating and \isi{post-editing} the same \ili{English} ST into \ili{Chinese}, \ili{German}, \ili{Hindi}, and \ili{Spanish} (all data sets are taken from the \isi{CRITT~TPR database}).}



After a brief overview of the field of translation process research and its methods, the software \isi{Translog~II}, and the CRITT~TPR database, they introduce the concepts \isi{Alignment Unit} (AU), \isi{Fixation Unit} (FU), and \isi{Production Unit} (PU), because the main aim of the study is to explain differences in AUs. An AU represents (a) word(s) in the ST and the corresponding word(s) in the TT. The tokens in this unit do not need to be coherent, i.~e. the AU might be separated by parts of another unit. Therefore, AUs can be grouped into continuous and discontinuous units.



In the next chapter, a detailed analysis of the considered variables is presented as well as the multiple regression model. Variables that were not significant were removed. Finally, the chapter discusses the results, which include amongst others that TfS takes longer than PE in all languages; that ST words with a low frequency take longer to be produced in the TT, especially by students; that a high number of translation possibilities has slowing effects in PE but not in TfS; that parallel processing (shifting the attention between ST, TT, and the keyboard) is time-consuming; and that the overall translation time is different for the different target languages (translations into \ili{Hindi} take the most time, while \ili{Spanish} translators were the quickest).


\largerpage
{It is assumed in numerous theories that one-to-one translations are less difficult to produce than differently phrased translations. The \isi{literal translation hypothesis} assumes that translators translate a source text unit literally first and then develop a looser version for the target text (cf.} %\label{ref:ZOTEROITEMCSLCITATIONcitationIDx9olxGKXpropertiesformattedCitationChesterman2011plainCitationChesterman2011citationItemsid143urishttpzoteroorgusers1255332itemsSFBN69GJurihttpzoteroorgusers1255332itemsSFBN69GJitemDataid143typechaptertitleReflectionsontheliteraltranslationhypothesiscontainertitleMethodsandstrategiesofprocessresearchintegrativeapproachesintranslationstudiespublisherJohnBenjaminsPublishingCompanypublisherplaceAmsterdamPhiladelphiapage2335volume94eventplaceAmsterdamPhiladelphiaauthorfamilyChestermangivenAndreweditorfamilyAlvstadgivenCeciliafamilyHildgivenAdelinafamilyTiseliusgivenissueddateparts2011schemahttpsgithubcomcitationstylelanguageschemarawmastercslcitationjsonRNDwyr0GwIgcD}
{\citealt{Chesterman2011})}{. In a paper by} %\label{ref:ZOTEROITEMCSLCITATIONcitationID3UwXlrZXpropertiesformattedCitationSchaefferandCarl2014plainCitationSchaefferandCarl2014citationItemsid254urishttpzoteroorgusers1255332itemsVVRH2ZJUurihttpzoteroorgusers1255332itemsVVRH2ZJUitemDataid254typearticlejournaltitleMeasuringtheCognitiveEffortofLiteralTranslationProcessescontainertitleEACL2014page2937authorfamilySchaeffergivenMoritzfamilyCarlgivenMichaelissueddateparts2014schemahttpsgithubcomcitationstylelanguageschemarawmastercslcitationjsonRNDfuXJ8ufQYc}
\citet{Schaeffer2014}{, a metric is introduced that measures how literal translations are. Further, they evaluate the effort that is necessary for non-literal translation. To investigate the issue, they use the \isi{gaze behaviour} of translators for different language pairs from the CRITT TPR. Different lexical realisations of source words were counted and for some words a higher variation was detected than for others. Therefore, some words require more effort to realise in the \isi{target language} than others. Next, on a syntactic basis, alignment crossing is introduced: The metric computes the} {\isi{\textit{Cross}} }{values for single words, based on their position in the source and} {the \isi{target language}. Depending on the point of view, the} {\textit{Cross}}{ value can be realised from the source text as a reference and the target text as output (}{\isi{\textit{CrossS}}}{) or the other way around (}{\isi{\textit{CrossT}}}{). The smaller the} {\textit{Cross}}{ value, the more similar the texts are in terms of structure; when the} {\textit{Cross}}{ value is high, syntax varies significantly.}



{In chapter three of the article, where translator behaviour is analysed for various aspects, different parts of the database were used. To map the alignment crossing, 313 translation sessions (source languages: \ili{Danish} and \ili{English}; target languages: \ili{Chinese}, \ili{Danish}, \ili{English}, \ili{German}, \ili{Hindi}, and \ili{Spanish}) were used for analysis. The analysis showed that higher} {\textit{Cross} }{values strongly correlate with \isi{total reading time} on source and target words and, therefore, prove that high syntactic variation takes more effort to produce. In the next subchapter, alignment crossing in the PE tasks is analysed (96 sessions of nine \ili{English} target texts that were post-edited for \ili{German}, \ili{Hindi}, and \ili{Spanish}). Strong correlations were found between negative} {\textit{CrossT}}{ values and \isi{total reading time} on the source text as well as between} {\textit{CrossS} }{and \isi{total reading time} on the target text. Finally, 24 TfS sessions from \ili{English} into \ili{Danish} and 65 PE sessions from \ili{English} into \ili{Spanish} and \ili{German} were considered for the translation choices analysis. Different realisations of source text items were counted and, for the analysis, only items were taken into consideration that were realised at least in nine different ways. A strong correlation was found between production time of target text word and number of alternative translations. Further, “[w]ith few choices posteditors are quicker than translators, but this distance decreases as the number of translation choices increase.” (ibid.: 34). Additionally, a strong correlation was detected between \isi{total reading time} on target text word and number of alternative translations. For translation from scratch, a correlation was found between \isi{total reading time} on source text and translation variations. However, no correlation was detected for PE.}



The role of co-activation of languages in translation and its influence on the translator’s behaviour is investigated in %\label{ref:ZOTEROITEMCSLCITATIONcitationIDbJIO9mRGpropertiesformattedCitationBangaloreetal2016plainCitationBangaloreetal2016citationItemsid152urishttpzoteroorgusers1255332itemsVQKM26ZXurihttpzoteroorgusers1255332itemsVQKM26ZXitemDataid152typechaptertitleSyntacticVarianceandPrimingEffectsinTranslationcontainertitleNewDirectionsinEmpiricalTranslationProcessResearchpublisherSpringerpublisherplaceHeidelbergNewYorkDordrechtLondonpage211238eventplaceHeidelbergNewYorkDordrechtLondonauthorfamilyBangaloregivenSrinivasfamilyBehrensgivenBergljotfamilyCarlgivenMichaelfamilyGhankotgivenMaheshwarfamilyHeilmanngivenArndtfamilyNitzkegivenJeanfamilySchaeffergivenMoritzfamilySturmgivenAnnegreteditorfamilyCarlgivenMichaelfamilyBangaloregivenSrinivasfamilySchaeffergivenMoritzissueddateparts2016schemahttpsgithubcomcitationstylelanguageschemarawmastercslcitationjsonRNDGj4PirzrNe}
\citet{BangaloreEtAl2016}. Four subsets of the database were used for evaluation: the translations and post-edits of the \ili{English} source text into \ili{Danish}, \ili{German}, and \ili{Spanish} as well as the \ili{English}-\ili{English} copying study, which was used as baseline. The syntactic variations in those data sets were measured with the help of manually created annotations that include three features – valency of the verb, voice, and clause type. On the basis of this annotation, \isi{entropy} values were calculated for the target sentences, which were then used to correlate them with measures of cognitive effort, in this case \isi{total reading time} on source and target text as well as coherent typing activity. The findings were compared across languages.


\largerpage
In TfS, syntactic variation could be positively correlated with \isi{total reading time} per source word and production time. However, no effect of syntactic variation on the \isi{total reading time} was found for PE, which indicates that the MT output primes the participants in the PE tasks. A highly significant positive correlation between syntactic \isi{entropy} and lexical translation \isi{entropy} confirms that lexical and semantic aspects cannot be examined completely autonomously. In addition, the study supports the hypothesis that source and target texts are co-activated and that different levels of co-activation can be detected during translation.



This overview shows the potential of the multilingual \isi{CRITT TPR database}. Many different research questions can be addressed concerning translation products, translation processes, and cognitive effort during translation. The study at hand will expand on the existing studies and explore \isi{problem solving behaviour} in the \ili{English}-\ili{German} data set.


\section{Session durations}
\label{sec:7:5}

The first analysis of this study will be on the time the participants spent on each text and task. This was similarly assessed by %\label{ref:ZOTEROITEMCSLCITATIONcitationIDCkcTrWrspropertiesformattedCitationCarlGutermuthandHansenSchirra2014plainCitationCarlGutermuthandHansenSchirra2014citationItemsid165urishttpzoteroorgusers1255332items2ME62S8Turihttpzoteroorgusers1255332items2ME62S8TitemDataid165typechaptertitlePostEditingMachineTranslationaUsabilityTestforProfessionalTranslationSettingscontainertitlePsycholinguisticandcognitiveinquiriesintranslationandinterpretationstudiespublisherCambridgeScholarsPublishingpublisherplaceNewcastleuponTynepage145174eventplaceNewcastleuponTyneauthorfamilyCarlgivenMichaelfamilyGutermuthgivenSilkefamilyHansenSchirragivenSilviaeditorfamilyFerreiragivenAlinefamilySchwietergivenJohnWissueddateparts2014schemahttpsgithubcomcitationstylelanguageschemarawmastercslcitationjsonRND0yW9M3IqrM}
\citet[157--158]{CarlSchwieter2014} for the data set and will be considered here, too. They showed that for most participants TfS took the longest. Only one participant (P18) needed more time on average for PE than for TfS, and three participants (P13, P14, P20) needed longer for MPE. However, due to illustration purposes, the data will be re-composed. \figref{fig:key:7:1} shows the times for the single sessions per participant. Four MPE, four PE, and three TfS sessions were missing, so 44 MPE and PE sessions, and 45 TfS sessions were available for the evaluation. Considering all complete session, P05 and P22 needed most time to finish all tasks and texts.


\begin{figure}
\caption{Length of sessions per participant in seconds}
\label{fig:key:7:1}
\resizebox{.6\textwidth}{!}{
\begin{tikzpicture}[trim axis right,trim axis left]
\pgfplotstableread{data/Fig7.1.csv}{\table}
    \pgfplotstablegetcolsof{\table}
    \pgfmathtruncatemacro\numberofcols{\pgfplotsretval-1}
        \begin{axis}[
                    ybar stacked,
                    xtick=data,
                    axis lines*=left,
%                     nodes near coords,
                    ymin=0,
                    xticklabels from table={\table}{P},
                    bar width=3mm,
                    width=\textwidth,
                    xticklabel style={rotate=90, anchor=east},
                    ticklabel style={font=\footnotesize},
                    enlarge x limits={0.05},
                    colormap/Accent,
                    cycle list/Accent,
                    legend pos=outer north east,
                    reverse legend,
                    legend style={font=\footnotesize}
                    ]
            \foreach \i in {1,...,\numberofcols} {
                \addplot+[
                    /pgf/number format/read comma as period, fill
                    ] table [x index={1},y index={\i},x expr=\coordindex] {\table};
                \pgfplotstablegetcolumnnamebyindex{\i}\of{\table}\to{\colname} % Adding column headers to legend
                \addlegendentryexpanded{\colname}
            }   
            \end{axis}    
\end{tikzpicture}}
% % \includegraphics[width=\textwidth]{figures/DissertationNitzkeberarbeitet-img3.jpg}
\end{figure}

 


As can be seen in \figref{fig:key:7:2}, most of the time was spent in the TfS task and the least in MPE. The differences between PE and MPE are not very noticeable. The mean time spent was highest for TFS (mean: 989.4~s, sd: 258.8~s), then PE (mean: 748.1, sd: 206~s), and the least time was spent on MPE (mean: 668.9~s, sd: 200.1~s). As the visualisation already suggests, the difference between PE and MPE is not significant ($t=-1.32$, $p=0.1931$). However, the differences between PE and TfS ($t=3.544$, $p<0.001$) as well as between MPE and TfS ($t=4.76$, $p<0.0001$) are significant, proving that TfS takes significantly longer than both PE tasks.


\begin{figure}
\caption{Added length of sessions per task in seconds}
\label{fig:key:7:2}

\resizebox{.6\textwidth}{!}{
\begin{tikzpicture}[trim axis right,trim axis left]
\pgfplotstableread{data/Fig7.2.csv}{\table}
    \pgfplotstablegetcolsof{\table}
    \pgfmathtruncatemacro\numberofcols{\pgfplotsretval-1}
        \begin{axis}[
                    ybar stacked,
                    xtick=data,
                    axis lines*=left,
%                     nodes near coords,
                    ymin=0,
                    xticklabels from table={\table}{P},
                    bar width=7mm,
                    width=\textwidth,
                    scaled y ticks = false,
                    ticklabel style={font=\footnotesize},                    
                    colormap/Accent,
                    cycle list/Accent,
                    legend pos=outer north east,
                    reverse legend,
                    legend style={font=\footnotesize}
                    ]
            \foreach \i in {1,...,\numberofcols} {
                \addplot+[
                    /pgf/number format/read comma as period, fill
                    ] table [x index={1},y index={\i},x expr=\coordindex] {\table};
                \pgfplotstablegetcolumnnamebyindex{\i}\of{\table}\to{\colname} % Adding column headers to legend
                \addlegendentryexpanded{\colname}
            }   
            \end{axis}    
\end{tikzpicture}
}
% % \includegraphics[width=\textwidth]{figures/DissertationNitzkeberarbeitet-img4.jpg}
\end{figure}

 


The differences between the the total required time per session is not very different for students and professionals (see \tabref{tab:key:7:2}) and do not show significance for the whole task set ($t=-0.3$, $p=0.7705$) or for the single tasks (TfS – $t=-0.04$, $p=0.9634$; PE – $t=-0.61$, $p=0.5504$; MPE – $t=-0.17$, $p=0.8682$). There is also no correlation between experience\footnote{Find further information on the experience vector in \sectref{sec:8:1}.} and time neither for the complete data set ($r=-0.117$, $p=0.334$) nor the single tasks (TfS – $r=-0.068$, $p=0.7506$; PE – $r=-0.162$, $p=0.4596$, MPE – $r=-0.312$, $p=0.1464$).


\begin{table}
\begin{tabular}{l *{4}{S[table-format=3.1]}}
\lsptoprule
 & \multicolumn{2}{c}{Professionals} & \multicolumn{2}{c}{Students}\\\cmidrule(lr){2-3}\cmidrule(lr){4-5}
 & \multicolumn{1}{c}{Mean} & \multicolumn{1}{c}{SD} & \multicolumn{1}{c}{Mean} & \multicolumn{1}{c}{SD}\\\midrule
 TfS & 986.9 & 276.4 & 991.9 & 252.3\\
 PE & 720.4 & 212.4 & 773.5 & 205.9\\
 MPE & 661.3 & 236.6 & 675.9 & 170.5\\
\lspbottomrule
\end{tabular}
\caption{Mean and standard deviation for the duration of the sessions according to task and status}
\label{tab:key:7:2}
\end{table}


The result that TfS takes more time than the PE tasks was to be expected. Saving time is supposed to be one of the main benefits of PE, which was indeed achieved in this experiment. Interestingly, MPE saves a significant amount of time, which may have been expected as the participants do not have to process the source text. Further, it is quite surprising that there is no significant time difference between students and professionals in all tasks. A hypothesis could have been that professionals are significantly faster at least for the TfS task, because they are more experienced in this task. One reason why there is no significant difference might be that professionals are unfamiliar with the text type, because they do not have to deal with general language texts in their professional life. Another explanation might be that the professional translators are more careful with their translations\slash post-edits and hence spent more time on revising.


\section{Complexity levels of the texts}
\label{sec:7:6}

Texts, even of the same text types, have different \isi{complexity levels}. Therefore, the assumption is that the more complex a text is, the harder the text is to translate and the more problems may occur that need to be overcome. The following table (\tabref{tab:key:7:3}) introduces the complexity of the six texts used for the experiments. The higher the score of a text the more complex the text, except for the Flesch reading ease score, where it is the other way around.

\begin{table}
\begin{tabular}{l *{6}{S[table-format=2.1]}}
\lsptoprule
 & \multicolumn{6}{c}{Text \#}\\\cmidrule(lr){2-7}
 & \multicolumn{1}{c}{1} & \multicolumn{1}{c}{2} & \multicolumn{1}{c}{3} & \multicolumn{1}{c}{4} & \multicolumn{1}{c}{5} & \multicolumn{1}{c}{6}\\
 \midrule 
 \isi{Flesch reading ease score} & 79.2 & 57.9 & 38.1 & 42.7 & 48.6 & 29.7\\
 \isi{Automated readability index} & 7.6 & 13.7 & 20.5 & 14.9 & 14.1 & 13.8\\
 \isi{Flesch-Kincaid grade level} & 5.6 & 10.9 & 16.1 & 12.3 & 11.5 & 13.1\\
 \isi{Coleman-Liau index} & 9.2 & 12.3 & 15.1 & 15.6 & 14.6 & 16.4\\
 \isi{Gunning fog index} & 8.9 & 14.7 & 20.2 & 16.4 & 15.3 & 16.8\\
 \isi{SMOG index} & 8.7 & 12.7 & 16.1 & 14.2 & 13.5 & 14.1\\
 Mean value\footnote{Excluding Flesch reading ease score} & 8 & 12.9 & 17.6 & 14.7 & 13.8 & 14.8\\
\lspbottomrule
\end{tabular}
% % \footnotetext{Excluding Flesch reading ease score}
%%please move \begin{table} just above \begin{tabular
\caption{Complexity levels according to different test scores}
\label{tab:key:7:3}
\end{table}


{The scores from \tabref{tab:key:7:3}\footnote{Calculated on \url{http://www.editcentral.com/gwt1/EditCentral.html} (last accessed on 18 November 2014).} show that the complexity level rises from Text 1 to Text 3, which was also stated in} %\label{ref:ZOTEROITEMCSLCITATIONcitationIDnRVjnf3upropertiesformattedCitationHvelplund2011plainCitationHvelplund2011citationItemsid164urishttpzoteroorgusers1255332itemsA5SKCFC4urihttpzoteroorgusers1255332itemsA5SKCFC4itemDataid164typebooktitleAllocationofcognitiveresourcesintranslationAneyetrackingandkeyloggingstudypublisherCopenhagenBusinessSchoolPhDSeriespublisherplaceCopenhagenDenmarkeventplaceCopenhagenDenmarkauthorfamilyHvelplundgivenKristianTangsgaardissueddateparts2011schemahttpsgithubcomcitationstylelanguageschemarawmastercslcitationjsonRNDcZDJt1beiW}
{\citet[88--93]{Hvelplund2011}}{, who did not only take reading scores into consideration, but also word frequencies, and the amount of non-literal expressions. Starting with Text 4, the picture is not that clear. According to the Flesh reading ease score, Text 4 and 5 would be less complex than Text 3, but more complex than Text 2; and Text 6 would be the most complex. Some of the other scores contradict the picture. The Colman-Liau index indicates that Text 4 and 6 are more complex than Text 3, while other scores state that Text 3 is the most complex. Therefore, the mean value of the scores was calculated to summarise the results of the single scores, excluding the Flesch reading ease score, because it represents complexity with low numbers and in general the numbers are much higher than for the other scores. The mean of the scores rank the texts in the following order (least complex to most complex): Text~1, Text~2, Text~5, Text~4, Text~6, Text~3 – see \tabref{tab:key:7:3}. Note that the difference between Text 4 and Text 6 is very low.}


\section{General keystroke effort for modifications}
\label{sec:7:7}

This chapter presents the number of tokens that were inserted and deleted in the different tasks, first for all participants, then for professionals and semi-profes\-sion\-als separately. The data were automatically generated in \isi{Translog II}. See also %\label{ref:ZOTEROITEMCSLCITATIONcitationIDLY94ik3BpropertiesformattedCitationCarletal2011plainCitationCarletal2011citationItemsid20urishttpzoteroorgusers1255332itemsMPWXCXVRurihttpzoteroorgusers1255332itemsMPWXCXVRitemDataid20typechaptertitleTheprocessofposteditingapilotstudycontainertitleCopenhagenStudiesinLanguage41publisherplaceCopenhagenDenmarkpage131142eventplaceCopenhagenDenmarkauthorfamilyCarlgivenMichaelfamilyDragstedgivenBarbarafamilyElminggivenJacobfamilyHardtgivenDanielfamilyJakobsengivenArntLykkeissueddateparts2011schemahttpsgithubcomcitationstylelanguageschemarawmastercslcitationjsonRNDF8tlyb4SWi}
\citet[132--133]{CarlEtAl2011} on keylogging data evaluation for the first three texts.


%%please move \begin{table} just above \begin{tabular
{\tabref{tab:key:7:4} displays the \isi{total token count} (\textit{Token}) for all participants, the percentage of the insertions\slash deletions in relation to the total modification count for the single tasks (\textit{Percent}), the mean (\textit{Mean),} the \isi{standard deviation} \textit{(SD)} and median (\textit{Median}) values for all participants, and finally the highest (\textit{Max Token}) and lowest (\textit{Min Token}) number of tokens that occurred in all the data. The corresponding participant is listed in the last two rows indicating the maximum and minimum token.

\begin{table}
\resizebox{\textwidth}{!}{\begin{tabular}{l *{6}{S[table-format=5.1,separate-uncertainty=true,table-space-text-post=(P22),group-digits=false]}}
\lsptoprule
 & \multicolumn{2}{c}{MPE (42)} & \multicolumn{2}{c}{PE (41)} & \multicolumn{2}{c}{TfS (45)}\\\cmidrule(lr){2-3}\cmidrule(lr){4-5}\cmidrule(lr){6-7}
 & \multicolumn{1}{c}{Insertions } & \multicolumn{1}{c}{Deletions} & \multicolumn{1}{c}{Insertions} & \multicolumn{1}{c}{Deletions} & \multicolumn{1}{c}{Insertions} & \multicolumn{1}{c}{Deletions}\\
 \midrule 
 Token & 13450 & 12401 & 14913 & 13888 & 52509 & 9468\\
 Percent & 52 & 48 & 51.8 & 48.2 & 84.7 & 15.3\\
 Mean & 305.7 & 281.8 & 346.8 & 323 & 1117.2 & 201.4\\
 SD & 188.9 & 176.1 & 170.7 & 168.9 & 200.6 & 121.8\\
 Median & 273 & 258 & 286 & 265 & 1097 & 175\\
 Max Token & 1159 (P22) & 1057 (P22) & 956 (P22) & 972 (P22) & 1695 (P12) & 598 (P22)\\
 Min Token & 92 (P1) & 61 (P1) & 138 (P13) & 121 (P23) & 750 (P12) & 32 (P18)\\
\lspbottomrule
\end{tabular}}
%%please move \begin{table} just above \begin{tabular
\caption{Total token count}
\label{tab:key:7:4}
\end{table}


MT produces a full target text, although it is often very error prone. Accordingly, it seems obvious that the number of inserted tokens is quite similar to the number of deleted tokens for MPE and PE. The defective target text segments are replaced by (expectedly) improved target text segments. However, it is interesting that in both tasks a few more tokens were inserted than deleted. In total, 68.2\% of the participants inserted more tokens than they deleted in MPE, similarly 67.4\% added tokens in PE. On average, the texts were 23.4 tokens longer after MPE and 23.8 tokens longer after PE. One reason might be that \ili{German} translations are in general longer than the corresponding \ili{English} source texts. The MT output, however, is often very literal. Therefore, it might be necessary to expand some text segments. Further, more insertions and deletions were made on average in the PE task compared to the MPE task, which might already indicate that the participants could recognise flaws in the MT output more easily and effectively when they were able to refer to the source text.


\largerpage
The nature of TfS explains that the number of insertions is much higher than the number of deletions. In contrast to the PE tasks, a target text has to be produced first. However, the number of deleted tokens is quite high, especially when we take into consideration that the number of inserted tokens includes the tokens that were deleted again at some point later in the session. This proves that human translators do not produce fully intact translations right away, but they may correct spelling mistakes as well as reconsider their initial decision. Of course the roots of the mistakes are very different, but a perfect translation does not come naturally, neither for the human nor for the machine. Finally, changes to the initial version and the final version are not only visible in the deletions but are also hidden in the insertions. Rephrasing does not necessarily mean deleting tokens and typing new tokens, but texts can be improved by insertions. One case could be that content was forgotten in the first text creation (unlikely with MT).



However, another very important characteristic of fluent texts, especially in \ili{German}, is the use of particles. One example can be found in P18\_T6 (participant~18 is the one with the lowest deletion number): The participant turned the sentence “Diese Gesellschaften sind folgerichtig oft nicht reich bevölkert”, which is a correct sentence grammatically and in terms of content, into “Diese Gesellschaften sind folgerichtig oft\textbf{mals} \textbf{auch} nicht reich bevölkert”,\footnote{Highlighted by the author.} which is an equally grammatically correct sentence with the same content. \textit{Oft} and \textit{oftmals} can be used synonymously, while \textit{auch} emphasises the relation between this sentence and the previous sentence. This participant apparently thought that the text would be more fluent, if these particles were included, although the PE guidelines implied that stylistic improvements should not be made (see \sectref{sec:7:3}).



The amount of changes in the MPE and PE tasks are impressive: While some participants only needed a few tokens to adjust the MT (92 insertions and 61 deletions in MPE and 138 insertions and 141 deletions in PE), the participant who changed the most in one MPE task (1159 inserted tokens and 1057 deleted tokens) has a higher modification rate than required for the average translation (1117.2 inserted tokens and 201.4 deleted tokens) and almost as high a number for PE (956 inserted tokens and 972 deleted tokens).\footnote{According to the numbers, one of the MPE sessions and the PE session of participant 22 would qualify as outliers. However, as no technical reasons could explain the data, the sessions are not excluded from the data.}


 \largerpage
{\tabref{tab:key:7:5} and \tabref{tab:key:7:6} divide the presented numbers into insertions and deletions for professional and semi-professional translators. The differences between both groups are not very great. As can be seen in \tabref{tab:key:7:7}, there is no significant differences between students and professionals, and no significant correlation between experience and keylogging behaviour. What catches the eye are the differences} {between max. token and min. token, which are not as major for the semi-}{professionals as those of the professionals. However, most of the max. token in the professional group were produced by one participant, which might suggest extensive editing behaviour of this participant.}
\clearpage 

\begin{table}[p]
\resizebox{\textwidth}{!}{\begin{tabular}{l *{6}{S[table-format=5.1,separate-uncertainty=true,table-space-text-post=(P22),group-digits=false]}}
\lsptoprule
 & \multicolumn{2}{c}{MPE (21)} & \multicolumn{2}{c}{ PE (19)} & \multicolumn{2}{c}{TfS (23)}\\\cmidrule(lr){2-3}\cmidrule(lr){4-5}\cmidrule(lr){6-7}
 & \multicolumn{1}{c}{Insertions } & \multicolumn{1}{c}{Deletions} & \multicolumn{1}{c}{Insertions} & \multicolumn{1}{c}{Deletions} & \multicolumn{1}{c}{Insertions} & \multicolumn{1}{c}{Deletions}\\
 \midrule 
 Token & 6942 & 6543 & 7326 & 6907 & 26935 & 5233\\
 Per cent & 51.5 & 48.5 & 51.5 & 48.5 & 83.7 & 16.3\\
 Mean & 330.6 & 311.6 & 385.6 & 363.5 & 1171.1 & 227.5\\
 Median & 259 & 266 & 329 & 284 & 1141 & 184\\
 SD & 252.4 & 233.3 & 200.8 & 202.4 & 175 & 134\\
 Max Token & 1159 (P22) & 1057 (P22) & 956 (P22) & 972 (P22) & 1632 (P07) & 598 (P22)\\
 Min Token & 92 (P01) & 61 (P01) & 138 (P13) & 121 (P23) & 882 (P21) & 69 (P19)\\
\lspbottomrule
\end{tabular}}
%%please move \begin{table} just above \begin{tabular
\caption{Total token count of professional translators}
\label{tab:key:7:5}
\end{table}

\begin{table}[p]
\resizebox{\textwidth}{!}{\begin{tabular}{l *{6}{S[table-format=5.1,separate-uncertainty=true,table-space-text-post=(P22),group-digits=false]}}
\lsptoprule
 & \multicolumn{2}{c}{ MPE (23)} & \multicolumn{2}{c}{ PE (24)} & \multicolumn{2}{c}{ TfS (24)}\\\cmidrule(lr){2-3}\cmidrule(lr){4-5}\cmidrule(lr){6-7}
& \multicolumn{1}{c}{Insertions } & \multicolumn{1}{c}{Deletions} & \multicolumn{1}{c}{Insertions} & \multicolumn{1}{c}{Deletions} & \multicolumn{1}{c}{Insertions} & \multicolumn{1}{c}{Deletions}\\\midrule 
 Token & 6508 & 5858 & 7587 & 6981 & 25574 & 4235\\
 Per cent & 52.6 & 47.4 & 52.1 & 47.9 & 85.8 & 14.2\\
 Mean & 283 & 254.7 & 316.1 & 290.9 & 1065.6 & 176.5\\
 Median & 276 & 249 & 268 & 254 & 1044 & 142\\
 SD & 103.7 & 97.5 & 139.5 & 132.5 & 213.3 & 105.7\\
 Max Token & 564 (P18) & 537 (P18) & 723 (P09) & 598 (P09) & 1695 (P12) & 546 (P12)\\
 Min Token & 104 (P4) & 101 (P20) & 138 (P14) & 123 (P14) & 750 (P12) & 32 (P18)\\
\lspbottomrule
\end{tabular}}
%%please move \begin{table} just above \begin{tabular
\caption{Total token count of semi-professional translators}
\label{tab:key:7:6}
\end{table}

\begin{table}[p]
\small
\begin{tabular}{l *{2}{S[table-format=4.1]S[table-format=1.4]}}
\lsptoprule
& \multicolumn{4}{c}{Mann-Whitney-U-test}\\
& \multicolumn{2}{c}{Insertions} & \multicolumn{2}{c}{Deletions}\\\cmidrule(lr){2-3}\cmidrule(lr){4-5}
& W & p & W & p\\\midrule
Total &  2643.5 &     0.3302  &  2560  &  0.5341\\
MPE   &  250.5  &     0.775   &  258   &  0.9037\\
PE    &  327    &     0.1693  &  277.5 &  0.7749\\
HT    &  365    &     0.05964 &  342.5 &  0.1601\\\midrule
\end{tabular}
\begin{tabular}{l *{2}{S[table-format=1.4]S[table-format=1.4]}}
& \multicolumn{4}{c}{Correlation with Experience}\\
& \multicolumn{2}{c}{Insertions} & \multicolumn{2}{c}{Deletions}\\\cmidrule(lr){2-3}\cmidrule(lr){4-5}
& r & p & r & p\\\midrule
Total & 0.067 & 0.2527 & 0.086 & 0.1493\\
MPE   & 0.126 & 0.2304 & 0.138 & 0.189\\
PE    & 0.125 & 0.2341 & 0.015 & 0.8864\\
HT    & 0.085 & 0.412  & 0.155 & 0.1353\\
\lspbottomrule
\end{tabular}
\caption{Significance tests for keylogging data}
\label{tab:key:7:7}
\end{table}

\clearpage   
\section{{ General} analysis of errors in the final texts}
\label{sec:7:8}

Before analysing the problem solving strategies, the translation product will be examined for errors in this chapter to assess the quality of the TfS and PE tasks. Quality assessments of translations have been widely discussed in translation studies, e.g. %\label{ref:ZOTEROITEMCSLCITATIONcitationID5bojDPYhpropertiesformattedCitationHouse1997plainCitationHouse1997citationItemsid119urishttpzoteroorgusers1255332items6E5HZKSIurihttpzoteroorgusers1255332items6E5HZKSIitemDataid119typebooktitleTranslationqualityassessmentAmodelrevisitedpublisherGunterNarrVerlagpublisherplaceTbingenvolume410eventplaceTbingenauthorfamilyHousegivenJulianeissueddateparts1997schemahttpsgithubcomcitationstylelanguageschemarawmastercslcitationjsonRNDNM3TimA2Ut}
\citet{House1997}, %\label{ref:ZOTEROITEMCSLCITATIONcitationIDLM2MsebapropertiesformattedCitationrtfHuc0u246nig1997plainCitationHnig1997citationItemsid187urishttpzoteroorgusers1255332itemsUEC63SHEurihttpzoteroorgusers1255332itemsUEC63SHEitemDataid187typearticlejournaltitlePositionspowerandpracticeFunctionalistapproachesandtranslationqualityassessmentcontainertitleCurrentIssuesinLanguageSocietypage634volume4issue1authorfamilyHniggivenHansGissueddateparts1997schemahttpsgithubcomcitationstylelanguageschemarawmastercslcitationjsonRNDLcG3khLE8I}
\citet{Honig1997}, %\label{ref:ZOTEROITEMCSLCITATIONcitationIDvsj6CnvdpropertiesformattedCitationMertin2006plainCitationMertin2006citationItemsid1441urishttpzoteroorggroups3587itemsAK86K2GIurihttpzoteroorggroups3587itemsAK86K2GIitemDataid1441typebooktitleProzessorientiertesQualittsmanagementimDienstleistungsbereichbersetzencollectiontitleLeipzigerStudienzurangewandtenLinguistikundTranslatologiepublisherLangpublisherplaceFrankfurtaMGermanyvolume2eventplaceFrankfurtaMGermanyISBN3631558597authorfamilyMertingivenElviraissueddateparts2006schemahttpsgithubcomcitationstylelanguageschemarawmastercslcitationjsonRNDaHQxbwiUJA}
\citet{Mertin2006} or %\label{ref:ZOTEROITEMCSLCITATIONcitationIDhAqqxtXrpropertiesformattedCitationReiss2014plainCitationReiss2014citationItemsid82urishttpzoteroorgusers1255332itemsVWT7TTRZurihttpzoteroorgusers1255332itemsVWT7TTRZitemDataid82typebooktitleTranslationCriticismPotentialsandLimitationsCategoriesandCriteriaforTranslationQualityAssessmentpublisherRoutledgepublisherplaceLondonNewYorkeventplaceLondonNewYorkauthorfamilyReissgivenKatharinaissueddateparts2014schemahttpsgithubcomcitationstylelanguageschemarawmastercslcitationjsonRNDUKLrmUicg1}
\citet{Reiss2014}, and some studies were published on the quality of MT, too, e.g. %\label{ref:ZOTEROITEMCSLCITATIONcitationIDaqPd10DspropertiesformattedCitationrtfFiedererandOuc0u8217Brien2009plainCitationFiedererandOBrien2009citationItemsid186urishttpzoteroorgusers1255332items38GM8QSPurihttpzoteroorgusers1255332items38GM8QSPitemDataid186typearticlejournaltitleQualityandmachinetranslationArealisticobjectivecontainertitleJournalofSpecialisedTranslationpage5274volume11authorfamilyFiederergivenRebeccafamilyOBriengivenSharonissueddateparts2009schemahttpsgithubcomcitationstylelanguageschemarawmastercslcitationjsonRNDKdyA0D4eQt}
\citet{FiedererO’Brien2009} or %\label{ref:ZOTEROITEMCSLCITATIONcitationIDxoUV89WWpropertiesformattedCitationLacruzDenkowskiandLavie2014plainCitationLacruzDenkowskiandLavie2014citationItemsid202urishttpzoteroorgusers1255332items4QGM2U2Zurihttpzoteroorgusers1255332items4QGM2U2ZitemDataid202typepaperconferencetitleCognitiveDemandandCognitiveEffortinPostEditingcontainertitleProceedingsoftheThirdWorkshoponPostEditingTechnologyandPracticepublisherAMTApage7384authorfamilyLacruzgivenIsabelfamilyDenkowskigivenMichaelfamilyLaviegivenAlonissueddateparts2014schemahttpsgithubcomcitationstylelanguageschemarawmastercslcitationjsonRND6QYsU0msBe}
\citet{LacruzEtAl2014}.\footnote{Furthermore, there are numerous publications on automatic MT output evaluation. Two of the most famous automatic matrices are BLEU (%\label{ref:ZOTEROITEMCSLCITATIONcitationIDquTjnkOJpropertiesformattedCitationPapinenietal2002plainCitationPapinenietal2002citationItemsid255urishttpzoteroorgusers1255332items2K7FS8J3urihttpzoteroorgusers1255332items2K7FS8J3itemDataid255typepaperconferencetitleBLEUamethodforautomaticevaluationofmachinetranslationcontainertitleProceedingsofthe40thannualmeetingonassociationforcomputationallinguisticspublisherAssociationforComputationalLinguisticspage311318authorfamilyPapinenigivenKishorefamilyRoukosgivenSalimfamilyWardgivenToddfamilyZhugivenWeiJingissueddateparts2002schemahttpsgithubcomcitationstylelanguageschemarawmastercslcitationjsonRNDHqqr0a8EzC}
\citealt{PapineniEtAl2002}) and METEOR %\label{ref:ZOTEROITEMCSLCITATIONcitationIDuHIPklDzpropertiesformattedCitationLavieandDenkowski2009plainCitationLavieandDenkowski2009citationItemsid185urishttpzoteroorgusers1255332itemsS2TUE4K7urihttpzoteroorgusers1255332itemsS2TUE4K7itemDataid185typearticlejournaltitleTheMETEORmetricforautomaticevaluationofmachinetranslationcontainertitleMachinetranslationpage105115volume23issue23authorfamilyLaviegivenAlonfamilyDenkowskigivenMichaelJissueddateparts2009schemahttpsgithubcomcitationstylelanguageschemarawmastercslcitationjsonRNDBUeuU4fq1H}
(\citealt{LavieDenkowski2009}). The MQM framework has been developed to evaluate both human translation and MT %\label{ref:ZOTEROITEMCSLCITATIONcitationID7UY3hktSpropertiesformattedCitationLommelUszkoreitandBurchardt2014plainCitationLommelUszkoreitandBurchardt2014citationItemsid4050urishttpzoteroorgusers1255332itemsP6KZU65Yurihttpzoteroorgusers1255332itemsP6KZU65YitemDataid4050typearticlejournaltitleMultidimensionalQualityMetricsMQMcontainertitleTradumticapage455463issue12authorfamilyLommelgivenArlefamilyUszkoreitgivenHansfamilyBurchardtgivenAljoschaissueddateparts2014schemahttpsgithubcomcitationstylelanguageschemarawmastercslcitationjsonRNDhhNy4ZKduI}
\citep{LommelEtAl2014}} However, too complex models and typical \isi{error categories} cannot be applied in this study, because the PE guidelines (see \sectref{sec:7:3}) stated clearly that not all linguistic aspects are important for the final target text; i.e. stylistic characteristics cannot be considered while analysing the target texts when the PE guidelines dictate that style is to be neglected.



To explain the nature of errors, Rasmussen's \isi{skill-rule-knowledge framework} differentiates three levels (cf. %\label{ref:ZOTEROITEMCSLCITATIONcitationIDHHxNInzdpropertiesformattedCitationReason1990plainCitationReason1990citationItemsid248urishttpzoteroorgusers1255332itemsCCN99H38urihttpzoteroorgusers1255332itemsCCN99H38itemDataid248typebooktitleHumanerrorpublisherCambridgeUniversityPresspublisherplaceCambridgeEnglandNewYorknumberofpages302sourceLibraryofCongressISBNeventplaceCambridgeEnglandNewYorkISBN0521306698callnumberBF323E7R421990authorfamilyReasongivenJTissueddateparts1990schemahttpsgithubcomcitationstylelanguageschemarawmastercslcitationjsonRNDceuAWRXRVG}
\citealt{Reason1990}: 42-44, %\label{ref:ZOTEROITEMCSLCITATIONcitationIDgWGQCUWqpropertiesformattedCitationJungermannPfisterandFischer2010plainCitationJungermannPfisterandFischer2010dontUpdatetruecitationItemsid109urishttpzoteroorgusers1255332itemsNPNAU6BFurihttpzoteroorgusers1255332itemsNPNAU6BFitemDataid109typebooktitleDiePsychologiederEntscheidungeineEinfhrungpublisherSpektrumAkadVerlpublisherplaceHeidelbergnumberofpages481edition3korrAuflsourceGemeinsamerBibliotheksverbundISBNeventplaceHeidelbergISBN9783827423863shortTitleDiePsychologiederEntscheidunglanguagegerauthorfamilyJungermanngivenHelmutfamilyPfistergivenHansRdigerfamilyFischergivenKatrinissueddateparts2010schemahttpsgithubcomcitationstylelanguageschemarawmastercslcitationjsonRNDjVuQE3VbV3}
\citealt{JungermannEtAl2010}: 38-40): skill-based, rule-based, and knowledge-based errors. The skill-based level describes errors that happen in every day situations by accident. Usually, the person knows what (s)he is doing and intends to do it correctly, but accidentally makes a mistake. In translation, a typical example would be typing errors. A set of rules for the situation is known to the person on the rule-based level, but (s)he applies those rules incorrectly. A translation-related example of rule-based errors would be the wrong application of grammar rules. Finally, errors on a knowledge-based level occur, when the situation is new to the person and (s)he does not have a predetermined set of rules to cope with the situation. This might happen in translation, when the translator chooses a wrong term in the context, because (s)he does not know better. As the examples showed, all error levels can be detected in translation situations. Therefore, they will be applied to the error categories later on.



In %\label{ref:ZOTEROITEMCSLCITATIONcitationID1iFab2dSpropertiesformattedCitationrtfSchuc0u228fer2003plainCitationSchfer2003citationItemsid37urishttpzoteroorgusers1255332itemsV4G72ZKTurihttpzoteroorgusers1255332itemsV4G72ZKTitemDataid37typepaperconferencetitleMTposteditinghowtoshedlightontheunknowntaskExperiencesatSAPcontainertitleControlledlanguagetranslationpublisherplaceDublinIrelandpage133140eventEAMTCLAW03eventplaceDublinIrelandauthorfamilySchfergivenFalkoissueddateparts2003schemahttpsgithubcomcitationstylelanguageschemarawmastercslcitationjsonRNDrg9rhEiVwZ}
\citet{Schafer2003}, error categories are introduced that were established at SAP AG to develop a standard PE guide. At the time the paper was published, four different MT systems were used at SAP AG for different languages. The guide is intended to help the translators with the new task of PE as well as to encourage the translators to keep an open-mind towards MT and should be applicable for the output and workflow of all four MT systems. The four error categories introduced are: \textit{lexical errors}, \textit{syntactic errors}, \textit{grammatical mistakes}, and \textit{mistakes due to defective source texts}. The latter did not occur in the six source texts in this study, which is to be expected in an experimental setting. However, this is an important aspect when technical texts are translated in practice (cf. %\label{ref:ZOTEROITEMCSLCITATIONcitationIDW47LkQMwpropertiesformattedCitationHornHelf1999plainCitationHornHelf1999citationItemsid121urishttpzoteroorgusers1255332itemsF8ICVJGTurihttpzoteroorgusers1255332itemsF8ICVJGTitemDataid121typebooktitleTechnischesbersetzeninTheorieundPraxispublisherFranckepublisherplaceTbingenBaseleventplaceTbingenBaselauthorfamilyHornHelfgivenBrigitteissueddateparts1999schemahttpsgithubcomcitationstylelanguageschemarawmastercslcitationjsonRNDSZzgmF95IV}
\citealt{Horn-helf1999,Horn-helf2007}) %\label{ref:ZOTEROITEMCSLCITATIONcitationIDDEhxbwappropertiesformattedCitationHornHelf2007plainCitationHornHelf2007citationItemsid120urishttpzoteroorgusers1255332itemsAU6QBFKCurihttpzoteroorgusers1255332itemsAU6QBFKCitemDataid120typebooktitleKulturdifferenzinFachtextsortenkonventionenAnalyseundTranslationeinLehrundArbeitsbuchcollectiontitleLeipzigerStudienzurangewandtenLinguistikundTranslatologiepublisherPeterLangpublisherplaceFrankfurtaMvolume4eventplaceFrankfurtaMauthorfamilyHornHelfgivenBrigitteissueddateparts2007schemahttpsgithubcomcitationstylelanguageschemarawmastercslcitationjsonRNDvRkORuxtSO}
. Another aspect is that MT output was analysed in %\label{ref:ZOTEROITEMCSLCITATIONcitationIDDxSrqKVkpropertiesformattedCitationrtfSchuc0u228fer2003plainCitationSchfer2003citationItemsid37urishttpzoteroorgusers1255332itemsV4G72ZKTurihttpzoteroorgusers1255332itemsV4G72ZKTitemDataid37typepaperconferencetitleMTposteditinghowtoshedlightontheunknowntaskExperiencesatSAPcontainertitleControlledlanguagetranslationpublisherplaceDublinIrelandpage133140eventEAMTCLAW03eventplaceDublinIrelandauthorfamilySchfergivenFalkoissueddateparts2003schemahttpsgithubcomcitationstylelanguageschemarawmastercslcitationjsonRNDCk79rDcegd}
Schäfer's study, while the study at hand will focus on mistakes in the final target texts. Therefore, syntactic mistakes were not included, because they appear less often in the post-edited text than in MT output and some syntactic structures could instead be categorised as bad style, which is not included in this analysis.



Further, %\label{ref:ZOTEROITEMCSLCITATIONcitationIDGrJUMnAlpropertiesformattedCitationMertin2006plainCitationMertin2006citationItemsid1441urishttpzoteroorggroups3587itemsAK86K2GIurihttpzoteroorggroups3587itemsAK86K2GIitemDataid1441typebooktitleProzessorientiertesQualittsmanagementimDienstleistungsbereichbersetzencollectiontitleLeipzigerStudienzurangewandtenLinguistikundTranslatologiepublisherLangpublisherplaceFrankfurtaMGermanyvolume2eventplaceFrankfurtaMGermanyISBN3631558597authorfamilyMertingivenElviraissueddateparts2006schemahttpsgithubcomcitationstylelanguageschemarawmastercslcitationjsonRNDmhN8XpG6AN}
\citet[232--258]{Mertin2006} error categories were consulted, as well. Due to the inclusion of the PE and MPE task as well as the experiment setting, most translation relevant criteria, all reference relevant, formal, and job specific criteria had to be excluded. However, all criteria concerning language rules were included: Spelling mistakes and typos were condensed into one category (spelling), punctuation was extended for the category “spaces”, and grammar was considered. Only two categories could be included from the translation relevant criteria, namely content mistakes and word mistakes, which were summarised in the category “lexical mistakes”.


\largerpage[-2]
To put it in a nutshell, this chapter will only focus on superficial error categories to make the analysis more objective and adaptable for both tasks.\footnote{The MPE task will be excluded from this analysis, because the nature of the tasks requires assessments on a content level, where most or the more severe mistakes are expected. However, more details can be found in \citet{Nitzke2016mono}.} This also means that the error analysis is incomplete. The following error categories were established: spelling, grammar, punctuation, spaces, and lexical mistakes. Other common categories like style and collocations were not included, because they are usually, at least to some degree, subjective. Further, in the PE instructions the participants were asked to correct only the most important mistakes, keep as much of the MT output as possible, and disregard style and personal preferences. Finally, those mistakes can be detected without consulting the source text which makes the counting process faster. Unfortunately, this procedure had the side effect that content mistakes could not be included.\footnote{While assessing the texts, hardly any content-based mistakes became obvious. If content had appeared to be a serious problem in these two tasks, the source text and the error category would have been included. Despite the familiarity with the texts, some content mistakes could be found nonetheless, e.g.:  In Text 3, the source texts says “which includes one minister charged with crimes against humanity”, which was realised by one participant as “zu der auch ein Minister […] mit der Einhaltung der Menschenrechte beauftragt wurde”. Next to the misuse of “zu” or the missing verb (could be both and was counted as one lexical mistake) the content is not correct (“to which one minister […] was mandated to adhere to the human rights”).} However, an elaboration on content mistakes in the \ili{English}-\ili{German} data set can be found in %\label{ref:ZOTEROITEMCSLCITATIONcitationIDgqeTM1CRpropertiesformattedCitationNitzke2016monoplainCitationNitzke2016monodontUpdatetruecitationItemsid183urishttpzoteroorgusers1255332itemsUPJZPQ8Rurihttpzoteroorgusers1255332itemsUPJZPQ8RitemDataid183typechaptertitleMonolingualposteditingAnexploratorystudyonresearchbehaviourandtargettextqualitycontainertitleEyetrackingandAppliedLinguisticspublisherLanguageSciencePresspublisherplaceBerlinpage83108eventplaceBerlinauthorfamilyNitzkegivenJeaneditorfamilyHansenSchirragivenSilviafamilyGruzcagivenSamborissueddateparts2016schemahttpsgithubcomcitationstylelanguageschemarawmastercslcitationjsonRNDwf11SkPnLW}
\citet{Nitzke2016mono}, where there is also more information on the mistakes in the MPE texts. The latter produced many more content mistakes than the other two tasks, while the non-content related mistakes presented in this chapter occurred almost equally often in MPE as in the other two tasks.



\isi{\textit{Spelling} mistakes} refer to typos in most cases. \isi{Translog~II} does not provide an automatic spell-checker as it is a component of most document and word processing tools as well as \isi{translation memory systems}. Therefore, most spelling mistakes may not have occurred if another software had been used. However, the repeated occurrence of spelling mistakes reflects on the fact that translators (and probably all other computer users, too) have become used to this kind of assistance and how challenging it is to find typos in self-produced texts. Most spelling mistakes should be skill-based errors, but might occasionally be knowledge-based as well (e.g. the correct writing of a new term is unknown).



The same applies to \isi{\textit{grammar} mistakes}. Most of these probably have their source in either typos or the reorganisation of the sentence structure. When not enough attention is paid during the latter, suffixes of grammatical cases or articles might survive the reorganisation that are wrong in the new syntactic structure. These mistakes occur more often in \ili{German} texts than in \ili{English} texts because of the more diverse grammatical inflection in \ili{German}. Grammar mistakes may either be on a skill-based or rule-based level. The knowledge-based level can be ruled out as all participants are \ili{German} native speakers and trained translators.



Most \isi{\textit{punctuation} mistakes} can be traced back to missing or too many commas, or missing hyphens etc. Mistakes concerning \textit{spaces} include two or more spaces where there should only have been one and missing spaces behind hyphens. For the same reason as mentioned above for grammar mistakes, punctuation and space mistakes should occur on a skill-based or rule-based level as well.



\isi{\textit{Lexical} mistakes} only concern errors that can be detected without consulting the source text in this analysis. This means that the translator may have chosen the wrong lexical realisation and this mistake is not included in this category. Only mistakes that could be detected without the source texts were counted. That may include words that do not exist in \ili{German} or their meaning does not suit the context, e. g. compromise was translated as “kompromittieren” which means expose\slash denounce\slash put someone in a bad light etc. rather than impair or endanger. Other lexical mistakes were wrong realisations of the chosen words, e.g. the official abbreviation of “Jahrhundert” (century) is “Jh.” and not “Jhdt” in \ili{German}. Most lexical mistakes should either be skill-based or knowledge-based errors.


\newpage 
In all 92 sessions (45 PE and 47 TfS), 139 mistakes were counted (overall data set – mean: 1.51, sd: 1.32; TfS – mean: 1.47, sd: 1.28; PE – mean: 1.56, sd: 1.36), which were distributed as follows:


\begin{table}
\begin{tabularx}{\textwidth}{XXXlXX}
\lsptoprule
 mistake & spelling & grammar & punctuation & spaces & lexical\\
 \midrule 
 count & 51 & 29 & 23 & 14 & 22\\
 \% & 36.7\% & 20.9\% & 16.5\% & 10.1\% & 15.8\%\\
\lspbottomrule
\end{tabularx}
%%please move \begin{table} just above \begin{tabular
\caption{Mistake count}
\label{tab:key:7:8}
\end{table}


Of those mistakes, 70 were made in the PE, 69 in TfS; 81 by semi-professionals, 58 by professionals. However, professionals (44) completed fewer sessions than students (48). Therefore, the following table shows the mean values:


\begin{table}
\begin{tabularx}{\textwidth}{XXXXX}
\lsptoprule
 & TfS – mean & TfS – SD & PE – mean & PE – SD\\
 \midrule 
 professionals & 1.217 & 1.204 & 1.429 & 1.287\\
 students & 1.708 & 1.334 & 1.667 & 1.434\\
\lspbottomrule
\end{tabularx}
%%please move \begin{table} just above \begin{tabular
\caption{Mistakes (mean and SD) per text related to task and status}
\label{tab:key:7:9}
\end{table}


{The \isi{standard deviation} is quite high, which indicates that the results for the individual participants are very different and cannot be back-tracked to task or status of the participant. However, a correlation test shows that there is a statistically significant correlation between the experience factor}\footnote{See further information on the experience vector in \sectref{sec:8:1}} and the number of mistakes when the whole data set is taken into consideration. The correlation is negative and very weak ($r_\pi=-0.175$, $p=0.0294$), which means that the more experienced the {translator is, the (slightly) fewer mistakes (s)he makes}{. However, there is no significant correlation when the data are separated by tasks (TfS}\footnote{It seems plausible that the data for TfS might have reached significance if more participants had taken part in the experiment.} – $r_\pi=-0.212$, $p=0.0612$, PE – $r_\pi=-0.108$, $p=0.3504$). A Mann-Whitney-U-test showed that there is no significant connection when the data set is divided by groups ($W=877$, $p=0.1481$), nor between task and number of mistakes ($W=1092.5$, $p=0.78$), or between previous PE experience and the amount of mistakes ($W=1053.5$, $p=0.3573$).}


\section{Criticism of the data set}
\label{sec:7:9}

When we want to find out what happens in the translators mind, we want to keep the experiment situation as natural as possible to mirror translation behaviour that is as close to real-life behaviour as possible. On the other hand, experiments on cognitive processes need to be as controlled as possible to achieve generalisable results. These conflicting interests can hardly be united in translation process studies. Though the study at hand aimed to be as controlled and natural as possible, at the same time, some points in setup and conduction of the experiments can be criticised, which will be done in the scope of this chapter.



To create \isi{natural translation environments}, the keylogging system runs in the background so that the participant is not aware that his\slash her keystrokes are recorded. Furthermore, a desktop eyetracker is used in the experiments. Compared to a head-mounted eyetracker or an eyetracker with a chinrest, a desktop eyetracker does not physically influence the participant and allows him\slash her to move relatively freely.\footnote{{The area in which the participant can move is of course restricted, but translators tend to sit quite still when they work (see \sectref{sec:7:1:4}).}} However, there are still a lot of factors that change the work environment, like %\label{ref:ZOTEROITEMCSLCITATIONcitationIDLVv5kpbkpropertiesformattedCitationrtfOuc0u8217Brien2009plainCitationOBrien2009citationItemsidxhQX8r7SbRtqzaAdurishttpzoteroorgusers1255332items9WCPJ74Xurihttpzoteroorgusers1255332items9WCPJ74XitemDataidxhQX8r7SbRtqzaAdtypearticlejournaltitleEyetrackingintranslationprocessresearchmethodologicalchallengesandsolutionscontainertitleMethodologyTechnologyandInnovationinTranslationProcessResearchpage251266volume38authorfamilyOBriengivenSharonissuedyear2009pagefirst251containertitleshortMethodolTechnolInnovTranslProcessResschemahttpsgithubcomcitationstylelanguageschemarawmastercslcitationjsonRNDnAV6oIYkqx}
\citet{OBrien2009} already pointed out: the monitor is most likely different to the one translators use at home, the computer might be equipped with an unfamiliar operating system or different software (or different versions), and the non-identical keyboard might result in typing errors and the translation processes might be slowed down (at least until the participant has adjusted to the keyboard). These factors probably influence professional translators more than student translators, because the latter most likely do not have a working environment as fixed as the professional translators. Students may, e.g., be required to use computers at the university for particular courses. A completely natural work environment could only be possible if the participants participated in the experiments with their hard- and software at home or in their offices. This, however, is not possible with the eyetracking system, which needs to be installed and adjusted. On the other hand, it would be difficult to guarantee that all participants finished all texts in a row and under the same conditions.



Another very critical point in this experiments is the choice of the \isi{text type}. Professional translators hardly deal with newspaper articles – it is almost insignificant for the field. Most professionals specialise in a certain domain at the beginning or during their career (cf. %\label{ref:ZOTEROITEMCSLCITATIONcitationID24mT5tgTpropertiesformattedCitationrtfSchmittGerstmeyerandMuc0u252ller2016plainCitationSchmittGerstmeyerandMller2016citationItemsid163urishttpzoteroorgusers1255332itemsGFE8FPHRurihttpzoteroorgusers1255332itemsGFE8FPHRitemDataid163typebooktitlebersetzerundDolmetscherEineinternationaleUmfragezurBerufspraxispublisherBDFachverlagpublisherplaceBerlineventplaceBerlinauthorfamilySchmittgivenPeterAfamilyGerstmeyergivenLinafamilyMllergivenSarahissueddateparts2016schemahttpsgithubcomcitationstylelanguageschemarawmastercslcitationjsonRNDnqIMc2aprF}
\citealt{SchmittEtAl2016}), which might be technical, IT, economic, law, medical translations etc. and may have not translated newspaper articles for years or decades, if at all. In an online survey published by %\label{ref:ZOTEROITEMCSLCITATIONcitationIDZH7CFhk8propertiesformattedCitationrtfHommerichandReiuc0u2232011plainCitationHommerichandRei2011citationItemsid6urishttpzoteroorgusers1255332items536WSTZXurihttpzoteroorgusers1255332items536WSTZXitemDataid6typearticletitleErgebnissederBDMitgliederbefragungauthorfamilyHommerichgivenChristophfamilyReigivenNicoleissueddateparts20114schemahttpsgithubcomcitationstylelanguageschemarawmastercslcitationjsonRNDcmnxmoT32W}
Hommerich and Reiß (cf. 2011: 71), which was conducted on behalf of the BDÜ (see \sectref{sec:4:4}), the authors reported that 49\% of the members that participated in the study (in total 1570) specialised in the field “Industry and Technology (general)”, 45\% in “Law and Administration”, 41\% in “Economics, Trade, and Finances“, 25\% in “Medicine and Pharmacy”, and 23\% in “Information Technology”. Only few translators specialised in fields that might require the use of general language like “Culture and Education” (13\%), “Sports, Recreation, and Tourism” (10\%), or “Media and Art” (9\%), although most of these fields might require domain-specific language and terminology as well.\footnote{The follow-up study from 2016 presents very similar numbers.} However, student translators often have\slash had to translate newspaper articles in courses to gain general \isi{translation competence} in Germersheim.\footnote{{This approach is controversial from a didactics point of view, as well.}} Therefore, some of the statements about \isi{translation competence} that will follow may not be meaningful to a full extent, because we cannot judge the balance between familiarity with the text source and \isi{translation competence} (cf. %\label{ref:ZOTEROITEMCSLCITATIONcitationIDjYBHmJsupropertiesformattedCitationrtfOuc0u8217Brien2009plainCitationOBrien2009citationItemsidxhQX8r7SbRtqzaAdurishttpzoteroorgusers1255332items9WCPJ74Xurihttpzoteroorgusers1255332items9WCPJ74XitemDataidxhQX8r7SbRtqzaAdtypearticlejournaltitleEyetrackingintranslationprocessresearchmethodologicalchallengesandsolutionscontainertitleMethodologyTechnologyandInnovationinTranslationProcessResearchpage251266volume38authorfamilyOBriengivenSharonissuedyear2009pagefirst251containertitleshortMethodolTechnolInnovTranslProcessResschemahttpsgithubcomcitationstylelanguageschemarawmastercslcitationjsonRNDbdDaDMTFzx}
\citealt{OBrien2009}).



If we consider \isi{real life translation situations}, a \isi{MT system} other than Google Translate should have been chosen to prepare the source text for PE. A much higher quality could be achieved with MT systems that were trained with com\-pa\-ny-spe\-ci\-fic corpora, so the PE task would be much more efficient. However, the resources are very limited at a university. Therefore, it is not plausible to train a system for only one experiment. On the other hand, as was already discussed, newspaper articles are not a typical text type in professional translation and it is highly unlikely that a company would train a MT system to translate these text types. Google Translate might, therefore, be a quite sophisticated MT system for our purposes, because it is trained with all kinds of text from the world wide web.



{Last but not least, the data were collected to gather comparable translation process data for different language combinations that can be accessed for free via an online platform to tackle various research questions. Therefore, the data existed before the research question and the hypotheses were formulated. Methodologically, this could be judged as quite critically because the tasks were not precisely tailored to assess the research question and the conditions cannot be controlled. On the other hand, relatively natural texts were used; and due to the large scope of the experiments, the study at hand can be expanded to other languages in the future. Further, as was already presented in \sectref{sec:7:4}, many other studies have been conducted with the data set. Another aspect is that all participants started with the TfS task, followed by the PE task and finished with the MPE, which might have influenced the task behaviour, because the participants became tired towards the end of the experiment or first had to get used to the text type at the beginning of the experiment. Further, this might have influenced the answering behaviour in the retrospective questionnaires, which will be discussed in \sectref{sec:8:3} on flaws in the questionnaire.}


