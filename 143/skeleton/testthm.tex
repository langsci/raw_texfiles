% -*- coding: utf-8 -*-

La \kwo{conséquence logique}\indexs{consequence logique@conséquence logique} 
n'est peut-être pas la relation la plus spectaculaire, mais elle a le mérite d'être
solidement et objectivement définie, à l'aide de la logique (comme son nom l'indique).  Sa définition est la suivante :


\begin{defi}[Conséquence logique]%
\label{d:conseq}
\indexs{consequence logique@conséquence logique}
Une phrase $B$ est une conséquence logique d'une phrase $A$,
ssi\footnotemark\ \emph{à
chaque fois} que $A$ est vraie, alors $B$ est vraie aussi.
\end{defi}%
%\addtocounter{footnote}{-1}
\footnotetext{J'utiliserai cette abréviation pour \sicut{si et seulement si} tout au long de l'ouvrage.}%
%\addtocounter{footnote}{1}%


Si $B$ est une conséquence logique de $A$, nous dirons alors que $A$ \kwo{entraîne}\footnote{Le terme anglais usuel
est \emph{entail} et la conséquence logique est appelée \alien{entailment}.  En français on peut également dire «~$A$ implique $B$~», mais nous éviterons généralement de le faire ici car
nous réserverons cette terminologie pour un autre emploi.} $B$.\indexs{implication!\elid\ logique|see{conséquence logique}}



\begin{nota}
Si $A$ entraîne $B$, nous écrirons : $A\satisf B$.
\end{nota}

Voici quelques exemples en guise d'illustration.  Dans chaque groupe
de phrases, la phrase (a) entraîne chacune des phrases qui suit.  

\begin{exo}
Ceci est un exercice. Bla bla bla bla bla bla bla bla bla bla bla bla bla bla bla bla bla bla bla bla bla bla bla bla bla bla bla bla bla bla bla bla bla bla bla bla bla.
\end{exo}
