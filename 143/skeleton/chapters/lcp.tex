% -*- coding: utf-8 -*-
\chapter{Sémantique vériconditionnelle et calcul des prédicats}
%%%%%%%%%%%%%%%%%%%%%%%%%%%%%%%%%%%%%%%%%%%%%%%%%%%%%%%%%%%%%%%
\label{LCP}
\Writetofile{solf}{\protect\section{Chapitre \protect\ref{LCP}}}




\section{Introduction à la sémantique vériconditionnelle}
%========================================================
\is{semantique@sémantique!\elid\ vericonditionnelle@\elid\ vériconditionnelle|sqq}

\subsection{Sens,  dénotation et référence}
%-----------------------------------------
\label{ss:SuB}

Les phrases (ou énoncés) de la langue nous permettent de parler des
choses, du monde, c'est-à-dire de la réalité.  Elles nous permettent
assurément d'exprimer et de faire d'\emph{autres} choses, mais il semble
très difficile d'écarter cette vocation à commenter le réel.
Par conséquent  la sémantique doit être amenée à s'intéresser --~entre
autres~-- aux 
rapports entre les entités linguistiques (mots, syntagmes, phrases) et
les entités du monde (choses).  Il y a beaucoup de manières
(philosophiques ou cognitives, pratiques ou théoriques) d'envisager ces
rapports%%%%%%% 
\footnote{Sur cette  incontournable relation 
qui se tient entre la
langue, en tant que système, et ce qui lui est, d'une manière ou d'une
autre, extérieur,
et en particulier sur ce qu'il convient de concevoir
comme réalité extra-linguistique dans une perspective sémantique, je
ne peux que recommander la lecture de
\citet[chap.~1]{Kleiber:99}.\Andexn{Kleiber, G.}}.
%
 Nous allons nous en tenir
à une vision assez claire et très classique, initiée par
\textcite{Frege:SuB}\Andex{Frege, G.}%%%%%
\footnote{\citet{Frege:SuB,Frege:COfr} sont des
  lectures très abordables et incontournables. Les travaux de Frege
  peuvent être tenus pour le point de départ de la sémantique formelle
  (ainsi que de la logique moderne).} %%%%  
dans son célèbre article «~\alien{\"Uber {S}inn und {B}edeutung}~»
  (en français «Sens et dénotation»).


Partant d'une démarche avant tout philosophique, en cherchant à
asseoir solidement une théorie générale de la connaissance, 
Frege pose une distinction fondamentale entre d'une part le \kw{sens} 
et  d'autre part la
\kwa{dénotation}{denotation} d'une expression linguistique
--~les termes originaux employés par Frege sont
respectivement \emph{Sinn} et 
\emph{Bedeutung}.   Notons tout de suite que \emph{Bedeutung} est
parfois traduit par \kwa{référence}{reference} (ou \kwa{référent}{referent})%
\footnote{Il faut ajouter que dans le langage courant,
  c'est-à-dire hors du vocabulaire technique des linguistes,
  \alien{Bedeutung} est aussi traduit par \emph{signification}. Mais
  cette traduction n'est guère appropriée pour nommer la notion
  identifiée par Frege ; de plus nous essaierons ici de ne pas retenir
  \emph{signification} comme un terme technique, c'est-à-dire que
  nous ne lui attribuerons pas une définition précise et scientifique,
  et nous l'utiliserons dans son acception quotidienne.}, 
cela se présente parfois comme une variante terminologique libre ;
  cependant  nous 
ferons ici (dans les pages qui suivent) une distinction subtile entre 
référence et dénotation.
%ça change
%selon les auteurs, et certains apportent des nuances propres. 

\largerpage[-1]

\begin{defi}[Dénotation]
La \kwa{dénotation}{denotation} d'une expression linguistique est l'objet du monde que
cette expression désigne.  
\end{defi}
Cette définition est une première
approximation et nous la raffinerons plus tard.  Regardons tout de
suite l'exemple de Frege : \sicut{l'étoile du matin} et
\sicut{l'étoile du soir}.  Ces deux expressions ont la même
dénotation car elles 
désignent le même objet céleste : la planète Vénus%%%%%%
\footnote{À noter que
le nom propre \sicut{Vénus} a ici, lui aussi, cette même
dénotation.}.  
%%%%%%%%%
La dénotation est donc cette notion qui incarne directement la
relation entre les éléments linguistiques et les entités de la réalité.
Cependant Frege montre que la dénotation ne peut pas suffire à décrire le
rôle (qu'on appellera plus tard \emph{sémantique}) d'une expression
dans une phrase,
 et que ce qui importe c'est, finalement, le
\emph{sens} de l'expression.  Et \sicut{l'étoile du matin} et
\sicut{l'étoile du soir} n'ont pas le même sens.  C'est ce que révèle,
par exemple, le test de substitution suivant :

\ex.  
\a. \label{xesoir}
L'étoile du matin est l'étoile du soir.
\b.
\label{xemat}
L'étoile du matin est l'étoile du matin.


La phrase \ref{xemat} est obtenue en remplaçant dans la phrase
\ref{xesoir} le second groupe nominal (\sicut{l'étoile du soir}) par
un autre groupe nominal qui a la même dénotation (en l'occurrence
\sicut{l'étoile du matin}).  Cette phrase \ref{xemat} est vraie par
nécessité, elle le sera toujours, sa vérité est inscrite dans sa
forme : c'est  une \kw{tautologie}.  Et en
tant que telle, elle n'apporte aucune information pertinente.  En
revanche, la phrase \ref{xesoir}, elle, apporte de l'information,
elle est intéressante, elle n'est pas une tautologie.  Ces deux
phrases sont loin d'être synonymes : elles ne disent pas la même
chose.  Puisque \sicut{l'étoile du matin} et \sicut{l'étoile du soir} ont la même dénotation, rien ne distingue \ref{xemat} de
\ref{xesoir} du strict point de vue dénotationnel.  D'ailleurs si l'on
s'en tenait uniquement aux dénotations, les deux phrases reviendraient
simplement à un schéma équatif de la forme : ``vénus = vénus''.  On
voit bien que ce qui les distingue, c'est autre chose : c'est le sens
des expressions qu'elles contiennent.

\largerpage[-1]

Comment Frege définit-il le \kw{sens} ? Si la dénotation est l'objet du
monde désigné 
par l'expression, son sens est «le mode de donation» de
cet objet.  Par mode de donation, il faut comprendre \emph{ce qui nous
  donne la dénotation de l'expression}.  
On peut  voir le sens comme une construction linguistique%%%%%%%
\footnote{Frege
insiste sur la distinction entre le sens et la représentation mentale
(et donc subjective) que le locuteur peut se faire d'un 
objet. Le sens est conventionnel, consensuel dans la mesure où il est
ancré au code de la langue.}, %%%%%%%%
 c'est-à-dire la fonction 
%sémantique 
intellectuelle qui permet
d'appréhender un objet du monde à partir d'une expression.
Adoptons cette définition.



\begin{defi}[Sens]
\label{def:sens}
Le \kw{sens} d'une expression est ce qui nous permet de connaître la
dénotation de cette expression.  
\end{defi}

Le sens n'est donc pas la dénotation, mais il est défini à partir d'elle.
On peut le voir comme une sorte de
mécanisme d'\emph{attribution} d'objets à des expressions de la  langue.  
Cela peut paraître un peu abstrait de prime abord (mais après tout, le
sens est une notion relativement abstraite), mais nous verrons au fil
des pages de ce manuel (jusqu'au chapitre~\ref{Ch:t+m}) que la
définition de Frege est suffisante pour développer notre théorie
sémantique et pour recevoir une formalisation précise.

Voici une autre paire d'expressions qui ont la même dénotation mais
des sens différents : \sicut{le vainqueur d'Iéna} et 
\sicut{le vaincu de Waterloo}\footnote{Exemple tiré de
  \citet{Lyons:77Fr},\Andex{Lyons, J.} qui
l'emprunte à Husserl.}.  Ici la différence de sens ne réside pas dans
l'éventuel contraste de connotations «affectives» ou appréciatives
qui pourraient venir se greffer sur les mots \sicut{vainqueur} et
\sicut{vaincu}.  Les sens de ces deux groupes nominaux diffèrent car
pour en connaître les dénotations, il faut être au fait, d'une part, de
l'issue de la bataille d'Iéna, savoir qui l'a remportée (et donc
savoir ce qui signifie \sicut{vainqueur}), et d'autre
part, savoir qui a perdu la bataille de Waterloo (en sachant donc ce que
signifie \sicut{vaincu}).  Il se trouve que
dans les deux cas, la réponse est Napoléon, qui est la dénotation
commune des deux expressions, mais on voit bien qu'on y  accède par
deux cheminements (c'est-à-dire deux sens) distincts.

Frege souligne que différentes expressions de la langue peuvent avoir le
même sens (\sicut{la somme de 1 et 3} et \sicut{la somme de 3 et
  1}\footnote{Cet exemple est loin d'être trivial. On le sait, une simple
  modification de l'ordre des mots peut avoir des répercussions
  sémantiques non négligeables, y compris dans les expressions qui
  parlent d'opérations mathématiques.  Comparez \sicut{le produit de 3
par 2} vs. \sicut{le produit de 2 par 3} avec \sicut{la division de 3
    par 2} vs. \sicut{la division de 2 par 3}.}), que 
des expressions de sens distincts peuvent avoir la même dénotation
(par exemple \sicut{l'étoile du matin} et \sicut{l'étoile du soir}),
et des expressions peuvent avoir un sens mais pas de dénotation (par
exemple on peut concevoir le sens de \sicut{la suite qui converge le
moins rapidement} bien qu'elle n'existe pas ; de même pour \sicut{le
plus grand nombre premier} ou \sicut{un cercle carré} ; et c'est
précisément parce que l'on perçoit leurs sens que l'on sait que ces
expressions n'ont pas de dénotation).


\largerpage[-1]

\paragraph*{Référence et référent.}
Avant de poursuivre, ouvrons une petite parenthèse terminologique et
notionnelle qui pourra s'avérer utile par la suite.  Le terme de
\kwa{référence}{reference} est souvent employé comme synonyme de
dénotation, et ce en vertu de la définition qu'on lui assigne
couramment.  En effet on appelle référence la relation qui s'établit
entre une expression et ce dont parle un locuteur en utilisant cette
expression.  Ainsi on dira que \sicut{le vainqueur d'Iéna} fait
référence, ou réfère, à Napoléon.  Et on dira conséquemment que Napoléon
est le \kwa{référent}{referent} de l'expression.  Cela coïncide bien
avec la dénotation.  

Cependant on peut (et même on doit) appréhender ces deux notions de
deux points de vue différents.  D'une certaine manière, la référence a
un pied dans la pragmatique, dans la mesure où sa définition fait
intervenir le locuteur.  On peut d'ailleurs faire ici une remarque
similaire à celle faite en \S\ref{s:spacts} (p.~\pageref{s:spacts})
sur l'idée de \emph{vouloir dire} : si l'on peut dire que le groupe
nominal \sicut{le vainqueur d'Iéna} fait référence à Napoléon,
on peut dire 
que c'est également le locuteur qui, en prononçant ce groupe nominal,
fait référence à Napoléon.  Il s'ensuit que la référence se présente
comme un lien \emph{contingent} entre une expression employée et un
objet du monde ; elle pointe sur ce à quoi le locuteur pense lorsqu'il
utilise telle ou telle expression.  C'est donc un \emph{fait} attaché
à un énoncé et dont est responsable le locuteur ; certains philosophes
parlent d'ailleurs à ce propos d'\emph{acte de
  référence}\footnote{Le titre de l'article de \citet{Strawson:50fr},\Andexn{Strawson, P.}
  «~\alien{On referring}~», a été traduit en français par «De
  l'acte de référence», et le chapitre~4 de
  \citet{Sea:69Frb}\Andexn{Searle, J.} s'intitule, dans la traduction française, «La
    référence comme acte de langage».  Sur la distinction entre
  référence et dénotation, on peut également se reporter, entre autres, à %au chapitre~7 de
 \citet[][chap.~7]{Lyons:77Fr},\Andexn{Lyons, J.} (en prenant garde au fait qu'il a un usage
terminologique légèrement différent de celui que nous adoptons ici --~notamment concernant la dénotation) ainsi qu'à \cite[][chap.~1 \& 2]{Abbott:10} qui distingue la référence pragmatique et la référence sémantique (cette dernière étant assimilée à la dénotation).}.
Au contraire on considérera que la dénotation est une \emph{propriété}
sémantique (donc linguistique) des expressions et que cette propriété
est envisageable indépendamment du contexte et de ce que pense le
locuteur.  Bien entendu il ne s'agit pas là d'une violente opposition
de notions ; c'est, comme nous l'avons dit, une question de points de
vue.  Et dans la plupart des cas, c'est parce que telle expression a
la propriété de dénoter tel objet qu'un locuteur la choisira pour faire
référence à l'objet en question.  C'est pourquoi les deux notions se
superposent très simplement, surtout lorsque l'on se focalise sur les
propriétés sémantiques des productions linguistiques, comme nous le
ferons dans l'essentiel de ce manuel (même si nous revendrions un peu
sur la question dans le chapitre~\ref{Ch:contexte}, vol.~2).





\subsection{Sens et conditions de vérité}
%-------------------------------------

Revenons à l'opposition entre sens et dénotation. Ce que nous avons
illustré jusqu'ici s'applique à des groupes nominaux (\GN), et plus
précisément à des {\GN} qui fonctionnent comme des noms propres. 
%(ce que
%l'on appelle des \alien{names} en anglais)
 Dans son article, Frege
précise également ce que sont le sens et la dénotation d'une phrase
déclarative. Dans ses termes, le sens d'une phrase est une pensée
(\emph{Gedanke}, en v.o.), et sa 
dénotation la \kwo{valeur de vérité}\indexs{valeur!\elid\ de vérité} de la phrase (cf. \S\ref{sss:pspDf}).  

Rappelons que la valeur de vérité d'une phrase, c'est %tout simplement 
\emph{vrai} ou
\emph{faux}, selon que la phrase en question est jugée vraie ou
fausse.  Nous retrouvons ici le rôle important que joue la vérité dans
la théorie sémantique (cf. \S\ref{s:conseql} au chapitre précédent).
Cela peut paraître un peu contre-intuitif de considérer ces concepts
abstraits, le vrai et le faux, comme des dénotations, c'est-à-dire des
objets du mondes, au même titre que ce que nous avons déjà vu, par exemple
une planète ou un être humain (qui eux sont des objets tangibles).
Mais examinons un instant ce qu'est au juste la vérité d'une phrase.
Une phrase est vraie, ou jugée vraie, si elle est conforme à la
réalité, c'est-à-dire aux faits qui constituent l'agencement du monde
tel qu'il est ; et la phrase est jugée fausse dans le cas contraire.
On constate donc que les valeurs de vérité ont à voir avec l'univers
des dénotations, la réalité.  Il y a d'autres justifications à cette
proposition de Frege.  L'une d'elles est que la  valeur de vérité d'une 
phrase ne change pas si dans cette phrase on  remplace un constituant,
par exemple un 
\GN, par un autre qui à la même dénotation.  
\label{leibniz}
C'est ce qu'illustrent les
exemples \ref{x:1821a} et \ref{x:1821b} : si \ref{x:1821a} est
vraie, alors forcément \ref{x:1821b} l'est aussi, car les deux {\GN}
sujets ont la même dénotation (les deux phrases parlent du même
individu).  En revanche si l'on leur substitue un 
{\GN} qui diffère par sa dénotation, comme en \ref{x:1821c}, la
valeur de vérité de la phrase peut changer.  Cela vient appuyer l'idée
que la dénotation du tout (i.e.\ la valeur de vérité de la phrase) est
étroitement déterminée par la dénotation de ses parties (entre autres
ses \GN).  


\ex.  \label{x:1821}
\a. \label{x:1821a}
Le vainqueur d'Iéna est mort en 1821.
\b. \label{x:1821b}
Napoléon est mort en 1821.
\c. \label{x:1821c}
Jules César est mort en 1821.


Enfin ajoutons que l'assimilation de la dénotation d'une phrase
déclarative à sa valeur de vérité permet de qualifier sémantiquement
les différents types (grammaticaux) de phrases de la langue.  Toutes
les phrases déclaratives, et seulement elles, ont (ou dénotent) une
valeur de vérité, et en cela elles s'opposent aux interrogatives,
impératives et exclamatives.  En vertu de la proposition de Frege, on
peut donc établir que les différents types de phrase se caractérisent
sémantiquement par différents types de dénotation\footnote{Nous
  n'aborderons pas dans ce manuel la sémantique des phrases non
  déclaratives (au delà des remarque d'ordre pragmatique faites au
  chapitre précédent).  Mais nous indiquerons quelques pistes et
  quelques renvois bibliographiques dans les chapitres \ref{Ch:contexte} et \ref{ch:conclu} (vol.~2). À
  présent, dans ce qui suit, le terme \emph{phrases} tout court sera
  utilisé simplement pour faire référence aux phrases déclaratives.}.  

Venons-en maintenant au sens des phrases déclaratives.
L'interrelation entre sens et dénotation, que pose Frege, vaut
toujours ici :  le sens est ce qui nous permet de trouver 
(systématiquement) la dénotation (possible) d'une expression.
Cela va nous permettre d'appréhender un peu plus clairement la notion
de sens, qui jusqu'ici restait un peu énigmatique.
Appliqué aux phrases, cela nous donne le principe
suivant. 

\begin{princ}\label{p:vcond1}
Comprendre une phrase c'est être \emph{capable}, \emph{d'une manière
  ou d'une autre}, de juger de sa vérité.
\end{princ}

Et comprendre une phrase signifie en percevoir le sens. Le principe ne
fait donc rien d'autre que de reformuler la définition \ref{def:sens}
pour le cas particulier des phrases.  Mais il faut expliciter un peu
ce «d'une manière ou d'une autre».  Souvenons-nous que la
vérité d'une phrase n'est jamais absolue.  Elle \emph{dépend} de
circonstances (\ie\ des cas de figure), et précisément les circonstances qui
forment la réalité, qui font que le monde est tel qu'il est.  Ainsi si
l'on sait qu'une phrase donnée est vraie ou fausse, c'est que l'on
connaît certaines informations sur l'état du monde.  Et
réciproquement, si on dispose des informations nécessaires, on saura
dire si une phrase (que l'on comprend) est vraie ou fausse.  En bref,
comprendre une phrase, c'est savoir effectuer un appariement entre
elle et la configuration du monde.  Cela nous permet d'expliciter le
principe précédent de la manière suivante.


\begin{princ}[Sémantique vériconditionnelle]\label{p:vcond2}
Connaître  le sens d'une
phrase (déclarative) c'est savoir comment doit, ou devrait, être le monde pour
que cette phrase soit vraie. 
\end{princ}


Pour dire les choses  autrement : savoir ce que signifie une phrase,
c'est savoir \emph{dans quelles conditions} elle est vraie.  C'est pourquoi
on considère que le sens d'une phrase consiste en ses \kw{conditions
  de vérité}. 
Lorsqu'une théorie sémantique s'appuie sur ce principe,
elle est dite \kwi{vériconditionnelle}{semantique@sémantique!\elid\ vericonditionnelle@\elid\ vériconditionnelle}.
Nous allons voir un exemple de formulation de conditions de vérité (et
nous en verrons beaucoup d'autres tout au long de ce manuel), mais
auparavant il est nécessaire de bien appuyer ce point :
savoir dans quelles conditions une phrase est vraie ne signifie pas
savoir si elle est effectivement vraie.  C'est pour cette raison que
le principe~\ref{p:vcond1} insiste sur «être capable».  Ce «être
capable» sous-entend «si on a les informations nécessaires sur le
monde».  Mais en soi, ces informations ne sont pas utiles pour
connaître le sens d'une phrases. Prenons par exemple \ref{x:orion} :

\ex. \label{x:orion}
Il y quatorze planètes qui gravitent autour de l'étoile Sirius.


Personne, y compris les astrophysiciens, ne connaît aujourd'hui la
dénotation de cette phrase, mais elle est parfaitement compréhensible :
son sens est clair.  C'est-à-dire que l'on sait dire dans quels cas
elle s'avérera vraie : \ref{x:orion} est vraie si et seulement s'il
existe dans la galaxie au moins quatorze objets qui sont de grandes
masses de matière à peu près sphériques, \ie\ des planètes, qui se
déplacent sur des trajectoires circulaires ou elliptiques autour d'un
objet très grand et lumineux, une étoile, connu sous le nom de
Sirius.  Cette description constitue précisément les conditions de
vérité de \ref{x:orion}.  Reste à savoir si notre univers vérifie
bien ces conditions --~auquel cas \ref{x:orion} sera admise comme
vraie~-- et c'est l'observation astronomique qui pourra nous le dire,
et se faisant, augmentera notre connaissance du monde (en l'occurrence
de l'univers ou du cosmos).

Ainsi, établir la vérité ou la fausseté, c'est-à-dire la dénotation,
des phrases, d'une manière générale, n'est pas l'affaire des
sémanticiens\footnote{Sauf bien sûr lorsqu'il s'agit de phrases qui
portent sur le sens linguistique.}  mais plutôt des scientifiques de
la nature, des
historiens, des philosophes etc., voire des juges ou des détectives.
Connaître les dénotations n'est pas une fin en soi en sémantique, mais
s'interroger sur les dénotations est un moyen de décrire le sens.
C'est ce que dit le principe \ref{p:vcond2}, car on peut en tirer la
règle méthodologique suivante : décrire le sens d'une expression c'est
spécifier les règles de calcul de sa dénotation ; on n'a pas besoin de
mener ce calcul jusqu'au bout, car pour cela il faudrait connaître des
informations plus ou moins précises sur le monde. C'est pourquoi il
faut s'assurer que ces règles de calcul puissent marcher pour
n'importe quelle configuration du monde.

Et il est très important d'être également conscient que le principe \ref{p:vcond2} peut se reformuler en l'envisageant sous sa réciproque. Il nous dit alors que si l'on connaît le sens d'une phrase et que l'on admet que cette phrase est vraie, alors on apprend des informations sur le monde. Et c'est bien ainsi que fonctionne basiquement la communication, lorsque des interlocuteurs s'échangent des informations en énonçant des assertions.


Pour terminer et résumer cette section, voici un autre exemple.  Soit la
phrase :

\ex.  \label{x:Alex}
Le père d'Alexandre le Grand était boiteux.


Et supposons que je vous demande la dénotation de cette phrase, \ie\ que
je vous demande si elle est vraie ou fausse.  {A priori}, vous ne
savez probablement pas répondre. Mais si vous voulez trouver la
solution, que pouvez-vous faire ? Vous pouvez, par exemple, aller
consulter une \emph{encyclopédie} ou un livre d'Histoire, qui donnera
des informations historiques sur le monde ; mais vous n'irez certainement pas
consulter un \emph{dictionnaire} de langue comme le Petit Robert.  Vous iriez
regarder un dictionnaire seulement si vous aviez un doute sur le sens
de cette phrase, ou de certaines de ses parties ; et le dictionnaire
vous aiderait ainsi à déterminer les \emph{conditions} de vérités de
\ref{x:Alex}.   Mais ce n'est pas ce dont vous avez besoin, car vous
connaissez ces conditions de vérité; elles disent approximativement
qu'il faut et il 
suffit qu'il ait existé un individu de sexe masculin qui avait,
disons, un jambe plus courte que l'autre et qui avait un fils connu
sous le nom d'Alexandre le Grand\footnote{Ces conditions de vérité
  sont approximatives car d'abord on peut être boiteux pour une autre
  raison que celle indiquée ici, et aussi car ce n'est pas exactement
  ainsi que l'on spécifie les conditions de dénotation pour un {\GN}
  défini comme \sicut{le père de...} ; nous y reviendrons au chapitre
  suivant, \S\ref{s:GNdef}.}.
Ce dont vous avez besoin, et que vous
pourrez trouver dans une encyclopédie, c'est une information sur les
\emph{faits} réels (historiques) : une information d'ordre
«anatomique» sur l'homme qui était le père d'Alexandre.  Ensuite
vous n'avez plus qu'à vérifier si cette information encyclopédique
respecte ou non les conditions de vérité de la phrase, pour en déduire
la dénotation.



\subsection{La compositionnalité}
%-------------------------------
\indexs{compositionnalité}

Jusqu'à présent, nous avons vu les dénotations des {\GN} et des phrases.
Nous savons quel genre de dénotation correspond à certains
{\GN} : des objets du monde ; et quel genre de dénotation correspond aux
phrases déclaratives : le vrai ou le faux.  Évidemment, on sera intéressé de
connaître aussi à quels genres de dénotation correspondent les autres
catégories de constituants (parties du discours) de la langue : les
verbes, les noms (substantifs), les adjectifs, les déterminants, les
prépositions, etc.   
Nous verrons cela au fur et à mesure dans les pages qui suivent, car
en fait il sera assez simple de déduire ces types de dénotations les
unes des autres.  Pour le moment, je vais introduire un principe
important de la sémantique vériconditionnelle, et qui justement
permettra de déduire les types de dénotation des autres catégories
grammaticales. 

Il s'agit du \kw{principe de compositionnalité}%
\footnote{Le principe est parfois aussi appelé \emph{principe de
    Frege}, ce qui est sujet à caution car Frege n'en a jamais fait
    mention dans ses écrits.  Cependant on considère habituellement
    que le principe est particulièrement fregéen dans l'esprit.  Pour
    plus détails sur la compositionnalité, voir \citet{Part:84c}.\Andexn{Partee, B.}}.
    Nous l'avons 
entr'aperçu dans la section précédente en disant que la dénotation du
tout dépend de la dénotation de ses parties.  Mais le principe va plus
loin.  En voici une première formulation que l'on retrouve assez
couramment dans la littérature :


\begin{princ}[Principe de compositionnalité (\textsc{i})]
\label{p:compo1}
La signification d'une expression \emph{dépend} de la signification de ses
parties. 
\end{princ}


Notons tout d'abord que cette formulation utilise un terme que nous
n'avons guère manipulé : «signification».  Il s'agit ici de ce que
nous avons défini précédemment comme le sens.  Mais si le
principe~\ref{p:compo1} est ainsi formulé, c'est souvent pour tirer
profit du caractère peu technique du terme de signification, car le
principe s'applique également à la dénotation des expressions (donc
ici signification recouvre à la fois sens et dénotation).  

Maintenant examinons ce qu'exprime et implique ce principe.  La
compositionnalité repose naturellement sur l'idée de composition, et
de son opération inverse, la décomposition. En cela elle incarne
simplement le procédé d'\emph{analyse} (au sens premier du terme)
sémantique : on reconstitue le sens d'une expression en la décomposant
en ses parties, ces parties ont elles-mêmes un sens et ces
«petits» sens se composent (ou combinent) entre eux pour donner le
sens de l'expression globale.  On peut voir cela comme une
arithmétique sémantique élémentaire.  Mais le principe a des
fondements plus importants, car il illustre en quelque sorte le
pendant sémantique de l'aspect \emph{génératif} de la description
grammaticale défini par \citet{Chom:57}.\Andex{Chomsky, N.}  Nous
sommes capables 
d'interpréter une infinité potentielle de phrases de la langue, même
si nous n'avons jamais entendu ou lu ces phrases.  Il est évident que
nous n'apprenons pas des listes de sens de phrases les uns après les
autres, puisque ces phrases sont en nombre infini.  Nous comprenons
une phrase parce que nous connaissons le sens de ses constituants
(syntagmes, mots, morphèmes) et que nous savons assembler ces sens.
C'est ce que pose le principe de compositionnalité, de manière assez
stricte car il semble assujettir fermement le sens d'une phrase à
celui de ses unités constituantes.

À cet égard, plusieurs critiques ont été émises à l'encontre du
principe.  J'en mentionnerai et commenterai ici trois\footnote{Voir
  \citet{Godard:06}\Andexn{Godard, D.} pour un panorama plus détaillé.}.  La première, et
la moindre, est qu'il ne suffit pas de connaître le sens des mots pour
établir le sens d'une phrase, il est également nécessaire de tenir
compte de son ordonnancement syntaxique.  Cela paraît évident si l'on
compare \ref{x:helico1} et \ref{x:helico2} où les mêmes mots n'occupent pas la même
position syntaxique.

\ex.  \a. \label{x:helico1}
Un hélicoptère a survolé un porte-avions.
\b.  \label{x:helico2}
Un porte-avions a survolé un hélicoptère.

C'est pourquoi le principe de compositionnalité est généralement
amendé de la façon suivante :

\begin{princ}[Principe de compositionnalité (\textsc{ii})]
\label{p:compo2}\indexs{principe de compositionnalité}
La signification d'une expression est \emph{fonction} de la
signification de ses parties et de leur mode de combinaison
syntaxique.
\end{princ}

C'est cette version du principe que nous prendrons en compte
dorénavant.  Notons au passage que, dans sa formulation,
«dépendre» a été remplacé par «être fonction», ce qui revient
au même ; on dit aussi parfois que la signification d'une expression
est \emph{une fonction} de la signification de ses parties etc., ce
qui donne un petit air mathématique au principe, mais nous prépare
également à des éléments de formalisation que nous verrons au
chapitre~\ref{ch:types}.

Une autre critique repose sur le fait que le sens d'une phrase dépend
également (et parfois totalement) du contexte, ce que le principe ne
mentionne pas.  Nous en avions fait l'expérience au chapitre précédent
notamment avec les implicatures, les significations illocutoires et
les actes de langage indirects.  Mais, comme je l'avais précisé, ces
significations pragmatiques se distinguent du sens littéral, qui est ce
qui fait l'objet de l'étude sémantique (dans ce manuel).  La
compositionnalité est une propriété sémantique du langage et ne
concerne véritablement que les sens littéraux.  Et c'est au-delà de
l'application du principe (et donc de l'analyse sémantique) que
s'opère un calcul interprétatif pragmatique, qui n'annule cependant
pas la validité du principe de compositionnalité car le sens littéral
sert de point de départ à ce calcul.

Enfin, la troisième critique concerne les expressions idiomatiques,
c'est-à-dire des expressions plus ou moins figées et souvent imagées
telles que  \sicut{casser sa pipe}, \sicut{prendre une veste},
\sicut{sucrer les fraises} ou \sicut{jeter le bébé avec l'eau du bain},
etc.  La particularité de ces expressions est que non seulement leur
sens ne se résume pas à la composition du sens de leurs parties, mais
en plus qu'il n'a souvent rien à voir.  Lorsque l'on dit de quelqu'un
qu'il va jeter le bébé avec l'eau du bain, il n'y a souvent aucun bébé
dans l'histoire. Mais si à ces expressions on  applique le
principe de compositionnalité «jusqu'au bout», on obtient un sens
littéral qui n'est généralement pas celui voulu par le locuteur.  Et
il ne s'agit pas ici d'implicatures, car les sens des idiomes est
parfaitement codifié et conventionnel, on peut le trouver dans un
dictionnaire.  C'est là une réelle mise en échec de la
compositionnalité ; mais il y a néanmoins une manière de contourner ce
problème en prenant la précaution suivante : on considère que les
idiomes sont des expressions composées (des locutions) et qui à ce
titre ne doivent pas être décomposées sémantiquement, mais plutôt être
traitées d'un bloc (après tout \sicut{casser sa pipe}, par exemple,
n'est qu'une manière coquette de dire \sicut{mourir}) en y empêchant
l'application du principe.   

En conclusion, même si le principe de compositionnalité peut être mis
en défaut et sembler parfois insuffisant, nous considérerons ici qu'il
est nécessaire à l'analyse sémantique et nous en tiendrons compte
comme outil d'investigation.

\subsection{Analyse sémantique formelle}

Récapitulons. Toute expression interprétable a une dénotation, ou est
au moins susceptible d'en avoir une (n'oublions pas que certaines
expressions ne dénotent rien).  La dénotation d'une expression est
toujours relative à un certain état du monde, celui par rapport auquel
le locuteur situe le contenu de son propos.  Le sens d'une expression
est ce qui détermine sa dénotation pour tout état du monde possible.
Décrire le sens d'une expression, ce qui est l'objet de la sémantique,
revient donc à donner les règles de calcul de sa dénotation (ou de son
absence de dénotation, le cas échéant) ; c'est-à-dire, dans le cas
particulier des phrases, les conditions de vérité.  Les conditions de
vérité sont des stipulations sur le monde, elles ne disent pas comment
est le monde mais comment il devrait être pour qu'une phrase soit
vraie.  Nous pouvons, comme nous l'avons fait précédemment, exprimer ces
conditions dans notre langage de tous les jours, le français, mais
cela deviendrait rapidement lourd et incommode lorsque nous serons amenés
à analyser des phrase un peu compliquées, de plus nous risquerions d'y
perdre en précision, sachant que les langues naturelles, comme le
français, comportent leur part inévitable de vague et de polysémie.
Pour exprimer des descriptions sémantiques, nous nous devons d'être
précis et explicites.  C'est pourquoi une méthode couramment employée
en sémantique consiste à utiliser un langage symbolique, artificiel et
formel dans lequel est formulé de manière concise et précise le sens
des expressions de la langue que l'on étudie.  L'intérêt d'un tel
langage, assez éloigné des langues naturelles dans sa structure, est
qu'il permet, lorsque l'on en est familier, de lire immédiatement et
directement dans ses formulations les conditions de vérité qu'il
représente, aussi facilement que l'on voit immédiatement une opération
d'addition dans l'écriture mathématique «~$4+3$~».  D'ailleurs le
langage de représentation sémantique que nous allons voir est emprunté
aux mathématiques et plus exactement à la logique et ce que l'on
appelle le \kw{calcul des prédicats}.  On lui donne donc
habituellement le nom de langage du calcul des prédicats, mais dans ce
manuel nous l'appellerons le \kw{langage objet} ou {\LO},\is{LO@\LO} d'abord pour
bien insister sur le fait qu'il sera en grande partie l'objet de notre
étude\footnote{Pour être tout à fait exact, l'objet de notre étude
sémantique ici est également, et probablement avant tout, le français.
Mais nous étudierons la sémantique du français par l'intermédiaire du
langage symbolique \LO, qui fera l'objet d'observations et de
commentaires.}, et ensuite parce qu'il sera amené à évoluer au cours
des pages jusqu'à s'éloigner un peu de ce que l'on appelle
traditionnellement le langage du calcul des prédicats (même s'il en
conservera les fondements).  Ce chapitre cependant se consacre à la
présentation des bases classiques du calcul des prédicats, d'où les
intitulés du chapitre et de la section qui suit. Une des tâches de la
description sémantique va donc maintenant consister à traduire des
expressions du français dans ce langage formel {\LO}, et pour mener à
bien ces opérations de traduction, il nous faut apprendre à «parler
couramment le {\LO}~».


\section{Le langage du calcul des prédicats}
%===========================================
\is{calcul des prédicats}

Le calcul des prédicats est une branche assez ancienne de la logique
classique\footnote{Il est déjà présent chez Aristote et dans la
syllogistique médiévale.}, mais il a commencé à être formalisé, au
moyen de systèmes d'écriture symbolique, appelés aussi des
idéographies, à partir de la fin du \textsc{xix}\up{e} et au début du \textsc{xx}\up{e}
siècle, notamment par les travaux de C. Pierce, G. Frege, G. Peano,
B. Russell et de bien d'autres par la suite.  Ce type de langages
visait surtout à exprimer sans équivoque des énoncés mathématiques et
à formaliser des principes de logique.  Mais son application à la
représentation du sens des phrases (plus ou moins courantes) des
langues naturelles fut également assez précoce et prit un essor
important à partir des années 1960 notamment (mais pas seulement) avec les
travaux de R. Montague.  Montague, qui était philosophe et logicien,
et bien qu'il ne s'inscrivait pas du tout dans le cadre de la
grammaire générative,\is{grammaire!\elid\ générative} a eu un rôle important dans le développement de
la linguistique formelle (et plus exactement la sémantique formelle)
telle qu'a pu l'initier Chomsky.  Chomsky a défendu la thèse que le
langage naturel peut être traité (\ie\ analysé) comme un système formel ;
Montague est allé plus loin dans cette idée en posant que le langage
naturel peut être traité comme un système formel
\emph{interprété}\footnote{La déclaration, volontiers provocatrice, de
  Montague que l'on cite souvent à cet égard est : «Il
    n'y a selon   moi aucune différence théorique importante entre les langues
  naturelles et les langages artificiels des logiciens; en effet, je considère que l’on peut comprendre la syntaxe et la sémantique de ces deux types de langage au sein d’une même théorie naturelle et mathématiquement précise»
  \citep[373]{Montague:UG}.
Voir aussi \citet[6--7]{Bach:89}.},
c'est-à-dire un système muni d'une sémantique (ce qu'avait écarté
Chomsky).  Et ce système est bien sûr celui du calcul des prédicats.
Les théories d'analyse sémantique développées par \Andex{Montague, R.}Montague\footnote{Voir notamment \citet{Montague:EFL,Montague:UG,PTQ}.\Andexn{Montague, R.} Pour un panorama historiographique et épistémologique du développement de la sémantique formelle sous l'impulsion de Montague, on pourra se reporter notamment à \cite{Partee:96,Partee:05}. } sont
regroupées aujourd'hui sous le nom de \emph{Grammaire de Montague}\is{grammaire!\elid\ de Montague} (il
s'agit plus d'une famille de grammaires que d'un formalisme unique).
Et c'est dans ce paradigme d'analyse sémantique que se place le
présent manuel.

Comme nous l'avons vu plus haut, l'analyse sémantique consiste donc à
expliciter les conditions de vérité des phrases de la langue en les
formulant dans le langage formel du calcul des prédicats.  Comme tout
langage, bien que celui-ci soit artificiel, il possède une syntaxe et
une sémantique.  C'est ce que vont présenter les sections qui suivent,
d'abord informellement (\S\ref{s:elf} et \ref{s:seminf}) puis de
façon plus rigoureuse (\S\ref{s:syntlf} et \ref{s:reglsem}).


\subsection{Les éléments du langage formel}
%------------------------------------------
\label{s:elf}

\subsubsection{Prédicats}
%''''''''''''''''''''''''
\label{sss:pred}

Comme son nom l'indique, le langage du calcul des prédicats repose
fondamentalement sur la notion de... \kwa{prédicat}{predicat}.  Dans la
perspective que nous adoptons ici, qui va consister à traduire (ou
transcrire) le sens de phrases du français dans un langage formel (\LO), les
prédicats seront les symboles par lesquels nous traduirons les
\emph{verbes},\is{verbe}
les \emph{noms communs}\is{nom!\elid\ commun} (substantifs), les \emph{adjectifs}.\is{adjectif}\is{substantif|see{nom commun}}
Nous utiliserons souvent aussi les prédicats pour traduire directement les
tournures attributives de la forme \sicut{être Adj} et
\sicut{être $($un$)$ N}.
Voici quelques exemples de matériaux linguistiques qui seront traduits
par des prédicats :


\ex.  \label{xpredicats}
\begin{tabular}[t]{l@{\qquad}l}
... est gentil & ... dort\\
... est un canard & ... a faim\\
... est (un) acteur & ... fume\\
... est ventru & ... aime ...\\
... est italien & ... connaît ...\\
... est le frère de ... & ... embrasse ...\\
\end{tabular}


D'un point de vue plus général, les
prédicats seront définis ici comme ces symboles qui correspondent à des
\emph{propriétés} ou des \emph{relations}.  Et le propre d'une
propriété est d'être \emph{attribuée} à quelque chose, de même qu'une
relation \emph{porte sur} des choses.  Plus généralement, un prédicat
est censé  \emph{concerner},  \emph{porter sur} ou encore
\emph{s'appliquer à} une ou plusieurs choses.  Et 
ces choses sont ce que l'on appelle les \kwi{arguments}{argument} du prédicat.
Cela transparaît déjà dans la formulation des exemples
\ref{xpredicats} via les points de suspensions : en fait les
arguments des prédicats sont «ce qui manque» en
\ref{xpredicats}.
%\footnote{Pour être tout à fait exact, on peut dire
%  qu'il  manque  autre chose dans les exemples \ref{xpredicats} : les
%flexions des verbes.  Nous y reviendrons dans le chapitre \ref{Ch:temps2}.}. 

\begin{defi}[Argument]
On appelle \kw{argument} d'un prédicat ce à quoi s'applique ou ce que
concerne le prédicat.
%\\
%Nous appellerons \kw{prédication} «l'alliance» d'un prédicat et de
%son ou ses argument(s).
\end{defi}

Il apparaît clairement, 
de par la série \ref{xpredicats}, que la notion d'argument est (à
peu de chose près) le pendant sémantique de la notion syntaxique de
\emph{complément} (en y incluant le sujet).
Et
nous constatons ainsi que la langue
naturelle, le français ici, nous invite à envisager des prédicats qui
attendent un seul argument (\sicut{est gentil}, \sicut{dort}, etc), pour
traduire certains noms\is{nom!\elid\ commun} et les verbes ou {\GV} intransitifs,\is{verbe!\elid\ intransitif} et des
prédicats qui  attendent deux arguments (\sicut{aime}, \sicut{est le frère
  de}, etc), pour traduire des constructions transitives et relationnelles.  On pourra
même considérer des prédicats à trois 
arguments (\sicut{... préfère ... à~...}), voire à quatre
(\sicut{... vend ... à ... pour ...}) ou plus...
La relation entre un prédicat et ses arguments est caractérisée par une propriété importante que l'on appelle l'\kw{arité}\footnote{On trouve également le terme de \emph{valence}\is{valence} pour désigner cette propriété.} du prédicat.
\addtocounter{footnote}{1}\footnotetext{En particulier, pour $n=1$ on prononcera \emph{unaire}, pour
$n=2$ \emph{binaire}, pour $n=3$ \emph{ternaire}, etc.}
\addtocounter{footnote}{-1}

\begin{defi}[Arité]\label{arite}
On appelle \kw{arité} d'un prédicat le nombre d'arguments qu'il
prend.  Ainsi, si un prédicat attend $n$ arguments, on dit que son
arité est $n$ ; on dit aussi que le prédicat est $n$-aire%
\footnotemark, ou encore
qu'il s'agit d'un prédicat à $n$ places.
\\
Un prédicat donné a une et une seule arité ; l'arité est une
caractéristique déterminante des prédicats.
\end{defi}

\begin{nota}[Prédicats]
Dans {\LO}, un prédicat est représenté par un symbole, dit \emph{symbole
de prédicat},  et ses arguments sont indiqués à sa suite entre
parenthèses\footnotemark, séparés par des virgules s'il y en a plusieurs.
\\
Par convention, les symboles de prédicats seront écrits en
\prd{\textcolor{black}{gras}}.
Et, toujours par convention, on utilisera des mots ou des
abréviations de mots de la langue pour représenter ces symboles dans {\LO}%
\footnotemark.  
\end{nota}
\addtocounter{footnote}{-1}\footnotetext{Il s'agit donc d'une notation de type
  \emph{fonctionnelle} ; %cf. \S\ref{foncteur}, p.~\pageref{foncteur}. 
d'ailleurs le terme d'argument se retrouve
  dans le vocabulaire mathématique concernant les fonctions --~ce
  n'est pas un hasard.}
\addtocounter{footnote}{1}\footnotetext{Il existe une autre convention de notation très couramment
  utilisée dans la littérature en sémantique formelle qui consiste à
  écrire les symboles de prédicats en les marquant d'un prime ($'$),
  par exemple, $\Xlo\text{gentil}'$, $\Xlo\text{aimer}'$, etc.}

Voici par exemple comment l'on peut traduire dans {\LO} les
expressions de \ref{xpredicats} --~pour l'instant je n'y représente
pas explicitement les arguments, à la place je note, provisoirement,
un espace vide, \slot, à l'endroit qui leur est réservé :

\ex.  \label{xpredicats'}
\(\Xlo
\begin{array}[t]{l@{\qquad}l}
\prd{gentil}(\slot) & \prd{dormir}(\slot)\\
\prd{canard}(\slot) &  \prd{avoir-faim}(\slot)\\
\prd{acteur}(\slot) &  \prd{fumer}(\slot)\\
\prd{ventru}(\slot) & \prd{aimer}(\slot,\slot)\\
\prd{italien}(\slot) & \prd{connaître}(\slot,\slot)\\
\prd{frère-de}(\slot,\slot) & \prd{embrasser}(\slot,\slot)\\
\end{array}\)



Les symboles de prédicats en \ref{xpredicats'} sont dénommés par
des mots du français, mots qui reprennent ceux des expressions de
\ref{xpredicats}.  Mais ce n'est là qu'une simple commodité.  Pour
écrire ces prédicats dans {\LO}, nous aurions pu choisir d'emprunter des
mots de l'anglais ou du latin, etc., ou même d'utiliser des symboles
abstraits comme par exemple $\Xlo G_3$, $\Xlo D_1$, $\Xlo C_{14}$, $\Xlo F_7$, $\Xlo A$,
$\Xlo\heartsuit$, etc.  Cela ne ferait fondamentalement aucune différence.
Le choix des noms de symboles pour transcrire les prédicats est
complètement \emph{conventionnel} et \emph{arbitraire}.  Et c'est
simplement pour des raisons pratiques, de transparence et de facilité
de lecture, que nous choisirons ici de reprendre des mots du français
pour écrire les symboles de prédicats.  Il est alors très important de
bien faire la distinction entre le (symbole de) prédicat \prd{aimer}
--~que l'on peut aussi écrire $\Xlo\prd{aimer}(\slot,\slot)$~-- et le
verbe transitif  
\sicut{aimer}. Car le premier appartient au langage objet {\LO}, alors
que le second appartient à la langue naturelle qu'est le
français%
\footnote{Remarquez que, par conséquent, rien ne nous
  empêcherait formellement de traduire dans  {\LO} par exemple  l'adjectif
  français \sicut{gentil} par le symbole de prédicat 
  \prd{méchant} ou par le symbole \prd{manchot} ou encore
  $\prdi{schtroumpf}{257}$... Cela ne poserait aucun problème
  théorique ; seulement, dans la pratique, nos notations deviendraient
  alors  bêtement obscures.}.  
Pour conclure cette remarque, il faut
bien être conscient que, malgré les apparences, ce n'est pas le nom du
symbole de prédicat qui porte en soi la sémantique  du prédicat qu'il
représente, cette sémantique est définie par ailleurs (en
\S\ref{S:semCP} \alien{infra}).

La définition \ref{arite} pose qu'un prédicat donné a une arité fixe.
\label{H:aritéfixe}
On postule également qu'un prédicat donné de {\LO} a un et un seul
sens.  Cela a pour conséquence de multiplier le nombre de
prédicats susceptibles de traduire un mot donné du français.  Par
exemple \ref{xpredicats'} donne le prédicat \prd{fumer} (à un
argument) pour traduire le verbe intransitif \sicut{fumer}, mais
\sicut{fumer} a aussi un emploi transitif (comme dans \sicut{fumer
un cigare}) qui devra se traduire par un prédicat binaire.  Même si
ces deux verbes \sicut{fumer} sont sémantiquement reliés, on ne pourra
pas les traduire par le même prédicat \prd{fumer}, car il doit avoir
une arité fixe.  C'est pourquoi lorsque l'on a besoin de refléter dans
{\LO} la polyvalence ou la polysémie d'un mot du français, on peut
adopter une notation qui décore les symboles de prédicat d'indices
numériques pour distinguer les différentes acceptions.  Ainsi on
pourra écrire $\Xlo\prd{fumer}_{\prd{1}}(\slot)$ pour traduire le verbe
intransitif (signifiant être un fumeur),
$\Xlo\prdi{fumer}{2}(\slot,\slot)$ pour le verbe transitif (brûler
du tabac en aspirant la fumée), et pourquoi pas
$\Xlo\prdi{fumer}{3}(\slot)$ pour le sens de dégager de la fumée
(\sicut{la cheminée fume}), etc. Les indices n'ont aucune signification
en soi, ils servent juste à poser des distinctions dans les noms de
symboles de prédicats.


\subsubsection{Constantes d'individus}
%'''''''''''''''''''''''''''''''''''''

Les prédicats sont des symboles de {\LO} ; le terme d'argument, quant à
lui, ne désigne pas une catégorie de symboles mais plutôt un
\emph{rôle} que certains symboles peuvent jouer vis à vis des prédicats. 
Nous devons maintenant voir quels sont les éléments du langage objet qui
peuvent jouer ce rôle.  
Les exemples précédents nous ont montré que, au moins dans certains
cas, les arguments correspondent à des expressions qui dénotent des
choses, des personnes, des objets au sens large, ce que nous regroupons
ici sous l'appellation d'\kwi{individus}{individu}.  Dans {\LO}, ces
expressions s'appellent des \kwi{termes}{terme}\footnote{Ne confondons
pas termes et individus : les termes sont des symboles de {\LO}, les
individus sont les objets du monde.}.  Il existe deux
grands types de termes, et le premier que nous allons regarder est
celui des \kwo{constantes d'individus}\indexs{constante!\elid\ d'individu}.


Les constantes d'individus sont  l'équivalent dans {\LO} des
\emph{noms propres}\is{nom!\elid\  propre} de la langue naturelle.  Ainsi, comme un nom
propre, une constante désigne directement un individu du monde, à ceci
près que nos constantes n'auront pas l'ambiguïté que peuvent avoir
certains noms propres comme \sicut{Pierre Dupont}, \sicut{John Smith}
ou  \sicut{Nathalie Lebrun}, etc. Par définition, chaque constante
d'individu a une dénotation unique, et c'est notamment pour cela qu'on l'appelle
une constante.


\begin{nota}[Constantes d'individus]
Dans {\LO}, les constantes d'individus seront notées par des lettres
minuscules, prises plutôt au début de l'alphabet%
\footnotemark, et éventuellement
complétées par des indices numériques.  Elles seront également écrites
  en \cns{\textcolor{black}{gras}}.
\end{nota}
\footnotetext{Les lettres de la fin de l'alphabet, $u$, $v$, $w$, $x$,
  $y$, $z$, seront réservées à un autre usage.}

Par exemple si nous souhaitons parler des individus suivants :  Joey,
Chandler, Monica, Phoebe, Rachel, Ross, 
nous pourrons les désigner respectivement par les constantes  \cns{j},
\cns{c},  \cns{m},  \cns{p},  $\cnsi{r}{1}$,  $\cnsi{r}{2}$.  Là encore,
le choix des lettres est complètement arbitraire : il se trouve que
c'est  pratique de reprendre les initiales des prénoms traduits sous
formes de constantes, mais nous aurions pu tout aussi bien prendre
$\cnsi{a}{1}$, 
$\cnsi{a}{2}$, 
$\cnsi{a}{3}$, 
$\cnsi{a}{4}$, 
$\cnsi{a}{5}$, 
$\cnsi{a}6$.

\sloppy
Remarquez aussi que notre convention d'écriture n'est pas innocente :
les constantes d'individus seront notées comme les prédicats, en
caractères gras, car en fait les prédicats sont considérés eux aussi comme
des constantes ; et on parle alors de constantes prédicatives (ou
constantes de
prédicats)\indexs{constante!\elid\ de prédicat} par opposition aux constantes d'individus.  Ces deux types
de constantes (c'est-à-dire tout ce que nous écrirons en gras dans
{\LO}) sont regroupés sous le terme de \kwo{constantes non logiques}\indexs{constante!\elid\ non logique}%
\footnote{Constantes non logiques, car il existe encore d'autres
  constantes, qui sont les constantes logiques\indexs{constante!\elid\ logique@\elid\ logique} et qui comprennent
  notamment les symboles de connecteurs et de quantification que nous
  allons voir en \S\ref{ss:ConnLog} et \S\ref{ss:vQ}.}. 

\fussy

Avec ces premiers éléments de {\LO}, nous pouvons déjà commencer à traduire
quelques phrases simples du français (la flèche {\trad} servira à
indiquer les traductions du français vers le langage objet).  Il
suffit pour cela de placer des constantes dans les positions
argumentales de prédicats :

\ex. \label{x:phrases}
\a.  Joey a faim {\trad} \(\Xlo\prd{avoir-faim}(\cns{j})\)
\b. Joey est acteur {\trad} \(\Xlo\prd{acteur}(\cns{j})\)
\c. Chandler fume {\trad} \(\Xlo\prd{fumer}(\cns{c})\)
\d. Ross aime Rachel {\trad} \(\Xlo\prd{aimer}(\cnsi{r}2,\cnsi{r}1)\)



Lorsque l'on fournit des arguments à un prédicat, on dit alors que
l'on \emph{sature} le prédicat, et on obtient une expression qui peut être
vraie ou fausse.  De telles expressions, qui ont donc bien le type de
dénotation des phrases déclaratives, s'appellent des
\kwi{formules}{formule}.   Les formules sont les «phrases» de {\LO}.

\subsubsection{Connecteurs logiques}
%'''''''''''''''''''''''''''''''''''
\label{ss:ConnLog}
\indexs{connecteur!\elid\ logique|(}

Certes, pour l'instant, les formules que nous pouvons écrire dans
{\LO} sont un peu rudimentaires.  Pour gagner en sophistication nous
avons besoin d'augmenter notre vocabulaire.  Une manière d'obtenir des
formules plus complexes est de \emph{combiner} entre elles des
formules simples.  La manière la plus basique de procéder à ces
combinaisons utilise des \kwo{connecteurs logiques}\indexs{connecteur!\elid\ logique}.  Les connecteurs
sont en quelque sorte le pendant en {\LO} des conjonctions de
coordination d'une langue comme le français.  Mais il sera dès à
présent prudent de ne pas trop pousser la comparaison. Car d'une part
les connecteurs de {\LO} ne «coordonnent» que des formules,
c'est-à-dire des phrases.  Et d'autre part, toutes les conjonctions du
français ne se traduisent 
pas forcément par un connecteur logique\footnote{Nous reviendrons plus
  en détail sur l'explication de ce point en \S\ref{ConnVfctel}.}.



\begin{defi}[Connecteur logique]
Un connecteur logique est un symbole de {\LO} qui permet d'assembler
deux formules pour former une nouvelle formule.\\
Une formule est dite complexe si elle est construite à l'aide d'un ou
plusieurs connecteurs.
\end{defi}

Les connecteurs que nous allons utiliser dans {\LO} sont au nombre de
quatre.  Le premier s'appelle la \kwi{conjonction}{conjonction}, il sera noté par le symbole $\Xlo\wedge$ %\indexs{$\wedge$} 
et
il traduit la conjonction de coordination \sicut{et} du français.  On
l'appelle d'ailleurs aussi le «et logique», car la conjonction est
ce connecteur qui permet de construire une formule (ou phrase) qui
sera vraie si et seulement si les formules qu'il connecte sont toutes
deux vraies.

Le second connecteur est la \kwi{disjonction}{disjonction},
il est noté par le symbole $\Xlo\vee$ %\indexs{$\vee$} 
et traduit la
conjonction française \sicut{ou}.  C'est donc le «ou logique»,
et il construit une formule qui est vraie si au moins l'une des deux
formules qu'il connecte est vraie.

Le troisième connecteur est l'\kwo{implication}\indexs{implication!\elid\ matérielle} %{implication matérielle ($\implq$)} 
dite \kwo{matérielle}.  Il traduit notamment la relation de
condition que l'on exprime en français avec des structures en
\sicut{si..., alors...} ou simplement \sicut{si..., ...}  On
le note par le symbole $\Xlo\implq$. %\indexs{$\implq$}  
Attention : $\Xlo\implq$
ne se place pas «au même endroit» que \sicut{si} en français ; en
{\LO}, on
place $\Xlo\implq$ entre deux formules.

Enfin le dernier est l'\kwo{équivalence}\indexs{equivalence@équivalence!\elid\ matérielle} dite aussi \kwo{matérielle}.  Il correspond à l'expression
française \sicut{si et seulement si}, et se note par $\Xlo\ssi$
%\indexs{$\ssi$} 
(et on
comprend pourquoi ce connecteur est parfois appelé aussi la
\emph{bi-implication}).  Ce connecteur semble plus approprié pour
traduire des énoncés de type mathématique que pour rendre compte de la
sémantique de la langue naturelle, dans la mesure où la tournure
\sicut{si et seulement si} est assez marginale dans le discours
courant.  Mais nous le retenons malgré tout dans le jeu de nos
connecteurs car il sera utile pour formaliser certains éléments de
sémantique du français.

Voyons maintenant quelques exemples de phrases que nous pouvons
transcrire dans {\LO} au moyen de connecteurs :

\ex.\a.Ross aime Rachel et Chandler aime Monica.\\{\trad}
\(\Xlo[\prd{aimer}(\cnsi{r}2,\cnsi{r}1) \wedge \prd{aimer}(\cns{c},\cns{m})]\)
\b. Si Chandler a faim, il (Chandler) fume.\\ {\trad}
\(\Xlo[\prd{avoir-faim}(\cns{c}) \implq \prd{fumer}(\cns{c})]\)


Notez  l'usage des crochets $[\,]$
que nous avons introduit au passage  pour parenthéser chaque
connexion de formules.  Ces crochets sont d'une grande utilité
lorsqu'une formule complexe comprend plusieurs connecteurs, exactement
comme en mathématiques les parenthèses sont importantes pour marquer
la priorité d'une opération sur une autre (exemple : $(4+1)\times 2 \neq 4+(1\times2)$).

La finalité du langage objet  est de nous permettre d'expliciter le
sens des énoncés, et pas seulement de refléter leur construction
syntaxique ou grammaticale.  C'est pourquoi les connecteurs vont
également nous servir à faire des traductions analytiques comme en
\ref{x:mpred}.

\ex.  \label{x:mpred}
\a. Joey est un acteur italien.\\\label{x:actital}%
 {\trad} \(\Xlo[\prd{acteur}(\cns{j}) \wedge \prd{italien}(\cns{j})]\)
\b. Ross et Rachel s'aiment.\\\label{x:rrsaiment}%
 {\trad} \(\Xlo[\prd{aimer}(\cnsi{r}2,\cnsi{r}1) \wedge
 \prd{aimer}(\cnsi{r}1,\cnsi{r}2)]\) 

 
En effet, la phrase \ref{x:actital} par exemple nous dit bien
conjointement deux 
choses : que l'individu nommé Joey est acteur \emph{et} qu'il est
italien.  De même \ref{x:rrsaiment} dit que Ross aime Rachel et que
Rachel aime Ross.  Notons que \ref{x:rrsaiment} est en fait ambiguë :
Ross et Rachel peuvent s'aimer soi-même sans s'aimer l'un l'autre.
Dans ce cas on traduira la phrase par
\(\Xlo[\prd{aimer}(\cnsi{r}2,\cnsi{r}2) \wedge
  \prd{aimer}(\cnsi{r}1,\cnsi{r}1)]\). Cela illustre que {\LO}
fait bien son travail d'explicitation du sens des phrases : si une
phrase a plusieurs sens, elle reçoit différentes traductions en {\LO}.

Enfin pour compléter au moins minimalement l'expressivité de {\LO},
nous allons avoir besoin de pouvoir traduire des phrases négatives,
c'est-à-dire de \emph{nier}  des propos ou, plus précisément, des
formules.  On fabrique des formules négatives dans {\LO} grâce à
l'opérateur de \kwi{négation}{negation@négation} que nous avons déjà
vu et qui se note par le symbole $\Xlo\neg$.  %\indexs{$\neg$}
Il
ne s'agit plus à proprement parler d'un \emph{connecteur} logique : il
ne connecte rien, il se place devant une formule pour former sa
négation.   Mais comme les connecteurs présentés \alien{supra}, la
négation «agit» sur des formules et sa contribution sémantique est
aussi définie en termes de valeurs de vérité : la négation est cet
opérateur qui rend vraies les formules fausses et fausses les formules
vraies.  

\ex. \label{x:phraseneg}
\a. Phoebe n'a pas faim.\\
 {\trad} \(\Xlo\neg \prd{avoir-faim}(\cns{p})\)
\b. Si Chandler dort, il ne fume pas.\\
 {\trad}
\(\Xlo[\prd{dormir}(\cns{c}) \implq \neg\prd{fumer}(\cns{c})]\)


\indexs{connecteur!\elid\ logique|)}


\subsubsection{Variables et quantificateurs}
%'''''''''''''''''''''''''''''''''''''''''''
\label{ss:vQ}

Jusqu'ici, dans {\LO} pour parler des individus, de ces choses sur
lesquelles on fait porter les prédicats, nous ne disposons que des
constantes d'individus, qui rappelons-le sont en quelque sorte les
noms propres du langage objet.  Or, nous nous en doutons bien, nous pouvons
difficilement nous permettre de donner un nom propre à tout individu du
monde que nous souhaiterions évoquer dans des énoncés. Ce ne serait pas
très réaliste, car rappelons aussi que nous comptons également les
objets inanimés parmi ce que nous appelons ici les individus.  Même si
certains objets reçoivent les honneurs de l'onomastique (comme par
exemple la Tour Eiffel, Saturne, Hubble, Durandal, Deep Blue, etc.),
la pratique n'est pas courante et surtout absolument pas
indispensable.  En effet, la langue nous permet d'évoquer des
individus, y compris des êtres humains ou animaux, \emph{sans}
mentionner leur nom.  Soit que nous ne le connaissons pas, soit que nous
n'avons pas envie ou pas besoin de faire mention du nom de l'individu en
question.  On utilise à cet effet des procédés linguistiques
 bien connus : c'est l'usage que l'on fait des syntagmes
nominaux, comme par exemple \sicut{une chaise}, \sicut{le voisin d'en face}, \sicut{un
  employé de banque}, \sicut{les pains au lait}, etc.  Et nous devons
donc nous donner les moyens dans {\LO} de faire référence aux
individus de façon similaire. 

Pour ce faire, nous avons d'abord besoin d'un autre type de termes,
les \kwi{variables}{variable}. Comme les constantes, les variables
dénotent des individus.  Mais alors qu'une constante donnée dénotera
toujours le même individu précis, par un lien conventionnel et permanent, 
une variable permettra de 
désigner tout et n'importe quoi, sans contrainte particulière \emph{a
priori}.  
%Les variables sont en quelque sorte des désignateurs anonymes et
%libres.  
D'où leur nom, variables, car ce qu'elles désignent peut
varier librement \emph{a priori}.  «~\emph{A priori}~» signifie que
si ce que désigne une variable est contraint dans une phrase, ce n'est
pas la variable en soi qui pose cette contrainte.  Ou pour dire encore
les choses autrement, on sait qu'une variable est censée dénoter
quelque chose, mais on ne sait pas quoi, du moins pas
systématiquement.  C'est pourquoi les variables de {\LO} fonctionnent
un peu comme les pronoms personnels de la langue naturelle
(et plus précisément les pronoms de troisième personne, \sicut{il},
\sicut{elle}, \sicut{le}, \sicut{la}, \sicut{lui}, etc.).

\begin{nota}[Variables]
Dans {\LO}, les variables seront notées par des lettres, de préférence
minuscules, en italiques, prises à la fin de l'alphabet : principalement
$\vrb x$, $\vrb y$, $\vrb z$, mais aussi parfois $\vrb u$, $\vrb v$, $\vrb w$.  Elles pourront être
éventuellement complétées d'indices numériques : $\vrbi x1$, $\vrbi x2$, $\vrbi x3$,
$\vrbi y1$..., ou de «primes» : $\vrb{x'}$, $\vrb{x''}$...
\end{nota}

Voici alors une manière simple de traduire dans {\LO} une phrase française
contenant un pronom personnel\footnote{Notez cependant que
  contrairement aux pronoms du français, les variables de {\LO} ne
  portent pas le trait grammatical de genre, de nombre ou de personne.} :\is{pronom}

\ex.  \label{x:xdort}
Il dort.
{\trad}  \(\Xlo\prd{dormir}(x)\)





Cependant les variables ne vont pas être utilisées seulement pour
traduire les pronoms de la langue dans {\LO}.  Elles  nous
servent à établir des références anonymes aux choses dont nous voulons
parler.  
Mais pour compenser le manque d'information dû à cet anonymat, nous
avons besoin  d'apporter d'autres éléments qui vont contraindre 
%et discipliner 
la référence des variables.  Pour ce faire on utilise des
prédicats (comme en \ref{x:xdort}), mais aussi des opérateurs
logiques particuliers qui imposent aux variables un «mode de
variation» précis.  Ce sont les \kwi{quantificateurs}{quantificateur}, qu'on appelle
aussi (plus précisément d'ailleurs) les \kw{symboles de
  quantification}.  Les deux quantificateurs de la logique classique
(sur laquelle est fondé {\LO}) sont :

\begin{itemize}
\item le \kwi{quantificateur existentiel}{quantificateur!{\elid}
  existentiel ($\exists$)}, noté $\Xlo\exists$, %\indexs{$\exists$}
qui impose à
  une variable de dénoter \emph{au moins un individu} ; ainsi $\Xlo\exists x$...
  se lit «il existe un $x$ tel que...» ;
\item le \kwi{quantificateur universel}{quantificateur!{\elid}
  universel ($\forall$)}, noté $\Xlo\forall$, %\indexs{$\forall$}
qui impose à
  une variable de dénoter \emph{successivement tous les individus} ;
  ainsi $\Xlo\forall x$... se lit «quel que soit $x$...» ou «pour
  tout $x$...».
\end{itemize}


%les pronoms ne constituent pas le seul moyen de faire de la référence
%anonyme.  

Voyons tout de suite des exemples :  

\ex.\a.  \(\Xlo\exists x\, \prd{acteur}(x)\) {\rtrad} il existe un $x$ (\ie\ un
individu ou une chose quelconque) qui est un acteur ou tel que $x$ est
un acteur : il existe
un acteur, ou il y a un acteur. 
\b. \(\Xlo\exists x\, [\prd{acteur}(x) \wedge \prd{fumer}(x)]\) {\rtrad}  il
existe un $x$ qui est un acteur et qui fume : un acteur fume.
\c. \(\Xlo\exists x\, [\prd{acteur}(x) \wedge \prd{connaître}(\cns{m},x)]\)
{\rtrad} 
Monica connaît un acteur.
\d. \(\Xlo\exists x\, \prd{aimer}(x,\cns{p})\) {\rtrad} il y a quelqu'un qui aime Phoebe.


\ex.\label{x:exforall}
\a. \(\Xlo\forall x\, \prd{avoir-faim}(x)\) {\rtrad} quel que soit $x$,
$x$ a faim : tout le monde à faim.\label{x:exforalla}
\b. \(\Xlo\forall x\, [\prd{acteur}(x) \implq \prd{fumer}(x)]\) {\rtrad}
quel que soit $x$, si $x$ est un acteur, alors $x$ fume : tous
les acteurs fument.



Remarquons cependant que \ref{x:exforalla} n'est pas une traduction très
rigoureuse de \emph{tout le monde à faim}, car en français
\sicut{tout le monde} ne concerne que les êtres humains.  Pour un
bonne traduction, il faut employer un prédicat signifiant «être
humain», par exemple \prd{humain}, et écrire $\Xlo\forall x
[\prd{humain}(x) \implq \prd{avoir-faim}(x)]$.

\subsection{La syntaxe du langage formel}
%----------------------------------------
\label{s:syntlf}

Nous allons à présent faire la synthèse que ce que nous venons de voir
sur le langage formel {\LO} en définissant
précisément ce langage.   En particulier nous devons définir
précisément comment les éléments vus précédemment s'organisent au sein
du langage.  Commençons par en redonner la liste, ils forment le
\emph{vocabulaire} de notre langage, ses atomes, sa matière première.
\largerpage

\begin{defi}[Vocabulaire de \LO]\label{def:vocLO}
Le vocabulaire de {\LO} comporte :
\begin{itemize}
\item un ensemble de variables : \set{\vrb x;\vrb y;\vrb
  z;\vrb{x_1};\vrb{x_2};\dotsc;\vrb{y_1};\vrb{y_2};\dots} ;
%****
\item un ensemble de constantes d'individus :\\ 
\set{\cns{a};\cns{b};\cns{c};\cns{d};\dotsc;\cnsi{a}1;\cnsi{a}2;\cnsi{a}3;\dotsc;\cnsi{b}1;\cnsi{b}2;\dots} ;
\item un ensemble de constantes de prédicats :\\
 \set{\prd{acteur};\prd{gentil};\prd{dormir};\prd{canard};\prd{aimer};\prd{connaître};\dots} ;
\item un ensemble de symboles logiques : \set{\xlo{\neg}; \xlo{\wedge}; \xlo{\vee} ;
\xlo{\implq}; \xlo{\ssi}; \xlo{=}; \xlo{\forall}; \xlo{\exists}} ;
\item les crochets $\Xlo[\,]$ et les parenthèses $\Xlo(\,)$.
\end{itemize}
\end{defi}

Notons que nous avons ajouté le symbole $\Xlo =$ parmi les symboles
logiques ; il servira à représenter l'identité,\is{identite@identité} c'est-à-dire l'égalité
de dénotation.

En principe, les ensembles de variables et de constantes d'individus
et de prédicats sont supposés être finis (ce qui est bien raisonnable,
nous n'allons pas manipuler un langage au vocabulaire infini) et être
présentés exhaustivement, ce que masque l'usage des points de
suspension dans la définition ci-dessus.
%\footnote
La tradition
«montagovienne»%
\footnote{\emph{Montagovien} signifie relatif à Montague ou à ses travaux.} 
a l'habitude (depuis les travaux de Montague
lui-même) de proposer des \emph{fragments} de langage, c'est-à-dire
des sous-parties (très) petites %\emph{mais complètes} 
de ce que serait
un langage objet réaliste de l'envergure d'une langue naturelle.
Procéder par fragments constitue une méthodologie de travail très
rigoureuse car tout y est complètement et précisément défini, aucun
élément du langage ne risque de rester dans l'ombre ou le vague.  Je
ne compte pas ici m'autoriser une quelconque mollesse ou désinvolture
dans la méthode de travail, simplement, pour des raisons de clarté et
de commodité, je me permettrai d'utiliser des ensembles de
constantes et de variables «ouverts», c'est-à-dire suffisamment grands pour
permettre de la variété dans les exemples de phrases traduites en
{\LO}.  Autant que faire se peut, chaque nouveau prédicat introduit
dont la sémantique ne serait pas immédiate sera explicitement commenté.

Un autre élément d'information attaché au vocabulaire et qui
n'apparaît pas dans la définition \ref{def:vocLO} doit être précisé : nous faisons
l'hypothèse que pour chaque constante de prédicat nous connaissons son
arité, et qu'elle est unique.  Cela veut dire que l'ensemble de
symboles de prédicat est en fait partitionné en plusieurs sous-ensembles
(celui des prédicats unaires, celui des prédicats binaires etc.) et
que nous connaissons cette partition.
Cette hypothèse est importante et
précieuse, car dans nos notations  rien ne nous indique formellement le
nombre d'arguments qu'attend un prédicat\footnote{Nous verrons comment
  perfectionner tout cela au chapitre \ref{ch:types}.}. 

Nous allons également en profiter pour définir tout de suite la notion de
\emph{terme}, qui va 
nous être utile par la suite.  Il s'agit juste de regrouper sous une
même appellation les variables et les constantes.

\begin{defi}[Termes]
Les variables et les constantes d'individus sont des \kwi{termes}{terme}.
\label{SynPTermes}
\end{defi}

Définir un langage, au sens des langages
formels, consiste à indiquer, {d'une manière ou d'une autre},
toutes les séquences  correctes
qu'admet ce langage.  Et bien entendu, la manière que l'on retient
consiste à spécifier l'ensemble des règles qui permettent de
construire n'importe quelle séquence admissible, c'est-à-dire la
\emph{syntaxe} du langage.  Nous reconnaissons là, naturellement, la démarche
de la grammaire générative, sauf que
nous n'allons pas ici exprimer les règles syntaxiques\is{regle@règle!\elid\ syntaxique} sous formes de
règles de réécriture  comme dans la tradition chomskienne par exemple,
mais par ce que l'on appelle la méthode inductive.  Définir une
syntaxe par induction consiste à spécifier des règles (ou des
«recettes») qui indiquent comment construire récursivement des
formules de plus en plus complexes à partir de formules plus simples
que l'on sait déjà construire.

La définition~\ref{SynP} présente la syntaxe de {\LO}, et c'est en
fait ni plus ni moins que la définition détaillée et complète de ce
qu'est une formule de {\LO}.  Dans ces règles, j'utilise des lettres
grecques, $\vrb\alpha$, $\vrb\beta$, $\vrb\gamma$, $\vrb\phi$, $\vrb\psi$, ainsi que $\vrb P$
pour désigner des expressions plus ou moins quelconques de {\LO}.  Ces
symboles en soi ne font pas partie de {\LO} (tous les symboles de
{\LO} sont donnés par le vocabulaire~\ref{def:vocLO}), ce sont des
sortes de macro-symboles ou méta-symboles qui nous permettent de dire
des choses générales sur {\LO}.  Ainsi, en toute rigueur, on ne
devrait pas dire que $\vrb\phi$ est une formule, mais que $\vrb\phi$
représente une formule (cependant, par la suite, nous nous
autoriserons quelque liberté de langage en qualifiant $\vrb\phi$ de formule).  


\begin{defi}[Syntaxe de LO]
%^^^^^^^^^^^^^^^^^^^^
\label{SynP}
\begin{enumerate}[syn,series=RglSyn1] %[(\RSyn1)]
\item 
\begin{enumerate}
\item Si $\Xlo\alpha$ est un terme et $\Xlo P$ un symbole de prédicat à une
place, alors $\Xlo P(\alpha)$ est une formule ;
\item Si $\Xlo\alpha$ et $\Xlo\beta$ sont des termes et $\Xlo P$ un symbole de
prédicat à deux places, alors $\Xlo P(\alpha,\beta)$ est une formule ;
\item Si $\Xlo\alpha$, $\Xlo\beta$ et $\Xlo\gamma$ sont des termes et $\Xlo P$ un symbole de
prédicat à trois places, alors $\Xlo P(\alpha,\beta,\gamma)$ est une
formule ;
\item etc.
\end{enumerate}
\label{SynPApp}
\item Si $\Xlo\alpha$ et $\Xlo\beta$ sont des termes, alors $\Xlo\alpha=\beta$
     est une formule ;
\label{SynP=}
\item Si $\Xlo\phi$  est une formule, alors
 \(\Xlo\neg\phi\) est une formule ;
\label{SynPNeg}
\item Si $\Xlo\phi$ et $\Xlo\psi$  sont des formules, alors
      \(\Xlo[\phi \wedge \psi], [\phi \vee \psi], [\phi \implq
     \psi]\) et \(\Xlo[\phi \ssi \psi]\) sont des formules ;
\label{SynPConn}
\item Si $\Xlo\phi$ est une formule  et $\vrb v$ une variable, alors $\Xlo\forall v \phi$ et
     $\Xlo\exists v \phi$ sont des formules.
\label{SynPQ}
\setcounter{RglSynt}{\value{enumi}}
\end{enumerate}
\end{defi}

Il convient ensuite de conclure cet ensemble de règles par une
dernière règle de \emph{clôture} qui dit ceci : rien d'autre que ce qui
peut être construit en un nombre fini d'étapes par les règles de
(\RSyn\ref{SynPApp}) à 
(\RSyn\ref{SynPQ}) n'est une formule.  Cela permet de  garantir
que notre syntaxe définit bien \emph{toutes} les formules de {\LO}.

Les règles (\RSyn\ref{SynPApp}) sont celles qui permettent de
construire les formules les plus simples (dites atomiques) en
fournissant aux symboles de prédicat des arguments en quantité
nécessaire.  La règle (\RSyn\ref{SynP=}) construit aussi des formules
simples, qui pose une identité\is{identite@identité} entre deux termes ; en fait le symbole $\Xlo=$
pourrait tout aussi bien être rangé parmi les prédicats binaires (on
devrait alors plutôt écrire $\Xlo\mathord{=}(\alpha,\beta)$) et
(\RSyn\ref{SynP=}) ne présente qu'une variante particulière de ce que
recouvre (\RSyn\ref{SynPApp}b) ; simplement on préfère utiliser la
notation habituelle, plus claire et naturelle, pour le symbole $\Xlo=$.
Les règles (\RSyn\ref{SynPNeg}) et (\RSyn\ref{SynPConn}) introduisent
les connecteurs logiques dans la construction des formules ; rappelons
que (pour le moment) les crochets $\Xlo[\,]$ sont indispensables dans
(\RSyn\ref{SynPConn}). Enfin la (double) règle (\RSyn\ref{SynPQ})
introduit les symboles de quantification ; notons que cette règle
autorise d'accoler une séquence quantificateur--variable devant
n'importe qu'elle formule $\vrb\phi$, c'est-à-dire qu'on n'a pas à
vérifier si la variable $\vrb v$ figure ou non dans $\vrb\phi$ (nous y
reviendrons \alien{infra}).

Illustrons à présent le fonctionnement de notre syntaxe avec un exemple.  La
syntaxe indique comment construire correctement des formules et en
même temps comment reconnaître si des séquences de symboles du vocabulaire
de {\LO} sont ou non des formules bien formées (ce que l'on appelle
aussi des expressions bien formées ou \acro{ebf}).  Essayons donc de
montrer que \ref{x:fbf} est bien formée.

\ex.  \label{x:fbf}
\(\Xlo\exists x \forall y [\prd{aimer}(\cns c,x) \wedge \prd{gentil}(y)]\)


Pour vérifier la bonne formation de \ref{x:fbf}, il suffit de la
reconstruire en appliquant les règles de la syntaxe, et seulement
elles, qui sont nécessaires.  On peut également déconstruire la formule
en appliquant les règles syntaxiques, mais dans l'autre sens, jusqu'à
ce que l'on n'obtienne  que des éléments du vocabulaire de base.  Voici
les étapes de construction de \ref{x:fbf} :

\begin{itemize}
\item \prd{aimer} est un prédicat à deux places, \cns c est une constante et
$\vrb x$ est une variable, donc $\Xlo\prd{aimer}(\cns c,x)$ est une formule
bien formée en vertu de la règle \RSyn\ref{SynPApp}b ;
\item de même, \prd{gentil} est un prédicat à une place et $\vrb y$ est une
  variable, donc $\Xlo\prd{gentil}(y)$ est une formule bien formée, en
  vertu de la règle \RSyn\ref{SynPApp}a ;
\item donc en vertu de la règle \RSyn\ref{SynPConn}, avec le
  connecteur $\Xlo\wedge$, $\Xlo[\prd{aimer}(\cns c,x) \wedge \prd{gentil}(y)]$
  est une formule bien formée ;
\item donc, comme $\vrb y$ est une variable, \(\Xlo\forall y [\prd{aimer}(\cns
  c,x) \wedge \prd{gentil}(y)]\) est une formule bien formée, en vertu
  de la règle \RSyn\ref{SynPQ} pour $\Xlo\forall$ ;
\item et donc, comme $\vrb x$ est une variable,  \(\Xlo\exists x \forall y
  [\prd{aimer}(\cns c,x) \wedge \prd{gentil}(y)]\) est une formule
  bien formée, en vertu de la règle \RSyn\ref{SynPQ} pour
  $\Xlo\exists$. %\hfl \alien{qed}.
\end{itemize}


\largerpage[-1] 
On peut illustrer graphiquement une telle démonstration à l'aide de ce
qu'on appelle l'\kwo{arbre de construction}\is{arbre!\elid\ de construction d'une formule} d'une formule.  L'arbre de
\ref{x:fbf} est donné en Figure~\ref{f:Axfbf}.  


\begin{figure}[h]
\begin{center}
\leaf{$\vrb x$}
\leaf{$\vrb y$}
\leaf{\prd{aimer}}
\leaf{\cns c}
\leaf{$\vrb x$}
\branch{3}{\zrbox{$_{\text{\RSyn\ref{SynPApp}b}}$\;}\(\Xlo\prd{aimer}(\cns c,x)\)}
\leaf{\prd{gentil}}
\leaf{$\vrb y$}
\branch{2}{\(\Xlo\prd{gentil}(y)\)\zbox{ $_{\text{\RSyn\ref{SynPApp}a}}$}}
\branch{2}{\(\Xlo[\prd{aimer}(\cns c,x) \wedge \prd{gentil}(y)]\)\zbox{ $_{\text{\RSyn\ref{SynPConn}}\; (\wedge)}$}}
\branch{2}{\(\Xlo\forall y [\prd{aimer}(\cns c,x) \wedge \prd{gentil}(y)]\)\zbox{ $_{\text{\RSyn\ref{SynPQ}}\; (\forall)}$}}
\branch{2}{\(\Xlo\exists x \forall y [\prd{aimer}(\cns c,x) \wedge
    \prd{gentil}(y)]\)\zbox{ $_{\text{\RSyn\ref{SynPQ}}\; (\exists)}$}}
\qobitree
\end{center}
\caption{Arbre de construction de \ref{x:fbf}}\label{f:Axfbf}
\end{figure}


Chaque n\oe ud de
l'arbre représente une étape de la construction de la formule, la
racine (ou n\oe ud supérieur) correspondant à la formule complète.
Chaque branche et embranchement représente l'application d'une règle
syntaxique, et les n\oe uds directement inférieurs dans un
embranchement représentent les «ingrédients» à partir desquels on
construit ce qui est dans le n\oe ud directement supérieur.


Il y a un théorème qui dit que toute formule bien formée est
représentable par un et un seul arbre de construction.  Donc pour
montrer qu'une séquence donnée est une formule bien formée, il suffit
de dessiner son arbre de construction.  Et si on n'y parvient pas,
c'est que la séquence n'est pas une formule bien formée\footnote{Ou
  bien qu'on s'est trompé dans l'analyse !}.  Ainsi on peut montrer que
\ref{x:nfbf} n'est pas bien formée, car son arbre de construction complet
(Figure~\ref{f:nfbf}) n'est pas possible.

\ex.  \label{x:nfbf}
\(\Xlo\neg \prd{aimer}(\cns bx)\)


\begin{figure}[h]
\begin{center}
\leaf{???}
\branch{1}{\(\Xlo\prd{aimer}(\cns bx)\)}
\branch{1}{\(\Xlo\neg \prd{aimer}(\cns bx)\)\zbox{ $_{\text{\RSyn\ref{SynPNeg}}}$}}
\qobitree
\end{center}
\caption{Échec de la construction de \ref{x:nfbf}}\label{f:nfbf}
\end{figure}


Aucune règle de la syntaxe ne permet de construire \(\Xlo\prd{aimer}(\cns
bx)\) ; celle qui s'en rapprocherait le plus serait
(\RSyn\ref{SynPApp}b), mais elle exige de placer une virgule entre deux
arguments d'un prédicat binaire.


En revanche, \ref{x:fbf2} est parfaitement bien formée (comme le
prouve l'arbre en Figure~\ref{f:fbf2}), même si elle semble un peu
bizarre ; mais cela n'est (et ne doit être) qu'une impression, car
\ref{x:fbf2} ne pose aucun problème pour le système du calcul des
prédicats\footnote{Nous verrons, en examinant la sémantique de {\LO},
  que cette formule est en fait équivalente à $\Xlo\prd{aimer}(\cns a,\cns
  b)$.}. 

\ex.  \label{x:fbf2}
\(\Xlo\exists y\, \prd{aimer}(\cns a,\cns b)\)


\begin{figure}[h]
\begin{center}
\leaf{$\vrb y$}
\leaf{\prd{aimer}}
\leaf{\cns a}
\leaf{\cns b}
\branch{3}{\(\Xlo\prd{aimer}(\cns a,\cns b)\)}
\branch{2}{\(\Xlo\exists y\, \prd{aimer}(\cns a,\cns b)\)}
\qobitree
\end{center}
\caption{Arbre de construction de \ref{x:fbf2}}\label{f:fbf2}
\end{figure}

Le concept d'arbre de construction d'une formule permet de définir
facilement une notion qui sera très utile par la suite, celle de
\kw{sous-formule} d'une formule.  Une sous-formule est tout simplement
une formule bien formée que l'on trouve à l'intérieur d'une formule
plus grande.

\begin{defi}[Sous-formule]
On dit que $\vrb\psi$ est une \kw{sous-formule} de la formule $\vrb\phi$ si et
seulement si $\vrb\psi$ est une formule qui apparaît dans l'arbre de
construction de $\vrb\phi$.
\end{defi}

Ainsi \(\Xlo\prd{aimer}(\cns c,x)\),
\(\Xlo\prd{gentil}(y)\),
\(\Xlo[\prd{aimer}(\cns c,x) \wedge \prd{gentil}(y)]\)
et
\(\Xlo\forall y [\prd{aimer}(\cns c,x) \wedge \prd{gentil}(y)]\)
sont des sous-formules de \ref{x:fbf}, mais pas \(\Xlo\exists x \forall
y\, \prd{aimer}(\cns c,x)\). 

\smallskip

% -*- coding: utf-8 -*-
\begin{exo}\label{e:LCPebf}
Parmi les séquences suivantes, lesquelles sont des formules correctes
de {\LO}? \pagesolution{crg:LCPebf}
\begin{exolist}
\item \(\Xlo\exists z\, [[\prd{connaître}(\cnsi{r}2,z) \wedge
\prd{gentil}(z)] \wedge \neg\prd{dormir}(z)]\)
\item \(\Xlo\exists x \forall y \exists z\, [\prd{aimer}(y,z) \vee
\prd{aimer}(z,x)]\)
\item \(\Xlo\forall x y\, \prd{aimer}(x,y)\)
\item \(\Xlo\neg\neg \prd{aimer}(\cnsi{r}1,\cns{m})\)
\item \(\Xlo\exists x \neg\exists z\, \prd{connaître}(x,z)\)
\item \(\Xlo\exists x [\prd{acteur}(x) \wedge \prd{dormir}(x) \vee
  \prd{avoir-faim}(x)] \)
\item \(\Xlo[\prd{aimer}(\cns{a},\cns{b}) \implq \neg\prd{aimer}(\cns{a},\cns{b})]\)
\item \(\Xlo\exists y [\prd{acteur}(x) \wedge \prd{dormir}(x)]\)
\end{exolist}
%'''''''''''
\begin{solu} (p.~\pageref{e:LCPebf}) \label{crg:LCPebf}
%Formules bien formées de {\LO}.

L'exercice consiste à produire l'arbre de construction (cf. p.~\pageref{f:Axfbf}) de chaque séquence en appliquant les règles syntaxiques de la définition \ref{SynP}, p.~\pageref{SynP}.

\begin{exolist}
\item \(\Xlo\exists z\, [[\prd{connaître}(\cnsi{r}2,z) \wedge
\prd{gentil}(z)] \wedge \neg\prd{dormir}(z)]\)

Cette formule est bien formée, on le montre à l'aide de son arbre de
construction:

\begin{center}
{\small
\leaf{\(\prd{connaître}\)}
\leaf{\(\cnsi{r}2\)}
\leaf{\(\vrb z\)}
\branch{3}{\(\Xlo\prd{connaître}(\cnsi{r}2,z)\)}
\leaf{\(\prd{gentil}\)}
\leaf{\(\vrb z\)}
\branch{2}{\(\Xlo\prd{gentil}(z)\)}
\branch{2}{\(\Xlo[\prd{connaître}(\cnsi{r}2,z) \wedge \prd{gentil}(z)]\)}
\leaf{\(\prd{dormir}\)}
\leaf{\(\Xlo z\)}
\branch{2}{\(\Xlo\prd{dormir}(z)\)}
\branch{1}{\(\Xlo\neg\prd{dormir}(z)\)}
\branch{2}{\(\Xlo[[\prd{connaître}(\cnsi{r}2,z) \wedge \prd{gentil}(z)] \wedge \neg\prd{dormir}(z)]\)}
\branch{1}{\(\Xlo\exists z\, [[\prd{connaître}(\cnsi{r}2,z) \wedge
      \prd{gentil}(z)] \wedge \neg\prd{dormir}(z)]\)}
\qobitree}
\end{center}

\item \(\Xlo\exists x \forall y \exists z\, [\prd{aimer}(y,z) \vee \prd{aimer}(z,x)]\)

Cette formule est bien formée:

\begin{center}
{\small
\leaf{\(\prd{aimer}\)}
\leaf{\(\Xlo y\)}
\leaf{\(\Xlo z\)}
\branch{3}{\(\Xlo\prd{aimer}(y,z)\)}
\leaf{\(\prd{aimer}\)}
\leaf{\(\Xlo z\)}
\leaf{\(\Xlo x\)}
\branch{3}{\(\Xlo\prd{aimer}(z,x)\)}
\branch{2}{\(\Xlo[\prd{aimer}(y,z) \vee \prd{aimer}(z,x)]\)}
\branch{1}{\(\Xlo\exists z\, [\prd{aimer}(y,z) \vee \prd{aimer}(z,x)]\)}
\branch{1}{\(\Xlo\forall y \exists z\, [\prd{aimer}(y,z) \vee \prd{aimer}(z,x)]\)}
\branch{1}{\(\Xlo\exists x \forall y \exists z\, [\prd{aimer}(y,z) \vee \prd{aimer}(z,x)]\)}
\qobitree}
\end{center}

\item \(\Xlo\forall x y\, \prd{aimer}(x,y)\)

Cette séquence  n'est pas une formule bien formée.  En effet $\Xlo\forall x$
peut être introduit par la règle (\RSyn\ref{SynPQ}), mais cette règle
doit opérer sur une \emph{formule}.  Or \(\Xlo\forall x y\,
\prd{aimer}(x,y)\) se décomposerait alors en \(\Xlo\forall x\) et \(\Xlo y\,
\prd{aimer}(x,y)\) et cette seconde expression n'est pas une formule
bien formée; on ne peut pas construire \(\Xlo y\, \prd{aimer}(x,y)\),
aucune règle n'autorise à placer une variable seule devant une
expression. 

\item \(\Xlo\neg\neg \prd{aimer}(\cnsi{r}1,\cns{m})\)

Cette formule est bien formée:

\begin{center}
{\small
\leaf{\(\prd{aimer}\)}
\leaf{\(\cnsi{r}1\)}
\leaf{\(\cns{m}\)}
\branch{3}{\(\Xlo\prd{aimer}(\cnsi{r}1,\cns{m})\)}
\branch{1}{\(\Xlo\neg \prd{aimer}(\cnsi{r}1,\cns{m})\)}
\branch{1}{\(\Xlo\neg\neg \prd{aimer}(\cnsi{r}1,\cns{m})\)}
\qobitree}
\end{center}


\item \(\Xlo\exists x \neg\exists z\, \prd{connaître}(x,z)\)

Cette formule est bien formée:

\begin{center}
{\small
\leaf{\(\prd{connaître}\)}
\leaf{\(\Xlo x\)}
\leaf{\(\Xlo z\)}
\branch{3}{\(\Xlo\prd{connaître}(x,z)\)}
\branch{1}{\(\Xlo\exists z\, \prd{connaître}(x,z)\)}
\branch{1}{\(\Xlo\neg\exists z\, \prd{connaître}(x,z)\)}
\branch{1}{\(\Xlo\exists x \neg\exists z\, \prd{connaître}(x,z)\)}
\qobitree}
\end{center}


\item \(\Xlo\exists x [\prd{acteur}(x) \wedge \prd{dormir}(x) \vee
  \prd{avoir-faim}(x)] \)

Cette expression n'est pas une formule bien formée car il manque une
paire de crochets dans \(\Xlo[\prd{acteur}(x) \wedge \prd{dormir}(x) \vee
  \prd{avoir-faim}(x)]\).  En effet les connecteurs comme $\Xlo\wedge$ et
$\Xlo\vee$ sont introduits par les règles (\RSyn\ref{SynPConn}) et ces
règles introduisent en même temps une paire de crochets pour chaque connecteur.

\item \(\Xlo[\prd{aimer}(\cns{a},\cns{b}) \implq \neg\prd{aimer}(\cns{a},\cns{b})]\)

Cette formule est bien formée:

\begin{center}
{\small
\leaf{\(\prd{aimer}\)}
\leaf{\(\cns{a}\)}
\leaf{\(\cns{b}\)}
\branch{3}{\(\Xlo\prd{aimer}(\cns{a},\cns{b})\)}
\leaf{\(\Xlo\prd{aimer}\)}
\leaf{\(\Xlo\cns{a}\)}
\leaf{\(\Xlo\cns{b}\)}
\branch{3}{\(\Xlo\prd{aimer}(\cns{a},\cns{b})\)}
\branch{1}{\(\Xlo\neg\prd{aimer}(\cns{a},\cns{b})\)}
\branch{2}{\(\Xlo[\prd{aimer}(\cns{a},\cns{b}) \implq \neg\prd{aimer}(\cns{a},\cns{b})]\)}
\qobitree}
\end{center}

\item \(\Xlo\exists y [\prd{acteur}(x) \wedge \prd{dormir}(x)]\)

Cette formule est bien formée:

\begin{center}
{\small
\leaf{\(\prd{acteur}\)}
\leaf{\(\Xlo x\)}
\branch{2}{\(\Xlo\prd{acteur}(x)\)}
\leaf{\(\prd{dormir}\)}
\leaf{\(\Xlo x\)}
\branch{2}{\(\Xlo\prd{dormir}(x)\)}
\branch{2}{\(\Xlo[\prd{acteur}(x) \wedge \prd{dormir}(x)]\)}
\branch{1}{\(\Xlo\exists y [\prd{acteur}(x) \wedge \prd{dormir}(x)]\)}
\qobitree}
\end{center}

Remarque: la variable $\vrb y$ <<~utilisée~>> par le quantificateur
$\Xlo\exists$ ne réapparaît pas dans la (sous-)formule qui suit, mais cela
n'empêche pas l'expression d'être une formule bien formée, comme le
permet la règle (\RSyn\ref{SynPQ}).

\end{exolist}
\end{solu}
%'''''''''
\end{exo}


% -*- coding: utf-8 -*-
\begin{exo}\label{e:versionLO}
Traduisez dans {\LO} les phrases françaises ci-dessous.  
\pagesolution{crg:versionLO}
Vous choisirez
les noms de prédicats et de constantes comme cela vous arrange, mais
vous indiquerez à chaque fois ce qu'ils traduisent du français.  Si
une phrase contient une présupposition, ne traduisez que 
son  contenu asserté (\ie\ non présupposé).  On ne tiendra pas compte
de valeur sémantique des temps verbaux (on les néglige pour l'instant).
\begin{enumerate}
\item Antoine n'est plus barbu.
\item Tout est sucré ou salé.
\item Soit tout est sucré, soit tout est salé.
\item Le chien qui aboie ne mord pas. (proverbe)
\item C'est Marie que Jérôme a embrassée.
\item Il y a des hommes et des femmes qui ne sont pas unijambistes.
\item Tout le monde aime quelqu'un.
\item Si tous les homards sont gauchers alors Alfred aussi est
  gaucher. 
\item Quelqu'un a envoyé une lettre anonyme à Anne.
\item Seule Chloé est réveillée.
\end{enumerate}
\begin{solu} (p.~\pageref{e:versionLO})\label{crg:versionLO}
%Traduction Fr $\leadsto$ \LO.

\begin{enumerate}
\item Antoine n'est plus barbu.\\ $\leadsto$
\(\Xlo\neg\prd{barbu}(\cns{a})\)
\item Tout est sucré ou salé.\\ $\leadsto$
\(\Xlo\forall x [\prd{sucré}(x) \vee \prd{salé}(x)]\)
\item Soit tout est sucré, soit tout est salé.\\ $\leadsto$
\(\Xlo [\forall x\, \prd{sucré}(x) \vee \forall x\, \prd{salé}(x)]\)
\item Le chien qui aboie ne mord pas. (proverbe) \\ $\leadsto$
\(\Xlo\forall x [[\prd{chien}(x) \wedge \prd{aboyer}(x)] \implq
  \neg\prd{mordre}(x)]\) \\ou
\(\Xlo\forall x [\prd{chien}(x) \implq [\prd{aboyer}(x) \implq
  \neg\prd{mordre}(x)]]\)
\item C'est Marie que Jérôme a embrassée. \\$\leadsto$
\(\Xlo\prd{embrasser}(\cns{j},\cns{m})\)
\item Il y a des hommes et des femmes qui ne sont pas
  unijambistes. \\$\leadsto$ 
\(\Xlo\exists x \exists y [[\prd{homme}(x) \wedge \prd{femme}(y)] \wedge
  [\neg\prd{unij}(x) \wedge \neg\prd{unij}(y)]]\)
\item Tout le monde aime quelqu'un. 

Cette phrase est ambiguë.  Elle peut signifier que pour chaque personne, il y a une personne que la première aime (et donc possiblement autant de personnes aimées que de personnes aimantes), cela correspond à la traduction (a) ci-dessous ; mais elle peut signifier aussi qu'il existe une personne aimée de tout le monde, ce qui correspond à la traduction (b). 
  \begin{enumerate}
    \item $\leadsto$
      \(\Xlo\forall x \exists y\, \prd{aimer}(x,y)\)\\ou
      \(\Xlo\forall x [\prd{hum}(x) \implq \exists y [\prd{hum}(y) \wedge
    \prd{aimer}(x,y)]]\)
    \item $\leadsto$
       \(\Xlo\exists y\forall x\, \prd{aimer}(x,y)\)\\ou
      \(\Xlo\exists y [\prd{hum}(y) \wedge\forall x [\prd{hum}(x) \implq 
    \prd{aimer}(x,y)]]\)
  \end{enumerate}
\item Si tous les homards sont gauchers alors Alfred aussi est
  gaucher.\\ $\leadsto$
\(\Xlo\forall x [\prd{homard}(x) \implq \prd{gaucher}(x)] \implq \prd{gaucher}(\cns{a})\)
\item Quelqu'un a envoyé une lettre anonyme à Anne. \\$\leadsto$
\(\Xlo\exists x [\prd{hum}(x) \wedge \exists y [[\prd{lettre}(y) \wedge
      \prd{anon}(y)] \wedge \prd{envoyer}(x,y,\cns{a})]]\)
\item Seule Chloé est réveillée.\\$\leadsto$
\(\Xlo\forall x [\prd{réveillé}(x) \ssi x=\cns{c}]\)\\
mais on peut aussi proposer \(\Xlo\forall x [\prd{réveillé}(x) \implq
  x=\cns{c}]\), qui n'a pas le même sens, et qui là exclut le présupposé.
\end{enumerate}
\end{solu}
\end{exo}

\begin{exo}\label{e:version2LO}
Même exercice. \pagesolution{crg:version2LO}
\begin{enumerate}
\item Il existe des éléphants roses.
\item Quelque chose me gratouille et me chatouille.
\item Quelque chose me gratouille et quelque chose me chatouille.
\item Nimes est entre Avignon et Montpellier.
\item S'il y a des perroquets ventriloques, alors Jacko en est un.
\item Anne a reçu une lettre de Jean, mais elle n'a rien reçu de
  Pierre.
\item Tout fermier qui possède un âne est riche.
\item Il y a quelqu'un qui a acheté une batterie et qui est en train
  d'en jouer.
\item Il y a un seul océan.
\item Personne n'aime personne.
\end{enumerate}
\begin{solu} (p.~\pageref{e:version2LO}) %Traductions.
\label{crg:version2LO}

\begin{enumerate}
\item Il existe des éléphants roses.\\$\leadsto$
\(\Xlo\exists x [\prd{éléphant}(x) \wedge \prd{rose}(x)]\)
\item Quelque chose me gratouille et me chatouille.\\$\leadsto$
\(\Xlo\exists x [\prd{gratouiller}(x,\cns l) \wedge
  \prd{chatouiller}(x,\cns l)]\)  (avec $\cns l$ pour le locuteur)
\item Quelque chose me gratouille et quelque chose me
  chatouille.\\$\leadsto$
\(\Xlo[\exists x\,\prd{gratouiller}(x,\cns l) \wedge \exists x\,
  \prd{chatouiller}(x,\cns l)]\) 
\item Nimes est entre Avignon et Montpellier.\\$\leadsto$
\(\Xlo\prd{être-entre}(\cns n,\cns a, \cns m)\)
\item S'il y a des perroquets ventriloques, alors Jacko en est
  un.\\$\leadsto$
\(\Xlo[\exists x [\prd{perroquet}(x) \wedge \prd{ventriloque}(x)] \implq
  [\prd{perroquet}(\cns j) \wedge \prd{ventriloque}(\cns j)]]\)
\item Anne a reçu une lettre de Jean, mais elle n'a rien reçu de
  Pierre. \\$\leadsto$
\(\Xlo\exists x [\prd{lettre}(x) \wedge \prd{recevoir}(\cns{a},x,\cns{j})]
  \wedge \neg\exists y \, \prd{recevoir}(\cns{a},y,\cns{p})\)
\item Tout fermier qui possède un âne est riche.\\$\leadsto$
\(\Xlo\forall x [[\prd{fermier}(x) \wedge \exists y [\prd{âne}(y) \wedge
    \prd{posséder}(x,y)] ] \implq \prd{riche}(x)]\)
\item Il y a quelqu'un qui a acheté une batterie et qui est en train
  d'en jouer.\\$\leadsto$
\(\Xlo\exists x [\prd{humain}(x) \wedge \exists y [\prd{batterie}(y)
  \wedge \prd{acheter}(x,y) \wedge \prd{jouer}(x,y)]]\)
\item Il y a un seul océan.\\$\leadsto$
\(\Xlo\exists x [\prd{océan}(x) \wedge \forall y [\prd{océan}(y) \implq y=x]]\)
\item Personne n'aime personne.
  \begin{enumerate}
  \item $\leadsto$
    \(\Xlo\neg \exists x [\prd{humain}(x) \wedge \exists y [\prd{humain}(y)
    \wedge \prd{aimer}(x,y)]]\)\\
    ou
    \(\Xlo\forall x [\prd{humain}(x) \implq \forall y [\prd{humain}(y) \implq
    \neg\prd{aimer}(x,y)]]\)
    \\ou
    \(\Xlo\forall x  \forall y [[\prd{humain}(x) \wedge \prd{humain}(y)] \implq
    \neg\prd{aimer}(x,y)]\)
  \item $\leadsto$
    \(\Xlo\neg\exists x [\prd{humain}(x) \wedge \forall y [\prd{humain}(y) \implq \neg\prd{aimer}(x,y)]]\)\\
ou
    \(\Xlo\forall x [\prd{humain}(x) \implq \neg\forall y [\prd{humain}(y) \implq \neg\prd{aimer}(x,y)]]\)
  \end{enumerate}
Il est important de noter que cette dernière phrase est ambiguë (d'où les deux séries de traductions).  Dans une première interprétation, elle signifie que pour chaque individu du modèle, celui-ci n'aime personne (autrement dit, il n'y a pas d'amour dans le modèle) ; c'est ce que donnent les traductions (a).  Dans la seconde interprétation, la phrase signifie qu'il est faux qu'il y a des gens qui n'aiment personne (autrement dit, tout le monde aime au moins une personne) ; elle est peut-être un peu moins spontanée, mais elle apparaît naturellement dans le dialogue \sicut{--- Albert n'aime personne. --- Mais non voyons, \textsc{personne} n'aime personne} (facilitée par l'accent intonatif sur le premier \sicut{personne}) ; c'est ce que donnent les traductions (b).
\end{enumerate}
\end{solu}
\end{exo}




Récapitulons. La syntaxe présentée ici nous permet de faire le
tri entre 
ce qui est une formule correcte de {\LO} et ce qui est une simple
séquence écrite n'importe comment avec les éléments du vocabulaire.
Les formules correctes, reconnues ou admises par la syntaxe, sont
souvent appelées des expressions bien formées (\acro{ebf}), par opposition à
des expressions qui seraient mal formées.  Cette discrimination entre
expressions 
bien formées et mal formées n'est pas purement normative ou
arbitraire.  N'oublions pas que nous sommes en sémantique ; et le but
du jeu n'est pas simplement de dire que telle expression est bien
écrite et que telle autre ne l'est pas.  Ici, la syntaxe
anticipe sur la sémantique, car ce que la syntaxe reconnaît comme
expressions bien formées sont en fait des expressions qui ont du sens
(\ie\ des expressions que l'on peut interpréter).  Nous les appellerons
alors des \kwo{expressions interprétables}\is{expression!\elid\ interprétable} 
(trad.\ de \textit{meaningful expressions}).  Donc si une expression bien formée
est une expression qui a un sens,  une expression mal formée est une
expression à laquelle on ne peut pas (ou on ne saurait pas) attribuer
un sens.   
Remarquons que cela a des implications importantes : qu'une expression
mal formée soit non interprétable, cela ne pose guère de problème ; en
revanche, ce qui est moins évident, c'est que toute expression que nous nous
autoriserons à manipuler dans {\LO} (c'est-à-dire que nous accepterons
comme bien formée) devra obligatoirement être clairement et
systématiquement interprétable ; de plus, tout ce que nous souhaiterons
interpréter dans notre théorie devra recevoir une écriture (bien
formée) dans {\LO}.

Nous allons maintenant voir ce qu'est le sens d'une expression de
{\LO}, c'est-à-dire la sémantique de ce langage.



\section{Interprétation dans le calcul des prédicats}
%===================================================
\label{S:semCP}


\subsection{Sémantique informelle}
%---------------------------------
\label{s:seminf}

Le sens est ce qui nous permet de trouver la dénotation d'une
expression.  Donc décrire le sens des expressions revient à spécifier
les règles de calcul des dénotations des expressions.
La finalité de l'analyse sémantique ici est principalement d'exprimer
les conditions de dénotation des formules (leurs conditions de
vérité), et comme les formules peuvent être plus ou moins complexes,
construites à partir d'éléments plus simples, nous devons d'abord
avoir une idée précise de ce que sont les dénotations de ces éléments,
c'est-à-dire les expressions de base du langage {\LO}.

Par définition, la dénotation d'une constante d'individu est
simplement l'individu (du monde) désigné par cette constante.  Par
exemple si nous nous intéressons à un individu, humain de sexe
masculin, nommé Chandler Bing, etc. nous pourrons choisir de lui faire
correspondre le symbole de constante \cns{c} de {\LO}.  Et donc nous
saurons ainsi que \cns{c}  dénote cet individu particulier.

Venons-en à la dénotation des prédicats et d'abord des prédicats à une
place comme \prd{acteur}.  Rappelons que la dénotation est en quelque
sorte la projection dans le monde de ce que «vaut» l'expression.
Le prédicat \prd{acteur} traduit le substantif français
\sicut{acteur}, il représente le concept abstrait d'être acteur et
dans le monde, concrètement, ce concept se réalise par l'ensemble de
tous les individus du monde qui sont acteurs.  Cet ensemble est la dénotation
du prédicat.  De manière générale, la dénotation d'un prédicat à une
place est l'\emph{ensemble de tous les individus} qui le vérifient,
c'est-à-dire qui tombent sous le chef du concept représenté par le prédicat.

Il en va de même pour les prédicats qui traduisent des adjectifs
qualificatifs, comme \prd{gentil}, dont la dénotation est l'ensemble de
tous les individus gentils, ainsi que des prédicats qui traduisent des
verbes intransitifs,\is{verbe!\elid\ intransitif} comme \prd{dormir}, qui dénote l'ensemble de tous
les individus qui sont en train de dormir dans le monde.

Faisons ici deux remarques importantes.  D'abord parler d'ensembles
tels que l'ensemble de \emph{tous} les acteurs peut paraître
contre-intuitif et peu naturel, car ces ensembles sont immenses et
personne ne peut prétendre sérieusement connaître exhaustivement %complètement
l'ensemble de tous les acteurs du monde.  C'est vrai, mais en fait
cela ne pose aucun problème pour la théorie. Nous reviendrons sur ce
point un peu plus loin, mais disons pour l'instant que savoir que la
dénotation du
prédicat \prd{acteur} est l'ensemble de tous les acteurs ce n'est pas
la même chose que savoir exhaustivement et précisément qui sont les
individus qui appartiennent à cette dénotation.  Dans la théorie
sémantique, ce qui va compter c'est ce premier savoir, bien plus que le
second. 


Ensuite, les dénotations
présentées ici sous forme d'ensembles sont celles de prédicats qui
traduisent des termes \emph{singuliers} du français : \sicut{acteur},
\sicut{gentil}, \sicut{dort}, et non pas \sicut{acteurs},
\sicut{gentils}, \sicut{dorment}.  On pourrait penser que lorsque l'on
dit quelque chose comme \sicut{Joey est acteur} ou simplement
\sicut{cet acteur}, il n'est question que d'un seul individu et
pas d'un ensemble d'acteurs.  Mais dans ces expressions, le caractère
singulier de la dénotation n'est pas porté par le prédicat lui-même,
mais d'une certaine manière, il est imposé par un autre élément de la
phrase, comme la structure syntaxique ou un déterminant\footnote{Pour
  le moment, nous laissons de côté l'analyse sémantique des
  pluriels ; cela sera abordé plus précisément dans le chapitre~\ref{GN++} (vol.~2).}.
En soi et pris isolément, un prédicat comme \prd{acteur} n'a aucune
raison de distinguer un acteur parmi d'autres, puisqu'il ne fait que
présenter le concept ou la propriété d'être acteur.


Nous pouvons nous convaincre de cela en nous interrogeant sur la dénotation
d'une formule, car nous avons à présent en main ce qu'il nous faut
pour effectuer le calcul de la valeur de vérité d'une formule simple
comme \ref{x:jact}, qui traduit la phrase \sicut{Joey est acteur}.

\ex.  \label{x:jact}
\(\Xlo\prd{acteur}(\cns j)\)


\cns j dénote un individu (celui qui s'appelle Joey dans le cas de
figure que nous examinons ici) et \prd{acteur} dénote un ensemble
d'individus (les acteurs).  Maintenant dans quels cas  \ref{x:jact}
est-elle vraie ?  Simplement si l'individu dénoté par \cns j
appartient à l'ensemble dénoté par \prd{acteur}. 
Et cela n'est rien d'autre que la condition de vérité de
\ref{x:jact}.  La «rencontre» dénotationnelle du prédicat et de
son argument, qui donne la dénotation de la formule, se fait par la
relation d'appartenance (le $\in$ de la théorie des ensembles) entre
un objet et un ensemble d'objets.


La dénotation d'un prédicat binaire, comme \prd{aimer} ou
\prd{frère-de}, est un peu plus complexe, mais poursuit le même genre
de formalisation.  Un tel prédicat exprime une relation qui se joue
entre deux individus ; sa dénotation ne peut donc pas être directement
un ensemble d'individus où chacun vérifie un concept indépendemment de
ses «compagnons d'ensemble».  Il faut pouvoir rendre compte du
fait que, par exemple, Ross peut être le frère de Monica mais pas de
Chandler, et aussi que Monica n'est pas le frère de Ross.  

Par conséquent, la dénotation d'un prédicat binaire est un ensemble
dans lequel des individus sont explicitement mis en relation avec
d'autres.\indexs{relation}  
Un moyen technique d'indiquer une mise en relation est
d'utiliser des listes \indexs{liste} ou ce qu'on appelle des
\kwi{\textit{n}-uplets}{n-uplets@$n$-uplets} ; un $n$-uplet
est une liste qui contient exactement $n$ objets.  Un $n$-uplet ou une
liste est une manière de représenter mathématiquement une collection
d'objets, mais c'est très différent d'un ensemble, car dans un
$n$-uplet les objets qu'il contient sont présentés dans un
\emph{ordre} précis et déterminant, alors que la notion d'ordre n'a
aucune pertinence à l'intérieur d'un ensemble ; de même, un objet donné
peut apparaître plusieurs fois dans un $n$-uplet, c'est-à-dire y
occuper plusieurs positions, alors qu'un objet appartient une fois
pour toutes à un ensemble.

Pour expliciter une relation binaire, nous utiliserons des $n$-uplets à
deux éléments, ce que l'on appelle des
\kwi{couples}{couple}\footnote{\emph{Couple} ici est un terme technique. Si l'on
veut, c'est la manière normale de prononcer «2-uplet» en français.  
Notons en passant qu'en français, un
\emph{ensemble} à deux éléments s'appelle une \kw{paire}.}.  Ainsi la
dénotation d'un prédicat binaire est un \kwo{ensemble de couples
d'individus}, tous les couples tels que leur premier membre vérifie
vis-à-vis de leur second membre la relation exprimée par le prédicat.
Par exemple, on peut supposer que la dénotation de \prd{frère-de}
contiendra, entre autres, le couple composé de Ross et de Monica, mais
pas le couple composé de Monica et de Ross (les couples sont
ordonnés), qui lui pourra appartenir à la dénotation de \prd{s\oe
ur-de}.

Et le calcul de la dénotation d'une formule comme \ref{x:rossmon}
reste fondamentalement le même, sauf qu'elle s'applique à des couples.  

\ex.  \label{x:rossmon}
\(\Xlo\prd{frère-de}(\cnsi{r}{2},\cns m)\)


\ref{x:rossmon} est vraie si le couple constitué de la dénotation de
\cnsi{r}{2} et de celle de \cns m appartient à la dénotation de
\prd{frère-de}, qui est l'ensemble de tous les couples frère--s\oe ur
ou frère--frère du
monde.
Remarque : si un individu a plusieurs s\oe urs, mettons trois, alors il
apparaîtra en tête de trois couples différents dans la dénotation de
\prd{frère-de}, un couple pour chaque s\oe ur.

On peu maintenant généraliser : la dénotation d'un prédicat à $n$
arguments (avec $n \geqslant 2$) est un ensemble de $n$-uplets d'individus.  
Pour un prédicat ternaire, on parlera d'ensembles de triplets\indexs{triplet}, pour un
prédicat quaternaire, d'ensembles de quadruplets\indexs{quadruplet}, etc.

Nous allons voir maintenant comment représenter précisément
(c'est-à-dire formellement) ce système de dénotations au moyen de
l'outil mathématique que sont les modèles.


\subsection{Les modèles}
%-----------------------
\indexs{modele@modèle|(}

Nous avons vu que la dénotation de la plupart des expressions
interprétables n'est pas absolue : elle \emph{dépend} d'un certain
nombre de circonstances, de cas de figure, bref de \emph{comment est
le monde}.  Par exemple, pour que \(\Xlo\prd{acteur}(\cns{j})\) soit
vraie, il faut que Joey soit effectivement un acteur dans le monde
que nous envisageons.  C'est peut-être le cas, ça peut ne pas l'être.
Les choses sont ce qu'elles sont, mais elles pourraient être
autrement ; et de toute façon, personne ne sait tout sur tout.

En fait si la dénotation d'une expression dépend d'un certain état du
monde, c'est qu'il ne convient pas de parler de dénotation seule, et
dorénavant nous prendrons soin d'envisager la dénotation d'une expression
toujours \emph{relativement} à une certaine configuration du monde.

Nous avons donc besoin de tenir compte de cette «certaine
configuration du monde».  Cela veut dire que nous allons devoir nous
donner un moyen de \emph{représenter} suffisamment précisément le
monde et son «état».
Pour ce faire, l'outil que nous allons utiliser est celui des
\kwi{modèles}{modele@modèle}. 

Un modèle est une figuration mathématique du monde ; c'est une
représentation \kwo{ensembliste} et \kwo{structurée}.  Dans sa version
la plus simple, la notion de modèle nous fournit ce que l'on peut
vraiment appeler une image du monde, au sens d'un cliché ou d'un
instantané.  Et pour des raisons avant tout pratiques, les modèles que
nous examinerons de près ne constituent que des images \emph{partielles}
du monde, comme les photographies qui ne restituent nécessairement que
de minuscules portions de la réalité.  Mais qu'un modèle décrive
fidèlement, scrupuleusement et donc démesurément le monde ou qu'il n'en
donne qu'une image partielle, les caractéristiques définitoires qui le
sous-tendent restent exactement les mêmes.  Les modèles partiels,
souvent minuscules, que nous verrons en exemples par la suite, je les
appellerai des {«modèles-jouets»} ; un véritable modèle quant à lui est un
objet plutôt théorique et est supposé décrire \emph{complètement} le
monde. 

Qu'est-ce qui fait que le monde, ou \emph{un} monde, est tel qu'il est
et pas autrement ?  Ce qui le caractérise c'est tout d'abord
l'ensemble des choses, c'est-à-dire des individus, qui le peuplent.
Un monde dans lequel existe King-Kong est assurément différent du
monde dans lequel existent les lecteurs de ces pages.  Un modèle
indique ce genre d'information en définissant un \kw{domaine} --~ce
que l'on appelle aussi un \kwo{univers}%
\footnote{Les appellations
complètes et plus précise que l'on trouve souvent sont
\kwo{domaine}
ou \kwo{univers de quantification}, ou encore \kwo{domaine} ou 
\kw{univers d'interprétation}.}.\indexs{domaine!\elid\ de quantification}%
\indexs{domaine!\elid\ d'interprétation}  
Le domaine d'un modèle donné  est un \emph{ensemble d'individus}, l'ensemble
de tous les individus qui 
appartiennent au
%«peuplent»%
%\footnote{Entre guillements, car, rappelons-le, par individus nous ne
% comptons pas que les humains, mais toute chose concrète.} le 
monde que décrit ce modèle.

Remarquons qu'une façon de se donner des modèles partiels est de les
définir sur de petits domaines, qui ne comportent que très peu
d'individus.  Nous dirons alors qu'ils ne contiennent que les individus
qui nous intéressent dans notre modélisation du monde, que les
individus dont nous sommes susceptibles de parler dans un contexte de
discours particulier\footnote{En réalité, les choses sont un peu plus compliquées que cela ; nous y reviendrons en \S\ref{ss:RestrDQuant}.}.

\newpage

Voici un exemple de (petit) domaine d'individus --~appelons-le \Unv{A} :\is{A@\Unv A}
\ex.  
\(\Unv{A} = \set{\Obj{MonicaG};
  \Obj{PhoebeB};\Obj{RachelG};\Obj{ChandlerB};\Obj{JoeyT};\Obj{RossG}}\)%
\footnote{On utilise la notation mathématique habituelle qui
  représente le contenu d'un ensemble encadré d'accolades \set{\,},
  ses éléments étant séparés par des points-virgules.}


Ce que contient l'ensemble \Unv{A}, en tant que domaine, ce sont des
individus, et pas des noms.  Il se trouve que les individus de cet
exemple sont des 
personnes, mais nous aurions pu aussi faire figurer des objets ou des
animaux dans \Unv{A}.

\begin{nota}\label{'Obj}
Pour bien marquer la distinction entre les mots ou noms de la
   langue et les individus qui appartiennent à la réalité décrite dans
   un modèle, ces derniers seront représentés à l'aide d'une
   {forme de caractères spéciale}, en  \Obj{Petites Capitales}. % dite \Obj{sans empattement}
%   (ou \Obj{sans serif} en anglais).
\end{nota}

Il faudra donc bien veiller à ne pas confondre le prénom \sicut{Joey},
qui appartient à la langue naturelle (le français ou l'anglais), la
constante \cns{j} qui appartient au langage objet {\LO}, et l'individu
\Obj{JoeyT} qui appartient au modèle.  Il serait probablement plus
pédagogique de présenter un dessin ou une photo de l'individu au lieu
de la séquence \Obj{JoeyT} pour bien montrer qu'il s'agit là d'une portion de
réalité.  Mais cette stratégie, qui deviendrait vite graphiquement
encombrante, ne resterait somme toute qu'un moyen terme tout aussi
artificiel. 
Accommodons-nous plutôt des contraintes matérielles
imposées par un ouvrage écrit en faisant usage de la convention de
notation typographique \ref{'Obj}. 

Donc un modèle établit une description (éventuellement partielle) du
monde en spécifiant un domaine d'individus, la «population» du
monde.   Un domaine n'est qu'un ensemble, dans lequel les individus
nous sont présentés en vrac.  Une description du monde ne se ramène
pas qu'à cela ; cela se doit d'être plus sophistiqué, plus informatif.
C'est pour cela que j'ai dit plus haut qu'un modèle est
\emph{structuré}.  Les individus du domaine sont organisés
d'une certaine manière, et cette «certaine manière» n'est ni plus
ni moins qu'un certain état du monde.
Le modèle spécifie une telle organisation en indiquant  «qui est
qui», «qui est quoi» et «qui 
fait quoi», etc.\ dans le domaine.  

En d'autres termes, un modèle décrit l'état du monde en explicitant le
lien entre le 
domaine et le langage (en l'occurrence {\LO}).  Ce lien est en fait la
dénotation du vocabulaire.
En effet, savoir dans quel état est le monde, ou savoir ce qui se passe dans le
monde, cela revient finalement à connaître les dénotations de tous les
prédicats du langage, puisque les prédicats servent à donner, dans le
langage, des informations sur le monde\footnote{À noter que l'on pourrait connaître
  certaines informations sur le monde qui soient indicibles ; elles ne
  correspondraient donc à la dénotation d'aucun prédicat.  Cela est
  tout à fait concevable (encore que les langues naturelles ont un
  pouvoir expressif immense).  Mais comme nous étudions ici la sémantique
  de la langue, nous ne nous intéresserons qu'aux informations
  «dicibles».}. 

Illustrons cela en reprenant «notre histoire» introduite via les
exemples \ref{xpredicats}--\ref{x:phraseneg}.
Un modèle qui décrit fidèlement la réalité d'un monde devra par
exemple nous dire qui est italien parmi les individus du domaine,
autrement dit quel est l'ensemble des italiens,... autrement dit quelle
est dénotation du prédicat \prd{italien}.  Et de même, il devra nous
dire qui est une femme, qui est un homme, qui est acteur, qui fume,
qui dort, etc.  Mais aussi qui aime qui, qui connaît qui, qui embrasse
qui, qui est le frère de qui, etc.  Bref, on attend d'un modèle qu'il
réponde à ce genre de question pour tout prédicat du langage. 

Par exemple, notre modèle en cours pourra nous faire savoir que
\ref{x:dital} est la dénotation de \prd{italien}, \ref{x:dact}
celle de \prd{acteur}, \ref{x:dfem} celle de \prd{femme}, \ref{x:dhom}
celle de \prd{homme}, etc.  Ces ensembles sont forcément des
sous-ensembles de \Unv{A}.

\ex.\a. \(\set{\Obj{JoeyT}}\) \label{x:dital}
\b. \(\set{\Obj{JoeyT}}\) \label{x:dact}
\c. \(\set{\Obj{MonicaG};\Obj{PhoebeB};\Obj{RachelG}}\) \label{x:dfem}
\d. \(\set{\Obj{JoeyT};\Obj{ChandlerB};\Obj{RossG}}\) \label{x:dhom}


Dans ce mini-modèle, \prd{italien} et \prd{acteur} ont la même
dénotation, mais c'est justement parce que le modèle est très petit.
Cela ne tire donc pas vraiment à conséquence.

Pour les prédicats $n$-aires, nous représenterons les $n$-uplets
\indexs{n-uplets@$n$-uplets}
en les encadrant de chevrons \tuple{\,} en séparant leurs membres par des
virgules. 
\sloppy  Par exemple : \tuple{\Obj{RossG},\Obj{RachelG}} est un
couple d'individus. Le modèle pourra ainsi nous dire que
\ref{x:daim} est la dénotation de \prd{aimer} et \ref{x:dfrr} celle
de \prd{frère-de}.


\ex. \raggedright
\a. { \set{\tuple{\Obj{RachelG}, \Obj{RossG}};
     \tuple{\Obj{RossG}, \Obj{RachelG}};
     \tuple{\Obj{MonicaG}, \Obj{ChandlerB}};
     \tuple{\Obj{ChandlerB}, \Obj{MonicaG}}} } \label{x:daim}
\b. \set{\tuple{\Obj{RossG},\Obj{MonicaG}}} \label{x:dfrr}


\fussy


Ce que nous devons voir pour terminer, c'est comment  le modèle nous
fournit ces réponses.  On utilise à cet effet un outil formel simple :
une \kwo{fonction}\is{fonction!\elid\ d'interprétation} qui à chaque constante
non logique de {\LO} associe sa dénotation dans le modèle.  Une telle
fonction s'appelle une \kwo{fonction d'interprétation}\indexs{fonction!\elid\ d'interprétation}, et elle est, au
côté du domaine, un élément constitutif du modèle.  Notons-la $\FI$.\is{F@\FI}
Un début de description (possible) du monde apparaîtra alors comme
ci-dessous ; de manière générale, nous représenterons graphiquement une
fonction sous forme de «matrice» à deux colonnes, la première
correspondant à l'ensemble de départ de la fonction, la seconde à son
ensemble d'arrivée et chaque élément de l'ensemble de départ est mis en
regard de son image par une flèche ($\mapsto$).

\ex.[]
\(\FI : \left[\begin{array}{l@{\;\longmapsto\;}>{\PBS\raggedright\hspace{0pt}}p{.55\textwidth}}
\prd{italien} & \set{\Obj{JoeyT}}\\
\prd{acteur} & \set{\Obj{JoeyT}}\\
\prd{femme} & \set{\Obj{MonicaG};\Obj{PhoebeB};\Obj{RachelG}}\\
\prd{homme} & \set{\Obj{JoeyT};\Obj{ChandlerB};\Obj{RossG}}\\
\prd{fumer} & \set{\Obj{ChandlerB}}\\
\prd{aimer} & 
\set{\tuple{\Obj{RachelG}, \Obj{RossG}};
     \tuple{\Obj{RossG}, \Obj{RachelG}};
     \tuple{\Obj{MonicaG}, \Obj{ChandlerB}};
     \tuple{\Obj{ChandlerB}, \Obj{MonicaG}}}\\
\prd{frère-de} & \set{\tuple{\Obj{RossG},\Obj{MonicaG}}}\\
\multicolumn{2}{l}{\text{etc.}}
\end{array}\right]\)


Nous pouvons également expliciter {\FI} avec la notation
fonctionnelle habituelle, en écrivant par exemple :
\(\FI(\prd{italien})= \set{\Obj{JoeyT}}\), \(\FI(\prd{femme}) =
\set{\Obj{MonicaG};\Obj{PhoebeB};\Obj{RachelG}}\), etc.

Le modèle nous dit aussi «qui est qui».  Cela signifie que la
fonction d'interprétation indique également à quel individu du domaine
correspond chaque constante d'individu du langage.  Donc {\FI} doit
être complétée par les attributions suivantes :

\ex.[]
\(%\FI:
\left[%
  \begin{array}{c@{\;\longmapsto\;}l}
\cns{m} & \Obj{MonicaG}\\
\cns{p} & \Obj{PhoebeB}\\
\cnsi{r}1 & \Obj{RachelG}\\
\cns{c} & \Obj{ChandlerB}\\
\cns{j} &\Obj{JoeyT}\\
\cnsi{r}2 &\Obj{RossG}
  \end{array}\right]
\)





Résumons par la définition suivante :

\begin{defi}[Modèle]
Un \kwa{modèle}{modele} (simple) d'interprétation sémantique d'un langage {\LO}  est
défini par un couple \tuple{\Unv{A},\FI} où 
\Unv{A}\is{A@\Unv A} est un ensemble non vide et {\FI} une fonction
 qui projette les constantes non logiques (\ie\
d'individus et de prédicats) de {\LO} dans \Unv{A}.\\
Si {\Modele}\is{M@\Modele} est le nom du modèle, on pose \(\Modele =
\tuple{\Unv{A},\FI}\) en guise de définition.\\
On dit que \Unv{A} est le \kw{domaine}  de {\Modele} et on le
tient pour l'ensemble de tous les individus du monde décrit par
{\Modele}.\\
{\FI} s'appelle la \kwo{fonction d'interprétation}\indexs{fonction!\elid\ d'interprétation} de {\LO} dans {\Modele}.
\end{defi}


\indexs{modele@modèle|)}


\subsection{Les règles sémantiques}
%----------------------------------
\label{s:reglsem}\is{regle@règle!\elid\ semantique@\elid\ sémantique}

Nous pouvons maintenant définir la sémantique du langage {\LO},
c'est-à-dire ses règles d'interprétation.  Comme nous le savons
maintenant, il s'agit de règles de calcul, et plus précisément du
calcul de la dénotation de toute expression de {\LO}.  Pour exprimer
ces calculs, introduisons d'abord un élément de notation.


\begin{nota}[Valeur sémantique]
%\indexs{\denote{\,}}
Soit $\vrb\alpha$ une expression interprétable quelconque de {\LO}.\\
\denote{\vrb\alpha} représente la \kwo{valeur sémantique}\indexs{valeur!\elid\ sémantique} de
l'expression $\vrb\alpha$. On l'appelle également l'\kwo{interprétation} de $\vrb\alpha$.
\end{nota}

\newpage

Nous allons nous servir de cette notation pour représenter la
dénotation des expressions.  La dénotation est, comme il se doit, la
valeur sémantique relativisée à un certain modèle.  

\begin{nota}[Dénotation]
Soit $\vrb\alpha$ une expression interprétable quelconque (de {\LO}) et
{\Modele} un modèle.\\
$\denote{\vrb\alpha}^{\Modele}$ représente la \kwa{dénotation}{denotation} de
%l'expression 
$\vrb\alpha$ \emph{relativement} au modèle {\Modele}.
\end{nota}


Remarque : \denote{\,} est une notation symbolique, mais elle
n'appartient pas au langage {\LO} ; là encore c'est un méta-symbole que
les sémanticiens utilisent pour parler de la sémantique d'une
expression.



Pour décrire le sens dans {\LO}, nous devons définir la valeur
sémantique de \emph{toute} expression interprétable relativement à
\emph{tout} modèle possible.  Pour ce faire, il suffit d'exprimer les
règles de calcul par 
rapport à un modèle absolument quelconque, c'est-à-dire complètement
général ; nous sommes sûrs ainsi qu'elles vaudront pour n'importe quel modèle.
Comme l'ensemble des expressions de {\LO} est déterminé par les
règles de syntaxe, le plus efficace est de reprendre chacune de ces
règles et de lui associer une règle d'interprétation sémantique.
Ainsi nous aurons la garantie d'avoir défini une sémantique pour toutes
les expressions possibles de {\LO}.

\largerpage[-2]

Commençons d'abord par définir l'interprétation du «lexique» de base
de {\LO}, c'est-à-dire les constantes d'individus et de prédicats.
Nous l'avons vu, la valeur sémantique de ces constantes nous est
directement donnée par la fonction d'interprétation $\FI$ du modèle
par rapport auquel nous avons choisi d'effectuer le calcul.


\begin{defi}[Interprétation des constantes non logiques]\label{RIcl}
Soit un modèle \(\Modele = \tuple{\Unv{A},\FI}\).
\begin{enumerate}
\item Si $\vrb\alpha$ est une constante d'individu, alors
\(\denote{\vrb\alpha}^{\Modele}=\FI(\vrb\alpha)\) ; \ie\ l'individu de $\Unv{A}$ assigné à $\vrb\alpha$ par $\FI$.
\item Si $\vrb P$ est une constante de prédicat, alors
  \(\denote{\vrb P}^{\Modele}=\FI(\vrb P)\) ; \ie\ un 
ensemble d'individus de $\Unv{A}$ si $\vrb P$ est unaire, un ensemble de couples
d'individus de $\Unv{A}$ si $\vrb P$ est binaire, etc.
\end{enumerate}
\end{defi}


%\sloppy
Ensuite, nous devons spécifier comment obtenir la dénotation de
n'importe quelle formule  syntaxiquement bien construite.  Les règles
d'interprétation sont récursives (comme celles de la syntaxe) et
compositionnelles car l'interprétation de toute formule est définie
via l'interprétation de ses constituants.
Comme la dénotation d'une formule est une valeur de vérité, nous
allons devoir manipuler le vrai et le faux, et nous le ferons au moyen
des symboles $1$ et $0$\footnote{$0$ et $1$ sont eux
  aussi des méta-symboles, il n'appartiennent pas à {\LO}.  Nous aurions pu  à la place utiliser
  $\mathrm{F}$ et $\mathrm{V}$, mais par expérience je trouve que $0$
  et $1$ permettent une lecture plus fluide, ils se distinguent mieux
  graphiquement (surtout dans les tables de vérité,
  cf. \S\ref{ConnVfctel}). Et puis nous avons déjà {\FI} pour la fonction
  d'interprétation, évitons autant que possible les
  homographies.}
 :

\fussy

\begin{nota}[Valeurs de vérité]
On note \set{0;1} l'ensemble des valeurs de vérité. $1$ représente
«vrai» et $0$ représente «faux». \\
Ainsi \(\denote{\vrb\phi}^{\Modele} = 1\) signifie que $\vrb\phi$ est vraie
par rapport au (ou dans le) modèle {\Modele}, et \(\denote{\vrb\phi}^{\Modele} = 0\)
que $\vrb\phi$ est fausse par rapport à \Modele.
\end{nota}%

\begin{defi}[Interprétation des formules]\label{d:Sem1}
\label{RI1}
Soit un modèle \(\Modele = \tuple{\Unv{A},\FI}\).
\begin{enumerate}[sem,series=RglSem1]  %[(\RSem1)]
  \item 
\label{RIprd}
\begin{enumerate}
\item
Si $\vrb P$ est un prédicat à une place et si $\vrb\alpha$ est une constante, alors
$\denote{\Xlo P(\alpha)}^{\Modele}=1$ ssi \(\denote{\Xlo\alpha}^{\Modele} \in
\denote{\Xlo P}^{\Modele}\) ; 
%
\item Si $\Xlo P$ est un prédicat à deux places et si $\Xlo\alpha$ et $\Xlo\beta$ sont des
constantes, alors 
$\denote{\Xlo P(\alpha,\beta)}^{\Modele}=1$ ssi
\(\tuple{\denote{\Xlo\alpha}^{\Modele},\denote{\Xlo\beta}^{\Modele}} \in 
\denote{\Xlo P}^{\Modele}\) ; 
%
\item Si $\Xlo P$ est un prédicat à trois places et si $\Xlo\alpha$, $\Xlo\beta$ et
  $\Xlo\gamma$ sont des constantes, alors 
$\denote{\Xlo P(\alpha,\beta,\gamma)}^{\Modele}=1$ ssi
\(\tuple{\denote{\Xlo\alpha}^{\Modele},\denote{\Xlo\beta}^{\Modele},\denote{\Xlo\gamma}^{\Modele}} 
\in \denote{\Xlo P}^{\Modele}\) ; 
%
\item etc.
  \end{enumerate}
\item \label{RI=}
Si $\Xlo t_1$ et $\Xlo t_2$ sont des termes, alors $\denote{\Xlo t_1 =
  t_2}^{\Modele}=1$ ssi $\denote{\Xlo t_1}^{\Modele}=\denote{\Xlo t_2}^{\Modele}$.
\item \label{RIneg}
Si $\Xlo\phi$ est une formule, alors $\denote{\Xlo\neg\phi}^{\Modele}=1$ ssi
\(\denote{\Xlo\phi}^{\Modele}=0\). 
\item \label{RIcon}
Si $\Xlo\phi$ et $\Xlo\psi$ sont des formules, alors 
  \begin{enumerate}
\item $\denote{\Xlo[\phi \wedge \psi]}^{\Modele}=1$ ssi $\denote{\Xlo\phi}^{\Modele}=1$ \emph{et} $\denote{\Xlo\psi}^{\Modele}=1$.
\item $\denote{\Xlo[\phi \vee \psi]}^{\Modele}=1$ ssi $\denote{\Xlo\phi}^{\Modele}=1$ \emph{ou} $\denote{\Xlo\psi}^{\Modele}=1$.
\item $\denote{\Xlo[\phi \implq \psi]}^{\Modele}=1$ ssi $\denote{\Xlo\phi}^{\Modele}=0$ \emph{ou} $\denote{\Xlo\psi}^{\Modele}=1$.
\item $\denote{\Xlo[\phi \ssi \psi]}^{\Modele}=1$ ssi $\denote{\Xlo\phi}^{\Modele}=\denote{\Xlo\psi}^{\Modele}$.
  \end{enumerate}
\setcounter{RglSem}{\value{enumi}}
\end{enumerate}
\end{defi}

Arrêtons-nous là pour le moment et laissons de côté provisoirement
l'interprétation des formules quantifiées, nous y reviendrons en temps voulu.

Ces règles nous disent dans quels cas (et seulement quels cas)
une formule de telle ou telle structure est vraie.  Il s'agit bien de
\emph{conditions de vérité}.  Et ces règles d'interprétation
définissent donc bien le \emph{sens} des formules.

Les règles (\RSem\ref{RIprd}) ne font que reformuler précisément ce que
nous avons vu en \S\ref{s:seminf} : la dénotation d'une formule
atomique s'obtient en vérifiant l'appartenance ensembliste ($\in$) de
la dénotation des arguments (seuls ou en $n$-uplets) à la dénotation
des prédicats.  
La règle (\RSem\ref{RI=}) est assez simple : elle dit que $\Xlo t_1=t_2$ est
vrai dans {\Modele} si et seulement si les dénotations de $\Xlo t_1$ et de
$\Xlo t_2$ sont identiques.

Ce qui est un peu plus nouveau, ce sont les règles (\RSem\ref{RIneg})
et (\RSem\ref{RIcon}) qui interprètent le rôle des connecteurs
logiques.  Mais d'après ce que nous avons vu en \S\ref{ss:ConnLog}, elles
sont assez simples à comprendre, sauf peut-être (\RSem\ref{RIcon}c)
que nous commenterons un peu plus en \S\ref{ConnVfctel}.


Pour illustrer le fonctionnement de ces règles, regardons un exemple
 d'application.  Une façon de mettre en \oe uvre les calculs
sémantiques consiste à se donner un modèle petit mais détaillé et
 d'évaluer des formules par rapport à ce modèle.
Soit donc  le modèle-jouet $\Modele_1=\tuple{\Unv{A}_1,\FI_1}$, défini comme
suit (nous lui mettons l'indice~${}_1$ pour montrer que c'est un modèle bien
particulier). \label{Modele1}

\ex.
\(\Unv{A}_1=\{\Obj{Thésée}; \Obj{Hippolyta}; \Obj{Hermia}; \Obj{Héléna};
\Obj{Lysandre}; \Obj{Démétrius};\Obj{Egée}; \Obj{Puck}; \Obj{Obéron};
\Obj{Titania}; \Obj{Bottom}\}\)
\\
\begin{minipage}[t]{.45\textwidth}
\(\FI_1(\cnsi{t}{1})=\Obj{Thésée}\),\\
\(\FI_1(\cnsi{h}{1})=\Obj{Hippolyta}\),\\
\(\FI_1(\cnsi{h}{2})=\Obj{Hermia}\),\\
\(\FI_1(\cnsi{h}{3})=\Obj{Héléna}\),\\
\(\FI_1(\cns l)=\Obj{Lysandre}\),\\
\(\FI_1(\cns d)=\Obj{Démétrius}\),
\end{minipage}
\
\begin{minipage}[t]{.45\textwidth}
\(\FI_1(\cns e)=\Obj{Egée}\),\\
\(\FI_1(\cns p)=\Obj{Puck}\),\\
\(\FI_1(\cns o)=\Obj{Obéron}\),\\
\(\FI_1(\cnsi{t}{2})=\Obj{Titania}\),\\
\(\FI_1(\cns b)=\Obj{Bottom}\)
\end{minipage}
\\
\(\FI_1(\prd{elfe})=\Set{\Obj{Obéron};\Obj{Titania};\Obj{Puck}}\)
\\
\(\FI_1(\prd{âne})=\Set{\Obj{Bottom}}\)
\\
\(\FI_1(\prd{farceur})=\Set{\Obj{Thésée};\Obj{Obéron};\Obj{Titania};\Obj{Puck}}\)
\\
\(\FI_1(\prd{triste})=\Set{\Obj{Héléna}}\)
\\
\(\FI_1(\prd{mari-de})=\Set{\begin{array}{@{}l@{}}
\tuple{\Obj{Thésée},\Obj{Hippolyta}};\\
\tuple{\Obj{Obéron},\Obj{Titania}}\\
		      \end{array}}
\)
\\
\(\FI_1(\prd{père-de})=\Set{\begin{array}{@{}l@{}}
\tuple{\Obj{Egée},\Obj{Hermia}}\\
		      \end{array}}
\)
\\
\(\FI_1(\prd{aimer})=\Set{\begin{array}{@{}l@{}}
\tuple{\Obj{Thésée},\Obj{Hippolyta}};\\
\tuple{\Obj{Hippolyta},\Obj{Thésée}};\\
\tuple{\Obj{Lysandre},\Obj{Hermia}};\\
\tuple{\Obj{Hermia},\Obj{Lysandre}};\\
\tuple{\Obj{Démétrius},\Obj{Hermia}};\\
\tuple{\Obj{Héléna},\Obj{Démétrius}};\\
\tuple{\Obj{Titania},\Obj{Bottom}}%
		      \end{array}}
\)\label{M1:aimer}


Cette présentation de \(\Modele_1\) implique que nous travaillons
présentement sur une petite portion d'un langage de type {\LO} qui ne
comprend que les constantes d'individu
\cnsi{t}{1},
\cnsi{h}{1},
\cnsi{h}{2},
\cnsi{h}{3},
\cns l,
\cns d,
\cns e,
\cns p,
\cns o,
\cnsi{t}{2},
\cns b, 
et les prédicats
\prd{elfe}, \prd{âne}, \prd{farceur}, \prd{triste}, \prd{mari-de},
\prd{père-de} et \prd{aimer}.  Ce petit fragment est loin de nous
permettre de rédiger en {\LO} un résumé d'une pièce de Shakespeare,
mais il nous suffira pour évaluer la vérité d'une bonne variété de
formules.  Notons aussi que \(\Modele_1\) sous-entend aussi (mais sans
équivoque possible) que tous les prédicats sont à un argument sauf
\prd{mari-de}, \prd{père-de} et \prd{aimer} qui sont binaires. 

Commençons avec une formule très simple ; nous pouvons nous demander quelle
est la dénotation de \ref{x:snd1} 
relativement à \(\Modele_1\) :

\ex.  \label{x:snd1}
\(\Xlo\prd{aimer}(\cns d,\cnsi{h}{2})\)


\sloppy
Autrement dit, que vaut   %\sloppy 
\(\denote{\Xlo\prd{aimer}(\cns d,\cnsi{h}{2})}^{\Modele_1}\) ?
La règle (\RSem\ref{RIprd}b) nous
dit que \(\denote{\Xlo\prd{aimer}(\cns d,\cnsi{h}{2})}^{\Modele_1} = 1\)
ssi \(\tuple{\denote{\cns
    d}^{\Modele_1},\denote{\cnsi{h}{2}}^{\Modele_1}} \in
\denote{\prd{aimer}}^{\Modele_1}\).  
Or, d'après la règle 1 de l'interprétation des constantes non
logiques, \(\denote{\cns d}^{\Modele_1} = \FI_1(\cns d)\) c'est-à-dire
\Obj{Démétrius} d'après la donnée de $\Modele_1$, et
\(\denote{\cnsi{h}{2}}^{\Modele_1} = \FI_1(\cnsi{h}{2}) =
\Obj{Hermia}\).  Donc \(\tuple{\denote{\cns
    d}^{\Modele_1},\denote{\cnsi{h}{2}}^{\Modele_1}} =
\tuple{\Obj{Démétrius},\Obj{Hermia}}\).   Quant à
$\denote{\prd{aimer}}^{\Modele_1}$ c'est égal à $\FI_1(\prd{aimer})$,
qui vaut l'ensemble de couples donné plus haut.  Et nous constatons que
\tuple{\Obj{Démétrius},\Obj{Hermia}} fait bien partie de cet
ensemble.  Nous avons donc bien montré que \(\denote{\Xlo\prd{aimer}(\cns
  d,\cnsi{h}{2})}^{\Modele_1} = 1\), \ref{x:snd1} est vraie dans
\(\Modele_1\). 

\fussy
Essayons avec \ref{x:snd2} :

\ex.  \label{x:snd2}
\(\Xlo[\prd{triste}(\cns b) \wedge \prd{aimer}(\cnsi{t}{2},\cns b)]\)


\sloppy
La règle (\RSem\ref{RIcon}a) nous dit que \(\denote{\Xlo[\prd{triste}(\cns b)
    \wedge \prd{aimer}(\cnsi{t}{2},\cns b)]}^{\Modele_1}=1\) ssi
    \(\denote{\Xlo\prd{triste}(\cns b)}^{\Modele_1}=1\) et
    \(\denote{\Xlo\prd{aimer}(\cnsi{t}{2},\cns b)}^{\Modele_1}=1\).
    Regardons d'abord \(\denote{\Xlo\prd{triste}(\cns b)}^{\Modele_1}\). 
\(\denote{\cns b}^{\Modele_1} = \FI_1(\cns b) = \Obj{Bottom}\) et 
\(\denote{\prd{triste}}^{\Modele_1} = \FI_1(\prd{triste}) =
    \Set{\Obj{Héléna}}\).  Donc évidemment \(\denote{\cns
    b}^{\Modele_1}  \not\in \denote{\prd{triste}}^{\Modele_1}\) et
    donc, en vertu de (\RSem\ref{RIprd}a), nous en concluons que
    \(\denote{\prd{triste}(\cns b)}^{\Modele_1} = 0\).  Nous pouvons
    nous arrêter là, car la condition demandée par (\RSem\ref{RIcon}a)
    n'est pas remplie : il faut que les \emph{deux} sous-formules
    soient vraies pour que \ref{x:snd2} le soit ; si la première est
    fausse, il est inutile d'évaluer la seconde, nous savons que
    \(\denote{\Xlo[\prd{triste}(\cns b) 
    \wedge \prd{aimer}(\cnsi{t}{2},\cns b)]}^{\Modele_1}=0\).  

\fussy
Remarquons en passant que ce qui est faux par rapport à \(\Modele_1\),
comme $\prd{triste}(\cns b)$, c'est ce qui n'est pas donné comme
information par $\Modele_1$.

\smallskip

Regardons une formule négative, \ref{x:snd3} :

\ex.  \label{x:snd3}
\(\Xlo\neg [\prd{âne}(\cns p) \vee \prd{farceur}(\cns p)]\)


\sloppy 
(\RSem\ref{RIneg}) nous dit que \(\denote{\Xlo\neg [\prd{âne}(\cns
    p) \vee \prd{farceur}(\cns p)]}^{\Modele_1}=1\) ssi
\(\oden\Xlo[\prd{âne}(\cns p) \vee \prd{farceur}(\cns
    p)]\fden\color{black}^{\Modele_1}=0\).  Calculons donc
\(\denote{\Xlo[\prd{âne}(\cns p) \vee \prd{farceur}(\cns
    p)]}^{\Modele_1}\).  (\RSem\ref{RIcon}b) dit que cette
sous-formule est vraie si \(\denote{\Xlo\prd{âne}(\cns
  p)}^{\Modele_1}=1\) ou si \(\denote{\Xlo\prd{farceur}(\cns
  p)}^{\Modele_1}=1\).  Là nous pouvons gagner du temps en observant
$\Modele_1$ qui nous conseille de regarder d'abord la valeur de
\prd{farceur}(\cns p).  \(\denote{\cns p}^{\Modele_1}=\FI_1(\cns p) =
\Obj{Puck}\) et
\(\denote{\prd{farceur}}^{\Modele_1}=\FI_1(\prd{farceur})=\Set{\Obj{Thésée};\Obj{Obéron};\Obj{Titania};\Obj{Puck}}\),
donc \(\denote{\cns p}^{\Modele_1} \in
\denote{\prd{farceur}}^{\Modele_1}\), et donc
\(\denote{\Xlo\prd{farceur}(\cns p)}^{\Modele_1}=1\).  Et cela suffit
à montrer que \(\denote{\Xlo[\prd{âne}(\cns p) \vee \prd{farceur}(\cns
    p)]}^{\Modele_1}=1\), puisque selon (\RSem\ref{RIcon}b) il suffit
que l'un des deux membres d'une disjonction soit vrai pour que
l'ensemble soit vrai.  Maintenant la règle (\RSem\ref{RIneg}) --~comme
toutes les règles d'interprétation~-- est en \sicut{si et seulement si
} ; cela veut dire que \(\denote{\Xlo\neg\phi}^{\Modele_1}=1\) si
\(\denote{\Xlo\phi}^{\Modele_1}=0\) \emph{et}
\(\denote{\Xlo\neg\phi}^{\Modele_1}=0\) si
\(\denote{\Xlo\phi}^{\Modele_1}=1\).  Par conséquent, \(\denote{\Xlo\neg
  [\prd{âne}(\cns p) \vee \prd{farceur}(\cns p)]}^{\Modele_1}=0\).

\fussy

\newpage

% -*- coding: utf-8 -*-
\begin{exo}\label{exo:2denot}
Calculez la dénotation relativement à $\Modele_1$ des formules
suivantes : \pagesolution{crg:2denot}
\begin{enumerate}
\item \(\Xlo[\prd{père-de}(\cns o,\cns p) \ssi \prd{elfe}(\cns b)]\)
\item \(\Xlo[\neg\prd{aimer}(\cns d, \cnsi{h}{3}) \implq \prd{triste}(\cnsi{h}{3})]\)
\item \(\Xlo\neg [\prd{elfe}(\cns p) \wedge \prd{farceur}(\cns p)]\)
\item \(\Xlo[\prd{farceur}(\cnsi t1) \implq [\prd{elfe}(\cnsi t1) \implq
    \prd{âne}(\cnsi t1)]]\)
\end{enumerate}
\begin{solu}
(p.~ \pageref{exo:2denot})\label{crg:2denot}

Les dénotations sont calculées en appliquant les règles d'interprétation de la définition \ref{d:Sem1}, p.~\pageref{d:Sem1}.

\begin{enumerate}
\item \(\Xlo[\prd{père-de}(\cns o,\cns p) \ssi \prd{elfe}(\cns b)]\)

\sloppy
La règle  (\RSem\ref{RIcon}d) nous dit que cette formule est vraie dans $\Modele_1$ ssi \(\Xlo\prd{père-de}(\cns o,\cns p)\) et \(\Xlo\prd{elfe}(\cns b)\) ont la même dénotation. \(\denote{\Xlo\prd{père-de}(\cns o,\cns p)}^{\Modele_1}=0\) car \(\tuple{\Obj{Obéron},\Obj{Puck}} \not\in \FI_1(\prd{père-de})\) ; et 
\(\denote{\Xlo\prd{elfe}(\cns b)}^{\Modele_1}=0\) car $\Obj{Bottom}\not\in\FI_1(\prd{elfe})$.  Par conséquent, la formule est vraie dans $\Modele_1$.

\fussy

\item \(\Xlo[\neg\prd{aimer}(\cns d, \cnsi{h}{3}) \implq \prd{triste}(\cnsi{h}{3})]\)

Selon la règle (\RSem\ref{RIcon}c), cette formule est vraie dans $\Modele_1$ ssi 
$\Xlo\neg\prd{aimer}(\cns d, \cnsi{h}{3})$ est fausse \emph{ou} 
$\Xlo\prd{triste}(\cnsi{h}{3})$ est vraie.
Il se trouve que nous remarquons assez rapidement que $\Xlo\prd{triste}(\cnsi{h}{3})$ est vraie dans $\Modele_1$, car $\Obj{Héléna}\in\FI_1(\prd{triste})$ (sachant que \cnsi h3 dénote \Obj{Héléna}). Cela suffit à montrer que la formule est vraie dans $\Modele_1$.

\sloppy

\item \(\Xlo\neg [\prd{elfe}(\cns p) \wedge \prd{farceur}(\cns p)]\)

Cette formule est d'abord une négation, donc nous calculons sa dénotation en consultant d'abord la règle (\RSem\ref{RIneg}) qui dit que la formule est vraie dans $\Modele_1$ ssi \(\Xlo [\prd{elfe}(\cns p) \wedge \prd{farceur}(\cns p)]\) est fausse dans $\Modele_1$. Or \(\denote{\Xlo\prd{elfe}(\cns p)}^{\Modele_1} = 1\) car $\Obj{Puck}\in\FI_1(\prd{efle})$, et
\(\denote{\Xlo\prd{farceur}(\cns p)}^{\Modele_1} = 1\) car $\Obj{Puck}\in\FI_1(\prd{farceur})$.  Donc, en vertu de la règle (\RSem\ref{RIcon}a), 
\(\denote{\Xlo\prd{farceur}(\cns p) \wedge \prd{farceur}(\cns p)}^{\Modele_1} = 1\), et nous en concluons que 
\(\denote{\Xlo\neg[\prd{farceur}(\cns p) \wedge \prd{farceur}(\cns p)]}^{\Modele_1} = 0\).

\fussy

\item \(\Xlo[\prd{farceur}(\cnsi t1) \implq [\prd{elfe}(\cnsi t1) \implq
    \prd{âne}(\cnsi t1)]]\)

D'après (\RSem\ref{RIcon}c), la formule est vraie dans $\Modele_1$ ssi 
\(\Xlo\prd{farceur}(\cnsi t1)\) est fausse ou 
\(\Xlo[\prd{elfe}(\cnsi t1) \implq \prd{âne}(\cnsi t1)]\) est vraie.
Or \(\Xlo\prd{farceur}(\cnsi t1)\) est vraie dans $\Modele_1$ car 
$\Obj{Thésée}\in\FI_1(\prd{farceur})$. 
Nous devons calculer la dénotation de \(\Xlo[\prd{elfe}(\cnsi t1) \implq \prd{âne}(\cnsi t1)]\), qui est vraie ssi 
\(\Xlo\prd{elfe}(\cnsi t1)\) est fausse ou 
\(\Xlo\prd{âne}(\cnsi t1)\) est vraie.
Et \(\Xlo\prd{elfe}(\cnsi t1)\) est effectivement fausse dans $\Modele_1$ ($\Obj{Thésée}\not\in\FI_1(\prd{elfe})$), donc \(\Xlo[\prd{elfe}(\cnsi t1) \implq \prd{âne}(\cnsi t1)]\) est vraie, ce qui fait que la formule globale est vraie aussi.
\end{enumerate}

\end{solu}
\end{exo}



\section{Un peu de logique}
%-----------------------------
\label{Logique}

Afin de bien maîtriser la sémantique de {\LO}, il est temps de faire
un petit détour vers la logique, car en se fondant sur le calcul des
prédicats, {\LO} met surtout en avant les propriétés logiques de la
structure sémantique des phrases.  Nous allons donc examiner de plus
près les connecteurs logiques, ainsi que des notions formelles
attachées aux formules de {\LO}, ce qui nous permettra de nous donner
des règles d'interprétation assez simples pour les formules
quantifiées.  Cette section permettra par ailleurs de s'entraîner un
peu à l'analyse par conditions de vérité.


\subsection{Tables de vérité}
%'''''''''''''''''''''''''''''''
\label{ConnVfctel}\label{TabV}\indexs{connecteur!\elid\ logique|(}

Les connecteurs logiques ($\Xlo\neg$, $\Xlo\wedge$, $\Xlo\vee$, $\Xlo\implq$, $\Xlo\ssi$)
de {\LO} sont également appelés des \kwo{connecteurs vérifonctionnels}\indexs{connecteur!\elid\ vérifonctionnel}.
Cela veut dire que leur dénotation peut être considérée comme une
\emph{fonction} à un (pour $\Xlo\neg$) ou deux (pour les autres)
argument(s) pris dans \set{0;1} et qui retourne une valeur de
\set{0;1}.  C'est ce qu'on appelle une \kwo{fonction de vérité}\indexs{fonction!\elid\ de vérité}.  C'est
juste une façon mathématique de dire que la valeur de vérité d'une
formule construite avec un connecteur dépend simplement de la valeur
de vérité de ses sous-formules connectées.  Et la définition de ces
fonctions nous est donnée dans les règles (\RSem\ref{RIneg}) et
(\RSem\ref{RIcon}). 

Ces règles peuvent s'expliciter  de manière systématique à l'aide de ce
qu'on appelle des \kwo{tables de vérité}.\is{table de vérité}  Une table de vérité est un
tableau qui permet de visualiser très facilement toutes les
dénotations que peuvent prendre une formule (ou un schéma de formules)
construite avec un connecteur. 
Le principe est le suivant :  les  premières colonnes d'une table
vont correspondre aux sous-formules connectées ; on y fait figurer toutes
les combinaisons de valeurs de vérité qu'elles peuvent prendre
(lorsqu'on examine un connecteur binaire, il y a quatre combinaisons
possibles) ; et dans la  colonne suivante, en face de chaque combinaison
de valeurs, on pose la valeur de vérité résultante pour la formule
 complexe.  Les tables de vérité sont aux connecteurs logiques ce que
nos vieilles tables de multiplication sont à la multiplication.
Le tableau~\ref{tv:conn} regroupe les tables de vérité des  quatre
connecteurs logiques de {\LO}.


\begin{table}[h]
\begin{center}
\(
\begin{array}{c|c||c|c|c|c}
\Xlo\phi & \Xlo\psi & \Xlo[\phi \wedge \psi] & \Xlo[\phi \vee \psi] & \Xlo[\phi \implq\psi] & \Xlo[\phi \ssi \psi] \\\hline\hline
1 & 1 & 1 & 1 & 1 & 1\\
1 & 0 & 0 & 1 & 0 & 0\\
0 & 1 & 0 & 1 & 1 & 0\\
0 & 0 & 0 & 0 & 1 & 1\\
\end{array}
\)
\end{center}
\caption{Tables de vérité des connecteurs logiques de {\LO}}\label{tv:conn}
\end{table}

Chaque ligne d'une table est donc une combinaison possible des valeurs
de $\Xlo\phi$ et $\Xlo\psi$.  Par rapport à notre formalisation sémantique,
cela veut dire que chaque ligne résume une certaine
\emph{catégorie} de modèles : la première ligne représente tous les
modèles par rapport auxquels $\Xlo\phi$ et $\Xlo\psi$ sont toutes deux vraies,
la deuxième tous les modèles par rapport auxquels $\Xlo\phi$ est vraie
et $\Xlo\psi$ est fausse, etc.  Quand on regarde la dénotation d'une
formule construite avec un connecteur binaire, on ne s'intéresse qu'à
quatre grandes catégories de modèles.  

On peut aussi dresser la table
de vérité de la négation, et là on n'a à regarder que deux
catégories de modèles : ceux où la formule de départ est vraie et ceux
où elle est fausse ; cf. tableau~\ref{tv:neg}.

\begin{table}[h]
\begin{center}
\(
\begin{array}{c||c}
\Xlo\phi &  \Xlo\neg \phi\\\hline\hline
1 & 0 \\
0 & 1 \\
\end{array}
\)
\end{center}
\caption{Table de vérité de la négation}\label{tv:neg}
\end{table}

Le principe des tables de vérité permet également de calculer toutes
les dénotations possibles de formules complexes, ou plus exactement de
schémas généraux de formules, qui comprennent plusieurs connecteurs%
\footnote{Cependant l'outil des
  tables de vérité n'est opérant que sur les formules qui ne
  contiennent pas de quantificateurs ni de variable, car alors la
  variété de modèles à prendre en compte est immense et ne peut pas se
  résumer à quelques lignes.}.
Techniquement il suffit d'ajouter des colonnes intermédiaires pour les
différentes étapes du calcul et ainsi on obtient progressivement le
résultat final en réutilisant les résultats intermédiaires%
\footnote{Mais attention : il ne faut pas oublier d'énumérer  sur les
  lignes de la table \emph{toutes} les combinaisons 
possibles de valeurs de vérité pour les sous-formules de base prises
en compte.  Plus une formule complexe connecte de formules simples
différentes, plus elle aura de lignes.}.
Ces calculs sont courants en logique, et ici ils peuvent nous être
  utiles car il y a une règle de logique qui dit que deux formules qui
  ont la même table de vérité sont logiquement (et sémantiquement)
  équivalentes\indexs{equivalence@équivalence!\elid\ logique}, puisque cela veut dire qu'elles sont vraies dans
  exactement les mêmes modèles\footnote{Il ne s'agit que d'une reformulation pour {\LO} de ce que nous avions vu au chapitre~\ref{Ch:1} sur l'équivalence logique ; cf.\ la définition~\ref{d:EquLog}, \S\ref{ss:EquivalenceLogique}, p.~\pageref{d:EquLog} ; voir aussi la définition~\ref{d:EquLogLO}, p.~\pageref{d:EquLogLO}.}.

Regardons, par exemple, les dénotations de $\Xlo[\neg \phi \wedge \neg\psi]$
  puis de $\Xlo\neg[\phi \vee \psi]$ données dans le tableau~\ref{tv:morg1} (page suivante).

\begin{table}[h]
\begin{center}
\(
\begin{array}{c|c||c|c|c}
\Xlo\phi & \Xlo\psi & \Xlo\neg\phi & \Xlo\neg \psi & \Xlo[\neg \phi \wedge \neg\psi]  \\\hline\hline
1 & 1 & 0 & 0 & 0\\
1 & 0 & 0 & 1 & 0\\
0 & 1 & 1 & 0 & 0\\
0 & 0 & 1 & 1 & 1\\
\end{array}
\qquad\quad
\begin{array}{c|c||c|c}
\Xlo\phi & \Xlo\psi & \Xlo[\phi \vee \psi] & \Xlo\neg [\phi \vee \psi]  \\\hline\hline
1 & 1 & 1 & 0 \\
1 & 0 & 1 & 0 \\
0 & 1 & 1 & 0 \\
0 & 0 & 0 & 1 \\
\end{array}
\)
\end{center}
\caption{Tables de vérité de $\xlo{[\neg \phi \wedge \neg\psi]}\text{ et de }\xlo{\neg [\phi \vee \psi]}$}\label{tv:morg1}
\end{table}

Dans la table de gauche on calcule d'abord les valeurs pour $\Xlo\neg\phi$
et $\Xlo\neg\psi$, en se servant de la table~\ref{tv:neg} de la
négation ; puis sur ces deux colonnes on applique les règles de calcul
de la conjonction $\Xlo\wedge$ données dans la table~\ref{tv:conn} et cela
nous donne les valeurs pour $\Xlo[\neg \phi \wedge \neg\psi]$.  La
construction de la table de vérité suit les étapes de la construction
syntaxique de la formule.  Et donc dans la table de droite, on
commence par indiquer les valeurs de $\Xlo[\phi \vee \psi]$, et à partir
de cette colonne on calcule les valeurs de $\Xlo\neg[\phi \vee \psi]$.
Les  colonnes de résultat final des deux tables sont identiques, les
deux formules sont 
logiquement (et sémantiquement) équivalentes.  Pour illustrer cela,
reprenons l'exemple~\ref{x:snd3} :

\ex.[\ref{x:snd3}]
\(\Xlo\neg [\prd{âne}(\cns p) \vee \prd{farceur}(\cns p)]\)


Cette formule peut littéralement traduire la phrase française \sicut{il
est faux que Puck est un âne ou un farceur} ; en clair, cela signifie
la même chose que \sicut{Puck n'est ni un âne, ni un farceur}, autrement
dit \sicut{Puck n'est pas un âne et Puck n'est pas un farceur}, ce qui se
traduit bien en \ref{x:snd3}$'$ :

\ex.[\ref{x:snd3}$'$]
\(\Xlo[\neg\prd{âne}(\cns p) \wedge \neg\prd{farceur}(\cns p)]\)


Cette équivalence logique fait partie de ce qui est connu sont le nom
de \kwo{lois de  Morgan}\indexs{loi!\elid\ de morgan@\elid s de  Morgan}.  Il y en a une seconde qui est que
$\Xlo\neg[\phi \wedge \psi]$ et $\Xlo[\neg \phi \vee \neg \psi]$ sont
aussi logiquement équivalentes.  On peut l'illustrer avec les exemples
suivants : \sicut{il est faux que Puck est un âne et un farceur},
c'est-à-dire \sicut{Puck n'est pas un âne farceur}, ce qui veut bien dire
\sicut{Puck n'est pas un âne ou bien Puck n'est pas farceur}.
Les lois de  Morgan peuvent donc s'avérer utiles pour comprendre
certaines formules ou phrases négatives.

\begin{theo}[Lois de  Morgan]
Pour toutes formules \vrb\phi\ et {\vrb\psi} :
\begin{enumerate}
\item  $\Xlo\neg[\phi \wedge \psi]$ et  $\Xlo[\neg \phi \vee \neg \psi]$ sont logiquement équivalentes.
\item
 $\Xlo\neg[\phi \vee \psi]$ et  $\Xlo[\neg \phi \wedge \neg \psi]$ sont logiquement équivalentes.
\end{enumerate}
\end{theo}


% -*- coding: utf-8 -*-
\begin{exo}\label{exo:[]}
\begin{enumerate}
\item Montrez, par la méthode des tables de vérité, 
\pagesolution{crg:[]}%
que $\Xlo[[\phi \wedge \psi]
\wedge \chi]$ et  $\Xlo[\phi \wedge [\psi \wedge \chi]]$ sont logiquement
équivalentes.  NB : ici les tables auront 8 lignes.
\item
De même pour $\Xlo[[\phi \vee \psi] \vee \chi]$ et  $\Xlo[\phi
  \vee [\psi \vee \chi]]$. 
\item
Montrez que $\Xlo[[\phi \implq \psi] \implq \chi]$ et $\Xlo[\phi
  \implq [\psi \implq \chi]]$ \emph{ne sont pas} logiquement
  équivalentes.
\end{enumerate}
\begin{solu} (p.~\pageref{exo:[]})\label{crg:[]}

Nous dressons les tables de vérité (\S\ref{TabV}, p.~\pageref{TabV}) des formules de chaque paire et nous comparons les colonnes de résultats.
\begin{enumerate}
\item Tables de vérité de $\Xlo[[\phi \wedge \psi] \wedge \chi]$ 
et de $\Xlo[\phi \wedge [\psi \wedge \chi]]$ :
\small\[
\begin{array}{c|c|c||c|>{\columncolor[gray]{.9}}c||c|>{\columncolor[gray]{.9}}c}
\Xlo\phi & \Xlo\psi & \Xlo\chi & \Xlo[\phi \wedge \psi] & \cellcolor{white}\Xlo[[\phi \wedge \psi] \wedge
  \chi] & \Xlo[\psi \wedge \chi] & \cellcolor{white}\Xlo[\phi \wedge [\psi \wedge \chi]]
\\\hline\hline
1 & 1 & 1 & 1 & 1 & 1 & 1\\
1 & 1 & 0 & 1 & 0 & 0 & 0\\
1 & 0 & 1 & 0 & 0 & 0 & 0\\
1 & 0 & 0 & 0 & 0 & 0 & 0\\
0 & 1 & 1 & 0 & 0 & 1 & 0\\
0 & 1 & 0 & 0 & 0 & 0 & 0\\
0 & 0 & 1 & 0 & 0 & 0 & 0\\
0 & 0 & 0 & 0 & 0 & 0 & 0\\
\end{array}
\]\normalsize
Les deux colonnes de résultats sont identiques, donc les deux formules
sont bien équivalentes.

\item Tables de vérité de 
 $\Xlo[[\phi \vee \psi] \vee \chi]$ et  $\Xlo[\phi
  \vee [\psi \vee \chi]]$: 
\small\[
\begin{array}{c|c|c||c|>{\columncolor[gray]{.9}}c||c|>{\columncolor[gray]{.9}}c}
\Xlo\phi & \Xlo\psi & \Xlo\chi & \Xlo[\phi \vee \psi] & \cellcolor{white}\Xlo[[\phi \vee \psi] \vee
  \chi] & \Xlo[\psi \vee \chi] & \cellcolor{white}\Xlo[\phi \vee [\psi \vee \chi]]
\\\hline\hline
1 & 1 & 1 & 1 & 1 & 1 & 1\\
1 & 1 & 0 & 1 & 1 & 1 & 1\\
1 & 0 & 1 & 1 & 1 & 1 & 1\\
1 & 0 & 0 & 1 & 1 & 0 & 1\\
0 & 1 & 1 & 1 & 1 & 1 & 1\\
0 & 1 & 0 & 1 & 1 & 1 & 1\\
0 & 0 & 1 & 0 & 1 & 1 & 1\\
0 & 0 & 0 & 0 & 0 & 0 & 0\\
\end{array}
\]\normalsize
Même conclusion que précédemment.

\item Tables de vérité de
 $\Xlo[[\phi \implq \psi] \implq \chi]$ et $\Xlo[\phi
  \implq [\psi \implq \chi]]$ :
\small\[
\begin{array}{c|c|c||c|>{\columncolor[gray]{.9}}c||c|>{\columncolor[gray]{.9}}c}
\Xlo\phi & \Xlo\psi & \Xlo\chi & \Xlo[\phi \implq \psi] & \cellcolor{white}\Xlo[[\phi \implq \psi] \implq
  \chi] & \Xlo[\psi \implq \chi] & \cellcolor{white}\Xlo[\phi \implq [\psi \implq \chi]]
\\\hline\hline
1 & 1 & 1 & 1 & 1 & 1 & 1\\
1 & 1 & 0 & 1 & 0 & 0 & 0\\
1 & 0 & 1 & 0 & 1 & 1 & 1\\
1 & 0 & 0 & 0 & 1 & 1 & 1\\
0 & 1 & 1 & 1 & 1 & 1 & 1\\
0 & 1 & 0 & 1 & \cellcolor[gray]{.8}0 & 0 & \cellcolor[gray]{.8}1\\
0 & 0 & 1 & 1 & 1 & 1 & 1\\
0 & 0 & 0 & 1 & \cellcolor[gray]{.8}0 & 1 & \cellcolor[gray]{.8}1\\
\end{array}
\]\normalsize
Les deux colonnes de résultats sont différentes, donc les deux
formules ne sont pas équivalentes.
\end{enumerate}
\end{solu}
\end{exo}


Les deux équivalences de l'exercice \ref{exo:[]} disent finalement
que par exemple lorsqu'on a deux conjonctions qui «se suivent»
dans une formule, peu importe comment on les regroupe, c'est-à-dire
comment on place les crochets, la dénotation de la formule reste
toujours la même.  On dit que la conjonction et la disjonction sont
des connecteurs \emph{associatifs} (comme le sont l'addition et la
multiplication sur les nombres), et pour cette raison on peut
s'autoriser à faire l'économie de ces crochets dans l'écriture des formules :
\is{regle@règle!\elid\ de suppression des crochets}
ils n'ont pas d'impact sémantique.  Ainsi nous pourrons écrire de temps en
temps  $\Xlo[\phi \wedge \psi \wedge \chi]$ et $\Xlo[\phi \vee \psi \vee
  \chi]$  à la place des quatre premières formules de l'exercice
\ref{exo:[]}.  Mais il faut bien garder à l'esprit que $\Xlo[\phi \wedge
  \psi \wedge \chi]$ n'est qu'une commodité d'écriture : en toute
rigueur, ce n'est pas une formule bien formée de {\LO}, mais juste un
raccourci qui vaut indifféremment pour $\Xlo[[\phi \wedge \psi]
\wedge \chi]$ ou  $\Xlo[\phi \wedge [\psi \wedge \chi]]$.

En revanche, comme le montre la troisième partie de l'exercice, $\Xlo[\phi
\implq \psi \implq \chi]$ n'est pas une écriture autorisée, cette
(pseudo-)formule n'a vraiment aucun sens.  De même on n'a pas le droit
d'écrire $\Xlo[\phi \wedge \psi \vee \chi]$.

Il y a une autre règle de suppression des crochets que nous pouvons nous
accorder pour alléger les écritures de formules.  Les crochets
servent à regrouper les sous-formules à l'intérieur d'une formule
complexe pour éviter toute équivoque et pour bien marquer le «champ
d'application» d'un connecteur.  Par conséquent lorsqu'une formule
(mais pas une sous-formule !) est entièrement encadrée de crochets,
ceux-ci ne servent pas à grand chose, car justement cette formule
globale n'est connectée à aucune autre.  Ainsi il est sémantiquement
inoffensif de supprimer les crochets les plus extérieurs d'une formule
et on peut écrire \(\Xlo\phi \implq \psi\) à la place de \(\Xlo[\phi
  \implq\psi]\) tant qu'il s'agit d'une formule autonome.  Mais
attention, cela ne concerne que les crochets complètement extérieurs :
on n'a pas le droit de les supprimer dans \(\Xlo\neg[\phi \wedge \psi]\)
ou dans \(\Xlo\exists x [\prd{elfe}(x) \wedge \prd{farceur}(x)]\) (mais on
a le droit d'écrire \(\Xlo\exists x\,\prd{elfe}(x) \wedge \exists x\,
  \prd{farceur}(x)\) pour \(\Xlo[\exists x\, \prd{elfe}(x) \wedge \exists x\,
  \prd{farceur}(x)]\)).  


\medskip

% -*- coding: utf-8 -*-
\begin{exo}\label{exo:equivlog}
Montrez que les paires de formules suivantes sont chacune logiquement
\pagesolution{crg:equivlog}%
équivalentes : 
\begin{enumerate}
\item \(\Xlo\phi\) et \(\Xlo\neg\neg\phi\)
\label{Nex:2neg}
\item \(\Xlo[\phi \implq \psi]\) et \(\Xlo[\neg\phi \vee \psi]\) 
\label{Nex:implq}
\item \(\Xlo[\phi \implq \psi]\) et \(\Xlo[\neg\psi \implq \neg\phi]\) 
\label{Nex:contraposition}
\item \(\Xlo[\phi \implq [\psi \implq \chi]]\) et \(\Xlo[[\phi \wedge \psi]
  \implq \chi]\)
\label{Nex:2implq}
\item \(\Xlo[\phi \wedge [\psi \vee \chi]]\) et \(\Xlo[[\phi \wedge \psi] \vee
  [\phi \wedge \chi]]\)
 \item \(\Xlo[\phi \vee [\psi \wedge \chi]]\) et \(\Xlo[[\phi \vee \psi] \wedge
  [\phi \vee \chi]]\)
\end{enumerate}
Il n'est pas inutile de connaître par c\oe ur ces équivalences (la \numero
\ref{Nex:2neg} s'appelle d'ailleurs la \emph{loi de la double
  négation}\index[sbj]{loi!\elid\ de la double négation} 
et la \numero \ref{Nex:contraposition}  la \emph{loi de
  contraposition}%
\footnote{Un exemple d'équivalence via la double négation est
  donnée par \sicut{Marie fume} et \sicut{il est faux que
  Marie ne fume pas}.  La loi de contraposition peut s'illustrer
  par \sicut{s'il y a du feu, il y a de la fumée} et
  \sicut{s'il n'y a pas de fumée, il n'y a pas de feu}.
}%
).\index[sbj]{loi!\elid\ de contraposition} 
%
\begin{solu} (p.~\pageref{exo:equivlog}) %Équivalences logiques.
\label{crg:equivlog}

Par défaut, les équivalences se démontrent par la méthode des tables de vérité comme dans l'exercice précédent, mais par endroits il est possible de procéder autrement.
\begin{enumerate}
\item \(\Xlo\phi\) et \(\Xlo\neg\neg\phi\)
\small\[
\begin{array}{>{\columncolor[gray]{.9}}c||c|>{\columncolor[gray]{.9}}c}
\cellcolor{white}\Xlo\phi & \Xlo\neg\phi & \cellcolor{white}\Xlo\neg\neg\phi\\\hline\hline
1&0&1\\
0&1&0\\
\end{array}
\]\normalsize
En fait, c'est presque laborieux de faire la table de vérité, car la démonstration est triviale et immédiate : $\Xlo\neg$ inverse les valeurs de vérité, donc si on inverse deux fois (ou un nombre paire de fois), on retombe sur la valeur initiale (comme retourner deux fois une pièce de monnaie).

\item \(\Xlo[\phi \implq \psi]\) et \(\Xlo[\neg\phi \vee \psi]\)
\small\[
\begin{array}{c|c||>{\columncolor[gray]{.9}}c||c|>{\columncolor[gray]{.9}}c}
\Xlo\phi & \Xlo\psi & \cellcolor{white}\Xlo[\phi\implq\psi] & \Xlo\neg\phi& \cellcolor{white}\Xlo[\neg\phi\vee\phi]
\\\hline\hline
1 & 1 & 1 & 0 & 1\\
1 & 0 & 0 & 0 & 0\\
0 & 1 & 1 & 1 & 1\\
0 & 0 & 1 & 1 & 1\\ 
\end{array}
\]\normalsize

\item \(\Xlo[\phi \implq \psi]\) et \(\Xlo[\neg\psi \implq \neg\phi]\) 

Ici on pourrait assez rapidement et facilement faire la table de
vérité, mais on peut aussi démontrer l'équivalence très simplement,
en raisonnant.  On sait, par la démonstration précédente, que $\Xlo[\phi
  \implq \psi]$ équivaut à $\Xlo[\neg\phi \vee \psi]$. Pour la même
raison, on sait que $\Xlo[\neg\psi \implq \neg\phi]$ équivaut à  
$\Xlo[\neg\neg\psi \vee \neg\phi]$, ce qui équivaut à
$\Xlo[\psi \vee \neg\phi]$ en vertu de la loi de double négation démontrée
ci-dessus en 1.  Et comme la disjonction est {commutative}, 
% Attention : commutatif pas défini dans le chapitre
cela équivaut à $\Xlo[\neg\phi \vee \psi]$, et donc à $\Xlo[\phi
  \implq \psi]$.

\item \(\Xlo[\phi \implq [\psi \implq \chi]]\) et \(\Xlo[[\phi \wedge \psi]
  \implq \chi]\)

Là aussi on peut démontrer l'équivalence en utilisant celle démontrée
ci-dessus en 2, ainsi que d'autres démontrées précédemment. Par
l'équivalence 2, on sait que
$\Xlo[\phi \implq [\psi \implq \chi]]$ équivaut à
$\Xlo[\neg\phi \vee [\psi \implq \chi]]$, qui équivaut à
$\Xlo[\neg\phi \vee [\neg\psi \vee \chi]]$.  Or on a démontré que $\Xlo\vee$
est commutative, donc la formule équivaut aussi à 
$\Xlo[[\neg\phi \vee \neg\psi] \vee \chi]$.  Et par une des lois de
Morgan, on sait aussi que $\Xlo[\neg\phi \vee \neg\psi]$ équivaut à
$\Xlo\neg[\phi \wedge \psi]$.  Donc, par remplacement, notre formule
équivaut à $\Xlo[\neg[\phi \wedge \psi] \vee \chi]$. Or cette formule
répond au schéma $\Xlo[\neg X\vee Y]$ de l'équivalence 2. Donc on en
conclut que notre formule équivaut à \(\Xlo[[\phi \wedge \psi]
  \implq \chi]\).

\item \(\Xlo[\phi \wedge [\psi \vee \chi]]\) et \(\Xlo[[\phi \wedge \psi] \vee
  [\phi \wedge \chi]]\)
\small\[
\begin{array}{c|c|c||c|>{\columncolor[gray]{.9}}c||c|c||>{\columncolor[gray]{.9}}c}
\Xlo\phi & \Xlo\psi & \Xlo\chi & \Xlo[\psi \vee \chi] & \cellcolor{white}\Xlo[\phi \wedge [\psi \vee
  \chi]] & \Xlo[\phi \wedge \psi] & \Xlo[\phi \wedge \chi] & \cellcolor{white}\Xlo[[\phi \wedge\psi]\vee[\phi\wedge\chi]]
\\\hline\hline
1 & 1 & 1 & 1 & 1 & 1 & 1 & 1\\
1 & 1 & 0 & 1 & 1 & 1 & 0 & 1\\
1 & 0 & 1 & 1 & 1 & 0 & 1 & 1\\
1 & 0 & 0 & 0 & 0 & 0 & 0 & 0\\
0 & 1 & 1 & 1 & 0 & 0 & 0 & 0\\
0 & 1 & 0 & 1 & 0 & 0 & 0 & 0\\
0 & 0 & 1 & 1 & 0 & 0 & 0 & 0\\
0 & 0 & 0 & 0 & 0 & 0 & 0 & 0\\
\end{array}
\]\normalsize

\item \(\Xlo[\phi \vee [\psi \wedge \chi]]\) et \(\Xlo[[\phi \vee \psi] \wedge
  [\phi \vee \chi]]\)
\small\[
\begin{array}{c|c|c||c|>{\columncolor[gray]{.9}}c||c|c||>{\columncolor[gray]{.9}}c}
\Xlo\phi & \Xlo\psi & \Xlo\chi & \Xlo[\psi \wedge \chi] & \cellcolor{white}\Xlo[\phi \vee
  [\psi \wedge \chi]] & \Xlo[\phi \vee \psi] & \Xlo[\phi \vee \chi] & \cellcolor{white}\Xlo[[\phi \vee\psi]\wedge[\phi\vee\chi]]
\\\hline\hline
1 & 1 & 1 & 1 & 1 & 1 & 1 & 1\\
1 & 1 & 0 & 0 & 1 & 1 & 1 & 1\\
1 & 0 & 1 & 0 & 1 & 1 & 1 & 1\\
1 & 0 & 0 & 0 & 1 & 1 & 1 & 1\\
0 & 1 & 1 & 1 & 1 & 1 & 1 & 1\\
0 & 1 & 0 & 0 & 0 & 1 & 0 & 0\\
0 & 0 & 1 & 0 & 0 & 0 & 1 & 0\\
0 & 0 & 0 & 0 & 0 & 0 & 0 & 0\\
\end{array}
\]\normalsize
\end{enumerate}
\end{solu}
\end{exo}



\subsection{Disjonctions inclusive et exclusive}
%''''''''''''''''''''''''''''''''''''''''''''''''''
%Sur les disjonction inclusives vs. exclusives

La table de vérité de la disjonction $\Xlo\vee$, répétée dans le
tableau~\ref{tv:ou}, de même que la règle (\RSem\ref{RIcon}b) nous
disent que $\Xlo[\phi \vee \psi]$ est vraie si l'une des deux
sous-formules est vraie et aussi lorsqu'elles sont vraies toutes
deux.  C'est ce qu'on appelle la \kwo{disjonction inclusive}\is{disjonction!\elid\ inclusive}, et c'est
pourquoi elle est parfois prononcée «et/ou». Il existe
en logique un autre connecteur disjonctif, qu'on appelle la
\kwo{disjonction exclusive}\is{disjonction!\elid\ exclusive} ; notons-la par le symbole $\Xlo\xou$%
\footnote{Ce symbole n'est pas spécialement consacré dans la
  littérature logique.  Mais à ma connaissance, il n'y a aucun
  consensus de notation clairement établi ; on trouve facilement les
  variantes suivantes : $\veebar$, $\triangledown$, $+$, $\oplus$,
  $\not\equiv$, etc., ce qui tend à montrer le côté relativement marginal de ce connecteur. } ; sa table
de vérité est donnée dans le tableau~\ref{tv:ou}.

\begin{table}[h]
\begin{center}
\(
\begin{array}{c|c||c|c}
\Xlo\phi & \Xlo\psi & \Xlo[\phi \vee \psi] & \Xlo[\phi \xou \psi] \\\hline\hline
1 & 1 & 1 & 0 \\
1 & 0 & 1 & 1 \\
0 & 1 & 1 & 1 \\
0 & 0 & 0 & 0 \\
\end{array}
\)
\end{center}
\caption{Tables de vérité des disjonctions inclusive et exclusive}\label{tv:ou}
\end{table}


Nous voyons que $\Xlo[\phi \xou \psi]$ est vraie si l'une des deux
sous-formules est vraie mais pas les deux en même temps. C'est fromage
ou dessert.

Une question importante qui  se pose en sémantique, c'est de
savoir si les expressions de la langue qui expriment la disjonction, en
français \sicut{ou}, \sicut{ou bien}, \sicut{soit...,   soit...}, doivent s'analyser par une disjonction inclusive ou
exclusive.  Nous pouvons deviner la réponse puisque c'est $\Xlo\vee$ et pas
$\Xlo\xou$ qui a été
introduit dans {\LO}, mais regardons les choses plus précisément.

En fait, il peut sembler que la plupart (ou au moins un grand nombre)
des phrases disjonctives du français sont plus naturellement comprises
comme exprimant une
disjonction exclusive plutôt qu'inclusive.  Un exemple classique est
\ref{x:disj1} :

\ex.  \label{x:disj1}
Alice est dans sa chambre ou dans la salle de bain.


Il paraît assez évident que cette phrase ne suggère pas qu'Alice est
peut-être à la fois dans sa chambre et dans la salle de bain.  La
disjonction en \ref{x:disj1} a bien l'air exclusive.  
Remarquons au passage que, parmi les raisons qui peuvent amener un
locuteur à utiliser une disjonction dans son énoncé, l'une des plus
courantes est lorsqu'il ne dispose pas d'une information absolument
certaine, que pour lui deux possibilités sont en balance.  Mais cela
ne veut pas dire que ces énoncés ne sont pas informatifs ; ils ne sont
pas vagues ; \ref{x:disj1} exclut tous les autres endroits où Alice
peut être.

Les phrases suivantes fonctionnent de la même manière, en illustrant
des disjonctions exclusives :

\ex.  
\a. 
Ce type, il s'appelle Hubert ou Herbert.\label{x:disj2}
\b.  
Daniel travaille à Biarritz ou à Bayonne. \label{x:disj3}
\b.  
Cet été pour les vacances, Adèle est allée en Espagne ou en Italie.\label{x:disj4}


Mais il faut faire une remarque importante ici : ce n'est pas forcément
parce que l'on comprend ces phrases comme exprimant une disjonction
exclusive ou qu'elles nous semblent exprimer une disjonction exclusive
que leur \emph{sens} doit être analysé de la sorte.  Nous savons à
présent comment décrire le sens d'une phrase : ce sont ses conditions
de vérité.  Or demandons-nous dans quels cas la phrase \ref{x:disj4}
est vraie.  Elle est vraie bien sûr si Adèle est effectivement allée
en Espagne ou si elle est allée en Italie.  Mais qu'en est-il si elle
a fait deux voyages pendant ses vacances et qu'elle a visité les deux
pays ?  Dans ce cas, force est d'admettre que \ref{x:disj4} est
également vraie.  Autrement dit, les conditions de vérité nous donnent
clairement une disjonction inclusive.  La phrase \ref{x:disj5} en
est un autre exemple.  Imaginez un inspecteur menant une enquête et
qui, après avoir recueilli des indices et des témoignages, énonce la
conclusion suivante :

\ex. \label{x:disj5}
Le coupable est roux ou il porte une perruque rousse.


Devra-t-on alors écarter un suspect qui à la fois serait roux et
porterait une perruque rousse ?  Bien sûr que non.  Par conséquent,
même si elle n'en a pas l'air, la phrase \ref{x:disj5} a une
structure sémantique de disjonction inclusive.


On peut cependant faire une double objection : d'abord les disjonctions
des phrases \ref{x:disj1}, \ref{x:disj2} et\ref{x:disj3} ont l'air «plus»
exclusives qu'en \ref{x:disj4} et \ref{x:disj5}, ensuite comment
expliquer que malgré tout il y a une lecture exclusive qui se présente
nettement lorsqu'on lit tous ces exemples ?  Il y a plusieurs manières
de répondre à cela.

Une première explication serait d'ordre strictement sémantique et
grammaticale.  Elle dirait que la conjonction \sicut{ou} du
français\footnote{Mais cela se retrouverait probablement dans toutes
  les langues.} est ambiguë : il y aurait deux  \sicut{ou}, un
inclusif et un exclusif.  Ainsi \ref{x:disj4} recevrait deux
traductions possibles en {\LO} :

\ex. %[\ref{x:disj4}$'$]   
\a. \(\Xlo[\prd{aller}(\cns a,\cns e) \vee \prd{aller}(\cns a,\cns i)]\)
\b. \(\Xlo[\prd{aller}(\cns a,\cns e) \xou \prd{aller}(\cns a,\cns i)]\)


Mais il a été montré (par \citet{Gzdr:79}\Andex{Gazdar, G.} notamment)
que cette analyse 
est difficilement défendable et on retient préférentiellement des
explications plus pragmatiques.

\largerpage[2]

D'abord il ne faut pas manquer d'observer dans les exemples
\ref{x:disj1} et \ref{x:disj2} que, dans le cours normal des
choses, les cas où les deux membres de la disjonction sont vrais
(Alice est à deux endroits à la fois, le type porte deux prénoms
usuels) sont hautement improbables, si ce n'est impossibles ou
absurdes.  
Il s'agit là en fait d'une sorte de \emph{présupposé} de bon sens,
quelque chose que le locuteur et l'allocutaire savent ou présument
ensembles et tacitement.   Et cela se reflète dans les conditions de
vérité : savoir si \ref{x:disj1} est vraie lorsqu'Alice est dans les deux
pièces à la fois est tout simplement sans intérêt, sans pertinence.
Et on peut même dire que dans un tel cas, \ref{x:disj1} n'est ni
vraie ni fausse, mais hors de propos --~ce qui est bien une propriété
des présuppositions.  Si on tient compte de cette présupposition, la
disjonction n'a de valeur que pour dire qu'un et un seul des deux
membres est vrai, ce qui provoque bien une lecture exclusive.
Simplement ce qui fait l'exclusivité de la disjonction n'est pas
affirmé par le locuteur mais présupposé, et ce n'est donc pas dans la
structure sémantique de sa phrase.
En bref, nos phrases ici sont sémantiquement des disjonctions
inclusives qui par présupposition reviennent à des disjonctions
exclusives\footnote{Mais attention : il s'agit bien là de
  présuppositions \emph{pragmatiques}, en ce sens qu'elles constituent
des informations évidentes.  Le fait qu'on est enclin à présumer
qu'une même personne ne peut pas être à la fois dans deux endroits
différents est indépendant de la sémantique de la disjonction en
français.  Il serait donc fautif de considérer ici la conjonction
\sicut{ou} comme un déclencheur de présupposition.}. 

Il y a une autre explication pragmatique, qui ne s'oppose pas à la
première, et selon laquelle les lectures disjonctives exclusives sont
des effets d'implicatures conversationnelles.  L'idée est la suivante :
une conjonction de deux formules $\Xlo[\phi \wedge \psi]$ est plus
informative, plus précise que la disjonction de ces mêmes formules
$\Xlo[\phi \vee \psi]$, car la conjonction est vraie dans moins de cas que
la disjonction (cf. tableau~\ref{tv:conn}) et car $\Xlo[\phi \vee\psi]$
est une conséquence logique de $\Xlo[\phi \wedge \psi]$ (quand la
conjonction est vraie la disjonction l'est aussi).  Par exemple, dire
qu'Adèle est allée en Espagne \emph{et} en Italie est évidemment plus
précis que de dire qu'elle est allée en Espagne \emph{ou} en Italie.
Par conséquent, si
un locuteur énonce la formule la moins informative, c'est qu'il est
probable qu'il ne pense pas que la formule la plus informative soit
vraie.  Autrement dit, une implicature que l'on peut tirer normalement
de $\Xlo[\phi \vee \psi]$ c'est $\Xlo\neg[\phi \wedge \psi]$.  Et si on réunit
les deux, cela nous donne : $\Xlo\phi$ ou $\Xlo\psi$ et pas les deux à la fois,
ce qui est bien l'interprétation de la disjonction exclusive.
Précédemment, l'exclusivité était donnée a priori par une
présupposition, ici elle est obtenue a posteriori par une implicature.


Terminons ces observations avec le coup de grâce que nous pouvons
porter à une analyse sémantique de la disjonction exclusive.
Reprenons l'hypothèse qui suggérait que \sicut{ou} soit
sémantiquement (\ie\ lexicalement) ambigu entre $\Xlo\vee$ et $\Xlo\xou$.  Cela
veut dire que toute phrase en \sicut{ou} reçoit deux traductions
\emph{plausibles} et qu'éventuellement ensuite l'une des
deux soit écartée parce que peu appropriée dans le contexte. 

Regardons alors ce qui se passe avec une disjonction multiple,
c'est-à-dire à plus de deux membres, comme par exemple
\ref{x:disjn} :

\ex.  \label{x:disjn}
Marie a prévenu Pierre ou Albert ou Jacques.


Si \sicut{ou} est ambigu, alors \ref{x:disjn} peut se traduire
au moins 
de deux façons :

\ex. %[\ref{x:disjn}$'$]
\a. \(\Xlo[[\prd{prévenir}(\cns m,\cns p) \vee \prd{prévenir}(\cns m, \cns a)]
  \vee \prd{prévenir}(\cns m,\cns j)]\)
\quad ou
\b. \(\Xlo[[\prd{prévenir}(\cns m,\cns p) \xou \prd{prévenir}(\cns m, \cns a)]
  \xou \prd{prévenir}(\cns m,\cns j)]\)

c'est-à-dire les schémas de formules $\Xlo[[\phi \vee \psi] \vee \chi]$ ou
$\Xlo[[\phi \xou \psi] \xou \chi]$.  Avant même d'analyser ces formules,
on peut déjà expliciter les conditions de vérité de \ref{x:disjn}
simplement en regardant la phrase.
On peut y voir effectivement deux lectures, une inclusive et une
exclusive.  Pour la lecture disjonctive inclusive, \ref{x:disjn} est
vraie si et seulement si Marie a prévenu \emph{au moins un} des trois garçons --~et
donc c'est  vrai aussi si elle en a prévenu deux ou les trois.  Pour
la lecture disjonctive exclusive c'est moins simple car  il y a deux
manières de voir les choses. On 
pourrait envisager une lecture exclusive faible qui dirait que
\ref{x:disjn} est vraie ssi Marie a prévenu au moins un des
trois garçons mais pas tous les trois --~ce serait donc vrai si elle
  en a prévenu deux.  Mais cette interprétation semble peu naturelle
  et plutôt compliquée.  L'autre lecture serait exclusive forte, elle
  dirait que   \ref{x:disjn} est vraie ssi Marie a prévenu \emph{un seul} des
trois garçons.
Examinons maintenant ce que nous disent les tables de vérité de nos
deux schémas de formules.  Elles sont donnée dans le tableau~\ref{tv:2ou}.

\begin{table}[h]
\begin{center}
\(
\begin{array}{c|c|c||c|c}
\Xlo\phi & \Xlo\psi & \chi & \Xlo[\phi \vee \psi] & \Xlo[[\phi \vee \psi] \vee
  \chi]\\\hline\hline 
1 & 1 & 1 & 1 & 1\\
1 & 1 & 0 & 1 & 1\\
1 & 0 & 1 & 1 & 1\\
1 & 0 & 0 & 1 & 1\\
0 & 1 & 1 & 1 & 1\\
0 & 1 & 0 & 1 & 1\\
0 & 0 & 1 & 0 & 1\\
0 & 0 & 0 & 0 & 0
\end{array}
\qquad
\begin{array}{c|c}
[\Xlo\phi \xou \psi] & \Xlo[[\phi \xou \psi] \xou \chi] \\\hline\hline
0 & 1 \\
0 & 0 \\
1 & 0 \\
1 & 1 \\
1 & 0 \\
1 & 1 \\
0 & 1 \\
0 & 0 \\
\end{array}
\)
\end{center}
\caption{Tables de vérité de disjonctions à trois termes}\label{tv:2ou}
\end{table}

La table de $\Xlo[[\phi \vee \psi] \vee \chi]$ est bien conforme à nos
attentes.  Quant à celle de $\Xlo[[\phi \xou \psi] \xou \chi]$ elle
correspond \emph{presque} à la lecture exclusive forte ;
«presque» à cause de la première ligne qui ne donne pas le
résultat escompté.  Et c'est beaucoup plus grave qu'il n'y paraît car
si on recommence l'exercice avec une disjonction à quatre ou cinq
membres, on constatera une étonnante régularité : une disjonction
exclusive multiple (en $\Xlo\xou$) est vraie ssi il y a un \emph{nombre
impair} de sous-formules atomiques connectées qui sont vraies.  Et
cette règle n'a décidément rien à voir avec ce que peut vouloir dire
un locuteur qui énonce une phrase disjonctive.  Autrement dit, le
connecteur {$\Xlo\xou$} donne des résultats erronés quand il s'agit
d'exprimer le sens des phrases du français.

La conclusion de tout cela est que d'un point vue strictement
sémantique, c'est-à-dire concernant uniquement les conditions de
vérité, l'expression d'une disjonction en français sera toujours
traduite par une disjonction inclusive, $\Xlo\vee$.  C'est à un autre
niveau de l'analyse (pragmatique) que se manifeste le caractère
exclusif de l'interprétation d'une disjonction.  Nous n'utiliserons
donc pas $\Xlo\xou$ dans {\LO}.



\subsection{Implication matérielle}
%'''''''''''''''''''''''''''''''''''''
\indexs{implication!\elid\ matérielle}

Le connecteur de l'\kwo{implication matérielle}, $\Xlo\implq$, a été présenté
comme étant ce qui permet de traduire en {\LO} les constructions
conditionnelles  en
\sicut{si..., (alors)...}.  Voici un exemple :

\ex. 
Si Démétrius n'aime pas Héléna, alors elle sera triste.
\\
 \(\Xlo[\neg\prd{aimer}(\cns d, \cnsi{h}{3}) \implq \prd{triste}(\cnsi{h}{3})]\)


\largerpage[-1]
Introduisons d'abord deux éléments de vocabulaire attachés aux implications :

\begin{defi}[Antécédent et conséquent d'une implication]
Dans une implication de la forme $\Xlo[\phi \implq \psi]$, la sous-formule
$\Xlo\phi$ s'appelle l'\kwo{anté\-cédent}\indexs{antecedent@antécédent!\elid\ d'une implication}
de l'implication, et la sous-formule $\Xlo\psi$ son
\kwo{conséquent}\indexs{consequent@conséquent (d'une implication)}. \\
Par extension on parle aussi de l'antécédent et du conséquent d'une
phrase conditionnelle pour désigner respectivement la proposition
subordonnée (en \sicut{si}) et la principale\footnotemark.
\end{defi}%
\footnotetext{L'antécédent d'une conditionnelle est parfois aussi appelé la \emph{protase}\indexs{protase} et son conséquent l'\emph{apodose}.\indexs{apodose}}

Les conditions de vérité données en (\RSem\ref{RIcon}c) et la table de
vérité, reproduite dans le tableau~\ref{tv:implq}, peuvent paraître un
peu éloignées de ce que l'on attend de la sémantique d'une structure
conditionnelle du français en \sicut{si}.  Il faut d'abord
savoir que la formulation de (\RSem\ref{RIcon}c) n'est qu'une façon
de dire parmi d'autres variantes équivalentes ; on aurait pu tout aussi
bien écrire : 
$\denote{\Xlo[\phi \implq \psi]}^{\Modele}=1$ si et seulement si, si
$\denote{\Xlo\phi}^{\Modele}=1$ alors $\denote{\Xlo\psi}^{\Modele}=1$ aussi.
Mais la version (\RSem\ref{RIcon}c) a le mérite d'être plus analytique
et moins alambiquée. 


\begin{table}[h]
\begin{center}
\(
\begin{array}{c|c||c}
\Xlo\phi & \Xlo\psi & \Xlo[\phi \implq \psi]  \\\hline\hline
1 & 1 & 1 \\
1 & 0 & 0  \\
0 & 1 & 1  \\
0 & 0 & 1  \\
\end{array}
\)
\end{center}
\caption{Table de vérité de {$\Xlo\implq$}}\label{tv:implq}
\end{table}

Ce qui gêne souvent dans la table de vérité de $\Xlo\implq$, c'est sa
troisième ligne : comment le faux peut-il impliquer le vrai ? ou
comment de quelque chose de faux peut-on déduire quelque chose de vrai
? 

D'abord, il faut bien être avisé du fait que l'implication matérielle
n'est pas la même chose que la déduction ; ce sont deux concepts
orthogonaux.  L'implication matérielle est un symbole de {\LO} qui
permet d'écrire des formules ; la déduction est un type de
raisonnement, ou éventuellement un jugement que l'on porte sur une
forme de raisonnement (on dit par exemple que tel raisonnement est
déductif).  Une manière d'établir une déduction est d'utiliser la
relation de conséquence logique que nous avons vue au chapitre
précédent.  On la notait par $\satisf$, et ce symbole ne fait pas
partie de {\LO}, c'est un méta-symbole.  Il sert à \emph{affirmer}
qu'il y a une conséquence logique entre des phrases, et nous avons bien vu
que cette relation ne s'établit pas n'importe comment (nous y
reviendrons en \S\ref{s:deflog}).  Au contraire,
la règle syntaxique (\RSyn\ref{SynPConn}) nous autorise à placer le
connecteur $\Xlo\implq$ entre n'importe quelles formules, pourvu qu'elles
soient bien formées.  C'est que $\Xlo[\phi \implq \psi]$ en soi n'affirme
rien, quand on l'écrit on ne dit pas si elle est vraie ou
fausse (d'où l'utilité des modèles et de \denote{\,}).  Ainsi on a
tout à fait le droit d'écrire $\Xlo[\phi \implq \neg\phi]$ puisque c'est une
\acro{ebf}, alors qu'écrire $P \satisf \neg P$ c'est commettre une
flagrante erreur de logique.  Car ces deux écritures ne se situent pas
sur le même plan : la première fait partie de {\LO}, le langage objet
qu'observent et étudient les sémanticiens, alors que la seconde appartient
au métalangage, dans lequel s'expriment les sémanticiens.

\newpage

Ensuite, on doit remarquer   que le faux n'implique pas que le vrai,
il implique aussi le faux (cf. la table de vérité).  Autrement dit le
faux implique n'importe quoi%
\footnote{\label{efsq}C'est même une loi logique fameuse et depuis longtemps
  identifiée sous le joli nom de \alien{e falso sequitur quodlibet}
  (du faux s'ensuit n'importe quoi, ou le faux implique tout).}
: lorsque l'antécédent d'une implication
est faux, la valeur de son conséquent n'est pas discriminante.
Ce qui est surtout déterminant, ce sont les deux premières lignes de
  la table de vérité.
Cela peut s'illustrer avec l'exemple suivant :

\ex. \label{x:si1}
Si je pars de chez moi après 8h, je rate mon train.


C'est une phrase que je peux énoncer très raisonnablement si je dois
prendre un train qui part à 8h20 et que je sais pertinemment qu'il me
faut au minimum 20 minutes pour aller à la gare\footnote{Pour les
besoins de la démonstration, nous faisons aussi l'hypothèse extravagante
que les trains partent toujours à l'heure.}.  Maintenant imaginons
que finalement je suis parti à 7h40 et que malgré tout, à cause d'une
panne de métro ou d'un embouteillage, j'ai quand-même raté mon train
(c'est-à-dire on imagine un modèle où les choses se sont passées
ainsi).  Ainsi l'antécédent de \ref{x:si1} est faux et son
conséquent est vrai.  Devons-nous alors en conclure que dans ce cas
\ref{x:si1} est fausse ?  Pas du tout, elle reste vraie, simplement
parce qu'elle ne se prononce pas sur ce qui se passe dans le cas où je
pars avant 8h (\ie\ lorsque l'antécédent est faux).

Il faut tout de même mentionner un effet de sens qui se produit assez
souvent avec des phrases conditionnelles, et qui là encore est d'ordre
pragmatique. Prenons l'exemple de parents qui disent à leurs enfants :

\ex. \label{x:si2}
Si vous êtes sages, on ira au parc d'attraction cet après-midi.


Cet énoncé est une sorte de promesse.
Et comme précédemment,  plaçons-nous dans une situation où
l'antécédent est faux et le conséquent vrai : les enfants n'ont pas été
sages du tout et les parents les ont emmenés au parc.  Dans ce cas, on
ne peut pas vraiment dire qu'ils n'ont pas tenu leur promesse,
techniquement ils ne se sont pas parjurés (ça aurait été le cas si les
enfants avaient été sages mais n'avaient pas été emmenés au parc).  En
revanche, on peut estimer que les parents sapent ainsi dangereusement
leur autorité et leur crédibilité.  La raison en est que, comme l'a
montré, entre autres, \citet{Ducrot:84}, \Andex{Ducrot, O.} une
affirmation de la forme de 
\ref{x:si2} s'accompagne habituellement d'un \emph{sous-entendu} (et
donc probablement d'une implicature conversationnelle) de la forme de
\ref{x:si3} : 

\ex. \label{x:si3}
Si vous n'êtes pas sages, on n'ira pas au parc d'attraction cet après-midi.


\largerpage[2]
C'est pourquoi une phrase comme \ref{x:si2} se comprend souvent
comme exprimant une équivalence matérielle ($\Xlo\ssi$) plutôt qu'une
implication, car l'affirmation \ref{x:si2} ($\Xlo[\phi \implq \psi]$)
complétée du sous-entendu \ref{x:si3} ($\Xlo[\neg\phi \implq \neg\psi]$)
revient à \sicut{on ira au parc d'attraction, si et seulement si vous êtes
sages}.  Cet effet d'équivalence, comme l'effet d'exclusivité
pour la disjonction, est le fruit d'un raisonnement pragmatique%
\footnote{Très informellement, ce raisonnement peut se résumer ainsi :
  si les parents énoncent \ref{x:si2} tout en envisageant la
  possibilité d'emmener les enfants au parc quoi qu'il arrive, alors
  ça ne sert à rien de mettre une condition à la promesse, car dans ce
  cas, il leur   suffirait de dire simplement : \sicut{on ira au parc cet
  après-midi}.}, mais 
il n'est pas inscrit dans la structure sémantique de la phrase de
départ \ref{x:si2}.

Pour conclure sur l'implication matérielle, disons qu'il faut toujours
penser à évaluer \emph{globalement} la dénotation d'une phrase
conditionnelle.  Une implication ou une conditionnelle met en place une
hypothèse exprimée par l'antécédent qui, seulement si elle est vraie,
nous invite à examiner (\ie\ à vérifier) une conséquence, c'est-à-dire
le conséquent.  Si l'hypothèse est fausse, alors globalement
l'implication n'a rien à signaler et donc elle est --~trivialement~-- 
vraie.  L'hypothèse, donc l'antécédent, a le statut de \kwo{condition
  suffisante} dans l'implication.  C'est pourquoi on peut s'aider à
bien interpréter une phrase conditionnelle en prononçant l'implication
par \sicut{il suffit que}%
\footnote{C'est-à-dire que $\Xlo[\phi \implq \psi]$ peut se prononcer en
 \sicut{si $\Xlo\phi$, $\Xlo\psi$} ou \sicut{il suffit que $\Xlo\phi$
 pour que $\Xlo\psi$}.}
 (mais surtout pas \sicut{il faut
  que} !), comme par exemple {\Next} %\ref{x:si2}$'$ 
qui est une bonne
variante logique de \ref{x:si2}.

\ex. %[\ref{x:si2}$'$]
{
Il suffit que vous soyez sages pour que nous allions au parc d'attraction cet après-midi.
}


\indexs{connecteur!\elid\ logique|)}

\subsection{La quantification}
%''''''''''''''''''''''''''''''''
\is{quantification|(}
% Sur les quantificateurs et leur affinité avec les connecteurs (le
% paradoxe de Donald Duck).\indexs{paradoxe de Donald Duck}

Il est temps de revenir aux formules quantifiées de {\LO} et à leur
sémantique.  Pour cela, il nous faut examiner quelques propriétés
formelles et quelques notions attachées aux formules qui comportent
des symboles de quantification.  La plus fondamentale est celle de
\kwa{portée}{portee}. 


\begin{defi}[Portée d'un quantificateur]\label{d:portee}
Si une formule $\Xlo\phi$ contient une sous-formule de la forme $\Xlo\exists v
\psi$ ou $\Xlo\forall v \psi$, on dit que $\Xlo\psi$ est la \kwa{portée}{portee}
respectivement du quantificateur $\Xlo\exists v$ ou  $\Xlo\forall v$ dans
$\Xlo\phi$. 
\end{defi}

Regardons tout de suite un exemple (volontairement compliqué, et
laissons de côté ce que la formule peut bien signifier) :

\ex. \label{x:CPEBF}
\(\Xlo\neg \exists x \exists y [\forall z [\exists w\, \prd{aimer}(z,w)
    \implq \prd{aimer}(y,z)] \wedge \prd{aimer}(x,y)]\)


\sloppy

En \ref{x:CPEBF}, la portée de $\Xlo\exists w$ est
$\Xlo\prd{aimer}(z,w)$, celle de $\Xlo\forall z$ est $\Xlo[\exists w\,
  \prd{aimer}(z,w)\implq \prd{aimer}(y,z)]$, celle de $\Xlo\exists y$ est
$\Xlo[\forall z [\exists w\, \prd{aimer}(z,w) \implq \prd{aimer}(y,z)]
  \wedge \prd{aimer}(x,y)]$ et celle de $\Xlo\exists x$ est $\Xlo\exists y
[\forall z [\exists w\, \prd{aimer}(z,w) \implq \prd{aimer}(y,z)]
  \wedge \prd{aimer}(x,y)]$.

\fussy

La portée d'un quantificateur $\Xlo\exists v$ ou $\Xlo\forall v$ est
simplement la sous-formule (complète !) qui le suit immédiatement dans
la formule globale, ou, si l'on préfère, la sous-formule qui a
été utilisée en appliquant la règle (\RSyn\ref{SynPQ}) au moment
d'introduire $\Xlo\exists v$ ou $\Xlo\forall v$ lors de la construction de la
formule globale. 

\largerpage

Remarquez qu'ici on appelle \emph{quantificateur} une séquence
composée d'un symbole de quantification suivi d'une variable.  À ce
propos, la formulation de la définition~\ref{d:portee} est un peu
simplifiée ; pour être tout à fait précis il faut en fait l'énoncer en
disant : «...~on dit que $\Xlo\psi$ est la {portée} \emph{de cette
occurrence particulière} du quantificateur $\Xlo\exists v$ ou $\Xlo\forall v$,
respectivement, dans $\Xlo\phi$».  En effet, rien n'empêche d'avoir
plusieurs fois par exemple $\Xlo\exists x$ dans une formule, et ce qui
nous intéresse ici c'est le rôle du quantificateur à un certain endroit
de la formule.  Ainsi dans \ref{x:portées39},
$\Xlo\prd{mari-de}(x,\cnsi{t}{2})$ est la portée de la première occurrence
de $\Xlo\exists x$ et $\Xlo[\prd{aimer}(\cnsi{t}{2},x) \wedge
  \neg\prd{mari-de}(x,\cnsi{t}{2})]$ la portée de sa seconde occurrence.

\ex.
\(\Xlo\exists x\, \prd{mari-de}(x,\cnsi{t}{2}) \wedge \exists x
	  [\prd{aimer}(\cnsi{t}{2},x) \wedge
	    \neg\prd{mari-de}(x,\cnsi{t}{2})]\) \label{x:portées39}



\sloppy

La quantification, par nature, concerne les variables.
Sémantiquement la portée d'un quantificateur  c'est, en quelque sorte, son
rayon d'action : elle délimite la zone où se trouvent les
variables qui sont 
concernées par ce quantificateur dans une formule. 
«Concerner» n'est pas un terme retenu en sémantique formelle ; on
parle plutôt des \kwo{variables liées}\indexs{variable!liee@{\elid} liée} par un
quantificateur.  Et lorsqu'une variable n'est pas liée dans une
formule, on dit qu'elle y est \kwo{libre}\indexs{variable!{\elid} libre}.

\fussy

\begin{defi}[Variables libres, variables liées]\label{d:vlibr}
\begin{enumerate}
\item L'occurrence d'une variable $\Xlo v$ dans une formule $\Xlo\phi$ est dite
\kwo{libre}\indexs{variable!{\elid} libre} dans $\Xlo\phi$ si elle n'est dans la portée d'aucun
quantificateur $\Xlo\exists v$ ou $\Xlo\forall v$.
\item
Si $\Xlo\exists v \psi$ (ou $\Xlo\forall v \psi$) est une sous-formule
de $\Xlo\phi$ et si $\Xlo v$ est libre dans $\Xlo\psi$, alors cette occurrence de
$\Xlo v$ est dite \kwo{liée}\indexs{variable!liee@{\elid} liée} par le
quantificateur $\Xlo\exists v$ (ou $\Xlo\forall v$).
\end{enumerate}
\end{defi}

Là aussi, \emph{libre} et \emph{liée} sont des propriétés
d'\emph{occurrences} de variables, c'est-à-dire de variables situées à
un certain endroit dans une formule\footnote{Lorsque l'on parle d'occurrences de variables (libres ou liées), on ne prend pas en compte les variables accolées à un symbole quantification (comme dans $\Xlo\exists x$ ou $\Xlo\forall x$) ; car celles-ci font partie du quantificateur.}.  Regardons par exemple les
variables de \ref{x:varll} :


\ex. \label{x:varll}
\(\Xlo\forall x [\prd{aimer}(x,y) \wedge \exists y\, \prd{elfe}(y)]\)



Et examinons les choses pas à pas.  Localement dans
$\Xlo\prd{elfe}(y)$, $\vrb y$ est libre puisque dans cette
sous-formule il n'y a pas de quantificateur.  Donc cette occurrence de
$\vrb y$ est liée par $\Xlo\exists y$ dans $\Xlo\exists y\,
\prd{elfe}(y)$ en vertu de la définition~\ref{d:vlibr}--2.  De même
dans $\Xlo[\prd{aimer}(x,y) \wedge \exists y\, \prd{elfe}(y)]$, $\vrb
x$ et la première occurrence de $\vrb y$ sont libres (et la seconde
occurrence de $\vrb y$ est liée, comme on vient de le voir).  Donc
$\vrb x$ est liée par $\Xlo\forall x$ dans \ref{x:varll}.  Par contre,
la première occurrence de $\vrb y$, elle, reste libre car elle est
dans la portée d'un $\Xlo\forall x$ mais pas dans celle d'un
$\Xlo\forall y$ ou d'un $\Xlo\exists y$.

On pourrait se demander pourquoi la définition~\ref{d:vlibr} est si
compliquée et pourquoi faut-il définir la notion de variable liée à
partir de celle de variable libre.  Ne suffirait-il pas de dire
simplement qu'une variable $\vrb v$ est liée par $\Xlo\exists v$ ou $\Xlo\forall v$
si elle se trouve dans sa portée ?  Eh non, cela ne suffirait pas, car
des quantificateurs sur $\vrb v$ peuvent se trouver eux-mêmes dans la
portée d'un autre quantificateur sur $\vrb v$.  Par exemple :




\ex. 
\(\Xlo\forall x [\prd{aimer}(x,y) \wedge \exists x\, \prd{elfe}(x)]\)



\newpage

Ici $\vrb y$ est libre et  le
premier $\vrb x$ est lié par $\Xlo\forall x$, comme en \ref{x:varll}.  Mais le
second $\vrb x$, lui,  n'est pas lié par 
$\Xlo\forall x$, même s'il est dans sa portée. Car il n'est pas libre dans
$\Xlo[\prd{aimer}(x,y) \wedge \exists x\, \prd{elfe}(x)]$, il est lié
par $\Xlo\exists x$. 

\sloppy

Le principe de la définition~\ref{d:vlibr}, en fait, est qu'une variable
$\vrb v$ libre dans une \mbox{(sous-)formule} $\Xlo\phi$ est toujours
susceptible d'être ensuite liée par un $\Xlo\exists v$ ou $\Xlo\forall v$ qui
serait placé devant $\Xlo\phi$.  
De par cette définition, on appelle d'ailleurs les symboles de
quantification des \kwi{lieurs}{lieur}.  C'est une notion
fondamentalement sémantique, mais il est aussi très simple de
la définir syntaxiquement.  Nous rencontrerons plus tard d'autres
lieurs que $\Xlo\exists$ et $\Xlo\forall$ ; c'est pourquoi la
définition suivante utilise le méta-symbole $\Xlo\ell$, pour une
formulation générique. 

\fussy

\begin{defi}[Lieur]
Un symbole $\Xlo\ell$ de {\LO} est appelé un \kw{lieur} s'il est
introduit dans le langage par une règle syntaxique qui dit que si $\Xlo v$
est une variable et $\Xlo\alpha$ une expression bien formée de {\LO},
alors $\Xlo\ell v\alpha$ est aussi une expression bien formée de {\LO}\footnotemark.
\end{defi}%
\footnotetext{On parle ici d'expressions bien formées et pas
  simplement de  formules, afin d'avoir une définition suffisamment
  générale (et donc valide aussi pour les autres lieurs que nous verrons).}

Tout lieur a une portée (c'est $\Xlo\alpha$ ci-dessus), et il suffit
d'adapter la définition~\ref{d:vlibr} en la généralisant à tout lieur
$\Xlo\ell$ pour savoir que, par définition, un lieur
est simplement un symbole qui lie des variables.

Il est très important de savoir
reconnaître les variables liées, notamment pour l'interprétation de la
quantification puisqu'évidemment, les quantificateurs quantifient sur
les variables qu'ils lient.

% -*- coding: utf-8 -*-
\begin{exo}\label{exo:vlibr}
Pour chacune des formules suivantes, dites :
\pagesolution{crg:vlibr}%
$i$) quelle est la portée de chaque quantificateur, 
$ii$) quelles sont les occurrences de variables libres (s'il y en a),
et $iii$) et par quels quantificateurs sont liées les autres variables.

 \begin{enumerate}
 \item \(\Xlo\exists x [\prd{aimer}(x,y) \wedge \prd{âne}(x)]\)
 \item \(\Xlo\exists x\, \prd{aimer}(x,y) \wedge \prd{âne}(x)\)
 \item \(\Xlo\exists x \exists y\,  \prd{aimer}(x,y) \implq \prd{âne}(x)\)
 \item \(\Xlo\forall x [\exists y \, \prd{aimer}(x,y) \implq \prd{âne}(x)]\)
 \item \(\Xlo\neg\exists x \exists y\,  \prd{aimer}(x,y) \implq \prd{âne}(x)\)
\item \(\Xlo\neg \prd{âne}(x) \implq [\neg \forall y [\neg \prd{aimer}(x,y) \vee \prd{âne}(x)] \implq
\prd{elfe}(y)]\)
\item \(\Xlo\neg \exists x [\prd{aimer}(x,y) \vee \prd{âne}(y)]\)
\item \(\Xlo\neg \exists x\, \prd{aimer}(x,x) \vee \exists y\, \prd{âne}(y)\)
\item \(\Xlo\forall x \forall y [[\prd{aimer}(x,y) \wedge \prd{âne}(y)] \implq \exists z\, \prd{mari-de}(x,z)]\)
\item \(\Xlo\forall x [\forall y\, \prd{aimer}(y,x) \implq \prd{âne}(y)]\)
 \end{enumerate}
\begin{solu} 
(p.~\pageref{exo:vlibr})\label{crg:vlibr} 

Nous encadrons chaque quantificateur et sa portée.  Les
variables libres sont notées en gras ($\vfree{x}$); les autres sont liées par le quantificateur indiqué par une flèche. L'exercice exploite les définitions \ref{d:portee} (p.~\pageref{d:portee}) et \ref{d:vlibr} (p.~\pageref{d:vlibr}).

\medskip

\newpsstyle{liage}{nodesepA=1pt,nodesepB=2pt,arrows=<-,linecolor=gray,angle=90,armA=1.2ex,armB=1ex}
 \begin{enumerate}[itemsep=2.5ex]
 \item \(\Xlo\fscp{\rnode{Q1}{\exists x} [\prd{aimer}(\rnode{x1}{x},\vfree{y}) \wedge \prd{âne}(\rnode{x2}{x})]}\)%
\ncbar[style=liage,offsetA=-2pt]{Q1}{x1}%
\ncbar[style=liage,offsetA=2pt,armA=1.8ex]{Q1}{x2}%
\hfill formule existentielle

 \item \(\Xlo\fscp{\rnode{Q1}{\exists x}\, \prd{aimer}(\rnode{x1}{x},\vfree{y})} \Xlo\wedge \prd{âne}(\vfree{x})\)
\ncbar[style=liage]{Q1}{x1}%
\hfill conjonction

 \item \(\Xlo\fscp{\rnode{Q1}{\exists x} \fscp{\rnode{Q2}{\exists y}\,  \prd{aimer}(\rnode{x1}{x},\rnode{y1}{y})}} \implq \prd{âne}(\vfree{x})\)
\ncbar[style=liage,armA=2ex]{Q1}{x1}%
\ncbar[style=liage,armA=1.5ex]{Q2}{y1}%
\hfill implication

 \item \(\Xlo\fscp{\rnode{Q1}{\exists x} [\fscp{\rnode{Q2}{\exists y} \, \prd{aimer}(\rnode{x1}{x},\rnode{y1}{y})} \implq \prd{âne}(\rnode{x2}{x})]}\)
\ncbar[style=liage,armA=2ex,offsetA=-2pt]{Q1}{x1}%
\ncbar[style=liage,armA=1.5ex]{Q2}{y1}%
\ncbar[style=liage,armA=2.5ex,offsetA=2pt]{Q1}{x2}%
\hfill formule existentielle

 \item \(\Xlo\neg\,\fscp{\rnode{Q1}{\exists x} \fscp{\rnode{Q2}{\exists y}\,  \prd{aimer}(\rnode{x1}{x},\rnode{y1}{y})}} \implq \prd{âne}(\vfree{x})\)
\ncbar[style=liage,armA=2ex]{Q1}{x1}%
\ncbar[style=liage,armA=1.5ex]{Q2}{y1}%
\hfill implication

\item \(\Xlo\neg \prd{âne}(\vfree{x}) \implq [\neg \fscp{\rnode{Q1}{\forall y} [\neg \prd{aimer}(\vfree{x},\rnode{y1}{y}) \vee \prd{âne}(\vfree{x})]} \implq
\prd{elfe}(\vfree{y})]\)
\ncbar[style=liage,armA=1.3ex]{Q1}{y1}%
\hfill implication

\item \(\Xlo\neg\, \fscp{\rnode{Q1}{\exists x} (\prd{aimer}(\rnode{x1}{x},\vfree{y}) \vee \prd{âne}(\vfree{y}))}\)
\ncbar[style=liage,armA=1.3ex]{Q1}{x1}%
\hfill négation

\item \(\Xlo\neg\, \fscp{\rnode{Q1}{\exists x}\, \prd{aimer}(\rnode{x1}{x},\rnode{x2}{x})} \vee \fscp{\rnode{Q2}{\exists y}\, \prd{âne}(\rnode{y1}{y})}\)
\ncbar[style=liage,offsetA=-2pt]{Q1}{x1}%
\ncbar[style=liage,armA=1.8ex,offsetA=2pt]{Q1}{x2}%
\ncbar[style=liage]{Q2}{y1}%
\hfill disjonction

\item \(\Xlo\fscp{\rnode{Q1}{\forall x} \fscp{\rnode{Q2}{\forall y} [[\prd{aimer}(\rnode{x1}{x},\rnode{y1}{y}) \wedge \prd{âne}(\rnode{y2}{y})] \implq \fscp{\rnode{Q3}{\exists z}\, \prd{mari-de}(\rnode{x2}{x},\rnode{z1}{z})}]}}\)
\ncbar[style=liage,offsetA=-2pt,armA=2.5ex]{Q1}{x1}%
\ncbar[style=liage,angle=-90,armA=1.8ex]{Q2}{y1}%
\ncbar[style=liage,armA=2ex]{Q2}{y2}%
\ncbar[style=liage,offsetA=2pt,armA=3ex]{Q1}{x2}%
\ncbar[style=liage,armA=2ex]{Q3}{z1}%
\hfill formule universelle

\item \(\Xlo\fscp{\rnode{Q1}{\forall x} [\fscp{\rnode{Q2}{\forall y}\, \prd{aimer}(\rnode{y1}{y},\rnode{x1}{x})} \implq \prd{âne}(\vfree{y})]}\)
\ncbar[style=liage,armA=1.5ex]{Q2}{y1}%
\ncbar[style=liage,armA=2ex]{Q1}{x1}%
\hfill formule universelle
 \end{enumerate}

\end{solu}
\end{exo}



\sloppy
Maintenant intéressons-nous aux conditions de vérité des formules (et
des phrases) qui contiennent des quantificateurs.  Quand ces formules
sont simples, leurs conditions de vérité sont assez faciles à
caractériser, et elles nous donnent le principe général
d'interprétation de la quantification.  Par exemple, la formule
\(\Xlo\exists x\, \prd{âne}(x)\) est vraie dans un modèle $\Modele =
\tuple{\Unv{A}, \FI}$ ssi il y a un individu, au moins, de \Unv{A} qui
appartient à $\FI(\prd{âne})$, l'ensemble des ânes de \Modele.  Et
$\Xlo\forall x \, \prd{âne}(x)$ est vraie dans {\Modele} ssi tous les
individus de \Unv{A} appartiennent à $\FI(\prd{âne})$ (tout le monde
est un âne).  Et \emph{très} informellement, on généralisera en disant
que $\denote{\Xlo\exists x \phi}^{\Modele}=1$ ssi la formule $\Xlo\phi$ est
vraie pour au moins un individu de \Unv{A} et $\denote{\Xlo\forall x
\phi}^{\Modele}=1$ ssi $\Xlo\phi$ est vraie pour tout individu de \Unv{A}.
Ainsi comme avec les autres règles sémantiques de {\LO}
l'interprétation se fait par simplification progressive de la formule :
on se débarrasse du quantificateur et on examine la dénotation de
$\Xlo\phi$ --~une fois avec $\Xlo\exists x$ et autant de fois que nécessaire
avec $\Xlo\forall x$.   

\fussy

Mais il nous faut ici être précis sur ce que cela signifie lorsqu'on
dit qu'une formule $\vrb\phi$ est \emph{vraie pour tel ou tel individu} de
\Unv{A}.  D'abord si le quantificateur lie la variable $\vrb x$, on doit
regarder les individus qui \emph{en tant que $\vrb x$} rendent vraie la
formule.  Cela veut dire simplement que les individus à tester doivent
en quelque sorte prendre la place de la variable dans la formule
$\vrb \phi$ pour qu'ensuite on vérifie sa dénotation.  Mais les individus
appartiennent au modèle, pas à {\LO}, ils ne peuvent pas intervenir
eux-mêmes dans les formules.  C'est pourquoi une manière de procéder
consiste à utiliser les constantes comme des représentants des individus dans
{\LO}.  Le mécanisme interprétatif de la quantification peut alors
s'exprimer facilement pour toute formule de {\LO}, il suffit de dire
que $\denote{\Xlo\exists x \phi}^{\Modele}=1$ ssi il y a au moins une
constante d'individu telle que si on remplace $\vrb x$ par cette constante
dans $\Xlo\phi$, $\Xlo\phi$ devient alors vraie dans {\Modele}, et
$\denote{\Xlo\forall x \phi}^{\Modele}=1$ ssi quand on remplace $\vrb x$
successivement par toutes les constantes d'individus,  $\Xlo\phi$ est
alors à chaque fois vraie dans {\Modele}.


Cette façon d'interpréter les formules quantifiées n'est pas complètement
satisfaisante sur le plan théorique, car elle se débarrasse des
variables (en les remplaçant par des constantes) simplement parce que
nous ne savons pas ce qu'est $\denote{\vrb x}^{\Modele}$ (effectivement
nous n'avons jamais défini la dénotation d'une variable dans les
règles des définitions~\ref{RIcl} et \ref{RI1}).  C'est pourquoi nous
verrons au chapitre suivant une autre façon, plus rigoureuse et
régulière, d'interpréter la quantification.  Cependant, sur le plan
pratique, la méthode présentée ici est tout à fait opérationnelle,
\emph{à condition} de poser une contrainte particulière sur {\LO} qui
est qu'à chaque individu du domaine \Unv{A} d'un modèle est associée au
moins une constante de {\LO}.  C'est le cas par exemple dans le
modèle-jouet $\Modele_1$ (p.~\pageref{Modele1}), mais a priori rien
n'oblige à ce que cela soit toujours ainsi ; ce serait même peu
réaliste tant qu'on assimile les constantes aux noms propres.  Mais
cette contrainte est nécessaire ici puisque les constantes sont
censées jouer le rôle des individus.  Disons qu'elle ajoute une
propriété formelle au système de {\LO} sans pour autant avoir un
impact sur l'interprétation sémantique de la langue naturelle : nous
considérerons que chaque nom propre de la langue se traduit par une
constante de {\LO}, mais que toute constante ne traduit pas forcément
un nom propre.

\sloppy

Pour formaliser proprement cette méthode d'interprétation  par
substitution de constantes, nous avons simplement besoin d'introduire
explicitement la procédure de remplacement des variables par 
des constantes.  C'est une opération générale sur la structure des
formules, que nous noterons comme suit :

\fussy

\begin{nota}[Substitution]
Soit $\Xlo\phi$  une formule 
de {\LO}, $\vrb v$ une variable et $\vrb t$ un terme.  On note $[\vrb t/\vrb v]\vrb\phi$ le
résultat de la \kw{substitution} dans $\Xlo\phi$ de \emph{toutes les
  occurrences libres} de $\vrb v$ par $\vrb t$. 
\end{nota}

La substitution ne doit concerner que les occurrences libres de la
variable, car elle sera déclenchée par un quantificateur ; et les
occurrences libres de $\vrb x$ dans $\Xlo\phi$ sont bien celles qui sont liées
par $\Xlo\exists x$ dans $\Xlo\exists x \phi$.  \ref{x:substi} nous donne un exemple
de l'opération\footnote{L'opérateur de remplacement noté $[\cns b/\vrb
    x]$
  ne fait pas 
  partie de {\LO}, c'est juste un moyen notationnel qui fait passer d'une
  formule à une autre.} :

\ex. \label{x:substi}
\([\cns b/\vrb x] \Xlo[\prd{âne}(x) \wedge \forall y [\prd{aimer}(x,y) \vee
    \exists x\, \prd{mari-de}(y,x)]]\)  
\\=\Xlo
\([\prd{âne}(\cns b) \wedge \forall y [\prd{aimer}(\cns b,y) \vee
    \exists x\, \prd{marie-de}(y,x)]]\) 


À présent nous pouvons formuler les règles d'interprétation
systématiques des formules quantifiées :

\begin{defi}[Interprétation des formules quantifiées]\label{d:Sem1Q}
\begin{enumerate}[sem,resume*=RglSem1] %[(\RSem1)]{\setcounter{enumi}{\value{RglSem}}}
\item\label{RIQ}
\begin{enumerate}
\item $\denote{\Xlo\exists v \phi}^{\Modele}=1$ ssi on trouve (au moins)
  une constante $\kappa$ telle que $\denote{[\kappa/\vrb v]\vrb\phi}^{\Modele}=1$
\item $\denote{\Xlo\forall v \phi}^{\Modele}=1$ ssi pour toute constante
  $\kappa$, $\denote{[\kappa/\vrb v]\vrb\phi}^{\Modele}=1$ 
\end{enumerate}
%\setcounter{RglSem}{\value{enumi}}
\end{enumerate}
\end{defi}

En somme, ces règles transposent sur des constantes les quantifications
qui sont indiquées sur des variables dans les formules ; et comme
chaque individu est représenté par une constante, cela revient bien à
effectuer les quantifications sur les individus du modèle.

Illustrons l'application de ces règles en calculant la dénotation des
formules suivantes par rapport à $\Modele_1$ (p.~\pageref{Modele1}).

\ex. \label{x:QE1}
\(\Xlo\exists x [\prd{elfe}(x) \wedge \prd{farceur}(x)]\)

\newcommand{\mkappa}{\ensuremath{\textcolor{black}{\kappa}}}

La règle (\RSem\ref{RIQ}a) nous dit que \(\denote{\Xlo\exists x
  [\prd{elfe}(x) \wedge \prd{farceur}(x)]}^{\Modele_1}=1\) ssi il
  existe une constante $\mkappa$ telle que \(\denote{\Xlo\prd{elfe}(\mkappa)
  \wedge \prd{farceur}(\mkappa)}^{\Modele_1}=1\).  Dans cette écriture,
  $\mkappa$ n'est qu'une constante virtuelle, et pour montrer que
  \ref{x:QE1} est vraie dans $\Modele_1$, il suffit donc de trouver
  une (véritable) constante qui fonctionne en tant que $\mkappa$.  En
  examinant $\Modele_1$ nous voyons que nous pouvons prendre par exemple la
  constante \cns o (et poser ainsi $\mkappa=\cns o$).  En effet
  \(\denote{\Xlo\prd{elfe}(\cns o) \wedge \prd{farceur}(\cns
  o)}^{\Modele_1}=1\), car \(\denote{\Xlo\prd{elfe}(\cns
  o)}^{\Modele_1}=1\) et \(\denote{\Xlo\prd{farceur}(\cns
  o)}^{\Modele_1}=1\), car \(\FI_1(\cns o)= \Obj{Obéron} \in
  \FI_1(\prd{elfe})\) et \(\Obj{Obéron} \in \FI_1(\prd{farceur})\) (nous
  appliquons ici les règles (\RSem\ref{RIcon}a) et (\RSem\ref{RIprd}a)).
  Nous avons ainsi montré que \(\denote{\ref{x:QE1}}^{\Modele_1}=1\) : il
  existe bien un elfe farceur dans $\Modele_1$.

Le mécanisme d'interprétation est similaire lorsqu'on a plusieurs
quantificateurs existentiels :

\ex. \label{x:QE2}
\(\Xlo\exists x \exists y\, \prd{aimer}(x,y)\)


\sloppy
Toujours selon (\RSem\ref{RIQ}a), \(\denote{\Xlo\exists x \exists y\,
  \prd{aimer}(x,y)}^{\Modele_1}=1\) ssi 
nous trouvons une constante $\mkappa$ telle que \(\denote{\Xlo\exists y\,
  \prd{aimer}(\mkappa,y)}^{\Modele_1}=1\).  Prenons
$\mkappa=\cnsi{t}{1}$ ; nous avons donc maintenant à calculer la
valeur de \(\denote{\Xlo\exists y\, \prd{aimer}(\cnsi{t}{1},y)}^{\Modele_1}\).  Là
encore (\RSem\ref{RIQ}a) nous dit que \(\denote{\Xlo\exists y\,
  \prd{aimer}(\cnsi{t}{1},y)}^{\Modele_1}=1\)  ssi nous trouvons une
constante $\mkappa$ telle que
\(\denote{\Xlo\prd{aimer}(\cnsi{t}{1},\mkappa)}^{\Modele_1}=1\).  Et là
encore, il suffit de trouver une constante qui marche parmi celles
dont nous disposons.  Prenons donc maintenant $\mkappa=\cnsi{h}{1}$. Et
nous avons bien
\(\denote{\Xlo\prd{aimer}(\cnsi{t}{1},\cnsi{h}{1})}^{\Modele_1}=1\), car
$\FI_1(\cnsi{t}{1}) =\Obj{Thésée}$, $\FI_1(\cnsi h1)=\Obj{Hippolyta}$
et $\tuple{\Obj{Thésée},\Obj{Hippolyta}} \in \FI_1(\prd{aimer})$
(cf. p.~\pageref{M1:aimer}).  Nous avons donc trouvé deux constantes
qui font l'affaire et cela prouve que
\(\denote{\ref{x:QE2}}^{\Modele_1}=1\) : il y a quelqu'un qui aime
quelqu'un dans $\Modele_1$.  


Bien sûr, nous aurions pu mener cette démonstration en
utilisant d'autres constantes, par exemple \cns l et \cnsi h2, ou \cns
d et \cnsi h2, ou \cnsi h1 et \cnsi t1, etc.  Si nous avions choisi \cns
p pour la première constante (\ie\ pour remplacer $\vrb x$), nous n'aurions pas
trouvé de seconde 
constante adéquate pour $\vrb y$, mais cela n'a pas d'importance car pour
qu'une formule existentielle soit vraie, il suffit qu'au moins une
constante la satisfasse ; peu importe celles qui échouent.  L'exercice
consiste donc à bien choisir les constantes qui prouvent la vérité de
la formule. 

\fussy

Évidemment c'est différent lorsqu'il s'agit de prouver qu'une formule
existentielle est fausse ou qu'une formule universelle est vraie.
Commençons par calculer la dénotation dans $\Modele_1$ de
l'universelle \ref{x:QA1} : 

\ex. \label{x:QA1}
\(\Xlo\forall x [\prd{elfe}(x) \implq \prd{farceur}(x)]\)


\sloppy

La règle (\RSem\ref{RIQ}b) dit que \(\denote{\Xlo\forall x [\prd{elfe}(x)
    \implq \prd{farceur}(x)]}^{\Modele_1}=1\) ssi pour \emph{toutes
    les} constantes $\mkappa$, on a \(\denote{\Xlo\prd{elfe}(\mkappa) \implq
    \prd{farceur}(\mkappa)}^{\Modele_1}=1\).  La démonstration ici est
    plus longue : il va falloir effectuer le calcul pour \cnsi t1,
    \cnsi h1, \cnsi h2, \cnsi h3, \cns l, \cns d, \cns e, \cns p, \cns
    o, \cnsi t2 et \cns b (donc 11 calculs !).  Heureusement la table
    de vérité de $\Xlo\implq$ va nous permettre de sauter rapidement des
    étapes.  Souvenons-nous que lorsque l'antécédent d'une implication
    est faux, alors l'implication entière est vraie, quelle que soit
    la valeur du conséquent.  Or dans les cas où $\mkappa$ est \cnsi
    t1, \cnsi h1, \cnsi h2, \cnsi h3, \cns l, \cns d, \cns e ou \cns
    b, on sait que \(\denote{\Xlo\prd{elfe}(\mkappa)}^{\Modele_1}=0\) car
    aucun des individus dénotés par ces constantes n'est dans
    $\FI_1(\prd{elfe})$.  Donc pour ces huit constantes, on sait tout
    de suite que \(\denote{\Xlo\prd{elfe}(\mkappa) \implq
    \prd{farceur}(\mkappa)}^{\Modele_1}=1\).  Ce qu'il reste à
    vérifier, et ce qui est déterminant pour la formule, ce sont les
    cas où $\mkappa$ est \cns p ou \cns o ou \cnsi t2 (constantes pour
    lesquelles \(\denote{\Xlo\prd{elfe}(\mkappa)}^{\Modele_1}=1\)).  Et
    c'est bien normal puisque \ref{x:QA1} traduit la phrase
    \sicut{tous les elfes sont farceurs} --~phrase qui ne
    s'intéresse qu'aux individus qui sont des elfes.  Commençons par
    \cns p ; \(\denote{\Xlo\prd{farceur}(\cns p)}^{\Modele_1}=1\) car
    \(\Obj{Puck} \in \FI_1(\prd{farceur})\), et donc
    \(\denote{\Xlo\prd{elfe}(\cns p) \implq \prd{farceur}(\cns
    p)}^{\Modele_1}=1\).  La même démonstration vaut pour \cns o, car
    \(\Obj{Obéron} \in \FI_1(\prd{farceur})\), et pour \cnsi t2, car
    \(\Obj{Titania} \in \FI_1(\prd{farceur})\).  Ainsi
    \(\denote{\Xlo\prd{elfe}(\cns o) \implq \prd{farceur}(\cns
    o)}^{\Modele_1}=1\) et \(\denote{\Xlo\prd{elfe}(\cnsi t2) \implq
    \prd{farceur}(\cnsi t2)}^{\Modele_1}=1\).  Donc
    \(\denote{\Xlo\prd{elfe}(\mkappa) \implq
    \prd{farceur}(\mkappa)}^{\Modele_1}=1\) pour toute constante
    $\mkappa$, ce qui prouve bien que
    \(\denote{\ref{x:QA1}}^{\Modele_1}=1\). 

\fussy

Pour résumer la méthode d'interprétation des formules quantifiées, on
peut considérer l'algorithme extrêmement minutieux suivant : 
\begin{itemize}
\item  pour calculer \(\denote{\Xlo\exists x \phi}^{\Modele}\), on passe
  en revue chaque constante $\mkappa$ pour calculer chaque \(\denote{[\mkappa/\vrb x]
  \Xlo\phi}^{\Modele}\) et on s'arrête dès qu'on trouve le résultat $1$ ;
  dans ce cas cela montre que \(\denote{\Xlo\exists x \phi}^{\Modele}=1\) ;
  en revanche, si pour tous les $\mkappa$ on a trouvé $0$ (\ie\ on a
  jamais trouvé $1$), alors c'est
  que \(\denote{\Xlo\exists x \phi}^{\Modele}=0\) ;
\item  pour calculer \(\denote{\Xlo\forall x \phi}^{\Modele}\), on passe
en revue chaque constante $\mkappa$ pour calculer chaque \(\denote{[\mkappa/\vrb x]
  \Xlo\phi}^{\Modele}\) et si à chaque fois le résultat est $1$, c'est que
\(\denote{\Xlo\forall x \phi}^{\Modele}=1\) ; au contraire dès qu'on trouve
le résultat $0$, on peut s'arrêter, car cela suffit à prouver que \(\denote{\Xlo\forall x \phi}^{\Modele}=0\).
\end{itemize}

Cette procédure nous indique du même coup les «conditions de
fausseté» d'une formule quantifiée, ou si l'on préfère, les
conditions  de vérité de sa négation.  

\ex. \label{x:QE3}
\(\Xlo\exists x [\prd{âne}(x) \wedge \prd{triste}(x)]\)


\(\denote{\ref{x:QE3}}^{\Modele_1}=0\) car \emph{il n'existe pas} de
constante $\mkappa$ telle que \(\denote{\Xlo\prd{âne}(\mkappa) \wedge
  \prd{triste}(\mkappa)}^{\Modele_1}=1\).  Pour le démontrer très
rigoureusement, il faudrait effectuer le calcul pour les onze
constantes et montrer que le résultat est toujours $0$. 

\ex. \label{x:QA2}
\(\Xlo\forall x [\prd{farceur}(x) \implq \prd{elfe}(x)]\)


\sloppy
\(\denote{\ref{x:QA2}}^{\Modele_1}=0\) car \emph{il existe au moins
  une} constante $\mkappa$ telle que \(\oden\Xlo\prd{farceur}(\mkappa)
  \implq \prd{elfe}(\mkappa)\color{black}\fden^{\Modele_1}=0\).  Cette constante est
  \cnsi t1, car  \(\denote{\Xlo\prd{farceur}(\cnsi t1)}^{\Modele_1}=1\) et
  \(\denote{\Xlo\prd{elfe}(\cnsi t1)}^{\Modele_1}=0\).  En effet, si
  quelque chose n'est pas vrai de tous les individus, c'est qu'il
  existe au moins un individu pour lequel c'est faux.



\fussy
Ces exemples illustrent la dualité bien connue qu'entretiennent entre
eux les deux types de quantificateurs : la négation d'une formule
existentielle est une formule universelle, et la négation d'une
formule universelle est une formule existentielle.   
Je ne vais pas donner ici le détail de la démonstration, mais nous
pouvons facilement nous convaincre de ces équivalences en décortiquant
un peu les exemples \ref{x:QE3} et \ref{x:QA2}.  

La négation de \ref{x:QE3} (\ie\ \(\Xlo\neg \exists x [\prd{âne}(x) \wedge
  \prd{triste}(x)]\)) 
signifie qu'il n'existe pas d'individu qui soit à la fois un âne et triste,
autrement dit pour tout individu (ou toute constante $\mkappa$) ou bien
ce n'est pas un âne ou il n'est pas triste, ce qui peut se reformuler
en : pour tout individu, s'il est un âne, alors il n'est pas triste
  (tout âne est non-triste).
Et cela, ce sont bien les conditions de vérité d'une formule
universelle, à savoir : 

\ex. \label{x:QA3}
\(\Xlo\forall x [\neg\prd{âne}(x) \vee \neg\prd{triste}(x)]\)\\
ou (c'est équivalent)\footnote{Cf.\ l'équivalence logique \numero
  \ref{Nex:implq} dans l'exercice \ref{exo:equivlog},
  p.~\pageref{exo:equivlog}.} :\\ 
\(\Xlo\forall x [\prd{âne}(x) \implq \neg\prd{triste}(x)]\)


Remarquons aussi que \ref{x:QA3} (qui équivaut à
$\neg$\ref{x:QE3}) traduit également la phrase \sicut{aucun âne
  n'est triste}.  Une phrase en \sicut{aucun} s'analyse par une
quantification universelle du type \ref{x:QA3} ou, ce qui revient au
même, par la négation d'une existentielle.

Quant à la négation de \ref{x:QA2}, c'est-à-dire \(\Xlo\neg\forall x
[\prd{farceur}(x) \implq \prd{elfe}(x)]\), ses conditions de vérité
disent qu'il y a au moins un individu qui est farceur mais pas un
elfe, ce qui correspond bien à la formule existentielle suivante :

\ex. \label{x:QE4}
\(\Xlo\exists x [\prd{farceur}(x) \wedge  \neg\prd{elfe}(x)]\)


En effet $\neg$\ref{x:QA2}, \sicut{il n'est pas vrai que tous les farceurs sont des elfes}, veut dire la même chose que \sicut{il
  y a au moins un farceur qui n'est pas elfe} \ref{x:QE4}.


Récapitulons cette dualité entre $\Xlo\exists$ et $\Xlo\forall$ par le
théorème suivant :

\begin{theo}
Les quatre paires de formules suivantes sont des équivalences logiques :
\begin{enumerate}
\item \(\Xlo\neg\exists x \phi\) et \(\Xlo\forall x \neg\phi\)
\item \(\Xlo\neg\forall x \phi\) et \(\Xlo\exists x \neg\phi\)
\item \(\Xlo\neg\exists x \neg\phi\) et \(\Xlo\forall x \phi\)
\item \(\Xlo\neg\forall x \neg\phi\) et \(\Xlo\exists x \phi\)
\end{enumerate}
\end{theo}

Les deux dernières équivalences se déduisent directement des deux
premières et de la loi de double négation vue \alien{supra} dans
l'exercice~\ref{exo:equivlog}. 


\begin{exo}\label{exo:2denot2}
Calculez  par rapport à $\Modele_1$ (cf. p.~\pageref{Modele1}) la
\pagesolution{crg:2denot2}
dénotation des formules suivantes :
\begin{exolist}
\item \(\Xlo\forall x [\prd{elfe}(x) \wedge \prd{farceur}(x)]\)\label{Qex:f1}
\item \(\Xlo\forall x [\prd{elfe}(x) \implq \neg\prd{triste}(x)]\)
\item \(\Xlo\neg\exists x [\prd{âne}(x) \wedge \prd{elfe}(x)]\)
\item \(\Xlo\exists x \forall y \, \prd{aimer}(y,x)\)
\item \(\Xlo\forall y \exists x \, \prd{aimer}(y,x)\)
\end{exolist}
Pour chaque formule, proposer une phrase en français qui peut se
traduire par cette formule.
\\
Comparer la formule \numero \ref{Qex:f1} avec la formule \ref{x:QA1}
\alien{supra}. 
%
\begin{solu} (p.~\pageref{exo:2denot2})\label{crg:2denot2}
\begin{exolist}
\item \(\Xlo\forall x [\prd{elfe}(x) \wedge \prd{farceur}(x)]\)

La règle (\RSem\ref{RIQ}b) nous dit que cette formule est vraie ssi pour \emph{toute} constante $\kappa$ du langage, nous trouvons \(\denote{\Xlo\prd{elfe}(\kappa) \wedge \prd{farceur}(\kappa)}^{\Modele_1}=1\).   Mais évidemment cette sous-formule est fausse pour de nombreuses constantes, par exemple \cnsi t1, puisque \Obj{Thésée} n'est pas un elfe.  La formule \(\Xlo\forall x [\prd{elfe}(x) \wedge \prd{farceur}(x)]\) est donc fausse dans $\Modele_1$.  En français, elle correspondra à \sicut{toute chose est un elfe farceur} ou \sicut{tout ce qui existe est un elfe farceur}. 
Elle se distingue donc crucialement de \ref{x:QA1}, \(\Xlo\forall x [\prd{elfe}(x) \implq \prd{farceur}(x)]\), qui elle correspond à \sicut{tous les elfes sont farceurs} et qui est vraie dans $\Modele_1$ (cf.\ p.~\pageref{x:QA1}).


\item \(\Xlo\forall x [\prd{elfe}(x) \implq \neg\prd{triste}(x)]\)

\sloppy

Comme précédemment, la formule sera vraie ssi pour toute constante $\kappa$, on trouve \(\denote{\Xlo\prd{elfe}(\kappa) \implq \neg\prd{triste}(\kappa)}^{\Modele_1}=1\).  Pour toutes les constantes qui dénotent des individus qui ne sont pas des elfes, nous savons déja que cela sera vrai (puisque $\Xlo\prd{elfe}(\kappa)$ sera faux).  Il reste à examiner les constantes \cns o, \cns p et \cnsi t2.  Comme aucun des individus dénotés par ces trois constantes (\Obj{Obéron}, \Obj{Puck} et \Obj{Tita\-nia}) ne sont dans $\FI_1(\prd{triste})$, nous obtiendrons, dans les trois cas, \(\denote{\Xlo\neg\prd{triste}(\kappa)}^{\Modele_1}=1\).  Cela suffit à montrer que \(\Xlo\Xlo\prd{elfe}(\kappa) \implq \neg\prd{triste}(\kappa)\) est toujours vraie et donc que la formule globale est vraie.
Elle correspond en français à \sicut{aucun elfe n'est triste}.

\fussy

\item \(\Xlo\neg\exists x [\prd{âne}(x) \wedge \prd{elfe}(x)]\)

Étant donné $\FI_1(\prd{âne})$ et $\FI_1(\prd{elfe})$, nous constatons qu'il n'y a pas d'individu du modèle qui est à la fois dans les deux ensembles.  Autrement dit, il n'existe pas de constante $\kappa$ telle que \(\Xlo[\prd{âne}(\kappa) \wedge \prd{elfe}(\kappa)]\) soit vraie.  Nous en concluons donc, par la règle (\RSem\ref{RIQ}a), que \(\Xlo\exists x [\prd{âne}(x) \wedge \prd{elfe}(x)]\) est fausse et que \(\Xlo\neg\exists x [\prd{âne}(x) \wedge \prd{elfe}(x)]\) est vraie dans $\Modele_1$.
En français, la formule correspond à \sicut{il est faux qu'il y a un âne qui est un elfe} (ou \sicut{un elfe qui est un âne}), ce qui plus simplement peut se formuler en \sicut{aucun âne n'est un elfe} ou \sicut{aucun elfe n'est un âne}.

\item \(\Xlo\exists x \forall y \, \prd{aimer}(y,x)\)

Cette formule est vraie ssi il existe une constante $\kappa_1$ telle que $\Xlo\forall y \, \prd{aimer}(y,\kappa_1)$ est vraie dans $\Modele_1$.  Et $\Xlo\forall y \, \prd{aimer}(y,\kappa_1)$ est vraie ssi pour toutes les constantes $\kappa_2$ nous trouvons $\Xlo\prd{aimer}(\kappa_2,\kappa_1)$.
Une telle constante $\kappa_1$ devrait donc dénoter un individu qui se retrouvait 11 fois en seconde position d'un couple $\tuple{\Obj x,\Obj y}$ dans la dénotation de \prd{aimer} (puisqu'il y a 11 constantes possibles pour $\kappa_2$).  Mais en regardant $\FI_1(\prd{aimer})$, nous voyons immédiatement qu'il n'y a pas autant de couples dans l'ensemble et donc qu'une telle constante $\kappa_1$ n'existe pas.  Par conséquent la formule est fausse dans $\Modele_1$.  En français elle correspond à \sicut{il y a quelqu'un que tout le monde aime}.

\item \(\Xlo\forall y \exists x \, \prd{aimer}(y,x)\) \sloppy

Cette formule est vraie ssi pour toute constante $\kappa_1$ \(\Xlo\exists x \, \prd{aimer}(\kappa_1,x)\) est vraie dans $\Modele_1$.  Cela fait donc, \emph{en théorie}, 11 calculs à effectuer, et chaque \(\Xlo\exists x \, \prd{aimer}(\kappa_1,x)\) sera vraie si à chaque fois on trouve une constante $\kappa_2$ telle que \(\Xlo\prd{aimer}(\kappa_1,\kappa_2)\) est vraie.  Nous pouvons rapidement montrer que cela ne se produit pas dans $\Modele_1$ en prenant, par exemple, d'abord \cns p pour $\kappa_1$. Dans $\FI_1(\prd{aimer})$ il n'y a pas de couple de la forme \tuple{\Obj{Puck},\dots}, donc nous ne trouverons pas de constante $\kappa_2$ telle que \(\Xlo\prd{aimer}(\cns p,\kappa_2)\) soit vraie.  Ce qui veut dire qu'il y a au moins une constante $\kappa_1$ telle que \(\Xlo\exists x \, \prd{aimer}(\kappa_1,x)\) est fausse, ce qui suffit à montrer que la formule globale est fausse dans $\Modele_1$.  En français elle correspond à \sicut{tout le monde aime quelqu'un} (dans le sens de \sicut{chaque personne a quelqu'un qu'elle aime}).
\end{exolist}

\fussy

\end{solu}
\end{exo}




La sémantique de la quantification (\RSem\ref{RIQ}) nous permet aussi de faire une remarque très importante sur l'interprétation et l'usage des variables liées\indexs{variable!liee@{\elid} liée} dans {\LO}.  Ceci est illustré par la formule {\Next} :

\ex. \(\Xlo\exists x [\prd{elfe}(x)\wedge\prd{mari-de}(x,\cnsi t2)] \wedge \exists x [\prd{âne}(x)\wedge\prd{aimer}(\cnsi t2,x)]\)

Cette formule est vraie par rapport à $\Modele_1$ (je ne détaille pas la démonstration, j'invite les lecteurs à la faire). Son sens peut à peu près se restituer en français par \sicut{Titania est mariée à un elfe et elle aime un âne}. L'elfe et l'âne qui permettent de la vérifier dans $\Modele_1$ ne sont pas le même individu, mais cela ne nous empêche nullement d'utiliser la même variable \vrb x dans les deux sous-formules quantifiées existentiellement. 
Car dans une étape du calcul interprétatif, les diverses occurrences d'une variable liée renvoient toujours au même individu, \emph{mais seulement} au sein de la portée du quantificateur (ou lieur) qui les lie.  C'est comme si, à l'extérieur de cette portée, les compteurs étaient remis à zéro, que la variable «oubliait» la constante qui l'a remplacée précédemment, et ainsi elle peut, par la suite, être réutilisée pour désigner autre chose. Cela est une conséquence directe de (\RSem\ref{RIQ}) où les substitutions de variables par des constantes se fait uniquement dans la portée du quantificateur en cours d'interprétation.  Il s'agit d'un théorème du système.

\begin{theo}[Renommage des variables liées]\label{th:renomvar}
Soit \vrb\phi\ une formule, $\xlo\ell$ un lieur, \vrb x une variable et \vrb y une variable \emph{non libre}\footnotemark\ dans \vrb\phi. 
Alors $\Xlo\ell x\phi$ est logiquement équivalente à $\xlo{\ell y}[\vrb y/ \vrb x]\Xlo\phi$.
\end{theo}%
\footnotetext{Non libre dans \vrb\phi\ signifie soit que \vrb y est liée dans \vrb\phi, soit qu'elle n'apparaît pas du tout dans \vrb\phi. Cette précaution est nécessaire pour ne pas conclure que $\Xlo\exists x\,\prd{aimer}(x,y)$ serait équivalent à $\Xlo\exists y\,\prd{aimer}(y,y)$.}


Ce théorème dit que l'on peut toujours, et sans risque, renommer une variable liée (par une nouvelle variable) au sein de la portée où elle est liée. Ainsi {\Last} est équivalent à, par exemple, \(\Xlo\exists z [\prd{elfe}(z)\wedge\prd{mari-de}(z,\cnsi t2)] \wedge \exists y [\prd{âne}(y)\wedge\prd{aimer}(\cnsi t2,y)]\). 
\indexs{variable!renommage des {\elid}s}
La pratique du renommage des variables nous sera très utile par la suite.
Notons aussi que, pour les mêmes raisons que précédemment, dans une formule comme \(\Xlo\exists x\,\prd{elfe}(x)\wedge\exists y\,\prd{farceur}(y)\), rien n'interdit de la vérifier en choisissant la même constante pour remplacer \vrb x et \vrb y. Les noms des variables liées n'ont globalement pas d'importance.

\smallskip

Pour conclure cette partie sur la quantification, voici en
tableau~\ref{T:vademecum} (ci-contre) un 
petit \alien{vade-mecum} de traduction français--{\LO} de phrases
quantifiées typiques.  $N$ représente un substantif quelconque et $V$
un verbe intransitif (ou éventuellement un groupe verbal simple) et
\prd{n} et  \prd{v} représentent les prédicats qui traduisent
respectivement $N$ et $V$.

\begin{table}[h]
\begin{center}
\begin{tabular}{ll}\lsptoprule
\bfseries Schémas de phrases &\bfseries Schémas de formules de {\LO}\\\midrule
\begin{tabular}{@{}l}
Un $N$ $V$ \\
Des $N$ $V$
\end{tabular}
& \(\Xlo\exists x [\prd n(x) \wedge \prd v(x)]\)\\
\smidrule
\begin{tabular}{@{}l}
Tout/chaque $N$ $V$ \\
Tous les $N$ $V$ \\
Les $N$ $V$
\end{tabular}
& \(\Xlo\forall x [\prd n(x) \implq \prd v(x)]\)\\
\smidrule
\begin{tabular}{@{}l}
Un $N$ ne $V$ pas\\
Des $N$ ne $V$ pas\\
Pas tous les $N$ ne $V$\\
Tout/chaque/tous les $N$ ne $V$ pas
\end{tabular}
& \begin{tabular}{@{}l@{}}
  \(\Xlo\exists x [\prd n(x) \wedge \neg\prd v(x)]\)\\
\(\Xlo\neg\forall x [\prd n(x) \implq \prd v(x)]\)
  \end{tabular}
\\
\smidrule
\begin{tabular}{@{}l}
Aucun $N$ ne $V$\\
Tout/chaque/tous les $N$ ne $V$ pas\\
Les $N$ ne $V$ pas
\end{tabular}
& \begin{tabular}{@{}l@{}}
  \(\Xlo\neg \exists x [\prd n(x) \wedge\prd v(x)]\)\\
\(\Xlo\forall x [\prd n(x) \implq \neg\prd v(x)]\)
  \end{tabular}
\\
\lspbottomrule
\end{tabular}
\end{center}
\caption{Schémas de traductions du français en {\LO}}\label{T:vademecum}
\end{table}

Les formules qui sont dans une même cellule du tableau sont
équivalentes, ce sont donc de simples variantes de traduction en
{\LO}, cela ne marque pas d'ambiguïté.  En revanche, la tournure
\sicut{tous les $N$ ne $V$ pas} est réellement ambiguë.  Nous y
reviendrons au chapitre suivant.


\Writetofile{solf}{\protect\newpage}
% -*- coding: utf-8 -*-
\begin{exo}[Quantificateurs et connecteurs]
\label{exo:2q+c}
Indiquez (informellement\footnote{C'est-à-dire en français, sans
  entrer dans les détails techniques.}) 
\pagesolution{crg:2q+c}
les conditions de vérité des formules
suivantes :
\addtolength{\multicolsep}{-8pt}
%\smallskip
\begin{multicols}{2}
\begin{exolist}
\item \(\Xlo\exists x [\prd{homard}(x) \wedge \prd{gaucher}(x)]\)
\item \(\Xlo\exists x [\prd{homard}(x) \vee \prd{gaucher}(x)]\)
\item \(\Xlo\exists x [\prd{homard}(x) \implq \prd{gaucher}(x)]\)
\item \(\Xlo\exists x [\prd{homard}(x) \ssi \prd{gaucher}(x)]\)
\item \(\Xlo\forall x [\prd{homard}(x) \wedge \prd{gaucher}(x)]\)
\item \(\Xlo\forall x [\prd{homard}(x) \vee \prd{gaucher}(x)]\)
\item \(\Xlo\forall x [\prd{homard}(x) \implq \prd{gaucher}(x)]\)
\item \(\Xlo\forall x [\prd{homard}(x) \ssi \prd{gaucher}(x)]\)
\end{exolist}
\end{multicols}
%

Quelles sont celles qui peuvent être des traductions de phrases
simples et naturelles du français ?

\begin{solu} (p.~\pageref{exo:2q+c})\label{crg:2q+c}
\begin{exolist}
\item \(\Xlo\exists x [\prd{homard}(x) \wedge \prd{gaucher}(x)]\) :
il existe un individu qui est un homard et qui est gaucher ; en français cela donnera \sicut{il y a un homard gaucher} ou, éventuellement, \sicut{il existe des homards gauchers}.

\item \(\Xlo\exists x [\prd{homard}(x) \vee \prd{gaucher}(x)]\) :
il existe un individu qui est un homard ou qui est gaucher ; cette formule est vraie du moment que les homards existent (même s'il n'y a pas de gauchers dans le modèle)\footnote{Ou inversement, cette formule est vraie aussi du moment que les gauchers existent.}.  Pas de phrase naturelle en français pour cette formule.

\item \(\Xlo\exists x [\prd{homard}(x) \implq \prd{gaucher}(x)]\) :
il existe un individu tel que \emph{si} c'est homard alors il est gaucher ; cette formulation des conditions de vérité est un peu alambiquée, mais il faut se souvenir que cela équivaut à : il existe un individu qui n'est pas un homard ou qui est gaucher (car $\Xlo\phi\implq\psi$ équivaut à $\Xlo\neg\phi\vee\psi$) ; cette formule est vraie du moment qu'il existe des individus (par exemple vous et moi) qui ne sont pas des homards... Pas de phrase naturelle en français pour cette formule.

\item \(\Xlo\exists x [\prd{homard}(x) \ssi \prd{gaucher}(x)]\)
il existe un individu qui est un homard gaucher ou bien qui n'est ni homard ni gaucher. Pas de phrase naturelle en français pour cette formule.

\item \(\Xlo\forall x [\prd{homard}(x) \wedge \prd{gaucher}(x)]\) :
tout individu du modèle est un homard gaucher. Pas vraiment de phrase naturelle en français pour cette formule.

\item \(\Xlo\forall x [\prd{homard}(x) \vee \prd{gaucher}(x)]\) :
tout individu est un homard ou est gaucher ; autrement dit, pour tout individu, si ce n'est pas un homard, alors il doit forcément être gaucher (et vice-versa). Pas de phrase naturelle en français pour cette formule.

\item \(\Xlo\forall x [\prd{homard}(x) \implq \prd{gaucher}(x)]\) :
pour tout individu, s'il est un homard alors il est gaucher. En français : \sicut{tous les homards sont gauchers}.

\item \(\Xlo\forall x [\prd{homard}(x) \ssi \prd{gaucher}(x)]\) :
pour tout individu, si c'est un homard, alors il est gaucher et s'il est gaucher, alors c'est un homard. Pas vraiment de phrase naturelle en français pour cette formule, si ce n'est \sicut{«homard» et «gaucher», c'est la même chose}...

\end{exolist}
\end{solu}
\end{exo}



Il est bon de se souvenir que lorsqu'on traduit des phrases du
français (ou de toute autre langue naturelle) il faut toujours s'attendre à
ce que $\Xlo\exists$ aille de paire avec $\Xlo\wedge$ et $\Xlo\forall$ avec
$\Xlo\implq$. 

\is{quantification|)}





\subsection{Quelques définitions logiques}
%''''''''''''''''''''''''''''''''''''''''''''
\label{s:deflog}
Nous avons maintenant les moyens formels de définir certaines notions vues
dans le chapitre~\ref{Ch:1}.


\begin{nota}[Satisfaction]\indexs{satisfaction}\label{def:satis1}
Si une formule $\Xlo\phi$ est vraie dans un modèle $\Modele$, c'est-à-dire
si $\denote{\Xlo\phi}^{\Modele}=1$, on dit que $\Xlo\phi$ est \kwo{satisfaite} par
$\Modele$, ou encore que $\Modele$ \kwo{satisfait} $\Xlo\phi$.\\
On note alors : $\Modele \satisf \Xlo\phi$. %\indexs{$\satisf$}
\end{nota}

Attention : le symbole $\satisf$ est à double emploi.
Il exprime soit la satisfaction d'une formule par un modèle, soit la
conséquence logique entre des phrases ou des formules que nous avions
vue au chapitre~\ref{Ch:1}.  En toute rigueur nous
devrions utiliser deux symboles différents.  D'ailleurs on trouve
parfois la variante de notation $\satisf_{\Modele} \Xlo\phi$ pour exprimer
la satisfaction de $\Xlo\phi$ par $\Modele$.  Mais normalement il n'y a
pas de confusion à craindre : si on trouve un modèle à la gauche de
$\satisf$, le symbole désigne la satisfaction, si on trouve une ou
plusieurs formules, il désigne la conséquence logique.

\largerpage

Dans le chapitre~\ref{Ch:1}, la conséquence logique entre deux phrases
(ou deux formules) était définie en disant que dans tous les cas où la
première phrase est vraie, la seconde l'est aussi.  Maintenant nous
savons ce qu'est formellement un \emph{cas} : c'est un modèle.  La définition
précise se donne donc dans ces termes : $\vrb\phi \satisf \vrb\psi$ si et
seulement si dans tous les modèles par rapport auxquels $\vrb\phi$ est
vraie, $\vrb\psi$ est vraie aussi.

\begin{defi}[Conséquence logique]\label{def:conseqlog}
La formule $\vrb\psi$ est une \kwi{conséquence logique}{consequence logique@conséquence logique} de la formule
$\vrb\phi$, ssi pour tout modèle $\Modele$ tel que $\Modele\satisf\vrb\phi$
alors $\Modele\satisf\vrb\psi$.
\\
Plus généralement, $\vrb\psi$ est une conséquence logique de l'ensemble de
formules \set{\vrbi\phi1;\vrbi\phi2;\dotsc;\vrbi\phi n}, ssi pour tout modèle
$\Modele$ tel que $\Modele\satisf\vrbi\phi1$ et $\Modele\satisf\vrbi\phi2$... et $\Modele\satisf\vrbi\phi n$ alors $\Modele\satisf\vrb\psi$.\\
On note alors : $\vrbi\phi1,\vrbi\phi2,\dotsc,\vrbi\phi n  \satisf \vrb\psi$.
\end{defi}

Partant, on peut aussi redéfinir les notions de tautologies et de
contradiction en termes de modèles.  Une tautologie est vraie dans
tout modèle et une contradiction dans aucun.

\begin{defi}[Tautologie]
Une formule $\vrb\phi$ est une \kw{tautologie}, ssi pour tout modèle
$\Modele$, $\Modele \satisf \vrb\phi$. 
\\
On note alors : $\satisf \vrb\phi$
\end{defi}

\begin{defi}[Contradiction]
Une formule $\vrb\phi$ est une \kw{contradiction}, ou est contradictoire,
ssi pour tout modèle 
$\Modele$, $\Modele \satisf \Xlo\neg\phi$ (c'est-à-dire
$\denote{\vrb\phi}^{\Modele}=0$).  
\\
On note alors : $\satisf \Xlo\neg\phi$
\end{defi}

Il reste toutes les autres formules, celles qui ne sont ni tautologiques ni contradictoires. On les appelles des \kwo{formules contingentes}\indexs{formule!\elid\ contingente}. Ce sont toutes ces formules qui sont parfois vraies, parfois fausses.

\begin{defi}[Formule contingente]
Une formule $\vrb\phi$ est  \kwo{contingente}, ssi il existe au moins un modèle $\Modele_1$ tel que  $\Modele_1 \satisf \vrb\phi$ et un modèle $\Modele_2$ tel que $\Modele_2\satisf \Xlo\neg\phi$.
\end{defi}


De même, la notion d'équivalence logique entre deux formules peut également être définie précisément maintenant.

\begin{defi}[Équivalence logique] \label{d:EquLogLO}
Deux formules $\vrb\phi$ et $\vrb\psi$ sont  \kwo{logiquement
  équivalentes}\indexs{equivalence@équivalence!\elid\ logique}, ou on pourra
  dire aussi \kwo{sémantiquement équivalentes}, ssi pour tout modèle
  $\Modele$, \(\denote{\vrb\phi}^{\Modele}=\denote{\vrb\psi}^{\Modele}\).

 On peut également définir la notion à l'aide de la conséquence
  logique, comme on l'avait bu au chapitre \ref{Ch:1}, \S\ref{ss:EquivalenceLogique} : $\vrb\phi$ et $\vrb\psi$ sont logiquement équivalentes, ssi $\vrb\phi
  \satisf \vrb\psi$ et $\vrb\psi \satisf\vrb\phi$.
\end{defi}

Si $\vrb\phi$ et $\vrb\psi$ sont logiquement équivalentes, cela veut dire qu'elles ont exactement le même sens dans \LO. C'est une conséquence directe de la définition du sens que nous nous sommes donnée : le sens est ce qui détermine la dénotation pour n'importe quel modèle.  Donc si $\vrb\phi$ et $\vrb\psi$ ont une dénotation identique par rapport à tout modèle, c'est bien que leur sens est le même.  C'est aussi ce qui explique la méthode de démonstration des ambiguïtés que nous avons vue au chapitre~\ref{Ch:1}, \S\ref{s:Ambiguïté}.   Si une phrase est ambiguë, c'est qu'elle peut être traduite par (au moins)  deux formules $\vrb\phi$ et $\vrb\psi$ qui représentent des sens différents. Et si ces sens sont différents, c'est que $\vrb\phi$ et $\vrb\psi$ ne sont pas logiquement équivalentes. Ce qui, par la définition \ref{d:EquLogLO}, veut dire qu'il existe au moins un modèle $\Modele$ tel que $\denote{\vrb\phi}^{\Modele}\neq\denote{\vrb\psi}^{\Modele}$. 
Et comme $\vrb\phi$ et $\vrb\psi$ sont des formules, si leurs dénotations sont différentes, c'est qu'il y en a une qui est vraie et l'autre qui est fausse. 
Il s'agit bien là de la méthode présentée en \S\ref{s:Ambiguïté} : on cherche un cas de figure, c'est-à-dire un modèle, pour lequel la phrase, ambiguë, sera jugée à la fois vraie et fausse.


Enfin terminons avec le théorème suivant :

\begin{theo}[$\satisf$ et $\implq$]
$\vrb\phi \satisf \vrb\psi$ si et seulement
si $\satisf \Xlo[\phi \implq \psi]$.
\end{theo}

\largerpage

Nous n'allons pas chercher à démontrer ce théorème ici, 
je le mentionne juste à titre d'entraînement à la
lecture et à la manipulation des notions et des symboles que nous
avons vus jusqu'ici.  Ce théorème montre le rapport qui existe entre
la conséquence logique et l'implication matérielle : il dit que $\vrb\psi$
est une conséquence de $\vrb\phi$ si et seulement si $\Xlo[\phi \implq \psi]$
est une tautologie ; autrement dit
la conséquence
logique correspond à une implication \emph{toujours} vraie.


\begin{exo}\label{exotcc}
Supposons que les  \vrb\phi\ et \vrb\psi\ représentent des formules qui ne contiennent pas %de quantificateurs et 
de variables libres.
En utilisant les tables de vérité, dites si chacune des formules suivantes est une tautologie,  une contradiction ou une formule contingente :

\addtolength{\multicolsep}{-8pt}
\begin{multicols}{3}
\begin{enumerate}
\item \(\Xlo[\phi \wedge \neg\phi]\)
\item \(\Xlo[\phi \vee \neg\phi]\)\footnote{Indice : cette formule illustre ce qui s'appelle la \kwo{loi du tiers exclu}.\indexs{loi!\elid\ du tiers exclu} Nous la retrouverons plus tard.}
\item \(\Xlo[\phi \implq \neg\phi]\)
\item \(\Xlo[\phi \implq \phi]\)
\item \(\Xlo[\neg\phi\implq[\phi\implq\psi]]\)
\item \(\Xlo[[\phi\implq\psi] \vee [\psi\implq\phi]]\)
\end{enumerate}
\end{multicols}
\begin{solu} (p.~\pageref{exotcc})

Dressons les table de vérité de ces formules. 

\small
\[\begin{array}{c||c|c|c|c|c}
\multicolumn{2}{@{}r@{}}{\text{formules:}}&
\multicolumn{1}{c}{1}&
\multicolumn{1}{c}{2}&
\multicolumn{1}{c}{3}&
\multicolumn{1}{c}{4}\\
\Xlo\phi & \Xlo\neg\phi 
& \Xlo[\phi \wedge \neg\phi]
& \Xlo[\phi \vee \neg\phi]
& \Xlo[\phi \implq \neg\phi]
& \Xlo[\phi \implq \phi]\\\hline\hline
1 & 0 & 0 & 1 & 0 & 1\\
0 & 1 & 0 & 1 & 1 & 1\\
\end{array}\]
\normalsize

\smallskip

La formule 1 est une contradiction, c'est même \emph{la loi de contradiction} vue en \S\ref{sss:contrad} (p.~\pageref{sss:contrad}); la formule 2 est une tautologie classique (\emph{la loi du tiers exclu}, qui dit qu'une formule est soit vraie, soit fausse, et qu'il n'y a donc pas de troisième --~ou \emph{tierce}~-- possibilité).  Malgré les apparences (et malgré nos attentes), la formule 4 n'est pas contradictoire : l'interprétation logique de l'implication matérielle en fait une formule contingente (qui est vraie quand son antécédent est faux); la formule 4 est, elle, bien une tautologie. 

\small
\[\begin{array}{c||c|c|c|c|c|c}
\multicolumn{2}{@{}r@{}}{\text{formules:}}&
\multicolumn{1}{c}{}&
\multicolumn{1}{c}{}&
\multicolumn{1}{c}{5}&
\multicolumn{1}{c}{}&
\multicolumn{1}{c}{6}\\
\Xlo\phi & \Xlo\psi & \Xlo\neg\phi 
& \Xlo[\phi \implq \psi]
& \Xlo\Xlo[\neg\phi\implq[\phi\implq\psi]]
& \Xlo[\psi\implq\phi]
& \Xlo\Xlo[[\phi\implq\psi] \vee [\psi\implq\phi]]\\\hline\hline
1 & 1 & 0 & 1 & 1 & 1 & 1\\
1 & 0 & 0 & 0 & 1 & 1 & 1\\
0 & 1 & 1 & 1 & 1 & 0 & 1\\
0 & 0 & 1 & 1 & 1 & 1 & 1\\
\end{array}\]
\normalsize

La formule 5 est une tautologie très remarquable de la logique classique, et on l'appelle \alien{e falso sequitur quodlibet} (cf. note~\ref{efsq}, p.~\pageref{efsq}), \ie\ \emph{du faux s'ensuit n'importe quoi}, ou \emph{à partir d'une hypothèse fausse on peut tout déduire}. En effet la formule commence par poser l'hypothèse que \vrb\phi\ est fausse ($\Xlo\neg\phi$), puis place \vrb\phi\ en antécédent d'une implication dont le conséquent est quelconque ($\Xlo\phi\implq\psi$). 
La formule 6 est une tautologie un peu curieuse (par rapport à notre intuition) qui dit que pour deux formules quelconques, il y en a forcément une qui implique l'autre.  Comme la formule 3, elle tend à montrer que l'implication matérielle ne traduit pas idéalement le sens que nous attribuons aux structures conditionnelles (en \sicut{si}) de la langue.
\end{solu}
\end{exo}

\begin{exo}\label{exoconseql}
En utilisant la définition~\ref{def:conseqlog}, démontrez les conséquences logiques suivantes :

\addtolength{\multicolsep}{-8pt}
\begin{multicols}{2}\raggedcolumns
\begin{enumerate}
\item \(\xlo{[\phi\wedge\psi]} \satisf \xlo{\phi}\)
\item \(\xlo{\phi} \satisf \xlo{[\phi\vee\psi]}\)
\item \(\xlo{[\phi\implq\psi]} \satisf \xlo{[\neg\psi\implq\neg\phi]}\)
\item \(\xlo{[\phi\wedge[\phi\implq\psi]]}\satisf\xlo{\psi}\)
\item \(\xlo{[\phi\implq[\psi\wedge\neg\psi]]} \satisf \xlo{\neg\phi}\)
\end{enumerate}
\end{multicols}
\begin{solu} (p.~\pageref{exoconseql})

En pratique, il y a plusieurs façons d'utiliser la définition~\ref{def:conseqlog} (p.~~\pageref{def:conseqlog}) pour démontrer une conséquence logique.  On pourrait, par exemple, reprendre la méthode des contre-exemples vue en \S\ref{s:conseql}; on pourrait également dressez les tables de vérités des formules en n'examinant que les lignes pour lesquelles la formule de gauche est vraie.  Mais il est tout aussi simple et rapide de démontrer les conséquences par raisonnement, en partant de l'hypothèse que la formule de gauche est vraie (dans un modèle quelconque) et de déduire la vérité de la formule de droite.

\begin{enumerate}
\item \(\xlo{[\phi\wedge\psi]} \satisf \xlo{\phi}\). 
Supposons que $\Xlo[\phi\wedge\psi]$ est vraie. Cela veut donc dire que les deux sous-formules sont vraies, et donc que \vrb\phi\ est vraie.  Évidemment, on a aussi \(\xlo{[\phi\wedge\psi]} \satisf \xlo{\psi}\). 
\item \(\xlo{\phi} \satisf \xlo{[\phi\vee\psi]}\).
Supposons que \vrb\phi\ est vraie. Alors $\Xlo[\phi\wedge\psi]$ puisqu'il suffit qu'une des deux sous-formules soit vraie pour qu'une disjonction soit vraie.
\item \(\xlo{[\phi\implq\psi]} \satisf \xlo{[\neg\psi\implq\neg\phi]}\).
Supposons que $\Xlo[\phi\implq\psi]$ est vraie. Cela veut dire alors, en vertu de l'interprétation de $\Xlo\implq$,  que soit \vrb\phi\ est fausse, soit \vrb\psi\ est vraie.  Si \vrb\psi\ est vraie, alors $\Xlo\neg\psi$ est fausse et donc $\Xlo[\neg\psi\implq\neg\phi]$ est vraie.  Si $\Xlo\phi$ est fausse, alors $\Xlo\neg\phi$ est vraie et donc $\Xlo[\neg\psi\implq\neg\phi]$ est encore vraie.
\item \(\xlo{[\phi\wedge[\phi\implq\psi]]}\satisf\xlo{\psi}\).
Supposons que $\Xlo[\phi\wedge[\phi\implq\psi]]$ est vraie. Cela veut dire, d'abord, que \vrb\phi\  et  $\Xlo[\phi\implq\psi]$ sont toutes les deux vraies. Comme \vrb\phi\ est vraie, alors pour que $\Xlo[\phi\implq\psi]$ le soit aussi, il faut nécessairement \vrb\psi\ soit vraie.  Cette conséquence est, en fait, la célèbre règle logique dite du \alien{modus ponens}\is{modus ponens@\alien{modus ponens}}.
\item \(\xlo{[\phi\implq[\psi\wedge\neg\psi]]} \satisf \xlo{\neg\phi}\).
Supposons que $\Xlo[\phi\implq[\psi\wedge\neg\psi]$ est vraie. On sait par ailleurs que $\Xlo[\psi\wedge\neg\psi]$ est fausse puisque c'est une contradiction (cf. l'exercice précédent). Et le seul cas où une implication est vraie lorsque sont conséquent est faux est celui où sont antécédente est faux aussi. Donc \vrb\phi\ est fausse, et  $\Xlo\neg\phi$ est vraie.
\end{enumerate}
\end{solu}
\end{exo}


%\newpage

\section{Conclusion}
%===================
\label{conclu:LCP}


Dans ce chapitre, nous avons introduit un système sémantique qui s'appuie sur un langage formel {\LO} issu de la logique du calcul des prédicats. Le langage {\LO} est \emph{interprété} ; cela veut dire qu'il est muni d'une sémantique rigoureusement définie par un ensemble de règles d'interprétation (donné dans les définitions \ref{d:Sem1} et \ref{d:Sem1Q}).
La démarche d'analyse qu'engendre ce type de système est souvent qualifiée de \emph{méthode d'interprétation indirecte},\is{interprétation directe/indirecte} parce que {\LO} est utilisé comme un langage intermédiaire ou pivot pour représenter le sens des expressions linguistiques. C'est ce qu'illustre l'architecture du processus d'analyse sémantique schématisé en figure \ref{fig:pas1}. 
Nous y voyons que  ce sont les expressions de {\LO} (les formules)  qui possèdent une interprétation directe (représentée par la flèche de droite), pas les expressions de la langue (par exemples les phrases).  Le processus nécessite donc en amont une phase de \emph{traduction} des expressions linguistiques (flèche de gauche) afin que par transitivité le sens défini formellement pour une expression de {\LO} soit identifiée au sens de l'expression linguistique dont elle est la traduction.  


\begin{figure}[h]
\begin{center}
\begin{pspicture}(9,2)
\rput[l](0,1.2){\rnode{F}{\boitealu{\textbm{Français}}}}
\rput(.8,.5){\small\itshape phrases}
\rput(5,1.2){\rnode{L}{\boitealu{\textbf{\LO}}}}
\rput(5,.5){\small\itshape formules}
\rput[r](9,1.2){\rnode{M}{\boitealu{$\boldsymbol{\Modele}$}}}
\rput(8.6,.3){\small\itshape\pile{valeurs\\sémantiques}}
\ncline[nodesep=3pt,linewidth=1.5pt,linecolor=darkgray]{->}{F}{L}%\Aput{\Ftrad}
\Bput{\small\sffamily traduction}
\ncline[nodesep=3pt,linewidth=1.5pt,linecolor=darkgray]{->}{L}{M}%\Aput{\denote{\,}}
\Bput{\small\sffamily interprétation}
\end{pspicture}
\caption{Processus d'analyse sémantique (1)}\label{fig:pas1}
\end{center}
\end{figure}

\sloppy

L'opération de traduction peut se formaliser comme une fonction qui à chaque expression du français associe une expression de {\LO} qui représente son sens.  Appelons-la {\Ftrad}.\is{F@\Ftrad}  Nous aurons ainsi, par exemple, $\Ftrad(\sicut{un homme dort})={\Xlo\exists x[\prd{homme}(x)\wedge\prd{dormir}(x)]}$. D'autre part, l'opération d'interprétation de {\LO} dans un modèle est également une fonction ; c'est ce que nous représentons par $\denote{\,}^{\Modele}$ qui à toute expression de {\LO} associe sa valeur sémantique (\ie\ sa dénotation) dans \Modele.  Autrement dit analyser sémantiquement une expression $E$ de la langue revient à définir $\denote{\Ftrad(E)}^{\Modele}$, qui consiste en un enchaînement des deux fonctions \Ftrad\ et $\denote{\,}^{\Modele}$.  Ainsi l'analyse se fait effectivement  en deux passes.

\fussy

Il est important ici de faire une remarque sur le statut de \Ftrad.\is{F@\Ftrad}  Elle vient d'être présentée comme une fonction ; cela implique qu'à toute expression de la langue elle associe une et une seule traduction dans {\LO}.  Et les expressions de {\LO}, par définition, ont un et un seul sens.  Or 
nous savons pertinemment que de très nombreuses expressions linguistiques 
sont ambiguës ou polysémiques (c'est le propre des langues naturelles), c'est-à-dire qu'elles peuvent être associées à plusieurs sens et donc recevoir plusieurs traductions différentes.  En tant que fonction, \Ftrad\ ne peut pas faire cela.  C'est pourquoi, pour la cohérence globale du système, il est nécessaire de considérer qu'il existe en fait une grande pluralité de fonctions \Ftrad, chacune sélectionnant un sens particulier pour chaque expression ambiguë de la langue.  
Il faudra donc se souvenir que les analyses sémantiques que nous réalisons se font toujours par rapport au choix (implicite) d'une certaine fonction de traduction\footnote{Sachant qu'à partir des chapitres \ref{ch:types} et \ref{ch:ISS} nous verrons que les traductions s'effectuent à partir d'expressions linguistiques analysées syntaxiquement, ce qui réduit considérablement le nombre d'ambiguïtés potentielles.}.

La méthode indirecte est couramment utilisée en sémantique formelle et c'est celle qui est adoptée notamment par \citet{PTQ}\Andex{Montague, R.} et les très nombreux travaux qui y ont fait suite.
Elle se distingue bien sûr de la méthode d'interprétation \emph{directe}\is{interprétation directe/indirecte}
qui consiste à définir des règles d'interprétation directement sur les expressions de la langue, avec des formulation comme, par exemple, $\denote{\sicut{Joey dort}}^{\Modele}=1$ ssi $\denote{\sicut{Joey}}^{\Modele}\in\denote{\sicut{dort}}^{\Modele}$.  Cette méthode est employée notamment par \citet{Montague:EFL} et \citet{HeimKratzer:97}.\Andex{Heim, I.}\Andex{Kratzer, A.}  Sur le fond il n'y a pas vraiment de différence entre les deux méthodes (souvent, beaucoup des notations utilisées dans le langage intermédiaire se retrouvent réimportées au niveau du modèle dans la méthode directe), elles se distinguent essentiellement sur la manière de mettre forme les descriptions sémantiques.  La méthode indirecte, que nous adoptons ici, a généralement l'avantage de présenter des formulations de sens à la fois explicites et concises.

\medskip

Le langage {\LO} est en quelque sorte le langage sémantique le plus simple que nous pouvons développer pour commencer à formaliser le sens d'une bonne variété de phrases de la langue.  Il sera amené à évoluer et à se perfectionner tout au long des chapitres qui suivent.
Tel qu'il est défini ici, {\LO} est \kwo{extensionnel}\indexs{extensionnel}.  \emph{Extension}\indexs{extension} est synonyme de \emph{dénotation}, et un langage extensionnel est un langage qui satisfait le principe dit d'extensionnalité\is{extensionnalité!principe d'\elid}.  Ce principe dit que si, dans une expression $E$, on remplace une de ses sous-expressions $A$ par une expression $B$ qui a la même dénotation que $A$, alors la dénotation de $E$ ne change pas.  C'est ce que nous avons appliqué avec les exemples \ref{x:1821} p.~\pageref{leibniz} \alien{supra}.  Nous verrons au chapitre \ref{Ch:t+m} qu'un langage extensionnel n'est pas entièrement adéquat pour rendre compte des propriétés sémantiques de la langue, et ce sera une occasion de  modifier {\LO} en conséquence.

L'extensionnalité de {\LO} fait aussi que, techniquement, les règles d'interprétation nous font calculer la dénotation des expressions par rapport à un modèle (avec $\denote{\,}^{\Modele}$).  La dénotation n'est pas le sens et on peut être tenté alors de se demander : où est exactement le sens dans le système ? 
En sémantique, ça n'est jamais le résultat du calcul de la dénotation d'une expression par rapport à un modèle donné qui importe vraiment, ce qui compte c'est la \emph{règle} qui a permis de réussir ce calcul pour n'importe quel modèle.  Et comme cela a été mentionné plusieurs fois dans ce chapitre, c'est cela le sens d'une expression : \emph{la règle de calcul} de sa dénotation --~ce que l'on appelle \emph{les conditions de vérité} dans le cas d'une phrase ou une formule.  
Le sens est donc d'une certaine manière une généralisation sur la dénotation ; cette généralisation consiste à faire abstraction du modèle en regardant ce qui reste constant dans le calcul quel que soit modèle utilisé.  
Ainsi si $\denote{\vrb\phi}^{\Modele}$ représente la dénotation d'une formule \vrb\phi\ dans \Modele, on peut dire que \denote{\vrb\phi} et même simplement \vrb\phi\ représentent le sens de \vrb\phi.  Les formules de {\LO} et les conditions de vérité sont finalement la même chose, dans la mesure où elles donnent la même information.

Un reproche qui est souvent adressé à cette manière de théoriser le sens est qu'elle trouverait ses limites quand on se place au niveau des prédicats et qu'elle ferait ainsi l'impasse sur la sémantique lexicale.\is{semantique@sémantique!\elid\ lexicale}  Mais formellement ce n'est pas exact.  
Certes les systèmes de sémantique formelle se concentrent prioritairement sur les propriétés logiques de l'interprétation des phrases et  
développent moins souvent la dimension lexicale du sens%
\footnote{Principalement parce que c'est une question formidablement vaste et complexe, lestée notamment par l'épineux problème de la polysémie.}.
Et il est vrai que lorsque l'on dit que \prd{dormir} représente le sens de \sicut{dormir}, on n'a pas dit grand chose sur le sens de ce verbe... si ce n'est qu'il est (probablement) différent du sens de, par exemple, \prd{écumoire}, simplement parce qu'on a utilisé des symboles de prédicats différents.  
Cependant le sens lexical est formellement présent dans le système, ne serait-ce que par hypothèse.  Il est encapsulé dans la fonction d'interprétation \FI\ du modèle.  En effet pour tout prédicat de {\LO}, \FI\ sait dire quels individus appartiennent à sa dénotation dans le modèle ; cette connaissance, par définition, est le sens du prédicat.  
Le sens lexical n'est donc pas explicité mais il a néanmoins sa place %est néanmoins défini 
dans le système.
C'est une erreur de supposer que la sémantique lexicale et la sémantique formelle seraient orthogonales ou disjointes parce qu'elles auraient des objets d'étude distincts.  
L'objet de la sémantique formelle est le sens linguistique, et ce qui la caractérise fondamentalement c'est un cadre méthodologique scientifique qui garantit la formulation d'hypothèses d'analyses falsifiables et dont on peut tirer les conséquences avec fiabilité. 
Depuis longtemps la sémantique formelle aborde le lexique dans ce cadre méthodologique\footnote{Cf. par exemple \cite{Dow:79}\Andexn{Dowty, D.} pour un exemple particulièrement précoce.} et de nos jours les travaux dans ce domaine sont de plus en plus nombreux.  
Cet intérêt apparemment tardif est en réalité seulement ordonné : il est plus raisonnable de s'attaquer aux problèmes complexes une fois que nous sommes équipés d'un système d'analyse suffisamment élaboré que de procéder dans l'ordre inverse.  En l'état, tel que {\LO} est défini dans ce chapitre et tant que nous n'avons pas avancé sur l'intensionnalité (chapitre \ref{Ch:t+m}) et la compositionnalité (chapitres \ref{ch:types} \& \ref{ch:ISS}),
il est probablement prématuré de soulever en profondeur la question du lexique.



\largerpage

\nocite{DWP:81,ChierchiaMcCG:90,Corb:13}\nocite{LBdG:03}\nocite{Gamut:1}%

\subsection*{Repères bibliographiques}
%-------------------------------------
Ce chapitre aborde assez longuement des notions de logiques indispensables à une bonne maîtrise de la sémantique formelle. Mais il ne peut se substituer entièrement à une véritable et complète introduction à la logique\footnote{En particulier le chapitre n'aborde pas les systèmes de démonstration, comme par exemple les \emph{systèmes axiomatiques} ou la \emph{déduction naturelle}, qui formalisent des méthodes pour construire «mathématiquement» les conséquences logiques d'un ensemble de formules données. Même si la connaissance de ces systèmes et des théories de la preuve  est importante (et recommandée) pour une bonne maîtrise de la logique, ils sortent un peu de la portée d'un ouvrage principalement consacré à l'analyse sémantique des énoncés de la langue.}. 
Il existe de nombreux ouvrages dédiés à la discipline, en anglais et en français, qui permettront d'approfondir  avantageusement ce que présente le chapitre. Dans une perspective explicitement orientée vers la sémantique formelle, \citet{Gamut:1} est particulièrement approprié ; en français, \citet{LBdG:03} constitue une introduction très pédagogique.

Les premiers chapitres de la plupart des manuels de sémantique formelle \citep[p. ex.][]{DWP:81,ChierchiaMcCG:90,Corb:13} peuvent servir de lectures complémentaires, sachant cependant que beaucoup abordent rapidement des notions que nous verrons à partir des chapitres~\ref{ch:gn} et \ref{ch:types}.




%\newpage
\section*{Exercices supplémentaires}
%==================================

\subsection*{Traductions}
%-----------------------

\Writetofile{solf}{\protect\newpage}
% -*- coding: utf-8 -*-
\begin{exo}[QCM de traductions]\label{e:QCM1}
Pour chaque phrase suivante,\pagesolution{crg:QCM1}
 indiquez quelle est sa (ou ses) traduction(s) correcte(s) 
dans \LO\ parmi celles proposées.
Pour chacune des phrases, il y a au moins une traduction correcte, mais pour certaines phrases, il peut y en avoir plusieurs.

\SetEnumitemKey{qcm}{label=\Square\ \alph*.,align=left}
\begin{enumerate}[leftmargin=*]
\item Quelques étudiants connaissent toutes les réponses.
  \begin{enumerate}[qcm]
    \item \(\Xlo\forall r [\prd{étudiant}(x) \wedge \prd{connaître}(x,r)]\)
    \item \(\Xlo\exists x\forall y [[\prd{étudiant}(x) \wedge \prd{réponse}(y)]\implq \prd{connaître}(x,y)]\)
    \item \(\Xlo\forall x [\prd{réponse}(x) \implq \exists y [\prd{étudiant}(y) \wedge \prd{connaître}(y,x)]]\)
    \item \(\Xlo\exists x [\prd{étudiant}(x) \wedge \forall y [\prd{réponse}(y) \implq \prd{connaître}(x,y)]]\)
  \end{enumerate}

\item Tous les étudiants connaissent une réponse.
  \begin{enumerate}[qcm]
    \item \(\Xlo\forall x \exists y[[\prd{étudiant}(x)\wedge\prd{réponse}(y)]\implq \prd{connaître}(x,y)]\)
    \item \(\Xlo\forall x \exists y[\prd{étudiant}(x)\implq [\prd{réponse}(y) \wedge \prd{connaître}(x,y)]]\)
    \item \(\Xlo\forall x [[\prd{étudiant}(x)\wedge\exists y\,\prd{réponse}(y)]\implq \prd{connaître}(x,y)]\)
    \item \(\Xlo\forall x [\prd{étudiant}(x)\implq \exists y[\prd{réponse}(y) \implq \prd{connaître}(x,y)]]\)
  \end{enumerate}

\item Alice connaît des réponses, mais pas toutes.
  \begin{enumerate}[qcm] 
    \item \(\Xlo \forall x [\prd{réponse}(x) \implq [\prd{connaître}(\cns a,x) \vee \neg\prd{connaître}(\cns a,x)]]\)
      \item \(\Xlo\exists x [\prd{réponse}(x) \wedge \prd{connaître}(\cns a,x)] \wedge \neg\forall x [\prd{réponse}(x) \implq \prd{connaître}(\cns a,x)]\)
      \item \(\Xlo\exists x [[\prd{réponse}(x) \wedge \prd{connaître}(\cns a,x)] \wedge [\prd{réponse}(x)  \wedge \neg\prd{connaître}(\cns a,x)]]\)
      \item \(\Xlo\exists x [\prd{réponse}(x) \wedge \prd{connaître}(\cns a,x)] \wedge \neg\exists y [\prd{réponse}(y) \wedge \prd{connaître}(\cns a,y)]\)
  \end{enumerate}

\item Aucun étudiant ne connaît toutes les réponses.
  \begin{enumerate}[qcm]
    \item \(\Xlo\forall x [\prd{étudiant}(x) \implq \forall y [\prd{réponse}(y) \implq \neg\prd{connaître}(x,y)]]\)
    \item \(\Xlo\forall x [\prd{étudiant}(x) \implq \neg\forall y [\prd{réponse}(y) \implq \prd{connaître}(x,y)]]\)
    \item \(\Xlo\neg\forall x [\prd{étudiant}(x) \implq \forall y [\prd{réponse}(y) \implq \prd{connaître}(x,y)]]\)
    \item \(\Xlo\forall y [\prd{réponse}(y) \implq \neg\forall x [\prd{étudiant}(x) \implq \prd{connaître}(x,y)]]\)
  \end{enumerate}

\item Il est faux que quelques étudiants ne connaissent aucune réponse.
  \begin{enumerate}[qcm]
    \item \(\Xlo\neg\exists x[\prd{étudiant}(x)\wedge \forall y [\prd{réponse}(y)\implq\neg\prd{connaître}(x,y)]]\)
    \item \(\Xlo\forall x [\prd{étudiant}(x) \implq \forall y [\prd{réponse}(y) \implq \prd{connaître}(x,y)]]\)
    \item \(\Xlo\forall x [\prd{étudiant}(x) \implq \exists y [\prd{réponse}(y) \wedge \neg\prd{connaître}(x,y)]]\)
    \item \(\Xlo\neg\forall y [\prd{réponse}(y) \implq \exists x [\prd{étudiant}(x) \wedge \neg\prd{connaître}(x,y)] ]\)
  \end{enumerate}

\item Alice connaît une seule réponse.
  \begin{enumerate}[qcm]
    \item \(\Xlo\exists x [\prd{réponse}(x) \wedge \prd{connaître}(\cns a,x)] \wedge \forall y [\prd{réponse}(y)\implq\neg\prd{connaître}(\cns a,y)]\)
    \item \(\Xlo\exists x [\prd{réponse}(x) \wedge \prd{connaître}(\cns a,x)] \wedge \exists y [\prd{réponse}(y) \wedge \neg\prd{connaître}(\cns a,y)] \)
    \item \(\Xlo\exists x [\prd{réponse}(x) \wedge \forall y [[\prd{réponse}(y) \wedge \prd{connaître}(\cns a,y)]\implq y=x]]\)
    \item \(\Xlo\exists x [[\prd{réponse}(x) \wedge \prd{seul}(x)] \wedge \prd{connaître}(\cns a,x)]\)
  \end{enumerate}

\item Seuls les tricheurs connaissent toutes les réponses.
  \begin{enumerate}[qcm]
    \item \(\Xlo\forall x [\prd{tricheur}(x) \implq \forall y [\prd{réponse}(y) \implq \prd{connaître}(x,y)]]\)
    \item \(\Xlo\forall x [\forall y [\prd{réponse}(y) \implq \prd{connaître}(x,y)] \implq \prd{tricheur}(x)]\)
    \item \(\Xlo\forall x \forall y [[\prd{réponse}(y) \implq \prd{connaître}(x,y)] \implq \prd{tricheur}(x)]\)
    \item \(\Xlo\neg\exists x [\neg\prd{tricheur}(x) \wedge \forall y [\prd{réponse}(y) \implq \prd{connaître}(x,y)]]\)
  \end{enumerate}
\end{enumerate}

\begin{solu}(p.~\pageref{e:QCM1})\label{crg:QCM1}

\begin{enumerate}
\item d. %1
Remarque : c pourrait également être une traduction correcte, mais cette formule signifie  \sicut{pour chaque réponse, il y a au moins un étudiant qui la connaît} et il semble que ce ne soit pas une interprétation naturelle de la phrase 1.
\item b. %2
Sachant que cette formule est équivalente à \(\Xlo\forall x [\prd{étudiant}(x)\implq \exists y[\prd{réponse}(y) \wedge \prd{connaître}(x,y)]]\)

\item b. %3
Les formules c et d sont contradictoires. La formule a est une tautologie. 

\item b. %4
La formule a signifie que chaque étudiant ne connaît aucune réponse. 


\item a. %5
\item c. %6
\item b, d.  %7
(équivalentes) 

\end{enumerate}
\end{solu}
\end{exo}


\medskip

% -*- coding: utf-8 -*-
\begin{exo}[Structure logique des formules]\label{e:elf}
Pour embellir {\LO}, j'ai écrit les constantes non logiques en runes elfiques.
\pagesolution{crg:elf}
En vous basant sur la structure logique des formules 1--5, retrouvez quelle formule est la traduction de quelle phrase du français (indiquez simplement les correspondances lettres--chiffres).

NB : Ci-dessous, seuls \vrb x et \vrb y sont des variables.
\newcommand{\cnelf}[1]{\ensuremath{\text{\texttw{#1}}}}

%\hspace{-2cm}
%\begin{minipage}[t]{.55\textwidth}
\begin{enumerate}[label=\alph*.]
\item Il n'y a que les élus qui voient la Dame du Lac.
\item Si Merlin réussit un sort, Arthur sera étonné.
\item Lancelot n'est le cousin d'aucun chevalier.
\item Les dames ne portent jamais d'armure.
\item Les chevaliers n'ont pas tous une épée.
\end{enumerate}
%\end{minipage}
%\begin{minipage}[t]{.6\textwidth}

\begin{enumerate}
\item \(\Xlo \forall x [\cnelf{}(x) \implq \neg\exists y[\cnelf{}(y) \wedge \cnelf{}(x,y)]]\)
\item \(\Xlo\neg\forall x [\cnelf{}(x) \implq \exists y[\cnelf{}(y) \wedge \cnelf{}(x,y)]]\)
\item \(\Xlo [\exists x [\cnelf{}(x) \wedge \cnelf{}(\cnelf ,x)] \implq \cnelf{}(\cnelf )]\)
\item
\(\Xlo \forall x [\cnelf{}(x,\cnelf{}) \implq \cnelf{}(x)]\)
\item
\(\Xlo \forall x [\cnelf{}(x) \implq \neg \cnelf{}(\cnelf{},x)]\)
\end{enumerate}
%\end{minipage}

\begin{solu}(p.~\pageref{e:elf})\label{crg:elf}

a--4 ; b--3 ; c--5 ; d--1 ; e--2.
\end{solu}
\end{exo}



\subsection*{Interprétation}
%--------------------------


% -*- coding: utf-8 -*-
\begin{exo}[Modèle]\label{exo:casso}
Représentez l'état de choses représenté %la disposition représentée 
\pagesolution{crg:casso}%
dans l'image en figure~\ref{fig:exo:Modele} sous la forme d'un modèle \tuple{\Unv A,\FI}.

\begin{figure}[h!]
\begin{center}
\fbox{\Large
\(\begin{array}{cccc}
\Box & \heartsuit & & \blacktriangle\\
& \bigcirc & \blacksquare & \\
\bigstar & \Box & & \clubsuit\\
&& \triangle
  \end{array}%
\)%
}
\caption{Modèle «casseaux»}\label{fig:exo:Modele}
\end{center}
\end{figure}

Vous poserez
\(\Unv A = \set{\Box_1; \heartsuit; \blacktriangle;
 \bigcirc; \blacksquare; 
\bigstar ; \Box_2 ; \clubsuit ; \triangle}\), avec $\Box_1$ pour le carré en au haut à gauche et $\Box_2$ pour l'autre carré blanc.
Vous définirez $\FI$ pour les prédicats unaires
\prd{carré}, \prd{rond}, \prd{c\oe ur}, \prd{triangle}, \prd{étoile}, \prd{trèfle}, \prd{blanc}, \prd{noir}, \prd{objet}
et les prédicats binaires
\prd{à-gauche-de}, \prd{à-droite-de}, \prd{au-dessus-de}, \prd{en-dessous-de}.

\begin{solu} (p.~\pageref{exo:casso})\label{crg:casso}

\raggedright

\(\Unv A = \set{\Box_1; \heartsuit; \blacktriangle;
 \bigcirc; \blacksquare; 
\bigstar ; \Box_2 ; \clubsuit ; \triangle}\).

\noindent
\(\FI(\prd{carré}) = \set{\Box_1; \blacksquare;\Box_2}\), 
\(\FI(\prd{rond}) = \set{\bigcirc}\), 
\(\FI(\prd{c\oe ur}) = \set{\heartsuit}\), 
\(\FI(\prd{triangle}) = \set{\blacktriangle; \triangle}\), 
\(\FI(\prd{étoile}) = \set{\bigstar}\), 
\(\FI(\prd{trèfle}) = \set{\clubsuit}\), 
\(\FI(\prd{blanc}) = \set{\Box_1; \heartsuit;\bigcirc;\Box_2;\triangle}\), 
\(\FI(\prd{noir}) = \set{\blacktriangle; \blacksquare;\bigstar;\clubsuit}\), 
\(\FI(\prd{objet}) = \Unv A\), 
\(\FI(\prd{à-gauche-de}) = \set{\tuple{\Box_1,\heartsuit} ; \tuple{\Box_1,\blacktriangle}; \tuple{\Box_1,\blacktriangle} ;
\tuple{\bigcirc,\blacksquare} ; 
\tuple{\bigstar,\Box_2} ; \tuple{\bigstar,\clubsuit} ; \tuple{\Box_2,\clubsuit}}\),
\(\FI(\prd{à-droite-de}) = \set{\tuple{\heartsuit,\Box_1} ; \tuple{\blacktriangle,\Box_1}; \tuple{\blacktriangle,\Box_1} ;
\tuple{\blacksquare,\bigcirc} ; 
\tuple{\Box_2,\bigstar} ; \tuple{\clubsuit,\bigstar} ; \tuple{\clubsuit,\Box_2}}\),
\(\FI(\prd{au-dessus-de}) = 
\set{\tuple{\Box_1,\bigstar}; \tuple{\heartsuit,\bigcirc} ; \tuple{\heartsuit,\Box_2} ; \tuple{\bigcirc,\Box_2} ; \tuple{\blacksquare,\triangle} ; \tuple{\blacktriangle , \clubsuit}}
\),
\(\FI(\prd{en-dessous-de}) = 
\set{\tuple{\bigstar,\Box_1}; \tuple{\bigcirc,\heartsuit} ; \tuple{\Box_2,\heartsuit} ; \tuple{\Box_2,\bigcirc} ; \tuple{\triangle,\blacksquare} ; \tuple{\clubsuit,\blacktriangle}}
\).
 


\end{solu}
\end{exo}


% -*- coding: utf-8 -*-
\begin{exo}[Interprétation dans un modèle]
\label{exo:figaro}
Soit le modèle suivant \(\Modele=\tuple{\Unv A,\FI}\), avec :
\pagesolution{crg:figaro}

%\sloppy
\begin{itemize}\raggedright
\item \(\Unv A=\set{\Obj{Almv}; \Obj{Rosn}; \Obj{Figr}; \Obj{Suzn}; \Obj{Mrcl};
  \Obj{Chrb}; \Obj{Fnch}; \Obj{Antn}; \Obj{Bart}}\) ;

\item \(\FI(\cnsi a1) = \Obj{Almv} ; 
\FI(\cns r) = \Obj{Rosn} ; 
\FI(\cnsi f1) = \Obj{Figr} ; 
\FI(\cns s) = \Obj{Suzn} ; 
\FI(\cns m) = \Obj{Mrcl} ;
\FI(\cns c) = \Obj{Chrb} ; 
\FI(\cnsi f2) = \Obj{Fnch} ; 
\FI(\cnsi a2) = \Obj{Antn} ; 
\FI(\cns b) = \Obj{Bart}\) ;

\item 
\(\FI(\prd{homme})=\set{\Obj{Almv}; \Obj{Figr}; \Obj{Chrb}; \Obj{Antn}; \Obj{Bart}}\) ;
\item \label{modele1}
\(\FI(\prd{femme})=\set{\Obj{Rosn};\Obj{Suzn}; \Obj{Mrcl}; \Obj{Fnch}}\) ;

\item 
\(\FI(\prd{domestique})=\set{\Obj{Figr}; \Obj{Suzn}; \Obj{Fnch}; \Obj{Antn}}\) ;
\item \(\FI(\prd{noble}) = \set{\Obj{Almv}; \Obj{Rosn};  \Obj{Chrb} }\) ;

\item 
\(\FI(\prd{roturier}) = \set{\Obj{Figr}; \Obj{Suzn}; \Obj{Mrcl};
 \Obj{Fnch}; \Obj{Antn}; \Obj{Bart}}\) ;

\item 
\(\FI(\prd{comte}) = \set{\Obj{Almv}}\) ;
\qquad \quad 
\(\FI(\prd{comtesse}) = \set{\Obj{Rosn}}\) ;

\item 
\(\FI(\prd{infidèle}) = \set{\Obj{Almv}}\) ;
\qquad 
%\item 
\(\FI(\prd{volage}) = \set{\Obj{Almv}; \Obj{Chrb}}\) ;
\item
%\item 
\(\FI(\prd{triste}) = \set{\Obj{Rosn}}\) ;

\item 
\(\FI(\prd{époux-de}) = \set{\tuple{\Obj{Almv}, \Obj{Rosn}};\tuple{\Obj{Figr}, \Obj{Suzn}}}\) ;
\item 
\(\FI(\prd{épouse-de}) = \set{\tuple{\Obj{Rosn},\Obj{Almv}};\tuple{\Obj{Suzn},\Obj{Figr}}}\) ;

\item 
\(\FI(\prd{père-de}) = \set{\tuple{\Obj{Antn},\Obj{Fnch}}}\) ;

\item 
\(\FI(\prd{aimer}) = \set{\tuple{\Obj{Figr},
    \Obj{Suzn}};\tuple{\Obj{Suzn},\Obj{Figr}}; \tuple{\Obj{Almv},
    \Obj{Suzn}}; \tuple{\Obj{Rosn},\Obj{Almv}}; \tuple{\Obj{Chrb},
    \Obj{Rosn}}; \tuple{\Obj{Chrb}, \Obj{Suzn}}; \tuple{\Obj{Chrb},
    \Obj{Fnch}}; \tuple{\Obj{Fnch},\Obj{Chrb}}; \tuple{\Obj{Mrcl},\Obj{Figr}}}\)
\end{itemize}

\fussy

%% On considère également les traductions suivantes des noms propres (du
%% français) en constantes de {\LCP}:
%% \begin{itemize}
%% \item 
%% Almaviva $\leadsto$ $\cns a_1$;
%% Rosine $\leadsto$ $\cns r$;
%% Figaro $\leadsto$ $\cns f_1$;
%% Suzanne $\leadsto$ $\cns s$;
%% Marceline $\leadsto$ $\cns m$;
%% Chérubin $\leadsto$ $\cns c$;
%% Fanchette $\leadsto$ $\cns f_2$;
%% Antonio $\leadsto$ $\cns a_2$;
%% Bartholo $\leadsto$ $\cns b$;
%% \end{itemize}

\medskip

\begin{enumerate}
\item Donnez la dénotation dans {\Modele} de chacune des formules suivantes.
Vous justifierez vos réponses sans entrer dans le détail du calcul formel, mais en explicitant les étapes de votre raisonnement. 

%, puis pour chacune proposez une
%  traduction possible en français.

\begin{enumerate}
%\item \([\prd{noble}(\cns a_1) \wedge \prd{roturier}(\cns f_1)]\)
\item \(\Xlo\exists x [\prd{femme}(x) \wedge \neg\prd{triste}(x)]\)

\item \(\Xlo\forall x [\prd{homme}(x) \implq [\prd{infidèle}(x) \vee \prd{volage}(x)]]\)

\item \(\Xlo\exists x\exists y [\prd{aimer}(x,y) \wedge \prd{aimer}(y,x)]\)

\item \(\Xlo[\prd{aimer}(\cnsi a1, \cns s) \implq \exists x\, \neg\prd{aimer}(x,\cnsi a1)]\)

\item \(\Xlo\neg\exists x [\prd{femme}(x) \wedge \exists y\,\prd{aimer}(x,y)]\)

\item  \(\Xlo\forall x [[\prd{homme}(x) \wedge \exists y
  [\prd{épouse-de}(y,x) \wedge \neg\prd{aimer}(x,y)]] \implq \prd{volage}(x)]\)
\end{enumerate}

\textbf{Conseil :} commencez par traduire les formules en phrases de la langue, puis demandez-vous si ces phrases sont vraies ou fausses dans {\Modele} (ainsi vous savez à l'avance la réponse que vous devez trouver). 

\bigskip

\item 
Modifiez le modèle {\Modele} pour faire en sorte que : \Obj{Rosn} ne soit plus triste, que \Obj{Fnch} et \Obj{Chrb} soient mariés et que tous les maris aiment leur femme et seulement elle.
\end{enumerate}

\begin{solu} (p.~\pageref{exo:figaro})\label{crg:figaro}
%\begin{enumerate}\sloppy
%\item 

1. Dénotation des formules.
\begin{enumerate}[label=\alph*.]
%\sloppy
\item \(\Xlo\exists x [\prd{femme}(x) \wedge \neg\prd{triste}(x)]\)

Pour que \(\denote{\Xlo\exists x [\prd{femme}(x) \wedge \neg\prd{triste}(x)]}^{\Modele}=1\), il faut trouver au moins une valeur de \vrb x telle que \(\Xlo\prd{femme}(x)\) et \(\Xlo\neg\prd{triste}(x)\) soient vraies dans \Modele.  Une valeur de \vrb x est une constante de \LO\ ou, ce qui revient au même, un individu de \Unv A. 

Une telle valeur existe bien, par exemple \Obj{Suzn}.  En effet \Obj{Suzn} appartient à \(\FI(\prd{femme})\) (l'ensemble des femmes) et n'appartient pas à \(\FI(\prd{triste})\) (l'ensemble des individus tristes).

Donc \(\denote{\Xlo\exists x [\prd{femme}(x) \wedge \neg\prd{triste}(x)]}^{\Modele}=1\).

La formule correspond à la phrase \sicut{il y a une femme qui n'est pas triste}.

\item \(\Xlo\forall x [\prd{homme}(x) \implq [\prd{infidèle}(x) \vee \prd{volage}(x)]]\)

\sloppy

Pour que 
\(\denote{\Xlo\forall x [\prd{homme}(x) \implq [\prd{infidèle}(x) \vee \prd{volage}(x)]]}^{\Modele}=1\), il faut que, pour chaque valeur que l'on peut assigner à  \vrb x, lorsque l'on répète le calcul de \(\denote{\Xlo [\prd{homme}(x) \implq [\prd{infidèle}(x) \vee \prd{volage}(x)]]}^{\Modele}\) on trouve $1$ à chaque fois.

On peut montrer rapidement que la formule est fausse en choisissant une valeur de \vrb x telle que \(\denote{\Xlo [\prd{homme}(x) \implq [\prd{infidèle}(x) \vee \prd{volage}(x)]]}^{\Modele}=0\) ; ce sera un individu qui appartient à l'ensemble des hommes mais qui n'appartient ni à l'ensemble des infidèles ni à celui des volages.  Par exemple \Obj{Figr} fait l'affaire.

Donc \(\denote{\Xlo\forall x [\prd{homme}(x) \implq [\prd{infidèle}(x) \vee \prd{volage}(x)]]}^{\Modele}=0\).

La formule correspond à \sicut{tous les hommes sont infidèles ou volages}.

\item \(\Xlo\exists x\exists y [\prd{aimer}(x,y) \wedge \prd{aimer}(y,x)]\)


\(\denote{\Xlo\exists x\exists y [\prd{aimer}(x,y) \wedge \prd{aimer}(y,x)]}^{\Modele}=1\)  ssi on trouve une valeur pour \vrb x telle que 
\(\denote{\Xlo\exists y [\prd{aimer}(x,y) \wedge \prd{aimer}(y,x)]}^{\Modele}=1\).  Et cette sous-formule est vraie ssi on trouve une valeur pour \vrb y telle que 
\(\denote{\Xlo [\prd{aimer}(x,y) \wedge \prd{aimer}(y,x)]}^{\Modele}=1\).

Il faut donc trouver deux individus \Obj x et \Obj y dans \Unv A, tels que \tuple{\Obj x, \Obj y} \emph{et} \tuple{\Obj y, \Obj x} appartiennent à \(\FI(\prd{aimer})\).  Il en existe : par exemple \Obj{Figr} et \Obj{Suzn}.

Donc \(\denote{\Xlo\exists x\exists y [\prd{aimer}(x,y) \wedge \prd{aimer}(y,x)]}^{\Modele}=1\).

Le sens de la formule se retrouve dans le sens de \sicut{il y a des gens qui s'aiment mutuellement}.

\item \(\Xlo[\prd{aimer}(\cnsi a1, \cns s) \implq \exists x\, \neg\prd{aimer}(x,\cnsi a1)]\)

La formule est une implication, donc 
\(\denote{\Xlo[\prd{aimer}(\cnsi a1, \cns s) \implq \exists x\, \neg\prd{aimer}(x,\cnsi a1)]}^{\Modele}=1\) ssi 
l'antécédent \(\Xlo\prd{aimer}(\cnsi a1, \cns s)\) est faux, \emph{ou} (donc si l'antécédent est vrai) le conséquent \(\Xlo\exists x\, \neg\prd{aimer}(x,\cnsi a1)\) est vrai.

L'antécédent est vrai, car \Obj{Almv} aime \Obj{Suzn} dans {\Modele}.  Vérifions donc que le conséquent est vraie aussi.


\(\denote{\Xlo\exists x\, \neg\prd{aimer}(x,\cnsi a1)}^{\Modele}=1\) ssi on trouve au moins une valeur pour \vrb x telle que 
\(\denote{\Xlo\neg\prd{aimer}(x,\cnsi a1)}^{\Modele}=1\), c'est-à-dire telle que 
\(\denote{\Xlo\prd{aimer}(x,\cnsi a1)}^{\Modele}=0\), c'est-à-dire un individu qui n'aime pas \Obj{Almv}.  Cette valeur est facile à trouver : par exemple \Obj{Figr} (en fait tous les individus de \Unv A sauf \Obj{Rosn} feront l'affaire).


Donc \(\denote{\Xlo[\prd{aimer}(\cnsi a1, \cns s) \implq \exists x\, \neg\prd{aimer}(x,\cnsi a1)]}^{\Modele}=1\).

Cette formule correspond à la phrase \sicut{Si Almaviva aime Suzanne, alors quelqu'un n'aime pas Almaviva}.

\item \(\Xlo\neg\exists x [\prd{femme}(x) \wedge \exists y\,\prd{aimer}(x,y)]\)

\sloppy
\(\denote{\Xlo\neg\exists x [\prd{femme}(x) \wedge \exists y\,\prd{aimer}(x,y)]}^{\Modele}=1\) 
ssi on trouve que 
\(\dlb\Xlo\exists x [\prd{femme}(x) \wedge\allowbreak\linebreak[4] \exists y\,\prd{aimer}(x,y)]\color{black}\drb^{\Modele}=0\).

Or  
\(\denote{\Xlo\exists x [\prd{femme}(x) \wedge \exists y\,\prd{aimer}(x,y)]}^{\Modele}=1\) ssi
on trouve une valeur pour \vrb x telle que 
\(\denote{\Xlo[\prd{femme}(x) \wedge \exists y\,\prd{aimer}(x,y)]}^{\Modele}=1\) (et si on ne trouve pas une telle valeur dans \Unv A, alors on aura montré que 
\(\denote{\Xlo\exists x [\prd{femme}(x) \wedge \exists y\,\prd{aimer}(x,y)]}^{\Modele}=0\)).

Essayons avec la valeur \Obj{Rosn} (ou \cns r) pour \vrb x. Dans ce cas \(\Xlo\prd{femme}(x)\) est vrai. Reste à vérifier que \(\Xlo\exists y\,\prd{aimer}(x,y)\) l'est aussi. Or cette sous-formule est vraie également car on trouve bien une valeur «qui marche» pour \vrb y, par exemple \Obj{Almv} car dans \Modele, \Obj{Rosn} aime \Obj{Almv}.  Cela montre donc  
\(\denote{\Xlo\exists x [\prd{femme}(x) \wedge \exists y\,\prd{aimer}(x,y)]}^{\Modele}=1\).  Et par conséquent, 
\(\denote{\Xlo\neg\exists x [\prd{femme}(x) \wedge \exists y\,\prd{aimer}(x,y)]}^{\Modele}=0\).

La formule correspond à quelque chose comme \sicut{aucune femme n'aime quelqu'un} (car elle est la négation de \sicut{il y a au moins une femme qui aime quelqu'un}). 

\item  \(\Xlo\forall x [[\prd{homme}(x) \wedge \exists y
  [\prd{épouse-de}(y,x) \wedge \neg\prd{aimer}(x,y)]] \implq \prd{volage}(x)]\)

Pour montrer que \(\Xlo\forall x [[\prd{homme}(x) \wedge \exists y
  [\prd{épouse-de}(y,x) \wedge \neg\prd{aimer}(x,y)]] \implq \prd{volage}(x)]\)
est vraie dans \Modele, il faut (en théorie) répéter le calcul que 
\(\Xlo [[\prd{homme}(x) \wedge \exists y
  [\prd{épouse-de}(y,x) \wedge \neg\prd{aimer}(x,y)]] \implq \prd{volage}(x)]\)
pour toute les valeurs de \vrb x (c'est-à-dire tous les individus de \Unv A) et trouver $1$ à chaque fois.

Pour les individus qui ne sont pas des hommes, on sait tout de suite que cette implication est vraie car \(\Xlo[\prd{homme}(x) \wedge \exists y
  [\prd{épouse-de}(y,x) \wedge \neg\prd{aimer}(x,y)]]\) est faux (vu que \(\Xlo\prd{homme}(x)\) est faux).

Pour les autres (les hommes, \Obj{Almv}, \Obj{Figr}, \Obj{Chrb}, \Obj{Antn}, \Obj{Bart}), il y a deux cas de figure : soit \(\Xlo \exists y
  [\prd{épouse-de}(y,x) \wedge \neg\prd{aimer}(x,y)]\) est faux, et dans ce cas l'implication est vraie ; soit \(\Xlo \exists y
  [\prd{épouse-de}(y,x) \wedge \neg\prd{aimer}(x,y)]\) et dans ce cas il faut que \(\Xlo\prd{volage}(x)\) soit vrai aussi pour que l'implication soit vraie.

Dans le cas de \Obj{Chrb}, \Obj{Antn} et \Obj{Bart},  \(\Xlo \exists y
  [\prd{épouse-de}(y,x) \wedge \neg\prd{aimer}(x,y)]\) car il n'existe pas de valeur pour \vrb y qui marche (aucun des trois n'apparaît dans \(\FI(\prd{épouse-de})\), i.e aucun n'est marié).  L'implication est donc vraie pour ces trois valeurs de \vrb x.

Dans le cas \Obj{Figr} \(\Xlo \exists y
  [\prd{épouse-de}(y,x) \wedge \neg\prd{aimer}(x,y)]\) est faux aussi, car bien que \Obj{Figr} ait une épouse (\Obj{Suzn}) il est faux qu'il ne l'aime pas.  Donc quand \vrb x est \Obj{Figr}, il n'y a pas de valeur de \vrb y qui marche. Et pour \Obj{Figr}, l'implication est encore vraie.

Enfin pour la valeur \Obj{Almv}, alors \(\Xlo \exists y
  [\prd{épouse-de}(y,x) \wedge \neg\prd{aimer}(x,y)]\)  est vrai, car \Obj{Rosn} est une valeur de \vrb y qui marche. Et il se trouve qui \(\Xlo\prd{volage}(x)\) est vrai aussi car \Modele\ nous dit que \Obj{Almv} est volage.  L'implication est donc vraie pour la valeur \Obj{Almv} de \vrb x.


%Conclusion : 
Donc l'implication \(\Xlo [[\prd{homme}(x) \wedge \exists y
  [\prd{épouse-de}(y,x) \wedge \neg\prd{aimer}(x,y)]] \implq \prd{volage}(x)]\)
est vraie pour toutes les valeurs de \vrb x, ce qui prouve que
la formule f dénote $1$ dans~\Modele. 
%\(\denote{\Xlo\forall x [[\prd{homme}(x) \wedge \exists y [\prd{épouse-de}(y,x) \wedge \neg\prd{aimer}(x,y)]] \implq \prd{volage}(x)]}^{\Modele}=1\).


La formule correspond à \sicut{tout homme qui n'aime pas son épouse est volage}.

\end{enumerate}

\bigskip

%\item

2. 
Voici la nouvelle version du modèle 
 \(\Modele=\tuple{\Unv A,\FI}\) (les modifications sont soulignées) : %marquées en rouge) :

\begin{itemize}\raggedright
\item \(\Unv A=\set{\Obj{Almv}; \Obj{Rosn}; \Obj{Figr}; \Obj{Suzn}; \Obj{Mrcl};
  \Obj{Chrb}; \Obj{Fnch}; \Obj{Antn}; \Obj{Bart}}\) ;

\item \(\FI(\cnsi a1) = \Obj{Almv} ; 
\FI(\cns r) = \Obj{Rosn} ; 
\FI(\cnsi f1) = \Obj{Figr} ; 
\FI(\cns s) = \Obj{Suzn} ; 
\FI(\cns m) = \Obj{Mrcl} ;
\FI(\cns c) = \Obj{Chrb} ; 
\FI(\cnsi f2) = \Obj{Fnch} ; 
\FI(\cnsi a2) = \Obj{Antn} ; 
\FI(\cns b) = \Obj{Bart}\) ;

\item 
\(\FI(\prd{homme})=\set{\Obj{Almv}; \Obj{Figr}; \Obj{Chrb}; \Obj{Antn}; \Obj{Bart}}\) ;
\item 
\(\FI(\prd{femme})=\set{\Obj{Rosn};\Obj{Suzn}; \Obj{Mrcl}; \Obj{Fnch}}\) ;

\item 
\(\FI(\prd{domestique})=\set{\Obj{Figr}; \Obj{Suzn}; \Obj{Fnch}; \Obj{Antn}}\) ;
\item \(\FI(\prd{noble}) = \set{\Obj{Almv}; \Obj{Rosn};  \Obj{Chrb} }\) ;

\item 
\(\FI(\prd{roturier}) = \set{\Obj{Figr}; \Obj{Suzn}; \Obj{Mrcl};
 \Obj{Fnch}; \Obj{Antn}; \Obj{Bart}}\) ;

\item 
\(\FI(\prd{comte}) = \set{\Obj{Almv}}\) ;
\qquad \quad 
\(\FI(\prd{comtesse}) = \set{\Obj{Rosn}}\) ;

\item 
\(\FI(\prd{infidèle}) = \set{\Obj{Almv}}\) ;
\qquad 
%\item 
\(\FI(\prd{volage}) = \set{\Obj{Almv}; \Obj{Chrb}}\) ;

\item 
\(\uline{\FI(\prd{triste}) = \varnothing}\) ;

\item 
\(\FI(\prd{époux-de}) = \set{\tuple{\Obj{Almv}, \Obj{Rosn}};\tuple{\Obj{Figr}, \Obj{Suzn}} ; \uline{\tuple{\Obj{Chrb},\Obj{Fnch}}}}\) ;
\item 
\(\FI(\prd{épouse-de}) = \set{\tuple{\Obj{Rosn},\Obj{Almv}};\tuple{\Obj{Suzn},\Obj{Figr}} ; \uline{\tuple{\Obj{Fnch},\Obj{Chrb}}}}\) ;

\item 
\(\FI(\prd{père-de}) = \set{\tuple{\Obj{Antn},\Obj{Fnch}}}\) ;

\item 
\(\FI(\prd{aimer}) = \set{\tuple{\Obj{Figr},
    \Obj{Suzn}};\tuple{\Obj{Suzn},\Obj{Figr}}; 
\tuple{\Obj{Almv},\uline{\Obj{Rosn}}}; 
\tuple{\Obj{Rosn},\Obj{Almv}}; \tuple{\Obj{Chrb},
    \Obj{Fnch}}; \tuple{\Obj{Fnch},\Obj{Chrb}}; \tuple{\Obj{Mrcl},\Obj{Figr}}}\)
\\
  \uline{(et on a enlevé \tuple{\Obj{Chrb},
    \Obj{Rosn}} et \tuple{\Obj{Chrb}, \Obj{Suzn}})}
\end{itemize}
%\end{enumerate}

\fussy

\end{solu}

\end{exo}




