% -*- coding: utf-8 -*-
\chapter{Sens des phrases et sens des énoncés}
%#############################################
\label{Ch:1}
\Writetofile{solf}{\protect\section{Chapitre \protect\ref{Ch:1}}}
\begin{refsegment}

\section{Objets de l'étude}
%==========================


\subsection{Qu'est-ce que le sens ?}
%-----------------------------------



La sémantique est l'étude du sens\is{sens|sqq} linguistique.  Et, disons-le tout de suite, le sens est mystérieux.  De toutes les disciplines de la linguistique descriptive, la sémantique est probablement celle qui est la moins directement au contact de son objet d'étude. 
Les disciplines comme la phonétique, la phonologie, la morphologie ou la syntaxe sont, d'une manière ou d'une autre, en contact avec la matérialité de la langue ; ce n'est pas le cas de la sémantique. 
À l'aube de la linguistique moderne, \citet{Saussure:16}\Index{Saussure, F. de} définit le signe linguistique\indexs{signe linguistique} par la coexistence d'un signifiant\indexs{signifiant} et d'un signifié\indexs{signifie@signifié} ; le signifiant, l'enveloppe matérielle, est en quelque sorte la face observable du signe alors que le signifié, c'est-à-dire le sens, en est la face cachée.  
Saussure assimile le signifié à la notion de \emph{concept}, et l'on trouve également souvent les termes de \emph{pensée}, \emph{idée} et \emph{contenu}\is{contenu} pour caractériser intuitivement le sens.   Ces notions laissent entendre que le sens renvoie à quelque chose de mental et d'intérieur.  Autrement dit, le sens semble avant tout confiné dans la tête des locuteurs, ce qui n'arrange pas les choses pour l'étudier : il ne s'offre pas immédiatement à l'observation.

Le sens est mental, mais cela ne nous aiderait pas énormément de chercher à le détecter en examinant ce qui se passe dans le cerveau d'un locuteur.  Ce n'est pas une pratique vaine, au contraire : la neurolinguistique nous donne des résultats précieux sur les processus physiologiques à l'\oe uvre dans l'activité langagière et aide  à évaluer des hypothèses d'analyse linguistique, mais elle ne répond pas exactement à la question «qu'est-ce que le sens ?».  
Pour découvrir le sens, une démarche naturelle consiste donc à l'attaquer «de l'extérieur», en l'objectivant, par une médiation : nous ne pouvons pas observer directement les sens, mais nous pouvons observer les objets qui sont censés en être porteurs : les mots, les phrases ou les énoncés de la langue.  Un des enjeux de la description sémantique est ainsi de se donner les moyens 
%et les méthodes 
de détecter ce qui est révélateur du sens dans les énoncés%
\footnote{Voir par exemple \cite{Krfk:11x} pour une présentation des méthodes scientifiques d'investigation du sens.}. 


Il n'est pas inutile ici d'ouvrir une petite parenthèse méthodologique. 
La question «qu'est-ce que le sens ?» %, qui 
est bien la première à laquelle la sémantique doit s'attacher à répondre.
%,  peut encore se compliquer 
Et répondre à cette question revient en quelque sorte à compléter l'équation «sens = ?», où le «~?~» devra être remplacé par une description, suffisamment claire, de cette notion ou ce phénomène que nous appelons \emph{sens}.  
Faire cela, c'est bien donner une \emph{définition} de ce qu'est le sens. 
Mais il y a un important risque de fourvoiement à suivre cette démarche, telle qu'elle est présentée ici. Car elle nous fait partir du mot \emph{sens} et attend que nous en dégagions une définition.  Ce n'est pas ainsi qu'il faut procéder scientifiquement.  Et originellement personne ne s'est jamais dit : «tiens, il y a le mot \emph{sens} dans la langue ; il n'a pas de définition, trouvons-en lui une !». 
Au contraire, la démarche scientifique consiste \emph{d'abord} à identifier empiriquement un certain phénomène (par exemple le phénomène du sens), \emph{puis} à le décrire et le délimiter le plus soigneusement possible, et \emph{enfin} à lui affecter une étiquette, un nom (par exemple le mot \emph{sens} ou \emph{signification}) --~parce que c'est effectivement bien commode de nommer synthétiquement les choses pour pouvoir en parler.  
Par conséquent, savoir si une définition est vraie ou fausse est une question qui ne se pose pas : une définition est par nature arbitraire et, de ce fait, porte en soi le statut d'être nécessairement vraie, en posant (ou imposant) une équivalence entre un nom et une description\footnote{En revanche, on peut juger qu'une définition donnée est bonne ou mauvaise selon que sa formulation est ou non suffisamment claire et objective.}.
Mais l'entreprise de définir le sens se complique particulièrement pour au moins deux raisons. D'une part, comme suggéré ci-dessus, le phénomène du sens (point de départ de la définition) est mystérieux et caché, ce qui rend difficile son observation directe et donc sa description.  D'autre part nous  avons déjà une intuition ferme de ce qu'est le sens (puisque nous connaissons le mot), mais cette intuition est probablement vague parce que les termes \emph{sens} et \emph{signification} sont très polyvalents dans leurs usages ordinaires et peuvent renvoyer à beaucoup de notions différentes. Cela risque éventuellement de biaiser et flouter notre manière d'appréhender le sens.  Finalement, pour dire les choses autrement, cet ouvrage ne va pas véritablement répondre à la question «qu'est-ce que le sens ?», parce qu'elle n'est pas bien posée.  Bien entendu nous allons parler du sens ; dans les lignes qui suivent nous allons esquisser une première proposition de définition et nous en poserons une plus précise au chapitre \ref{LCP}.  Mais ce que nous aurons fait alors, c'est identifier une certaine propriété des énoncés et décider très arbitrairement de nommer cette propriété le sens.  On pourra éventuellement ne pas être d'accord avec cette définition, mais cela ne pourra pas constituer une critique de fond, tout au mieux ce sera une objection sur le choix terminologique et l'usage fait ici du mot \emph{sens}.  Mais cet usage est un de ceux très couramment adoptés en sémantique. 

Pour reprendre maintenant le fil de notre projet, à savoir définir et caractériser le sens en observant ses manifestations à travers les formes linguistiques, je propose, dans un premier temps, de partir du constat suivant : le sens, c'est ce que l'on comprend.\indexs{comprehension@compréhension} Ou, pour être plus précis : les locuteurs ont la faculté de comprendre les énoncés de leur langue, et nous appelons \emph{sens} ce que cette compréhension permet de reconnaître.  Comprendre un énoncé, c'est l'\emph{interpréter}, c'est-à-dire lui assigner un certain sens.
De cette façon nous abordons le sens indirectement, par l'intermédiaire  de la compréhension.  Cependant la compréhension est également un phénomène intimement mental et nous ne pouvons pas non plus l'observer directement, il nous faudra aussi l'objectiver en nous intéressant à ses manifestations extérieures, ses «symptômes».  C'est ce à quoi se consacre la section \ref{s:RelInfer}.  Mais auparavant il est nécessaire de faire un petit point sur ces objets linguistiques porteurs de sens et que nous allons étudier.





\subsection{Phrases, énoncés et contextes}
%-----------------------------------------

Lorsque l'on entend parler de sémantique, il est très courant d'avoir le réflexe de penser %en premier lieu 
initialement
à l'étude du sens des mots\footnote{Il faut préciser que \sicut{mot} n'est pas un terme consacré en théorie linguistique. Il n'a pas de définition nette et précise car il renvoie à différentes réalités. Il est plus approprié de parler d'\emph{unités lexicales}\is{unite lexicale@unité lexicale} ou de \sicut{lexèmes}\is{lexeme@lexème} lorsque l'on réfère à ces entités linguistiques qui composent un lexique ou un vocabulaire, et de \emph{formes lexicales}\is{forme!\elid\ lexicale} pour désigner les emplois concrets des unités lexicales dans des énoncés.}.\indexs{mot} 
Bien évidemment les mots ont un sens (et souvent plusieurs) et l'assemblage de ces sens permet de former le sens d'entités linguistiques plus grandes, les phrases et les discours (nous reviendrons fréquemment sur cette idée).  Il peut donc sembler naturel de se consacrer en premier lieu à la sémantique lexicale.  Mais ce n'est pas ce que nous allons faire ici.
Car, en fait, il n'est pas moins naturel d'amorcer l'étude en s'interrogeant directement sur \emph{le sens des phrases}.\indexs{phrase} D'abord parce qu'il n'est pas très réaliste ni très efficace d'étudier le sens des mots en isolation, il convient mieux de les observer dans leur «milieu naturel» c'est-à-dire leurs emplois dans des phrases.  Et pour ce faire, il est nécessaire d'avoir dès le départ une idée assez claire de ce qu'est le sens d'une phrase.  De même, si nous voulons aborder le sens via le phénomène de la compréhension, nous obtiendrons certainement des résultats tout aussi  précis, fins  et fiables (et peut-être plus) en considérant les manifestations de la compréhension de phrases. 

Nous allons donc, dans ces pages, principalement nous occuper du sens des phrases,  en partant du principe que les phrases sont porteuses d'un sens qui est ce que nous saisissons lorsque nous les comprenons. Mais... est-ce vraiment bien le cas ?  Est-ce que, lorsqu'un locuteur s'exprime, ce sont vraiment ses phrases que nous comprenons ? 
Prenons un exemple.  Supposons qu'après une journée de cours, j'aie dit : 

\ex.  \label{x6h} 
Ce matin, j'ai écrit mon nom au tableau.
%Ce matin, je me suis levé à 6h.  


Un auditeur francophone de cette phrase la comprendra sans problème, c'est-à-dire qu'il est en mesure d'accéder à son sens.  Et il peut le montrer en reformulant ce que j'ai dit, par exemple en relatant : «Laurent Roussarie a inscrit \sicut{Laurent Roussarie} sur le  tableau d'une salle de cours ce mercredi 25 septembre 2013 entre 8h et 12h....».  On devine facilement où je veux en venir.  Car 
si quelqu'un d'autre prononce la phrase \ref{x6h}, l'auditeur ne pourra pas faire exactement le même récit ; autrement dit il aura compris des choses différentes.  Il en va de même si j'avais prononcé cette phrase un autre jour (que le 25 septembre 2013).  Par conséquent il y a de très nombreuses manières différentes de comprendre précisément \ref{x6h}, ce qui peut laisser supposer que \ref{x6h} a de très nombreux sens différents.
Et comme c'est exactement la même phrase qui aura été prononcée à chaque fois, nous devons admettre que considérer
une phrase en soi n'est pas suffisant pour déterminer tout le sens de
ce qui est dit.  Il y a autre chose qui entre en jeu et qui fait la différence entre les diverses manières de comprendre \ref{x6h}.  Cette autre chose est bien sûr le \kwo{contexte d'énonciation}\is{contexte|(}. 
C'est pourquoi il est, en général, plus approprié de s'interroger sur le sens de ce que nous appellerons un \kwo{énoncé}\indexs{enonce@énoncé} plutôt que sur le sens d'une phrase. 
Ce qui distingue un énoncé d'une phrase peut se résumer informellement par l'équation suivante :
 
\ex. \(\text{énoncé} = \text{phrase} + \text{contexte}\)


Un énoncé est une phrase immergée dans son
contexte d'énonciation, c'est-à-dire accompagnée d'informations procurées par le contexte. 
À partir de là nous pouvons commencer à nous faire une idée du processus d'explicitation du sens  en observant le
schéma de la figure \ref{schemaDucrot1}. Ce schéma, repris de
\citet[p.~14]{Ducrot:84}\footnote{J'y renomme quelques éléments pour les harmoniser avec la terminologie employée ici.},\Index{Ducrot, O.}
montre que le contexte est un paramètre à prendre en compte pour caractériser le sens d'une phrase (ou plus exactement de son énoncé dans ce contexte)\footnote{Somme toute, cela ne devra pas nous étonner : nous savons bien que lorsque nous nous interrogeons sur le sens d'un mot ou d'une phrase, la réponse la plus fréquente qui nous vient est que «ça dépend du contexte».}.


%Son point de vue est légèrement différent, mais les idées de bases sont les mêmes. 



 \begin{figure}[h]
 \begin{center}
%\scalebox{.9}{\input{fig/ducrot1.pstex_t}}
\scalebox{.9}{\pstree[treemode=U,arrows=<-,levelsep=1.7cm,nodesepA=3pt,linewidth=1.6pt,linecolor=darkgray]{\TR{sens de $A$ dans $X$}}{%
  \pstree[nodesepB=3pt,nodesepA=0pt]{\TR{\psset{linewidth=1pt}\boitealu{\sffamily\begin{tabular}{c}Description sémantique\\d'une langue \textit{L}\end{tabular}}}}{%
\TR{phrase $A$} %\TR{énoncé $A$}
\TR{contexte $X$}} %\TR{circonstances $X$}}
}}
 \end{center}
\caption{Schéma d'interprétation (d'après \citealt{Ducrot:84})}\label{schemaDucrot1}
 \end{figure}



Nous devons ici faire plusieurs remarques sur les notions d'énoncé, de phrase et de contexte.  Un énoncé, comme indiqué ci-dessus, est  une occurrence particulière d'une phrase.  On considère généralement aussi qu'un énoncé est une production linguistique qui se suffit à elle-même d'un point de vue communicationnel (c'est-à-dire pour faire passer un certain message).  De fait, il existe des énoncés qui ne correspondent pas à des phrases \alien{stricto sensu}, grammaticalement parlant.  Ça peut être le cas, par exemple, de
\sicut{merci}, \sicut{sapristi !}, \sicut{aïe !}, \sicut{espèce d'andouille !}, \sicut{encore une fois}...
Le schéma de la figure~\ref{schemaDucrot1} doit donc être généralisé en laissant $A$ représenter non seulement des phrases mais aussi toute forme linguistique, simple ou complexe, pouvant donner lieu à  un sens.  J'en profite pour introduire un élément de terminologie supplémentaire : dans cet ouvrage j'utiliserai fréquemment le terme d'\kwo{expression linguistique}\indexs{expression!\elid\ linguistique} pour désigner indifféremment les phrases, les syntagmes et les mots (et, le cas échéant, les séquences de phrases ou discours). 

Une phrase est une entité abstraite définie par et au sein de la grammaire, selon des critères notamment syntaxiques.  À cet égard, \ref{x6h} est une certaine phrase, unique en soi, de la langue française.
Et il lui correspond virtuellement une infinité d'énoncés
différents  puisqu'un énoncé est une certaine instance contextualisée d'une phrase (ou d'une expression)%
\footnote{Cf. par exemple \citet[p.~174]{Ducrot:84}.\Indexn{Ducrot}}.
À chaque fois qu'une même phrase est prononcée ou écrite (voire lue) elle donne lieu à un nouvel énoncé.

Par contexte d'énonciation,\is{contexte} il faut entendre la situation du monde dans laquelle se produit l'acte d'énonciation.  Un peu comme on parle du contexte historique d'un événement par exemple, cela recouvre une quantité d'entités, informations et circonstances qui entourent et environnent cet acte.   
Parmi tous ces éléments, les plus notables et influents dans la détermination du sens des énoncés sont ceux qui nous renseignent sur les points suivants :

\begin{itemize}[before*=\raggedright]  
\item qui parle ? (protagoniste que nous nommerons le \kw{locuteur}\indexs{locuteur}) 
\item à qui s'adresse-t-il ? (protagoniste que nous nommerons l'\kw{allocutaire})
\item où  et quand se passe l'énonciation ?
\item quels autres objets, personnes ou entités saillants sont présents dans la situation ?
\item quelles informations et connaissances les interlocuteurs partagent (ou pensent partager) ?
\item qu'est-ce qui a été dit ou écrit précédemment dans la conversation ou le discours ?
\end{itemize}

\smallskip

La liste n'est certainement pas exhaustive mais elle montre que le contexte se présente comme un ensemble de paramètres nécessaire à la compréhension d'une phrase et qui 1) ne sont pas présents dans la phrase et 2) sont ancrés, d'une manière ou d'une autre, à une certaine situation (spatio-temporelle) concrète.  Et nous voyons ainsi en quoi certains de ces paramètres jouent un rôle déterminant dans la compréhension précise de notre exemple \ref{x6h}.
Et cela est dû au fait que cette phrase comporte des expressions spéciales, notamment \sicut{ce}, \sicut{je}, \sicut{mon}, qui, pour être interprétées, nous demandent de consulter directement le contexte; les expressions qui entretiennent ce type de lien étroit avec le contexte sont appelées des \kwo{déictiques}\is{deictique@déictique} ou \kwo{indexicaux}.\is{indexical}


Nous avons donc établi que le sens d'une phrase dépend du contexte dans lequel elle est énoncée (\ie\ produite).  Est-ce à dire qu'une phrase en soi, hors de tout contexte spécifique, n'a pas de sens ?  Certaines positions philosophiques et linguistiques en défendent l'idée, mais nous ne suivrons pas ici cette extrémité.  D'abord que le sens \emph{dépende} du contexte n'implique nullement que le contexte soit le seul élément producteur de sens.  La proposition illustrée en figure~\ref{schemaDucrot1} suggère qu'il y a au moins deux ingrédients qui interviennent dans l'explicitation du sens : le contexte $X$ mais aussi la phrase $A$ elle-même.  Par ailleurs il existe des phrases que nous comprenons parfaitement sans avoir à consulter le contexte. C'est le cas par exemple des phrases \ref{x:357deg} :


\ex.  \label{x:357deg}
\a. Le mercure bout à 357\,\degres C.
\b. Louis \textsc{xiv} est mort à l'âge de 76 ans.
\b. Alice poursuit un lapin blanc.

Peu importe dans quelles conditions sont énoncées ces phrases, il semble que nous comprendrons chacune d'elles toujours de la même manière.
C'est donc qu'elles sont capables par elles-mêmes de véhiculer un sens.

\largerpage[-1]
Et méthodologiquement, s'il existe virtuellement une infinité de contextes pouvant accueillir une infinité d'énoncés différents d'une phrase donnée, nous serons bien obligés de faire un peu abstraction du contexte pour nous lancer dans l'étude du sens sans nous retrouver immobilisés dans une surabondance de cas particuliers. 

Nous allons donc admettre qu'une phrase, en tant qu'objet abstrait de la grammaire, possède un sens. 
Simplement parfois ce sens sera partiel, incomplet. Par exemple le sens de la phrase \ref{x6h} sera l'\emph{invariant} ou le \emph{dénominateur commun} que l'on retrouve dans la compréhension de tous ses énoncés, quel que soit le contexte. 
Ensuite, pour compléter la détermination du sens d'un énoncé, on doit faire intervenir le contexte dans le processus. 


Cette idée est illustrée dans le second schéma d'interprétation que propose Ducrot
(figure~\ref{schemaDucrot2}) et qui montre que la «grammaire du sens» se divise en deux composants que Ducrot appelle \emph{linguistique} et \emph{rhétorique}.
L'interprétation procède ainsi en deux passes : d'abord on dégage un premier sens (général) $A'$ propre à la phrase, puis on élucide un sens plus complet et précis de l'énoncé en tenant compte des informations fournies par le contexte $X$. 


 \begin{figure}[h]
 \begin{center}
%\scalebox{.9}{\input{fig/ducrot2.pstex_t}}
\scalebox{.9}{\pstree[treemode=U,arrows=<-,levelsep=1.7cm,nodesepA=3pt,linewidth=1.6pt,linecolor=darkgray]{\TR{sens de $A$ dans $X$}}{%
  \pstree[nodesepB=3pt,nodesepA=0pt]{\TR{\psset{linewidth=1pt}\boitealu{\sffamily\begin{tabular}{c}Composant rhétorique\\(pragmatique)\end{tabular}}}}{%
  \pstree[nodesepB=0pt,nodesepA=3pt]{\TR{$A'$}}{%
     \pstree[nodesepB=3pt,nodesepA=0pt]{\TR{\psset{linewidth=1pt}\boitealu{\sffamily\begin{tabular}{c}Composant linguistique\\(sémantique)\end{tabular}}}}{\TR{phrase $A$}}}
\skiplevels{2}\TR{contexte $X$}\endskiplevels}}}
 \end{center}
\caption{Schéma d'interprétation, seconde version \citep{Ducrot:84}}\label{schemaDucrot2}
 \end{figure}


Pour employer une terminologie plus standard, nous renommerons les composants linguistique et rhétorique de Ducrot en, respectivement, composants \emph{sémantique} et \emph{pragmatique}.  La pragmatique,\indexs{pragmatique} comme la sémantique, étudie le sens linguistique (et d'ailleurs la délimitation des deux disciplines n'est pas toujours absolument franche) ; par convention on considère généralement que la pragmatique se concentre sur le sens des énoncés \emph{en contexte} et \emph{en usage}.
Dans ces pages nous nous consacrerons principalement au $A'$ du schéma, c'est-à-dire, en quelque sorte, le «sens sémantique» directement attaché aux phrases.  C'est une démarche qui, au vu de l'architecture du schéma, est ordonnée puisque le composant pragmatique opère après et sur les résultats issus du composant sémantique.

\is{contexte|)}





\section{La compréhension comme inférence}
%=========================================
\label{s:RelInfer}



J'ai proposé que nous pouvions entreprendre une investigation du sens
via la compréhension, en soulignant cependant que  la compréhension est également un
phénomène intimement mental. Là encore nous devons essayer de
l'aborder de l'extérieur. %, en tentant d'être suffisamment objectif.
Par exemple, nous pouvons nous interroger sur les symptômes de la
compréhension, en nous demandant ce qui peut manifester le fait qu'un
auditeur a compris une phrase ou un énoncé.  Comprendre une phrase (ou
un énoncé) c'est, entre autres, être capable de raisonner dessus.
Autrement dit, c'est être capable d'en tirer des conclusions, d'en
faire un résumé, de la reformuler, de la qualifier, de la comparer, l'opposer à
d'autres, etc.\ ce que nous allons désigner par le terme général d'\emph{inférences}.\indexs{inference@inférence}
Et là encore, nous n'allons pas nous intéresser
directement aux conclusions mentales, aux idées ou pensées ni
même aux réactions cérébrales que déclenchent une phrase.
Heureusement, ces conclusions peuvent à leur tour être exprimées sous
forme de phrases, qui sont, elles, directement observables.  Ce que nous
allons donc faire, c'est nous intéresser à des \emph{propriétés} et des \emph{relations} qui manifestent ces inférences.  Il s'agit de propriétés et de relations de sens qui s'établissent sur ou entre des phrases et qui, lorsqu'elles sont précisément identifiées, peuvent nous guider dans la découverte et la caractérisation du sens des phrases concernées.





En guise d'exemple et d'illustration, nous allons d'abord envisager une
relation assez simple : la \kw{paraphrase}.  
C'est  certainement une des relations les plus intuitives, et nous l'abordons ici  sous un angle informel, 
afin de nous faire une idée de ce qu'est une relation de sens
entre phrases.
La paraphrase, c'est la synonymie au niveau de la phrase.  Nous dirons donc
que deux phrases sont en relation de paraphrase (ou qu'une phrase est
une paraphrase d'une autre), si elles ont le même
sens.\label{d:paraphrase}  Voici quelques exemples de paires de phrases qui peuvent être jugées des paraphrases l'une de l'autre :



\ex.\label{xparaph1}
\a.  Anne a acheté une lampe au brocanteur.
\b. Le brocanteur a vendu une lampe à Anne.

\ex.\label{xparaph2}
\a.  La glace  fond à 0\,\degres C.
%\b. L'eau gèle à 0\degres C.
\c. La température de fusion de l'eau est de 0\,\degres C.

\ex.\label{xparaph2.5}
\a. Tous les étudiants ont eu la moyenne à l'examen.
\b. Aucun étudiant n'a été recalé à l'examen.

\ex.\label{xparaph3}
\a. Vous savez bien que l'existence précède l'essence.
\b. Vous n'ignorez pas que l'existence précède l'essence.




Mais il y a un petit problème méthodologique ici : nous risquons de tourner en rond, en inversant la prémisse et la conclusion.  
La définition de la
paraphrase esquissée ci-dessus présume qu'on se fasse \emph{a priori} une idée du sens des
phrases.  En effet, nous ne disons pas : si deux phrases sont des paraphrases, alors nous en concluons qu'elles ont le même sens ; 
nous disons : si elles ont le même sens, alors ce
sont des paraphrases.  
Autrement dit, établir que deux phrases sont des paraphrases ne nous \emph{apprend} pas grand chose sur leur sens, si ce n'est ce que nous savons déjà : qu'elles veulent dire la même chose.

\largerpage[-1]

Nous pourrions contourner ce problème en estimant que nous avons, en tant que locuteurs, une intuition forte et fiable de la relation de paraphrase, qui nous vient naturellement sans avoir à nous interroger consciemment et précisément sur le sens des phrases (et c'est peut-être ce qui s'est produit quand nous avons lu 
les exemples \ref{xparaph1}--\ref{xparaph3}).  Dans ce cas, la conclusion que des phrases ont le même sens pourrait être vue comme un gain dans l'observation. Mais scientifiquement cette démarche a ses limites, car justement l'intuition n'est, par nature, pas entièrement fiable.  Que se passe-t-il si un observateur qui a la ferme intuition que les deux phrases de \ref{xparaph1} ou de \Next\ sont de bonnes paraphrases fait face à un autre observateur qui a la ferme intuition du contraire ? Comment départager raisonnablement deux intimes convictions ?  Dans le simple affrontement d'une opinion contre une autre, il n'y a pas de vainqueur honnête.

\ex.
\a. La glace fond à 0\,\degres C.
\b. L'eau gèle à 0\,\degres C.


Ce dont nous avons besoin c'est d'étayer nos jugements au moyen d'\emph{arguments} et de \emph{justifications} suffisamment univoques et objectifs.  Et pour ce faire, il nous faut travailler sur des définitions de relations elles-mêmes univoques et objectives, et si possible qui n'anticipent pas sur la notion de sens. 






\subsection{La conséquence logique}
%---------------------------------
\label{s:conseql}
\indexs{consequence logique@conséquence logique}

La \kwo{conséquence logique}\indexs{consequence logique@conséquence logique} 
n'est peut-être pas la relation la plus spectaculaire, mais elle a le mérite d'être
solidement et objectivement définie, à l'aide de la logique (comme son nom l'indique).  Sa définition est la suivante :


\begin{defi}[Conséquence logique]%
\label{d:conseq}
\indexs{consequence logique@conséquence logique}
Une phrase $B$ est une conséquence logique d'une phrase $A$,
ssi%\footnotemark\ 
\emph{à
chaque fois} que $A$ est vraie, alors $B$ est vraie aussi.
\end{defi}%
%\addtocounter{footnote}{-1}
%\footnotetext{J'utiliserai cette abréviation pour \sicut{si et seulement si} tout au long de l'ouvrage.}%
%\addtocounter{footnote}{1}%

\largerpage

Si $B$ est une conséquence logique de $A$, nous dirons alors que $A$ \kwo{entraîne}\footnote{Le terme anglais usuel
est \emph{entail} et la conséquence logique est appelée \alien{entailment}.  En français on peut également dire «~$A$ implique $B$~», mais nous éviterons généralement de le faire ici car
nous réserverons cette terminologie pour un autre emploi.} $B$.\indexs{implication!\elid\ logique|see{conséquence logique}}



\begin{nota}
Si $A$ entraîne $B$, nous écrirons : $A\satisf B$.
\end{nota}

Voici quelques exemples en guise d'illustration.  Dans chaque groupe
de phrases, la phrase (a) entraîne chacune des phrases qui suit.  


\ex.\label{xbag}
\a. L'objet qui est dans la boîte  est une bague en or. \label{xbaga}
\b. L'objet qui est dans la boîte  est en or. \label{xbagb}
\b. L'objet qui est dans la boîte  est une bague. \label{xbagc}

\ex.\label{xkiss}
\a.  Kim a embrassé Elise avec fougue.
\b. Kim a embrassé Elise.
\c. Elise a été embrassée.
\d. Quelqu'un a embrassé Elise.
\e. Kim a fait quelque chose (à Elise).

\ex.
\a.  Alice a marché sur un schtroumpf.
%\b. Il existe (au moins) un schtroumpf. (les schtroumpfs existent)
\b. Les schtroumpfs existent.


\ex.\a.  Bob est marié.
\b. Bob n'est pas pas célibataire.

\ex.\label{x12etu}
\a.  Douze étudiants ont eu la mention TB.\label{x12etua}
\b. Trois étudiants ont eu la mention TB.\label{x12etub}


\sloppy 

Afin de voir comment la définition~\ref{d:conseq} s'applique ici, 
pour chaque groupe d'exemples, commencez par imaginer ou supposer que
la phrase (a) est vraie ; sous cette hypothèse vous ne pouvez alors pas faire autrement que
d'admettre que les phrases suivantes sont vraies aussi.  
% Il y a donc bien une relation de conséquence logique de (a) vers ses suivantes.
Et c'est bien le cas, y compris pour \ref{x12etu} qui peut à première vue sembler un peu contre-intuitif.  En effet si nous supposons qu'il y a effectivement douze étudiants qui ont eu la mention TB, alors \ref{x12etub} ne peut pas être fausse car cela voudrait dire qu'il n'est pas possible de trouver trois étudiants qui ont la mention TB. Mais parmi les douze c'est très facile d'en extraire trois.  Autrement dit, \ref{x12etub} est forcément vraie sous l'hypothèse \ref{x12etua}\footnote{Nous reviendrons plus longuement sur l'interprétation des numéraux en \S\ref{ss:implicatures} car, bien sûr, l'histoire n'est pas finie.}. 

\fussy

Notons que pour «sentir» une conséquence logique entre deux
phrases, nous pouvons nous aider en les enchaînant avec le connecteur  \sicut{donc}\footnote{Mais attention, ceci n'est qu'une aide, l'emploi de \sicut{donc} n'a pas
valeur de démonstration.} ; ainsi  \sicut{donc} peut être une
manière de prononcer le symbole $\satisf$.

L'avantage de cette relation est qu'elle échappe au cercle vicieux
aperçu avec la paraphrase.  Il s'agit bien d'une relation de sens,
dans la mesure où elle s'avère caractéristique de la compréhension
que l'on a de la phrase qui en entraîne une autre, mais la définition
de la relation ne fait pas intervenir la notion de sens.  Elle
s'appuie à la place sur la notion de \kwo{vérité}\indexs{verite@vérité} d'une phrase.  On
pourra arguer que pour être capable de déterminer si une phrase est
vraie ou fausse, il faut en connaître le sens (et c'est exactement
cette vision des choses que nous adopterons dans cet ouvrage, nous y
reviendrons en détail dans le chapitre~\ref{LCP}).  Pour autant, il
n'en demeure pas moins que l'on peut appliquer la
définition~\ref{d:conseq} aux phrases \ref{xbag}--\ref{x12etu} sans
s'interroger \emph{a priori} sur leur sens et surtout sans partir d'une idée
préconçue de ce que cela pourrait être.  
Empiriquement, la
conséquence logique semble plus immédiatement et objectivement accessible que le sens.


\newpage
%\largerpage[-1]

Mentionnons dès à présent un point très important de la théorie (nous
y reviendrons avec insistance dans les chapitres suivants) et qui 
apparaît dans la définition via le «à chaque fois que».  Nous
aurions pu dire aussi «dans tous les cas où».  Ces «fois»
ou ces «cas» correspondent à ce que nous pouvons appeler des
\emph{circonstances}, et il importe de toujours avoir à l'esprit
qu'une phrase est rarement vraie ou fausse dans l'absolu, mais que sa
vérité ou sa fausseté \emph{dépend} de certaines circonstances.  Cette
importance des circonstances qui «accompagnent» une phrase doit
nous faire penser à ce qui a été évoqué dans la section précédente sur
le rôle du contexte.\is{contexte}  Il nous faudra beaucoup avancer dans la théorie
pour montrer précisément que les circonstances évoquées ici ne
coïncident pas exactement avec les circonstances qui définissent le
contexte d'énonciation et qui ont été présentées précédemment (ceci
sera abordé au chapitre~\ref{Ch:contexte}, vol.~2).  Pour l'instant, disons
simplement qu'il conviendra de faire la distinction entre, d'une part, les circonstances par
rapport auxquelles on peut dire si une phrase est vraie ou fausse et, d'autre part, les
circonstances qui permettent de déterminer ou de compléter le sens
d'un énoncé.  Les premières n'influent pas sur le sens ; les secondes
constituent les paramètres que nous avons regroupés sous le chef de
contexte d'énonciation. 


Maintenant, que peut nous apprendre une conséquence logique sur le sens
des phrases qu'elle relie ?  
Nous sommes en train de faire l'hypothèse que
c'est le sens des phrases
qui est responsable de leur relation de conséquence.  
D'ailleurs nous nous en apercevons,
au moins informellement, par le fait que nous pouvons dire, lorsque $A$
entraîne $B$,
%% ($A\satisf B$), 
que $A$  «signifie», entre
autres, $B$ (sachant bien sûr que $A$ ne signifie pas que
cela), ou encore que ce qui est «dit» dans  $B$ est déjà dit
dans  $A$.  Et ce qui est dit dans une phrase, c'est son
contenu, la pensée qu'elle véhicule conventionnellement, c'est-à-dire
très probablement son sens.  Nous pouvons alors conclure que si $A$
entraîne $B$, alors tout le sens de $B$ est inclus, d'une manière ou d'une autre, dans le sens de
$A$.
%
%
$B$ peut donc se présenter comme un extrait du sens de $A$.  Et
donc si nous parvenons à obtenir suffisamment de 
conséquences logiques 
pour une phrase donnée, cela peut, dans certains cas, nous permettre
de décomposer le sens 
de cette phrase en unités plus petites et plus simples.
Inversement, si le sens est responsable des conséquences logiques, alors une description sémantique adéquate d'une phrase doit être en mesure de rendre compte et de prédire toutes ses conséquences logiques.

La conséquence logique peut aussi parfois aider à établir que deux phrases n'ont pas le même sens. Supposons, par exemple, que $A$ et $B$ nous semblent assez proches sémantiquement, au point que nous serions prêts à les qualifier d'assez bonnes paraphrases l'une de l'autre.  Si cependant nous trouvons une phrase $C$ qui est une conséquence de $A$ mais pas de $B$, alors nous pourrons conclure que $A$ et $B$ n'ont pas exactement le même sens.

Enfin c'est sur la conséquence logique entre phrases que se définissent les relations sémantiques lexicales d'hyperonymie\indexs{hyperonymie} et hyponymie\indexs{hyponymie} \citep[cf.][p.~89]{Cruse:86}.\Index{Cruse, D. A.} Un hyponyme est une unité lexicale plus spécifique que son hyperonyme qui lui est plus général, comme par exemple \sicut{truite} par rapport à \sicut{poisson}, qui se vérifie par le fait que \sicut{j'ai mangé une truite} $\satisf$ \sicut{j'ai mangé un poisson}.  Cela confirme l'idée qu'il n'est pas déraisonnable d'étudier le sens des phrases en amont de celui des mots. 



\medskip %\smallskip

S'il est important de savoir reconnaître une conséquence logique
entre deux phrases, il est tout aussi important d'être bien avisé de ce qui
n'en est pas une.  À cet effet, donnons-nous d'abord
un élément de notation.  

\begin{nota}
S'il n'y a pas de relation de conséquence logique de $A$ vers
$B$, nous écrirons $A \not\satisf B$.
\end{nota}



Mais qu'est-ce que cela signifie que $B$ n'est pas une conséquence
de $A$ ?  Quand peut-on établir $A \not\satisf B$ ? Tout simplement
lorsque la définition~\ref{d:conseq} ne s'applique pas ; c'est-à-dire
s'il existe au moins un cas de figure où $A$ est vraie et $B$ est fausse
(car alors il n'est pas vrai que dans \emph{tous} les cas où $A$ est vraie
$B$ l'est aussi).  Un tel cas, s'il existe,  s'appelle un
\kw{contre-exemple} à la conséquence logique et il suffit à lui seul à
montrer que $A$ n'entraîne pas $B$.

Cela permet de faire une distinction franche entre ce qui est
rigoureusement une conséquence logique et ce qui est une simple
conjecture.  Ainsi \ref{xalbb} n'est pas une conséquence logique de
\ref{xalba}. 

\ex.\label{xalb}
\a. Albert est un homme. \label{xalba}
\b. Albert a deux bras. \label{xalbb}

Nous pouvons facilement lui  trouver un contre-exemple :  un cas de
figure où Albert est un homme manchot ; \ref{xalba} est alors vraie mais
\ref{xalbb} est fausse.  
%C'est aussi pour cela que \ref{xkiss:e}
%peut sembler un peu limite.



La notion de contre-exemple est particulièrement précieuse car nous en
tirons une efficace méthode  de vérification (ou d'infirmation, le cas
échéant) d'une 
relation de conséquence logique.\label{p.contrex} % entre phrases.
La méthode est la suivante :  pour vérifier si une phrase $B$ est bien une
conséquence d'une phrase $A$, nous cherchons un \emph{contre-exemple},
c'est-à-dire : nous essayons d'imaginer une situation pour laquelle $A$ vraie et $B$ est fausse. Si nous y parvenons, c'est que $A\not\satisf B$  ; et si au contraire cela nous conduit forcément à une incohérence ou une contradiction, c'est que $A \satisf B$.

Et il y a un autre point qui mérite notre attention ici.  C'est qu'il
ne suffit pas d'extraire un «bout» de phrase pour obtenir une
conséquence logique.  Regardons les exemples suivants :

\ex. \label{x:celmout}
\a. Jean croit que c'est le colonel Moutarde le coupable. \label{x:celmouta} 
\b. C'est le colonel Moutarde le coupable. \label{x:celmoutb}

\ex. \label{x:cachalot}
\a.  Cet animal est un petit cachalot. \label{x:cachalota}
\b. Cet animal est un cachalot. \label{x:cachalotb}
\c. Cet animal est petit. \label{x:cachalotc}



Pour \ref{x:celmout}, c'est assez simple. Imaginons une situation où
un crime a eu lieu et où nous \emph{savons} que le colonel Moutarde est
innocent (car nous sommes bien informés), et dans cette
situation, Jean croit à tort que le colonel est coupable. Evidemment
alors \ref{x:celmouta} est vraie, mais \ref{x:celmoutb}  est
fausse. C'est un contre-exemple.
Avec \ref{x:cachalot}, c'est un peu différent. Certes
\ref{x:cachalota} $\satisf$ \ref{x:cachalotb}.  En revanche, dans une
situation où \ref{x:cachalota} est vraie (par exemple nous sommes
devant un cachalot pygmée, à peu près de la taille d'un hippopotame),
il est tout à fait légitime d'admettre que \ref{x:cachalotc} est
fausse (à partir du moment où l'on considère qu'un petit animal est
plutôt du coté des souris par exemple, voire des
insectes)\footnote{Nous reviendrons sur ce type de phénomène dans les
  chapitres \ref{Ch:t+m} et \ref{Ch:adj} (vol.~2).}. 

\smallskip

% -*- coding: utf-8 -*-
\begin{exo}\label{exo:1CL}
Pour chaque paire de phrases suivantes,\pagesolution{crg:1CL}
dites, en le démontrant, si la phrase (b) est ou non une conséquence logique de la phrase (a).
\begin{enumerate}
\item 
  \begin{enumerate}
  \item Ta soupe est chaude.
  \item Ta soupe n'est pas froide.
  \end{enumerate}
\item 
  \begin{enumerate}
  \item Ta soupe est chaude.
  \item Ta soupe n'est pas brûlante.
  \end{enumerate}
\item 
  \begin{enumerate}
  \item Joseph a prouvé que c'est le colonel Moutarde le coupable.
  \item C'est le colonel Moutarde le coupable.
  \end{enumerate}
\item 
  \begin{enumerate}
  \item Des étudiants ont eu la moyenne au partiel. 
  \item Des étudiants n'ont pas eu la moyenne au partiel. 
  \end{enumerate}
\end{enumerate}
\begin{solu}(p.~\pageref{exo:1CL})\label{crg:1CL}

\sloppy

Nous allons démontrer les réponses en appliquant la méthode des contre-exemples (\S\ref{s:conseql}, p~\pageref{p.contrex}):  pour chaque paire de phrases, nous essayons d'imaginer une situation par rapport à laquelle (a) est vraie \emph{et} (b) est fausse.

\fussy
\begin{enumerate}
\item 
La situation serait telle que la soupe est chaude \emph{et} la soupe est froide (puisqu'il est faux qu'elle n'est pas froide). C'est contradictoire, et nous en concluons que (a) $\satisf$ (b).
\item 
Ici la situation serait telle que la soupe est chaude et brûlante. C'est tout à fait possible (si la soupe est précisément brûlante) puisque quelque chose de brûlant est forcément chaud. C'est un contre-exemple, et donc (a) $\not\satisf$ (b).
\item 
Si nous sommes dans une situation où le colonel Moutarde \emph{n'est pas} coupable, alors on ne peut pas \emph{prouver} qu'il est coupable. À la rigueur, Joseph pourrait argumenter ou soutenir (par erreur ou malhonnêteté) que le colonel est coupable, mais objectivement nous n'aurons pas le droit de dire qu'il a \emph{prouvé} la culpabilité.  Par conséquent (a) $\satisf$ (b).
\item 
Ici il faut être vigilant : une situation où (b) est fausse  (il est faux que des étudiants n'ont pas eu la moyenne au partiel) est telle que \emph{tous} les étudiants ont eu la moyenne.  Dans une telle situation, (a) peut-elle être vraie ? Oui, nécessairement, car si (a) était fausse cela voudrait dire qu'aucun étudiant n'a eu la moyenne (ce qui serait contradictoire avec la première hypothèse).  Nous pouvons avoir une situation où (a) est vraie et (b) fausse, et donc (a) $\not\satisf$ (b).
\\
NB : le fait que nous comprenons souvent (a) comme s'accompagnant aussi de la vérité de (b) ne relève pas de la conséquence logique mais d'une implicature conversationnelle (cf. \S\ref{ss:implicatures}).
\end{enumerate}
\end{solu}
\end{exo}




\subsection{Les équivalences}
%----------------------------
\label{ss:EquivalenceLogique}
Revenons aux paraphrases.  Nous  avons vu que la définition donnée en
page~\pageref{d:paraphrase} était
un peu  circulaire (du moins pour l'objectif que nous poursuivons
ici).  Avec la 
conséquence logique, nous pouvons donner une définition plus rigoureuse,
même si elle risque d'être un peu plus restrictive.  D'ailleurs nous
ne reprendrons pas ici le terme de paraphrase ; nous parlerons, à la
place, d'\kwo{équivalence logique}\indexs{equivalence@équivalence!\elid\ logique}.

\begin{defi}[Équivalence logique (1)]
Deux phrases $A$ et $B$ sont dites (logiquement) équivalentes ssi elles s'entraînent mutuellement, {\ie} ssi $A
\satisf B$ et $B \satisf A$.
\end{defi}

Deux phrases logiquement équivalentes peuvent
({a priori} théoriquement\footnote{En fait c'est inexact : les phrases peuvent
avoir des propriétés syntaxiques ou pragmatiques différentes qui les
rendent plus ou moins naturelles dans un discours ou une discussion
donnés.  Nous ne regardons ici que l'effet sémantique de la relation
d'équivalence.}) être remplacées l'une par l'autre dans un discours ou
une conversation en préservant la vérité de ce qui est dit ;
c'est-à-dire qu'elles
disent la même la
chose, et nous en déduirons qu'elles ont le même sens.  Mais là
encore, puisque nous utilisons la conséquence logique pour la définir,
la relation d'équivalence s'offre à l'observation sans être
directement tributaire d'une définition {a priori} du sens.  

A ce propos, la définition ci-dessus est un peu indirecte, découlant
de la conséquence.  Nous pouvons donner une définition plus primaire,
qui reflète mieux les propriétés de l'équivalence.

\begin{defi}[Équivalence logique (2)] \label{d:EquLog}
Deux phrases $A$ et $B$ sont dites (logiquement) équivalentes,  ssi
dans tous les cas où $A$ est vraie, $B$ est vraie aussi, et dans tous
les cas où $A$ est fausse, $B$ est fausse aussi\footnotemark.
\end{defi}%
\footnotetext{Une autre manière de le dire est : dans tous les cas où $A$ est vraie, $B$ est vraie
  et dans tous les cas où $B$ est vraie, $A$ est vraie aussi.  Cela
  revient exactement au même.}

La conséquence logique ne prend pas en compte les cas où $A$ est
fausse. Lorsque nous avons $A\satisf B$, nous pouvons trouver des cas où $A$ est
fausse et $B$ vraie, c'est juste qu'ils ne sont pas pertinents pour
la relation.  Nous voyons donc que l'équivalence logique, quant à elle,  est une relation
plus «étroite», car elle pose plus de contraintes dans sa
définition : elle regarde aussi les cas où $A$ est fausse.
De là nous pouvons tirer un corollaire.




\begin{coro}
$A$ et $B$ sont  (logiquement) équivalentes ssi on ne peut
jamais avoir \emph{dans les mêmes circonstances} $A$ vraie et $B$ fausse (ou
inversement).
\end{coro}


Autrement dit, si $A$ et $B$ sont logiquement équivalentes, alors elles sont soit toutes les deux vraies en même temps, soit toutes les deux fausses en même temps (selon les circonstances par rapport auxquelles nous les jugeons). 
Cela nous conduit à nous intéresser à une autre relation, qui manifeste le comportement inverse, la \emph{contradiction}.


\begin{exo}\label{exo:1EqLog}
Vérifiez si les exemples données en \ref{xparaph1}--\ref{xparaph3}
sont bien des équivalences logiques.
\begin{solu} (p. \pageref{exo:1EqLog})

La méthode consiste à appliquer la définition \ref{d:EquLog} et son corollaire donnés à la p.~\pageref{d:EquLog} en cherchant un contre-exemple à l'équivalence, c'est-à-dire un scénario par rapport auquel on pourra juger que l'une des deux phrases est vraie et l'autre fausse.  Si nous y arrivons, alors c'est que les deux phrases ne sont pas équivalentes; sinon nous concluons qu'elles le sont probablement.  Il faut noter qu'en toute rigueur,  cette conclusion  ne peut être que provisoire\footnote{Comme toute conclusion scientifique sérieuse.}, car si nous ne trouvons pas de contre-exemple, cela ne veut pas nécessairement dire qu'il n'en existe pas, mais simplement que nous n'avons peut-être pas assez cherché ou pas encore trouvé.

\ex.[\ref{xparaph1}]
\a.  Anne a acheté une lampe au brocanteur.
\b. Le brocanteur a vendu une lampe à Anne.

Peut-on imaginer un scénario où la phrase (a) est vraie et (b) fausse? 
Une piste serait d'envisager qu'Anne a effectué l'achat mais que le brocanteur, lui, n'a rien fait dans l'histoire.  Par exemple, si Anne a acheté la lampe sur internet via un site de vente par correspondance... Mais pour autant, même dans ce cas, peut-on vraiment dire que (b) est fausse?  Si, un peu plus tard, le brocanteur consulte son compte sur le site, constatant la transaction, il pourrait dire : «Ah super, aujourd'hui j'ai vendu une lampe».  Si l'on accepte cette possibilité, alors on n'aura pas prouvé qu'il s'agit là d'un contre-exemple. Inversement, pour avoir (b) vraie et (a) fausse, il faudrait s'orienter vers un scénario où Anne n'a rien fait ou éventuellement qu'elle ne s'est pas rendu compte que l'achat a eu lieu. Mais là aussi, on imagine mal comment le brocanteur aura réussi à vendre la lampe ou comment l'opération ne pourra pas être qualifiée d'achat en ce qui concerne Anne. À ce stade, nous conclurons (au moins provisoirement) que ces phrases semblent bien équivalentes, tout en retenant que les suggestions évoquées ci-dessus indiquent des directions vers lesquelles poursuivre et approfondir la question. 

\ex.[\ref{xparaph2}]
\a.  La glace  fond à 0\,\degres C.
%\b. L'eau gèle à 0\degres C.
\c. La température de fusion de l'eau est de 0\,\degres C.

Notons d'abord que des physiciens relèveront sans doute que ces deux phrases peuvent être fausses en même temps car elles ne disent rien des conditions de pression ambiante.  Mais d'un point de vue sémantique, cela ne nous concerne guère : nous avons seulement à \emph{supposer} la vérité et la fausseté de l'une et l'autre phrase.  En procédant de la sorte, une situation où (a) est jugée vraie rend également (b) vraie.  De même, inversement, en partant d'une situation où (b) est vraie, car une telle situation n'est pas seulement un cas où la science établit une certaine propriété de l'eau, c'est aussi un cas où cette propriété est définitivement vérifiée dans la réalité.  Ce qui perturbe l'exercice, en revanche, c'est que les phrases sont polysémiques (cf. p.~\pageref{p.polysem}) : \sicut{glace} peut désigner diverses choses (de l'eau ou un autre liquide à l'état solide, une crème glacée, un miroir...), les phrases peuvent chacune référer à des matériaux en général ou à des entités particulières observées dans le contexte (un certain bloc de glace, une certaine quantité d'eau), etc. Tant que ces acceptions ne sont pas fixées, il faudra conclure leur variabilité de sens empêche les phrases d'être équivalentes.  Elles ne pourront l'être que sous certaines hypothèses qui les désambiguïsent.


\ex.[\ref{xparaph2.5}]
\a. Tous les étudiants ont eu la moyenne à l'examen.
\b. Aucun étudiant n'a été recalé à l'examen.

Supposons une situation où les règles d'évaluation de l'examen en question stipulent qu'il faut une note supérieure ou égale à 12/20 pour être admis. Et supposons de plus que tous les étudiants ont eu une note supérieure à 10/20 mais que certains ont eu 11/20.  Dans ce cas (a) est vraie, mais (b) est fausse. Les deux phrases ne sont donc pas logiquement équivalentes.


\ex.[\ref{xparaph3}]
\a. Vous savez bien que l'existence précède l'essence.
\b. Vous n'ignorez pas que l'existence précède l'essence.

Plaçons nous dans une situation où il est vrai que «l'existence précède l'essence» (quoi que cela veuille dire) et où (a) est fausse. Dans cette situation, l'allocutaire ne sait donc pas que l'existence précède l'essence, c'est-à-dire qu'il ignore que l'existence précède l'essence.  Mais alors (b) est fausse aussi. Un raisonnement similaire peut être mené en partant d'une situation où (b) est fausse (s'il est faux que l'allocutaire n'ignore pas, c'est qu'il sait).  Les deux phrases sont donc équivalentes.

%Équivalences logiques.
\end{solu}
\end{exo}

%Cela nous amène à la notion opposée : la contradiction.

\subsection{Les contradictions et les anomalies}
%------------------------------------------------

\subsubsection{Contradictions}
%'''''''''''''''''''''''''''''
\label{sss:contrad}

Deux phrases sont dites \kwo{contradictoires} ssi lorsque l'une est vraie,
l'autre est fausse et réciproquement, c'est-à-dire ssi les deux phrases
ne peuvent jamais être vraies \emph{en même temps}.

\begin{defi}[Contradiction (relation)]\label{d:contra1}
Deux phrases $A$ et $B$ sont contradictoires (ou en relation de
contradiction), ssi dans tous les cas où $A$ est vraie, $B$ est
fausse et dans tous les cas où $A$ est fausse, $B$ est vraie.
\end{defi}

La contradiction la plus élémentaire est celle qui relie une phrase et
sa négation.\indexs{negation@négation}  Profitons en pour introduire un élément de notation.

\begin{nota}[Négation]
Soit $A$  une phrase (déclarative) quelconque, on note $\neg A$ la
\kwa{négation}{negation} de $A$.
\end{nota}

Le symbole $\neg$ est un \kwo{opérateur}, l'opérateur de négation ; il
«transforme» une phrase en sa négation, et  $\neg A$ se prononce
généralement «~\sicut{non} $A$~»%
\footnote{Astuce :  il n'est pas toujours facile de «prononcer»
  $\neg A$ en français avec les
tournures négatives standards ({\ie} \sicut{ne... pas}).  Dans ce cas,
  il est prudent d'utiliser une version non équivoque qui est : \emph{il est faux que
  $A$} ou \emph{il n'est pas vrai que $A$}. Par exemple, si $A$
  représente \sicut{Léo n'a aucun ennemi}, alors $\neg A$ se
  prononcera \sicut{il est faux que Léo n'a aucun ennemi}.}.
Nous reviendrons dessus
aux chapitre~\ref{LCP}, mais nous pouvons dire dès à présent que,
\emph{par définition}, cet opérateur a pour fonction d'\emph{inverser}
la vérité d'une phrase ; c'est-à-dire que dans tous les cas où $A$ est
vraie alors $\neg A$ est fausse et réciproquement.  Nous constatons donc
que la définition~\ref{d:contra1} s'applique immédiatement à $A$ et
$\neg A$.  Ainsi, par exemple, les phrases \ref{x:D//a} et \ref{x:D//b} sont contradictoires :

\ex. \label{x:D//}
\a.  Les droites $\mathscr D_1$ et $\mathscr D_2$ sont parallèles. \label{x:D//a}
\b. Les droites $\mathscr D_1$ et $\mathscr D_2$ ne sont pas parallèles. \label{x:D//b}


Ce phénomène, illustré par \ref{x:D//}, s'appelle la \kwo{loi de
  contradiction}\indexs{loi!\elid\ de contradiction}.  Elle peut sembler évidente, mais elle met néanmoins
en avant une propriété sémantique fondamentale de la négation ; et nous
  aurons donc intérêt à définir formellement le sens de la négation de
  manière %sorte 
à respecter cette loi (cf.\ \S\ref{s:reglsem}, p.~\pageref{s:reglsem}).  Par ailleurs, nous verrons que cette
  loi constitue un outil intéressant lorsque nous aborderons les
  propriétés sémantiques des groupes nominaux (cf.\ \S\ref{s:CatGN},
  p.~\pageref{s:CatGN}). 


\begin{defi}[Loi de contradiction]\label{loi:contrad}
Si $A$ représente une phrase déclarative, alors  $A$ et $\neg A$ sont contradictoires.
\end{defi}


Bien sûr nous pouvons trouver des paires de phrases contradictoires qui ne
suivent pas (du moins explicitement) le schéma de la loi de
contradiction. Ainsi :

\ex.
\a. Mercredi matin, à la réunion, Pierre était présent.
\b. Mercredi matin, à la réunion, Pierre était absent.


\largerpage

Dans la définition~\ref{d:contra1}, la contradiction est définie comme
une relation sémantique, c'est-à-dire qui s'établit entre deux
phrases.  Mais il est également courant de faire usage du terme de
contradiction pour désigner une  \emph{propriété} sémantique de phrase, c'est-à-dire
quelque chose que l'on peut dire au sujet d'une seule phrase.  

\begin{defi}[Contradiction (propriété)]
On dit qu'une phrase est \kw{contradictoire} ou que c'est une
\kw{contradiction} ssi elle  ne peut, en aucune circonstance,
être vraie.  
\end{defi}

Autrement dit, une phrase contradictoire est une phrase \emph{toujours} fausse.

Un moyen simple de construire une phrase contradictoire consiste à
coordonner par les
conjonctions \sicut{et} ou \sicut{mais} deux phrases qui sont entre
elles en relation de contradiction, comme en \ref{x:contr6} : 
%et \ref{x:contr7} : 


\ex.  \label{x:contr6}
\a. Alice est intelligente mais Alice n'est pas intelligente.
\b. La Terre est sphérique et  plate.
 \label{x:contr7b}


L'exemple \ref{x:contr7b} montre aussi que la notion de contradiction peut s'avérer utile pour tirer des informations de sémantique lexicale
(en l'occurrence sur le sens des adjectifs \sicut{sphérique} et \sicut{plat}).  
C'est ce que montre également la contradiction \Next.


\ex.
Ce carré est un triangle.




\subsubsection{Tautologies}
%''''''''''''''''''''''''''

S'il y a des phrases qui sont toujours fausses, il est tout aussi 
utile de considérer celles qui sont toujours vraies et qui s'appellent les \emph{tautologies}.


\begin{defi}[Tautologie]
Une phrase est une  \kw{tautologie} ssi elle est vraie en toute
circonstance. 
\end{defi}

Il s'ensuit que si $A$ est une tautologie, alors $\neg A$ est une contradiction et vice-versa.  Et à l'instar de la conséquence logique, il y a une méthode assez simple de démontrer qu'une phrase est une tautologie : il suffit d'essayer d'imaginer un cas de figure où cette phrase serait fausse ; si cela s'avère impossible ou incohérent, c'est que nous avons une tautologie.  C'est ce que montrent les exemples suivants :

\ex.
\a. Mon cheval est un cheval.
\b. Peut-être qu'en ce moment il pleut à Cherbourg, et peut-être pas.
\b. Si Alice a la grippe, alors elle est malade.


Les tautologies jouent un rôle important en logique  --~où elles sont souvent appelées des \kwo{lois}\is{loi}~-- pour formaliser les raisonnements et les démonstrations.  
D'un point de vue linguistique, on peut s'interroger sur l'intérêt de les examiner car, comme l'illustrent les exemples ci-dessus,  elles paraissent souvent sémantiquement triviales (nous allons immédiatement y revenir). 
Pourtant notre capacité à juger qu'une phrase est tautologique est révélateur de la compréhension que nous en avons et donc de son sens.  C'est à la sémantique d'expliquer pourquoi ces phrases sont des tautologies.


\subsubsection{Absurdités et anomalies sémantiques}
%''''''''''''''''''''''''''''''''''''''''''''''''''
\label{sss:anomalies}

Les exemples de tautologies et de contradictions que nous avons vus ont la particularité de nous apparaître comme un peu saugrenus, un peu absurdes.\indexs{absurde}  
Il se trouve qu'en logique, la contradiction est aussi nommée l'\emph{absurde}
(comme dans les
raisonnements \emph{par l'absurde}). 
Mais il ne faudra pas confondre cet \emph{absurde} logique avec ce que nous qualifions ordinairement d'absurde pour désigner quelque chose d'insensé, d'incohérent ou de ridicule.  D'ailleurs \sicut{insensé}, comme son nom l'indique, suggère l'idée de quelque chose qui n'a pas de sens.  Mais est-ce que ce que nous pouvons parfois trouver insensé est vraiment dépourvu de sens ?  Il est possible (et légitime) de défendre que les tautologies et les contradictions ont un sens et que c'est justement ce sens qui nous permet de les reconnaître comme telles.  Tautologies et contradictions nous frappent par leur absurdité mais pour des raisons un peu différentes.  Les tautologies sont dramatiquement évidentes et dans une conversation, du point de vue de l'échange d'information (du moins si nous les prenons littéralement), elles n'apportent rien, n'apprennent rien, ne servent à rien.  Le cas des contradictions est un peu plus grave : en matière d'informations, elles s'auto-détruisent puisqu'elles disent quelque chose qui ne peut jamais être vrai. 

De fait, nous pouvons concevoir des phrases qui «clochent» de par leur sens et que nous pouvons juger comme problématiques du point de vue de leur compréhension.  Il est dans l'intérêt de la sémantique d'étudier et d'expliquer ces dérèglements de sens car ils peuvent nous en apprendre beaucoup sur les mécanismes d'interprétation dans la langue\footnote{Tout comme il est dans l'intérêt de la médecine et de la biologie d'étudier les maladies, c'est-à-dire les dérèglements de la santé, pour comprendre le fonctionnement des organismes vivants.} : savoir juger qu'une phrase est sémantiquement dysfonctionnelle est révélateur de notre faculté de compréhension.
Autrement dit, à l'instar de la syntaxe qui porte des jugements d'(in)acceptabilité grammaticale, il peut nous être utile de porter des jugements d'\emph{anomalies sémantiques}\indexs{anomalie sémantique}.  Mais il convient alors de prendre certaines précautions.  Contrairement aux inacceptabilités grammaticales qui sanctionnent une malformation syntaxique, les anomalies sémantiques ne témoignent pas nécessairement d'une malformation du sens.  
Cela est dû en partie à ce que, comme nous l'avons vu ci-dessus, il existe différents types d'anomalies, différents phénomènes sémantiques qui provoquent des sentiments d'étrangeté lors de la compréhension.  Il convient donc de savoir correctement les distinguer afin de ne pas se méprendre sur les conclusions que nous pourrions en tirer.  À cet effet je reprends ici (en partie) la typologie proposée par \citet[10--14]{Cruse:86}\Index{Cruse, D. A.} et qui est illustrée par les exemples suivants :

\ex. \label{x:anom}
\a. Il est tombé de la neige blanche ce matin.  \label{x:anoma}
\b. L'avion monte vers le haut.  \label{x:anomb}
\b. Mon écrevisse apprivoisée a fini tous les mots-croisés du journal.  \label{x:anomc}
\b. Le monde est un grand disque posé sur le dos de quatre immenses éléphants se tenant sur la carapace d'une gigantesque tortue.   \label{x:anomd}
\b. Les émois intriqués des synapses rougeoient géologiquement.  \label{x:anome}
\b. La racine cubique du canapé de Mireille a bu notre altérité. \label{x:anomf} 
%\b.   \label{x:anomg}
%\b. J'ai voulu prendre l'avion pour Chicago, mais il était trop lourd. (G. Marx)  \label{x:anomh}


Les phrases \ref{x:anoma} et \ref{x:anomb} contiennent des \emph{pléonasmes}\indexs{pleonasme@pléonasme}, c'est-à-dire des redondances, des répétitions superflues d'informations.  Les tautologies sont souvent des pléonasmes, mais les pléonasmes ne sont pas tous des tautologies\footnote{Par exemple \ref{x:anoma} peut être fausse s'il n'a pas neigé ce matin.}.
Ils ne posent pas de graves problèmes de compréhension et nous les jugeons probablement plus lourds qu'anormaux, parce que mal équilibrés sémantiquement.  Mais ils sont utiles pour l'analyse car ils signalent une répétition à laquelle la sémantique est sensible.

Les phrases \ref{x:anomc} et \ref{x:anomd} font partie de ce que Cruse appelle les \emph{invraisemblances}\indexs{invraisemblance} (ang.\ \alien{improbablities}) et il arrive souvent qu'elles soient, dans un réflexe néophyte, rejetées comme sémantiquement déviantes.  Mais en réalité elles sont parfaitement compréhensibles et il n'y a donc pas de raison de voir une anomalie dans leur sens.  Les anomalies se trouvent dans l'histoire racontée ou le tableau dépeint, qui se trouvent s'écarter singulièrement de notre réalité quotidienne, mais cela n'est pas un problème linguistique.  Cela fait partie des propriétés du langage que d'avoir le pouvoir expressif de mettre en scène les fruits de notre imagination\footnote{Et c'est ce qui nous permet de lire tout naturellement de la fiction sans sombrer dans des abîmes de perplexité. D'ailleurs une recette pour aider à porter ce type de jugement  est de se dire que si la scène décrite par la phrase peut être dessinée ou «filmée» (même avec des effets spéciaux virtuoses), alors il s'agit d'une invraisemblance, pas d'une véritable  anomalie sémantique.}.  Nous aurons donc intérêt à ne pas traiter les invraisemblances comme des anomalies sémantiques.

Au contraire, les phrases \ref{x:anome} et \ref{x:anomf} présentent, elles, des anomalies flagrantes.  Nous peinons à les comprendre précisément et cela implique que nous ne parvenons pas (du moins pas entièrement) à leur attribuer un sens\footnote{Certes il est souvent possible, si nous faisons preuve de suffisamment de créativité (et de charité envers l'auteur des phrases), de finir par leur trouver un sens imagé (mais généralement peu conventionnel et pas forcément consensuel). Pour autant cela ne change rien au jugement initial que nous portons, car si nous devons faire des efforts particuliers pour les déchiffrer (contrairement à des phrases ordinaires) c'est justement parce qu'elles sont au départ sémantiquement déficientes.}.
Cruse les appelle des \emph{dissonances}\indexs{dissonance} pour souligner qu'elles présentent des incompatibilités ou des discordances entre plusieurs éléments de la phrase qui en bloquent la compréhension.  C'est bien sûr lié à l'idée que le sens d'une phrase se construit et que parfois cette construction peut échouer.  C'est généralement à la sémantique d'expliquer et de prédire de tels échecs. 

Souvent les dissonances les plus susceptibles d'intéresser les sémanticiens ne sont pas aussi spectaculaires que \ref{x:anome} et \ref{x:anomf}.  Elles peuvent être spécifiquement localisées et ne pas contaminer complètement la compréhension de la phrase. Mais nous avons tout de même la faculté de juger qu'elles comportent des défauts sémantiques, comme par exemple dans les phrases   \ref{x:anom3}. 
\largerpage

\ex.  \label{x:anom3}
\a. Ce trou est {entièrement profond}.
\b. Mon voisin est très veuf.
\b. Seuls tous les étudiants ont écouté le professeur.



Ce sont les dissonances que nous qualifierons véritablement de sémantiquement mal formées ou anormales, car nous pouvons estimer que, globalement, elles ne veulent à peu près rien dire.  Et il est à remarquer que la malformation sémantique n'est pas liée à une malformation syntaxique\footnote{C'est par exemple l'idée que défend \citet[\S2.3]{Chom:57}.\Indexn{Chomsky, N.}} : les phrases \ref{x:anome} et \ref{x:anomf} sont tout à fait correctes pour ce qui est de leur syntaxe.  En syntaxe, il est usuel de marquer les jugements d'agrammaticalité en préfixant d'un $^*$ les séquences rejetées.  En sémantique, on utilise souvent la marque  {\zarb} pour signaler une anomalie.  Mais il faut faire attention car ce symbole est aussi employé pour marquer des phrases parfaitement bien formées et compréhensibles mais qui sont \emph{inappropriées} dans le contexte\is{contexte} où elles sont énoncées. C'est le cas par exemple des réponses en \Next.

\ex.
\a. -- Tu aurais de la monnaie sur 10\,\myeuro\ s'il te plaît ?\\
\juge{\zarb} -- J'ai passé d'excellentes vacances en Lozère il y a deux ans.
\b. -- Qui est-ce qui a invité Jeremy à mon anniversaire ?\\
\juge{\zarb} -- C'est Jeremy que Thierry a invité.

Cela pourrait constituer une catégorie supplémentaire à l'inventaire de Cruse,  que nous pourrions appeler les \emph{incongruïtés} (\emph{contextuelles}), mais il faudra bien garder à l'esprit que ce type d'anomalies relève au moins autant de la pragmatique que de la sémantique, puisque le contexte entre en jeu. 






\subsection{Les ambiguïtés}
%------------------------
\label{s:Ambiguïté}\indexs{ambiguïté|(}

\largerpage[2]

Nous venons de voir que les phrases les plus anormales sémantiquement sont difficiles (voire impossibles) à comprendre.   Il existe par ailleurs des phrases qui peuvent nous sembler difficiles à comprendre sans pour autant que nous les jugions vraiment mal formées sémantiquement.  Ce sont ce que nous pourrons appeler des phrases obscures ou absconses. Il peut y avoir diverses causes à cet effet, et l'une de celles qui viennent le plus souvent à l'esprit est que ces phrases sont \emph{ambiguës}.  
Les ambiguïtés peuvent effectivement être des sources d'incompréhension, mais \sicut{ambigu} ne signifie pas nécessairement «peu clair» ou «difficile à comprendre».  Une phrase ambiguë, par définition, est une phrase à laquelle on peut attribuer (au moins) deux sens différents.  
C'est ce qui crée parfois un flottement dans la compréhension, même si souvent le contexte nous aide beaucoup à résoudre les ambiguïtés (et parfois sans que nous en ayons conscience).
Évidemment cette définition, comme celle de la paraphrase, boucle sur une connaissance préalable  du sens, et nous aurons donc intérêt à donner une caractérisation plus objective.  
Le but de la sémantique est d'expliciter le sens des expressions de la langue, et donc si une expression est ambiguë, la sémantique se doit d'en rendre compte en en proposant des sens distincts. 
Et cela implique que la sémantique doit se munir d'un critère précis et rigoureux qui permette d'établir que deux sens sont différents.  Nous voyons ainsi que les ambiguïtés jouent un rôle important pour nous\footnote{C'est même une des principales missions de la sémantique : repérer et identifier les ambiguïtés. Il faut savoir en revanche que la sémantique théorique et descriptive n'a pas pour objectif de \emph{résoudre} (on dit aussi \emph{lever}) les ambiguïtés, même si, bien sûr, les principes de résolutions doivent fondamentalement s'appuyer sur la sémantique et la pragmatique.}.  


Nous allons commencer par examiner quelques exemples, qui montrent que les ambiguïtés peuvent se nicher à divers niveaux de l'analyse linguistique. 



\ex.
\a. L'obstination de cet homme brave la tourmente. \label{x:amb1}
\b. Les figurants ne sont pas dans le champ. \label{x:amb2}
\b. Dans le parc, j'ai vu un singe avec un télescope. \label{x:telescope}
\b. Julie a proposé à Clara d'aller chez elle. \label{x:amb3}
\b. Voltaire et Rousseau s'admiraient beaucoup. \label{x:amb4}
\b. Alex a seulement présenté Sam à Maria. \label{x:amb5}
%Tous les nombres premiers sont plus grands qu'un nombre pair. 


Une façon pratique et simple de révéler des ambiguïtés consiste à donner à la phrase  différentes reformulations ou paraphrases.  Parfois quelques commentaires suffisent.  Ainsi \ref{x:amb1} nous parle soit d'un homme qui est brave et qui cause des tourments à quelqu'un (probablement une femme), soit d'un homme qui affronte une tempête ou une agitation.  Cette phrase joue malicieusement sur l'homonymie des mots \sicut{brave} (adjectif ou verbe), \sicut{la} (pronom ou déterminant) et \sicut{tourmente} (nom ou verbe).

L'exemple \ref{x:amb2} est également un cas d'ambiguïté lexicale, mais au sein d'une même catégorie grammaticale : le nom \sicut{champ} peut être compris comme désignant un terrain ou le champ d'une caméra. 
Notons à ce propos qu'en ce qui concerne les unités lexicales, il est habituel de distinguer l'ambiguïté de la \emph{polysémie}\indexs{polysemie@polysémie}.\label{p.polysem} 
L'ambiguïté correspond généralement à de l'homonymie, c'est-à-dire deux unités lexicales distinctes (et donc de sens différents) qui se trouvent avoir, souvent accidentellement, la même forme ; la polysémie, qui littéralement signifie «plusieurs sens», qualifie \emph{une} unité lexicale qui possède plusieurs acceptions, différentes mais sémantiquement apparentées. 
À cet égard, \sicut{champ} est polysémique (puisqu'il renvoie toujours à l'idée d'un espace où se produit une certaine activité). Ici nous ne parlerons pas de phrases polysémiques, et nous dirons que l'emploi d'une unité lexicale polysémique peut rendre une phrase ambiguë.

\ref{x:telescope} est un exemple classique d'ambiguïté syntaxique portant sur le rattachement structurel du groupe prépositionnel %(\GP) 
\sicut{avec un télescope} : soit c'est moi qui avais un télescope à l'aide duquel j'ai vu le singe, %(le \GP\ sera rattaché au verbe \sicut{voir}), 
soit c'est le singe qui tenait un télescope. %(le \GP\ sera rattaché au groupe nominal «un singe»). 

L'ambiguïté de \ref{x:amb3}, quant à elle, n'est pas liée à l'organisation syntaxique de la phrase ; elle réside dans l'interprétation du pronom\is{pronom} personnel \sicut{elle} qui peut renvoyer à Julie ou à Clara.  Cela nous rappelle, au passage, qu'un pronom comme \sicut{il} ou \sicut{elle} ne s'interprète pas en isolation :  pour permettre une compréhension complète, il a besoin d'être mis en relation avec un élément du contexte (la phrase, le discours précédent ou la situation d'énonciation), élément que nous appellerons (pour faire simple) l'\emph{antécédent}\indexs{antecedent@antécédent!\elid\ d'un pronom} du pronom.
Pour lever ce type d'ambiguïté, il est courant d'insérer, dans l'écriture de la phrase, des \kwo{indices} numériques dits référentiels\is{indice!\elid\ referentiel@\elid\ référentiel} pour faire apparaître ce lien entre le pronom et son antécédent, et ainsi sélectionner l'une ou l'autre des interprétations disponibles : \sicut{Julie$_1$ a proposé à Clara$_2$ d'aller chez elle$_1$} {\vs} \sicut{Julie$_1$ a proposé à Clara$_2$ d'aller chez elle$_2$}.

C'est aussi un pronom qui engendre l'ambiguïté de \ref{x:amb4}, le pronom réfléchi \sicut{se} qui peut donner lieu à une interprétation réflexive (Voltaire admirait Voltaire et Rousseau admirait Rousseau) ou réciproque (Voltaire admirait Rousseau et Rousseau admirait \mbox{Voltaire}).

Enfin \ref{x:amb5} est au moins triplement ambiguë, ce que nous pouvons faire ressortir au moyen de paraphrases qui utilisent la tournure \sicut{ne...que} au lieu de \sicut{seulement} : soit Alex n'a fait qu'une seule présentation, Sam à Maria ; soit il n'a présenté que Sam à Maria ; soit il n'a présenté Sam qu'à Maria.


Donner des paraphrases nous aide à révéler les différents sens de phrases ambiguës, mais cela ne peut pas constituer une méthode de démonstration qu'une phrase est réellement ambiguë. Tout simplement parce qu'en théorie, rien ne nous garantit que deux paraphrases, même formellement distinctes, aient effectivement des sens distincts. 
Nous ne pourrons pas non plus fonder la démonstration de l'existence d'une ambiguïté sur la possibilité d'assigner à une phrase deux analyses syntaxiques différentes. 
Car là encore rien ne prouve que deux analyses syntaxiques distinctes donnent lieu nécessairement à deux sens distincts.  Et réciproquement, il ne va pas de soi qu'une distinction de sens coïncide nécessairement avec une distinction de structure syntaxique.  
S'il est assez évident par exemple que  \ref{x:amb1} et \ref{x:telescope} recouvrent chacun plusieurs analyses syntaxiques, ça l'est moins pour \ref{x:amb2}, \ref{x:amb3} et \ref{x:amb5}.  Pour dire les choses autrement, si la syntaxe est en mesure de rendre compte de \emph{tout} type d'ambiguïté c'est qu'elle présuppose la notion de sens (et qu'à ce titre, elle n'est peut-être pas \emph{que} de la syntaxe). 
En fin de compte, une méthode objective (et fondamentalement sémantique) de démontrer qu'une phrase est ambiguë a intérêt à contourner les {a priori} d'une analyse linguistique préalable. 
Et pour ce faire, nous allons, là encore, utiliser la notion de vérité, en  appuyant notre méthode sur la définition suivante\footnote{Voir \citet{Gillon:04x}\Indexn{Gillon, B.} sur ce point et sur la notion d'ambiguïté en général.} :




\begin{defi}[Phrase ambiguë] %[Ambiguïté sémantique]
\label{d:ambig}
Une phrase (déclarative) $A$ est sémantiquement ambiguë ssi il existe au moins un même cas
de figure par rapport auquel  $A$ peut être jugée à la fois vraie et fausse.
\end{defi}


Cette définition s'applique seulement aux phrases déclaratives mais en avançant dans la théorie (notamment avec les chapitres \ref{LCP} et \ref{Ch:t+m}) nous pourrons voir comment elle peut s'étendre à toute catégorie d'expressions ambiguës.  
Elle est certes «plus seconde» que celle qui précédemment évoquait plusieurs sens, mais 
nous verrons également que cette définition est en fait une conséquence directe de la première ainsi que de la définition formelle du sens que nous adopterons dès le chapitre suivant.   De plus elle a le mérite de s'affranchir d'une idée préconçue de ce que serait le ou les sens d'une phrase.

La définition repose sur le fait qu'une phrase comprise avec \emph{un sens déterminé} ne peut jamais être à la fois vraie et fausse quand nous la jugeons par rapport un cas de figure donné (cf.\ la loi de contradiction p.~\pageref{loi:contrad}).  
Par conséquent, si nous arrivons à la trouver vraie \emph{et} fausse (sans changer le cas de figure), c'est que nous parvenons à la comprendre avec \emph{deux}  sens différents.
Nous avons ainsi une méthode opérationnelle pour \emph{démontrer} qu'une phrase est ambiguë. 
Il suffit d'inventer un scénario, nous avons le droit de le choisir comme nous le souhaitons, mais en nous arrangeant de telle sorte qu'il rende la phrase vraie selon un certain sens et fausse selon un autre.
Par exemple, reprenons \ref{x:telescope}. 
Supposons donc que je me promène un matin
dans un parc, sans télescope, et que j'aperçoive dans un arbre un singe qui tient un  télescope. Dans ce cas de figure, la phrase \ref{x:telescope} est vraie car ce que j'ai vu est bien \sicut{un singe avec un télescope}, mais elle est également fausse car je n'ai pas utilisé un télescope pour voir le singe.  

Il est tout à fait crucial ici de bien insister sur le fait que cette méthode consiste à inventer un \emph{et un seul} scénario. Inventer deux scénarios différents qui rendent chacun la phrase vraie selon un sens puis l'autre \emph{ne prouve absolument rien}.  Car une phrase non ambiguë peut être vraie par rapport à une multitude de scénarios différents.   Par exemple, la phrase simple \sicut{j'ai vu un singe} sera vraie dans des scénarios où le singe est orang-outan, un gibbon ou un gorille, où je me trouvais dans un zoo, une forêt ou le métro, etc.  


Cela nous donne l'occasion de souligner une distinction importante à faire entre d'une part les ambiguïtés et d'autre part les propriétés que sont le \kw{vague} et l'\kwo{imprécision}.\is{imprecision@imprécision}  
Ces deux propriétés ne recouvrent pas les mêmes phénomènes, mais il s'agit surtout ici de les opposer à l'ambiguïté.
%C'est le propre des langues naturelles d'offrir la possibilité d'énoncer des phrases ou des expressions intrinsèquement et délibérément vagues.  
Par exemple, nous savons que le nom \sicut{singe} peut renvoyer à de nombreuses espèces différentes ; cela ne veut pas dire que ce nom est ambigu, son sens est bien déterminé, il est simplement assez général, moins précis que le sens de certains autres mots. 
Il n'est pas vague non plus, car le vague correspond à l'existence de zones d'incertitudes et de frontières floues lorsqu'il s'agit par exemple de juger de la vérité ou de la fausseté d'une phrase.  C'est ainsi le cas des noms de couleurs : sur un spectre chromatique (ou un arc-en-ciel) à partir de quel endroit pouvons nous être sûr de ne plus être dans le bleu et déjà dans le vert ?
Souvent les expressions vagues sont également imprécises.  \sicut{Beaucoup} et \sicut{nombreux} en sont d'autres exemples\footnote{Il faut cependant être prudent car \emph{par ailleurs} ces termes peuvent aussi être ambigus, compris comme signifiant soit «une grande quantité» soit «une grande proportion».  Nous y reviendrons au chapitre \ref{ch:ISS}, \S\ref{ss:QGDet}.} :

\ex.  \label{x:vaches}
Il y a beaucoup de vaches dans ce pré.


Dans \Last, \sicut{beaucoup} est imprécis dans la mesure où il ne nous donne pas le nombre exact de vaches et qu'il ne répond pas idéalement à la question \sicut{combien il y a de vaches ?}.  \sicut{Beaucoup} est aussi vague car nous ne pouvons pas dire clairement à partir de quel nombre $n$ de vaches la phrase \Last\ est vraie (sachant alors que pour $n-1$ vaches, elle sera fausse).  
Et cela se complique d'autant plus que \sicut{beaucoup} comporte certainement un élément de subjectivité variable qui entre en jeu dans l'évaluation de la quantité. 
Ainsi, par rapport à un scénario qui spécifie un nombre précis de vaches, par exemple 15, sans autre information sur ce que pense le locuteur, nous ne saurons pas dire si \ref{x:vaches} est vraie ou fausse, et donc nous ne pourrons pas dire qu'elle est à la fois vraie et fausse.  
Alternativement, si nous savons que le locuteur estime que 15 n'est pas un nombre élevé de vaches pour un troupeau, alors nous jugerons la phrase fausse (et donc pas vraie).  Dans les deux cas, nous n'aurons pas  démontré que \ref{x:vaches} est ambiguë. 

%***

Profitons-en pour introduire un outil d'observation qui permet de distinguer empiriquement une véritable ambiguïté sémantique de ce qui n'en est pas une.  C'est ce que nous appellerons le test des ellipses.\is{ellipse}\is{test!\elid\ des ellipses} Il consiste d'abord à ajouter à la phrase une proposition elliptique, par exemple avec \sicut{aussi}, \sicut{non plus} ou \sicut{mais pas}, qui reprend implicitement (c'est-à-dire sans la répéter) l'expression ambiguë.  Par exemple, pour \ref{x:telescope}, nous aurons :

\ex. 
Dans le parc, j'ai vu un singe avec un télescope, et Alice aussi. \label{x:telescope2}


Pour comprendre globalement \ref{x:telescope2}, nous devons mentalement restituer ce qui n'est pas exprimé dans la seconde partie, afin d'obtenir \sicut{et Alice aussi a vu un singe avec un télescope}.  Or il se trouve que cette restitution se fait en «recopiant» \emph{le sens} d'une expression de la première partie de la phrase.\label{test:ellipse}
Autrement dit, si cette expression est ambiguë, elle aura néanmoins le même sens dans les deux parties de la phrase : nous ne pouvons pas panacher en mêlant les deux sens, \ref{x:telescope2}  ne peut pas se comprendre comme disant que j'ai vu, à l'\oe il nu, un singe qui tenait un télescope, et qu'Alice a vu, dans un télescope, un singe (qui n'avait pas de télescope). 
En revanche, \ref{x:telescope3} est acceptable pour décrire une situation où j'ai vu un ourang-outan et Alice un macaque.  \sicut{Singe} n'est (évidemment) pas ambigu à cet égard. 

\ex. 
J'ai vu un singe, et Alice aussi. \label{x:telescope3}


Il en va de même pour \ref{x:vaches2} où il n'y a pas de raison de penser qu'il y a forcément autant de vaches dans le pré et dans l'étable.

\ex. \label{x:vaches2}
Il y a beaucoup de vaches dans le pré, et dans l'étable aussi.


Repérer précisément les ambiguïtés et savoir expliciter leurs différentes interprétations est donc une tâche essentielle de la sémantique.  
Cela dessine des éléments importants de son programme : une théorie sémantique doit être en mesure de produire des descriptions de sens suffisamment fines et précises pour faire apparaître les ambiguïtés, et aussi de rendre compte \emph{en tant que tels} de l'imprécision et du vague lorsqu'ils sont présents, sans chercher à les rendre plus précis ou plus nets.

\medskip

% -*- coding: utf-8 -*-
\begin{exo}[Ambiguïtés]\label{exo:1Ambig}
Montrez, en utilisant la définition \ref{d:ambig} ci-dessus, que les phrases suivantes
sont ambiguës.
\pagesolution{crg:1Ambig}

\begin{enumerate}
\item J'ai rempli ma bouteille d'eau.
\item Pierre s'est fait arrêter par un policier en pyjama. 
\item Daniel n'est pas venu à la fête parce que le président était là.
\item Alice ne mange que des yaourts au chocolat. 
\item Kevin dessine tous les gens moches.
\end{enumerate}
\begin{solu}(p.~\pageref{exo:1Ambig})\label{crg:1Ambig}

En accord avec la définition \ref{d:ambig} p.~\pageref{d:ambig}, pour chaque phrase, nous construisons un scénario par rapport auquel nous pouvons juger que la phrase est à la fois vraie et fausse (selon le sens retenu).
\begin{enumerate}
\item %J'ai rempli ma bouteille d'eau.\\
Scénario : \emph{j'ai une bouteille d'eau minérale en plastique, initialement vide ; je l'ai remplie de limonade.} %\\
La phrase est vraie, car c'est bien ma bouteille d'eau que j'ai remplie.  Mais elle est aussi fausse, car je ne l'ai pas remplie d'eau.
\item %Pierre s'est fait arrêter par un policier en pyjama. \\
Scénario : \emph{Pierre, très distrait, se promène dans la rue en pyjama ; un agent de police, en uniforme, l'arrête pour lui demander la raison de cette tenue étrange.} %\\
La phrase est vraie car Pierre, alors qu'il était en pyjama, s'est fait arrêter par un policier. Elle est aussi fausse car ce n'est pas un policier en pyjama qui l'a arrêté. 
\item %Daniel n'est pas venu à la fête parce que le président était là.
Scénario : \emph{Daniel est venu à la fête, le président aussi ; Daniel ne savait même pas que le président serait présent et il est venu parce qu'il aime les fêtes et n'en manque aucune.}
La phrase est vraie car ce n'est pas parce que le président était là que Daniel est venu.  Elle est fausse parce que Daniel est venu à la fête.  Les deux sens de la phrase peuvent se paraphraser en : i) la raison pour laquelle Daniel est venu n'est pas que le président était là ; ii) la raison pour laquelle Daniel n'est pas venu est que le président était là.
\item %Alice ne mange que des yaourts au chocolat. 
Scénario : \emph{Alice est une omnivore accomplie et a un régime alimentaire varié et équilibré, mais pour ce qui est des yaourts, elle n'en mange qu'au chocolat.}
La phrase est vraie car les seuls yaourts qu'Alice mange sont au chocolat, et elle est fausse car Alice ne se nourrit pas exclusivement de yaourts au chocolat.
\item %Kevin dessine tous les gens moches.
Scénario : \emph{Kevin aime bien dessiner, et en particulier il aime faire le portrait des gens qui sont plutôt beaux (ils ne dessine jamais les gens qu'il trouve moches) ; mais il a un mauvais coup de crayon et ses dessins enlaidissent toujours les sujets.}
La phrase est vraie car, les gens qu'il dessine, il les dessine moches. Et elle est fausse car il ne dessine pas les gens qui sont moches.  Les deux sens se paraphrasent ainsi : i) si Kevin dessine quelqu'un, il le dessine moche ; ii) si quelqu'un est moche, Kevin le dessine.
\end{enumerate}
\end{solu}
\end{exo}


% -*- coding: utf-8 -*-
\begin{exo}\label{exo:1Ambig2}
La phrase \sicut{ce bijou n'est pas une bague en or} peut servir à communiquer que le bijou n'est pas une bague ou %à communiquer 
qu'il n'est pas en or.  Est-ce à dire que la phrase est ambiguë ? 
\pagesolution{crg:1Ambig2}
\begin{solu}(p.~\pageref{exo:1Ambig2})\label{crg:1Ambig2}

La phrase \sicut{ce bijou n'est pas une bague en or} n'est pas ambiguë, elle a juste une signification assez large pour couvrir les cas où le bijou n'est pas une bague et ceux où il n'est pas en or.  Pour nous en assurer, nous pouvons ici appliquer le test des ellipses vu dans le chapitre p.~\pageref{test:ellipse}.  Supposons qu'un premier bijou $B_1$ est une bague en argent et qu'un second bijou $B_2$ est un bracelet en or ; nous pouvons alors tout à fait énoncer \sicut{ce bijou $B_1$ n'est pas une bague en or, et ce bijou $B_2$ non plus}. 
\end{solu}
\end{exo}





\indexs{ambiguïté|)}




\section{Dire et vouloir dire}
%=========================
\label{s:pragmat}

Les relations, propriétés et jugements de sens que nous avons vus précédemment constituent les premiers composants d'une boîte à outils pour notre étude et commencent à dresser le contenu d'un cahier des charges de la sémantique : la théorie devra, si possible, se donner les moyens de rendre compte de ces phénomènes que sont la conséquence logique, les anomalies, les ambiguïtés, etc.
Pour compléter notre investigation, nous allons aborder maintenant des  relations qui vont nous conduire progressivement vers le domaine de la pragmatique mais qui jouent un rôle très important dans l'observation des  symptômes de la compréhension. 

\subsection{Projections et présuppositions}
%-----------------------------
\label{s:presuppositions}
\indexs{presupposition@présupposition|(}


\subsubsection{Contenus projectifs {\vs} contenus en jeu}
%''''''''''''''''''''''''''''''''''''''''''''''''''''''''''
\label{ss:projections}

%Il existe un type de relation entre phrases qui ressemble beaucoup à
%la conséquence logique, mais qui s'en distingue singulièrement.
%Faisons d'abord une première 
Commençons par faire une observation sur la conséquence logique.
Mettons la phrase \ref{xbaga} à la forme négative \ref{xbag'a} et confrontons-la 
aux conséquences que nous avions précédemment en \ref{xbag} :

\ex. \label{xbag'} %[\ref{xbag}$'$]
\a. L'objet qui est dans la boîte \emph{n'est pas} une bague en or. \label{xbag'a}
\b. L'objet qui est dans la boîte est en or. \label{xbag'b}
\c. L'objet qui est dans la boîte est une bague. \label{xbag'c}


Nous pouvons montrer très facilement que \(\text{\ref{xbag'a}}
\not\satisf  \text{\ref{xbag'b}}\) et que \(\text{\ref{xbag'a}}
\not\satisf \text{\ref{xbag'c}}\).  Il suffit de trouver un
contre-exemple.  Supposons que l'objet en question est
une boucle d'oreille en argent.  Alors \ref{xbag'a} est vraie,
mais \ref{xbag'b} et \ref{xbag'c} sont fausses.
Cela semble être une règle assez générale : lorsque $A$ entraîne
$B$, la négation de $A$ pour autant n'entraîne pas forcément $B$.
Nous dirons que la conséquence
logique \emph{ne résiste pas} à la négation.


Maintenant examinons les exemples \ref{xmoutarde}--\ref{xgonc} :

\ex.\label{xmoutarde}
\a. Jean sait que c'est le colonel Moutarde le coupable.
\b. C'est le colonel Moutarde le coupable.

\ex.\label{xfred}
\a. Fred est allé chercher son fils à l'école. \label{xfreda}
\b. Fred a un fils. \label{xfredb}

\ex.\label{xpanda}
\a. Marie regrette d'avoir accepté de garder mon panda pendant les
vacances.
\b. Marie a accepté de garder mon panda pendant les
vacances.

\ex.\label{xgonc}
\a. Annie, qui était au lycée avec moi, a reçu le prix Goncourt.
\b. Annie était au lycée avec moi.


\largerpage[-1]

Ces exemples semblent faire apparaître des conséquences logiques  des
phrases (a) vers les phrases (b).  
Par exemple nous ne pouvons pas imaginer un scénario cohérent où Jean sait que le colonel est le coupable et où, en même temps, le colonel n'est pas le coupable.
Mais il y 
a une différence frappante avec les exemples que nous avons vus en \S\ref{s:conseql} et avec ce que montre \ref{xbag'} ci-dessus : les formes négatives des phrases (a) semblent aussi entraîner chacune des phrases (b).\label{p.resNeg}

\ex.[\ref{xmoutarde}]
\a.[a$'$.] Jean ne sait pas que c'est le colonel Moutarde le coupable.
\b.[b.] C'est le colonel Moutarde le coupable.

\ex.[\ref{xfred}]
\a.[a$'$.] Fred n'est pas allé chercher son fils à l'école.
\b.[b.] Fred a un fils.

\ex.[\ref{xpanda}]
\a.[a$'$.] Marie ne regrette pas d'avoir accepté de garder mon panda
  pendant les vacances.
\b.[b.] Marie a accepté de garder mon panda pendant les
vacances.

\ex.[\ref{xgonc}]
\a.[a$'$.] Annie, qui était au lycée avec moi, n'a pas reçu le prix Goncourt.
\b.[b.] Annie était au lycée avec moi.


Autrement dit, ces phrases (b) \emph{résistent à la négation} dans (a$'$).  
Avec \ref{xbag'}   nous avons montré qu'à partir de $A\satisf B$ nous ne devions
\emph{pas nécessairement} conclure $\neg A\satisf B$. 
Allons plus loin en regardant ce qu'implique la résistance à la négation, c'est-à-dire la coexistence de $A\satisf B$ et $\neg A\satisf B$.  
La première conséquence dit qu'à chaque fois que $A$ est vraie, $B$ est vraie aussi.  La seconde dit qu'à chaque fois que $\neg A$ est vraie, c'est-à-dire que $A$ est fausse, $B$ est vraie aussi.  Y a-t-il alors des cas où $B$ pourrait être fausse ?  Visiblement pas, car $A$ et $\neg A$ recouvrent à eux deux tous les cas de figure possibles.  Si $B$ ne peut pas être fausse, c'est une tautologie.  
Ainsi la logique prédit que les seules phrases qui résistent à la négation sont des tautologies.  Mais, de toute évidence, les phrases (b) de \ref{xmoutarde}--\ref{xgonc} n'en sont pas.

Ce paradoxe logique nous amène à conclure que la relation à l'\oe uvre dans  \ref{xmoutarde}--\ref{xgonc} n'est pas véritablement la conséquence logique.  Il s'agit de ce que nous allons appeler, dans un premier temps, une relation de \kwo{projection}\is{projection}.
C'est un peu une autre façon de dire qu'il y a là une résistance à la négation :  le sens des phrases (b) de \ref{xmoutarde}--\ref{xgonc} «se projette hors» des phrases (a) et  (a$'$), échappant ainsi à l'effet de la négation.
%et cette appellation capte l'idée (informellement et intuitivement) que ces sens (b) échappent à la structure sémantique  de (a) et (a$'$) en fonctionnant de manière quelque peu autonome dans le flux des informations communiquées.

Le phénomène de projection est \emph{relativement} robuste, car il ne se manifeste pas seulement avec la négation. 
Une bonne façon de le diagnostiquer, inspirée de \citet[23--27]{ChierchiaMcCG:90}, consiste à observer une famille de phrases comme celle illustrée en \ref{x:FamCMCG}. 
Les phrases qui la composent sont: la phrase originale à la simple forme affirmative \ref{x:FamCMCGa}, sa forme négative \ref{x:FamCMCGb}, son enchâssement dans une supposition (une subordonnée conditionnelle en \sicut{si}) \ref{x:FamCMCGd}, et sa version interrogative \ref{x:FamCMCGe} :


\ex. \label{x:FamCMCG}
\a. Fred est allé chercher son fils à l'école. \label{x:FamCMCGa}
\b. Fred n'est pas allé chercher son fils à l'école. \label{x:FamCMCGb}
%\b. Il se peut que Fred soit allé chercher son fils à l'école.\label{x:FamCMCGc}
\b. Si Fred est allé chercher son fils à l'école, il sera rentré vers 17h.\label{x:FamCMCGd}
\b. Est-ce que Fred est allé chercher son fils à l'école?\label{x:FamCMCGe}


Une famille comme \ref{x:FamCMCG} fonctionne un peu comme une grille ou un tamis: elle «laisse passer» certaines inférences mais pas d'autres.
Ainsi nous pouvons remarquer que \ref{xfreda}, répété en \ref{x:FamCMCGa}, a pour (réelle) conséquence logique que \sicut{Fred s'est rendu à l'école}, mais cette conséquence disparaît des phrases \ref{x:FamCMCGb}--\ref{x:FamCMCGe}%
\footnote{Évidemment pour \ref{x:FamCMCGe}, la démonstration n'est pas triviale puisque notre définition de $\satisf$ en \S\ref{s:conseql} ne porte que sur les phrases déclaratives; néanmoins nous comprenons bien que la réponse à cette question peut être \sicut{oui} ou \sicut{non}, et que  ce second cas de figure (\sicut{non}) peut être justement un cas où Fred ne s'est pas rendu à l'école.}.
En contraste, les projections sont ces inférences que l'on  tire de la phrase de départ (comme  \ref{x:FamCMCGa}) \emph{et} qui se maintiennent (ou plus exactement qui \emph{peuvent} se maintenir)  dans les autres phrases de la famille.
Ainsi, dans \ref{x:FamCMCGb}--\ref{x:FamCMCGe}, nous continuons à comprendre que Fred a un fils.


Il s'agit là, il faut bien le reconnaître,  plus d'une caractérisation empirique des projections que d'une véritable définition; mais cela nous permet dès à présent de faire une observation cruciale sur une certaine manière dont le sens des phrases est organisé.  
Si nous considérons, somme toute assez naturellement, que les conséquences logiques et les projections d'une phrase donnée dépendent du sens de cette phrase, alors nous pouvons poser l'hypothèse qu'il y a une partie de ce sens qui est responsable des conséquences logiques et une autre partie responsable des projections.  
Nous appellerons cette première partie le \kwo{contenu en jeu}\footnote{J'utilise cette appellation pour traduire l'anglais \alien{at-issue} et \alien{at-issueness}.}\is{contenu!\elid\ en jeu} de la phrase.  On admet couramment que c'est ce qui constitue la part centrale et principale de ce qui est dit, ce qui est mis au premier plan dans la communication.  Quant à l'autre partie, en quelque sorte le contenu «hors-jeu», nous la désignerons par le terme général de \kwo{contenu projectif}.  
Celui-ci regroupe donc les éléments de sens dont relèvent les projections, et par exemple en \ref{x:FamCMCGe}, c'est ce sur quoi la question \emph{ne porte pas}. En effet, si on répond \sicut{non} à \ref{x:FamCMCGe}, cela ne remet pas en cause que Fred a un fils (mais seulement qu'il n'est pas allé le chercher).  De même c'est ce qui n'est pas nié en \ref{x:FamCMCGb} et ce qui n'est pas supposé en \ref{x:FamCMCGd}.

La distinction entre contenu en jeu et contenu projectif va jouer un rôle important dans cet ouvrage, ne serait-ce parce que nous nous concentrerons essentiellement sur le premier en laissant un peu en marge le second, pour des raisons qui seront données dans les lignes qui suivent.
Et c'est précisément pourquoi il importe ici de se donner une idée suffisamment claire de ce que sont les contenus projectifs, au moins pour bien savoir les identifier et, le cas échéant, les mettre de côté.  
Or il se trouve que ce terme, «contenu projectif», n'est pas spécialement courant en sémantique, notamment du fait qu'il recouvre un ensemble hétérogène de phénomènes\footnote{Sur ce sujet, voir \citet{TBRS:13} dont je reprends ici quelques éléments de terminologie et de classification.}.
En particulier les projections de \ref{xmoutarde}--\ref{xpanda} sont plus classiquement identifiées sous le nom de  \kwo{présuppositions}\indexs{presupposition@présupposition}.
On dit que, dans de tels exemples, les phrases (b) sont des présuppositions des phrases (a), ou que les phrases (a) présupposent les phrases (b)\footnote{Il est important de noter qu'ici \emph{présupposition} et \sicut{présupposer} sont des termes techniques qui renvoient à des notions spécifiques de la théorie et qui donc ne coïncident pas nécessairement avec l'usage ordinaire qu'ils peuvent avoir dans la langue quotidienne.}.
Les présuppositions constituent, depuis longtemps, un vaste domaine d'étude pour la sémantique, la pragmatique et la philosophie,  qui mérite que nous nous y attardions un peu dans les sous-sections qui suivent.
L'exemple \ref{xgonc} illustre, quant à lui, une catégorie de projections qui, depuis \citet{Potts:05}, sont désignées sous le nom d'\kwo{implicatures conventionnelles}.\is{implicature!\elid\ conventionnelle} Je ne vais pas expliquer immédiatement cette dénomination particulière car elle a eu une histoire terminologique fluctuante  et il sera plus cohérent d'y revenir en \S\ref{ss:implicatures}.
Prenons-la pour l'instant comme une simple étiquette  regroupant des projections qui se distinguent des présuppositions, comme nous le verrons en  \ref{sss:pIC}.



\subsubsection{Présuppositions sémantiques et pragmatiques}
%''''''''''''''''''''''''''''''''''''''''''''''''''''''''''
\label{sss:pspDf}

Il y a plusieurs manières d'appréhender le phénomène des présuppositions, ce qui fait qu'il n'en existe pas forcément une définition unique et consensuelle --~même si à peu près tout le monde s'accorde à leur reconnaître les mêmes propriétés remarquables.  
Nous allons ici retenir deux grands types d'approches : une vision que, pour faire simple, nous appellerons sémantique, et une vision pragmatique.  Nous verrons qu'elles ne s'opposent pas catégoriquement, mais plutôt qu'elles se complètent avantageusement.


L'approche dite sémantique des présuppositions remonte traditionnellement aux travaux de  P. Strawson\footnote{Voir \citet{Strawson:50fr}, sachant que Strawson n'y emploie jamais le terme de \emph{présupposition}, mais il y pose les bases de sa définition.}\Index{Strawson, P.}
et s'appuie sur ce que nous appellerons la \kwo{valeur de vérité}\indexs{valeur!\elid\ de vérité} d'une phrase (déclarative).  La valeur de vérité d'une phrase, c'est tout simplement le  \emph{vrai} ou
le \emph{faux}, selon que nous jugeons la phrase en question vraie ou
fausse (par rapport à un cas de figure donné).

\begin{defi}[Présupposition sémantique]
Une phrase $A$ présuppose une phrase $B$ ssi $B$ doit nécessairement être vraie pour que $A$ possède une valeur de vérité définie.  %Si $B$ est fausse, alors $A$ n'est ni vraie ni fausse.
\end{defi}


Posséder une valeur de vérité définie veut dire être vraie ou être fausse.  
Cette définition a un corollaire immédiat qui est que si $A$ présuppose $B$ et si nous nous trouvons dans une situation où $B$ est fausse, alors $A$ se retrouve sans valeur de vérité définie, c'est-à-dire que $A$ n'est ni vraie ni fausse. 
À ce titre, $A$ devient contextuellement inappropriée car sans pertinence. 
Reprenons, par exemple, \ref{xfreda} et supposons que nous sommes dans une situation où Fred n'a pas de fils, c'est-à-dire où \ref{xfredb} est fausse.  
Dans ce cas \ref{xfreda} ne peut pas être vraie, mais elle ne peut pas être fausse non plus (car cela voudrait dire qu'il n'est pas allé chercher son fils à l'école).  En fait nous voyons bien que si Fred n'a pas de fils, ça n'a tout simplement pas de sens d'énoncer \ref{xfreda}. 
Ce phénomène, que nous appelons le \emph{défaut de valeur de vérité}\indexs{valeur!\elid\ de vérité!défaut de \elid} (en anglais \alien{truth-value gap}),\label{p.tvgap}
donne lieu à une sévère anomalie sémantique mais aussi pragmatique.

Dans l'approche pragmatique, principalement développée par 
R. Stalnaker\footnote{Voir, entre autres, \citet{Stlnk:73,Stlnk:74}.},\Index{Stalnaker, R.} la présupposition n'est pas vue comme une relation entre phrases ou entre contenus, mais comme une attitude mentale des locuteurs et interlocuteurs d'une conversation.  
Selon cette vision, on ne dira pas qu'une phrase en présuppose une autre, ce sont les locuteurs qui présupposent telle ou telle phrase. 

\begin{defi}[Présupposition pragmatique]
Une présupposition est une phrase dont le locuteur tient la vérité pour acquise en supposant que les autres participants de la conversation font de même.
\end{defi}

Il s'agit là d'une définition parmi d'autres de la présupposition pragmatique, et de surcroît dans une version assez simplifiée%
\footnote{Citons pour information, deux définitions plus détaillées ; 
dans \citet[448]{Stlnk:73} :
«Un locuteur présuppose que $P$ 
à un moment donné de la conversation
%dans le cas où 
ssi
il est disposé à agir comme s'il tenait la vérité de
$P$ pour acquise, et comme s'il supposait que son auditoire reconnaît
qu'il agit ainsi»{} ; dans \citet[200]{Stlnk:74} :
«~$P$ est une 
présupposition pragmatique d'un locuteur
dans un contexte donné ssi le locuteur suppose ou croit que $P$,
suppose ou croit que son auditoire suppose ou croit que $P$, et
suppose ou croit que son auditoire reconnaît qu'il fait ces
suppositions ou a ces croyances».
}.  
Mais l'idée centrale est qu'une présupposition est une connaissance commune, admise  et partagée tacitement par les interlocuteurs dans le contexte d'une conversation. 
Toute communication langagière se déroule forcément sur un fond de présuppositions, qui constituent ainsi un arrière-plan informationnel indispensable sur lequel se produisent et s'interprètent les énoncés.  Par exemple, en écrivant ces lignes, je présuppose, entre autres choses, que mes lecteurs parlent et comprennent le français. 
En bref, les présuppositions sont ce qu'ensemble nous savons déjà (ou plus exactement ce que nous \emph{supposons} savoir communément). Il s'ensuit que les présuppositions sont des éléments constitutifs d'un contexte d'énonciation\is{contexte} et qu'elles y ont le statut d'évidences, des vérités que les interlocuteurs sont censés s'accorder à ne pas remettre en question. 

Un point commun essentiel entre la définition sémantique et la définition pragmatique est qu'une présupposition jouit d'une certaine antériorité par rapport à l'énoncé que l'on considère : les interlocuteurs savent (ou sont censés savoir) qu'elle est \emph{déjà} vraie.  Autrement dit, une présupposition est une information qui est manipulée comme \emph{préalablement vraie} dans le processus de compréhension d'un énoncé donné.  D'où le nom de «\emph{pré}-supposition».


Dans la vision sémantique, nous disons qu'une phrase $A$ présuppose une phrase $B$.  
Cette relation est habituellement \emph{déclenchée} par un élément formel de $A$, c'est-à-dire un mot ou une construction syntaxique ; à cet effet on parle généralement de \kwo{déclencheurs de présuppositions} pour désigner ces éléments\footnote{Dans \ref{xmoutarde}--\ref{xpanda}, les déclencheurs sont respectivement le verbe \sicut{savoir}, le possessif \sicut{son} et le verbe \sicut{regretter}.}.
Cela laisse entendre qu'il s'agit d'une propriété sémantique générale, quasiment grammaticale, attachée à $A$ en tant que phrase (plus qu'en tant qu'énoncé). La vision pragmatique aide à expliciter cette propriété. Dire que $A$ présuppose $B$ signifie que $B$ représente une information contenue dans $A$ mais qui est censée être déjà connue dans le contexte. Autrement dit, le sens de $B$ est une partie du sens de $A$ qui n'est pas nouvelle. Cela permet à $A$ de s'ancrer avec pertinence et cohérence dans le contexte en se reliant avec des choses que les interlocuteurs savent déjà ou ont en tête.

\subsubsection{Propriétés des présuppositions}
%'''''''''''''''''''''''''''''''''''''''''''''
\label{sss:ptépsp}

La vérité préalable des présuppositions est  ce qui explique leur comportement projectif; par exemple pourquoi elles résistent (ou échappent) à la négation.  Si elles sont déjà reconnues comme vraies, la négation qui apparaît dans la phrase ne peut pas les concerner ; elles sont hors d'atteinte.
Et c'est aussi ce qui explique, de la même façon,  leur projection des interrogatives et des conditionnelles que nous avons vue en \S\ref{ss:projections}. 

%*** un ensemble d'autres propriétés qui caractérisent leur comportement linguistique. Nous allons en commenter quelques unes.




Dans le même ordre d'idée, et pour les mêmes raisons, dans une conversation, une présupposition contenue dans une phrase déclarative n'est pas ce sur quoi l'on peut directement et simplement enchaîner (par exemple pour approuver ou pour objecter) :

\ex. \label{x:pspE}
\a. --- Marie regrette d'avoir accepté de garder mon panda pendant les
vacances.
\b. --- En effet. / Oui c'est vrai. / Je confirme.
\b.[b$'$.] --- Non, c'est pas vrai. / Je ne te crois pas.

Dans ce dialogue, les répliques (b) et (b$'$) ne portent que sur le regret de Marie, pas sur le fait qu'elle ait accepté de garder mon panda.
Si le second locuteur veut contester la présupposition du premier, il ne peut pas le faire avec une simple négation équivalant à \sicut{il est faux que}, il doit s'attaquer globalement à la parole de son interlocuteur par un commentaire revenant à \sicut{il n'est pas pertinent de dire que} : 

\ex.[\ref{x:pspE}] 
\a.[b$''$.] --- N'importe quoi, tu ne peux pas dire ça : elle n'a rien accepté du tout.

Ce faisant il signale que le contexte courant est défectueux car le premier locuteur y fait une présupposition qui, en fait, n'en est pas une (puisque les interlocuteurs ne sont pas d'accord à son sujet). 
Dans cet exemple, le second locuteur défait une présupposition du premier.  Dans certains cas, un locuteur peut lui-même révoquer une présupposition sémantique déclenchée par l'une de ses phrases, à condition que celle-ci soit négative :

\ex.
Fred n'est pas allé chercher son fils à l'école, puisqu'il n'a pas de fils.


Ce mécanisme dit d'\kwo{annulation}%
\footnote{Il est en fait plus approprié de parler de \emph{méta-annulation} (cf. \citet{Amsili:07}\Indexn{Amsili, P.} sur ce sujet) dans la mesure où cela fait intervenir une négation dite \emph{métalinguistique},\indexs{negation@négation!{\elid} métalinguistique}
\ie\ une négation qui ne nie pas le sens d'une phrase mais conteste la pertinence et la validité de l'énoncé en soi. }%
\indexs{annulation!\elid\ d'une présupposition} des présuppositions n'est pas toujours possible, mais il existe une autre manière par laquelle un locuteur peut faire disparaître une présupposition sémantique, par le phénomène de \kwo{suspension}.\indexs{suspension!\elid\ d'une présupposition}\label{p.suspen}  
Par exemple, comparons \ref{x:pspSa} et \ref{x:pspSb} :

\ex.
\a. Si Steve est toujours aussi colérique, je plains sa femme. \label{x:pspSa}
\b. Si Steve est marié, je plains sa femme. \label{x:pspSb}

\ref{x:pspSa} présuppose que Steve est marié à cause du groupe nominal \sicut{sa femme}, mais globalement \ref{x:pspSb}  ne le présuppose pas, précisément parce que cette information est suspendue, rendue hypothétique dans la proposition conditionnelle (\sicut{\emph{si} Steve est marié}).  
Pour dire les choses autrement, dans \ref{x:pspSb} la présupposition ne se projette pas, contrairement à ce que nous observions dans les exemples \ref{x:FamCMCG} de \S\ref{ss:projections}. Les présuppositions sont projectives parce qu'elles \emph{peuvent} se projeter (notamment de la famille de constructions du type de \ref{x:FamCMCG}) mais il y a des situations où elles ne le font pas.
Ce fait relève d'une problématique plus générale% (et centrale en sémantique)
, appelée la \kwo{projection des présuppositions},\indexs{projection!\elid\ d'une présupposition}\label{p.projpsp} et qui consiste à déterminer, lorsqu'une expression $E$ déclenche une présupposition $B$, ce qui est encore présupposé (ou projeté) par une expression plus grande qui inclut $E$. 
%On dira que la présupposition se projette au niveau de la phrase \ref{x:pspSa}, mais qu'elle ne se projette pas au niveau de \ref{x:pspSb}.

Comme annoncé en \S\ref{ss:projections}, %et comme le confirme ces propriétés,
dans le mécanisme de la communication, les contenus présupposés sont singulièrement hors jeu. 
À tel point que, comme l'explique Stalnaker, si le contenu de $B$ est pragmatiquement présupposé dans un contexte donné, alors il est particulièrement inapproprié de la part d'un locuteur d'affirmer simplement $B$, car dans la conversation ce serait «un coup pour rien».  S'il tient à mentionner le contenu de $B$, il est obligé de le faire en le présentant sous la forme d'une présupposition sémantique (en utilisant des déclencheurs adéquats).  C'est la réciproque de ce que nous avons évoqué ci-dessus : tout ce qui est sémantiquement présupposé est déjà connu, et tout ce qui est déjà connu doit être sémantiquement présupposé.
Il s'agit d'une règle élémentaire de pragmatique de la conversation. Et bien sûr, comme toute règle, il peut lui arriver d'être transgressée.  On ne doit normalement pas présupposer (sémantiquement) ce qui n'est pas une connaissance partagée des interlocuteurs, mais si on le fait, ce peut être une stratégie, parfois un peu sournoise, de faire passer une information nouvelle sur l'air du «tu es censé le savoir».  Il peut aussi arriver que le locuteur transmette plus innocemment une information nouvelle sous une forme présupposée, par mégarde, négligence ou simplement parce que c'est une information secondaire. Cela ne crée pas nécessairement de conflit dans la conversation et l'allocutaire peut alors choisir de réparer le contexte courant par le mécanisme dit d'\kw{accommodation}\label{p.accomm} (introduit par \citealt{Lewis:79}) qui consiste à intégrer rétroactivement l'information nouvelle parmi les présuppositions pragmatiques de la conversation comme si elle était connue dès le début\footnote{Il convient cependant de préciser que l'accommodation doit être vue comme une stratégie de dernier recours ou de «sauvetage» par l'allocutaire, car la règle qui dissuade de présupposer une information nouvelle dans le contexte reste normalement prioritaire dans le jeu pragmatique de la conversation.}.

Le fait qu'une présupposition pragmatique ne doit normalement pas être directement affirmée permet d'établir un %autre 
test%
\footnote{Ce test dérive d'observations faites notamment par \citet{Ducrot:72}, \citet{Lewis:79},
%ainsi que 
\citet{Krfk:93}.}\label{p.testAB}\is{test!\elid\ de présuppositions} 
pour vérifier qu'une phrase $B$ est bien (sémantiquement) présupposée par une phrase $A$ : si $A$ présuppose $B$,  l'enchaînement de phrases \sicut{$B$ et/mais $A$} nous paraîtra acceptable (si nous négligeons d'éventuelles maladresses stylistiques), alors que \sicut{$A$ et/mais $B$} sera perçu comme étrangement et péniblement redondant, une anomalie sémantique et pragmatique comparable aux pléonasmes.

\ex.\label{x:pspR1}
\a. Fred a un fils et il est allé chercher son fils à l'école.\label{x:pspRa} 
\b. \juge{\zarb} Fred est allé chercher son fils à l'école et il a un fils.\label{x:pspRb}


Dans la première partie de \ref{x:pspRa}, \sicut{Fred a un fils} est simplement affirmé, c'est donc une information nouvelle dans le discours.  Mais une fois que cette proposition est énoncée publiquement, elle devient une information connue de tous les interlocuteurs et donc une présupposition pragmatique.  Le locuteur peut alors énoncer la seconde partie qui présuppose légitimement que Fred a un fils.  
À l'inverse \ref{x:pspRb} commence par une phrase qui présuppose que Fred a un fils ; c'est donc une information connue dès le départ (ou \emph{accommodée} alors comme telle).  Énoncer ensuite \sicut{il a un fils} ne fait que la répéter inutilement.


\subsubsection{Présuppositions {\vs} implicatures conventionnelles}
%'''''''''''''''''''''''''''''''''''''''''''''''''''''''''''''''''
\label{sss:pIC}
\is{implicature!\elid\ conventionnelle}

En \S\ref{ss:projections}, les contenus projectifs ont été présentés comme comportant d'une part les présuppositions et d'autre part les implicatures conventionnelles.  Revenons un instant sur ces dernières pour montrer en quoi elles se distinguent des présuppositions.  Globalement, les implicatures conventionnelles servent au locuteur à véhiculer un commentaire sur le contenu \emph{en jeu} de son énoncé. \citet{Potts:05} les subdivise en deux grandes catégories : les contenus additionnels \ref{x:ICa1}--\ref{x:ICa2} et les contenus expressifs\is{contenu!\elid\ expressif} \ref{x:ICe1}--\ref{x:ICe2}.

\ex.
\a. Annie, \emph{qui était au lycée avec moi}, a reçu le prix Goncourt.\label{x:ICa1}
\b. \emph{Entre nous}, Annie n'écrit pas si bien que ça.\label{x:ICa2}
\b. \emph{Cet abruti de} Donald est en train de déjeuner.\label{x:ICe1}
\b. Assey\emph{ez-vous}, \emph{cher ami}.\label{x:ICe2}


Les contenus additionnels sont habituellement réalisés par des constructions dites appositives, parenthétiques ou incidentes, qui apportent un commentaire supplémentaire, souvent utile à la compréhension de l'ensemble du discours, mais relativement accessoire vis à vis du propos principal, comme une sorte de brève digression.  Les contenus expressifs sont ce par quoi le locuteur manifeste son ressenti ou une disposition personnelle par rapport à un ou plusieurs éléments du contenu en jeu.  Par exemple en \ref{x:ICe1}, le locuteur montre, par implicature conventionnelle, qu'il tient ce Donald en très basse estime.   Les ressentis ou dispositions ainsi exprimés peuvent être d'ordre affectif ou émotif mais aussi relever de signaux qui marquent un rapport de rang social, de respect ou de politesse\footnote{C'est ce qui est plus ou moins richement encodé dans les systèmes d'\emph{honorifiques}, présents dans des langues comme le japonnais, le coréen, le thaï... En français, on peut penser à l'opposition entre le vouvoiement et le tutoiement.}.




En tant que projections, les implicatures conventionnelles sont proches des présuppositions, mais on peut montrer cependant  qu'elles s'en distinguent sur plusieurs points essentiels.
Nous avons vu en \S\ref{sss:ptépsp} que, par le phénomène de suspension, des présuppositions peuvent parfois ne pas se projeter au niveau d'une phrase complexe. C'est ce qu'illustre cet autre exemple de suspension \ref{x:susp2} :

\ex.
Sylvie croit que Fred est allé chercher son fils à l'école. \label{x:susp2}


Cette phrase peut très bien être jugée pertinente et vraie dans un scénario où Fred n'a en fait pas de fils mais où Sylvie croit que c'est le cas (par exemple parce qu'elle l'a vu rentrer dans l'école et en a tiré une inférence rapide et erronée).  Selon une telle interprétation, \ref{x:susp2} globalement ne présuppose pas que Fred a un fils.  
À l'opposé, les implicatures conventionnelles, elles, se projettent \emph{toujours}.  Par exemple en \ref{x:suspic}, l'information sur la scolarité passée d'Annie ne peut pas être circonscrite aux croyances (éventuellement fausses) de Sylvie, elle est forcément assumée par le locuteur.

\ex.
Sylvie croit qu'Annie, qui était au lycée avec moi, a reçu le prix Goncourt.\label{x:suspic}


\sloppy

De plus, contrairement aux présuppositions, les implicatures conventionnelles ne semblent pas sujettes au défaut de valeur de vérité\indexs{valeur!\elid\ de vérité!défaut de \elid} (p.~\pageref{p.tvgap}).  Si une phrase $A$ a pour implicature conventionnelle un contenu $B$ et si nous sommes dans une situation où nous savons ou estimons que $B$ est faux, nous aurons tendance à juger que le locuteur de $A$ se trompe en partie, mais pas que l'énoncé $A$ est absurde ou sans pertinence.  Par exemple, même si nous savons que Thomas est en fait né à Rouen, nous considérerons que \ref{x:ICa2'} donne une information vraie, simplement entachée d'une inexactitude.

\fussy

\ex.
Thomas, \emph{natif du Havre}, a passé six mois et demi dans l'espace.\label{x:ICa2'}


Enfin, même si, comme nous l'avons vu p.~\pageref{p.accomm}, le mécanisme d'accommodation permet d'accepter une information nouvelle sous forme d'une présupposition, celle-ci se présente néanmoins sur le mode du «déjà connu».  
Au contraire, les implicatures conventionnelles sont plutôt destinées à transmettre des contenus qui ne sont pas préalablement partagés par les interlocuteurs.  On peut s'en rendre compte en étendant le test vu p.~\pageref{p.testAB} : si $B$ est une implicature conventionnelle de $A$, alors les \emph{deux} séquences \sicut{$B$ et/mais $A$} et \sicut{$A$ et/mais $B$} seront perçues comme anormalement redondantes.  C'est ce qui apparaît en \ref{x:icR1}, à  comparer  avec le test des présuppositions en \ref{x:pspR1}, p.~\pageref{x:pspR1}.\is{test!\elid\ de présuppositions}

\ex. \label{x:icR1}
\a. \juge{\zarb} Félix est le fils de Fred, et Félix, le fils de Fred, est rentré de l'école.
\b. \juge{\zarb} Félix, le fils de Fred, est rentré de l'école, et Félix est le fils de Fred.


C'est ainsi que nous pouvons établir la ligne de démarcation pragmatique la plus notable entre présuppositions et implicatures conventionnelles : les présuppositions s'occupent des contenus «anciens», déjà connus, formant l'arrière-plan communicationnel du contenu en jeu ; les implicatures conventionnelles se chargent des contenus nouveaux que le locuteur ajoute en contrepoint du contenu en jeu.


\largerpage[-1]
Pour conclure, ces observations sur le «hors-jeu» des contenus projectifs nous permettent de faire l'hypothèse que le sens véhiculé par une phrase ou un énoncé est structuré sur (au moins)  deux ou trois niveaux, deux ou trois dimensions communicatives.  
En ce qui concerne les présuppositions, nous aurons d'une part (et d'abord) ce qui est présupposé,\is{presuppose@présupposé} et d'autre part ce qui est en jeu, c'est-à-dire ce qui est réellement affirmé (ou questionné, demandé, etc.), que j'appellerai ici ce qui est \emph{proféré}%
\footnote{Je m'inspire ici du terme anglais \alien{proffered} judicieusement choisi par \citet{Roberts:96}.\Indexn{Roberts, C.}  C'est un faux-ami, sauf si l'on revient à l'étymologie latine qui signifie «porter en avant, présenter». Et ici, pour simplifier, \emph{proféré} et \emph{en jeu} seront utilisés comme renvoyant à la même chose.},\is{profere@proféré} 
autrement dit la partie centrale et cruciale de ce qui est dit. 
De la sorte, nous pouvons décomposer le contenu d'une phrase en séparant les deux modes de contribution sémantique dans l'ordre suivant :

\ex. Fred est allé chercher son fils à l'école.\label{x:xd1}
\a. présupposé : \emph{Fred a un fils.}
\b. proféré : \emph{Fred est allé le chercher à l'école.}

D'après cette  décomposition, la partie proférée se comprend \emph{à la suite de} et \emph{par rapport à} la présupposition, comme le montre le lien qu'établit le pronom \sicut{le}.   Cela apparaît d'autant plus dans le cas de phrases qui possèdent plusieurs présuppositions qui s'emboîtent un peu comme des poupées russes :

\ex.
Marie regrette d'avoir accepté de garder mon panda pendant les
vacances.\label{x:xd2}
\a. présupposé : \emph{J'ai un panda.}
\b. présupposé : \emph{Marie a accepté de le garder pendant les vacances.} (\sicut{le} = \sicut{mon panda})
\b. proféré : \emph{Marie le regrette.} (\sicut{le} = \sicut{d'avoir accepté de garder etc.})


Et il est possible de faire apparaître les implicatures conventionnelles en les dissociant du contenu en jeu de façon tout à fait similaire.  Elles interviendront sur une dimension parallèle à celle de ce qui est proféré et que nous pouvons appeler, par exemple, le \emph{commentaire}.\is{commentaire} 

\ex.
Annie, qui était au lycée avec moi, a reçu le prix Goncourt.\label{x:xd3}
\a. proféré : \emph{Annie a reçu le prix Goncourt.}
\b. commentaire : \emph{Annie était au lycée avec moi.}


Ainsi nous voyons que pour rendre compte, entre autres, du rapport sémantique entre \ref{x:xd1}, \ref{x:xd2} ou \ref{x:xd3} avec leurs formes négatives respectives, nous saurons que la négation n'opère que sur le proféré.

Comme les contenus projectifs ne sont pas l'élément principal de ce qui est dit et de ce qui est compris, la démarche que nous adopterons ici pour décrire et expliciter le sens des phrases se concentrera essentiellement sur leur partie proférée.  Cela ne veut pas dire que nous ignorerons complètement les contenus projectifs, mais simplement que nous ferons l'hypothèse de travail que 1) elles sont \emph{déjà là} dans le contexte lorsque nous interprétons une phrase, et 2) que les implicatures conventionnelles sont calculables en parallèle du contenu en jeu.   
C'est une démarche assez simplificatrice mais il faut avant tout la voir comme une étape préliminaire de l'analyse. Elle permet, méthodologiquement, d'alléger le fardeau de la sémantique (qui a déjà beaucoup à faire sans cela) en développant une théorie cohérente du sens des phrases qu'il est possible ensuite de perfectionner en y intégrant un traitement adéquat et compatible des présuppositions et des implicatures conventionnelles\footnote{L'option présentée ici de répartir le sens d'un énoncé sur plusieurs dimensions se retrouve, sous une forme ou une autre, dans divers travaux, comme \citet{KartPet:79}, \citet{vdS:92}, \citet{Roberts:96}, \citet{KampPrsp:01}, \citet{Potts:05}.}. 

\newpage

% -*- coding: utf-8 -*-
\begin{exo}\label{exo:1psp1}
Explicitez les projections de chacune des phrases suivantes.
\begin{enumerate}
\item %\ex.[\ref{x:expsp1}]
Si Pierre était venu, Marie serait partie.

\item %\ex.[\ref{x:expsp2}]
Hélène aussi fait de la linguistique.\label{x:expsp2}

\item %\ex.[\ref{x:expsp3}]
Hélène fait aussi  de la linguistique.\label{x:expsp3}

\item %\ex.[\ref{x:expsp4}]
Marianne a arrêté de fumer.

\item %\ex.[\ref{x:expsp5}]
Lorsque le téléphone a sonné, j'étais dans mon bain.

\item %\ex.[\ref{x:expsp6}]
Laurence a pris seulement une salade.

\item %\ex.[\ref{x:expsp7}]
Antoine n'est plus barbu.

\item %\ex.[\ref{x:expsp8}]
Jean a réussi à intégrer l'ENA.

\item %\ex.[\ref{x:expsp9}]
C'est Pierre qui a apporté des fleurs.\label{x:expsp9}

\item %\ex.[\ref{x:expsp10}]
Même Robert a eu la moyenne au partiel.\label{x:expsp10}

\item %\ex.[\ref{xmoutarde}]
Jean sait que c'est le colonel Moutarde le coupable.
\end{enumerate}
%--------
\begin{solu} (p.~\pageref{exo:1psp1})

La méthode la plus simple pour trouver les projections d'une
phrase $P$ est de la comparer avec sa forme négative en \sicut{il est
  faux que $P$}, comme ce que nous avons vu avec les exemples \ref{xmoutarde}--\ref{xgonc} en \S\ref{ss:projections}, p.~\pageref{p.resNeg}.

\begin{enumerate}
\item %\ex. \label{x:expsp1}
Projection : Pierre n'est pas venu. 

\item %\ex.  \label{x:expsp2}
Projection : Quelqu'un d'autre qu'Hélène fait de la linguistique.

\item %\ex.  \label{x:expsp3}
Projection : Hélène fait autre chose que de la linguistique.

\item %\ex.  \label{x:expsp4}
Projection : Marianne fumait avant.

\item %\ex. \label{x:expsp5}
Projection : Le téléphone a sonné.

\item %\ex.  \label{x:expsp6}
Projection : Laurence a pris une salade (c'est-à-dire la phrase sans \sicut{seulement} ;  remarquons que dans cette
phrase, le  contenu proféré est : Laurence n'a rien pris d'autre qu'une
salade). 

\item %\ex.  \label{x:expsp7}
Projection : Antoine a été barbu.

\item %\ex.  \label{x:expsp8}
Projections : Jean a essayé d'intégrer l'ENA (c'est-à-dire il s'est présenté au
concours d'entrée).  Mais il y a une seconde présupposition déclenchée par \sicut{essayer} qui est : il n'est pas facile d'intégrer l'ENA.

\item %\ex.  \label{x:expsp9}
Projection : Quelqu'un a apporté des fleurs.

\item %\ex.  \label{x:expsp10}
Projection : Robert faisait partie de ceux qui avaient le moins de chances d'avoir la moyenne au partiel.  Notons ici que le test de la négation s'applique un peu
difficilement. Il faut bien prendre soin d'utiliser la formulation en
\sicut{il est faux que}.

\item %\ex.[\ref{xmoutarde}] \a. 
Projections : 1) Quelqu'un est le coupable (c'est-à-dire  il y a un coupable) et 2)
le coupable est le colonel Moutarde. 
\end{enumerate}

Notons que la plupart des projections de cet exercice se trouvent être des présuppositions. Pour certaines, on peut s'en assurer assez facilement en appliquant le test de redondance (\S\ref{sss:ptépsp}, p.~\pageref{p.testAB}) ; pour d'autres, comme \ref{x:expsp9} et \ref{x:expsp10}, le test est moins concluant, ce qui peut soulever la question de leur statut véritablement présuppositionnel. Quant à \ref{x:expsp2} et \ref{x:expsp3}, le test fonctionne bien à condition d'utiliser une formulation plus précise du contenu projectif, par exemple \sicut{Lucie fait de la linguistique et Hélène aussi fait de la linguistique} {\vs} {\zarb}\sicut{Hélène aussi fait de la linguistique et Lucie fait de la linguistique}.

\end{solu}
\end{exo}


% -*- coding: utf-8 -*-
\begin{exo}\label{exo:1psp2}\sloppy
Beaucoup de présuppositions sont attachées à la structure «syntactico-sémantique» de la phrase, c'est-à-dire au choix de certains mots-outils (déterminants, conjonctions, adverbes...). Pour les identifier, il peut alors être pratique d'utiliser des mots (pleins) inventés, dont on ne connaît pas le sens.  On ne comprendra pas entièrement les énoncés observés, mais cela ne nous empêchera pas d'établir certaines relations de sens entre phrases qui contiennent ces mêmes mots inventés. Cela permet, par la même occasion, de neutraliser les inférences que nous pourrions tirer à partir de nos connaissances du monde, et ainsi de nous focaliser uniquement sur ce que déclenche la structure des phrases.

\fussy

Trouvez, s'il y en a, les présuppositions des phrases suivantes :
\begin{enumerate}
\item Il y a des verchons qui ont bourniflé.
\item Tous les verchons ont bourniflé.
\item Aucun verchon n'a bourniflé.
\end{enumerate}
\begin{solu}
(p.~\pageref{exo:1psp2})

Les résultats que nous pouvons tirer de cet exercice sont à prendre avec précaution car ils attendent des jugements sémantiques particulièrement fins, qui peuvent varier d'un locuteur à l'autre. Pour obtenir des conclusions plus assurées, il serait utile, par exemple, de mettre sur pied des expérimentations à grande échelle.  Nous allons donc ici seulement nous concentrer sur la méthode sous-jacente.  Celle-ci consiste, en premier lieu, à comparer les inférences que nous tirons de chaque phrase avec celles que nous tirons de leur négation.  Pour ces trois phrases, les présuppositions potentielles portent sur l'existence des verchons.
\begin{enumerate}
\item \emph{Il y a des verchons qui ont bourniflé.} 
Nous en inférons évidemment que les verchons existent.\\
Négation : \emph{Il est faux qu'il y a des verchons qui ont bourniflé.}
Certes, les contextes les plus naturels dans lesquels cette phrase peut être énoncée, sont ceux où l'on sait que les verchons existent ; cependant (et c'est ce qui importe ici) elle peut également l'être dans des cas de figure où les verchons n'existent pas.  En d'autres termes, si les verchons n'existent pas, il semblera assez légitime de dire que cette phrase est vraie. 

Il semble donc que l'expression \sicut{il y a des verchons} ne présuppose pas \emph{sémantiquement} qu'il existe des verchons (mais ça l'affirme).

\item \emph{Tous les verchons ont bourniflé.}  Si cette phrase est vraie, alors nous devons en inférer que les verchons existent (notamment du fait de l'emploi du passé composé).\\
Négation : \emph{Il est faux que tous les verchons ont bourniflé.} Cela signifie qu'il y a au moins un verchon qui n'a pas bourniflé, et donc que les verchons existent. 

Nous pouvons également nous aider du test de la redondance (\S\ref{sss:ptépsp}, p.~\pageref{p.testAB}).  Si nous comparons \sicut{les verchons existent et tous les verchons ont bourniflé} et \sicut{tous les verchons ont bourniflé et les verchons existent}, nous constatons que cette seconde phrase est particulièrement redondante.

Il est donc raisonnable de conclure que la phrase présuppose (sémantiquement) que les verchons existent.

\item \emph{Aucun verchon n'a bourniflé.}  Pouvons-nous inférer de (la vérité de) cette phrase que forcément les verchons existent ? Les jugements peuvent être fluctuants, mais nous pouvons estimer que si les verchons n'existent pas, alors la phrase sera vraie (comme pour la négation de la phrase 1 ci-dessus). Si tel est bien le cas, alors il n'est pas utile d'examiner la négation de la phrase\footnote{Celle-ci signifie qu'il y a des verchons qui ont bourniflé, ce qui entraîne directement l'existence des verchons.}, nous pourrons tout de suite conclure que la phrase ne présuppose pas que les verchons existent. Le test de la redondance va-t-il dans ce sens ? \sicut{Les verchons existent mais aucun verchon n'a bourniflé} est acceptable, comme prévu ; quant à \sicut{aucun verchon n'a bourniflé, mais les verchons existent} nous pouvons y voir une redondance, mais elle est probablement moins saillante que dans la version avec \sicut{tous les} en 2. 
\end{enumerate}
\end{solu}
\end{exo}




\indexs{presupposition@présupposition|)}



\subsection{Les implicatures conversationnelles}
%-----------------------------------------------
\label{ss:implicatures}
\indexs{implicature|(}

\subsubsection{Suppléments de sens}

Jusqu'ici 
nous avons essayé d'approcher quelques propriétés sémantiques des
phrases en nous servant d'un outil sérieux et intraitable : la
logique.  Cette façon de procéder a l'intérêt d'éviter de nous
laisser aveugler ou fourvoyer par des préjugés ou des intuitions
légères que nous pourrions avoir au sujet du sens des phrases.  
Pour autant, nous ne pouvons probablement pas tout expliquer dans le phénomène de compréhension seulement par %***
 le genre de relations logiques que nous avons
examinées précédemment.  
De plus, si nous devons prendre des précautions vis-à-vis de nos intuitions, il n'en est pas moins utile d'expliquer pourquoi et comment ces intuitions (ou moins certaines d'entre elles) existent.
C'est que nous allons faire ici en nous penchant sur un phénomène éminemment pragmatique qui peut se présenter sous la forme de relations de sens entre phrases.


En \S\ref{s:conseql}, nous avons montré que \ref{x12etub'} se trouve être une conséquence logique \ref{x12etua'}. 


\ex.\label{x12etu'}
\a.  Douze étudiants ont eu la mention TB.\label{x12etua'}
\b. Trois étudiants ont eu la mention TB.\label{x12etub'}

Cela provient du fait que \ref{x12etub'} doit être jugée vraie si le nombre d'étudiants qui ont eu la mention est supérieur ou égal à 3.  Il s'ensuit que \ref{x12etub'} est logiquement équivalente à \sicut{au moins trois étudiants ont eu la mention TB}. Autrement dit la logique nous oblige à considérer que, par la force des choses, \sicut{trois} est synonyme de \sicut{au moins trois}.   
Si nous peinons à nous en convaincre complètement, remplaçons les étudiants par des euros.  
Supposons que je flâne dans une librairie avec exactement douze euros sur moi en liquide (et pas de carte bancaire) et que je trouve un livre que je cherchais depuis longtemps et qui coûte trois euros.  Dans cette situation, il serait parfaitement ridicule que je m'exclame :

\ex.  Ah flûte ! Je ne peux pas l'acheter, je n'ai \emph{pas} \emph{trois} euros : j'ai \emph{douze} euros !


Nous n'avons pas le choix : si j'ai douze euros, la phrase \sicut{j'ai trois euros} est vraie.   Et pourtant, le plus souvent ce n'est pas ainsi que nous la comprenons, nous l'interprétons naturellement comme signifiant \sicut{j'ai exactement trois euros}\footnote{Pour être tout à fait précis, nous pouvons parfois la comprendre comme \sicut{j'ai trois euros et quelques centimes}, c'est-à-dire que, par ailleurs, nous nous autorisons à passer sous silence certaines quantités négligeables. Mais ce n'est pas vraiment ce qui est discuté ici. Et dans le cas de \sicut{trois étudiants} cet effet, évidemment, ne se produit pas.}, comme par exemple dans le contexte du dialogue :

\ex.
--- Tu as combien sur toi ?\\
--- Voyons voir... j'ai trois euros.


Notons que ce sens \sicut{exactement trois} peut se décomposer en \sicut{au moins trois} (qui est l'équivalence logique observée ci-dessus) complété de \sicut{pas plus de trois}.  
Ce complément ou ajout de sens est une inférence pragmatique qui, à la suite des travaux fondateurs du philosophe H. P. \citet{Grc:75}\Index{Grice, H. P.}, porte le nom d'\kwo{implicature conversationnelle}\indexs{implicature!\elid\ conversationnelle} --~à ne pas confondre avec les implicatures \emph{conventionnelles}  présentées \S\ref{s:presuppositions} (voir aussi \alien{infra} \S\ref{sss:ICbis})%
\footnote{À noter que, comme c'est assez souvent le cas dans la littérature, l'emploi du terme \emph{implicature} seul renverra ici toujours aux implicatures \emph{conversationnelles}.}.
Une implicature \emph{n'est pas} pas une implication (qui, comme nous l'avons vu, peut être une manière de nommer la conséquence logique), et il faudra bien prendre soin de bannir le verbe \sicut{impliquer} pour y renvoyer ; faute de mieux nous dirons qu'une implicature est «implicatée», ou sous-entendue\footnote{Terme qui permet ainsi de rejoindre la notion de \kwo{sous-entendu}\indexs{sous-entendu} développée par Ducrot (cf. par exemple \citealt{Ducrot:84})\Indexn{Ducrot, O.} et qui se trouve coïncider à peu près avec celle d'implicature de Grice.}.


\subsubsection{Les implicatures et principes conversationnels}
%'''''''''''''''''''''''''''''''''''''''''''''''''''''''''''''

\largerpage
La notion d'implicature repose sur l'idée qu'en s'exprimant, un locuteur en dit  généralement plus que ce que  disent ses phrases.  
Cela conduit à concevoir une distinction entre d'une part \emph{ce qui est dit}, que nous pouvons appelé le sens conventionnel voire littéral %
%\footnote{Cependant, il ne faut alors pas assimiler ce sens littéral à ce que l'on appelle en sémantique lexicale le sens propre ou sens premier, car celui-ci résulte justement d'une conventionnalisation.} 
et qui est, en sorte, le sens de la phrase, et d'autre part \emph{ce qui est communiqué}, qui est le sens visé par le locuteur et qui correspond ainsi au sens effectif de l'énoncé.  
Les implicatures conversationnelles font partie de ce sens communiqué et ainsi relèvent fondamentalement de la pragmatique, puisqu'elles caractérisent la compréhension d'un énoncé.
En voici une définition générale.



\begin{defi}[Implicature conversationnelle]\label{d:ic}
Une \kwo{implicature conversationnelle}\indexs{implicature!\elid\ conversationnelle} est une inférence que l'on tire par défaut en raisonnant à partir du sens d'une phrase, d'hypothèses sur le contexte\is{contexte} et des «règles du jeu» de la conversation.
\end{defi}


«Par défaut» signale qu'il s'agit d'une inférence qui a normalement cours \emph{sauf} si l'on se trouve dans une situation où l'on a de bonnes raisons de ne pas la tirer.  Autrement dit, une implicature, contrairement à la conséquence logique, est une relation de sens qui «ne marche pas à tous les coups», elle peut ne pas avoir lieu, c'est une sorte de pari que nous faisons en interprétant un énoncé. 
Cela est en partie dû à ce qu'elle ne découle pas automatiquement du sens conventionnel de la phrase ; elle s'obtient, comme l'indique la définition, par un raisonnement systématique, une sorte de calcul, qui exploite plusieurs sources d'information, et notamment ce que j'ai appelé les règles du jeu de la conversation. 

Ces règles ont été mises au jour par Grice en partant du constat que la communication langagière est un processus \emph{rationnel et coopératif}.
Dans un échange verbal, les interlocuteurs sont naturellement attachés à ce que la communication réussisse.  Ils sont disposés en quelque sorte à s'associer pour atteindre leur objectif ou pour progresser dans la direction qu'ils ont acceptée de prendre dans leur conversation.  
Et parmi ces objectifs, il y a %bien sûr 
celui de bien se comprendre mutuellement. 
C'est ce que Grice définit comme le \emph{principe de coopération},\indexs{coopération (principe de \elid)} qui s'applique normalement, naturellement et par défaut dans une conversation.
Quelqu'un qui ne serait pas capable de suivre et reconnaître ce principe s'exclurait de lui-même de la communication (un peu comme vouloir jouer au basket en s'attachant les mains dans le dos).

En pratique, la coopération dans la communication consiste à se conformer à une série de règles, des \kwo{maximes}\indexs{maxime conversationnelle} (c'est-à-dire de petits commandements que l'on s'assigne à soi-même) qui organisent et disciplinent le mécanisme de la conversation. Converser est donc un peu comme jouer à un jeu dont voici les règles élémentaires : 

\begin{defi}[Maximes conversationnelles]
\begin{description}[itemsep=0pt]
\item[Quantité :] (a) Fais en sorte que ta contribution soit aussi
informative que nécessaire (pour l'échange linguistique en cours).
(b) Fais en sorte que ta contribution ne soit pas plus informative que
nécessaire. 
\item[Qualité :] (a) Ne dis pas ce que tu crois être faux. (b) Ne dis pas ce
pour quoi tu manques de preuve.
\item[Relation :]  Sois pertinent.  %(ne passe pas du coq à l'âne).
\item[Manière :] (a) Évite d'être obscur. (b)  Évite d'être ambigu. (c)
Sois bref. (d) Sois ordonné.
\end{description}
\end{defi}

Ces maximes peuvent nous sembler banales et évidentes, mais c'est justement parce nous les connaissons tous. Elles sont universelles et prennent racine dans cette qualité évolutive qu'est notre instinct de coopération.  Elles sont caractéristiques de notre coopération mais aussi de notre rationalité : quelqu'un qui enfreint ces maximes \emph{sans motif explicable} nous semblera un peu dérangé ou du moins inapte à la communication. %\footnote{Voir par exemple \citet[]{Levinson:83}}
La précision de «sans motif explicable» (ou sans raison valable) est fondamentale car, précisément, le moteur de la théorie de Grice est que, même si nous savons que nous sommes supposés respecter ces maximes, nous passons notre temps à y déroger ; mais ces transgressions ne sont qu'apparentes et superficielles.  Et c'est en sachant cela que nous sommes en mesure de «calculer» les implicatures qui se greffent sur un énoncé%
\footnote{Bien entendu, les menteurs\indexs{mensonge} violent délibérément les maximes (en particulier celles de qualité), dans une stratégie qui est fondamentalement non coopérative. Mais les menteurs sont des tricheurs du jeu de la conversation et, pour atteindre leur objectif, les tricheurs compétents se doivent de connaître et maîtriser les règles du jeu.  Autrement dit, dans le cas du mensonge (surtout s'il est habile), le principe de coopération reste présent, à ceci près qu'il est \emph{simulé} malhonnêtement par le locuteur. C'est différent de l'ironie\indexs{ironie} (qui peut être vue comme un cas d'implicature conversationnelle) où le locuteur prononce quelque chose de faux mais en comptant sur le principe de coopération pour que son allocutaire reconnaisse l'intention initiale de communiquer le contraire de ce qui est dit.}.   



\subsubsection{Calcul des implicatures}
%''''''''''''''''''''''''''''''''''''''

Dans ses grandes lignes le mode de raisonnement qui fait émerger les implicatures conversationnelles peut se résumer de la manière suivante.
Nous partons d'abord de l'hypothèse (ou du constat) que nous nous trouvons dans un contexte où le locuteur n'a pas de raison de ne pas respecter les maximes conversationnelles.  Ensuite nous constatons que ce qui est dit par le locuteur (c'est-à-dire le sens conventionnel de sa phrase) semble transgresser une ou plusieurs maximes. Mais comme, par hypothèse, il est coopératif, c'est que les maximes sont transgressées en surface seulement, et qu'elles sont respectées à un niveau plus profond de la communication.  À partir de là, nous en inférons un élément de sens qui complète ou rectifie le sens de la phrase initiale en respectant les maximes apparemment enfreintes.  Cet élément de sens est l'implicature de la phrase dans le contexte courant.

Il y a plusieurs façons de mettre \oe uvre un tel raisonnement, selon ce qui est dit dans la phrase, les maximes transgressées, le contexte, etc.  En voici une application assez courante qui nous sera utile dans les illustrations qui suivent.

\ex. \label{raisonIC}
\a. Le locuteur énonce la phrase $A$ dans un contexte C ;
\b. $B$ est une phrase alternative, pertinente dans C et compatible avec $A$ ; %(par ex. $B\satisf A$) ;
\b. si $B$ était vraie dans C (ou tenue pour vraie par le locuteur), alors en disant $A$, et pas $B$, le locuteur transgresserait, dans C, des maximes conversationnelles ; \label{raisonICc}
\b. par hypothèse, le locuteur est  rationnel et coopératif (\ie\ il respecte les maximes) ;
\b. nous en inférons alors que $B$ est fausse dans C (ou que le but du locuteur est de ne pas présenter $B$ comme vraie ou valable) ;
\b. donc $\neg B$ est l'implicature %conversationnelle 
de $A$ dans le contexte C. 


Ce raisonnement sert notamment à identifier un certain type d'implicatures très courantes et beaucoup étudiées, appelées \kwo{implicatures scalaires}\footnote{Cf.\ \citet{Horn:72,Horn:89}.\Indexn{Horn, L.}}.\indexs{implicature!\elid\ scalaire}  L'implicature est ici la négation d'une ou plusieurs alternatives $B$ plus informatives que $A$.  
En bref, elle procède de la devise «si $B$ était vraie, le locuteur l'aurait dite».
Et c'est ainsi que nous pouvons expliquer l'implicature de notre exemple initial \ref{x12etub'} \sicut{trois étudiants ont eu la mention TB}.  Dans une situation où quatre étudiants ont eu la mention, \ref{x12etub'} est vraie mais n'est pas optimalement coopérative (cf.\ la première maxime de quantité) ; de même pour cinq, six, sept... étudiants.  L'implicature est donc qu'il n'y a pas quatre, ni cinq, ni six... étudiants qui ont la mention, autrement dit : pas plus de trois.

\sloppy

En guise d'illustration, voici quelques exemples d'implicatures conversationnelles.  Dans chaque groupe de phrases, l'implicature de (a) est donnée en (c), et (b) indique une alternative possible utilisée pour le calcul et présentée de telle sorte que (c) est généralement équivalente à la négation de (b).

\fussy

\largerpage[-1]

\newcommand{\IC}{\textsc{ic}}

\ex.
\a. Quelques étudiants ont bien réussi l'examen.
\b. $B$ : Tous les étudiants ont bien réussi l'examen.
\b. {\IC} : \emph{Les étudiants n'ont pas tous bien réussi l'examen}, c'est-à-dire :
\emph{il y a des étudiants qui n'ont pas bien réussi l'examen.}

\ex.
\a. J'ai croisé une femme dans les escaliers.
\b. $B$ : J'ai croisé ma femme/ma fiancée/ma s\oe ur/ma mère/une amie/une voisine/...
\b. {\IC} : \emph{J'ai croisé une femme que je ne connais pas.}

\ex.
\a. Pierre et Anne sont allés au cinéma hier soir.
\b. $B$ : Pierre et Anne sont allés au cinéma séparément.
\b. {\IC} : \emph{Ils y sont allés ensembles.}

\ex.
\a. Léa a trouvé un boulot au Parlement Européen et elle a déménagé à Strasbourg. 
\b.  $B$ : Léa a trouvé un boulot au Parlement Européen après avoir déménagé à Strasbourg. 
\b. {\IC} : \emph{Léa a %d'abord 
trouvé un boulot au Parlement Européen, puis elle a déménagé à Strasbourg.}

\ex.
\a.  Je crois que Sylvie est partie.
\b. $B$ : Je sais que Sylvie est partie.
\b. {\IC} : \emph{Je n'en suis pas sûr.}

\ex.
\a. Paul croit que Sylvie est partie.
\b. $B$ : Paul sait que Sylvie est partie.
\b. {\IC} : \emph{Paul se trompe.}



Les raisonnements comme \ref{raisonIC} sont très dépendants du contexte où la phrase est énoncée, comme le montre en particulier la condition \ref{raisonICc} où le jugement qu'une maxime est transgressée n'est pas absolu et prend place au milieu d'informations et d'hypothèses qui proviennent du contexte C : une même phrase $A$ peut violer des maximes dans un contexte mais pas dans d'autres. C'est pour cela que les implicatures conversationnelles sont des inférences \emph{défectibles}, c'est-à-dire qu'elles ne s'obtiennent pas toujours.  En d'autres termes, elles peuvent être \emph{annulées}\indexs{annulation!\elid\ d'une implicature} explicitement par le locuteur : 

\ex.
\a. Oui Bertrand a trois enfants. Il en a même six.   
\b. Plusieurs étudiants ont eu la moyenne au partiel.  En fait ils ont tous eu la moyenne.


Elles peuvent également être \emph{suspendues}\indexs{suspension!\elid\ d'une implicature} dans certains environnement grammaticaux, comme %en 
\ref{x:suspIC} qui ne se comprend pas comme \sicut{si vous avec exactement trois enfants, vous avez le droit à une réduction} (ce qui serait le cas si l'implicature était maintenue).

\ex. \label{x:suspIC}
Si vous avez trois enfants, vous avez le droit à une réduction.


L'annulation et la suspension nous font penser aux présuppositions, mais elles ne se produisent pas dans les mêmes conditions avec les implicatures\footnote{Voir là encore \citet{Amsili:07}.\Indexn{Amsili, P.}}. D'autre part, et contrairement aux présuppositions et aux conséquences logiques, les implicatures sont également \emph{renforçables}, c'est-à-dire que le locuteur peut expliciter le contenu implicaté sans créer de redondance pragmatique.  C'est ce que montre \ref{x:renfICa}, à comparer avec \ref{x:renfICb} qui explicite une conséquence logique de la première phrase.

\ex. 
\a.  
Plusieurs étudiants ont réussi l'examen, mais pas tous. \label{x:renfICa}
\b.  Plusieurs étudiants ont réussi l'examen, \zarb et pas aucun. \label{x:renfICb}


Mentionnons également qu'il est courant, depuis Grice, de diviser les implicatures conversationnelles en deux grandes sous-catégories.  Il y a d'une part les \kwo{implicatures conversationnelles généralisées},\indexs{implicature!\elid\ conversationnelle!\elid\ généralisée} ce sont celles que nous avons examinées jusqu'ici (et qui sont le plus souvent étudiées en pragmatique formelle). Elles sont \emph{régulières}, en ce sens que \emph{lorsqu'}elles se produisent, leur calcul donne toujours le même résultat.
D'autre part, les \kwo{implicatures conversationnelles particularisées},\indexs{implicature!\elid\ conversationnelle!\elid\ particularisée} ont, elles, au contraire, un contenu qui dépend spécifiquement du contexte d'énonciation ; il est parfois assez difficile de les deviner en n'examinant que la phrase qui les déclenche.  \citet{Levinson:83}\Index{Levinson, S.} donne ainsi l'exemple \ref{x:ICP1}, auquel nous serions bien en peine d'associer une implicature sans plus d'information.

\ex. \label{x:ICP1}
Le chien a l'air content.

Mais si \ref{x:ICP1} est donnée en réponse à la question \sicut{Mais où est donc passé le rôti ?}, nous pourrons comprendre que la phrase suggère une explication en sous-entendant que le rôti a été probablement mangé par le chien (et que c'est pour cela qu'il a l'air si content).

\subsubsection{Rapide retour sur les implicatures conventionnelles}
%''''''''''''''''''''''''''''''''''''''''''''''''''''''''''''''''''
\label{sss:ICbis}

\sloppy

Revenons un instant sur les implicatures conventionnelles\is{implicature!\elid\ conventionnelle}, présentées en \S\ref{sss:pIC}, pour faire un petit point de clarification épistémologique.  
Ce terme d'implicature conventionnelle a été également introduit par \citet{Grc:75}, avec cette idée que les implicatures en général (conversationnelles et conventionnelles) désignent un effet de sens qui n'est pas capté par les propriétés logiques de la phrase.  
Mais au delà de ce point commun, les implicatures conventionnelles s'opposent diamétralement aux conversationnelles, en ce qu'elles ne sont pas inférées par raisonnement mais codées dans le lexique ou la grammaire (d'où leur nom de \emph{conventionnelles}), et qu'elles ne sont ni annulables, ni suspendables, ni renforçables.  
Les exemples typiques %d'implicatures conventionnelles 
que donne Grice sont l'idée de concession ou d'opposition véhiculée par \sicut{mais} et celle de conséquence accompagnant \sicut{donc}.  
Nous pouvons y ajouter les valeur concessives d'autres connecteurs comme \sicut{pourtant}, \sicut{cependant}, et les valeurs additives de \sicut{de plus}, \sicut{en outre}.

\fussy

Les implicatures conventionnelles ne jouent pas un rôle central dans la théorie de Grice ; il les identifie principalement pour signaler des phénomènes interprétatifs particuliers à tenir à l'écart de ceux dérivés des maximes conversationnelles.
Par la suite, plusieurs travaux se sont attachés à raviver l'intérêt de leur étude.  Leur caractère projectif (cf. \S\ref{ss:projections}) a amené, pendant un certain temps, à les assimiler simplement aux présuppositions%
\footnote{Par exemple, \citet{KartPet:79}, contribution majeure sur le problème de la projection des présuppositions, est intitulé «Conventional implicature».}.
 Mais, nous l'avons vu (\S\ref{sss:pIC}), cette assimilation s'est avérée impropre, et la notion d'implicature conventionnelle a progressivement gagné en autonomie, notamment en voyant s'étendre sa couverture empirique\footnote{Voir, par exemple, \citet[127--132]{Levinson:83} et \citet[chap.~2]{Potts:05}.}: on a pu y ajouter les commentaires parenthétiques, les  valeurs expressives, émotives et affectives  d'adverbes ou d'épithètes, certaines interjections, les honorifiques et autres signaux de relations sociales ou interpersonnelles, etc.
De la sorte, le terme d'\emph{implicature conventionnelle} s'est finalement réactualisé pour prendre une acception plus large que ce que visait  probablement Grice initialement. 


\bigskip

\sloppy

Pour conclure, il faut reconnaître que les implicatures (conversationnelles, mais aussi conventionnelles) jouent un rôle fondamental dans la compréhension des énoncés, ne serait-ce que par leur grande fréquence dans la pratique langagière.  En complétant le sens littéral et conventionnel des phrases, elles permettent d'expliciter ce qui relève du non-dit, du sous-entendu, du  sens «entre les lignes».  Elles relèvent prioritairement du champ de la pragmatique, en particulier parce que ce sont des inférences qui ne sont pas inéluctables, elles peuvent être annulées ou suspendues.  Mais ce ne sont pas pour autant des associations d'idées énigmatiques : elles procèdent de raisonnements relativement bien identifiés.  Dans les chapitres suivants,  un peu comme avec les présuppositions,  nous laisserons les implicatures de côté pour nous concentrer sur le sens conventionnel des phrases, précisément parce que ces raisonnements qui les font émerger exploitent ce sens conventionnel.  Autrement dit, le calcul pragmatique des implicatures conversationnelles intervient \emph{après} la construction sémantique du sens «de base»\footnote{Cela ne veut pas dire que le calcul des implicatures doive nécessairement attendre que le sens de la phrase soit complètement construit, il peut s'effectuer localement sur des parties interprétées d'une phrase, mais toujours sur la base d'un sens conventionnel.}. Il est donc méthodologiquement sain et cohérent de commencer par faire le point sur la dimension strictement sémantique (hors contexte).  
Par la même occasion, en divisant le travail de la sorte, cela soulage encore la sémantique d'un certain nombre de complications qu'il ne serait pas utile d'aborder à ce niveau de l'analyse.
En particulier, la théorie des implicatures nous dispense de postuler des ambiguïtés lexicales ou grammaticales artificielles, comme celle qui associerait à \sicut{trois} tantôt le sens de \sicut{au moins trois} tantôt celui de \sicut{exactement trois}\footnote{Nous pouvons montrer que \sicut{trois} n'est pas sémantiquement ambigu en reprenant le test de l'ellipse.\is{test!\elid\ des ellipses} Dans un contexte approprié, la phrase \sicut{J'ai trois euros, toi aussi (nous pouvons donc aller à la piscine)} peut être énoncée si j'ai exactement trois euros et que mon allocutaire en a cinq.}. 

\fussy

\indexs{implicature|)}


\bigskip
%% Exercice
\begin{exo}[Relations sémantiques]\label{exo:1RelSem}
Pour chaque paire suivante, indiquez quelle est la \emph{relation de sens}
qui relie la phrase (a) à la phrase (b).

\begin{enumerate}
\item\begin{enumerate}
\item Jack est retourné en Pantagonie.
\item Jack a  déjà été en Patagonie.
\end{enumerate}

\item\begin{enumerate}
\item Charlie a failli réveiller Lucy. 
\item Charlie n'a pas  réveillé Lucy. 
\end{enumerate}

\item\begin{enumerate}
\item J'ai des amis qui aiment le chocolat.
\item J'ai des amis qui n'aiment pas le chocolat.
\end{enumerate}

\item\begin{enumerate}
\item J'ai des amis qui aiment le chocolat.
\item Je n'ai pas d'amis qui aiment le chocolat.
\end{enumerate}

\item\begin{enumerate}
\item Patty est déjà allée en Inde.
\item Il est faux que Patty n'a jamais été en Inde.
\end{enumerate}

\item\begin{enumerate}
\item François est allé à New York.
\item François n'est pas allé à New York à la nage.
\end{enumerate}
\end{enumerate}

\begin{solu}
(p.~\pageref{exo:1RelSem})

\begin{enumerate}
\item Présupposition.
\item Conséquence logique.
\item Implicature conversationnelle (scalaire).
\item Contradiction.
\item Équivalence logique.
\item Implicature conversationnelle (particularisée).
\end{enumerate}
\end{solu}
\end{exo}



\subsection{Les actes de langage}
%--------------------------------
\label{s:spacts}
\indexs{acte de langage|(}

Nous venons de voir que les implicatures conversationnelles rendent compte de la distinction entre ce qui est dit et ce qui est communiqué.  Nous pouvons la reformuler en parlant de la distinction entre ce que \emph{veut dire} une phrase et ce que \emph{veut dire} le locuteur qui l'énonce. 
En jouant ainsi sur la signification de \sicut{vouloir dire} ({«avoir pour sens»} {\vs} «avoir l'intention de communiquer»),  nous replaçons le locuteur et ses intentions au c\oe ur de la problématique du sens des énoncés. 
C'est central pour la théorie des implicatures et ça l'est probablement encore plus pour celle des \kwo{actes de langage}%
\footnote{Le terme anglais universellement adopté depuis \citet{Aus:65Frb} est \alien{speech acts}. Diverses traductions françaises se rencontrent dans la littérature, comme \emph{actes de discours}, \emph{actes de parole}, \emph{actes d'énonciation},  \emph{actes de langage} ; je reprends ici celle qui est la plus fréquemment utilisée et qui se retrouve dans la traduction du titre de \citet{Sea:69Frb}.}%
.\indexs{acte de langage}

Cette théorie, initiée par J. L. \citet{Aus:65Frb}\Index{Austin, J. L.} puis reprise et systématisée par J.~\citet{Sea:69Frb},\Index{Searle, J.} se fonde sur le principe que parler, c'est avant tout \emph{agir socialement}.  Cela permet d'appréhender la compréhension et la question du sens en étudiant les énoncés comme le produit d'une action et en s'interrogeant sur ce que fait un locuteur lorsqu'il énonce quelque chose.

Le point de départ fameux de l'étude d'Austin est d'attirer l'attention sur une catégorie d'énoncés qui  fonctionnent de manière assez particulière, que l'on appelle les énoncés \kwo{performatifs}\indexs{performatif}, illustrés en \ref{x:perfo1} (les exemples sont d'Austin).

\ex. \label{x:perfo1}
\a. Je baptise ce bateau le Queen Elizabeth.
\b. Je donne et lègue ma montre à mon frère.
\b. Je vous unis par les liens du mariage.
\b. Je vous parie six pence qu'il pleuvra demain.
\e. Je promets de venir.


Un performatif est un énoncé qui accomplit ce qu'il dit par le simple fait de le dire.  Nous voyons ainsi que les phrases de \ref{x:perfo1} ne se contentent pas d'accompagner une certaine action, elles en sont une condition indispensable et fondatrice.   Comme le déclare la traduction française du titre d'Austin, «dire, c'est faire».  Mais toute phrase à la première personne qui met en scène le locuteur dans une certaine activité ne donne pas systématiquement lieu à un performatif.  Par exemple, si effectivement l'énonciation de \ref{x:perfo2a} vaut en soi pour une félicitation, on n'insultera jamais personne en énonçant \ref{x:perfo2b}.

\ex.
\a. Je te félicite. \label{x:perfo2a}
\b. Je t'insulte. \label{x:perfo2b}


En étudiant les conséquences logiques, les ambiguïtés, les présuppositions et même les implicatures, nous nous interrogions sur les cas de figure où des phrases pouvaient être jugées vraies ou fausses.  Avec les performatifs, cette question semble hors sujet, ou du moins très secondaire.  Ce qui compte ici, c'est plutôt de savoir si l'action décrite par un performatif a réellement été accomplie lors de son énonciation.  Cela justifie l'intérêt à décrire ces énoncés en explicitant leurs conditions de «félicité» c'est-à-dire de réussite et de légitimité. 


Mais il faut remarquer que les énoncés ordinaires, non performatifs, qu'Austin nomme \kwo{constatifs}\indexs{constatif}, réalisent eux aussi des actions, même si celles-ci ne sont pas décrites explicitement par les phrases énoncées.  Par exemple, en énonçant \ref{x:AL1a}, le locuteur fait ce que nous appellerons une certaine \kw{assertion} (c'est-à-dire une affirmation), avec \ref{x:AL1b} il pose une certaine question\is{question} et avec \ref{x:AL1c} il donne un certain ordre. 

\ex. \label{x:AL1}
\a. J'ai oublié d'acheter du pain. \label{x:AL1a} %$\longrightarrow$ le locuteur fait une affirmation/assertion
\b. À quelle heure est la réunion ? \label{x:AL1b} %$\longrightarrow$ le locuteur pose une question
\b. Range ta chambre ! \label{x:AL1c}  %$\longrightarrow$ le locuteur donne un ordre


Ces actions semblent conventionnellement associées à la forme (grammaticale) des phrases et appartenir seulement à la sphère linguistique et communicative (contrairement aux actions de \ref{x:perfo1}). 
Mais cette perspective rend possible la généralisation qui consiste à étudier ces énoncés de la même façon que les performatifs, en décrivant leurs conditions de félicité qui sont, en quelque sorte, leur «mode d'emploi» linguistique. 

Les actes de langage sont donc ces actions qu'accomplit le locuteur en s'exprimant (qu'elles soient comme celles de \ref{x:perfo1} ou de \ref{x:AL1}) ou, si l'on veut, ce sont les énoncés vus globalement comme des actions.  
Austin les décrit comme la réunion indissociable de trois actes% 
\footnote{Il est important de noter que ces trois actes ne sont pas des actions indépendantes qui se produiraient simultanément (comme regarder la télé en mâchant du chewing-gum ) ; il faut les voir comme trois facettes constitutives d'\emph{un même}  acte.} :

\begin{itemize}
\item un \kwo{acte locutoire}\is{acte de langage!\elid\ locutoire} qui est l'acte concret d'énonciation, la production d'une expression linguistique avec une forme (phonétique ou graphique), une structure (grammaticale) et un sens (conventionnel) ;
\item un \kwo{acte illocutoire}\is{acte de langage!\elid\ illocutoire} qui est l'acte de faire une assertion, poser une question, émettre une demande, donner un ordre ou un conseil, promettre, remercier, féliciter, etc. ; il est souvent caractérisé comme étant l'acte accompli \emph{en} disant ce qui est dit ;
\item un \kwo{acte perlocutoire}\is{acte de langage!\elid\ perlocutoire} qui est l'acte de causer ou provoquer un effet, par exemple sur l'auditoire, comme informer, convaincre, obliger, émouvoir, faire peur, séduire, amuser, rassurer, etc. ; il est caractérisé comme l'acte obtenu \emph{par} le fait de dire ce qui est dit.
\end{itemize}

\smallskip

La pragmatique s'intéresse principalement aux actes illocutoires, car c'est à leur niveau que se situe le sens que le locuteur souhaite communiquer par son énoncé.  Les actes illocutoires sont traditionnellement décrits au moyen de deux ingrédients fondamentaux : une certaine \kwo{force illocutoire}\indexs{force!\elid\ illocutoire} et un \kwo{contenu propositionnel}.\is{contenu!\elid\ propositionnel}  
La force illocutoire est ce qui correspond aux étiquettes \sicut{assertion},\is{assertion} \sicut{question},\is{question} \sicut{requête}...\is{requête} que nous avons utilisées ci-dessus.  Nous pouvons généralement la faire apparaître à travers le verbe que nous choisissons si nous relatons «en vue objective» un acte de langage dont nous avons été témoin.  Par exemple, pour les actes de \ref{x:AL1}, en supposant que le locuteur s'appelle Loïc et l'allocutaire Aline, nous pourrons relater :

\ex. \label{x:AL2}
\a. Loïc \emph{affirme} à Aline qu'il a oublié d'acheter du pain.
\b. Loïc \emph{demande} à Aline à quelle heure est la réunion.
\b. Loïc \emph{ordonne} à Aline de ranger sa chambre.

Dans sa définition (en termes de conditions de félicité), une force illocutoire est toujours caractérisée, entre autres choses, par une certaine \emph{intention}\indexs{intention} du locuteur.  Par exemple, avec une assertion\is{assertion} il a l'intention d'apprendre quelque chose à l'allocutaire, avec une question\is{question} celle d'obtenir une information, avec un ordre\is{ordre (acte de langage)} ou une requête\is{requête} celle de voir une action être accomplie, etc.

Le contenu propositionnel, quant à lui,  serait l'ensemble d'informations sur lequel portent l'assertion, la question, l'ordre... et en \ref{x:AL2} il correspond à ce qui est exprimé par les subordonnées. 

À partir de là, nous pouvons voir comment la question du sens (pragmatique) s'articule dans la théorie des actes de langage.  En fait, comprendre un énoncé, c'est reconnaître l'acte illocutoire\is{acte de langage!\elid\ illocutoire} que le locuteur accomplit effectivement dans le contexte, autrement dit, c'est retrouver l'intention qu'il avait en énonçant ce qu'il dit.  Cela implique que la compréhension a pour tâche de restituer la force illocutoire et le contenu propositionnel en jeu.

Le contenu propositionnel \emph{semble} dériver assez directement du sens conventionnel composé dans l'acte locutoire, et parmi les questions majeures qui occupent les études sur les actes de langage, il y a surtout celles de savoir comment la force illocutoire est encodée dans l'énoncé, comment elle est reconnue dans la compréhension et dans quelle mesure le contenu propositionnel (lié au sens littéral récupéré via la grammaire) et la force illocutoire sont dissociés dans les mécanismes linguistiques.  Ce sont des problématiques plus complexes qu'il n'y paraît et qui ont donné lieu à diverses approches théoriques souvent très concurrentes. 
Nous n'entrerons pas ici dans le détail de ces débats\footnote{Voir \citet[chap. 5]{Levinson:83} pour un panorama approfondi.}, nous nous contenterons de faire quelques remarques générales sur l'illocution et de donner quelques arguments pour la position qui sera adoptée dans cet ouvrage vis-à-vis des actes de langage.

\sloppy

Les exemples \ref{x:AL1}  semblent indiquer que les forces illocutoires sont conventionnellement attachées à la structure grammaticale, ce que l'on appelle des \kwo{types de phrases syntaxiques}\is{type de phrases} (déclaratives, interrogatives, impératives, exclamatives...).  Mais cette corrélation n'est pas suffisante, car les inventaires d'actes illocutoires proposent souvent plus de variété et de granularité que ceux des catégories syntaxiques de phrases : les assertions, avertissements, menaces, promesses, félicitations, remerciements, nominations, etc. peuvent être réalisés au moyen de phrases déclaratives.
De plus, les forces illocutoires peuvent parfois être aussi marquées par des éléments qui ne relèvent pas directement de la structure syntaxique. Par exemple, \sicut{évidemment}, \sicut{hélas}, \sicut{après tout} ne sont compatibles qu'avec des assertions et surtout pas avec des questions ou des requêtes ; \sicut{s'il te plaît} signale forcément une requête, etc. 
Ajoutons qu'il est parfois possible de réaliser ce que nous appellerons des \kwo{actes de langage complexes}\indexs{acte de langage!\elid\ complexe} qui sont la conjonction de plusieurs actes illocutoires dans une même phrase (une manière pour le locuteur de faire d'une pierre deux coups) :

\fussy

\ex.
\a.  Est-ce que Jean, qui a arrêté l'école en 4ème, veut vraiment s'inscrire en licence ? \label{x:ALCa}
\b. Austin était anglais, n'est-ce pas ? \label{x:ALCb}

\ref{x:ALCa} réalise une question, mais également une assertion via la proposition relative. Et \ref{x:ALCb} est le croisement d'une assertion et d'une demande de confirmation (l'ensemble n'étant pas illocutoirement équivalent à la question \sicut{Est-ce qu'Austin était anglais ?}).

Enfin la situation se complique dramatiquement par l'existence des \kwo{actes de langage indirects}\indexs{acte de langage!\elid\ indirect} dont la valeur illocutoire réelle est différente de celle exprimée littéralement :



\ex. \label{x:ALI}
\a. Peux-tu me passer le sel ?  \label{x:ALIa}
\b. Tu te tais maintenant ! \label{x:ALIb}
\b. Tu ne tueras point. \label{x:ALIc}
\b. J'aimerais savoir à quelle heure ferme le guichet. \label{x:ALId}
\b. Franchement, qui regarde encore la télé de nos jours ? \label{x:ALIe}


L'exemple fameux \ref{x:ALIa} est, en surface, une question mais le locuteur n'attend pas qu'on lui réponde oui ou non, il attend qu'on lui donne le sel ; l'énoncé est compris comme un acte illocutoire de requête.  Les apparentes assertions \ref{x:ALIb} et \ref{x:ALIc} sont comprises comme des ordres, et celle de \ref{x:ALId} comme une question.  Et \ref{x:ALIe} est une question rhétorique, qui n'attend pas de réponse et fonctionne comme une assertion. Ces phénomènes de réinterprétation pragmatique d'un sens littéral font évidemment penser aux implicatures conversationnelles et il est assez clair que la théorie de Grice rejoint celle des actes de langage au moins sur ce point. 
Même si les implicatures n'expliquent peut-être pas entièrement le fonctionnement des actes de langage indirects et la compréhension des actes illocutoires en général, les remarques qui concluaient la section précédente peuvent être reprises ici.

Pour conclure, je souhaiterais apporter une petite clarification attenante à cette question de l'appariement entre le sens %(et la forme) 
de surface et l'acte illocutoire réalisé.  
%Laissons de côté provisoirement les actes de langage indirects.
On trouve, assez fréquemment et depuis assez longtemps, l'idée que les énoncés de \ref{xilloc} réalisent des actes de langage différents parce qu'ils ont des forces illocutoires différentes --~ce qui est juste~-- tout en ayant le même contenu propositionnel --~ce qui est très probablement faux.


\ex. \label{xilloc}
\a. Félix range sa chambre. \label{xilloca}
\b. Est-ce que Félix  range sa chambre ? \label{xillocb}
\c. Félix, range ta chambre ! \label{xillocc}


Les avancées en sémantique formelle, depuis plusieurs décennies, permettent de soutenir que les structures logico-sémantiques précises de ces trois phrases ne sont pas les mêmes.  
Leur sens  (que nous l'appelions de surface, littéral, conventionnel ou locutoire) comporte un composant, souvent appelé le \kwo{mode} de la phrase\footnote{À ne pas confondre avec ce que l'on appelle plus traditionnellement les modes de conjugaison\is{mode!\elid\ de conjugaison} comme l'indicatif, le subjonctif, le conditionnel...}\is{mode!\elid\ de phrase}, qui est le pendant sémantique des types syntaxiques (déclaratives, interrogatives, impératives...).
Le mode fait partie intégrante du contenu propositionnel, qui est ainsi un objet sémantique suffisamment spécifiée pour décrire ce qu'est le sens d'une phrase.  
La force illocutoire s'obtient ensuite par un raisonnement pragmatique qui exploite ce contenu (entre autres choses).  Par exemple, c'est notamment parce que nous savons que la \emph{phrase} \ref{x:ALIa} a un contenu de question qui interroge sur la capacité de l'allocutaire à passer le sel que nous comprenons l'\emph{acte} \ref{x:ALIa} comme une requête  qui revient à \sicut{passe-moi le sel}.  
Avec les exemples \ref{xilloc}, c'est encore plus simple : ainsi  parce que la phrase \ref{xillocc} a un contenu propositionnel de requête que nous comprenons que l'acte illocutoire \ref{xillocc} est une requête ou un ordre.  
La force illocutoire est donc  quelque chose qui \emph{s'ajoute} à un sens de phrase déjà disponible. Et comme avec les implicatures, il s'avère normal de commencer par expliciter le contenu propositionnel (assimilable au sens conventionnel des phrases).  C'est ce à quoi nous nous attacherons dans cet ouvrage.  Comme le montre la théorie des actes de langage, ce n'est pas suffisant pour rendre entièrement compte du sens effectif d'un énoncé, mais c'est nécessaire.




\indexs{acte de langage|)}





\section{Conclusion}
%===================

Dans ce chapitre, nous avons fait apparaître diverses manifestations objectives du sens des phrases et des énoncés en mettant en \oe uvre quelques méthodes d'observation et d'investigation qui constitueront nos premiers outils pour l'étude sémantique que nous allons entreprendre dans la suite de cet ouvrage. 
Cela nous aura aussi permis  d'esquisser les premières grandes lignes d'un programme scientifique de la théorie sémantique, notamment en dessinant une délimitation, arbitraire mais utile, entre ce qui relève de son domaine et ce qu'il est plus raisonnable de traiter, en aval, dans le cadre de la pragmatique.  
Pour résumer, les phénomènes présentés en \S\ref{s:RelInfer} appartiendront au cahier des charges de la sémantique, et ceux de \S\ref{s:pragmat} seront, en sachant bien les identifier, mis de côté, en réserve pour la pragmatique. 
La théorie de sémantique formelle qui sera présentée dans les chapitres suivants sera donc consacrée au sens des phrases (en privilégiant presque exclusivement les phrases déclaratives).

%\fixme{***}







\nocite{Simons:12x,MoeReb:94,Beav:96,BeavGeurts:12x,Horn:04,DavisGillon:04}

%\defbibnote{debibchI}{\small 
\subsection*{Repères bibliographiques}
Il n'est pas inutile, à ce stade, de compléter les présentations de ce chapitre par des lectures qui approfondissent les notions de pragmatique qui ne seront plus tellement discutées dans les chapitres qui suivent.  En français,  de bonnes introductions étendues à ces notions sont fournies, en français, par \citet{MoeReb:94} et, en anglais, par \citet{Levinson:83}. Sur les présuppositions, on pourra consulter \citet{Ducrot:72,Ducrot:84}, \citet[chap.~4]{Levinson:83} et, pour un panorama plus actualisé,  \citet{Beav:96} et \citet{BeavGeurts:12x} ; sur les implicatures, outre \citet{Grc:75}, on pourra se reporter à \citet[chap.~3]{Levinson:83}, \citet{Horn:04}, \citet[intro \& chap.~1]{Spector:06} et \citet{Simons:12x}. Ce ne sont que quelques indications bibliographiques parmi d'autres, la littérature sur ces sujets étant assez abondante. 
%}

%\newpage
%\printbibliography[segment=\therefsegment,heading=subbibliography,prenote=debibchI]

\end{refsegment}
