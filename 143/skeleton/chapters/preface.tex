% -*- coding: utf-8 -*-
\addchap{Avant-propos}
\begin{refsection}

%content goes here

\sloppy
Il existe de nombreux manuels de sémantique formelle ; citons notamment  \citet{DWP:81,ChierchiaMcCG:90,Gamut:1,Gamut:2,Cann:94,HeimKratzer:97,dSwrt:98nls,Corb:13,Jacobson:14,Winter:16}.
Ces excellents ouvrages sont en anglais\footnote{\citet{Corb:13} est une exception notable et très appréciable. Cependant sa couverture n'en fait pas une introduction complète à la {Grammaire de Montague} (par exemple l'intensionnalité et le \lcalcul\ n'y sont pas abordés). Le présent manuel se propose d'être, espérons-le, un complément naturel à celui de mon estimé confrère.  Mentionnons aussi \citet{Galm:91} qui, à l'inverse, offre un commentaire détaillé et raisonné de \citet{PTQ}, mais sans vraiment adopter la démarche générale d'un manuel de sémantique formelle.}.
Il nous a semblé, à plusieurs de mes collègues et à moi-même, qu'il était regrettable que les étudiants francophones ne disposent pas, dans leur langue, d'un ouvrage de référence d'une portée comparable aux références susmentionnées.
J'ai donc entrepris, il y a quelques années, de mettre par écrit les cours que je donnais à l'Université Paris~8 et à l'École Normale Supérieure, ne serait-ce que  pour offrir aux étudiants un support utile, et parfois secourable,  pour leurs révisions, leurs besoins d'éclaircissement et leurs velléités d'approfondissement.  
Les cours d'introduction à la sémantique formelle apparaissent parfois, aux néophytes,  comme âpres et rédhibitoires, de par l'exigence scientifique de la discipline et la couverture réduite des faits linguistiques traités ; j'ai pu constater qu'en général, la matière ne récompense  les étudiants persévérants  qu'à l'issue d'un long apprentissage, lorsque le formalisme  enseigné acquiert suffisamment de pouvoir expressif pour aborder un grand éventail et une réelle variété de phénomènes sémantiques.
Pour ces raisons, j'ai trouvé opportun d'inclure dans une présentation \emph{unifiée} un certain nombre d'avancées importantes qui ont enrichi le domaine depuis plus d'un demi-siècle (temporalité, événements, modalités, pluralités, degrés, etc.) en prêtant une attention particulière à la cohérence globale du formalisme exposé
et notamment en veillant, autant que possible, à bien identifier les conséquences formelles que chaque amendement peut engendrer dans le système. 
Ainsi, avec le temps, le projet a pris de l'ampleur,  
et il m'est apparu qu'il suscitait
également  l'intérêt de chercheurs (jeunes ou confirmés) désireux de s'initier, se mettre à jour ou se perfectionner dans la discipline.
Il devenait alors plus raisonnable de diviser le manuel en deux volumes ; le présent volume expose les notions de base de la théorie et l'essentiel de l'outillage formel qui la met en \oe uvre ; le second présentera plusieurs applications et développements qui étendent la portée du formalisme   (son sommaire est indiqué \alien{infra}).


\fussy



Le principal objectif du manuel est de permettre aux lecteurs de se familiariser progressivement avec la définition, le fonctionnement et la manipulation d'un \emph{système sémantique formel}. Il s'agit plus précisément d'un système dans lequel «s'incarne» le paradigme théorique couramment désigné par l'appellation de \emph{Grammaire de Montague}\is{grammaire!\elid\ de Montague} \citep{Montague:EFL,Montague:UG,PTQ}%
\footnote{À cet égard, il convient de préciser que le manuel ne fournit pas un exposé exhaustif du fragment défini dans \citet{PTQ} (on trouvera des présentations dédiées et détaillées dans \citealt{DWP:81}, \citealt{Gamut:2} et \citealt{Galm:91}), mais il rend justice à la plupart de ses principaux ingrédients.} tel qu'il a pu se développer au cours des dernières décennies.   
Un autre enjeu important est de proposer, au fil des pages, une introduction à des éléments de \emph{méthodologie} scientifique afin de donner un aperçu des pratiques d'analyse propres à la sémantique formelle. 
En revanche, cet ouvrage n'a pas nécessairement pour vocation de défendre des analyses sémantiques particulières\footnote{On trouvera de nombreux \alien{handbooks} de qualité qui remplissent précieusement cette fonction.}, même s'il en présente et explique plusieurs. 
Elles ne sont données qu'à titre d'illustration et pour motiver les divers développements qui viennent régulièrement amender le formalisme.
Enfin, comme cela apparaîtra assez clairement dans les pages qui suivent, j'ai pris le parti de privilégier la dimension pédagogique du texte, en m'attachant à anticiper les questionnements et difficultés que j'ai pu voir se poser aux étudiants durant mon expérience d'enseignement, et en ayant à c\oe ur d'offrir une présentation stimulante et détendue de cette discipline encore riche de perspectives. 


\medskip

\noindent
Sommaire du volume 2 :

\begin{enumerate}[start=7]
\item Temporalité et événements\\
{\small Le temps retrouvé -- Vers une analyse des temps verbaux -- Évènements --  Temporalité, événements et compositionnalité.}
\item Pronoms et contextes\\
{\small Traitements formels des pronoms -- Mondes, indices et {\LOz} -- Indexicaux et logique des démonstratifs.}
\item Modalités\\
{\small Retour aux modalités -- Conditionnelles -- Attitudes propositionnelles.}
\item Pluriels et pluralités\\
{\small Des pluralités dans le modèle -- Du pluriel dans {\LO} -- Pluralités sans pluriels.}
\item Adjectifs et degrés\\
{\small Sens et dénotation des adjectif -- Comparaison et gradabilité %-- Échelles 
-- Les autres degrés.}
\item Conclusion et perspectives
\end{enumerate}
%% Ref to volume 2 chapters
\manuallabel{Ch:temps2}{7}
\manuallabel{Ch:contexte}{8}
\manuallabel{Ch:modalites}{9}
\manuallabel{GN++}{10}
\manuallabel{Ch:adj}{11}
\manuallabel{ch:conclu}{12}
\manuallabel{s:conditionnelles}{9.2}


\printbibliography[heading=subbibliography]
\end{refsection}

