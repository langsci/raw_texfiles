% -*- coding: utf-8 -*-
%%% Déclarations theorem %%%%%%%%%%%%%%%%%%%%%%%%%%%%
\newcommand{\fonttitre}{}

\definecolor{deficolor}{rgb}{0.88,0.88,0.88}
\definecolor{theocolor}{rgb}{0.95,0.95,0.95}
%\colorlet{colorexo}{SteelBlue4!60!black}
\colorlet{colorexo}{black}
\newcommand{\colexo}[1]{\textcolor{colorexo}{#1}}


\theorembodyfont{\rmfamily}
\addtolength{\theorempostskipamount}{-4pt}
\theoremstyle{break}\theoremheaderfont{\color{black}\fonttitre\bfseries}
\shadecolor{deficolor}\newshadedtheorem{defi}{Définition}[chapter]
\shadecolor{theocolor}\newshadedtheorem{princ}{Principe}[chapter]
\shadecolor{theocolor}\newshadedtheorem{point}{Point}[chapter]
\shadecolor{theocolor}\newshadedtheorem{postu}{Postulat}[chapter]
\shadecolor{theocolor}\newshadedtheorem{prop}{Proposition}[chapter]
\shadecolor{theocolor}\newshadedtheorem{theo}{Théorème}[chapter]
\shadecolor{theocolor}\newshadedtheorem{coro}{Corollaire}[chapter]
\shadecolor{theocolor}\newshadedtheorem{nota}{Notation}[chapter]
\shadecolor{theocolor}\newshadedtheorem{conv}[nota]{Convention de notation}
%% Pour les exercices :
%\setlength{\theorempreskipamount}{0pt}\setlength{\theorempostskipamount}{0pt}
\theoremheaderfont{\fonttitre\bfseries\color{colorexo}}
\theoremindent1.3cm\theorembodyfont{\small}\newtheorem{exo}{Exercice}[chapter]
