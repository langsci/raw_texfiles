\protect \section {Chapitre \protect \ref {Ch:1}}
\begin{Solution}{1.{1}}
(p.~\pageref{exo:1CL})\label{crg:1CL}

\sloppy

Nous allons démontrer les réponses en appliquant la méthode des contre-exemples (\S\ref{s:conseql}, p~\pageref{p.contrex}):  pour chaque paire de phrases, nous essayons d'imaginer une situation par rapport à laquelle (a) est vraie \emph{et} (b) est fausse.

\fussy
\begin{enumerate}
\item
La situation serait telle que la soupe est chaude \emph{et} la soupe est froide (puisqu'il est faux qu'elle n'est pas froide). C'est contradictoire, et nous en concluons que (a) $\satisf$ (b).
\item
Ici la situation serait telle que la soupe est chaude et brûlante. C'est tout à fait possible (si la soupe est précisément brûlante) puisque quelque chose de brûlant est forcément chaud. C'est un contre-exemple, et donc (a) $\not\satisf$ (b).
\item
Si nous sommes dans une situation où le colonel Moutarde \emph{n'est pas} coupable, alors on ne peut pas \emph{prouver} qu'il est coupable. À la rigueur, Joseph pourrait argumenter ou soutenir (par erreur ou malhonnêteté) que le colonel est coupable, mais objectivement nous n'aurons pas le droit de dire qu'il a \emph{prouvé} la culpabilité.  Par conséquent (a) $\satisf$ (b).
\item
Ici il faut être vigilant : une situation où (b) est fausse  (il est faux que des étudiants n'ont pas eu la moyenne au partiel) est telle que \emph{tous} les étudiants ont eu la moyenne.  Dans une telle situation, (a) peut-elle être vraie ? Oui, nécessairement, car si (a) était fausse cela voudrait dire qu'aucun étudiant n'a eu la moyenne (ce qui serait contradictoire avec la première hypothèse).  Nous pouvons avoir une situation où (a) est vraie et (b) fausse, et donc (a) $\not\satisf$ (b).
\\
NB : le fait que nous comprenons souvent (a) comme s'accompagnant aussi de la vérité de (b) ne relève pas de la conséquence logique mais d'une implicature conversationnelle (cf. \S\ref{ss:implicatures}).
\end{enumerate}
\end{Solution}
\begin{Solution}{1.{2}}
 (p. \pageref{exo:1EqLog})

La méthode consiste à appliquer la définition \ref{d:EquLog} et son corollaire donnés à la p.~\pageref{d:EquLog} en cherchant un contre-exemple à l'équivalence, c'est-à-dire un scénario par rapport auquel on pourra juger que l'une des deux phrases est vraie et l'autre fausse.  Si nous y arrivons, alors c'est que les deux phrases ne sont pas équivalentes; sinon nous concluons qu'elles le sont probablement.  Il faut noter qu'en toute rigueur,  cette conclusion  ne peut être que provisoire\footnote{Comme toute conclusion scientifique sérieuse.}, car si nous ne trouvons pas de contre-exemple, cela ne veut pas nécessairement dire qu'il n'en existe pas, mais simplement que nous n'avons peut-être pas assez cherché ou pas encore trouvé.

\ex.[\ref{xparaph1}]
\a.  Anne a acheté une lampe au brocanteur.
\b. Le brocanteur a vendu une lampe à Anne.

Peut-on imaginer un scénario où la phrase (a) est vraie et (b) fausse?
Une piste serait d'envisager qu'Anne a effectué l'achat mais que le brocanteur, lui, n'a rien fait dans l'histoire.  Par exemple, si Anne a acheté la lampe sur internet via un site de vente par correspondance... Mais pour autant, même dans ce cas, peut-on vraiment dire que (b) est fausse?  Si, un peu plus tard, le brocanteur consulte son compte sur le site, constatant la transaction, il pourrait dire : «Ah super, aujourd'hui j'ai vendu une lampe».  Si l'on accepte cette possibilité, alors on n'aura pas prouvé qu'il s'agit là d'un contre-exemple. Inversement, pour avoir (b) vraie et (a) fausse, il faudrait s'orienter vers un scénario où Anne n'a rien fait ou éventuellement qu'elle ne s'est pas rendu compte que l'achat a eu lieu. Mais là aussi, on imagine mal comment le brocanteur aura réussi à vendre la lampe ou comment l'opération ne pourra pas être qualifiée d'achat en ce qui concerne Anne. À ce stade, nous conclurons (au moins provisoirement) que ces phrases semblent bien équivalentes, tout en retenant que les suggestions évoquées ci-dessus indiquent des directions vers lesquelles poursuivre et approfondir la question.

\ex.[\ref{xparaph2}]
\a.  La glace  fond à 0\,\degres C.
%\b. L'eau gèle à 0\degres C.
\c. La température de fusion de l'eau est de 0\,\degres C.

Notons d'abord que des physiciens relèveront sans doute que ces deux phrases peuvent être fausses en même temps car elles ne disent rien des conditions de pression ambiante.  Mais d'un point de vue sémantique, cela ne nous concerne guère : nous avons seulement à \emph{supposer} la vérité et la fausseté de l'une et l'autre phrase.  En procédant de la sorte, une situation où (a) est jugée vraie rend également (b) vraie.  De même, inversement, en partant d'une situation où (b) est vraie, car une telle situation n'est pas seulement un cas où la science établit une certaine propriété de l'eau, c'est aussi un cas où cette propriété est définitivement vérifiée dans la réalité.  Ce qui perturbe l'exercice, en revanche, c'est que les phrases sont polysémiques (cf. p.~\pageref{p.polysem}) : \sicut{glace} peut désigner diverses choses (de l'eau ou un autre liquide à l'état solide, une crème glacée, un miroir...), les phrases peuvent chacune référer à des matériaux en général ou à des entités particulières observées dans le contexte (un certain bloc de glace, une certaine quantité d'eau), etc. Tant que ces acceptions ne sont pas fixées, il faudra conclure leur variabilité de sens empêche les phrases d'être équivalentes.  Elles ne pourront l'être que sous certaines hypothèses qui les désambiguïsent.


\ex.[\ref{xparaph2.5}]
\a. Tous les étudiants ont eu la moyenne à l'examen.
\b. Aucun étudiant n'a été recalé à l'examen.

Supposons une situation où les règles d'évaluation de l'examen en question stipulent qu'il faut une note supérieure ou égale à 12/20 pour être admis. Et supposons de plus que tous les étudiants ont eu une note supérieure à 10/20 mais que certains ont eu 11/20.  Dans ce cas (a) est vraie, mais (b) est fausse. Les deux phrases ne sont donc pas logiquement équivalentes.


\ex.[\ref{xparaph3}]
\a. Vous savez bien que l'existence précède l'essence.
\b. Vous n'ignorez pas que l'existence précède l'essence.

Plaçons nous dans une situation où il est vrai que «l'existence précède l'essence» (quoi que cela veuille dire) et où (a) est fausse. Dans cette situation, l'allocutaire ne sait donc pas que l'existence précède l'essence, c'est-à-dire qu'il ignore que l'existence précède l'essence.  Mais alors (b) est fausse aussi. Un raisonnement similaire peut être mené en partant d'une situation où (b) est fausse (s'il est faux que l'allocutaire n'ignore pas, c'est qu'il sait).  Les deux phrases sont donc équivalentes.

%Équivalences logiques.
\end{Solution}
\begin{Solution}{1.{3}}
(p.~\pageref{exo:1Ambig})\label{crg:1Ambig}

En accord avec la définition \ref{d:ambig} p.~\pageref{d:ambig}, pour chaque phrase, nous construisons un scénario par rapport auquel nous pouvons juger que la phrase est à la fois vraie et fausse (selon le sens retenu).
\begin{enumerate}
\item %J'ai rempli ma bouteille d'eau.\\
Scénario : \emph{j'ai une bouteille d'eau minérale en plastique, initialement vide ; je l'ai remplie de limonade.} %\\
La phrase est vraie, car c'est bien ma bouteille d'eau que j'ai remplie.  Mais elle est aussi fausse, car je ne l'ai pas remplie d'eau.
\item %Pierre s'est fait arrêter par un policier en pyjama. \\
Scénario : \emph{Pierre, très distrait, se promène dans la rue en pyjama ; un agent de police, en uniforme, l'arrête pour lui demander la raison de cette tenue étrange.} %\\
La phrase est vraie car Pierre, alors qu'il était en pyjama, s'est fait arrêter par un policier. Elle est aussi fausse car ce n'est pas un policier en pyjama qui l'a arrêté.
\item %Daniel n'est pas venu à la fête parce que le président était là.
Scénario : \emph{Daniel est venu à la fête, le président aussi ; Daniel ne savait même pas que le président serait présent et il est venu parce qu'il aime les fêtes et n'en manque aucune.}
La phrase est vraie car ce n'est pas parce que le président était là que Daniel est venu.  Elle est fausse parce que Daniel est venu à la fête.  Les deux sens de la phrase peuvent se paraphraser en : i) la raison pour laquelle Daniel est venu n'est pas que le président était là ; ii) la raison pour laquelle Daniel n'est pas venu est que le président était là.
\item %Alice ne mange que des yaourts au chocolat.
Scénario : \emph{Alice est une omnivore accomplie et a un régime alimentaire varié et équilibré, mais pour ce qui est des yaourts, elle n'en mange qu'au chocolat.}
La phrase est vraie car les seuls yaourts qu'Alice mange sont au chocolat, et elle est fausse car Alice ne se nourrit pas exclusivement de yaourts au chocolat.
\item %Kevin dessine tous les gens moches.
Scénario : \emph{Kevin aime bien dessiner, et en particulier il aime faire le portrait des gens qui sont plutôt beaux (ils ne dessine jamais les gens qu'il trouve moches) ; mais il a un mauvais coup de crayon et ses dessins enlaidissent toujours les sujets.}
La phrase est vraie car, les gens qu'il dessine, il les dessine moches. Et elle est fausse car il ne dessine pas les gens qui sont moches.  Les deux sens se paraphrasent ainsi : i) si Kevin dessine quelqu'un, il le dessine moche ; ii) si quelqu'un est moche, Kevin le dessine.
\end{enumerate}
\end{Solution}
\begin{Solution}{1.{4}}
(p.~\pageref{exo:1Ambig2})\label{crg:1Ambig2}

La phrase \sicut{ce bijou n'est pas une bague en or} n'est pas ambiguë, elle a juste une signification assez large pour couvrir les cas où le bijou n'est pas une bague et ceux où il n'est pas en or.  Pour nous en assurer, nous pouvons ici appliquer le test des ellipses vu dans le chapitre p.~\pageref{test:ellipse}.  Supposons qu'un premier bijou $B_1$ est une bague en argent et qu'un second bijou $B_2$ est un bracelet en or ; nous pouvons alors tout à fait énoncer \sicut{ce bijou $B_1$ n'est pas une bague en or, et ce bijou $B_2$ non plus}.
\end{Solution}
\begin{Solution}{1.{5}}
 (p.~\pageref{exo:1psp1})

La méthode la plus simple pour trouver les projections d'une
phrase $P$ est de la comparer avec sa forme négative en \sicut{il est
  faux que $P$}, comme ce que nous avons vu avec les exemples \ref{xmoutarde}--\ref{xgonc} en \S\ref{ss:projections}, p.~\pageref{p.resNeg}.

\begin{enumerate}
\item %\ex. \label{x:expsp1}
Projection : Pierre n'est pas venu.

\item %\ex.  \label{x:expsp2}
Projection : Quelqu'un d'autre qu'Hélène fait de la linguistique.

\item %\ex.  \label{x:expsp3}
Projection : Hélène fait autre chose que de la linguistique.

\item %\ex.  \label{x:expsp4}
Projection : Marianne fumait avant.

\item %\ex. \label{x:expsp5}
Projection : Le téléphone a sonné.

\item %\ex.  \label{x:expsp6}
Projection : Laurence a pris une salade (c'est-à-dire la phrase sans \sicut{seulement} ;  remarquons que dans cette
phrase, le  contenu proféré est : Laurence n'a rien pris d'autre qu'une
salade).

\item %\ex.  \label{x:expsp7}
Projection : Antoine a été barbu.

\item %\ex.  \label{x:expsp8}
Projections : Jean a essayé d'intégrer l'ENA (c'est-à-dire il s'est présenté au
concours d'entrée).  Mais il y a une seconde présupposition déclenchée par \sicut{essayer} qui est : il n'est pas facile d'intégrer l'ENA.

\item %\ex.  \label{x:expsp9}
Projection : Quelqu'un a apporté des fleurs.

\item %\ex.  \label{x:expsp10}
Projection : Robert faisait partie de ceux qui avaient le moins de chances d'avoir la moyenne au partiel.  Notons ici que le test de la négation s'applique un peu
difficilement. Il faut bien prendre soin d'utiliser la formulation en
\sicut{il est faux que}.

\item %\ex.[\ref{xmoutarde}] \a.
Projections : 1) Quelqu'un est le coupable (c'est-à-dire  il y a un coupable) et 2)
le coupable est le colonel Moutarde.
\end{enumerate}

Notons que la plupart des projections de cet exercice se trouvent être des présuppositions. Pour certaines, on peut s'en assurer assez facilement en appliquant le test de redondance (\S\ref{sss:ptépsp}, p.~\pageref{p.testAB}) ; pour d'autres, comme \ref{x:expsp9} et \ref{x:expsp10}, le test est moins concluant, ce qui peut soulever la question de leur statut véritablement présuppositionnel. Quant à \ref{x:expsp2} et \ref{x:expsp3}, le test fonctionne bien à condition d'utiliser une formulation plus précise du contenu projectif, par exemple \sicut{Lucie fait de la linguistique et Hélène aussi fait de la linguistique} {\vs} {\zarb}\sicut{Hélène aussi fait de la linguistique et Lucie fait de la linguistique}.

\end{Solution}
\begin{Solution}{1.{6}}
(p.~\pageref{exo:1psp2})

Les résultats que nous pouvons tirer de cet exercice sont à prendre avec précaution car ils attendent des jugements sémantiques particulièrement fins, qui peuvent varier d'un locuteur à l'autre. Pour obtenir des conclusions plus assurées, il serait utile, par exemple, de mettre sur pied des expérimentations à grande échelle.  Nous allons donc ici seulement nous concentrer sur la méthode sous-jacente.  Celle-ci consiste, en premier lieu, à comparer les inférences que nous tirons de chaque phrase avec celles que nous tirons de leur négation.  Pour ces trois phrases, les présuppositions potentielles portent sur l'existence des verchons.
\begin{enumerate}
\item \emph{Il y a des verchons qui ont bourniflé.}
Nous en inférons évidemment que les verchons existent.\\
Négation : \emph{Il est faux qu'il y a des verchons qui ont bourniflé.}
Certes, les contextes les plus naturels dans lesquels cette phrase peut être énoncée, sont ceux où l'on sait que les verchons existent ; cependant (et c'est ce qui importe ici) elle peut également l'être dans des cas de figure où les verchons n'existent pas.  En d'autres termes, si les verchons n'existent pas, il semblera assez légitime de dire que cette phrase est vraie.

Il semble donc que l'expression \sicut{il y a des verchons} ne présuppose pas \emph{sémantiquement} qu'il existe des verchons (mais ça l'affirme).

\item \emph{Tous les verchons ont bourniflé.}  Si cette phrase est vraie, alors nous devons en inférer que les verchons existent (notamment du fait de l'emploi du passé composé).\\
Négation : \emph{Il est faux que tous les verchons ont bourniflé.} Cela signifie qu'il y a au moins un verchon qui n'a pas bourniflé, et donc que les verchons existent.

Nous pouvons également nous aider du test de la redondance (\S\ref{sss:ptépsp}, p.~\pageref{p.testAB}).  Si nous comparons \sicut{les verchons existent et tous les verchons ont bourniflé} et \sicut{tous les verchons ont bourniflé et les verchons existent}, nous constatons que cette seconde phrase est particulièrement redondante.

Il est donc raisonnable de conclure que la phrase présuppose (sémantiquement) que les verchons existent.

\item \emph{Aucun verchon n'a bourniflé.}  Pouvons-nous inférer de (la vérité de) cette phrase que forcément les verchons existent ? Les jugements peuvent être fluctuants, mais nous pouvons estimer que si les verchons n'existent pas, alors la phrase sera vraie (comme pour la négation de la phrase 1 ci-dessus). Si tel est bien le cas, alors il n'est pas utile d'examiner la négation de la phrase\footnote{Celle-ci signifie qu'il y a des verchons qui ont bourniflé, ce qui entraîne directement l'existence des verchons.}, nous pourrons tout de suite conclure que la phrase ne présuppose pas que les verchons existent. Le test de la redondance va-t-il dans ce sens ? \sicut{Les verchons existent mais aucun verchon n'a bourniflé} est acceptable, comme prévu ; quant à \sicut{aucun verchon n'a bourniflé, mais les verchons existent} nous pouvons y voir une redondance, mais elle est probablement moins saillante que dans la version avec \sicut{tous les} en 2.
\end{enumerate}
\end{Solution}
\begin{Solution}{1.{7}}
(p.~\pageref{exo:1RelSem})

\begin{enumerate}
\item Présupposition.
\item Conséquence logique.
\item Implicature conversationnelle (scalaire).
\item Contradiction.
\item Équivalence logique.
\item Implicature conversationnelle (particularisée).
\end{enumerate}
\end{Solution}
\protect \section {Chapitre \protect \ref {LCP}}
\begin{Solution}{2.{1}}
 (p.~\pageref{e:LCPebf}) \label{crg:LCPebf}
%Formules bien formées de {\LO}.

L'exercice consiste à produire l'arbre de construction (cf. p.~\pageref{f:Axfbf}) de chaque séquence en appliquant les règles syntaxiques de la définition \ref{SynP}, p.~\pageref{SynP}.

\begin{exolist}
\item \(\Xlo\exists z\, [[\prd{connaître}(\cnsi{r}2,z) \wedge
\prd{gentil}(z)] \wedge \neg\prd{dormir}(z)]\)

Cette formule est bien formée, on le montre à l'aide de son arbre de
construction:

\begin{center}
{\small
\leaf{\(\prd{connaître}\)}
\leaf{\(\cnsi{r}2\)}
\leaf{\(\vrb z\)}
\branch{3}{\(\Xlo\prd{connaître}(\cnsi{r}2,z)\)}
\leaf{\(\prd{gentil}\)}
\leaf{\(\vrb z\)}
\branch{2}{\(\Xlo\prd{gentil}(z)\)}
\branch{2}{\(\Xlo[\prd{connaître}(\cnsi{r}2,z) \wedge \prd{gentil}(z)]\)}
\leaf{\(\prd{dormir}\)}
\leaf{\(\Xlo z\)}
\branch{2}{\(\Xlo\prd{dormir}(z)\)}
\branch{1}{\(\Xlo\neg\prd{dormir}(z)\)}
\branch{2}{\(\Xlo[[\prd{connaître}(\cnsi{r}2,z) \wedge \prd{gentil}(z)] \wedge \neg\prd{dormir}(z)]\)}
\branch{1}{\(\Xlo\exists z\, [[\prd{connaître}(\cnsi{r}2,z) \wedge
      \prd{gentil}(z)] \wedge \neg\prd{dormir}(z)]\)}
\qobitree}
\end{center}

\item \(\Xlo\exists x \forall y \exists z\, [\prd{aimer}(y,z) \vee \prd{aimer}(z,x)]\)

Cette formule est bien formée:

\begin{center}
{\small
\leaf{\(\prd{aimer}\)}
\leaf{\(\Xlo y\)}
\leaf{\(\Xlo z\)}
\branch{3}{\(\Xlo\prd{aimer}(y,z)\)}
\leaf{\(\prd{aimer}\)}
\leaf{\(\Xlo z\)}
\leaf{\(\Xlo x\)}
\branch{3}{\(\Xlo\prd{aimer}(z,x)\)}
\branch{2}{\(\Xlo[\prd{aimer}(y,z) \vee \prd{aimer}(z,x)]\)}
\branch{1}{\(\Xlo\exists z\, [\prd{aimer}(y,z) \vee \prd{aimer}(z,x)]\)}
\branch{1}{\(\Xlo\forall y \exists z\, [\prd{aimer}(y,z) \vee \prd{aimer}(z,x)]\)}
\branch{1}{\(\Xlo\exists x \forall y \exists z\, [\prd{aimer}(y,z) \vee \prd{aimer}(z,x)]\)}
\qobitree}
\end{center}

\item \(\Xlo\forall x y\, \prd{aimer}(x,y)\)

Cette séquence  n'est pas une formule bien formée.  En effet $\Xlo\forall x$
peut être introduit par la règle (\RSyn\ref{SynPQ}), mais cette règle
doit opérer sur une \emph{formule}.  Or \(\Xlo\forall x y\,
\prd{aimer}(x,y)\) se décomposerait alors en \(\Xlo\forall x\) et \(\Xlo y\,
\prd{aimer}(x,y)\) et cette seconde expression n'est pas une formule
bien formée; on ne peut pas construire \(\Xlo y\, \prd{aimer}(x,y)\),
aucune règle n'autorise à placer une variable seule devant une
expression.

\item \(\Xlo\neg\neg \prd{aimer}(\cnsi{r}1,\cns{m})\)

Cette formule est bien formée:

\begin{center}
{\small
\leaf{\(\prd{aimer}\)}
\leaf{\(\cnsi{r}1\)}
\leaf{\(\cns{m}\)}
\branch{3}{\(\Xlo\prd{aimer}(\cnsi{r}1,\cns{m})\)}
\branch{1}{\(\Xlo\neg \prd{aimer}(\cnsi{r}1,\cns{m})\)}
\branch{1}{\(\Xlo\neg\neg \prd{aimer}(\cnsi{r}1,\cns{m})\)}
\qobitree}
\end{center}


\item \(\Xlo\exists x \neg\exists z\, \prd{connaître}(x,z)\)

Cette formule est bien formée:

\begin{center}
{\small
\leaf{\(\prd{connaître}\)}
\leaf{\(\Xlo x\)}
\leaf{\(\Xlo z\)}
\branch{3}{\(\Xlo\prd{connaître}(x,z)\)}
\branch{1}{\(\Xlo\exists z\, \prd{connaître}(x,z)\)}
\branch{1}{\(\Xlo\neg\exists z\, \prd{connaître}(x,z)\)}
\branch{1}{\(\Xlo\exists x \neg\exists z\, \prd{connaître}(x,z)\)}
\qobitree}
\end{center}


\item \(\Xlo\exists x [\prd{acteur}(x) \wedge \prd{dormir}(x) \vee
  \prd{avoir-faim}(x)] \)

Cette expression n'est pas une formule bien formée car il manque une
paire de crochets dans \(\Xlo[\prd{acteur}(x) \wedge \prd{dormir}(x) \vee
  \prd{avoir-faim}(x)]\).  En effet les connecteurs comme $\Xlo\wedge$ et
$\Xlo\vee$ sont introduits par les règles (\RSyn\ref{SynPConn}) et ces
règles introduisent en même temps une paire de crochets pour chaque connecteur.

\item \(\Xlo[\prd{aimer}(\cns{a},\cns{b}) \implq \neg\prd{aimer}(\cns{a},\cns{b})]\)

Cette formule est bien formée:

\begin{center}
{\small
\leaf{\(\prd{aimer}\)}
\leaf{\(\cns{a}\)}
\leaf{\(\cns{b}\)}
\branch{3}{\(\Xlo\prd{aimer}(\cns{a},\cns{b})\)}
\leaf{\(\Xlo\prd{aimer}\)}
\leaf{\(\Xlo\cns{a}\)}
\leaf{\(\Xlo\cns{b}\)}
\branch{3}{\(\Xlo\prd{aimer}(\cns{a},\cns{b})\)}
\branch{1}{\(\Xlo\neg\prd{aimer}(\cns{a},\cns{b})\)}
\branch{2}{\(\Xlo[\prd{aimer}(\cns{a},\cns{b}) \implq \neg\prd{aimer}(\cns{a},\cns{b})]\)}
\qobitree}
\end{center}

\item \(\Xlo\exists y [\prd{acteur}(x) \wedge \prd{dormir}(x)]\)

Cette formule est bien formée:

\begin{center}
{\small
\leaf{\(\prd{acteur}\)}
\leaf{\(\Xlo x\)}
\branch{2}{\(\Xlo\prd{acteur}(x)\)}
\leaf{\(\prd{dormir}\)}
\leaf{\(\Xlo x\)}
\branch{2}{\(\Xlo\prd{dormir}(x)\)}
\branch{2}{\(\Xlo[\prd{acteur}(x) \wedge \prd{dormir}(x)]\)}
\branch{1}{\(\Xlo\exists y [\prd{acteur}(x) \wedge \prd{dormir}(x)]\)}
\qobitree}
\end{center}

Remarque: la variable $\vrb y$ <<~utilisée~>> par le quantificateur
$\Xlo\exists$ ne réapparaît pas dans la (sous-)formule qui suit, mais cela
n'empêche pas l'expression d'être une formule bien formée, comme le
permet la règle (\RSyn\ref{SynPQ}).

\end{exolist}
\end{Solution}
\begin{Solution}{2.{2}}
 (p.~\pageref{e:versionLO})\label{crg:versionLO}
%Traduction Fr $\leadsto$ \LO.

\begin{enumerate}
\item Antoine n'est plus barbu.\\ $\leadsto$
\(\Xlo\neg\prd{barbu}(\cns{a})\)
\item Tout est sucré ou salé.\\ $\leadsto$
\(\Xlo\forall x [\prd{sucré}(x) \vee \prd{salé}(x)]\)
\item Soit tout est sucré, soit tout est salé.\\ $\leadsto$
\(\Xlo [\forall x\, \prd{sucré}(x) \vee \forall x\, \prd{salé}(x)]\)
\item Le chien qui aboie ne mord pas. (proverbe) \\ $\leadsto$
\(\Xlo\forall x [[\prd{chien}(x) \wedge \prd{aboyer}(x)] \implq
  \neg\prd{mordre}(x)]\) \\ou
\(\Xlo\forall x [\prd{chien}(x) \implq [\prd{aboyer}(x) \implq
  \neg\prd{mordre}(x)]]\)
\item C'est Marie que Jérôme a embrassée. \\$\leadsto$
\(\Xlo\prd{embrasser}(\cns{j},\cns{m})\)
\item Il y a des hommes et des femmes qui ne sont pas
  unijambistes. \\$\leadsto$
\(\Xlo\exists x \exists y [[\prd{homme}(x) \wedge \prd{femme}(y)] \wedge
  [\neg\prd{unij}(x) \wedge \neg\prd{unij}(y)]]\)
\item Tout le monde aime quelqu'un.

Cette phrase est ambiguë.  Elle peut signifier que pour chaque personne, il y a une personne que la première aime (et donc possiblement autant de personnes aimées que de personnes aimantes), cela correspond à la traduction (a) ci-dessous ; mais elle peut signifier aussi qu'il existe une personne aimée de tout le monde, ce qui correspond à la traduction (b).
  \begin{enumerate}
    \item $\leadsto$
      \(\Xlo\forall x \exists y\, \prd{aimer}(x,y)\)\\ou
      \(\Xlo\forall x [\prd{hum}(x) \implq \exists y [\prd{hum}(y) \wedge
    \prd{aimer}(x,y)]]\)
    \item $\leadsto$
       \(\Xlo\exists y\forall x\, \prd{aimer}(x,y)\)\\ou
      \(\Xlo\exists y [\prd{hum}(y) \wedge\forall x [\prd{hum}(x) \implq
    \prd{aimer}(x,y)]]\)
  \end{enumerate}
\item Si tous les homards sont gauchers alors Alfred aussi est
  gaucher.\\ $\leadsto$
\(\Xlo\forall x [\prd{homard}(x) \implq \prd{gaucher}(x)] \implq \prd{gaucher}(\cns{a})\)
\item Quelqu'un a envoyé une lettre anonyme à Anne. \\$\leadsto$
\(\Xlo\exists x [\prd{hum}(x) \wedge \exists y [[\prd{lettre}(y) \wedge
      \prd{anon}(y)] \wedge \prd{envoyer}(x,y,\cns{a})]]\)
\item Seule Chloé est réveillée.\\$\leadsto$
\(\Xlo\forall x [\prd{réveillé}(x) \ssi x=\cns{c}]\)\\
mais on peut aussi proposer \(\Xlo\forall x [\prd{réveillé}(x) \implq
  x=\cns{c}]\), qui n'a pas le même sens, et qui là exclut le présupposé.
\end{enumerate}
\end{Solution}
\begin{Solution}{2.{3}}
 (p.~\pageref{e:version2LO}) %Traductions.
\label{crg:version2LO}

\begin{enumerate}
\item Il existe des éléphants roses.\\$\leadsto$
\(\Xlo\exists x [\prd{éléphant}(x) \wedge \prd{rose}(x)]\)
\item Quelque chose me gratouille et me chatouille.\\$\leadsto$
\(\Xlo\exists x [\prd{gratouiller}(x,\cns l) \wedge
  \prd{chatouiller}(x,\cns l)]\)  (avec $\cns l$ pour le locuteur)
\item Quelque chose me gratouille et quelque chose me
  chatouille.\\$\leadsto$
\(\Xlo[\exists x\,\prd{gratouiller}(x,\cns l) \wedge \exists x\,
  \prd{chatouiller}(x,\cns l)]\)
\item Nimes est entre Avignon et Montpellier.\\$\leadsto$
\(\Xlo\prd{être-entre}(\cns n,\cns a, \cns m)\)
\item S'il y a des perroquets ventriloques, alors Jacko en est
  un.\\$\leadsto$
\(\Xlo[\exists x [\prd{perroquet}(x) \wedge \prd{ventriloque}(x)] \implq
  [\prd{perroquet}(\cns j) \wedge \prd{ventriloque}(\cns j)]]\)
\item Anne a reçu une lettre de Jean, mais elle n'a rien reçu de
  Pierre. \\$\leadsto$
\(\Xlo\exists x [\prd{lettre}(x) \wedge \prd{recevoir}(\cns{a},x,\cns{j})]
  \wedge \neg\exists y \, \prd{recevoir}(\cns{a},y,\cns{p})\)
\item Tout fermier qui possède un âne est riche.\\$\leadsto$
\(\Xlo\forall x [[\prd{fermier}(x) \wedge \exists y [\prd{âne}(y) \wedge
    \prd{posséder}(x,y)] ] \implq \prd{riche}(x)]\)
\item Il y a quelqu'un qui a acheté une batterie et qui est en train
  d'en jouer.\\$\leadsto$
\(\Xlo\exists x [\prd{humain}(x) \wedge \exists y [\prd{batterie}(y)
  \wedge \prd{acheter}(x,y) \wedge \prd{jouer}(x,y)]]\)
\item Il y a un seul océan.\\$\leadsto$
\(\Xlo\exists x [\prd{océan}(x) \wedge \forall y [\prd{océan}(y) \implq y=x]]\)
\item Personne n'aime personne.
  \begin{enumerate}
  \item $\leadsto$
    \(\Xlo\neg \exists x [\prd{humain}(x) \wedge \exists y [\prd{humain}(y)
    \wedge \prd{aimer}(x,y)]]\)\\
    ou
    \(\Xlo\forall x [\prd{humain}(x) \implq \forall y [\prd{humain}(y) \implq
    \neg\prd{aimer}(x,y)]]\)
    \\ou
    \(\Xlo\forall x  \forall y [[\prd{humain}(x) \wedge \prd{humain}(y)] \implq
    \neg\prd{aimer}(x,y)]\)
  \item $\leadsto$
    \(\Xlo\neg\exists x [\prd{humain}(x) \wedge \forall y [\prd{humain}(y) \implq \neg\prd{aimer}(x,y)]]\)\\
ou
    \(\Xlo\forall x [\prd{humain}(x) \implq \neg\forall y [\prd{humain}(y) \implq \neg\prd{aimer}(x,y)]]\)
  \end{enumerate}
Il est important de noter que cette dernière phrase est ambiguë (d'où les deux séries de traductions).  Dans une première interprétation, elle signifie que pour chaque individu du modèle, celui-ci n'aime personne (autrement dit, il n'y a pas d'amour dans le modèle) ; c'est ce que donnent les traductions (a).  Dans la seconde interprétation, la phrase signifie qu'il est faux qu'il y a des gens qui n'aiment personne (autrement dit, tout le monde aime au moins une personne) ; elle est peut-être un peu moins spontanée, mais elle apparaît naturellement dans le dialogue \sicut{--- Albert n'aime personne. --- Mais non voyons, \textsc{personne} n'aime personne} (facilitée par l'accent intonatif sur le premier \sicut{personne}) ; c'est ce que donnent les traductions (b).
\end{enumerate}
\end{Solution}
\begin{Solution}{2.{4}}
(p.~ \pageref{exo:2denot})\label{crg:2denot}

Les dénotations sont calculées en appliquant les règles d'interprétation de la définition \ref{d:Sem1}, p.~\pageref{d:Sem1}.

\begin{enumerate}
\item \(\Xlo[\prd{père-de}(\cns o,\cns p) \ssi \prd{elfe}(\cns b)]\)

\sloppy
La règle  (\RSem\ref{RIcon}d) nous dit que cette formule est vraie dans $\Modele_1$ ssi \(\Xlo\prd{père-de}(\cns o,\cns p)\) et \(\Xlo\prd{elfe}(\cns b)\) ont la même dénotation. \(\denote{\Xlo\prd{père-de}(\cns o,\cns p)}^{\Modele_1}=0\) car \(\tuple{\Obj{Obéron},\Obj{Puck}} \not\in \FI_1(\prd{père-de})\) ; et
\(\denote{\Xlo\prd{elfe}(\cns b)}^{\Modele_1}=0\) car $\Obj{Bottom}\not\in\FI_1(\prd{elfe})$.  Par conséquent, la formule est vraie dans $\Modele_1$.

\fussy

\item \(\Xlo[\neg\prd{aimer}(\cns d, \cnsi{h}{3}) \implq \prd{triste}(\cnsi{h}{3})]\)

Selon la règle (\RSem\ref{RIcon}c), cette formule est vraie dans $\Modele_1$ ssi
$\Xlo\neg\prd{aimer}(\cns d, \cnsi{h}{3})$ est fausse \emph{ou}
$\Xlo\prd{triste}(\cnsi{h}{3})$ est vraie.
Il se trouve que nous remarquons assez rapidement que $\Xlo\prd{triste}(\cnsi{h}{3})$ est vraie dans $\Modele_1$, car $\Obj{Héléna}\in\FI_1(\prd{triste})$ (sachant que \cnsi h3 dénote \Obj{Héléna}). Cela suffit à montrer que la formule est vraie dans $\Modele_1$.

\sloppy

\item \(\Xlo\neg [\prd{elfe}(\cns p) \wedge \prd{farceur}(\cns p)]\)

Cette formule est d'abord une négation, donc nous calculons sa dénotation en consultant d'abord la règle (\RSem\ref{RIneg}) qui dit que la formule est vraie dans $\Modele_1$ ssi \(\Xlo [\prd{elfe}(\cns p) \wedge \prd{farceur}(\cns p)]\) est fausse dans $\Modele_1$. Or \(\denote{\Xlo\prd{elfe}(\cns p)}^{\Modele_1} = 1\) car $\Obj{Puck}\in\FI_1(\prd{efle})$, et
\(\denote{\Xlo\prd{farceur}(\cns p)}^{\Modele_1} = 1\) car $\Obj{Puck}\in\FI_1(\prd{farceur})$.  Donc, en vertu de la règle (\RSem\ref{RIcon}a),
\(\denote{\Xlo\prd{farceur}(\cns p) \wedge \prd{farceur}(\cns p)}^{\Modele_1} = 1\), et nous en concluons que
\(\denote{\Xlo\neg[\prd{farceur}(\cns p) \wedge \prd{farceur}(\cns p)]}^{\Modele_1} = 0\).

\fussy

\item \(\Xlo[\prd{farceur}(\cnsi t1) \implq [\prd{elfe}(\cnsi t1) \implq
    \prd{âne}(\cnsi t1)]]\)

D'après (\RSem\ref{RIcon}c), la formule est vraie dans $\Modele_1$ ssi
\(\Xlo\prd{farceur}(\cnsi t1)\) est fausse ou
\(\Xlo[\prd{elfe}(\cnsi t1) \implq \prd{âne}(\cnsi t1)]\) est vraie.
Or \(\Xlo\prd{farceur}(\cnsi t1)\) est vraie dans $\Modele_1$ car
$\Obj{Thésée}\in\FI_1(\prd{farceur})$.
Nous devons calculer la dénotation de \(\Xlo[\prd{elfe}(\cnsi t1) \implq \prd{âne}(\cnsi t1)]\), qui est vraie ssi
\(\Xlo\prd{elfe}(\cnsi t1)\) est fausse ou
\(\Xlo\prd{âne}(\cnsi t1)\) est vraie.
Et \(\Xlo\prd{elfe}(\cnsi t1)\) est effectivement fausse dans $\Modele_1$ ($\Obj{Thésée}\not\in\FI_1(\prd{elfe})$), donc \(\Xlo[\prd{elfe}(\cnsi t1) \implq \prd{âne}(\cnsi t1)]\) est vraie, ce qui fait que la formule globale est vraie aussi.
\end{enumerate}

\end{Solution}
\begin{Solution}{2.{5}}
 (p.~\pageref{exo:[]})\label{crg:[]}

Nous dressons les tables de vérité (\S\ref{TabV}, p.~\pageref{TabV}) des formules de chaque paire et nous comparons les colonnes de résultats.
\begin{enumerate}
\item Tables de vérité de $\Xlo[[\phi \wedge \psi] \wedge \chi]$
et de $\Xlo[\phi \wedge [\psi \wedge \chi]]$ :
\small\[
\begin{array}{c|c|c||c|>{\columncolor[gray]{.9}}c||c|>{\columncolor[gray]{.9}}c}
\Xlo\phi & \Xlo\psi & \Xlo\chi & \Xlo[\phi \wedge \psi] & \cellcolor{white}\Xlo[[\phi \wedge \psi] \wedge
  \chi] & \Xlo[\psi \wedge \chi] & \cellcolor{white}\Xlo[\phi \wedge [\psi \wedge \chi]]
\\\hline\hline
1 & 1 & 1 & 1 & 1 & 1 & 1\\
1 & 1 & 0 & 1 & 0 & 0 & 0\\
1 & 0 & 1 & 0 & 0 & 0 & 0\\
1 & 0 & 0 & 0 & 0 & 0 & 0\\
0 & 1 & 1 & 0 & 0 & 1 & 0\\
0 & 1 & 0 & 0 & 0 & 0 & 0\\
0 & 0 & 1 & 0 & 0 & 0 & 0\\
0 & 0 & 0 & 0 & 0 & 0 & 0\\
\end{array}
\]\normalsize
Les deux colonnes de résultats sont identiques, donc les deux formules
sont bien équivalentes.

\item Tables de vérité de
 $\Xlo[[\phi \vee \psi] \vee \chi]$ et  $\Xlo[\phi
  \vee [\psi \vee \chi]]$:
\small\[
\begin{array}{c|c|c||c|>{\columncolor[gray]{.9}}c||c|>{\columncolor[gray]{.9}}c}
\Xlo\phi & \Xlo\psi & \Xlo\chi & \Xlo[\phi \vee \psi] & \cellcolor{white}\Xlo[[\phi \vee \psi] \vee
  \chi] & \Xlo[\psi \vee \chi] & \cellcolor{white}\Xlo[\phi \vee [\psi \vee \chi]]
\\\hline\hline
1 & 1 & 1 & 1 & 1 & 1 & 1\\
1 & 1 & 0 & 1 & 1 & 1 & 1\\
1 & 0 & 1 & 1 & 1 & 1 & 1\\
1 & 0 & 0 & 1 & 1 & 0 & 1\\
0 & 1 & 1 & 1 & 1 & 1 & 1\\
0 & 1 & 0 & 1 & 1 & 1 & 1\\
0 & 0 & 1 & 0 & 1 & 1 & 1\\
0 & 0 & 0 & 0 & 0 & 0 & 0\\
\end{array}
\]\normalsize
Même conclusion que précédemment.

\item Tables de vérité de
 $\Xlo[[\phi \implq \psi] \implq \chi]$ et $\Xlo[\phi
  \implq [\psi \implq \chi]]$ :
\small\[
\begin{array}{c|c|c||c|>{\columncolor[gray]{.9}}c||c|>{\columncolor[gray]{.9}}c}
\Xlo\phi & \Xlo\psi & \Xlo\chi & \Xlo[\phi \implq \psi] & \cellcolor{white}\Xlo[[\phi \implq \psi] \implq
  \chi] & \Xlo[\psi \implq \chi] & \cellcolor{white}\Xlo[\phi \implq [\psi \implq \chi]]
\\\hline\hline
1 & 1 & 1 & 1 & 1 & 1 & 1\\
1 & 1 & 0 & 1 & 0 & 0 & 0\\
1 & 0 & 1 & 0 & 1 & 1 & 1\\
1 & 0 & 0 & 0 & 1 & 1 & 1\\
0 & 1 & 1 & 1 & 1 & 1 & 1\\
0 & 1 & 0 & 1 & \cellcolor[gray]{.8}0 & 0 & \cellcolor[gray]{.8}1\\
0 & 0 & 1 & 1 & 1 & 1 & 1\\
0 & 0 & 0 & 1 & \cellcolor[gray]{.8}0 & 1 & \cellcolor[gray]{.8}1\\
\end{array}
\]\normalsize
Les deux colonnes de résultats sont différentes, donc les deux
formules ne sont pas équivalentes.
\end{enumerate}
\end{Solution}
\begin{Solution}{2.{6}}
 (p.~\pageref{exo:equivlog}) %Équivalences logiques.
\label{crg:equivlog}

Par défaut, les équivalences se démontrent par la méthode des tables de vérité comme dans l'exercice précédent, mais par endroits il est possible de procéder autrement.
\begin{enumerate}
\item \(\Xlo\phi\) et \(\Xlo\neg\neg\phi\)
\small\[
\begin{array}{>{\columncolor[gray]{.9}}c||c|>{\columncolor[gray]{.9}}c}
\cellcolor{white}\Xlo\phi & \Xlo\neg\phi & \cellcolor{white}\Xlo\neg\neg\phi\\\hline\hline
1&0&1\\
0&1&0\\
\end{array}
\]\normalsize
En fait, c'est presque laborieux de faire la table de vérité, car la démonstration est triviale et immédiate : $\Xlo\neg$ inverse les valeurs de vérité, donc si on inverse deux fois (ou un nombre paire de fois), on retombe sur la valeur initiale (comme retourner deux fois une pièce de monnaie).

\item \(\Xlo[\phi \implq \psi]\) et \(\Xlo[\neg\phi \vee \psi]\)
\small\[
\begin{array}{c|c||>{\columncolor[gray]{.9}}c||c|>{\columncolor[gray]{.9}}c}
\Xlo\phi & \Xlo\psi & \cellcolor{white}\Xlo[\phi\implq\psi] & \Xlo\neg\phi& \cellcolor{white}\Xlo[\neg\phi\vee\phi]
\\\hline\hline
1 & 1 & 1 & 0 & 1\\
1 & 0 & 0 & 0 & 0\\
0 & 1 & 1 & 1 & 1\\
0 & 0 & 1 & 1 & 1\\
\end{array}
\]\normalsize

\item \(\Xlo[\phi \implq \psi]\) et \(\Xlo[\neg\psi \implq \neg\phi]\)

Ici on pourrait assez rapidement et facilement faire la table de
vérité, mais on peut aussi démontrer l'équivalence très simplement,
en raisonnant.  On sait, par la démonstration précédente, que $\Xlo[\phi
  \implq \psi]$ équivaut à $\Xlo[\neg\phi \vee \psi]$. Pour la même
raison, on sait que $\Xlo[\neg\psi \implq \neg\phi]$ équivaut à
$\Xlo[\neg\neg\psi \vee \neg\phi]$, ce qui équivaut à
$\Xlo[\psi \vee \neg\phi]$ en vertu de la loi de double négation démontrée
ci-dessus en 1.  Et comme la disjonction est {commutative},
% Attention : commutatif pas défini dans le chapitre
cela équivaut à $\Xlo[\neg\phi \vee \psi]$, et donc à $\Xlo[\phi
  \implq \psi]$.

\item \(\Xlo[\phi \implq [\psi \implq \chi]]\) et \(\Xlo[[\phi \wedge \psi]
  \implq \chi]\)

Là aussi on peut démontrer l'équivalence en utilisant celle démontrée
ci-dessus en 2, ainsi que d'autres démontrées précédemment. Par
l'équivalence 2, on sait que
$\Xlo[\phi \implq [\psi \implq \chi]]$ équivaut à
$\Xlo[\neg\phi \vee [\psi \implq \chi]]$, qui équivaut à
$\Xlo[\neg\phi \vee [\neg\psi \vee \chi]]$.  Or on a démontré que $\Xlo\vee$
est commutative, donc la formule équivaut aussi à
$\Xlo[[\neg\phi \vee \neg\psi] \vee \chi]$.  Et par une des lois de
Morgan, on sait aussi que $\Xlo[\neg\phi \vee \neg\psi]$ équivaut à
$\Xlo\neg[\phi \wedge \psi]$.  Donc, par remplacement, notre formule
équivaut à $\Xlo[\neg[\phi \wedge \psi] \vee \chi]$. Or cette formule
répond au schéma $\Xlo[\neg X\vee Y]$ de l'équivalence 2. Donc on en
conclut que notre formule équivaut à \(\Xlo[[\phi \wedge \psi]
  \implq \chi]\).

\item \(\Xlo[\phi \wedge [\psi \vee \chi]]\) et \(\Xlo[[\phi \wedge \psi] \vee
  [\phi \wedge \chi]]\)
\small\[
\begin{array}{c|c|c||c|>{\columncolor[gray]{.9}}c||c|c||>{\columncolor[gray]{.9}}c}
\Xlo\phi & \Xlo\psi & \Xlo\chi & \Xlo[\psi \vee \chi] & \cellcolor{white}\Xlo[\phi \wedge [\psi \vee
  \chi]] & \Xlo[\phi \wedge \psi] & \Xlo[\phi \wedge \chi] & \cellcolor{white}\Xlo[[\phi \wedge\psi]\vee[\phi\wedge\chi]]
\\\hline\hline
1 & 1 & 1 & 1 & 1 & 1 & 1 & 1\\
1 & 1 & 0 & 1 & 1 & 1 & 0 & 1\\
1 & 0 & 1 & 1 & 1 & 0 & 1 & 1\\
1 & 0 & 0 & 0 & 0 & 0 & 0 & 0\\
0 & 1 & 1 & 1 & 0 & 0 & 0 & 0\\
0 & 1 & 0 & 1 & 0 & 0 & 0 & 0\\
0 & 0 & 1 & 1 & 0 & 0 & 0 & 0\\
0 & 0 & 0 & 0 & 0 & 0 & 0 & 0\\
\end{array}
\]\normalsize

\item \(\Xlo[\phi \vee [\psi \wedge \chi]]\) et \(\Xlo[[\phi \vee \psi] \wedge
  [\phi \vee \chi]]\)
\small\[
\begin{array}{c|c|c||c|>{\columncolor[gray]{.9}}c||c|c||>{\columncolor[gray]{.9}}c}
\Xlo\phi & \Xlo\psi & \Xlo\chi & \Xlo[\psi \wedge \chi] & \cellcolor{white}\Xlo[\phi \vee
  [\psi \wedge \chi]] & \Xlo[\phi \vee \psi] & \Xlo[\phi \vee \chi] & \cellcolor{white}\Xlo[[\phi \vee\psi]\wedge[\phi\vee\chi]]
\\\hline\hline
1 & 1 & 1 & 1 & 1 & 1 & 1 & 1\\
1 & 1 & 0 & 0 & 1 & 1 & 1 & 1\\
1 & 0 & 1 & 0 & 1 & 1 & 1 & 1\\
1 & 0 & 0 & 0 & 1 & 1 & 1 & 1\\
0 & 1 & 1 & 1 & 1 & 1 & 1 & 1\\
0 & 1 & 0 & 0 & 0 & 1 & 0 & 0\\
0 & 0 & 1 & 0 & 0 & 0 & 1 & 0\\
0 & 0 & 0 & 0 & 0 & 0 & 0 & 0\\
\end{array}
\]\normalsize
\end{enumerate}
\end{Solution}
\begin{Solution}{2.{7}}
(p.~\pageref{exo:vlibr})\label{crg:vlibr}

Nous encadrons chaque quantificateur et sa portée.  Les
variables libres sont notées en gras ($\vfree{x}$); les autres sont liées par le quantificateur indiqué par une flèche. L'exercice exploite les définitions \ref{d:portee} (p.~\pageref{d:portee}) et \ref{d:vlibr} (p.~\pageref{d:vlibr}).

\medskip

\newpsstyle{liage}{nodesepA=1pt,nodesepB=2pt,arrows=<-,linecolor=gray,angle=90,armA=1.2ex,armB=1ex}
 \begin{enumerate}[itemsep=2.5ex]
 \item \(\Xlo\fscp{\rnode{Q1}{\exists x} [\prd{aimer}(\rnode{x1}{x},\vfree{y}) \wedge \prd{âne}(\rnode{x2}{x})]}\)%
\ncbar[style=liage,offsetA=-2pt]{Q1}{x1}%
\ncbar[style=liage,offsetA=2pt,armA=1.8ex]{Q1}{x2}%
\hfill formule existentielle

 \item \(\Xlo\fscp{\rnode{Q1}{\exists x}\, \prd{aimer}(\rnode{x1}{x},\vfree{y})} \Xlo\wedge \prd{âne}(\vfree{x})\)
\ncbar[style=liage]{Q1}{x1}%
\hfill conjonction

 \item \(\Xlo\fscp{\rnode{Q1}{\exists x} \fscp{\rnode{Q2}{\exists y}\,  \prd{aimer}(\rnode{x1}{x},\rnode{y1}{y})}} \implq \prd{âne}(\vfree{x})\)
\ncbar[style=liage,armA=2ex]{Q1}{x1}%
\ncbar[style=liage,armA=1.5ex]{Q2}{y1}%
\hfill implication

 \item \(\Xlo\fscp{\rnode{Q1}{\exists x} [\fscp{\rnode{Q2}{\exists y} \, \prd{aimer}(\rnode{x1}{x},\rnode{y1}{y})} \implq \prd{âne}(\rnode{x2}{x})]}\)
\ncbar[style=liage,armA=2ex,offsetA=-2pt]{Q1}{x1}%
\ncbar[style=liage,armA=1.5ex]{Q2}{y1}%
\ncbar[style=liage,armA=2.5ex,offsetA=2pt]{Q1}{x2}%
\hfill formule existentielle

 \item \(\Xlo\neg\,\fscp{\rnode{Q1}{\exists x} \fscp{\rnode{Q2}{\exists y}\,  \prd{aimer}(\rnode{x1}{x},\rnode{y1}{y})}} \implq \prd{âne}(\vfree{x})\)
\ncbar[style=liage,armA=2ex]{Q1}{x1}%
\ncbar[style=liage,armA=1.5ex]{Q2}{y1}%
\hfill implication

\item \(\Xlo\neg \prd{âne}(\vfree{x}) \implq [\neg \fscp{\rnode{Q1}{\forall y} [\neg \prd{aimer}(\vfree{x},\rnode{y1}{y}) \vee \prd{âne}(\vfree{x})]} \implq
\prd{elfe}(\vfree{y})]\)
\ncbar[style=liage,armA=1.3ex]{Q1}{y1}%
\hfill implication

\item \(\Xlo\neg\, \fscp{\rnode{Q1}{\exists x} (\prd{aimer}(\rnode{x1}{x},\vfree{y}) \vee \prd{âne}(\vfree{y}))}\)
\ncbar[style=liage,armA=1.3ex]{Q1}{x1}%
\hfill négation

\item \(\Xlo\neg\, \fscp{\rnode{Q1}{\exists x}\, \prd{aimer}(\rnode{x1}{x},\rnode{x2}{x})} \vee \fscp{\rnode{Q2}{\exists y}\, \prd{âne}(\rnode{y1}{y})}\)
\ncbar[style=liage,offsetA=-2pt]{Q1}{x1}%
\ncbar[style=liage,armA=1.8ex,offsetA=2pt]{Q1}{x2}%
\ncbar[style=liage]{Q2}{y1}%
\hfill disjonction

\item \(\Xlo\fscp{\rnode{Q1}{\forall x} \fscp{\rnode{Q2}{\forall y} [[\prd{aimer}(\rnode{x1}{x},\rnode{y1}{y}) \wedge \prd{âne}(\rnode{y2}{y})] \implq \fscp{\rnode{Q3}{\exists z}\, \prd{mari-de}(\rnode{x2}{x},\rnode{z1}{z})}]}}\)
\ncbar[style=liage,offsetA=-2pt,armA=2.5ex]{Q1}{x1}%
\ncbar[style=liage,angle=-90,armA=1.8ex]{Q2}{y1}%
\ncbar[style=liage,armA=2ex]{Q2}{y2}%
\ncbar[style=liage,offsetA=2pt,armA=3ex]{Q1}{x2}%
\ncbar[style=liage,armA=2ex]{Q3}{z1}%
\hfill formule universelle

\item \(\Xlo\fscp{\rnode{Q1}{\forall x} [\fscp{\rnode{Q2}{\forall y}\, \prd{aimer}(\rnode{y1}{y},\rnode{x1}{x})} \implq \prd{âne}(\vfree{y})]}\)
\ncbar[style=liage,armA=1.5ex]{Q2}{y1}%
\ncbar[style=liage,armA=2ex]{Q1}{x1}%
\hfill formule universelle
 \end{enumerate}

\end{Solution}
\begin{Solution}{2.{8}}
 (p.~\pageref{exo:2denot2})\label{crg:2denot2}
\begin{exolist}
\item \(\Xlo\forall x [\prd{elfe}(x) \wedge \prd{farceur}(x)]\)

La règle (\RSem\ref{RIQ}b) nous dit que cette formule est vraie ssi pour \emph{toute} constante $\kappa$ du langage, nous trouvons \(\denote{\Xlo\prd{elfe}(\kappa) \wedge \prd{farceur}(\kappa)}^{\Modele_1}=1\).   Mais évidemment cette sous-formule est fausse pour de nombreuses constantes, par exemple \cnsi t1, puisque \Obj{Thésée} n'est pas un elfe.  La formule \(\Xlo\forall x [\prd{elfe}(x) \wedge \prd{farceur}(x)]\) est donc fausse dans $\Modele_1$.  En français, elle correspondra à \sicut{toute chose est un elfe farceur} ou \sicut{tout ce qui existe est un elfe farceur}.
Elle se distingue donc crucialement de \ref{x:QA1}, \(\Xlo\forall x [\prd{elfe}(x) \implq \prd{farceur}(x)]\), qui elle correspond à \sicut{tous les elfes sont farceurs} et qui est vraie dans $\Modele_1$ (cf.\ p.~\pageref{x:QA1}).


\item \(\Xlo\forall x [\prd{elfe}(x) \implq \neg\prd{triste}(x)]\)

\sloppy

Comme précédemment, la formule sera vraie ssi pour toute constante $\kappa$, on trouve \(\denote{\Xlo\prd{elfe}(\kappa) \implq \neg\prd{triste}(\kappa)}^{\Modele_1}=1\).  Pour toutes les constantes qui dénotent des individus qui ne sont pas des elfes, nous savons déja que cela sera vrai (puisque $\Xlo\prd{elfe}(\kappa)$ sera faux).  Il reste à examiner les constantes \cns o, \cns p et \cnsi t2.  Comme aucun des individus dénotés par ces trois constantes (\Obj{Obéron}, \Obj{Puck} et \Obj{Tita\-nia}) ne sont dans $\FI_1(\prd{triste})$, nous obtiendrons, dans les trois cas, \(\denote{\Xlo\neg\prd{triste}(\kappa)}^{\Modele_1}=1\).  Cela suffit à montrer que \(\Xlo\Xlo\prd{elfe}(\kappa) \implq \neg\prd{triste}(\kappa)\) est toujours vraie et donc que la formule globale est vraie.
Elle correspond en français à \sicut{aucun elfe n'est triste}.

\fussy

\item \(\Xlo\neg\exists x [\prd{âne}(x) \wedge \prd{elfe}(x)]\)

Étant donné $\FI_1(\prd{âne})$ et $\FI_1(\prd{elfe})$, nous constatons qu'il n'y a pas d'individu du modèle qui est à la fois dans les deux ensembles.  Autrement dit, il n'existe pas de constante $\kappa$ telle que \(\Xlo[\prd{âne}(\kappa) \wedge \prd{elfe}(\kappa)]\) soit vraie.  Nous en concluons donc, par la règle (\RSem\ref{RIQ}a), que \(\Xlo\exists x [\prd{âne}(x) \wedge \prd{elfe}(x)]\) est fausse et que \(\Xlo\neg\exists x [\prd{âne}(x) \wedge \prd{elfe}(x)]\) est vraie dans $\Modele_1$.
En français, la formule correspond à \sicut{il est faux qu'il y a un âne qui est un elfe} (ou \sicut{un elfe qui est un âne}), ce qui plus simplement peut se formuler en \sicut{aucun âne n'est un elfe} ou \sicut{aucun elfe n'est un âne}.

\item \(\Xlo\exists x \forall y \, \prd{aimer}(y,x)\)

Cette formule est vraie ssi il existe une constante $\kappa_1$ telle que $\Xlo\forall y \, \prd{aimer}(y,\kappa_1)$ est vraie dans $\Modele_1$.  Et $\Xlo\forall y \, \prd{aimer}(y,\kappa_1)$ est vraie ssi pour toutes les constantes $\kappa_2$ nous trouvons $\Xlo\prd{aimer}(\kappa_2,\kappa_1)$.
Une telle constante $\kappa_1$ devrait donc dénoter un individu qui se retrouvait 11 fois en seconde position d'un couple $\tuple{\Obj x,\Obj y}$ dans la dénotation de \prd{aimer} (puisqu'il y a 11 constantes possibles pour $\kappa_2$).  Mais en regardant $\FI_1(\prd{aimer})$, nous voyons immédiatement qu'il n'y a pas autant de couples dans l'ensemble et donc qu'une telle constante $\kappa_1$ n'existe pas.  Par conséquent la formule est fausse dans $\Modele_1$.  En français elle correspond à \sicut{il y a quelqu'un que tout le monde aime}.

\item \(\Xlo\forall y \exists x \, \prd{aimer}(y,x)\) \sloppy

Cette formule est vraie ssi pour toute constante $\kappa_1$ \(\Xlo\exists x \, \prd{aimer}(\kappa_1,x)\) est vraie dans $\Modele_1$.  Cela fait donc, \emph{en théorie}, 11 calculs à effectuer, et chaque \(\Xlo\exists x \, \prd{aimer}(\kappa_1,x)\) sera vraie si à chaque fois on trouve une constante $\kappa_2$ telle que \(\Xlo\prd{aimer}(\kappa_1,\kappa_2)\) est vraie.  Nous pouvons rapidement montrer que cela ne se produit pas dans $\Modele_1$ en prenant, par exemple, d'abord \cns p pour $\kappa_1$. Dans $\FI_1(\prd{aimer})$ il n'y a pas de couple de la forme \tuple{\Obj{Puck},\dots}, donc nous ne trouverons pas de constante $\kappa_2$ telle que \(\Xlo\prd{aimer}(\cns p,\kappa_2)\) soit vraie.  Ce qui veut dire qu'il y a au moins une constante $\kappa_1$ telle que \(\Xlo\exists x \, \prd{aimer}(\kappa_1,x)\) est fausse, ce qui suffit à montrer que la formule globale est fausse dans $\Modele_1$.  En français elle correspond à \sicut{tout le monde aime quelqu'un} (dans le sens de \sicut{chaque personne a quelqu'un qu'elle aime}).
\end{exolist}

\fussy

\end{Solution}
\protect \newpage 
\begin{Solution}{2.{9}}
 (p.~\pageref{exo:2q+c})\label{crg:2q+c}
\begin{exolist}
\item \(\Xlo\exists x [\prd{homard}(x) \wedge \prd{gaucher}(x)]\) :
il existe un individu qui est un homard et qui est gaucher ; en français cela donnera \sicut{il y a un homard gaucher} ou, éventuellement, \sicut{il existe des homards gauchers}.

\item \(\Xlo\exists x [\prd{homard}(x) \vee \prd{gaucher}(x)]\) :
il existe un individu qui est un homard ou qui est gaucher ; cette formule est vraie du moment que les homards existent (même s'il n'y a pas de gauchers dans le modèle)\footnote{Ou inversement, cette formule est vraie aussi du moment que les gauchers existent.}.  Pas de phrase naturelle en français pour cette formule.

\item \(\Xlo\exists x [\prd{homard}(x) \implq \prd{gaucher}(x)]\) :
il existe un individu tel que \emph{si} c'est homard alors il est gaucher ; cette formulation des conditions de vérité est un peu alambiquée, mais il faut se souvenir que cela équivaut à : il existe un individu qui n'est pas un homard ou qui est gaucher (car $\Xlo\phi\implq\psi$ équivaut à $\Xlo\neg\phi\vee\psi$) ; cette formule est vraie du moment qu'il existe des individus (par exemple vous et moi) qui ne sont pas des homards... Pas de phrase naturelle en français pour cette formule.

\item \(\Xlo\exists x [\prd{homard}(x) \ssi \prd{gaucher}(x)]\)
il existe un individu qui est un homard gaucher ou bien qui n'est ni homard ni gaucher. Pas de phrase naturelle en français pour cette formule.

\item \(\Xlo\forall x [\prd{homard}(x) \wedge \prd{gaucher}(x)]\) :
tout individu du modèle est un homard gaucher. Pas vraiment de phrase naturelle en français pour cette formule.

\item \(\Xlo\forall x [\prd{homard}(x) \vee \prd{gaucher}(x)]\) :
tout individu est un homard ou est gaucher ; autrement dit, pour tout individu, si ce n'est pas un homard, alors il doit forcément être gaucher (et vice-versa). Pas de phrase naturelle en français pour cette formule.

\item \(\Xlo\forall x [\prd{homard}(x) \implq \prd{gaucher}(x)]\) :
pour tout individu, s'il est un homard alors il est gaucher. En français : \sicut{tous les homards sont gauchers}.

\item \(\Xlo\forall x [\prd{homard}(x) \ssi \prd{gaucher}(x)]\) :
pour tout individu, si c'est un homard, alors il est gaucher et s'il est gaucher, alors c'est un homard. Pas vraiment de phrase naturelle en français pour cette formule, si ce n'est \sicut{«homard» et «gaucher», c'est la même chose}...

\end{exolist}
\end{Solution}
\begin{Solution}{2.{10}}
 (p.~\pageref{exotcc})

Dressons les table de vérité de ces formules.

\small
\[\begin{array}{c||c|c|c|c|c}
\multicolumn{2}{@{}r@{}}{\text{formules:}}&
\multicolumn{1}{c}{1}&
\multicolumn{1}{c}{2}&
\multicolumn{1}{c}{3}&
\multicolumn{1}{c}{4}\\
\Xlo\phi & \Xlo\neg\phi
& \Xlo[\phi \wedge \neg\phi]
& \Xlo[\phi \vee \neg\phi]
& \Xlo[\phi \implq \neg\phi]
& \Xlo[\phi \implq \phi]\\\hline\hline
1 & 0 & 0 & 1 & 0 & 1\\
0 & 1 & 0 & 1 & 1 & 1\\
\end{array}\]
\normalsize

\smallskip

La formule 1 est une contradiction, c'est même \emph{la loi de contradiction} vue en \S\ref{sss:contrad} (p.~\pageref{sss:contrad}); la formule 2 est une tautologie classique (\emph{la loi du tiers exclu}, qui dit qu'une formule est soit vraie, soit fausse, et qu'il n'y a donc pas de troisième --~ou \emph{tierce}~-- possibilité).  Malgré les apparences (et malgré nos attentes), la formule 4 n'est pas contradictoire : l'interprétation logique de l'implication matérielle en fait une formule contingente (qui est vraie quand son antécédent est faux); la formule 4 est, elle, bien une tautologie.

\small
\[\begin{array}{c||c|c|c|c|c|c}
\multicolumn{2}{@{}r@{}}{\text{formules:}}&
\multicolumn{1}{c}{}&
\multicolumn{1}{c}{}&
\multicolumn{1}{c}{5}&
\multicolumn{1}{c}{}&
\multicolumn{1}{c}{6}\\
\Xlo\phi & \Xlo\psi & \Xlo\neg\phi
& \Xlo[\phi \implq \psi]
& \Xlo\Xlo[\neg\phi\implq[\phi\implq\psi]]
& \Xlo[\psi\implq\phi]
& \Xlo\Xlo[[\phi\implq\psi] \vee [\psi\implq\phi]]\\\hline\hline
1 & 1 & 0 & 1 & 1 & 1 & 1\\
1 & 0 & 0 & 0 & 1 & 1 & 1\\
0 & 1 & 1 & 1 & 1 & 0 & 1\\
0 & 0 & 1 & 1 & 1 & 1 & 1\\
\end{array}\]
\normalsize

La formule 5 est une tautologie très remarquable de la logique classique, et on l'appelle \alien{e falso sequitur quodlibet} (cf. note~\ref{efsq}, p.~\pageref{efsq}), \ie\ \emph{du faux s'ensuit n'importe quoi}, ou \emph{à partir d'une hypothèse fausse on peut tout déduire}. En effet la formule commence par poser l'hypothèse que \vrb\phi\ est fausse ($\Xlo\neg\phi$), puis place \vrb\phi\ en antécédent d'une implication dont le conséquent est quelconque ($\Xlo\phi\implq\psi$).
La formule 6 est une tautologie un peu curieuse (par rapport à notre intuition) qui dit que pour deux formules quelconques, il y en a forcément une qui implique l'autre.  Comme la formule 3, elle tend à montrer que l'implication matérielle ne traduit pas idéalement le sens que nous attribuons aux structures conditionnelles (en \sicut{si}) de la langue.
\end{Solution}
\begin{Solution}{2.{11}}
 (p.~\pageref{exoconseql})

En pratique, il y a plusieurs façons d'utiliser la définition~\ref{def:conseqlog} (p.~~\pageref{def:conseqlog}) pour démontrer une conséquence logique.  On pourrait, par exemple, reprendre la méthode des contre-exemples vue en \S\ref{s:conseql}; on pourrait également dressez les tables de vérités des formules en n'examinant que les lignes pour lesquelles la formule de gauche est vraie.  Mais il est tout aussi simple et rapide de démontrer les conséquences par raisonnement, en partant de l'hypothèse que la formule de gauche est vraie (dans un modèle quelconque) et de déduire la vérité de la formule de droite.

\begin{enumerate}
\item \(\xlo{[\phi\wedge\psi]} \satisf \xlo{\phi}\).
Supposons que $\Xlo[\phi\wedge\psi]$ est vraie. Cela veut donc dire que les deux sous-formules sont vraies, et donc que \vrb\phi\ est vraie.  Évidemment, on a aussi \(\xlo{[\phi\wedge\psi]} \satisf \xlo{\psi}\).
\item \(\xlo{\phi} \satisf \xlo{[\phi\vee\psi]}\).
Supposons que \vrb\phi\ est vraie. Alors $\Xlo[\phi\wedge\psi]$ puisqu'il suffit qu'une des deux sous-formules soit vraie pour qu'une disjonction soit vraie.
\item \(\xlo{[\phi\implq\psi]} \satisf \xlo{[\neg\psi\implq\neg\phi]}\).
Supposons que $\Xlo[\phi\implq\psi]$ est vraie. Cela veut dire alors, en vertu de l'interprétation de $\Xlo\implq$,  que soit \vrb\phi\ est fausse, soit \vrb\psi\ est vraie.  Si \vrb\psi\ est vraie, alors $\Xlo\neg\psi$ est fausse et donc $\Xlo[\neg\psi\implq\neg\phi]$ est vraie.  Si $\Xlo\phi$ est fausse, alors $\Xlo\neg\phi$ est vraie et donc $\Xlo[\neg\psi\implq\neg\phi]$ est encore vraie.
\item \(\xlo{[\phi\wedge[\phi\implq\psi]]}\satisf\xlo{\psi}\).
Supposons que $\Xlo[\phi\wedge[\phi\implq\psi]]$ est vraie. Cela veut dire, d'abord, que \vrb\phi\  et  $\Xlo[\phi\implq\psi]$ sont toutes les deux vraies. Comme \vrb\phi\ est vraie, alors pour que $\Xlo[\phi\implq\psi]$ le soit aussi, il faut nécessairement \vrb\psi\ soit vraie.  Cette conséquence est, en fait, la célèbre règle logique dite du \alien{modus ponens}\is{modus ponens@\alien{modus ponens}}.
\item \(\xlo{[\phi\implq[\psi\wedge\neg\psi]]} \satisf \xlo{\neg\phi}\).
Supposons que $\Xlo[\phi\implq[\psi\wedge\neg\psi]$ est vraie. On sait par ailleurs que $\Xlo[\psi\wedge\neg\psi]$ est fausse puisque c'est une contradiction (cf. l'exercice précédent). Et le seul cas où une implication est vraie lorsque sont conséquent est faux est celui où sont antécédente est faux aussi. Donc \vrb\phi\ est fausse, et  $\Xlo\neg\phi$ est vraie.
\end{enumerate}
\end{Solution}
\protect \newpage 
\begin{Solution}{2.{12}}
(p.~\pageref{e:QCM1})\label{crg:QCM1}

\begin{enumerate}
\item d. %1
Remarque : c pourrait également être une traduction correcte, mais cette formule signifie  \sicut{pour chaque réponse, il y a au moins un étudiant qui la connaît} et il semble que ce ne soit pas une interprétation naturelle de la phrase 1.
\item b. %2
Sachant que cette formule est équivalente à \(\Xlo\forall x [\prd{étudiant}(x)\implq \exists y[\prd{réponse}(y) \wedge \prd{connaître}(x,y)]]\)

\item b. %3
Les formules c et d sont contradictoires. La formule a est une tautologie.

\item b. %4
La formule a signifie que chaque étudiant ne connaît aucune réponse.


\item a. %5
\item c. %6
\item b, d.  %7
(équivalentes)

\end{enumerate}
\end{Solution}
\begin{Solution}{2.{13}}
(p.~\pageref{e:elf})\label{crg:elf}

a--4 ; b--3 ; c--5 ; d--1 ; e--2.
\end{Solution}
\begin{Solution}{2.{14}}
 (p.~\pageref{exo:casso})\label{crg:casso}

\raggedright

\(\Unv A = \set{\Box_1; \heartsuit; \blacktriangle;
 \bigcirc; \blacksquare;
\bigstar ; \Box_2 ; \clubsuit ; \triangle}\).

\noindent
\(\FI(\prd{carré}) = \set{\Box_1; \blacksquare;\Box_2}\),
\(\FI(\prd{rond}) = \set{\bigcirc}\),
\(\FI(\prd{c\oe ur}) = \set{\heartsuit}\),
\(\FI(\prd{triangle}) = \set{\blacktriangle; \triangle}\),
\(\FI(\prd{étoile}) = \set{\bigstar}\),
\(\FI(\prd{trèfle}) = \set{\clubsuit}\),
\(\FI(\prd{blanc}) = \set{\Box_1; \heartsuit;\bigcirc;\Box_2;\triangle}\),
\(\FI(\prd{noir}) = \set{\blacktriangle; \blacksquare;\bigstar;\clubsuit}\),
\(\FI(\prd{objet}) = \Unv A\),
\(\FI(\prd{à-gauche-de}) = \set{\tuple{\Box_1,\heartsuit} ; \tuple{\Box_1,\blacktriangle}; \tuple{\Box_1,\blacktriangle} ;
\tuple{\bigcirc,\blacksquare} ;
\tuple{\bigstar,\Box_2} ; \tuple{\bigstar,\clubsuit} ; \tuple{\Box_2,\clubsuit}}\),
\(\FI(\prd{à-droite-de}) = \set{\tuple{\heartsuit,\Box_1} ; \tuple{\blacktriangle,\Box_1}; \tuple{\blacktriangle,\Box_1} ;
\tuple{\blacksquare,\bigcirc} ;
\tuple{\Box_2,\bigstar} ; \tuple{\clubsuit,\bigstar} ; \tuple{\clubsuit,\Box_2}}\),
\(\FI(\prd{au-dessus-de}) =
\set{\tuple{\Box_1,\bigstar}; \tuple{\heartsuit,\bigcirc} ; \tuple{\heartsuit,\Box_2} ; \tuple{\bigcirc,\Box_2} ; \tuple{\blacksquare,\triangle} ; \tuple{\blacktriangle , \clubsuit}}
\),
\(\FI(\prd{en-dessous-de}) =
\set{\tuple{\bigstar,\Box_1}; \tuple{\bigcirc,\heartsuit} ; \tuple{\Box_2,\heartsuit} ; \tuple{\Box_2,\bigcirc} ; \tuple{\triangle,\blacksquare} ; \tuple{\clubsuit,\blacktriangle}}
\).



\end{Solution}
\begin{Solution}{2.{15}}
 (p.~\pageref{exo:figaro})\label{crg:figaro}
%\begin{enumerate}\sloppy
%\item

1. Dénotation des formules.
\begin{enumerate}[label=\alph*.]
%\sloppy
\item \(\Xlo\exists x [\prd{femme}(x) \wedge \neg\prd{triste}(x)]\)

Pour que \(\denote{\Xlo\exists x [\prd{femme}(x) \wedge \neg\prd{triste}(x)]}^{\Modele}=1\), il faut trouver au moins une valeur de \vrb x telle que \(\Xlo\prd{femme}(x)\) et \(\Xlo\neg\prd{triste}(x)\) soient vraies dans \Modele.  Une valeur de \vrb x est une constante de \LO\ ou, ce qui revient au même, un individu de \Unv A.

Une telle valeur existe bien, par exemple \Obj{Suzn}.  En effet \Obj{Suzn} appartient à \(\FI(\prd{femme})\) (l'ensemble des femmes) et n'appartient pas à \(\FI(\prd{triste})\) (l'ensemble des individus tristes).

Donc \(\denote{\Xlo\exists x [\prd{femme}(x) \wedge \neg\prd{triste}(x)]}^{\Modele}=1\).

La formule correspond à la phrase \sicut{il y a une femme qui n'est pas triste}.

\item \(\Xlo\forall x [\prd{homme}(x) \implq [\prd{infidèle}(x) \vee \prd{volage}(x)]]\)

\sloppy

Pour que
\(\denote{\Xlo\forall x [\prd{homme}(x) \implq [\prd{infidèle}(x) \vee \prd{volage}(x)]]}^{\Modele}=1\), il faut que, pour chaque valeur que l'on peut assigner à  \vrb x, lorsque l'on répète le calcul de \(\denote{\Xlo [\prd{homme}(x) \implq [\prd{infidèle}(x) \vee \prd{volage}(x)]]}^{\Modele}\) on trouve $1$ à chaque fois.

On peut montrer rapidement que la formule est fausse en choisissant une valeur de \vrb x telle que \(\denote{\Xlo [\prd{homme}(x) \implq [\prd{infidèle}(x) \vee \prd{volage}(x)]]}^{\Modele}=0\) ; ce sera un individu qui appartient à l'ensemble des hommes mais qui n'appartient ni à l'ensemble des infidèles ni à celui des volages.  Par exemple \Obj{Figr} fait l'affaire.

Donc \(\denote{\Xlo\forall x [\prd{homme}(x) \implq [\prd{infidèle}(x) \vee \prd{volage}(x)]]}^{\Modele}=0\).

La formule correspond à \sicut{tous les hommes sont infidèles ou volages}.

\item \(\Xlo\exists x\exists y [\prd{aimer}(x,y) \wedge \prd{aimer}(y,x)]\)


\(\denote{\Xlo\exists x\exists y [\prd{aimer}(x,y) \wedge \prd{aimer}(y,x)]}^{\Modele}=1\)  ssi on trouve une valeur pour \vrb x telle que
\(\denote{\Xlo\exists y [\prd{aimer}(x,y) \wedge \prd{aimer}(y,x)]}^{\Modele}=1\).  Et cette sous-formule est vraie ssi on trouve une valeur pour \vrb y telle que
\(\denote{\Xlo [\prd{aimer}(x,y) \wedge \prd{aimer}(y,x)]}^{\Modele}=1\).

Il faut donc trouver deux individus \Obj x et \Obj y dans \Unv A, tels que \tuple{\Obj x, \Obj y} \emph{et} \tuple{\Obj y, \Obj x} appartiennent à \(\FI(\prd{aimer})\).  Il en existe : par exemple \Obj{Figr} et \Obj{Suzn}.

Donc \(\denote{\Xlo\exists x\exists y [\prd{aimer}(x,y) \wedge \prd{aimer}(y,x)]}^{\Modele}=1\).

Le sens de la formule se retrouve dans le sens de \sicut{il y a des gens qui s'aiment mutuellement}.

\item \(\Xlo[\prd{aimer}(\cnsi a1, \cns s) \implq \exists x\, \neg\prd{aimer}(x,\cnsi a1)]\)

La formule est une implication, donc
\(\denote{\Xlo[\prd{aimer}(\cnsi a1, \cns s) \implq \exists x\, \neg\prd{aimer}(x,\cnsi a1)]}^{\Modele}=1\) ssi
l'antécédent \(\Xlo\prd{aimer}(\cnsi a1, \cns s)\) est faux, \emph{ou} (donc si l'antécédent est vrai) le conséquent \(\Xlo\exists x\, \neg\prd{aimer}(x,\cnsi a1)\) est vrai.

L'antécédent est vrai, car \Obj{Almv} aime \Obj{Suzn} dans {\Modele}.  Vérifions donc que le conséquent est vraie aussi.


\(\denote{\Xlo\exists x\, \neg\prd{aimer}(x,\cnsi a1)}^{\Modele}=1\) ssi on trouve au moins une valeur pour \vrb x telle que
\(\denote{\Xlo\neg\prd{aimer}(x,\cnsi a1)}^{\Modele}=1\), c'est-à-dire telle que
\(\denote{\Xlo\prd{aimer}(x,\cnsi a1)}^{\Modele}=0\), c'est-à-dire un individu qui n'aime pas \Obj{Almv}.  Cette valeur est facile à trouver : par exemple \Obj{Figr} (en fait tous les individus de \Unv A sauf \Obj{Rosn} feront l'affaire).


Donc \(\denote{\Xlo[\prd{aimer}(\cnsi a1, \cns s) \implq \exists x\, \neg\prd{aimer}(x,\cnsi a1)]}^{\Modele}=1\).

Cette formule correspond à la phrase \sicut{Si Almaviva aime Suzanne, alors quelqu'un n'aime pas Almaviva}.

\item \(\Xlo\neg\exists x [\prd{femme}(x) \wedge \exists y\,\prd{aimer}(x,y)]\)

\sloppy
\(\denote{\Xlo\neg\exists x [\prd{femme}(x) \wedge \exists y\,\prd{aimer}(x,y)]}^{\Modele}=1\)
ssi on trouve que
\(\dlb\Xlo\exists x [\prd{femme}(x) \wedge\allowbreak\linebreak[4] \exists y\,\prd{aimer}(x,y)]\color{black}\drb^{\Modele}=0\).

Or
\(\denote{\Xlo\exists x [\prd{femme}(x) \wedge \exists y\,\prd{aimer}(x,y)]}^{\Modele}=1\) ssi
on trouve une valeur pour \vrb x telle que
\(\denote{\Xlo[\prd{femme}(x) \wedge \exists y\,\prd{aimer}(x,y)]}^{\Modele}=1\) (et si on ne trouve pas une telle valeur dans \Unv A, alors on aura montré que
\(\denote{\Xlo\exists x [\prd{femme}(x) \wedge \exists y\,\prd{aimer}(x,y)]}^{\Modele}=0\)).

Essayons avec la valeur \Obj{Rosn} (ou \cns r) pour \vrb x. Dans ce cas \(\Xlo\prd{femme}(x)\) est vrai. Reste à vérifier que \(\Xlo\exists y\,\prd{aimer}(x,y)\) l'est aussi. Or cette sous-formule est vraie également car on trouve bien une valeur «qui marche» pour \vrb y, par exemple \Obj{Almv} car dans \Modele, \Obj{Rosn} aime \Obj{Almv}.  Cela montre donc
\(\denote{\Xlo\exists x [\prd{femme}(x) \wedge \exists y\,\prd{aimer}(x,y)]}^{\Modele}=1\).  Et par conséquent,
\(\denote{\Xlo\neg\exists x [\prd{femme}(x) \wedge \exists y\,\prd{aimer}(x,y)]}^{\Modele}=0\).

La formule correspond à quelque chose comme \sicut{aucune femme n'aime quelqu'un} (car elle est la négation de \sicut{il y a au moins une femme qui aime quelqu'un}).

\item  \(\Xlo\forall x [[\prd{homme}(x) \wedge \exists y
  [\prd{épouse-de}(y,x) \wedge \neg\prd{aimer}(x,y)]] \implq \prd{volage}(x)]\)

Pour montrer que \(\Xlo\forall x [[\prd{homme}(x) \wedge \exists y
  [\prd{épouse-de}(y,x) \wedge \neg\prd{aimer}(x,y)]] \implq \prd{volage}(x)]\)
est vraie dans \Modele, il faut (en théorie) répéter le calcul que
\(\Xlo [[\prd{homme}(x) \wedge \exists y
  [\prd{épouse-de}(y,x) \wedge \neg\prd{aimer}(x,y)]] \implq \prd{volage}(x)]\)
pour toute les valeurs de \vrb x (c'est-à-dire tous les individus de \Unv A) et trouver $1$ à chaque fois.

Pour les individus qui ne sont pas des hommes, on sait tout de suite que cette implication est vraie car \(\Xlo[\prd{homme}(x) \wedge \exists y
  [\prd{épouse-de}(y,x) \wedge \neg\prd{aimer}(x,y)]]\) est faux (vu que \(\Xlo\prd{homme}(x)\) est faux).

Pour les autres (les hommes, \Obj{Almv}, \Obj{Figr}, \Obj{Chrb}, \Obj{Antn}, \Obj{Bart}), il y a deux cas de figure : soit \(\Xlo \exists y
  [\prd{épouse-de}(y,x) \wedge \neg\prd{aimer}(x,y)]\) est faux, et dans ce cas l'implication est vraie ; soit \(\Xlo \exists y
  [\prd{épouse-de}(y,x) \wedge \neg\prd{aimer}(x,y)]\) et dans ce cas il faut que \(\Xlo\prd{volage}(x)\) soit vrai aussi pour que l'implication soit vraie.

Dans le cas de \Obj{Chrb}, \Obj{Antn} et \Obj{Bart},  \(\Xlo \exists y
  [\prd{épouse-de}(y,x) \wedge \neg\prd{aimer}(x,y)]\) car il n'existe pas de valeur pour \vrb y qui marche (aucun des trois n'apparaît dans \(\FI(\prd{épouse-de})\), i.e aucun n'est marié).  L'implication est donc vraie pour ces trois valeurs de \vrb x.

Dans le cas \Obj{Figr} \(\Xlo \exists y
  [\prd{épouse-de}(y,x) \wedge \neg\prd{aimer}(x,y)]\) est faux aussi, car bien que \Obj{Figr} ait une épouse (\Obj{Suzn}) il est faux qu'il ne l'aime pas.  Donc quand \vrb x est \Obj{Figr}, il n'y a pas de valeur de \vrb y qui marche. Et pour \Obj{Figr}, l'implication est encore vraie.

Enfin pour la valeur \Obj{Almv}, alors \(\Xlo \exists y
  [\prd{épouse-de}(y,x) \wedge \neg\prd{aimer}(x,y)]\)  est vrai, car \Obj{Rosn} est une valeur de \vrb y qui marche. Et il se trouve qui \(\Xlo\prd{volage}(x)\) est vrai aussi car \Modele\ nous dit que \Obj{Almv} est volage.  L'implication est donc vraie pour la valeur \Obj{Almv} de \vrb x.


%Conclusion :
Donc l'implication \(\Xlo [[\prd{homme}(x) \wedge \exists y
  [\prd{épouse-de}(y,x) \wedge \neg\prd{aimer}(x,y)]] \implq \prd{volage}(x)]\)
est vraie pour toutes les valeurs de \vrb x, ce qui prouve que
la formule f dénote $1$ dans~\Modele.
%\(\denote{\Xlo\forall x [[\prd{homme}(x) \wedge \exists y [\prd{épouse-de}(y,x) \wedge \neg\prd{aimer}(x,y)]] \implq \prd{volage}(x)]}^{\Modele}=1\).


La formule correspond à \sicut{tout homme qui n'aime pas son épouse est volage}.

\end{enumerate}

\bigskip

%\item

2.
Voici la nouvelle version du modèle
 \(\Modele=\tuple{\Unv A,\FI}\) (les modifications sont soulignées) : %marquées en rouge) :

\begin{itemize}\raggedright
\item \(\Unv A=\set{\Obj{Almv}; \Obj{Rosn}; \Obj{Figr}; \Obj{Suzn}; \Obj{Mrcl};
  \Obj{Chrb}; \Obj{Fnch}; \Obj{Antn}; \Obj{Bart}}\) ;

\item \(\FI(\cnsi a1) = \Obj{Almv} ;
\FI(\cns r) = \Obj{Rosn} ;
\FI(\cnsi f1) = \Obj{Figr} ;
\FI(\cns s) = \Obj{Suzn} ;
\FI(\cns m) = \Obj{Mrcl} ;
\FI(\cns c) = \Obj{Chrb} ;
\FI(\cnsi f2) = \Obj{Fnch} ;
\FI(\cnsi a2) = \Obj{Antn} ;
\FI(\cns b) = \Obj{Bart}\) ;

\item
\(\FI(\prd{homme})=\set{\Obj{Almv}; \Obj{Figr}; \Obj{Chrb}; \Obj{Antn}; \Obj{Bart}}\) ;
\item
\(\FI(\prd{femme})=\set{\Obj{Rosn};\Obj{Suzn}; \Obj{Mrcl}; \Obj{Fnch}}\) ;

\item
\(\FI(\prd{domestique})=\set{\Obj{Figr}; \Obj{Suzn}; \Obj{Fnch}; \Obj{Antn}}\) ;
\item \(\FI(\prd{noble}) = \set{\Obj{Almv}; \Obj{Rosn};  \Obj{Chrb} }\) ;

\item
\(\FI(\prd{roturier}) = \set{\Obj{Figr}; \Obj{Suzn}; \Obj{Mrcl};
 \Obj{Fnch}; \Obj{Antn}; \Obj{Bart}}\) ;

\item
\(\FI(\prd{comte}) = \set{\Obj{Almv}}\) ;
\qquad \quad
\(\FI(\prd{comtesse}) = \set{\Obj{Rosn}}\) ;

\item
\(\FI(\prd{infidèle}) = \set{\Obj{Almv}}\) ;
\qquad
%\item
\(\FI(\prd{volage}) = \set{\Obj{Almv}; \Obj{Chrb}}\) ;

\item
\(\uline{\FI(\prd{triste}) = \varnothing}\) ;

\item
\(\FI(\prd{époux-de}) = \set{\tuple{\Obj{Almv}, \Obj{Rosn}};\tuple{\Obj{Figr}, \Obj{Suzn}} ; \uline{\tuple{\Obj{Chrb},\Obj{Fnch}}}}\) ;
\item
\(\FI(\prd{épouse-de}) = \set{\tuple{\Obj{Rosn},\Obj{Almv}};\tuple{\Obj{Suzn},\Obj{Figr}} ; \uline{\tuple{\Obj{Fnch},\Obj{Chrb}}}}\) ;

\item
\(\FI(\prd{père-de}) = \set{\tuple{\Obj{Antn},\Obj{Fnch}}}\) ;

\item
\(\FI(\prd{aimer}) = \set{\tuple{\Obj{Figr},
    \Obj{Suzn}};\tuple{\Obj{Suzn},\Obj{Figr}};
\tuple{\Obj{Almv},\uline{\Obj{Rosn}}};
\tuple{\Obj{Rosn},\Obj{Almv}}; \tuple{\Obj{Chrb},
    \Obj{Fnch}}; \tuple{\Obj{Fnch},\Obj{Chrb}}; \tuple{\Obj{Mrcl},\Obj{Figr}}}\)
\\
  \uline{(et on a enlevé \tuple{\Obj{Chrb},
    \Obj{Rosn}} et \tuple{\Obj{Chrb}, \Obj{Suzn}})}
\end{itemize}
%\end{enumerate}

\fussy

\end{Solution}
\protect \section {Chapitre \protect \ref {ch:gn}}
\begin{Solution}{3.{1}}
(p.~\pageref{exo:3portee})\label{crg:3portee}
\begin{enumerate}
\item
Chaque convive a raconté plusieurs histoires qui impliquaient un
membre de la famille royale.

\begin{enumerate}
\item Chaque convive $>$ plusieurs histoires $>$ un membre :
pour chaque convive \Obj x, \Obj x a raconté plusieurs histoires, et chacune de ces histoires implique un membre (possiblement différent) de la famille royale.

\item Chaque convive $>$ un membre $>$ plusieurs histoires :
pour chaque convive \Obj x, il y a un membre (particulier) de la famille royale au sujet duquel \Obj x a raconté plusieurs histoires.

\item Un membre $>$ chaque convive $>$ plusieurs histoires :
il y a un membre (particulier) de la famille royale au sujet duquel tous les convives ont raconté plusieurs histoires (possiblement différentes).

\item Un membre $>$ plusieurs histoires $>$ chaque convive :
il y a un membre (particulier) de la famille royale au sujet duquel plusieurs histoires ont été racontées par tous les convives (et chacun racontait les mêmes histoires que les autres).

\item Plusieurs histoires $>$ un membre $>$ chaque convive :
il y a plusieurs histoires impliquant divers membres de la famille royale qui ont été racontées par tous les convives (et chacun racontait les mêmes histoires que les autres).

\sloppy
\item Il y a une dernière lecture théoriquement possible mais pragmatiquement étrange et qui correspond à : plusieurs histoires $>$ chaque convive $>$ un membre.  Cette lecture apparaît spécifiquement dans un scénario comme le suivant : il y a une certaine série d'histoires et tous les convives racontent cette même série, mais chacun change le personnage sur lequel portent les histoires.
  À la rigueur cette lecture peut passer si on comprend \sicut{histoire} comme dénotant des types ou modèles d'histoires mais pas d'anecdotes précises.

\fussy
\end{enumerate}

\item
Chaque sénateur a raconté à plusieurs journalistes qu'un membre du
cabinet était corrompu.

\begin{enumerate}
\item Chaque sénateur $>$ plusieurs journalistes $>$ un membre :
chaque sénateur \Obj x s'adresse à plusieurs journalistes et à chaque journaliste, \Obj x parle d'un membre corrompu (possiblement différent pour chaque journaliste).

\item Chaque sénateur $>$ un membre $>$ plusieurs journalistes :
pour chaque sénateur \Obj x il y a un membre \Obj y du cabinet dont \Obj x parle à plusieurs journalistes.

\item  Un membre $>$ chaque sénateur $>$ plusieurs journalistes :
il y a un membre corrompu \Obj y et chaque sénateur parle de \Obj y à plusieurs journalistes (les journalistes peuvent être différents selon les sénateurs).

\item  Un membre $>$ plusieurs journalistes $>$ chaque sénateur :
il y a un membre corrompu \Obj y et un groupe particulier de journaliste et tous les sénateurs parlent de \Obj y à ce groupe de journalistes.

\item  Plusieurs journalistes $>$ un membre $>$ chaque sénateur :
il y a un groupe de journalistes, pour chacun de ces journalistes il y a un membre corrompu (possiblement différent d'un journaliste à l'autre) et chaque sénateur parle de ce membre au journaliste qui lui est «associé».

\item  Plusieurs journalistes $>$ chaque sénateur $>$ un membre :
il y a un groupe donné de journalistes, tous les sénateurs s'adressent à ce même groupe et chacun parle d'un membre corrompu différent.

\end{enumerate}

NB : il existe encore d'autres lectures possibles (et subtiles) de cette phrase qui mettent en jeu un phénomène que nous aborderons plus précisément en \S\ref{ss:re/dicto}, mais la troisième phrase ci-dessous nous en donne un petit aperçu (avec notamment la lecture~3c).

\item
Un professeur croit que chaque étudiant a lu un roman de Flaubert.

\begin{enumerate}
\item Un professeur $>$ (que) chaque étudiant $>$ un roman :
il y a un professeur qui pense que chaque étudiant a choisi librement un roman de Flaubert et l'a lu.

\item Un roman $>$ un professeur $>$ (que) chaque étudiant :
il y a un roman écrit par Flaubert et un professeur qui pense que tous les étudiants ont lu ce roman.

\item Un professeur $>$ (que) un roman $>$ chaque étudiant :
il y a un professeur qui pense à un roman particulier, par exemple \emph{Le rouge et le noir}, il pense que ce roman est de Flaubert (mais il se trompe, en l'occurrence) et pense que tous les étudiants l'ont lu.
Notons cependant que cette lecture n'implique nécessairement que le professeur se trompe sur l'auteur du roman, mais elle est compatible avec un tel cas de figure.

\end{enumerate}

\end{enumerate}
\end{Solution}
\begin{Solution}{3.{2}}
(p.~\pageref{exo:speci})\label{crg:speci}

Si nous voulons montrer que \ref{x:vampire} est ambiguë selon la définition \ref{d:ambig}, nous devons construire un modèle par rapport auquel la phrase sera jugée vraie et fausse selon que le \GN\ a une lecture spécifique ou non.
Un tel modèle doit comporter minimalement l'individu \Obj{Alice}, une jeune fille,  un autre individu, appelons-le \Obj{Lestat}, qui est un vampire et \Obj{Lestat} a mordu \Obj{Alice} pendant la nuit.  Mais dans ce modèle \ref{x:vampire} sera toujours vraie, que \sicut{un vampire} soit spécifique ou non.  La lecture (ou l'usage) non spécifique émerge quand le locuteur ne connaît pas l'identité de \Obj{Lestat}, mais cette information n'est pas codée dans le modèle (tels que les modèles sont définis dans notre système).  Nous devons donc conclure que \ref{x:vampire} n'est pas sémantiquement ambiguë (si elle l'est, il s'agira plutôt d'une ambiguïté pragmatique).
\end{Solution}
\begin{Solution}{3.{3}}
(p.~\pageref{exo:specdicto})\label{crg:specdicto}

Dans la phrase (\ref{x:specpourqui}), \sicut{un vampire} à forcément un usage spécifique du fait de l'apposition.  D'après la définition de la spécificité (p.~\pageref{def:spécificité}), nous conclurons que le locuteur pense à un certain vampire, et plus précisément Dracula, et affirme qu'Alice pense que ce vampire l'a mordue.  Mais il y a, par ailleurs, une façon très naturelle de comprendre (\ref{x:specpourqui}) selon laquelle c'est avant tout Alice qui pense à un vampire particulier, Dracula, et pense qu'il l'a mordue.  Autrement dit \sicut{un vampire} s'avère d'abord spécifique pour Alice avant de l'être pour locuteur.  Foncièrement cela ne change probablement pas grand chose pour l'analyse sémantique de la phrase (si le \GN\ est spécifique pour Alice, il l'est a fortiori aussi pour le locuteur), mais du point de vue de son adéquation descriptive,   l'analyse passe un peu à côté de cette caractéristique qui est que la source de la spécificité n'est pas toujours le fait du locuteur.
\end{Solution}
\begin{Solution}{3.{4}}
 (p.~\pageref{exo:Spec3}) \label{crg:Spec3}

Évidemment, la lecture la plus naturelle de la phrase est celle avec un usage spécifique du \GN\ puisque le locuteur raconte une expérience vécue personnellement (il a donc une idée précise de l'acteur en question).   Un contexte permettant cet usage serait par exemple une situation où le locuteur est invité à une soirée, il y rencontre Bryan Cranston, qui est un acteur américain, il le reconnaît pour l'avoir vu dans des films (sans forcément se souvenir de son nom) et il nous rapporte cet épisode avec la phrase (1).

Pour que le \GN\ ait un usage non spécifique, il faudrait que le locuteur sache qu'il a rencontré un individu qui est un acteur américain tout en ignorant de quel individu il s'agit (c'est-à-dire de qui il parle précisément).  C'est là que réside la difficulté.  Supposons d'abord que le locuteur ait rencontré de nombreuses personnes au cours de la soirée, dont Bryan Cranston (que le locuteur ne connaît pas et n'a jamais vu en film ou à la télévision).  Dans ce contexte, le locuteur peut prononcer la phrase (1) sans avoir une idée précise de qui il parle (puisqu'il a rencontré plusieurs personnes ce qui laisse le choix dans l'identité de l'individu).  Mais comment sait-il alors qu'il s'agit d'un acteur américain ?  Une possibilité est qu'il l'apprenne par une tierce personne : un ami du locuteur présent à la soirée l'a vu discuter avec Bryan Cranston et l'informe le lendemain en lui disant «hier tu as parlé avec un acteur américain» (sans plus de précision). À partir de là, le locuteur peut alors prononcer la phrase (1) avec un usage non spécifique du \GN, du moment qu'il ne sait toujours pas lequel des invités était l'acteur.
\end{Solution}
\begin{Solution}{3.{5}}
 (p.~\pageref{exo:DefPlur})\label{crg:DefPlur}

\noindent\emph{Test de covariation} (cf. \ref{xGNsubitcv}, p.~\pageref{xGNsubitcv}): on place le GN défini pluriel dans une phrase qui commence, par exemple, par \sicut{à plusieurs reprises}.

\begin{enumerate}[label=(\arabic*)]
\item À plusieurs reprises, \emph{les prisonniers} ont tenté de \label{xpriso-f}s'évader.
\end{enumerate}

Cette phrase ne semble pas manifester une covariation du défini pluriel : il est question d'une certaine compagnies de prisonniers et c'est toujours à celle-ci que sont attribuées les tentatives d'évasion.
Il est très important de noter ici que l'on peut voir dans cette phrase ce que l'on appelle parfois la lecture de groupe, ou encore lecture solidaire (en anglais on parle de \alien{team credit}), du défini pluriel. Par cette lecture, il n'est pas nécessaire, dans les faits, que \emph{tous} les prisonniers aient été impliqués dans la tentative d'évasion pour imputer l'action à l'ensemble du groupe. Les responsables réels de la tentative peuvent donc ne pas être toujours les mêmes. Mais pour nous, cela ne change rien, il n'y a pas covariation.  Car le propre de cette lecture est justement d'assigner (à tort ou à raison) le prédicat verbal  à tout le groupe, et ce groupe reste toujours le même (sous encore l'hypothèse de la note \ref{fnmonpriso} p.~\pageref{fnmonpriso} du chapitre).

\smallskip

\noindent\emph{Test de la négation} (cf. \ref{xGNambigNeg}, p.~\pageref{xGNambigNeg}): on place le GN défini pluriel en position d'objet dans une phrase négative.

\begin{enumerate}[label=(\arabic*),resume]
\item Jean n'a pas lu \emph{les dossiers}.\label{jnapaslulesd}
\end{enumerate}

Nous voyons apparaître ici une distinction sémantique subtile et non triviale entre \sicut{tous les} et \sicut{les}.
Il est généralement admis que \ref{jnapaslulesd} n'est pas ambiguë et qu'elle signifie uniquement que Jean n'a lu aucun des dossiers ; une situation où il aurait lu certains dossiers mais pas tous est le plus souvent perçue comme rendant la phrase ni vraie ni fausse. C'est donc qu'il y a probablement une affaire de présupposition ; nous ne la développerons pas ici mais elle sera abordée au chapitre~\ref{GN++} (vol.~2).

\smallskip

\noindent\emph{Test de consistance} (cf. \ref{test:contra}, p.~\pageref{test:contra}): on vérifie si la conjonction \sicut{les $N$ GV et les $N$ non-GV} est une contradiction.

\begin{enumerate}[label=(\arabic*),resume]
\item Les candidats sont barbus et les candidats sont imberbes.\label{test:contraf}
\end{enumerate}


On constate que si l'une des deux phrases connectées est vraie, l'autre est forcément fausse.  La phrase \ref{test:contraf} ne peut donc jamais être vraie.
Et si l'une des deux est fausse, la phrase \ref{test:contraf} est bien sûr immédiatement fausse.  Le défini pluriel semble donc se comporter ici comme \sicut{tous les N}.   Mais que se passe-t-il dans
 un modèle où il y a, par exemple, 50\% de barbus et 50\% d'imberbes ?
Ici les jugements sont un peu délicats ; cependant il se dégage généralement  une tendance  qui est que les deux propositions seront jugées ni vraies ni fausses, mais plutôt inappropriées. Cela est, encore une fois,  lié à l'effet présuppositionnel mentionné \alien{supra} et qui fait que le test de consistance réussit, mais seulement «à moitié» (la phrase n'est jamais vraie, mais elle n'est pas exactement toujours fausse).   Cet effet présuppositionnel se retrouve aussi dans le test de complétude.

\smallskip

\noindent\emph{Test de complétude} (cf. \ref{test:compl}, p.~\pageref{test:compl}) : on vérifie si la disjonction \sicut{les $N$ GV ou les $N$ non-GV} est une tautologie.

\begin{enumerate}[label=(\arabic*),resume]
\item Les candidats sont barbus ou les candidats sont imberbes.\label{test:complf}
\end{enumerate}

Certes si les candidats sont tous barbus ou s'ils sont tous imberbes, la phrase \ref{test:complf} sera globalement vraie.  Inversement, s'il y a, par exemple, 50\% de barbus et 50\% d'imberbes, aucune des deux phrases connectées ne sera vraie, mais, comme précédemment, il sera difficile de les tenir pour fausses pour autant.  Et donc là encore, le test réussit seulement à moitié.

Ces tests, et notamment les deux derniers, montrent que les définis pluriels ont un comportement un peu à part, comparés aux autres \GN, même si, globalement, ils se rapprochent plus de la catégorie de \sicut{le N} que de \sicut{tous les N}.
\end{Solution}
\begin{Solution}{3.{6}}
(p.~\pageref{exoInduCovar})\label{crg:InduCovar}

Une façon de définir un test de la covariation induite par un \GN\ consiste à placer ce \GN\ dans une position sujet d'une phrase avec un indéfini singulier en position de complément d'objet.  Nous observons ensuite si cet indéfini peut ou non être multiplié par le sujet.

\ex.[]
\a.  \emph{Julie} a corrigé une page wikipédia.
\b.  \emph{Elle} a corrigé une page wikipédia.
\b.  \emph{L'étudiante} a corrigé une page wikipédia.
\b.  \emph{Cette étudiante} a corrigé une page wikipédia.
\b.  \emph{Mon étudiante} a corrigé une page wikipédia.
\b.  \jcovar \emph{Elles} ont corrigé une page wikipédia.
\b.  \jcovar \emph{Les étudiants} ont corrigé une page wikipédia.
\b.  \jcovar \emph{Ces étudiants} ont corrigé une page wikipédia.
\b.  \jcovar \emph{Mes étudiants} ont corrigé une page wikipédia.
\b.  \emph{Une étudiante} a corrigé une page wikipédia.
\b.  \jcovar \emph{Trois étudiants} ont corrigé une page wikipédia.
\b.  \jcovar \emph{Des étudiants} ont corrigé une page wikipédia.
\b.  \jcovar \emph{Plusieurs étudiants} ont corrigé une page wikipédia.
\b.  \jcovar \emph{Quelques étudiants} ont corrigé une page wikipédia.
\b.  \jcovar \emph{Aucun étudiant} n'a corrigé une page wikipédia.
\b.  \jcovar \emph{La plupart des étudiants} ont corrigé une page wikipédia.
\b.  \jcovar \emph{Tous les étudiants} ont corrigé une page wikipédia.
\b.  \jcovar \emph{Chaque étudiant} a corrigé une page wikipédia.
\b. etc.


Globalement, nous remarquons que les \GN\ pluriels induisent de la covariation (ainsi que \sicut{chaque $N$}).  Faisons tout de même deux remarques.  Pour les définis pluriels (g--i), la covariation de l'objet semble peut-être moins naturelle et moins courante, mais a priori nous ne pouvons pas complètement en exclure la possibilité\footnote{Bien sûr, la covariation devient nette si l'on ajoute le quantificateur dit flottant \sicut{tous} (comme dans \sicut{ces étudiants ont tous corrigé une page wikipédia}) mais cela ne montre rien puisque ça peut très probablement être \sicut{tous} qui est responsable du phénomène et pas le \GN\ en soi.}.  Le jugement porté sur \sicut{aucun étudiant} (o) est discutable ; nous pouvons certes y observer une multiplication par $0$, mais celle-ci peut être induite par la négation (\sicut{ne}) qui accompagne \sicut{aucun}.
\end{Solution}
\begin{Solution}{3.{7}}
(p.~\pageref{exo:CatGN})\label{crg:CatGN}

\sicut{Un tiers des candidats} est un indéfini car, d'un point de vue strictement sémantique, il ne satisfait pas le test de consistance (p.~\pageref{test:contra}) : \sicut{un tiers des candidats sont barbus et un tiers des candidats sont imberbes} peut être jugé vrai dans une situation qui comporte, par exemple, 50\% de barbus et 50\% d'imberbes, si nous considérons qu'\sicut{un tiers} signifie \sicut{au moins un tiers} (cf. \S\ref{ss:implicatures}).

\sicut{Trois quarts des candidats} est quantificationnel, car il satisfait le test de consistance (\sicut{trois quarts des candidats sont barbus et trois quarts des candidats sont imberbes} ne peut pas être vrai), mais il ne satisfait pas le test de complétude (p.~\pageref{test:compl}): \sicut{trois quarts des candidats sont barbus ou trois quarts des candidats sont imberbes} est faux par exemple dans une situation avec 50\% de barbus et 50\% d'imberbes.

Pour \sicut{beaucoup de candidats}, la classification est moins simple, car en fait le déterminant est ambigu.  Commençons avec le test de consistance : \sicut{beaucoup de candidats sont barbus et beaucoup de candidats sont imberbes}.  Supposons que nous sommes dans une situation où il y a 200 candidats barbus et 200 candidats imberbes et que nous estimons que 200 est un nombre important de candidats (par exemple parce que nous en attendions seulement une soixantaine), dans ce cas la phrase pourra être jugée vraie.  Et cela suffit à prouver que le GN\ est un indéfini.

Mais on peut comprendre le déterminant \sicut{beaucoup de} d'une
autre manière. Dans les phrases ci-dessus, \sicut{beaucoup de} est
interprété comme signifiant «une grande quantité de» et la
«grandeur» de cette quantité dépend du contexte. L'autre interprétation de
\sicut{beaucoup de} est celle qui signifie quelque chose comme «une grande proportion de» ; là encore la grandeur de cette proportion dépend habituellement du contexte.  Selon le seuil que fixe le contexte pour estimer qu'on est en présence d'une \emph{grande} proportion,  \GN\ apparaîtra alors comme quantificationnel (si le seuil est supérieur à 50\%) ou indéfini (si le seuil est inférieur à 50\%) (cf. \S\ref{ss:QGDet}).   Dans certains cas, il est possible cependant d'associer à \sicut{beaucoup de} un sens qui ne dépend pas du contexte en l'interprétant comme signifiant «la plus grande proportion de».  Dans ce cas le \GN\ sera quantificationnel.  En
effet dans ce cas \sicut{beaucoup de candidats sont barbus et beaucoup de candidats sont imberbes} est contradictoire, car quel que soit le
nombre de candidats au total, ce sont soit les barbus, soit les
imberbes qui représentent la plus grande proportion, mais pas les deux
à la fois.  Quant à \sicut{beaucoup de candidats sont barbus et beaucoup de candidats sont imberbes}, elle est fausse dans le cas où il y a
autant de barbus que d'imberbes dans le groupe de candidats.
\end{Solution}
\begin{Solution}{3.{8}}
 (p.~\pageref{exo:DefCovar})\label{crg:DefCovar}

Il y a différentes façons de construire un {\GN} défini qui covarie
nettement avec une autre expression de la phrase, mais toutes se
ramènent généralement au même phénomène.

Premier type de constructions : une expression quantifiée apparaît en
complément ou en modifieur de la tête nominale du {\GN} défini.  Exemple :

\begin{enumerate}[label=(\arabic*)]
\item \label{x:def1}
Paul a contrefait \underline{la signature de \emph{chaque membre
  de sa famille}}.
\end{enumerate}


Dans (\ref{x:def1}), il est bien question d'autant de signatures qu'il
y a de membres de la famille, il y a donc bien une multiplication,
c'est-à-dire un effet de covariation.

Deuxième type de construction : le {\GN} défini contient un pronom (ou un
élément anaphorique) qui est lié d'une manière ou d'une autre (par
exemple par son antécédent) à une expression quantifée.  Exemple :

\begin{enumerate}[label=(\arabic*),resume]
\item \label{x:def2}
\emph{Tout dompteur$_i$} craint \underline{le lion qu'\emph{il$_i$}
est en train de dompter}.

\item
\emph{Chaque candidat$_i$} doit remplir \underline{le formulaire qui
\emph{lui$_i$} a été remis}.
\end{enumerate}

Là encore, dans (\ref{x:def2}) par exemple, il y a autant de lions que
de dompteurs, bien que le {\GN} soit singulier.
Remarque : l'élément anaphorique en question peut être implicite :

\begin{enumerate}[label=(\arabic*),resume]
\item \emph{Dans chaque appartement que nous avons visité}, \underline{la salle de bain} était minuscule.
\end{enumerate}

Ici il s'agit de la salle de bain «de lui», où \sicut{lui}
est l'appartement dont il est question à chaque fois\footnote{Ce phénomène est usuellement désigné par le terme d'\emph{anaphore associative}.}.

Pour les éléments d'explication, voir \S\ref{sss:defdep} p. \pageref{sss:defdep}.

\end{Solution}
\begin{Solution}{3.{9}}
 (p.~\pageref{exo:tradiota})\label{crg:tradiota}

%Traductions avec $\Xlo\atoi$.
D'après la règle (\RSem\ref{RIatoi}) p.~\pageref{RIatoi}, \(\Xlo\atoi x\phi\) dénote l'unique individu \vrb x tel que \vrb\phi\ est vraie.  Donc si $N$ se traduit par \vrb\alpha, \sicut{le $N$} se traduira par $\Xlo\atoi x\,\alpha(x)$.
\begin{enumerate}
\item Le maire a rencontré le pharmacien.
\\$\leadsto$ \(\Xlo\prd{rencontrer}(\atoi x\,\prd{maire}(x),\atoi x\,\prd{pharmacien}(x))\)

\item Gontran a perdu son chapeau.
%\\$\leadsto$ \(\Xlo\prd{perdre}(\cns g,\atoi x [\prd{chapeau}(x)\wedge \prd{poss}(y,x)]) \wedge y=\cns g\)
%\\ou directement
\\$\leadsto$ \(\Xlo\prd{perdre}(\cns g,\atoi x [\prd{chapeau}(x)\wedge \prd{poss}(\cns g,x)])\)

Le prédicat \prd{poss} exprime la possession : $\Xlo\prd{poss}(x,y)$ signifie que \vrb x possède \vrb y.

\item Le boulanger a prêté l'échelle au cordonnier.
\\$\leadsto$ \(\Xlo\prd{prêter}(\atoi x\,\prd{boulanger}(x),\atoi x\,\prd{échelle}(x),\atoi x \,\prd{cordonnier}(x))\)

\item La maîtresse a confisqué le lance-pierres de l'élève.
\\$\leadsto$ \(\Xlo\prd{confisquer}(\atoi x\,\prd{maîtresse}(x),\atoi x [\prd{lance-pierres}(x)\wedge \prd{poss}(\atoi y \,\prd{élève}(y),x)])\)

Remarque : là encore on peut traduire \sicut{le lance-pierres de
  l'élève} en utilisant toujours la variable $\vrb x$ : \(\Xlo\atoi x
[\prd{lance-pierres}(x)\wedge \prd{poss}(\atoi x
  \,\prd{élève}(x),x)]\), car on peut reconnaître que les différentes
occurrences de $\vrb x$ ne sont pas liées par le même $\Xlo\atoi x$.
Cependant, pour des raisons de lisibilité et de confort, il est
naturel d'utiliser ici deux variables distinctes.

\item Gontran a attrapé le singe qui avait volé son chapeau.
%\\$\leadsto$ \(\Xlo\prd{attraper}(\cns g, \atoi x [\prd{singe}(x)\wedge
%  \prd{voler}(x,\atoi y [\prd{chapeau}(y)\wedge \prd{poss}(z,y)])])\wedge z=\cns g\)
%\\ou directement
\\$\leadsto$ \(\Xlo\prd{attraper}(\cns g, \atoi x [\prd{singe}(x)\wedge
  \prd{voler}(x,\atoi y [\prd{chapeau}(y)\wedge \prd{poss}(\cns g,y)])])\)

Remarque :  le contenu de la relative, \sicut{qui avait volé son
  chapeau}, se retrouve dans la portée de la première description
définie dès lors que l'on comprend cette relative comme une relative
\emph{restrictive} : on parle ici de l'unique l'individu qui est à la
fois un
singe \emph{et}  a volé le chapeau de Gontran. Dans ce cas, la
phrase pourra être vraie même s'il y a plusieurs singes dans le modèle
(et le contexte), du moment qu'un seul ait volé le chapeau.

Au contraire, si on comprend la relative comme une relative
\emph{descriptive} (on mettrait alors plus volontiers une virgule :
\sicut{Gontran a attrapé le singe, qui avait volé son chapeau}), la
phrase ne peut être vraie que s'il n'y a qu'un seul singe dans le
contexte. La traduction serait alors :
\\$\leadsto$ \(\Xlo\prd{attraper}(\cns g, \atoi x\,\prd{singe}(x)) \wedge
  \prd{voler}(\atoi x\,\prd{singe}(x),\atoi y
      [\prd{chapeau}(y)\wedge \prd{poss}(\cns g,y)])\)

\item Celui qui a gagné le gros-lot, c'est Fabrice.
\\$\leadsto$ \(\Xlo\atoi x\, \prd{gagner}(x,\atoi y \,\prd{gros-lot}(y)) =
\cns f\)

\item Celui qui a gagné un cochon, c'est Fabrice.
\\$\leadsto$ \(\Xlo\atoi x \exists y [\prd{cochon}(y) \wedge \prd{gagner}(x,y)] =
\cns f\)

\item Tout cow-boy aime son cheval.
%\\$\leadsto$ \(\Xlo\forall x [\prd{cow-boy}(x)\implq [\prd{aimer}(x,\atoi
%    y [\prd{cheval}(y)\wedge\prd{poss}(z,y)]) \wedge z=x]]\)
%\\ ou directement:
\\$\leadsto$ \(\Xlo\forall x [\prd{cow-boy}(x)\implq \prd{aimer}(x,\atoi
    y [\prd{cheval}(y)\wedge\prd{poss}(x,y)])]\)

\item Georges a vendu son portrait de Picasso.
\\ Comme vu dans le chapitre, cette phrase est ambiguë\footnote{NB :  ces traductions incluent un argument \vrb z pour \prd{vendre} en considérant que $\Xlo\prd{vendre}(x,y,z)$ signifie \vrb x vend \vrb y à \vrb z.} :
\begin{enumerate}
\item $\leadsto$ \(\Xlo\exists z\, \prd{vendre}(\cns g,\atoi x[\prd{portrait}(x,\cns p)\wedge\prd{posséder}(\cns g,x)],z)\)\\
le portrait représente Picasso et appartient à Georges
\item $\leadsto$ \(\Xlo\exists z\, \prd{vendre}(\cns g,\atoi x[\prd{portrait}(x,\cns p)\wedge\prd{réaliser}(\cns g,x)],z)\)\\
le portrait représente Picasso et a été réalisé par Georges (bien sûr si Georges le vend, c'est aussi qu'il le possédait, mais dans ce cas, c'est une inférence supplémentaire qui n'est pas directement liée au déterminant possessif)
\item $\leadsto$ \(\Xlo\exists z\, \prd{vendre}(\cns g,\atoi x[\exists y\,\prd{portrait}(x,y)\wedge\prd{réaliser}(\cns p,x) \wedge \prd{posséder}(\cns g,x)],z)\)\\
le portrait a été réalisé par Picasso et appartient à Georges (et on ne sait pas qui il représente)
\item $\leadsto$ \(\Xlo\exists z\, \prd{vendre}(\cns g,\atoi x[\prd{portrait}(x,\cns g)\wedge\prd{réaliser}(\cns p,x)],z)\)\\
le portrait représente Georges et a été réalisé par Picasso (cette interprétation est moins naturelle, en français on dira plutôt \sicut{son portrait par Picasso})
\end{enumerate}

\end{enumerate}

\smallskip

Remarque : en toute rigueur, dans tous ces exemples, la traduction des {\GN} définis devrait toujours comporter une sous-formule supplémentaire qui restreint les valeurs pertinentes de la variable, comme vu en \S\ref{ss:RestrDQuant}.  Autrement dit, la traduction complète de la phrase 1 est en fait :

\begin{enumerate}
\item \(\Xlo\prd{rencontrer}(\atoi x[\prd{maire}(x) \wedge C_1(x)],\atoi x[\prd{pharmacien}(x)\wedge C_2(x)])\)
\end{enumerate}
Cela permet de faire référence à l'unique individu qui est maire \emph{dans l'ensemble \vrbi C1} et l'unique individu qui est pharmacien \emph{dans l'ensemble \vrbi C2}.  Il est probable que la phrase s'interprète naturellement avec \vrbi C1 $=$ \vrbi C2 (il peut s'agir, par exemple, de l'ensemble des habitants d'un village ou d'un quartier -- où il n'y aura qu'un seul pharmacien), mais par souci de généralité, il est prudent de considérer que chaque {\GN} introduit son propre ensemble de restriction.
\end{Solution}
\protect \section {Chapitre \protect \ref {Ch:t+m}}
\begin{Solution}{4.{1}}
(p.~\pageref{exo:derededicto})\label{crg:derededicto}

Pour illustrer les différentes lectures, je propose un scénario qui rend vrai chacune d'elles (mais cela ne veut pas dire nécessairement que chacune de ces lectures n'est vraie \emph{que} dans le scénario particulier qui lui est associé).  Il s'agit bien sûr d'alternances combinées \dedicto/\dere\ (cf. \S\ref{ss:re/dicto}, p.~\pageref{p.re/dicto}).

\begin{enumerate}
\item Eugène$_1$ trouve que le Pape ressemble à son$_1$ arrière-grand-père.
  \begin{enumerate}
  \item \sicut{le Pape}, \sicut{son arrière-grand-père} : \dedicto\\
  Eugène se dit : «c'est marrant, le Pape ressemble beaucoup à mon arrière-grand-père».
  \item \sicut{le Pape} : \dere, \sicut{son arrière-grand-père} : \dedicto\\
  Eugène voit une photo du Pape, sans savoir de qui il s'agit, et il se dit : «ce type ressemble à mon arrière-grand-père».
  \item \sicut{le Pape} : \dedicto, \sicut{son arrière-grand-père} : \dere\\
  Eugène voit une photo de son arrière-grand-père, sans savoir de qui il s'agit, et il se dit : «Le Pape, il ressemble à ce type».
  \item \sicut{le Pape}, \sicut{son arrière-grand-père} : \dere\\
  Eugène voit une photo du Pape, sans savoir de qui il s'agit, et une photo de son arrière-grand-père, sans le reconnaître non plus, et il se dit : «ces deux types se ressemblent beaucoup».
  \end{enumerate}

\item \OE dipe$_1$ ne {savait} pas que {sa$_1$ mère} était {sa$_1$ mère}.
\\
NB : comme déjà mentionné dans le chapitre, il convient ici de faire abstraction de la polysémie du nom \sicut{mère} (mère biologique {\vs} mère adoptive {\vs} mère sociale etc.) afin de ne pas surmultiplier les interprétations.

\begin{enumerate}
\item  \sicut{sa mère} : \dedicto\ $\times 2$\\
Quelqu'un dit à \OE dipe : «Ta mère est ta mère» (ce qui est une tautologie), et il répond (sans ironie)  : «Ah ? Je ne savais pas».  C'est évidemment une situation très absurde.

\item 1\ier\ \sicut{sa mère} : \dere, 2\ieme\ \sicut{sa mère} : \dedicto\\
Jocaste est la mère d'\OE dipe et quelqu'un dit à \OE dipe, en désignant Jocaste : «Cette femme est ta mère».  \OE dipe répond : «Ah ? Je ne savais pas».

\item 1\ier\ \sicut{sa mère} : \dedicto, 2\ieme\ \sicut{sa mère} : \dere\\
Jocaste est la mère d'\OE dipe, \OE dipe ne sait pas qui est sa mère (il ne l'a jamais connue et n'a aucune information sur elle) et quelqu'un lui dit, en désignant Jocaste : «Ta mère, c'est cette femme».  \OE dipe répond : «Ah ? Je ne savais pas».  Naturellement c'est similaire à la situation précédente.

\item  \sicut{sa mère} : \dere\ $\times 2$\\
Jocaste est la mère d'\OE dipe et quelqu'un dit à \OE dipe : «Jocaste est Jocaste».   \OE dipe répond : «Ah ? Je ne savais pas».  C'est une situation absurde similaire à la première.

\end{enumerate}

\end{enumerate}
\end{Solution}
\begin{Solution}{4.{2}}
(p.~\pageref{exo:PF})\label{crg:PF}

Nous appliquons les règles (\RSem\ref{RSemTps}) de la définition \ref{Def:SemPF} p.~\pageref{RSemTps} qui, en substance, disent que  \(\denote{\Xlo\mP\phi}^{\Modele,i,g}=1\) ssi il y a un instant $i'$ avant $i$ auquel \vrb\phi\ est vraie et \(\denote{\Xlo\mF\phi}^{\Modele,i,g}=1\) ssi il y a un instant $i'$ après $i$ auquel \vrb\phi\ est vraie.

\begin{enumerate}
\item \(\denote{\Xlo\mP\prd{dormir}(\cns d) \wedge \mF\prd{dormir}(\cns d)}^{\Modele,i_3,g}=0\) parce que $\Xlo\mF\prd{dormir}(\cns d)$ est faux à $i_3$ (personne ne dort à $i_4$).

\item \(\denote{\Xlo\mP[\prd{dormir}(\cns b) \wedge \prd{dormir}(\cns c)]}^{\Modele,i_4,g}=0\) parce qu'il n'y a pas d'instants avant $i_4$ où Bruno et Charles dorment en même temps.

\item \(\denote{\Xlo\mP\prd{dormir}(\cns b) \wedge \mP\prd{dormir}(\cns c)}^{\Modele,i_4,g} =1\)
parce que $\Xlo\prd{dormir}(\cns b)$ est vraie à $i_1$ (avant $i_4$) et $\Xlo\mP\prd{dormir}(\cns c)$ est vraie à $i_2$ (avant $i_4$).

\item \(\denote{\Xlo\prd{dormir}(\cns c) \implq \mF\prd{dormir}(\cns c)}^{\Modele,i_2,g}=1\) parce que Charles dort à $i_2$ et aussi à $i_3$ qui est après $i_2$.

\item \(\denote{\Xlo\mF\neg\exists x\,\prd{dormir}(x)}^{\Modele,i_2,g}=1\) parce que personne ne dort à $i_4$.

\item \(\denote{\Xlo\neg\exists x\,\mF\prd{dormir}(x)}^{\Modele,i_2,g}=0\) parce que Charles dort à $i_3$.

\end{enumerate}

\sloppy

Pour traduire \sicut{tout le monde a dormi}, nous avons le choix entre \(\Xlo\mP\forall x\,\prd{dormir}(x)\) et \(\Xlo\forall x\,\mP\prd{dormir}(x)\).  La première formule dit qu'il existe un moment dans le passé où tout le monde dort ; autrement dit, tout le monde a dormi au même moment, et il n'est pas certain que la phrase du français véhicule cette condition.  La seconde formule dit que pour chaque individu, il y a un moment dans le passé durant lequel il dort ; c'est une traduction plus générale que la première et qui convient probablement mieux aux conditions de vérité de la phrase.

\fussy

\end{Solution}
\begin{Solution}{4.{3}}
(p.~\pageref{exo:PPP})\label{crg:PPP}

Commençons par poser les conditions
de vérité de $\Xlo\mP\phi$ et $\Xlo\mP\mP\mP\phi$ en appliquant la règle (\RSem\ref{RSemTps}) (déf. \ref{Def:SemPF} p.~\pageref{RSemTps}).
\begin{itemize}
\item \(\denote{\Xlo\mP\phi}^{\Modele,i,g}=1\) ssi il existe un instant $i'$ tel que $i'\tprec i$ et \(\denote{\Xlo\phi}^{\Modele,i',g}=1\).

\item \(\denote{\Xlo\mP\mP\mP\phi}^{\Modele,i,g}=1\) ssi il existe trois instants $i'$, $i''$ et $i'''$ tels que $i'''\tprec i'' \tprec i' \tprec i$ et \(\denote{\Xlo\phi}^{\Modele,i''',g}=1\). En effet, $\Xlo\mP\mP\mP$ nous fait faire trois bonds dans le passé, ce qui peut donc se résumer par cette formulation de conditions de vérité.
\end{itemize}

À partir de là, on constate rapidement que \(\xlo{\mP\mP\mP\phi}\satisf\xlo{\mP\phi}\).  Si $\Xlo\mP\mP\mP$ est vraie à $i$, alors $\Xlo\phi$ est vraie à $i'''$, or $i'''$ est un instant du  passé de $i$. Donc il existe bien un instant antérieur à $i$ où $\Xlo\phi$ est vraie, ce qui fait que $\Xlo\mP\phi$ est vraie à $i$. Autrement dit : si l'on fait trois bonds dans le passé pour aller vérifier $\Xlo\phi$, on peut tout aussi bien le faire avec un seul grand bond -- qui couvre les trois précédents.

En revanche a-t-on \(\xlo{\mP\phi}\satisf\xlo{\mP\mP\mP\phi}\) ? Pour s'en assurer, cherchons un contre-exemple. Il s'agirait d'un cas (et même d'un instant $i$) où $\Xlo\mP\phi$ est vraie et $\Xlo\mP\mP\mP\phi$ est fausse. Pour cela, il faut que {\Tps} comporte un instant $i'$ tel que 1) $i'\tprec i$, 2) $\Xlo\phi$ est vraie à $i'$, 3) il existe \emph{au maximum} un instant intercalé entre $i'$ et $i$  (mais pas deux !), et 4) $\Xlo\phi$ n'est vraie à aucun instant antérieur à $i'$.  C'est donc un cas de figure très particulier où $\Xlo\phi$ n'est vraie que dans un passé très proche de $i$.

Autrement dit, dans tous les cas où $\Xlo\mP\mP\mP\phi$ est vraie, $\Xlo\mP\phi$ est vraie, et dans \emph{presque} tous les cas où $\Xlo\mP\phi$ est vraie, $\Xlo\mP\mP\mP\phi$ est vraie aussi. «Presque» tous les cas ne suffit pas pour conclure à une équivalence logique, mais on n'en est pas loin.  D'autant plus que si on ajoute l'hypothèse que {\Tps} contient une infinité d'instants et surtout que $\tprec$ est un ordre dense, cela implique, par définition de la densité, que pour toute paire d'instants de {\Tps}, il en existe toujours un troisième (et donc une infinité) situé entre les deux. Et dans ce cas la condition 3 ci-dessus ne peut pas être vérifiée ; il n'y a pas de contre-exemple à \(\xlo{\mP\phi}\satisf\xlo{\mP\mP\mP\phi}\) et les deux formules sont alors sémantiquement équivalentes. Si on ne pose pas cette hypothèse, elles ne sont pas équivalentes, mais elles sont sémantiquement très proches.
\end{Solution}
\begin{Solution}{4.{4}}
 (p.~\pageref{exo4:mod1})\label{crg:mod1}

Les dénotations sont calculées en utilisant les règles d'interprétation (\RSem\ref{RSemMod}) de la définition \ref{d:semMod1}, p.~\pageref{d:semMod1}.
\sloppy\begin{enumerate}
\item \(\denote{\Xlo\peut\prd{dormir}(\cns a)}^{\Modele,\w_1,i_1,g}= 1\) car Alice dort dans $\w_2$.
\item \(\denote{\Xlo\peut\prd{dormir}(\cns a)}^{\Modele,\w_2,i_1,g}=1\) pour la même raison.
\item \(\denote{\Xlo\doit\prd{dormir}(\cns a)}^{\Modele,\w_1,i_1,g}=0\) car il y a des mondes où Alice ne dort pas à $i_1$, par exemple~$\w_1$.
\item \(\denote{\Xlo\doit\prd{dormir}(\cns b)}^{\Modele,\w_1,i_1,g}=1\) car Bruno dort dort dans les quatre mondes à $i_1$.
\item \(\denote{\Xlo\peut\prd{dormir}(\cns c)}^{\Modele,\w_3,i_1,g}=0\) car Charles ne dort dans aucun monde à $i_1$.
\item \(\denote{\Xlo\neg\doit\prd{dormir}(\cns d)}^{\Modele,\w_4,i_1,g}=1\) car Dina ne dort pas dans tous les mondes.
\item \(\denote{\Xlo\peut\prd{dormir}(\cns a)\wedge\peut\prd{dormir}(\cns d)}^{\Modele,\w_4,i_1,g}=1\) car Alice dort dans $\w_2$ et Dina dort dans~$\w_3$.
\item \(\denote{\Xlo\peut[\prd{dormir}(\cns a)\wedge\prd{dormir}(\cns d)]}^{\Modele,\w_4,i_1,g}=0\) car il n'y a pas de monde dans lequel à la fois Alice et Dina dorment.
\end{enumerate}
\fussy
\end{Solution}
\begin{Solution}{4.{5}}
 (p.~\pageref{exo:4equiv})\label{crg:4equiv}

Les équivalences s'infèrent à partir des règles d'interprétation des modalités  (\RSem\ref{RSemMod}) p.~\pageref{RSemMod} et de celle de la négation (\RSem\ref{RIneg}), \S\ref{s:reglsem} p.~\pageref{RIneg}.
Ce sont les suivantes :

\begin{exolist}
\item $\Xlo\peut\neg\phi$ et $\Xlo\neg\doit\phi$ : la première dit qu'il y a un monde où \vrb\phi\ est fausse, et la seconde qu'il est faux que \vrb\phi\ est vraie dans tous les mondes.

$\Xlo\peut\neg\phi$ {\rtrad} \sicut{il est possible que non {\vrb\phi}}
et
$\Xlo\neg\doit\phi$ {\rtrad} \sicut{il n'est pas nécessaire que {\vrb\phi}}.

\item $\Xlo\doit\neg\phi$ et $\Xlo\neg\peut\phi$ : la première dit que \vrb\phi\ est fausse dans tous les mondes, et la seconde qu'il n'y a pas de monde où \vrb\phi\ est vraie.

$\Xlo\doit\neg\phi$ {\rtrad} \sicut{il est nécessaire que non {\vrb\phi}}
et $\Xlo\neg\peut\phi$ {\rtrad} \sicut{il est impossible que {\vrb\phi}}.

\item $\Xlo\neg\peut\neg\phi$ et $\Xlo\doit\phi$ : la première dit qu'il n'y a pas de monde où \vrb\phi\ est fausse, et la seconde que \vrb\phi\ est vraie dans tous les mondes.

$\Xlo\neg\peut\neg\phi$ {\rtrad} \sicut{il est impossible que non {\vrb\phi}}
et $\Xlo\doit\phi$ {\rtrad} \sicut{il est nécessaire que {\vrb\phi}}.

\item $\Xlo\neg\doit\neg\phi$ et $\Xlo\peut\phi$ : la première dit que \vrb\phi\ n'est pas fausse dans tous les mondes et la seconde qu'il y a un monde où \vrb\phi\ est vraie.

$\Xlo\neg\doit\neg\phi$ {\rtrad} \sicut{il n'est pas nécessaire que non {\vrb\phi}}
et $\Xlo\peut$ {\rtrad} \sicut{il est possible que {\vrb\phi}}.

\end{exolist}


\end{Solution}
\begin{Solution}{4.{6}}
(p.~\pageref{exo:mHisto})\label{crg:mHisto}

Selon la règle d'interprétation de $\mF$ (\RSem\ref{RSemTps}), p.~\pageref{RSemTps}, et celle de $\Xlo\doit$ (\RSem\ref{RSemMod}), p.~\pageref{d:semMod2}, appliquée aux modalités historiques (\S\ref{s:branchants}, p.~\pageref{p.mhisto}),  $\Xlo\doit\mF\phi$ dit que pour tout  monde actuellement identique au monde d'évaluation courant, il y a un instant du futur où \vrb\phi\ est vraie. $\Xlo\mF\doit\phi$ dit qu'il existe un instant du futur durant lequel, dans tous les mondes qui seront alors identiques au monde d'évaluation courant, \vrb\phi\ est vraie.  Reprenons la figure \ref{futurs4}, p.~\pageref{futurs4}, et supposons que $\vrb\phi$ est vraie dans les états \tuple{\w_3,i_5} et \tuple{\w_4,i_5} et que $\vrb\phi$ est fausse dans tous les états de $\w_5$ et $\w_6$.  Plaçons nous dans l'état \tuple{\w_4,i_4}.  De là, \(\denote{\Xlo\mF\doit\phi}^{\Modele,\w_4,i_4}=1\) car $i_5$ est postérieur à $i_4$ et dans tous les mondes en relation avec $\w_4$ à l'instant $i_5$ (c'est-à-dire $\w_3$ et $\w_4$), \vrb\phi\ est vraie. Mais \(\denote{\Xlo\doit\mF\phi}^{\Modele,\w_4,i_4}=0\) car tous les mondes reliés à $\w_4$ à l'instant $i_4$ sont $\w_3$, $\w_4$, $\w_5$ et $\w_6$, et dans $\w_5$ et $\w_6$ il n'y a pas d'instant postérieur à $i_4$ où \vrb\phi\ est vraie.

Soit \tuple{w,i} un état du monde quelconque.  Et supposons que $\Xlo\mP\phi$ est vraie dans \tuple{w,i}.  Donc il existe un instant $i'$ antérieur à $i$ tel que \vrb\phi\ est vraie dans \tuple{w,i'}.  Par définition des relations d'accessibilité historiques, tous les mondes $w'$ reliés à $w$ par $\RK_i$ sont aussi reliés à $w$ par $\RK_{i'}$ (parce que $i'\tprec i$), et tous les états \tuple{w',i'} sont identiques à \tuple{w,i'}.  Donc comme $\vrb\phi$ est vraie dans \tuple{w,i'}, nous savons que $\vrb\phi$ est vraie dans tous les états \tuple{w',i'} ; cela permet de conclure que $\Xlo\mP\phi$ est vraie dans tous les états \tuple{w',i} et donc que $\Xlo\doit\mP\phi$ est vraie dans \tuple{w,i}. Ainsi, si $\Xlo\mP\phi$ est vraie dans \tuple{w,i}, alors $\Xlo\doit\mP\phi$ est vraie dans \tuple{w,i}, c'est-à-dire que $\Xlo[\mP\phi\implq\doit\mP\phi]$ est vraie dans \tuple{w,i}.

\end{Solution}
\begin{Solution}{4.{7}}
 (p.~\pageref{exo:4winp})\label{crg:4winp}

La démonstration est immédiate :
en sémantique intensionnelle, comme le pose la notation \ref{nota:4.3} p.~\pageref{nota:4.3} (avec la simplification de \S\ref{ss:s&ig}, p.~\pageref{ss:s&ig}),
\(\denote{\Xlo\phi}^{\Modele,w,g}=1\) signifie que \vrb\phi\ est vraie dans $w$ ;
d'après le point \ref{pt:prop°}, p.~\pageref{pt:prop°}, \(\Ch{\denote{\Xlo\phi}^{\Modele,g}}\) est l'ensemble de tous les mondes où \vrb\phi\ est vraie ; donc si
\(w \in
\Ch{\denote{\Xlo\phi}^{\Modele,g}}\) c'est que $w$ est un monde où \vrb\phi\ est vraie.
\end{Solution}
\begin{Solution}{4.{8}}
(p.~\pageref{exo:redictoSue})\label{crg:redictoSue}
\begin{enumerate}
\item Nous allons présenter les intensions de formules (\ie\ les propositions) sous la forme d'ensembles de mondes.
L'intension de \(\Xlo\exists x [\prd{républicain}(x) \wedge \prd{élu}(x)]\) est, par définition (p.~\pageref{pt:prop°}), l'ensemble de tous les mondes dans lesquels celui qui est élu est un républicain. D'après le tableau~\ref{t:republ}, p.~\pageref{t:republ}, il s'agit de l'ensemble \set{\w_1;\w_2;\w_3;\w_6}.

\item Avec $g_1$ qui donne $g_1(\vrb x)=\Obj{Barry}$, l'intension de \(\Xlo\prd{élu}(x)\) est l'ensemble de tous les mondes où \Obj{Barry} est élu. C'est-à-dire : \set{\w_1;\w_3;\w_5;\w_7}.
Et avec $g_2$ qui donne $g_2(\vrb x)=\Obj{Johnny}$, l'intension de \(\Xlo\prd{élu}(x)\) est l'ensemble de tous les mondes où \Obj{Johnny} est élu : \set{\w_2;\w_4;\w_6;\w_8}.
\item Par rapport à $g_1$, l'intension de \(\Xlo\prd{penser}(\cns s,\Intn\prd{élu}(x))\)  est \set{\w_2;\w_3;\w_4} (d'après les indications de l'énoncé de l'exercice).  Et par rapport à $g_2$, l'intension de la formule  est \set{\w_5;\w_6;\w_8}.
\item Donc dans $\w_2$, $\w_3$ et $\w_4$, \Obj{Sue} pense que \Obj{Barry} va être élu, et dans $\w_5$, $\w_6$ et $\w_8$ elle pense que \Obj{Johnny} va être élu.  Dans $\w_2$, $\w_3$ et $\w_4$, \Obj{Barry} est républicain et dans $\w_5$ et $\w_6$  \Obj{Johnny} est républicain.  Donc l'intension de \(\Xlo\exists x [\prd{républicain}(x) \wedge \prd{penser}(\cns s,\Intn\prd{élu}(x))]\) est \set{\w_2;\w_3;\w_4;\w_5;\w_6}.
\item D'après l'énoncé de l'exercice et le résultat de la question 1, l'intension de la formule \(\Xlo\prd{penser}(\cns s,\Intn\exists x [\prd{républicain}(x) \wedge \prd{élu}(x)])\) est \set{\w_1;\w_7}.
\end{enumerate}
\end{Solution}
\begin{Solution}{4.{9}}
(p.~\pageref{exo:redictoPsp})\label{crg:redictoPsp}

%\sloppy
Sachant, par présupposition, que \vrb x dénote la mère d'\OE dipe dans le monde d'évaluation, la lecture \dere\ de la phrase se traduit par \(\Xlo\prd{vouloir}(\cns{\oe},\Intn\prd{épouser}(x))\) (ou éventuellement \(\Xlo\prd{mère}(x,\cns{\oe}) \wedge \prd{vouloir}(\cns{\oe},\Intn\prd{épouser}(x))\)).


Pour la lecture \dedicto, le mieux que nous pouvons proposer (en traduisant le \GN\ par une variable libre) serait \(\Xlo\prd{vouloir}(\cns{\oe},\Intn[\prd{mère}(x,\cns{\oe}) \wedge \prd{épouser}(x)])\).  Mais du fait de la présupposition, nous savons déjà que \vrb x dénote la mère réelle d'\OE dipe ; cela ne convient donc pas à l'interprétation recherchée. Pour cette lecture, nous sommes en fait obligés de suspendre la présupposition (cf. \S\ref{sss:ptépsp}, p.~\pageref{p.suspen}). C'est-à-dire que nous devons l'empêcher de se projeter (\ie\ de figurer dans le contexte général de la phrase), et de la confiner à l'intérieur de la traduction de la subordonnée (pour que les informations relatives à l'identité de la mère d'\OE dipe soient localisées dans les croyances d'\OE dipe).  Autrement dit, la traduction correcte devrait être : \(\Xlo\prd{vouloir}(\cns{\oe},\Intn\exists x [[\prd{mère}(x,\cns\oe)\wedge \forall y [\prd{mère}(y,\cns\oe)\ssi y=x]] \wedge \prd{épouser}(x)])\).  Mais ce n'est pas quelque chose qui s'obtient de façon simple en sémantique compositionnelle\footnote{C'est une question qui a directement trait au problème de la projection des présuppositions (cf. p.~\pageref{p.projpsp}) et qui est abordée, notamment, par \citet{Heim:92} et qui nous dirige vers la sémantique dynamique.}.

\fussy
\end{Solution}
\begin{Solution}{4.{10}}
(p.~\pageref{exo:homfem})\label{crg:homfem}
\begin{enumerate}
\item Il s'agit bien sûr d'une ambiguïté \alien{de re}/\alien{de dicto} sur le {\GN} \sicut{tous les hommes (qui étaient là)}.
Pour la lecture \alien{de re} la glose sera :
  \begin{enumerate}
  \item Pour chaque individu qui est un homme et qui était là, Paul a cru qu'il s'agissait d'une femme.
  \end{enumerate}
Pour la lecture \alien{de dicto} la glose sera :
  \begin{enumerate}
  \item[b.] Paul s'est dit : «~tiens, tous les hommes qui sont là sont des femmes~».
  \end{enumerate}
\item Construisons un modèle $\Modele$ par rapport auquel la phrase avec la lecture \alien{de re} sera vraie et celle avec la lecture \alien{de dicto} sera fausse.  $\Modele$ contient les données suivantes :
\begin{itemize}
\item \Obj{Paul}, un homme ;
\item \Obj{Antoine} et \Obj{Mickaël}, des hommes, amis de \Obj{Paul} ;
\item \Obj{Julie} et \Obj{Sarah}, des femmes, amies de \Obj{Paul} ;
\item \Obj{Antoine} est déguisé en Catwoman ;
\item \Obj{Mickaël} est déguisé en Batgirl ;
\item \Obj{Julie} est déguisée en Iron Man ;
\item \Obj{Sarah} est déguisée en marsupilami ;
\item \Obj{Paul} ne reconnaît aucun de ses amis ;
\item \Obj{Paul} croit que \Obj{Antoine} et \Obj{Mickaël} sont des femmes ;
\item \Obj{Paul} croit que \Obj{Julie} et \Obj{Sarah} sont des hommes ;
\end{itemize}

Avec la lecture \alien{de dicto} de \sicut{tous les hommes}, la phrase est fausse, car les individus qui sont des hommes dans les croyances de Paul sont Julie et Sarah et Paul pense que ces deux individus sont des hommes et non pas des femmes.  Mais avec la lecture \alien{de re} de \sicut{tous les hommes}, la phrase est vraie, car les individus qui sont des hommes dans $\Modele$ sont Antoine et Mickaël et Paul croit bien qu'Antoine et Mickaël sont des femmes\footnote{NB : dans cette démonstration, on exclut Paul de l'ensemble des hommes quand on fait la quantification de \sicut{tous les hommes}. C'est un problème annexe ; en fait il s'agit d'un phénomène courant lorsqu'une phrase contient un quantificateur universel et un GN référentiel, ce dernier se retrouve exclu de la quantification. Cf.\ par exemple \sicut{dans la classe, Jean est plus grand que tout le monde} ; techniquement cette phrase devrait toujours être fausse car Jean fait partie de \sicut{tout le monde} et Jean n'est pas plus grand que lui-même, mais par pragmatique on interprète la phrase en partitionnant l'ensemble des individus en faisant en sorte que \sicut{tout le monde} signifie «tout individu sauf Jean».}.

\item
  \begin{enumerate}
  \item \alien{De re} :\\
\(\Xlo\forall x [\prd{homme}(x) \implq \prd{croire}(\cns p,\Intn\prd{femme}(x))]\)
  \item \alien{De dicto} :\\
\(\Xlo\prd{croire}(\cns p,\Intn\forall x [\prd{homme}(x) \implq \prd{femme}(x)])\)
  \end{enumerate}
\item Comme l'indique la glose ci-dessus, la lecture \alien{de dicto} est la moins naturelle des deux car elle attribue à Paul la pensée (i.e. la proposition) qui dit que tous les hommes sont des femmes. Mais dans quels mondes possibles la phrase \sicut{tous les hommes sont des femmes} est vraie ? Probablement aucun (si on ne change pas le sens des mots), et cette phrase est absurde (au sens logique, c'est-à-dire que c'est une contradiction, une phrase qui n'est jamais vraie).  Et il est fort probable que Paul n'a pas ce genre de pensée : cela voudrait dire qu'il considère que le monde dans lequel il se trouve appartient à l'ensemble vide...
\item L'interprétation la plus naturelle est celle avec la lecture \alien{de re} du GN \sicut{tous les hommes}.  Mais, comme le montre la formule (3a), cela veut dire que ce GN est interprété avec une portée large, \emph{en dehors de la proposition syntaxique où il apparaît}. C'est un contre-exemple à l'observation que nous avions faite en \S\ref{sss:limiteportée}, p.~\pageref{pt:Portee2}, et qui disait que les GN quantificationnels forts (par ex. universels) ne peuvent pas «~traverser~»\ une frontière de proposition.
\end{enumerate}
\end{Solution}
\protect \newpage 
\protect \section {Chapitre \protect \ref {ch:types}}
\begin{Solution}{5.{1}}
 (p.~\pageref{exo:fxregarder})\label{crg:fxregarder}

\sloppy

La figure \ref{f:regardfexo} indique graphiquement les trajets à suivre dans la fonction \(\denote{\prd{regarder}}^{\Modele,\w_1,g}\) pour chaque calcul, avec la légende suivante :
\begin{enumerate}
\item \psline[linestyle=dotted,dotsep=2pt]{->}(0,.5ex)(1.2,.5ex)\hspace{12mm}
donc \(\denote{\Xlo\prd{regarder}(\cns a,\cns c)}^{\Modele,\w_1,g}=0\)
\item \psline[linestyle=dashed,dash=6pt 2pt]{->}(0,.5ex)(1.2,.5ex)\hspace{12mm}
donc \(\denote{\Xlo\prd{regarder}(\cns b,\cns d)}^{\Modele,\w_1,g}=1\)
\item \psline[linestyle=dotted,linewidth=1.4pt]{->}(0,.5ex)(1.2,.5ex)\hspace{12mm}
donc \(\denote{\Xlo\prd{regarder}(\cns d,\cns a)}^{\Modele,\w_1,g}=1\)
\item \psline[linestyle=dashed,dash=3pt 5pt]{->}(0,.5ex)(1.2,.5ex)\hspace{12mm}
donc \(\denote{\Xlo\prd{regarder}(\cns b,\cns b)}^{\Modele,\w_1,g}=0\)
\end{enumerate}
\newcommand{\fxregardeBX}
{\left[
\begin{array}{l}
\begin{array}{@{}l}
\Obj{Alice}\rnode{a}{\stx}\\
\Obj{Bruno}\rnode{b}{\stx}\\
\Obj{Charles}\rnode{c}{\stx}\\
\Obj{Dina}\rnode{d}{\stx}\\
%\dots\\
\end{array}%
\rule{2.5cm}{0pt}
\begin{array}{l@{}}
\rnode{1}{1}\\[2ex]\rnode{0}{0}
\end{array}
\end{array}\right]
\ncline[nodesep=3pt]{->}{a}{1}
\ncline[nodesep=3pt,offsetB=2pt,linestyle=dashed,dash=3pt 5pt]{->}{b}{0}
\ncline[nodesep=3pt]{->}{c}{0}
\ncline[nodesep=3pt,offsetB=-2pt]{->}{d}{0}%
}
\newcommand{\fxregardeAX}
{\left[
\begin{array}{l}
\begin{array}{@{}l}
\Obj{Alice}\rnode{a}{\stx}\\
\Obj{Bruno}\rnode{b}{\stx}\\
\Obj{Charles}\rnode{c}{\stx}\\
\Obj{Dina}\rnode{d}{\stx}\\
%\dots\\
\end{array}\rule{2.5cm}{0pt}
\begin{array}{l@{}}
\rnode{1}{1}\\[2ex]\rnode{0}{0}
\end{array}
\end{array}\right]
\ncline[nodesep=3pt,offsetB=1pt]{->}{a}{0}
\ncline[nodesep=3pt,offsetB=-1pt]{->}{b}{0}
\ncline[nodesep=3pt,offsetB=1pt]{->}{c}{1}
\ncline[nodesep=3pt,offsetB=-1pt,linestyle=dotted,linewidth=1.4pt]{->}{d}{1}%
}
\newcommand{\fxregardeCX}
{\left[
\begin{array}{l}
\begin{array}{@{}l}
\Obj{Alice}\rnode{a}{\stx}\\
\Obj{Bruno}\rnode{b}{\stx}\\
\Obj{Charles}\rnode{c}{\stx}\\
\Obj{Dina}\rnode{d}{\stx}\\
%\dots\\
\end{array}\rule{2.5cm}{0pt}
\begin{array}{l@{}}
\rnode{1}{1}\\[2ex]\rnode{0}{0}
\end{array}
\end{array}\right]
\ncline[nodesep=3pt,offsetB=4pt,linestyle=dotted,dotsep=2pt]{->}{a}{0}
\ncline[nodesep=3pt,offsetB=2pt]{->}{b}{0}
\ncline[nodesep=3pt]{->}{c}{0}
\ncline[nodesep=3pt,offsetB=-2pt]{->}{d}{0}%
}
\newcommand{\fxregardeDX}
{\left[
\begin{array}{l}
\begin{array}{@{}l}
\Obj{Alice}\rnode{a}{\stx}\\
\Obj{Bruno}\rnode{b}{\stx}\\
\Obj{Charles}\rnode{c}{\stx}\\
\Obj{Dina}\rnode{d}{\stx}\\
%\dots\\
\end{array}\rule{2.5cm}{0pt}
\begin{array}{l@{}}
\rnode{1}{1}\\[2ex]\rnode{0}{0}
\end{array}
\end{array}\right]
\ncline[nodesep=3pt,offsetB=2pt]{->}{a}{0}
\ncline[nodesep=3pt,linestyle=dashed,dash=6pt 2pt]{->}{b}{1}
\ncline[nodesep=3pt]{->}{c}{0}
\ncline[nodesep=3pt,offsetB=-2pt]{->}{d}{0}%
}
\begin{figure}[h]
\begin{bigcenter}
\scalebox{.9}%
{
\(\denote{\prd{regarder}}^{\Modele,\w_1,g} =
\left[
\begin{array}{l}
\\[2ex]
\Obj{Alice}\rnode{a1}{\stx}\\[2ex]
\Obj{Bruno}\rnode{b1}{\stx}\\[2ex]
\Obj{Charles}\rnode{c1}{\stx}\\[2ex]
\Obj{Dina}\rnode{d1}{\stx}\\[2ex]
%\dots\\
\end{array}\rule{2.5cm}{0pt}
%
\begin{array}{l@{\;}}
\rnode{q1}{\stx}%
\fxregardeAX
\\
\rnode{q2}{\stx}%
\fxregardeBX
\\
\rnode{q3}{\stx}%
\fxregardeCX
\\
\rnode{q4}{\stx}%
\fxregardeDX %\left[\begin{array}{l}\dots\end{array}\right]
\end{array}
\right]
\ncline[nodesep=3pt,linestyle=dotted,linewidth=1.4pt]{->}{a1}{q1}
\ncline[nodesep=3pt,linestyle=dashed,dash=3pt 5pt]{->}{b1}{q2}
\ncline[nodesep=3pt,linestyle=dotted,dotsep=2pt]{->}{c1}{q3}
\ncline[nodesep=3pt,linestyle=dashed,dash=6pt 2pt]{->}{d1}{q4}\)
}
\end{bigcenter}
\caption{Récupération de la dénotation des formules de 1--4}\label{f:regardfexo}
\end{figure}

N'oubliez pas qu'on commence toujours le trajet par la dénotation du \emph{dernier} argument de la liste accolée au prédicat (\ie\ ici le second argument).
\fussy
\end{Solution}
\begin{Solution}{5.{2}}
\label{lcombin[]solu}
(p.~\pageref{lcombin[]})\largerpage

Pour corriger \(\Xlo\lambda x\lambda y \alpha (\beta)(\gamma)\) nous
devons faire des hypothèses successives choisissant quelle fonction s'applique
aux arguments $\Xlo\beta$ et $\Xlo\gamma$, et ensuite essayer de réaliser ces hypothèses via la règle de l'application fonctionnelle (définition \ref{def:f@}, p.~\pageref{def:f@}), qui ajoutera des crochets aux endroits appropriés.
\begin{enumerate}
\item $\Xlo\beta$ est appliqué à la fonction $\Xlo\alpha$.
Nous devrons alors écrire
\(\Xlo\lambda x\lambda y [\alpha (\beta)](\gamma)\).
\\
Mais il reste encore
à donner $\Xlo\gamma$ à une fonction :
  \begin{enumerate}
    \item $\Xlo\gamma$ est appliqué à la fonction $\Xlo[\alpha (\beta)]$ (car
    cela peut encore être une fonction, si $\Xlo\alpha$ est une fonction à
    plusieurs arguments -- au moins deux).  Nous écrirons alors :
\(\Xlo\lambda x\lambda y [[\alpha (\beta)](\gamma)]\).
    \item $\Xlo\gamma$ est appliqué à la fonction $\Xlo\lambda y [\alpha
    (\beta)]$ (c'est évidemment une fonction, à cause de $\Xlo\lambda
    y$).  Nous écrirons :
    \(\Xlo\lambda x[\lambda y [\alpha (\beta)](\gamma)]\).
    \item $\Xlo\gamma$ est appliqué à la fonction $\Xlo\lambda x \lambda y [\alpha(\beta)]$.  Nous écrirons :
\(\Xlo[\lambda x\lambda y [\alpha (\beta)](\gamma)]\).
  \end{enumerate}
\item $\Xlo\beta$ est appliqué à la fonction $\Xlo\lambda y \alpha$.  La
  première correction à faire est donc :
\(\Xlo\lambda x[\lambda y \alpha (\beta)](\gamma)\).
Ensuite :
  \begin{enumerate}
    \item $\Xlo\gamma$ est appliqué à la fonction $\Xlo[\lambda y \alpha
    (\beta)]$ (c'est possible si on suppose que $\Xlo\alpha$ en soit est
    une fonction à au moins deux arguments).  Nous écrirons :
\(\Xlo\lambda x[[\lambda y \alpha (\beta)](\gamma)]\).
    \item $\Xlo\gamma$ est appliqué à la fonction
$\Xlo\lambda x[\lambda y \alpha(\beta)]$.  Nous écrirons :
\(\Xlo[\lambda x[\lambda y \alpha (\beta)](\gamma)]\).
  \end{enumerate}
\item $\Xlo\beta$ est appliqué à la fonction $\Xlo\lambda x \lambda y \alpha$.
Cela donne d'abord :
\(\Xlo[\lambda x\lambda y \alpha (\beta)](\gamma)\).
Puis :
  \begin{enumerate}
    \item $\Xlo\gamma$ est appliqué à la fonction $\Xlo[\lambda x \lambda y \alpha (\beta)]$.  Nous écrirons :
\(\Xlo[[\lambda x\lambda y \alpha (\beta)](\gamma)]\).  Ici c'est la seule
    correction possible.
  \end{enumerate}
\end{enumerate}
Pour récapituler, nous obtenons six combinaisons, toutes
différentes :

\(\Xlo\begin{array}[t]{@{}l@{\qquad}l@{\qquad}l}
\lambda x\lambda y [[\alpha (\beta)](\gamma)]&
\lambda x[\lambda y [\alpha (\beta)](\gamma)]&
{}[\lambda x\lambda y [\alpha (\beta)](\gamma)]\\
\lambda x[[\lambda y \alpha (\beta)](\gamma)]&
{}[\lambda x[\lambda y \alpha (\beta)](\gamma)]&
{}[[\lambda x\lambda y \alpha (\beta)](\gamma)]
\end{array}\)
\largerpage
\end{Solution}
\begin{Solution}{5.{3}}
 (p.~\pageref{exo:types:crochets})\label{crg:types:crochets}

L'important est de vérifier que la suppression des crochets (convention \ref{c:suppr[]}, p.~\pageref{c:suppr[]}) ne crée
pas d'ambiguïté.

\sloppy
\begin{enumerate}
\item \(\Xlo[[\prd{aimer}(x)](y)]\).
%
$\prd{aimer}$ est une fonction qui reçoit l'argument $\vrb x$, et comme
  $\prd{aimer}$ est un prédicat à deux arguments
%% de  type \type{e,\type{e,t}},
$\Xlo[\prd{aimer}(x)]$ est encore une fonction qui reçoit
  l'argument $\vrb y$.  Nous pouvons donc simplifier en $\Xlo\prd{aimer}(x)(y)$ puis en $\Xlo\prd{aimer}(y,x)$.

\item \(\Xlo\lambda x [[\prd{aimer}(x)](y)]\).
%
Pour les mêmes raisons, cela se simplifie en
$\Xlo\lambda x\,\prd{aimer}(y,x)$.

\item \(\Xlo[\lambda x [\prd{aimer}(x)](y)]\).  $\prd{aimer}$ est une
  fonction qui reçoit l'argument $\vrb x$.  Nous pouvons donc déjà
  simplifier en \(\Xlo[\lambda x\, \prd{aimer}(x)(y)]\).  Notons que
  $\Xlo\prd{aimer}(x)$ est encore une fonction (elle n'attend plus que
  son <<~second~>> argument).  Mais maintenant il faut faire
  attention: car \(\Xlo\lambda x\, \prd{aimer}(x)\) est aussi une
  fonction, et elle est différente de $\Xlo\prd{aimer}(x)$.  Si nous
  supprimons les crochets extérieurs, nous risquons de ne plus savoir à
  laquelle de ces deux fonctions $\vrb y$ est appliqué.  Dans
  l'exemple, $\vrb y$ est appliqué à \(\Xlo\lambda x\,
  \prd{aimer}(x)\), nous le savons grâce aux crochets extérieurs.
  Remarque: si $\vrb y$ était appliqué à $\Xlo\prd{aimer}(x)$, nous
  devrions écrire: $\Xlo\lambda x [\prd{aimer}(x)(y)]$ (ou par
  simplification $\Xlo\lambda x\,\prd{aimer}(y,x)$).

%% Pour illustrer, remplaçons $\prd{aimer}$ par la constante \prd{regarder}.
%% $\prd{regarder}(x)$ représente <<~regarde $x$~>>, ou <<~le$_{x}$
%% regarde~>>.  C'est une fonction car ça attend encore un sujet.  En
%% revanche $\lambda x\, \prd{regarder}(x)$ représente <<~regarde~>>,
%% c'est une fonction à deux arguments.
%% \prd{aimer}uant à \([\lambda x\,\prd{regarder}(x)(y)]\) sera $\beta$-réduit en
%% \(\prd{regarder}(y)\); ça représente donc <<~le$_{y}$
%% regarde~>>. Enfin $\lambda x\,\prd{regarder}(y,x)$  représente
%% <<~il$_{y}$ regarde~>>, qui est une fonction car ça attend un GN objet.

\item \(\Xlo\peut[[\prd{aimer}(x)](y)]\).  Se simplifie en
$\Xlo\peut\prd{aimer}(y,x)$.

\item \(\Xlo[\Intn[\prd{aimer}(x)](y)]\). Ne peut se simplifier qu'en
  \(\Xlo\Intn[\prd{aimer}(x)](y)\).  En effet \(\Xlo\Intn
  \prd{aimer}(x)(y)\) serait ambigu: nous ne saurions pas si $\Xlo\Intn$
  porte sur $\prd{aimer}$ ou $\Xlo\prd{aimer}(x)$ ou
  $\Xlo\prd{aimer}(x)(y)$.  Ici nous savons qu'il porte sur
  $\Xlo\prd{aimer}(x)$.

\item \(\Xlo\Intn[[\prd{aimer}(x)](y)]\).  Se simplifie en
  $\Xlo\Intn[\prd{aimer}(y,x)]$.

\item \(\Xlo[\lambda x [[\prd{aimer}(y)](x)](z)]\). Nous savons que nous pouvons
  simplifier $\Xlo[[\prd{aimer}(y)](x)]$ en $\Xlo\prd{aimer}(y,x)$.  Nous avons donc
  \(\Xlo[\lambda x\, \prd{aimer}(x,y) (z)]\).  Il vaut mieux garder les
  crochets extérieurs pour bien indiquer que $\vrb z$ est appliqué à
  $\Xlo\lambda x\, \prd{aimer}(x,y)$.

\item \(\Xlo\lambda x [\lambda y [\prd{aimer}(y)](x)]\).  Se simplifie
  uniquement en
  \(\Xlo\lambda x [\lambda y\,\prd{aimer}(y)(x)]\) (à la rigueur).
%Remarque: par   $\beta$-réduction, on obtiendra $\lambda x\, \prd{aimer}(x)$.
\end{enumerate}

\fussy
\end{Solution}
\begin{Solution}{5.{4}}
(p.~\pageref{exo:types:varcrochets})\label{soluvarlambdacrochet}
\begin{enumerate}
\item Variante de notation :
\addtolength{\multicolsep}{-10pt}
\begin{multicols}{2}
\begin{enumerate}[label=\arabic*]
\item \(\Xlo\prd{aimer}(x)(y)\)
\item \(\Xlo\lambda x [\prd{aimer}(x)(y)]\)
\item \(\Xlo\lambda x [\prd{aimer}(x)](y)\)
\item \(\Xlo\peut\prd{aimer}(x)(y)\)
\item \(\Xlo\Intn\prd{aimer}(x)(y)\)\label{exo:var:[af]:i5}
\item \(\Xlo\Intn\prd{aimer}(x)(y)\)\label{exo:var:[af]:i6}
\item \(\Xlo\lambda x [\prd{aimer}(y)(x)](z)\)
\item \(\Xlo\lambda x [\lambda y [\prd{aimer}](y)(x)]\)
\end{enumerate}
\end{multicols}
\item Le problème qui se pose est que les expressions
  {5} et {6} deviennent identiques
  alors que dans l'exercice~\ref{exo:types:crochets} nous avons vu
  qu'elles ne se simplifiaient pas de la même manière. Et en fait
  elles n'ont pas le même sens.  Pour éviter que se pose ce problème,
  il faut donc que dans cette variante de notation, la règle
  d'introduction de l'opérateur $\Xlo\Intn$  (p.~\pageref{p:^}) soit
  révisée en: si $\Xlo\alpha$ est une expression de {\LO}, alors
  $\Xlo\Intn[\alpha]$ aussi.
\end{enumerate}
\end{Solution}
\begin{Solution}{5.{5}}
\label{sol:Ilabs@}
(p.~\pageref{exo:Ilabs@})

%%Pour calculer les valeurs sémantiques de  \lterme s, il y a deux
%%façons de procéder,  selon que l'on part du terme global pour le
%%décomposer ou que l'on ***

\begin{enumerate}
\item \(\denote{\Xlo\lambda x\lambda y[[\prd{regarder}(y)](x)]}^{\Modele,\w_1,g}\)
%%  \begin{enumerate}
%%  \item
\end{enumerate}
\sloppy

D'après la définition~\ref{d:slabs}, p.~\pageref{d:slabs},
\(\denote{\Xlo\lambda x\lambda y[[\prd{regarder}(y)](x)]}^{\Modele,\w_1,g}\)
est la fonction qui, à tout objet \Obj x, associe
\(\denote{\Xlo\lambda y[[\prd{regarder}(y)](x)]}^{\Modele,\w_1,g_{[\Obj
    x/\vrb x]}}\).
De même, \(\denote{\Xlo\lambda y[[\prd{regarder}(y)](x)]}^{\Modele,\w_1,g_{[\Obj
    x/\vrb x]}}\) est la fonction qui, à tout objet \Obj y, associe
\(\denote{\Xlo[[\prd{regarder}(y)](x)]}^{\Modele,\w_1,g_{[\Obj
    x/\vrb x]_{[\Obj y/\vrb y]}}}\).

Ensuite, d'après la définition~\ref{d:Sem@}, p.~\pageref{d:Sem@},
\(\denote{\Xlo[[\prd{regarder}(y)](x)]}^{\Modele,\w_1,g_{[\Obj
      x/\vrb x]_{[\Obj y/\vrb y]}}}
= %\linebreak
\denote{\Xlo[\prd{regarder}(y)]}^{\Modele,\w_1,g_{[\Obj
      x/\vrb x]_{[\Obj y/\vrb y]}}}(\denote{\Xlo x}^{\Modele,\w_1,g_{[\Obj
      x/\vrb x]_{[\Obj y/\vrb y]}}})
\).
Et comme l'assignation courante spécifie la valeur de \vrb x, nous savons que \(\denote{\Xlo x}^{\Modele,\w_1,g_{[\Obj
      x/\vrb x]_{[\Obj y/\vrb y]}}} = g_{[\Obj
      x/\vrb x]_{[\Obj y/\vrb y]}}(\vrb x) =\Obj x\).
Donc
\(\denote{\Xlo[[\prd{regarder}(y)](x)]}^{\Modele,\w_1,g_{[\Obj
      x/\vrb x]_{[\Obj y/\vrb y]}}}
= \denote{\Xlo[\prd{regarder}(y)]}^{\Modele,\w_1,g_{[\Obj
      x/\vrb x]_{[\Obj y/\vrb y]}}}(\Obj x)\).

Toujours d'après la définition~\ref{d:Sem@}, nous avons
\(\denote{\Xlo[\prd{regarder}(y)]}^{\Modele,\w_1,g_{[\Obj
      x/\vrb x]_{[\Obj y/\vrb y]}}} =
\denote{\Xlo\prd{regarder}}^{\Modele,\w_1,g_{[\Obj
      x/\vrb x]_{[\Obj y/\vrb y]}}}
(\denote{\Xlo y}^{\Modele,\w_1,g_{[\Obj
      x/\vrb x]_{[\Obj y/\vrb y]}}})
=
\denote{\Xlo\prd{regarder}}^{\Modele,\w_1,g_{[\Obj
      x/\vrb x]_{[\Obj y/\vrb y]}}}
(\Obj y)
\).
Et le fait est que
\(\denote{\Xlo\prd{regarder}}^{\Modele,\w_1,g_{[\Obj
      x/\vrb x]_{[\Obj y/\vrb y]}}}\) est la même chose que
\(\denote{\Xlo\prd{regarder}}^{\Modele,\w_1,g}\) (car il n'y a plus de
variable ici), c'est-à-dire c'est la
fonction
décrite dans la
figure~\ref{f:regardf}, p.~\pageref{f:regardf}.
Par conséquent,
\(\denote{\Xlo[[\prd{regarder}(y)](x)]}^{\Modele,\w_1,g_{[\Obj
      x/\vrb x]_{[\Obj y/\vrb y]}}} =
\denote{\Xlo\prd{regarder}}^{\Modele,\w_1,g}(\Obj y)(\Obj x)
\), c'est-à-dire la valeur que cette fonction assigne quand on lui
donne \Obj y comme premier argument et \Obj x comme second
argument.  Par exemple, si \Obj x est \Obj{Alice} et \Obj y est
\Obj{Dina}, la valeur sera $0$, etc.

Pour conclure : en reprenant le calcul depuis le début,
\(\denote{\Xlo\lambda x\lambda y[[\prd{regarder}(y)](x)]}^{\Modele,\w_1,g}\)
est la fonction qui à tout objet \Obj x associe la fonction qui à tout
objet \Obj y associe la valeur
\(\denote{\Xlo\prd{regarder}}^{\Modele,\w_1,g}(\Obj y)(\Obj x)\).
Mais attention, cela n'est pas la même fonction que
\(\denote{\Xlo\prd{regarder}}^{\Modele,\w_1,g}\), car les arguments ne
sont pas attendus dans le même ordre\footnote{Rappelons que
  \(\denote{\Xlo\prd{regarder}}^{\Modele,\w_1,g}\)
équivaut à \(\denote{\Xlo\lambda y\lambda
  x[[\prd{regarder}(y)](x)]}^{\Modele,\w_1,g}\), la fonction  qui à
tout objet \Obj y associe la fonction qui à tout
objet \Obj x associe la valeur
\(\denote{\Xlo\prd{regarder}}^{\Modele,\w_1,g}(\Obj y)(\Obj x)\).}.
%%  \end{enumerate}

%\fussy

\begin{enumerate}[resume]
\item
\(\denote{\Xlo\lambda x[\lambda y[\prd{regarder}(y)](x)]}^{\Modele,\w_1,g}\)
\end{enumerate}

D'après la définition~\ref{d:slabs},
\(\denote{\Xlo\lambda x[\lambda y[\prd{regarder}(y)](x)]}^{\Modele,\w_1,g}\)
est la fonction qui, à tout objet \Obj x, associe
\(\denote{\Xlo[\lambda y[\prd{regarder}(y)](x)]}^{\Modele,\w_1,g_{[\Obj
x/\vrb x]}}\).  Maintenant pour interpréter \(\Xlo[\lambda
  y[\prd{regarder}(y)](x)]\), nous pouvons utiliser le théorème
\ref{th:Seml@}.
Pour ce faire, nous devons connaître la dénotation de \vrb x par
rapport aux indices via lesquels nous évaluons l'application fonctionnelle  (en
l'occurrence $\w_1$ et $g_{[\Obj x/\vrb x]}$) ;
et nous la connaissons : \(\denote{\Xlo x}^{\Modele,\w_1,g_{[\Obj
x/\vrb x]}} = \Obj x\).
Partant, le théorème nous dit que
\(\denote{\Xlo[\lambda y[\prd{regarder}(y)](x)]}^{\Modele,\w_1,g_{[\Obj
x/\vrb x]}}
=
\denote{\Xlo[\prd{regarder}(y)]}^{\Modele,\w_1,g_{[\Obj
x/\vrb x]_{[\Obj x/\vrb y]}}}
\).

\sloppy
Ensuite, la définition~\ref{d:Sem@} nous indique que
\(\denote{\Xlo[\prd{regarder}(y)]}^{\Modele,\w_1,g_{[\Obj
x/\vrb x]_{[\Obj x/\vrb y]}}}
=
\denote{\Xlo\prd{regarder}}^{\Modele,\w_1,g_{[\Obj
x/\vrb x]_{[\Obj x/\vrb y]}}}
(\denote{\Xlo y}^{\Modele,\w_1,g_{[\Obj
x/\vrb x]_{[\Obj x/\vrb y]}}})
\).
Et nous avons \(\denote{\Xlo y}^{\Modele,\w_1,g_{[\Obj
x/\vrb x]_{[\Obj x/\vrb y]}}} = \Obj x\) (à cause de l'assignation $g_{[\Obj
    x/\vrb x]_{[\Obj x/\vrb y]}}$ qui fixe la valeur de \vrb y à \Obj x).
Par conséquent,
\(\denote{\Xlo[\lambda y[\prd{regarder}(y)](x)]}^{\Modele,\w_1,g_{[\Obj
x/\vrb x]}}
=
\denote{\Xlo\prd{regarder}}^{\Modele,\w_1,g}(\Obj x)\).

Conclusion : \(\denote{\Xlo\lambda x[\lambda y[\prd{regarder}(y)](x)]}^{\Modele,\w_1,g}\)
est la fonction qui à tout objet \Obj x associe
\(\denote{\Xlo\prd{regarder}}^{\Modele,\w_1,g}(\Obj x)\).
C'est globalement une fonction à un seul argument et qui renvoie comme
résultat une fonction elle aussi à un argument.

\fussy
\end{Solution}
\begin{Solution}{5.{6}}
 (p. \pageref{exo:betared3})

Les \breduc s (cf. définition \ref{d:breduc1}, p.~\pageref{d:breduc1}) sont présentées ci-dessous  sous forme d'égalités.
\begin{enumerate}
\item \(\Xlo[[[\lambda z\lambda y\lambda x\,\prd{présenter}(x,y,z)(\cns a)](\cns b)](\cns c)]\)
\\
= \(\Xlo[[\lambda y\lambda x\,\prd{présenter}(x,y,\cns a)(\cns b)](\cns c)]\)
\hfill{\small (\breduc\ sur \vrb z)}
\\
= \(\Xlo[\lambda x\,\prd{présenter}(x,\cns b,\cns a)(\cns c)]\)
\hfill{\small (\breduc\ sur \vrb y)}
\\
= \(\Xlo\prd{présenter}(\cns c,\cns b,\cns a)\)
\hfill{\small (\breduc\ sur \vrb x)}


\item \(\Xlo[[\lambda z[\lambda y\lambda x\,\prd{présenter}(x,y,z)(\cns a)](\cns b)](\cns c)]\)\\
= \(\Xlo[[\lambda z\lambda x\,\prd{présenter}(x,\cns a,z)(\cns b)](\cns c)]\)
\hfill{\small (\breduc\ sur \vrb y)}
\\
= \(\Xlo[\lambda x\,\prd{présenter}(x,\cns a,\cns b)(\cns c)]\)
\hfill{\small (\breduc\ sur \vrb z)}
\\
= \(\Xlo\prd{présenter}(\cns c,\cns a,\cns b)\)
\hfill{\small (\breduc\ sur \vrb x)}

Ici on aurait pu aussi commencer par la \breduc\ sur \vrb z.

\item \(\Xlo[\lambda z[[\lambda y\lambda x\,\prd{présenter}(x,y,z)(\cns a)](\cns b)](\cns c)]\)\\
= \(\Xlo[\lambda z[\lambda x\,\prd{présenter}(x,\cns a,z)(\cns b)](\cns c)]\)
\hfill{\small (\breduc\ sur \vrb y)}
\\
= \(\Xlo[\lambda z\,\prd{présenter}(\cns b,\cns a,z)(\cns c)]\)
\hfill{\small (\breduc\ sur \vrb x)}
\\
= \(\Xlo\prd{présenter}(\cns b,\cns a,\cns c)\)
\hfill{\small (\breduc\ sur \vrb z)}

Là aussi plusieurs ordres sont possibles pour effectuer les \breduc s, par exemple : sur \vrb y, puis \vrb z, puis \vrb x, ou sur \vrb z, puis \vrb y, puis \vrb x.

\item \(\Xlo[\lambda z[\lambda y[\lambda x\,\prd{présenter}(x,y,z)(\cns a)](\cns b)](\cns c)]\)\\
= \(\Xlo[\lambda z[\lambda y\,\prd{présenter}(\cns a,y,z)(\cns b)](\cns c)]\)
\hfill{\small (\breduc\ sur \vrb x)}
\\
= \(\Xlo[\lambda z\,\prd{présenter}(\cns a,\cns b,z)(\cns c)]\)
\hfill{\small (\breduc\ sur \vrb y)}
\\
= \(\Xlo\prd{présenter}(\cns a,\cns b,\cns c)\)
\hfill{\small (\breduc\ sur \vrb z)}

Ici on peut effectuer les trois \breduc s dans l'ordre qu'on veut.

\end{enumerate}
\end{Solution}
\begin{Solution}{5.{8}}
 (p. \pageref{exo:betared1})


%$\beta$-réductions.
\begin{enumerate}
\item \(\Xlo[\lambda x\,\prd{aimer}(x,y)(y)] \beq \prd{aimer}(y,y)\)
%
\item \(\Xlo[\lambda x\,\prd{aimer}(x,x)(\cns a)] \beq \prd{aimer}(\cns
  a,\cns a)\)
%
\item \(\Xlo[\lambda y \lambda x\,\prd{aimer}(x,y)(\cns c)] \beq
\lambda x\,\prd{aimer}(x,\cns c)\)
%
\item \(\Xlo\lambda y [\lambda x\,\prd{aimer}(x,y)(\cns c)] \beq
\lambda y\, \prd{aimer}(\cns c,y)\)
%
\item \(\Xlo[[\lambda y \lambda x\,\prd{aimer}(x,y)(\cns a)](\cns b)] \beq
[\lambda x\,\prd{aimer}(x,\cns a)(\cns b)] \beq \prd{aimer}(\cns b,\cns a)\)
%
\item \(\Xlo[\lambda y [\lambda x\,\prd{aimer}(x,y)(\cns a)](\cns b)] \beq
[\lambda x\,\prd{aimer}(x,\cns b)(\cns a)] \beq
\prd{aimer}(\cns a,\cns b)\)
\\
Ici on peut aussi procéder en commençant par $\beta$-réduire sur
  $\Xlo\lambda x$ :\\
\(\Xlo[\lambda y [\lambda x\,\prd{aimer}(x,y)(\cns a)](\cns b)] \beq
[\lambda y \,\prd{aimer}(\cns a,y)(\cns b)] \beq
\prd{aimer}(\cns a,\cns b)\)
%
\item \(\Xlo[\lambda z \lambda y [\lambda x\,\prd{aimer}(x,y)(z)](\cns
  b)] \beq
\lambda y [\lambda x\,\prd{aimer}(x,y)(\cns b)] \beq
\lambda y \,\prd{aimer}(\cns b,y)\)
\\ Ou bien :\\
\(\Xlo[\lambda z \lambda y [\lambda x\,\prd{aimer}(x,y)(z)](\cns
  b)] \beq
[\lambda z \lambda y \,\prd{aimer}(z,y)(\cns b)]
\beq
\lambda y \,\prd{aimer}(\cns b,y)\)
%
\item \(\Xlo[\lambda P\, P(x,y)(\prd{aimer})] \beq
\prd{aimer}(x,y)\)
%
\item \(\Xlo[\lambda Q \lambda y \lambda x [Q(x,y) \wedge
    Q(y,x)](\prd{aimer})] \beq
\lambda y \lambda x [\prd{aimer}(x,y) \wedge \prd{aimer}(y,x)]\)
%
\item \(\Xlo[\lambda P [P(\cns a)](\lambda x\, \prd{pleurer}(x))] \beq
[\lambda x\,\prd{pleurer}(x)(\cns a)] \beq
\prd{pleurer}(\cns a) \)
%
\item \(\Xlo[[\lambda X \lambda x [X(x)](\prd{chat})](\cns d)] \beq
[\lambda x [\prd{chat}(x)](\cns d)] \beq
[\prd{chat}(\cns d)]\)
%
\item
\(\Xlo[\lambda X \lambda z [[X(z)](z)](\lambda y \lambda x\,
  Q(x,y))] \beq
\lambda z [[\lambda y \lambda x\,Q(x,y)(z)](z)]\)
\\\(\beq\Xlo
\lambda z [\lambda x\,Q(x,z)(z)]
\beq
\lambda z \,Q(z,z)
\)
\end{enumerate}
\end{Solution}
\begin{Solution}{5.{9}}
 (p. \pageref{exo:betared})

%$\beta$-réductions.
\begin{enumerate}\raggedright
\item \(\Xlo[[\lambda y \lambda x\, \prd{aimer}(x,y)(\cns{j})](\cns{a})]
\beq [\lambda x\, \prd{aimer}(x,\cns{j})(\cns{a})]
\beq
\prd{aimer}(\cns{a},\cns{j})\)

\item \(\Xlo[\lambda y [\lambda x\, \prd{aimer}(x,y)(\cns{j})](\cns{a})]
\beq
[\lambda y\, \prd{aimer}(\cns{j},y) (\cns{a})]
\beq
\prd{aimer}(\cns{j},\cns{a})\)


\item \(\Xlo[\lambda Q \lambda y [Q(y)](\lambda x\, \prd{dormir}(x))]
\beq
\lambda y [\lambda x\, \prd{dormir}(x)(y)]
\beq
\lambda y\, \prd{dormir}(y)\)


\item \(\Xlo[[\lambda P \lambda x\, [[P(x)](y)](\prd{aimer})](\cns{a})]
\beq
[\lambda x\, [[\prd{aimer}(x)](y)](\cns{a})]
\beq
[[\prd{aimer}(\cns{a})](y)]
=
\prd{aimer}(y,\cns{a})\) (par suppression des crochets)


%\sloppy
\item \(\Xlo[\lambda P \lambda x\, [[P(x)](y)](\lambda y \lambda z\,
  \prd{aimer}(y,z))]
\beq
\lambda x\, [[\lambda y \lambda z\, \prd{aimer}(y,z)(x)](y)]
\beq
\lambda x\, [\lambda z\, \prd{aimer}(x,z)(y)]
\beq
\lambda x\, \prd{aimer}(x,y)\)

\item \(\Xlo[\lambda X \lambda x [X(\lambda y\, \prd{aimer}(x,y))]
(\lambda Q\, [Q(\cns{a})])]
\beq
\lambda x [\lambda Q\, [Q(\cns{a})](\lambda y\, \prd{aimer}(x,y))]
\beq
\lambda x [\lambda y\, \prd{aimer}(x,y)(\cns{a})]
\beq
\lambda x \,\prd{aimer}(x,\cns{a})
\)

\fussy

\end{enumerate}
\end{Solution}
\begin{Solution}{5.{10}}
 (p. \pageref{exo:types:ebf})

Rappel des types :\\
$\vrb x, \vrb y \in \VAR_{\mathrm{e}}$, $\vrb \phi \in \VAR_{\mathrm{t}}$, $\vrb P \in
\VAR_{\type{e,t}}$, $\vrb Q \in \VAR_{\type{e,\type{e,t}}}$, $\vrb K \in
\VAR_{\type{\type{s,t},\type{e,t}}}$, $\vrb S \in \VAR_{\type{s,\type{e,t}}}$.\\
Les règles syntaxiques utilisées ci-dessous sont celles de la définition \ref{RSyn:ltype}, p.~\pageref{RSyn:ltype}.

\sloppy
\begin{enumerate}
\item  \(\Xlo\lambda x\, Q(x,y)\)\\
$\Xlo Q(x,y)$ est de type \typ{t} car $\vrb Q \in \VAR_{\type{e,\type{e,t}}}$ et
  a ses deux arguments de type \typ{e} ; donc \(\Xlo\lambda x\, Q(x,y)\)
  est de type \type{e,t} (\RSyn\ref{SynTlamb}).

\item \(\Xlo\lambda Q \exists x\, Q(x,y)\)\\
$\Xlo Q(x,y)$ est de type \typ{t}, donc $\Xlo\exists x\, Q(x,y)$ aussi
  (\RSyn\ref{SynTQ}) et   \(\Xlo\lambda Q \exists x\, Q(x,y)\) est de type
  \type{\type{e,\type{e,t}},t}  (\RSyn\ref{SynTlamb}).

\item \(\Xlo[\lambda x\, Q(x,y)(y)]\)\\
$\Xlo Q(x,y)$ est de type \typ{t}, donc $\Xlo\lambda x\, Q(x,y)$ est de type
  \type{e,t} (\RSyn\ref{SynTlamb}) et donc \(\Xlo[\lambda x\, Q(x,y)(y)]\)
  est de type \typ{t} (\RSyn\ref{SynTApp}).\\  {NB : on voit bien que
  les crochets sont importants pour retrouver le bon ordre de
  composition de l'expression. }

\item \(\Xlo\peut\lambda x\, P(x)\)\\
$\Xlo P(x)$ est de type \typ{t} car $\vrb P  \in \VAR_{\type{e,t}}$
  (\RSyn\ref{SynTApp}), donc $\Xlo\lambda x\, P(x)$ est de type
  \type{e,t} ; or d'après (\RSyn\ref{SynTMod}) {\Xlo\peut} ne peut
  s'accoler qu'à une expression de type \typ{t} ; donc l'expression
  n'est pas bien formée.


\item \(\Xlo\lambda P\, P\)\\
$\vrb P$ est de type \type{e,t}, donc $\Xlo\lambda P\, P$ est de type
  \type{\type{e,t},\type{e,t}}, en vertu de (\RSyn\ref{SynTlamb}).


\item \(\Xlo[\lambda P\, P(x)(y)]\)\\
Ici l'expression est ambiguë quant à la suppression d'une paire de
crochets\footnote{Ce qui devrait laisser deviner d'emblée qu'elle n'est pas
bien formée, puisqu'on n'a pas le droit de supprimer des crochets
en rendant une expression ambiguë.} : on peut supposer qu'elle est la
simplification soit de \(\Xlo[\lambda P [P(x)](y)]\), soit de \(\Xlo[[\lambda
  P\, P(x)](y)]\) ; dans les deux cas elle est mal formée.  Dans le
premier cas, $\Xlo P(x)$ est de type \typ{t}, $\Xlo\lambda P [P(x)]$ est de type
\type{\type{e,t},t} et ne peut donc pas se combiner avec $\vrb y$ de type
\typ{e}.  Dans le second cas, $\Xlo\lambda P\, P$ est de type
\type{\type{e,t},\type{e,t}} et ne peut donc pas se combiner avec $\vrb x$.

\item \(\Xlo\lambda x [\Extn S(x)]\)\\
$\Xlo\Extn S$ est de type \type{e,t} (\RSyn\ref{SynTExt}) car $\vrb S$ est de
  type \type{s,\type{e,t}} ; donc $\Xlo[\Extn S(x)]$ est de type \typ{t} et
  \(\Xlo\lambda x [\Extn S(x)]\) de type \type{e,t}.

\item \(\Xlo\lambda x\, \Extn [S(x)]\)\\
Cette expression est mal formée, car $\vrb S$ ne peut pas se combiner
directement avec $\vrb x$.

\item \(\Xlo[\lambda x\, \Extn S(x)]\)\\
$\Xlo\Extn S$ est de type \type{e,t}, donc $\Xlo\lambda x\, \Extn S$ est de
  type \type{e,\type{e,t}} et \(\Xlo[\lambda x\, \Extn S(x)]\) est de type
  \type{e,t}.

\item \(\Xlo K(\Intn [\lambda x\, P(x)(y)])\)\\
Quelle que soit la manière d'analyser $\Xlo[\lambda x\, P(x)(y)]$ on
trouvera le type \typ{t} pour cette sous-expression.  En effet $\Xlo P(x)$
est de type \typ t, $\Xlo\lambda x\, P(x)$  de type \type{e,t} et donc
$\Xlo[\lambda x\, P(x)(y)]$ de type \typ t ; ou bien $\Xlo\lambda x\, P$ est de
type \type{e,\type{e,t}} et $\Xlo\lambda x\, P(x)$ est de type
\type{e,t}.  Ensuite $\Xlo\Intn[\lambda x\,P(x)(y)]$ est de type
\type{s,t} (\RSyn\ref{SynTInt}), et donc \(\Xlo K(\Intn [\lambda x\,
  P(x)(y)])\) est de type \type{e,t}, car $\vrb K$ est de type
  \type{\type{s,t},\type{e,t}}.

\item \(\Xlo\lambda \phi\, K(x,\Intn \phi)\)\\
$\vrb \phi$ est de type \typ t, donc $\Xlo\Intn \phi$ est de type \type{s,t}, et
  $\Xlo K(x,\Intn \phi)$ est de type \typ t, car c'est la simplification de
  $\Xlo[[K(\Intn \phi)](x)]$ ; donc \(\Xlo\lambda \phi\, K(x,\Intn \phi)\) est de type
  \type{t,t}.


\item \(\Xlo\lambda \phi \lambda x \, K(x,\Intn \phi)\)\\
$\Xlo\lambda x\, K(x,\Intn \phi)$ est de type \type{e,t}, donc \(\Xlo\lambda \phi \lambda x \, K(x,\Intn \phi)\) est de type \type{t,\type{e,t}}.

\item \(\Xlo\lambda x \lambda \phi \, K(x,\Intn \phi)\)\\
$\Xlo\lambda \phi \, K(x,\Intn \phi)$ est de type \type{t,t}, donc \(\Xlo\lambda x   \lambda \phi \, K(x,\Intn \phi)\) est de type \type{e,\type{t,t}}.

\item \(\Xlo\lambda K \lambda \phi\, K(x,\Intn \phi)\)\\
$\Xlo\lambda \phi \, K(x,\Intn \phi)$ est de type \type{t,t}, donc \(\Xlo\lambda K
  \lambda \phi\, K(x,\Intn \phi)\) est de type
  \type{\type{\type{s,t},\type{e,t}},\type{t,t}}.

\item \(\Xlo\lambda K \lambda \phi\, \Intn [[K(\Intn \phi)](x)]\)\\
$\Xlo\Intn \phi$ est de type \type{s,t},\\ donc $\Xlo[K(\Intn \phi)]$ est de type   \type{e,t},\\ donc $\Xlo[[K(\Intn \phi)](x)]$ est de type \typ t,\\ donc $\Xlo\Intn[[K(\Intn \phi)](x)]$ est de type \type{s,t},\\ donc $\Xlo\lambda \phi\,\Intn [[K(\Intn \phi)](x)]$ est de type \type{t,\type{s,t}}, \\et \(\Xlo\lambda K \lambda \phi\, \Intn [[K(\Intn \phi)](x)]\) de type   \type{\type{\type{s,t},\type{e,t}},\type{t,\type{s,t}}}.

\largerpage[-1]

\item \(\Xlo\lambda K \lambda \phi\, [\Intn [K(\Intn \phi)](x)]\)\\
$\Xlo[K(\Intn \phi)]$ est de type \type{e,t}, donc $\Xlo\Intn [K(\Intn \phi)]$ est   de type \type{s,\type{e,t}} et cela ne peut pas se combiner avec
  $\vrb x$.  L'expression est donc mal formée.

\item \(\Xlo\lambda x [P(x) \wedge Q(x)]\)\\
$\Xlo P(x)$ est de type \typ t, $\Xlo Q(x)$ est de type \type{e,t} (car $\vrb Q$ est   de type \type{e,\type{e,t}}) ; $\Xlo\wedge$ ne peut donc pas les combiner
  car il ne coordonne que des expressions de type \typ t   (\RSyn\ref{SynTConn}). L'expression est mal formée.

\item \(\Xlo\lambda y\lambda x  [P(x) \wedge Q(x,y)](x)\)\\
$\Xlo Q(x,y)$ est de type \typ t, donc $\Xlo[P(x) \wedge Q(x,y)]$ est de type
  \typ t, $\Xlo\lambda x  [P(x) \wedge Q(x,y)]$ est de type \type{e,t},
  $\Xlo\lambda y\lambda x  [P(x) \wedge Q(x,y)]$ de type \type{e,\type{e,t}}
  et \(\Xlo\lambda y\lambda x  [P(x) \wedge Q(x,y)](x)\) de type \type{e,t}.


\item \(\Xlo[\lambda Q\, Q(x,y)(\lambda x\, P)]\)\\
$\Xlo Q(x,y)$ est de type \typ t, donc $\Xlo\lambda Q\, Q(x,y)$ est de type
  \type{\type{e,\type{e,t}},t} ; et $\Xlo\lambda x\, P$ est de type
  \type{e,\type{e,t}}, donc \(\Xlo[\lambda Q\, Q(x,y)(\lambda x\, P)]\)
  est de type \typ t.

\item \(\Xlo\lambda y\, Q(\atoi x\, P(x),y)\)\\
$\Xlo P(x)$ est de type \typ t, donc $\Xlo\atoi x\, P(x)$ est de type \typ e
  (\RSyn\ref{SynTQ}), donc $\Xlo Q(\atoi x\, P(x),y)$ est de type \typ t,
  et \(\Xlo\lambda y\, Q(\atoi x\, P(x),y)\) est de type \type{e,t}.

\end{enumerate}
\fussy
\end{Solution}
\begin{Solution}{5.{11}}
  (p. \pageref{exo:typesQA})

Dans ce qui suit, $a$ et $b$ représentent des types
a priori  «inconnus».
\begin{enumerate}
\item \(\Xlo Q(A)\)

Comme \(\Xlo Q(A)\)  est une application fonctionnelle, c'est que $\vrb Q$
dénote une fonction et donc que $\vrb Q$ est de type \mtype{a,b}. Et on
sait que $b=\typ t$ car c'est le résultat attendu par hypothèse dans
cet exercice.  Donc $\Xlo Q$ est de type \mtype{a,\typ t}.  Quand à $a$ on
sait aussi que c'est le type de $\vrb A$, puisque $\vrb A$ est un argument
licite de $\vrb Q$.  Il n'y a pas d'autre contrainte sur son type ; on peut
prendre ce qu'on veut pour $a$, par exemple \typ e.  Ainsi $\vrb Q$ est de
type \type{e,t} et $\vrb A$ de type \typ e.

Mais on peut aussi donner
\type{\type{e,t},t} pour $\vrb Q$ et \type{e,t} pour $\vrb A$ ; ou encore
\type{t,t} et \typ t, etc.

\item \(\Xlo[\Extn Q(A)]\)

Ici il faut reconnaître une application fonctionnelle de la fonction
$\Xlo\Extn Q$ sur l'argument $\vrb A$.  On en déduit plusieurs choses : d'abord
que $\vrb Q$ est de type \mtype{\typ s,b} pour qu'on ait le droit de lui
accoler $\Xlo\Extn$ par la règle (\RSyn\ref{SynTExt}) (p.~\pageref{SynTExt}). Et par cette règle on
sait que $b$ est le type de $\Xlo\Extn Q$.  Mais $b$ est en fait un type
complexe puisqu'on est présence d'une fonction.  Là on se ramène un
peu au cas précédent : $\Xlo\Extn Q$ doit être de type \mtype{a,\typ t}
(donc $b=\mtype{a,\typ t}$) et $a$ est le type de $\vrb A$.  Si on prend
$e$ pour le type de $\vrb A$, alors $b =\type{e,t}$, alors le type de $\vrb Q$
est \type{s,\type{e,t}} (puisqu'on a posé que ce type était
\mtype{\typ s,b}).

Mais on peut aussi proposer par exemple \type{s,\type{\type{e,t},t}}
pour $Q$ et \type{e,t} pour $A$, etc.

\item \(\Xlo\Extn[Q(A)]\)

Cette fois, $\Xlo\Extn$ s'accole à tout $\Xlo[Q(A)]$.  Pour que cela soit
possible, on en déduit donc que c'est $\Xlo[Q(A)]$ qui est de type
\mtype{\typ s,b}. Et donc  \(\Xlo\Extn[Q(A)]\) est de type $b$.  Mais on
sait aussi par hypothèse que $b=\typ t$.  Il reste donc à trouver les
types de $\vrb Q$ et $\vrb A$ sachant que $\Xlo[Q(A)]$ est de type \type{s,t}.
Maintenant cela ressemble au premier cas ci-dessus : $\vrb Q$ est de type
\mtype{a,\type{s,t}}, puisque c'est la fonction qui doit retourner un
résultat de type \type{s,t} ; et $a$ est le type de $\vrb A$.  On peut le
choisir librement.  Si $a=\typ e$, alors $\vrb Q$ est de type
\type{e,\type{s,t}}.

Si $a=\type{e,t}$, alors $\vrb Q$ est de type \type{\type{e,t},\type{s,t}}, etc.
\end{enumerate}

\end{Solution}
\begin{Solution}{5.{12}}
 (p. \pageref{exo:lEns})

\begin{enumerate}
\item \(\prd{élément-de} = \Xlo\lambda P\lambda x[P(x)]\)
\item \(\prd{inclus} = \Xlo\lambda Q\lambda P\forall x[[P(x)]\implq[Q(x)]]\)
\item \(\prd{disjoint} = \Xlo\lambda Q\lambda P\neg\exists x[[P(x)]\wedge[Q(x)]]\)
\item \(\prd{intersection} = \Xlo\lambda Q\lambda P\lambda x[[P(x)]\wedge[Q(x)]]\)
\item \(\prd{union} = \Xlo\lambda Q\lambda P\lambda x[[P(x)]\vee[Q(x)]]\)
\item \(\prd{vide} = \Xlo\lambda P\neg\exists x[P(x)]\)
\item \(\prd{singleton} = \Xlo\lambda P\exists x[[P(x)]\wedge \forall y[[P(y)]\implq y=x]]\)
\item \(\prd{parties-de} = \Xlo\lambda P\lambda Q\forall x[[Q(x)]\implq[P(x)]]\)

NB : ce dernier \lterme\ est identique à \prd{inclus} (c'est normal car $A\in\powerset(B)$ ssi $A\inclus B$).
\end{enumerate}
\end{Solution}
\begin{Solution}{5.{13}}
 (p. \pageref{exo:notaPTQ})

\begin{enumerate}
\item $\Xlo\cns a^*$ est de type \type{\type{s,\type{\type{s,e},t}},t}
\\
=
$\Xlo\lambda P[P\{\Intn\alpha\}]$ =
$\Xlo\lambda P[\Extn P(\Intn\alpha)]$ (avec \vrb P de type  \type{s,\type{\type{s,e},t}})

\item $\Xlo\lambda x\, \Intn\prd{aimer}\{\cns a,x\}$ est de type \et
\\
=
$\Xlo\lambda x[\Extn\Intn\prd{aimer}(\cns a,x)]$ =
$\Xlo\lambda x\,\prd{aimer}(\cns a,x)$

\item $\Xlo\hat{\vrbS P}[\vrbS P\{x\}]$ est de type \type{s,\type{\type{\type{s,e},t},t}}
\\
= $\Xlo\Intn\lambda{\vrbS P}[\Extn\vrbS P(x)]$

\item $\Xlo\mathit{x̑}\,\prd{citron}(x)$ est de type \et
\\
=
$\Xlo\lambda x\,\prd{citron}(x)$

\item $\Xlo\hat x\, \prd{aimer}(\cns a,x)$ est de type \type{s,\et}
\\
=
$\Xlo\Intn\lambda x\, \prd{aimer}(\cns a,x)$

\item $\Xlo[\cns a^*(\Intn\vrbS P)]$ est de type \typ t
\\
=
$\Xlo[\lambda P[P\{\Intn\cns a\}](\Intn\vrbS P)]$ =
$\Xlo[\lambda P[\Extn P(\Intn\cns a)](\Intn\vrbS P)]$ =
$\Xlo[\Extn \Intn\vrbS P(\Intn\cns a)]$ =
$\Xlo[\vrbS P(\Intn\cns a)]$
\end{enumerate}

\smallskip

\cns a dénote \Obj{Alice} dans $w$ ; pour savoir ce que dénote
$\Xlo\cns a^*$, commençons par restituer ce que cette écriture
abrège : \(\Xlo\lambda P[P\{\Intn\cns a\}]\).
On sait que
\(\Xlo\Intn\cns a\) dénote le concept individuel d'Alice -- appelons
cela le concept-Alice.   Ensuite \(\Xlo[P\{\Intn\cns a\}]\) équivaut à
\(\Xlo[\Extn P(\Intn\cns a)]\).  On sait que \vrb P est de type
\type{s,\type{\type{s,e},t}}, ce qui veut dire que c'est une variable
de propriété (intensionnelle) de concepts d'individus, et donc
$\Xlo\Extn P$ est de type \type{\type{s,e},t}, c'est l'extension de la
propriété \vrb P dans le monde $w$ (autrement dit un ensemble de
concepts d'individus).  Donc  \(\Xlo[\Extn P(\Intn\cns a)]\) est une
formule, et elle est vraie dans $w$ si le concept-Alice vérifie la
propriété \vrb P dans $w$.  Donc
\(\Xlo\lambda P[P\{\Intn\cns a\}]\), \ie\ $\Xlo\cns a^*$, est
l'ensemble de toutes les propriétés intensionnelles possédées par le
concept-Alice dans le monde $w$.
\end{Solution}
\protect \section {Chapitre \protect \ref {ch:ISS}}
\begin{Solution}{6.{1}}
(p.~\pageref{exo:6adj})\label{crg:6adj}

Il y a a priori deux grands types de structures syntaxiques pour analyser \sicut{petit tigre édenté}, schématisés en (a) et (b) ci-dessous ; sans entrer dans les détails de ces analyses, nous allons retenir la version (b) (qui est la plus couramment adoptée pour le français).

\ex.[(a)]
\small
\Tree[.N$'$
  [.AP petit ]
  [.N$'$
    [.N$'$ tigre ]
    [.AP édenté ]
  ]
]
\normalsize\qquad
(b)\quad\small
\Tree[.N$'$
  [.N$'$
    [.AP petit ]
    [.N$'$ tigre ]
  ]
    [.AP édenté ]
]
\normalsize

\sloppy

\S\ref{ss:ISSmodifieurs} nous offre plusieurs options d'analyses sémantiques ; choisissons ici celle qui traduit \sicut{petit} par \(\Xlo\lambda P\lambda x [[\prd{petit}(P)](x)]\), de type \type{\et,\et} (cf. p.~\pageref{p.petitA}).  \sicut{Tigre} se traduit par \(\Xlo\lambda x\,\prd{tigre}(x)\) ou, pour simplifier immédiatement, \prd{tigre}.  La traduction de \sicut{petit tigre} est donc \(\Xlo[\lambda P\lambda x [[\prd{petit}(P)](x)](\prd{tigre})]\) par application fonctionnelle, puis \(\Xlo\lambda x [[\prd{petit}(\prd{tigre})](x)]\) par \breduc.
Nous pouvons traduire \sicut{édenté} comme \sicut{petit} (\ie\ \(\Xlo\lambda P\lambda y [[\prd{édenté}(P)](y)]\)), mais puisque c'est un adjectif intersectif, nous pouvons tout aussi bien le traiter comme de type \et, \(\Xlo\lambda x\,\prd{édenté}(x)\), et par la règle de \emph{modification de prédicat} (règle \ref{ri:PM}, p.~\pageref{ri:PM}), \sicut{petit tigre édenté} se traduira par \(\Xlo\lambda y[[\lambda x [[\prd{petit}(\prd{tigre})](x)](y)]\wedge[\lambda x\,\prd{édenté}(x)(y)]]\), ce qui après \breduc\ donnera :
\(\Xlo\lambda y[ [[\prd{petit}(\prd{tigre})](y)]\wedge\prd{édenté}(y)]\).


À noter que si nous avions traduit \sicut{édenté} par \(\Xlo\lambda P\lambda y [[\prd{édenté}(P)](y)]\), nous aurions obtenu au final : \(\Xlo\lambda y [[\prd{édenté}(\lambda x [[\prd{petit}(\prd{tigre})](x)])](y)]\) (ou \(\Xlo\lambda y [[\prd{édenté}([\prd{petit}(\prd{tigre})])](y)]\) par $\eta$-réduction), ce qui aura les mêmes conditions de vérité que le résultat précédent si nous posons que la dénotation de \(\Xlo\lambda P\lambda y [[\prd{édenté}(P)](y)]\) renvoie l'ensemble qui est l'intersection de l'ensemble des édentés avec l'ensemble qui est la dénotation de l'argument~(\vrb P).

\fussy
\end{Solution}
\begin{Solution}{6.{2}}
(p.~\pageref{exo:6VPf})\label{crg:6VPf}

\sloppy

Si  nous voulons que les VP soient la fonction principale de la phrase, alors ceux-ci devront être de type \type{\ett,t}, attendant un DP sujet de type \ett\ pour produire une expression de type \typ t.  \sicut{Dormir} se traduira alors par \(\Xlo\lambda X[X(\lambda x\,\prd{dormir}(x))]\), avec $\vrb X\in\VAR_{\ett}$.
Ainsi nous pourrons dériver la traduction de \sicut{Alice dort} en composant
\(\Xlo[\lambda X[X(\lambda x\,\prd{dormir}(x))](\lambda P[P(\cns a)])]\) qui, par \breduc s, se simplifie en
\(\Xlo[\lambda P[P(\cns a)](\lambda x\,\prd{dormir}(x))]\)\footnote{Ce qui nous fait revenir évidemment à ce dont nous partons p.~\pageref{p.PaV}.},
puis en
\(\Xlo[\lambda x\,\prd{dormir}(x)(\cns a)]\) et
\(\Xlo\prd{dormir}(\cns a)\).

\fussy
\end{Solution}
\begin{Solution}{6.{3}}
(p.~\pageref{exo:6iotad})\label{crg:6iotad}

\sloppy
Pour obtenir un DP défini singulier de type \typ e, le déterminant \sicut{le} devra être de type \type{\et,e} et se traduire par \(\Xlo\lambda P\atoi x[P(x)]\).
Ainsi, \sicut{le Pape} se traduira par \(\Xlo[\lambda P\atoi x[P(x)](\lambda y\,\prd{pape}(y))]\) qui, par \breduc s, se simplifie en
\(\Xlo\atoi x[\lambda y\,\prd{pape}(y)(x)]\) et
\(\Xlo\atoi x\,\prd{pape}(x)\) (qui est bien de type \typ e).

\fussy
\end{Solution}
\begin{Solution}{6.{4}}
(p.~\pageref{exo:6deriv})\label{crg:6deriv}

Comme d'habitude, nous procédons à des renommage de variables pour éviter tout conflit et toute confusion.
\begin{enumerate}
\item Tous les enfants mangent une glace.
\begin{enumerate}
\item \(\sicut{une} \leadsto \Xlo\lambda Q\lambda P\exists y[[P(y)]\wedge[Q(y)]]\)
\item \(\sicut{glace} \leadsto \Xlo\lambda z\,\prd{glace}(z)\)

\item \(\sicut{une glace} \leadsto \Xlo[\lambda Q\lambda P\exists y[[P(y)]\wedge[Q(y)]](\lambda z\,\prd{glace}(z))]\)\\
\(=\Xlo\lambda P\exists y[[P(y)]\wedge[\lambda z\,\prd{glace}(z)(y)]]\)
\hfill{\small(\breduc\ sur \vrb Q)}\\
\(=\Xlo\lambda P\exists y[[P(y)]\wedge \prd{glace}(y)]\)
\hfill{\small(\breduc\ sur \vrb z)}

\item \(\sicut{mangent} \leadsto \Xlo \lambda Y\lambda v[Y(\lambda u\,\prd{manger}(v,u))]\)

\item \(\sicut{mangent une glace} \leadsto \Xlo [\lambda Y\lambda v[Y(\lambda u\,\prd{manger}(v,u))](\lambda P\exists y[[P(y)]\wedge \prd{glace}(y)])]\)\\
\(= \Xlo \lambda v[\lambda P\exists y[[P(y)]\wedge \prd{glace}(y)](\lambda u\,\prd{manger}(v,u))]\)
\hfill{\small(\breduc\ sur \vrb Y)}\\
\(= \Xlo \lambda v\exists y[[\lambda u\,\prd{manger}(v,u)(y)]\wedge \prd{glace}(y)]\)
\hfill{\small(\breduc\ sur \vrb P)}\\
\(= \Xlo \lambda v\exists y[\prd{manger}(v,y)\wedge \prd{glace}(y)]\)
\hfill{\small(\breduc\ sur \vrb u)}

\item \(\sicut{tous les} \leadsto \Xlo\lambda Q\lambda P\forall x[[Q(x)]\implq[P(x)]]\)
\item \(\sicut{enfants} \leadsto \Xlo\lambda z\,\prd{enfant}(z)\)

\item \(\sicut{tous les enfants} \leadsto \Xlo[\lambda Q\lambda P\forall x[[Q(x)]\implq[P(x)]](\lambda z\,\prd{enfant}(z))]\)\\
\(=\Xlo\lambda P\forall x[[\lambda z\,\prd{enfant}(z)(x)]\implq[P(x)]]\)
\hfill{\small(\breduc\ sur \vrb Q)}\\
\(=\Xlo\lambda P\forall x[\prd{enfant}(x)\implq[P(x)]]\)
\hfill{\small(\breduc\ sur \vrb z)}

\item \(\sicut{tous les enfants mangent une glace} \leadsto\) \\
\(\Xlo[\lambda P\forall x[\prd{enfant}(x)\implq[P(x)]](\lambda v\exists y[\prd{manger}(v,y)\wedge \prd{glace}(y)])]\)\\
\(=\Xlo\forall x[\prd{enfant}(x)\implq[\lambda v\exists y[\prd{manger}(v,y)\wedge \prd{glace}(y)](x)]]\)
\hfill{\small(\breduc\ sur \vrb P)}\\
\(=\Xlo\forall x[\prd{enfant}(x)\implq\exists y[\prd{manger}(x,y)\wedge \prd{glace}(y)]]\)
\hfill{\small(\breduc\ sur \vrb v)}
\end{enumerate}

\item Peter Parker est Spiderman.
\begin{enumerate}
\item \(\sicut{Spiderman} \leadsto \Xlo\lambda P[P(\cns s)]\)
\item \(\sicut{est} \leadsto \Xlo\lambda Y\lambda x[Y(\lambda y[x=y])]\)

\item \(\sicut{est Spiderman} \leadsto \Xlo[\lambda Y\lambda x[Y(\lambda y[x=y])](\lambda P[P(\cns s)])]\)\\
\(=\Xlo\lambda x[\lambda P[P(\cns s)](\lambda y[x=y])]\)
\hfill{\small(\breduc\ sur \vrb Y)}\\
\(=\Xlo\lambda x[\lambda y[x=y](\cns s)]\)
\hfill{\small(\breduc\ sur \vrb P)}\\
\(=\Xlo\lambda x[x=\cns s]\)
\hfill{\small(\breduc\ sur \vrb y)}

\item \(\sicut{Peter Parker} \leadsto \Xlo\lambda P[P(\cns p)]\)

\item \(\sicut{Peter Parker est Spiderman} \leadsto \Xlo[\lambda P[P(\cns p)](\lambda x[x=\cns s])]\)\\
\(=\Xlo[\lambda x[x=\cns s](\cns p)]\)
\hfill{\small(\breduc\ sur \vrb P)}\\
\(=\Xlo[\cns p=\cns s]\)
\hfill{\small(\breduc\ sur \vrb x)}
\end{enumerate}
\end{enumerate}
\end{Solution}
\begin{Solution}{6.{5}}
(p.~\pageref{exo:6HK})\label{crg:6HK}

Assigner le type \type{\eet,\et} à un DP revient à prévoir que celui-ci attend un V transitif (de type \eet) pour produire un VP de type \et.  Posons la variable \vrb R  de type \eet, pour jouer le rôle du V transitif. \sicut{Une glace} se traduira alors par \(\Xlo\lambda R\lambda x\exists y[\prd{glace}(y)\wedge [[R(y)](x)]]\).
La dérivation du VP \sicut{mange une glace} sera alors la suivante (le  DP objet dénote la fonction et le V transitif est l'argument) :

\begin{enumerate}
\item[] \(\sicut{mange une glace} \leadsto
\Xlo[\lambda R\lambda x\exists y[\prd{glace}(y)\wedge [[R(y)](x)]](\lambda v\lambda u\,\prd{manger}(u,v))]\)\\
= \(\Xlo\lambda x\exists y[\prd{glace}(y)\wedge [[\lambda v\lambda u\,\prd{manger}(u,v)(y)](x)]]\)
\hfill{\small(\breduc\ sur \vrb R)}
\\
= \(\Xlo\lambda x\exists y[\prd{glace}(y)\wedge [\lambda u\,\prd{manger}(u,y)(x)]]\)
\hfill{\small(\breduc\ sur \vrb v)}
\\
= \(\Xlo\lambda x\exists y[\prd{glace}(y)\wedge \prd{manger}(x,y)]\)
\hfill{\small(\breduc\ sur \vrb u)}
\end{enumerate}

\end{Solution}
\begin{Solution}{6.{6}}
(p.~\pageref{exo:6Vdit})\label{crg:6Vdit}

\small\noindent
1. \(\Xlo\lambda Y\lambda Z\lambda x[Z(\lambda z[Y(\lambda y\,\prd{donner}(x,y,z))])]\)
\quad
2. \(\Xlo\lambda Y\lambda Z\lambda x[Y(\lambda y[Z(\lambda z\,\prd{donner}(x,y,z))])]\)
\\
3. \(\Xlo\lambda Z\lambda Y\lambda x[Z(\lambda z[Y(\lambda y\,\prd{donner}(x,y,z))])]\)
\quad
4. \(\Xlo\lambda Z\lambda Y\lambda x[Y(\lambda y[Z(\lambda z\,\prd{donner}(x,y,z))])]\)
\normalsize

\sloppy

Ces quatre \lterme s attendent deux quantificateurs généralisés \vrb Y et \vrb Z.  \vrb Y correspond au complément direct du verbe (car il se combine avec une expression qui fait abstraction de \vrb y, le second argument de \prd{donner}) et \vrb Z correspond au complément d'objet indirect (datif).  Ce qui distingue, d'une part, 1 de 3 et, d'autre part, 2 de 4, c'est l'ordre de leurs \labstraction s qui reflète l'ordre dans lequel le verbe rencontre syntaxiquement ses compléments.  Quant à ce qui distingue 1 de 2 (ainsi que 3 de 4) c'est qu'en 1 le quantificateur \vrb Y se trouve dans la portée de \vrb Z, et 2 présente les portées inverses.  Si nous traduisons \sicut{donne un exercice à tous les élèves}, avec 1 (ou 3) nous obtiendrons :
\(\Xlo\lambda x \forall z[\prd{élève}(z)\implq \exists y[\prd{exercice}(y)\wedge \prd{donner}(x,y,z)]]\), et avec 2 (ou 4) :
\(\Xlo\lambda x \exists y[\prd{exercice}(y)\wedge \forall z[\prd{élève}(z)\implq \prd{donner}(x,y,z)]]\).  Le problème est que ces deux interprétations sont possibles, et donc que nous ne devons pas choisir entre 1 et 2 (ou 3 et 4 selon l'analyse syntaxique) ; en d'autres termes nous devrions postuler une ambiguïté de traduction pour les verbes ditransitifs -- mais ce n'est pas la manière la plus efficace de procéder comme le montre la suite du chapitre.

\fussy

\end{Solution}
\begin{Solution}{6.{7}}
(p.~\pageref{exo:6mvt})\label{crg:6mvt}
\begin{enumerate}
\item Nous reprenons l'analyse syntaxique de \ref{x:montéesujet}, p.~\pageref{x:montéesujet}, rappelée ici :\\ {}
[\Stag{TP} [\Stag{DP} \sicut{Alice}]$_1$ [\Stag{T$'$} \sicut{a} [\Stag{VP} $t_1$ [\Stag{V$'$} \sicut{appelé Bruno}]]]], et nous allons traduire tous les DP par des quantificateurs généralisés de type \ett.
\begin{enumerate}
\item \(\text{[\Stag{V$'$} appelé Bruno]} \leadsto
\Xlo[\lambda Y\lambda x[Y(\lambda y\,\prd{appeler}(x,y))](\lambda P[P(\cns b)])]\)\\
\(=\Xlo\lambda x[\lambda P[P(\cns b)](\lambda y\,\prd{appeler}(x,y))]\)
\hfill{\small(\breduc\ sur \vrb Y)}\\
\(=\Xlo\lambda x[\lambda y\,\prd{appeler}(x,y)(\cns b)]\)
\hfill{\small(\breduc\ sur \vrb P)}\\
\(=\Xlo\lambda x\,\prd{appeler}(x,\cns b)\)
\hfill{\small(\breduc\ sur \vrb y)}

\item \(t_1 \leadsto \Xlo\lambda P[P(x_1)]\)
\hfill{\small(cf. p.~\pageref{p.trace1})}

\item \(\text{[\Stag{VP} $t_1$ appelé Bruno]} \leadsto
\Xlo[\lambda P[P(x_1)](\lambda x\,\prd{appeler}(x,\cns b))]\)\\
\(=\Xlo[\lambda x\,\prd{appeler}(x,\cns b)(x_1)]\)
\hfill{\small(\breduc\ sur \vrb P)}\\
\(=\Xlo\prd{appeler}(x_1,\cns b)\)
\hfill{\small(\breduc\ sur \vrb x)}

\item Pour l'instant nous considérons que l'auxiliaire n'a pas de contribution sémantique et donc T$'$ se traduit comme VP.

\item \(\text{[\Stag{TP} Alice$_1$ [\Stag{T$'$} a $t_1$ appelé Bruno]]} \leadsto\)\\
\(\Xlo [\lambda P[P(\cns a)](\lambda x_1\,\prd{appeler}(x_1,\cns b))]\)
\hfill {\small (montée du sujet, cf. \ref{ri:MontSuj}, p.~\pageref{ri:MontSuj})}\\
\(=\Xlo [\lambda x_1\,\prd{appeler}(x_1,\cns b)(\cns a)]\)
\hfill{\small(\breduc\ sur \vrb P)}\\
\(=\Xlo \prd{appeler}(\cns a,\cns b)\)
\hfill{\small(\breduc\ sur \vrbi x1)}
\end{enumerate}

\item En nous inspirant de ce qui a été vu dans le chapitre, nous posons l'analyse syntaxique suivante : %\\{}
[\Stag{DP} \sicut{tous les} [\Stag{NP} [\Stag{NP} \sicut{écrivains}] [\Stag{CP} \sicut{que}$_2$ [\Stag{TP} \sicut{Charles}$_1$ [\Stag{VP} $t_1$ \sicut{connaît} $t_2$]]]]].\\
Les traces $t_1$ et $t_2$ sont traduites respectivement par $\Xlo\lambda P[P(x_1)]$ et $\Xlo\lambda P[P(x_2)]$.
\begin{enumerate}
\item \(\text{[\Stag{V$'$} connaît $t_2$]} \leadsto
\Xlo[\lambda Y\lambda x[Y(\lambda y\,\prd{connaître}(x,y))](\lambda P[P(x_2)])]\)\\
\(=\Xlo\lambda x[\lambda P[P(x_2)](\lambda y\,\prd{connaître}(x,y))]\)
\hfill{\small(\breduc\ sur \vrb Y)}\\
\(=\Xlo\lambda x[\lambda y\,\prd{connaître}(x,y)(x_2)]\)
\hfill{\small(\breduc\ sur \vrb P)}\\
\(=\Xlo\lambda x\,\prd{connaître}(x,x_2)\)
\hfill{\small(\breduc\ sur \vrb y)}

\item \(\text{[\Stag{VP} $t_1$ connaît $t_2$]} \leadsto
\Xlo[\lambda P[P(x_1)](\lambda x\,\prd{connaître}(x,x_2))]\)\\
\(=\Xlo[\lambda x\,\prd{connaître}(x,x_2)(x_1)]\)
\hfill{\small(\breduc\ sur \vrb P)}\\
\(=\Xlo\prd{connaître}(x_1,x_2)\)
\hfill{\small(\breduc\ sur \vrb x)}

\item \(\text{[\Stag{TP} Charles$_1$ $t_1$ connaît $t_2$]} \leadsto\)\\
\(\Xlo[\lambda P[P(\cns c)](\lambda x_1\,\prd{connaître}(x_1,x_2))]
\)
\hfill{\small(montée du sujet, p.~\pageref{ri:MontSuj})}\\
\(=\Xlo[\lambda x_1\,\prd{connaître}(x_1,x_2)(\cns c)]
\)
\hfill{\small(\breduc\ sur \vrb P)}\\
\(=\Xlo\prd{connaître}(\cns c,x_2)
\)
\hfill{\small(\breduc\ sur \vrbi x1)}

\item \(\text{que} \leadsto \Xlo \lambda P\lambda x[P(x)]\)
\hfill{\small(cf. p.~\pageref{p.prorel})}%
\footnote{Nous pourrions aussi choisir de traduire le pronom relatif par $\Xlo\lambda Q\lambda P[[P(x)]\wedge[Q(x)]]$ comme suggéré dans le chapitre. Dans ce cas, nous n'aurons pas à appliquer la règle de modification de prédicat.}

\item \(\text{[\Stag{CP} que$_2$ Charles$_1$ $t_1$ connaît $t_2$]} \leadsto\)\\
\(\Xlo[\lambda P\lambda x[P(x)](\lambda x_2\,\prd{connaître}(\cns c,x_2))]\)
\hfill{\small(montée du pronom, p.~\pageref{p.montprorel})}\\
\(=\Xlo\lambda x[\lambda x_2\,\prd{connaître}(\cns c,x_2)(x)]\)
\hfill{\small(\breduc\ sur \vrb P)}\\
\(=\Xlo\lambda x\,\prd{connaître}(\cns c,x)\)
\hfill{\small(\breduc\ sur \vrbi x2)}\\
CP est de type \et\ comme il se doit, et va se combiner avec [\Stag{NP} \sicut{écrivains}] également de type {\et} ($\Xlo\lambda x\,\prd{écrivain}(x)$) ; nous devons donc appliquer la règle de modification de prédicat vue en \S\ref{ss:ISSmodifieurs} (règle \ref{ri:PM}, p.~\pageref{ri:PM}):

\item \(\text{[\Stag{NP} [\Stag{NP} écrivains] que$_2$ Charles$_1$ $t_1$ connaît $t_2$]} \leadsto \)\\
\(\Xlo\lambda y[[\lambda x\,\prd{écrivain}(x)(y)]\wedge [\lambda x\,\prd{connaître}(\cns c,x)(y)]]\)
\hfill{\small(modification de prédicat)}\\
\(=\Xlo\lambda y[\prd{écrivain}(y)\wedge \prd{connaître}(\cns c,y)]\)
\hfill{\small(\breduc s sur \vrb x)}

\item \(\text{[\Stag{DP} tous les écrivains que$_2$ Charles$_1$ $t_1$ connaît $t_2$]} \leadsto \)\\
\(\Xlo [\lambda Q\lambda P\forall x[[Q(x)]\implq[P(x)]](\lambda y[\prd{écrivain}(y)\wedge \prd{connaître}(\cns c,y)])]\)\\
\(=\Xlo \lambda P\forall x[[\lambda y[\prd{écrivain}(y)\wedge \prd{connaître}(\cns c,y)](x)]\implq[P(x)]]\)
\hfill{\small(\breduc\ sur \vrb Q)}\\
\(=\Xlo \lambda P\forall x[[\prd{écrivain}(x)\wedge \prd{connaître}(\cns c,x)]\implq[P(x)]]\)
\hfill{\small(\breduc\ sur \vrb y)}
\end{enumerate}
\end{enumerate}
\end{Solution}
\begin{Solution}{6.{8}}
(p.~\pageref{exo:6VaM})\label{crg:6VaM}
%[\Stag{TP} Alice$_1$ semble [\Stag{VP} $t_1$ dormir]].

Nous allons supposer l'analyse syntaxique donnée en figure \ref{f:VaMontée} où \sicut{Alice} monte de la position de sujet de \sicut{dormir} (Spec de VP) jusqu'à la position de sujet de la phrase (Spec de TP) en trois étapes successives%
\footnote{Peu importe ici que cette analyse soit ou non parfaitement correcte sur le plan syntaxique ; l'objectif de l'exercice est de nous faire manipuler des traces d'un même constituant déplacé plusieurs fois.}.

\begin{figure}[h]
\begin{center}
{\small
\Tree
[.TP
  [.DP$_1$ \rnode{a}{Alice} ]
  [.T$'$
    [.VP \rnode{t13}{$t_1$}
      [.V$'$
        [.V semble ]
        [.TP
          \rnode{t12}{$t_1$}
          [.T$'$
            [.T avoir ]
            [.VP
              \rnode{t11}{$t_1$}
              [.V$'$ dormi ]
            ]
          ]
        ]
      ]
    ]
  ]
]
}%
\ncbar[angle=-90,linecolor=darkgray,nodesep=2pt,offsetB=-1.5pt]{->}{t11}{t12}%
\ncbar[angle=-90,linecolor=darkgray,nodesep=2pt,offset=-1.5pt]{->}{t12}{t13}%
\ncbar[angle=-90,linecolor=darkgray,nodesep=2pt,offsetA=-1.5pt]{->}{t13}{a}
\end{center}
\caption{Hypothèse syntaxique pour \sicut{Alice semble avoir dormi}}\label{f:VaMontée}
\end{figure}

%\bigskip

Les traces $t_1$ marquent des positions anciennement occupées par [\Stag{DP} Alice]$_1$, c'est pourquoi elles ont le même indice et se traduiront donc de la même manière par $\Xlo\lambda P[P(x_1)]$.

\begin{enumerate}
\item {} [\Stag{VP} $t_1$ dormi] $\leadsto$
\(\Xlo[\lambda P[P(x_1)](\lambda x\,\prd{dormir}(x))]\)\\
\(=\Xlo\prd{dormir}(x_1)\)
\hfill{\small (\breduc s sur \vrb P puis \vrb x)}

\item Là encore, nous considérons provisoirement que l'auxiliaire n'a pas de contribution sémantique et donc que T$'$ se traduit comme VP, de type \typ t.
Nous allons donc ensuite devoir appliquer la règle de montée du sujet (cf. \ref{ri:MontSuj} p.~\pageref{ri:MontSuj}) qui ajoute $\Xlo\lambda x_1$, ce qui fait que la deuxième trace $t_1$ est traitée comme un constituant déplacé.

\item {} [\Stag{TP} $t_1$ avoir $t_1$ dormi] $\leadsto$
\(\Xlo[\lambda P[P(x_1)](\lambda x_1\,\prd{dormir}(x_1))]\)
\hfill{\small (montée du sujet)}\\
\(=\Xlo[\lambda x_1\,\prd{dormir}(x_1)(x_1)]\)
\hfill{\small (\breduc\ sur \vrb P)}\\
\(=\Xlo\prd{dormir}(x_1)\)\footnote{Notons qu'ici il n'a pas été nécessaire de renommer les occurrences liées de \vrbi x1 avant d'effectuer la \breduc\ puisqu'à l'arrivée la variable argument \vrbi x1 reste libre.}
\hfill{\small (\breduc\ sur \vrbi x1)}

\item \(\sicut{semble}\leadsto \Xlo\lambda p\,\prd{sembler}(p)\), avec $\vrb p\in\VAR_{\type{s,t}}$.  Comme \prd{sembler} est de type \type{\type{s,t},t}\footnote{Le prédicat \prd{sembler} a une sémantique modale qui se rapproche de celle des verbes d'attitude pro\-po\-si\-tion\-nel\-le.% ; il peut d'ailleurs être raisonnable de généraliser en ajoutant un argument
}, il va falloir utiliser l'application fonctionnelle intensionnelle (définition \ref{d:AFI}, p.~\pageref{d:AFI}).

\item {} [\Stag{V$'$} semble $t_1$ avoir $t_1$ dormi] $\leadsto$
\(\Xlo[\lambda p\,\prd{sembler}(p)(\Intn\prd{dormir}(x_1))]\)\footnote{NB : ici $\Xlo\Intn\prd{dormir}(x_1)$ est en fait la simplification (un peu abusive) de $\Xlo\Intn[\prd{dormir}(x_1)]$ (mais sachant que l'expression est bien formée, il n'y a pas de risque de confondre avec $\Xlo[\Intn\prd{dormir}(x_1)]$).}
\hfill{\small (AFI)}\\
\(=\Xlo\prd{sembler}(\Intn\prd{dormir}(x_1))\)
\hfill{\small (\breduc\ sur \vrb p)}

\item Ici, pour combiner la troisième trace $t_1$ avec V$'$ de type \typ t, nous devons encore une fois ajouter l'abstraction $\Xlo\lambda x_1$, tout en sachant qu'il ne s'agit pas là d'une instance de la règle \ref{ri:MontSuj}, mais d'une règle qui reconnaît que le sujet de \prd{dormir} a quitté la subordonnée pour venir occuper (provisoirement) la position sujet de \sicut{sembler} :\\
{} [\Stag{VP} $t_1$ semble $t_1$ avoir $t_1$ dormi] $\leadsto$\\
\(\Xlo[\lambda P[P(x_1)](\lambda x_1\,\prd{sembler}(\Intn\prd{dormir}(x_1)))]\)
\hfill{\small (ajout de $\Xlo\lambda x_1$)}\\
\(=\Xlo\prd{sembler}(\Intn\prd{dormir}(x_1))\)
\hfill{\small (\breduc s sur \vrb P puis \vrbi x1)}

\item {} [\Stag{TP} Alice $t_1$ semble $t_1$ avoir $t_1$ dormi] $\leadsto$\\
\(\Xlo[\lambda P[P(\cns a)](\lambda x_1\,\prd{sembler}(\Intn\prd{dormir}(x_1)))]\)
\hfill{\small (montée du sujet, règle \ref{ri:MontSuj})}\\
\(=\Xlo\prd{sembler}(\Intn\prd{dormir}(\cns a))\)
\hfill{\small (\breduc s sur \vrb P puis \vrbi x1)}
\end{enumerate}

Remarque : Cet exercice, et principalement l'étape 6 de la dérivation, nous permet de constater que les traces intermédiaires doivent se comporter à la fois comme des traces ordinaires mais aussi comme si elles étaient elles-mêmes des éléments déplacés.  Ce genre de traitement ne va pas entièrement de soi dans la formalisation précise de l'interface syntaxe sémantique (même si l'exercice montre qu'il est réalisable) ; cependant il s'intègre très simplement dans la variante formelle (dite «HK») présentée en \S\ref{sss:VarianteMvt}.

\end{Solution}
\begin{Solution}{6.{9}}
(p.~\pageref{exo:6QG})\label{crg:6QG}
  \begin{enumerate}
    \item Il y a quatre 2\textsc{CV} vertes dans le parking.\\
\(\Xlo\prd{Quatre}(\lambda x[\prd{2cv}(x)\wedge\prd{vert}(x)])(\lambda x\,\prd{dans}(x,\atoi y\,\prd{parking}(y)))\)

Évidemment le déterminant \prd{Quatre} est calqué, \alien{mutatis mutandis},  sur \prd{Deux} et \prd{Trois} vus en \S\ref{ss:QGDet}, p.~\pageref{xd:Deux}.

    \item Moins de la moitié des candidats ont répondu à toutes les questions.\\
Commençons par traduire \sicut{ont répondu à toutes les questions}, c'est un VP de type {\et} : \(\Xlo\lambda x\forall y[\prd{question}(y)\implq\prd{répondre}(x,y)]\). Ensuite, nous devons fournir une définition du déterminant \sicut{moins de la moitié} :\\
\(\denote{\Xlo \prd{Moins-de-la-moitié}(R)(P)}^{\Modele,w,g}=1\) ssi
\(\Card{\Ch{\denote{\Xlo R}}^{\Modele,w,g} \cap \Ch{\denote{\Xlo P}}^{\Modele,w,g}} <
\frac{\nbr1}{\nbr2}\Card{\Ch{\denote{\Xlo R}}^{\Modele,w,g}}\).
\\
À partir de là, nous pouvons traduire la phrase :\\
\(\Xlo\prd{Moins-de-la-moitié}(\prd{candidat})(\lambda x\forall y[\prd{question}(y)\implq\prd{répondre}(x,y)])\)
\\
Cette formule dit que le nombre d'éléments dans l'ensemble des candidats qui ont répondu à toutes les questions est inférieur à la moitié du nombre total de candidats. Mais il y a une autre interprétation, avec les portées inversées des quantificateurs :\\
\(\Xlo\forall y[\prd{question}(y)\implq\prd{Moins-de-la-moitié}(\prd{candidat})(\lambda x\,\prd{répondre}(x,y))]\)\\
Cette traduction dit que pour chaque question, il y a, à chaque fois, moins de la moitié des candidats qui y ont répondu.  Cette lecture exclut par exemple les scénarios où il y  aurait eu quelques questions très faciles auxquelles la plupart des candidats ont su répondre.


    \item Trois stagiaires apprendront deux langues étrangères.

\(\Xlo\prd{Trois}(\prd{stagiaire})(\lambda x\,\prd{Deux}(\lambda y[\prd{langue}(y)\wedge\prd{étranger}(y)])(\lambda y\,\prd{apprendre}(x,y)))\)\\
Cette formule est vraie ssi l'intersection de l'ensemble des stagiaires et de ceux qui ont appris deux langues étrangères contient (au moins) 3 individus.  Il peut donc y avoir 6 langues différentes en jeu dans ce genre de situations.

\(\Xlo\prd{Deux}(\lambda y[\prd{langue}(y)\wedge\prd{étranger}(y)])(\lambda y\,\prd{Trois}(\prd{stagiaire})(\lambda x\,\prd{apprendre}(x,y)))\)\\
Cette traduction présente la lecture avec portées inversées que l'on obtient en appliquant \QRa\ (cf. \S\ref{ss:QR}).  Elle est vraie ssi l'intersection de l'ensemble des langues étrangères et de l'ensemble des choses qui ont été apprises par trois stagiaires contient (au moins) 2 éléments.  Il peut donc y avoir au total 6 stagiaires dans l'histoire (par exemple trois qui apprennent le japonais, et trois autres qui apprennent le russe).
  \end{enumerate}
\end{Solution}
\begin{Solution}{6.{10}}
(p.~\pageref{exo:LOWER})\label{crg:LOWER}

\sloppy
Par définition (\ref{xd:lower}, p.~\pageref{xd:lower}), \prdk{lower} équivaut à $\Xlo \lambda X\atoi y \forall P[[X(P)] \implq [P(y)]]$ ; par conséquent $\Xlo\prdk{lower}(\lambda P[P(\cns a)])$ équivaut à
$\Xlo [\lambda X\atoi y \forall P[[X(P)] \implq [P(y)]](\lambda P'[P'(\cns a))]$ qui, par \breduc s successives, se simplifie en
$\Xlo\atoi y \forall P[[\lambda P'[P'(\cns a)(P)] \implq [P(y)]]$
puis en
$\Xlo\atoi y \forall P[[P(\cns a)] \implq [P(y)]]$.
Ce \atoi-terme dénote dans \w\ l'individu qui possède \emph{toutes} les propriétés (extensionnelles) que possède Alice dans \w.  Cet individu existe, est unique et est forcément Alice, ne serait-ce parce que parmi les propriétés d'Alice, il y a celle qui s'exprime par $\Xlo\lambda x[x=\cns a]$ et que seule Alice satisfait.  Donc $\Xlo\prdk{lower}(\lambda P[P(\cns a)])$ est bien équivalent à \cns a.


De la même façon, \(\Xlo\prdk{lower}(\lambda P\forall x[\prd{enfant}(x)\implq[P(x)]])\)
équivaut à
\(\Xlo[\lambda X\atoi y \forall P[[X(P)] \implq [P(y)]](\lambda P'\forall x[\prd{enfant}(x)\implq[P'(x)]])]\)
qui se réduit en
\(\Xlo\atoi y \forall P[\forall x[\prd{enfant}(x)\implq[P(x)]] \implq [P(y)]]\).
Ce \atoi-terme dénote l'unique individu qui possède toutes les propriétés communes à  tous les enfants ; mais s'il y a plusieurs enfants dans le monde d'évaluation, il ne peut pas y avoir un seul individu qui satisfait cette condition (chaque enfant a toutes les propriétés de tous les enfants). Donc la dénotation de \(\Xlo\prdk{lower}(\lambda P\forall x[\prd{enfant}(x)\implq[P(x)]])\)
n'est pas définie (tant qu'il y a plusieurs individus dans la dénotation de \sicut{enfant}).


De même, \(\Xlo\prdk{lower}(\lambda P\exists x[\prd{enfant}(x)\wedge[P(x)]])\) se réduit en
\(\Xlo\atoi y \forall P[\exists x[\prd{enfant}(x)\wedge[P(x)]] \implq [P(y)]]\).
Cette fois ce \atoi-terme dénote l'unique individu qui possède toute propriété satisfaite par au moins un enfant, mais un tel individu n'existe pas (s'il y a plusieurs enfants) car parmi ces propriétés il y en a beaucoup qui sont contradictoires entre elles (par exemple, il y a des enfants bruns, des enfants blonds, des filles, des garçons, etc.). Ainsi, comme précédemment, la dénotation de \(\Xlo\prdk{lower}(\lambda P\exists x[\prd{enfant}(x)\wedge[P(x)]])\) n'est pas définie.

\fussy
\end{Solution}
\begin{Solution}{6.{11}}
(p.~\pageref{exo:BE-Deux})\label{crg:BE-Deux}

\sloppy

Par définition (cf.\ \ref{xd:BE}, p.~\pageref{xd:BE}), \prdk{be} est équivalent à \(\Xlo\lambda Y\lambda x [Y(\lambda y [y=x])]\). Sachant que $\Xlo\prdk{be}(\prd{Deux}(\prd{enfant}))$ est de type \ett\ (c'est un quantificateur généralisé),
\(\Xlo\prdk{be}(\prd{Deux}(\prd{enfant}))\) peut donc se réécrire en :
\(\Xlo[\lambda Y\lambda x [Y(\lambda y [y=x])](\prd{Deux}(\prd{enfant}))]\).
Par \breduc, cela se simplifie en
\(\Xlo\lambda x [\prd{Deux}(\prd{enfant})(\lambda y [y=x])]\).
\(\Xlo\prd{Deux}(\prd{enfant})(\lambda y [y=x])\) est une structure tripartite de type \typ t, et sa portée \(\Xlo\lambda y [y=x]\) dénote le singleton \(\set{\denote{\vrb x}^{\Modele,w,g}}\).  Par définition, \(\Xlo\prd{Deux}(\prd{enfant})(\lambda y [y=x])\) est donc vraie ssi l'intersection de l'ensemble de tous les enfants et de l'ensemble \(\set{\denote{\vrb x}^{\Modele,w,g}}\) contient au moins deux éléments ; mais cela est, bien sûr, mathématiquement impossible puisque \(\set{\denote{\vrb x}^{\Modele,w,g}}\) ne contient qu'un seul élément.
Par conséquent, \(\Xlo\prd{Deux}(\prd{enfant})(\lambda y [y=x])\) sera toujours faux et \(\Xlo\lambda x [\prd{Deux}(\prd{enfant})(\lambda y [y=x])]\) dénotera toujours l'ensemble vide.

\fussy

\end{Solution}
\begin{Solution}{6.{12}}
(p.~\pageref{exo:6FC})\label{crg:6FC}

\sloppy
$\Xlo f(\prd{enfant})$, de type \typ e, dénote un enfant particulier choisi par \vrb f. Appelons, provisoirement, cet enfant Fanny. Le quantificateur $\Xlo \prd{Un}(\prd{enfant})$ dénote l'ensemble de tous les ensembles qui contiennent au moins un enfant.
Nous pouvons déjà remarquer que $\Xlo\lambda P[P(f(\prd{enfant}))]$, même si c'est bien un quantificateur généralisé, n'est pas équivalent à $\Xlo \prd{Un}(\prd{enfant})$, puisque le premier dénote l'ensemble de tous les ensembles qui contiennent \Obj{Fanny}, alors que nous cherchons l'ensemble de tous les ensembles qui contiennent n'importe quel enfant.  Pour prendre en compte n'importe quel enfant, nous devons donc faire en sorte que \vrb f ne soit plus libre (afin que sa dénotation ne dépende plus fixement de l'assignation $g$ globale), et nous obtenons cela en quantifiant sur \vrb f.  Mais il ne faut pas se tromper : $\Xlo\forall f[P(f(\prd{enfant}))]$ est une formule qui est vraie ssi \vrb P est une propriété qui est commune à tous les enfants (car $\Xlo\forall f$ va nous faire parcourir toutes les façons de choisir un enfant dans la dénotation de \prd{enfant}).  C'est bien une quantification existentielle qu'il faut utiliser ici : $\Xlo\exists f[P(f(\prd{enfant}))]$ est vraie ssi \vrb P est une propriété satisfaite par au moins un enfant\footnote{NB : $\Xlo[P(\exists f\,f(\prd{enfant}))]$ et $\Xlo[P(\forall f\,f(\prd{enfant}))]$ sont des expressions mal formées de \LO, car $\Xlo f(\prd{enfant})$ n'est pas de type \typ t.}
 ; et donc ce qui équivaut à $\Xlo \prd{Un}(\prd{enfant})$ est $\Xlo\lambda P\exists f[P(f(\prd{enfant}))]$.

\fussy

L'opérateur qui nous ferait passer du terme $\Xlo f(\prd{enfant})$ au quantificateur généralisé devrait donc être le \lterme\ $\Xlo\lambda x\lambda P\exists f[P(x)]$.  Or cela semble présupposer que nous connaîtrions à l'avance le nom de la variable \vrb f de fonction de choix utilisée «dans \vrb x~», ce qui n'a pas de raison d'être.  Mais, en fait, la situation est encore plus grave car, formellement, dans ce \lterme, $\Xlo\exists f$ ne lie rien du tout, sémantiquement il ne sert à rien et en soi $\Xlo\lambda x\lambda P\exists f[P(x)]$ équivaut  à $\Xlo\lambda x\lambda P[P(x)]$\footnote{Sans compter que dans $\Xlo[\lambda x\lambda P\exists f[P(x)](f(\prd{enfant}))]$, nous n'aurions pas le droit d'effectuer la \breduc\ (alors que c'est précisément ce qu'il nous faudrait) car le \vrb f libre de l'argument ne doit pas se retrouver lié après réduction (déf.~\ref{d:breduc2} p.~\pageref{d:breduc2}).}.
Par conséquent, si nous admettons que les \alien{type-shifteurs} sont des opérateurs \emph{compositionnels} formalisables par des fonctions, alors le passage de $\Xlo f(\prd{enfant})$ au quantificateur généralisé n'est pas du \alien{type-shifting}.
\end{Solution}
