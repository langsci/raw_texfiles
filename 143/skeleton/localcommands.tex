%add all your local new commands to this file


\newcommand{\tbc}[1]{}
\newcommand{\fixme}[1]{#1}

\newcommand{\up}[1]{\ensuremath{^{\text{#1}}}}

%\newcommand{\nn}[1]{#1\textit{n}} %% pour renvoi d'index dans des footnotes
\newcommand{\nn}[1]{#1} %% pour renvoi d'index dans des footnotes
\newcommand{\sq}[1]{#1\,\textit{sq.}} %% pour renvoi d'index à sq.
\newcommand{\sqq}[1]{#1\,\textit{sqq.}} %% pour renvoi d'index à sqq. (pluriel)

%\newcommand{\Index}[1]{\index{#1}}
%\newcommand{\Indexn}[1]{\index{#1|nn}}
\newcommand{\Index}[1]{}
\newcommand{\Indexn}[1]{}
\newcommand{\Andex}[1]{} % provisoires
\newcommand{\Andexn}[1]{}
%\newcommand{\indexs}[1]{\index[sbj]{#1}}
\newcommand{\indexs}[1]{\is{#1}}

%\newcommand{\og}{<<}
%\renewcommand{\fg}{>>}

\newcommand{\sicut}[1]{\emph{#1}}
\newcommand{\sicuto}[1]{\emph{#1}}
\newcommand{\alien}[1]{\emph{#1}}
\newcommand{\alienx}[1]{#1} %% pas sûr de laisser en italiques


\newcommand{\degres}{°}
\newcommand{\ie}{i.e.}
\newcommand{\vs}{\alien{vs}}
\newcommand{\No}{N\no}
\newcommand{\numero}{n\no}
\newcommand{\myeuro}{euros}%\newcommand{\myeuro}{\texteuro}

\newcommand{\cache}[1]{}

% Macros
\newcommand\vartextvisiblespace[1][.3em]{%
  \mbox{\kern.03em\vrule height.3ex}%
  \vbox{\hrule width#1}%
  \hbox{\vrule height.3ex}\kern.02em}
\newcommand{\slot}{\ensuremath{\text{\vartextvisiblespace[.4em]}}}
\newcommand{\implq}{\ensuremath{\rightarrow}}		% ->
\newcommand{\ssi}{\ensuremath{\leftrightarrow}}		% <->
\newcommand{\set}[1]{\ensuremath{\{ #1 \}}} % { }
\newcommand{\Set}[1]{\ensuremath{\left\{ #1 \right\}}} % { }
\newcommand{\tuple}[1]{\ensuremath{\langle #1 \rangle}} % < >
\newcommand{\acro}[1]{\textsc{#1}}

\newlength{\xboxwidth}
\newcommand{\zbox}[2][l]{\makebox[0pt][#1]{#2}}  %% text box with 0 width
\newcommand{\zcbox}[1]{\makebox[0pt][c]{#1}}
\newcommand{\zrbox}[1]{\makebox[0pt][r]{#1}}
\newcommand{\zzbox}[2][bl]{\raisebox{.6em}{\begin{picture}(0,0)(0,0)\put(0,0){\makebox(0,0)[#1]{#2}}\end{picture}}}  %% text box with 0 width and 0 height
\newcommand{\zzboxb}[2][bl]{\begin{picture}(0,0)(0,0)\put(0,0){\makebox(0,0)[#1]{#2}}\end{picture}}  %% text box with 0 width and 0 height
\newcommand{\xbox}[3][c]{\settowidth{\xboxwidth}{#2}\makebox[\xboxwidth][#1]{#3}}
\newcommand{\pile}[2][c]{\begin{tabular}[#1]{@{}c@{}}#2\end{tabular}}
\newcommand{\piler}[2][c]{\begin{tabular}[#1]{@{}r@{}}#2\end{tabular}}
\newcommand{\pilel}[2][c]{\begin{tabular}[#1]{@{}l@{}}#2\end{tabular}}
\newcommand{\mpile}[2][c]{\ensuremath{\begin{array}[#1]{@{}c@{}}#2\end{array}}}
\newcommand{\PreserveBackslash}[1]{\let\temp=\\#1\let\\=\temp}
\let\PBS=\PreserveBackslash
\newcommand{\Ftrad}{\ensuremath{\mathfrak{F}}}
\newcommand{\Ftradf}[1]{\ensuremath{\Ftrad(\text{#1})}}
\newcommand{\Ftradi}[1]{\ensuremath{\Ftrad(\sicut{#1})}}
\newcommand{\atoi}{\raisebox{\totalheight}{\rotatebox{180}{$\iota$}}} %\newcommand{\atoi}{℩}
\newcommand{\eVide}{\ensuremath{\varnothing}}		% 
\newcommand{\Evide}{\ensuremath{\varnothing}}
\newcommand{\doit}{\ensuremath{\Box}}		% op. modal []
\newcommand{\peut}{\ensuremath{\Diamond}}	% op. modal <>
\newcommand{\doitn}[1]{\ensuremath{{\scriptstyle[#1]}}}		% op. modal []
\newcommand{\peutn}[1]{\ensuremath{{\scriptstyle\langle#1\rangle}}}	% op. modal <>
\newcommand{\cf}{cf.\ }
\newcommand{\powerset}{\ensuremath{\mathalpha{\mathscr{P}}}}
\newcommand{\inclus}{\ensuremath{\subseteq}}
\newcommand{\Vers}{\ensuremath{\longrightarrow}}		% ->
\newcommand{\Stag}[1]{\ensuremath{_{\text{#1}}}}
\newcommand{\Azero}{\ensuremath{\emptyset}}
\newcommand{\rond}{\ensuremath{\mathbin{\circ}}}
\newcommand{\orond}{\ensuremath{\mathord{\circ}}}
\newcommand{\himn}[1]{\ensuremath{\textup{\textsc{h}}_{#1}}}
%\newcommand{\QUIN}[3]{\ensuremath{{}_{#1}(\text{#2},\text{#3})}}
\newcommand{\QUIN}[3]{${}_{#1}(${#2}$,\,${#3}$)$}

\newcommand{\colonrel}{%
  \DeclareMathSymbol{;}{\mathrel}{operators}{`;}}
%\DeclareMathSymbol{;}{\mathrel}{operators}{"3B}}
\newcommand{\colonpunct}{%
\DeclareMathSymbol{;}{\mathpunct}{operators}{"3B}}
\newcommand{\colonord}{%
\DeclareMathSymbol{;}{\mathord}{operators}{"3B}}
\colonrel


% Des espaces blancs horizontaux, verticaux, et carrés.
\newcommand{\hstrab}[1][1ex]{\rule{#1}{0pt}}
\newcommand{\vstrab}[1][1ex]{\rule{0pt}{#1}}
\newcommand{\cstrab}[1][1ex]{\rule{#1}{0pt}\rule{0pt}{#1}}


% Sem
%\newcommand{\satisf}{\ensuremath{\models}}
\newcommand{\satisf}{\ensuremath{\mathrel|\mathrel{\mkern -3.5mu}=}}
\newcommand{\trad}{\ensuremath{\leadsto}}
\newcommand{\rtrad}{\ensuremath{\reflectbox{\ensuremath{\leadsto}}}}
\newcommand{\Unv}[1]{\ensuremath{\mathcal{#1}}}
\newcommand{\VAR}{\ensuremath{\mathcal{V}\kern-.12em\textit{ar}}}
\newcommand{\CON}{\ensuremath{\mathcal{C}\kern-.12em\textit{ns}}}
\newcommand{\IND}{\ensuremath{\mathcal{I}\kern-.12em\textit{nd}}}
%\newcommand{\obj}[1]{\ensuremath{\textsc{#1}}}
\newcommand{\textobj}[1]{\textsc{\upshape#1}}
\newcommand{\Obj}[1]{\ensuremath{\textsc{#1}}}
\newcommand{\Obji}[2]{\ensuremath{\textsc{#1}_{#2}}}
\newcommand{\FI}{\ensuremath{{F}}}  % Fct d'interprétation
\newcommand{\Modele}{\ensuremath{\mathcal{M}}}
\newcommand{\denote}[1]{%
   \ensuremath{\dlb{#1}\drb}}  %% de newtxmath
\newcommand{\oden}{\ensuremath{\dlb}}
\newcommand{\fden}{\ensuremath{\drb}}
\newcommand{\doublevee}{\vee\!\!\vee}
\newcommand{\xou}{\ensuremath{\mathbin{\doublevee}}}
\renewcommand{\phi}{\varphi}
\newcommand{\Tps}{\ensuremath{\mathcal{I}}}
\newcommand{\tprec}{\ensuremath{<}}
\newcommand{\tpreceq}{\ensuremath{\leqslant}}
%\newcommand{\mOp}[1]{\ensuremath{\mathord{\textup{\textbf{\textsf{#1}}}}}}
\newcommand{\mOp}[1]{\xlo{\ensuremath{\mathord{\textup{\textsf{#1}}}}}}
\newcommand{\mF}{\mOp{F}}
\newcommand{\mP}{\mOp{P}}
\newcommand{\mG}{\mOp{G}}
\newcommand{\mH}{\mOp{H}}
\newcommand{\indw}[1]{\ensuremath{\mathrm{#1}}}
\newcommand{\w}{\indw{w}}
\newcommand{\wo}{\ensuremath{\indw{w}_0}}  %% le monde réel
\newcommand{\RK}{\ensuremath{\mathrel{R}}}
\newcommand{\tq}{\ensuremath{\,|\,}}
\newcommand{\extn}[1]{\ensuremath{\check{}#1}}	% 
\newcommand{\Extn}{\ensuremath{\mathord{^{\scriptscriptstyle\vee}}\kern-.1em}}	
\newcommand{\textExtn}{\ensuremath{\mathord{^{\scriptscriptstyle\vee}}}}	
%\newcommand{\Types}{\ensuremath{\mathbf{T}}}
\newcommand{\Types}{\ensuremath{\textbf{T}}}
\newcommand{\TypesB}{\ensuremath{\Types_{\textsc b}}}
\newcommand{\type}[1]{\ensuremath{\tuple{\mathrm{#1}}}}
\newcommand{\typ}[1]{\ensuremath{\mathrm{#1}}}
\newcommand{\mtype}[1]{\ensuremath{\tuple{#1}}}
\newcommand{\mtyp}[1]{\ensuremath{\mathit{#1}}}
\newcommand{\typeevent}{a}
\newcommand{\typev}{\typ\typeevent}
\newcommand{\tev}{\typev}
\newcommand{\et}{\type{e,t}}
\newcommand{\ett}{\type{\et,t}}
\newcommand{\eet}{\type{e,\et}}
\newcommand{\utyp}[2]{\ensuremath{\underbrace{#1}_{#2}}}
\newcommand{\ME}{\ensuremath{\textrm{M\!E}}}
\newcommand{\DoM}{\ensuremath{\mathcal{D}}} % Domaine(s)
\newcommand{\Rel}[1][]{\ensuremath{\mathrel{R_{#1}}}}
\newcommand{\Deux}{\ensuremath{\mathbf{2}}}
\newcommand{\DEUX}{\ensuremath{\set{0;1}}}
\newcommand{\Df}{\ensuremath{\mathord{:}}}
\newcommand{\Tyz}{\ensuremath{\text{Ty}{\text 2}}}
\newcommand{\Fz}[1][]{\ensuremath{\mathfrak f_{#1}}}

\newcommand{\reecr}{\ensuremath{\longrightarrow}}
\newcommand{\seecr}{\ensuremath{\longleftarrow}}
 
\newcommand{\intn}[1]{\ensuremath{\hat{}#1}}	% 
\newcommand{\Intn}{\ensuremath{\mathord{^{\scriptscriptstyle\wedge}}\kern-.1em}} 
\newcommand{\textIntn}{\ensuremath{\mathord{^{\scriptscriptstyle\wedge}}}}	 

\newcommand{\Cantor}{\ensuremath{\scriptscriptstyle\euscr C}}
\newcommand{\Ch}[1]{\ensuremath{#1_{\Cantor}}}
\newcommand{\fsynsem}[1]{\fcolorbox{black!75!white}{gray!10}{#1}}
\newcommand{\RISS}[2]{\begin{tabular}[t]{@{}l@{}}{\,\fonttitre#1}\\\fsynsem{#2}\end{tabular}}

\newcommand{\QRa}{\alien{QR}}

\newcommand{\sysind}[3]{\ensuremath{[%
#1\mathord{:}%
#2\mathord{:}%
#3%
]}}
\newcommand{\icat}[4]{\ensuremath{%
\text{#1}_{\sysind{#2}{#3}{#4}}}}

\newcommand{\epsi}{\ensuremath{\varepsilon}}
\newcommand{\Card}[1]{\ensuremath{\lvert#1\rvert}}
%\newcommand{\nbr}[1]{\ensuremath{\boldsymbol{#1}}}
\newcommand{\nbr}[1]{\ensuremath{\textbf{#1}}}
\newcommand{\gands}{\ensuremath{\mathaccent\ldotp\cap}}
\newcommand{\gand}{\ensuremath{\mathbin{\mathaccent\ldotp\cap}}}
\newcommand{\gor}{\ensuremath{\mathbin{\mathaccent\cdot\cup}}}
\newcommand{\gneg}{\ensuremath{\mathord{\sim}}}
\newcommand{\texttype}[1]{\ensuremath{\tuple{\text{#1}}}}

\newcommand{\RegleSyntaxique}{Syn.}
\newcommand{\RegleSemantique}{Sém.}
\newcommand{\RSyn}{{\small \RegleSyntaxique}}
\newcounter{RglSynt}
\newcommand{\RSem}{{\small \RegleSemantique}}
\newcounter{RglSem}

\newcommand{\GN}{GN}
\newcommand{\GV}{GV}
\newcommand{\LO}{LO}
\newcommand{\Lo}{LO}
\newcommand{\LOz}{\ensuremath{\text{LO}_{2}}}
\newcommand{\Loz}{\ensuremath{\text{LO}_{2}}}

\newcommand{\mcovar}{$^{\textsc c}$}
\newcommand{\jcovar}{\juge{\mcovar}}
\newcommand{\mambig}{$^{\textsc a}$}
\newcommand{\jambig}{\juge{\mambig}}


\newcommand{\textsb}[1]{\textbf{#1}}
\newcommand{\textbm}[1]{\textsb{#1}} %% avec cfr-lm


\colorlet{LOcolor}{blue!45!black}
%\definecolor{vert}{named}{DarkGreen}
\newcommand{\xlo}[1]{\textcolor{LOcolor}{#1}}
\newcommand{\Xlo}{\color{LOcolor}}
%\colorlet{LO2color}{black!79!green!90!olive}
\colorlet{LO2color}{black!78!green!91!yellow}
\newcommand{\xloz}[1]{\textcolor{LO2color}{#1}}
\newcommand{\Xloz}{\color{LO2color}}

\newcommand{\cns}[1]{\xlo{\ensuremath{\text{\textbm{#1}}}}}
%\newcommand{\cnsi}[2]{\xlo{\ensuremath{\cns{#1}_{\mathbf{#2}}}}}
\newcommand{\cnsi}[2]{\xlo{\ensuremath{\cns{#1}_{\bm{#2}}}}}
\newcommand{\prd}[1]{\xlo{\ensuremath{\text{\textbm{#1}}}}}
%\newcommand{\prdi}[2]{\xlo{\ensuremath{\prd{#1}_{\mathbf{#2}}}}}
\newcommand{\prdi}[2]{\xlo{\ensuremath{\prd{#1}_{\bm{#2}}}}}
\newcommand{\prdx}[1]{\xlo{\ensuremath{\prd{#1}_{\boldsymbol{*}}}}}
\newcommand{\prdk}[1]{\xlo{\ensuremath{\text{\textbm{\scshape#1}}}}}
\newcommand{\prdki}[2]{\xlo{\ensuremath{\text{\textbm{\scshape#1}}_{#2}}}}
\newcommand{\vrb}[1]{\xlo{\ensuremath{#1}}}
\newcommand{\vrbi}[2]{\xlo{\ensuremath{#1_{#2}}}}
\newcommand{\vrbS}[1]{\xlo{\ensuremath{\euscr{#1}}}}
\newcommand{\vrbSz}[1]{\xloz{\ensuremath{\euscr{#1}}}}
%\newcommand{\vrbP}[1]{\xlo{\ensuremath{\mathcal{#1}}}}
%% Pour Ty2 :
\newcommand{\cnsz}[1]{\xloz{\ensuremath{\text{\textbm{#1}}}}}
\newcommand{\cnszi}[2]{\xloz{\ensuremath{\cnsz{#1}_{\mathbf{#2}}}}}
\newcommand{\prdz}[1]{\xloz{\ensuremath{\text{\textbm{#1}}}}}
\newcommand{\prdzi}[2]{\xloz{\ensuremath{\prdz{#1}_{\mathbf{#2}}}}}
\newcommand{\prdzz}[1]{\xloz{\ensuremath{\text{\textbm{#1}}'}}}
\newcommand{\prdzzi}[2]{\xloz{\ensuremath{\prdz{#1}'_{\mathbf{#2}}}}}
\newcommand{\vrbz}[1]{\xloz{\ensuremath{#1}}}
\newcommand{\vrbzi}[2]{\xloz{\ensuremath{#1_{#2}}}}

\makeatletter
\newcommand{\urgh}{\ensuremath{{}^{\text{\nofrench@punctuation\symbol{"3F}\french@punctuation}}}}
\newcommand{\uurgh}{\ensuremath{{}^{\text{\nofrench@punctuation\symbol{"3F}\symbol{"3F}\french@punctuation}}}}
\makeatother
\newcommand{\zarb}{\ensuremath{{}^{\text{\#}}}}

\newcommand{\lamb}{\ensuremath{\lambda}}
\newcommand{\lterme}{$\lambda$-terme}
\newcommand{\labstraction}{$\lambda$-abstraction}
\newcommand{\lcalcul}{$\lambda$-calcul}
\newcommand{\breduc}{$\beta$-réduction}
\newcommand{\dere}{\alien{de re}}
\newcommand{\dedicto}{\alien{de dicto}}

\newcommand{\ulambda}{\textit{λ}}
\newcommand{\ubeta}{\textit{β}}

\newcommand{\stx}{\rule{0pt}{1ex}}
\newcommand{\taquet}[2]{\hfill\zbox{#2}\rule{#1}{0pt}}

\newcommand{\smidrule}{\arrayrulecolor{gray}\midrule\arrayrulecolor{black}}

\newcounter{lastexo}
\newenvironment{exolist}%
  {\begin{enumerate}}%
%  {\begin{enumerate}\renewcommand{\theenumi}{\alph{enumi}}\renewcommand{\labelenumi}{(\theenumi)}}%
  {\setcounter{lastexo}{\value{enumi}}\end{enumerate}}

\newcommand{\contexo}{\setcounter{enumi}{\value{lastexo}}}


  \newcommand{\juge}[1]{\AddInfo{#1}\recTestForGramm} %% complète linguex.sty
  \renewcommand{\firstrefdash}{} %% pour linguex.sty
  \renewcommand{\philarge}{4.5\mindigitwidth}
  \newcommand{\ExNBP}{\setlength{\Extopsep}{.35\baselineskip}\setlength{\Exindent}{1.5em}\setlength{\Exlabelsep}{.85em}\setlength{\SubExleftmargin}{1.6em}} %% correctif pour les exemples en footnotes

% Keywords
\newcommand{\textkw}[1]{\textsc{#1}}
%\newcommand{\kw}[1]{\textkw{#1}\index[sbj]{#1}}
%\newcommand{\kwi}[2]{\textkw{#1}\index[sbj]{#2}}
%\newcommand{\kwa}[2]{\textkw{#1}\index[sbj]{#2@#1}}
\newcommand{\kw}[1]{\textkw{#1}\indexs{#1}}
\newcommand{\kwi}[2]{\textkw{#1}\indexs{#2}}
\newcommand{\kwa}[2]{\textkw{#1}\indexs{#2@#1}}
\newcommand{\kwo}[1]{\textkw{#1}}
\newcommand{\kwab}[1]{\textsc{#1}}  % for abbreviations.tex


%% Définitions theorems
% -*- coding: utf-8 -*-
%%% Déclarations theorem %%%%%%%%%%%%%%%%%%%%%%%%%%%%
\newcommand{\fonttitre}{}

\definecolor{deficolor}{rgb}{0.88,0.88,0.88}
\definecolor{theocolor}{rgb}{0.95,0.95,0.95}
%\colorlet{colorexo}{SteelBlue4!60!black}
\colorlet{colorexo}{black}
\newcommand{\colexo}[1]{\textcolor{colorexo}{#1}}


%% Repris de langsci-tbls.sty
\mdfsetup{skipabove=\baselineskip,skipbelow=\baselineskip,frametitlefont=\bfseries, needspace=4.5\baselineskip, splittopskip=1.5\baselineskip, frametitlealignment=\raggedright}
\mdfsetup{apptotikzsetting={\tikzset{mdfbackground/.append style={draw=none}}}}


\mdfdefinestyle{defstyle}{%
   frametitlefont=\normalsize\sffamily\bfseries,
   hidealllines=true, backgroundcolor=deficolor,
   frametitleaboveskip=5mm, frametitlebelowskip=0mm, frametitlerule=false,
   innerleftmargin=5mm, innerrightmargin=5mm, innerbottommargin=5mm, innertopmargin=1mm
}

\mdfdefinestyle{theostyle}{%
   frametitlefont=\normalsize\sffamily\bfseries,
   hidealllines=true, backgroundcolor=theocolor,
   frametitleaboveskip=5mm, frametitlebelowskip=0mm, frametitlerule=false,
   innerleftmargin=5mm, innerrightmargin=5mm, innerbottommargin=5mm, innertopmargin=1mm
}


%\theorembodyfont{\rmfamily}
%\theoremheaderfont{\color{black}\fonttitre\bfseries}
\mdtheorem[style=defstyle]{defi}{Définition}[chapter]
\mdtheorem[style=theostyle]{princ}{Principe}[chapter]
\mdtheorem[style=theostyle]{point}{Point}[chapter]
\mdtheorem[style=theostyle]{postu}{Postulat}[chapter]
\mdtheorem[style=theostyle]{prop}{Proposition}[chapter]
\mdtheorem[style=theostyle]{theo}{Théorème}[chapter]
\mdtheorem[style=theostyle]{coro}{Corollaire}[chapter]
\mdtheorem[style=theostyle]{nota}{Notation}[chapter]
\mdtheorem[style=theostyle]{conv}[nota]{Convention de notation}

%% Pour les exercices :
%\addtolength{\theorempostskipamount}{-4pt}
\theoremstyle{break}
%\setlength{\theorempreskipamount}{0pt}\setlength{\theorempostskipamount}{0pt}
\theoremheaderfont{\fonttitre\bfseries\color{colorexo}}
\theoremindent1.3cm\theorembodyfont{\small}
\mdfdefinestyle{exostyle}{%
hidealllines=true,
font=\small,
frametitlefont=\bfseries,
frametitlebelowskip=0mm,
	innerleftmargin=13mm, innerrightmargin=0mm, innerbottommargin=5mm,
}

%\mdtheorem[style=exostyle]{exo}{Exercice}[chapter]
\newtheorem{exo}{Exercice}[chapter]





% Pour faire plus facilement des centrages
\def\hfl{\hspace*{\fill}}
\def\vfl{\vspace*{\fill}}

%%  Couper une formule inline 
\newcommand{\coupe}{\raggedright\linebreak[4]\rule{.6em}{0pt}}


%%% Symbole d'élision pour les sous-entrées de l'index 
%%% ==================================================

\newcommand{\elid}{\ensuremath{\sim}}


% Solutions avec answers.sty
\Newassociation{solu}{Solution}{solf}
%\renewcommand{\Solutionlabel}[1]{{\color{couleurtitre}\fonttitre\large\bfseries Exercice #1}}
\renewcommand{\Solutionlabel}[1]{{\bfseries Exercice #1}}

\newcommand{\pagesolution}[1]{}

%% Macros pour exercices sur portées, variables libres et liées
\newlength{\origfboxsep}\setlength{\origfboxsep}{\fboxsep}
\newcommand{\fscp}[1]{\setlength{\fboxsep}{2pt}\fcolorbox{darkgray}{white}{\ensuremath{#1}}\setlength{\fboxsep}{\origfboxsep}}
\newcommand{\vfree}[1]{{\textcolor{green!30!black}{\boldsymbol#1}}}

\newcommand{\beq}{\ensuremath{\mathrel{\textcolor{black}{=}}}}


%%
% Jolis boîtes avec pst
\newcommand{\boitealu}[1]{\psframebox[framearc=.2,fillstyle=gradient,gradbegin=lightgray,gradend=white,gradmidpoint=0,linecolor=gray]{\vphantom{Ij}#1}}

%% Couleurs
\colorlet{Red2}{red!90!darkgray}

%%%
%%% Biblatex tuning
%%% ===============
\ExecuteBibliographyOptions{related=true}

\DefineBibliographyExtras{french}{\restorecommand\mkbibnamefamily} % pour enlever les smallcaps
\DeclareCaseLangs*{french}

\DefineBibliographyStrings{french}{%
bibliography = {Références bibliographiques},
  volume = {vol.},  %% "t." for "tome" looks weird
  volumes = {vol.},
}

\renewbibmacro*{institution+location+type+date}{%
%% reordering 'type, loc : inst'; more natural in French
  \printfield{type}%
  \setunit{\addcomma\space}% 
  \printlist{location}%
  \iflistundef{institution}
    {}
    {\setunit*{\addcolon\space}}%
  \printlist{institution}%
  \setunit*{\addcomma\space}%
  \usebibmacro{date}%
  \newunit}

\defbibheading{references}{\chapter{Références bibliographiques}}


%%% ===============




%% Font tengwar (Exercice chap 2)
\newfontfamily{\telcontar}[Renderer=Graphite]{Tengwar Telcontar}
%% http://freetengwar.sourceforge.net/tengtelc.html
\newcommand{\texttw}[1]{{\telcontar#1}}


%% Set references manually (-> vol. 2)
\makeatletter
\newcommand{\manuallabel}[2]{\def\@currentlabel{#2}\label{#1}}
\makeatother


%% Patch for fixing a bug in gloss-french.ldf with TeXlive 2016
\makeatletter
\patchcmd\french@punctuation{255}{4095}{}{}
\patchcmd\french@punctuation{255}{4095}{}{}
\makeatother

% Second level in enumerate
\renewcommand{\labelenumii}{\theenumii.}

% French "bullets" and French spacing in itemize
\setlist[itemize,1]{label=--,noitemsep,topsep=.3ex}
\setlist[enumerate]{noitemsep,topsep=.5ex}

\SetEnumitemKey{syn}{label=(\RSyn\arabic*),ref=\arabic*,leftmargin=3.5em} %% with enumitem.sty v3.0
\SetEnumitemKey{synz}{label=(\RSyn$'$\arabic*),ref=\arabic*,leftmargin=3.5em} 
\SetEnumitemKey{sem}{label=(\RSem\arabic*),ref=\arabic*,leftmargin=3.5em} 
\SetEnumitemKey{semi}{label=(\RSem\arabic*$'$),ref=\arabic*,leftmargin=3.5em} 
\SetEnumitemKey{semz}{label=(\RSem$'$\arabic*),ref=\arabic*,leftmargin=3.5em} 


\renewcommand{\leq}{\leqslant}
\renewcommand{\geq}{\geqslant}


%%% ----------debut de bigcenter.sty--------------
\makeatletter
%%% nouvel environnement bigcenter
%%% pour centrer sur toute la page (sans overfull)

\newskip\@bigflushglue \@bigflushglue = -100pt plus 1fil

\def\bigcenter{\trivlist \bigcentering\item\relax}
\def\bigcentering{\let\\\@centercr\rightskip\@bigflushglue%
\leftskip\@bigflushglue
\parindent\z@\parfillskip\z@skip}
\def\endbigcenter{\endtrivlist}

%%% ----------fin de bigcenter.sty--------------
\makeatother
