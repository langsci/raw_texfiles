% -*- coding: utf-8 -*-

Depuis près d'un demi-siècle, la sémantique formelle, à l'intersection féconde de la linguistique, la logique et la philosophie, occupe une place de premier plan dans le domaine de la linguistique théorique.  Ce manuel, le premier en français de cette ampleur, en propose une présentation à la fois introductive et approfondie.  La sémantique formelle est parfois appréhendée avec hésitation (voire redoutée) du fait de sa rigueur mathématique et de l'apparente complexité de son symbolisme ; le présent ouvrage vise à en donner une approche pédagogique, stimulante et dédramatisée.  Il est conçu pour offrir un apprentissage et une maîtrise autonomes et progressifs des principaux aspects conceptuels et formels de la théorie, en accordant une attention particulière aux innovations marquantes qui ont structuré le domaine depuis les travaux fondateurs de R. Montague.  Il a également pour objectif de proposer, au fil des pages, une introduction à des éléments de méthodologie scientifique afin de donner un aperçu des pratiques d'analyse propres à la sémantique formelle. 

\sloppy

Les six chapitres, accompagnés de nombreux exercices corrigés, y abordent les notions de base de logique et de sémantique vériconditionnelle, les phénomènes de quantification, la logique intensionnelle appliquée à la temporalité et les modalités, le $\lambda$-calcul typé et l'analyse sémantique compositionnelle à l'interface syntaxe-sémantique.   Le volume sera complété d'un second qui présentera plusieurs applications et développements qui étendent la portée du formalisme.

\fussy

L'ouvrage s'adresse aux étudiants de licence et master de sciences de langage, mais aussi d'autres disciplines comme, par exemple, la philosophie, la logique, l'informatique, ainsi qu'aux chercheurs désireux de s'initier, se mettre à jour ou se perfectionner dans la discipline.
