% -*- coding: utf-8 -*-
\begin{exo}\label{exo:[]}
\begin{enumerate}
\item Montrez, par la méthode des tables de vérité, 
\pagesolution{crg:[]}%
que $\Xlo[[\phi \wedge \psi]
\wedge \chi]$ et  $\Xlo[\phi \wedge [\psi \wedge \chi]]$ sont logiquement
équivalentes.  NB : ici les tables auront 8 lignes.
\item
De même pour $\Xlo[[\phi \vee \psi] \vee \chi]$ et  $\Xlo[\phi
  \vee [\psi \vee \chi]]$. 
\item
Montrez que $\Xlo[[\phi \implq \psi] \implq \chi]$ et $\Xlo[\phi
  \implq [\psi \implq \chi]]$ \emph{ne sont pas} logiquement
  équivalentes.
\end{enumerate}
\begin{solu} (p.~\pageref{exo:[]})\label{crg:[]}

Nous dressons les tables de vérité (\S\ref{TabV}, p.~\pageref{TabV}) des formules de chaque paire et nous comparons les colonnes de résultats.
\begin{enumerate}
\item Tables de vérité de $\Xlo[[\phi \wedge \psi] \wedge \chi]$ 
et de $\Xlo[\phi \wedge [\psi \wedge \chi]]$ :
\small\[
\begin{array}{c|c|c||c|>{\columncolor[gray]{.9}}c||c|>{\columncolor[gray]{.9}}c}
\Xlo\phi & \Xlo\psi & \Xlo\chi & \Xlo[\phi \wedge \psi] & \cellcolor{white}\Xlo[[\phi \wedge \psi] \wedge
  \chi] & \Xlo[\psi \wedge \chi] & \cellcolor{white}\Xlo[\phi \wedge [\psi \wedge \chi]]
\\\hline\hline
1 & 1 & 1 & 1 & 1 & 1 & 1\\
1 & 1 & 0 & 1 & 0 & 0 & 0\\
1 & 0 & 1 & 0 & 0 & 0 & 0\\
1 & 0 & 0 & 0 & 0 & 0 & 0\\
0 & 1 & 1 & 0 & 0 & 1 & 0\\
0 & 1 & 0 & 0 & 0 & 0 & 0\\
0 & 0 & 1 & 0 & 0 & 0 & 0\\
0 & 0 & 0 & 0 & 0 & 0 & 0\\
\end{array}
\]\normalsize
Les deux colonnes de résultats sont identiques, donc les deux formules
sont bien équivalentes.

\item Tables de vérité de 
 $\Xlo[[\phi \vee \psi] \vee \chi]$ et  $\Xlo[\phi
  \vee [\psi \vee \chi]]$: 
\small\[
\begin{array}{c|c|c||c|>{\columncolor[gray]{.9}}c||c|>{\columncolor[gray]{.9}}c}
\Xlo\phi & \Xlo\psi & \Xlo\chi & \Xlo[\phi \vee \psi] & \cellcolor{white}\Xlo[[\phi \vee \psi] \vee
  \chi] & \Xlo[\psi \vee \chi] & \cellcolor{white}\Xlo[\phi \vee [\psi \vee \chi]]
\\\hline\hline
1 & 1 & 1 & 1 & 1 & 1 & 1\\
1 & 1 & 0 & 1 & 1 & 1 & 1\\
1 & 0 & 1 & 1 & 1 & 1 & 1\\
1 & 0 & 0 & 1 & 1 & 0 & 1\\
0 & 1 & 1 & 1 & 1 & 1 & 1\\
0 & 1 & 0 & 1 & 1 & 1 & 1\\
0 & 0 & 1 & 0 & 1 & 1 & 1\\
0 & 0 & 0 & 0 & 0 & 0 & 0\\
\end{array}
\]\normalsize
Même conclusion que précédemment.

\item Tables de vérité de
 $\Xlo[[\phi \implq \psi] \implq \chi]$ et $\Xlo[\phi
  \implq [\psi \implq \chi]]$ :
\small\[
\begin{array}{c|c|c||c|>{\columncolor[gray]{.9}}c||c|>{\columncolor[gray]{.9}}c}
\Xlo\phi & \Xlo\psi & \Xlo\chi & \Xlo[\phi \implq \psi] & \cellcolor{white}\Xlo[[\phi \implq \psi] \implq
  \chi] & \Xlo[\psi \implq \chi] & \cellcolor{white}\Xlo[\phi \implq [\psi \implq \chi]]
\\\hline\hline
1 & 1 & 1 & 1 & 1 & 1 & 1\\
1 & 1 & 0 & 1 & 0 & 0 & 0\\
1 & 0 & 1 & 0 & 1 & 1 & 1\\
1 & 0 & 0 & 0 & 1 & 1 & 1\\
0 & 1 & 1 & 1 & 1 & 1 & 1\\
0 & 1 & 0 & 1 & \cellcolor[gray]{.8}0 & 0 & \cellcolor[gray]{.8}1\\
0 & 0 & 1 & 1 & 1 & 1 & 1\\
0 & 0 & 0 & 1 & \cellcolor[gray]{.8}0 & 1 & \cellcolor[gray]{.8}1\\
\end{array}
\]\normalsize
Les deux colonnes de résultats sont différentes, donc les deux
formules ne sont pas équivalentes.
\end{enumerate}
\end{solu}
\end{exo}
