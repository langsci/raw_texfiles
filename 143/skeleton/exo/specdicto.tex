% -*- coding: utf-8 -*-
\begin{exo} \label{exo:specdicto}
\begin{solu}(p.~\pageref{exo:specdicto})\label{crg:specdicto}

Dans la phrase (\ref{x:specpourqui}), \sicut{un vampire} à forcément un usage spécifique du fait de l'apposition.  D'après la définition de la spécificité (p.~\pageref{def:spécificité}), nous conclurons que le locuteur pense à un certain vampire, et plus précisément Dracula, et affirme qu'Alice pense que ce vampire l'a mordue.  Mais il y a, par ailleurs, une façon très naturelle de comprendre (\ref{x:specpourqui}) selon laquelle c'est avant tout Alice qui pense à un vampire particulier, Dracula, et pense qu'il l'a mordue.  Autrement dit \sicut{un vampire} s'avère d'abord spécifique pour Alice avant de l'être pour locuteur.  Foncièrement cela ne change probablement pas grand chose pour l'analyse sémantique de la phrase (si le \GN\ est spécifique pour Alice, il l'est a fortiori aussi pour le locuteur), mais du point de vue de son adéquation descriptive,   l'analyse passe un peu à côté de cette caractéristique qui est que la source de la spécificité n'est pas toujours le fait du locuteur.
\end{solu}
Quel problème peut éventuellement 
\pagesolution{crg:specdicto}%
poser la phrase ci-dessous vis-à-vis de la définition de la
spécificité vue dans cette section ?
%%Spécifique pour qui ?

\begin{enumerate}
\item \label{x:specpourqui}
Alice croit qu'un vampire, à savoir Dracula en personne, l'a mordue
pendant la nuit.
\end{enumerate}
\end{exo}
