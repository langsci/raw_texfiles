% -*- coding: utf-8 -*-
\begin{exo}\label{exo:2denot}
Calculez la dénotation relativement à $\Modele_1$ des formules
suivantes : \pagesolution{crg:2denot}
\begin{enumerate}
\item \(\Xlo[\prd{père-de}(\cns o,\cns p) \ssi \prd{elfe}(\cns b)]\)
\item \(\Xlo[\neg\prd{aimer}(\cns d, \cnsi{h}{3}) \implq \prd{triste}(\cnsi{h}{3})]\)
\item \(\Xlo\neg [\prd{elfe}(\cns p) \wedge \prd{farceur}(\cns p)]\)
\item \(\Xlo[\prd{farceur}(\cnsi t1) \implq [\prd{elfe}(\cnsi t1) \implq
    \prd{âne}(\cnsi t1)]]\)
\end{enumerate}
\begin{solu}
(p.~ \pageref{exo:2denot})\label{crg:2denot}

Les dénotations sont calculées en appliquant les règles d'interprétation de la définition \ref{d:Sem1}, p.~\pageref{d:Sem1}.

\begin{enumerate}
\item \(\Xlo[\prd{père-de}(\cns o,\cns p) \ssi \prd{elfe}(\cns b)]\)

\sloppy
La règle  (\RSem\ref{RIcon}d) nous dit que cette formule est vraie dans $\Modele_1$ ssi \(\Xlo\prd{père-de}(\cns o,\cns p)\) et \(\Xlo\prd{elfe}(\cns b)\) ont la même dénotation. \(\denote{\Xlo\prd{père-de}(\cns o,\cns p)}^{\Modele_1}=0\) car \(\tuple{\Obj{Obéron},\Obj{Puck}} \not\in \FI_1(\prd{père-de})\) ; et 
\(\denote{\Xlo\prd{elfe}(\cns b)}^{\Modele_1}=0\) car $\Obj{Bottom}\not\in\FI_1(\prd{elfe})$.  Par conséquent, la formule est vraie dans $\Modele_1$.

\fussy

\item \(\Xlo[\neg\prd{aimer}(\cns d, \cnsi{h}{3}) \implq \prd{triste}(\cnsi{h}{3})]\)

Selon la règle (\RSem\ref{RIcon}c), cette formule est vraie dans $\Modele_1$ ssi 
$\Xlo\neg\prd{aimer}(\cns d, \cnsi{h}{3})$ est fausse \emph{ou} 
$\Xlo\prd{triste}(\cnsi{h}{3})$ est vraie.
Il se trouve que nous remarquons assez rapidement que $\Xlo\prd{triste}(\cnsi{h}{3})$ est vraie dans $\Modele_1$, car $\Obj{Héléna}\in\FI_1(\prd{triste})$ (sachant que \cnsi h3 dénote \Obj{Héléna}). Cela suffit à montrer que la formule est vraie dans $\Modele_1$.

\sloppy

\item \(\Xlo\neg [\prd{elfe}(\cns p) \wedge \prd{farceur}(\cns p)]\)

Cette formule est d'abord une négation, donc nous calculons sa dénotation en consultant d'abord la règle (\RSem\ref{RIneg}) qui dit que la formule est vraie dans $\Modele_1$ ssi \(\Xlo [\prd{elfe}(\cns p) \wedge \prd{farceur}(\cns p)]\) est fausse dans $\Modele_1$. Or \(\denote{\Xlo\prd{elfe}(\cns p)}^{\Modele_1} = 1\) car $\Obj{Puck}\in\FI_1(\prd{efle})$, et
\(\denote{\Xlo\prd{farceur}(\cns p)}^{\Modele_1} = 1\) car $\Obj{Puck}\in\FI_1(\prd{farceur})$.  Donc, en vertu de la règle (\RSem\ref{RIcon}a), 
\(\denote{\Xlo\prd{farceur}(\cns p) \wedge \prd{farceur}(\cns p)}^{\Modele_1} = 1\), et nous en concluons que 
\(\denote{\Xlo\neg[\prd{farceur}(\cns p) \wedge \prd{farceur}(\cns p)]}^{\Modele_1} = 0\).

\fussy

\item \(\Xlo[\prd{farceur}(\cnsi t1) \implq [\prd{elfe}(\cnsi t1) \implq
    \prd{âne}(\cnsi t1)]]\)

D'après (\RSem\ref{RIcon}c), la formule est vraie dans $\Modele_1$ ssi 
\(\Xlo\prd{farceur}(\cnsi t1)\) est fausse ou 
\(\Xlo[\prd{elfe}(\cnsi t1) \implq \prd{âne}(\cnsi t1)]\) est vraie.
Or \(\Xlo\prd{farceur}(\cnsi t1)\) est vraie dans $\Modele_1$ car 
$\Obj{Thésée}\in\FI_1(\prd{farceur})$. 
Nous devons calculer la dénotation de \(\Xlo[\prd{elfe}(\cnsi t1) \implq \prd{âne}(\cnsi t1)]\), qui est vraie ssi 
\(\Xlo\prd{elfe}(\cnsi t1)\) est fausse ou 
\(\Xlo\prd{âne}(\cnsi t1)\) est vraie.
Et \(\Xlo\prd{elfe}(\cnsi t1)\) est effectivement fausse dans $\Modele_1$ ($\Obj{Thésée}\not\in\FI_1(\prd{elfe})$), donc \(\Xlo[\prd{elfe}(\cnsi t1) \implq \prd{âne}(\cnsi t1)]\) est vraie, ce qui fait que la formule globale est vraie aussi.
\end{enumerate}

\end{solu}
\end{exo}
