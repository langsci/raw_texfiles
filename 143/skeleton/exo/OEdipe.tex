% -*- coding: utf-8 -*-
\begin{exo}
Cet exercice a pour but de donner la solution au problème de la
lecture de re que nous n'arrivions pas à représenter de manière
satisfaisante dans \LO. L'enjeu est de comparer minutieusement la
traduction de la lecture de dicto avec celle de la lecture de re.
Cela va nous permettre d'introduire une nouvelle contrainte très
importante sur la $\beta$-réduction.

Reprenons notre phrase.  Comme nous l'avions vu en ***,
l'interprétation de {\Next} avec la lecture de dicto de \sicut{sa$_1$
  mère} est la formule \Next[a].  

\ex. \OE dipe$_1$ voulait épouser sa$_1$ mère. 
\a. \(\Xlo\prd{vouloir}(\cns{\oe},\Intn\prd{épouser}(\cns{\oe},\atoi z\,\prd{mère}(z,\cns{\oe})))\)
\b. \(\Xlo[\lambda x\,\prd{vouloir}(\cns{\oe},\Intn\prd{épouser}(\cns{\oe},x))(\atoi z\,\prd{mère}(z,\cns{\oe}))]\)


Remarque : on a l'impression que ces deux formules sont équivalentes,
que \Last[b] se ramène à \Last[a] si on opère la $\beta$-réduction.
Mais nous allons voir que ce n'est pas le cas.  Donc gardons ces deux
formules pour le moment.

Donnons-nous un modèle-jouet \(\Modele =\tuple{\Unv{A},\Unv{W},\FI}\).

\(\Unv{A} = \set{\Obj{\OE dipe}; \Obj{Jocaste}; \Obj{Hélène}; \Obj{Ariane}}\)

\(\Unv{W} = \set{\w_1;\w_2;\w_3;\w_4}\)

Comme nous ne tenons pas compte de la temporalité, nous considérerons
que dans les mondes possibles que nous avons le désir d'\OE dipe à
épouser quelqu'un et le mariage d'\OE dipe coexistent.

\noindent
\(\FI(\w_1,\prd{mère})(\Obj{\OE dipe}) = \FI(\w_2,\prd{mère})(\Obj{\OE dipe}) = \small
\left[\begin{array}{@{}cc@{}}   
   \begin{array}{l}
   \Obj{\OE dipe}\pnode(0,.5ex){o1}\\
   \Obj{Jocaste}\pnode(0,.5ex){j1}\\
   \Obj{Hélène}\pnode(0,.5ex){h1}\\
   \Obj{Ariane}\pnode(0,.5ex){a1}
   \end{array}
& \qquad
   \begin{array}{c}
     \rnode{01}{0} \\ ~ \\ 
     \rnode{11}{1}
   \end{array}
\end{array}\right]%
\psset{nodesep=2pt}\ncline{->}{o1}{01}\ncline{->}{j1}{11}\ncline{->}{h1}{01}\ncline{->}{a1}{01}
\)

\noindent
\(\FI(\w_3,\prd{mère})(\Obj{\OE dipe}) = \small
\left[\begin{array}{@{}cc@{}}   
   \begin{array}{l}
   \Obj{\OE dipe}\pnode(0,.5ex){o2}\\
   \Obj{Jocaste}\pnode(0,.5ex){j2}\\
   \Obj{Hélène}\pnode(0,.5ex){h2}\\
   \Obj{Ariane}\pnode(0,.5ex){a2}
   \end{array}
& \qquad
   \begin{array}{c}
     \rnode{02}{0} \\ ~ \\ 
     \rnode{12}{1}
   \end{array}
\end{array}\right]%
\psset{nodesep=2pt}\ncline{->}{o2}{02}\ncline{->}{j2}{02}\ncline{->}{h2}{12}\ncline{->}{a2}{02}
\)
\hfill
\(\FI(\w_4,\prd{mère})(\Obj{\OE dipe}) = \small
\left[\begin{array}{@{}cc@{}}   
   \begin{array}{l}
   \Obj{\OE dipe}\pnode(0,.5ex){o3}\\
   \Obj{Jocaste}\pnode(0,.5ex){j3}\\
   \Obj{Hélène}\pnode(0,.5ex){h3}\\
   \Obj{Ariane}\pnode(0,.5ex){a3}
   \end{array}
& \qquad
   \begin{array}{c}
     \rnode{03}{0} \\ ~ \\ 
     \rnode{13}{1}
   \end{array}
\end{array}\right]%
\psset{nodesep=2pt}\ncline{->}{o3}{03}\ncline{->}{j3}{03}\ncline{->}{h3}{03}\ncline{->}{a3}{13}
\)



\end{exo}
