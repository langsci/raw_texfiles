% -*- coding: utf-8 -*-
\begin{exo}\label{exo:typesQA}
Quels types peut-on proposer pour $\vrb Q$ et $\vrb A$ dans chaque expression
suivante pour faire en sorte que l'expression soit à chaque fois de
type \typ t ?  Il y a à chaque fois plusieurs réponses possibles.

\begin{enumerate}
\item \(\Xlo Q(A)\)
\item \(\Xlo[\Extn Q(A)]\)
\item \(\Xlo\Extn[Q(A)]\)
\end{enumerate}

Montrez qu'il n'est pas possible de faire la même chose pour \(\Xlo Q(Q)\).

\begin{solu}  (p. \pageref{exo:typesQA})

Dans ce qui suit, $a$ et $b$ représentent des types
a priori  «inconnus». 
\begin{enumerate}
\item \(\Xlo Q(A)\) 

Comme \(\Xlo Q(A)\)  est une application fonctionnelle, c'est que $\vrb Q$
dénote une fonction et donc que $\vrb Q$ est de type \mtype{a,b}. Et on
sait que $b=\typ t$ car c'est le résultat attendu par hypothèse dans
cet exercice.  Donc $\Xlo Q$ est de type \mtype{a,\typ t}.  Quand à $a$ on
sait aussi que c'est le type de $\vrb A$, puisque $\vrb A$ est un argument
licite de $\vrb Q$.  Il n'y a pas d'autre contrainte sur son type ; on peut
prendre ce qu'on veut pour $a$, par exemple \typ e.  Ainsi $\vrb Q$ est de
type \type{e,t} et $\vrb A$ de type \typ e.  

Mais on peut aussi donner
\type{\type{e,t},t} pour $\vrb Q$ et \type{e,t} pour $\vrb A$ ; ou encore
\type{t,t} et \typ t, etc.

\item \(\Xlo[\Extn Q(A)]\)

Ici il faut reconnaître une application fonctionnelle de la fonction
$\Xlo\Extn Q$ sur l'argument $\vrb A$.  On en déduit plusieurs choses : d'abord
que $\vrb Q$ est de type \mtype{\typ s,b} pour qu'on ait le droit de lui
accoler $\Xlo\Extn$ par la règle (\RSyn\ref{SynTExt}) (p.~\pageref{SynTExt}). Et par cette règle on
sait que $b$ est le type de $\Xlo\Extn Q$.  Mais $b$ est en fait un type
complexe puisqu'on est présence d'une fonction.  Là on se ramène un
peu au cas précédent : $\Xlo\Extn Q$ doit être de type \mtype{a,\typ t}
(donc $b=\mtype{a,\typ t}$) et $a$ est le type de $\vrb A$.  Si on prend
$e$ pour le type de $\vrb A$, alors $b =\type{e,t}$, alors le type de $\vrb Q$
est \type{s,\type{e,t}} (puisqu'on a posé que ce type était
\mtype{\typ s,b}).  

Mais on peut aussi proposer par exemple \type{s,\type{\type{e,t},t}}
pour $Q$ et \type{e,t} pour $A$, etc. 

\item \(\Xlo\Extn[Q(A)]\)

Cette fois, $\Xlo\Extn$ s'accole à tout $\Xlo[Q(A)]$.  Pour que cela soit
possible, on en déduit donc que c'est $\Xlo[Q(A)]$ qui est de type
\mtype{\typ s,b}. Et donc  \(\Xlo\Extn[Q(A)]\) est de type $b$.  Mais on
sait aussi par hypothèse que $b=\typ t$.  Il reste donc à trouver les
types de $\vrb Q$ et $\vrb A$ sachant que $\Xlo[Q(A)]$ est de type \type{s,t}.  
Maintenant cela ressemble au premier cas ci-dessus : $\vrb Q$ est de type
\mtype{a,\type{s,t}}, puisque c'est la fonction qui doit retourner un
résultat de type \type{s,t} ; et $a$ est le type de $\vrb A$.  On peut le
choisir librement.  Si $a=\typ e$, alors $\vrb Q$ est de type
\type{e,\type{s,t}}. 

Si $a=\type{e,t}$, alors $\vrb Q$ est de type \type{\type{e,t},\type{s,t}}, etc.
\end{enumerate}

\end{solu}

\end{exo}
