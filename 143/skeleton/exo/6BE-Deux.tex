% -*- coding: utf-8 -*-
\begin{exo}\label{exo:BE-Deux}
Calculez $\Xlo\prdk{be}(\prd{Deux}(\prdk{enfant}))$ et 
\pagesolution{crg:BE-Deux}
décrivez la dénotation du résultat obtenu.
\begin{solu}(p.~\pageref{exo:BE-Deux})\label{crg:BE-Deux}

\sloppy

Par définition (cf.\ \ref{xd:BE}, p.~\pageref{xd:BE}), \prdk{be} est équivalent à \(\Xlo\lambda Y\lambda x [Y(\lambda y [y=x])]\). Sachant que $\Xlo\prdk{be}(\prd{Deux}(\prd{enfant}))$ est de type \ett\ (c'est un quantificateur généralisé), 
\(\Xlo\prdk{be}(\prd{Deux}(\prd{enfant}))\) peut donc se réécrire en :
\(\Xlo[\lambda Y\lambda x [Y(\lambda y [y=x])](\prd{Deux}(\prd{enfant}))]\). 
Par \breduc, cela se simplifie en 
\(\Xlo\lambda x [\prd{Deux}(\prd{enfant})(\lambda y [y=x])]\).
\(\Xlo\prd{Deux}(\prd{enfant})(\lambda y [y=x])\) est une structure tripartite de type \typ t, et sa portée \(\Xlo\lambda y [y=x]\) dénote le singleton \(\set{\denote{\vrb x}^{\Modele,w,g}}\).  Par définition, \(\Xlo\prd{Deux}(\prd{enfant})(\lambda y [y=x])\) est donc vraie ssi l'intersection de l'ensemble de tous les enfants et de l'ensemble \(\set{\denote{\vrb x}^{\Modele,w,g}}\) contient au moins deux éléments ; mais cela est, bien sûr, mathématiquement impossible puisque \(\set{\denote{\vrb x}^{\Modele,w,g}}\) ne contient qu'un seul élément.
Par conséquent, \(\Xlo\prd{Deux}(\prd{enfant})(\lambda y [y=x])\) sera toujours faux et \(\Xlo\lambda x [\prd{Deux}(\prd{enfant})(\lambda y [y=x])]\) dénotera toujours l'ensemble vide.

\fussy

\end{solu}
\end{exo}
