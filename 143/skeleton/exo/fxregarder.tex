% -*- coding: utf-8 -*-
\begin{exo}\label{exo:fxregarder}
En suivant la définition de la fonction en figure~\ref{f:regardf}, 
calculez :\pagesolution{crg:fxregarder}
%
\addtolength{\multicolsep}{-9pt}
\begin{multicols}{2}
\begin{exolist}
\item \(\denote{\Xlo\prd{regarder}(\cns a,\cns c)}^{\Modele,\w_1,g}\)
\item \(\denote{\Xlo\prd{regarder}(\cns b,\cns d)}^{\Modele,\w_1,g}\)
\item \(\denote{\Xlo\prd{regarder}(\cns d,\cns a)}^{\Modele,\w_1,g}\)
\item \(\denote{\Xlo\prd{regarder}(\cns b,\cns b)}^{\Modele,\w_1,g}\)
\end{exolist}
\end{multicols}
\begin{solu} (p.~\pageref{exo:fxregarder})\label{crg:fxregarder}

\sloppy

La figure \ref{f:regardfexo} indique graphiquement les trajets à suivre dans la fonction \(\denote{\prd{regarder}}^{\Modele,\w_1,g}\) pour chaque calcul, avec la légende suivante : 
\begin{enumerate}
\item \psline[linestyle=dotted,dotsep=2pt]{->}(0,.5ex)(1.2,.5ex)\hspace{12mm}  
donc \(\denote{\Xlo\prd{regarder}(\cns a,\cns c)}^{\Modele,\w_1,g}=0\)
\item \psline[linestyle=dashed,dash=6pt 2pt]{->}(0,.5ex)(1.2,.5ex)\hspace{12mm}
donc \(\denote{\Xlo\prd{regarder}(\cns b,\cns d)}^{\Modele,\w_1,g}=1\)
\item \psline[linestyle=dotted,linewidth=1.4pt]{->}(0,.5ex)(1.2,.5ex)\hspace{12mm}
donc \(\denote{\Xlo\prd{regarder}(\cns d,\cns a)}^{\Modele,\w_1,g}=1\)
\item \psline[linestyle=dashed,dash=3pt 5pt]{->}(0,.5ex)(1.2,.5ex)\hspace{12mm}
donc \(\denote{\Xlo\prd{regarder}(\cns b,\cns b)}^{\Modele,\w_1,g}=0\)
\end{enumerate}
\newcommand{\fxregardeBX}
{\left[
\begin{array}{l}
\begin{array}{@{}l}
\Obj{Alice}\rnode{a}{\stx}\\
\Obj{Bruno}\rnode{b}{\stx}\\
\Obj{Charles}\rnode{c}{\stx}\\
\Obj{Dina}\rnode{d}{\stx}\\
%\dots\\
\end{array}%
\rule{2.5cm}{0pt}
\begin{array}{l@{}}
\rnode{1}{1}\\[2ex]\rnode{0}{0}
\end{array}
\end{array}\right]
\ncline[nodesep=3pt]{->}{a}{1}
\ncline[nodesep=3pt,offsetB=2pt,linestyle=dashed,dash=3pt 5pt]{->}{b}{0}
\ncline[nodesep=3pt]{->}{c}{0}
\ncline[nodesep=3pt,offsetB=-2pt]{->}{d}{0}%
}
\newcommand{\fxregardeAX}
{\left[
\begin{array}{l}
\begin{array}{@{}l}
\Obj{Alice}\rnode{a}{\stx}\\
\Obj{Bruno}\rnode{b}{\stx}\\
\Obj{Charles}\rnode{c}{\stx}\\
\Obj{Dina}\rnode{d}{\stx}\\
%\dots\\
\end{array}\rule{2.5cm}{0pt}
\begin{array}{l@{}}
\rnode{1}{1}\\[2ex]\rnode{0}{0}
\end{array}
\end{array}\right]
\ncline[nodesep=3pt,offsetB=1pt]{->}{a}{0}
\ncline[nodesep=3pt,offsetB=-1pt]{->}{b}{0}
\ncline[nodesep=3pt,offsetB=1pt]{->}{c}{1}
\ncline[nodesep=3pt,offsetB=-1pt,linestyle=dotted,linewidth=1.4pt]{->}{d}{1}%
}
\newcommand{\fxregardeCX}
{\left[
\begin{array}{l}
\begin{array}{@{}l}
\Obj{Alice}\rnode{a}{\stx}\\
\Obj{Bruno}\rnode{b}{\stx}\\
\Obj{Charles}\rnode{c}{\stx}\\
\Obj{Dina}\rnode{d}{\stx}\\
%\dots\\
\end{array}\rule{2.5cm}{0pt}
\begin{array}{l@{}}
\rnode{1}{1}\\[2ex]\rnode{0}{0}
\end{array}
\end{array}\right]
\ncline[nodesep=3pt,offsetB=4pt,linestyle=dotted,dotsep=2pt]{->}{a}{0}
\ncline[nodesep=3pt,offsetB=2pt]{->}{b}{0}
\ncline[nodesep=3pt]{->}{c}{0}
\ncline[nodesep=3pt,offsetB=-2pt]{->}{d}{0}%
}
\newcommand{\fxregardeDX}
{\left[
\begin{array}{l}
\begin{array}{@{}l}
\Obj{Alice}\rnode{a}{\stx}\\
\Obj{Bruno}\rnode{b}{\stx}\\
\Obj{Charles}\rnode{c}{\stx}\\
\Obj{Dina}\rnode{d}{\stx}\\
%\dots\\
\end{array}\rule{2.5cm}{0pt}
\begin{array}{l@{}}
\rnode{1}{1}\\[2ex]\rnode{0}{0}
\end{array}
\end{array}\right]
\ncline[nodesep=3pt,offsetB=2pt]{->}{a}{0}
\ncline[nodesep=3pt,linestyle=dashed,dash=6pt 2pt]{->}{b}{1}
\ncline[nodesep=3pt]{->}{c}{0}
\ncline[nodesep=3pt,offsetB=-2pt]{->}{d}{0}%
}
\begin{figure}[h]
\begin{bigcenter}
\scalebox{.9}%
{
\(\denote{\prd{regarder}}^{\Modele,\w_1,g} = 
\left[
\begin{array}{l}
\\[2ex]
\Obj{Alice}\rnode{a1}{\stx}\\[2ex]
\Obj{Bruno}\rnode{b1}{\stx}\\[2ex]
\Obj{Charles}\rnode{c1}{\stx}\\[2ex]
\Obj{Dina}\rnode{d1}{\stx}\\[2ex]
%\dots\\
\end{array}\rule{2.5cm}{0pt}
%
\begin{array}{l@{\;}}
\rnode{q1}{\stx}%
\fxregardeAX
\\
\rnode{q2}{\stx}%
\fxregardeBX
\\
\rnode{q3}{\stx}%
\fxregardeCX
\\
\rnode{q4}{\stx}%
\fxregardeDX %\left[\begin{array}{l}\dots\end{array}\right]
\end{array}
\right]
\ncline[nodesep=3pt,linestyle=dotted,linewidth=1.4pt]{->}{a1}{q1}
\ncline[nodesep=3pt,linestyle=dashed,dash=3pt 5pt]{->}{b1}{q2}
\ncline[nodesep=3pt,linestyle=dotted,dotsep=2pt]{->}{c1}{q3}
\ncline[nodesep=3pt,linestyle=dashed,dash=6pt 2pt]{->}{d1}{q4}\)
}
\end{bigcenter}
\caption{Récupération de la dénotation des formules de 1--4}\label{f:regardfexo}
\end{figure}

N'oubliez pas qu'on commence toujours le trajet par la dénotation du \emph{dernier} argument de la liste accolée au prédicat (\ie\ ici le second argument).
\fussy
\end{solu}
\end{exo}
