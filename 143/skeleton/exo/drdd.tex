% -*- coding: utf-8 -*-
\begin{exo}\label{exo:derededicto}
Trouvez les quatre lectures (théoriquement) possibles de :\pagesolution{crg:derededicto}

\begin{enumerate}
\item Eugène$_1$ trouve que le Pape ressemble à son$_1$ arrière-grand-père.
\item \OE dipe$_1$ ne {savait} pas que {sa$_1$ mère} était {sa$_1$ mère}.
\end{enumerate}

\begin{solu}(p.~\pageref{exo:derededicto})\label{crg:derededicto} 

Pour illustrer les différentes lectures, je propose un scénario qui rend vrai chacune d'elles (mais cela ne veut pas dire nécessairement que chacune de ces lectures n'est vraie \emph{que} dans le scénario particulier qui lui est associé).  Il s'agit bien sûr d'alternances combinées \dedicto/\dere\ (cf. \S\ref{ss:re/dicto}, p.~\pageref{p.re/dicto}).

\begin{enumerate}
\item Eugène$_1$ trouve que le Pape ressemble à son$_1$ arrière-grand-père.
  \begin{enumerate}
  \item \sicut{le Pape}, \sicut{son arrière-grand-père} : \dedicto\\
  Eugène se dit : «c'est marrant, le Pape ressemble beaucoup à mon arrière-grand-père».
  \item \sicut{le Pape} : \dere, \sicut{son arrière-grand-père} : \dedicto\\
  Eugène voit une photo du Pape, sans savoir de qui il s'agit, et il se dit : «ce type ressemble à mon arrière-grand-père».
  \item \sicut{le Pape} : \dedicto, \sicut{son arrière-grand-père} : \dere\\
  Eugène voit une photo de son arrière-grand-père, sans savoir de qui il s'agit, et il se dit : «Le Pape, il ressemble à ce type».
  \item \sicut{le Pape}, \sicut{son arrière-grand-père} : \dere\\
  Eugène voit une photo du Pape, sans savoir de qui il s'agit, et une photo de son arrière-grand-père, sans le reconnaître non plus, et il se dit : «ces deux types se ressemblent beaucoup».
  \end{enumerate}

\item \OE dipe$_1$ ne {savait} pas que {sa$_1$ mère} était {sa$_1$ mère}.
\\
NB : comme déjà mentionné dans le chapitre, il convient ici de faire abstraction de la polysémie du nom \sicut{mère} (mère biologique {\vs} mère adoptive {\vs} mère sociale etc.) afin de ne pas surmultiplier les interprétations.

\begin{enumerate}
\item  \sicut{sa mère} : \dedicto\ $\times 2$\\
Quelqu'un dit à \OE dipe : «Ta mère est ta mère» (ce qui est une tautologie), et il répond (sans ironie)  : «Ah ? Je ne savais pas».  C'est évidemment une situation très absurde.

\item 1\ier\ \sicut{sa mère} : \dere, 2\ieme\ \sicut{sa mère} : \dedicto\\
Jocaste est la mère d'\OE dipe et quelqu'un dit à \OE dipe, en désignant Jocaste : «Cette femme est ta mère».  \OE dipe répond : «Ah ? Je ne savais pas».

\item 1\ier\ \sicut{sa mère} : \dedicto, 2\ieme\ \sicut{sa mère} : \dere\\
Jocaste est la mère d'\OE dipe, \OE dipe ne sait pas qui est sa mère (il ne l'a jamais connue et n'a aucune information sur elle) et quelqu'un lui dit, en désignant Jocaste : «Ta mère, c'est cette femme».  \OE dipe répond : «Ah ? Je ne savais pas».  Naturellement c'est similaire à la situation précédente.

\item  \sicut{sa mère} : \dere\ $\times 2$\\
Jocaste est la mère d'\OE dipe et quelqu'un dit à \OE dipe : «Jocaste est Jocaste».   \OE dipe répond : «Ah ? Je ne savais pas».  C'est une situation absurde similaire à la première.

\end{enumerate}

\end{enumerate}
\end{solu}
\end{exo}
