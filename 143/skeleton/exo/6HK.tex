% -*- coding: utf-8 -*-
\begin{exo}\label{exo:6HK}
\sloppy
Certains auteurs,\pagesolution{crg:6HK}
par exemple \citet{HeimKratzer:97},\Index{Heim, I.}\Index{Kratzer, A.} conservent le type \eet\ pour les verbes transitifs et examinent la possibilité d'assigner aux DP  le type \type{\eet,\et} (en plus de \ett).  Quelle serait alors la traduction de \sicut{une glace} dans ce cas ?

\fussy
\begin{solu}(p.~\pageref{exo:6HK})\label{crg:6HK}

Assigner le type \type{\eet,\et} à un DP revient à prévoir que celui-ci attend un V transitif (de type \eet) pour produire un VP de type \et.  Posons la variable \vrb R  de type \eet, pour jouer le rôle du V transitif. \sicut{Une glace} se traduira alors par \(\Xlo\lambda R\lambda x\exists y[\prd{glace}(y)\wedge [[R(y)](x)]]\).
La dérivation du VP \sicut{mange une glace} sera alors la suivante (le  DP objet dénote la fonction et le V transitif est l'argument) :

\begin{enumerate}
\item[] \(\sicut{mange une glace} \leadsto
\Xlo[\lambda R\lambda x\exists y[\prd{glace}(y)\wedge [[R(y)](x)]](\lambda v\lambda u\,\prd{manger}(u,v))]\)\\
= \(\Xlo\lambda x\exists y[\prd{glace}(y)\wedge [[\lambda v\lambda u\,\prd{manger}(u,v)(y)](x)]]\)
\hfill{\small(\breduc\ sur \vrb R)}
\\
= \(\Xlo\lambda x\exists y[\prd{glace}(y)\wedge [\lambda u\,\prd{manger}(u,y)(x)]]\)
\hfill{\small(\breduc\ sur \vrb v)}
\\
= \(\Xlo\lambda x\exists y[\prd{glace}(y)\wedge \prd{manger}(x,y)]\)
\hfill{\small(\breduc\ sur \vrb u)}
\end{enumerate}

\end{solu}
\end{exo}
