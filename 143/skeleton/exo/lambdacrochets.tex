% -*- coding: utf-8 -*-
\begin{exo}[Suppression des crochets]
\label{exo:types:crochets}
Dans les formules suivantes, sachant que \prd{aimer} est un prédicat à 
\pagesolution{crg:types:crochets}%
deux arguments, supprimez les crochets à chaque fois que
c'est possible, en appliquant la règle de
simplification.

%On considère que $x,y,z \in \ME_{\mathrm{e}}$, $\prd{aimer} \in \ME_{\type{e,\type{e,t}}}$
\addtolength{\multicolsep}{-9pt}
\begin{multicols}{2}
\begin{enumerate}
\item \(\Xlo[[\prd{aimer}(x)](y)]\)
\item \(\Xlo\lambda x [[\prd{aimer}(x)](y)]\)
\item \(\Xlo[\lambda x [\prd{aimer}(x)](y)]\)
\item \(\Xlo\peut[[\prd{aimer}(x)](y)]\)
\item \(\Xlo[\Intn[\prd{aimer}(x)](y)]\)
\item \(\Xlo\Intn[[\prd{aimer}(x)](y)]\)
\item \(\Xlo[\lambda x [[\prd{aimer}(y)](x)](z)]\)
\item \(\Xlo\lambda x [\lambda y [\prd{aimer}(y)](x)]\)
\end{enumerate}
\end{multicols}
\begin{solu} (p.~\pageref{exo:types:crochets})\label{crg:types:crochets}

L'important est de vérifier que la suppression des crochets (convention \ref{c:suppr[]}, p.~\pageref{c:suppr[]}) ne crée
pas d'ambiguïté.

\sloppy
\begin{enumerate}
\item \(\Xlo[[\prd{aimer}(x)](y)]\).
%
$\prd{aimer}$ est une fonction qui reçoit l'argument $\vrb x$, et comme
  $\prd{aimer}$ est un prédicat à deux arguments
%% de  type \type{e,\type{e,t}}, 
$\Xlo[\prd{aimer}(x)]$ est encore une fonction qui reçoit
  l'argument $\vrb y$.  Nous pouvons donc simplifier en $\Xlo\prd{aimer}(x)(y)$ puis en $\Xlo\prd{aimer}(y,x)$.

\item \(\Xlo\lambda x [[\prd{aimer}(x)](y)]\).
%
Pour les mêmes raisons, cela se simplifie en 
$\Xlo\lambda x\,\prd{aimer}(y,x)$. 

\item \(\Xlo[\lambda x [\prd{aimer}(x)](y)]\).  $\prd{aimer}$ est une
  fonction qui reçoit l'argument $\vrb x$.  Nous pouvons donc déjà
  simplifier en \(\Xlo[\lambda x\, \prd{aimer}(x)(y)]\).  Notons que
  $\Xlo\prd{aimer}(x)$ est encore une fonction (elle n'attend plus que
  son <<~second~>> argument).  Mais maintenant il faut faire
  attention: car \(\Xlo\lambda x\, \prd{aimer}(x)\) est aussi une
  fonction, et elle est différente de $\Xlo\prd{aimer}(x)$.  Si nous
  supprimons les crochets extérieurs, nous risquons de ne plus savoir à
  laquelle de ces deux fonctions $\vrb y$ est appliqué.  Dans
  l'exemple, $\vrb y$ est appliqué à \(\Xlo\lambda x\,
  \prd{aimer}(x)\), nous le savons grâce aux crochets extérieurs.
  Remarque: si $\vrb y$ était appliqué à $\Xlo\prd{aimer}(x)$, nous
  devrions écrire: $\Xlo\lambda x [\prd{aimer}(x)(y)]$ (ou par
  simplification $\Xlo\lambda x\,\prd{aimer}(y,x)$).

%% Pour illustrer, remplaçons $\prd{aimer}$ par la constante \prd{regarder}.
%% $\prd{regarder}(x)$ représente <<~regarde $x$~>>, ou <<~le$_{x}$
%% regarde~>>.  C'est une fonction car ça attend encore un sujet.  En
%% revanche $\lambda x\, \prd{regarder}(x)$ représente <<~regarde~>>,
%% c'est une fonction à deux arguments. 
%% \prd{aimer}uant à \([\lambda x\,\prd{regarder}(x)(y)]\) sera $\beta$-réduit en
%% \(\prd{regarder}(y)\); ça représente donc <<~le$_{y}$
%% regarde~>>. Enfin $\lambda x\,\prd{regarder}(y,x)$  représente
%% <<~il$_{y}$ regarde~>>, qui est une fonction car ça attend un GN objet.

\item \(\Xlo\peut[[\prd{aimer}(x)](y)]\).  Se simplifie en 
$\Xlo\peut\prd{aimer}(y,x)$.

\item \(\Xlo[\Intn[\prd{aimer}(x)](y)]\). Ne peut se simplifier qu'en
  \(\Xlo\Intn[\prd{aimer}(x)](y)\).  En effet \(\Xlo\Intn
  \prd{aimer}(x)(y)\) serait ambigu: nous ne saurions pas si $\Xlo\Intn$
  porte sur $\prd{aimer}$ ou $\Xlo\prd{aimer}(x)$ ou
  $\Xlo\prd{aimer}(x)(y)$.  Ici nous savons qu'il porte sur
  $\Xlo\prd{aimer}(x)$.

\item \(\Xlo\Intn[[\prd{aimer}(x)](y)]\).  Se simplifie en
  $\Xlo\Intn[\prd{aimer}(y,x)]$.

\item \(\Xlo[\lambda x [[\prd{aimer}(y)](x)](z)]\). Nous savons que nous pouvons
  simplifier $\Xlo[[\prd{aimer}(y)](x)]$ en $\Xlo\prd{aimer}(y,x)$.  Nous avons donc
  \(\Xlo[\lambda x\, \prd{aimer}(x,y) (z)]\).  Il vaut mieux garder les
  crochets extérieurs pour bien indiquer que $\vrb z$ est appliqué à
  $\Xlo\lambda x\, \prd{aimer}(x,y)$.

\item \(\Xlo\lambda x [\lambda y [\prd{aimer}(y)](x)]\).  Se simplifie
  uniquement en 
  \(\Xlo\lambda x [\lambda y\,\prd{aimer}(y)(x)]\) (à la rigueur).  
%Remarque: par   $\beta$-réduction, on obtiendra $\lambda x\, \prd{aimer}(x)$.
\end{enumerate}

\fussy
\end{solu}
\end{exo}
