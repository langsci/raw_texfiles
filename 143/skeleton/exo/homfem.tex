% -*- coding: utf-8 -*-
\begin{exo}\label{exo:homfem}
Cet exercice, en plus d'être une application de ce qui est présenté
\pagesolution{crg:homfem}
dans cette section, met le doigt sur un «petit» problème d'analyse
sémantique. 
\begin{enumerate}[label=(\alph*)]
\item {[\itshape{C'est l'anniversaire de Paul. Ses amis ont décidé d'organiser une fête surprise chez lui.  Et ils ont opté pour une soirée déguisée. Tous ses amis hommes ont choisi de se déguiser en femmes : en héroïnes de bande-dessinée. Et leurs déguisements étaient si bien réussis que sur le moment, quand il est arrivé...}}] Paul a cru que tous les hommes qui étaient là étaient des femmes.
\end{enumerate}

La phrase qui nous intéresse ici est la dernière de ce paragraphe ; le
texte en italique est là pour fixer un contexte et orienter la
compréhension.  Cette phrase est ambiguë (au moins théoriquement). %, et l'ambiguïté repose sur les lectures \alien{de re} vs.\ \alien{de dicto} du groupe nominal \sicut{tous les hommes qui étaient là} (ou simplement \sicut{tous les hommes}).

\begin{enumerate}
\item Explicitez les deux lectures en donnant, en français, deux paraphrases (ou gloses) précises et suffisamment distinctes de la dernière phrase de (a).
\item Prouvez qu'il s'agit bien d'une ambiguïté en utilisant la méthode vue au chapitre~\ref{Ch:1}. %en cours.
\item Traduisez les deux lectures en \LO\footnote{Ne traduisez pas la relative \sicut{qui étaient là}, elle n'est pas déterminante pour l'exercice.}.
\item L'ambiguïté de la phrase ne saute pas aux yeux car une des deux lectures est assez peu naturelle. De laquelle s'agit-il ? Et essayer d'expliquer pourquoi elle est si peu naturelle. 
\item Cet exercice montre que la dernière phrase de (a) est intrigante car elle remet en cause quelque chose que nous avons vu dans un chapitre précédent. De quoi s'agit-il ?
\end{enumerate}
\begin{solu}(p.~\pageref{exo:homfem})\label{crg:homfem}
\begin{enumerate}
\item Il s'agit bien sûr d'une ambiguïté \alien{de re}/\alien{de dicto} sur le {\GN} \sicut{tous les hommes (qui étaient là)}.
Pour la lecture \alien{de re} la glose sera :
  \begin{enumerate}
  \item Pour chaque individu qui est un homme et qui était là, Paul a cru qu'il s'agissait d'une femme. 
  \end{enumerate}
Pour la lecture \alien{de dicto} la glose sera :
  \begin{enumerate}
  \item[b.] Paul s'est dit : «~tiens, tous les hommes qui sont là sont des femmes~».
  \end{enumerate}
\item Construisons un modèle $\Modele$ par rapport auquel la phrase avec la lecture \alien{de re} sera vraie et celle avec la lecture \alien{de dicto} sera fausse.  $\Modele$ contient les données suivantes :
\begin{itemize}
\item \Obj{Paul}, un homme ;
\item \Obj{Antoine} et \Obj{Mickaël}, des hommes, amis de \Obj{Paul} ;
\item \Obj{Julie} et \Obj{Sarah}, des femmes, amies de \Obj{Paul} ;
\item \Obj{Antoine} est déguisé en Catwoman ;
\item \Obj{Mickaël} est déguisé en Batgirl ;
\item \Obj{Julie} est déguisée en Iron Man ;
\item \Obj{Sarah} est déguisée en marsupilami ;
\item \Obj{Paul} ne reconnaît aucun de ses amis ;
\item \Obj{Paul} croit que \Obj{Antoine} et \Obj{Mickaël} sont des femmes ;
\item \Obj{Paul} croit que \Obj{Julie} et \Obj{Sarah} sont des hommes ;
\end{itemize}

Avec la lecture \alien{de dicto} de \sicut{tous les hommes}, la phrase est fausse, car les individus qui sont des hommes dans les croyances de Paul sont Julie et Sarah et Paul pense que ces deux individus sont des hommes et non pas des femmes.  Mais avec la lecture \alien{de re} de \sicut{tous les hommes}, la phrase est vraie, car les individus qui sont des hommes dans $\Modele$ sont Antoine et Mickaël et Paul croit bien qu'Antoine et Mickaël sont des femmes\footnote{NB : dans cette démonstration, on exclut Paul de l'ensemble des hommes quand on fait la quantification de \sicut{tous les hommes}. C'est un problème annexe ; en fait il s'agit d'un phénomène courant lorsqu'une phrase contient un quantificateur universel et un GN référentiel, ce dernier se retrouve exclu de la quantification. Cf.\ par exemple \sicut{dans la classe, Jean est plus grand que tout le monde} ; techniquement cette phrase devrait toujours être fausse car Jean fait partie de \sicut{tout le monde} et Jean n'est pas plus grand que lui-même, mais par pragmatique on interprète la phrase en partitionnant l'ensemble des individus en faisant en sorte que \sicut{tout le monde} signifie «tout individu sauf Jean».}. 

\item 
  \begin{enumerate}
  \item \alien{De re} :\\
\(\Xlo\forall x [\prd{homme}(x) \implq \prd{croire}(\cns p,\Intn\prd{femme}(x))]\)
  \item \alien{De dicto} :\\
\(\Xlo\prd{croire}(\cns p,\Intn\forall x [\prd{homme}(x) \implq \prd{femme}(x)])\)
  \end{enumerate}
\item Comme l'indique la glose ci-dessus, la lecture \alien{de dicto} est la moins naturelle des deux car elle attribue à Paul la pensée (i.e. la proposition) qui dit que tous les hommes sont des femmes. Mais dans quels mondes possibles la phrase \sicut{tous les hommes sont des femmes} est vraie ? Probablement aucun (si on ne change pas le sens des mots), et cette phrase est absurde (au sens logique, c'est-à-dire que c'est une contradiction, une phrase qui n'est jamais vraie).  Et il est fort probable que Paul n'a pas ce genre de pensée : cela voudrait dire qu'il considère que le monde dans lequel il se trouve appartient à l'ensemble vide...
\item L'interprétation la plus naturelle est celle avec la lecture \alien{de re} du GN \sicut{tous les hommes}.  Mais, comme le montre la formule (3a), cela veut dire que ce GN est interprété avec une portée large, \emph{en dehors de la proposition syntaxique où il apparaît}. C'est un contre-exemple à l'observation que nous avions faite en \S\ref{sss:limiteportée}, p.~\pageref{pt:Portee2}, et qui disait que les GN quantificationnels forts (par ex. universels) ne peuvent pas «~traverser~»\ une frontière de proposition.
\end{enumerate}
\end{solu}
\end{exo}

