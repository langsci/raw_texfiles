% -*- coding: utf-8 -*-
\begin{exo}[Modèle]\label{exo:casso}
Représentez l'état de choses représenté %la disposition représentée 
\pagesolution{crg:casso}%
dans l'image en figure~\ref{fig:exo:Modele} sous la forme d'un modèle \tuple{\Unv A,\FI}.

\begin{figure}[h!]
\begin{center}
\fbox{\Large
\(\begin{array}{cccc}
\Box & \heartsuit & & \blacktriangle\\
& \bigcirc & \blacksquare & \\
\bigstar & \Box & & \clubsuit\\
&& \triangle
  \end{array}%
\)%
}
\caption{Modèle «casseaux»}\label{fig:exo:Modele}
\end{center}
\end{figure}

Vous poserez
\(\Unv A = \set{\Box_1; \heartsuit; \blacktriangle;
 \bigcirc; \blacksquare; 
\bigstar ; \Box_2 ; \clubsuit ; \triangle}\), avec $\Box_1$ pour le carré en au haut à gauche et $\Box_2$ pour l'autre carré blanc.
Vous définirez $\FI$ pour les prédicats unaires
\prd{carré}, \prd{rond}, \prd{c\oe ur}, \prd{triangle}, \prd{étoile}, \prd{trèfle}, \prd{blanc}, \prd{noir}, \prd{objet}
et les prédicats binaires
\prd{à-gauche-de}, \prd{à-droite-de}, \prd{au-dessus-de}, \prd{en-dessous-de}.

\begin{solu} (p.~\pageref{exo:casso})\label{crg:casso}

\raggedright

\(\Unv A = \set{\Box_1; \heartsuit; \blacktriangle;
 \bigcirc; \blacksquare; 
\bigstar ; \Box_2 ; \clubsuit ; \triangle}\).

\noindent
\(\FI(\prd{carré}) = \set{\Box_1; \blacksquare;\Box_2}\), 
\(\FI(\prd{rond}) = \set{\bigcirc}\), 
\(\FI(\prd{c\oe ur}) = \set{\heartsuit}\), 
\(\FI(\prd{triangle}) = \set{\blacktriangle; \triangle}\), 
\(\FI(\prd{étoile}) = \set{\bigstar}\), 
\(\FI(\prd{trèfle}) = \set{\clubsuit}\), 
\(\FI(\prd{blanc}) = \set{\Box_1; \heartsuit;\bigcirc;\Box_2;\triangle}\), 
\(\FI(\prd{noir}) = \set{\blacktriangle; \blacksquare;\bigstar;\clubsuit}\), 
\(\FI(\prd{objet}) = \Unv A\), 
\(\FI(\prd{à-gauche-de}) = \set{\tuple{\Box_1,\heartsuit} ; \tuple{\Box_1,\blacktriangle}; \tuple{\Box_1,\blacktriangle} ;
\tuple{\bigcirc,\blacksquare} ; 
\tuple{\bigstar,\Box_2} ; \tuple{\bigstar,\clubsuit} ; \tuple{\Box_2,\clubsuit}}\),
\(\FI(\prd{à-droite-de}) = \set{\tuple{\heartsuit,\Box_1} ; \tuple{\blacktriangle,\Box_1}; \tuple{\blacktriangle,\Box_1} ;
\tuple{\blacksquare,\bigcirc} ; 
\tuple{\Box_2,\bigstar} ; \tuple{\clubsuit,\bigstar} ; \tuple{\clubsuit,\Box_2}}\),
\(\FI(\prd{au-dessus-de}) = 
\set{\tuple{\Box_1,\bigstar}; \tuple{\heartsuit,\bigcirc} ; \tuple{\heartsuit,\Box_2} ; \tuple{\bigcirc,\Box_2} ; \tuple{\blacksquare,\triangle} ; \tuple{\blacktriangle , \clubsuit}}
\),
\(\FI(\prd{en-dessous-de}) = 
\set{\tuple{\bigstar,\Box_1}; \tuple{\bigcirc,\heartsuit} ; \tuple{\Box_2,\heartsuit} ; \tuple{\Box_2,\bigcirc} ; \tuple{\triangle,\blacksquare} ; \tuple{\clubsuit,\blacktriangle}}
\).
 


\end{solu}
\end{exo}
