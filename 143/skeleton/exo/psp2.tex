% -*- coding: utf-8 -*-
\begin{exo}\label{exo:1psp2}\sloppy
Beaucoup de présuppositions sont attachées à la structure «syntactico-sémantique» de la phrase, c'est-à-dire au choix de certains mots-outils (déterminants, conjonctions, adverbes...). Pour les identifier, il peut alors être pratique d'utiliser des mots (pleins) inventés, dont on ne connaît pas le sens.  On ne comprendra pas entièrement les énoncés observés, mais cela ne nous empêchera pas d'établir certaines relations de sens entre phrases qui contiennent ces mêmes mots inventés. Cela permet, par la même occasion, de neutraliser les inférences que nous pourrions tirer à partir de nos connaissances du monde, et ainsi de nous focaliser uniquement sur ce que déclenche la structure des phrases.

\fussy

Trouvez, s'il y en a, les présuppositions des phrases suivantes :
\begin{enumerate}
\item Il y a des verchons qui ont bourniflé.
\item Tous les verchons ont bourniflé.
\item Aucun verchon n'a bourniflé.
\end{enumerate}
\begin{solu}
(p.~\pageref{exo:1psp2})

Les résultats que nous pouvons tirer de cet exercice sont à prendre avec précaution car ils attendent des jugements sémantiques particulièrement fins, qui peuvent varier d'un locuteur à l'autre. Pour obtenir des conclusions plus assurées, il serait utile, par exemple, de mettre sur pied des expérimentations à grande échelle.  Nous allons donc ici seulement nous concentrer sur la méthode sous-jacente.  Celle-ci consiste, en premier lieu, à comparer les inférences que nous tirons de chaque phrase avec celles que nous tirons de leur négation.  Pour ces trois phrases, les présuppositions potentielles portent sur l'existence des verchons.
\begin{enumerate}
\item \emph{Il y a des verchons qui ont bourniflé.} 
Nous en inférons évidemment que les verchons existent.\\
Négation : \emph{Il est faux qu'il y a des verchons qui ont bourniflé.}
Certes, les contextes les plus naturels dans lesquels cette phrase peut être énoncée, sont ceux où l'on sait que les verchons existent ; cependant (et c'est ce qui importe ici) elle peut également l'être dans des cas de figure où les verchons n'existent pas.  En d'autres termes, si les verchons n'existent pas, il semblera assez légitime de dire que cette phrase est vraie. 

Il semble donc que l'expression \sicut{il y a des verchons} ne présuppose pas \emph{sémantiquement} qu'il existe des verchons (mais ça l'affirme).

\item \emph{Tous les verchons ont bourniflé.}  Si cette phrase est vraie, alors nous devons en inférer que les verchons existent (notamment du fait de l'emploi du passé composé).\\
Négation : \emph{Il est faux que tous les verchons ont bourniflé.} Cela signifie qu'il y a au moins un verchon qui n'a pas bourniflé, et donc que les verchons existent. 

Nous pouvons également nous aider du test de la redondance (\S\ref{sss:ptépsp}, p.~\pageref{p.testAB}).  Si nous comparons \sicut{les verchons existent et tous les verchons ont bourniflé} et \sicut{tous les verchons ont bourniflé et les verchons existent}, nous constatons que cette seconde phrase est particulièrement redondante.

Il est donc raisonnable de conclure que la phrase présuppose (sémantiquement) que les verchons existent.

\item \emph{Aucun verchon n'a bourniflé.}  Pouvons-nous inférer de (la vérité de) cette phrase que forcément les verchons existent ? Les jugements peuvent être fluctuants, mais nous pouvons estimer que si les verchons n'existent pas, alors la phrase sera vraie (comme pour la négation de la phrase 1 ci-dessus). Si tel est bien le cas, alors il n'est pas utile d'examiner la négation de la phrase\footnote{Celle-ci signifie qu'il y a des verchons qui ont bourniflé, ce qui entraîne directement l'existence des verchons.}, nous pourrons tout de suite conclure que la phrase ne présuppose pas que les verchons existent. Le test de la redondance va-t-il dans ce sens ? \sicut{Les verchons existent mais aucun verchon n'a bourniflé} est acceptable, comme prévu ; quant à \sicut{aucun verchon n'a bourniflé, mais les verchons existent} nous pouvons y voir une redondance, mais elle est probablement moins saillante que dans la version avec \sicut{tous les} en 2. 
\end{enumerate}
\end{solu}
\end{exo}
