% -*- coding: utf-8 -*-
\begin{exo}\label{exo:notaPTQ}
\citet{PTQ} utilise les «raccourcis» de notation
suivants\footnote{Les notations de Montague ne sont pas exactement
  celles présentées ici : j'ai procédé aux ajustements appropriés pour
  rester cohérent avec les conventions utilisées dans ce manuel
  (définition~\ref{RSyn:ltype}) ; en l'occurrence, les crochets
  $[\,]$ ne sont pas placés de la même façon.} :
\begin{itemize}
\item Si $\xlo{\gamma} \in \ME_{\type{s,\type{\mtyp a,t}}}$ et $\xlo{\alpha}
  \in \ME_{\mtyp a}$, alors  $\Xlo[\gamma\{\alpha\}]$ vaut pour
  $\Xlo[\Extn\gamma(\alpha)]$ ; 
%et donc $[\gamma\{\alpha\}] \in  \ME_{\typ t}$ ; 
\item Si $\xlo{\gamma} \in \ME_{\type{s,\type{\mtyp a,\type{\mtyp
        b,t}}}}$,  
$\xlo{\alpha}   \in \ME_{\mtyp a}$ et   $\xlo{\beta}
  \in \ME_{\mtyp b}$, alors  $\Xlo[\gamma\{\beta,\alpha\}]$ vaut pour
  $\Xlo[\Extn\gamma(\beta,\alpha)]$ ; 
  %et donc $[\gamma\{\alpha\}] \in  \ME_{\typ t}$ ; 
\item Si $\vrb x\in\VAR_a$ et $\vrb{\phi} \in \ME_{\typ t}$, alors
  $\Xlo\mathit{x̑}\phi$ vaut pour $\Xlo\lambda x\phi$,
\item et $\Xlo\hat{x}\phi$ vaut pour $\Xlo\Intn\lambda x\phi$ ;
\item Si $\vrb{\alpha}\in\ME_{\typ e}$ et  $\vrb P\in
  \ME_{\type{s,\type{\type{s,e},t}}}$, alors $\Xlo\alpha^*$ vaut pour
  $\Xlo\lambda P[P\{\Intn\alpha\}]$. 
%  $\Xlo\mathit{̑Ȓ}[R\{\Intn\alpha\}]$. 
\end{itemize}

%Ainsi, par exemple, $\Xlo\mathit{$x$}\,\prd{citron}(x)$ dénote
%l'ensemble de   tous les citrons (pour un modèle et un monde donné),
%et $\Xlo\hat x \,\prd{citron}(x)$ dénote la propriété d'être un citron.

%Si $\cns a$ dénote l'individu \Obj{Alice} dans {\Modele} et $w$, que
%dénote $\Xlo\cns a^*$ ?

Sachant que \vrb x et \cns a sont de type \typ e, \prd{citron} de type \et, \prd{aimer} de type \eet\ et \vrbS P de type \type{\type{s,e},t},
pour chaque expression ci-dessous, donnez son type et retranscrivez la
selon notre syntaxe habituelle de la définition ~\ref{RSyn:ltype}
(p.~\pageref{RSyn:ltype}).  
\addtolength{\multicolsep}{-10pt}
\begin{multicols}{2}
\begin{enumerate}
\item $\Xlo\cns a^*$
\item $\Xlo\lambda x\, \Intn\prd{aimer}\{\cns a,x\}$
\item $\Xlo\hat{\vrbS P}[\vrbS P\{x\}]$
\item $\Xlo\mathit{x̑}\,\prd{citron}(x)$
\item $\Xlo\hat x\, \prd{aimer}(\cns a,x)$
\item $\Xlo[\cns a^*(\Intn\vrbS P)]$
\end{enumerate}
\end{multicols}

Détaillez ce que dénote $\Xlo\cns a^*$ sachant que \cns a dénote \Obj{Alice}.
%
%
\begin{solu} (p. \pageref{exo:notaPTQ})

\begin{enumerate}
\item $\Xlo\cns a^*$ est de type \type{\type{s,\type{\type{s,e},t}},t}
\\
= 
$\Xlo\lambda P[P\{\Intn\alpha\}]$ = 
$\Xlo\lambda P[\Extn P(\Intn\alpha)]$ (avec \vrb P de type  \type{s,\type{\type{s,e},t}})

\item $\Xlo\lambda x\, \Intn\prd{aimer}\{\cns a,x\}$ est de type \et
\\
=
$\Xlo\lambda x[\Extn\Intn\prd{aimer}(\cns a,x)]$ =
$\Xlo\lambda x\,\prd{aimer}(\cns a,x)$ 

\item $\Xlo\hat{\vrbS P}[\vrbS P\{x\}]$ est de type \type{s,\type{\type{\type{s,e},t},t}}
\\
= $\Xlo\Intn\lambda{\vrbS P}[\Extn\vrbS P(x)]$

\item $\Xlo\mathit{x̑}\,\prd{citron}(x)$ est de type \et
\\
=
$\Xlo\lambda x\,\prd{citron}(x)$

\item $\Xlo\hat x\, \prd{aimer}(\cns a,x)$ est de type \type{s,\et}
\\
=
$\Xlo\Intn\lambda x\, \prd{aimer}(\cns a,x)$ 

\item $\Xlo[\cns a^*(\Intn\vrbS P)]$ est de type \typ t
\\
=
$\Xlo[\lambda P[P\{\Intn\cns a\}](\Intn\vrbS P)]$ =
$\Xlo[\lambda P[\Extn P(\Intn\cns a)](\Intn\vrbS P)]$ =
$\Xlo[\Extn \Intn\vrbS P(\Intn\cns a)]$ =
$\Xlo[\vrbS P(\Intn\cns a)]$
\end{enumerate}

\smallskip

\cns a dénote \Obj{Alice} dans $w$ ; pour savoir ce que dénote
$\Xlo\cns a^*$, commençons par restituer ce que cette écriture
abrège : \(\Xlo\lambda P[P\{\Intn\cns a\}]\). 
On sait que
\(\Xlo\Intn\cns a\) dénote le concept individuel d'Alice -- appelons
cela le concept-Alice.   Ensuite \(\Xlo[P\{\Intn\cns a\}]\) équivaut à 
\(\Xlo[\Extn P(\Intn\cns a)]\).  On sait que \vrb P est de type
\type{s,\type{\type{s,e},t}}, ce qui veut dire que c'est une variable
de propriété (intensionnelle) de concepts d'individus, et donc
$\Xlo\Extn P$ est de type \type{\type{s,e},t}, c'est l'extension de la
propriété \vrb P dans le monde $w$ (autrement dit un ensemble de
concepts d'individus).  Donc  \(\Xlo[\Extn P(\Intn\cns a)]\) est une
formule, et elle est vraie dans $w$ si le concept-Alice vérifie la
propriété \vrb P dans $w$.  Donc
\(\Xlo\lambda P[P\{\Intn\cns a\}]\), \ie\ $\Xlo\cns a^*$, est
l'ensemble de toutes les propriétés intensionnelles possédées par le
concept-Alice dans le monde $w$.
\end{solu}
\end{exo}
