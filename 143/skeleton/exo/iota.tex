% -*- coding: utf-8 -*-
\begin{exo}\label{exo:tradiota}
Traduisez dans {\LO} les phrases suivantes, en utilisant l'opérateur
$\Xlo\atoi$.\pagesolution{crg:tradiota}
\begin{enumerate}
\item Le maire a rencontré le pharmacien.
\item Gontran a perdu son chapeau.
\item Le boulanger a prêté l'échelle au cordonnier.
\item La maîtresse a confisqué le lance-pierres de l'élève.
\item Gontran a attrapé le singe qui avait volé son chapeau.
\item Celui qui a gagné le gros-lot, c'est Fabrice.
\item Celui qui a gagné un cochon, c'est Fabrice.
\item Tout cow-boy aime son cheval.
\item Georges a vendu son portrait de Picasso.
\end{enumerate}
\begin{solu} (p.~\pageref{exo:tradiota})\label{crg:tradiota}

%Traductions avec $\Xlo\atoi$.
D'après la règle (\RSem\ref{RIatoi}) p.~\pageref{RIatoi}, \(\Xlo\atoi x\phi\) dénote l'unique individu \vrb x tel que \vrb\phi\ est vraie.  Donc si $N$ se traduit par \vrb\alpha, \sicut{le $N$} se traduira par $\Xlo\atoi x\,\alpha(x)$.
\begin{enumerate}
\item Le maire a rencontré le pharmacien.
\\$\leadsto$ \(\Xlo\prd{rencontrer}(\atoi x\,\prd{maire}(x),\atoi x\,\prd{pharmacien}(x))\) 

\item Gontran a perdu son chapeau.
%\\$\leadsto$ \(\Xlo\prd{perdre}(\cns g,\atoi x [\prd{chapeau}(x)\wedge \prd{poss}(y,x)]) \wedge y=\cns g\)
%\\ou directement
\\$\leadsto$ \(\Xlo\prd{perdre}(\cns g,\atoi x [\prd{chapeau}(x)\wedge \prd{poss}(\cns g,x)])\)

Le prédicat \prd{poss} exprime la possession : $\Xlo\prd{poss}(x,y)$ signifie que \vrb x possède \vrb y.

\item Le boulanger a prêté l'échelle au cordonnier.
\\$\leadsto$ \(\Xlo\prd{prêter}(\atoi x\,\prd{boulanger}(x),\atoi x\,\prd{échelle}(x),\atoi x \,\prd{cordonnier}(x))\)

\item La maîtresse a confisqué le lance-pierres de l'élève.
\\$\leadsto$ \(\Xlo\prd{confisquer}(\atoi x\,\prd{maîtresse}(x),\atoi x [\prd{lance-pierres}(x)\wedge \prd{poss}(\atoi y \,\prd{élève}(y),x)])\)

Remarque : là encore on peut traduire \sicut{le lance-pierres de
  l'élève} en utilisant toujours la variable $\vrb x$ : \(\Xlo\atoi x
[\prd{lance-pierres}(x)\wedge \prd{poss}(\atoi x
  \,\prd{élève}(x),x)]\), car on peut reconnaître que les différentes
occurrences de $\vrb x$ ne sont pas liées par le même $\Xlo\atoi x$.
Cependant, pour des raisons de lisibilité et de confort, il est
naturel d'utiliser ici deux variables distinctes. 

\item Gontran a attrapé le singe qui avait volé son chapeau.
%\\$\leadsto$ \(\Xlo\prd{attraper}(\cns g, \atoi x [\prd{singe}(x)\wedge
%  \prd{voler}(x,\atoi y [\prd{chapeau}(y)\wedge \prd{poss}(z,y)])])\wedge z=\cns g\)
%\\ou directement
\\$\leadsto$ \(\Xlo\prd{attraper}(\cns g, \atoi x [\prd{singe}(x)\wedge
  \prd{voler}(x,\atoi y [\prd{chapeau}(y)\wedge \prd{poss}(\cns g,y)])])\)

Remarque :  le contenu de la relative, \sicut{qui avait volé son
  chapeau}, se retrouve dans la portée de la première description
définie dès lors que l'on comprend cette relative comme une relative
\emph{restrictive} : on parle ici de l'unique l'individu qui est à la
fois un
singe \emph{et}  a volé le chapeau de Gontran. Dans ce cas, la
phrase pourra être vraie même s'il y a plusieurs singes dans le modèle
(et le contexte), du moment qu'un seul ait volé le chapeau.

Au contraire, si on comprend la relative comme une relative
\emph{descriptive} (on mettrait alors plus volontiers une virgule :
\sicut{Gontran a attrapé le singe, qui avait volé son chapeau}), la
phrase ne peut être vraie que s'il n'y a qu'un seul singe dans le
contexte. La traduction serait alors :
\\$\leadsto$ \(\Xlo\prd{attraper}(\cns g, \atoi x\,\prd{singe}(x)) \wedge
  \prd{voler}(\atoi x\,\prd{singe}(x),\atoi y
      [\prd{chapeau}(y)\wedge \prd{poss}(\cns g,y)])\) 

\item Celui qui a gagné le gros-lot, c'est Fabrice.
\\$\leadsto$ \(\Xlo\atoi x\, \prd{gagner}(x,\atoi y \,\prd{gros-lot}(y)) =
\cns f\)

\item Celui qui a gagné un cochon, c'est Fabrice.
\\$\leadsto$ \(\Xlo\atoi x \exists y [\prd{cochon}(y) \wedge \prd{gagner}(x,y)] =
\cns f\)

\item Tout cow-boy aime son cheval.
%\\$\leadsto$ \(\Xlo\forall x [\prd{cow-boy}(x)\implq [\prd{aimer}(x,\atoi
%    y [\prd{cheval}(y)\wedge\prd{poss}(z,y)]) \wedge z=x]]\)
%\\ ou directement:
\\$\leadsto$ \(\Xlo\forall x [\prd{cow-boy}(x)\implq \prd{aimer}(x,\atoi
    y [\prd{cheval}(y)\wedge\prd{poss}(x,y)])]\)

\item Georges a vendu son portrait de Picasso.
\\ Comme vu dans le chapitre, cette phrase est ambiguë\footnote{NB :  ces traductions incluent un argument \vrb z pour \prd{vendre} en considérant que $\Xlo\prd{vendre}(x,y,z)$ signifie \vrb x vend \vrb y à \vrb z.} :
\begin{enumerate}
\item $\leadsto$ \(\Xlo\exists z\, \prd{vendre}(\cns g,\atoi x[\prd{portrait}(x,\cns p)\wedge\prd{posséder}(\cns g,x)],z)\)\\
le portrait représente Picasso et appartient à Georges
\item $\leadsto$ \(\Xlo\exists z\, \prd{vendre}(\cns g,\atoi x[\prd{portrait}(x,\cns p)\wedge\prd{réaliser}(\cns g,x)],z)\)\\
le portrait représente Picasso et a été réalisé par Georges (bien sûr si Georges le vend, c'est aussi qu'il le possédait, mais dans ce cas, c'est une inférence supplémentaire qui n'est pas directement liée au déterminant possessif)
\item $\leadsto$ \(\Xlo\exists z\, \prd{vendre}(\cns g,\atoi x[\exists y\,\prd{portrait}(x,y)\wedge\prd{réaliser}(\cns p,x) \wedge \prd{posséder}(\cns g,x)],z)\)\\
le portrait a été réalisé par Picasso et appartient à Georges (et on ne sait pas qui il représente)
\item $\leadsto$ \(\Xlo\exists z\, \prd{vendre}(\cns g,\atoi x[\prd{portrait}(x,\cns g)\wedge\prd{réaliser}(\cns p,x)],z)\)\\
le portrait représente Georges et a été réalisé par Picasso (cette interprétation est moins naturelle, en français on dira plutôt \sicut{son portrait par Picasso})
\end{enumerate}

\end{enumerate}

\smallskip

Remarque : en toute rigueur, dans tous ces exemples, la traduction des {\GN} définis devrait toujours comporter une sous-formule supplémentaire qui restreint les valeurs pertinentes de la variable, comme vu en \S\ref{ss:RestrDQuant}.  Autrement dit, la traduction complète de la phrase 1 est en fait :

\begin{enumerate}
\item \(\Xlo\prd{rencontrer}(\atoi x[\prd{maire}(x) \wedge C_1(x)],\atoi x[\prd{pharmacien}(x)\wedge C_2(x)])\) 
\end{enumerate}
Cela permet de faire référence à l'unique individu qui est maire \emph{dans l'ensemble \vrbi C1} et l'unique individu qui est pharmacien \emph{dans l'ensemble \vrbi C2}.  Il est probable que la phrase s'interprète naturellement avec \vrbi C1 $=$ \vrbi C2 (il peut s'agir, par exemple, de l'ensemble des habitants d'un village ou d'un quartier -- où il n'y aura qu'un seul pharmacien), mais par souci de généralité, il est prudent de considérer que chaque {\GN} introduit son propre ensemble de restriction.
\end{solu}
\end{exo}
