% -*- coding: utf-8 -*-
\begin{exo}\label{exo:6iotad}
Supposons que nous souhaitions maintenir
\pagesolution{crg:6ioatd}%
que les DP définis singuliers soient de type~\typ e.  Quels devraient alors être le type et la traduction du déterminant \sicut{le} ?
\begin{solu}(p.~\pageref{exo:6iotad})\label{crg:6iotad}

\sloppy
Pour obtenir un DP défini singulier de type \typ e, le déterminant \sicut{le} devra être de type \type{\et,e} et se traduire par \(\Xlo\lambda P\atoi x[P(x)]\).
Ainsi, \sicut{le Pape} se traduira par \(\Xlo[\lambda P\atoi x[P(x)](\lambda y\,\prd{pape}(y))]\) qui, par \breduc s, se simplifie en 
\(\Xlo\atoi x[\lambda y\,\prd{pape}(y)(x)]\) et 
\(\Xlo\atoi x\,\prd{pape}(x)\) (qui est bien de type \typ e).

\fussy
\end{solu}
\end{exo}
