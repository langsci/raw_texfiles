% -*- coding: utf-8 -*-
\begin{exo}\label{exo:DefPlur}
Dans le travail de classification des {\GN} que nous venons d'opérer, 
\pagesolution{crg:DefPlur}%
le défini pluriel, \sicut{les N}, a été volontairement omis. 
C'est qu'il manifeste des propriétés interprétatives assez particulières.
Appliquez les quatre tests présentés ci-dessus (covariation, ambiguïté avec la négation, consistance et complétude) sur un défini pluriel pour tenter de le rattacher à l'une de nos trois classes de {\GN}.  Qu'observe-t-on ? 
%
\begin{solu} (p.~\pageref{exo:DefPlur})\label{crg:DefPlur}

\noindent\emph{Test de covariation} (cf. \ref{xGNsubitcv}, p.~\pageref{xGNsubitcv}): on place le GN défini pluriel dans une phrase qui commence, par exemple, par \sicut{à plusieurs reprises}.

\begin{enumerate}[label=(\arabic*)]
\item À plusieurs reprises, \emph{les prisonniers} ont tenté de \label{xpriso-f}s'évader.
\end{enumerate}

Cette phrase ne semble pas manifester une covariation du défini pluriel : il est question d'une certaine compagnies de prisonniers et c'est toujours à celle-ci que sont attribuées les tentatives d'évasion.
Il est très important de noter ici que l'on peut voir dans cette phrase ce que l'on appelle parfois la lecture de groupe, ou encore lecture solidaire (en anglais on parle de \alien{team credit}), du défini pluriel. Par cette lecture, il n'est pas nécessaire, dans les faits, que \emph{tous} les prisonniers aient été impliqués dans la tentative d'évasion pour imputer l'action à l'ensemble du groupe. Les responsables réels de la tentative peuvent donc ne pas être toujours les mêmes. Mais pour nous, cela ne change rien, il n'y a pas covariation.  Car le propre de cette lecture est justement d'assigner (à tort ou à raison) le prédicat verbal  à tout le groupe, et ce groupe reste toujours le même (sous encore l'hypothèse de la note \ref{fnmonpriso} p.~\pageref{fnmonpriso} du chapitre).  

\smallskip

\noindent\emph{Test de la négation} (cf. \ref{xGNambigNeg}, p.~\pageref{xGNambigNeg}): on place le GN défini pluriel en position d'objet dans une phrase négative.

\begin{enumerate}[label=(\arabic*),resume]
\item Jean n'a pas lu \emph{les dossiers}.\label{jnapaslulesd}
\end{enumerate}

Nous voyons apparaître ici une distinction sémantique subtile et non triviale entre \sicut{tous les} et \sicut{les}.
Il est généralement admis que \ref{jnapaslulesd} n'est pas ambiguë et qu'elle signifie uniquement que Jean n'a lu aucun des dossiers ; une situation où il aurait lu certains dossiers mais pas tous est le plus souvent perçue comme rendant la phrase ni vraie ni fausse. C'est donc qu'il y a probablement une affaire de présupposition ; nous ne la développerons pas ici mais elle sera abordée au chapitre~\ref{GN++} (vol.~2).

\smallskip

\noindent\emph{Test de consistance} (cf. \ref{test:contra}, p.~\pageref{test:contra}): on vérifie si la conjonction \sicut{les $N$ GV et les $N$ non-GV} est une contradiction.

\begin{enumerate}[label=(\arabic*),resume]
\item Les candidats sont barbus et les candidats sont imberbes.\label{test:contraf}
\end{enumerate}


On constate que si l'une des deux phrases connectées est vraie, l'autre est forcément fausse.  La phrase \ref{test:contraf} ne peut donc jamais être vraie. 
Et si l'une des deux est fausse, la phrase \ref{test:contraf} est bien sûr immédiatement fausse.  Le défini pluriel semble donc se comporter ici comme \sicut{tous les N}.   Mais que se passe-t-il dans 
 un modèle où il y a, par exemple, 50\% de barbus et 50\% d'imberbes ?
Ici les jugements sont un peu délicats ; cependant il se dégage généralement  une tendance  qui est que les deux propositions seront jugées ni vraies ni fausses, mais plutôt inappropriées. Cela est, encore une fois,  lié à l'effet présuppositionnel mentionné \alien{supra} et qui fait que le test de consistance réussit, mais seulement «à moitié» (la phrase n'est jamais vraie, mais elle n'est pas exactement toujours fausse).   Cet effet présuppositionnel se retrouve aussi dans le test de complétude.

\smallskip

\noindent\emph{Test de complétude} (cf. \ref{test:compl}, p.~\pageref{test:compl}) : on vérifie si la disjonction \sicut{les $N$ GV ou les $N$ non-GV} est une tautologie.

\begin{enumerate}[label=(\arabic*),resume]
\item Les candidats sont barbus ou les candidats sont imberbes.\label{test:complf}
\end{enumerate}

Certes si les candidats sont tous barbus ou s'ils sont tous imberbes, la phrase \ref{test:complf} sera globalement vraie.  Inversement, s'il y a, par exemple, 50\% de barbus et 50\% d'imberbes, aucune des deux phrases connectées ne sera vraie, mais, comme précédemment, il sera difficile de les tenir pour fausses pour autant.  Et donc là encore, le test réussit seulement à moitié.

Ces tests, et notamment les deux derniers, montrent que les définis pluriels ont un comportement un peu à part, comparés aux autres \GN, même si, globalement, ils se rapprochent plus de la catégorie de \sicut{le N} que de \sicut{tous les N}.
\end{solu}
\end{exo}
