% -*- coding: utf-8 -*-
\begin{exo}\label{exo:PPP}
En supposant un modèle suffisamment réaliste \pagesolution{crg:PPP}
(c'est-à-dire avec {\Tps}
comprenant un très grand nombre d'instants), comparez les conditions
de vérité de $\Xlo\mP\phi$ et $\Xlo\mP\mP\mP\phi$ (en particulier en vous intéressant à ce qui distingue sémantiquement ces deux formules). 
\begin{solu}(p.~\pageref{exo:PPP})\label{crg:PPP} 

Commençons par poser les conditions
de vérité de $\Xlo\mP\phi$ et $\Xlo\mP\mP\mP\phi$ en appliquant la règle (\RSem\ref{RSemTps}) (déf. \ref{Def:SemPF} p.~\pageref{RSemTps}).
\begin{itemize} 
\item \(\denote{\Xlo\mP\phi}^{\Modele,i,g}=1\) ssi il existe un instant $i'$ tel que $i'\tprec i$ et \(\denote{\Xlo\phi}^{\Modele,i',g}=1\).

\item \(\denote{\Xlo\mP\mP\mP\phi}^{\Modele,i,g}=1\) ssi il existe trois instants $i'$, $i''$ et $i'''$ tels que $i'''\tprec i'' \tprec i' \tprec i$ et \(\denote{\Xlo\phi}^{\Modele,i''',g}=1\). En effet, $\Xlo\mP\mP\mP$ nous fait faire trois bonds dans le passé, ce qui peut donc se résumer par cette formulation de conditions de vérité.
\end{itemize}

À partir de là, on constate rapidement que \(\xlo{\mP\mP\mP\phi}\satisf\xlo{\mP\phi}\).  Si $\Xlo\mP\mP\mP$ est vraie à $i$, alors $\Xlo\phi$ est vraie à $i'''$, or $i'''$ est un instant du  passé de $i$. Donc il existe bien un instant antérieur à $i$ où $\Xlo\phi$ est vraie, ce qui fait que $\Xlo\mP\phi$ est vraie à $i$. Autrement dit : si l'on fait trois bonds dans le passé pour aller vérifier $\Xlo\phi$, on peut tout aussi bien le faire avec un seul grand bond -- qui couvre les trois précédents. 

En revanche a-t-on \(\xlo{\mP\phi}\satisf\xlo{\mP\mP\mP\phi}\) ? Pour s'en assurer, cherchons un contre-exemple. Il s'agirait d'un cas (et même d'un instant $i$) où $\Xlo\mP\phi$ est vraie et $\Xlo\mP\mP\mP\phi$ est fausse. Pour cela, il faut que {\Tps} comporte un instant $i'$ tel que 1) $i'\tprec i$, 2) $\Xlo\phi$ est vraie à $i'$, 3) il existe \emph{au maximum} un instant intercalé entre $i'$ et $i$  (mais pas deux !), et 4) $\Xlo\phi$ n'est vraie à aucun instant antérieur à $i'$.  C'est donc un cas de figure très particulier où $\Xlo\phi$ n'est vraie que dans un passé très proche de $i$.  

Autrement dit, dans tous les cas où $\Xlo\mP\mP\mP\phi$ est vraie, $\Xlo\mP\phi$ est vraie, et dans \emph{presque} tous les cas où $\Xlo\mP\phi$ est vraie, $\Xlo\mP\mP\mP\phi$ est vraie aussi. «Presque» tous les cas ne suffit pas pour conclure à une équivalence logique, mais on n'en est pas loin.  D'autant plus que si on ajoute l'hypothèse que {\Tps} contient une infinité d'instants et surtout que $\tprec$ est un ordre dense, cela implique, par définition de la densité, que pour toute paire d'instants de {\Tps}, il en existe toujours un troisième (et donc une infinité) situé entre les deux. Et dans ce cas la condition 3 ci-dessus ne peut pas être vérifiée ; il n'y a pas de contre-exemple à \(\xlo{\mP\phi}\satisf\xlo{\mP\mP\mP\phi}\) et les deux formules sont alors sémantiquement équivalentes. Si on ne pose pas cette hypothèse, elles ne sont pas équivalentes, mais elles sont sémantiquement très proches.
\end{solu}
\end{exo}
