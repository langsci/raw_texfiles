% -*- coding: utf-8 -*-
\begin{exo}\label{exo:6adj}
Dérivez compositionnellement \pagesolution{crg:6adj}%
(donc à partir d'hypothèses syntaxiques) la traduction sémantique de \sicut{petit tigre édenté}.
\begin{solu}(p.~\pageref{exo:6adj})\label{crg:6adj}

Il y a a priori deux grands types de structures syntaxiques pour analyser \sicut{petit tigre édenté}, schématisés en (a) et (b) ci-dessous ; sans entrer dans les détails de ces analyses, nous allons retenir la version (b) (qui est la plus couramment adoptée pour le français).

\ex.[(a)]
\small
\Tree[.N$'$ 
  [.AP petit ]
  [.N$'$ 
    [.N$'$ tigre ]
    [.AP édenté ]
  ] 
]
\normalsize\qquad
(b)\quad\small
\Tree[.N$'$ 
  [.N$'$ 
    [.AP petit ]
    [.N$'$ tigre ]
  ] 
    [.AP édenté ]
]
\normalsize

\sloppy

\S\ref{ss:ISSmodifieurs} nous offre plusieurs options d'analyses sémantiques ; choisissons ici celle qui traduit \sicut{petit} par \(\Xlo\lambda P\lambda x [[\prd{petit}(P)](x)]\), de type \type{\et,\et} (cf. p.~\pageref{p.petitA}).  \sicut{Tigre} se traduit par \(\Xlo\lambda x\,\prd{tigre}(x)\) ou, pour simplifier immédiatement, \prd{tigre}.  La traduction de \sicut{petit tigre} est donc \(\Xlo[\lambda P\lambda x [[\prd{petit}(P)](x)](\prd{tigre})]\) par application fonctionnelle, puis \(\Xlo\lambda x [[\prd{petit}(\prd{tigre})](x)]\) par \breduc.  
Nous pouvons traduire \sicut{édenté} comme \sicut{petit} (\ie\ \(\Xlo\lambda P\lambda y [[\prd{édenté}(P)](y)]\)), mais puisque c'est un adjectif intersectif, nous pouvons tout aussi bien le traiter comme de type \et, \(\Xlo\lambda x\,\prd{édenté}(x)\), et par la règle de \emph{modification de prédicat} (règle \ref{ri:PM}, p.~\pageref{ri:PM}), \sicut{petit tigre édenté} se traduira par \(\Xlo\lambda y[[\lambda x [[\prd{petit}(\prd{tigre})](x)](y)]\wedge[\lambda x\,\prd{édenté}(x)(y)]]\), ce qui après \breduc\ donnera :
\(\Xlo\lambda y[ [[\prd{petit}(\prd{tigre})](y)]\wedge\prd{édenté}(y)]\).


À noter que si nous avions traduit \sicut{édenté} par \(\Xlo\lambda P\lambda y [[\prd{édenté}(P)](y)]\), nous aurions obtenu au final : \(\Xlo\lambda y [[\prd{édenté}(\lambda x [[\prd{petit}(\prd{tigre})](x)])](y)]\) (ou \(\Xlo\lambda y [[\prd{édenté}([\prd{petit}(\prd{tigre})])](y)]\) par $\eta$-réduction), ce qui aura les mêmes conditions de vérité que le résultat précédent si nous posons que la dénotation de \(\Xlo\lambda P\lambda y [[\prd{édenté}(P)](y)]\) renvoie l'ensemble qui est l'intersection de l'ensemble des édentés avec l'ensemble qui est la dénotation de l'argument~(\vrb P).

\fussy
\end{solu}
\end{exo}
