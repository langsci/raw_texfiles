% -*- coding: utf-8 -*-
\begin{exo}\label{exo:PF}
Toujours à partir du modèle $\Modele$ donné en \ref{x:ModeleI}, \pagesolution{crg:PF}
calculez :\addtolength{\multicolsep}{-9pt}
\begin{multicols}{2}
\begin{enumerate}
\item \(\denote{\Xlo\mP\prd{dormir}(\cns d) \wedge \mF\prd{dormir}(\cns d)}^{\Modele,i_3,g}\)
\item \(\denote{\Xlo\mP[\prd{dormir}(\cns b) \wedge \prd{dormir}(\cns c)]}^{\Modele,i_4,g}\)
\item \(\denote{\Xlo\mP\prd{dormir}(\cns b) \wedge \mP\prd{dormir}(\cns c)}^{\Modele,i_4,g}\)
\item \(\denote{\Xlo\prd{dormir}(\cns c) \implq \mF\prd{dormir}(\cns c)}^{\Modele,i_2,g}\)
\item \(\denote{\Xlo\mF\neg\exists x\,\prd{dormir}(x)}^{\Modele,i_2,g}\)
\item \(\denote{\Xlo\neg\exists x\,\mF\prd{dormir}(x)}^{\Modele,i_2,g}\)
\end{enumerate}
\end{multicols}

\smallskip

\noindent
Quelle est la meilleure manière de traduire dans {\LO}  \sicut{tout le monde a dormi} ?

\begin{solu}(p.~\pageref{exo:PF})\label{crg:PF} 

Nous appliquons les règles (\RSem\ref{RSemTps}) de la définition \ref{Def:SemPF} p.~\pageref{RSemTps} qui, en substance, disent que  \(\denote{\Xlo\mP\phi}^{\Modele,i,g}=1\) ssi il y a un instant $i'$ avant $i$ auquel \vrb\phi\ est vraie et \(\denote{\Xlo\mF\phi}^{\Modele,i,g}=1\) ssi il y a un instant $i'$ après $i$ auquel \vrb\phi\ est vraie.

\begin{enumerate}
\item \(\denote{\Xlo\mP\prd{dormir}(\cns d) \wedge \mF\prd{dormir}(\cns d)}^{\Modele,i_3,g}=0\) parce que $\Xlo\mF\prd{dormir}(\cns d)$ est faux à $i_3$ (personne ne dort à $i_4$).

\item \(\denote{\Xlo\mP[\prd{dormir}(\cns b) \wedge \prd{dormir}(\cns c)]}^{\Modele,i_4,g}=0\) parce qu'il n'y a pas d'instants avant $i_4$ où Bruno et Charles dorment en même temps.

\item \(\denote{\Xlo\mP\prd{dormir}(\cns b) \wedge \mP\prd{dormir}(\cns c)}^{\Modele,i_4,g} =1\)
parce que $\Xlo\prd{dormir}(\cns b)$ est vraie à $i_1$ (avant $i_4$) et $\Xlo\mP\prd{dormir}(\cns c)$ est vraie à $i_2$ (avant $i_4$).
 
\item \(\denote{\Xlo\prd{dormir}(\cns c) \implq \mF\prd{dormir}(\cns c)}^{\Modele,i_2,g}=1\) parce que Charles dort à $i_2$ et aussi à $i_3$ qui est après $i_2$.

\item \(\denote{\Xlo\mF\neg\exists x\,\prd{dormir}(x)}^{\Modele,i_2,g}=1\) parce que personne ne dort à $i_4$.

\item \(\denote{\Xlo\neg\exists x\,\mF\prd{dormir}(x)}^{\Modele,i_2,g}=0\) parce que Charles dort à $i_3$.

\end{enumerate}

\sloppy

Pour traduire \sicut{tout le monde a dormi}, nous avons le choix entre \(\Xlo\mP\forall x\,\prd{dormir}(x)\) et \(\Xlo\forall x\,\mP\prd{dormir}(x)\).  La première formule dit qu'il existe un moment dans le passé où tout le monde dort ; autrement dit, tout le monde a dormi au même moment, et il n'est pas certain que la phrase du français véhicule cette condition.  La seconde formule dit que pour chaque individu, il y a un moment dans le passé durant lequel il dort ; c'est une traduction plus générale que la première et qui convient probablement mieux aux conditions de vérité de la phrase.

\fussy

\end{solu}
\end{exo}
