% -*- coding: utf-8 -*-
\begin{exo}[Structure logique des formules]\label{e:elf}
Pour embellir {\LO}, j'ai écrit les constantes non logiques en runes elfiques.
\pagesolution{crg:elf}
En vous basant sur la structure logique des formules 1--5, retrouvez quelle formule est la traduction de quelle phrase du français (indiquez simplement les correspondances lettres--chiffres).

NB : Ci-dessous, seuls \vrb x et \vrb y sont des variables.
\newcommand{\cnelf}[1]{\ensuremath{\text{\texttw{#1}}}}

%\hspace{-2cm}
%\begin{minipage}[t]{.55\textwidth}
\begin{enumerate}[label=\alph*.]
\item Il n'y a que les élus qui voient la Dame du Lac.
\item Si Merlin réussit un sort, Arthur sera étonné.
\item Lancelot n'est le cousin d'aucun chevalier.
\item Les dames ne portent jamais d'armure.
\item Les chevaliers n'ont pas tous une épée.
\end{enumerate}
%\end{minipage}
%\begin{minipage}[t]{.6\textwidth}

\begin{enumerate}
\item \(\Xlo \forall x [\cnelf{}(x) \implq \neg\exists y[\cnelf{}(y) \wedge \cnelf{}(x,y)]]\)
\item \(\Xlo\neg\forall x [\cnelf{}(x) \implq \exists y[\cnelf{}(y) \wedge \cnelf{}(x,y)]]\)
\item \(\Xlo [\exists x [\cnelf{}(x) \wedge \cnelf{}(\cnelf ,x)] \implq \cnelf{}(\cnelf )]\)
\item
\(\Xlo \forall x [\cnelf{}(x,\cnelf{}) \implq \cnelf{}(x)]\)
\item
\(\Xlo \forall x [\cnelf{}(x) \implq \neg \cnelf{}(\cnelf{},x)]\)
\end{enumerate}
%\end{minipage}

\begin{solu}(p.~\pageref{e:elf})\label{crg:elf}

a--4 ; b--3 ; c--5 ; d--1 ; e--2.
\end{solu}
\end{exo}
