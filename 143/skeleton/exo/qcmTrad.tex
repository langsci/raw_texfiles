% -*- coding: utf-8 -*-
\begin{exo}[QCM de traductions]\label{e:QCM1}
Pour chaque phrase suivante,\pagesolution{crg:QCM1}
 indiquez quelle est sa (ou ses) traduction(s) correcte(s) 
dans \LO\ parmi celles proposées.
Pour chacune des phrases, il y a au moins une traduction correcte, mais pour certaines phrases, il peut y en avoir plusieurs.

\SetEnumitemKey{qcm}{label=\Square\ \alph*.,align=left}
\begin{enumerate}[leftmargin=*]
\item Quelques étudiants connaissent toutes les réponses.
  \begin{enumerate}[qcm]
    \item \(\Xlo\forall r [\prd{étudiant}(x) \wedge \prd{connaître}(x,r)]\)
    \item \(\Xlo\exists x\forall y [[\prd{étudiant}(x) \wedge \prd{réponse}(y)]\implq \prd{connaître}(x,y)]\)
    \item \(\Xlo\forall x [\prd{réponse}(x) \implq \exists y [\prd{étudiant}(y) \wedge \prd{connaître}(y,x)]]\)
    \item \(\Xlo\exists x [\prd{étudiant}(x) \wedge \forall y [\prd{réponse}(y) \implq \prd{connaître}(x,y)]]\)
  \end{enumerate}

\item Tous les étudiants connaissent une réponse.
  \begin{enumerate}[qcm]
    \item \(\Xlo\forall x \exists y[[\prd{étudiant}(x)\wedge\prd{réponse}(y)]\implq \prd{connaître}(x,y)]\)
    \item \(\Xlo\forall x \exists y[\prd{étudiant}(x)\implq [\prd{réponse}(y) \wedge \prd{connaître}(x,y)]]\)
    \item \(\Xlo\forall x [[\prd{étudiant}(x)\wedge\exists y\,\prd{réponse}(y)]\implq \prd{connaître}(x,y)]\)
    \item \(\Xlo\forall x [\prd{étudiant}(x)\implq \exists y[\prd{réponse}(y) \implq \prd{connaître}(x,y)]]\)
  \end{enumerate}

\item Alice connaît des réponses, mais pas toutes.
  \begin{enumerate}[qcm] 
    \item \(\Xlo \forall x [\prd{réponse}(x) \implq [\prd{connaître}(\cns a,x) \vee \neg\prd{connaître}(\cns a,x)]]\)
      \item \(\Xlo\exists x [\prd{réponse}(x) \wedge \prd{connaître}(\cns a,x)] \wedge \neg\forall x [\prd{réponse}(x) \implq \prd{connaître}(\cns a,x)]\)
      \item \(\Xlo\exists x [[\prd{réponse}(x) \wedge \prd{connaître}(\cns a,x)] \wedge [\prd{réponse}(x)  \wedge \neg\prd{connaître}(\cns a,x)]]\)
      \item \(\Xlo\exists x [\prd{réponse}(x) \wedge \prd{connaître}(\cns a,x)] \wedge \neg\exists y [\prd{réponse}(y) \wedge \prd{connaître}(\cns a,y)]\)
  \end{enumerate}

\item Aucun étudiant ne connaît toutes les réponses.
  \begin{enumerate}[qcm]
    \item \(\Xlo\forall x [\prd{étudiant}(x) \implq \forall y [\prd{réponse}(y) \implq \neg\prd{connaître}(x,y)]]\)
    \item \(\Xlo\forall x [\prd{étudiant}(x) \implq \neg\forall y [\prd{réponse}(y) \implq \prd{connaître}(x,y)]]\)
    \item \(\Xlo\neg\forall x [\prd{étudiant}(x) \implq \forall y [\prd{réponse}(y) \implq \prd{connaître}(x,y)]]\)
    \item \(\Xlo\forall y [\prd{réponse}(y) \implq \neg\forall x [\prd{étudiant}(x) \implq \prd{connaître}(x,y)]]\)
  \end{enumerate}

\item Il est faux que quelques étudiants ne connaissent aucune réponse.
  \begin{enumerate}[qcm]
    \item \(\Xlo\neg\exists x[\prd{étudiant}(x)\wedge \forall y [\prd{réponse}(y)\implq\neg\prd{connaître}(x,y)]]\)
    \item \(\Xlo\forall x [\prd{étudiant}(x) \implq \forall y [\prd{réponse}(y) \implq \prd{connaître}(x,y)]]\)
    \item \(\Xlo\forall x [\prd{étudiant}(x) \implq \exists y [\prd{réponse}(y) \wedge \neg\prd{connaître}(x,y)]]\)
    \item \(\Xlo\neg\forall y [\prd{réponse}(y) \implq \exists x [\prd{étudiant}(x) \wedge \neg\prd{connaître}(x,y)] ]\)
  \end{enumerate}

\item Alice connaît une seule réponse.
  \begin{enumerate}[qcm]
    \item \(\Xlo\exists x [\prd{réponse}(x) \wedge \prd{connaître}(\cns a,x)] \wedge \forall y [\prd{réponse}(y)\implq\neg\prd{connaître}(\cns a,y)]\)
    \item \(\Xlo\exists x [\prd{réponse}(x) \wedge \prd{connaître}(\cns a,x)] \wedge \exists y [\prd{réponse}(y) \wedge \neg\prd{connaître}(\cns a,y)] \)
    \item \(\Xlo\exists x [\prd{réponse}(x) \wedge \forall y [[\prd{réponse}(y) \wedge \prd{connaître}(\cns a,y)]\implq y=x]]\)
    \item \(\Xlo\exists x [[\prd{réponse}(x) \wedge \prd{seul}(x)] \wedge \prd{connaître}(\cns a,x)]\)
  \end{enumerate}

\item Seuls les tricheurs connaissent toutes les réponses.
  \begin{enumerate}[qcm]
    \item \(\Xlo\forall x [\prd{tricheur}(x) \implq \forall y [\prd{réponse}(y) \implq \prd{connaître}(x,y)]]\)
    \item \(\Xlo\forall x [\forall y [\prd{réponse}(y) \implq \prd{connaître}(x,y)] \implq \prd{tricheur}(x)]\)
    \item \(\Xlo\forall x \forall y [[\prd{réponse}(y) \implq \prd{connaître}(x,y)] \implq \prd{tricheur}(x)]\)
    \item \(\Xlo\neg\exists x [\neg\prd{tricheur}(x) \wedge \forall y [\prd{réponse}(y) \implq \prd{connaître}(x,y)]]\)
  \end{enumerate}
\end{enumerate}

\begin{solu}(p.~\pageref{e:QCM1})\label{crg:QCM1}

\begin{enumerate}
\item d. %1
Remarque : c pourrait également être une traduction correcte, mais cette formule signifie  \sicut{pour chaque réponse, il y a au moins un étudiant qui la connaît} et il semble que ce ne soit pas une interprétation naturelle de la phrase 1.
\item b. %2
Sachant que cette formule est équivalente à \(\Xlo\forall x [\prd{étudiant}(x)\implq \exists y[\prd{réponse}(y) \wedge \prd{connaître}(x,y)]]\)

\item b. %3
Les formules c et d sont contradictoires. La formule a est une tautologie. 

\item b. %4
La formule a signifie que chaque étudiant ne connaît aucune réponse. 


\item a. %5
\item c. %6
\item b, d.  %7
(équivalentes) 

\end{enumerate}
\end{solu}
\end{exo}
