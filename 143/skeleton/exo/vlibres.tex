% -*- coding: utf-8 -*-
\begin{exo}\label{exo:vlibr}
Pour chacune des formules suivantes, dites :
\pagesolution{crg:vlibr}%
$i$) quelle est la portée de chaque quantificateur, 
$ii$) quelles sont les occurrences de variables libres (s'il y en a),
et $iii$) et par quels quantificateurs sont liées les autres variables.

 \begin{enumerate}
 \item \(\Xlo\exists x [\prd{aimer}(x,y) \wedge \prd{âne}(x)]\)
 \item \(\Xlo\exists x\, \prd{aimer}(x,y) \wedge \prd{âne}(x)\)
 \item \(\Xlo\exists x \exists y\,  \prd{aimer}(x,y) \implq \prd{âne}(x)\)
 \item \(\Xlo\forall x [\exists y \, \prd{aimer}(x,y) \implq \prd{âne}(x)]\)
 \item \(\Xlo\neg\exists x \exists y\,  \prd{aimer}(x,y) \implq \prd{âne}(x)\)
\item \(\Xlo\neg \prd{âne}(x) \implq [\neg \forall y [\neg \prd{aimer}(x,y) \vee \prd{âne}(x)] \implq
\prd{elfe}(y)]\)
\item \(\Xlo\neg \exists x [\prd{aimer}(x,y) \vee \prd{âne}(y)]\)
\item \(\Xlo\neg \exists x\, \prd{aimer}(x,x) \vee \exists y\, \prd{âne}(y)\)
\item \(\Xlo\forall x \forall y [[\prd{aimer}(x,y) \wedge \prd{âne}(y)] \implq \exists z\, \prd{mari-de}(x,z)]\)
\item \(\Xlo\forall x [\forall y\, \prd{aimer}(y,x) \implq \prd{âne}(y)]\)
 \end{enumerate}
\begin{solu} 
(p.~\pageref{exo:vlibr})\label{crg:vlibr} 

Nous encadrons chaque quantificateur et sa portée.  Les
variables libres sont notées en gras ($\vfree{x}$); les autres sont liées par le quantificateur indiqué par une flèche. L'exercice exploite les définitions \ref{d:portee} (p.~\pageref{d:portee}) et \ref{d:vlibr} (p.~\pageref{d:vlibr}).

\medskip

\newpsstyle{liage}{nodesepA=1pt,nodesepB=2pt,arrows=<-,linecolor=gray,angle=90,armA=1.2ex,armB=1ex}
 \begin{enumerate}[itemsep=2.5ex]
 \item \(\Xlo\fscp{\rnode{Q1}{\exists x} [\prd{aimer}(\rnode{x1}{x},\vfree{y}) \wedge \prd{âne}(\rnode{x2}{x})]}\)%
\ncbar[style=liage,offsetA=-2pt]{Q1}{x1}%
\ncbar[style=liage,offsetA=2pt,armA=1.8ex]{Q1}{x2}%
\hfill formule existentielle

 \item \(\Xlo\fscp{\rnode{Q1}{\exists x}\, \prd{aimer}(\rnode{x1}{x},\vfree{y})} \Xlo\wedge \prd{âne}(\vfree{x})\)
\ncbar[style=liage]{Q1}{x1}%
\hfill conjonction

 \item \(\Xlo\fscp{\rnode{Q1}{\exists x} \fscp{\rnode{Q2}{\exists y}\,  \prd{aimer}(\rnode{x1}{x},\rnode{y1}{y})}} \implq \prd{âne}(\vfree{x})\)
\ncbar[style=liage,armA=2ex]{Q1}{x1}%
\ncbar[style=liage,armA=1.5ex]{Q2}{y1}%
\hfill implication

 \item \(\Xlo\fscp{\rnode{Q1}{\exists x} [\fscp{\rnode{Q2}{\exists y} \, \prd{aimer}(\rnode{x1}{x},\rnode{y1}{y})} \implq \prd{âne}(\rnode{x2}{x})]}\)
\ncbar[style=liage,armA=2ex,offsetA=-2pt]{Q1}{x1}%
\ncbar[style=liage,armA=1.5ex]{Q2}{y1}%
\ncbar[style=liage,armA=2.5ex,offsetA=2pt]{Q1}{x2}%
\hfill formule existentielle

 \item \(\Xlo\neg\,\fscp{\rnode{Q1}{\exists x} \fscp{\rnode{Q2}{\exists y}\,  \prd{aimer}(\rnode{x1}{x},\rnode{y1}{y})}} \implq \prd{âne}(\vfree{x})\)
\ncbar[style=liage,armA=2ex]{Q1}{x1}%
\ncbar[style=liage,armA=1.5ex]{Q2}{y1}%
\hfill implication

\item \(\Xlo\neg \prd{âne}(\vfree{x}) \implq [\neg \fscp{\rnode{Q1}{\forall y} [\neg \prd{aimer}(\vfree{x},\rnode{y1}{y}) \vee \prd{âne}(\vfree{x})]} \implq
\prd{elfe}(\vfree{y})]\)
\ncbar[style=liage,armA=1.3ex]{Q1}{y1}%
\hfill implication

\item \(\Xlo\neg\, \fscp{\rnode{Q1}{\exists x} (\prd{aimer}(\rnode{x1}{x},\vfree{y}) \vee \prd{âne}(\vfree{y}))}\)
\ncbar[style=liage,armA=1.3ex]{Q1}{x1}%
\hfill négation

\item \(\Xlo\neg\, \fscp{\rnode{Q1}{\exists x}\, \prd{aimer}(\rnode{x1}{x},\rnode{x2}{x})} \vee \fscp{\rnode{Q2}{\exists y}\, \prd{âne}(\rnode{y1}{y})}\)
\ncbar[style=liage,offsetA=-2pt]{Q1}{x1}%
\ncbar[style=liage,armA=1.8ex,offsetA=2pt]{Q1}{x2}%
\ncbar[style=liage]{Q2}{y1}%
\hfill disjonction

\item \(\Xlo\fscp{\rnode{Q1}{\forall x} \fscp{\rnode{Q2}{\forall y} [[\prd{aimer}(\rnode{x1}{x},\rnode{y1}{y}) \wedge \prd{âne}(\rnode{y2}{y})] \implq \fscp{\rnode{Q3}{\exists z}\, \prd{mari-de}(\rnode{x2}{x},\rnode{z1}{z})}]}}\)
\ncbar[style=liage,offsetA=-2pt,armA=2.5ex]{Q1}{x1}%
\ncbar[style=liage,angle=-90,armA=1.8ex]{Q2}{y1}%
\ncbar[style=liage,armA=2ex]{Q2}{y2}%
\ncbar[style=liage,offsetA=2pt,armA=3ex]{Q1}{x2}%
\ncbar[style=liage,armA=2ex]{Q3}{z1}%
\hfill formule universelle

\item \(\Xlo\fscp{\rnode{Q1}{\forall x} [\fscp{\rnode{Q2}{\forall y}\, \prd{aimer}(\rnode{y1}{y},\rnode{x1}{x})} \implq \prd{âne}(\vfree{y})]}\)
\ncbar[style=liage,armA=1.5ex]{Q2}{y1}%
\ncbar[style=liage,armA=2ex]{Q1}{x1}%
\hfill formule universelle
 \end{enumerate}

\end{solu}
\end{exo}
