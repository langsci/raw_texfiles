% -*- coding: utf-8 -*-
\begin{exo}\label{exo:6QG}
  Traduisez dans {\LO} les phrases suivantes :\pagesolution{crg:6QG}
  \begin{enumerate}
    \item Il y a quatre 2\textsc{CV} vertes dans le parking.
    \item Moins de la moitié des candidats ont répondu à toutes les questions.
    \item Trois stagiaires apprendront deux langues étrangères.\label{xx:QG3x2}
  \end{enumerate}
  Explicitez les conditions de vérités des formules obtenues pour \ref{xx:QG3x2}.  %Qu'observez-vous ?
\begin{solu}(p.~\pageref{exo:6QG})\label{crg:6QG}
  \begin{enumerate}
    \item Il y a quatre 2\textsc{CV} vertes dans le parking.\\
\(\Xlo\prd{Quatre}(\lambda x[\prd{2cv}(x)\wedge\prd{vert}(x)])(\lambda x\,\prd{dans}(x,\atoi y\,\prd{parking}(y)))\)

Évidemment le déterminant \prd{Quatre} est calqué, \alien{mutatis mutandis},  sur \prd{Deux} et \prd{Trois} vus en \S\ref{ss:QGDet}, p.~\pageref{xd:Deux}.

    \item Moins de la moitié des candidats ont répondu à toutes les questions.\\
Commençons par traduire \sicut{ont répondu à toutes les questions}, c'est un VP de type {\et} : \(\Xlo\lambda x\forall y[\prd{question}(y)\implq\prd{répondre}(x,y)]\). Ensuite, nous devons fournir une définition du déterminant \sicut{moins de la moitié} :\\
\(\denote{\Xlo \prd{Moins-de-la-moitié}(R)(P)}^{\Modele,w,g}=1\) ssi
\(\Card{\Ch{\denote{\Xlo R}}^{\Modele,w,g} \cap \Ch{\denote{\Xlo P}}^{\Modele,w,g}} <
\frac{\nbr1}{\nbr2}\Card{\Ch{\denote{\Xlo R}}^{\Modele,w,g}}\).
\\
À partir de là, nous pouvons traduire la phrase :\\
\(\Xlo\prd{Moins-de-la-moitié}(\prd{candidat})(\lambda x\forall y[\prd{question}(y)\implq\prd{répondre}(x,y)])\)
\\
Cette formule dit que le nombre d'éléments dans l'ensemble des candidats qui ont répondu à toutes les questions est inférieur à la moitié du nombre total de candidats. Mais il y a une autre interprétation, avec les portées inversées des quantificateurs :\\
\(\Xlo\forall y[\prd{question}(y)\implq\prd{Moins-de-la-moitié}(\prd{candidat})(\lambda x\,\prd{répondre}(x,y))]\)\\
Cette traduction dit que pour chaque question, il y a, à chaque fois, moins de la moitié des candidats qui y ont répondu.  Cette lecture exclut par exemple les scénarios où il y  aurait eu quelques questions très faciles auxquelles la plupart des candidats ont su répondre.


    \item Trois stagiaires apprendront deux langues étrangères.

\(\Xlo\prd{Trois}(\prd{stagiaire})(\lambda x\,\prd{Deux}(\lambda y[\prd{langue}(y)\wedge\prd{étranger}(y)])(\lambda y\,\prd{apprendre}(x,y)))\)\\
Cette formule est vraie ssi l'intersection de l'ensemble des stagiaires et de ceux qui ont appris deux langues étrangères contient (au moins) 3 individus.  Il peut donc y avoir 6 langues différentes en jeu dans ce genre de situations.

\(\Xlo\prd{Deux}(\lambda y[\prd{langue}(y)\wedge\prd{étranger}(y)])(\lambda y\,\prd{Trois}(\prd{stagiaire})(\lambda x\,\prd{apprendre}(x,y)))\)\\
Cette traduction présente la lecture avec portées inversées que l'on obtient en appliquant \QRa\ (cf. \S\ref{ss:QR}).  Elle est vraie ssi l'intersection de l'ensemble des langues étrangères et de l'ensemble des choses qui ont été apprises par trois stagiaires contient (au moins) 2 éléments.  Il peut donc y avoir au total 6 stagiaires dans l'histoire (par exemple trois qui apprennent le japonais, et trois autres qui apprennent le russe). 
  \end{enumerate}
\end{solu}
\end{exo}
