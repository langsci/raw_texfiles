% -*- coding: utf-8 -*-
\begin{exo}\label{exo:1CL}
Pour chaque paire de phrases suivantes,\pagesolution{crg:1CL}
dites, en le démontrant, si la phrase (b) est ou non une conséquence logique de la phrase (a).
\begin{enumerate}
\item 
  \begin{enumerate}
  \item Ta soupe est chaude.
  \item Ta soupe n'est pas froide.
  \end{enumerate}
\item 
  \begin{enumerate}
  \item Ta soupe est chaude.
  \item Ta soupe n'est pas brûlante.
  \end{enumerate}
\item 
  \begin{enumerate}
  \item Joseph a prouvé que c'est le colonel Moutarde le coupable.
  \item C'est le colonel Moutarde le coupable.
  \end{enumerate}
\item 
  \begin{enumerate}
  \item Des étudiants ont eu la moyenne au partiel. 
  \item Des étudiants n'ont pas eu la moyenne au partiel. 
  \end{enumerate}
\end{enumerate}
\begin{solu}(p.~\pageref{exo:1CL})\label{crg:1CL}

\sloppy

Nous allons démontrer les réponses en appliquant la méthode des contre-exemples (\S\ref{s:conseql}, p~\pageref{p.contrex}):  pour chaque paire de phrases, nous essayons d'imaginer une situation par rapport à laquelle (a) est vraie \emph{et} (b) est fausse.

\fussy
\begin{enumerate}
\item 
La situation serait telle que la soupe est chaude \emph{et} la soupe est froide (puisqu'il est faux qu'elle n'est pas froide). C'est contradictoire, et nous en concluons que (a) $\satisf$ (b).
\item 
Ici la situation serait telle que la soupe est chaude et brûlante. C'est tout à fait possible (si la soupe est précisément brûlante) puisque quelque chose de brûlant est forcément chaud. C'est un contre-exemple, et donc (a) $\not\satisf$ (b).
\item 
Si nous sommes dans une situation où le colonel Moutarde \emph{n'est pas} coupable, alors on ne peut pas \emph{prouver} qu'il est coupable. À la rigueur, Joseph pourrait argumenter ou soutenir (par erreur ou malhonnêteté) que le colonel est coupable, mais objectivement nous n'aurons pas le droit de dire qu'il a \emph{prouvé} la culpabilité.  Par conséquent (a) $\satisf$ (b).
\item 
Ici il faut être vigilant : une situation où (b) est fausse  (il est faux que des étudiants n'ont pas eu la moyenne au partiel) est telle que \emph{tous} les étudiants ont eu la moyenne.  Dans une telle situation, (a) peut-elle être vraie ? Oui, nécessairement, car si (a) était fausse cela voudrait dire qu'aucun étudiant n'a eu la moyenne (ce qui serait contradictoire avec la première hypothèse).  Nous pouvons avoir une situation où (a) est vraie et (b) fausse, et donc (a) $\not\satisf$ (b).
\\
NB : le fait que nous comprenons souvent (a) comme s'accompagnant aussi de la vérité de (b) ne relève pas de la conséquence logique mais d'une implicature conversationnelle (cf. \S\ref{ss:implicatures}).
\end{enumerate}
\end{solu}
\end{exo}
