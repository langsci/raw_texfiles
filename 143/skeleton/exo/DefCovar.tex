% -*- coding: utf-8 -*-
\begin{exo}\label{exo:DefCovar}
Trouvez deux exemples (suffisamment différents) de phrases avec un {\GN}
\pagesolution{crg:DefCovar}%
défini singulier (en \sicut{le} ou \sicut{la}) qui (en dépit de ce que nous avons
dit et montré ci-dessus) covarie avec un autre élément de la phrase.
Et tentez une explication.
\begin{solu} (p.~\pageref{exo:DefCovar})\label{crg:DefCovar}

Il y a différentes façons de construire un {\GN} défini qui covarie
nettement avec une autre expression de la phrase, mais toutes se
ramènent généralement au même phénomène.

Premier type de constructions : une expression quantifiée apparaît en
complément ou en modifieur de la tête nominale du {\GN} défini.  Exemple :

\begin{enumerate}[label=(\arabic*)]
\item \label{x:def1}
Paul a contrefait \underline{la signature de \emph{chaque membre
  de sa famille}}.
\end{enumerate}


Dans (\ref{x:def1}), il est bien question d'autant de signatures qu'il
y a de membres de la famille, il y a donc bien une multiplication,
c'est-à-dire un effet de covariation.

Deuxième type de construction : le {\GN} défini contient un pronom (ou un
élément anaphorique) qui est lié d'une manière ou d'une autre (par
exemple par son antécédent) à une expression quantifée.  Exemple :

\begin{enumerate}[label=(\arabic*),resume]
\item \label{x:def2}
\emph{Tout dompteur$_i$} craint \underline{le lion qu'\emph{il$_i$}
est en train de dompter}. 

\item
\emph{Chaque candidat$_i$} doit remplir \underline{le formulaire qui
\emph{lui$_i$} a été remis}.
\end{enumerate}

Là encore, dans (\ref{x:def2}) par exemple, il y a autant de lions que
de dompteurs, bien que le {\GN} soit singulier.
Remarque : l'élément anaphorique en question peut être implicite :

\begin{enumerate}[label=(\arabic*),resume]
\item \emph{Dans chaque appartement que nous avons visité}, \underline{la salle de bain} était minuscule.
\end{enumerate}

Ici il s'agit de la salle de bain «de lui», où \sicut{lui}
est l'appartement dont il est question à chaque fois\footnote{Ce phénomène est usuellement désigné par le terme d'\emph{anaphore associative}.}.

Pour les éléments d'explication, voir \S\ref{sss:defdep} p. \pageref{sss:defdep}.

\end{solu}
\end{exo}
