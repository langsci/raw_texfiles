% -*- coding: utf-8 -*-
\begin{exo}\label{exo:6VPf}
Nous traitons uniformément tous les DP comme étant de type \ett.  
\pagesolution{crg:6VPf}
Supposons maintenant que nous tenions à tout prix à maintenir que dans la composition du TP, le VP continue à dénoter la fonction et que le DP sujet soit son argument.  Quels devraient être alors le type et la traduction d'un verbe comme \sicut{dormir}?
\begin{solu}(p.~\pageref{exo:6VPf})\label{crg:6VPf}

\sloppy

Si  nous voulons que les VP soient la fonction principale de la phrase, alors ceux-ci devront être de type \type{\ett,t}, attendant un DP sujet de type \ett\ pour produire une expression de type \typ t.  \sicut{Dormir} se traduira alors par \(\Xlo\lambda X[X(\lambda x\,\prd{dormir}(x))]\), avec $\vrb X\in\VAR_{\ett}$.
Ainsi nous pourrons dériver la traduction de \sicut{Alice dort} en composant 
\(\Xlo[\lambda X[X(\lambda x\,\prd{dormir}(x))](\lambda P[P(\cns a)])]\) qui, par \breduc s, se simplifie en 
\(\Xlo[\lambda P[P(\cns a)](\lambda x\,\prd{dormir}(x))]\)\footnote{Ce qui nous fait revenir évidemment à ce dont nous partons p.~\pageref{p.PaV}.}, 
puis en 
\(\Xlo[\lambda x\,\prd{dormir}(x)(\cns a)]\) et
\(\Xlo\prd{dormir}(\cns a)\).

\fussy
\end{solu}
\end{exo}
