% -*- coding: utf-8 -*-
\begin{exo}\label{exo:redictoSue}
Reprenons en détail l'analyse d'un {\GN} indéfini ambigu comme en :
\pagesolution{crg:redictoSue}
\begin{enumerate}[label=(\alph*)]
\item Sue pense qu'un républicain va remporter l'élection.
\end{enumerate}

Pour simplifier l'exercice, nous considérerons que le groupe verbal
\sicut{va remporter l'élection} se traduit par le prédicat à une place
\prd{élu} (ainsi $\Xlo\prd{élu}(x)$ signifie $\vrb x$ est ou sera élu) ; de plus
nous ne tiendrons pas compte du temps verbal.

Soit le modèle-jouet suivant
\(\Modele=\tuple{\Unv{A},\Unv{W},\FI}\), avec
\(\Unv{A}=\set{\Obj{Barry} ; \Obj{Johnny} ; \Obj{Sue}}\), et
\(\Unv{W}=\set{\w_1; \w_2; \w_3; \w_4; \w_5; \w_6; \w_7; \w_8}\).

On supposera que dans le modèle {\Modele}, on ne peut pas être à la
fois républicain et démocrate, et qu'une seule personne peut remporter
l'élection. 
Le tableau~\ref{t:republ} nous donne les dénotations de \prd{républicain}, \prd{démocrate} et
  \prd{élu} dans {\Modele}.
Et complétons ce modèle, en ajoutant que dans les mondes $\w_1$ et $\w_7$, \Obj{Sue} croit la proposition \set{\w_1;\w_2;\w_3;\w_6}, dans les mondes $\w_2$, $\w_3$ et $\w_4$, elle croit la proposition \set{\w_2;\w_4;\w_6;\w_8}, et dans les mondes $\w_5$, $\w_6$ et $\w_8$, elle croit la proposition \set{\w_1;\w_3;\w_5;\w_7}.
%\fixme{Croyances de Sue}

\begin{table}[h]
\begin{bigcenter}
\caption{Interprétations de \prd{républicain}, \prd{démocrate} et
  \prd{élu} dans {\Modele}}\label{t:republ}
{\small
\begin{tabular}{@{}l@{}l@{}}\lsptoprule
\begin{tabular}{l}
\(\FI(\w_1,\prd{républicain})=\set{\Obj{Barry} ; \Obj{Johnny}}\), \\
\(\FI(\w_1,\prd{démocrate})=\eVide\),\\
\(\FI(\w_1,\prd{élu})=\set{\Obj{Barry}}\) ;
\\%\hline
\(\FI(\w_2,\prd{républicain})=\set{\Obj{Barry} ; \Obj{Johnny}}\), \\
\(\FI(\w_2,\prd{démocrate})=\eVide\),\\
\(\FI(\w_2,\prd{élu})=\set{\Obj{Johnny}}\) ;
\\%\hline
\(\FI(\w_3,\prd{républicain})=\set{\Obj{Barry}}\), \\
\(\FI(\w_3,\prd{démocrate})=\set{\Obj{Johnny}}\),\\
\(\FI(\w_3,\prd{élu})=\set{\Obj{Barry}}\) ;
\\
\(\FI(\w_4,\prd{républicain})=\set{\Obj{Barry}}\), \\
\(\FI(\w_4,\prd{démocrate})=\set{\Obj{Johnny}}\),\\
\(\FI(\w_4,\prd{élu})=\set{\Obj{Johnny}}\) ;
\end{tabular}
&
\begin{tabular}{l}
\(\FI(\w_5,\prd{républicain})=\set{\Obj{Johnny}}\),\\ 
\(\FI(\w_5,\prd{démocrate})=\set{\Obj{Barry}}\),\\
\(\FI(\w_5,\prd{élu})=\set{\Obj{Barry}}\) ;
\\%\hline
\(\FI(\w_6,\prd{républicain})=\set{\Obj{Johnny}}\), \\
\(\FI(\w_6,\prd{démocrate})=\set{\Obj{Barry}}\),\\
\(\FI(\w_6,\prd{élu})=\set{\Obj{Johnny}}\) ;
\\%\hline
\(\FI(\w_7,\prd{républicain})=\eVide\), \\
\(\FI(\w_7,\prd{démocrate})=\set{\Obj{Barry} ; \Obj{Johnny}}\),\\
\(\FI(\w_7,\prd{élu})=\set{\Obj{Barry}}\) ;
\\%\hline
\(\FI(\w_8,\prd{républicain})=\eVide\), \\
\(\FI(\w_8,\prd{démocrate})=\set{\Obj{Barry} ; \Obj{Johnny}}\),\\
\(\FI(\w_8,\prd{élu})=\set{\Obj{Johnny}}\) ;
\end{tabular}
\\\lspbottomrule
\end{tabular} }
\end{bigcenter}
\end{table}

Dans tout cet exercice, nous manipulerons les intensions des formules (\ie\ les propositions) directement comme des ensembles de mondes.


\begin{enumerate}
\item Quelle est l'intension de \(\Xlo\exists x [\prd{républicain}(x)
    \wedge \prd{élu}(x)]\) dans {\Modele} et par rapport à une
    assignation $g$ quelconque ?
\item Quelle est l'intension de \(\Xlo\prd{élu}(x)\) dans {\Modele}, par
  rapport à une assignation $g_1$ telle que $g_1(\vrb x)=\Obj{Barry}$ ? Et par rapport à l'assignation $g_2$ telle que $g_2(\vrb x)=\Obj{Johnny}$ ?
\item Quelle est l'intension de \(\Xlo\prd{penser}(\cns s,\Intn\prd{élu}(x))\) par rapport à $g_1$ ?  Et par rapport à $g_2$ ? (\cns s dénote \Obj{Sue}, naturellement)
\item Quelle est l'intension de \(\Xlo\exists x [\prd{républicain}(x) \wedge \prd{penser}(\cns s,\Intn\prd{élu}(x))]\) ?
\item Quelle est l'intension de \(\Xlo\prd{penser}(\cns s,\Intn\exists x [\prd{républicain}(x) \wedge \prd{élu}(x)])\) ?
\end{enumerate}

\begin{solu}(p.~\pageref{exo:redictoSue})\label{crg:redictoSue}
\begin{enumerate}
\item Nous allons présenter les intensions de formules (\ie\ les propositions) sous la forme d'ensembles de mondes. 
L'intension de \(\Xlo\exists x [\prd{républicain}(x) \wedge \prd{élu}(x)]\) est, par définition (p.~\pageref{pt:prop°}), l'ensemble de tous les mondes dans lesquels celui qui est élu est un républicain. D'après le tableau~\ref{t:republ}, p.~\pageref{t:republ}, il s'agit de l'ensemble \set{\w_1;\w_2;\w_3;\w_6}.

\item Avec $g_1$ qui donne $g_1(\vrb x)=\Obj{Barry}$, l'intension de \(\Xlo\prd{élu}(x)\) est l'ensemble de tous les mondes où \Obj{Barry} est élu. C'est-à-dire : \set{\w_1;\w_3;\w_5;\w_7}. 
Et avec $g_2$ qui donne $g_2(\vrb x)=\Obj{Johnny}$, l'intension de \(\Xlo\prd{élu}(x)\) est l'ensemble de tous les mondes où \Obj{Johnny} est élu : \set{\w_2;\w_4;\w_6;\w_8}.
\item Par rapport à $g_1$, l'intension de \(\Xlo\prd{penser}(\cns s,\Intn\prd{élu}(x))\)  est \set{\w_2;\w_3;\w_4} (d'après les indications de l'énoncé de l'exercice).  Et par rapport à $g_2$, l'intension de la formule  est \set{\w_5;\w_6;\w_8}.
\item Donc dans $\w_2$, $\w_3$ et $\w_4$, \Obj{Sue} pense que \Obj{Barry} va être élu, et dans $\w_5$, $\w_6$ et $\w_8$ elle pense que \Obj{Johnny} va être élu.  Dans $\w_2$, $\w_3$ et $\w_4$, \Obj{Barry} est républicain et dans $\w_5$ et $\w_6$  \Obj{Johnny} est républicain.  Donc l'intension de \(\Xlo\exists x [\prd{républicain}(x) \wedge \prd{penser}(\cns s,\Intn\prd{élu}(x))]\) est \set{\w_2;\w_3;\w_4;\w_5;\w_6}.
\item D'après l'énoncé de l'exercice et le résultat de la question 1, l'intension de la formule \(\Xlo\prd{penser}(\cns s,\Intn\exists x [\prd{républicain}(x) \wedge \prd{élu}(x)])\) est \set{\w_1;\w_7}.
\end{enumerate}
\end{solu}
\end{exo}
