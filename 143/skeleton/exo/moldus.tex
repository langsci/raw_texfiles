\begin{exo}\label{exo:moldu}
Cet exercice, en plus d'être une application de ce qui est présenté
\pagesolution{crg:moldu}
dans cette section, met le doigt sur un petit problème d'analyse
sémantique. 

\begin{enumerate}
\item {\itshape{\small Parmi le personnel qui travaille au palais de
    Buckingham, il y a  
principalement des gens ordinaires, comme vous et moi, c'est-à-dire
des \emph{moldus}\footnote{En VO, des \emph{muggles}, si vous préférez.} ; mais il y a aussi quelques magiciens.  Bien sûr ceux-ci
font très attention de garder secrète leur véritable identité.}
Si bien que} la Reine d'Angleterre elle-même croit que tous les
magiciens qui la côtoient  sont des moldus. 
\end{enumerate}

La phrase qui nous intéresse ici est la dernière de ce paragraphe ; le
texte en italique est là pour fixer un contexte et orienter la
compréhension.  Cette phrase est ambiguë (au moins théoriquement). Commencez par expliciter 
informellement cette ambiguïté à l'aide de deux paraphrases précises et suffisamment distinctes, puis donnez les deux traductions
possibles de la phrase dans {\LO}\footnote{Négligez la contribution de \sicut{elle-même}.}. 

\begin{solu}(p.~\pageref{exo:moldu})\label{crg:moldu}
L'ambiguïté de cette phrase repose bien sûr sur l'opposition de re
vs.\ de dicto du {\GN} \sicut{tous les magiciens qui la côtoient}.

Selon la lecture de re, la phrase signifie : \og pour tout magicien qui côtoie la reine, dans le monde où nous nous plaçons, celle-ci croit qu'il est un moldu \fg.

Preuve : on décrit un monde $w$.

Les deux lectures se traduisent ainsi :

\begin{enumerate}
\item de dicto :\\
 \(\Xlo\prd{croire}(\atoi x\,\prd{rda}(x),\Intn\forall y
          [[\prd{magicien}(y)\wedge\prd{côtoyer}(y,\atoi x\,\prd{rda}(x))]\implq \prd{muggle}(y)])\)

\item de re :\\
 \(\Xlo\forall y
          [[\prd{magicien}(y)\wedge\prd{côtoyer}(y,\atoi
              x\,\prd{rda}(x))]\implq  \prd{croire}(\atoi x\,\prd{rda}(x),\Intn\prd{muggle}(y))]\)
\end{enumerate}


Le problème vient de la lecture de re : il semblerait bien que le
{\GN} universel a une portée large, \emph{au-delà de la proposition
  syntaxique où il apparaît}.

C'est d'autant plus embêtant que c'est cette lecture qui est vraiment
la plus naturelle des deux.  En effet la lecture de dicto attribue une
pensée contradictoire à la Reine d'Angleterre.

\end{solu}
\end{exo}
