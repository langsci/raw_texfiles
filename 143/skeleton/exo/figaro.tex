% -*- coding: utf-8 -*-
\begin{exo}[Interprétation dans un modèle]
\label{exo:figaro}
Soit le modèle suivant \(\Modele=\tuple{\Unv A,\FI}\), avec :
\pagesolution{crg:figaro}

%\sloppy
\begin{itemize}\raggedright
\item \(\Unv A=\set{\Obj{Almv}; \Obj{Rosn}; \Obj{Figr}; \Obj{Suzn}; \Obj{Mrcl};
  \Obj{Chrb}; \Obj{Fnch}; \Obj{Antn}; \Obj{Bart}}\) ;

\item \(\FI(\cnsi a1) = \Obj{Almv} ; 
\FI(\cns r) = \Obj{Rosn} ; 
\FI(\cnsi f1) = \Obj{Figr} ; 
\FI(\cns s) = \Obj{Suzn} ; 
\FI(\cns m) = \Obj{Mrcl} ;
\FI(\cns c) = \Obj{Chrb} ; 
\FI(\cnsi f2) = \Obj{Fnch} ; 
\FI(\cnsi a2) = \Obj{Antn} ; 
\FI(\cns b) = \Obj{Bart}\) ;

\item 
\(\FI(\prd{homme})=\set{\Obj{Almv}; \Obj{Figr}; \Obj{Chrb}; \Obj{Antn}; \Obj{Bart}}\) ;
\item \label{modele1}
\(\FI(\prd{femme})=\set{\Obj{Rosn};\Obj{Suzn}; \Obj{Mrcl}; \Obj{Fnch}}\) ;

\item 
\(\FI(\prd{domestique})=\set{\Obj{Figr}; \Obj{Suzn}; \Obj{Fnch}; \Obj{Antn}}\) ;
\item \(\FI(\prd{noble}) = \set{\Obj{Almv}; \Obj{Rosn};  \Obj{Chrb} }\) ;

\item 
\(\FI(\prd{roturier}) = \set{\Obj{Figr}; \Obj{Suzn}; \Obj{Mrcl};
 \Obj{Fnch}; \Obj{Antn}; \Obj{Bart}}\) ;

\item 
\(\FI(\prd{comte}) = \set{\Obj{Almv}}\) ;
\qquad \quad 
\(\FI(\prd{comtesse}) = \set{\Obj{Rosn}}\) ;

\item 
\(\FI(\prd{infidèle}) = \set{\Obj{Almv}}\) ;
\qquad 
%\item 
\(\FI(\prd{volage}) = \set{\Obj{Almv}; \Obj{Chrb}}\) ;
\item
%\item 
\(\FI(\prd{triste}) = \set{\Obj{Rosn}}\) ;

\item 
\(\FI(\prd{époux-de}) = \set{\tuple{\Obj{Almv}, \Obj{Rosn}};\tuple{\Obj{Figr}, \Obj{Suzn}}}\) ;
\item 
\(\FI(\prd{épouse-de}) = \set{\tuple{\Obj{Rosn},\Obj{Almv}};\tuple{\Obj{Suzn},\Obj{Figr}}}\) ;

\item 
\(\FI(\prd{père-de}) = \set{\tuple{\Obj{Antn},\Obj{Fnch}}}\) ;

\item 
\(\FI(\prd{aimer}) = \set{\tuple{\Obj{Figr},
    \Obj{Suzn}};\tuple{\Obj{Suzn},\Obj{Figr}}; \tuple{\Obj{Almv},
    \Obj{Suzn}}; \tuple{\Obj{Rosn},\Obj{Almv}}; \tuple{\Obj{Chrb},
    \Obj{Rosn}}; \tuple{\Obj{Chrb}, \Obj{Suzn}}; \tuple{\Obj{Chrb},
    \Obj{Fnch}}; \tuple{\Obj{Fnch},\Obj{Chrb}}; \tuple{\Obj{Mrcl},\Obj{Figr}}}\)
\end{itemize}

\fussy

%% On considère également les traductions suivantes des noms propres (du
%% français) en constantes de {\LCP}:
%% \begin{itemize}
%% \item 
%% Almaviva $\leadsto$ $\cns a_1$;
%% Rosine $\leadsto$ $\cns r$;
%% Figaro $\leadsto$ $\cns f_1$;
%% Suzanne $\leadsto$ $\cns s$;
%% Marceline $\leadsto$ $\cns m$;
%% Chérubin $\leadsto$ $\cns c$;
%% Fanchette $\leadsto$ $\cns f_2$;
%% Antonio $\leadsto$ $\cns a_2$;
%% Bartholo $\leadsto$ $\cns b$;
%% \end{itemize}

\medskip

\begin{enumerate}
\item Donnez la dénotation dans {\Modele} de chacune des formules suivantes.
Vous justifierez vos réponses sans entrer dans le détail du calcul formel, mais en explicitant les étapes de votre raisonnement. 

%, puis pour chacune proposez une
%  traduction possible en français.

\begin{enumerate}
%\item \([\prd{noble}(\cns a_1) \wedge \prd{roturier}(\cns f_1)]\)
\item \(\Xlo\exists x [\prd{femme}(x) \wedge \neg\prd{triste}(x)]\)

\item \(\Xlo\forall x [\prd{homme}(x) \implq [\prd{infidèle}(x) \vee \prd{volage}(x)]]\)

\item \(\Xlo\exists x\exists y [\prd{aimer}(x,y) \wedge \prd{aimer}(y,x)]\)

\item \(\Xlo[\prd{aimer}(\cnsi a1, \cns s) \implq \exists x\, \neg\prd{aimer}(x,\cnsi a1)]\)

\item \(\Xlo\neg\exists x [\prd{femme}(x) \wedge \exists y\,\prd{aimer}(x,y)]\)

\item  \(\Xlo\forall x [[\prd{homme}(x) \wedge \exists y
  [\prd{épouse-de}(y,x) \wedge \neg\prd{aimer}(x,y)]] \implq \prd{volage}(x)]\)
\end{enumerate}

\textbf{Conseil :} commencez par traduire les formules en phrases de la langue, puis demandez-vous si ces phrases sont vraies ou fausses dans {\Modele} (ainsi vous savez à l'avance la réponse que vous devez trouver). 

\bigskip

\item 
Modifiez le modèle {\Modele} pour faire en sorte que : \Obj{Rosn} ne soit plus triste, que \Obj{Fnch} et \Obj{Chrb} soient mariés et que tous les maris aiment leur femme et seulement elle.
\end{enumerate}

\begin{solu} (p.~\pageref{exo:figaro})\label{crg:figaro}
%\begin{enumerate}\sloppy
%\item 

1. Dénotation des formules.
\begin{enumerate}[label=\alph*.]
%\sloppy
\item \(\Xlo\exists x [\prd{femme}(x) \wedge \neg\prd{triste}(x)]\)

Pour que \(\denote{\Xlo\exists x [\prd{femme}(x) \wedge \neg\prd{triste}(x)]}^{\Modele}=1\), il faut trouver au moins une valeur de \vrb x telle que \(\Xlo\prd{femme}(x)\) et \(\Xlo\neg\prd{triste}(x)\) soient vraies dans \Modele.  Une valeur de \vrb x est une constante de \LO\ ou, ce qui revient au même, un individu de \Unv A. 

Une telle valeur existe bien, par exemple \Obj{Suzn}.  En effet \Obj{Suzn} appartient à \(\FI(\prd{femme})\) (l'ensemble des femmes) et n'appartient pas à \(\FI(\prd{triste})\) (l'ensemble des individus tristes).

Donc \(\denote{\Xlo\exists x [\prd{femme}(x) \wedge \neg\prd{triste}(x)]}^{\Modele}=1\).

La formule correspond à la phrase \sicut{il y a une femme qui n'est pas triste}.

\item \(\Xlo\forall x [\prd{homme}(x) \implq [\prd{infidèle}(x) \vee \prd{volage}(x)]]\)

\sloppy

Pour que 
\(\denote{\Xlo\forall x [\prd{homme}(x) \implq [\prd{infidèle}(x) \vee \prd{volage}(x)]]}^{\Modele}=1\), il faut que, pour chaque valeur que l'on peut assigner à  \vrb x, lorsque l'on répète le calcul de \(\denote{\Xlo [\prd{homme}(x) \implq [\prd{infidèle}(x) \vee \prd{volage}(x)]]}^{\Modele}\) on trouve $1$ à chaque fois.

On peut montrer rapidement que la formule est fausse en choisissant une valeur de \vrb x telle que \(\denote{\Xlo [\prd{homme}(x) \implq [\prd{infidèle}(x) \vee \prd{volage}(x)]]}^{\Modele}=0\) ; ce sera un individu qui appartient à l'ensemble des hommes mais qui n'appartient ni à l'ensemble des infidèles ni à celui des volages.  Par exemple \Obj{Figr} fait l'affaire.

Donc \(\denote{\Xlo\forall x [\prd{homme}(x) \implq [\prd{infidèle}(x) \vee \prd{volage}(x)]]}^{\Modele}=0\).

La formule correspond à \sicut{tous les hommes sont infidèles ou volages}.

\item \(\Xlo\exists x\exists y [\prd{aimer}(x,y) \wedge \prd{aimer}(y,x)]\)


\(\denote{\Xlo\exists x\exists y [\prd{aimer}(x,y) \wedge \prd{aimer}(y,x)]}^{\Modele}=1\)  ssi on trouve une valeur pour \vrb x telle que 
\(\denote{\Xlo\exists y [\prd{aimer}(x,y) \wedge \prd{aimer}(y,x)]}^{\Modele}=1\).  Et cette sous-formule est vraie ssi on trouve une valeur pour \vrb y telle que 
\(\denote{\Xlo [\prd{aimer}(x,y) \wedge \prd{aimer}(y,x)]}^{\Modele}=1\).

Il faut donc trouver deux individus \Obj x et \Obj y dans \Unv A, tels que \tuple{\Obj x, \Obj y} \emph{et} \tuple{\Obj y, \Obj x} appartiennent à \(\FI(\prd{aimer})\).  Il en existe : par exemple \Obj{Figr} et \Obj{Suzn}.

Donc \(\denote{\Xlo\exists x\exists y [\prd{aimer}(x,y) \wedge \prd{aimer}(y,x)]}^{\Modele}=1\).

Le sens de la formule se retrouve dans le sens de \sicut{il y a des gens qui s'aiment mutuellement}.

\item \(\Xlo[\prd{aimer}(\cnsi a1, \cns s) \implq \exists x\, \neg\prd{aimer}(x,\cnsi a1)]\)

La formule est une implication, donc 
\(\denote{\Xlo[\prd{aimer}(\cnsi a1, \cns s) \implq \exists x\, \neg\prd{aimer}(x,\cnsi a1)]}^{\Modele}=1\) ssi 
l'antécédent \(\Xlo\prd{aimer}(\cnsi a1, \cns s)\) est faux, \emph{ou} (donc si l'antécédent est vrai) le conséquent \(\Xlo\exists x\, \neg\prd{aimer}(x,\cnsi a1)\) est vrai.

L'antécédent est vrai, car \Obj{Almv} aime \Obj{Suzn} dans {\Modele}.  Vérifions donc que le conséquent est vraie aussi.


\(\denote{\Xlo\exists x\, \neg\prd{aimer}(x,\cnsi a1)}^{\Modele}=1\) ssi on trouve au moins une valeur pour \vrb x telle que 
\(\denote{\Xlo\neg\prd{aimer}(x,\cnsi a1)}^{\Modele}=1\), c'est-à-dire telle que 
\(\denote{\Xlo\prd{aimer}(x,\cnsi a1)}^{\Modele}=0\), c'est-à-dire un individu qui n'aime pas \Obj{Almv}.  Cette valeur est facile à trouver : par exemple \Obj{Figr} (en fait tous les individus de \Unv A sauf \Obj{Rosn} feront l'affaire).


Donc \(\denote{\Xlo[\prd{aimer}(\cnsi a1, \cns s) \implq \exists x\, \neg\prd{aimer}(x,\cnsi a1)]}^{\Modele}=1\).

Cette formule correspond à la phrase \sicut{Si Almaviva aime Suzanne, alors quelqu'un n'aime pas Almaviva}.

\item \(\Xlo\neg\exists x [\prd{femme}(x) \wedge \exists y\,\prd{aimer}(x,y)]\)

\sloppy
\(\denote{\Xlo\neg\exists x [\prd{femme}(x) \wedge \exists y\,\prd{aimer}(x,y)]}^{\Modele}=1\) 
ssi on trouve que 
\(\dlb\Xlo\exists x [\prd{femme}(x) \wedge\allowbreak\linebreak[4] \exists y\,\prd{aimer}(x,y)]\color{black}\drb^{\Modele}=0\).

Or  
\(\denote{\Xlo\exists x [\prd{femme}(x) \wedge \exists y\,\prd{aimer}(x,y)]}^{\Modele}=1\) ssi
on trouve une valeur pour \vrb x telle que 
\(\denote{\Xlo[\prd{femme}(x) \wedge \exists y\,\prd{aimer}(x,y)]}^{\Modele}=1\) (et si on ne trouve pas une telle valeur dans \Unv A, alors on aura montré que 
\(\denote{\Xlo\exists x [\prd{femme}(x) \wedge \exists y\,\prd{aimer}(x,y)]}^{\Modele}=0\)).

Essayons avec la valeur \Obj{Rosn} (ou \cns r) pour \vrb x. Dans ce cas \(\Xlo\prd{femme}(x)\) est vrai. Reste à vérifier que \(\Xlo\exists y\,\prd{aimer}(x,y)\) l'est aussi. Or cette sous-formule est vraie également car on trouve bien une valeur «qui marche» pour \vrb y, par exemple \Obj{Almv} car dans \Modele, \Obj{Rosn} aime \Obj{Almv}.  Cela montre donc  
\(\denote{\Xlo\exists x [\prd{femme}(x) \wedge \exists y\,\prd{aimer}(x,y)]}^{\Modele}=1\).  Et par conséquent, 
\(\denote{\Xlo\neg\exists x [\prd{femme}(x) \wedge \exists y\,\prd{aimer}(x,y)]}^{\Modele}=0\).

La formule correspond à quelque chose comme \sicut{aucune femme n'aime quelqu'un} (car elle est la négation de \sicut{il y a au moins une femme qui aime quelqu'un}). 

\item  \(\Xlo\forall x [[\prd{homme}(x) \wedge \exists y
  [\prd{épouse-de}(y,x) \wedge \neg\prd{aimer}(x,y)]] \implq \prd{volage}(x)]\)

Pour montrer que \(\Xlo\forall x [[\prd{homme}(x) \wedge \exists y
  [\prd{épouse-de}(y,x) \wedge \neg\prd{aimer}(x,y)]] \implq \prd{volage}(x)]\)
est vraie dans \Modele, il faut (en théorie) répéter le calcul que 
\(\Xlo [[\prd{homme}(x) \wedge \exists y
  [\prd{épouse-de}(y,x) \wedge \neg\prd{aimer}(x,y)]] \implq \prd{volage}(x)]\)
pour toute les valeurs de \vrb x (c'est-à-dire tous les individus de \Unv A) et trouver $1$ à chaque fois.

Pour les individus qui ne sont pas des hommes, on sait tout de suite que cette implication est vraie car \(\Xlo[\prd{homme}(x) \wedge \exists y
  [\prd{épouse-de}(y,x) \wedge \neg\prd{aimer}(x,y)]]\) est faux (vu que \(\Xlo\prd{homme}(x)\) est faux).

Pour les autres (les hommes, \Obj{Almv}, \Obj{Figr}, \Obj{Chrb}, \Obj{Antn}, \Obj{Bart}), il y a deux cas de figure : soit \(\Xlo \exists y
  [\prd{épouse-de}(y,x) \wedge \neg\prd{aimer}(x,y)]\) est faux, et dans ce cas l'implication est vraie ; soit \(\Xlo \exists y
  [\prd{épouse-de}(y,x) \wedge \neg\prd{aimer}(x,y)]\) et dans ce cas il faut que \(\Xlo\prd{volage}(x)\) soit vrai aussi pour que l'implication soit vraie.

Dans le cas de \Obj{Chrb}, \Obj{Antn} et \Obj{Bart},  \(\Xlo \exists y
  [\prd{épouse-de}(y,x) \wedge \neg\prd{aimer}(x,y)]\) car il n'existe pas de valeur pour \vrb y qui marche (aucun des trois n'apparaît dans \(\FI(\prd{épouse-de})\), i.e aucun n'est marié).  L'implication est donc vraie pour ces trois valeurs de \vrb x.

Dans le cas \Obj{Figr} \(\Xlo \exists y
  [\prd{épouse-de}(y,x) \wedge \neg\prd{aimer}(x,y)]\) est faux aussi, car bien que \Obj{Figr} ait une épouse (\Obj{Suzn}) il est faux qu'il ne l'aime pas.  Donc quand \vrb x est \Obj{Figr}, il n'y a pas de valeur de \vrb y qui marche. Et pour \Obj{Figr}, l'implication est encore vraie.

Enfin pour la valeur \Obj{Almv}, alors \(\Xlo \exists y
  [\prd{épouse-de}(y,x) \wedge \neg\prd{aimer}(x,y)]\)  est vrai, car \Obj{Rosn} est une valeur de \vrb y qui marche. Et il se trouve qui \(\Xlo\prd{volage}(x)\) est vrai aussi car \Modele\ nous dit que \Obj{Almv} est volage.  L'implication est donc vraie pour la valeur \Obj{Almv} de \vrb x.


%Conclusion : 
Donc l'implication \(\Xlo [[\prd{homme}(x) \wedge \exists y
  [\prd{épouse-de}(y,x) \wedge \neg\prd{aimer}(x,y)]] \implq \prd{volage}(x)]\)
est vraie pour toutes les valeurs de \vrb x, ce qui prouve que
la formule f dénote $1$ dans~\Modele. 
%\(\denote{\Xlo\forall x [[\prd{homme}(x) \wedge \exists y [\prd{épouse-de}(y,x) \wedge \neg\prd{aimer}(x,y)]] \implq \prd{volage}(x)]}^{\Modele}=1\).


La formule correspond à \sicut{tout homme qui n'aime pas son épouse est volage}.

\end{enumerate}

\bigskip

%\item

2. 
Voici la nouvelle version du modèle 
 \(\Modele=\tuple{\Unv A,\FI}\) (les modifications sont soulignées) : %marquées en rouge) :

\begin{itemize}\raggedright
\item \(\Unv A=\set{\Obj{Almv}; \Obj{Rosn}; \Obj{Figr}; \Obj{Suzn}; \Obj{Mrcl};
  \Obj{Chrb}; \Obj{Fnch}; \Obj{Antn}; \Obj{Bart}}\) ;

\item \(\FI(\cnsi a1) = \Obj{Almv} ; 
\FI(\cns r) = \Obj{Rosn} ; 
\FI(\cnsi f1) = \Obj{Figr} ; 
\FI(\cns s) = \Obj{Suzn} ; 
\FI(\cns m) = \Obj{Mrcl} ;
\FI(\cns c) = \Obj{Chrb} ; 
\FI(\cnsi f2) = \Obj{Fnch} ; 
\FI(\cnsi a2) = \Obj{Antn} ; 
\FI(\cns b) = \Obj{Bart}\) ;

\item 
\(\FI(\prd{homme})=\set{\Obj{Almv}; \Obj{Figr}; \Obj{Chrb}; \Obj{Antn}; \Obj{Bart}}\) ;
\item 
\(\FI(\prd{femme})=\set{\Obj{Rosn};\Obj{Suzn}; \Obj{Mrcl}; \Obj{Fnch}}\) ;

\item 
\(\FI(\prd{domestique})=\set{\Obj{Figr}; \Obj{Suzn}; \Obj{Fnch}; \Obj{Antn}}\) ;
\item \(\FI(\prd{noble}) = \set{\Obj{Almv}; \Obj{Rosn};  \Obj{Chrb} }\) ;

\item 
\(\FI(\prd{roturier}) = \set{\Obj{Figr}; \Obj{Suzn}; \Obj{Mrcl};
 \Obj{Fnch}; \Obj{Antn}; \Obj{Bart}}\) ;

\item 
\(\FI(\prd{comte}) = \set{\Obj{Almv}}\) ;
\qquad \quad 
\(\FI(\prd{comtesse}) = \set{\Obj{Rosn}}\) ;

\item 
\(\FI(\prd{infidèle}) = \set{\Obj{Almv}}\) ;
\qquad 
%\item 
\(\FI(\prd{volage}) = \set{\Obj{Almv}; \Obj{Chrb}}\) ;

\item 
\(\uline{\FI(\prd{triste}) = \varnothing}\) ;

\item 
\(\FI(\prd{époux-de}) = \set{\tuple{\Obj{Almv}, \Obj{Rosn}};\tuple{\Obj{Figr}, \Obj{Suzn}} ; \uline{\tuple{\Obj{Chrb},\Obj{Fnch}}}}\) ;
\item 
\(\FI(\prd{épouse-de}) = \set{\tuple{\Obj{Rosn},\Obj{Almv}};\tuple{\Obj{Suzn},\Obj{Figr}} ; \uline{\tuple{\Obj{Fnch},\Obj{Chrb}}}}\) ;

\item 
\(\FI(\prd{père-de}) = \set{\tuple{\Obj{Antn},\Obj{Fnch}}}\) ;

\item 
\(\FI(\prd{aimer}) = \set{\tuple{\Obj{Figr},
    \Obj{Suzn}};\tuple{\Obj{Suzn},\Obj{Figr}}; 
\tuple{\Obj{Almv},\uline{\Obj{Rosn}}}; 
\tuple{\Obj{Rosn},\Obj{Almv}}; \tuple{\Obj{Chrb},
    \Obj{Fnch}}; \tuple{\Obj{Fnch},\Obj{Chrb}}; \tuple{\Obj{Mrcl},\Obj{Figr}}}\)
\\
  \uline{(et on a enlevé \tuple{\Obj{Chrb},
    \Obj{Rosn}} et \tuple{\Obj{Chrb}, \Obj{Suzn}})}
\end{itemize}
%\end{enumerate}

\fussy

\end{solu}

\end{exo}
