% -*- coding: utf-8 -*-
\begin{exo}[Variante de notation]
\label{exo:types:varcrochets}
La convention de notation présentée ici pour l'application
\pagesolution{soluvarlambdacrochet}%
fonctionnelle, \(\Xlo[\alpha(\beta)]\), n'est pas universelle, mais on la
retrouve chez certains auteurs, par exemple \citet{Gamut:2}%
\footnote{Sauf que \citet{Gamut:2} utilise des parenthèses à la place
de nos crochets : $\Xlo(\alpha(\beta))$.}.\Index{Gamut, L. T. F.}  
Je choisis de l'utiliser dans   cet ouvrage car elle permet de repérer
assez facilement quelle expression est l'argument de quelle fonction,
et en particulier quel argument vient saturer quelle variable
$\lambda$-abstraite.  Cependant, on trouve très souvent dans la
littérature la règle d'application fonctionnelle directement formulée
en $\Xlo\alpha(\beta)$.  Dans ce cas, les auteurs les plus prudents, comme
par exemple \citet{ChierchiaMcCG:90} et \citet{PtMW:90}, 
\Index{Chierchia, G.}\Index{McConnell-Ginet, S.}\Index{Partee, B.}
prennent alors
soin de définir la règle de $\lambda$-abstraction sous la forme de
\(\Xlo\lambda v [\alpha]\)%
\footnote{\citet{PtMW:90} utilisent des parenthèses à la place des
  crochets : \(\Xlo\lambda v (\alpha)\).}.
\begin{enumerate}
\item Réécrivez les expressions de l'exercice~\ref{exo:types:crochets} en
utilisant cette variante de notation.
\item Quel problème cela pose-t-il ? Comment devrait-on encore
  modifier la syntaxe de {\LO} pour éviter ce genre de problème ?
\end{enumerate}
\begin{solu}(p.~\pageref{exo:types:varcrochets})\label{soluvarlambdacrochet}
\begin{enumerate}
\item Variante de notation :
\addtolength{\multicolsep}{-10pt}
\begin{multicols}{2}
\begin{enumerate}[label=\arabic*] 
\item \(\Xlo\prd{aimer}(x)(y)\)
\item \(\Xlo\lambda x [\prd{aimer}(x)(y)]\)
\item \(\Xlo\lambda x [\prd{aimer}(x)](y)\)
\item \(\Xlo\peut\prd{aimer}(x)(y)\)
\item \(\Xlo\Intn\prd{aimer}(x)(y)\)\label{exo:var:[af]:i5}
\item \(\Xlo\Intn\prd{aimer}(x)(y)\)\label{exo:var:[af]:i6}
\item \(\Xlo\lambda x [\prd{aimer}(y)(x)](z)\)
\item \(\Xlo\lambda x [\lambda y [\prd{aimer}](y)(x)]\)
\end{enumerate}
\end{multicols}
\item Le problème qui se pose est que les expressions
  {5} et {6} deviennent identiques
  alors que dans l'exercice~\ref{exo:types:crochets} nous avons vu
  qu'elles ne se simplifiaient pas de la même manière. Et en fait
  elles n'ont pas le même sens.  Pour éviter que se pose ce problème,
  il faut donc que dans cette variante de notation, la règle
  d'introduction de l'opérateur $\Xlo\Intn$  (p.~\pageref{p:^}) soit
  révisée en: si $\Xlo\alpha$ est une expression de {\LO}, alors
  $\Xlo\Intn[\alpha]$ aussi.
\end{enumerate}
\end{solu}
\end{exo}
