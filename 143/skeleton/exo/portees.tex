\begin{exo}\label{exo:3portee}
Déterminez, en les glosant, 
\pagesolution{crg:3portee}
toutes les lectures possibles de chacune des phrases suivantes -- en prenant bien soin de repérer notamment la lecture avec portée intermédiaire de l'indéfini singulier.
\begin{enumerate}
\item
Chaque convive a raconté plusieurs histoires qui impliquaient un
membre de la famille royale.
\item
Chaque sénateur a raconté à plusieurs journalistes qu'un membre du
cabinet était corrompu.
\item 
Un professeur croit que chaque étudiant a lu un roman de Flaubert.
\end{enumerate}
\begin{solu}(p.~\pageref{exo:3portee})\label{crg:3portee}
\begin{enumerate}
\item
Chaque convive a raconté plusieurs histoires qui impliquaient un
membre de la famille royale.

\begin{enumerate}
\item Chaque convive $>$ plusieurs histoires $>$ un membre : 
pour chaque convive \Obj x, \Obj x a raconté plusieurs histoires, et chacune de ces histoires implique un membre (possiblement différent) de la famille royale.

\item Chaque convive $>$ un membre $>$ plusieurs histoires : 
pour chaque convive \Obj x, il y a un membre (particulier) de la famille royale au sujet duquel \Obj x a raconté plusieurs histoires.

\item Un membre $>$ chaque convive $>$ plusieurs histoires : 
il y a un membre (particulier) de la famille royale au sujet duquel tous les convives ont raconté plusieurs histoires (possiblement différentes). 

\item Un membre $>$ plusieurs histoires $>$ chaque convive : 
il y a un membre (particulier) de la famille royale au sujet duquel plusieurs histoires ont été racontées par tous les convives (et chacun racontait les mêmes histoires que les autres).

\item Plusieurs histoires $>$ un membre $>$ chaque convive : 
il y a plusieurs histoires impliquant divers membres de la famille royale qui ont été racontées par tous les convives (et chacun racontait les mêmes histoires que les autres).

\sloppy
\item Il y a une dernière lecture théoriquement possible mais pragmatiquement étrange et qui correspond à : plusieurs histoires $>$ chaque convive $>$ un membre.  Cette lecture apparaît spécifiquement dans un scénario comme le suivant : il y a une certaine série d'histoires et tous les convives racontent cette même série, mais chacun change le personnage sur lequel portent les histoires. 
  À la rigueur cette lecture peut passer si on comprend \sicut{histoire} comme dénotant des types ou modèles d'histoires mais pas d'anecdotes précises.

\fussy
\end{enumerate}

\item
Chaque sénateur a raconté à plusieurs journalistes qu'un membre du
cabinet était corrompu.

\begin{enumerate}
\item Chaque sénateur $>$ plusieurs journalistes $>$ un membre :
chaque sénateur \Obj x s'adresse à plusieurs journalistes et à chaque journaliste, \Obj x parle d'un membre corrompu (possiblement différent pour chaque journaliste).

\item Chaque sénateur $>$ un membre $>$ plusieurs journalistes :
pour chaque sénateur \Obj x il y a un membre \Obj y du cabinet dont \Obj x parle à plusieurs journalistes.

\item  Un membre $>$ chaque sénateur $>$ plusieurs journalistes : 
il y a un membre corrompu \Obj y et chaque sénateur parle de \Obj y à plusieurs journalistes (les journalistes peuvent être différents selon les sénateurs).

\item  Un membre $>$ plusieurs journalistes $>$ chaque sénateur :
il y a un membre corrompu \Obj y et un groupe particulier de journaliste et tous les sénateurs parlent de \Obj y à ce groupe de journalistes.

\item  Plusieurs journalistes $>$ un membre $>$ chaque sénateur :
il y a un groupe de journalistes, pour chacun de ces journalistes il y a un membre corrompu (possiblement différent d'un journaliste à l'autre) et chaque sénateur parle de ce membre au journaliste qui lui est «associé».

\item  Plusieurs journalistes $>$ chaque sénateur $>$ un membre :
il y a un groupe donné de journalistes, tous les sénateurs s'adressent à ce même groupe et chacun parle d'un membre corrompu différent.

\end{enumerate}

NB : il existe encore d'autres lectures possibles (et subtiles) de cette phrase qui mettent en jeu un phénomène que nous aborderons plus précisément en \S\ref{ss:re/dicto}, mais la troisième phrase ci-dessous nous en donne un petit aperçu (avec notamment la lecture~3c).

\item 
Un professeur croit que chaque étudiant a lu un roman de Flaubert.

\begin{enumerate}
\item Un professeur $>$ (que) chaque étudiant $>$ un roman :
il y a un professeur qui pense que chaque étudiant a choisi librement un roman de Flaubert et l'a lu.

\item Un roman $>$ un professeur $>$ (que) chaque étudiant :
il y a un roman écrit par Flaubert et un professeur qui pense que tous les étudiants ont lu ce roman.

\item Un professeur $>$ (que) un roman $>$ chaque étudiant :
il y a un professeur qui pense à un roman particulier, par exemple \emph{Le rouge et le noir}, il pense que ce roman est de Flaubert (mais il se trompe, en l'occurrence) et pense que tous les étudiants l'ont lu.
Notons cependant que cette lecture n'implique nécessairement que le professeur se trompe sur l'auteur du roman, mais elle est compatible avec un tel cas de figure.

\end{enumerate}

\end{enumerate}
\end{solu}
\end{exo}
