% -*- coding: utf-8 -*-
\begin{exo}\label{exo:1Ambig2}
La phrase \sicut{ce bijou n'est pas une bague en or} peut servir à communiquer que le bijou n'est pas une bague ou %à communiquer 
qu'il n'est pas en or.  Est-ce à dire que la phrase est ambiguë ? 
\pagesolution{crg:1Ambig2}
\begin{solu}(p.~\pageref{exo:1Ambig2})\label{crg:1Ambig2}

La phrase \sicut{ce bijou n'est pas une bague en or} n'est pas ambiguë, elle a juste une signification assez large pour couvrir les cas où le bijou n'est pas une bague et ceux où il n'est pas en or.  Pour nous en assurer, nous pouvons ici appliquer le test des ellipses vu dans le chapitre p.~\pageref{test:ellipse}.  Supposons qu'un premier bijou $B_1$ est une bague en argent et qu'un second bijou $B_2$ est un bracelet en or ; nous pouvons alors tout à fait énoncer \sicut{ce bijou $B_1$ n'est pas une bague en or, et ce bijou $B_2$ non plus}. 
\end{solu}
\end{exo}
