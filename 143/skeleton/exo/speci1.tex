% -*- coding: utf-8 -*-
\begin{exo}\label{exo:speci}
%
\begin{solu}(p.~\pageref{exo:speci})\label{crg:speci}

Si nous voulons montrer que \ref{x:vampire} est ambiguë selon la définition \ref{d:ambig}, nous devons construire un modèle par rapport auquel la phrase sera jugée vraie et fausse selon que le \GN\ a une lecture spécifique ou non.
Un tel modèle doit comporter minimalement l'individu \Obj{Alice}, une jeune fille,  un autre individu, appelons-le \Obj{Lestat}, qui est un vampire et \Obj{Lestat} a mordu \Obj{Alice} pendant la nuit.  Mais dans ce modèle \ref{x:vampire} sera toujours vraie, que \sicut{un vampire} soit spécifique ou non.  La lecture (ou l'usage) non spécifique émerge quand le locuteur ne connaît pas l'identité de \Obj{Lestat}, mais cette information n'est pas codée dans le modèle (tels que les modèles sont définis dans notre système).  Nous devons donc conclure que \ref{x:vampire} n'est pas sémantiquement ambiguë (si elle l'est, il s'agira plutôt d'une ambiguïté pragmatique).
\end{solu}
En reprenant la définition \ref{d:ambig} 
\pagesolution{crg:speci}%
donnée en \S\ref{s:Ambiguïté}
(p.~\pageref{d:ambig}), essayez de montrer si \ref{x:vampire} est, ou
non, un cas de véritable ambiguïté sémantique.

\ex.[\ref{x:sonnvamp}]
\a.[b.] Un vampire a mordu Alice.

\end{exo}
