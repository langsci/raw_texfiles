% -*- coding: utf-8 -*-
\begin{exo}
\label{exo:types:ebf}
%% Les constantes:
%% $\prd{manger} \in \CON_{\type{e,\type{e,t}}}$, $\prd{croire} \in \ME_{\type{\type{s,t},\type{e,t}}}$

On se donne les variables suivantes : 
$\vrb x, \vrb y \in \VAR_{\mathrm{e}}$, $\vrb \phi \in \VAR_{\mathrm{t}}$, $\vrb P \in \VAR_{\type{e,t}}$, 
\newline
$\vrb Q \in \VAR_{\type{e,\type{e,t}}}$, $\vrb K \in
\VAR_{\type{\type{s,t},\type{e,t}}}$, $\vrb S \in \VAR_{\type{s,\type{e,t}}}$.
 
Dites si les expressions suivantes sont bien formées, et si oui quel
est leur type.  On considère que la règle de suppression des crochets
s'est appliquée quand c'était utile.

\begin{multicols}{2}
\begin{enumerate}
\item \(\Xlo\lambda x\, Q(x,y)\)
\item \(\Xlo\lambda Q \exists x\, Q(x,y)\)
\item \(\Xlo[\lambda x\, Q(x,y)(y)]\)
\item \(\Xlo\peut\lambda x\, P(x)\)
\item \(\Xlo\lambda P\, P\)
\item \(\Xlo[\lambda P\, P(x)(y)]\)
\item \(\Xlo\lambda x [\Extn S(x)]\)
\item \(\Xlo\lambda x\, \Extn [S(x)]\)
\item \(\Xlo[\lambda x\, \Extn S(x)]\)
\item \(\Xlo K(\Intn [\lambda x\, P(x)(y)])\)
\item \(\Xlo\lambda \phi\, K(x,\Intn \phi)\)
\item \(\Xlo\lambda \phi \lambda x \, K(x,\Intn \phi)\)
\item \(\Xlo\lambda x \lambda \phi \, K(x,\Intn \phi)\)
\item \(\Xlo\lambda K \lambda \phi\, K(x,\Intn \phi)\)
\item \(\Xlo\lambda K \lambda \phi\, \Intn [[K(\Intn \phi)](x)]\)
\item \(\Xlo\lambda K \lambda \phi\, [\Intn [K(\Intn \phi)](x)]\)
\item \(\Xlo\lambda x [P(x) \wedge Q(x)]\)
\item \(\Xlo\lambda y\lambda x  [P(x) \wedge Q(x,y)](x)\)
\item \(\Xlo[\lambda Q\, Q(x,y)(\lambda x\, P)]\)
\item \(\Xlo\lambda y\, Q(\atoi x\, P(x),y)\)
\end{enumerate}
\end{multicols}

\begin{solu} (p. \pageref{exo:types:ebf})

Rappel des types :\\
$\vrb x, \vrb y \in \VAR_{\mathrm{e}}$, $\vrb \phi \in \VAR_{\mathrm{t}}$, $\vrb P \in
\VAR_{\type{e,t}}$, $\vrb Q \in \VAR_{\type{e,\type{e,t}}}$, $\vrb K \in
\VAR_{\type{\type{s,t},\type{e,t}}}$, $\vrb S \in \VAR_{\type{s,\type{e,t}}}$.\\
Les règles syntaxiques utilisées ci-dessous sont celles de la définition \ref{RSyn:ltype}, p.~\pageref{RSyn:ltype}.

\sloppy
\begin{enumerate}
\item  \(\Xlo\lambda x\, Q(x,y)\)\\
$\Xlo Q(x,y)$ est de type \typ{t} car $\vrb Q \in \VAR_{\type{e,\type{e,t}}}$ et
  a ses deux arguments de type \typ{e} ; donc \(\Xlo\lambda x\, Q(x,y)\)
  est de type \type{e,t} (\RSyn\ref{SynTlamb}).

\item \(\Xlo\lambda Q \exists x\, Q(x,y)\)\\
$\Xlo Q(x,y)$ est de type \typ{t}, donc $\Xlo\exists x\, Q(x,y)$ aussi
  (\RSyn\ref{SynTQ}) et   \(\Xlo\lambda Q \exists x\, Q(x,y)\) est de type
  \type{\type{e,\type{e,t}},t}  (\RSyn\ref{SynTlamb}).

\item \(\Xlo[\lambda x\, Q(x,y)(y)]\)\\
$\Xlo Q(x,y)$ est de type \typ{t}, donc $\Xlo\lambda x\, Q(x,y)$ est de type
  \type{e,t} (\RSyn\ref{SynTlamb}) et donc \(\Xlo[\lambda x\, Q(x,y)(y)]\)
  est de type \typ{t} (\RSyn\ref{SynTApp}).\\  {NB : on voit bien que
  les crochets sont importants pour retrouver le bon ordre de
  composition de l'expression. }

\item \(\Xlo\peut\lambda x\, P(x)\)\\
$\Xlo P(x)$ est de type \typ{t} car $\vrb P  \in \VAR_{\type{e,t}}$
  (\RSyn\ref{SynTApp}), donc $\Xlo\lambda x\, P(x)$ est de type
  \type{e,t} ; or d'après (\RSyn\ref{SynTMod}) {\Xlo\peut} ne peut
  s'accoler qu'à une expression de type \typ{t} ; donc l'expression
  n'est pas bien formée.


\item \(\Xlo\lambda P\, P\)\\
$\vrb P$ est de type \type{e,t}, donc $\Xlo\lambda P\, P$ est de type
  \type{\type{e,t},\type{e,t}}, en vertu de (\RSyn\ref{SynTlamb}).


\item \(\Xlo[\lambda P\, P(x)(y)]\)\\
Ici l'expression est ambiguë quant à la suppression d'une paire de
crochets\footnote{Ce qui devrait laisser deviner d'emblée qu'elle n'est pas
bien formée, puisqu'on n'a pas le droit de supprimer des crochets
en rendant une expression ambiguë.} : on peut supposer qu'elle est la
simplification soit de \(\Xlo[\lambda P [P(x)](y)]\), soit de \(\Xlo[[\lambda
  P\, P(x)](y)]\) ; dans les deux cas elle est mal formée.  Dans le
premier cas, $\Xlo P(x)$ est de type \typ{t}, $\Xlo\lambda P [P(x)]$ est de type
\type{\type{e,t},t} et ne peut donc pas se combiner avec $\vrb y$ de type
\typ{e}.  Dans le second cas, $\Xlo\lambda P\, P$ est de type
\type{\type{e,t},\type{e,t}} et ne peut donc pas se combiner avec $\vrb x$.

\item \(\Xlo\lambda x [\Extn S(x)]\)\\
$\Xlo\Extn S$ est de type \type{e,t} (\RSyn\ref{SynTExt}) car $\vrb S$ est de
  type \type{s,\type{e,t}} ; donc $\Xlo[\Extn S(x)]$ est de type \typ{t} et
  \(\Xlo\lambda x [\Extn S(x)]\) de type \type{e,t}.

\item \(\Xlo\lambda x\, \Extn [S(x)]\)\\
Cette expression est mal formée, car $\vrb S$ ne peut pas se combiner
directement avec $\vrb x$.

\item \(\Xlo[\lambda x\, \Extn S(x)]\)\\
$\Xlo\Extn S$ est de type \type{e,t}, donc $\Xlo\lambda x\, \Extn S$ est de
  type \type{e,\type{e,t}} et \(\Xlo[\lambda x\, \Extn S(x)]\) est de type
  \type{e,t}. 

\item \(\Xlo K(\Intn [\lambda x\, P(x)(y)])\)\\
Quelle que soit la manière d'analyser $\Xlo[\lambda x\, P(x)(y)]$ on
trouvera le type \typ{t} pour cette sous-expression.  En effet $\Xlo P(x)$
est de type \typ t, $\Xlo\lambda x\, P(x)$  de type \type{e,t} et donc
$\Xlo[\lambda x\, P(x)(y)]$ de type \typ t ; ou bien $\Xlo\lambda x\, P$ est de
type \type{e,\type{e,t}} et $\Xlo\lambda x\, P(x)$ est de type
\type{e,t}.  Ensuite $\Xlo\Intn[\lambda x\,P(x)(y)]$ est de type
\type{s,t} (\RSyn\ref{SynTInt}), et donc \(\Xlo K(\Intn [\lambda x\,
  P(x)(y)])\) est de type \type{e,t}, car $\vrb K$ est de type
  \type{\type{s,t},\type{e,t}}. 

\item \(\Xlo\lambda \phi\, K(x,\Intn \phi)\)\\
$\vrb \phi$ est de type \typ t, donc $\Xlo\Intn \phi$ est de type \type{s,t}, et
  $\Xlo K(x,\Intn \phi)$ est de type \typ t, car c'est la simplification de
  $\Xlo[[K(\Intn \phi)](x)]$ ; donc \(\Xlo\lambda \phi\, K(x,\Intn \phi)\) est de type
  \type{t,t}. 


\item \(\Xlo\lambda \phi \lambda x \, K(x,\Intn \phi)\)\\
$\Xlo\lambda x\, K(x,\Intn \phi)$ est de type \type{e,t}, donc \(\Xlo\lambda \phi \lambda x \, K(x,\Intn \phi)\) est de type \type{t,\type{e,t}}.

\item \(\Xlo\lambda x \lambda \phi \, K(x,\Intn \phi)\)\\
$\Xlo\lambda \phi \, K(x,\Intn \phi)$ est de type \type{t,t}, donc \(\Xlo\lambda x   \lambda \phi \, K(x,\Intn \phi)\) est de type \type{e,\type{t,t}}.

\item \(\Xlo\lambda K \lambda \phi\, K(x,\Intn \phi)\)\\
$\Xlo\lambda \phi \, K(x,\Intn \phi)$ est de type \type{t,t}, donc \(\Xlo\lambda K
  \lambda \phi\, K(x,\Intn \phi)\) est de type
  \type{\type{\type{s,t},\type{e,t}},\type{t,t}}. 

\item \(\Xlo\lambda K \lambda \phi\, \Intn [[K(\Intn \phi)](x)]\)\\
$\Xlo\Intn \phi$ est de type \type{s,t},\\ donc $\Xlo[K(\Intn \phi)]$ est de type   \type{e,t},\\ donc $\Xlo[[K(\Intn \phi)](x)]$ est de type \typ t,\\ donc $\Xlo\Intn[[K(\Intn \phi)](x)]$ est de type \type{s,t},\\ donc $\Xlo\lambda \phi\,\Intn [[K(\Intn \phi)](x)]$ est de type \type{t,\type{s,t}}, \\et \(\Xlo\lambda K \lambda \phi\, \Intn [[K(\Intn \phi)](x)]\) de type   \type{\type{\type{s,t},\type{e,t}},\type{t,\type{s,t}}}.

\largerpage[-1]

\item \(\Xlo\lambda K \lambda \phi\, [\Intn [K(\Intn \phi)](x)]\)\\
$\Xlo[K(\Intn \phi)]$ est de type \type{e,t}, donc $\Xlo\Intn [K(\Intn \phi)]$ est   de type \type{s,\type{e,t}} et cela ne peut pas se combiner avec
  $\vrb x$.  L'expression est donc mal formée.

\item \(\Xlo\lambda x [P(x) \wedge Q(x)]\)\\
$\Xlo P(x)$ est de type \typ t, $\Xlo Q(x)$ est de type \type{e,t} (car $\vrb Q$ est   de type \type{e,\type{e,t}}) ; $\Xlo\wedge$ ne peut donc pas les combiner
  car il ne coordonne que des expressions de type \typ t   (\RSyn\ref{SynTConn}). L'expression est mal formée.

\item \(\Xlo\lambda y\lambda x  [P(x) \wedge Q(x,y)](x)\)\\
$\Xlo Q(x,y)$ est de type \typ t, donc $\Xlo[P(x) \wedge Q(x,y)]$ est de type
  \typ t, $\Xlo\lambda x  [P(x) \wedge Q(x,y)]$ est de type \type{e,t},
  $\Xlo\lambda y\lambda x  [P(x) \wedge Q(x,y)]$ de type \type{e,\type{e,t}}
  et \(\Xlo\lambda y\lambda x  [P(x) \wedge Q(x,y)](x)\) de type \type{e,t}.


\item \(\Xlo[\lambda Q\, Q(x,y)(\lambda x\, P)]\)\\
$\Xlo Q(x,y)$ est de type \typ t, donc $\Xlo\lambda Q\, Q(x,y)$ est de type
  \type{\type{e,\type{e,t}},t} ; et $\Xlo\lambda x\, P$ est de type
  \type{e,\type{e,t}}, donc \(\Xlo[\lambda Q\, Q(x,y)(\lambda x\, P)]\)
  est de type \typ t.

\item \(\Xlo\lambda y\, Q(\atoi x\, P(x),y)\)\\
$\Xlo P(x)$ est de type \typ t, donc $\Xlo\atoi x\, P(x)$ est de type \typ e
  (\RSyn\ref{SynTQ}), donc $\Xlo Q(\atoi x\, P(x),y)$ est de type \typ t,
  et \(\Xlo\lambda y\, Q(\atoi x\, P(x),y)\) est de type \type{e,t}.

\end{enumerate}
\fussy
\end{solu}

\end{exo}
