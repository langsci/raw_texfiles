% -*- coding: utf-8 -*-
% add all extra packages you need to load to this file  
\usepackage{tabularx} 

%%%%%%%%%%%%%%%%%%%%%%%%%%%%%%%%%%%%%%%%%%%%%%%%%%%%
%%%                                              %%%
%%%           Examples                           %%%
%%%                                              %%%
%%%%%%%%%%%%%%%%%%%%%%%%%%%%%%%%%%%%%%%%%%%%%%%%%%%% 
%% to add additional information to the right of examples, uncomment the following line
% \usepackage{jambox}
%% if you want the source line of examples to be in italics, uncomment the following line
% \renewcommand{\exfont}{\itshape}
%\usepackage{./LSP/lsp-styles/lsp-gb4e}
\usepackage{listings}

\lstset{ %
  backgroundcolor=\color{white},   % choose the background color; you must add \usepackage{color} or \usepackage{xcolor}
  basicstyle=\footnotesize\ttfamily,        % the size of the fonts that are used for the code 
  keywordstyle=\color{blue!60!black},       % keyword style
  language=XML,                 % the language of the code 
  stringstyle=\color{green!60!black},     % string literal style 
  morekeywords={token,xlink:href, Action, Value, Cursor,LogEvent}
} 
 
%\usepackage{polyglossia}		% TODO remove this?
%\setdefaultlanguage{french}

%\frenchspacing			


\usepackage{textcomp}  %% font math
\usepackage{amsmath}   %% extensions math

\usepackage{wasysym}  %% symboles math supplémentaires ou rectifiés (notamment \leadsto, \Diamond, \Box) 

\usepackage[mathscr]{euscript}
\let\euscr\mathscr \let\mathscr\relax% just so we can load this and rsfs
\usepackage[scr]{rsfso}
        
\usepackage[libertine]{newtxmath}  %% Must be loaded AFTER the other math-fonts packages. NB: many symbols (digits, punctuation, delimiters...) were typeset in CM font, not liberine; I had to add the following:

\usepackage{mathspec}
  \setmathsfont[Path=\fontpath,ItalicFont={LinLibertine_RI_B.otf}]{LinLibertine_R_B.otf}
  \setmathsfont(Digits)[Path=\fontpath,BoldFont=LinLibertine_RZ_B.otf]{LinLibertine_R_B.otf}
  \setmathrm[Path=\fontpath,ItalicFont={LinLibertine_RI_B.otf}]{LinLibertine_R_B.otf}
  \setboldmathrm[Path=\fontpath]{LinLibertine_RZ_B.otf}


\usepackage{bm}

\usepackage{answers}   %% exercices et corrigés
\usepackage[nocenter]{qtree}\usepackage{qtreefr} %% arbres
\renewcommand{\qroofpadding}{0.1em}

%\usepackage{framed}\usepackage[framed,hyperref]{ntheorem}
\usepackage{ntheorem}
%\usepackage{amsthm}
%\usepackage{LSP/lsp-styles/langsci-tbls}
\usepackage[framemethod=tikz]{mdframed}


%\usepackage%[dvipsnames,svgnames,x11names,table]
%{xcolor}


%\usepackage[libertine]{newtxmath}\let\forall\forallAlt\let\exists\existsAlt\newcommand{\textsb}[1]{{\libertineSB#1}}

%\newcommand{\textbm}[1]{\textsb{#1}} %% avec cfr-lm

%\useosf


\usepackage{pstricks}
\usepackage{pst-node}
\usepackage{pst-grad}
\usepackage{pst-tree}
\usepackage{rotating}

\usepackage%[shortlabels]
{enumitem} % >= v3.0
%\usepackage{enumerate} %les 2 sont incompatibles
\usepackage{multicol}

\usepackage{colortbl}

\usepackage[normalem]{ulem}

%% \usepackage[mathscr]{euscript}
%% \let\euscr\mathscr \let\mathscr\relax% just so we can load this and rsfs
%% \usepackage[scr]{rsfso}


\usepackage{linguex}
        
\usepackage{csquotes}


%\usepackage{bigcenter}%% à changer

