\documentclass[output=paper,colorlinks,citecolor=brown]{langscibook}
\ChapterDOI{10.5281/zenodo.15775163}
\author{Judith Purkarthofer\orcid{0000-0002-2650-2274}\affiliation{University of Duisburg-Essen} and         Rosemarie Tracy\orcid{0000-0002-6683-3481}\affiliation{University of Mannheim} and         Sofia Grigoriadou\orcid{}\affiliation{University of Duisburg-Essen} and         Johanna Tausch\orcid{0009-0008-1945-3356}\affiliation{University of Mannheim; Leibniz Institute for the German Language}}
        
\title[Family language dynamics]
      {Family language dynamics: Strengthening heritage speakers’ linguistic resources}

\abstract{Research on heritage languages is not only empirically and theoretically relevant: It also touches a nerve with families, minority speech communities and educators, teachers and other agents within the majority language educational and health systems. This contribution presents the rationale of a specific transfer project based on the findings of the Research Unit \textit{Emerging Grammars in Language Contact Situations} (RUEG) as well as on multilingualism research in general. The need for differentiated outreach activities is highlighted; first steps of implementation in close cooperation with non-academic partners and their target groups are addressed. Target groups include parents with very rudimentary to excellent majority language skills, professionals providing advice to parents or involved in language fostering as well as instructors in qualifying programs for educators and teachers. Initiatives include workshops, video lectures, interviews and other material, eventually all available online via www.ruegram.de as well as suggestions for specific citizen science activities already conducted or under way.

\keywords{family languages, heritage languages, transfer and outreach, language repertoire, multilingual education}
}

\IfFileExists{../localcommands.tex}{
  \addbibresource{../localbibliography.bib}
  % add all extra packages you need to load to this file

\usepackage{tabularx,multicol}
\usepackage{url}
\urlstyle{same}

\usepackage{listings}
\lstset{basicstyle=\ttfamily,tabsize=2,breaklines=true}

\usepackage{langsci-basic}
\usepackage{langsci-optional}
\usepackage{langsci-lgr}
\usepackage{langsci-osl}
% \usepackage{./langsci/styles/langsci-lgr}
% \usepackage{./langsci/styles/langsci-osl}
% \usepackage{langsci-gb4e}

\usepackage{tikz}
\usetikzlibrary{patterns,calc}
\pgfdeclarepatternformonly{south east lines}{\pgfqpoint{-0pt}{-0pt}}{\pgfqpoint{3pt}{3pt}}{\pgfqpoint{3pt}{3pt}}{
    \pgfsetlinewidth{0.6pt}
    \pgfpathmoveto{\pgfqpoint{0pt}{3pt}}
    \pgfpathlineto{\pgfqpoint{3pt}{0pt}}
    \pgfpathmoveto{\pgfqpoint{.2pt}{-.2pt}}
    \pgfpathlineto{\pgfqpoint{-.2pt}{.2pt}}
    \pgfpathmoveto{\pgfqpoint{3.2pt}{2.8pt}}
    \pgfpathlineto{\pgfqpoint{2.8pt}{3.2pt}}
    \pgfusepath{stroke}}
    
\usepackage{stmaryrd}
\usepackage{wasysym}
\usepackage{multirow}
\usepackage{caption}
\usepackage{subcaption}
\usepackage{mathrsfs}
\usepackage{qtree}

\usepackage{linguex}


  %pminos do not split footnotes
% \interfootnotelinepenalty=10000 %Footnote in Laporte chapters has to be split SN


%\DeclareIndexNameFormat{default}{%
%\nameparts{#1}%
%\usebibmacro{index:name}%
%{\index[names]}%
%{\namepartfamily}%
%{\namepartgiveni}%
% {}% L1
% {}% L2
%{\namepartprefix}% generates spurious space L3
%{\namepartsuffix}% generates spurious space L4
%}

%  {\DeclareIndexNameFormat{default}{%
%     \usebibmacro{index:name}{\index[names]}{#1}{#3}{#5}{#7}}}

%\DeclareIndexNameFormat{default}{%
%  \usebibmacro{index:name}{\sindex[nom]}{#1}{#3}{#5}{#7}}

%\DeclareIndexNameFormat{default}{%
%  \usebibmacro{index:name}{\sindex[person]}{#1}{#3}{#5}{#7}}
%\DeclareIndexNameFormat{default}{%
%\nameparts{#1} \usebibmacro{index:name}{\sindex[person]]}{\namepartfamily}{‌​\namepartgiven}{\nam‌​epartprefix}{\namepa‌​rtsuffix}}

%\newcommand{\smiley}{:)}

%\renewbibmacro*{index:name}[5]{%
%\usebibmacro{index:entry}{#1}%
%{\iffieldundef{usera}{}{\thefield{usera}\actualoperator}\mkbibindexname{#2}{#3}{#4}{#5}}}

% \newcommand{\noop}[1]{}

%remove for final
%\overfullrule=1mm

\newcommand{\tobi}[2]}}
\renewcommand{\S}[1]{\tobi{#1}{\textsc{*}}}

% this volume references
% puts: [this volume]
% already defined: \citetv
%\newcommand{\citepv}[1]{(\citeauthor{#1} \citeyear*{#1} [this volume])}
\newcommand{\citealtv}[1]{\citeauthor{#1} \citeyear*{#1} [this volume]}

%parentheses around example number
\newcommand{\pref}[1]{(\ref{#1})}

% in-text examples

\newcommand{\lnex}[1]{\textit{#1}} %target lang word
\newcommand{\lnlit}[1]{(lit.: `#1')} %literal reading
\newcommand{\lnlat}[1]{(#1)} % latinization
\newcommand{\lntrans}[1]{`#1'} %translation
\newcommand{\lnexl}[2]%
{\lnex{#1}{} \lnlat{#2}} % ex with latinization
\newcommand{\lnexlat}[3]{\lnex{#1}{} \lnlat{#2}{} \lntrans{#3}} % ex with latinization and tranl.

%ch01
\newcommand{\co}[1]{\mbox{\textbf{#1}}}

%ch09

\newcommand{\cyrbulg}[1]{\begin{otherlanguage*}{bulgarian}#1\end{otherlanguage*}}


%ch10
\newcommand{\nlp}{{\small NLP}}
\newcommand{\mwe}{{\small MWE}}
\newcommand{\rae}{{\small RAE}}
\newcommand{\lvc}{{\small LVC}}
\newcommand{\pos}{{\small P}o{\small S}}
%\newcommand{\todo}[1]{ \textcolor{red}{#1} }

%\renewcommand{\labelenumi}{\theenumi}
%\ainamefmt{{vv}{ll}{, ff}{, jj}} % fullname

\newcommand{\biberror}[1]{{\color{red}#1}}

\newcommand{\osenovaitem}{--~}
  %% hyphenation points for line breaks
%% Normally, automatic hyphenation in LaTeX is very good
%% If a word is mis-hyphenated, add it to this file
%%
%% add information to TeX file before \begin{document} with:
%% %% hyphenation points for line breaks
%% Normally, automatic hyphenation in LaTeX is very good
%% If a word is mis-hyphenated, add it to this file
%%
%% add information to TeX file before \begin{document} with:
%% %% hyphenation points for line breaks
%% Normally, automatic hyphenation in LaTeX is very good
%% If a word is mis-hyphenated, add it to this file
%%
%% add information to TeX file before \begin{document} with:
%% \include{localhyphenation}
\hyphenation{
    Beck-man
    Ngu-yen
    back-chan-nel
    back-chan-nels
    mo-not-o-nous
    ste-reo-typ-i-cal
}

\hyphenation{
    Beck-man
    Ngu-yen
    back-chan-nel
    back-chan-nels
    mo-not-o-nous
    ste-reo-typ-i-cal
}

\hyphenation{
    Beck-man
    Ngu-yen
    back-chan-nel
    back-chan-nels
    mo-not-o-nous
    ste-reo-typ-i-cal
}

  \boolfalse{bookcompile}
  \togglepaper[4] %%chapternumber
}{}

\begin{document}
\maketitle

\section{Introduction}
Heritage languages play a crucial role in multilingual societies: they are important social facts for families, but also for educational institutions, workplaces and social interaction in general. At the same time, they contribute to a multilingual language ecology, thereby at times challenging assumptions about language use and development. Attitudes towards majority, minority and minoritised languages influence how parents and children valorise or disregard their own and their friends’ languages. To this day, and maybe even more so than 50 years ago~– a time when individual heritage speakers were typically immersed (or rather submersed) in majority language classrooms – immigrant families need to defend their wish to pass on their minority languages, even if researchers generally agree on the positive outcomes of bi- and trilingual family language policies \parencite{Schwartz&Verschick2013, Tracy2014, Tracy&al.2018, Arnaus&al.2019}. While early research on language transmission \parencite{Fishman1991} was particularly interested in generational transfer in order to evaluate the vitality of heritage languages, recent research has paid more attention to the parallel use of several languages as part of the linguistic repertoire of speakers (i.e., on multilingual speakers’ biographies, \citealt{Pavlenko&Blackledge2004, Busch2012, Purkarthofer&Flubacher2022}; on language, globalisation and superdiversity, \citealt{Arnaut&al.2016}).

Despite its ubiquity, multilingualism within the family and the wish to enable children to grow up with more than one language is still considered a private matter – albeit sometimes heavily policed by society. In Germany, as in many other countries, questions of multilingual upbringing often come with a strong focus on supposedly successful strategies for integrating minorities into the majority society \parencite{Tracy2014, Schroeder2017, Wiese&al.2020}. This is happening in parallel to and hardly disturbed by research findings indicating that bilingual schooling can be beneficial for children even with regard to test results in maths and sciences. In general, however, there is still little public awareness of what it takes for language acquisition, monolingual or bilingual, to get off the ground.

Hard-to-die ideologies and, above all, discriminatory discourses work against multilingualism in childhood, except for cases of elite bilingualism involving languages with undisputed status, for instance English or French in Germany. Already in her 1998 volume on bilingual education programs, \textcite[1]{Brisk1998} pointed out that ``[t]he paradox of bilingual education is that when it is employed in private schools for the children of elites throughout the world it is accepted as educationally valid […]. However, when public schools implemented education for language minority students over the past 50 years, bilingual education became highly controversial.''

This means that families, but also those they turn to for advice – early childhood educators, teachers, paediatricians – often are at a loss with respect to crucial language-related and educational issues, regardless of the languages involved (cf. \cite{Hopp&al.2010, Edwards2015, Purkarthofer2019}). From the perspective of some institutions and majority language speakers, the visibility and the maintenance of heritage languages are seen as threats to societal peace and solidarity (\cite{Tracy2014}, \cite{Wiese&al.2020}). Concerns about societal rift are even voiced by those who one might expect to know better, namely immigrants, in Germany and beyond. Linguist and US-senator Hayakawa, for example, was a stout supporter of the English-Only movement and opponent of bilingual education programs, which he considered “a costly and confusing bureaucratic nightmare”, likely to enhance political disloyalty and destabilization \parencite[44]{Hayakawa1992}. To this day, many agents in the educational system and the public in general lack information on what shapes languages, what is needed for acquisition and what brings about language change apart from the presence of ``other'' languages. In general, it is fair to say that beyond complaints about decreasing standards of politeness, formality, or writing skills, there is little awareness that languages are highly dynamic, hence – somewhat ironically formulated! – constantly ``on the move'', with or without language contact or migration.
Transmission of research-backed knowledge is the main objective of the transfer project entitled ``Family language dynamics: empowering speakers of majority and heritage languages'' of the Research Unit \textit{Emerging Grammars in Language Contact Situations} (RUEG; \url{https://hu.berlin/rueg}). Target groups are parents and educators of mono- and multilingual children as well as children and adolescents themselves. All activities focus on speakers’ multilingual repertoires and relate to RUEG’s main research question ``What are the linguistic dynamics in heritage speakers’ repertoires?''

The RUEG transfer activities are organised in five categories (see \figref{fig:purkarthoferetal:five-modules}). We will refer to this pentagon and its modules throughout the rest of this chapter. The categories cover:

\begin{enumerate}
\item Knowledge about languages and language use (green),
\item Specific research findings illustrating language ‘live’ in language contact situations (blue),
\item Experiences of multilingual speakers (purple),
\item Myths/Discourses about languages and multilingual language use (red) and finally
\item Citizen science activities, researching in one's own family and community (yellow).
\end{enumerate}


\begin{figure}
     \includegraphics[width=0.4\linewidth]{figures/Ch4_figure_1_five-modules.png}
     \caption{Five modules: (1) knowledge (green), specific research findings (blue), experiences (purple), myths/discourses (red) as well as citizen science (yellow).}
     \label{fig:purkarthoferetal:five-modules}
\end{figure}
 

Taken together, these components provide research-backed facts for interested parents and educators and invite speakers into meaningful dialogue about their experiences, attitudes and language practices. Our contribution starts in the red area (in \sectref{sec:purkathoferetal:2}) with myths and misconceptions about the coexistence of two or more languages in individuals and society. We then move to our own experiences (\sectref{sec:purkathoferetal:3}, purple) to describe the specific research process and methods of the transfer project, before we introduce family language dynamics in its relation to heritage languages more broadly (in blue and green Sections \ref{sec:purkathoferetal:4} and \ref{sec:purkathoferetal:5}, covering specific results and broader knowledge about family languages). In \sectref{sec:purkathoferetal:6} (yellow), we present a range of outreach and citizen science activities backed by linguistic research and developed to strengthen heritage language speakers’ resources in the context of the project. Finally, in \sectref{sec:purkathoferetal:7}, we draw conclusions with regard to audiences, investments, needs and goals.

\section{Myth busting: Tackling ghosts from the past?} \label{sec:purkathoferetal:2}

In 1988, parents speaking another language at home besides German were given the following advice in an information leaflet published by a municipal Youth Welfare Department (\textit{Jugendamt}) of a major German city, here translated into English:

\begin{quote}
If father and mother speak different languages, what should the children learn? Both languages at the same time? Or better one after the other? And in that case which language first? Because every family situation is different, no definite advice can be given. There's only one thing that can be said; it would be good if the children could feel one language to be their so-called mother tongue, i.e. could feel at home in one language. The advantage of growing up bilingually is often bought at the expense of insecurity in both languages, and this development should be avoided if at all possible. […] In most cases, the right thing will be for the children to grow up with the language of the country where they live, where they are growing up. Of course they will learn the national language later on anyway playing with their peers and at school, but if they have not always spoken it they won't come to recognize it as their mother tongue. The situation is different if the family intends to move in the near future, if, for example, an Italian father wants to return to Italy with his German wife and child. In such a case, it would of course be good if mother and child got used to the Italian language early on. Often, the foreign father will also wish his children to speak his, the father's language, even when he has left his homeland forever. In such a case, the father should try to put aside his understandable feelings in favour of his child's interests. (translated/cited by \cite{Saunders1988})
\end{quote}


While it is unlikely that Youth Welfare Departments still openly voice recommendations like this today, it is more than a fair guess that many childhood educators and teachers throughout the educational system, and even parents themselves, embrace it. Even among those who no longer think that humans need “one” language to “feel at home in” and to avoid linguistic insecurity, there is still a strong sense as to which languages are considered worth acquiring or supporting. The opinion that ``other'' languages are obstacles to learning majority languages and that their use can be interpreted as signs of failed integration or reluctance to integrate may even be more prevalent nowadays than ever, boosted by the outcomes of large panel studies testing language-related skills (OECD, PISA, etc.). Despite best efforts of research-driven initiatives (like BIVEM), of parents’ associations, and of highly motivated educators, many families still feel short of answers four decades after the advice quoted above. They often worry about language-related questions such as how to do ``the right thing'' and how to enable their children to become successful members of the society they live in without losing the language needed for communicating with family members and friends living in the countries of origin or in other parts of the world. Even people parents consider authorities and turn to, i.e. early childhood educators, teachers, and paediatricians, are often at a loss with respect to providing informed answers, except for those cases where language and multilingualism\hyp related topics have been integrated into curricula of educator and teacher training.

So who, if not linguists with expertise in acquisition, language use, and language change, could and should have an important message to convey to all these stakeholders? Yet, more is needed than expertise and willingness to engage with various interest groups. In Germany, like in many other countries, knowledge transfer (as mentioned in the Higher Education Framework Acts (``Hochschulrahmengesetze'')), is one of three missions, alongside research and teaching. While knowledge and technology transfer have been integral parts of the interface between the Applied Sciences and non-academic institutions and companies, the liberal arts or humanities have only gradually recognized what they have to offer in terms of relevant research findings, thereby also contributing to their departments’ and faculties’ “impact”. Experience with third-mission activities (community engagement, voluntary work, and the like) are often included in the list of desirable job qualifications. Increasing awareness of what linguistics has to offer by way of transfer can also be seen in the rise of funding programs through federal and local ministries of research, foundations, new conferences, academic and parental associations, and private organizations, for instance family networks across languages or for specific heritage languages. For a number of years, federally funded research initiatives~– like RUEG~– successfully received transfer project funding. This way researchers could engage in cooperations with a partner outside academia, a point to which we will return.

%%hier weiter: Gonser and Zimmer finden!!
In the following, “transfer” will be used as an umbrella term for activities aiming at sharing knowledge and expertise between researchers and non-academic partners. According to Gonser and Zimmer, these activities typically cover three main aspects of engagement. They distinguish communication (the presentation of results and materials for the public), consultation (the adaptation of scientific knowledge for use in concrete practice contexts) and cooperative actions, where further knowledge is jointly produced via cooperation between researchers and practice partners (\cite[19]{Gonser&Zimmer2020}). Gluns (\citeyear[249]{Gluns2020}) highlights the opportunities inherent in the latter, since in a “functionally diversified society, no actor is in possession of all knowledge relevant to a specific area like migration. Each actor has instead a particular perspective on the topic and can contribute implicit (experience-based) knowledge and/or scientific knowledge” (transl. by the authors).
\largerpage

In a very concrete sense, usefulness of and need for transfer activities in the language domain should be evident. After all, spoken, written, or signed: Whoever wants to reach out to potential audiences or cooperation partners has to use the very medium that is simultaneously part of the challenge and of the solution. This also means that all speakers (writers, signers) have first-hand experience to contribute. As competent speakers of at least one language, they are indispensable sources of data and can provide intuitions about formal well-formedness, paraphrases, ambiguity, as well as more or less appropriate formal and informal conditions of use in different registers. Regardless of specific sub-disciplines, empirical research with living subjects acutely depends on cooperation, i.e. on more or less collaborative contributions by those investigated. Hence, what used to be called ``subjects'' or “informants”, more recently “participants”, can quite rightly be considered “consultants”. This especially holds true for cases where linguists crucially depend on competent speakers providing data and reactions to specific questions – e.g. in the case of field work with unwritten languages. Moreover, in applied linguistics, given its concerns with ‘real-world problems’ and the forms and functions of verbal interaction, the move to explicit audience involvement should be a natural step to take. By drawing on the current project as an example, we will proceed to outline challenges and opportunities that such transfer initiatives entail.

\section{Methodological reflections: Setting out to outreach} \label{sec:purkathoferetal:3}
\largerpage

Already before the start of the RUEG transfer project focused on here, RUEG members~– individually or jointly~– had been contributing talks and workshops on multilingualism\slash heritage languages at conferences and for special interest groups like childhood educators, teacher and parents’ organisations, speech and language therapists, pediatricians and regional governments and foundations. Our previous involvement in various transfer activities had provided us with a network of partners to build on.

Reaching audiences outside of academia requires long-term investment, and given the often mobile nature of academic employment and changes in administrative staff in schools and other local agents, assuring continuity of tried and tested cooperations is as challenging as it is necessary. The transfer activities described here profited greatly from the fact that several team members already brought various academic and non-academic, local and international networks into the project, including both national and international academics who agreed to act as consultants and cooperation partners and, in some cases, co-authored papers. In order to establish cooperations, persistence and some flexibility in timing are necessary – in particular when it comes to matching plans for events with the partners. The RUEG transfer project’s main practice partners are aim (Akademie für innovative Bildung und Management\slash Academy for innovative education and management), Heilbronn, and the Servicezentrum der Berliner Volkshochschulen. By choosing to work with them, we opted for highly skilled and rather large partner organisations. Both our partner organizations are deeply invested in education, in the case of VHS Berlin of parents and the general public and in the case of aim Heilbronn mainly of educators and teachers. Both institutions offer a number of language-related courses and trainings. Our partners’ contributions were thus not just a prerequisite for funding on part of the DFG, but also indispensable in terms of reaching our diverse target groups.

Within the overall RUEG context, research findings on the ‘dynamics’ of Turkish, Russian, Greek, and German as heritage languages and shifting patterns of language use in diaspora settings, both in Germany and in the United States (see the other chapters, this volume) provided data for exemplification. At the same time, our activities were guided by what parents, teachers and language fostering staff themselves considered relevant questions and concerns. In order to tailor activities and materials accordingly, we started with a needs analysis.  Questionnaires distributed via our two partner organizations were filled out by about 50 parents enrolled in classes of the Servicezentrum der Berliner Volkshochschulen (Service Centre for Adult Education Berlin) and by about the same number of instructors of courses for early childhood educators and language support staff employed by the Akademie für Innovative Bildung und Management (aim, Academy for Innovative Education and Management) in Heilbronn, our Southern partner.

The questionnaires consisted of closed questions about attitudes towards multilingualism (i.e. ‘In which contexts do you use your family language?’), open questions about the language use of the speakers on a daily basis (i.e. ‘Which languages do you consider part of your everyday life?’), and they addressed either issues related to language transmission and use within the family in the case of our Berlin partners. Questionnaires targeting instructors revolved around issues they were confonted with in their own classroom contexts.

In the following, we mainly draw on open questions such as  – here roughly  translated – “How do you feel when you speak your different languages?”, “What kind of language\hyp related topics do you discuss in your family?” and “Are there questions about multilingualism that you yourself or participants in your classes would be interested in?”. Based on responses given, we identified repeatedly voiced issues as well as topics that we could confidently address on the basis of available evidence. In a more practical section, we also asked questions related to personal learning strategies and preferred learning contexts and materials, including videos, texts, and classes.
Overall, responses provided in the questionnaires demonstrated considerable awareness of the crucial role of the family as the first major agent in language socialization. Responses also revealed a high degree of insecurity in how to deal best with language-related issues and conflicts in the family.  Therefore, we next turn to family language dynamics in \sectref{sec:purkathoferetal:4} and to educational contexts in \sectref{sec:purkathoferetal:5}. Both times we take as our starting point the information gained through our questionnaires.


\section{Family language dynamics: Linguistic resources and repertoires} \label{sec:purkathoferetal:4}

Families, qua primary caregivers, are the most crucial players in early language socialization. They provide the input relevant for the dynamics of language acquisition to unfold in the direction of specific languages. The plural in ‘languages’ is fully intended here since there may well be more than one language involved right from birth. In addition to our biological predisposition and cognitive abilities, caregivers’ input (spoken or signed) supplies what is needed for L1 acquisition to take off. Whether and how development continues beyond early childhood depends on exposure as well as on the minority or majority status of the languages concerned.
Sooner or later, multilingual parents tend to raise, or are confronted with, questions of how best to support their children’s development (\cite{Lanza2004}). The spectrum of circumstances in which languages are acquired is basically without limits, especially if one considers exposure-related quantitative and qualitative variation: Is there, – to take one of the questions addressed in parent classes – a difference in outcome between children exposed to a specific language for only two to three hours a day vs. a full day a week? We simply don’t know. Questions of such granularity may be irrelevant, anyway. What we do know, however, for a number of languages, is which patterns (e.g. with respect to word order, morphological and phonological subsystems) can get firmly established in language A despite competition from language B. We also know what properties are harder to figure out, even regardless of language contact, and which therefore need more exposure time and more access to specific contrasts in the input, even in the monolingual (cf. \cite{Tsimpli2014, chapters/05}). Research on the combination of different heritage languages within the same majority context, and attention to individual differences in exposure (due to varying family size and differences in family language policy) therefore offer a natural laboratory for investigating both acquisition processes and outcomes.
In order to illustrate parental attitudes, we selected written examples from the questionnaires collected in Berlin. We quote the German original and provide a rough English translation.
Excerpt \REF{ex:04:1}, mother:

\ea \label{ex:04:1}
ich fühle mich total entspannt und es ist mir total lustig hören wie meine tochter gleischzeitig beide sprachen lerne\\
\trans`I am completely relaxed and I find it completely funny how my daughter learns both languages at the same time.'
\z

\begin{sloppypar}
Parents adopt very different family language policies \citep{Romaine1989, Gawlitzek-Maiwald&Tracy1996, Lanza2004, DeHouwer2007, Tracy2008, Mueller&al.2018}, often building on their own experiences of multilingual upbringing (\cite{Purkarthofer&Steien2019}). While some families decide on one person-one language strategies or select one language as a common family language used by both parents, others~– especially if they feel confident in both languages~– may freely alternate. However, initial decisions in favour of a particular policy do not preclude changes later on, especially if children react differently than what parents hoped for (\cite{Gawlitzek-Maiwald&Tracy1996}). An often disheartening experience on part of parents arises when children refuse to speak one of their languages even though they are likely to react appropriately when addressed in it.
In one questionnaire, a mother provided a clear description of her feelings in this respect.
Excerpt \REF{ex:04:2}, mother:
\end{sloppypar}

\ea \label{ex:04:2}
Ich bin enttäuscht, dass meine Kinder nicht mehr so gut Türkisch können. Sie verstehen manche Wörter nicht\\
\trans `I am disappointed that my children cannot speak Turkish so well any longer. They don’t understand some words.'
\z

While shifts – temporary or permanent – in children’s language preferences may occur in all bilingual settings, minority languages which neither enjoy the status of a language taught in schools nor peer or community support are harder to cultivate. Hence, it is important to reach out to families, for instance by setting up regular meetings involving parents and kindergarten teachers, where games and activities aimed at fostering the majority language can then be replicated in diverse heritage language settings at home (\cite{Tracy&al.2009}; for an initiative of bringing parents into classrooms, see cf. \cite{Prasad2017}).
Another parent, speaking about Italian as a heritage language, considers change inevitable.
Excerpt \REF{ex:04:3}, mother:\largerpage

\ea\label{ex:04:3}
In diese Zeit ist es mehr wichtig für mein Kind Deutsch zu sprechen und schreiben zu lernen. Er kennt sehr gut Italienisch, aber ist wichtig für andere italienische Kinder zu besuchen.\\
\trans `At this point in time, it is more important for me that my child learns to speak and write German. He knows Italian very well but it is important for him to see other Italian children.'
\z

From this parent’s perspective, language use is necessarily tied to interaction, and she recognizes that heritage language maintenance need peer-support. In light of the societal attitudes towards bi- and multilingualism, family languages – heritage languages of immigrants as well as many indigenous minority languages and regional dialects – are threatened (cf. overviews in \cite{Baker2011}, \cite{Brehmer&Treffers-Daller2020}). However, there are cases of the successful revitalization of indigenous languages by conscious efforts and investments in the education system and the consultation of families. In Wales, for example, midwives provide first language guidance on the bilingual upbringing of new-born children (\cite{Edwards2015}), and in New Zealand and Hawaii, family practices are actively supported by early childhood education, schools and immersion-oriented ‘language nests’ (\cite{Hinton2013}). These early childcare facilities operate with a strong language policy to foster the minority language and provide language input for children and parents alike by setting a positive example. This is particularly relevant where parents themselves were no longer in close contact with non-emigrated family members. In other cases, heritage languages are used as the main language of encounter and have served as a mediating lingua franca between family settings and a multilingual community. In recent years, due to war-driven migration to Germany, we have seen an added awareness of i.e. Arabic as a heritage language since speakers and community infrastructure were crucial to deal with needs of recently arrived refugees, at times including unaccompanied minors.
Teachers and educators in early childcare are important target groups for transfer activities since parents often consider them authorities with respect to developmental and educational issues. The same goes for medical professionals as key figures who, for instance in regular medical checks, assess the ‘typical’ development of pre-school children. Even though medical professionals rarely have access to up-to-date information about language acquisition and even less so on specific questions of simultaneous bilingualism, their opinions are crucial, for instance when it comes to decisions on speech therapy. Not surprisingly, various projects have shown that there is a considerable risk of over- and under-diagnosing multilingual children and their need for therapy (e.g. the CAMINO and MILA projects by \cite{Schulz2013, VoetCornelli&al.2013, Tracy&al.2018}).
In the following we provide a brief outline of educational contexts  impinging on family language dynamics. Insight into these linguistic landscapes helps us identify the potential for outreach and transfer.


\section{Family languages in educational contexts} \label{sec:purkathoferetal:5}\largerpage

Among the resources available and relevant for the development of family languages are all kinds of educational institutions (cf. \cite{Polinsky&Kagan2007, Baker2011, Kasstan&al.2018, Montrul2018, Polinsky2018, Gagarina&Milano2021}). For parents, the availability and goals of heritage language education can be crucial in finding their own position towards multilingual family language practices. In Germany, few heritage languages feature in the mainstream education system: the bilingual European Schools with different heritage languages in addition to German (e.g., Italian, Portuguese, Turkish, and Russian) started in the 1970s but have since not found a high number of followers (\cite{Niedrig2001}). A recent publication addressed Turkish and Russian as particularly important heritage languages in Germany, featuring to some extent in the educational system (\cite{Yildiz&al.2017}). With respect to Italian speakers in Germany, \citet{Caloi&Torregrossa2021} also stress the importance of school-based support for heritage language maintenance.

For students whose home languages are not school languages, HL education is organised under regional regulations and often depends on funding from the countries where these languages have majority language status. Teaching is typically focussed on standard majority contexts, e.g. Russian spoken in Russia, aiming at ensuring children’s competence in registers which they may not really perceive as “their” heritage due to social or dialectical variation (\cite{Woerfel&al.2020}). Therefore, it seems that teachers in heritage language education and in majority language classes as well as in preschool education need to develop a wider understanding of bilingual language acquisition and multilingual learning (\cite{Baker2011}, \cite{Krifka&al.2014}, \cite{Seals&Olsen-Reader2019}). In addition, the use of vernaculars and (often imagined) connections between social groups and linguistic resources raise relevant questions which have not been answered, e.g. for Germany: How are vernaculars present in the school context (\cite{Nero&Ahmad2014})? In which ways are social distinctions addressed with regard to heritage languages (\cite{Flores&Rosa2015})? How can discrimination based on assumed immigrant or language background be addressed (\cite{Bonefeld&Dickhaeuser2018})? How should schools respond to political demands requesting constraints on language use (\cite{Wiese&al.2020})?
In the questionnaires distributed among lecturers running classes for educators in Heilbronn, we see the same issues and questions addressed.
Excerpt \REF{ex:04:4}, lecturer:
\largerpage[2]

\ea\label{ex:04:4}
Wie sich der Loyalitätskonflikt minimieren lässt, in dem sich Kinder befinden, deren Eltern sich in Deutschland nicht wohlfühlen. Weil Kinder das spüren, kann es dazu führen, dass sie ihre eigene Sprachentwicklung blockieren, um ihren Eltern nicht in den Rücken zu fallen. Letztlich steht dahinter eine Integrationsfrage. Wie kann Integration besser gelingen, vor allem bei Personen, deren Aufenthaltsstatus nicht geklärt ist.\\
  
\trans `How can the loyalty conflict be minimised that affects children whose parents do not feel at home in Germany. Because children can feel that and this in turn can lead to them blocking their language development in order to not turn against their parents. But in principal, this is a question of integration. How can integration work better, in particular for persons whose status of residence is unclear.'
\z

Language skills are indispensable for all school contexts. Regardless of subject, teachers need to be aware of typical language-related ‘hidden’ barriers to comprehension and learning. Beyond this general requirement, insights into types of language acquisition and into properties of heritage languages are necessary for students in teacher training programs in Germany.
On the whole, the following players can be identified as immediately relevant for language-related concerns:
1) preschool educators (who are in charge of language fostering and increasingly of language-related diagnostics), 2) teachers of German as a school subject, including German as a second and as a foreign language, 3) teachers of heritage languages studying in university programmes (for example at the University Duisburg Essen, and at Berlin universities where such a program is planned), 4) foreign language teachers of typical school languages (English, French) but also of other subjects, among them increasing numbers of heritage speakers (which is, for instance, the case in Mannheim, where a curriculum for future highschool teachers and teachers of professional schools (“Berufsschulen”) has been established, taking into account the linguistic heterogeneity of both highschool students and their future teachers (cf. \cite{Karst&al.2021}), 5) teachers in training programs for other languages, where no institutionalised teacher training exists (e.g., Arabic, Kurdish), and, finally 6) teachers brought in from other countries for complementary schooling (e.g. HL education in Saturday schools) (\cite{Woerfel&al.2020}).
These target groups are important as they work with children and adolescents, and have ample opportunity and legitimate reasons for reaching out to parents. Their take on speakers and learners of any language, including heritage languages, impacts children, their families and communities (\cite{DeKorne2017}, \cite{Seals2018}).
Awareness of acquisition processes in general and of heritage language development in Germany (and in other countries) prepares teachers for their multilingual teaching realities. It also offers an additional basis for reflection to the growing number of future teachers who are themselves heritage language speakers. However, this latter fact does not automatically make them more efficient or insightful teachers since they themselves have often internalized stereotypes which have to be addressed (\cite{Thoma&Ofner2021}).
Excerpt \REF{ex:04:5}, lecturer: 
\largerpage

\ea\label{ex:04:5}
Nicht wenige Teilnehmende gehen davon aus, dass mehrsprachig aufwachsende Kinder, vor allem wenn ihre Familiensprache(n) keine der Bildungssprachen des globalen Nordens ist, weniger verstehen. Bildungssprachliche Aspekte des (dialogischen) Vorlesens gehören deshalb fast immer dazu.\\

\trans `A relevant number of participants assumes that multilingual children whose family languages are not from the Global North understand less. Therefore, aspects like (dialogically) reading to them are almost always needed.'
\z

Transferring research results to educational settings and to encounters with parents calls for an approach that takes speakers’ agency seriously and shows awareness of their multilingual environments. In the following section, we expand on specific strategies for working with educators and parents.

\section{Family languages in outreach and transfer: Potential momentum} \label{sec:purkathoferetal:6}

\subsection{Starting from an information hub}

In this section, we describe specific proposals to strengthen heritage speakers’ resources, firstly through communication and consultation and secondly through citizen science activities. We set out with concrete insights into the website that serves as our information hub and will then go in detail about the categories that structure our teaching and learning material.

Educational resources and a modular curriculum for workshops with parents of children in majority or heritage language education are our main tool to transmit knowledge. In order to illustrate the approach taken in communication and consultation activities, a screenshot from the website can be found in \figref{fig:purkarthoferetal:website-entry}.


\begin{figure}
    \includegraphics[width=\linewidth]{figures/Ch4_figure_2_website_entry.png}
    \caption{Entry page for parents with links to the five categories coded in the respective colours.}
    \label{fig:purkarthoferetal:website-entry}
\end{figure}


Highlighted in this screenshot is the link to resources on multilingual speakers’ experiences, as in the case of a video with an international researcher and Mercator fellow of the RUEG group who speaks about her own growing-up as a heritage speaker of Portuguese in Germany.\footnote{\url{https://www.uni-due.de/germanistik/rueg/videodrei.php}} She explains how she experienced heritage language education as a child, and she also comments on the fact that her bilingual upbringing helped in the pursuit of her career as a professor of German in Portugal (see \figref{fig:purkarthoferetal:website-detail}). The video is recorded in German, with subtitles available in German, Ukrainian, Greek, and English, with more languages to be added.


\begin{figure}
    \includegraphics[width=\linewidth]{figures/Ch4_figure_3_website_detail.png}
    \caption{Example of a video recording of a heritage language speaker reflecting on her experiences.}
    \label{fig:purkarthoferetal:website-detail}
\end{figure}

In order to engage viewers with the contents of the video, activities, e.g. a set of questions or input for discussion for children and adults, are presented. Visitors can access the site themselves, but educators can also make use of the videos and activities suggested to e.g. initiate a discussion during a parents’ meeting. Additional resources are provided, such as a chart on languages in the German school system that can inform parents about the status of different languages. Alternatively, specific information on heritage language education is available for contexts where parents would want to make decisions about signing their children up for classes. As an entertaining activity for all, a quiz is added with facts and myths about heritage languages (e.g.: “Who is considered a native speaker?”, ``How many speakers of heritage languages live in Germany?'', etc.). The quiz can be taken individually or in a group.

Finally, website links are given to activities that invite participants to think about their own upbringing and to reflect on their experience with schooling in more than one language. These citizen science explorations can, if a speaker wishes to do so, be uploaded and added to the collection of the project (as is described in \sectref{sec:purkathoferetal:6.2}).

All information, tasks, and discussion prompts are included in at least two or three languages, thereby making the materials suitable for diverse audiences. In particular, the subtitling helps groups to work on the same videos but to access them according to language preference.

\subsection{Communication and consultation} \label{sec:purkathoferetal:6.2}

Our encounters with large numbers of heritage speakers interested in their own family languages encouraged us to widen the range of information offered and to encourage speaker groups to select tools specific to their languages and professional background. This information was put in place through a multilingual website that offers access to different topics.\footnote{\url{https://www.ruegram.de}} The texts, videos, and audios are accessible via guided paths, and they are framed differently depending on languages chosen, roles (parents, teachers, children, adolescents) and media format. Each group is addressed explicitly through introductory texts, and the language used in descriptions is adapted in length and complexity. Certain activities are specifically designed for children, but many are just framed differently in order to respond to needs in a particular moment in life. Short video clips of multilingual speakers, most of them from the RUEG project group, offer an accessible way to understand language repertoires in specific contexts and to engage in discussions from there. In general, we cover the five categories already presented in the colourful pentagon above, with some detail following.

\subsubsection{Knowledge about languages and language use}

What early childhood educators need and interested lay persons, for instance parents, typically want to know are basic facts about language acquisition, typical developmental phases, and the relevance of input. They also need to know that bilingual children may go through phases of intensive language mixing and that this is neither chaotic nor an indication of identity or language confusion. Extensive previous experience with workshops geared to parents, childhood educators and elementary school teachers have shown that they can be demonstrated the linguistic and metalinguistic abilities of very small children (\citealt{Tracy2008, VoetCornelli&al.2013}).

\subsubsection{Research findings illustrating heritage speakers’ language use ‘live’ in language contact situations}

Illustrations of language contact phenomena that are highly accessible to non-linguists typically include the borrowing of words from the majority language and code-switching as normal bilingual practices. What does the set of RUEG projects contribute to these contents? In addition to what is known from previous research, transfer activities include specific illustrations from language combinations consisting of German or English and the heritage languages Turkish, Greek, Russian in Germany and in the US, including German as a heritage language in the US. Also, families speaking heritage languages are typically interested in learning about how their own language and language use have changed since immigration. Thus they can be shown in what respect the languages in their countries of origin have changed independently of the influence of language contact. Among the changes most easily demonstrable are changes in word order and morphological subsystems as well as the adoption of specific discourse markers, as described in detail in other chapters of this volume.

\subsubsection{Multilingual speakers’ experience}


As long as monolingualism is considered a perfect and the most natural state of the human mind, bi- and multilinguals are likely to continue doubting their own abilities. Non-experts easily find their own behavioural repertoires deficient, unaware of non-selective lexical access and the natural competition between simultaneously activated resources. Among the many experiences typical for heritage and also dialect speakers is the sense of having been shamed for inappropriate or supposedly flawed language use.
Heritage speakers of any age are often faced with negative expectations, underestimation, discrimination, and unfair evaluations based on suspected immigration background (\cite{Bonefeld&Dickhaeuser2018}).
Faced with specific texts, teachers often don’t realize that what is heard or read may be based on a particular informal register, and that this does not mean that the producer of this text was incapable of formulations in other styles or registers (\cite{Keim&Tracy2007}, \cite{Keim2008}). 

\subsubsection{Myths and discourses about languages and multilingual language use}

Throughout this text, various misconceptions were mentioned. Some reflect opinions no longer supported by the current state of the art of multilingualism research, others were ideological from the start. This module therefore highlights particularly persistent beliefs, such as the claim that bilingualism results in deficits in both languages with monolingualism being the “ideal”, and “more natural” state, that language mixing is a sign of incompetence, that children should not have to deal with more than one language in early childhood, that only one language should be acquired first, that people speaking a language we don’t know in our presence must want to hide something from us or talk about us \citep{Wiese&al.2020}. On the other hand, there certainly are advantages of sharing the privilege of a private code, already appreciated by bilingual children who early on speculate on who speaks what \citep{Tracy2008}.


\subsection{Citizen science}

One important goal of the transfer project was the involvement of interested people of all ages as lay researchers. To qualify as Citizen science (CS), research projects have to involve lay persons in data collection or analysis, or, as \citet[5]{Rymes2020} puts it: “Citizen science is the study of the world by the people who live in it [...]”. Golumbic and colleagues identify three key aspects of CS, namely (1) inclusion of citizens in the scientific process through shared activities; (2) contributions to both science and the public, and (3) reciprocity of information in both directions, between researchers and stakeholders outside academia \citep[6]{Golumbic&al.2017}. In their own study about researcher investment in CS, researchers were positive towards the idea that audience members can be educated through taking part in transfer activities and in particular in citizen science projects. Not surprisingly, they were met with scepticism on the part of other researchers who doubted that untrained members of the public can actually contribute to science. Moreover, engagement with the public was often considered a time\hyp consuming activity of lesser value than doing “actual science”. Robinson et al.’s study summarises many challenges faced by CS projects across disciplines, and scientists have since addressed them, for example by establishing principles CS projects should adhere to \citep{Robinson2018}. These include more than a commitment to making the audience participate in data collection: They call for explicitly acknowledging and crediting work and creative contributions, and to care for scientific quality and credibility. \citet{Freitag&al.2016} published on strategies of CS programs in order to increase data quality and found this to be a topic of considerable concern for research projects dealing with environmental issues on Californian beaches: ``While all science can face challenges to its credibility, the specific context of citizen science, along with external assumptions about citizen science, can make establishment of its credibility particularly difficult. Implementing science projects with volunteers, often outside traditional science institutions and typically with limited resources, contributes to the credibility challenge. Preconceptions about citizen science also can be an issue.'' (\cite[2]{Freitag&al.2016}) In light of this, CS is necessarily embedded in discourses of scientific quality, and throughout the research process reflective practices should be included.

In the humanities, CS has been less of a topic \citep{Shirk&Bonney2018}, with the exception of digital humanities that have, for example, used reading skills of the public for deciphering old texts. For linguistics, advantages are not hard to see, for instance with respect to questions related to sociolinguistics (\cite{Rymes&Leone2014}, \cite{Svendsen2018}, \cite{Rymes2020}), or “the study of the world of language and communication by the people who use it and, as such, have devised ways to understand it that may be more relevant than the ways professional sociolinguists have developed” (\cite[5]{Rymes2020}). In our case, CS initiatives create opportunities for speakers of heritage and majority languages to actively engage with scientific results and to reflect on their own environment and communicative practices.

In addition to the activities sketched in \sectref{sec:purkathoferetal:6.2}, with their focus on research\hyp backed and yet necessarily low-threshold knowledge transfer about language and multilingualism, we successfully piloted CS activities suitable for future roll-out in cooperation with schools and other relevant agents, such as our cooperation partners. In a first step, data collection tools were made available on our website, followed by demonstrating how to use them in university classes, schools, as well as with parents.\footnote{RUEGram citizen science: \url{https://www.uni-due.de/germanistik/rueg/citizen_science.php}.}

Not only do we rely on participants using the website on their own but we also introduce CS tasks in our workshops and teaching. Students are encouraged to provide and work with CS data as part of their course assignments, educators are invited to bring their own collections of pictures to our workshops. Our experience shows that this also promotes discussions on data quality, methodology, and awareness of one’s own role and responsibility as researcher.  Students and teachers who previously felt rather timid about trying out projects and methods on their own profit greatly from the supervised use of online tools made available.
In a playful approach to linguistic research and language repertoires we invite participants to contribute their own experience, expertise and ideas. The topic ``Research in my own biography, family, and community'' contains six different tasks, as stated below. Each task is introduced by a short video (subtitled in up to six languages) and linked to texts that contain explanations and further material necessary, like an empty silhouette or a short questionnaire.

\begin{enumerate}
\item Language portrait: Participants are encouraged to map their language resources onto a body silhouette and to write and reflect on their own language biography, on language use in the family but also in educational and professional settings.
\item My language story – multilingual and multimodal narration: This task asks participants to upload their story as text/audio/video and to talk/write about their multilingual or monolingual experience.
\item My family language biography: Data about language genealogy is collected with the help of structured questions. This task is of particular interest in combination with the (1) portrait, (2) story, or (5) photo elicitation.
\item Lived experience of language: This task asks for a text about one episode or event linked to growing up with two or more languages. An example would be: ‘Write about one event where your languages helped you cope with a situation.’
\item Photo elicitation: Participants are asked to take four pictures from their environment that are closely linked to their family language(s). They are also asked to write a short explanation with each picture.
\item Citizen science census: Participants familiarise themselves with current micro-census practices and contribute their own anonymised family language profile. Blueprints for mapping activities in speakers’ own environments are discussed.
\end{enumerate}

Individual CS researchers can then either complete tasks for themselves or involve family members for answers. Once drawings are completed, texts written and questionnaire answered, the website provides an upload form where participants enter a self-selected code and upload their data. This way, we encourage engagement with the website and expand the overall data pool for new CS activities.
Via the self-selected code, data from the same person can be linked and analysed on the basis of methods developed for language biographical projects (e.g. \cite{Purkarthofer2016}) and with regard to language use in specific languages, registers and modes. In contrast to the standardized data elicitation procedure for the RUEG corpus (see \cite{chapters/02}), our CS activities encourage contributions in a multilingual mode and hence language mixing. As the data collection is still ongoing, results cannot be published yet, but an overview of tasks, corresponding instructions, and tools is available.

\section{Conclusion} \label{sec:purkathoferetal:7}

This contribution dealt with steps undertaken in a specific transfer project in collaboration with two “practice” partners who were already deeply involved in adult education in two areas of Germany. What the members of the RUEG transfer project and, with them indirectly, the whole research group brought into this cooperation was state-of-the art insight into current research on multilingualism, both from a sociolinguistic and a psycholinguistic perspective, and previous experience with outreach and public relation activities.

In the domain of language, transfer projects are not just compelling but also self-perpetuating because they can draw on readily available or arousable interest of speakers and their experience of themselves as speakers of at least one language. From early childhood on, speakers are interested in language, enjoy language play, and prove highly sensitive to the way they are spoken to and comment on language use (see \cite{Tracy2008}, \cite{Tracy&Gawlitzek2023}).
Lay people can easily understand that building up rich repertoires for expressing themselves presupposes encounters with situations in which these repertoires are needed and that the environment must make an effort to provide these learning opportunities. It clearly works in our favour that communication and language use are among our  favourite activities, as very aptly formulated by Levelt (\citeyear[169]{Levelt1989}): “We, homo sapiens, are fanatic speakers. Most of us talk for several hours a day, and when we are not chatting with others, we are probably talking to ourselves.” For those living with more than one language, talking to themselves in more than one language is part of their multilingual reality and to be expected, as is occasional competition between resources co-activated.
Transfer initiatives in the domain of language appeal to what people know and can do. This differs from other academic areas where researchers convey results of their work to the public. Thus they can make explicit remarkable skills both readily available and ‘by their very nature’ of interest to the individual. For anyone wishing to empower individuals, families, educators, and our already multilingual societies, the importance of this message cannot be overstated.


\section*{Acknowledgements}
The support of the overall RUEG team is gratefully acknowledged, this includes our internal reviewers Heike Wiese and Natalia Gagarina as well as the external reviewers. Heartfelt thanks to our research assistents Anne Mölders, Geylan Ahmed Daud, Özge Zar, Serpil Kuzay, Diana Lehl, and Bahar Yilmaz, as well as to Lea Coy for help with pre-publication formatting.

The authors thank all participants in outreach and citizen science activities: as learners and parents, as educators and teachers, as students and creative co-workers and finally as multilingual subjects. The support of other RUEG projects is gratefully acknowledged. Most importantly, without the external cooperation partners in Berlin and in Heilbronn, their investment of time and resources, including access to their target groups, the project would not have been funded by the DFG, and without their excitement and contribution of ideas it would certainly have been less exciting to run.

The research was supported through funding by the Deutsche Forschungsgemeinschaft (DFG, German Research Foundation) for the Research Unit \textit{Emerging Grammars in Language Contact Situations}, project Pt (grant number: 313607803).

\printbibliography[heading=subbibliography,notkeyword=this]

\end{document}
