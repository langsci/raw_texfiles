\documentclass[output=paper,colorlinks,citecolor=brown]{langscibook}
\ChapterDOI{10.5281/zenodo.15775187}

\author{Annika Labrenz\orcid{0000-0002-6235-9321}\affiliation{Humboldt-Universität zu Berlin} and Kateryna Iefremenko\orcid{0000-0003-3711-0935}\affiliation{University of Potsdam; Leibniz-Centre General Linguistics} and Kalliopi Katsika\orcid{0000-0002-6736-4963}\affiliation{University of Kaiserslautern-Landau} and Shanley E.M. Allen\orcid{0000-0002-5421-6750}\affiliation{University of Kaiserslautern-Landau} and Christoph Schroeder\orcid{0000-0003-1188-7746}\affiliation{Humboldt-Universität zu Berlin, University of Potsdam} and Heike Wiese\orcid{0000-0002-6310-3045}\affiliation{Humboldt-Universität zu Berlin}}

\title{Dynamics of discourse markers in language contact}

\abstract{This chapter addresses the dynamics of discourse markers (DMs) in language contact. DMs are functionally variable elements that are grammatically not integrated and hence easily transferable – in terms of matter as well as in terms of pattern. This makes them especially dynamic in language contact and, in particular, in heritage languages. We present findings from the \textsc{rueg} corpus on verbal DMs and on the three-dot sign, which we approach as a graphic DM, in different heritage languages (German, Greek, Turkish, Russian) and majority languages (English, German). Qualitative findings indicate an impact of societal status differences on frequencies, functional extension/restriction, and (inverted) pragmaticalisation processes.

\keywords{discourse markers, heritage language, majority language, pattern replication, functional convergence}
}



\IfFileExists{../localcommands.tex}{
   \addbibresource{../localbibliography.bib}
   \usepackage{langsci-optional}
\usepackage{langsci-gb4e}
\usepackage{langsci-lgr}

\usepackage{listings}
\lstset{basicstyle=\ttfamily,tabsize=2,breaklines=true}

%added by author
% \usepackage{tipa}
\usepackage{multirow}
\graphicspath{{figures/}}
\usepackage{langsci-branding}

   
\newcommand{\sent}{\enumsentence}
\newcommand{\sents}{\eenumsentence}
\let\citeasnoun\citet

\renewcommand{\lsCoverTitleFont}[1]{\sffamily\addfontfeatures{Scale=MatchUppercase}\fontsize{44pt}{16mm}\selectfont #1}
  
   %% hyphenation points for line breaks
%% Normally, automatic hyphenation in LaTeX is very good
%% If a word is mis-hyphenated, add it to this file
%%
%% add information to TeX file before \begin{document} with:
%% %% hyphenation points for line breaks
%% Normally, automatic hyphenation in LaTeX is very good
%% If a word is mis-hyphenated, add it to this file
%%
%% add information to TeX file before \begin{document} with:
%% %% hyphenation points for line breaks
%% Normally, automatic hyphenation in LaTeX is very good
%% If a word is mis-hyphenated, add it to this file
%%
%% add information to TeX file before \begin{document} with:
%% \include{localhyphenation}
\hyphenation{
affri-ca-te
affri-ca-tes
an-no-tated
com-ple-ments
com-po-si-tio-na-li-ty
non-com-po-si-tio-na-li-ty
Gon-zá-lez
out-side
Ri-chárd
se-man-tics
STREU-SLE
Tie-de-mann
}
\hyphenation{
affri-ca-te
affri-ca-tes
an-no-tated
com-ple-ments
com-po-si-tio-na-li-ty
non-com-po-si-tio-na-li-ty
Gon-zá-lez
out-side
Ri-chárd
se-man-tics
STREU-SLE
Tie-de-mann
}
\hyphenation{
affri-ca-te
affri-ca-tes
an-no-tated
com-ple-ments
com-po-si-tio-na-li-ty
non-com-po-si-tio-na-li-ty
Gon-zá-lez
out-side
Ri-chárd
se-man-tics
STREU-SLE
Tie-de-mann
}
   \boolfalse{bookcompile}
   \togglepaper[14]%%chapternumber
}{}

\begin{document}
\lehead{Annika Labrenz et al.}
\maketitle

\section{Introduction}
\label{sec:Introduction}
Discourse markers (DMs) are linguistic devices that speakers use to signal a range of pragmatic and communicative functions in their speech and writing. These markers are functionally variable elements, and their polyfunctional nature \citep*[e.g.][]{moosegard_hansen_function_1998,schiffrin_discourse_1987,schourup_discourse_1999} makes them a highly versatile lexical domain in general \citep{auer_entstehung_2005}. This inherent characteristic of DMs may be conducive to innovation, especially in informal communicative situations where there is less normative pressure compared to more formal situations. In language contact settings, the use of DMs can be particularly interesting, as speakers’ repertoires contain DMs from different languages whose functions might overlap. Research suggests that bilingual speakers profit from shared mental planning processes while using different resources in their repertoire \citep[cf.][69]{matras_contact_2010}. As a consequence, bilingual speakers might have access to a shared functional pool of such DMs. Previous studies discuss two main reasons for the dynamics of DMs in language contact: 1) their detachability and 2) their tendency for functional convergence.

\begin{itemize}
    \item[(1)] Detachability: According to Matras, DMs are highly detachable not only in a structural sense but also from the bulk of the lexicon (\citealt*{matras_utterance_1998,matras_language_2020}; see also \citealt{fuller_principle_2001}).
    Because of their “pragmatic role […] as highly automatic conversational routines” \citep[209]{matras_language_2020} and their gesture-like quality, Matras suggests the involvement of distinct control mechanisms compared to those governing lexical items or grammatical inflections \citep[209]{matras_language_2020}. In bilinguals, this easily leads to the tendency to integrate DMs from one language into the other, as various studies on borrowings have shown (e.g. \citealt{salmons_bilingual_1990, goss_evolution_2000, fuller_principle_2001, matras_utterance_1998}, see also \cite{chapters/02}). This phenomenon is also called matter replication \citep[see][]{matras_grammatical_2007, matras_contact_2010}.
    
    \item[(2)] Functional convergence: DMs are not only susceptible to matter replication (or borrowing) but also to pattern replication (see \citealt{matras_grammatical_2007}, \citealt[212]{heine_rise_2021}). For the discourse domain and specifically DMs, this means that the functional spectra of specific DMs have the potential to converge. This can lead to functional extensions, restrictions and/or an increase in frequency.
\end{itemize}

\noindent While previous studies on DMs in language contact settings mainly focused on borrowings (e.g. \citealt{salmons_bilingual_1990}; contributions in \citealt{maschler_discourse_2000}), this chapter focuses on functional convergence in the discourse domain. We examine the dynamics and variability of DMs across different languages and contact settings where languages have a different societal status: They have either the status of a minoritised heritage language that is mainly spoken in families and in certain communities but is not the language of the larger society or as a majority language, i.e. the language spoken by the larger society in which administrative processes and school education usually take place. A typical scenario for bilingual heritage speakers (HSs) usually results in them being dominant in their majority language, at least in certain communicative situations, at some point after the start of school \citep[cf.][]{rothman_understanding_2009, flores_different_2019}. As a consequence, one might expect a stronger influence of the majority on the heritage language. In this chapter, we will look at how this influence manifests itself in the case of DMs. More specifically, we will investigate how overlapping functional spectra of DMs in one language influence the use of DMs in a contact language. This leads us to our first research question: What are the patterns of functional variation, extension, and restriction of DMs in language contact? 

We also introduce a new kind of DM, namely graphic DMs, such as the three-dot sign, from instant messaging (\citealt{labrenz_three-dot_2022}; see also \citealt{wiese_emoji_2021} on emoji as graphic DMs), thus including the (informal) written domain into DM research. By so doing, we follow Imo's call to broaden DM studies \citep{bluhdorn_diskursmarker_2017-1}. Graphic DMs are an interesting case for studies in language contact because they are not confined to any specific language, but rather keep their form or matter across languages, which makes them translinguistic elements. In view of this fact, it is particularly interesting to see if language-specific uses or functions still occur or if the same matter also leads to cross-linguistic universal patterns. This leads us to our second research question: Are there language-specific uses of graphical DMs even if they have the same form across languages? If so, what are the dynamics in bilinguals?
In pursuit of these questions, we conduct a cross-linguistic study that includes a range of contact settings which allows us to disentangle the effects of societal language status and bilingualism in general.
The chapter is structured as follows: \sectref{sec:labrenzetal:Verbal_and_graphic_DMs} discusses conceptual aspects of DMs, followed by an overview of the functions of the verbal and graphic DMs explored in this chapter based on prior literature. \sectref{sec:labrenzetal:Method} presents the database and outlines the procedure of the corpus study. In \sectref{sec:labrenzetal:Findings}, we present the findings regarding specific uses in language contact. Following a brief discussion and the synthesis of the findings in \sectref{sec:labrenzetal:Discussion}, we conclude in \sectref{sec:labrenzetal:Conclusion}.

\section{Verbal and graphic DMs}
\label{sec:labrenzetal:Verbal_and_graphic_DMs}
DMs typically come from a variety of part-of-speech categories, such as adverbs and conjunctions (cf. \cites{schourup_discourse_1999}[69]{crible_discourse_2018} for an extended enumeration). According to the literature, such polyfunctional lexical items carry a core meaning that has been partly lost through a process of semantic bleaching and syntactic disintegration \citep[cf.][94]{hopper_grammaticalization_2003}, a process also referred to as pragmaticalisation or grammaticalisation (see \citealt{brinton_evolution_2017} and \citealt{diewald_pragmaticalization_2011} for a discussion of these terms). A consequence of this process is that DMs contribute solely to the discourse-functional level. And this can be said for verbal as well as graphic DMs alike (cf. \citealt{wiese_emoji_2021} on emoji, \citealt{labrenz_three-dot_2022} on the three-dot sign). We define DMs 
\begin{quote}
    […] as elements that are not fully syntactically integrated and do not directly contribute to the propositional meaning and truth value of an utterance but rather operate on the level of discourse \citep[cf.][]{blakemore_relevance_2004, fraser_towards_2006, bluhdorn_diskursmarker_2017}. \hfill \citep[3]{wiese_emoji_2021}
\end{quote}

\noindent Throughout the literature there are different approaches to what should be included in the category of DMs. This terminological inconsistency in the field \citep[e.g.][]{andersen_pragmatic_2001, fraser_what_1999} makes it particularly challenging to work cross-linguistically on discourse-pragmatic markers. In this paper, we address this challenge by taking on a functional perspective on DMs \citep[cf.][]{maschler_metalanguage_2009}. For this purpose, we consider a range of functions that are prototypical for the class of DMs (see also \citealt{pons_borderia_using_2021} on DM features). 

For instance, connectors that mark semantic relations are more lexicalised and, in our view, less prototypical as DMs than markers that solely structure discourse. This is in line with \citet{ariel_pragmatic_1994}, who distinguishes between “semantically transparent markers as opposed to relatively opaque ones” \citep[3251]{ariel_pragmatic_1994}. Pragmatic detachability \citep{matras_utterance_1998} in terms of degree of lexicalisation and semantical transparency can thus serve as indicators for prototypicality of DMs. Following this logic, English \textit{so} as a semantically transparent marker, is less typical for the class of DMs than English \textit{well} (\cites{ariel_pragmatic_1994}[62]{muller_discourse_2005}).

Below we give an overview of the canonical functional spectrum of the items under investigation – that are German \textit{so}, Turkish \textit{yani} and \textit{işte}, Greek \textit{etsi} and \textit{lipon}, and the three-dot sign based on previous descriptions in the literature, and list those functions that are relevant for later analysis. Since most of the verbal DMs overlap functionally with either English \textit{so} or with German \textit{so} or \textit{also}, we expect convergent developments between these markers. As they are important counterparts of the items under investigation, we first give a brief overview of their functional spectrum.

English \textit{so} is widely being acknowledged as an adverb of degree and manner (like German \textit{so}) and as a conjunction \citep[e.g.][62]{muller_discourse_2005}. As a conjunction, it indicates an inference or a consequence. In this function, it is seen as a DM by most researchers, although it might not be the most prototypical representative of its class (see \citealt[62]{muller_discourse_2005} and discussion in the Introduction). Another function of \textit{so} in peripheral position is that of initiating a new narrative sequence \citep[cf.][]{bolden_implementing_2009}. The functional spectrum of English \textit{so} encompasses initialising a narrative, indicating the end of a narrative, thematic switch, and [indicating a consequence]\footnote{Functions in square brackets indicate that they are not the most prototypical cases of the DM category.}.

\noindent German \textit{also} is a highly polyfunctional lexical item \citep[e.g.][]{alm_also_2007, fernandez-villanueva_uses_2007, labrenz_functional_2023}. A central use of \textit{also} is as a consequence indicating adverbial connector and as marker for indicating inferences \citep{konerding_semantische_2004}. It is also commonly used as a repair marker, including the indication of elaborations/specifications, corrections, restarts and hesitations \citep[see][]{pfeiffer_uber_2017}. Similar to English \textit{so}, it can be used to initiate a new narrative sequence. In addition, it can indicate the speaker’s stance (evaluation) \citep{labrenz_functional_2023}. The functional spectrum of \textit{also} includes initialising a narrative; indicating evaluations/inferences, elaborations, or corrections/restarts; and [indicating a consequence (adverbial connector)]\footnote{German \textit{also} is not a DM in this case: When it is used for indicating a consequence on the propositional level it is syntactically integrated.}.

German \textit{so} is a highly polyfunctional lexical item and its classification has been widely discussed in a range of grammatical descriptions and studies (see \citealt{schumann_fokusmarker_2021} for an overview). A central function of \textit{so} is its deictic use \citep[e.g][]{thurmair_vergleiche_2001,ehlich_so_2007}.
As a modal indexical marker, it can indicate manner, quality, or intensity \citep{wiese_so_2011, schumann_fokusmarker_2021}. Additional uses are as a comparison particle, as a quotative marker, as a progressive marker, and to indicate approximations and hedging \citep[cf.][]{hennig_grammatik_2006, wiese_so_2011, wiese_kiezdeutsch_2012, schumann_fokusmarker_2021}. In addition, \citet{hennig_grammatik_2006} mentions the use of \textit{so} in the function of initialising a new narrative sequence in left peripheral position \citep[cf.][]{hennig_grammatik_2006}. A non-canonical use of \textit{so} as a focus marker has been described by \citet{wiese_so_2011} and explored in detail by \citet{schumann_fokusmarker_2021}. In this use, \textit{so} is an optional element that is semantically bleached such that it does not contribute any meaning and solely contributes on the level of discourse \citep{wiese_so_2011}. As a focus marker and in hedges (in \textit{und so} ‘and so’ or \textit{oder so} ‘or so’), \textit{so} can be located in utterance final position (\cites{wiese_kiezdeutsch_2012}[48]{schumann_fokusmarker_2021}). Hence, in these uses it fits our criteria for DMs. In this paper, the focus function of \textit{so} in the right peripheral position will be treated as an assertion because, in contrast to medial positions, the scope is extended from a phrase to the entire preceding clause. German \textit{so} functionally overlaps with English \textit{so}, but the only overlap in discourse functions is the initialising function. The discourse-functional spectrum of German \textit{so} includes initialising a narrative, assertion, and hedging. Examples are provided in \sectref{sec:sub:labrenzetal:German_so}.

Turkish \textit{işte} is one of the most frequent discourse markers in Turkish \citep{yilmaz_pragmatic_2004}. Its main functions are marking shared knowledge between discourse participants, claiming the turn in a conversation, marking topic boundary by indicating the end of a discourse unit and emphasizing the speaker’s points \citep{ozbek_yani_2000, yilmaz_pragmatic_2004}. In the latter function which we call \textit{assertion}, it overlaps with German \textit{so}. In addition, \textit{işte} has functional overlap with English \textit{so} and \textit{well}. The discourse-functional spectrum of \textit{işte} encompasses indicating a thematic switch, ending a narrative, and assertion. Examples are provided in \sectref{sec:sub:labrenzetal:Turkish_iste}.

Turkish \textit{yani} is the most frequent discourse marker in conversational Turkish \citep*{yilmaz_descriptive_1994, yilmaz_pragmatic_2004, ozbek_discourse_1995}. Turkish \textit{yani} is originally a loanword from Arabic (\textit{yaʿnī}, “[he/she/it] means”) and entered into Turkish directly as a DM with predominant linking functions, such as indicating elaborations. However, in modern Turkish \textit{yani} is multifunctional, and in addition to the function of elaboration, it is used to summarize ideas, to emphasize the speaker’s stance on something (indicate an evaluation), to initiate a turn, and to hold the floor \citep{yilmaz_pragmatic_2004}. It functionally overlaps with German \textit{also}. Depending on its functional contribution in discourse, it overlaps functionally with English \textit{well} or \textit{so}. The discourse-functional spectrum of \textit{yani} includes indicating elaboration, evaluation, the end of a narrative, and hesitation. Examples are provided in \sectref{sec:sub:labrenzetal:Turkish_yani}.

Greek \textit{lipon} is used most frequently as a DM according to \citet{georgakopoulou_conjunctions_1998}. In formal grammars and Greek-English dictionaries, \textit{lipon} is characterised as a deductive conjunction \citep{triantafyllidis_modern_2019} which can be translated as “so”, “then”, “therefore”, “hence” or as an interjection meaning “well”, “so”, “now” expressing surprise, relief, query, or decision \citep[119]{stavropoulos_oxford_1988}. In spoken narratives, \textit{lipon} functions as a hypotactic marker \citep{redeker_ideational_1990} linking secondary parts of speech (such as comments, corrections, etc.) to the main narrative part \citep{antoniou_discourse_2008}. On the basis of their analysis of spoken and written narratives and non-narratives, \citet{georgakopoulou_conjunctions_1998} claim that \textit{lipon} is used for the transition from a secondary to the main line of narration but never the other way around. Such uses of \textit{lipon} can also be found in classroom environments where \textit{lipon} is used by teachers in turn-initial position to return to previously interrupted narration or discussion \citep{christodoulidou_meaning_2014}. There is thus functional overlap with German \textit{also} and with English \textit{so} or \textit{well}. The discourse-functional spectrum of \textit{lipon} includes initialising a narrative, and indicating a thematic switch. Examples are provided in \sectref{sec:sub:labrenzetal:Greek_lipon}.

Greek \textit{etsi}’s main function in Modern Greek is that of an adverb of manner or quantity \citep{tzartzanos_greek_2002}. Regarding its function as a DM, \citet{georgakopoulou_conjunctions_1998} did not find evidence that \textit{etsi} functions as a DM in spoken modality and only found limited use of \textit{etsi} as a DM in written modality. After conducting a corpus analysis of spoken Greek texts, however, \citet{tsampoukas_grammaticalization_2015} found a variety of DM functions of \textit{etsi}. More specifically, \citet{tsampoukas_grammaticalization_2015} provides evidence and argumentation that \textit{etsi} is currently in the beginning of a grammaticalisation process. The main DM function that \citet{tsampoukas_grammaticalization_2015} identified for \textit{etsi} is that it indicates a result or a consequence of a previously mentioned event or series of events in the discourse. Thus, similarly to German \textit{also} and English \textit{so}, \textit{etsi} may have the function of a consecutive connector~-- a function which we do not see as the most prototypical for the category of DMs, but which could be a step with potential for further pragmaticalisation. The discourse-functional spectrum of \textit{etsi} includes [indicating a consequence (emerging)]. Examples are provided in \sectref{sec:sub:labrenzetal:Greek_etsi}.

The three-dot sign's cross-linguistic canonical function is its use as a placeholder for omitted material (see \cites[301]{raclaw_punctuation_2006}[60]{baron_necessary_2011} on English; \cites[125]{bredel_interpunktion_2008}[68]{meibauer_how_2019}[154]{androutsopoulos_auslassungspunkte_2020} on Greek, \citealt[90]{rosenthal_spravochnik_2012} on Russian; \citealt[44 ff.]{turan_turk_2014} on Turkish). A number of additional functions come into play, especially when writing digitally. These functions closely resemble those of verbal DMs. Consequently, we approached the three-dot sign as a graphical DM. Like verbal DMs the three-dot sign is positioned at the periphery and does not contribute directly to the propositional meaning, but rather at the level of discourse. This perspective allowed us to subsume its functions under textual, subjective and intersubjective discourse functions. Examples are given in \sectref{sec:sub:labrenzetal:Procedure}.

\section{Method}
\label{sec:labrenzetal:Method}
\subsection{Database}
\label{sec:sub:labrenzetal:Database}

The informal and formal-spoken productions in the English, German, Greek, and Turkish subcorpora snapshot versions 1.0\footnote{\url{https://hu.berlin/rueg-corpus} (last accessed January 10th, 2024)} of the \textsc{rueg} corpus constitute the database for the study of verbal DMs. The informal-written productions (instant messages) of the \textsc{rueg} corpus version 0.4.0 \citep{wiese_rueg_2021} are the database for the study of the three-dot sign, and also include Russian. 

The data comprise naturalistic and comparable productions that were elicited using the Language Situations setup \citep{wiese_language_2020}. As described in more detail in \textcitetv{chapters/02}, in this setup, participants watched a video about a car accident and were asked to imagine themselves as eyewitnesses, and to tell different interlocutors about it in four different communicative situations. We focus on the spoken mode in formal and informal communicative situations for the verbal DMs (\tabref{tab:labrenzetal:1}). In these conditions, participants were asked to act out leaving a voicemail for the police and a voicemail via WhatsApp for a friend. For the three-dot sign, we focus on the informal-written communicative situation in which participants were asked to write an instant message to a friend (\tabref{tab:labrenzetal:2}).

\noindent Tables \ref{tab:labrenzetal:1} and \ref{tab:labrenzetal:2} give an overview of the word tokens and speakers included in the respective analyses. We included heritage German (\tabref{tab:labrenzetal:1}), Turkish, Greek (Tables~\ref{tab:labrenzetal:1} and~\ref{tab:labrenzetal:2}) and Russian (\tabref{tab:labrenzetal:2}) in contact with majority German and English and additionally looked at the respective languages spoken in a majority context (e.g. German in Germany, Turkish in Turkey). Each subcorpus contains bilingually and monolingually raised adolescents (age 14--18 years) and adults (22--35 years). The monolingually raised speakers grew up with English, German, Greek, Russian, or Turkish as their only language in the family. These languages have the societal status of a majority language in the respective countries of elicitation. The bilinguals were HSs of Greek, Russian, and Turkish, raised in Germany or the US and HSs of German in the US. Thus, they grew up with either German or English as a majority language, in addition to their respective heritage languages.

\begin{table}
\caption{Database for the analysis of verbal DMs, \textsc{rueg} corpus Snapshot 1.0: Number of overall word tokens in informal (is) and formal-spoken (fs) data and speakers across languages and language status (excluding heritage Russian speakers of the English and German subcorpus)}
\label{tab:labrenzetal:1}
 \begin{tabular}{l lrr}
  \lsptoprule
      Language      & Societal status & Tokens (is, fs)  & Speakers\\
  \midrule
  English  &   majority  &    63,813  &    222\\
  \midrule
  German  &   majority  &   59,300  &    175\\
          &   heritage  &  9,187  &  35\\
  \midrule
  Greek  &  majority  &  15,560  &  64\\
         & heritage  &  23,505  &  108\\
  \midrule
  Turkish  &  majority  &  11,538  &  66\\
           &  heritage  &  25,370  &  126\\
  \lspbottomrule
 \end{tabular}
\end{table}

\begin{table}
\caption{Database for the three-dot sign, \textsc{rueg} corpus version 0.4.0: Word tokens of informal-written (iw) data and speaker numbers across languages, societal language status, and countries}
\label{tab:labrenzetal:2}
 \begin{tabular}{l lrr}
  \lsptoprule
      Language      & Societal status & Tokens (iw)  & Speakers\\
  \midrule
  English  &   majority  &    16,363  &    187\\
  \midrule
  German  &   majority  &   25,094  &    165\\
  \midrule
  Greek  &  majority    &  5,488  &  64\\
         & heritage-DE  &  3,819  &  47\\
         & heritage-US  &  3,391  &  64\\
  \midrule
  Russian  &  majority     &  4,398  &  67\\
           &  heritage-DE  &  4,927  &  58\\
           &  heritage-US  &  3,889  &  66\\
  \midrule
  Turkish  &  majority     &  4,222  &  64\\
           &  heritage-DE  &  4,502  &  65\\
           &  heritage-US  &  3,575  &  56\\
  \lspbottomrule
 \end{tabular}
\end{table}

\subsection{Procedure}
\label{sec:sub:labrenzetal:Procedure}
For the verbal DMs, we selected specific polyfunctional lexical items \citep[1]{pichler_introduction_2016} from each language that serve discourse-pragmatic functions that partly overlap with their counterpart in the respective contact language. In particular, we look at German \textit{so}, Turkish \textit{yani} and \textit{işte}, and Greek \textit{etsi} and \textit{lipon}. We chose these markers because they are among the most frequent DMs in the respective languages, and our corpus data indicated interesting patterns in heritage language use. We extracted all occurrences of these items along with their verbal context. Two annotators independently annotated positions and functions. The annotated positions include left \REF{ex:labrenzetal:1} and right \REF{ex:labrenzetal:2} peripheral and medial \REF{ex:labrenzetal:3} positions. To guide our functional annotation process, we referred to an established annotation catalogue derived from functions of these DMs as described in the existing literature (see \sectref{sec:labrenzetal:Verbal_and_graphic_DMs}). We regularly discussed cases of doubt. Special emphasis was placed on instances that did not align with the established functions of the respective DMs. In such cases, we determined functions by analysing the preceding and subsequent verbal content and the role of the discourse marker in that specific context.

\ea left periphery \label{ex:labrenzetal:1}\\
{\textbf{so} i: just witnessed an accident}\hfill[USmo05FE\_isE]\footnote{The speaker codes are constructed as follows: The first two letters refer to the country of elicitation (DE=Germany, TU=Turkey, GR=Greece, US=USA RU=Russia, TU=Turkey); the next two to the speaker’s language background (bi=bilingual, mo=monolingual; followed by age group (1--49=adults, >50=adolescents); then gender self-identification (F=female, M=male), heritage language (G=Greek, D=German, T=Turkish, R=Russian) or majority language in case of German (D) and English (E) monolinguals; communicative situation (iw = informal-written, is = informal-spoken, fw=formal-written, fs = formal-spoken); language of elicitation (D=German, E=English, G=Greek, R=Russian, T=Turkish).}\\
\z

\ea right periphery \label{ex:labrenzetal:2} \\
\gll weil äh die müssen Rücksicht auf uns nehmen \textbf{so} \\ 
because uh they {have to} respect of us give \textsc{dm} \\
\glt `because they have to respect us \textbf{so}' \hfill[DEbi26FT\_isD]
\z

\ea medial \label{ex:labrenzetal:3}\\
\gll ama iki tane araba \textbf{işte} kaza yap-tı \\ but two piece car \textsc{DM} accident do-\textsc{3sg.pst} \\
\glt `but (-) two cars \textbf{işte} had an accident' \hfill [DEbi26FT\_isD]
\z

\noindent For the three-dot sign the procedure was similar, but additional factors were taken into account: We extracted all occurrences of the three-dot sign together with their preceding and following context and annotated the data manually. In our analysis, we considered the position (lone, message-final, CU-internal, between discourse units, see \xxref{ex:labrenzetal:4}{ex:labrenzetal:10}) and the discourse context on the macro- and micro-levels. On the level of macro-structure, the entire message was taken into account and divided into opening, main narration and closing (\REF{ex:labrenzetal:4}, see also \cite{chapters/15}). At the level of micro-structure, preceding and following discourse units of the three-dot sign were classified according to their functional contribution as real-world-referring (mainly referring to the retelling of the accident as in ‘witnessed a rear end collision’ in \REF{ex:labrenzetal:4}), subjective (convey attitudes, stance, evaluations as in ‘weird day today’ in \REF{ex:labrenzetal:9}), and intersubjective (concerned with speaker-hearer relationship for example through greeting in \REF{ex:labrenzetal:4} \citep[255]{labrenz_three-dot_2022}). Subsequently, we derived textual, subjective, and intersubjective discourse functions of the three-dot sign considering its position and contextual environment.

\ea macro-structure \\
$[$hey \dots  sorry im running a bit late \dots witnessed a rear end collision.$]$\textsc{opening}\\ $[$someone last their ball and it went onto oncoming traffic!$]$\\ 
\textsc{main narration}\\
$[$need to stay and give testimony to the officers about what i saw! see you soon]\textsc{closing} \hfill[USbi19FG\_iwE] \label{ex:labrenzetal:4}\\

\z

\noindent Based on previous literature \citep[e.g.][]{androutsopoulos_auslassungspunkte_2020, busch_digitale_2021, meibauer_how_2019} and in view of position and context, we identified the following DM functions of this graphic device in our data and subsumed them under three broader categories as follows. 

\begin{enumerate}
    \item textual: discourse organisation \REF{ex:labrenzetal:5} and segmentation \REF{ex:labrenzetal:6}, creating dramatic effect \REF{ex:labrenzetal:7}
    \item subjective: indicating speechlessness and/or emphasis \REF{ex:labrenzetal:8}
    \item intersubjective: general openness of the communication inviting the interlocutor to react \REF{ex:labrenzetal:9}, \REF{ex:labrenzetal:10}
\end{enumerate}

\noindent In \REF{ex:labrenzetal:5} and \REF{ex:labrenzetal:6}, the three-dot sign has a textual function. While in \REF{ex:labrenzetal:2} it indicates a major shift at the macro- and micro-structure levels, in \REF{ex:labrenzetal:6} it is simply segmenting the informational content in the main narration (see \citealt[252--58]{labrenz_three-dot_2022} for details of the method).

\ea Between discourse units and between main narration and closing \label{ex:labrenzetal:5}\\
$[$они плсвонили полицие$]$\textsc{main narration} \textbf{...} $[$как ты думаешь: мне тоже надо в полицию поехать и им как свидетел всё расказать?$]$\textsc{closing}\\
{oni plsvonili polizie ... kak ty dumaesh: mne tozhe nado v poliziyu poekhat’ I im kak svidetel vsjo rasskazat’?} \\
%\gll oni plsvonili polizie \textbf{...} kak ty dumaesh: mne tozhe nado v poliziyu poekhat’ I im kak svidetel vsjo rasskazat’? \\ they-\textsc{nom.pl} called-\textsc{pst.pl} police-\textsc{dat.sg} ... how you-\textsc{nom.sg} think-\textsc{prs.2.sg}: I-\textsc{dat.sg} also necessary in police-\textsc{acc.sg} go-\textsc{inf} and they-\textsc{dat.pl} as witness-\textsc{nom.sg} everything-\textsc{acc.sg} tell-\textsc{inf}? \\
\glt `they called the police ... Do you think I should go to the police and tell them everything as a witness?' \hfill[DEbi03FR\_iwR]
\z

\ea Between discourse units: preceding and following real-world referring DUs \label{ex:labrenzetal:6}\\ {this 1 guy was playing with a ball \textbf{...} it slipped from his hand} \\
\hfill[USbi57FG\_iwE]
\z

\ea Internal with preceding interjection \label{ex:labrenzetal:7}\\
{Whoa\textbf{...}just witnessed a chain reaction accident.} \hfill[USmo32ME\_iwE]
\z

\ea Between discourse units with preceding subjective DU \\ 
{bonitaaaa, tam auffahrunfall, aber sooo unnötig ... siehste man kann nicht kontrollieren habibi}\label{ex:labrenzetal:8} \\
\glt ‘bonitaaaa, tam rear-end collision, but sooo unnecessary ... you see you cannot control habibi’ \hfill[DEbi03FT\_iwD]
\z

\ea Message-final with preceding subjective DU \label{ex:labrenzetal:9}\\
{and crashed into each other, wierd day today ...}\hfill[USbi55MR\_iwE]
\z

\ea Message-final with preceding real-world referring DU \\ 
{ich bleib mal hier, falls ich aussagen muss ...}\label{ex:labrenzetal:10}\\
\glt ‘I’m staying here in case I have to give a witness statement…’\\
\hfill[DEbi37FR\_iwD]
\z


\noindent The following section includes an analysis of each of the DMs that we identified and analysed in the \textsc{rueg} corpus. 

\section{Findings}
\label{sec:labrenzetal:Findings}
\subsection{German \emph{so}}
\label{sec:sub:labrenzetal:German_so}

Although German \textit{so} 
%[\textipa{zoː}]
and English \textit{so} 
%[\textipa{so\textupsilon}]\footnote{Pronunciation in American English; in British English [\textipa{s\textschwa\textupsilon}]}
differ in their phonological representation, they are orthographically identical. As described in \sectref{sec:labrenzetal:Verbal_and_graphic_DMs}, the only functional overlap regarding DM uses is the initializing function. We start from the following hypothesis for German as a majority language\footnote{MSs of German integrated in this analysis are either monolingually or bilingually raised with Turkish as their (other) family language.} compared to German as a heritage language in a majority-English setting:
Due to the orthographic similarity of German \textit{so} and English \textit{so} and their divergent functional spectrum, we expect an impact of majority English on heritage German, leading to functional differences in heritage German compared to majority German. 
A close look at the positions and specific functions of \textit{so} reveals interesting trends. As shown in \tabref{tab:labrenzetal:3}, most of the uses of \textit{so} in all groups are in the integrated position. If we focus on peripheral uses, i.e. those cases that correspond to our DM definition, we see that while HSs prefer left peripheral positions, majority speakers (MSs) show a more frequent use in the right peripheral position. This positional difference between MSs and HSs is related to differences in functional use. Note that we included mono- and bilingual MSs in the analysis in order to disentangle bilingualism from language status effects. In the following, we first present specific functions and then show the proportional use of these functions per speaker group and how this is related to position.

\begin{table}
\caption{Raw numbers ($n$) and percentages of syntactically integrated, right and left peripheral positions of `so' across mono- and bilingual MSs of German, and HSs of German}
\label{tab:labrenzetal:3}
 \begin{tabularx}{.8\textwidth}{l YYYYYY}
  \lsptoprule
            & \multicolumn{2}{r}{integrated} &  \multicolumn{2}{l}{right peripheral}  & \multicolumn{2}{c}{left peripheral}\\
            & $n$  & \% & $n$  & \% &  $n$  & \% \\
  \midrule
  MS (mo)    &  306  &  88.0  &  41  &  11.8  &  1  &  0.3\\
  MS (bi)    &  282  &  87.9  &  34  &  10.6  &  5  &  1.68\\
  HS    &  55  &  70.5  &  1  &  1.3  &  22  &  28.2\\
  \lspbottomrule
 \end{tabularx}
\end{table}

In HSs of German, we observe the use of \textit{so} with a novel function in left peripheral position, namely as a connector indicating a consequence \REF{ex:labrenzetal:11a}. This is one of the functions of English \textit{so} \REF{ex:labrenzetal:11b}. In German it is an innovative use of \textit{so}, since this function would canonically be covered by German \textit{also}.

\pagebreak
\ea consecutive connector \label{ex:labrenzetal:11}
    \ea\label{ex:labrenzetal:11a}
\gll aber die Autos haben gekommen \textbf{so} sie $[$die Leute$]$ haben gestoppt\\
but the cars have.\textsc{aux.3pl} \textsc{ptcp}-come-\textsc{ptcp} so they $[$the people$]$ have.\textsc{aux.3pl} \textsc{ptcp}-stop-\textsc{ptcp}\\
\glt ‘but the cars came, \textit{so} they [the people] stopped’ \hfill[USbi57FD\_fsD]
  \ex \textit{it was cause this guy like lost control of his soccer ball and I know you love soccer \textbf{so} thought you’d be interested}\label{ex:labrenzetal:11b}\hfill[USmo11FE\_isE]
%\gll
%\glt
  \z
\z

\noindent We also observe a more frequent use of the ‘initialising narrative’ function in HSs compared to MSs (\citealt[cf.][]{hennig_grammatik_2006} on that function). This may also reflect an influence from English \textit{so}, for which the initialisation of a new narrative (sequence) has been described as a function \citep[cf.][]{bolden_implementing_2009}. We also find evidence for this function in our data for English \textit{so} \REF{ex:labrenzetal:12b}.

\ea initialising message/narrative\footnote{F16 in \REF{ex:labrenzetal:12a} refers to the case number of the accident that was provided by the elicitors.} \label{ex:labrenzetal:12}
  \ea\label{ex:labrenzetal:12a}
\gll \textbf{so} F16 da war ein Mann\\ so F16 there was a man\\
\glt ‘\textit{so} F16, there was a man’ \hfill[USbi75MD\_fsD]
  \ex {so i: just witnessed an accident it seemed like a minor accident there was a woman on the right side of the street opening her trunk} \label{ex:labrenzetal:12b}\hfill[USmo05FE\_isE]
%\gll
%\glt
  \z
\z

\noindent Additionally, as shown in \tabref{tab:labrenzetal:3}, there is only one case of right peripheral use of \textit{so} in HSs, while for MSs this is the most frequent peripheral position. This is related to functions. The one use in the right periphery is as a general extender or hedge\footnote{Note that English \textit{so} can be used as general extender in the construction ‘and so on (and so forth)’ in parallel to German ‘und so weiter und so fort’. In (colloquial) German this is often reduced to ‘und so’.} as in \REF{ex:labrenzetal:13}. We found no evidence though for the use as an utterance final assertion marker as in \REF{ex:labrenzetal:14} which we found to be very common in majority language use (see \tabref{tab:labrenzetal:4}). This might indicate that the non-overlap with the functional spectrum of English \textit{so} might have an additional impact on the functional use of German \textit{so} in HSs. However, the fact that these functions are (almost) absent in our data does not necessarily mean that HSs do not actually use such functions. 

\ea general extender/hedge\label{ex:labrenzetal:13}\\
\gll ich soll auch nochmal anrufen und \textbf{so}\\ I should also again call and so\\
\glt ‘I should also call again and \textit{so}’ \hfill[DEmo17MD\_isD]
\z

\ea (utterance final) assertive function\label{ex:labrenzetal:14}\\
\gll dieser Ball ist dann einfach auf die Straße gerollt \textbf{so}\\ this ball is then simply onto the street rolled \textsc{dm}\\
\glt ‘this ball then simply rolled onto the street \textit{so}’ \hfill[DEmo76FD\_isD]
\z

\begin{table}
\caption{Percentages of functions of `so' at the periphery across speaker groups. Columns should sum to 100\%, missing values are functions that occur less than 4 times}
\label{tab:labrenzetal:4}
 \begin{tabularx}{.8\textwidth}{X rrrr}
  \lsptoprule
  Function  & Peripheral & MS (mo)  & MS (bi) & HS\\
  & position & & & \\
  \midrule
  hedge  &   right  &    59.5  &    58.9      & -\\
  assertion  &   right &   38.1  &    28.2     & -\\
  consequence  &   left &   -  &    -     & 39.1\\
  initialising  &   left &   -  &    -     & 30.4\\
  \lspbottomrule
 \end{tabularx}
\end{table}

\noindent In summary, our analysis of German HSs’ use of German \textit{so} revealed functional extensions (indicating a consequence), and restrictions (hedge, assertion) when no overlapping functions were present. Furthermore, when the functional spectrum overlapped (initialising function), German HSs displayed a higher frequency of use compared to German MSs. For this item, we observe a unidirectional cross-linguistic influence from the majority into the heritage language which supports our initial hypothesis.

\subsection{Turkish \emph{işte}}
\label{sec:sub:labrenzetal:Turkish_iste}
Turkish \textit{işte} overlaps with German \textit{so}  in its utterance final assertive function. Thus, our hypothesis is the following:
Due to functional similarities of German \textit{so}  and Turkish \textit{işte}, we expect a more frequent use of overlapping functions in heritage Turkish in Germany.
In our data, addressee-oriented functions such as marking shared knowledge between discourse participants and claiming turn in a conversation are very rare, presumably because the elicited productions are monologues that do not presuppose an answer or a reaction from the interlocutor. In \tabref{tab:labrenzetal:5}, we present the percentage of functions of \textit{işte} across speaker groups. Only functions that occurred more than 2 times are considered.

\begin{table}
\caption{Percentages of functions of \textit{işte} across speaker groups; only functions that occurred more than 2 times are considered}
\label{tab:labrenzetal:5}
\centering
\begin{tabular}{ lrrr }
  \lsptoprule
  & \multicolumn{2}{c}{HS} & MS \\
  \cmidrule(lr){2-3} \cmidrule(lr){4-4}
  Function & Germany & US & Turkey \\
  \midrule
  assertion & 46.8 & 37.0 & 41.9 \\
  thematic switch & 13.9 & 21.9 & 12.9 \\
  connector & 11.4 & 13.7 & 9.7 \\
  hesitation & 12.7 & 4.1 & 8.1 \\
  initializing narrative & 5.1 & 2.7 & 12.9 \\
  elaboration & 6.3 & 15.1 & 12.9 \\
  other & 3.2 & 5.5 & 1.6 \\
  \lspbottomrule
\end{tabular}
\end{table}

\noindent Among all three speaker groups in our data, \textit{işte} most frequently functions as an utterance-final assertion. Notably, Turkish HSs in Germany exhibit a slightly higher use of \textit{işte} in this function \REF{ex:labrenzetal:15a} compared to Turkish MSs and HSs in the US. In fact, this is the function that is shared by Turkish \textit{işte} \REF{ex:labrenzetal:15a}, \REF{ex:labrenzetal:15b} and German \textit{so} \REF{ex:labrenzetal:14}. Hence, the higher frequencies of \textit{işte} in this function might be driven by the functional overlap with German \textit{so}.

\ea assertive function\label{ex:labrenzetal:15}\\
  \ea \label{ex:labrenzetal:15a}
\gll arka-daki araba yetiş-e-me-di ve (-) kaza ol-du \textbf{işte}\\
back-\textsc{attr} car keep-up-\textsc{neg-pst.3sg} and (-) accident happen-\textsc{pst.3sg} \textsc{dm}\\
\glt ‘The car at the back wasn’t quick enough and the accident happened \textit{işte}’ 
\hfill[DEbi59MT\_isT]
  \ex \label{ex:labrenzetal:15b}
\gll karşı-dan (-) {bebek araba-lı} iki kişi geç-iyo-du (-) yan-ın-da {bi de} çocuğ-u var-dı \textbf{işte} \\ opposite-\textsc{abl} {} stroller-\textsc{attr} two person pass-\textsc{prog}-\textsc{pst.3sg} {} side-\textsc{poss}-\textsc{loc} also child-\textsc{poss} be-\textsc{pst.3sg} \textsc{dm}\\
\glt ‘From the opposite side two persons with a baby were passing. They also had a child \textit{işte}.’ \hfill [TUmo73MT\_fsT]
  \z
\z

\noindent In addition to the proportionally more frequent use of the assertive function in HSs in Germany, we find novel functions of \textit{işte} as a circum-connector \REF{ex:labrenzetal:16} and what could be called an emergent subordinate connector \REF{ex:labrenzetal:17} in heritage and majority Turkish; although the frequency of such examples is slightly higher in HSs. Drawing a parallel to the developments in Balkan Turkish, \citet{keskin_aspects_nodate} discuss similar examples where DMs are used in the function of connectors and argue that such innovation points to an ongoing shift of Turkish to Indo-European subordination patterns (see \cite{chapters/09}). 

The development of \textit{işte} from a marker with predominant discourse-pragmatic functions to novel connecting functions corresponds to an opposite development to typical pragmaticalisation processes characterised by semantic bleaching (cf. \sectref{sec:labrenzetal:Verbal_and_graphic_DMs}). Compared to DMs as defined in this article, connectors contribute more meaning to the logical coherence of utterances and are, therefore, at least less prototypical for the DM category than those that operate purely on a metacommunicative level indicating, for instance, a new narrative sequence.

\ea circum-connector\label{ex:labrenzetal:16}\\
\gll \textbf{işte} (bi) ilk araba da köpeğ-e es-me-sin diye bi fi/firen-e bas-tı \\ \textsc{dm} \textsc{art} first car also dog-\textsc{dat} hit-\textsc{neg}-\textsc{opt.3sg} \textsc{con} \textsc{art} break push-\textsc{pst.3sg}\\
\glt ‘\textit{işte} the first car not to crush the dog diye braked’ \hfill[DEbi25MT\_isT]
\z

\noindent Example \REF{ex:labrenzetal:16} demonstrates the use of \textit{işte} in combination with the subordinator \textit{diye} in the function of a circum-connector which links the finite adverbial clause introduced with \textit{işte} to the main clause. Similar use is also observed in the combination of the DM \textit{yani} with \textit{diye} (see \sectref{sec:sub:labrenzetal:Turkish_yani}).

\ea relative-connector\label{ex:labrenzetal:17}\\
\gll sol taraf-ta da bi tane kadın \textbf{işte} alışveriş falan yap-mış \\ left	side-\textsc{loc} also \textsc{art} piece woman \textsc{dm} groceries stuff do-\textsc{pst.3sg}\\
\glt ‘On the left, there was a woman \textit{işte} did groceries and stuff’ \\
\hfill[DEbi09FT\_isT]
\z

\noindent Example \REF{ex:labrenzetal:17} can be interpreted as a main clause which is followed by a relative clause with the relative connector \textit{işte}. Hence, the sentence follows a subordination pattern similar to the Indo-European one: The relative clause is postpositive finite and is linked to the main clause by means of a free subordinating element. Interestingly, the novel function of \textit{işte} is found also in majority Turkish, which suggests that Turkish in Turkey may also be undergoing syntactic change. At the same time, a higher number of \textit{işte} used in the function of a connector in heritage Turkish might be interpreted as a convergence phenomenon, where Turkish develops patterns of clause combining which follow the model of German and English, respectively (\citealt{ozsoy_shifting_2022}, see also \cite{chapters/08}). 

HSs in the US show the lowest frequency in the use of \textit{işte} as an assertion marker – a function that is typically used in informal situations and that is not in the functional spectrum of English \textit{so}. Instead, this speaker group has higher frequencies of \textit{işte} as a connector and as an indication of thematic switch – functions that are in the functional spectrum of English \textit{so}. In addition, we also find a cross-linguistic influence of the majority on the heritage language, but only in terms of the frequency of use of a specific function. The extension of the functional spectrum in terms of specific connective functions in HSs may be more indicative of a language-internal development, as it is also found in Turkish in Turkey, but this process may be accelerated by language contact. Thus, there is only partial support for our original hypothesis.

\subsection{Turkish \emph{yani}}
\label{sec:sub:labrenzetal:Turkish_yani}
Turkish \textit{yani} functionally overlaps with German \textit{also} in its repair marking functions, including elaborations. It also overlaps with English \textit{well} for marking hesitations and \textit{so}, especially for indicating the end of a narrative.

Due to functional similarities of German \textit{also} and English \textit{well/so} with Turkish \textit{yani}, we expect a more frequent use of overlapping functions in heritage Turkish. In \tabref{tab:labrenzetal:6}, we present percentages for use of the different functions of \textit{yani} across speaker groups. Only functions that occurred more than two times are considered.

\begin{table}
\caption{Percentages of use of the different functions of \textit{yani} across speaker groups; only functions that occurred more than 2 times are considered}
\label{tab:labrenzetal:6}
 \centering
 \begin{tabular}{ l rrr }
  \lsptoprule
            & \multicolumn{2}{c}{HS} & MS  \\\cmidrule(lr){2-3} \cmidrule(lr){4-4}
  Function  &   Germany  &    US  &    Turkey  \\
  \midrule
  elaboration  &   26.5  &    18.7  &    19.2  \\
  evaluation  &   8.8 &   4.4  &    32.7 \\
  correction  &   23.8 &   10.9  &    5.8   \\
  assertion  &   8.2 &   6.6  &    13.5   \\
  connector  &   10.2 &   5.5  &    5.8    \\
  hesitation  &   7.5 &   41.7  &    5.8    \\
  encapsulation  &   4.1 &   3.3  &    5.8    \\
  ending narration  &   3.4 &   3.3  &    11.5    \\
  thematic switch  &   4.8 &   5.5  &    -    \\
  \lspbottomrule
 \end{tabular}
\end{table}

\noindent HSs in the US show a more frequent use of \textit{yani} as a hesitation marker \REF{ex:labrenzetal:18} – a common function of English \textit{well} \citep[cf.][]{aijmer_well_2011} compared to the other two groups. HSs in Germany use \textit{yani} predominantly for indicating elaborations \REF{ex:labrenzetal:19} and corrections – two common functions of German \textit{also}. The functional overlap might favor a more frequent use of such functions in HSs compared to Turkish MSs and to HSs with English or German as a majority language respectively. 

\ea hesitations marker\label{ex:labrenzetal:18}\\
\gll çünkü \textbf{yani} o {a hab/aa} (-) \textbf{yani} (-) aile-ye doğru git-me-ye çalış-tı\\ because \textsc{dm} he hmm {} \textsc{dm} {} family-\textsc{dat} towards	go-\textsc{nmz}-\textsc{dat} try-\textsc{pst.3sg}\\
\glt `Because \textit{yani} hmm he \textit{yani} tried to go towards the family' \\ \hfill[USbi16MT\_fsT]
\ex elaboration\label{ex:labrenzetal:19}\\
\gll {ay/ ondan: ee dolayı} işte karşı taraf-a yuvarlan-ıyo	köppek de {on-a sa/} \textbf{yani}	top-a saldır-ıyo\\ {because of that}	\textsc{dm} opposite side-\textsc{dat} roll-\textsc{prs.3sg} dog also it-\textsc{dat} \textsc{dm} ball-\textsc{dat} attack-\textsc{prs.3sg}\\
\glt ‘That's why it rolled to the opposite side. The dog attacked it \textit{yani} the ball.’ \hfill[DEbi16FT\_isT]
\z

\noindent Similar to what was observed with regard to \textit{işte}, we find a novel function of \textit{yani} in the function of what might be called an emergent subordinate connector. Like with \textit{işte}, this may combine with the subordinator \textit{diye}, as in \REF{ex:labrenzetal:16}, and again this is found in heritage as well as majority Turkish speakers, but the number of such occurrences is higher in HSs.

\ea circum-connector\label{ex:labrenzetal:20}\\
\gll köpeğ-i-ni	hemen sıkı tut-tu \textbf{yani} {bi şey} ol-ma-sın \textbf{diye}\\ dog-\textsc{poss-acc} immediately tight hold-\textsc{pst.3sg} \textsc{dm} something happen-\textsc{neg-opt.3sg} \textsc{con}\\
\glt ‘S/he immediately held her/his dog tight so that nothing would happen (to it).’ \hfill[USbi16MT\_isT]
\ex relative connector\label{ex:labrenzetal:21}\\
\gll şu/ orda bi kadın vardı \textbf{yani} araba-sı-na (-) alışveriş torba-lar-ın-ı falan koyuyodu\\ th/ there one woman be-\textsc{pst.3sg} \textsc{dm} car-\textsc{poss-dat} (-) grocery bag\textsc{-pl-poss-acc} stuff put\textsc{-rog-pst.3sg}\\
\glt ‘There was a woman \textit{yani} (who) was putting grocery bags and stuff into her car.’ \hfill[DEbi87FT\_isT]
\z

\noindent In line with our hypothesis, the example of \textit{yani} shows the influence of the respective majority language: HSs show a more frequent use of exactly those functions that overlap with the functional spectrum of the respective majority language counterpart. In addition, similar to \textit{işte}, there may be language internal developments at play, which may be even more dynamic in HSs.

\subsection{Greek \emph{lipon}}
\label{sec:sub:labrenzetal:Greek_lipon}
Greek \textit{lipon} overlaps mainly with German \textit{also} in the initialising-narrative function and with English \textit{so} in indication of thematic switch. Our hypothesis is the following:  
Due to the overlapping functional spectrum of English \textit{so}, and German \textit{also} with \textit{lipon}, we expect a more frequent use of overlapping functions in heritage Greek.

Examining the use of \textit{lipon} across majority and heritage Greek, we see that although all groups use \textit{lipon} to initialise narrations, other uses of \textit{lipon}, such as the indication of thematic switch, are only found in majority Greek speakers. Data are shown in \tabref{tab:labrenzetal:7}.

Although the use of \textit{lipon} in the two HS groups is not as high as the use of \textit{lipon} in majority Greek speakers, all groups use \textit{lipon} to initialise their narrative. This means that they either use it in the very beginning of the text, as in Example \REF{ex:labrenzetal:22}, or they use it after they provide an opening (such as greeting) and then start their main narrations. In addition, only the majority Greek speakers also use \textit{lipon} to introduce thematic switch, such as in Example \REF{ex:labrenzetal:23}, in which the speaker uses \textit{lipon} to switch back to the main narration after having inserted a personal comment. 

\ea initialise narrative\label{ex:labrenzetal:22}\\
\gll \textbf{lipon} otan icha scholasi ap to scholio ke pijena spiti ksafnika icha dhi pos mia mpala \\ \textsc{dm} when have.\textsc{pst.1s} finish.\textsc{pst.1s} from the school and go.\textsc{pst.1s} home suddenly have.\textsc{pst.1s} see.\textsc{pst.ptcp} that a ball\\
\glt ‘\textit{lipon} when I finished school and was going home suddenly I saw that a ball.' \hfill[DEbi57MG\_fsG]
\z

\ea thematic switch\label{ex:labrenzetal:23}\\
\gll vevea tora entaksi ine dheka metra makria opote mu fanike paralogho e: i sinechia \textbf{lipon} ine oti apla chtipai to piso amaksi to mprosta amaksi \\ certainly now alright be.\textsc{prs.3s} ten meters away so me.\textsc{gen.1s} seem.\textsc{pst.13s} illogical uh the continuity \textsc{dm} be.\textsc{prs.3s} that simply hit.\textsc{prs.3s} the back car the front car\\
\glt ‘certainly now okay it’s ten meters away so it seemed crazy to me uh next \textit{lipon} is that the car at the back simply hits (-) the car in the front.’ \\
\hfill[GRmo10MG\_fsG]
\z

\begin{table}
\caption{Percentages of the most frequent functions of ‘lipon’ across countries\slash speaker groups}
\label{tab:labrenzetal:7}
 \centering
 \begin{tabular}{ l rrr }
  \lsptoprule
    & MS & \multicolumn{2}{c}{HS}  \\\cmidrule(lr){2-2} \cmidrule(lr){3-4}
  Function  &   Greece  &    Germany  &    US  \\
  \midrule
  initialise narrative  &   75.8  &    75.0  &    100  \\
  thematic switch  &   16.2 &   16.7  &   - \\
  \lspbottomrule
 \end{tabular}
\end{table}

\noindent On the basis of partial functional overlap between \textit{lipon} and the English \textit{so} and \textit{lipon} and the German \textit{also}, we expected to find extended functions of \textit{lipon} in the data of HSs. However, our analysis did not confirm this hypothesis. HSs used \textit{lipon} almost exclusively to initialise narrations. This restricted functional spectrum in the HSs’ use of \textit{lipon} might indicate that the initialising function is the most transparent function of this marker and that there is probably no real perceived functional overlap from the perspective of the HSs.

\subsection{Greek \emph{etsi}}
\label{sec:sub:labrenzetal:Greek_etsi}
\largerpage
Greek \textit{etsi} has the emerging function of indicating a consequence. In this function it overlaps with German \textit{also} and English \textit{so}. We hypothesise the following:
Due to functional similarities of English \textit{so}, and German \textit{also} with \textit{etsi}, we expect a more frequent use of overlapping functions in heritage Greek.
Across all speaker groups, we find functions of \textit{etsi} consistent with the literature. Summary data are given in \tabref{tab:labrenzetal:8}. Similar to German \textit{also}, \textit{etsi} indicates a consequence in Example \REF{ex:labrenzetal:25}. In Example \REF{ex:labrenzetal:24}, it is similar to the adverbial use of German \textit{so} that indicates a way or manner. In majority Greek, we exclusively find instances of \textit{etsi} in which \textit{etsi} functions as adverb, and hence not as a DM \REF{ex:labrenzetal:24}, and as a consecutive connector \REF{ex:labrenzetal:25}. \citet{tsampoukas_grammaticalization_2015} identifies the consecutive connector function as a DM function. 

\ea adverb \label{ex:labrenzetal:24} \\
\gll na sas apodhiksi oti ontos \textbf{etsi} sinevin/sinevisan ta pragmata \\ to you.\textsc{pl} prove.\textsc{sbj.3sg} that indeed like happen/happen.\textsc{pst.3pl} the things\\
\glt ‘to prove to you that things happened \textit{etsi/like} this’ \hfill[GRmo08FG\_fsG]
\ex consequence \label{ex:labrenzetal:25} \\
\gll to amaksi pu ine piso pu itan piso ap afto t= amaksi dhe prolave na patisi freno ki \textbf{etsi} trakaran  \\ the car that be.\textsc{prs.3sg} behind that be.\textsc{pst.3sg} behind from this the car not get.\textsc{pst.3sg} to push.\textsc{sbj.3sg} brake and so bump.\textsc{pst.3pl}\\
\glt ‘the car that is behind that was behind that car did not get to brake and \textit{etsi/so} they bumped’  \hfill[GRmo76MG\_isG]
\z

\noindent In heritage Greek in the US and Germany, however, we observe functional extension of \textit{etsi}. Although heritage Greek speakers in Germany use \textit{etsi} almost as frequently as majority Greek speakers, heritage Greek speakers in Germany extend the use of \textit{etsi} to indicate instances of corrections or restarts \REF{ex:labrenzetal:26} and elaborations/specification \REF{ex:labrenzetal:27}. This corresponds to common functions of German \textit{also}.
Heritage Greek speakers in the US use relatively fewer instances of \textit{etsi} compared to the two other speaker groups. Despite the low number of \textit{etsi} occurrences in the spoken narrations of heritage Greek speakers in the US, the use of \textit{etsi} is again extended. Thus, \textit{etsi} is used for indicating the end of a narrative, as in Example \REF{ex:labrenzetal:28} , which is a common function of majority English \textit{so}. Overall, the functional extension of \textit{etsi} seems to be a convergence with the respective majority languages.

\ea indicate correction\slash restart \label{ex:labrenzetal:26} \\
\gll ey ghia su ti kanis dhe pistevis ti idha simera itane \textbf{etsi} perpatusane dhio \\ ey hello you how do.\textsc{prs.2sg} not believe.\textsc{prs.2sg} what see.\textsc{pst.1sg} today be.\textsc{prt.3sg} \textsc{dm} walk.\textsc{pst.3sg} two\\
\glt ‘hey hello how are you you won’t believe what I saw today two people were \textit{etsi} were walking’  \hfill[DEbi07FG\_isG]
\z

\ea specification \label{ex:labrenzetal:27} \\
\gll ghiati etrekse ke o antras meta pros ti meria tu skiliu ki \textbf{etsi} itane oli mazi kapos ston aftokinitodromo \\ because run.\textsc{pst.3sg} and the man then towards the side the dog and \textsc{dm} be.\textsc{pst.3pl} all together somehow on.the motorway\\
\glt ‘because the man also ran towards the side of the dog and \textit{etsi} they were all together somehow on the driveway’ \hfill[DEbi03FG\_fsG]
\z

\ea end narrative \label{ex:labrenzetal:28} \\
\gll ke stamatise o enas o mprostinos \textbf{etsi} ksafnika ke o allos {ap=p/ apo} piso ton trakarise ke \textbf{etsi} \\ and stop.\textsc{pst.3sg} the one the front \textsc{dm} suddenly and the other {fr=f/from} behind him bump.\textsc{pst.3sg} and \textsc{dm}\\
\glt ‘and the one in the front stopped suddenly and the other one at the back bumped on him and \textit{etsi}’ \hfill[GRmo76MG\_isG]
\z
The findings on \textit{etsi} were different than what we expected: We did not find a more frequent use of overlapping functions, but HSs of Greek in Germany and the US extend the functional spectrum of \textit{etsi} according to common functions of the respective majority language counterpart.

\begin{table}
\caption{Percentages of the most frequent functions of \textit{etsi} across countries\slash speaker groups}
\label{tab:labrenzetal:8}
 \centering
 \begin{tabularx}{.6\textwidth}{X  c  c  c }
  \lsptoprule
            & MS & \multicolumn{2}{c}{HS}  \\
  \cmidrule(r){2-2} \cmidrule(l){3-4}
  Function  &   Greece  &    Germany  &    US  \\
  \midrule
  adverb       &   59.2 &   36.4  &    79.2  \\
  consequence  &   40.8 &   47.7  &   4.2 \\
  \lspbottomrule
 \end{tabularx}
\end{table}

\subsection{Three-dot sign}
\label{sec:sub:labrenzetal:Three_dot_sign}
\largerpage
In analyzing data containing the three-dot sign, we used the indicators of position and context to assign discourse functions. First, we report findings for individual languages. In \tabref{tab:labrenzetal:9} we present raw numbers and percentages for functions that we assigned based on the position. The main finding regarding language specifics is that textual functions are predominant across languages but with a larger variation in German, Russian and Turkish compared to English and Greek \citep[cf.][260--61]{labrenz_three-dot_2022}.

\begin{table}
\caption{Raw numbers ($n$) and percentage use (\%) of functions per language relative to all uses of the three-dot sign in that language}
\label{tab:labrenzetal:9}
 \centering
 \begin{tabular}{l  c  c  c  c  c  c}
  \lsptoprule
            & \multicolumn{2}{c}{Textual} & \multicolumn{2}{c}{Subjective} & \multicolumn{2}{c}{Intersubjective}  \\
  \cmidrule(lr){2-3} \cmidrule(lr){4-5}\cmidrule(lr){6-7}
            &  $n$  &  \%  &  $n$  &  \%  &  $n$ &  \%   \\
  \midrule
  English  &  22 &  88.0  &  0  &  0.0  &  3  &  12.0 \\
  German  &  52 &  68.4  &  4  &  5.3  &  20 &  26.3 \\
  Greek  &  17 &  81.0  &  0  &  0.0  &  4  &  19.0 \\
  Russian  &  15 &  62.5  &  1  &  4.2  &  8  &  33.3 \\
  English  &  12 &  63.2  &  1  &  5.3  &  6  &  31.6 \\
  \lspbottomrule
 \end{tabular}
\end{table}

The second indicator \citep[cf.][261--62]{labrenz_three-dot_2022} was the context which is especially relevant in message-final position and in the position between two discourse units. In these positions the three-dot sign has two sources of interpretation, and can therefore have additional subjective or intersubjective functions depending on the preceding discourse unit. Taken together, the findings from Tables~\ref{tab:labrenzetal:9} and~\ref{tab:labrenzetal:10} show a tendency for German, Russian, and Turkish to use the three-dot sign relatively more frequently compared to English and Greek in intersubjective and subjective functions and polyfunctionally.

\begin{table}
\caption{Raw numbers ($n$) and percentage use (\%) of monofunctional (either textual, subjective, or intersubjective) and polyfunctional (additional (inter)subjective functions) use per language relative to all uses of the three-dot sign in that language}
\label{tab:labrenzetal:10}
 \centering
 \begin{tabular}{l  c  c  c  c}
  \lsptoprule
            & \multicolumn{2}{c}{Monofunctional} & \multicolumn{2}{c}{Polyfunctional} \\
   \cmidrule(lr){2-3} \cmidrule(lr){4-5}
            &  $n$  &  \%  &  $n$ &  \%  \\
  \midrule
  English  &   17  &    68.0  &    8  & 32.0 \\
  German  &   39  &    51.3  &    37  & 48.7 \\
  Greek  &   17  &    81.0  &    4  & 19.1 \\
  Russian  &   12  &    50  &    12  & 50 \\
  \lspbottomrule
 \end{tabular}
\end{table}

Regarding potential dynamics in bilinguals \citep[cf.][264--66]{labrenz_three-dot_2022}, we found an overall tendency of bilinguals to adapt the three-dot use in their heritage language to that in the respective majority language in two areas: functional variation and frequency of use. With respect to frequencies, the following pattern emerged for HSs of Turkish and Russian: Just as MSs of German use the three-dot sign more frequently than MSs of English, so HSs of Turkish and Russian with German as their majority language use it more frequently than HSs of Turkish and Russian with English as their majority language (see \tabref{tab:labrenzetal:11}). The numbers in brackets are occurrences of the three-dot sign per 100 communicative units (roughly all independent sentences) in the respective group\slash language. MSs of German and English includes mono- as well as bilingually raised speakers).

\begin{table}
\caption{First pattern for HSs of Turkish and Russian}
\label{tab:labrenzetal:11}
 \begin{tabularx}{.75\textwidth}{X  r  r  r  r }
  \lsptoprule
    \multicolumn{2}{c}{Germany (DE)} &  & \multicolumn{2}{c}{US}  \\
  \midrule
  MSs-German  &  (2.71)  &  >  & MSs-English  &  (0.96)  \\
  %\midrule
  HSs-Turkish/DE  &  (1.89)  &  >  &  HSs-Turkish/US  &  (0.73)  \\
  %\midrule
  HSs-Russian/DE  &  (1.5)  &  >  &  HSs-Russian/US  &  (0.52) \\
  \lspbottomrule
 \end{tabularx}
\end{table}

\noindent Another pattern that emerged is the tendency within bilingual speakers with Greek and Russian as heritage languages to use the three-dot sign relatively more frequently in their majority language (German or English) than in their heritage language (see \tabref{tab:labrenzetal:12}).

\vfill
\begin{table}[H]
\caption{Second pattern for HSs of Greek and Russian}
\label{tab:labrenzetal:12}
 \begin{tabularx}{.82\textwidth}{Xrrrr }
  \lsptoprule
    \multicolumn{2}{c}{HS of Greek} &  & \multicolumn{2}{c}{}  \\
  \midrule
  majority-German   &  (1.1)  &  >  & heritage-Greek/DE  &  (0.73)  \\
  %\midrule
  majority-English   &  (1.52)  &  >  &  heritage-Greek/US   &  (0.79)  \\
   \midrule
    \multicolumn{2}{c}{HS of Russian} &  & \multicolumn{2}{c}{} \\
    \midrule
  majority-German   &  (3.0)  &  >  & heritage-Russian/DE   &  (1.5)  \\
  majority-English   &  (1.16)  &  >  &  heritage-Russian/US  &  (0.52)  \\
  \lspbottomrule
 \end{tabularx}
\end{table}
\vfill
\pagebreak

To sum up, we found similarities as well as slight differences in the use across countries and languages, as well as across bilingual speakers’ two languages. Cross-linguistically, the three-dot sign is used for discourse organisation and segmentation in digital informal writing (see also \citealt{busch_digitale_2021} on German, and \citealt{androutsopoulos_auslassungspunkte_2020} on Greek). In terms of language contact, our data indicate that patterns of use in the majority language influence the use in the heritage language. Additionally, we found that within bilingual speakers the use of the three-dot sign in the heritage language is less frequent and with a smaller functional spectrum compared to their majority language use, maybe pointing to an insecurity in using such salient markers of informality in their heritage language (see \citealt[259--266]{labrenz_three-dot_2022} for more details on results).

\section{Discussion}
\label{sec:labrenzetal:Discussion}

Taken together, our findings indicate similarities between verbal and graphic DMs concerning the impact of societal language status, that is, the status of a contact language as a majority language or as a minoritised heritage language: We observed an influence of the majority language on the use of DMs in the heritage language in terms of frequencies, functional extension/restriction, and pragmaticalisation processes. 

More specifically, we find functional extensions, restrictions and/or a more frequent use of specific functions in heritage German \textit{so}, heritage Turkish \textit{yani} and \textit{işte}, heritage Greek \textit{lipon} and \textit{etsi}, pointing to convergence with the functional spectrum of a counterpart DM in the majority language (either \textit{so} in English or \textit{so} or \textit{also} in German). This might point to a shared pool of functions in the bilingual mind with stronger activations of additional or overlapping functions of the (societally) dominant language when the heritage language is used than vice versa. Additionally, we find tendencies for language internal developments for Turkish \textit{yani} and \textit{işte}, and Greek \textit{etsi} which seem to be especially dynamic in contact situations. As for graphic DMs, specifically the three-dot sign, we observed language-specific trends, which we found particularly interesting given its translinguistic status. For this graphic DM, we also observed a tendency for the majority language to influence heritage language use, even though graphic markers have no specific language affiliation and exhibit common functions across languages (such as structuring discourse in the case of the three-dot sign).

The observed influence from the majority onto the heritage language might be on one hand favoured by the dominance of the majority language in countries with a strong monolingual bias such as Germany and the US. On the other hand, innovations in heritage languages might be generally favoured because of the lower presence of normative authorities such as school education in that language. Our findings indicate that this might be especially true for less salient phenomena, such as pattern replication. In contrast, more noticeable phenomena, like matter replication or the use of informal register markers such as the three-dot sign, might be used more carefully, at least in the context of a monolingual mode, which is a condition of the elicitation method (see also \cite{chapters/02}). This, in turn, might indicate that in a communicative situation characterised by a monolingual mode, HS might not only be careful in their choice of shared language resources, but also in their use of salient graphic register markers in their heritage language.

\section{Conclusion}
\label{sec:labrenzetal:Conclusion}
This chapter has shed light on the dynamics of DMs in language contact situations, both in formal and in informal-spoken discourse as well as in informal-written texts. By exploring the graphic domain through the inclusion of informal-written messages, we underline the crucial role of informality in DM usage, demonstrating its presence beyond spoken language.

Our key findings reveal similar trends in language contact for both verbal and graphic DMs. First, the societal status of a language has an impact on pattern replication and/or frequency in use of DMs. Secondly, pragmaticalisation processes can be accelerated in specific language contact scenarios. 

These two findings are highly relevant for persistent issues in the field. First, by addressing functional convergence in the discourse domain, we have ventured into a less explored aspect of DMs in language contact, which complements the extensively studied area of borrowings (see also \cite{chapters/02} for borrowed or translanguaged DMs in the \textsc{rueg} data). The fact that HS replicated relatively few actual word forms might be due to the monolingual mode deliberately induced by the \textsc{rueg} setup. This suggests that in situations with monolingual interlocutors, HSs are more likely to resort to pattern replication than to matter replication, since matter replication is more salient and may not be perceived as appropriate by the speakers themselves in such contexts \citep[cf.][]{matras_contact_2010}. This is also supported by the fact that translanguaging of DMs only occurred in the informal data, a communicative situation in which speakers are less exposed to normative pressures than in a more formal situation (see \cite{chapters/02}). Moreover, in our examination involving the majority language and the heritage language of bilingual speakers, we were able to distinguish the effects of bilingualism in general from the societal status of the languages: While we found no effects of bilingualism in the use of the majority language, we found an influence of the majority language on bilingual speakers’ use of the heritage language. 

Second, we identified a link between societal status and the process of pragmaticalisation, observing two tendencies in heritage languages. The first tendency exemplifies a typical path of pragmaticalisation: From an adverbial and/or semantically transparent connective through a process of semantic bleaching into uses with mere discourse functions. This trend holds for both majority and heritage languages. Notably, Greek \textit{etsi} is of particular interest as it appears to be in the early stages of this process, assuming a new function as a semantically transparent consecutive connector. In heritage Greek, this process seems to have progressed further, as more DM functions can already be found here. The second tendency is the transformation of relatively opaque DMs into more transparent connectors indicating an inverted pragmaticalisation process, a phenomenon primarily observed in language contact situations. This is in line with Matras \citep*{matras_grammatical_2007, matras_contact_2010}, who claims that “[…] a model of convergence must also be able to account for potential exceptions to the unidirectionality of grammaticalization” \citep[71]{matras_contact_2010}. In our data, Turkish in contact with German and English showcases this development, potentially influenced by typological differences between Turkic and Indo-European languages: German and English typically rely on finite clauses with pre-posed subordinating connectors, while in Turkish, non-finite subordination is preferred, without the employment of connectors with word status. Finite subordination with pre-posed or circum-connectors is possible; however, in the heritage varieties we note an increase in this pattern of subordination. DMs like \textit{işte} and \textit{yani} belong to the pool of words from which the language recruits new connectors following this dynamic. This leads to an inverted pragmaticalisation path from DMs to connectors, particularly in the contact situations.

Taken together, our findings point to three main factors important for studies on the dynamics of discourse markers: 1) contact scenarios, offering insights into their functional convergence, 2) the influence of societal language status, and 3) the impact of informality beyond spoken language. The study of both verbal and graphical DMs in different language contexts and contact scenarios contributes to a more comprehensive picture of the dynamics and evolution of these linguistic phenomena.

As a last point, we want to discuss the observed unidirectional influence of the majority language on the heritage language use of verbal and graphic DMs. This may be due to the dominant monolingual bias in Germany and the United States. In such societal settings, heritage languages are challenged by strong hegemonial majority languages. In contrast, in multilingually oriented societies (for instance, in many Asian and African countries), crosslinguistic influences may be more fluid, allowing for bidirectional pattern replication. In future research, it would be interesting to build on our results in comparative studies that include heritage languages from such settings.

\section*{Acknowledgements}
We would like to thank our amazing and dedicated student assistants, each of whom was able to bring their own unique strengths to project P6 and P9: Ilyuza Akhmetzioanova, Christian Anders, Yeşim Bayram, Franziska Groth, Oğuzan Kuyrukçu, Alexander Lehman, Ksenia Nowak, Hannah Plückebaum, Simge Türe, Sharon Rauschenbach, Guendalina Reul, Tjona Sommer, Charlott Thomas, and Barbara Zeyer. We further thank Lea Coy for help with pre-publication formatting.

We also thank two external anonymous reviewers and three internal reviewers Oliver Bunk, Johanna Tausch, and Wintai Tsehaye for their helpful and very constructive suggestions. The research was supported through funding by the Deutsche Forschungsgemeinschaft (\textsc{dfg}, German Research Foundation) for the Research Unit \textit{Emerging Grammars in Language Contact Situations}, projects P6 (394838878) and P9 (313607803)

\sloppy
\printbibliography[heading=subbibliography,notkeyword=this]
\cleardoublepage
\end{document}
