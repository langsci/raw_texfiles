\documentclass[output=paper,colorlinks,citecolor=brown]{langscibook}
\ChapterDOI{10.5281/zenodo.15775170}

%This is chapter 
\author{Vasiliki Rizou\orcid{0000-0002-5804-4976}\affiliation{Humboldt-Universität zu Berlin}    and Onur Özsoy\orcid{0000-0003-3617-4697}\affiliation{Leibniz-Centre General Linguistics; Humboldt-Universität zu Berlin}     and Maria Martynova\orcid{0000-0003-4833-9567}\affiliation{Humboldt-Universität zu Berlin}    and Luka Szucsich\orcid{0000-0003-0264-980X}\affiliation{Humboldt-Universität zu Berlin}   and Artemis Alexiadou\orcid{0000-0002-6790-232X}\affiliation{Leibniz-Centre General Linguistics; Humboldt-Universität zu Berlin}    and Natalia Gagarina\orcid{0000-0002-5136-1071}\affiliation{Leibniz-Centre General Linguistics; Humboldt-Universität zu Berlin}
    }
\title{Dynamics of verbal aspect in heritage Greek, Russian and Turkish}
\abstract{Aspect – the grammatical category associated with the internal temporal properties of a situation – is obligatorily marked in the grammar of many languages. Although it is used pervasively, innovative patterns commonly appear in situations of language contact. Thus, aspect provides an ideal window into the dynamic language patterns that occur in heritage grammars. This chapter reviews six studies conducted in the context of the Research Unit \textit{Emerging Grammars} that investigate dynamic patterns in the use of aspect in monolingually-raised speakers of Greek, Russian, and Turkish as well as heritage speakers of each of these languages living in the US and Germany. Each study uses production and/or comprehension tasks to explore one or more potential sources of these dynamic patterns, including cross-linguistic influence, verbal morphology and markedness of the aspectual features, formality (formal vs. informal), mode of expression (spoken vs. written), and participant age. The results show several innovative patterns and some differences in processing in heritage speakers, but largely find that monolingually-raised and heritage speakers produce and process aspect marking in a qualitatively similar way.

\keywords{grammatical aspect, heritage languages, Greek, Russian, Turkish}
}

\IfFileExists{../localcommands.tex}{
   \addbibresource{../localbibliography.bib}
   \usepackage{langsci-optional}
\usepackage{langsci-gb4e}
\usepackage{langsci-lgr}

\usepackage{listings}
\lstset{basicstyle=\ttfamily,tabsize=2,breaklines=true}

%added by author
% \usepackage{tipa}
\usepackage{multirow}
\graphicspath{{figures/}}
\usepackage{langsci-branding}

   
\newcommand{\sent}{\enumsentence}
\newcommand{\sents}{\eenumsentence}
\let\citeasnoun\citet

\renewcommand{\lsCoverTitleFont}[1]{\sffamily\addfontfeatures{Scale=MatchUppercase}\fontsize{44pt}{16mm}\selectfont #1}
  
   %% hyphenation points for line breaks
%% Normally, automatic hyphenation in LaTeX is very good
%% If a word is mis-hyphenated, add it to this file
%%
%% add information to TeX file before \begin{document} with:
%% %% hyphenation points for line breaks
%% Normally, automatic hyphenation in LaTeX is very good
%% If a word is mis-hyphenated, add it to this file
%%
%% add information to TeX file before \begin{document} with:
%% %% hyphenation points for line breaks
%% Normally, automatic hyphenation in LaTeX is very good
%% If a word is mis-hyphenated, add it to this file
%%
%% add information to TeX file before \begin{document} with:
%% \include{localhyphenation}
\hyphenation{
affri-ca-te
affri-ca-tes
an-no-tated
com-ple-ments
com-po-si-tio-na-li-ty
non-com-po-si-tio-na-li-ty
Gon-zá-lez
out-side
Ri-chárd
se-man-tics
STREU-SLE
Tie-de-mann
}
\hyphenation{
affri-ca-te
affri-ca-tes
an-no-tated
com-ple-ments
com-po-si-tio-na-li-ty
non-com-po-si-tio-na-li-ty
Gon-zá-lez
out-side
Ri-chárd
se-man-tics
STREU-SLE
Tie-de-mann
}
\hyphenation{
affri-ca-te
affri-ca-tes
an-no-tated
com-ple-ments
com-po-si-tio-na-li-ty
non-com-po-si-tio-na-li-ty
Gon-zá-lez
out-side
Ri-chárd
se-man-tics
STREU-SLE
Tie-de-mann
}
   \boolfalse{bookcompile}
   \togglepaper[7]%%chapternumber
}{}

\begin{document}
\lehead{Vasiliki Rizou et al.}
\maketitle
\section{Introduction}

In this chapter, we investigate how verbal aspect, either grammatical or lexical, is used in heritage and monolingual varieties of Greek, Russian, and Turkish and how it reflects the dynamic language patterns that occur in heritage grammars. In particular, we review six studies concerning the production and comprehension of verbal aspect that were conducted as part of the Research Unit \textit{Emerging Grammars in Language Contact Situations: A Comparative Approach}. Each study takes a different perspective on the use of aspect in these varieties (heritage and monolingual) and asks whether aspect use is affected by the language itself (Greek, Russian, Turkish), by the contact language (English, German), and/or by the mode (spoken, written) or formality (formal, informal) of the situation. The studies also cover a number of methods, including elicited narrative, sentence completion tasks, spontaneous speech production, and the Visual World paradigm.

Aspect is one of the most widely explored phenomena across different languages and different groups of speakers, including both monolingually-raised and bilingual speakers \parencite{montrul2002incomplete, laleko2010syntax, montrul2011assessing, cuza2013development}. It is described as representing the internal temporal structure of a situation, which is different from the situation's external time, referred to as tense. Researchers typically distinguish between lexical (situation\slash Aktionsart\slash inner) and grammatical (viewpoint\slash outer) aspect \parencite{comrie1976, kiyota2008situation, alexiadou_etal2010}. 
Grammatical aspect is defined as “the different ways of viewing the internal temporal constituency of a situation”, while lexical aspect denotes the inherent semantic category of a verb's eventuality \parencite{comrie1976}. Grammatical aspect is a (morpho-)syntactic category and can be variably marked on verbs, while lexical aspect can be determined by the whole verbal phrase. The studies reviewed in this chapter focus either on grammatical aspect, i.e. \textcite{RizouEtAl}, \textcite{alexiadou2022use}, \textcite{novelforms} and \textcite{Ozsoy_2023}, or on the interaction of both \textcite{rizou2021verbal}, \textcite{gagarina2020first}.

A major distinction for grammatical aspect is made between imperfective and perfective \parencite{comrie1976}. 
Imperfective aspect denotes an event or state as progressive, repeating or continuous. 
Perfective aspect is used to express that an event or an action is completed, or that is ceased without any reference to its tempus. 
In other words, perfective aspect indicates that an event is bounded in time. Although aspect has several further distinctions that differ across languages \parencite{gagarina2000acquisition}, these are not relevant for the current chapter so will not be elaborated here. 

Aspect  is an ideal phenomenon to investigate in language development because it is regarded as highly dynamic across the spectrum from early L1 acquisition through heritage language acquisition and L2 acquisition.
For example, L1 Greek-speaking children produce perfective aspect as early as 1;1,\footnote{We report child age as years;months. Thus, 1;1 means 1 year and 1 month old.} before the imperfective \parencite{stephany1997acquisition, konstantzou2013perfective}. Although imperfective aspect is also used relatively early, the semantics of imperfective aspect are only acquired later by 5;0 to 6;5 \parencite{delidaki2006acquisition, panitsa2010aspects}. L1 Russian-speaking children recognize aspectual meanings at an early stage and begin to use aspectual forms that resemble the target language from the onset of speech. However, they occasionally generate novel expressions or innovations, wherein aspectual forms deviate from the target language norms, that may persist until approximately the age of 6 years \parencite{ceytlin2000jazyk, Gagarina2007, gagarina2011acquisition, kistanova2019acquisition}. Finally, L1 Turkish-speaking children acquire the primary aspect/tense markers quite early, namely the perfective-past marker \textit{-(y)DI} and the progressive-imperfective marker -\textit{(I)yor} \parencite{aksukoc1988}. These distinct markers serve as a significant focal point for understanding the contrast between perfective and imperfective aspect and are productively used by around the age of 3 \parencite{acarlar2008adapting}. 

Innovative aspectual forms are also used regularly in situations of language contact, particularly in languages with concatenative morphology like Greek and Russian, as well as in languages with agglutinating structures like Turkish \parencite{laleko2008compositional, Montrul2015, antonova2016aspect,  gagarina2020first}. This is illustrated in the following example in \REF{Iwillwrite} taken from \textcite[115]{gagarina2020first}, where the perfective aspect form combines with the auxiliary \textit{budu} `I will (be)’ in the future, a form that is considered ungrammatical in monolingual productions:

\ea  \label{Iwillwrite}
\gll budu *napisat'\\
	will-\textsc{1sg} \hphantom{*}write-\textsc{pfv}\\
\glt `(I) will write' 
\z 
\il{Russian}

Given that aspect appears with dynamic and innovative forms across both L1 and L2 acquisition, it seems to be the ideal phenomenon to explore in heritage language use as well. Heritage speakers are typically described as individuals who are bilingual or multilingual, acquire their heritage language(s) during early childhood within their families, and use their heritage language(s) alongside the majority language(s) or official language(s) of the community they live in \parencite{Valdes2005, Rothman2009, benmamoun2013heritage, Guijarro-FuentesSchmitz2015, Montrul2015, Polinsky2015, polinsky2018heritage}.
Generally, heritage speakers are regarded as bilinguals whose family and majority languages are part of a nativeness continuum \parencite{wiese2022heritage}. 
However, the level of proficiency in their languages can vary throughout their lives, as discussed by \textcite{benmamoun2013heritage, polinsky2018heritage}, and \textcite{wiese2020language}. 
In this chapter, we follow this perspective and do not compare heritage speakers' data with monolingually-raised speakers' data to explore accuracy. 
Instead, we consider the heritage and monolingually-raised populations as distinct language varieties, in line with the view expressed in \textcite{rothman2023monolingual}.

Due to the informal nature of the acquisition of heritage languages, speakers are more accustomed to the spoken form of the language, while the written form requires formal instruction.
Research exploring register variation in heritage speakers suggests that when heritage languages are learned within the family context, without receiving any formal instruction in the heritage language, heritage speakers' repertoires exhibit a conversational and informal style, which is often limited to everyday topics, resulting in register narrowing \parencite{dressler1991sociolinguistic, chevalier2004heritage}. Therefore, several of the studies that we review specifically focus on whether aspect is used differently in spoken vs. written language, and in formal vs. informal settings.

The six studies reviewed in this chapter were conducted within the projects P1, P3 and P10 of the Research Unit \textit{Emerging Grammars} (RUEG)\footnote{\url{https://hu.berlin/rueg}. P1 dealt with “Nominal morphosyntax and word order in heritage Greek across majority languages”, P3 with “Nominal morphosyntax and word order in heritage Russian across majority languages”, and P10 with “Dynamics of verbal aspect and (pro)nominal reference in language contact”.}, aiming to address the overarching question: How do speakers of heritage and monolingual varieties of Greek, Russian, and Turkish use aspect? Specifically, we investigate whether heritage speakers' production and comprehension of aspect diverges from that of monolingually-raised speakers. We account for different sources of variability including cross-linguistic differences, and language transfer effects. The studies below shed light on different levels of the phenomenon. 

The structure of this chapter is as follows. In \sectref{cross-linguistic_study}, we review a comparative study on grammatical aspect in Greek, Russian, and Turkish \parencite{RizouEtAl}. This study investigates the monolingual varieties of Greek, Russian, and Turkish as well as their heritage variants in Germany and the US regarding the morphological marking of grammatical aspect.
By keeping the analysis as parallel as possible in the three languages, it explores how monolingually-raised and heritage speakers perform on perfective and imperfective aspect use, whether effects of cross-linguistic influence from the majority languages are evident, and how the parameters of formality and mode affect their aspect-related productions.
\sectref{periphrasesinGreek} highlights a study exploring the use of periphrastic aspectual constructions versus the use of aspect on lexical verbs in different formality levels by Greek heritage and monolingually-raised speakers \parencite{alexiadou2022use}. In \sectref{study_with_elicited_production_task}, we report a study of the accuracy of Greek monolingually-raised and heritage speakers in an elicited production task with items controlling the interaction of grammatical and lexical aspect \parencite{rizou2021verbal}.
The same methodology is employed by the study reported in \sectref{study-novel-forms}, in which morphologically novel forms that appear to be challenging in heritage speakers' grammars are analyzed \parencite{novelforms}.
This is followed in \sectref{study-on-Russian-aspect} by a review of a qualitative study by \textcite{gagarina2020first}, analyzing aspect use in child, adolescent, and adult productions of Russian heritage speakers in Germany and the US and contrasting it with the productions obtained from monolingually-raised Russian-speaking peers. 
The last study, in \sectref{eye-tracking_in_Turkish}, focuses on Turkish heritage and monolingually-raised speakers' incremental processing of grammatical aspect \parencite{Ozsoy_2023, Ozsoy_2024}. Using the Visual World Paradigm, this eye-tracking study aims to find the underlying factors that influence processing speed and picture-matching accuracy by individual participants.
Finally, in Section 8 we discuss the overall findings and their implications for broadening our understanding of the use of aspect in heritage and monolingually-raised speakers. The novelty of these studies lies in the systematic observation and categorization of verbal structures, providing an overview of verbal aspect cross-linguistically by employing different methodologies and controlling for intra- and extra-linguistic factors. 

\section{A cross-linguistic study on grammatical aspect in heritage and monolingual varieties of Greek, Russian and Turkish by \textcite{RizouEtAl}} \label{cross-linguistic_study}

The expression of verbal aspect in Greek, Russian, and Turkish varies in its morphological realization, but all three languages distinguish between perfective and imperfective aspect. In Greek, the stem of the morphologically marked aspect type is formed by suffixation, or weak/strong suppletion \parencite{spyropoulos2017root, revithiadou2019changing}. Example \REF{play_rizou} shows the addition of the suffix /s/ in the perfective, triggering stem allomorphy.\footnote{In order to keep the aspectual morphological description of the languages under the scope of this study parallel, more details about the Greek verbal morphology are mentioned in \sectref{study-novel-forms}, about Russian acquisition patterns of aspect in \sectref{study-on-Russian-aspect}, and about aspect in Turkish in \sectref{eye-tracking_in_Turkish}.}

\eal \label{play_rizou}
\ex \gll pez-o\\
 play.\textsc{ipfv-prs.1sg}\\
\glt `play' 
\ex 
\gll (na) peks-o \\
     \textsc{sbjv} play.\textsc{pfv-prs.1sg}\\
\glt `to play' 
\zl 

Most verbs in Russian form aspectual pairs and the distinction between perfective and imperfective aspect is primarily marked through morphology: the perfective aspect is derived from the stem of imperfective verbs through processes such as prefixation (as seen in example \ref{RUprefixation}), suffixation or suppletion, and additionally, imperfective verbs may be derived from perfective ones by suffixation leading to secondary imperfectivization \parencite{gagarina2008anaphoric}.\footnote{There are verbs that do not form aspectual pairs, such as \textit{perfectiva tantum}, as well as \textit{imperfectiva tantum} \parencite{Gagarina2007}.} 

\eal \label{RUprefixation}
 \ex \gll delat'\\
 do/make.\textsc{ipfv}\\
 \glt `to do/make' 
 \ex \gll s-delat' \\
 \textsc{pfv}-do/make\\
 \glt `to do/make' 
 \zl 

In Turkish, the copula in past tense -\textit{(y)DI}, the marker -\textit{DI} and the multifunctional evidential marker -\textit{mIş} are mainly applied to mark the perfective aspect, while for progressive-imperfective the verbal marker -\textit{(I)yor} is used. Regarding morphological markedness, the perfective aspect is considered morphologically marked in Greek and Russian, while the imperfective aspect is the marked one in Turkish as seen in example \REF{TurkishAspectExample}.

 \eal \label{TurkishAspectExample}
\ex \gll bitir-iyor-du-m\\
finish-\textsc{prog-pst-1sg}\\
\glt `I was finishing.'
\ex \gll bitir-di-m\\
go-\textsc{pst.pfv-1sg}\\
\glt `I finished.'
\zl

The aim of our investigation was to analyze how grammatical aspect is expressed in typologically different languages, namely Greek, Russian, and Turkish. By employing a consistent analytical framework for all three languages, the study explored the usage of perfective and imperfective aspect by both monolingually\hyp raised and heritage speakers of the respective languages in the US and Germany. An intriguing angle of this investigation is the examination of verbal aspect in heritage Greek, Russian and Turkish in contact with majority English, a language that marks aspect, and with majority German, which does not. The study seeks to understand whether and to what extent the use of aspect in heritage Greek, Russian, and Turkish is influenced by the language contact situation. Furthermore, the study addresses the question of whether the notion of morphological markedness impacts the production of grammatical aspect in heritage Greek, Russian and, Turkish, and if so, how. Additionally, the research delves into the impact of formality and mode on grammatical aspect productions. What distinguishes this study from previous research is its emphasis on cross-linguistically comparable and ecologically-valid data obtained from monolingual and heritage varieties of Greek, Russian, and Turkish through narration tasks.

The data analyzed in \textcite{RizouEtAl} 
were collected according to the \textit{language situations} approach by \textcite{wiese2020language}. This methodology provides comparable naturalistic data in both oral and written modes and in formal and informal registers. During the elicitation, the participants were shown a short video of a staged minor car accident, and their task was to narrate what happened to either a close friend or a police officer, imagining that they witnessed the incident. In that way, participants took part in four different communicative settings within one experimental session. Heritage speakers took part in two sessions at least three days apart, one in their majority and one in their heritage language, while monolingually-raised
participants took part only in one session in the majority language of the respective country of elicitation. To test how formality and mode affect narrations, we simulated formal-spoken, formal-written, informal-spoken, and
informal-written settings. The elicitation orders were balanced. 
\tabref{rizouetaltableparts} presents the overall information about the number of participants grouped by country of elicitation, their mean chronological age\footnote{Within each group, two age groups were tested, namely adults (age range 25--35) and adolescents (age range 14--18). In the present study, this further categorization was not taken into account, and all participants were analyzed together as the age variable was not a significant factor for grammatical aspect.} and their age of onset to bilingualism, and the number of tokens in the RUEG subcorpora of Greek, Russian, and Turkish.\footnote{A token is often used to mean something like a graphemic word but is technically defined as the smallest unit in a corpus, independent of what this smallest unit might be linguistically \parencite{schmid2008tokenizing, sauer2016flexible}.}

\begin{table}
\small
\begin{tabular}{llccccc}
\lsptoprule
Country & Group & Participants & \multicolumn{1}{c}{Tokens} 
& Verbs & M age & M AoO \\ 
\midrule
Greece & monolinguals & 64 & 27,931 & 4,954 & 21.4 & - \\  
Russia & monolinguals & 66 & 25,930 & 3,965 & 21.0 & - \\
Turkey & monolinguals & 64 & 20,947 & 4,609 & 22.2 & - \\ 
\midrule
Germany & \begin{tabular}[c]{@{}l@{}}heritage Greek \\ heritage  Russian\\ heritage Turkish\end{tabular} & \begin{tabular}[c]{@{}l@{}}48\\ 61\\ 64\end{tabular} & \begin{tabular}[c]{@{}r@{}}19,782\\  32,882 \\  23,722 \end{tabular} & \begin{tabular}[c]{@{}r@{}} 3,494\\  4,624 \\  4,986  \end{tabular}  & \begin{tabular}[c]{@{}r@{}} 22.6 \\  21.1 \\  21.5  \end{tabular} & \begin{tabular}[c]{@{}r@{}} 1.8 \\  1.3 \\  2.5  \end{tabular} \\
\midrule
US & \begin{tabular}[c]{@{}l@{}}heritage Greek\\ heritage  Russian\\ heritage Turkish\end{tabular} & \begin{tabular}[c]{@{}l@{}}63\\ 60\\ 58\end{tabular} & \begin{tabular}[c]{@{}r@{}}18,302\\  29,214 \\ 18,502\end{tabular} & \begin{tabular}[c]{@{}r@{}} 4,371\\  4,342 \\  4,257 \end{tabular} & \begin{tabular}[c]{@{}r@{}} 23.0 \\  22.2 \\  22.0  \end{tabular} & \begin{tabular}[c]{@{}r@{}} 1.3 \\  3.3 \\  2.5  \end{tabular} \\ 
\lspbottomrule
\end{tabular}
\caption{Participants information}
\label{rizouetaltableparts}
\end{table}

Verbs were tagged automatically as either perfective or imperfective and corrected manually by native speakers of Greek, Russian, and Turkish. The data extracted from the corpus were processed and analyzed in R using the \textit{tidyverse} and \textit{brms} packages \parencite{schmidt2009exmaralda, wickham2019welcome, Rcoreteam}. The observations of aspect were categorized as a binary dependent variable with two levels: imperfective and perfective. This resulted in a total of 26,788 data points, with 8,829 from Greek, 11,626 from Russian, and 6,333 from Turkish.
To draw cross-linguistic (Greek, Russian and Turkish heritage speaker groups) and cross-variational (monolingually-raised vs. heritage speakers) comparisons, we applied methods from a Bayesian statistical framework \parencite{gelman2013bayesian, kruschke2014doing, kruschke2015bayesian}. Bayesian regression models with 89 credibility intervals (CIs) were constructed for each language, incorporating random slopes for participants to accommodate individual differences. The independent variables in these models included Country of elicitation (Germany, US, and the native countries Greece, Russia, and Turkey), Formality (formal vs. informal), Mode (spoken vs. written), and Tense (past vs. present). Tense was included as one of the variables because of the interdependence between tense and grammatical aspect. Employing a uniform model structure across all languages ensures comparability in interpreting the outcomes. The availability of both data and analysis code in an open repository ensures the reproducibility of the analysis.

Applying descriptive statistics, we observed a different distribution of perfective and imperfective in the three languages. Greek, Turkish, and only two groups of Russian speakers showed a higher frequency of perfective aspect, whereas Russian heritage speakers in Germany did not align with this pattern. The results of the Bayesian statistical analysis of the estimates revealed that there were no meaningful effects for the Greek-speaking groups besides the increased production of perfective forms by the Greek heritage speakers in the US as compared to the grand mean [0.13, 0.31].
Russian heritage speakers in Germany opted for imperfective forms more likely than other Russian speaker groups [$-0.22$, $-0.09$]. In addition, all Russian-speaking groups tended to produce more imperfective forms in the spoken communicative situation compared to the written one [$-0.19$, $-0.13$]. For the Turkish-speaking groups, no meaningful effects of Country were observed, but at the level of formality, we found a small effect indicating that speakers produce
more imperfective in the formal compared to the informal condition [$-0.13$, $-0.04$]. For mode, Turkish speakers tended to produce more imperfective forms in the spoken compared to the written mode [$-0.12$, $-0.02$]. Finally, a common meaningful effect in all three languages was achieved by the Tense factor. Specifically, more verbs bearing the perfective aspect were produced in the past tense than in the present, which can be explained by a task effect initiated by the stimulus question ``What happened?". 

All in all, this cross-linguistic study offers an assessment of speakers' use of grammatical aspect. Although the comparison involves languages that all encode perfective and imperfective aspect grammatically, we observed divergent results for different groups of speakers. This could be driven by language contact or typological differences. The majority languages English and German in contact with heritage languages Greek, Russian, and Turkish seem to affect the use of perfective and imperfective in different ways. Additionally, we found effects of mode and formality. Providing a better overview of the distribution of grammatical aspect among heritage speakers of different languages and different varieties of the same language is the most important contribution to the overall research question.

\section{Periphrastic constructions in Greek by \textcite{alexiadou2022use}} \label{periphrasesinGreek}
Heritage varieties of Greek are claimed to show interesting dynamics in terms of aspectual formation and use. Several studies investigating verbal aspect show that this feature is challenging for Greek heritage speakers. The first scholar to identify discrepancies in the aspect category within heritage Greek was \textcite{seaman2017modern}, who noted a general simplification of verb forms and the application of periphrastic constructions among Greek heritage speakers residing in the US. Corroborating these findings, the study by \textcite{alexiadou2022use} that is reviewed in this section investigates more systematically two groups of Greek heritage speakers, namely one in the US and one in Germany, regarding the production of analytic forms. According to \textcite{haspelmath2000periphrasis}, a periphrasis refers to the linguistic phenomenon where a specific idea is conveyed using a combination of words rather than a single term. An example for Greek is shown in \REF{periphrases_rizou}, where the light verb like \textit{do} and a noun are used instead of a lexical verb. In Greek, there is a morphological syncretism in light verbs meaning that the perfective and imperfective verbal stems are identical in contrast to full verbs.  

\ea \label{periphrases_rizou}
\gll kano katathesi vs. katatheto \\ 
	do testify vs. testify \\
\glt `conduct a testimony' vs. `testify' 
\z 
\il{Greek}

In the study reported here, the use of periphrastic structures employing light and/or auxiliary verbs to convey past events was explored. These constructions stand in opposition to the synthetic representation of such events, which frequently entails morpho-phonological alterations within the verbal system. Finally, this study controlled for extra-linguistic factors such as mode and formality.  

The goal of the study was to detect whether the repertoires of different groups of heritage speakers differed in comparison to monolingually-raised speakers regarding the use of periphrastic constractions. \textcite{boon2014heritage} observed that heritage Welsh speakers resort to different strategies, such as code-mixing, in case they cannot retrieve morpho-phonological elements or lexical clusters. This has also been observed for heritage Greek by \textcite{alexiadou2018units}. Specifically, analytic structures were assembled from Greek light verbs that lack certain morphological features, and a borrowed infinitival verbal form from the majority language. \textcite{alexiadou2022use} investigated morphologically deviant patterns in heritage speakers' repertoires compared to monolinguals' productions, as well as emerging novel patterns and periphrastic constructions. Finally, the study explored the question of why heritage speakers make use of periphrastic constructions. What the authors proposed is that monolingually-raised Greek speakers utilize periphrasis to indicate different formality levels and distinguish between oral and written communicative settings, while heritage speakers tend to adopt forms linked with informal communicative settings, and this preference is influenced by the morpho-phonological intricacies of the formality of the communicative setting,\footnote{The Greek lexicon encompasses components that are delineated along the [+/−learned] spectrum, a trait that directly corresponds to the high and low register and the diglossic context prevalent in the 19th and 20th century.} leading to a tendency of register levelling overall.

Methodologically, the narration task in the frame of \textit{language situations} by \textcite{wiese2020language} was employed.\footnote{More information can be found in \sectref{cross-linguistic_study}. Detailed user guidelines and all experimental materials are available in an open-access repository: \url{https://osf.io/qhupg/}.} Two distinct age groups of Greek heritage speakers, namely adolescents and adults, residing in Germany and the US were tested. Additionally, a group comprising monolingually-raised speakers of similar age was included in the study. Their meta-linguistic data (self-ratings, mean age of onset, current input and literacy practices in the heritage language, hours and years of education in the heritage language, parents’ generation) were collected in a form of a questionnaire and correlated with the use of periphrastic constructions. 

The results showed a general preference for periphrastic constructions by Greek heritage speakers. These analytic constructions were analyzed in terms of appropriateness of use. This means that each construction was evaluated in terms of grammaticality and felicitousness in each context regarding the perfective/imperfective aspectual value. Only a few ambiguous and infelicitous cases were detected in heritage speakers' grammar. The most important finding is that heritage speakers use periphrasis to indicate the perfective aspect instead of using lexical verbs that require morpho-phonological changes as seen in example \REF{2nd_periphrases_rizou}. As the light verb \textit{do} in past tense has one identical stem for perfective and imperfective, speakers prefer to use this instead of lexical verbs that require morpho-phonological changes to form the perfective aspect. 

\ea \label{2nd_periphrases_rizou}
\gll ekanan bam vs. sigkrustikan \\ 
	did bam vs. collided \\
\glt `crashed' (attribution with the sound) vs. `collided' 
\z 
\il{Greek}

Another crucial tendency observed in the data is that heritage speakers utilized analytic forms interchangeably in the different formality settings, unlike monolingually-raised speakers who used them mainly in the informal setting. A common finding in both heritage and monolingually-raised speakers' narrations was that they both made use of analytic forms in the spoken mode across formality settings as an indicator of colloquial speech.

In sum, this study revealed some alternative strategies used by Greek heritage speakers for producing grammatical aspect. The investigation of analytic forms suggests a wider use of periphrastic constructions, pointing to a reorganization of heritage speakers' linguistic repertoire. Furthermore, the exploration of different communication settings highlights an emerging pattern, which is the use of periphrastic constructions across formality variation by heritage speakers in the US. The contribution to the overarching research question is that it brings to the fore a strategy systematically used by Greek heritage speakers. 

\section{Verbal aspect in heritage and monolingually-raised Greek speakers: Data from narration and production tasks by \textcite{rizou2021verbal}} \label{study_with_elicited_production_task}

Grammatical aspect in Greek can be either perfective or imperfective, as mentioned before. The perfective aspect in Greek denotes either completion or instantaneity of events, and it is marked on the verb stem, as we show below. The imperfective aspect is unmarked on the verb stem and corresponds to either continuous (example \ref{GRcontinuous}) or habitual (example \ref{GRhabitual}) interpretation. Imperfective aspect has the continuous or the habitual interpretation depending on the context. The use of lexical cues like adverbs or adverbial phrases clarifies the relevant interpretation as shown in examples \REF{GRcontinuous} and \REF{GRhabitual} \parencite{moser1994interaction}. As there is no aspectual distinction for the present tense in indicative mood in Greek, we illustrate this distinction by using past tense. 

\il{Greek}
\ea \label{GRcontinuous}
\gll I Maria magireve gia tris ores.\\
	The Mary cook.\textsc{ipfv.pst} for three hours.\\
\glt `Mary was cooking for three hours.'
\ex \label{GRhabitual}
\gll I Maria magireve tis Kiriakes.\\ 
	The Mary cook.\textsc{ipfv.pst} on Sundays.\\
\glt `Mary used to cook on Sundays.'
\z
\il{Greek}

The aim of the study by \textcite{rizou2021verbal} was to provide qualitative insights from the narration described in \sectref{cross-linguistic_study} and to measure the accuracy rate of Greek heritage speakers in an elicited production task as described below. Furthermore, the study investigated whether heritage speakers rely on the inherent semantics of the verbs as predicted by the Aspect Hypothesis by \textcite{andersen1994discourse} and combine the perfective aspect with telic complements and the imperfective with atelic in semi-spontaneous production tasks. Additionally, the differences between two age groups, namely adults and adolescents, within each speaker group are also explored in this study. 

The methodology employed for the narration task conforms with the \textit{language situations} approach by \textcite{wiese2020language}  as described in \sectref{cross-linguistic_study}.  To elicit verbal forms marked for grammatical aspect, an oral controlled production task in the form of sentence completion created by \textcite{agathopoulou2009morphological} was used. This task consisted of thirty items in total, mapping the three conditions of grammatical aspect, namely ten sentences denoting the perfective aspect, ten the imperfective continuous interpretation, and ten the imperfective habitual.\footnote{There are 9 strong verbs that require an exponent /s/ to mark the aspectual feature, 8 verbs that exhibit weak allomorphy requiring the exponent /s/, and  4 verbs that exhibit only weak allomorphy with vowel alternation patterns. Finally, there is one verb with strong suppletion, requiring a different stem altogether. 22 out of 30 verbs belong to the first conjugation class, and 8 of them belong to the second conjugation class.} The task was performed orally, meaning that the elicitor read the sentences out loud, leaving a 2-second pause in the place where the verb should be produced. Participants had to repeat the whole sentence with the answer, which was recorded.

Three groups of Greek speakers took part in this study, namely Greek monolin-gually-raised speakers and Greek heritage speakers in Germany and in the US; see \tabref{rizouetaltableparts}. A further categorization was conducted for age groups, namely adults and adolescents within the three groups.\footnote{Further information concerning the number of participants in each age group can be found in \textcite[41--42]{rizou2021verbal}.}  

The results from the qualitative analysis of the narrative data revealed patterns of perfectivization of verbs, mainly with atelic events and the verb class of activity verbs as shown in example \REF{perfectictivization_perpatisan} from the original publication.
This pattern was observed mostly in the narrations of heritage speakers rather than in those of monolingually-raised speakers, and it was found in both formal and informal communication settings. 

\ea \label{perfectictivization_perpatisan}
\gll Itan ena zevgari pu *perpatisan me ena karotsi. \\ 
	{there was} a couple who \hphantom{*}walk.\textsc{pst.pvf} with a stroller \\
\glt `There was a couple with a stroller that was walking.' 
\z 
\il{Greek}

Moving to the controlled elicited production task, a one-way multivariate analysis of variance test (MANOVA) showed no differences between the three independent groups. The outcome revealed a significant difference between the monolingually-raised and heritage speakers in the US across aspectual conditions. A weak three-way interaction between the groups and the age groups was reported, and so adults and adolescents in the respective groups were analyzed separately. It was found that heritage adults in the US performed more accurately in the three conditions of aspect compared to adolescent heritage speakers in the US. In contrast, Greek adult heritage speakers in Germany performed worse compared to adolescent heritage speakers in Germany. There was no statistical difference between the age groups in the different conditions of aspect. 

Summarizing, the overall contribution of the study is the exploration of different age groups, namely adult and adolescent heritage speakers' performance in an offline production task. Via the particular task, which has previously been used for different populations such as L2 speakers and bilingual children, the author obtained similar results to previous studies: the most challenging condition is the imperfective habitual one, for Greek heritage speakers both in Germany and in the US. The most important contribution, though, is the slight reverse pattern observed within and between the groups, namely the performance of adult and adolescent heritage speakers. This finding calls for future research on other factors affecting performance on verbal aspect (grammatical and lexical). 

\section{Novel forms for the expression of aspect in heritage Greek by \textcite{novelforms}} \label{study-novel-forms}

Greek has a rich verbal morphology; each verb is marked for tense, aspect, voice and agreement. Recent analyses of the verbs in the first and second conjugation classes (CC) have been conducted by \textcite{spyropoulos2017root} and \textcite{revithiadou2019changing}. The first CC comprises verbs in which the stress falls on the root of IPFV [−PAST] forms like \textit{graf-o} `write', while the second CC includes all verbs that exhibit non-root stress for example \textit{agap-ao/o} `love' and \textit{poth-o} `desire' \textcite[126--136]{holton1997greek}.

Focusing on the aspectual feature, the first pattern for the verbs of the first CC, in which the root ends either in a vowel or in a consonant in imperfective (e.g. \textit{idri-o} `establish'), is the addition of the exponent /s/ at the right edge of the root (e.g. \textit{(na) idri-s-o} `to establish'). These verbs are called regular/strong verbs in the literature \parencite{spyropoulos2017root, revithiadou2019changing}.

The second pattern concerns the irregular/weak roots, which undergo reshaping due to phonological adjustments, suppletion, and omission or addition of a consonant and syllables. There are two allomorphic vowel patterns of the same root, one for [−]PFV environments (e.g. \textit{sern-o} `drag') and one for [+]PFV, (e.g. \textit{(na) sir-o} `to drag').\footnote{The /-n/ does not belong to the root and thus, the root vowel changes into /i/ when the coronal /-n/ disappears.}

The third pattern requires suppletion under which verbs undergo a complete stem change or strong suppletion, (e.g. \textit{tro-o} `eat' for [−]PFV and \textit{(na) fa-o} `to eat' for [+]PFV environments), unlike the verbs that require weak suppletion.
The verbs of the second CC exhibit two allomorphic stems in
[−]PFV (e.g. \textit{pul(a)-o} `sell') and [+]PFV aspect (e.g. \textit{(na) puli-s-o} `to sell'). 

The study by \textcite{novelforms} explored whether the productions of heritage speakers of Greek differed from those of monolingually-raised speakers regarding the encoding of several morpho-phonological features given the concatenate verbal morphology and, in more complex cases, the addition of prefixes. Specifically, the goal of the study was to reveal whether there were mismatches in verbal features such as aspect, tense, voice, $phi$-features (person and number), and whether there were any other asymmetries in morpho-phonological alternations like vocalic elements or stress placement.
 
Previous studies have shown that certain morphological features seem to be open to dynamic patterns in bilingual/heritage grammar across a variety of languages. One of the first researchers to mention differences in Russian heritage speakers' verbal morphology was \textcite{romanova2008mechanisms}. \textcite{slabakova2014bottleneck} and \textcite{mikhaylova2018morphological} proposed the Bottleneck Hypothesis for L2 learners and heritage speakers, proposing that these speakers face difficulties with morphological features in languages with rich inflection. Furthermore, \textcite{perez2019differential} and \textcite{fernandez2021acquisition} report differences in mood and tense for Spanish heritage speakers while \textcite{johannsdottir2023and} reports an overstandardization of tense features in heritage Icelandic, meaning that the speakers resort to default strategies. It has been observed by \textcite{scontras2015heritage} that heritage speakers often alter the structure of the verbal morphosyntactic system, resulting in the simplification of intricate morpho-phonemic rules.

The methodology employed in \textcite{novelforms} is the same as that used by \textcite{rizou2021verbal}, and was described in \sectref{study_with_elicited_production_task}. In the elicited production task, there were twenty verbs of the first CC and ten of the second. Both strong verbs and verbs that require weak allomorphy were included in the experimental conditions. The participants who took part in this study were Greek monolingually-raised speakers recruited in Athens, Greece, and Greek heritage speakers recruited in Germany and in the US. They performed an elicited production task with fixed conditions for the production of perfective and imperfective aspect. More details can be found in the aforementioned publication.

The highlight of the analysis is that heritage speakers in the US produced more innovative verbal forms compared to heritage speakers in Germany. In Greek heritage speakers' productions, numerous patterns were detected pointing to a systematic restructuring of stem change, conjugation class change, stress and /s/ suffixation misplacement, and erroneous or dropped augment /e/ as seen in examples \REF{novel_examples_rizou} categorized in the second CC. Another tendency was detected regarding the voice feature, indicating that heritage speakers make use of the productive rule proposed by \textcite{oikonomou2022voice} of employing the non-active voice as opposed to the typical active voice, suggesting a potential generalization of the voice feature for the sake of morphological simplification.

\ea \label{novel_examples_rizou}
 \ex \gll *(è)-for-e vs. fòra-g-e\\
 \hphantom{*}augment-wear-\textsc{3sg} vs wear-\textsc{ipvf-3sg}\\
 \glt `was wearing' 
 \ex \gll pònak*-s-e vs ponù-s-e \\
 hurt-\textsubscript{vocalic element}-\textsc{3sg} vs hurt-\textsc{ipfv-3sg} \\
 \glt `was hurting' 
 \z
 
Overall, morphological analyses of verbal features are rare in Greek heritage linguistics. This study contributes to filling this gap through a detailed analysis of dynamic verbal features in Greek heritage grammars. Insights on the features encoded in verbs are provided in this qualitative analysis, shedding light on this unexplored field and pointing the way to further studies on other phenomena that are encoded with different features in different languages.

\section{Aspect acquisition and use in heritage Russian in Germany and the US by \textcite{gagarina2020first}} \label{study-on-Russian-aspect}

Aspect production in adult heritage Russian speakers is often characterized by systematic patterns that differ from standard Russian \parencite{pereltsvaig2004immigrant, laleko2008compositional}.
Besides, studies on child heritage Russian report innovations in the use and formation of perfective and imperfective verbs \parencite{gagarina2011acquisition, antonova2016aspect, kistanova2019acquisition}.
One explanation for this dynamic linguistic behavior might be the missing sustainability hypothesis proposed by \textcite{gagarina2020first}; see \textcite{swain1995problems} for a similar analysis.
Specifically, in language contact situations, the aspectual system seems to undergo restructuring due to a lower amount of input in the heritage language in the daily communication of young children, even though the use of aspect is acquired early in Russian \parencite{ceytlin2000jazyk, gagarina2008first, kistanova2019acquisition}. 
Additionally, the dominance shift from the heritage language to the majority language, which usually occurs when children enter monolingual education institutions, further contributes to the dynamicity of the aspectual category in the linguistic system of an asymmetrically bilingual child.

The aim of \textcite{gagarina2020first} was to investigate the productions of aspectual forms in heritage child and adult Russian in Germany and the US by contrasting them to the typical aspectual patterns of monolingual child Russian.
More specifically, the study dealt with the question of whether the observed changes in the aspectual system of heritage Russian are unique for heritage speakers or whether they are typical for the acquisitional patterns of aspect in monolingual acquisition.
Furthermore, the study attempted to gain insight into the degree of dynamicity of the aspectual system of heritage Russian, i.e., the degree of change in a given time period.
To achieve this goal, the study applied the usage-based theory of language acquisition \parencite{tomasello2005constructing} as well as frequency accounts of (first) language acquisition \parencite{gulzow2011frequency, lieven2010input}.

The investigation was carried out on longitudinal spontaneous production data collected from two heritage bilingual children in Germany (Russian\hyp German) in Germany and one heritage bilingual child (Russian\hyp English) in the US (age range: 1;6--4;0 years) on the one hand, as well as on semi-spontaneous production data elicited from 128 adolescent and adult heritage Russian speakers (age range: 15--35 years) in Germany and the US on the other hand.
Productions of perfective and imperfective verb forms of heritage speakers were compared with the productions of child monolingual speakers of Russian. 
Monolingual data were taken from studies by \textcite{bar2002tense, kistanova2019acquisition, bondarko2011kategorizacija}, and \textcite{Gagarina2007}.
Adult and adolescent heritage data were collected according to the \textit{language situations} approach by \textcite{wiese2020language}, described earlier in \sectref{cross-linguistic_study}. 
In the present study, only Russian heritage data were used, which were drawn from the 0.2.0 version of the RUEG-RU subcorpus \parencite{RUEGcorpus2019_v.0.2.0}.\footnote{Please note that the number of heritage Russian speakers reported in \textcite{RizouEtAl} slightly differs from the numbers in the current study due to the different corpus versions used.}

The findings indicate that in young children at the age of around two years, aspect production is very similar in heritage and monolingual Russian. 
However, with increasing age starting already from three to four years, aspect use of heritage speakers occasionally shows dynamic patterns and innovations, which are not or rarely attested in productions of monolingual peers. 
These patterns often represent speakers' strong preference for perfective aspectual forms in analytical constructions, and also for the substitution of perfective aspectual forms by imperfective ones in denoting concrete resultative events.
In example \REF{RUinnovations1}, the verb form \textit{pojti} ‘go' is perfective, which is an ungrammatical use in standard Russian. Instead, an imperfective form \textit{idti} ‘go' is canonically used here.
In \REF{RUinnovations2}, the imperfective use of ‘brake' is unexpected in the given context, in which the speaker was looking at the car driver braking suddenly to prevent an accident. The expected perfective verb form is \textit{za-tormozil} ‘brake'. 

\ea \label{RUinnovations1}
 \gll ne xoču *\textbf{pojti} v školu\\
 not want \hphantom{*}go.\textsc{pfv} to school\\
 \glt `(I) don't want to go to school'
 \z
 
 \ea \label{RUinnovations2}
 \gll voditel’ \#\textbf{tormozil} \\
 driver \hphantom{\#}brake.\textsc{impf.pst}\\
 \glt `The car driver was braking'
 \z

In adolescent and adult heritage speakers of Russian, those patterns consistently appear in language productions. 

Taken together, the results of the study demonstrate that acquisitional patterns in bilingual and monolingually-raised children converge until the age of three, and then rare innovations and dynamic patterns emerge in productions of heritage speakers. Those patterns include the use of perfective aspectual forms and modal verbs in analytical constructions, as well as substitutions of perfective forms by imperfective ones. Such linguistic behavior in heritage speakers might be explained by the \textit{missing sustainability hypothesis}, suggesting that the aspectual system undergoes restructuring caused by a low quantity of input in the heritage language during the daily communication to young children in situations of language contact.

\section{Processing of verbal aspect in a Visual World eye-tracking paradigm by \textcite{Ozsoy_2023}} \label{eye-tracking_in_Turkish}

The study by \textcite{Ozsoy_2023} reviewed in this section was conducted with Turkish-speaking groups, and can be thought of as a conceptual replication of two processing studies by \textcite{minor2022fine} and \textcite{minor2023aspect}. \textcite{minor2022fine} investigated the processing of grammatical aspect in monolingual Russian speakers. \textcite{minor2023aspect} extended this research to monolingual speakers of English and Spanish, comparing the results with the original data from \textcite{minor2022fine}. Minor and colleagues performed an initial picture selection task and a Visual World eye-tracking study focusing on Russian aspectual prefixes, revealing that participants already demonstrated a preference prior to the verb's completion for target images corresponding to either ongoing or completed events. This suggests incremental processing of grammatical aspect information on a word-internal scale. 


The study reviewed here, \textcite{Ozsoy_2023}, extended the aforementioned aspect processing paradigm to bilingual heritage speakers of Turkish. In doing so, this study aims to fill the gap in the literature on heritage Turkish concerning the comprehension of grammatical aspect and contributes to a paradigm shift towards online psycholinguistic investigations of heritage languages \parencite{Bayram_2021}. Specifically, the researchers aimed to investigate the extent to which the preference for the perfective aspect in processing completed events and the imperfective aspect for ongoing events would be displayed by Turkish heritage speakers in Germany and Turkish monolingually-raised speakers. They tested this distinction using two different suffixes in item pairs such as in \REF{TurkishPerfectiveVWP} and \REF{TurkishImperfectiveVWP}.

\il{Turkish}
\ea \label{TurkishPerfectiveVWP}
\gll Temiz gömleğini ütüle-di yaşlı amca {özene bezene}.\\
clean shirt iron-\textsc{pst.pfv} old uncle properly\\
\glt ‘The elderly uncle carefully ironed his clean shirt.’
\ex \label{TurkishImperfectiveVWP}
\gll Temiz gömleğini ütülü-yor-du yaşlı amca {özene bezene}.\\
clean shirt iron-\textsc{impfv.prog-pst} old uncle properly\\
\glt ‘The elderly uncle was carefully ironing his clean shirt.’
\z
\il{Turkish}

One of the questions that the study explored is whether aspectual distinctions would be recognized by heritage speakers of Turkish. These speakers have shown reduced sensitivity to TAM morphology in previous studies \parencite{arslan2017processing, aylin2018second}, which could lead to a reduced effect (i.e., slower processing of aspectual distinctions) in the processing patterns of heritage speakers of Turkish when compared to monolingually-raised speakers. Another related research question is whether aspectual distinctions would be processed incrementally as the sentence unfolds. Additionally, the study explored driving factors of individual differences in processing by examining individual cognitive skills (processing speed) and language proficiency (C-test scores). 

A statistical analysis of the eye-tracking results focused on eye gaze preferences and the accuracy of picture selection. With regard to the overarching research question of this review chapter, the results indicate that heritage speakers recognize aspectual distinctions between the perfective and the imperfective aspect. The results also suggested that heritage speakers and monolingual speakers of Turkish indeed processed aspectual distinctions incrementally \parencite[see also][]{Ozsoy_2024}. However, the researchers found that incremental processing is delayed in heritage speakers compared to monolingually-raised speakers. The effect of the incremental processing of aspectual distinctions arises shortly after the offset of the verb and most closely patterns with the findings on Spanish, which seems to be due to similar morphological characteristics. Both Turkish and Spanish mark grammatical aspect using suffixes after the verb stem.

Additionally, within the group of heritage speakers, the findings showed small significant effects of individual cognitive abilities (processing speed) and language proficiency, indicating that speakers with higher cognitive abilities and language proficiency recognized aspectual distinctions faster. An effect of language proficiency was also observed in the results of the picture selection task, showing that speakers with higher C-test scores reached higher accuracy.

Overall, this study replicates the findings of \textcite{minor2022fine, minor2023aspect} for Turkish, which is typologically different from languages in the Indo-European language family, such as English, Russian, and Spanish, which were investigated by the aforementioned researchers. Considering the aim to extend (aspect) processing studies to bilingual populations \parencite{Bayram_2021}, this study took a successful first step. Turkish heritage speakers were able to process aspectual marking and its distinctions incrementally. These findings contrast with previous studies, such as those by \textcite{karaca2024morphosyntactic} and \textcite{ozsoy2023turkish}, which found that heritage speakers show no or only marginal effects in processing grammatical factors such as case and aspect. This new study demonstrates that heritage speakers do engage in grammatical processing and this is modulated by individual differences factors. The authors offer valuable insights into the interplay of linguistic and cognitive factors in the processing of aspectual distinctions among Turkish heritage speakers, shedding light on potential effects of heritage language bilingualism and accounting for individual differences.

\section{Discussion} \label{aspectDiscussion}
    
This chapter provides reviews of six studies focusing on aspect in heritage and monolingual varieties of Greek, Russian, and Turkish that were conducted as part of the Research Unit \textit{Emerging Grammars}. The cross-linguistic study evaluates the use of perfective and imperfective aspectual forms among the aforementioned population in narrative tasks, while the rest of the studies provide different foci on the exploration of grammatical aspect in each language separately analyzed with different methods. Some of the studies investigate the influence of formality and mode variation on the expression of aspect forms as well. Analyzing large-scale data for each language with different statistical methods, we reveal interesting insights into the dynamicity of phenomena related to grammatical aspect that undergo systematic changes. Different factors that affect the acquisition, the preservation, and the production of aspect are analyzed in each study, offering different perspectives regarding the phenomenon of aspect.  

Aspect was one of the first phenomena addressed in heritage language research \parencite{montrul2002incomplete, kagan2005support,laleko2008compositional}. However, no previous research has used a comparative approach to explore the realization of aspect in heritage speakers of typologically distinct languages. The comparative approach in \textcite{RizouEtAl} reveals that heritage speakers are influenced by the morphological marking of aspect in the majority language in language contact situations. We expected the notion of morphological markedness as described by \textcite{comrie1976} to affect heritage speakers' preferences with regard to grammatical aspect. On the one hand, we expected the speakers of the heritage languages in contact with majority German to opt for the simpler and morphologically unmarked forms, as German does not have grammatical aspect; on the other hand, we expected heritage speakers to opt for the marked forms due to contact with majority English, which marks grammatical aspect.  
These predictions were confirmed for Russian heritage speakers in Germany and for Greek heritage speakers residing in the US, respectively. The findings speak in favor of the markedness account \parencite{comrie1976}, although it is not uniform for all languages under the scope of this study. For Turkish, no group effects were found, which is probably connected to the typologically different way of aspect marking, which is provided by the prominent markers \textit{-iyor} and \textit{-(y)di}. As the study by \textcite{Ozsoy_2023} revealed, the processing of the aspectual distinctions begins shortly after the offset of the verb and is incremental. Although heritage speakers perform slower compared to monolingually-raised speakers, this does not seem to be related to the salient aspectual markers in Turkish and rather might be explained by language use effects.

Another factor considered in the studies by \textcite{RizouEtAl} and \textcite{alexiadou2022use} is the communicative setting. The factors of formality and mode variation affect the realization of grammatical aspect in different ways in the languages discussed in this chapter. By mixing various levels of formality and mode, one can get varied results across different languages, as demonstrated in the study by \textcite{pescuma2023situating}. Thus, in the study by \textcite{RizouEtAl}, mode effects were detected in both Turkish and Russian speakers, who tended to produce more imperfectives in the spoken mode. Additionally, formality effects were reported only for Turkish, manifested in more imperfective forms in the formal setting. In the study by \textcite{alexiadou2022use} it was found that both heritage and monolingually-raised speakers clearly favor periphrastic constructions in the oral mode, but there was a tendency by Greek heritage speakers in the US to use these analytic structures more often than monolingually-raised speakers. Since there are no studies correlating the different formality levels and modes with the phenomenon of grammatical aspect, we approached mode and formality variation in an exploratory way. The non-uniformity of the results regarding the factors of formality and mode in the different languages is probably due to the fact that extra-linguistic situational-functional parameters which influence intra-individual variation (cf. \cite{ludeling2022register}) do not uniformly determine the distribution of linguistic means across different language communities, even if the phenomena involved are comparable.

A further factor taken into consideration in the studies investigating verbal aspect is the age factor. In the study by \textcite{rizou2021verbal}, different communities of Greek heritage speakers exhibit a divergent performance on production tasks concerning the phenomenon of verbal aspect. The Greek adolescent heritage group in Germany performed more accurately in the task compared to the adult group. However, the opposite pattern was detected in the US: Greek adult heritage speakers performed more accurately compared to the adolescent heritage group. The explanation of this finding might be hidden in community characteristics such as past and current input and literacy practices, which seem to be different in heritage Greek communities in Germany and the US.  

The study by \textcite{gagarina2020first} provides an overview of the use of aspect in L1 and heritage Russian, spanning from the initial occurrence of verb form production in child L1 to adulthood. The findings indicate a resemblance in the early acquisition and production of aspectual forms between monolinguals and bilinguals, while notable differences in aspect use between the two populations emerge systematically around the age of three to four. 
Differences in aspect usage from monolingually-raised speakers, such as the substitution of perfective with imperfective in analytical future constructions, as well as the use of imperfective in resultative contexts by heritage Russian speakers, are attributed to the lack of a sustainable input. In their adulthood, even fluent and proficient Russian heritage speakers overlook formal restrictions aligning with the prototypical semantics of aspect. These systematic patterns are in line with previous literature reporting a spread of non-canonical patterns in the aspectual system of Russian heritage speakers \parencite{polinsky2000composite}.

The study by \textcite{novelforms} highlights the challenges that heritage speakers face in encoding all verbal features within and outside the stem boundaries in a language with concatenate morphology, such as Greek. The features most susceptible to change seem to be the $\phi$-features and aspect. At the same time, stem alternations that involve an allomorphic vocalic element and morpho-phonological adjustments seem to pose difficulties for heritage speakers' productions even in experiments without time constraints. Thus, heritage speakers resort to morphologically novel patterns with simplified and even default verbal features. This is in line with the theory of representational economy proposed by \textcite{scontras2015heritage}. Another strategy followed by heritage speakers is revealed in the study by \textcite{alexiadou2022use}, which is a preference for analytic verbal forms. This periphrasis contains a light verb and either a noun or a code-switched phrase. Heritage speakers tend to resort to morphologically less complex verbs, as is the case with light verbs. Thus, they avoid difficulties with morpho-phonological changes that come with aspect formation of lexical verbs.

The aspectual preference observed in our speakers is not determined solely by one factor; rather, it is influenced by a complex interaction of both language-internal and language-external factors in addition to other background factors. The findings suggest that internal grammatical properties of languages in a contact situation, such as the typology of each language, affect bilingual speakers' selection of aspectual forms in language contact situations.

\begin{sloppypar}
This review chapter confirms the necessity of adopting a multifaceted approach to elucidate phenomena of the core grammatical system, like grammatical aspect. The methodologies and analyses outlined here underscore the importance of conducting further research to comprehensively explore grammatical aspect across various languages and populations in diverse situational contexts. By conducting experiments that account for the aforementioned factors, researchers can assess whether and to which degree these factors impact the processing and production of grammatical aspect. 
\end{sloppypar}

\section*{Acknowledgements}
We would like to thank everyone involved in projects P1, P3 and P10 of RUEG.
This work has grown out of a big team effort, with many student assistants spending hundreds of hours with detailed annotations. For this reason, we would like to thank all of them: Foteini-Maria Karkaletsou, Nikolaos Tsokanos, Ioanna Kolokytha, Iro Malta, Katerina Chatzitheoxari, Olga Kritharidou, Foteini Iliadou, Tsampika Psilou, Panagiota Papavasileiou, Jannis Spiekermann, Olga Buchmüller, Andrei Koniaev, Anastasia Rozova, Tony Müller, Yelizaveta Vlasova, Alexandra König, Gvantsa Rukhiashvili, Nina Bredereck, Daria Alkhimchenkova, Zeynep Özal, and Büşra Çiçek.
We also thank Kalliopi Katsika, Cem Keskin, Kateryna Iefremenko, and  two anonymous reviewers as well as audiences at the WILA 14 and the ISB 14 conferences for their helpful and constructive suggestions. We also thank the RUEG Team and RUEG's Mercator Fellows for their insightful comments. Finally, we thank Lea Coy for help with pre-publication formatting.

The research was supported through funding by the Deutsche Forschungsgemeinschaft (DFG, German Research Foundation) for the Research Unit \textit{Emerging Grammars in Language Contact Situations}, projects P1 (no. 394836232), P3 (no. 394839191) and P10 (no. 313607803).

\printbibliography[heading=subbibliography,notkeyword=this]
\end{document}
