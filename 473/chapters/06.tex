\documentclass[output=paper,colorlinks,citecolor=brown]{langscibook}
\ChapterDOI{10.5281/zenodo.15775167}
%This is chapter 6
\author{Onur Özsoy\orcid{0000-0003-3617-4697}\affiliation{Leibniz-Centre General Linguistics; Humboldt-Universität zu Berlin} and  Vasiliki Rizou\orcid{0000-0002-5804-4976}\affiliation{Humboldt-Universität zu Berlin} and Maria Martynova\orcid{0000-0003-4833-9567}\affiliation{Humboldt-Universität zu Berlin} and   Natalia Gagarina\orcid{0000-0002-5136-1071}\affiliation{Leibniz-Centre General Linguistics; Humboldt-Universität zu Berlin} and    Luka Szucsich\orcid{0000-0003-0264-980X}\affiliation{Humboldt-Universität zu Berlin} and  Artemis Alexiadou\orcid{0000-0002-6790-232X}\affiliation{Leibniz-Centre General Linguistics; Humboldt-Universität zu Berlin}
   }
\title{Null subjects in heritage Greek, Russian and Turkish}

\abstract{Greek, Russian, and Turkish represent three different types of null subject languages: Greek is characterized as a strict pro-drop language; Russian is claimed to be either a partial pro-drop, or a non-pro-drop language with abundant subject ellipsis; Turkish is classified as a pro-drop or topic pro-drop language in which overt subjects are required in unspecified contexts. This chapter reports results from corpus-based studies exploring the question of whether subjects in heritage varieties of these three languages in contact with English and German pattern alike regarding the realization of overt subjects. Moreover, the chapter examines whether heritage speakers of these three languages apply topic drop more widely in one of their majority languages, namely German. We consider different possibilities for our observations reaching from language contact effects, to language internal dynamics and most importantly a typological continuum within the different pro-drop types regarding the stability and dynamicity in their pro-drop systems.
\keywords{null subjects, heritage languages, Greek, Russian, Turkish}
}



\IfFileExists{../localcommands.tex}{
   \addbibresource{../localbibliography.bib}
   % add all extra packages you need to load to this file

\usepackage{tabularx,multicol}
\usepackage{url}
\urlstyle{same}

\usepackage{listings}
\lstset{basicstyle=\ttfamily,tabsize=2,breaklines=true}

\usepackage{langsci-basic}
\usepackage{langsci-optional}
\usepackage{langsci-lgr}
\usepackage{langsci-osl}
% \usepackage{./langsci/styles/langsci-lgr}
% \usepackage{./langsci/styles/langsci-osl}
% \usepackage{langsci-gb4e}

\usepackage{tikz}
\usetikzlibrary{patterns,calc}
\pgfdeclarepatternformonly{south east lines}{\pgfqpoint{-0pt}{-0pt}}{\pgfqpoint{3pt}{3pt}}{\pgfqpoint{3pt}{3pt}}{
    \pgfsetlinewidth{0.6pt}
    \pgfpathmoveto{\pgfqpoint{0pt}{3pt}}
    \pgfpathlineto{\pgfqpoint{3pt}{0pt}}
    \pgfpathmoveto{\pgfqpoint{.2pt}{-.2pt}}
    \pgfpathlineto{\pgfqpoint{-.2pt}{.2pt}}
    \pgfpathmoveto{\pgfqpoint{3.2pt}{2.8pt}}
    \pgfpathlineto{\pgfqpoint{2.8pt}{3.2pt}}
    \pgfusepath{stroke}}
    
\usepackage{stmaryrd}
\usepackage{wasysym}
\usepackage{multirow}
\usepackage{caption}
\usepackage{subcaption}
\usepackage{mathrsfs}
\usepackage{qtree}

\usepackage{linguex}


   %pminos do not split footnotes
% \interfootnotelinepenalty=10000 %Footnote in Laporte chapters has to be split SN


%\DeclareIndexNameFormat{default}{%
%\nameparts{#1}%
%\usebibmacro{index:name}%
%{\index[names]}%
%{\namepartfamily}%
%{\namepartgiveni}%
% {}% L1
% {}% L2
%{\namepartprefix}% generates spurious space L3
%{\namepartsuffix}% generates spurious space L4
%}

%  {\DeclareIndexNameFormat{default}{%
%     \usebibmacro{index:name}{\index[names]}{#1}{#3}{#5}{#7}}}

%\DeclareIndexNameFormat{default}{%
%  \usebibmacro{index:name}{\sindex[nom]}{#1}{#3}{#5}{#7}}

%\DeclareIndexNameFormat{default}{%
%  \usebibmacro{index:name}{\sindex[person]}{#1}{#3}{#5}{#7}}
%\DeclareIndexNameFormat{default}{%
%\nameparts{#1} \usebibmacro{index:name}{\sindex[person]]}{\namepartfamily}{‌​\namepartgiven}{\nam‌​epartprefix}{\namepa‌​rtsuffix}}

%\newcommand{\smiley}{:)}

%\renewbibmacro*{index:name}[5]{%
%\usebibmacro{index:entry}{#1}%
%{\iffieldundef{usera}{}{\thefield{usera}\actualoperator}\mkbibindexname{#2}{#3}{#4}{#5}}}

% \newcommand{\noop}[1]{}

%remove for final
%\overfullrule=1mm

\newcommand{\tobi}[2]}}
\renewcommand{\S}[1]{\tobi{#1}{\textsc{*}}}

% this volume references
% puts: [this volume]
% already defined: \citetv
%\newcommand{\citepv}[1]{(\citeauthor{#1} \citeyear*{#1} [this volume])}
\newcommand{\citealtv}[1]{\citeauthor{#1} \citeyear*{#1} [this volume]}

%parentheses around example number
\newcommand{\pref}[1]{(\ref{#1})}

% in-text examples

\newcommand{\lnex}[1]{\textit{#1}} %target lang word
\newcommand{\lnlit}[1]{(lit.: `#1')} %literal reading
\newcommand{\lnlat}[1]{(#1)} % latinization
\newcommand{\lntrans}[1]{`#1'} %translation
\newcommand{\lnexl}[2]%
{\lnex{#1}{} \lnlat{#2}} % ex with latinization
\newcommand{\lnexlat}[3]{\lnex{#1}{} \lnlat{#2}{} \lntrans{#3}} % ex with latinization and tranl.

%ch01
\newcommand{\co}[1]{\mbox{\textbf{#1}}}

%ch09

\newcommand{\cyrbulg}[1]{\begin{otherlanguage*}{bulgarian}#1\end{otherlanguage*}}


%ch10
\newcommand{\nlp}{{\small NLP}}
\newcommand{\mwe}{{\small MWE}}
\newcommand{\rae}{{\small RAE}}
\newcommand{\lvc}{{\small LVC}}
\newcommand{\pos}{{\small P}o{\small S}}
%\newcommand{\todo}[1]{ \textcolor{red}{#1} }

%\renewcommand{\labelenumi}{\theenumi}
%\ainamefmt{{vv}{ll}{, ff}{, jj}} % fullname

\newcommand{\biberror}[1]{{\color{red}#1}}

\newcommand{\osenovaitem}{--~}
   %% hyphenation points for line breaks
%% Normally, automatic hyphenation in LaTeX is very good
%% If a word is mis-hyphenated, add it to this file
%%
%% add information to TeX file before \begin{document} with:
%% %% hyphenation points for line breaks
%% Normally, automatic hyphenation in LaTeX is very good
%% If a word is mis-hyphenated, add it to this file
%%
%% add information to TeX file before \begin{document} with:
%% %% hyphenation points for line breaks
%% Normally, automatic hyphenation in LaTeX is very good
%% If a word is mis-hyphenated, add it to this file
%%
%% add information to TeX file before \begin{document} with:
%% \include{localhyphenation}
\hyphenation{
    Beck-man
    Ngu-yen
    back-chan-nel
    back-chan-nels
    mo-not-o-nous
    ste-reo-typ-i-cal
}

\hyphenation{
    Beck-man
    Ngu-yen
    back-chan-nel
    back-chan-nels
    mo-not-o-nous
    ste-reo-typ-i-cal
}

\hyphenation{
    Beck-man
    Ngu-yen
    back-chan-nel
    back-chan-nels
    mo-not-o-nous
    ste-reo-typ-i-cal
}

   \boolfalse{bookcompile}
   \togglepaper[6]%%chapternumber
}{}

\begin{document}
\lehead{Onur Özsoy et al.}
\maketitle
\section{Introduction}
The main aim of this chapter is to examine the phenomenon of null subjects in Greek, Russian, and Turkish majority and heritage language varieties using a comparative approach. The study analyzes data from both monolingual and bilingual populations, where the latter are heritage speakers. Heritage speakers are typically individuals who use their heritage language(s), learned during early childhood in their families, in addition to the majority language(s) spoken in the host community \parencite[cf. among many others][]{Valdes2005, Rothman2009, benmamoun2013heritage, Guijarro-FuentesSchmitz2015, Polinsky2015, Montrul2015}. Heritage speakers are considered bilinguals belonging to a nativeness continuum as native speakers of their heritage language, but the proficiency in the languages they speak can vary throughout their lifetime \parencite{benmamoun2013heritage, polinsky2018heritage, wiese2022heritage}. In this chapter, we adopt this perspective and we do not compare the heritage data to the monolingual data as a measure of accuracy. Rather, we view the heritage and monolingual populations as belonging to distinct language varieties in their own right, in line with \citegen{rothman2022monolingual} perspective.

Additionally, we aim to gain an understanding of subject realization in the heritage languages Greek, Russian, and Turkish, which differ typologically, and investigate whether they are influenced by contact with the majority languages English and German. The latter are both characterized as non-pro-drop languages with German allowing for the omission of subjects in topic positions \parencite{oppermann2021subject, schafer2021topic, trutkowski2016topic}. 
Homeland varieties of Greek, Russian and Turkish all allow subject omission to varying degrees. Heritage Greek, Russian, and Turkish have been argued to exhibit similar tendencies regarding their pro-drop properties. Notably, overtly realized pronominal subjects have been found to be more frequent in heritage varieties, particularly those in contact with Germanic languages, than in monolingual ones. This has been discussed in various studies such as \textcite{ArgyriSorace2007} and \textcite{TsimpliEtAl2004} for heritage Greek; \textcite{DubininaPolinsky2013,Gagarina2012}, and \textcite{Ivanova-Sullivan2014, Ivanova-Sullivan2015} for heritage Russian; and \textcite{Haznedar2010} and \textcite{Koc2016} for heritage Turkish (but see \cite{AzarEtAl2020} for results showing that heritage and monolingual speakers of Turkish omit subjects to a similar degree).

While previous studies have mostly focused on the heritage language of these speakers, there are very few studies that investigate the majority language too. This chapter adds to these studies and discusses whether HSs of Greek, Russian, and Turkish in Germany apply topic drop more widely in their majority language German. An expansion of topic drop would represent a creative extension of German grammar which points to a reorganization at the pragmatics-syntax interface.

The investigation of subject realization in heritage Greek, Russian, and Turkish in the US and Germany, and of topic drop in majority German is a promising area of study since it relates to the interface between internal linguistic domains (core grammatical system) and external linguistic domains (information structure and discourse organization). To achieve the above-mentioned objectives, the study uses a multi-factorial analysis of semi-naturalistic and ecologically more valid data from monolingual and heritage speakers, taking into account differences in formality and mode.

This study also extends previous research in directly addressing the effect on subject realization of animacy, mode, and formality --
three factors which are meaningful to subject realization but are not often considered in this field. Some studies investigating languages unrelated to the languages of our study point to animacy playing an important role in subject realization.
Animacy is an inherent feature of nominal referents \parencite{corbett1979agreement, comrie1989}. 
Numerous scholars have classified animacy on a hierarchy \parencite{Silverstein1986}, placing it on a continuum with animate human discourse participants on the one end and inanimate and abstract referents on the other (cited from \cite[90]{corbett2000number}), as illustrated below:

\ea
speaker (first person pronouns) $>$ addressee (second person pronouns) $>$ third person $>$ kin $>$ human $>$ animate $>$ inanimate
\z

\textcite{li2012variation} and \textcite{pu1997zero} report for the radical pro-drop language Mandarin that animate subjects and objects are more prone to be null subjects than inanimate ones. Similarly, for some creole languages like Tamambo, animacy plays a significant role in subject deletion with animate subjects being more likely to be omitted \parencite{meyerhoff2009replication}. 
Finally, it was reported that subject omission in the Austronesian languages Vera'a and Teop is affected by animacy \parencite{vollmer2019radical}.

Another uncharted territory that this study explores regarding null subjects is their relation to different communicative situations and speakers' repertoires. Previous studies have shown that certain text types such as short messages on social media or on the phone facilitate subject omission and ellipsis more generally. For example, \textcite{Frick_2017} shows that in German text messages, there are more null pronouns than realized ones. This is atypical for German generally, but it is typical for this specific text genre. We are not aware of a cross-modality and cross-formality study of subject omission, which makes the investigations in this chapter pioneering and exploratory.

However, there is related previous work from the domain of nominal reference which refers to demonstrative-marked noun phrases in Greek, Russian and Turkish mono- and bilingual speakers. \textcite{Martynova2024} show that different communicative settings impact the use of definite noun phrases in different ways. For instance, Greek speakers use more double definite structures in informal settings, Turkish speakers use demonstratives more in spoken than in written mode, and Russian speakers use demonstratives more frequently in informal and oral communication settings. These findings imply that effects of communicative situations could be language-specific in the domain of null subjects too. 

The chapter is structured as follows: \sectref{sec:oezsoy:6Theory} provides the theoretical background on the different types of null subject languages this chapter deals with.
In this section, we also summarize results of previous research that discusses animacy, mode and formality as possible factors influencing when subjects are omitted.
\sectref{sec:oezsoy6RQs} introduces the research questions and the corresponding predictions for every language in our study derived from results in the literature.
\sectref{sec:oezsoy:6Methodology} describes our methodology, including the design, the participant sample, the corpus annotation scheme and the statistical analysis. 
\sectref{sec:oezsoy:6Results} provides the results reported by the models and grouped according to the different factors accounted for in the analysis, namely typology (i.e., different pro-drop types), animacy, mode and formality.
Finally, \sectref{sec:oezsoy:6Discussion} presents a discussion of our results.

\section{Null subjects} \label{sec:oezsoy:6Theory}
Languages have been grouped into different types regarding their ability to omit subjects. Leaving aside languages with so-called radical pro-drop or expletive subjects, most authors distinguish between languages readily allowing pro-drop (strict pro-drop) and non-pro-drop languages which by and large disallow the omission of subjects. The former are claimed to require rich verbal morphology \parencite{Perlmutter1971-PERDAS, holmberg2005there, RobertsHolmberg2010}. A third class of languages are referred to as partial pro-drop; subject omission in these languages is more limited than in strict pro-drop languages, and may be restricted by (morpho)syntactic, information-structural, or lexical constraints \parencite{holmberg2005there, RobertsHolmberg2010, Frascarelli2018, madariaga2018diachronic}. Lastly, a fourth class of discourse-prominent languages allow for the omission of subjects when they are topics; this process, however, follows different constraints in different
languages \parencite{taylan_Turkish_2001, Oeztuerk2008, schafer2021topic}.
Greek, Russian, Turkish, English and German are classified as different types of null subject languages, as shown in \tabref{tab:oezsoy:ProDrop:TableLang}.\footnote{For a different view on language classification according to the pro-drop feature see \citet{wals-101}.} In the following subsections, we will discuss the languages in more detail.

\begin{table}
\begin{tabular}{ll}
\lsptoprule
Language & Pro-drop type   \\ \midrule
Greek    & consistent/strict pro-drop  \\ 
Russian  & partial pro-drop (or non-pro-drop) \\ 
Turkish  & topic pro-drop   \\ 
English  & non-pro-drop   \\ 
German   & topic drop   \\ 
\lspbottomrule
\end{tabular}
\caption{Pro-drop across languages}
\label{tab:oezsoy:ProDrop:TableLang}
\end{table}

\subsection{Null subjects in English}
English is a language that does not allow null subjects. This means that subjects, both expletive and referential, are obligatorily overtly realized.
Spoken English as well as diary English have been argued to contain null subjects, but it is a matter of controversy whether these registers involve truncated structures or are instances of topic drop; see \textcite{HaegemanIhsane2001} for discussion and references. A noticeable higher use of null subjects in informal conversational English has also been noted, particularly in the first person and in negated constructions such as \textit{don't know} and \textit{can't say} \parencite{harvie1998null, nagy2011null}. Additionally, English discourse fixed expressions in the first and second person singular, such as \textit{gotcha!} and \textit{wanna go?} permit null subjects \parencite{Haegeman+2010+167+174}.

Research on heritage languages with strict pro-drop properties that come into contact with English has shown that their speakers tend to have more realized subjects compared to their monolingual counterparts, indicating transfer effects on subject realization. For instance,  \textcite{paradis2003subject} report that bilingual Spanish-English children are more likely to use overt subjects in heritage Spanish than monolingual Spanish-speaking children. \textcite{austin2017null} observe the reverse transfer effects in sequential bilingual Spanish-English children who are more proficient in their heritage language Spanish than in their majority language English.
Second-generation heritage speakers of Polish in Toronto also tend to use more realized subjects in Polish compared to first-generation and monolingual speakers of Polish in subject continuity contexts \parencite{chociej2011polish}. A study on heritage Welsh by \textcite{boon2014heritage} found that heritage speakers omit subject pronouns only with the copula verb `to be' more often than the baseline, but not with other verbs, which she attributes to ``incomplete acquisition'' of the pro-drop parameter.

\subsection{Null subjects in German}

German is classified as a topic drop language in certain registers and under certain grammatical constraints \parencite{fries1988ueber, trutkowski2016topic, schafer2021topic}. Subject drop in German is possible only in the left periphery (pre-field) and is claimed to show restrictions concerning agreement, with first person being especially prone to subject drop resulting in V1 sentences \parencite{schmitz2012null, trutkowski2016topic, schafer2021topic}. Example \ref{ex:oezsoy:germantopicdrop} shows a sentence with prototypical German topic drop. The subject pronoun \textit{ich} `I' can be omitted because it is discourse-prevalent and the finite auxiliary verb marks number and person even though in this example first and third person singular would both be permitted. The reading for the first person singular is the default interpretation without further context, but in a context with a given third person singular topic or referent, the interpretation of the empty position would shift to that.

Based on a corpus study and two acceptability rating studies, \textcite{schafer2021topic} shows that topic drop in German is influenced by a mixture of factors such as the predictability of the omitted constituent in context, verb probability and distinct verbal inflection. 

\ea \label{ex:oezsoy:germantopicdrop}
\gll \textit{pro} kann die Verabredung leider nicht einhalten\\ 
	\textit{e} can the appointment unfortunately  not keep\\
\glt `Unfortunately, (I) cannot keep the appointment.' \parencite[116]{schafer2021topic}
\z 
\il{German}

Several studies have investigated subject realization in bilingual populations with a null subject heritage language and with German as a majority language \parencite{schmitz2012null, flores2020factors, brehmer2023same}. In contrast to what has been found for English as a majority language, these studies suggest that heritage speakers in contact with German do not show transfer effects in their majority language.
\textcite{brehmer2023same} found in their study that German-Polish bilingual children and their Polish monolingual peers behaved similarly in their narratives and dialogues regarding the realization of subject pronouns.
 
\subsection{Null subjects in Greek}
Greek is a consistent pro-drop language showing rich verbal morphology, which means that null subjects can be used in all grammatical feature combinations and syntactic environments \parencite{philippaki-warburton_1987, alexiadou1998parametrizing}.
In the generative tradition, this has been associated with overt movement of the verb to the T(ense) category licensing null subjects \parencite{alexiadou1998parametrizing}.
In Greek, the default strategy is to use null subjects, except in certain situations such as topicalization, focusing, and contrastive contexts where overt subject realization is required \parencite{TsimpliEtAl2004}.

Research on null subjects in Greek bilinguals has focused on both situations where both languages allow null subjects as well as situations where the additional language does not allow null subjects. The findings show that bilingual adults generally exhibit similar results to monolingual groups. For example, a study conducted by \textcite{giannakou2022journal} on adult Greek heritage speakers in Argentina found that these participants produced overt and null subjects in Greek similarly to the monolingual group of Greek speakers. There was also no overuse of overt subjects. The only significant difference was that heritage speakers used null subjects more often in topic shift contexts. 
\textcite{DaskalakiEtAl2019} focused on the realization of subjects in heritage Greek children residing in Canada and the US who spoke English as their majority language. The results showed patterns of subject realization to be comparable to monolingual Greek-speaking children. According to \textcite{andreou2020effects}, English-Greek, German-Greek and Greek-Albanian bilingual children exhibited overspecification patterns by using full noun phrases when null subjects were the more appropriate choice. These patterns were modulated either by low language experience in the case of English-Greek and German-Greek speakers with Greek as their heritage language or by language dominance in Greek in the case of Greek-Albanian children with Greek as their majority language. Turning to adult Greek-Italian heritage speakers, \textcite{di2019age} claim that the null subject in Greek is the most employed anaphoric device across speaker groups regardless of Greek being the heritage or the majority language. In contrast, L2 Greek speakers with Spanish as their L1 show diverging patterns in subject realization in Greek depending on the manipulated context (contrastive vs. non-contrastive) and their proficiency in Greek, as reported by \textcite{margaza2020null}. Finally \textcite{torregrossa2021bilingual} tested whether Greek and Italian monolingually-raised speakers differ in the use and interpretation of null and overt subject pronouns known as anaphora resolution and found that Greek null subjects are more flexible referring to subject and object antecedent than Italian null subjects.
In sum, heritage speakers' performance is similar to monolinguals, but with some inconsistencies.

\subsection{Null subjects in Russian}
Since Russian exhibits frequent realization of subjects, but may drop them, it is sometimes classified as a partial pro-drop language or as a non-pro-drop language with frequent subject ellipsis \parencite{fehrmann2008subjects, Shushurin2018, Budennaya2019, Budennaya2020}. Realization and omission of subject pronouns are determined by various factors, especially information structural categories, the distance between an antecedent and the null subject, and lexical choices such as the type of connector in subordination contexts \parencite{FougeronBreillard2004, Bizzarri2015, madariaga2018diachronic, Madariaga2022, Pekelis2018, Budennaya2020}. These factors often interact with each other, which makes it difficult to determine the weight of each factor, especially since pro-drop is optional in many cases.

For instance, it has been claimed that being part of an A(boutness)-Topic chain facilitates omission of the subject with additional requirements concerning the distance between the null subject and the antecedent \parencite{madariaga2018diachronic, Madariaga2022}. \textcite{Madariaga2022} claims that the subject in the second sentence in \REF{russianexamplelake} may be omitted, since it continues an A-Topic chain, whereas the subject in the third sentence does not, thus has to be overtly realized:

\ea \label{russianexamplelake}
\gll My\textsubscript{i} idëm na ozero. {pro\textsubscript{i}} Nadeemsja tam vstretit' Ivana\textsubscript{j}. *(On)\textsubscript{j} nam obeščal peredat’ ključi.\\ 
we.\textsc{nom} go.1.\textsc{pl} to lake (we) hope.1.\textsc{pl} there see.\textsc{inf} Ivan.\textsc{acc} he.\textsc{nom} we.\textsc{dat} promised.\textsc{m} pass.\textsc{inf} keys.\textsc{acc}\\
\glt `We are going to the lake. We hope to see Ivan there. He has promised us to pass the keys.'
\z
\il{Russian}

For subordinate contexts, \textcite{Pekelis2018} shows that certain connectors like \textit{kak} `how', \textit{čto} `that', \textit{potomu čto} `because', and \textit{esli} `if', sometimes with and without certain correlates in the main clause, may facilitate or prohibit null subjects. For example, the conditional complementizer \textit{esli} `if' without a correlate \textit{to} `then' disallows null subjects as shown in \REF{ex:oezsoy:optionaldropRU} \parencite[155]{Shushurin2018}.  

\ea \label{ex:oezsoy:optionaldropRU}
\gll [Esli on\textsubscript{i} dejstvitel’no idet po toj ulice] *(to) pro\textsubscript{i} skoro uvidit stanciju.\\ 
	\hphantom{[}if he really walks along that street then (he) soon will.see station\\
\glt `If he is really walking that street then he’ll see the station soon.'
\z
\il{Russian}

Research on heritage Russian speakers yields inconsistent findings concerning null vs. overt pronominal subjects. Some studies have shown that overt pronominal subjects are more common in heritage Russian varieties in the US than in monolingual ones, which is claimed to indicate transfer from English \parencite{isurin2011russian, DubininaPolinsky2013, Ivanova-Sullivan2014}. In contrast, \textcite{nagy2011null}, examining first vs. second/third generation Russian heritage speakers in Toronto, found no generational differences in the frequency of null subjects, leading the authors to conclude that English had not caused any changes in this heritage Russian population.

\subsection{Null subjects in Turkish}
Turkish is characterized as a discourse prominent language allowing topic pro-drop, meaning that subjects are not always necessary in certain highly specified contexts as in \REF{ex:oezsoy:Iamdrinkingwater}, or when there is continuity of the topic \parencite{taylan_Turkish_2001, Oeztuerk2008}. In these cases, the use of overt pronouns is considered to be pragmatically marked \parencite{enc1986topic}. The most widely accepted approaches to analyzing null subject in Turkish have followed the formal analysis that null subjects can be identified through the verbal agreement morphology, which specifies person and number \parencite{enc1986topic}. However, \textcite{taylan_Turkish_2001} argues that pro-drop in Turkish is guided entirely by pragmatic constraints and, as such, Turkish should be viewed as a topic drop language, rather than a typical pro-drop language. For the purposes of the current study, we do not take a stance on the specific formal analysis of null subjects in Turkish, but it is important to note that the omission of subjects in Turkish is common. We propose the term ``topic pro-drop'' to unify existing approaches and signify that Turkish is not a typical pro-drop language like Greek or Italian.

\ea \label{ex:oezsoy:Iamdrinkingwater}
\gll (Ben) su iç-iyor-um.\\ 
	{\textit{pro} (I)} water drink-\textsc{prog}-1\textsc{sg}\\
\glt `I am drinking water.'
\z
\il{Turkish}

Studies on Turkish heritage speakers have produced inconclusive results with regard to subject realization in heritage Turkish varieties. Some studies, such as those conducted by \textcite{Haznedar2010} and \textcite{Koc2016} on heritage Turkish in contact with majority English, found that pronominal subjects are more frequently realized in heritage varieties compared to the monolingual standard. However, other studies, such as those by \textcite{AzarEtAl2020} and \textcite{dikilitacsacquisition}, found contrasting results. 
\textcite{AzarEtAl2020} observed a slightly higher use of overt pronouns in the repertoire of heritage speakers in contact with Dutch, but there were no significant differences between the groups. They argued that the groups align in their frequency of realization of overt subjects and referents overall. Similarly, a longitudinal case study by \textcite{dikilitacsacquisition} of a Turkish-English bilingual child revealed no evidence of cross-linguistic influence from majority English, as the use of null and overt subjects did not deviate from those of monolingual Turkish peers. These divergent findings may be attributed to several factors, including sociolinguistic background variables that guide the rate of overt subject use, differences in the communities in which heritage speakers acquire the language, and differences in methodologies, including statistical tests and sample sizes.

\section{Research questions and hypotheses} \label{sec:oezsoy6RQs}
Our review of null subjects in the three languages under investigation demonstrates that there are clear differences in the type of pro-drop languages at hand. While Greek requires null subjects in most contexts which are licensed by agreement features, this is more limited in Russian and Turkish. We also outlined that the two majority contact languages in this study, German and English, are both non-pro-drop languages which only allow null subjects in very restricted contexts and to different degrees. The participants in this study exhibit interesting bilingual language combinations. These combinations, along with the unique cross-linguistic comparisons we can draw from this set of languages, prompt several lines of research. These research lines also capture grammatical and extra-linguistic factors, such as animacy and different communicative situations, which we have introduced. Drawing from the current body of research, we formulate the subsequent research questions and corresponding hypotheses:



\begin{enumerate}
\item[RQ1:] Is the way in which subjects are expressed in heritage Greek, Russian, and Turkish  similar to the way in which monolingual speakers express subjects?
\item[H1:] We anticipate that heritage speakers will show a higher use of overtly realized subjects.
    
There are several plausible explanations for a higher use of overt subjects by heritage speakers. One reason might be the effect of cross-linguistic influence from the dominant languages German and English. Bilinguals experience interaction between their language systems, according to the Non-Autonomous Version of the Separate Development Hypothesis (as described by \cite{hulk2000bilingual, muller2001crosslinguistic, tracy2000language, tracy2014mehrsprachigkeit}). Throughout a speaker's life, transfer from the dominant language may be used to fill in the gaps caused by reduced input in the heritage language. Heritage speakers may adopt subject realization strategies from their dominant language for their heritage language \parencite{gawlitzek1996bilingual, dopke2000generation}. At the same time, research has shown that heritage speakers tend to be more explicit, which could be another factor that leads them to use overt subjects more frequently than monolinguals \parencite{Pashkova2020}.

    
\item[RQ2:] Are there any similarities or differences in the patterns of subject omission observed among various groups of heritage speakers and, if so, what factors may account for these patterns?
%\sloppy
\item[H2.1:] It can be assumed that these different pro-drop types allow for varying levels of language-internal change or cross-linguistic influence from the majority languages. Therefore, we might anticipate higher use of overtly realized subjects from heritage Greek to heritage Russian to heritage Turkish. However, the design only includes two majority languages that are both non-pro-drop languages. This makes it unsuitable to test for a true effect of cross-linguistic influence as this would require a comparison with a pro-drop majority language. So this sub-hypothesis is only falsifiable outside of the framework of this study.
     
Apart from cross-linguistic influence, the typological differences might also indicate how stable or dynamic the pro-drop system of a language is internally, regardless of the majority language influence. A strict pro-drop system as in Greek might lead to a more stable use of null subjects in the heritage grammar too. Following the same logic, pro-drop systems that are guided more by pragmatic and situational factors such as those of Russian and Turkish might be open to more dynamicity in the use of null subjects in the heritage language.
\item[H2.2:] Alternatively, in the context of heritage languages, we could argue that there may be converging developments regardless of the pro-drop type of the languages. Since the (non)realization of pronominal subjects also involves the interface with external linguistic domains (information structure and discourse organization), we may expect uniform transfer effects from majority languages following the Interface Hypothesis \parencite{sorace2011pinning}. We expect to find more overtly realized subjects in heritage varieties for several reasons. First, there is great variability in linguistic phenomena at the syntax-discourse interface \parencite{TsimpliEtAl2004, sorace2011pinning}. Additionally, as the morphological repertoire weakens in heritage languages \parencite{polinsky2018heritage} and this might affect agreement marking which licenses pro-drop, we expect a higher use of overtly realized subjects across all heritage varieties of Greek, Russian and Turkish. Furthermore, some heritage speakers with limited language exposure may not be able to adequately mark verb agreement, leading them to avoid null subjects and use more overt subjects instead \parencite{isurin2011russian, DaskalakiEtAl2019}. Therefore, individual variance within each group needs to be considered to avoid false positives resulting from the analysis of aggregated frequencies \parencite{BAAYEN2008390, winter2011pseudoreplication, brezina2014significant, winter2019statistics}.

\pagebreak
\item[RQ3:] Do Greek, Russian and Turkish heritage speakers’ expressions of null subjects in majority German align with monolingual German speakers’ productions?

\item[H3:] German belongs to the non-pro-drop languages and typologically differs from the typology of the three heritage varieties, namely Greek, Russian, and Turkish, and thus we expect cross-linguistic influence with extended use of null subjects in majority German productions. Additionally, the heritage speakers might also use the reverse strategy and produce more standard-like and formal expressions of subject use in German to display that they are proficient users of their majority language German. This would lead to less topic drop and more overt subjects.

\sloppy
\item[RQ4:] Do intra-linguistic factors, such as animacy, or extra-linguistic factors, such as formality (formal vs. informal) and mode (written vs. spoken), have an impact on subject realization in both monolingual and heritage varieties?
    
\item[H4:] Based on the literature discussed earlier, we expect null subjects to be preferred for animate subjects. 
However, in light of a lack of prior research on the impact of the external factors formality and mode on null subjects for Greek and Turkish and conflicting results for Russian, we approach these factors in an exploratory way.
\end{enumerate}

\section{Methodology} \label{sec:oezsoy:6Methodology}
Following the \textit{Language Situations} approach \parencite{wiese2020language} described in \sectref{sec:oezsoy:ruegcorpus} below, we collected ecologically-valid semi-spontaneous data through narration tasks to identify emerging trends in the language systems.
The sections below provide detailed information on the composition of speakers in the study, corpus annotation and queries as well as on the statistical analysis applied.

\subsection{Participants}
\begin{sloppypar}
\tabref{tab:oezsoy:ProDrop:TableParts} presents the overall information about the  number of participants grouped by country of elicitation and the number of tokens in the different subcorpora. Tokens are used in the way they are defined by the TreeTagger tokenization script \parencite{schmid2013probabilistic}. 
All participants were recruited from urban areas to minimize the role of dialect. Urban areas facilitate regional dialect levelling and were therefore preferred \parencite{britain2010supralocal}. 
Monolingual\footnote{We use monolingual as a shorthand to refer to monolingually-raised participants who have not acquired another language other than their first language in childhood (and later potentially learned other languages in school settings) and, crucially, do not use another language than their first language in everyday life.}  participants were recruited from Athens, Greece; St. Petersburg, Russia; and İzmir and Eskişehir, Turkey. 
Bilingual participants were recruited from the United States and Germany. 
Heritage speakers in the US lived in the greater Washington DC area (including Virginia and Maryland), Chicago, and the greater New York City area (including New Jersey).
Heritage speakers in Germany were recruited from the Berlin and Brandenburg urban area.
\end{sloppypar}


\begin{table}
\begin{tabular}{llcc}
\lsptoprule
Country of elicitation & Group & N & \multicolumn{1}{c}{Tokens} \\ \midrule
Greece & monolinguals & 64 & 27,931 \\
Russia & monolinguals & 66 & 25,930 \\
Turkey & monolinguals & 64 & 20,947 \\ \midrule
Germany & \begin{tabular}[c]{@{}l@{}}heritage Greek \\ heritage  Russian\\ heritage Turkish\end{tabular} & \begin{tabular}[c]{@{}l@{}}48\\ 61\\ 64\end{tabular} & \begin{tabular}[c]{@{}r@{}}19,782\\  32,882\\  23,722\end{tabular} \\ \midrule
USA & \begin{tabular}[c]{@{}l@{}}heritage Greek\\ heritage  Russian\\ heritage Turkish\end{tabular} & \begin{tabular}[c]{@{}l@{}}63\\ 60\\ 58\end{tabular} & \begin{tabular}[c]{@{}r@{}}18,302\\  29,214\\ 18,502\end{tabular} \\ 
\lspbottomrule
\end{tabular}
\caption{Participant and corpus metadata based on the corpus version 1.0}
   \label{tab:oezsoy:ProDrop:TableParts}
\end{table}

Candidates were invited to take part in the study if they were a) either mono- or bilingual speakers, b) born and raised in the country of the majority language or, in case of bilingual speakers, moved there before the age of 48 months,\footnote{In exceptional cases participants who moved to the hosting country before the age of six years were admitted to the study.} c) bilinguals should use their heritage language with family and friends regularly, d) not diagnosed with hearing or speech disorders, e) exposed to the majority language (English or German) before the age of five.

In addition, for the written production in Greek and Russian, participants were given the option to use Latin script if they were not able to write in Greek or Cyrillic scripts. 
Prior to the study, participants were informed about their rights, data protection, and the procedure, and asked to sign a consent form in the majority language of the country where the elicitation took place. 
In cases where the participant was a minor, one of their parents or legal guardians was asked to sign the consent form.


Several considerations were taken to ensure demographic comparability between the participant groups. Two age groups of participants were included: adolescents (aged 14--18) and adults (aged 22--35). Adolescent participants were either currently attending school or had recently graduated. The two different age groups are not relevant for the present study as no effects of age on null subjects were expected. An additional criterion excluded bilingual candidates with extensive formal education in the heritage language, such as those who attended bilingual primary or secondary schools.

From a sociolinguistic and demographic standpoint, Greek, Russian, and Turkish migrant communities in the US and Germany exhibit similar characteristics in their language-related behaviors and experiences within the host countries. These communities often establish tight-knit groups, acting as crucial centers for cultural and linguistic preservation, actively contributing to language maintenance initiatives. Nevertheless, despite endeavors to preserve and pass on linguistic heritage to the younger generations, the heritage language of second, third and fourth generation migrants is observed to undergo substantial transformations \parencite{ benmamoun2013heritage, polinsky2018heritage}.

\subsection{RUEG corpus} \label{sec:oezsoy:ruegcorpus}
The approach employed in this research is a modification of the setup introduced by \textcite{wiese2020language}, known as the \textit{Language situations} paradigm. This paradigm facilitates the generation of semi-spontaneous data, offering comparable naturalistic information in both oral and written forms, as well as in formal and informal situations. In the elicitation process, participants were presented with a short video depicting a fictional minor car accident, and their task was to narrate the incident as if they had witnessed it, addressing either a close friend or a police officer.
Consequently, participants engaged in four distinct communication scenarios during a single session. Heritage speakers took part in two sessions separated by at least three days: one in the majority language and one in their heritage language. Monolingually-raised participants only participated in a single session conducted in the majority language of the respective country.
To examine the impact of formality and mode on narrations, we simulated formal-spoken, formal-written, informal spoken, and informal-written settings. The formal part of the elicitation occurred in an office setting, where the elicitor and participant faced each other. The elicitor wore formal attire, and used standardized language and honorifics. Spoken narration involved leaving a voice message on the police department's answering machine, while the written task required typing a witness report on a ``police laptop''.
For the informal part, another elicitor, stylized as talkative and casually dressed, engaged participants in casual conversation before instructing them to narrate the video's contents via a voice message on WhatsApp to a close friend. The written task involved sending a text message about the accident on WhatsApp to the same friend. The entire session was audiorecorded and the data were pseudonymized.
Elicitation orders for the communicative situations were balanced.
Once transcribed and annotated, the data were released as the RUEG Corpus \parencite{RUEGcorpus2024}, developed within the Research Unit \textit{Emerging Grammars in Language Contact Situations} (RUEG) funded by the German Research Foundation (\url{https://hu.berlin/rueg}). The multilevel annotated RUEG corpus is accessible through the ANNIS interface \parencite{ANNIS3}, comprising audio for spoken data and visualization options across six sub-corpora for English, German, Greek, Russian, Turkish, and Kurmanji. Additional sub-corpora feature special annotations such as aspect and tense. The present study utilized data from the RUEG corpus \parencite{RUEGcorpus2024} including narrations from a total of 548 speakers. This encompassed both monolingually-raised and heritage speakers of Greek, Russian, and Turkish. 

\subsection{Corpus annotation} \label{sec:oezsoy:Corpus}
The RUEG subcorpora for Greek, Russian and Turkish were manually annotated in EXMARaLDA \parencite{Exmaralda} with respect to subject realizations on the three following levels: 

\begin{itemize}
    \item \textit{denotation and animacy of the referents} derived from the context (the full list can be found at \url{https://osf.io/25tw6})
    \item \textit{syntactic realization} (overt vs. null subjects)
    \item \textit{expectedness} of subject (i.e., whether an overt subject is expected in the given context using the tags \textit{yes}, \textit{no}, \textit{not sure})\footnote{The judgments regarding the \textit{expectedness} of subjects were done by at least two native or near-native speakers of each language and cases where the annotators did not agree were further discussed.}
\end{itemize}

\subsubsection{Animacy of referents}

Subjects were identified in relation to a finite verb corresponding to the broad syntactic notion of subjecthood \parencite{mccloskey1997subjecthood}. The category of subject includes noun phrases such as lexical nouns, pronominal expressions, and demonstratives.

Regarding the denotation of the subjects, we considered human referents like \textit{man}, \textit{woman}, \textit{driver}, and animals like \textit{dog} and personal pronouns referring to them as animate subjects while physical objects like \textit{ball}, \textit{cars}, \textit{groceries} were considered as inanimate subjects.
Besides the individual referents, we found some conjoint referents in our data as well. 
We classified conjoints referents that included at least one animate referent as animate \parencite{adamson2021interpretability} as in the example below:

\ea 
\gll para s koljaskoj\\ 
	couple with stroller\\
\glt `A/the couple with a/the stroller.'
\z
\il{Russian}

\subsubsection{Overtness of subjects}

The following explains how the layers ``syntactic realization'' and ``expectedness'' were operationalized to assess the expression of subject reference in this study. The concept of expectedness was introduced in order to make the present analysis cross-linguistically comparable. Expectedness captures the factors that license and facilitate null subjects which are highly language-dependent as presented in \sectref{sec:oezsoy:6Theory}. By creating the concept of expectedness, we turn the variety of grammatical factors into a binary choice (Yes, an overt subject was expected according to the grammatical rules of the language; or No, an overt subject was not expected according to those extensive criteria). Some details from the guidelines are listed below. The elaborate annotation guidelines that explain the additional annotation layers for this study in depth are available in a PDF-file at \url{https://osf.io/pvmx5}.

Our working concept for expectedness differs from the widely used notion of accessibility \parencite{chafe1987cognitive, gundel1993cognitive, ariel2001accessibility, arnold2010speakers}. As \textcite{allen2015role} point out, accessibility, generally, is seen as a cognitive proxy to describe how likely a speaker is to produce something on the spectrum between a full form like a noun phrase and a null form like subject drop. Accessibility is one of the factors that we take into account when we code for expectedness which takes into account many additional factors such as topicality and language-specific grammatical factors. Topicality adds to a threeway interaction with accessibility and expectedness. Additionally, as our descriptions of pro-drop in Greek, Russian and Turkish illustrate, there seem to be stricter grammatical constrains between languages that determine whether a subject may be omitted or overt. This is true regardless of accessibility as Turkish speakers can, for example, realize subjects even when they are accessible, but this is not possible in Greek.

\subsubsection{Language specific aspects and exclusions}

In all languages of investigation, we excluded dative subjects, subjects in imperatives, rhetorical questions, interrogative as well as exclamative sentences, addressings, and impersonal constructions from the annotation.

For Greek, fixed expressions like \textit{ksero go} `I don't know' were also excluded as the subject is always expected post-verbally and this could influence the results for overt subjects. 
21 such cases were observed in the Greek data.

For the annotation of null subjects in Russian, various factors mentioned in \sectref{sec:oezsoy:6Theory} were taken into account. 
Particularly, \textit{communicative situation, discourse, distance between antecedent and subject, lexical markers, information structural categories} were considered. 
For instance, in an informal-spoken communicative situation, as a tendency, we would expect less overt subject realization in general. More specific situations like contrast or topic change were marked as expecting subjects.
Besides, subjects within discourse markers like \textit{(ty) ponimaeš'} `(you) understand' were annotated as not expected ones. 

Since Turkish is a topic pro-drop language, as specified earlier, the available number and person marking on the finite verb allow the subject to be omitted. In many instances, speakers also accept realized subjects in non-requiring contexts and positions as grammatical even though these utterances might be perceived as slightly redundant when speakers judge them. However, our data show that these kinds of overt subject realizations are common in different mono- and bilingual speaker's repertoires.
Additionally, Turkish requires subjects to be overtly realized to mark a new or later reoccurring topic, including contrast contexts which also can require overt subjects. Example (\ref{ex:oezsoy:temizleme}) demonstrates a contrastive sentence where the subjects of the main and the subordinate clause are both overtly realized:

\ea \label{ex:oezsoy:temizleme}
\gll ben evi temizlerken sen boş boş film izliyorsun.\\ 
	I house clean you empty empty movie watch\\
\glt `I'm cleaning the house and you're idly watching a movie.'
\z
\il{Turkish}

One exception in Turkish, however, is the first person pronoun \textit{ben} in narratives. 
When a finite verb is marked for 1SG, it is unambiguously clear that it refers to the narrator. 
In this case, it is often not necessary to overtly realize the pronoun, except when it is in a contrastive or emphatic use.

Additionally, 232 data points that were tagged as syntactically realized or not realized had to be excluded because their \textit{expectedness} status was not clear. 
This means that the formal grammars of the language do not allow a clear prediction as to whether a subject would be expected or not in a given context. 
Compared to the 31,539 data points that we accounted for, the number of excluded cases is marginal, i.e., $<$1\%. The exact number of data exclusion based on this case is as follows: 

\begin{table}
\begin{tabular}{lrr}
  \lsptoprule
 Language & N overt subjects & N null subjects \\ 
  \midrule
  Greek & 3 & 1 \\ 
  Russian & 115 & 51 \\ 
  Turkish & 32 & 30 \\ 
  \lspbottomrule
\end{tabular}
\caption{Excluded data points grouped by language and syntactic condition.}
\label{tab:oezsoy:ProDrop:ExcludedTable}
\end{table}

\subsection{Statistical Analysis} \label{sec:oezsoy:Stats}
To draw cross-linguistic (Greek, Russian and Turkish heritage speaker groups) and also across-variation (monolingual vs. heritage speakers) comparisons, we ran mixed-effects models in R using the lme4 package \parencite{batesetal2015}. As independent variables, we included \textit{language} (Greek, Russian, Turkish), \textit{country} (Germany, Greece, Russia, Turkey, US), \textit{expectedness}, \textit{animacy}, \textit{formality} (formal vs. informal), and \textit{mode} (spoken vs. written). Further, to capture the individual speaker variability, we specified random effects by participant. In R notation, the model for each language looked like follows:
 
\begin{lstlisting}[language=R,keywordstyle=\ttfamily]
glmer(realization ~ country + language + expectedness + mode + formality + animacy + (1|participant), data = ProdropModel, family = binomial, control = glmerControl(calc.derivs=FALSE))
\end{lstlisting}

In words, this code uses the \texttt{lme4} package function \texttt{glmer} to create a generalized linear mixed model of the binomial distribution family. The formula models Subject realization with its levels Null and Overt as a function of language group (heritage USA, heritage Germany, monolingual majority homeland), expectedness according to our annotation scheme (levels yes and no), mode with its levels spoken and written, formality with its levels formal and informal, and animacy with its levels animate and inanimate. 

The normalization of data points is conducted according to the number of finite verbs. 
\tabref{tab:oezsoy:ProDrop:NormTable} reports the number of finite verb tokens per group.

\begin{table}
\begin{tabular}{llc}
  \lsptoprule
 Country of elicitation & Group & N finite verbs \\ 
  \midrule
  Greece & monolinguals & 4,954 \\ 
  Russia & monolinguals & 3,965 \\ 
  Turkey & monolinguals & 4,609 \\ \midrule
   & heritage Greek & 3,494 \\ 
  Germany & heritage Russian & 4,624 \\ 
   & heritage Turkish & 4,986 \\ \midrule
 & heritage Greek & 3,471 \\ 
 USA & heritage Russian & 4,342 \\ 
  & heritage Turkish & 4,257 \\
   \lspbottomrule
\end{tabular}
\caption{Number of finite verbs per group.}
\label{tab:oezsoy:ProDrop:NormTable}
\end{table}

\section{Results} \label{sec:oezsoy:6Results}

This section provides the results of the statistical analysis on subject realization and subject drop in different conditions in Greek, Russian, and Turkish. 
To describe the results from different perspectives, we organize this section in the following subsections: descriptive results (\sectref{sec:oezsoy:descriptiveresults}) and intra- and extralinguistic factors (\sectref{sec:oezsoy:intraandextalinguisticfactors}) including formality, mode and animacy. 
Before we address those perspectives one by one, we present general results and model outcomes in the following. 

First, we show how the heritage and monolinguals speakers of Greek, Russian and Turkish use subjects along the two variables which both have two levels, namely \textit{realization} (null vs. overt) and \textit{expectedness} (YES = expected, NO = unexpected). 

\begin{figure}[ht]
    \centering
    \includegraphics[width=\textwidth]{figures/Ch6_ProdropRaw_paper.png}
    \caption{Normalized number of occurrences for four different combinations/categorizations.}
    \label{fig:oezsoy:ProDropRaw}
\end{figure}

\figref{fig:oezsoy:ProDropRaw} illustrates all three languages and all participant groups for a better comparison across the same scale. 
The figure represents a faceted plot with a grid for each of the nine speaker groups sorted by different countries of elicitation from left to right (Germany, respective countries of the monolingual varieties, US) and three languages from top to bottom (Greek, Russian, Turkish). 
Each row shows one language, and each column reflects one country. 
For instance, the first grid of the first column represents heritage speakers of Greek in Germany, the second grid in the first column represents heritage speakers of Russian in Germany and the third row in the first column represents heritage speakers of Turkish in Germany. 
The second column represents monolingual speakers in Greece, Russia and Turkey, respectively.
The third column shows heritage speakers of Greek, Russian and Turkish residing in the US, respectively.

Each subplot shows four conditions that are the result of combination of \textit{realization} (null vs. overt) and \textit{expectedness} (YES = expected, NO = unexpected) variables. We list and illustrate each of them below.
From left to right, each bar represents the following conditions:
\pagebreak
\begin{itemize}
    \item \textit{NO null}: overt subject is unexpected and it was null
    \item \textit{NO overt}: overt subject is unexpected, but it was realized
    \item \textit{YES null}: overt subject is expected, but it was null
    \item \textit{YES overt}: overt subject is expected and it was realized
\end{itemize}

The tag ``NO overt'' applies when an overt subject was observed even though a null subject was expected. In pro-drop languages, this would typically be the case when a referent was just introduced and is just maintained in the discourse. An overt subject in this position would be non-canonical given that it is a pro-drop language. This would be the case if the sentence in Example \REF{ex:oezsoy:NO overt} is a continuation of \REF{ex:oezsoy:YES overt}.

\ea \label{ex:oezsoy:NO overt}
\gll Köpek topu yakalıyor.\\ 
	dog ball catches\\
\glt `The dog catches the ball.'
\z
\il{Turkish}

 The combination ``NO null'' was annotated when a canonical case of null subject use was observed. In the given context of the example, this would be the case if the dog is maintained as a referent. We illustrate this in the canonical continuation in \REF{ex:oezsoy:NO null} as a contrast to non-canonical continuation in \REF{ex:oezsoy:NO overt}.   

 \ea \label{ex:oezsoy:NO null}
\gll Topu yakalıyor.\\ 
	ball catches\\
\glt `(The dog) catches the ball.'
\z
\il{Turkish}

The combination ``YES overt'' is used to tag canonically overtly expressed subjects in a language. In pro-drop languages, for example, subjects are often expressed overtly when they are introduced for the first time in discourse such as in Example \REF{ex:oezsoy:YES overt}. In this case, the dog is introduced in the narrative. We will use this context for all tag combinations.

\ea \label{ex:oezsoy:YES overt}
\gll Ondan sonra bir köpek topa doğru koşuyor.\\ 
	after that a dog ball towards runs\\
\glt `Then, a dog runs towards the ball.'
\z
\il{Turkish}

Finally, there is the less frequent combination ``YES null'' where a subject is null even though an overt form was expected. Given our example context, this would be the case if a new referent is introduced in the story but it is not overtly realized as a subject as in Example \REF{ex:oezsoy:YES null}.
    
\ea \label{ex:oezsoy:YES null}
\gll Köpeğe kızıyor.\\ 
	dog rant\\
\glt `(S)he talks angrily to the dog.'
\z
\il{Turkish}

The \textit{NO null} and \textit{YES overt} conditions represent the canonical way of subject realization, whereas \textit{NO overt} and \textit{YES null} conditions represent the non-canonical subject realization.

The first view on \figref{fig:oezsoy:ProDropRaw} provides the impression that on the one hand, Greek, Russian and Turkish behave differently from each other, and on the other hand, heritage and monolingual speakers of these languages seem to behave similarly to each other.

To address these patterns in a more interpretable way, we present the results for the generalized linear mixed-effects regression models for each language. 
We fitted three logistic mixed-effects models to predict syntactic realization of subjects (overt vs. null) with \textit{Country}, \textit{Expectedness}, \textit{Mode}, \textit{Formality} and \textit{Animacy} as fixed effects. 
The model included \textit{Participant} as random effect.

\subsection{Descriptive results grouped by language} \label{sec:oezsoy:descriptiveresults}
This subsection presents the descriptive results from heritage Greek, Russian, Turkish and majority German.

\subsubsection{Null subjects in Greek, Russian and Turkish} 
We first asked whether null subjects in the three languages work in the same way. Starting the exploration of our results based on the typological differences on pro-drop in the three languages under investigation we confirm the claim that Greek clearly belongs to strict pro-drop languages while Russian and Turkish do not. As seen in \figref{fig:oezsoy:ProDropRaw}, Greek preserves pro-drop in all varieties explored in this study while Russian and Turkish align in the sense that overt subjects are more prominent in all varieties. Russian speakers across all groups realize subjects to a similar extent in the condition where the subject realization is expected. Specifically, the amount of overt subjects is higher than the amount of null subjects. The same observation holds for Turkish speakers.

\subsubsection{Topic drop in majority German}
We then asked whether each group of heritage speakers produced topic drop in the same way in their majority language, German. An overview of the results from the German data is shown in \figref{fig:oezsoy:ProDropGerman}. The black dots that represent the means almost perfectly mirror each other in all subplots. This points to no differences between the groups with regard to subject realization strategies in the majority language German. We confirmed in a Bayesian model that we report extensively in \textcite{ozsoyGermanTopicDrop} that there are no meaningful differences between the groups. The only strong and important effect of topic drop that we find is based on mode. Specifically, we find that utterances in spoken language are much more likely to include null subjects than utterances in written language.

\begin{figure}[ht]
    \centering
    \includegraphics[width=\textwidth]{figures/Ch6_ProdropNorm.png}
    \caption{Normalized number of occurrences for four different combinations/categorizations in German data.}
    \label{fig:oezsoy:ProDropGerman}
\end{figure}

    \subsection{Intra- and extralinguistic factors} \label{sec:oezsoy:intraandextalinguisticfactors}
    In order to verify which factors affect the realization of subjects we conducted a regression analysis and the results are shown below.

\subsubsection{Country} \label{sec:oezsoy:Country}
One of the main aims of this study is the typological comparison between Greek, Russian, and Turkish as heritage languages (and partly German as a majority language). Since in our controlled design and study the participant groups live in similar sociodemographic and linguistic environments, this direct typological comparison can be drawn. For each language, there are three levels -- Germany and the USA as the host countries of the heritage language, and the respective homeland countries Greece, Russia and Turkey. We can report the effects of each of these levels per language because we sum-coded these variables which allows us to compare all groups equally.

For Greek, we observe that the effect of Country is statistically significant at the USA level ($\text{mean}=0.616$, $\text{SD}=0.205$, $p<0.01$) and non-significant at the Germany ($\text{mean}=0.348$, $\text{SD}=0.213$, $p>0.05$) and Greece ($\text{mean}=0.268$, $\text{SD}=0.194$, $p>0.05$) levels. The directionality of the effect is similar in all groups. This indicates that heritage speakers of Greek in the USA show a higher use of overt subjects compared to the groups in the other countries.

For Russian, the effect of Country is significant at the Germany ($\text{mean}=0.472$, $\text{SD}=0.099$, $p<0.01$) and Russia levels ($\text{mean}=-0.599$, $\text{SD}=0.101$, $p<0.01$), and non-significant at the USA level ($\text{mean}=0.127$, $\text{SD}=0.101$, $p>0.05$). The directionality of the effect is positive for the heritage groups and negative for the Russia group. This indicates that heritage speakers of Russian are more likely to produce overt subjects compared to monolingual speakers who show a reverse effect meaning that they drop subjects more frequently.

For Turkish, the model estimates a significant effect at the Germany ($\text{mean}=0.328$, $\text{SD}=0.115$, $p<0.001$) level and non-significant effects at the Turkey ($\text{mean}=-0.154$, $\text{SD}=0.114$, $p>0.05$) and USA ($\text{mean}=-0.174$, $\text{SD}=0.119$, $p>0.05$) levels. The directionality of the effect is positive for the Germany level and reversed for the Turkey and USA levels. This means that while heritage speakers of Turkish in Germany are likely to produce more overt subjects, this effect is not present for the other groups.

\subsubsection{Formality and mode} \label{sec:oezsoy:Registerandmodality}
To account for a broader repertoire of naturalistic language use, our study tests the effects of Mode and Formality which we utilize to capture register variation. For Greek, the effects are non-significant and non-meaningful (Mode: $\text{mean}=-0.048$, $\text{SD}=0.098$, $p>0.05$; Formality: $\text{mean}=0.118$, $\text{SD}=0.097$, $p>0.05$). For Russian (Mode: $\text{mean}=0.376$, $\text{SD}=0.048$, $p<0.001$; Formality: $\text{mean}=0.202$, $\text{SD}=0.045$, $p<0.001$) and Turkish (Mode: $\text{mean}=-0.114$, $\text{SD}=0.054$, $p<0.001$; Formality: $\text{mean}=0.186$, $\text{SD}=0.054$, $p<0.001$), the effects for Mode and Formality are all significant, and they partly overlap in their directionality. The effect for Formality indicates that Russian and Turkish speakers are more likely to overtly realize subjects in formal contexts compared to informal ones. The Mode effect goes in different directions for the two languages. Russian speakers are more likely to overtly realize subjects in spoken mode and drop more in written mode. On the contrary, Turkish speakers drop subjects more in spoken mode and overtly realize the subject more when writing.

\subsubsection{Animacy} \label{sec:oezsoy:Animacy}
In addition to situational and discourse-dependent factors such as Mode and Formality, our study tests the grammatical factor of Animacy for the realization of subjects. Animacy is statistically significant for all the languages under investigation (Greek: $\text{mean}=0.398$, $\text{SD}=0.201$, $p<0.01$; Russian: $\text{mean}=0.361$, $\text{SD}=0.099$, $p<0.01$; Turkish: $\text{mean}=0.254$, $\text{SD}=0.116$, $p<0.01$). The effect directionality indicates that inanimate subjects are more likely to be overtly realized and animate subjects are more likely to be null.


%%%%%%%%%%%%%%%%%%%%%%%%%%%%%%%%%%%%
\section{Discussion} \label{sec:oezsoy:6Discussion}
Whether bilinguals drop subjects to similar degrees as other groups has sparked a lively debate in bilingualism  research \parencite{TsimpliEtAl2004, ArgyriSorace2007, Haznedar2010, Gagarina2012, DubininaPolinsky2013, Ivanova-Sullivan2014, Ivanova-Sullivan2015, Koc2016, AzarEtAl2020}. A common observation has been that bilinguals tend to use a higher proportion of overt to null subjects than do monolingual speakers. Our study provides a comparative approach to three typologically different languages concerning the use of null subjects. We conducted an exploration in heritage and monolingual varieties of Greek, Russian, and Turkish to investigate intra- and inter-group convergence or divergence regarding pro-drop. The findings suggest that different patterns are observed depending on the type of pro-drop language in use. In Greek as a strict-pro-drop language, no differences are found between speakers in Germany and Greece, but speakers in the US produced more overt subjects. A more frequent realization of subjects is also observed in heritage Russian and heritage Turkish in Germany, which are partial pro-drop and topic pro-drop languages, respectively. No significant differences were observed in heritage Russian and heritage Turkish in the US. In addition, we controlled for other factors that might affect the production of overt or null subjects, and the most significant one across the languages turned out to be animacy. Furthermore, formality and mode appear to have a significant impact on the frequency of overt subjects in Russian and Turkish as well.

From early on, cross-linguistic influence has been discussed as a key facilitator of more overt subjects in heritage speakers of pro-drop languages with non-pro-drop majority languages \parencite{gawlitzek1996bilingual, dopke2000generation}. However, the literature for Greek as a pro-drop heritage language frequently showed no differences between bilinguals and monolinguals \parencite{di2019age, DaskalakiEtAl2019, andreou2020effects, torregrossa2021bilingual, giannakou2022journal}. Our study adds to this literature with the Greek heritage group in Germany producing similar levels of null subjects as monolingually-raised Greek speakers. Heritage speakers of Greek in the US in our sample align with the general trend in the literature towards a higher use of overt subjects (though this effect was not confirmed in our Bayesian re-analysis of the data in \textcite{ozsoyetal} which yielded no differences between the Greek groups). Similarly, Russian heritage speaker groups in the US and in Germany differ significantly from the monolingual variety regarding the frequency of overt subjects. This supports findings by \textcite{isurin2011russian, DubininaPolinsky2013} and contradict \textcite{nagy2011null} who found no influence of majority English on subject drop in Russian. As for Turkish, the findings in the literature have been split with \textcite{Haznedar2010} and \textcite{Koc2016} finding a higher use in overt subjects whereas \textcite{AzarEtAl2020, dikilitacsacquisition} do not observe any change in heritage Turkishes. This is particularly interesting since \textcite{Haznedar2010, Koc2016} and \textcite{dikilitacsacquisition} also investigated heritage Turkish speakers with English as a majority language and found mixed results. We find a higher use in Turkish in Germany but not in the US which confirms the mixed nature of the results. 

To contextualize these results better, we draw on an explanation that goes beyond cross-linguistic influence and takes the different types of null subject languages in this study, namely, strict pro-drop Greek, partial pro-drop Russian and topic pro-drop Turkish into account \parencite{philippaki-warburton_1987, Oeztuerk2008, Shushurin2018}. According to this perspective, we would find a continuum from more strict to less strict pro-drop languages. On the strict side of the continuum, we expect the pro-drop system to behave more stably under language contact influence in bilinguals. The more we move toward the less strict system, the more dynamicity in the pro-drop system we would expect. Dynamicity could be reflected as a higher use of overt subjects in bilingual speakers. Since the design of our study did not include pro-drop majority languages, we cannot infer whether the dynamics that we observed here are due to language internal change in bilingual grammars or due to cross-linguistic influence from the majority languages. We also were not able to attribute change to factors that we did not control for, such as sociolinguistic background factors.
 
There are fewer studies that investigate the dynamics in subject use from a perspective that goes from the heritage to the majority language. In this case, we could have expected bi-directional cross-linguistic influence given the interactions in the systems of bilingual grammars \parencite{zhou2021bidirectional}. However, this does not seem to be the case as our German topic drop results are in line with previous studies that found stability in heritage bilingual's majority German grammars \parencite{schmitz2012null, flores2020factors, brehmer2023same}. This confirms a general trend, in that more often the status of the language, majority or heritage, seems to play a role for linguistic innovation and change \parencite{wiese2022heritage}.

Besides the effects of bilingualism, we explored different extra- and intra\hyp linguistic factors which could influence the realization of pronominal subjects. Outside of the bilingualism literature, animacy has been extensively investigated as an intra\hyp linguistic factor influencing subject omission, with the clear finding that animate subjects are
omitted more often than inanimate subjects \parencite{pu1997zero, meyerhoff2009replication, li2012variation, vollmer2019radical}. However, to our knowledge, the present study is the first systematic investigation regarding the relation between animacy and its impact on the realization of subjects in (heritage) bilinguals. Our findings regarding animacy in Greek, Russian and Turkish indicate that this factor has a significant effect on subject realization. According to the animacy hierarchies  given in \textcite{corbett2000number}, animate subjects are more salient and thus can be easily omitted. Generally, the issue of animacy as a factor is related and overlapping with other concepts that have been found to be meaningful for the realization of subjects such as topicality or cognitive accessibility \parencite{lee2007contrastive, allen2015role, laleko2017silence}. All of these concepts relate to how present the subject referent is in the speaker's mind or the general context.

As for the contexts of formality and mode in our methodology, we explored whether these extra-linguistic factors affect the realization of subjects. These factors are largely uncharted in the framework of bilingual null subject use. In a study with German (monolingual) corpora, \textcite{Frick_2017} found that null subjects were more frequent in informal brief text messages such as SMS. In our study, the results suggest that the communicative situation does not affect the three languages in the same way. In Greek, neither formality nor mode seem to be significant factors. The picture is different for Russian and Turkish as formality and mode have an impact on subject realization. Specifically, in both languages participants produce more overt subjects in a formal communicative situation.
However, the realization patterns diverge regarding mode.
While Russian speakers produce more overt subjects in the spoken mode, the probability of overt subjects in the Turkish group is higher in the written mode. Generally, these findings support our suggested continuum from more strict and stable to less strict and dynamic pro-drop languages that align with the typological differences between Greek, Russian and Turkish. Specifically, the strict pro-drop language Greek does not seem to be affected by extra-linguistic factors and relies more on agreement-marking to license null subjects.
Leaving aside the typological differences concerning null subjects in Russian and Turkish, being less explicit in the informal communicative situation can be related to the relaxed tenor of discourse, the reduced attentiveness of the speaker and in general the nonchalant style of informal communicative situations \parencite{stowell2017introducing}. On the other hand, one would expect that explicitness conforms with written mode so any misunderstanding could be avoided as the paralinguistic features are absent from all communicative situations in written mode. Thus the question why the heritage groups diverge regarding the extra-linguistic factor mode remains an open issue and a desideratum for future research.

Based on our findings from the elicited production task at hand, we could build hypotheses for experimental tasks that focus on the processing and comprehension of pro-drop by speakers of languages that belong to different types of null subject languages. Specifically for the different groups of heritage Greek, Russian and Turkish speakers, we would expect that heritage speakers of Greek are more restricted in their acceptance of non-canonical overt subjects, but Russian and Turkish heritage speakers might allow a more free use of overt subjects. This would allow for testing whether the observation regarding typological differences in dynamicity in null subject use only holds for production or whether it extends to comprehension as well.

A core finding from our study on heritage Greek, Russian and Turkish is the observation that null subject use in the heritage language seems to be more stable or dynamic based on the underlying type of pro-drop language. While our study allows for more cross-linguistic comparisons than previous studies, it still largely relies on languages that can broadly be classified within a pro-drop theory which depends on agreement marking \parencite{rizzi1986null}. Therefore, expanding our observation to other languages within the typology of null subject languages would be interesting. For example, it would be interesting to see how radical-pro-drop languages like Mandarin Chinese, which lack agreement marking, behave in this cross-linguistic comparison \parencite{neeleman2007radical}. Similarly, we would need to explore the role of partial pro-drop languages like Hebrew and Finnish, which allow null subjects in first and second person but not in third \parencite{camacho2013null}. This wider cross-linguistic comparison may allow us to test and generalize our hypothesis that languages which require more null subjects will behave more stably, whereas languages that allow overt subjects more freely will show more dynamicity under language contact. For example, the few studies that exist on the acquisition of null subjects by Chinese heritage language speakers indicate that their use of null subjects stays stable \parencite{chou2020acquisition, zhang2021language}, which would be in line with our predictions. This would not only extend the typological diversity in the study of null subjects, but it could also open ways for the formalization and theorizing regarding the change of the pro-drop parameter in bi- and multilingual grammmars of heritage speakers. 

To our knowledge, this study is the first that compares subject realization in Greek, Russian and Turkish heritage speakers in Germany and the US and accounts for animacy, formality and mode variation. The study offers insights for the same phenomenon in three heritage varieties in contact with two majority languages, making the outcome comparable for the three populations of interest. So far, studies in the field of heritage languages have rarely used a unified methodology for different languages, which raises questions about the generalizability of the results \parencite{winter2021independence}. In contrast, our study offers cross-linguistically comparable data from different heritage communities residing in the same countries and selected according to the same demographic criteria. Its wider implications for bilingualism are manifold. First, since in many languages morphological agreement and syntactic marking are crucial for allowing null subjects, heritage speakers' variability in this domain informs us about the grammatical mechanisms in the bilingual mind. Second, the three languages under scope of this study represent different types of pro-drop languages (strict, partial, topic pro-drop). Combining these different types with two non-pro-drop languages (German or English) in a bilingual's mind gives us insights into diverse outcomes. We can provide evidence of the divergence of heritage languages from the monolingual variety and confirm the unlikeliness of some scenarios in language, for example, that a strict pro-drop language completely abolishes this feature due to language contact. Rather, there are slight adjustments which we labeled dynamicity, but there is not a complete reorganization of the system. 

\section*{Acknowledgements}
We would like to thank everyone involved in project P10 of the Research Unit \textit{Emerging Grammars in Language Contact Situations} (RUEG, DFG project no. 313607803).

This work has grown out of a big team effort with many student assistants spending hundreds of hours with detailed annotations. For this reason, we would like to thank all of them: Büşra Çiçek, Zeynep Özal, Borbála Sallai, Mariya Burbelko, Alona Prozorova, Ioanna Kolokytha, Panagiota Papavasiliou, Nina Bredereck. We additionally thank Lea Coy for help with pre-publication formatting.

We also thank the anonymous reviewers as well as audiences at the DGfS 2023, ISBPAC 2022, HL@cross, and GAL 2022 conferences for their helpful and very constructive suggestions.
The research was supported through funding by the Deutsche Forschungsgemeinschaft (DFG, German Research Foundation) for the Research Unit \textit{Emerging Grammars in Language Contact Situations}, project P10 (DFG grant number: 313607803).

\printbibliography[heading=subbibliography,notkeyword=this]
\end{document}
