\documentclass[output=paper,colorlinks,citecolor=brown]{langscibook}
\ChapterDOI{10.5281/zenodo.15775179}
\title{Dynamic properties of the heritage speaker lexicon} 
\author{    Nadine Zürn\orcid{0000-0003-4180-5414}\affiliation{University of Mannheim}    and  Mareike Keller\orcid{0000-0001-8054-1701}\affiliation{University of Mannheim}     and Rosemarie Tracy\orcid{0000-0002-6683-3481}\affiliation{University of Mannheim}      and Anke Lüdeling\orcid{0000-0001-5944-4595}\affiliation{Humboldt-Universität zu Berlin}}

\abstract{Against the backcloth of current research on heritage speakers' linguistic knowledge and behavior, this chapter focuses on a domain very much in flux in any speaker: the lexicon. We investigate lexical resources and resourcefulness in written and spoken descriptions of the same event by heritage speakers in both their languages, and by monolingual speakers of English and German. Spoken and written reports are based on a filmed staged accident and were elicited in standardized situations manipulated in order to encourage use of either formal or informal registers. Our line of argument moves along the following research questions: First, how comparable are different speaker groups with respect to lexical inventory and lexical diversity, and what trends can be identified? Second: How do heritage speakers of German who were raised in the US and monolingual speakers of German compare with respect to German particle verbs regarding syntactic, morphological, semantic, phonological, and pragmatic properties? Third: What insight into lexical resources can be gained by studying performance-related phenomena (self-interruptions, self-repairs, filler particles, etc.)?  Speaker group comparisons of lexical diversity and inventory are conducted via statistical modelling, whereas the particle verb analyses dealing with various interface phenomena are based on fine-grained qualitative analyses. Overall, our findings provide further evidence for tendencies towards explicitness and transparency discussed in heritage language research.

\keywords{heritage language, German, lexical diversity, particle verbs}

}

\IfFileExists{../localcommands.tex}{
   \addbibresource{../localbibliography.bib}
   % add all extra packages you need to load to this file

\usepackage{tabularx,multicol}
\usepackage{url}
\urlstyle{same}

\usepackage{listings}
\lstset{basicstyle=\ttfamily,tabsize=2,breaklines=true}

\usepackage{langsci-basic}
\usepackage{langsci-optional}
\usepackage{langsci-lgr}
\usepackage{langsci-osl}
% \usepackage{./langsci/styles/langsci-lgr}
% \usepackage{./langsci/styles/langsci-osl}
% \usepackage{langsci-gb4e}

\usepackage{tikz}
\usetikzlibrary{patterns,calc}
\pgfdeclarepatternformonly{south east lines}{\pgfqpoint{-0pt}{-0pt}}{\pgfqpoint{3pt}{3pt}}{\pgfqpoint{3pt}{3pt}}{
    \pgfsetlinewidth{0.6pt}
    \pgfpathmoveto{\pgfqpoint{0pt}{3pt}}
    \pgfpathlineto{\pgfqpoint{3pt}{0pt}}
    \pgfpathmoveto{\pgfqpoint{.2pt}{-.2pt}}
    \pgfpathlineto{\pgfqpoint{-.2pt}{.2pt}}
    \pgfpathmoveto{\pgfqpoint{3.2pt}{2.8pt}}
    \pgfpathlineto{\pgfqpoint{2.8pt}{3.2pt}}
    \pgfusepath{stroke}}
    
\usepackage{stmaryrd}
\usepackage{wasysym}
\usepackage{multirow}
\usepackage{caption}
\usepackage{subcaption}
\usepackage{mathrsfs}
\usepackage{qtree}

\usepackage{linguex}


   %pminos do not split footnotes
% \interfootnotelinepenalty=10000 %Footnote in Laporte chapters has to be split SN


%\DeclareIndexNameFormat{default}{%
%\nameparts{#1}%
%\usebibmacro{index:name}%
%{\index[names]}%
%{\namepartfamily}%
%{\namepartgiveni}%
% {}% L1
% {}% L2
%{\namepartprefix}% generates spurious space L3
%{\namepartsuffix}% generates spurious space L4
%}

%  {\DeclareIndexNameFormat{default}{%
%     \usebibmacro{index:name}{\index[names]}{#1}{#3}{#5}{#7}}}

%\DeclareIndexNameFormat{default}{%
%  \usebibmacro{index:name}{\sindex[nom]}{#1}{#3}{#5}{#7}}

%\DeclareIndexNameFormat{default}{%
%  \usebibmacro{index:name}{\sindex[person]}{#1}{#3}{#5}{#7}}
%\DeclareIndexNameFormat{default}{%
%\nameparts{#1} \usebibmacro{index:name}{\sindex[person]]}{\namepartfamily}{‌​\namepartgiven}{\nam‌​epartprefix}{\namepa‌​rtsuffix}}

%\newcommand{\smiley}{:)}

%\renewbibmacro*{index:name}[5]{%
%\usebibmacro{index:entry}{#1}%
%{\iffieldundef{usera}{}{\thefield{usera}\actualoperator}\mkbibindexname{#2}{#3}{#4}{#5}}}

% \newcommand{\noop}[1]{}

%remove for final
%\overfullrule=1mm

\newcommand{\tobi}[2]}}
\renewcommand{\S}[1]{\tobi{#1}{\textsc{*}}}

% this volume references
% puts: [this volume]
% already defined: \citetv
%\newcommand{\citepv}[1]{(\citeauthor{#1} \citeyear*{#1} [this volume])}
\newcommand{\citealtv}[1]{\citeauthor{#1} \citeyear*{#1} [this volume]}

%parentheses around example number
\newcommand{\pref}[1]{(\ref{#1})}

% in-text examples

\newcommand{\lnex}[1]{\textit{#1}} %target lang word
\newcommand{\lnlit}[1]{(lit.: `#1')} %literal reading
\newcommand{\lnlat}[1]{(#1)} % latinization
\newcommand{\lntrans}[1]{`#1'} %translation
\newcommand{\lnexl}[2]%
{\lnex{#1}{} \lnlat{#2}} % ex with latinization
\newcommand{\lnexlat}[3]{\lnex{#1}{} \lnlat{#2}{} \lntrans{#3}} % ex with latinization and tranl.

%ch01
\newcommand{\co}[1]{\mbox{\textbf{#1}}}

%ch09

\newcommand{\cyrbulg}[1]{\begin{otherlanguage*}{bulgarian}#1\end{otherlanguage*}}


%ch10
\newcommand{\nlp}{{\small NLP}}
\newcommand{\mwe}{{\small MWE}}
\newcommand{\rae}{{\small RAE}}
\newcommand{\lvc}{{\small LVC}}
\newcommand{\pos}{{\small P}o{\small S}}
%\newcommand{\todo}[1]{ \textcolor{red}{#1} }

%\renewcommand{\labelenumi}{\theenumi}
%\ainamefmt{{vv}{ll}{, ff}{, jj}} % fullname

\newcommand{\biberror}[1]{{\color{red}#1}}

\newcommand{\osenovaitem}{--~}
   %% hyphenation points for line breaks
%% Normally, automatic hyphenation in LaTeX is very good
%% If a word is mis-hyphenated, add it to this file
%%
%% add information to TeX file before \begin{document} with:
%% %% hyphenation points for line breaks
%% Normally, automatic hyphenation in LaTeX is very good
%% If a word is mis-hyphenated, add it to this file
%%
%% add information to TeX file before \begin{document} with:
%% %% hyphenation points for line breaks
%% Normally, automatic hyphenation in LaTeX is very good
%% If a word is mis-hyphenated, add it to this file
%%
%% add information to TeX file before \begin{document} with:
%% \include{localhyphenation}
\hyphenation{
    Beck-man
    Ngu-yen
    back-chan-nel
    back-chan-nels
    mo-not-o-nous
    ste-reo-typ-i-cal
}

\hyphenation{
    Beck-man
    Ngu-yen
    back-chan-nel
    back-chan-nels
    mo-not-o-nous
    ste-reo-typ-i-cal
}

\hyphenation{
    Beck-man
    Ngu-yen
    back-chan-nel
    back-chan-nels
    mo-not-o-nous
    ste-reo-typ-i-cal
}

   \boolfalse{bookcompile}
   \togglepaper[10]%%chapternumber
}{}

\begin{document}
\maketitle

%%%%%%%%%%%%%%%%%%%%%%%%%%%%%%%%%
%%%%%%%%%%%%%%%%%%%%%%%%%%%%%%%%%
%%%%%%%% start section 1 %%%%%%%%
%%%%%%%%%%%%%%%%%%%%%%%%%%%%%%%%%
%%%%%%%%%%%%%%%%%%%%%%%%%%%%%%%%%

\section{Introduction} \label{sec:kelleretal:intro}

Regardless of how researchers conceptualize the architecture of the mental lexicon, most will probably agree that it is the most dynamic subsystem of our overall linguistic knowledge. Once vocabulary growth speeds up in early childhood, our lexical repertoires continue to expand, but not necessarily in the languages we started out in, and certainly not just linearly. Words of our childhood are adjusted to target forms with respect to phonology, morphologically restructured, and recategorized with respect to contextual and cultural appropriateness.  

Whether and to what extent the lexicon of a first language (L1) develops across a speaker's lifetime  depends on many factors, including contact with and properties of other languages acquired from birth or later on, with each potentially influencing the others. Essentially then, the mental lexicon is a moving target, principally ``on the go'', with new words discoverable at any time and either holistically adopted or analyzed according to available productive word formation and inflection processes. The investigation of heritage languages (henceforth HLs), whose speakers often feel that their majority language (henceforth MajL) is the more proficient and dominant one, provides us with a natural laboratory for exploring how intricate word-related knowledge can be acquired in diverse acquisition scenarios.

What happens to immigrants' HLs in the long run, i.e. across generations in diasporic islands, has already been investigated for many language combinations. As for German as a heritage language, the adoption and adaptation of new vocabulary, especially freely importable discourse markers, leveling of irregular morphosyntactic paradigms, changes in argument structure, and word order have been identified as prominent outcomes of contact with English (\citealt{Matras1998Utterance, Muysken2000Bilingual, Fuller2001Detachability, Clyne2003Dynamic, Boas2009, Boas2010DiscourseMarkers, Muysken2013Outcomes, PutnamSalmons2013PassiveVoice, Riehl2014Einführung,HoppPutnam2015,Stolberg2015Changes, ZimmerEtAl2020Korpus}, etc.). Some of these long-term changes as well as convergence of similar forms, ample borrowing, and orthographic interaction have also been documented in first generation immigrants \citep{Tracy2001Contact,Clyne2003Dynamic,Schmid2011,Keller2014,Tracy2022Gemischtsprachiges}.

Complementary to these two research strands – long-term effects of language contact on immigrant languages in diaspora communities on the one hand and L1 change in first-generation immigrants on the other – this contribution focuses on second-generation immigrants. Our participants are early bilinguals, exposed to the HL within their family context. In some cases, contact with the HL is limited to communication with just one parent, as in the Tiny Language Island scenario discussed in \textcitetv{chapters/05}. Where the minority language can neither draw on a HL community outside the home nor on the educational system (as in mother-tongue classes, bilingual programs or foreign-language classes), its speakers may only rarely be exposed to functional varieties of their HL other than an informal-spoken register. As the background variables impacting quantity and quality of exposure can be very heterogeneous, the heterogeneity of linguistic outcomes, which is often mentioned in the literature, does not come as a surprise \citep{Montrul2006Competence, Moreno-Fernandez2007Anglicismos, Fairclough2010Availability, Polinsky2018HeritageLanguages}. However, it is important to recognize that similar inter-individual differences are noticeable in monolingual speakers (henceforth MSs) (\citealt{ShadrovaEtAl2021,WieseEtAl2022}, and other chapters in this volume).

Standardized data collection, corpus compilation and analysis took place in the context of the Research Unit  \textit{Emerging Grammars in Language Contact Situations} (RUEG), described in detail in \textcitetv{chapters/02}, \citet{Klotzetal2024} and briefly in \sectref{sec:kelleretal:data-method} below. The target were heritage varieties of Greek, Russian, and Turkish both in Germany and in the United States, as well as heritage German in the United States. We elicited the same type of data from monolingually raised majority language speakers in all countries, thereby minimizing the risk of attributing non-canonical HS utterances to language contact with one of the two MajLs, English or German.

In this contribution we report results based on a quantitative exploration of the HSs' lexical resources by means of lexical diversity calculations and lexical inventory assessments and identify group-specific patterns related to the lexicon. Further, two additional kinds of dynamics are addressed in qualitative analyses, both focusing on German particle verbs (PVs): First, we present canonical and non-canonical occurrences of PVs in order to capture patterns pointing towards innovation and change. Second, we ask what production phenomena, such as hesitations, filler items and overt repairs surrounding PVs reveal about word candidates considered at the moment of speaking and what they tell us about speakers' implicit judgement of the quality of their own utterances.

Despite the overall rise of interest in heritage languages and in what they contribute to general theories of learnability and language change, the question of how HSs use (non-)lexicalized forms in their productions still calls for an answer. We aim at contributing to closing this gap by analyzing the lexical items speakers resort to in situations where they are confronted with specific spoken and written tasks in both their languages. Our findings support the claims from previous literature that HSs, especially in contrast with second language learners, are ``comfortable in experimenting with the lexicon of their language'' \citep[294--295]{Polinsky2018HeritageLanguages} and that they display preferences for compositionality, semantic transparency, and explicitness (see \citealt{RakhilinaVyrenkovaPolinsky2016Creativity, PashkovaHodgeShanley2020Explicitness}). 

Our argumentation proceeds as follows. \sectref{sec:kelleretal:lextheory} starts from a conception of the mental lexicon as tightly interconnected with all levels of grammar. This section also provides our rationale for selecting German particle verbs for scrutiny later on. \sectref{sec:kelleretal:data-method} introduces the corpus and methodology. In \sectref{sec:kelleretal:lex-div-inventory}, the empirical portion of this chapter begins with a quantitative analysis of the lexical inventory and lexical diversity. \sectref{sec:kelleretal:ger-part-verbs} narrows the focus to German PVs and illustrates subtle differences between HSs’ and MSs’ productions. \sectref{sec:kelleretal:lexproduction} continues the exploration of PVs but shifts attention to production phenomena as additional ways in which speakers provide us with evidence for the lexical resources under their control. \sectref{sec:kelleretal:discussion} summarizes findings, points out limitations and raises new questions.

%%%%%%%%%%%%%%%%%%%%%%%%%%%%%%%%%
%%%%%%%%%%%%%%%%%%%%%%%%%%%%%%%%%
%%%%%%%% start section 2 %%%%%%%%
%%%%%%%%%%%%%%%%%%%%%%%%%%%%%%%%%
%%%%%%%%%%%%%%%%%%%%%%%%%%%%%%%%%

\section{The lexicon as a dynamic and interconnected resource}\label{sec:kelleretal:lextheory}
\largerpage

Current theories of the mental lexicon no longer consider it in isolation from the rest of the grammar or as a mere storage space for the non-productive and necessarily listed items, including multi-word idiomatic expressions. At the same time, many approaches go for ``the lexicon all the way down'', with all kinds of meaningful units, from morpheme to clauses and even larger discourse chunks considered more or less unique form-function pairings in a gigantic constructicon (\citealt{Goldberg2005Constructions,Tomasello2006, Bybee2010}; various contributions in \citealt{EngelbergHollerProost2011}). However, the absence of consensus on how to best capture item- vs. rule-generated properties of natural languages, or, more specifically, of the lexicon, is irrelevant for our concern at this moment. 

Following up on a metaphor by \citet{JackendoffAudring2019TextureLexicon}, we conceive words to be ``small bridges'' across phonology, morphology, semantics, syntax, pragmatics, and, in writing, orthography. This means that word knowledge is inherently relational. As far as individual lexical items are tied to specific registers, different dialects or languages, they have to be marked accordingly. The bridging function of words across linguistic interfaces within each language and across languages, as well as the co-activation potential and competition of formally and/or semantically similar candidates, make lexical items highly susceptible to fluctuation. At the same time, these multiple connections provide speakers with a rich source for creativity and innovation \citep{DeganiPriorTokowicz2011Bidirectional, PriorEtAl2017Suspectibility, RabinovitchTsvetkovWintner2018Effects}, which is  particularly relevant from our perspective.

Researchers inquiring into the HS lexicon concluded that it does not match the repertoire and behavior of monolingual peers in size or age-adequate use (e.g. \citealt{Montrul2006Competence, Polinsky2018HeritageLanguages}). Differences have also been identified with respect to lexical retrieval (\citealt{Moreno-Fernandez2007Anglicismos, Polinsky2018HeritageLanguages}). As stated repeatedly in the literature, reduced exposure beyond early childhood and decreasing relevance of the HL offer plausible explanations for differences between HSs and MSs. After all, in contrast with minority languages, the majority L1 does not have a status problem, hence is not questioned or threatened but supported by the education system. 
\largerpage

One way of assessing lexical resources is measuring lexical diversity (LD). LD is considered ``an important indicator of language learners’ active vocabulary and of how it is deployed'' \citep[85]{MalvernRichads2002AccommodationLD} to communicate effectively and appropriately. Numerous studies operationalize LD as an indicator of proficiency and ``a type of linguistic complexity'' \citep[95]{Jarvis2013DiversityinLD}, either as a stand-alone measure or in combination with others, for instance lexical density and sophistication of expression \citep{BonvinEtAl2018Entwicklung, GharibiBoers2019LexicalRichness, ElabdaliWeinOrtega2022LDBilingually} in both spoken and/or written productions \citep{LauferNation1995VocabSize, MalvernRichads2002AccommodationLD, PennockSpeckClavelArroitia2021LDWrittenSpoken}. To this day, various kinds of LD measures have been applied to monolingual, to L2, as well as to HS data either for grouping speakers into proficiency categories (e.g. \citealt{KopotevKisselevPolinsky2020Collocations}) or with the aim to validate the appropriateness of this measure for comparing speaker groups, also with respect to different settings \citep{DallervanHoutTreffersDaller2003Richness, Yu2009LDWritingSpeaking, HrzicaRoch2021LDCroatianItalian, PetersenFogetHansenThøgersenKühl2021Proficiency}. For instance, in a study on LD in reports of younger and older HSs in comparison with monolingual peers, \citet{GharibiBoers2019LexicalRichness} found lower LD values in younger HSs compared to monolinguals and to older HSs. The authors attribute higher LD values in the latter to longer exposure time. On the whole, measures aiming at the assessment of a speaker's lexicon only provide snapshots of a temporary state of lexical knowledge since that state is likely to change as a consequence of continued language exposure and use \citep{Yu2009LDWritingSpeaking, CzapkaTopajGagarina2021LongitudinalLexDevelopment, Lambelet2021LDDevelopment}, which is, in turn, connected to issues of the wider context and differences in status as minority or majority language \citep{TreffersKorybski2016Measures, TreffersDaller2019DominanceDef}.

Variability in language dominance is also reflected in setting- or register\hyp specific vocabulary. \citet{VanGijselSpeelmanGeeraerts2005Richness}, for instance, show an effect of register variation on lexical richness measured by type-token ratio, with lower values for informal settings compared to formal ones. This is in line with \citet{Alamillo2019LexicalSkills}'s findings on Spanish heritage and L2 speakers. Furthermore, topic familiarity, time pressure during production and self-confidence have been considered predictors of LD. Here, the intra-speaker comparison of spoken and written productions by \citet{Yu2009LDWritingSpeaking} makes a better prediction of LD in spoken compared to written language productions, with overall similar levels of LD between both modes. Written tasks, which usually provide more time, yielded higher LD values, especially when subjects were familiar with the topic and felt more confident \citep[250]{Yu2009LDWritingSpeaking}. Against this backdrop, \sectref{sec:kelleretal:lex-div-inventory} pursues the overarching research question of what we can deduce from LD and LI measures in group comparisons, given the diverse acquisition contexts attested for heritage speakers.

As we later move from the quantitative assessments of lexical repertoires and the pros and cons of LD and LI measures to a very specific but theoretically complex type of verb, German particle verbs, some justification for this move is called for. Particle verbs are Janus-faced: On the one hand they behave like complex words, on the other hand like phrasal syntagmas. What are commonly considered PVs, such as \textit{ankommen} (at-come, `arrive') and \textit{anrufen} (up-call, `phone'), do not form one homogeneous class. Their structural analysis is as controversial as it is intriguing. While some authors consider them words with strange properties, many analyze them as syntactic constructions -- see e.g. \citet{Mueller2002}, \citet{Lüdeling2001ParticleVerbs}, and \citet{Felfe2012}. As shown for example by \citet{Lüdeling2001ParticleVerbs}, it is even unclear what to count as a particle since various form classes behave similarly. As a theoretical discussion of PVs is far beyond the scope of this paper, our presentation of corpus data in Sections \ref{sec:kelleretal:ger-part-verbs} and \ref{sec:kelleretal:lexproduction} is limited to undisputed cases, namely verbal particles homonymous with adverbs or prepositions, and we treat PVs as lexical entries without any further comments on their structural status. In the context of HL acquistion and maintainance, PVs are an interesting research object with respect to (a) syntactic distribution within clauses, (b) semantic function, and (c) differences in form due to their co-occurrence and amalgamation with deictic elements. 

Example \REF{ex:kelleretal:loc_fahren} gives a first impression of the material available in the RUEG corpus. The short passage contains seven clauses, main clauses (\textsc{mc}) and subordinate clauses (\textsc{sc}) counted separately. They are all functionally assertions, as expected in reports. The altogether ten verbs occur in one of two possible positions for verbs in German: finite verbs appear in second position in declarative main clauses, clause-finally in subordinate clauses.\largerpage

\begin{exe}
    \ex \label{ex:kelleretal:loc_fahren} {[}\textsubscript{MC} Ein Auto \textbf{is} in einem anderen Auto hinten \textbf{reingefahren} {[}\textsubscript{SC} weil das erste Auto für ein Hund schnell bremsen \textbf{musste}{]}{]}. {[}\textsubscript{MC} Das erste Auto \textbf{war} blau{]} {[}\textsubscript{MC} und das Auto {[}\textsubscript{SC} das hinten \textbf{reingefahren} \textbf{ist}{]} \textbf{war} weiss oder vielleicht grau{]}. {[}\textsubscript{MC} Niemand \textbf{sah} verletzt \textbf{aus}{]} {[}\textsubscript{MC} und jemand \textbf{hat} die Polizei \textbf{angerufen}{]}.\\
    `One car rearended another car because the first car suddenly had to brake for a dog. The first car was blue, and the car that hit it from behind was white or maybe grey. Nobody seemed hurt and someone called the police.' (USbi50FD\_fwD)\footnote{See \sectref{sec:kelleretal:lex-div-inventory}, \tabref{tab:kelleretal:independentvars} for detailed information on how to interpret the speaker codes provided at the end of each example.}
\end{exe}

Non-finite verbs (infinitives, participles) and~-- crucial here~-- particles of PVs only appear in final position. In main clauses, finite base verbs and auxiliaries occur in second position. Hence, whenever a PV appears clause finally, whether non-finite in main clauses, or (non-)finite in subordinate clauses, the particle and the verb are adjacent and orthographically rendered without space, i.e., they look (and ``feel'') like a word. 

In addition to their distributional properties (continuous vs. discontinuous), PVs are semantically relevant in two ways. First, many of them are semantically intransparent and have to be learned as a whole. Others form patterns which can then be used for productive new formations. Second, we will see in later discussions of the way events are described (\sectref{sec:kelleretal:ger-part-verbs}) that verbs encode manner of motion, with the particle providing path information with respect to direction and goal, thereby also often contributing to a change in aspect \citep{Talmy1988, TennyPustejowski2000, Slobin2003Language}. \textit{Fahren} in Example (\ref{ex:kelleretal:fahren-atelisch}) is atelic (an activity in the sense of \citealt{Vendler1957}). The particle verb \textit{reinfahren} `crash into, hit' in Example (\ref{ex:kelleretal:fahren-part}) is telic. The particle \textit{rein} implies a goal (in this case \textit{ihm} `him', meaning the other driver and his car). The goal can also often be expressed by a full PP, as exemplified by \textit{auf den Parkplatz} `into the parking lot' in Example (\ref{ex:kelleretal:fahren-pp}).\footnote{Transcription conventions: Oral productions are transcribed according to project-internal transcription and annotation guidelines (\url{https://korpling.german.hu-berlin.de/rueg-docs/latest/annotations.html}. Mark-up irrelevant for the present discussion, (e.g. vowel length), has been removed. Pauses are marked by a hyphen in brackets. The spelling in written productions has been gently normalized to enhance readability and keep the focus on aspects relevant to this study.

Glossing: The glosses follow the basic principles of the Leipzig Glossing Rules (\url{https://www.eva.mpg.de/lingua/resources/glossing-rules.php}), but morphological boundaries and categories are only marked as far as they are relevant for our exposition. For verbal particles as the morphological feature in focus we introduced the label \textsc{vpart}. All items relevant for the discussion of production phenomena (hesitations, repetitions, self-corrections, etc.) are marked by a double asterisk (**) in the gloss.}


\ea
\label{ex:kelleretal:fahren-atelisch}
\gll an dem besagten tag \textbf{fuhr-en} zwei autos auf dem parkgelände hintereinander\\
on the said day drive-\Pst.\Tpl{} two cars on the.\Dat{} parking.area behind.each.other\\
\glt `on said day two cars drove behind each other in the parking area' (DEbi02FT\_fsD)

\ex 
\label{ex:kelleretal:fahren-part}
\gll und	er	\textbf{fähr-t}	ihm	\textbf{rein}\\
and he drive-\Prs.\Tsg{} him.\Dat{} into.\Vpart{}\\
\glt `and he hits him' (DEbi05FT\_iwD)

\ex \label{ex:kelleretal:fahren-pp}
\gll genau in dieser sekunde \textbf{fuhr-en} zwei autos \textbf{auf} \textbf{den} \textbf{parkplatz} \\
exactly in this second drive-\Pst.\Tpl{} two cars on the.\Acc{} parking.lot \\
\glt `exactly this second two cars entered the parking lot' (DEbi03FR\_isD)
\z

While PVs expressing motion events are often fairly transparent, complexity arises because speakers can choose between morphologically different forms of the particle, such as \textit{ein}/\textit{rein}/\textit{herein} (all meaning `in(-to)'). Sometimes choice is motivated by register parameters, in other cases choices have consequences for the expression (or comprehension) of argument structure  \citep{HärtlWitt1998KonzeptePartikelverben}. We saw in Example (\ref{ex:kelleretal:fahren-part}) that the PV \textit{reinfahren} can be used with a dative argument. In (\ref{ex:kelleretal:fuhrrein}), the same particle verb is used without an argument and with a slight shift in meaning: an overt locative argument (like a parking lot or a garage) of the particle is omitted but inferrable in shared non-verbal contexts. In Example (\ref{ex:kelleretal:Mazda}), the particle \textit{hinein} and the full PP \textit{in diesen} `into this one' are both used.

\ea
    \label{ex:kelleretal:fuhrrein}
    \gll auf einmal \textbf{fuhr} ein weißes auto \textbf{rein}. \\ 
    at once drive.\Pst.\Tsg{} a white car into.\Vpart{}\\
    \glt `Suddenly a white car drove in.' (DEbi52FT\_fsD)
    
    \ex \label{ex:kelleretal:Mazda}
    \gll Dieser	\textbf{fuhr}	nach	der	Voll-bremsung	des	Mazda	\textbf{in}	\textbf{diesen}	\textbf{hin-ein}. \\
    this drive.\Pst.\Tsg{} after the full-brake of.the Mazda in this-\Acc{} there-in.\Vpart{}\\
    \glt `That one ran into the Mazda after its emergency stop.' (DEbi34FR\_fwD) 
\z

Even though PV behavior is even more complicated, for our purpose here it suffices to say that particles can influence the argument structure of the base verb. Sometimes the particle satisfies an argument slot, sometimes it opens up an argument position, and sometimes it changes the argument function (see e.g. \cite{StiebelsWunderlich1994,Lüdeling2001ParticleVerbs,Zeller2001, Mueller2002, Boas2003, Felfe2012}). 

Although our current focus is on adolescent and adult heritage speakers, findings on L1 acquisition are worth pointing out. Interestingly, the syntactic behavior of PVs in both continuous and discontinuous transparent constellations is no acquisition hurdle \citep{SchulzTracy2011, Tracy2011Konstruktion, Tracy1991}. Separable telic particles are already part of children’s lexicon at the time when they only produce one-word utterances (\textit{weg} `away', \textit{auf} `up, open', \textit{zu} `closed', \textit{rein} `into'), and they are present in early two-word combinations with or without their verbal base, with or without deictic expressions, hence well before the appearance of finite V2-clauses with finite verbs and particles separated. 

The early emergence and stability of PVs can be attributed to the confluence of the following: (a) their consistent position at the end of clauses; (b) their contextually relevant and transparent semantic content compared to the rest of the verb, which is often a semantic lightweight, compare \textit{aufmachen} `to open' (lit: `to make open'), \textit{zumachen} `to close' (lit: `to make closed'), and \textit{wegmachen} `to remove' (lit: `to make gone'); (c) when combined with their verbal base, particles bear word stress (\textit{\textsc{auf}machen} \textit{\textsc{zu}machen}); and finally, (d), as particles, they remain uninflected, hence consistent in form, apart from combining with deictic elements which, in turn, contribute important information with respect to event specifics.

Likewise, comprehension studies provide evidence for children's early sensitivity to telicity, see \citet{VanHout2000} for Dutch, where particles are equally precocious, and for L1 and early L2 German \citep{SchulzTracy2011}.\footnote{Similarly, the formal properties of PVs do not seem to be difficult for older learners of German as a Foreign Language, see \citet{LuedelingHirschmannShadrova2017Productivity}.} 
In conclusion, the early emergence and prominence of PVs in a child's lexicon could explain which of the overall intricate features associated with particle verbs are mastered and remain resilient in HSs even though L1 exposure may decrease after early childhood.


%%%%%%%%%%%%%%%%%%%%%%%%%%%%%%%%%
%%%%%%%%%%%%%%%%%%%%%%%%%%%%%%%%%
%%%%%%%% start section 3 %%%%%%%%
%%%%%%%%%%%%%%%%%%%%%%%%%%%%%%%%%
%%%%%%%%%%%%%%%%%%%%%%%%%%%%%%%%%

\section{Data and methodology} \label{sec:kelleretal:data-method}

The corpus explored here is the outcome of a large comparative research initiative (\citealt{RUEGcorpus2024}, RUEG) investigating various HLs (Greek, Turkish, Russian) in Germany in comparison with the same HLs plus Heritage German in the United States. The core data gathered across all projects consists of reports elicited on the basis of a video stimulus showing a staged minor car accident (\citealt{Wiese2020LanguageSituations}, see also \cite{chapters/02}). Participants are monolingual and bilingual adolescents (age 13--19) and adults (age 20--37). As one of the joint research goals is the investigation of formality- and mode-specific linguistic repertoires, all participants were asked to relate what happened in the 40-second film clip to different imagined addressees, both in speech and in writing. Fictive addressees were the police, contacted via voicemail (formal-spoken) and written testimony (formal-written), as well as friends, again in a spoken voice message (informal-spoken) and in a written WhatsApp message (informal-written).\footnote{While we realize that differentiating degrees of formality is a complex matter, the descriptors ``formal'' and ``informal'' here refer to carefully arranged, formal or informal elicitation contexts, with even elicitators dressed accordingly.} Bilingual speakers performed all tasks in both their heritage and their majority language. During elicitation sessions and during casual encounters around these sessions, project members –  each in charge of eliciting either the heritage or the majority language data – did not engage in code-switching. Therefore code-switches and borrowing on part of participants was not primed by interlocutors. The repeated elicitation of reports on the same event in four different settings makes it possible to investigate consistency of lexical and grammatical choices, frequency of occurrence and cooccurrence of lexical items, type and frequency of morphological processes, as well as the adherence to and extension of the semantic scope of lexical items. In addition it provides us with insights into sensitivity towards different varieties, i.e. the specific registers which are of common concern to all projects (see the other chapters in this volume and \citealt{TsehayeEtAl2021,WieseEtAl2022, PashkovaEtAl2022}).\footnote{The notion \textit{register} -- roughly: situationally and functionally conditioned variation -- is complex and we cannot do it justice here \citep{BiberConrad2009RegisterGenreStyle, EgbertBiber2018RegiterVariation,Matthiessen2019Register, Lüdeling2022Register}. In the RUEG context, we operationalize register by the four situations created to elicit the data. For more detail see \textcitetv{chapters/05}.}

In order to allow for corpus searches targeting the morphological make-up of lexemes and the inclusion of performance phenomena, we decided to augment the existing corpus by additional annotations:

\begin{enumerate}
    \item Manual annotation of all verb tokens in the complete German sub-corpus for lemma (associates the separated particles of particle verbs with their base verb), morphological type (simplex, prefix, particle), and syntactic function (lexical, modal, auxiliary, copular)\footnote{The annotation guidelines are available online at \url{https://korpling.german.hu-berlin.de/rueg-docs/standalone/verb-analysis/}.}
    \item Selective manual span annotation of production phenomena (hesitations, filler particles, interruptions, repetitions, repairs, etc.)
\end{enumerate}

Although the RUEG data was not elicited specifically for investigating lexical inventories, the use of the same stimulus material across various conversational settings as well as across participants with varying language backgrounds makes it possible to analyze and compare lexical resources within and across speaker groups. Given the generally dynamic nature of lexical knowledge and the variability in exposure of HSs to registers, as well as differences in opportunity to use their HL, we expect differences between monolingually raised, majority and heritage speakers of a given HL, here German. More specifically, we expect a gradation effect in LD and LI size from monolingual speakers of a language, such as German, to majority speakers of German followed by heritage speakers of German, as well as differences between the situational and conversational settings in line with \citet{VanGijselSpeelmanGeeraerts2005Richness}, \citet{Yu2009LDWritingSpeaking}, or \citet{Alamillo2019LexicalSkills}. For majority language use, such as English in the US, we do not predict a similar gradation pattern. Even though the analyses discussed in later sections specifically focus on HSs of German raised in the United States (USbiGer), \sectref{sec:kelleretal:lex-div-inventory} includes data from other speaker groups for selective comparison. 

Overall, our comparisons include monolingually raised speakers of English (USmo) or German (DEmo) as well as HSs of Greek, Russian or Turkish dominant in English (USbiGreek, USbiRuss, USbiTurk) or dominant in German (DEbiGreek, DEbiRuss, DEbiTurk). \tabref{tab:kelleretal:sumstatstokencount} displays the number of speakers per speaker group, along with summary statistics on the token count calculated across the elicitations of all speakers per speaker group. A comparison of the mean token count across groups and languages indicates that the USbiGer group has the lowest mean token count of all German groups, while, as expected, their mean token counts are similar to the other US groups, which consist of further majority English as well as English monolingual speakers. The average token count ranges from 111.51 to 168.39 tokens per elicitation session, with considerable variation within all groups (\textit{SD} = 55.45--89.16), irrespective of acquisition type and language (see also \cite{shadrova2024LexDiv}).

\begin{table}[ht]
\caption{Summary statistics of speaker group \& token count}
\begin{tabularx}{\textwidth}{llrrrrrrr}
    \lsptoprule
    Language & Speaker & Group & Mean & Median & SD & Min & Max \\ 
    ~ & Group & Size & ~ & ~ & ~ & ~ & ~ \\ 
    \midrule
    German & DEmo & 64 & 159.93 & 142.50 & 80.84 &  34 & 524 \\
    ~ & USbiGer & 36 & 111.51 & 96.00 & 56.90 &  34 & 312 \\
    ~ & DEbiGreek & 45 & 134.02 & 121.50 & 57.04 &  42 & 420 \\
    ~ & DEbiRuss & 61 & 168.39 & 146.50 & 83.85 &  43 & 682 \\ 
    ~ & DEbiTurk & 65 & 150.21 & 137.50 & 74.54 &  36 & 595 \\ 
    \midrule
    English & USmo & 64 & 124.34 & 111.00 & 55.45 &  36 & 305 \\
    ~ & USbiGer & 34 & 131.80 & 120.00 & 59.33 &  46 & 318 \\
    ~ & USbiGreek & 65 & 128.48 & 117.00 & 60.85 &  40 & 412 \\ 
    ~ & USbiRuss & 65 & 149.26 & 133.00 & 89.16 &  38 & 880 \\ 
    ~ & USbiTurk & 59 & 143.46 & 130.00 & 66.73 &  37 & 446 \\ 
    \lspbottomrule
\end{tabularx}
\label{tab:kelleretal:sumstatstokencount}
\end{table}

%%%%%%%%%%%%%%%%%%%%%%%%%%%%%%%%%
%%%%%%%%%%%%%%%%%%%%%%%%%%%%%%%%%
%%%%%%%% start section 4 %%%%%%%%
%%%%%%%%%%%%%%%%%%%%%%%%%%%%%%%%%
%%%%%%%%%%%%%%%%%%%%%%%%%%%%%%%%%

\section{Lexical diversity and inventory} \label{sec:kelleretal:lex-div-inventory}

Our analysis begins with lexical diversity (LD), a common measure to determine language dominance in bilinguals \citep{TreffersKorybski2016Measures} as well as to examine language proficiency \citep{MalvernRichads2002AccommodationLD, Jarvis2013DiversityinLD}. Whereas LD measures have been applied to different types of language data, the present contribution uses retellings of events as a basis for assessing LD, which have been deemed useful for measuring lexical knowledge, also cross-linguistically \citep[321]{SimonCereijidoGutiérrezClellen2009LexicalGrammatical}. To assess LD both in English and in German, we employed the Moving Average Type-Token Ratio (MATTR, \citealt{ConvingtonMcFall2010MATTR}), which is considered a suitable LD measure for short texts of the type available in the RUEG data \citep{ZenkerKyle2021MinimumText},\footnote{Analyses were also performed with the Measure of Textual Lexical Diversity (MTLD, \citealt{MccarthyJarvis2010MTLD}), which do not show substantially different results.} on lemmatized tokens.\footnote{The lemmatized tokens include all content and function words, as well as repetitions or repairs but exclude hesitations and non-verbal material.} Preliminary descriptive analyses, the research design (i.e., the nature of the data), and theories about differences in LD between HSs and MSs (e.g., \citealt{BonvinEtAl2018Entwicklung,GharibiBoers2019LexicalRichness}) lead to the formulation of three linear mixed-effects models with MATTR as the dependent variable and contrast-coded independent variables \citep{Schadetal2019contrasts}, which are explained in \tabref{tab:kelleretal:independentvars}. The final model structures are given in \tabref{tab:kelleretal:modelstructure}.\footnote{All quantitative analyses discussed in the present section are available in the Open Science Framework project \href{https://osf.io/k89dc/?view_only=c2521da400ee4edd9667c264c5fad6ea}{``Quantitative Analyses of the Lexical Diversity and Lexical Inventory of Heritage Speakers in the RUEG Corpus''}. The analyses were implemented with R \citep{Rcitation} using the following packages: tidyverse \citep{tidyverse}, lme4 \citep{lme4}, emmeans \citep{emmeans}, sjPlot \citep{sjPlot}, \textsc{MASS} \citep{MASS}, hypr \citep{hypr}, performance \citep{performance}, kableExtra \citep{kableExtra}, and ggpubr \citep{ggpubr}.}

\begin{table}[!ht]
    \caption{Independent variables}
    \begin{tabular}{lll}
        \lsptoprule
        Speaker type & \multicolumn{2}{l}{German HSs with English as the MajL (USbiGer)}\\
        ~ & \multicolumn{2}{l}{majority language speaker (DE-/USbi),} \\
        ~ & \multicolumn{2}{l}{monolingual speaker (DE-/USmo),} \\
        Formality & \multicolumn{2}{l}{formal, informal} \\
        Mode & \multicolumn{2}{l}{spoken, written} \\
        Session & \multicolumn{2}{l}{first, second, third, fourth elicitation session} \\
        Language (=lang) & \multicolumn{2}{l}{German, English} \\
        Language order & \multicolumn{2}{l}{MajL-HL, HL-MajL}\\
        ID & \multicolumn{2}{l}{unique speaker identifier (e.g. USbi72FD) composed of\dots} \\
        ~ & \dots Elicitation country & DE (Germany), GR(eece), \\
        ~ & ~ & RU(ssia), TU(rkey), US(A) \\
        ~ & \dots Acquisition type & mo(nolingual), bi(lingual) \\
        ~ & \dots Age group & 01--49 (adolescent), 50--99 (adult) \\
        ~ & \dots Gender & F(emale), M(ale), X (diverse) \\
        ~ & \dots HL/L1 & D (German), E(nglish), G(reek), \\
        ~ & ~ & R(ussian), T(urkish) \\
        \lspbottomrule
    \end{tabular}
    \label{tab:kelleretal:independentvars}
\end{table}

\begin{table}[!ht]
    \caption{Final model structures}
    \small
    \fittable{\begin{tabular}{ll}
        \lsptoprule
        \texttt{1:} & \texttt{mattr $\sim$ speakertype * formality * mode + session + (1 | ID),} \\
        \texttt{~} & \texttt{data=German}\\
        \texttt{2:} & \texttt{mattr $\sim$ speakertype * formality * mode + session + (1 | ID),} \\
        \texttt{~} & \texttt{data=English} \\
        \texttt{3:} & \texttt{mattr $\sim$ lang + formality + mode + session + language\_order + (1 | ID)}, \\
        \texttt{~} & \texttt{data=USbiGer}\\
        \lspbottomrule
    \end{tabular}}
    \label{tab:kelleretal:modelstructure}
\end{table}

Models 1 and 2 evaluate the MATTR measure across all of the German (Appendix, \tabref{tab:kelleretal:MATTR_final_model_DE}) and the English data (Appendix, \tabref{tab:kelleretal:MATTR_final_model_EN}), respectively.\footnote{At this point, it is important to note that the IV ``speaker type'' is not based (solely) on a theoretical demarcation between different types of HSs. For analytical reasons, we distinguish the USbiGer group from the other DEbi and USbi speaker groups as we focus on this former subgroup, even though from an aquisitional perspective, the USbi and USbiGer speaker groups do not differ from each other apart from the respective HL. Hence, the ``US'' and the ``USbi'' designations exclude the USbiGer speaker group in subsequent analyses.} Additionally, Model 3 was set up to target both German and English only for the USbiGer speakers, the group in focus, and thus includes the IVs ``lang'' instead of ``speakertype'' (Appendix, \tabref{tab:kelleretal:MATTR_final_model_USHGer}). Since there are multiple LD measures per speaker, the variable ``ID'' is included as a random factor in all models. The three models which are reported result from the statistical evaluation of assumption tests and pairwise comparisons between model structures with and without the interactions of interest. The conditional r-squared (R\textsuperscript{2}\textsubscript{c}) values, visible in the model summary tables on the LD measurements, reveal that the models explain between $\sim$48\% and $\sim$56\% of the variance. 

The summary of Model 1 (Appendix, \tabref{tab:kelleretal:MATTR_final_model_DE}) on the LD in the German data shows significant simple effects for speaker type, formality, and mode. \figref{fig:kelleretal:MATTR_formality_mode_speakertype_DE} illustrates a clear gradation pattern between the three speaker types: DEmos display a higher LD compared to DEbis, while these two groups show a higher LD compared to the USbiGer group. Regarding formality and mode, the model indicates lower LD in the formal (opposite to \citealt{VanGijselSpeelmanGeeraerts2005Richness, Alamillo2019LexicalSkills}) and the spoken (in line with \citealt{VanGijselSpeelmanGeeraerts2005Richness}; contrasting \citealt{Yu2009LDWritingSpeaking}), respectively. Furthermore, there is an interaction effect between formality and mode, with lower MATTR values consistently in the spoken condition compared to the written one, however, less of a difference between spoken and written in the formal as opposed to the informal condition. This effect is mainly driven by the written LD values which are considerably lower in the formal condition in comparison to the informal condition. The independent variable "elicitation session" does not show a significant simple effect on the MATTR. There is considerable intra-level variation for the DEmo and USbiGer speaker groups compared to the DEbi speaker group.

Similar to the findings of Model 1, Model 2 on the LD in the English data (summary in \tabref{tab:kelleretal:MATTR_final_model_EN}, Appendix) shows simple and interaction effects for formality and mode. The speaker group comparisons reveal a significant difference between the USbiGer and the US speaker groups, i.e. all other groups in this data set, while the difference between the USmo and USbi groups (here, excluding USbiGers) lacks statistical significance. However, this result might be misleading, as plot C in \figref{fig:kelleretal:MATTR_formality_mode_speakertype_EN} (Appendix) suggests similarities in LD between the USbiGer and the USmo speaker groups but not between the USmo and USbi speaker groups. This contradiction can be explained by the large intra-group variation, estimated by 95\% confidence intervals. From this, we conclude that, despite the contradictory significance values, the USbiGer and the USmo groups have similar LD, whereas the USbi group shows a lower LD, but only by a slight margin.

Taken together, Models 1 and 2 show a clear gradation pattern between the three speaker groups in the German data, with the USbiGers showing the lowest LD values, whereas the English data indicates less gradation and more similarity between speaker types. The results of both models also highlight large intra-group variances for USbiGers, USmos, DEmos, calling into question the meaningfulness of LD as a group measure for both monolinguals and bilinguals in the varying situational and conversational settings. 

Model 3, which only calculates values for the USbiGer group (summary in \tabref{tab:kelleretal:MATTR_final_model_USHGer}, Appendix), reveals significant simple effects of language, formality and mode but no significant simple effect for language order and elicitation session (Appendix, \figref{fig:kelleretal:MATTR_formality_mode_speakertype_USHGer}). This confirms the significant effect of formality and mode present within the USbiGer group, with lower LD in the formal and the spoken condition, respectively. It also reveals that, as expected, the USbiGers as a group demonstrate a higher LD in English, their majority language, than in German, their HL.

Summing up, the MATTR calculations as a proxy for LD show that the number of different lexemes used in the reports making up the RUEG data vary in relation to speaker type, formality and mode. With respect to HSs of German (ie. USbiGer), analyses indicate that their vocabulary used is less diverse than that of the other speaker groups in their HL but equally diverse in their dominant language. Furthermore, the HSs of German show similar patterns in LD in the varying situational and conversational settings, comparing their majority and heritage language as well as the other English and German speaker groups.

As a next step, we looked at the lexical items used by the different speaker groups to get a clearer idea of the lexical inventory (LI). To this end, we conducted distinct- and shared-lemma analyses of three word classes (adjectives, nouns, and verbs) on a descriptive level. First, the size of the LI for German and for English is compared by counting the types\footnote{By \textit{types} we mean unique lemmas, by \textit{tokens} all occurrences, or word forms, of a lemma or class of lemmas in the corpus (see \citealt[10]{PustejovskyBatiukova2019Lexicon}). For example, the current version of the German sub-corpus contains 31.933 occurrences of verbs, i.e. verb tokens. These are instantiations of 1331 different verbs on the lemma layer, i.e. verb types.}, normalized by the number of speakers per group. Second, we focus on lemmas shared across speaker groups. These LI analyses are based on the same speaker groups as the LD analyses, and they target the same variables that appeared to be relevant with respect to LD, namely speaker type, language, formality and mode. Generally, the number of different lemmas a speaker uses has the same effect on both the LD and the LI measures used in this study. In other words, the two measures are positively correlated. Yet, they provide different insights into the speaker’s lexicon. LD is influenced by the lexeme repetition rate \citep[87]{Jarvis2013DiversityinLD}, i.e. it decreases if lexemes are repeated within the specified text span or window. Hence, the LD measure allows us to quantify \textit{how} speakers make use of the lexical inventory at their disposal. The LI value, in turn, is not affected by lexeme repetition and only gives insight into \textit{how many} word types are actively used by a speaker (group) in the four reports and thus indicates that the speakers hold ``\textit{at least} this number of words'' \citep[358]{NationAnthony2016VocabSize} in their repertoire.

The comparison of LI size between speaker groups in the German sub-corpus (\figref{fig:kelleretal:sizelexicalinventory}, plot A) reveals the smallest LI for the USbiGer and DEbiTurk speaker groups, while the largest LI is observed for the DEmo and DEbiRuss speaker groups. The DEbiGreek speaker group lies in-between. Hence, the clear gradation pattern between the DEmo, DEbi and USbiGer speaker groups we saw for the LD analyses is repeated. Furthermore, a small LI in combination with a low LD, as seen for the USbiGer group in the German data, indicates that the smaller set of lemmas the speakers have at their disposal is also used more repetitively compared to the other speaker groups. Looking at the three major word classes of nouns, vers and adjectives separately, the LI calculations show differences between the speaker groups concerning adjectives and nouns. In contrast, the size of the verb inventory is rather consistent across the speaker groups. In this data, the number of distinct adjectives is generally the lowest, while the verb inventory is approximately of the same size or smaller than the noun inventory.
\vfill
\begin{figure}[H]
  \footnotesize
  \renewcommand\thesubfigure{\Alph{subfigure}}
  \begin{subfigure}{\textwidth}
    \centering
    \begin{tikzpicture}
      \begin{axis}[
          ybar stacked,
          axis lines*=left,
          width=\textwidth,
          height=5cm,
          ymin=0,
          ymax=25,
          symbolic x coords={DEmo,USbiGer,DEbiGreek,DEbiRuss,DEbiTurk},
          xtick=data,
          bar width=3em,
          nodes near coords,
%           xlabel = {Speaker group},
          ylabel = {Relative lemma count\\(by speaker)},
          ylabel near ticks,
          ylabel style = {align=center},
          reverse legend,
          legend style={
              at={(0.5,1)},
              anchor=south,
              legend columns=-1
          },
      ]
      \addplot+ [black,fill=lsMidBlue!50, draw=lsMidBlue] coordinates {
        (DEmo,7.64) (USbiGer,8.14) (DEbiGreek,7.93) (DEbiRuss,7.51) (DEbiTurk,6.68)
      };
      \addlegendentry{\textsc{verb}}
      \addplot+ [black, fill=lsSoftGreen!50, draw=lsSoftGreen] coordinates {
        (DEmo,8.89) (USbiGer,6.14) (DEbiGreek,7.42) (DEbiRuss,9.18) (DEbiTurk,6.78)
      };
      \addlegendentry{\textsc{noun}}
      \addplot+ [black, fill=lsRed!50, draw=lsRed] coordinates {
        (DEmo,5.91) (USbiGer,2.72) (DEbiGreek,4.62) (DEbiRuss,5.97) (DEbiTurk,4.14)
      };
      \addlegendentry{\textsc{adj}}
      \end{axis}
    \end{tikzpicture}
    \caption{German}
    \end{subfigure}\smallskip\\
    \begin{subfigure}{\textwidth}
    \centering
    \begin{tikzpicture}
      \begin{axis}[
          ybar stacked,
          axis lines*=left,
          width=\textwidth,
          height=5cm,
          ymin=0,
          ymax=25,
          symbolic x coords={USmo,USbiGer,USbiGreek,USbiRuss,USbiTurk},
          xtick=data,
          bar width=3em,
          nodes near coords,
%           xlabel = {Speaker group},
          ylabel = {Relative lemma count\\(by speaker)},
          ylabel near ticks,
          ylabel style = {align=center},
      ]
      \addplot+ [black,fill=lsMidBlue!50, draw=lsMidBlue] coordinates {
        (USmo,4.36) (USbiGer,5.88) (USbiGreek,4.12) (USbiRuss,4.71) (USbiTurk,4.58)
      };
%       \addlegendentry{VERB}
      \addplot+ [black, fill=lsSoftGreen!50, draw=lsSoftGreen] coordinates {
        (USmo,4.44) (USbiGer,6.47) (USbiGreek,4.22) (USbiRuss,5.28) (USbiTurk,4.73)
      };
%       \addlegendentry{NOUN}
      \addplot+ [black, fill=lsRed!50, draw=lsRed] coordinates {
        (USmo,2.42) (USbiGer,2.71) (USbiGreek,2.45) (USbiRuss,3.03) (USbiTurk,2.51)
      };
%       \addlegendentry{ADJ}
      \end{axis}
    \end{tikzpicture}
    \caption{English}
    \end{subfigure}
% % %     \includegraphics[width=\textwidth]{figures/Ch10_arranged_distinct_lemma_count_DE_EN.pdf}
    \caption{Size of the lexical inventory: Count of adjective, noun, and verb lemma types in the German (A) and English (B) sub-corpus}
    \label{fig:kelleretal:sizelexicalinventory}
\end{figure}
\vfill\pagebreak

The English data (\figref{fig:kelleretal:sizelexicalinventory}, plot B) show quite a different distribution: There are differences between the speaker groups in relative frequencies across all categories, with the lowest relative frequencies in the USbiGreek, USmo, and USbiTurk speaker groups. The highest relative frequency is observed for the USbiGer speaker group, while the USbiRuss speaker group is in between, apart from the LI size of the adjectives, which is similar or higher than the one of the USbiGer group. This again contrasts with the results of the LD analyses which showed similar values for all speaker groups, especially the USmo and USbiGer speakers. 

A comparison of the distribution between speaker groups in the English and German data (\figref{fig:kelleretal:sizelexicalinventory}, plots A and B) shows that the LI used for the reporting task is bigger in German compared to English for all categories. For the USbiGer speaker group, this difference can mainly be attributed to the LI size of the verb inventory, which is considerably lower in the English data. This may largely be caused by the language-specific lemmatization guidelines applied to the data. It mainly affects German particle verbs (as well as nouns) such as \textit{davonrennen} or \textit{vorbeirennen} and and English phrasal verbs like \textit{run off} or \textit{run past} which due to differing orthographic conventions result in varying type counts if the basis for the lemma count is the orthographic word. We refrain from further quantitative cross-linguistic comparisons in this contribution due to this difference in lemmatization. Importantly, a larger LI value does not necessarily relate to the length of the elicited text (compare for instance the USbiRuss and DEbiRuss mean token counts in \tabref{tab:kelleretal:sumstatstokencount} and the respective LI values in \figref{fig:kelleretal:sizelexicalinventory}). Further, a larger LI value for a group does not necessarily imply that each individual speaker within this group uses more different lemmas than speakers of the other groups. A close look at the USbiGer speaker group shows, for instance, that the high average of different lemmas can be traced back to the high variety of nouns and verbs \textit{within} the group. In other words, the USbiGer speakers are very heterogeneous with respect to their choice of lexical items, whereas, for instance, the USmo group shares more lemmas.

The observed heterogeneity within the groups concerning lexical choice suggests two avenues to fathom the dynamic properties of the HS lexicon: (a) a closer look at the speakers behavior within a group (see Sections \ref{sec:kelleretal:ger-part-verbs} and \ref{sec:kelleretal:lexproduction}); (b) an analysis of the intra- and inter-group lexical overlap, or ``sharedness''.

We operationalize ``sharedness'' as the percentage of lemmas used by at least one speaker from group X and one speaker from group Y. Sharedness is positively correlated with LI size: The larger the LI of a group, the higher the chance for any lexeme in the inventory to overlap with a lexeme in the inventory of another group.

The sharedness calculations for the German data reveal that the USbiGer group shares the smallest number of lemmas with each of the other groups. (Appendix, Tables \ref{tab:kelleretal:all_Kable_TriangleMatrix_DE}--\ref{tab:kelleretal:VERB_Kable_TriangleMatrix_DE}). Hence, the gradation pattern between the three language profiles, or speaker types, USbiGer, DEbi and DEmo, is consistent with the one obtained from the LD analysis. Furthermore, the heterogeneity of the USbiGer group becomes even more evident when looking at the within-group sharedness of lexemes: Apart from the percentage of shared adjectives, where almost all speaker groups show similar percentages, the USbiGer group demonstrates the lowest percentage of shared lemmas in general and specifically for nouns and verbs. 

When we consider the sharedness of lemmas between the two levels of formality or mode (calculated as the percentage of lemmas shared at least once between the two levels of formality or mode, respectively) no clear patterns of sharedness \textit{between} groups arise. However, \textit{within} each speaker group, more lemmas, except for verbs, are shared between the two modes than between the two levels of formality (\figref{fig:kelleretal:arrangedformalityDE}). This suggests that speakers of all groups, including the USbiGers, select lexemes based more on formality than on mode. Additionally, particularly the number of nouns shared between the two levels of formality shows a negative correlation with LI size. In other words, a speaker group which uses larger number of nouns overall (see \figref{fig:kelleretal:sizelexicalinventory}, plot A), such as the DEmoGer and DEbiRuss group, uses more of these nouns in a setting-specific way, as opposed to, for instance, the USbiGer group with a smaller noun inventory and a higher percentage of shared noun lemmas between the formal and informal settings.

\begin{figure}[hp]
  \footnotesize
  \renewcommand\thesubfigure{\Alph{subfigure}}
  \begin{subfigure}{\textwidth}
    \centering
    \begin{tikzpicture}
      \begin{axis}[
          ybar,
          axis lines*=left,
          width=\textwidth,
          height=4cm,
          ymin=0,
          ymax=45,
          symbolic x coords={DEmo,USbiGer,DEbiGreek,DEbiRuss,DEbiTurk},
          xtick=data,
          bar width=3mm,
          nodes near coords,
          nodes near coords style = {rotate=90,anchor=east},
%           xlabel = {Speaker group},
          ylabel = {\% shared lemmas},
          ylabel near ticks,
          ylabel style = {align=center},
          legend style={
              at={(0.5,1)},
              anchor=south,
              legend columns=-1
          },
      ]
      \addplot+ [black,fill=lsRed!50, draw=lsRed] coordinates {
        (DEmo,21.96) (USbiGer,26.53) (DEbiGreek,32.21) (DEbiRuss,21.70) (DEbiTurk,28.62)
      };
      \addlegendentry{\textsc{adj}}
      \addplot+ [black, fill=lsSoftGreen!50, draw=lsSoftGreen] coordinates {
        (DEmo,24.08) (USbiGer,33.48) (DEbiGreek,35.03) (DEbiRuss,27.50) (DEbiTurk,34.01)
      };
      \addlegendentry{\textsc{noun}}
      \addplot+ [black, fill=lsMidBlue!50, draw=lsMidBlue] coordinates {
        (DEmo,35.79) (USbiGer,33.45) (DEbiGreek,35.57) (DEbiRuss,39.30) (DEbiTurk,40.09)
      };
      \addlegendentry{\textsc{verb}}
      \end{axis}
    \end{tikzpicture}
    \caption{formality}
    \end{subfigure}\smallskip\\
    \begin{subfigure}{\textwidth}
    \centering
    \begin{tikzpicture}
      \begin{axis}[
          ybar,
          axis lines*=left,
          width=\textwidth,
          height=4cm,
          ymin=0,
          ymax=45,
          symbolic x coords={DEmo,USbiGer,DEbiGreek,DEbiRuss,DEbiTurk},
          xtick=data,
          bar width=3mm,
          nodes near coords,
          nodes near coords style = {rotate=90,anchor=east},
          ylabel = {\% shared lemmas},
          ylabel near ticks,
          ylabel style = {align=center},
      ]
      \addplot+ [black,fill=lsRed!50, draw=lsRed] coordinates {
        (DEmo,30.69) (USbiGer,34.69) (DEbiGreek,39.42) (DEbiRuss,32.14) (DEbiTurk,34.57)
      };
%       \addlegendentry{\textsc{adj}}
      \addplot+ [black, fill=lsSoftGreen!50, draw=lsSoftGreen] coordinates {
        (DEmo,37.43) (USbiGer,44.80) (DEbiGreek,44.01) (DEbiRuss,40.00) (DEbiTurk,41.95)
      };
%       \addlegendentry{\textsc{noun}}
      \addplot+ [black, fill=lsMidBlue!50, draw=lsMidBlue] coordinates {
        (DEmo,39.88) (USbiGer,36.52) (DEbiGreek,39.50) (DEbiRuss,43.89) (DEbiTurk,43.09)
      };
%       \addlegendentry{\textsc{verb}}
      \end{axis}
    \end{tikzpicture}
    \caption{mode}
    \end{subfigure}
% % %     \includegraphics[width=\textwidth]{figures/Ch10_arranged_formality_mode_DE.pdf}
    \caption{Adjective, noun, and verb lemma types in the German sub-corpus: Shared lemma types across formality (A) and mode (B)}
    \label{fig:kelleretal:arrangedformalityDE}
\end{figure}

The proportion of shared lemma types between the different speaker groups in the English data (Appendix, Tables \ref{tab:kelleretal:all_Kable_TriangleMatrix_EN}--\ref{tab:kelleretal:VERB_Kable_TriangleMatrix_EN}) is higher than in the German data. Whereas proportions for German range between 15\% and 41\%, those for English lie between 27\% and 62\%. The pairwise comparisons between the USbiGer, the other majority English, and the monolingual English speaker groups return no substantial differences concerning the numbers of shared items between and within groups, which is not surprising as the USbiGer speakers are majority English speakers and are thus of the same speaker type as the other USbi groups, apart from the analytical distinction made in this contribution.
\largerpage

\begin{figure}[hp]
  \footnotesize
  \renewcommand\thesubfigure{\Alph{subfigure}}
  \begin{subfigure}{\textwidth}
    \centering
    \begin{tikzpicture}
      \begin{axis}[
          ybar,
          axis lines*=left,
          width=\textwidth,
          height=4cm,
          ymin=0,
          ymax=65,
          symbolic x coords={USmo,USbiGer,USbiGreek,USbiRuss,USbiTurk},
          xtick=data,
          bar width=3mm,
          nodes near coords,
          nodes near coords style = {rotate=90,anchor=east},
%           xlabel = {Speaker group},
          ylabel = {\% shared lemmas},
          ylabel near ticks,
          ylabel style = {align=center},
          legend style={
              at={(0.5,1)},
              anchor=south,
              legend columns=-1
          },
      ]
      \addplot+ [black,fill=lsRed!50, draw=lsRed] coordinates {
        (USmo,27.74) (USbiGer,22.83) (USbiGreek,25.79) (USbiRuss,30.46) (USbiTurk,31.76)
      };
      \addlegendentry{\textsc{adj}}
      \addplot+ [black, fill=lsSoftGreen!50, draw=lsSoftGreen] coordinates {
        (USmo,38.03) (USbiGer,36.82) (USbiGreek,40.15) (USbiRuss,38.48) (USbiTurk,42.29)
      };
      \addlegendentry{\textsc{noun}}
      \addplot+ [black, fill=lsMidBlue!50, draw=lsMidBlue] coordinates {
        (USmo,43.73) (USbiGer,49.00) (USbiGreek,50.00) (USbiRuss,47.06) (USbiTurk,50.37)
      };
      \addlegendentry{\textsc{verb}}
      \end{axis}
    \end{tikzpicture}
    \caption{formality}
    \end{subfigure}\smallskip\\
    \begin{subfigure}{\textwidth}
    \centering
    \begin{tikzpicture}
      \begin{axis}[
          ybar,
          axis lines*=left,
          width=\textwidth,
          height=4cm,
          ymin=0,
          ymax=65,
          symbolic x coords={USmo,USbiGer,USbiGreek,USbiRuss,USbiTurk},
          xtick=data,
          bar width=3mm,
          nodes near coords,
          nodes near coords style = {rotate=90,anchor=east},
%           xlabel = {Speaker group},
          ylabel = {\% shared lemmas},
          ylabel near ticks,
          ylabel style = {align=center},
      ]
      \addplot+ [black,fill=lsRed!50, draw=lsRed] coordinates {
        (USmo,40.65) (USbiGer,41.30) (USbiGreek,34.59) (USbiRuss,39.59) (USbiTurk,44.59)
      };
%       \addlegendentry{\textsc{adj}}
      \addplot+ [black, fill=lsSoftGreen!50, draw=lsSoftGreen] coordinates {
        (USmo,51.06) (USbiGer,48.64) (USbiGreek,46.72) (USbiRuss,51.60) (USbiTurk,57.35)
      };
%       \addlegendentry{\textsc{noun}}
      \addplot+ [black, fill=lsMidBlue!50, draw=lsMidBlue] coordinates {
        (USmo,55.91) (USbiGer,62.00) (USbiGreek,58.21) (USbiRuss,57.19) (USbiTurk,60.74)
      };
%       \addlegendentry{\textsc{verb}}
      \end{axis}
    \end{tikzpicture}
    \caption{mode}
    \end{subfigure}
% % %     \includegraphics[width=\textwidth]{figures/Ch10_arranged_formality_mode_EN.pdf}
    \caption{Adjective, noun, and verb lemma types in the English sub-corpus: Shared lemma types across formality (A) and mode (B)}
    \label{fig:kelleretal:arrangedformalityEN}
\end{figure}

With respect to formality and mode \textit{within} each group (\figref{fig:kelleretal:arrangedformalityEN}), verbs are consistently shared most, followed by nouns and then adjectives. In contrast to the high variability observed in the German data, this holds true for all speaker groups. As in the German data, a higher number of lemma types, irrespective of category, is shared between the two modes than between the two levels of formality. Yet, the inverse relationship between sharedness and LI size cannot be observed in the English data.

This quantitative and descriptive glance at the LI shows that the number of different lemmas used in German appears to be larger than in English across all groups speaking the language. However, as discussed, this could primarily be due to differences in lemmatization conventions of morphologically complex lexemes between the English and German sub-corpora. In addition, a further mitigating factor is speaker group heterogeneity, specifically observed in the sharedness analysis for the USbiGer group. The particularly low number of shared lexemes \textit{within} the USbiGer speaker group, as shown in Tables \ref{tab:kelleretal:all_Kable_TriangleMatrix_DE}--\ref{tab:kelleretal:VERB_Kable_TriangleMatrix_DE} (Appendix), suggests a high within-group variability that can be traced back to idiosyncratic rather than generalizable speaker group behavior.

Moreover, in both the German and English data, a larger number of lemmas is observed to be shared between modes than between levels of formality within groups, except for the verb category in the German data where the values are similar. This may be due to a between-elicitation priming effect since the spoken and written elicitation session within each formality setting were conducted consecutively, whereas the switch between formality settings was accompanied by separate introductions and a short break. 

Our comparison of the three largest lexical classes showed that verbs are shared more often between groups than nouns or adjectives with similar inventory sizes between groups in the German but not in the English data. These findings are particularly interesting in light of \citet{FridmanMeir2023LexProdInnov}'s research who conclude that ``noun performance was more likely to diverge from the baseline, while verb performance followed a more monolingual-like trend'' \citep[890]{FridmanMeir2023LexProdInnov}, which supports \citet{Polinsky2005WordClassIncompGrammar}'s argument that ``it is less ‘costly’ for an incomplete learner to lose a noun than it is to lose a verb'' \citep[430]{Polinsky2005WordClassIncompGrammar}. To explore this finding further and to examine whether differences in the lexical repertoire result mainly from differences in lexical choice, we select verbs for a close-up analysis. In the following we compare heritage German and monolingual German speakers. In addition, we decided on starting with a narrow focus on an intricate phenomenon: German particle verbs.

%%%%%%%%%%%%%%%%%%%%%%%%%%%%%%%%%
%%%%%%%%%%%%%%%%%%%%%%%%%%%%%%%%%
%%%%%%%% start section 5 %%%%%%%%
%%%%%%%%%%%%%%%%%%%%%%%%%%%%%%%%%
%%%%%%%%%%%%%%%%%%%%%%%%%%%%%%%%%

\section{Challenges in the inventory: German particle verbs} \label{sec:kelleretal:ger-part-verbs}

The aim of this section is to show how HSs of German handle the syntactic, semantic and pragmatic challenges posed by particle verbs (PVs) which were laid out in \sectref{sec:kelleretal:lextheory}. We show how a quantitative analysis of the lexicon like the one presented in \sectref{sec:kelleretal:lex-div-inventory} can be augmented by qualitative explorations into a specific aspect of the lexicon to broaden our understanding of the details that characterize the vocabulary choices of heritage speakers of German.

As a preparatory step, we divided all verbs in the heritage German and the monolingual corpus into three groups, according to their morphological characteristics: simplex verbs, prefix verbs (i.e. verbs containing a non-separable prefix) and PVs with a separable particle. Of the 31.933 verb tokens in the German subcorpus, only 16\% are PVs – which is more than the 7\% prefix verbs but very little compared to the 77\% made up by simplex verbs. However, in terms of verb types, or lemmas, the 598 different PV types in the corpus make up 57\%, compared to 271 simplex verbs (26\%) and 178 prefix verbs (17\%). There is no question, then, that they play an important role in verb inventories of our speakers.

The events in the stimulus lend themselves to coding options via particle verbs (see \sectref{sec:kelleretal:lextheory}). Their description requires the identification of conceptual primitives (see \citealt{Talmy1972Basis, Slobin2003Language}): figures, types of ground (street, parking lot, sidewalk), types of motion (driving, walking, running, rolling, falling, etc.) along various paths, with and without an already perceivable or imagined goal. For the sake of exposition we narrow our focus even more and turn to the most prominent German motion verb in the stimulus: \textit{fahren} (`drive'). Despite this concentration our observations also apply to other verbs in the corpus. Driving events are central to the storyline of the video and account for a large number of types and tokens across speaker groups and communicative situations. Examples and numbers are all based on event descriptions from two subgroups of speakers, namely HSs of German in the US (USbiGer, $N=36$) and German monolinguals in Germany (DEmo, $N=64$). The data contain 138 tokens of \textit{fahren} as a simplex verb, plus 369 tokens and 26 types of PVs with \textit{fahren} as their base.\largerpage

In the case of polysemous PVs, meaning can only be determined in context. Take the PV \textit{anfahren}, which is among the top-five most frequently used PVs in both speaker groups. According to one of the major German dictionaries, Duden online, the verb \textit{anfahren} has nine clearly differentiated subsenses, most of which cannot be inferred from the combined meanings of the particle and the verb. They are as diverse as 1. beginning to move, 2. approaching, 3. rearending, or 4. angrily shouting at someone. In the USbiGer data, \textit{anfahren} is used primarily in the sense of approaching and entering new ground, such as turning into a street or a parking lot (Duden online, Subsense 2; Example \ref{ex:kelleretal:angefahren}). This is perfectly idiomatic and is also used by our DEmo speakers. However, in the DEmo data, \textit{anfahren} is used much more often in the sense of accidentally hitting a target while driving a vehicle (Duden online, Subsense 6; Example \ref{ex:kelleretal:anzufahren}). This sense is barely used by the HSs. Instead, we find the verb \textit{treffen}, a loan translation of the English verb \textit{hit}, as in Example (\ref{ex:kelleretal:treffen}).{\interfootnotelinepenalty=10000\footnote{As mentioned before, in order to preserve a clear focus, we decided against comments on non-canonical realizations like the auxiliary `haben' in Example (\ref{ex:kelleretal:angefahren}) and other non-canonical forms (case, gender) not at issue here.}}

\ea
\label{ex:kelleretal:angefahren}
\gll als die zwei autos \textbf{an}-\textbf{ge}-\textbf{fahr}-\textbf{en} hatten\\
when the two cars on.\Vpart{}-\Ptcp{}-drive-\Ptcp{} have.\Pst.\Tpl{}\\
\glt `when the two cars were approaching' (USbi71MD\_fsD)

\ex
\label{ex:kelleretal:anzufahren}
\gll vermutlich aus Angst den Hund \textbf{an}-\textbf{zu}-\textbf{fahr}-\textbf{en}\\
probably for fear the.\Acc{} dog on.\Vpart{}-to-drive-\Inf{}\\
\glt `probably for fear of hitting the dog.' (DEmo47MD\_fwD)

\ex
\label{ex:kelleretal:treffen}
\gll um den hund nicht zu \textbf{treff-en}\\
for the.\Acc{} dog not to hit-\Inf{}\\
\glt `so as not to hit the dog' (USbi71MD\_fwD)
\z

Hence, even though a pure count of PVs suggests that both heritage speakers and monolinguals make frequent use of the same verb (see \sectref{sec:kelleretal:lex-div-inventory}), there may be subtle but crucial differences on the semantic level, as we proceed to show.

One difference between monolingual and heritage speakers concerns the semantics of particles with respect to argument structure. The two most frequent particles combined with \textit{fahren} are adverb-based \textit{rein} in the USbiGer data and preposition-based \textit{auf} in the DEmo data, along with its adverb-based variants \textit{rauf} and \textit{drauf}. However, both particles are used by either group. The subsense of the PV correlates in interesting ways with either NP complements expressing an affected object or PP adjuncts expressing detail on path of movement. When \textit{reinfahren} is used in the sense of entering new ground, i.e. crossing a boundary, both USbiGer (Example \ref{ex:kelleretal:in_reingefahren}) and DEmo speakers (Example \ref{ex:kelleretal:reingefahren_in}) frequently add a PP complement. However, most of the time the PV \textit{reinfahren} is used in the sense of rearending another car, a telic event and a momentary achievement in the sense of \citet{Vendler1957}. In these cases, DEmo speakers prefer dative NP complements (20 out of 31 clauses; Example \ref{ex:kelleretal:fuhr_rein}), while USbiGer speakers almost exclusively choose PP complements (33 out of 35 clauses; Example \ref{ex:kelleretal:dem_reingefahren}).


\ea
\label{ex:kelleretal:in_reingefahren}
\gll das blaue auto was \textbf{in} \textbf{den} \textbf{parkplatz} \textbf{rein}-\textbf{ge}-\textbf{fahr}-\textbf{en} ist\\
the blue car which in the.\Acc{} parking.lot into.\Vpart{}-\Ptcp{}-drive-\Ptcp{} be.\Prs.\Tsg{}\\
\glt `the blue car which drove into the parking lot' (USbi72FD\_isD)

\ex
\label{ex:kelleretal:reingefahren_in}
\gll und zwei autos sind \textbf{rein-ge-fahr-en} \textbf{in} \textbf{den} \textbf{parkplatz}\\
and two cars be.\Prs.\Tpl{} into.\Vpart{}-\Ptcp{}-drive-\Ptcp{} in the.\Acc{} parking.lot\\
\glt `and two cars drove into the parking lot' (Demo38FD\_isD)

\ex
\label{ex:kelleretal:fuhr_rein}
\gll der eine is \textbf{dem} \textbf{ander-en} hinten \textbf{rein-ge-fahr-en}\\
the one be.\Prs.\Tsg{} the.\Dat{} other-\Dat{} in.back into.\Vpart{}-\Ptcp{}-drive-\Ptcp{}\\
\glt `and the one rearended the other one' (DEmo19FD\_isD) 

\ex
\label{ex:kelleretal:dem_reingefahren}
\gll Und dann \textbf{fuhr} das zweite auto \textbf{in} \textbf{das} \textbf{erste} \textbf{rein}\\
and then drive.\Pst.\Tsg{} the second car in the first into.\Vpart{}\\
\glt `and then the second car rearended the first one' (USbi04FD\_fwD)

\z

The encoding of path information in German does not necessarily require a PV. It may also be expressed by a PP. When direction or goal of motion is already expressed by a simplex verb combined with a PP, adding a corresponding particle to the verb can be semantically redundant. At the same time, it is perfectly canonical in German to do so. It is quite interesting, therefore, that both speaker groups differ with respect to double marking in connection with motion events. In the DEmo data, simplex \textit{fahren} together with a directional \textit{auf}-PP is used to describe entering new ground, for example a parking lot (as in Example \ref{ex:kelleretal:auf_gefahren}). Double marking critical subevents of motion with the help of the verbal particle \textit{auf}- plus an \textit{auf}-PP is used when the focus is on the endpoint of the motion event, i.e. when the event is telic, as in our texts describing the second car hitting the first one (Example \ref{ex:kelleretal:auf_aufgefahren}). In other word, monolingual speakers use single versus double marking to convey subtle semantic differences in motion events. 

\ea
\label{ex:kelleretal:auf_gefahren}
\gll ein Auto, was gerade \textbf{auf} \textbf{den} \textbf{Parkplatz} \textbf{ge-fahr-en} ist\\
a car which just on the parking.lot \Ptcp{}-drive-\Ptcp{} be.\Prs.\Tsg\\
\glt `a car which was just driving into the parking lot' (DEmo88FD\_fwD)

\ex
\label{ex:kelleretal:auf_aufgefahren}
\gll Aufgrunddessen ist das hintere \textbf{auf} \textbf{das} \textbf{vordere} \textbf{Auto} \textbf{auf-ge-fahr-en}\\
Because.of.this be.\Prs.\Tsg{} the back on the.\Acc{} front car on.\Vpart{}-\Ptcp{}-drive-\Ptcp{}\\
\glt `For this reason, the car in back rearended the car in front' (DEmo53FD\_fwD)
\z

In the USbiGer data, the simplex \textit{fahren}, together with a directional PP, is used to describe entering new ground, like the parking lot in the stimulus video. Most often, however, the PV \textit{reinfahren} is used, along with the matching preposition \textit{in} (Example \ref{ex:kelleretal:reingefahren_im}).\footnote{The form of German verbal particles derived from prepositions is usually exactly the same as the preposition. Only the preposition \textit{in} changes to \textit{ein-} when it is used as a verbal particle. The same holds for the deictic adverbials derived from the preposition \textit{in} (\textit{herein, hinein, rein}).} The same construction of a PV together with a directional PP is used for describing the act of rearending another car (Example \ref{ex:kelleretal:dem_reingefahren}). 

\ea
\label{ex:kelleretal:reingefahren_im}
\gll Und es waren zwei autos die \textbf{im} \textbf{pa/} \textbf{parkplatz} \textbf{rein-ge-fahr-en} sind\\
and it be.Pst.\Tpl{} two cars which in.\Dat{} {} parking.lot into.\Vpart{}-\Ptcp{}-drive-\Ptcp{} be.\Prs.\Tpl{} \\
\glt `and there were two cars entering the parking lot' (USbi72FD\_fwD)
\z

Examples like (\ref{ex:kelleretal:reingefahren_im}) suggest that in the USbiGer data, entering new ground and endpoint orientation of motion verbs are lexicalized in the same way. Looking at other PVs as well, double marking of direction and endpoint of motion is the preferred option in the USbiGer data. Since \citet{PashkovaHodgeShanley2020Explicitness} have shown evidence for increased explicitness in heritage speaker productions in their majority language, double marking in their heritage language may reflect this as well. However, in the case of German-English bilinguals, structural parallelism (\textit{drive}/\textit{run} + PP) and potential cross-linguistic effects must not be disregarded either.

Since previous research suggests that lexical choice according to register differentiation is particularly challenging, we now briefly turn to the use of PVs in specific communicative situations (formality, mode). The most frequent PV with the base \textit{fahren} in the DEmo data, \textit{auffahren} occurs most often in the formal-written setting (see \ref{ex:kelleretal:auffahren}). In the informal settings, the particle is often realized as \textit{rauf}, evoking direction or target without overtly combining with with a deictic argument, resulting in \textit{rauffahren} (see \ref{ex:kelleretal:rauffahren}). 

\begin{exe}
    \ex {auffahren}: fw (40) \textgreater{} fs (28) \textgreater{} iw (16) \textgreater{} is (16) \label{ex:kelleretal:auffahren}
    \ex {rauffahren}: is (30) \textgreater{} iw (18) \textgreater{} fs (14) \textgreater{} fw (6) \label{ex:kelleretal:rauffahren}
\end{exe}

Even though the quantitative analysis of shared lemmas across different variables in \sectref{sec:kelleretal:lex-div-inventory} shows that the USbiGer speakers as a group are sensitive to formality distinctions in their selection of lexical items, differentiation according to formality and mode is not found for the most frequent \textit{fahren} PV in the data; \textit{reinfahren} is used across all four communicative situations for referring to the telic event of rearending a car. While this is perfectly acceptable in colloquial German, in more formal situations, like providing a witness report, \textit{auffahren} would seem more appropriate. Having grown up in Germany, DEmo speakers are more likely to know that the PV \textit{auffahren} is used specifically to describe the rearending of a car and that in official accounts accidents like this would be referred to as \textit{Auffahrunfall}.\footnote{Duden online lists rearending another car as the first subsense of \textit{auffahren}, whereas this subsense is not listed at all for (\textit{he}-)\textit{reinfahren}. Nevertheless, whether or not a PV is listed in the Duden does not say much about actual use. As the formation of PVs is productive and often transparent, the Duden mostly lists them once a specific meaning has become lexicalized.}  In contrast, many of our US-based HS participants are not likely to have encountered reference to many car accidents in German, especially not in formal contexts. Additionally, the frequent choice of \textit{reinfahren} might be influenced by the speakers’s dominant language, English: \textit{Reinfahren} combines with the preposition \textit{in}, which, as a homophonous diamorph, facilitates the transition between languages (see \cites{Clyne1967Transference}[133]{Muysken2000Bilingual}). Example (\ref{ex:kelleretal:reinfahren_homophdiamorph}) is a case in point.

\ea
\label{ex:kelleretal:reinfahren_homophdiamorph}
\gll als die zwei autos (-) die in einer (-) anderen straße fuhr-en (-) \textbf{in} ähm (--) \textbf{the} (-) \textbf{parking} \textbf{lot} \textbf{ein-ge-bog-en} sind \\
when the two cars {**} which in a.\Dat{} {**} different.\Dat{} street drive-\Pst.\Tpl{} {**} in {**} {**} the {**} parking lot in.\Vpart{}-\Ptcp{}-turn-\Ptcp{} be.\Prs.\Tpl{} \\
\glt `when the two cars which were driving along a different street turned into the parking lot' (USbi65MD\_fsD)
\z

The speaker chooses the particle \textit{ein}- together with the base \textit{biegen}, a canonical construction in German, but the complement of the PP, encoding the goal of the directed motion, is realized in English. The preposition itself is unspecified for language: It can be German or English or both at the same time. Furthermore, the utterance contains short pauses and hesitation particles, hence exactly the type of production phenomena we turn to in \sectref{sec:kelleretal:lexproduction}. 

This short illustration of the intricacies of German PVs that a speaker is faced with shows several minute differences between HSs and MSs which escape a quantitative assessment based on type and token counts. Given what is known about the early appearance of particles and particle verbs (see \cite{chapters/05}), it comes as no surprise that HSs have no problem with the most notorious syntactic feature of PVs, namely their distributional properties. Nevertheless, in the case of the specific motion verbs considered, individual heritage speakers arrive at slightly different conclusions than monolingual speakers of German with respect to particle choice, both in terms of meaning and register. The following third section of our analyis looks at production phenomena with the question in mind what they reveal to us about moments of choice. 

%%%%%%%%%%%%%%%%%%%%%%%%%%%%%%%%%
%%%%%%%%%%%%%%%%%%%%%%%%%%%%%%%%%
%%%%%%%% start section 6 %%%%%%%%
%%%%%%%%%%%%%%%%%%%%%%%%%%%%%%%%%
%%%%%%%%%%%%%%%%%%%%%%%%%%%%%%%%%

\section{Producing particle verbs in real time} \label{sec:kelleretal:lexproduction}

The speed with which speakers access their word store and within fractions of a second select from tens of thousands of available options those that fit an intended message is impressive. Thanks to efficient pro- and retroactive monitoring skills speakers can swiftly edit unintended messages and occasional slips of the tongue.\footnote{See \citet{Levelt1989Speaking}'s Main Interruption rule and different motivations for reformulations and repairs. As he stresses, ``[s]peakers can monitor for almost any output of their own speech'' \citep[436]{Levelt1989Speaking}.} In the context of our current discussion, both the very fact of self-initiated interventions in specific places and the overt details in which they play out provide insight into speakers' ``personal'' view on their own messages.

Speakers' self-monitoring manifests itself in various performance phenomena besides the actual reparans (i.e. the correction): interruptions with and without hesitating, syllable lengthening, iterations, and sometimes speaker-specific fillers, such as tongue-clicks. Essentially, these phenomena are proliferous in everybody’s unrehearsed speech (see \citealt{Fromkin1973Speech, HiekeCowallOConnell1983Trouble, Levelt1989Speaking,Clark2002Using, Belz2023filler}). 
The downside is that the identification, transcription and annotation of relevant data involves painstaking attention to phonetic detail and acoustic measurement with respect to timing. As shown by \citet{Belz2021Phonetik}, German hesitation particles alone – in the German data of the RUEG corpus orthographically transcribed as \textit{äh} or \textit{ähm} – occur in many phonetic shapes. Moreover, form is one thing, function another. Hence it is difficult, and may often be downright impossible, to unambiguously attribute a particular phonetic event to a specific challenge speakers face. Nevertheless, as we demonstrate here, along with other publications based on the RUEG corpus dealing with performance issues \citep{BoettcherZellers2023, TracyGibbon2023Timing}, self-initiated changes in utterances are revealing, especially with respect to what they tell us about lexical inventories. Similar to \citet{Levelt1989Speaking}'s elicitation of speech errors in descriptions of paths taken through a visual array, our event narrations yield useful information on what is there to choose from.

In the oral RUEG narrations, regardless of minority or majority speaker status or language, overt and covert performance phenomena are attributable to various types of pro- and retroactive repairs of word selection or message construction, and also to discourse-related motivations (e.g. change of topic, see other chapters of this volume). In the heritage German speaker data, we find an abundance of word iteration and modification in specific, predictable trouble spots related to challenges involving gender and case marking, in amalgamations of articles and prepositions, plural inflection in nouns, auxiliary choice, and participle morphology. None of these come as a surprise, given the eccentric, highly irregular nature of the paradigms involved. Some of these non-canonical features will be seen in the examples below, but we will not draw attention to them unless they are related to our immediate concern. While we maintain our main focus on German particle verbs, we now also include cues in their vicinity that provide insight into local troubleshooting. 

Before we start on our analysis of the HSs productions, it must be pointed out that some of the monitoring phenomena dicussed occur in monolingual speakers as well, as shown in Example (\ref{ex:kelleretal:langgefahren}). The speaker here replaces a partially uttered simplex verb of motion (\textit{fahren}, `drive', our model verb from \sectref{sec:kelleretal:ger-part-verbs}), with a PV. The particle \textit{lang}, a short form of \textit{entlang} `along' requires a deictic adverbial or object NP expressing path information concerning the region along which motion takes place. While reference to location is opaque, the obligatory argument position is filled, resulting in a syntactically well-formed clause conversationally adequate in informal contexts.

\ea
\label{ex:kelleretal:langgefahren}
\gll und dann sind halt (-) zwei autos \textbf{gefah/} \textbf{da} \textbf{lang-ge-fahr-en} \\
and then be.\Prs.\Tpl{} simply {**} two cars {**} there along.\Vpart{}-\Ptcp{}-drive-\Ptcp{} \\
\glt `and then two cars came driving along there'(DEmo57FD\_isD)
\z

Next, consider the formal and informal reports from a German HS in Examples (\REF{ex:kelleretal:hintun}--\REF{ex:kelleretal:hingefahrn}).

\ea
\label{ex:kelleretal:hintun}
\gll die war grad ähm (-) [tcl] äh einkauf-en ge-gang-en und ähm woll-te alles in/ in-s auto ähm \textbf{hin-tun} \\
she be.\Pst.\Tsg{} just {**} {**} {**} {**} shop-\Inf{} \Ptcp{}-go-\Ptcp{} and {**} want-\Pst.\Tsg{} everything {**} in-the.\Acc{} car {**} there.\Vpart{}-put.\Inf{}\\
\glt `she had just been shopping and wanted to put everything in the car' (USbi03FD\_fsD)

\pagebreak
\ex \label{ex:kelleretal:hinpacken}
\gll und ähm (-) sie woll-te ihre sachen im äh (-) auto ähm (-) [tcl] äh (-) \textbf{pa/} \textbf{hin-pack-en} (-) \\
and {**} {**} she want-\Pst.\Tsg{} her stuff in.\Dat{} {**} {**} car {**} {**} {**} {**} {**} {**} there.\Vpart{}-pack.\Inf{} {**}\\
\glt `and she wanted to put her stuff in the car' (USbi03FD\_isD)

\ex \label{ex:kelleretal:hingefahrn} 
\gll äh dieses auto äh muss-te ga/ a/ auch ganz schnell stopp-en und is eigentlich ähm in-s erst/ erstes auto (-) äh \textbf{hin-ge-fahr-n} äh \textbf{rein-ge-fahr-n}\\
{**} this car {**} must-\Pst.\Tsg{} {**} {**} also really fast stop-\Inf{} and be.\Prs.\Tsg{} actually {**} in-the {**} first car {**} {**} there.\Vpart{}-\Ptcp{}-drive-\Ptcp{} {**} into.\Vpart{}-\Ptcp{}-drive-\Ptcp{}\\
\glt `this car had to stop really fast and actually drove to uhm bumped into the first car' (USbi03FD\_fsD)
\z 

We can identify lavishly distributed hesitation particles, reiterations and tongue\hyp clicking leading up to proactively perceived troublespots. But what happens to the verbs? In the first two cases (Examples \REF{ex:kelleretal:hintun} and \REF{ex:kelleretal:hinpacken}), not going for a particle would actually have resulted in perfectly well-formed and contextually adequate expressions: \textit{ins Auto tun} `put into the car' and \textit{ins Auto packen} `pack/load into the car'. However, the result is marginally (\textit{hintun}) or more substantially (\textit{hinpacken}) odd, given the particular container, a car.

In the last case, the speaker produces what by anybody's standard -- and obviously by her own -- is in need of repair (\textit{hingefahrn} $\rightarrow$ \textit{reingefahrn}). Interestingly, both written reports -- controlled typing activities extending the time for corrections -- of the same participant confirm her preference for the structure repaired to in Example \REF{ex:kelleretal:hingefahrn}, for one car hitting the other (Examples \REF{ex:kelleretal:zweties} and \REF{ex:kelleretal:stoppen}).

\ea
\label{ex:kelleretal:zweties}
\gll Und dann ist ein zweites auto im parkplatz \textbf{ge-fahr-en} und ist im erstes auto \textbf{rein-ge-fahr-en}.\\
And then be.\Prs.\Tsg{} a second car in.\Dat{} parking.lot \Ptcp{}-drive-\Ptcp{} and be.\Prs.\Tsg{} in.\Dat{} first car into.\Vpart{}-\Ptcp{}-drive-\Ptcp{}. \\
\glt `And then a second car drove into the parking lot and rearended the first car' (USbi03FD\_fwD)

\ex \label{ex:kelleretal:stoppen}
\gll ein auto muss-te ganz schnell stopp-en und ein anderes auto hat in-s ersten auto \textbf{rein-ge-fahr-en}\\
a car must-\Pst.\Tsg{} really fast stop-\Inf{} and a different car have.\Prs.\Tsg{} in-the first car into.\Vpart{}-\Ptcp{}-drive-\Ptcp{} \\
\glt `a car had to stop really fast and another car rearended the first car' (USbi03FD\_iwD)
\z

With hindsight it is unfortunate that the spectrum of RUEG data elicitation methods did not include tracking self-corrections and timing in written narratives. Nevertheless, the fact that participants repeatedly refer to the same scenes makes it possible to identify candidates for stable lexicalizations differing from conventional items. Examples (\ref{ex:kelleretal:shintergerannt}) and (\ref{ex:kelleretal:whintergerannt}) serve as cases in point. At first sight the non-canonical participle \textit{hintergerannt} (lit. `behind-run', intended `follow', instead of the canonical \textit{hinterhergerannt} `follow') in Example (\ref{ex:kelleretal:shintergerannt}) sounds like a one-shot speech error due to syllable elision. Yet, the same participle recurs in the formal-written scenario, supporting the assumption that \textit{hinterrennen}, most likely strengthened by parallelism with the English \textit{run after}, except for order. though innovative from a canonical perspective, Example (\ref{ex:kelleretal:whintergerannt}) then constitutes an idiosyncratic conventionalized particle verb for this particular speaker.

\ea
\label{ex:kelleretal:shintergerannt}
\gll ... ist ein hund vom rand der sträße (-) andere seite der/ dem ball \textbf{hinter-ge-rann-t} \\
{} be.\Prs.\Tsg{} a dog from.\Dat{} edge the.\Gen{} street {**} other side {**} the.\Dat{} ball after.\Vpart-\Ptcp{}-run-\Ptcp{} \\
\glt `a dog from the curb of the street, the other side, ran after the ball' (USbi64MD\_fsD) 
    
\ex \label{ex:kelleretal:whintergerannt}
\gll ... ist ein Hund von der andere seite der strass den ball \textbf{hinter-ge-rann-t} \\
{} be.\Prs.\Tsg{} a dog from the other side the.\Gen{} street the.\Acc{} ball after.\Vpart-\Ptcp{}-run-\Ptcp{} \\
\glt `a dog from the other side of the street ran after the ball' (USbi64MD\_fwD) 
\z

Quite subtle cross-linguistic interactions can be seen in the following two examples. The particle verb \textit{aufringen} (Example \ref{ex:kelleretal:aufringen}) is documented both in speech and in writing (we only quote the written version here). The particle \textit{auf} is from German and the verb \textit{ringen} is a borrowing from English \textit{to ring} (to call). \textit{Aufringen} may be morphologically possible but is not a verb used in German -- it is a calque modelled after the English phrasal verb \textit{ring up}. Once more, it is the repeated use of this verb which allows us to consider it an innovation enriching the speaker's German repertoire.

\ea
\label{ex:kelleretal:aufringen}
\gll \textbf{ring-t} mich \textbf{auf} \\
ring-\Prs.\Spl{} me.\Acc{} up.\Vpart{} \\
\glt `Ring me up' (USbi77FD\_iwD)
\z

Our final illustration is more intricate. In Example (\ref{ex:kelleretal:hintreibt}) the PV \textit{vor sich hintreiben} (in-front-of oneself there-drive), which in German can take a collocate PP, is arguably based on English \textit{drive}. The PV construction augmented by `vor sich hin' exists in German, but with the meaning of either actively chasing something or passively drifting along. What makes this instance particularly relevant is the speaker’s prompt attempt at a repair of the first calque by resorting to an existing but equally ``off the mark'' PV.

\ea
\label{ex:kelleretal:hintreibt}
\gll ... der einen ball \textbf{vor} \textbf{sich} \textbf{hin} (-) ä:h hum/ (-) \textbf{vor} \textbf{sich} \textbf{hin-treib-t} (--) ä:h oder \textbf{hin-spiel-t}\\
{} who a-\Acc{} ball in.front.of self.\Refl{} there.\Vpart{} {**} {**} {**} {**} in.front.of self.\Refl{} there.\Vpart{}-drive-\Prs.\Tsg{} {**} {**} or there.\Vpart{}-play-\Prs.\Tsg{} \\
\glt `who is driving uhm playing a ball in front of himself uhm playing' (USbi68MD\_isD)
\z

The RUEG narratives provide a plethora of evidence for within-language and cross-linguistic networking, as can be seen in the spontaneous self-corrections presented here. With respect to German particle verbs, heritage speakers have all it takes in terms of the basic building blocks and combinatory principles, i.e. all of the morphological resources they need, and they struggle with details of choice if put on the spot, for example in a challenging experimental situation. 

%%%%%%%%%%%%%%%%%%%%%%%%%%%%%%%%%
%%%%%%%%%%%%%%%%%%%%%%%%%%%%%%%%%
%%%%%%%% start section 7 %%%%%%%%
%%%%%%%%%%%%%%%%%%%%%%%%%%%%%%%%%
%%%%%%%%%%%%%%%%%%%%%%%%%%%%%%%%%

\section{Discussion and conclusion} \label{sec:kelleretal:discussion}

Heritage language research provides us with privileged access to studying which properties of early grammars remain stable when the languages of our childhood are sent to the backstage and exposure decreases. HS data also provide clues to what is likely to change, either due to dynamics of internal language change (such as regularization of irregular word forms), or as a consequence of intensive contact with a specific majority language (cf. \cite{chapters/05}). Our contribution explored the lexicon of HSs, a domain of our linguistic competence which, regardless of speaker type, is highly dynamic: As we stated initially, the lexicon is a moving target.

For a quantitative assessment of lexical inventories we compared different groups of bilingual heritage speakers of Turkish, Russian, Greek and German in their English and German productions as well as the respective non-heritage monolingual speakers (\sectref{sec:kelleretal:lex-div-inventory}). First of all, this revealed considerable intra-group heterogeneity which can be attributed to significant differences in speaker background variables and to different interpretations of the elicitation tasks by the participants. Despite this heterogeneity, group comparisons show that specific subsets of the lexicon are not only comparable in size, \textit{within} a language, but are also shared between the majority and monolingual speaker groups of either German or English. This is the case particularly for the verb inventory in the German data. This is not surprising in light of the findings on verb maintenance for HSs \citep{Polinsky2005WordClassIncompGrammar, FridmanMeir2023LexProdInnov} and further in terms of task demand, since the elicited descriptions relate the same events, while reference to the animate and inanimate protagonists involved is more diverse. 

Additionally, the gradation pattern from MSs via MajS to HSs established by way of LD analysis confirmed our expectations based on previous research regarding reduced HL input and assumptions based on the available speaker metadata. The LD analyses without further qualitative assessment of the individual utterances do not support a tendeny of HSs towards register leveling, in contrast to previous findings on HSs (e.g. \citealt{WieseEtAl2022}) and LD in general \citep{VanGijselSpeelmanGeeraerts2005Richness, Yu2009LDWritingSpeaking, Alamillo2019LexicalSkills}. It is plausible that even though the diversity score (here MATTR) differs between levels of formality and mode, the lemma types embedded in that score do not, as we have seen for the most frequent particle verb \textit{reinfahren} in the HS productions. This indicates that HSs may have a limited register-specific repertoire, yet are still able differentiate between registers by using their resources in a diversified manner.

With respect to LD and LI analyses of data from different languages, a major methodological concern should be mentioned: In German and English contact situations, typological closeness not just creates descriptive challenges for transcribing and annotating the data. Local ambiguity makes automatic lemmatization difficult and requires considerable manual correction based on token-by-token-in-context decisions (see \cite{chapters/02}). However, in the RUEG corpus this detective work is supported by the availability of different texts on the same events produced by the same participants, allowing us to pursue questions relating to local problems (selecting from sets of highly similar particles, word searches, etc.) and individual coping strategies. In the case of particle verbs, our analysis shows how ambiguity due to different intended readings are only resolved by paying close attention to verbal contexts and to the sub-event of the accident focused on by the speaker.

The qualitative analysis of German particle verbs (\sectref{sec:kelleretal:ger-part-verbs}) confirms our initial assumption that the very nature of the input material, i.e. the RUEG video stimulus, is well suited for eliciting verb bases referring to types and manner of motion which select for semantically relevant satellites: PPs, verbal particles, and additional deictic elements satisfying argument positions. Verbal particles offer a substantial, though sometimes only minimally differing inventory of signs for identifying locations, paths, directions and goals, and for turning atelic processes into telic ones. In view of this multitude of formal-functional detail to be worked out in acquisition and managed in real-time tasks, the often-cited syntactically excentric status of combinations of verbs and particles seems downright insignificant. Our data provide no evidence that HSs struggle with the syntactic positioning of verbs and particles, no matter whether they are realized as one continuous string or split up between the left and right sentential bracket. We see, however, that HSs do not always choose the same lexical means as the MSs to convey meaning. We identified subtle differences between heritage and non-heritage speakers in terms of meaning shift and register. Moreover, we find that HSs, more so than monolinguals, tend to express specific subevents redundantly through both particles and prepositional phrases. This finding supports the hypothesis initially mentioned in \citet[294--295]{Polinsky2018HeritageLanguages} that HSs tend to prefer compositionally transparent and explicit formulations.

The findings concerning the analysis of PVs in \sectref{sec:kelleretal:ger-part-verbs} are corroborated by our exploratory discussion of speech production in \sectref{sec:kelleretal:lexproduction}, which shows that particle verbs provide a good starting point for investigating local challenges due to minute contrasts in form, as in word-onsets such as \textit{auffahren}, \textit{rauffahren}, \textit{drauffahren}. Both proactive signals of trouble and the direction and result of self-initiated change supply us with evidence for the individual lexical inventory and for the morphological tools needed for word formation. It cannot be overstated that all these performance phenomena are self-initiated, hence pointing to speaker awareness that alternative expressions were not just available but sometimes called for.

As mentioned in \sectref{sec:kelleretal:lextheory}, particle verbs play an important part in children's early lexicons, and so does, for German-speaking children, the expression of telicity. It may well be the case that the bias towards redundant marking of path and goal discussed here echos child-directed registers. As discussed by \citet[177]{Bryant2018}, parents tend to go for the redundant marking of location or goal, which means via both particles and prepositional phrases: \textit{Immer auf’n Tisch die Schalen draufwerfen} (lit. always on-the table the peels onto-throw, `always throw the peels onto the table'). Also, as stated by \citet[291--328]{Polinsky2018HeritageLanguages}, majority speakers tend to consider the HS variety of the same language pragmatically peculiar and inadequate in view of a speaker’s age. Unfortunately we did not meet our participants in early childhood and had no access to their parental baseline. Hypotheses concerning childhood input are thus waiting to be pursued in future studies.

Heritage speakers help us answer fundamental questions related to language learnability and maintainance: How can humans learn so much even under reduced input conditions and with the L1 under increasing pressure from a dominant, and possibly very similar, hence, distracting, language? As we have shown, heritage language speakers have an important part to play in solving puzzles related to acquisition, language change, and highly competent performance. Since HSs are not lost for words, as shown here, we can be optimistic.

%%%%%%%%%%%%%%%%%%%%%%%%%%%%%%%%%%%%%%%%
%%%%%%%%%%%%%%%%%%%%%%%%%%%%%%%%%%%%%%%%
%%%%%%%% start Acknowledgements %%%%%%%%
%%%%%%%%%%%%%%%%%%%%%%%%%%%%%%%%%%%%%%%%
%%%%%%%%%%%%%%%%%%%%%%%%%%%%%%%%%%%%%%%%

\section*{Acknowledgements} \label{sec:kelleretal:acknowledgements}

All authors gratefully acknowledge funding by the Deutsche Forschungsgemeinschaft (DFG, German Research Foundation) within the Research Unit \textit{Emerging Grammars}, grant number  313607803, Project P11 -- Lexicon. Additionally, Anke Lüdeling acknowledges funding by the Deutsche Forschungsgemeinschaft (DFG, German Research Foundation) – SFB 1412 \textit{Register}. Special thanks are owed to the RUEG team, our Mercator fellows, two anonymous external and our three internal reviewers Natalia Gagarina, Luka Szucsich and Sabine Zerbian, to RUEG cooperation partner Dafydd Gibbon for comments on earlier versions of this paper, to Elena Krotova for continuing support with data annotation and analysis in Python, as well as to our student research assistants Stella Baumann, Joshua Boivin, Anna Kuhn, Nils Picksak, Elena Unger for their tireless work from first data preparation to final proof-reading. We additionally thank Lea Coy for help with pre-publication formatting.

%%%%%%%%%%%%%%%%%%%%%%%%%%%%%%%%%%%%%
%%%%%%%%%%%%%%%%%%%%%%%%%%%%%%%%%%%%%
%%%%%%%% start Abbreviations %%%%%%%%
%%%%%%%%%%%%%%%%%%%%%%%%%%%%%%%%%%%%%
%%%%%%%%%%%%%%%%%%%%%%%%%%%%%%%%%%%%%

\section*{Abbreviations}

\begin{tabularx}{\textwidth}{@{}lQ@{}}
HS & Heritage Speaker\\
HL & Heritage Language\\
L1 & First Language\\
L2 & Second Language\\
LD & Lexical Diversity\\
LI & Lexical Inventory\\
MATTR & Moving Average Type-Token Ratio\\
MajL & Majority Language\\
MS & Monolingual Speaker\\
PV & Particle Verb\\
R\textsuperscript{2}\textsubscript{c} & conditional r-squared\\
RUEG & Research Unit Emerging Grammars\\
\end{tabularx}%



%%%%%%%%%%%%%%%%%%%%%%%%%%%%%%%%
%%%%%%%%%%%%%%%%%%%%%%%%%%%%%%%%
%%%%%%%% start Appendix %%%%%%%%
%%%%%%%%%%%%%%%%%%%%%%%%%%%%%%%%
%%%%%%%%%%%%%%%%%%%%%%%%%%%%%%%%
\newpage
\section*{Appendix}

\begin{figure}[H]
  \centering
  \includegraphics[width=\textwidth]{figures/arranged_MATTR_final_model_plot_DE_speakertype_formality_mode.png}
  \caption{Model 1, predicted values of MATTR in the German data: Formality (A), Mode (B), Speaker Type (C), and the Interaction (D)}
  \label{fig:kelleretal:MATTR_formality_mode_speakertype_DE}
\end{figure}
\pagebreak
\hbox{}
\vfill
\begin{figure}[H]
  \centering
  \includegraphics[width=\textwidth]{figures/arranged_MATTR_final_model_plot_EN_speakertype_formality_mode.png}
  \caption{Model 2, predicted values of MATTR in the English data: Formality (A), Mode (B), Speaker Type (C), and the Interaction (D)}
  \label{fig:kelleretal:MATTR_formality_mode_speakertype_EN}
\end{figure}
\vfill\pagebreak

\begin{sidewaystable}
    \robustify\bfseries
    \small
    \caption{Model on the MATTR measurements in the German data}
    \begin{tabular}{l S[table-format=-1.2]
                      S[table-format=1.2]
                      >{[}S[table-format={[}-1.2]
                      @{, }
                      S[table-format=-1.2{]}]<{]}
                      S[table-format=3.2]
                      S[table-format=<1.2]
                      r
                    }
        \lsptoprule
        Predictors & {$\beta$} & {SE} & \multicolumn{2}{c}{CI} & {Statistic} & {$p$} & {df} \\ \midrule
        (Intercept) & .66 & .00 & .65 & .67 & 202.94 & \bfseries <.01 & 246 \\
        speakertype: USbiGer-DE\footnote{The DE group includes all monolingual and majority German speakers, DEmo and DEbi respectively.} & -.02 & .01 & -.03 & -.01 & -3.53 & \bfseries <.01 & 246 \\ 
        speakertype: DEmo-DEbi & .06 & .01 & .04 & .07 & 6.14 & \bfseries <.01 & 246 \\ 
        formality: F-I & .02 & .00 & .01 & .02 & 4.61 & \bfseries <.01 & 738 \\ 
        mode: S-W & .05 & .00 & .04 & .05 & 13.12 & \bfseries <.01 & 738 \\ 
        session: 2-1 & .00 & .00 & -.00 & .01 & 1.03 & .30 & 738 \\ 
        session: 3-2 & .00 & .00 & -.01 & .01 & -.13 & .90 & 738 \\ 
        session: 4-3 & .00 & .00 & -.01 & .01 & .36 & .72 & 738 \\ 
        speakertype: USbiGer-DE × formality: F-I & .01 & .01 & -.00 & .02 & 1.62 & .11 & 738 \\ 
        speakertype: DEmo-DEbi × formality: F-I& -.02 & .01 & -.04 & .00 & -1.91 & .06 & 738 \\ 
        speakertype: USbiGer-DE × mode: S-W & .01 & .01 & -.01 & .02 & .83 & .41 & 738 \\ 
        speakertype: DEmo-DEbi × mode: S-W& -.01 & .01 & -.03 & .01 & -1.07 & .29 & 738 \\ 
        formality: F-I × mode: S-W& .02 & .01 & .01 & .04 & 3.29 & \bfseries <.01 & 738 \\ 
        speakertype: USbiGer-DE × formality: F-I  × mode: S-W & .00 & .01 & -.02 & .02 & -.06 & .95 & 738 \\ 
        speakertype: DEmo-DEbi × formality: F-I × mode: S-W & -.02 & .02 & -.06 & .02 & -.98 & .33 & 738 \\ 
        \midrule
        \begin{tabular}{@{} l r @{}}
        Random effects\\ 
        σ\textsuperscript{2}                & .00 \\
        τ\textsubscript{00 ID}              & .00 \\
        ICC                                 & .38 \\
        N\textsubscript{ID}                 & 246 \\
        Observations                        & 984 \\
        Marginal R\textsuperscript{2}       & .210  \\
        Conditional R\textsuperscript{2}    & .509 \\
        AIC                                 & -2804.619 \\ 
        \end{tabular} & \\
        \lspbottomrule
    \end{tabular}
    \label{tab:kelleretal:MATTR_final_model_DE}
\end{sidewaystable}

\begin{sidewaystable}
    \small
    \robustify\bfseries
    \caption{Model on the MATTR measurements in the English data}
    \begin{tabular}{l S[table-format=-1.2]
                      S[table-format=1.2]
                      >{[}S[table-format={[}-1.2]
                      @{, }
                      S[table-format=-1.2{]}]<{]}
                      S[table-format=3.2]
                      S[table-format=<1.2]
                      r
                    }
        \lsptoprule
        {Predictors} & {$\beta$} & {SE} & \multicolumn{2}{c}{CI} & {Statistic} & {$p$} & {df} \\ \midrule
        (Intercept) & .68 & .00 & .68 & .69 & 228 & \bfseries <.01 & 287 \\ 
        speakertype: USbiGer-US\footnote{The US group includes all monolingual and majority English speakers, USmo and USbi respectively, minus the heritage German speaker group with English as the majority language.} 
                               & .01 & .01 & .00 & .02 & 2.67 & \bfseries .01 & 287 \\ 
        speakertype: USmo-USbi & .00 & .01 & -.02 & .02 & -.04 & .97 & 287 \\
        formality: F-I & .03 & .00 & .02 & .04 & 9.40 & \bfseries <.01 & 861 \\ 
        mode: S-W & .04 & .00 & .03 & .04 & 11.59 & \bfseries <.01 & 861 \\
        session: 2-1 & .00 & .00 & -.00 & .01 & 1.00 & .32 & 861.45 \\
        session: 3-2 & .00 & .00 & -.01 & .01 & .50 & .62 & 861 \\
        session: 4-3 & .00 & .00 & -.01 & .01 & -.23 & .82 & 861.46 \\
        speakertype: USbiGer-US × formality: F-I& .01 & .01 & -.00 & .02 & 1.67 & .10 & 861 \\
        speakertype: USmo-USbi × formality: F-I& -.01 & .01 & -.03 & .01 & -1.28 & .20 & 861 \\
        speakertype: USbiGer-US × mode: S-W  & .01 & .01 & -.01 & .02 & .99 & .32 & 861 \\ 
        speakertype: USmo-USbi × mode: S-W & .00 & .01 & -.02 & .02 & -.02 & .99 & 861 \\
        formality: F-I × mode: S-W  & .04 & .01 & .02 & .05 & 5.54 & \bfseries <.01 & 861 \\
        speakertype: USbiGer-US × formality: F-I × mode: S-W & .02 & .01 & -.00 & .04 & 1.68 & .09 & 861 \\ 
        speakertype: USmo-USbi × formality: F-I × mode: S-W& -0.03 & .02 & -.06 & .01 & -1.46 & .15 & 861 \\ \midrule
        \begin{tabular}{@{} l r @{}}
        Random Effects \\ 
        σ\textsuperscript{2} & .00\\ 
        τ\textsubscript{00 ID} & .00 \\
        ICC                  & .37 \\
        N\textsubscript{ID} & 287 \\
        Observations        & 1148 \\
        Marginal R\textsuperscript{2}   & .179 \\
        Conditional R\textsuperscript{2} & .479 \\
        AIC                              & -3404.25 \\ 
        \end{tabular}\\
        \lspbottomrule
    \end{tabular}
    \label{tab:kelleretal:MATTR_final_model_EN}
\end{sidewaystable}

\begin{table}
    \robustify\bfseries
    \small
    \caption{Model on the MATTR measurements in the USbiGer data}
    \begin{tabular}{l S[table-format=-1.2]
                      S[table-format=1.2]
                      >{[}S[table-format={[}-1.2]
                      @{, }
                      S[table-format=-1.2{]}]<{]}
                      S[table-format=3.2]
                      S[table-format=<1.2]
                      r
                    }
        \lsptoprule
        {Predictors} & {$\beta$} & {SE} & \multicolumn{2}{c}{CI} & {Statistic} & {$p$} & {df} \\ \midrule
        (Intercept) & .65 & .01 & .64 & .67 & 97.13 & \bfseries <.01 & 36.06 \\ 
        lang: ENG-GER & -.06 & .01 & -.07 & -.05 & -9.96 & \bfseries <.01 & 248.00 \\ 
        formality: F-I & .03 & .01 & .02 & .04 & 4.75 & \bfseries <.01 & 244.26 \\ 
        mode: S-W & .05 & .01 & .03 & .06 & 7.54 & \bfseries <.01 & 244.26 \\ 
        session: 2-1 & .00 & .01 & -.02 & .01 & -.40 & .69 & 244.26 \\ 
        session: 3-2 & .01 & .01 & -.01 & .03 & 1.15 & .25 & 244.26 \\ 
        session: 4-3 & .00 & .01 & -.01 & .02 & .49 & .62 & 244.26 \\ 
        language order: H-M & -.02 & .01 & -.04 & .01 & -1.29 & .20 & 36.06 \\ \midrule
        \multicolumn{8}{l}{\begin{tabular}{@{} l r @{} }
            Random Effects  & \\ 
            σ\textsuperscript{2} &  .00\\
            τ\textsubscript{00 ID} &  .00\\
            ICC &  .32\\
            N\textsubscript{ID} &  36\\
            Observations &  280\\
            Marginal R\textsuperscript{2} &  .315\\
            Conditional R\textsuperscript{2} &  .535\\
            AIC &  -725.44\\ 
        \end{tabular}}\\
    \lspbottomrule
    \end{tabular}
    \label{tab:kelleretal:MATTR_final_model_USHGer}
\end{table}

\begin{figure}[ht]
  \centering
  \includegraphics[width=\textwidth]{figures/arranged_MATTR_final_model_plot_HS_lang_formality_mode.png}
  \caption{Model 3, Predicted Values of MATTR in the USbiGer Data: Formality (A), Mode (B), Speaker Type (C), and the Interaction (D)}
  \label{fig:kelleretal:MATTR_formality_mode_speakertype_USHGer}
\end{figure}

\begin{table}
\centering
\caption{Percentage of all shared lemmas across speaker groups in the German sub-corpus}
\begin{tabularx}{\textwidth}{Qrrrrr}
\lsptoprule
~ & DEmo & USbiGer & DEbiGreek & DEbiRuss & DEbiTurk\\\midrule
DEmo & \cellcolor[HTML]{f2f3f4} 41.02 & 17.24 & 23.55 & 24.88 & 25.00\\
USbiGer & ~ & \cellcolor[HTML]{f2f3f4} 35.29 & 20.32 & 18.25 & 18.85\\
DEbiGreek & ~ & ~ & \cellcolor[HTML]{f2f3f4} 42.38 & 24.46 & 25.35\\
DEbiRuss & ~ & ~ & ~ & \cellcolor[HTML]{f2f3f4} 40.81 & 25.65\\
DEbiTurk & ~ & ~ & ~ & ~ & \cellcolor[HTML]{f2f3f4} 44.41 \\
\lspbottomrule
\end{tabularx}
\label{tab:kelleretal:all_Kable_TriangleMatrix_DE}
\end{table}

\begin{table}
\caption{Percentage of shared adjective lemmas across speaker groups in the German sub-corpus}
\begin{tabularx}{\textwidth}{Qrrrrr}
\lsptoprule
~ & DEmo & USbiGer & DEbiGreek & DEbiRuss & DEbiTurk\\\midrule
DEmo & \cellcolor[HTML]{f2f3f4} 35.98 & 15.82 & 24.42 & 28.37 & 29.40\\
USbiGer & ~ & \cellcolor[HTML]{f2f3f4} 36.73 & 18.60 & 14.93 & 19.16\\
DEbiGreek & ~ & ~ & \cellcolor[HTML]{f2f3f4} 38.94 & 26.27 & 27.20\\
DEbiRuss & ~ & ~ & ~ & \cellcolor[HTML]{f2f3f4} 35.99 & 27.62\\
DEbiTurk & ~ & ~ & ~ & ~ & \cellcolor[HTML]{f2f3f4} 38.29 \\
\lspbottomrule
\end{tabularx}
\label{tab:kelleretal:ADJ_Kable_TriangleMatrix_DE}
\end{table}

\begin{table}
\caption{Percentage of shared noun lemmas across speaker groups in the German sub-corpus}
\centering
\begin{tabularx}{\textwidth}{Qrrrrr}
\lsptoprule
~ & DEmo & USbiGer & DEbiGreek & DEbiRuss & DEbiTurk\\\midrule
DEmo & \cellcolor[HTML]{f2f3f4} 37.61 & 20.24 & 28.45 & 32.67 & 33.07\\
USbiGer & ~ & \cellcolor[HTML]{f2f3f4} 34.84 & 26.71 & 21.65 & 22.37\\
DEbiGreek & ~ & ~ & \cellcolor[HTML]{f2f3f4} 43.71 & 30.32 & 32.71\\
DEbiRuss & ~ & ~ & ~ & \cellcolor[HTML]{f2f3f4} 38.04 & 33.64\\
DEbiTurk & ~ & ~ & ~ & ~ & \cellcolor[HTML]{f2f3f4} 43.99 \\
\lspbottomrule
\end{tabularx}
\label{tab:kelleretal:NOUN_Kable_TriangleMatrix_DE}
\end{table}

\begin{table}
\caption{percentage of shared verb lemmas across speaker groups in the German sub-corpus}
\centering
\begin{tabularx}{\textwidth}{Qrrrrr}
\lsptoprule
~ & DEmo & USbiGer & DEbiGreek & DEbiRuss & DEbiTurk\\\midrule
DEmo & \cellcolor[HTML]{f2f3f4} 48.88 & 24.72 & 38.46 & 37.65 & 36.54\\
USbiGer & ~ & \cellcolor[HTML]{f2f3f4} 35.15 & 27.95 & 28.16 & 26.22\\
DEbiGreek & ~ & ~ & \cellcolor[HTML]{f2f3f4} 43.14 & 39.55 & 39.75\\
DEbiRuss & ~ & ~ & ~ & \cellcolor[HTML]{f2f3f4} 48.03 & 40.92\\
DEbiTurk & ~ & ~ & ~ & ~ & \cellcolor[HTML]{f2f3f4} 48.62 \\
\lspbottomrule
\end{tabularx}
\label{tab:kelleretal:VERB_Kable_TriangleMatrix_DE}
\end{table}

\begin{table}
\centering
\caption{Percentage of all shared lemmas across speaker groups in the English sub-corpus}
\begin{tabularx}{\textwidth}{Qrrrrr}
\lsptoprule
~ & USmo & USbiGer & USbiGreek & USbiRuss & USbiTurk\\\midrule
USmo & \cellcolor[HTML]{f2f3f4} 51.95 & 28.62 & 28.75 & 29.09 & 29.75\\
USbiGer & ~ & \cellcolor[HTML]{f2f3f4} 52.15 & 28.94 & 27.32 & 28.62\\
USbiGreek & ~ & ~ & \cellcolor[HTML]{f2f3f4} 50.36 & 29.61 & 30.76\\
USbiRuss & ~ & ~ & ~ & \cellcolor[HTML]{f2f3f4} 51.18 & 29.75\\
USbiTurk & ~ & ~ & ~ & ~ & \cellcolor[HTML]{f2f3f4} 53.66 \\
\lspbottomrule
\end{tabularx}
\label{tab:kelleretal:all_Kable_TriangleMatrix_EN}
\end{table}

\begin{table}
\caption{Percentage of shared adjective lemmas across speaker groups in the English sub-corpus}
\begin{tabularx}{\textwidth}{Qrrrrr}
\lsptoprule
~ & USmo & USbiGer & USbiGreek & USbiRuss & USbiTurk\\\midrule
USmo & \cellcolor[HTML]{f2f3f4} 42.58 & 30.00 & 31.38 & 31.84 & 28.94\\
USbiGer & ~ & \cellcolor[HTML]{f2f3f4} 44.57 & 32.11 & 27.31 & 30.43\\
USbiGreek & ~ & ~ & \cellcolor[HTML]{f2f3f4} 35.85 & 31.85 & 36.44\\
USbiRuss & ~ & ~ & ~ & \cellcolor[HTML]{f2f3f4} 40.10 & 29.70\\
USbiTurk & ~ & ~ & ~ & ~ & \cellcolor[HTML]{f2f3f4} 43.24 \\
\lspbottomrule
\end{tabularx}
\label{tab:kelleretal:ADJ_Kable_TriangleMatrix_EN}
\end{table}

\begin{table}
\caption{Percentage of shared noun lemmas across speaker groups in the English sub-corpus}
\centering
\begin{tabularx}{\textwidth}{Qrrrrr}
\lsptoprule
~ & USmo & USbiGer & USbiGreek & USbiRuss & USbiTurk\\\midrule
USmo & \cellcolor[HTML]{f2f3f4} 52.11 & 39.23 & 39.15 & 39.96 & 41.10\\
USbiGer & ~ & \cellcolor[HTML]{f2f3f4} 48.64 & 38.38 & 36.65 & 39.39\\
USbiGreek & ~ & ~ & \cellcolor[HTML]{f2f3f4} 52.55 & 42.82 & 42.89\\
USbiRuss & ~ & ~ & ~ & \cellcolor[HTML]{f2f3f4} 51.90 & 44.65\\
USbiTurk & ~ & ~ & ~ & ~ & \cellcolor[HTML]{f2f3f4} 52.69 \\
\lspbottomrule
\end{tabularx}
\label{tab:kelleretal:NOUN_Kable_TriangleMatrix_EN}
\end{table}
\clearpage
\begin{table}[t]
\caption{Percentage of shared verb lemmas across speaker groups in the English sub-corpus}
\centering
\begin{tabularx}{\textwidth}{Qrrrrr}
\lsptoprule
~ & USmo & USbiGer & USbiGreek & USbiRuss & USbiTurk\\\midrule
USmo & \cellcolor[HTML]{f2f3f4} 56.99 & 46.93 & 47.44 & 48.48 & 52.50\\
USbiGer & ~ & \cellcolor[HTML]{f2f3f4} 59.50 & 48.57 & 45.40 & 46.42\\
USbiGreek & ~ & ~ & \cellcolor[HTML]{f2f3f4} 56.72 & 48.32 & 51.12\\
USbiRuss & ~ & ~ & ~ & \cellcolor[HTML]{f2f3f4} 57.52 & 48.45\\
USbiTurk & ~ & ~ & ~ & ~ & \cellcolor[HTML]{f2f3f4} 60.37 \\
\lspbottomrule
\end{tabularx}
\label{tab:kelleretal:VERB_Kable_TriangleMatrix_EN}
\end{table}

\printbibliography[heading=subbibliography,notkeyword=this]
\end{document}

