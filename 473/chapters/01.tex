\documentclass[output=paper,colorlinks,citecolor=brown]{langscibook}
\ChapterDOI{10.5281/zenodo.15775157}
\title{Introduction: Investigating the dynamics of language contact situations}
\author{Heike Wiese\orcid{0000-0002-6310-3045}\affiliation{Humboldt-Universität zu Berlin} and Shanley E. M. Allen\orcid{0000-0002-5421-6750}\affiliation{University of Kaiserslautern-Landau} and Mareike Keller\orcid{0000-0001-8054-1701}\affiliation{University of Mannheim} and Artemis Alexiadou\orcid{0000-0002-6790-232X}\affiliation{Leibniz-Centre General Linguistics; Humboldt-Universität zu Berlin}}

\abstract{In this chapter, we introduce the volume \textit{Linguistic dynamics in heritage speakers}~– the culmination of six years of coordinated work in the Research Unit \textit{Emerging Grammars in Language Contact Situations: A Comparative Approach}, funded by the German Research Foundation. The work reported in this volume explores productions of heritage speakers of German, Greek, Russian, and Turkish who have either English or German as a majority language as well as productions of monolingually-raised speakers of each of those languages. The research is based on a corpus of narrative data in communicative situations differing in formality (formal, informal) and mode (spoken, written). This chapter motivates the overall research program, presents relevant terminology, highlights the main findings across the studies reported in the volume, briefly introduces each chapter, and summarizes the contributions of our research to the field.

\keywords{heritage speakers, language contact, linguistic dynamics, comparative corpus study, register variation}
}

\IfFileExists{../localcommands.tex}{
   \addbibresource{../localbibliography.bib}
   \usepackage{langsci-optional}
\usepackage{langsci-gb4e}
\usepackage{langsci-lgr}

\usepackage{listings}
\lstset{basicstyle=\ttfamily,tabsize=2,breaklines=true}

%added by author
% \usepackage{tipa}
\usepackage{multirow}
\graphicspath{{figures/}}
\usepackage{langsci-branding}

   
\newcommand{\sent}{\enumsentence}
\newcommand{\sents}{\eenumsentence}
\let\citeasnoun\citet

\renewcommand{\lsCoverTitleFont}[1]{\sffamily\addfontfeatures{Scale=MatchUppercase}\fontsize{44pt}{16mm}\selectfont #1}
  
   %% hyphenation points for line breaks
%% Normally, automatic hyphenation in LaTeX is very good
%% If a word is mis-hyphenated, add it to this file
%%
%% add information to TeX file before \begin{document} with:
%% %% hyphenation points for line breaks
%% Normally, automatic hyphenation in LaTeX is very good
%% If a word is mis-hyphenated, add it to this file
%%
%% add information to TeX file before \begin{document} with:
%% %% hyphenation points for line breaks
%% Normally, automatic hyphenation in LaTeX is very good
%% If a word is mis-hyphenated, add it to this file
%%
%% add information to TeX file before \begin{document} with:
%% \include{localhyphenation}
\hyphenation{
affri-ca-te
affri-ca-tes
an-no-tated
com-ple-ments
com-po-si-tio-na-li-ty
non-com-po-si-tio-na-li-ty
Gon-zá-lez
out-side
Ri-chárd
se-man-tics
STREU-SLE
Tie-de-mann
}
\hyphenation{
affri-ca-te
affri-ca-tes
an-no-tated
com-ple-ments
com-po-si-tio-na-li-ty
non-com-po-si-tio-na-li-ty
Gon-zá-lez
out-side
Ri-chárd
se-man-tics
STREU-SLE
Tie-de-mann
}
\hyphenation{
affri-ca-te
affri-ca-tes
an-no-tated
com-ple-ments
com-po-si-tio-na-li-ty
non-com-po-si-tio-na-li-ty
Gon-zá-lez
out-side
Ri-chárd
se-man-tics
STREU-SLE
Tie-de-mann
}
   \boolfalse{bookcompile}
   \togglepaper[1]%%chapternumber
}{}

\begin{document}
\maketitle

%%%%%%%%%%%%%%%%%%%%%%%%%%%%%%%%%
%%%%%%%%%%%%%%%%%%%%%%%%%%%%%%%%%
%%%%%%%% start section 1 %%%%%%%%
%%%%%%%%%%%%%%%%%%%%%%%%%%%%%%%%%
%%%%%%%%%%%%%%%%%%%%%%%%%%%%%%%%%

\section{Introduction} \label{sec:introwieseetal:intro}
\subsection{Heritage speakers in focus} \label{sec:introwieseetal:focus}

Variation is a core characteristic of language use, and contact between different linguistic systems is always part of this. Accordingly, multilingualism is the normal condition for human language. Worldwide, most speakers today are multilingual (e.g., \citealt{Grosjean1982,Grosjean2010, Romaine1989, Myers-Scotton2006}), and many countries recognise a range of national languages (e.g., Switzerland has four, Namibia has 13, and India recognises 22 “regional languages”).

In contrast to this multilingual reality, it is often monolingualism that is treated as the norm. Such a monolingual bias is particularly common in societies in Europe and their former colonies such as the US or Australia (e.g., \citealt{Gogolin2002, Canagarajah2013, CookWei2016, OrtegaEtAl2016}). This is a heritage of European nation-state building in the 19\textsuperscript{th} century. At that time, an ideological link of “one country, one nation, one language” provided the basis for an imagined homogenous speech community as the basis of a nation.

Such a monolingual bias has also influenced linguistic research, in particular in the Global North. For instance, in second language acquisition, learners have traditionally been judged against monolingual language use, and structural approaches have, for a long time, focused on monolingual speakers as the primary source for native grammars (cf. criticism in \citealt{Brutt-GrifflerSamimy2001, Bonfiglio2010, Cook2016, OrtegaEtAl2016}). However, to fathom the reality of human languages in use, we should not restrict ourselves to the special case of monolinguals, but instead include multilingual speakers, who represent the normal case of linguistic competence \citep{WieseEtAl2022}.

One example of speakers who routinely use two or more languages in their daily lives is represented by the case of heritage speakers, that is, speakers who grew up in a bi- or multilingual home with a minority language that is part of their family’s linguistic heritage, typically as a result of immigration in an earlier generation. As minority languages, such heritage languages are spoken in the family, but they are not the language that is dominant in the larger society at the national or regional level. The dominant languages are typically part of speakers’ repertoires as well. In monolingually\hyp biased societies, heritage speakers hence bring back some multilingual normalcy: they do not restrict themselves to one societally\hyp recognised language, but make use of resources from different languages, and they do so from an early age.

The heritage language is typically first acquired through exposure to oral language and implicit learning and starts as a typically developing and dominant first language (L1), but often loses its dominant status after early childhood. The societal majority language is acquired as an additional L1 or an early L2; it typically starts as a less\hyp used language, but often becomes dominant when formal education begins (kindergarten, school), which is also the time when formal registers and written language are systematically established \citep{Rothman2007}. This interplay of multilingualism and language development challenges traditional L1/L2 distinctions and conceptualisations of native speakers based on constructions of monolingual speech communities (\citealt{RothmanTreffers-Daller2014, WieseEtAl2022}).

Heritage speakers and their communities thus offer a fascinating locus for research. The fact that both heritage and majority languages are used in daily life makes language contact ubiquitous. The effects of this constant contact are evident not only at levels of individual linguistic repertoires, but also at the larger level of the speech community, bringing about and spreading new patterns of language structure and use in a setting that is particularly open to linguistic variation and change.

In the public discussion of such countries as Germany or the US, the prevailing monolingual bias leads to a view of heritage speakers as a deviant case requiring special language support and intervention if speakers do not adhere to a perceived monolingual norm. This bias has also been visible in linguistic approaches where heritage speakers’ competence has often been judged against monolingual standards, rather than assessed in its own right (cf. criticism in \citealt{RothmanEtAl2022}). For instance, “attainment” in the heritage language is typically measured against a monolingual standard language, with deviations classified as “errors”, and linguistic areas characterised as “vulnerable” if they are likely to support such deviations, and “resilient” if they are not. Changes in a language are often often measured with different yardsticks, depending on whether the language is being used as a heritage language or as a majority language. If we observe changes in heritage speakers, but not in majority language speakers, this is typically characterised as a “vulnerability” of the heritage language. On the other hand, if we observe changes in majority language speakers, but not in heritage speakers, heritage speakers tend to be characterised as “conservative” (e.g., \citealt{Montrul2015}). In recent years, however, research has moved beyond supposed incomplete acquisition, attrition, and loss compared to monolingual norms, towards studies homing in on internal grammatical systematicity (e.g., \citealt{Grosjean2008, PascualRothman2012, RothmanTreffers-Daller2014, Guijarro-FuentesSchmitz2015, ScontrasEtAl2015, Schroeder2016, TsehayeEtAl2021, ZubanEtAl2021, WieseEtAl2022, AlexiadouRizou2023, AlexiadouEtAl2023, ÖzsoyBlum2023, Bunketal_ip_a, Bunketal_ip_b, Keskinetal_ir}; see \citealt{KupischRothman2018} for a detailed critique of accounts of what has been labelled “incomplete acquisition”). The following section illustrates the contribution of the Research Unit \textit{Emerging Grammars in Language Contact Situations} (RUEG; see \url{https://hu.berlin/rueg}) to reframing heritage speakers and their languages as systematic developments within the spectrum of native grammars.

\subsection{The Research Unit \textit{Emerging Grammars in Language Contact Situations} (RUEG)} \label{sec:introwieseetal:RUEG}

RUEG is a cluster of projects that have collaboratively investigated the dynamics of language contact in heritage speakers. Our research has been driven by a multilingual perspective: we approached heritage speakers’ languages in terms of dynamics, rather than vulnerability; of development, rather than incomplete acquisition; and of innovation, rather than attrition and loss. As is also emphasised in other recent approaches taking the same perspective, we regard heritage grammars as reflecting a type of variation within native grammars (\citealt{ScontrasEtAl2018, EmbickEtAl2020, FloresRinke2020}, in reply to \citealt{PolinskyScontras2020}) as well as principles of language variation and change (\citealt{Johannessen2018, AalberseEtAl2019, LohndalEtAl2019, Muysken2020}).

A central target has been to investigate noncanonical phenomena in heritage speakers’ repertoires as patterns in their own right in order to understand their status in the linguistic system and in language use. In a unified approach, projects examined the hypothesis that what is often regarded as speakers’ limited proficiency in both the heritage and the majority languages can be re-interpreted as the emergence of new grammatical options and structures and, possibly, new linguistic varieties. While contact-induced variation may at first reflect an individual phenomenon, variants may also stabilise, spread and become part of shared linguistic repertoires and practices that support the emergence of new varieties.

Our collaborative research was guided by two overarching questions over two research periods (RUEG1: 2018--2021; RUEG2: 2021--2024):

\begin{enumerate}
    \item What is the status of noncanonical phenomena in heritage speakers’ two languages from the perspective of emerging grammars?
    \item What are the linguistic dynamics in heritage speakers’ repertoires?
\end{enumerate}

In pursuit of the first research question, we investigated linguistic systematicity at different grammatical and pragmatic levels. Casting our net wide, we targeted noncanonical phenomena in general, that is, all those that differ from codified standard language norms. This allowed us to broadly take into account such phenomena in heritage speakers as well as in monolinguals, and to then assess which ones are characteristic for language contact situations and heritage speakers’ repertoires. To capture a wider portion of speakers’ repertoires, we included language use in different communicative situations: formal and informal, spoken and written. Building on this, the second research question targeted dynamics in repertoires. We looked at crosslinguistic interactions between heritage speakers’ two languages as well as developments that derive from existing language-internal tendencies, and at the dynamics of register in language contact.

The group encompassed altogether 16 research projects and a central project that was responsible for the overall coordination:

\begin{labeling}{P10}
\item[P1] \textit{Nominal morphosyntax and word order in Heritage Greek across majority languages} \\
    PI: Artemis Alexiadou \\
    Research Associate: Vasiliki Rizou
\item[P2] \textit{Morphosyntax and word order in majority English across heritage speakers} \\
    PI: Shanley E. M. Allen \\
    Research Associate: Tatiana Pashkova 
\item[P3] \textit{Nominal morphosyntax and word order in heritage Russian across majority languages} \\
    PIs: Natalia Gagarina, Luka Szucsich \\
    Research Associate: Maria Martynova
\item[P4] \textit{Clause combining and word order in heritage Turkish across majority languages} \\
    PI: Christoph Schroeder \\
    Research Associate: Kateryna Iefremenko
\item[P4a] \textit{Head directionality change in Turkic in contact situations: A diachronic comparison between heritage Turkish and Balkan Turkic} \\
    PI: Christoph Schroeder \\
    Research Associate: Cem Keskin
\item[P5] \textit{Clause structure in heritage German} \\
    PI: Rosemarie Tracy \\
    Research Associate: Wintai Tsehaye
\item[P6] \textit{Noncanonical constituent linearisation in German across heritage speakers} \\
    PI: Heike Wiese \\
    Research Associate: Oliver Bunk
\item[P7] \textit{Intonation and word order in majority English and heritage Russian across speaker populations} \\
    PI: Sabine Zerbian \\
    Research Associate: Yulia Zuban
\item[P8] \textit{Dynamics of information structure in language contact} \\
    PIs: Shanley E. M. Allen, Oliver Bunk, Sabine Zerbian \\
    Research Associates: Kristina Barabashova, Tatiana Pashkova, Yulia Zuban
\item[P9] \textit{Dynamics of discourse organisation in language contact} \\
    PIs: Shanley E. M. Allen, Christoph Schroeder, Heike Wiese \\
    Research Associates: Kateryna Iefremenko, Kalliopi Katsika, Annika Labrenz
\item[P9a] \textit{Clause combining in Balkan Turkic: Pathways and stages of contact-induced grammaticalization} \\
    PI: Christoph Schroeder \\
    Research Associate: Cem Keskin
\item[P10] \textit{Dynamics of verbal aspect and pronominal reference in language contact} \\
    PIs: Artemis Alexiadou, Natalia Gagarina, Luka Szucsich \\
    Research Associates: Maria Martynova, Onur Özsoy, Vasiliki Rizou
\item[P11] \textit{The heritage speaker lexicon: Dynamics and interfaces} \\
    PIs: Mareike Keller, Anke Lüdeling, Rosemarie Tracy \\
    Research Associate: Nadine Zürn
\item[Pc] \textit{Corpus linguistic methods} \\
    PIs: Anke Lüdeling, Anna Shadrova \\
    Research Associate: Martin Klotz
\item[Pd] \textit{``Emerging grammars'': A cross-linguistic corpus of comparative data in heritage and majority language use} \\
    PIs: Anke Lüdeling, Heike Wiese \\
    Research Associates: Martin Klotz, Annika Labrenz, Maria Pohle
\item[Pt] \textit{Family language dynamics: Empowering speakers of majority and heritage languages} \\
    PIs: Judith Purkarthofer, Rosemarie Tracy \\
    Research Associates: Sofia Grigoriadou, Johanna Tausch
\item[Pz] \textit{Coordination} \\
    PI: Heike Wiese \\
    Associates: Irem Duman Çakir, Esther Jahns, Pia Linscheid, Katrin Neuhaus
\end{labeling}

In addition to the researchers in individual projects, four Mercator Fellows cooperated with all projects: Maria Polinsky and Shana Poplack (Mercators in RUEG1); Cristina Flores and Jeanine Treffers-Daller (Mercators in RUEG2).

The remainder of the chapter is structured as follows. In \sectref{sec:introwieseetal:terminology}, we provide some definitions for central concepts used in our group. Based on this, we describe our shared methodology and collaborative data collection (\sectref{sec:introwieseetal:comparative}) and summarise central findings of our research (\sectref{sec:introwieseetal:results}). We then offer an overview of the contributions to this volume (\sectref{sec:introwieseetal:overview_chapters}), and conclude with a reflection on RUEG’s contributions and directions for future research (Section~\ref{sec:introwieseetal:conclusion}).

\subsection{Central concepts and terminology} \label{sec:introwieseetal:terminology}

We call \textsc{noncanonical} all phenomena that differ from codified standard language norms. Accordingly, phenomena falling within codified standard norms are \textsc{canonical}.

We understand a \textsc{heritage} \textsc{language} as a language spoken in a family, typically as a result of migration, in cases where that language is not dominant in the larger society. The counterpart to a heritage language is a \textsc{majority} \textsc{language}: the dominant language of the larger society. Note that the larger society need not necessarily be the society at a national level, but can also be the speech community in a specific region. For instance, in a Kurdish-dominant region of Turkey, the majority language might be Kurdish.

A \textsc{heritage} \textsc{speaker} is a speaker who combines heritage and majority languages in their repertoire – a speaker who grew up in a bi- or multilingual home with a minority/heritage language that is not the majority language dominant in the larger society.

The term \textsc{monolingual} \textsc{speaker} should be understood as a short form of “monolingually\hyp raised speaker”: this is a speaker who grew up without an additional minority/heritage language in the family.

Note that both heritage speakers and monolingual speakers will have the majority language as part of their repertoire. Accordingly, a \textsc{majority-language} \textsc{speaker} is to be understood as a speaker who speaks a majority language, and this includes both monolingual and bi-/multilingual speakers.

Hence, if we look at two languages, say, English and German, and two countries, say, the US and Germany, and compare majority and heritage language use for monolingual and bilingual speakers, then this is not a one-to-one correlation. For instance, in the US, we could compare majority-language use in bilingual vs. monolingual speakers: this would be English spoken as a majority language (a) by monolinguals speakers who grew up with only English regularly spoken in the family, and (b) by bilinguals who speak English as a majority language, and who are, at the same time, heritage speakers of another language, in this case German. On the other hand, we could compare German in the US and Germany. In the US, German is spoken by heritage speakers, that is, bilingual speakers who use German as a heritage language in addition to English (the majority language), whereas in Germany, German is spoken as a majority language, that is, by majority-language speakers, who might be monolingual (only German in the family) or bilingual (heritage speakers of other languages, for instance, Turkish).

If we look at speakers’ repertoires, we find a differentiation of language use across different \textsc{communicative} \textsc{situations}. This term refers to the setting of a communicative event. In this volume, it will be used as a superordinate term for the four conditions in which we elicited our data: informal-spoken, informal-written, formal-spoken, and formal-written (see \sectref{sec:introwieseetal:RUEG}; see \citealt{Wiese2020,Wiese2023}, for a detailed discussion of communicative situations). Communicative situations can be distinguished through two more specific terms for the contrasts we employed: \textsc{mode} refers to the contrast of spoken vs. written communicative situations, \textsc{formality} to that of formal vs. informal ones.

Against this background, we understand \textsc{register} as the linguistic counterpart of communicative situation, that is, the language use associated with different communicative situations. Specifically, register distinctions are socially motivated and recurring intra-individual linguistic variations in different communicative situations (cf. \citealt{LüdelingEtAl2022}).

For instance, if speakers use different linguistic patterns in informal vs. formal communicative situations, then they have different registers for this kind of contrast. If they do not differentiate between their linguistic options in these kinds of situations, they do not make a register distinction for informal vs. formal communicative situations for this kind of contrast. Such distinctions can also be specific to linguistic domains: if speakers, say, use different terms of address in informal and formal communicative situations, but they use noncanonical gender marking in both situations, then they make a register distinction with respect to address terms for informal vs. formal communicative situations, but not with respect to nominal gender.

The next section describes the kind of comparisons we made and the contrasts we targeted in more detail.

%%%%%%%%%%%%%%%%%%%%%%%%%%%%%%%%%
%%%%%%%%%%%%%%%%%%%%%%%%%%%%%%%%%
%%%%%%%% start section 2 %%%%%%%%
%%%%%%%%%%%%%%%%%%%%%%%%%%%%%%%%%
%%%%%%%%%%%%%%%%%%%%%%%%%%%%%%%%%

\section{RUEG’s comparative approach} \label{sec:introwieseetal:comparative}

Our collective research supported a large-scale comparative investigation based on a unified methodology and a shared empirical basis. We included five languages: English, German, Greek, Russian, and Turkish. We investigated bilinguals in the US and Germany (heritage speakers of Greek, Russian, and Turkish in both countries, plus heritage speakers of German in the US), and monolinguals of all five languages in Germany, Greece, Russia, Turkey, and the US

From a perspective of emerging grammars, analyses took into account the possibility that heritage speakers develop new grammatical options in both their languages in different communicative situations. Crucially, this meant that it was not sufficient to focus on standard language alone, but rather that we needed to investigate the larger repertoires of speakers, and their different choices across communicative situations. Accordingly, our approach included

\begin{enumerate}[label=(\alph*)]
\item heritage and majority language use;
\item language use in formal as well as informal situations;
\item written and spoken language;
\item language use in two different age groups: adolescents and adults.
\end{enumerate}

Taking into account both languages of heritage speakers enabled us to identify domain-specific skills in different languages that might complement each other (cf. \citealt{Grosjean1997} on the “complementarity principle” for bilinguals, and \citealt{BlommaertEtAl2005} for a discussion of “truncated multilingualism” for linguistic resources in different spatial contexts). It also allowed us to investigate mutual influence between the majority language and the heritage language, including the mutual reinforcement of linguistic options from different registers for majority and heritage languages. 

The inclusion of informal as well as formal communicative situations, and of spoken as well as written communicative situations, further supported a genuinely bilingual perspective and a holistic view of competence. For instance, a frequent finding reported from heritage language research has been the loss of abilities in a heritage language after majority language immersion at school. However, such findings might be coloured by a sole focus on formal registers (e.g., \citealt{Merino1983}; cf. also \citealt{KupischRothman2018} for a critique). By including language use in informal (spoken and written) communicative situations, we were able to capture possible register specialisations and interactions. This allowed us to assess speakers’ proficiency by register, rather than solely by comparison to formal standard language. Furthermore, it enabled us to identify the transfer of patterns across communicative situations as well as restrictions to certain registers.

For an appropriate assessment of noncanonical patterns, one needs to take into account the breadth of language use not only for heritage speakers, but also for monolinguals. Accordingly, we compared the repertoires of heritage speakers with language use of monolingual speakers.

If heritage language use is compared only to the respective standard language or to monolingual language from formal settings, noncanonical patterns might be attributed to the heritage context even if they represent linguistic variation that generally occurs in informal-spoken situations (see \cite{chapters/02}). Factoring this in, we incorporated the same breadth of language use in monolinguals as in heritage speakers. Our comparisons thus targeted

\begin{enumerate}[label=(\alph*)]
\sloppy
\item monolingual majority language speakers in countries where heritage speakers live, and
\item monolingual majority language speakers in the heritage countries (countries of origin for speakers’ ancestors),
\end{enumerate}

across the same written and spoken, formal and informal communicative situations.

Such register-sensitive, matched comparisons with monolingual language use allowed us to distinguish genuine language\hyp contact phenomena from general patterns of variation. At the same time, the observed linguistic structures need not be an outcome of contact-induced change in the sense of (lexical or structural) transfer, but could also be due to language-internal change triggered or at least accelerated in contexts of language contact (see \citealt{Silva-Corvalan1994, Wiese2013, KupischPolinsky2022}).\textsuperscript{} This possibility made the inclusion of different language pairs essential in order to identify specific language\hyp contact effects. Consequently, our collaborative research allowed us to integrate results on

\begin{enumerate}[label=(\alph*)]
\item speakers of the same heritage language in countries with different majority languages,
\item speakers of different heritage languages for the same majority language,
\item speakers of the same language as heritage vs. majority language.
\end{enumerate}

To support systematic, large-scale comparisons, we collected data in five countries in a closely synchronised endeavour, using the same methodology for eliciting naturalistic productions (see \cite{chapters/02} for methodological details). All data was processed and integrated into an annotated corpus that served as a joint empirical basis for all projects and has, from its first version, been completely open access (\url{https://hu.berlin/rueg-corpus}; see \cite{chapters/03} and \cite{Klotzetal2024} for corpus linguistic details). The cross-linguistic RUEG corpus is, to our knowledge, the first of its kind: it provides parallel data across countries, languages, contact-linguistic settings, and age groups, covering different communicative situations for heritage speakers and monolingual counterparts alike.

%%%%%%%%%%%%%%%%%%%%%%%%%%%%%%%%%
%%%%%%%%%%%%%%%%%%%%%%%%%%%%%%%%%
%%%%%%%% start section 3 %%%%%%%%
%%%%%%%%%%%%%%%%%%%%%%%%%%%%%%%%%
%%%%%%%%%%%%%%%%%%%%%%%%%%%%%%%%%

\section{Major results and contributions from RUEG1 and RUEG2} \label{sec:introwieseetal:results}

\subsection{Corpus} \label{sec:introwieseetal:corpus}

A first major outcome of RUEG is the just-mentioned corpus – an annotated, multilayer and multimodal open-access corpus of comparable cross-linguistic production data from adolescent and adult heritage, majority and monolingually-raised speakers in informal and formal settings and written and spoken modes (\citealt{RUEGcorpus2019}; see \cite{chapters/03} for more details).
The data in the final version of this corpus come from 736 speakers: 381 adolescents and 355 adults; 411 bilingual and 325 monolingual. These speakers together contributed 528,709 normalised tokens to the corpus (typically graphematic words – 150,999 for English, 165,396 for German, 69,450 for Greek, 76,930 for Russian, and 65,934 for Turkish). An additional subcorpus of Kurdish speakers in Turkey ($n=29$, 15169 tokens) and Germany ($n=9$, 17710 tokens) provides Kurdish and Turkish language productions.
In addition to the basic transcription, the entire corpus is annotated for language, lemma, and part of speech. Syntactic spans according to Universal Dependencies were added automatically. To pursue our specific research questions, manual annotations for topological Fields for German and Surface Syntactic Dependencies for German and Russian were added. Further, subsets of the data were annotated for specific phenomena such as intonation, nominal morphology, verb morphology, aspect marking, functions of discourse markers, referent introduction, and more.{\interfootnotelinepenalty=10000\footnote{The corpus is freely available at \url{https://zenodo.org/records/3236068} and is searchable with ANNIS (\citealt{KrauseZeldes2016,KrauseEtAl2023}). At the time of publication of this volume, an ANNIS instance at Humboldt University of Berlin containing the RUEG corpora is hosted at \url{https://korpling.german.hu-berlin.de/annis/}. Full documentation for the corpus can be consulted at \url{https://korpling.german.hu-berlin.de/rueg-docs/latest/}.}}

The corpus supports numerous types of systematic comparisons – across languages, countries, age groups, contact-linguistic settings, and communicative situations (see \cite{chapters/02}), as well as detailed analyses within a heritage language and of individual speaker profiles (e.g. \cite{chapters/05, chapters/10}). Many of our studies compare the use of a heritage language in a majority setting and two different minority settings (Germany and the US~-- e.g., null subjects in \cite{chapters/06}, aspect in \cite{chapters/07}, clause combining in \cite{chapters/08}, word order in \cite{chapters/11}, discourse markers in \cite{chapters/14}), while others compare the use of a majority language by three or four groups of bilingual speakers (clause combining in \cite{chapters/08} for English and German, several structures in \cite{chapters/11} for German and \cite{chapters/11, chapters/13} for English). Taken together, these comparisons allow for a more nuanced understanding of the factors underlying dynamic uses of language than is typically the case in other studies. Another strength of the corpus is that all of the studies on different areas of language (prosody, morphology, syntax, lexicon, pragmatics) come from the same speakers and the same data set, thus providing a more comprehensive picture of heritage language use than in previous research.

Many of the difficulties and rewards of the lived experience of our collaborative data collection, annotation and publication in an open-access corpus are detailed in \textcite{chapters/03}. From this experience, and in addition to the corpus itself, the RUEG project has several outcomes that are relevant for the field of corpus linguistics – especially for (distributed) interdisciplinary projects involving a variety of researchers and linguistic phenomena that are annotated manually in a multi-layer corpus. In particular, RUEG provides a model for how annotation, modeling, analysis, and the epistemological embedding of categories and analytical processes can proceed hand in hand. Our experience brings to light the importance of understanding that data acquisition, transcription and annotation are not trivial aspects of corpus work. Rather, all of the decisions involved in these steps are based on~-- sometimes implicit~-- assumptions and models which should be laid out explicitly to allow for scholarly sound use of the corpus data.

\subsection{Factors driving noncanonical patterns} \label{sec:introwieseetal:noncanonical}

A second major outcome of RUEG is the identification of numerous noncanonical patterns across the data, and insight into the various factors driving these patterns. We organised our corpus-linguistic studies within several joint ventures that involved all projects. These joint ventures targeted systematic developments, their association with internal vs. external interfaces, their association with communicative situations (spoken vs. written, formal vs. informal), and the distinction of contact-induced change vs. language-internal developments and variation. Our corpus\hyp linguistic and experimental studies revealed systematic noncanonical patterns in the domains of word order (\cite{chapters/11, chapters/13}), sentence structure (\cite{chapters/05, chapters/08, chapters/09, chapters/11, chapters/13}), morphosyntax (\cite{chapters/06, chapters/07}), intonation (\cite{chapters/12}), lexicon (\cite{chapters/10}), discourse organisation (\cite{chapters/14, chapters/15}), and information structure (\cite{chapters/11, chapters/12, chapters/13}).

\begin{sloppypar}
One key question was whether the noncanonical patterns we identified arose specifically due to language contact or if they emerged as part of the process of language\hyp internal development and variation. We found clear indications for both in different situations. Numerous instances of cross\hyp linguistic influence were evident~-- predominantly from the majority language to the heritage language (\cite{chapters/05, chapters/07, chapters/10, chapters/14}) but also from the heritage language to the majority language (\cite{chapters/13}). Other findings indicate change that bears no obvious relation to influence from another language but rather represents language\hyp internal development since we saw similar noncanonical patterns in both bilingual and monolingual speakers of the language in question, or in bilingual speakers in contact with typologically different languages (\cite{chapters/06, chapters/09, chapters/14}).
\end{sloppypar}

A second key question was whether noncanonical patterns would emerge mostly in structures at external interfaces incorporating knowledge from core grammar (e.g., phonology, morphology, syntax) and extragrammatical domains (e.g., discourse, pragmatics) (see Interface Hypothesis; \citealt{Sorace2011}), or whether they would also emerge in structures that only required knowledge from core areas of grammar. We found a range of noncanonical patterns at external interfaces, for example at the interfaces of information structure with intonation (\cite{chapters/12}), with word order in syntax (\cite{chapters/05, chapters/11, chapters/13}), or with referent form in morphology (\cite{chapters/06, chapters/13}). However, we also found noncanonical patterns in areas where arguably only core areas of the grammar were in play, such as with grammatical aspect or particle verbs (interface between morphology and semantics; \cite{chapters/07, chapters/10}).

A third key question addressed the role of communicative situations for the development of noncanonical patterns. Here we found more evidence of dynamic patterns in informal than formal situations, and in spoken than written situations. We also found evidence of register levelling in heritage speakers, such that patterns restricted to informal registers in monolingual speakers were also found in formal registers in heritage speakers, likely due to heritage speakers’ lack of experience with formal registers (\cite{chapters/05, chapters/08, chapters/10, chapters/12, chapters/15}). These findings underline the importance of considering register in future studies of heritage speakers and other bilinguals, since most research elicits data in contexts that are more formal than informal. Our findings suggest that this methodological practice may be hampering our insights into one of the most fertile grounds of language change, namely informal language.

Finally, we evaluated the role of the social context in which a heritage language is used. In particular, we saw an effect of the size and cohesion of the speech community in several studies. For instance, the Turkish community in Berlin, where we collected our German data, is large and cohesive, whereas the Turkish community in the New York area, where we collected the US data, is smaller and more spread out.

Interestingly, we found more diverse patterns in clause combining (\citealt{IefremenkoEtAl2021}) and more register levelling \parencite{chapters/08, chapters/15} in the heritage speakers in the US. As another example, the German heritage speakers we studied in the US were all from recently immigrated families rather than the long\hyp standing language islands that have existed in the US for generations. Our speakers were not part of a cohesive community, and we found considerably more inter-individual variation in clausal structure \parencite{chapters/05} and the lexicon \parencite{chapters/10} than has been found for heritage speaker groups living in more cohesive communities.

\subsection{Resource-oriented perspective} \label{sec:introwieseetal:resource}

Arguably the most important result of RUEG is its support and furtherance of a resource\hyp oriented rather than deficit\hyp oriented view of heritage speakers. \citet{WieseEtAl2022} offers a summary of our main results in this regard. First, we found noncanonical patterns not only in bilingual speakers but also in monolingual speakers, including patterns that have so far been considered absent from native grammars, across linguistic domains (morphology, syntax, intonation, and pragmatics). Second, we found a degree of lexical and morphosyntactic inter\hyp speaker variability in monolinguals that was sometimes higher than that of bilinguals. This finding challenges the model of the streamlined native speaker. Third, we observed that noncanonical patterns were dominant in spoken and/or informal communicative situations, and this was true for monolinguals and bilinguals. In some cases, bilingual speakers were leading quantitatively, but there was no qualitative difference in the noncanonical productions. In heritage settings where the language was not part of formal schooling, we found tendencies of register levelling such that patterns associated with informal situations in monolinguals were also used in formal communicative situations by heritage speakers. 

\begin{sloppypar}
Our findings thus indicate possible quantitative differences and different register distributions rather than distinct grammatical patterns in bilingual and monolingual speakers. Indeed, the linguistic systematicity that became clear here showed that, contrary to previous work, the language use of heritage speakers is not primarily characterised by errors, erosion and incomplete acquisition, but provides insights into options of developments and variation within the spectrum of a language. This further supports the integration of heritage speakers into the native-speaker continuum (e.g., \citealt{TsehayeEtAl2021, WieseEtAl2022}).
\end{sloppypar}

\subsection{Dissemination} \label{sec:introwieseetal:dissemination}

Through conference contributions, workshops, invited talks and publications (see references throughout the following chapters), we were able to disseminate our research findings across sub-disciplines and theoretical schools. We have provided new findings on heritage speakers and on grammatical and pragmatic patterns in a range of languages, and new impulses for the study and conceptualisation of multilinguals and their linguistic resources. 

\begin{sloppypar}
We also were engaged in extensive outreach activities to transfer the knowledge gained through RUEG (e.g., through exhibitions, lectures, publications, workshops, web portals, and social media posts; see \citetv{chapters/04} for details). Our primary audience for these efforts has been educators of heritage language children, mainly through training workshops and collaborations with schools, as well as parents of heritage language children through collaboration with adult education programs. Last but not least, a number of outreach activities resulted in successful engagement and dialog with the general public.
\end{sloppypar}

Our experience in these various activities points to the importance of an approach that takes speakers’ agency seriously and shows awareness of their multilingual environments (e.g., \citealt{Bunk_subm, Bunk_iprep, TauschTsehaye2023}). As a result, we highlight the strategy of providing interactive activities that can be engaged with on multiple levels – by multilingual individuals themselves, by parents, by educators and others – including hands-on tasks, discussion prompts, quizzes, and information charts all based on the same materials developed from research results. We further emphasize the advantages of having the same resources and activities available in multiple languages to most effectively reach multilingual audiences (e.g. \citealt{PurkarthoferEtAl2023}). We also underline the value of communicating easily accessible yet research-grounded knowledge about bi- and multilingual language development, including the use of citizen science research activities.


%%%%%%%%%%%%%%%%%%%%%%%%%%%%%%%%%
%%%%%%%%%%%%%%%%%%%%%%%%%%%%%%%%%
%%%%%%%% start section 4 %%%%%%%%
%%%%%%%%%%%%%%%%%%%%%%%%%%%%%%%%%
%%%%%%%%%%%%%%%%%%%%%%%%%%%%%%%%%

\section{Overview of individual projects and chapters} \label{sec:introwieseetal:overview_chapters}

Part~\ref{part:I} of this volume (Chapters 2 to 4) introduces the methodology and application concerning corpus-based studies of heritage languages in the RUEG project. Part~\ref{part:II} (Chapters 5 to 9) focuses on the dynamics of morphosyntactic structure under a multilingual perspective, with studies of individual heritage languages and majority languages. It is complemented by Chapter 10 on the dynamics of the heritage speaker lexicon. Part~\ref{part:III} (Chapters 11 to 15) revolves around pragmatic aspects of heritage and majority varieties, including information structure, intonation, discourse markers and discourse openings and closings.

In Chapter 2, Wiese, Labrenz and Roy introduce the volume with their contribution \textit{Tapping into speaker's repertoires}. They provide the conceptual\hyp methodological background to RUEG’s collaborative enterprise, arguing for comparative studies that take into account speaker repertoires across groups, including language from formal and informal communicative situations in heritage speakers and majority speakers, bilinguals and monolinguals alike. They present the Language Situations method used to elicit the data in all RUEG projects, as well as two studies that were conducted to evaluate the method. The authors argue that the set-up was successful in eliciting language productions which reflect register distinctions in different communicative situations, and that this data is naturalistic in the sense that it aligns with spontaneous data.

In Chapter 3, Shadrova, Klotz, Hartz and Lüdeling introduce the RUEG corpus and its infrastructure, providing insights into the methodological considerations concerning modeling, compilation and analysis of a structured and deeply annotated corpus. As is aptly foreshadowed in the title, \textit{Mapping the mappings and then containing them all}, instead of providing the reader with a manual of how to construct and maintain a complex linguistic corpus, the authors shine a light on a range of theoretical and practical concerns such as how to keep the corpus intact and sound within reasonable production cycles requiring the collaboration of dozens of researchers, and discerning which types of models we face in a large scale corpus project on heritage languages. Questions like these are seldom addressed in relation to large-scale corpus projects, even though they are fundamental to the quality, success and sustainability of open resources.

In Chapter 4, Purkarthofer, Tracy, Grigoriadou and Tausch address \textit{Family language dynamics} by discussing the societal status of heritage languages, and by offering best practices for providing research-based knowledge about heritage language dynamics and repertoires to educators and bilinguals. As heritage languages are still often regarded as an obstacle rather than an asset, the authors target myths and misconceptions about heritage speakers. They also present different kinds of transfer activities revolving around three key aspects, namely communication, consultation and cooperative action, to emphasise the value of multilingualism from an early age.

In Chapter 5, Tsehaye, Tracy and Tausch target \textit{Inter- and intra-individual variation} in German clause structure to show how a heritage language develops outside of historically established language islands in what they term "tiny language islands". The authors present results on clause-type optionality, finite subordinate clause distribution and clausal peripheries, concluding that in heritage speakers clausal architecture presents as a stable feature, and that variation is found mostly in the morphological domain. In their discussion of inter-individual variation they emphasise that generalisations over groups can be misleading given that, in many cases, non-canonical but nevertheless systematic patterns can clearly be traced to individual speakers and do not characterise the whole group.

In Chapter 6, Özsoy, Rizou, Martynova, Gagarina, Szucsich and Alexiadou discuss \textit{Null subjects in heritage Greek, Russian and Turkish}, focusing on modality and animacy as influencing factors. They review results from corpus-based and experimental studies exploring whether heritage varieties, compared to monolingual data, pattern alike regarding the frequency of overtly realised pronominal subjects, especially in contact with Germanic majority languages. Their results indicate that animacy is a common factor affecting subject realization across groups. Further, quantitative differences between speaker groups and also between different communicative situations emerge depending on the various language combinations in the study. 

In Chapter 7, Rizou, Martynova, Özsoy, Szucsich, Alexiadou and Gagarina discuss \textit{Dynamics of verbal aspect in heritage Greek, Russian and Turkish}. The authors show that intra-linguistic factors like markedness and extra-linguistic factors like formality and mode variation affect speakers’ preference for aspectual forms in heritage languages, although not to the same degree.~The results from off-line tasks suggest that heritage speakers’ repertoires vary depending on the languages in contact, motivating further discussions on aspect based on multilingual comparisons.

In Chapter 8, Schroeder, Iefremenko, Katsika, Labrenz and Allen take a comparative approach to \textit{Clause combining in narrative discourse}, exploring subordination strategies in relation to formality and mode in the adolescents across all the RUEG speaker groups. The data show that typological characteristics of clause combining are largely preserved in the contact settings, despite speakers’ bilingualism. Nevertheless, register levelling is observed in contact settings, less so in Germany and more so in the US where the heritage language has less community support. The authors advocate that future investigations should pursue individual variation in more detail to gain further insights into sociolinguistic variables explaining the observed effects.

In Chapter 9, Keskin takes \textit{Balkan Turkic as a model for understanding contact-induced change in Turkish}, focusing on changes in head directionality and subordination. Considering the dialectal variation in Balkan Turkic, the author lays out trajectories of syntactic change in Turkic that can be used as a model to predict the changes in heritage Turkish. As regards head directionality change, he proposes that the observed heterogeneity across and within varieties can be traced back to intensity of contact, rates of change in different subsystems and universal constraints on grammar. For the domain of subordination Keskin proposes that in situations of prolonged contact between the the Turkic varieties under study, two subordination templates are blended into one.

In Chapter 10, Zürn, Keller, Tracy and Lüdeling discuss \textit{Dynamic properties of the heritage speaker lexicon}, focusing on verbs in heritage German in an English majority context. The contribution begins with an analysis of the lexical inventory as a whole and then moves on to details concerning particle verbs with respect to their structural and semantic properties. They also include production phenomena like self-correction, repetition and and hesitation in the immediate environment of particle verbs to elucidate the process of lexical access and the selection of words from a pool of available options. The results challenge claims concerning the limited vocabulary of heritage speakers by showing the creative ways in which heritage speakers employ morphological building blocks to convey subtle nuances in meaning.

In Chapter 11, Bunk, Allen, Zerbian, Pashkova, Zuban and Conti illustrate the relationship between \textit{Information packaging and word order dynamics in language contact}. The chapter reviews several studies discussing referent introduction in English as well as verb-third word order and modal particles in German, and presents a cross-linguistic study on left\hyp dislocation constructions in English, German, and Russian. The authors propose that the greater flexibility exhibited by heritage speakers concerning their word order choices might be due to several factors, depending on their specific linguistic biographies. These factors include the constant exposure of heritage speakers to the various linguistic structures of their majority and heritage language, the different quantity and quality of input in the heritage languages, and an increased sensitivity to communicative situations. 

In Chapter 12, Zerbian, Zuban, Böttcher and Bunk present their research on \textit{Intonation in heritage languages}, a topic that has received less attention in heritage language research than, for example, morphosyntactic issues. The authors investigate different intonational features (phrasing and frequency of pitch accents) as they are used across oral narrations as well as at specific interfaces such as questions, focus and discourse linking. In accordance with other findings reported in this volume, they conclude that, overall, heritage language utterances show quantitative rather than qualitative differences, which could be attributed to register levelling and/or to maintenance of language features.

While all other chapters focus on heritage languages, Chapter 13 by Pashkova, Böttcher, Katsika, Zuban, Zerbian and Allen takes a closer look at \textit{Majority English of heritage speakers}. The authors review studies relevant to the interface between information structure and four linguistic domains (prosody, article use, reference, and clausal syntax), as well as discourse structure. Their findings provide evidence that heritage speakers and monolingual speakers do not differ significantly in their use of the majority language in most of these areas. They suggest that the few differences found between these speaker groups in majority language use could be attributed either to crosslinguistic influence, or to a desire for explicitness and strict differentiation between their two languages.

In Chapter 14, Labrenz, Iefremenko, Katsika, Allen, Schroeder and Wiese present studies on \textit{Dynamics of discourse markers in language contact} from English, German, Turkish, and Greek. While previous literature has mostly focused on the borrowability of discourse markers, this contribution highlights functional convergence in bilingual settings. Furthermore, the target domain is expanded to graphic discourse markers by including the three-dot sign. The qualitative analyses address language-specific developments, majority-language influence, and potential register levelling, expressed by functional extensions or restrictions of discourse markers in heritage languages. Overall, the findings indicate differences between speaker groups concerning the frequency with which specific functions of discourse markers are used. 

In Chapter 15, Katsika, Labrenz, Iefremenko and Allen discuss \textit{Discourse openings and closings across language in contact}. A detailed analysis of their textual, subjective and intersubjective functions in the light of intercultural communication indicates that openings and closings are crucial in discourse organisation across communicative situations. In particular, the high frequency of openings across speaker groups suggests that monolingual and heritage speakers alike use situation-specific openings. The results also show evidence of language-specific patterns of discourse organisation that are similar across monolingual and heritage speakers, with pockets of evidence for cross-linguistic influence, register leveling, and individual variation.


%%%%%%%%%%%%%%%%%%%%%%%%%%%%%%%%%
%%%%%%%%%%%%%%%%%%%%%%%%%%%%%%%%%
%%%%%%%% start section 4 %%%%%%%%
%%%%%%%%%%%%%%%%%%%%%%%%%%%%%%%%%
%%%%%%%%%%%%%%%%%%%%%%%%%%%%%%%%%

\section{Conclusion} \label{sec:introwieseetal:conclusion}

Out of the many contributions to the study of heritage speakers and their language repertoires made by RUEG, perhaps the most tangible one is our open-access corpus, covering comparable informal vs. formal and spoken vs. written productions by adolescent and adult bilingual speakers of heritage Greek, Russian, and Turkish in Germany and the US, and of heritage German in the US, in both of their languages, as well as matching data from monolinguals in Germany, the US, Greece, Russia, and Turkey. This empirical resource can be used by researchers world-wide to investigate a wide variety of linguistic phenomena from a comparative perspective.

Our large-scale cross-linguistic study has allowed us to approach bilingual and monolingual speakers as native speakers of both their languages and to tease apart the various sources of noncanonical patterns in bilingual productions: language-internal developments and variation, cross-linguistic influence, general effects of bilingualism, and register levelling. Moreover, our research has provided new insights on register variation in different communicative situations among monolingual speakers and unveiled previously unreported patterns that are worthy of further investigation inside and outside a heritage language context.

Taken together, our findings suggest many profitable future directions for research. One possible direction is to expand the populations under investigation. The Language Situations method used in RUEG would adapt well to a range of ages, including both children and older adults which we did not examine, potentially uncovering developmental trends in noncanonical patterns. In addition, it would be useful to extend this method to cross-linguistic studies involving typologically more varied languages in contact than the ones included in our sample. This could help to discern a sharper distinction between the sources of noncanonical patterns in heritage productions. We also hope that research on the monolingual grammar will continue to examine the sources of variation we have identified in our corpus work. Moreover, some of our studies have shown that experimental research can be useful to complement and build on our corpus linguistic findings, by targeting details of the noncanonical patterns our elicitations uncovered in different groups and communicative situations. Further experimental research could tackle the subtle differences observed among our various groups with quantitative paradigms designed to explore these in more depth.

Another interesting direction would be to expand the contexts of this research to different societal settings. One factor noted in several of our studies is the impact of the cohesiveness of the speech community on the uniformity of noncanonical patterns. It would be interesting to explore this in a more targeted way, such as in a study that explicitly compares one heritage-majority language pair in cohesive vs. more spread-out communities. Another direction would be to investigate heritage speakers in the Global South where many countries do not have such strong majority languages. This would allow us to tease apart what in our findings pertains to heritage speakers in general and what is particular to bilingual speakers under specific conditions in countries with a monolingual habitus in the style of European nation states.

We are pleased to present this volume offering a detailed overview of the contributions of RUEG to the field of heritage language studies. We hope it inspires further development in the field in addition to contributing knowledge in its own right.


\section*{Acknowledgements}
\begin{sloppypar}
We gratefully acknowledge funding of the Research Unit \textit{Emerging Grammars in Language Contact Situations} (FOR 2537) from the Deutsche Forschungsgemeinschaft. This includes the following grant numbers: 394836232, 394837597, 394839191, 394841858, 394995401, 394838878, 394844953, 394844736, 394821310, 313607803. Without this funding, such a large and complex project would never have been possible.
\end{sloppypar}

We also gratefully acknowledge the 736 participants who graciously gave their time to be involved in our study, the many individuals and organisations that helped us to contact these participants, the many research assistants who collected and transcribed and annotated and processed the data, the doctoral students and postdocs and principal investigators who led the individual studies reported here, the cooperation partners and Mercator Fellows involved in RUEG, and the many colleagues who have provided insights and encouragement over the years of the RUEG project. We also gratefully acknowledge the 30 external reviewers and 23 internal reviewers whose careful commentary on earlier versions have substantially improved the quality of the chapters in this volume. Finally, we thank Lea Coy for help with pre-publication formatting.

\printbibliography[heading=subbibliography,notkeyword=this]
\end{document}
