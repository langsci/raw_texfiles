\documentclass[output=paper,colorlinks,citecolor=brown]{langscibook}
\ChapterDOI{10.5281/zenodo.15775181}

\title{Information packaging and word order dynamics in language contact} 

\author{Oliver Bunk\orcid{0000-0003-4505-4873}\affiliation{Humboldt-Universität zu Berlin} and Shanley E. M. Allen\orcid{0000-0002-5421-6750}\affiliation{University of Kaiserslautern-Landau} and  Sabine Zerbian\orcid{0000-0002-4631-369X}\affiliation{University of Stuttgart} and Tatiana Pashkova\orcid{0000-0002-6676-9555}\affiliation{University of Kaiserslautern-Landau} and Yulia Zuban\orcid{0009-0009-3033-1760}\affiliation{University of Stuttgart} and Erica Conti\orcid{}\affiliation{Humboldt-Universität zu Berlin}}

\abstract{Word order is one of the linguistic resources speakers use to express specific meanings and present information. The literature considers information structure a major driving force behind word order and word order variation. The chapter argues that this is particularly true in language contact settings, enhancing the emergence of new word order patterns. We review studies that were conducted in the RUEG group, covering a) language-specific phenomena (referent introduction in English, V3 clauses and the placement of modal particles in German, and object-verb/verb-object patterns in Russian) and b) a cross-linguistically available construction -- left dislocation in English, German, and Russian.

\keywords{word order variation, information packaging, left dislocation, referent introduction, OV/VO}
}

\IfFileExists{../localcommands.tex}{
   \addbibresource{../localbibliography.bib}
   \usepackage{langsci-optional}
\usepackage{langsci-gb4e}
\usepackage{langsci-lgr}

\usepackage{listings}
\lstset{basicstyle=\ttfamily,tabsize=2,breaklines=true}

%added by author
% \usepackage{tipa}
\usepackage{multirow}
\graphicspath{{figures/}}
\usepackage{langsci-branding}

   
\newcommand{\sent}{\enumsentence}
\newcommand{\sents}{\eenumsentence}
\let\citeasnoun\citet

\renewcommand{\lsCoverTitleFont}[1]{\sffamily\addfontfeatures{Scale=MatchUppercase}\fontsize{44pt}{16mm}\selectfont #1}
  
   %% hyphenation points for line breaks
%% Normally, automatic hyphenation in LaTeX is very good
%% If a word is mis-hyphenated, add it to this file
%%
%% add information to TeX file before \begin{document} with:
%% %% hyphenation points for line breaks
%% Normally, automatic hyphenation in LaTeX is very good
%% If a word is mis-hyphenated, add it to this file
%%
%% add information to TeX file before \begin{document} with:
%% %% hyphenation points for line breaks
%% Normally, automatic hyphenation in LaTeX is very good
%% If a word is mis-hyphenated, add it to this file
%%
%% add information to TeX file before \begin{document} with:
%% \include{localhyphenation}
\hyphenation{
affri-ca-te
affri-ca-tes
an-no-tated
com-ple-ments
com-po-si-tio-na-li-ty
non-com-po-si-tio-na-li-ty
Gon-zá-lez
out-side
Ri-chárd
se-man-tics
STREU-SLE
Tie-de-mann
}
\hyphenation{
affri-ca-te
affri-ca-tes
an-no-tated
com-ple-ments
com-po-si-tio-na-li-ty
non-com-po-si-tio-na-li-ty
Gon-zá-lez
out-side
Ri-chárd
se-man-tics
STREU-SLE
Tie-de-mann
}
\hyphenation{
affri-ca-te
affri-ca-tes
an-no-tated
com-ple-ments
com-po-si-tio-na-li-ty
non-com-po-si-tio-na-li-ty
Gon-zá-lez
out-side
Ri-chárd
se-man-tics
STREU-SLE
Tie-de-mann
}
   \boolfalse{bookcompile}
   \togglepaper[11] %%chapternumber
}{}

\begin{document}
\lehead{Oliver Bunk et al.}
\maketitle

\section{Introduction} 
In a world of increasingly dynamic global migration and communication, language contact has become an important and pervasive phenomenon in linguistic research. The interaction of different languages can result in the emergence of new linguistic forms and the shaping of language use and communication. One area of particular interest in studying language contact is the relationship between information packaging and word order. Information packaging involves two key concepts: information structure and information status. Information structure refers to how speakers encode and present information in a sentence and includes notions such as focus and topic (see \cite{krifka_information_2012}). Information status means the status of referring expressions as referentially and/or lexically given or new in discourse (see \cite{BaumannStefan2006Tiog}). Both concepts affect the packaging of information in spoken and written texts in order to respond to the immediate communicative needs of the interlocutor (see \cite{chafe_1976}).

The impact of language contact on information packaging and word order is a topic of ongoing research and debate. Information structure and status have been identified as one of the main causes of word order variation in languages with relatively free word order, such as German (\cite{musan_informationsstrukturelle_2002}), Russian (\cite{jasinskaja_information_2016}), and Turkish (\cite{ozsoy_word_2019}), as well as languages with fixed word order, such as English (\cite{ward_information_2004}). Across different languages, word order has been shown to be influenced by specific information packaging principles including the given-before-new principle, the end-focus principle, the end-weight principle, and the complexity principle (see \cite{hilpert_information_2021}).

For example, according to the given-before-new principle (\citealt{halliday1967notes, hilpert_information_2021}), discourse-given referents tend to appear before discourse new referents across different languages, such as English and Russian (\cite{haviland_whats_1974, kathryn_bock_syntactic_1980, arnold_heaviness_2000, slioussar_grammar_2007}, \citeyear{slioussar_processing_2011}). In Russian, this pattern may lead to word orders with inversion ((XP) V S) or dislocation (XP S V), differing from the default subject-verb-object (SVO) pattern (e.g., \cite{king_configuring_1995, kallestinova_aspects_2007, bailyn_syntax_2012}). In German, constituents can scramble within the clause to change their information status (e.g., \cite{musan_informationsstrukturelle_2002}). English uses specific syntactic constructions for highlighting and focusing. For example, locative inversion and existential \textit{there} structures highlight the introduction of new information postverbally (\cite{ward_information_2004}).


The differences in word order flexibility and the different strategies in conveying information syntactically are particularly interesting from a cross-linguistic perspective, and language contact settings can be expected to enhance our understanding of the dynamic relationship between word order and information packaging. Various studies point to different factors influencing word order variation in bilinguals. For particular word order patterns, some studies suggest cross-linguistic influence as the explanatory factor (\cite{polinsky_heritage_2018}). However, for other phenomena, there is no evidence of such cross-linguistic influence. Rather, bilingual speakers seem more dynamic in language use than monolingual speakers (\cite{wiese_heritage_2022}). This claim is based on the idea that exposure to multiple languages and linguistic systems allows bilinguals to draw on a wider range of linguistic resources, resulting in a more flexible and adaptive approach to language use (\cite{WieseHeike2016Cinu}). In particular, bilinguals may be more likely to adapt information packaging and word order in response to language contact, as they can switch between different linguistic systems and draw on a range of linguistic resources to encode information, leading to either new emerging word order patterns or more frequent use of noncanonical patterns. In this chapter, we delve into the relation between information packaging, encompassing information structure and status, and word order, which seems to be dynamic in language contact situations and bilingual speakers.

This chapter presents studies within the RUEG group\footnote{The Research Unit \textit{Emerging Grammars in Language Contact Situations: A Comparative Approach} (FOR 2537, 2018--2024, \url{https://hu.berlin/RUEG}), funded by the DFG, investigated linguistic dynamics in monolingual and multilingual speakers' repertoires. The present article presents results from the subproject P8 (PIs: Shanley E. M. Allen, Oliver Bunk, Sabine Zerbian), focussing on information structural dynamics in contact situations.} investigating word order phenomena related to information packaging in English, Russian, and German. We consider different contact settings, namely 1) settings in which these languages are used as majority languages by monolinguals and by speakers of different heritage languages (i.e., Germany, the US, and Russia) or 2) settings in which two of these languages, German and Russian, are used as heritage languages (i.e., Germany and the US). We start by discussing the empirical basis of the analyses and findings for specific phenomena in English (referent introduction), German (V3 and modal particles), and Russian (word order in main/embedded clauses and OV/VO patterns). Then, we turn to the dynamics of information packaging from a cross-linguistic perspective, discussing a noncanonical word order pattern that occurs in all three languages, namely left dislocation constructions. To our knowledge, no study has systematically looked at the use and structure of left dislocation in different languages across different speaker groups (i.e., monolingual and bilingual) and different communicative settings. Thus, our study is one of the first to attempt such an endeavor and provides a broad perspective on noncanonical variation at the interface of word order and information packaging. Finally, we summarize our results and discuss implications for further research.

\section{Word order variation in English, German, and Russian}
English, German, and Russian exhibit various word order patterns related to information packaging. The languages differ concerning word order flexibility, making these language contact settings particularly interesting. As a language with flexible word order, Russian has a default SVO pattern, but all other alternations are possible, fulfilling information structural needs (e.g., \cite{sirotinina_porjadok_2003, slioussar_grammar_2007}, \citeyear {slioussar_processing_2011}). German is an SOV language with a V2 constraint, leading to finite V2 in declarative main clauses and finite verb last in subordinate clauses. English is strict SVO with residual V2. In the following sections, we present findings pronounced in bilingual speakers, indicating that this group is particularly prone to word order variation triggered by information packaging.  

\subsection{Data basis}

The empirical basis for the analyses is the RUEG corpus\footnote{\url{https://hu.berlin/RUEG-corpus}} \parencite{RUEGcorpus2024}. The data were elicited using one common experimental set-up (\cite{wiese_language_2020}), which allows for a systematic comparison of language in different communicative situations. Participants were asked to picture themselves as a witness to a car accident that they had just watched in a short video clip. They were tasked to describe the accident in four communicative situations, differing in mode (spoken vs. written) and formality (formal vs. informal). \tabref{bunk:ComSit-Wiese2020} illustrates these four contexts. 

\begin{table} [h!]
\label{tab:bunk:Lang-Sit_set-up}
  \begin{tabularx}{\textwidth}{llQ}
    \lsptoprule
    & Formal & Informal \\
    \midrule
    Spoken & voice recording to the police & WhatsApp$\copyright$ audio message to a friend \\
    Written & witness report to the police & WhatsApp$\copyright$ text message to a friend \\
    \lspbottomrule
  \end{tabularx}
  \caption{Communicative situations simulated in the Lang-Sit set-up (\cite{wiese_language_2020})}
  \label{bunk:ComSit-Wiese2020}
\end{table}

The corpus comprises data from adolescent and adult speakers from different countries and with different language biographies. Speakers were from Germany, Greece, Russia, Turkey, or the US. They were either monolingually or bilingually raised.\footnote{We consider monolingually-raised speakers as speakers who grew up with one language as their family language. Bilingually-raised speakers are considered speakers who grew up with another family language in addition to the majority language. These speakers were either born in Germany, Greece, Russia, Turkey, or the US with the respective majority languages or migrated to these countries before the age of four years.} Monolinguals spoke the majority language of their respective countries, i.e., German, Greek, Russian, Turkish, or English. Bilingual speakers were tested in Germany and the US. In Germany, bilingual speakers were speakers of the majority language (German) and Greek, Russian, or Turkish as their heritage language. In the US, bilingual speakers were speakers of the majority language (English) and German, Greek, Russian, or Turkish as their heritage language. \tabref{bunk:Num-speak-subcorp} gives an overview of the subcorpora (RU-RUEG = Russian subcorpus, EN-RUEG = English subcorpus, DE-RUEG = German subcorpus; h=heritage, mo=monolingual), including the numbers of speakers per group.


\begin{table}[h!]
\label{tab:bunk:speakers}
  \begin{tabular}{l lr}
    \lsptoprule
         & Speaker & Number \\
         & group   & of speakers\\
    \midrule
    \multicolumn{3}{l}{RUEG-RU} \\
         & mo-Russian (RUS) & 67\\
         & h-Russian (GER)  & 60\\
         & h-Russian (US)   & 69\\
    \multicolumn{3}{l}{RUEG-EN} \\
         & mo-English (US) & 64\\
         & h-German (US)   & 34\\
         & h-Greek (US)    & 65\\
         & h-Russian (US)  & 65\\
         & h-Turkish (US)  & 59\\
    \multicolumn{3}{l}{RUEG-DE} \\
         & mo-German (GER) & 64\\
         & h-German (US)   & 36\\
         & h-Greek (GER)   & 45\\
         & h-Russian (GER) & 61\\
         & h-Turkish (GER) & 65\\
    \lspbottomrule
  \end{tabular}
  \caption{Numbers of speakers in the different subcorpora}
  \label{bunk:Num-speak-subcorp}
\end{table}

While the RUEG corpus was the basis for all of our investigations, we used additional corpora for German, to conduct comparative analyses across different language contact scenarios. We provide information on these corpora in the respective sections.

\subsection{Referent introduction in English} 
\label{sec:bunk-ref-introduction-eng}

Even though English word order is canonically SVO, it can be changed according to pragmatic constraints, resulting in specific information packaging constructions (\cite{huddleston_syntactiv_2002}). For example, the given-before-new principle (\cite{biber_grammar_2021, hilpert_information_2021}) discussed in the introduction leads to a conflict for new subjects in English: a new subject referent has to be placed first because of SVO word order, but ideally, it would be placed closer to the end of the clause because the referent is new.


To resolve this tension, speakers can use several non-SVO constructions that put the original SVO subject referent after the finite verb, for instance, existential \textit{there} (\ref{ex:bunk-woman-a}), presentational \textit{there} (\ref{ex:bunk-woman-b}), locative inversion (\ref{ex:bunk-woman-c}), and passivization (\ref{ex:bunk-woman-d}). The referent might remain the syntactic subject (\ref{ex:bunk-woman-c}) or take a different syntactic role~-- a notional subject in (\ref{ex:bunk-woman-a}) and (\ref{ex:bunk-woman-b}) or an object of a preposition in (\ref{ex:bunk-woman-d}). 

\begin{exe}
\ex \label{ex:bunk-woman} Original clause: {\textbf{A woman} was unloading her groceries on the other side of the street.}
\begin{xlist}
        \ex \label{ex:bunk-woman-a} {There was \textbf{a woman} unloading her groceries on the other side of the street.}
        \ex \label{ex:bunk-woman-b} {There stood \textbf{a woman} unloading her groceries on the other side of the street.}
        \ex \label{ex:bunk-woman-c} {On the other side of the street was \textbf{a woman} unloading her groceries.}
        \ex \label{ex:bunk-woman-d} <I saw some groceries.> {They were unloaded by \textbf{a woman}.}
    \end{xlist}
\end{exe}

There is some evidence that bilingual speakers use information packaging constructions differently compared to monolingually-raised English speakers, possibly because of cross-linguistic influence. For example, speakers of Singapore and Jamaican English use fewer existential \textit{there} constructions than speakers of monolingual varieties (\cite{winkle_non-canonical_2015}).  

Our pilot findings in majority English showed differences and similarities between bilinguals  and monolinguals. Unlike monolinguals, bilinguals used locative inversion for new subjects. Both groups used existential \textit{there} to a similar extent. 

Based on these previous findings, \citet{pashkova_left_nodate} focused on the syntactic structures that are used for the introduction of new subjects in majority English. We asked if English bilinguals use more non-SVO constructions for the introduction of new subjects as compared to English monolinguals. We hypothesized that English bilinguals with heritage Russian and heritage Turkish, languages with flexible word orders, would use more non-SVO structures in English, possibly because of the cross-linguistic influence from the heritage languages. To evaluate the hypothesis, we compared the syntactic structures used for new vs. given subjects produced by 82 English bilinguals (40 heritage Russian and 42 heritage Turkish) vs. 40 English monolinguals. Each subject was annotated for its information status (new vs. given) and for the syntactic structure it was used in. Subsequently, the syntactic structures were divided into SVO-type structures (SVO and copular clauses without inversion) and non-SVO-type structures (existential and presentational \textit{there}, locative and non-locative inversions, right and left dislocations, passives, questions, \textit{it}- and pseudo-clefts). The results of the analysis indicated no difference between the frequency of use of non-SVO structures by English bilinguals and monolinguals. In contrast, there was a difference between new and given subjects in the frequency of non-SVO structures: new subjects were more likely to appear in such a structure than given subjects. There was no interaction between the information status of the subject and speaker group (English bilinguals vs. monolinguals). These results show that all speakers in our sample, regardless of their bilingualism, preferred to use non-SVO structures for new subjects more than for given subjects. The data indicate that new subjects appear more frequently in constructions referring to information status (i.e., non-SVO), and both bilinguals and monolinguals similarly follow this trend. Cross-linguistic influence, as suggested by other studies, does not seem to play a major role. Rather, dynamics in information packaging appear to influence word order patterns in both speaker groups. 

\subsection{V3 and modal particles in German}

In contrast to English, German is a more flexible language, allowing for a range of word order patterns. German is generally considered an SOV language with a V2 constraint, placing the finite verb in the second position in declarative clauses. Various constituents, such as subjects, objects, or adverbials, can occupy the preverbal position. V2 is considered a rigid constraint in German by the vast majority of the literature, and deviations from V2 are often claimed to be ungrammatical (see \ref{ex:bunk-Johann-danced}): 

\begin{exe}
 \ex \label{ex:bunk-Johann-danced} {*} {
 \gll {Gestern} {Johann} {hat} {getanzt.} \\
 yesterday  Johann  has  danced  \\
 \glt \hspace{0.25cm} ‘Yesterday, Johann danced.’ \citep[137]{roberts_extended_2002}
 }
\end{exe}

However, V3 patterns occur systematically in everyday language use, as (\ref{ex:bunk-drop-ball}) exemplifies:  

\begin{exe}
 \ex \label{ex:bunk-drop-ball}
 \gll {danach} {er} {lässt} {den} {ball} {fallen} \\
 after.that he lets the ball fall \\
 \glt ‘after that, he drops the ball’ (\nocite{wiese_rueg_2019}RUEG corpus, DEbi58MT\_isD) 
\end{exe}

These instances show an adverbial-subject-finite verb linearization (but see \cite{sluckin_non-canonical_2021} for other V3 orders, such as adverbial-adverbial-finite verb orders, sparsely occurring) and are not only reported for German (see \cite{WieseHeike2013Wcnu}), but also for other V2 languages in language contact settings. V3 occurs in urban contact dialects in Sweden (see \cite{kotsinas_immigrant_1992}), Denmark (see \cite{quist_ny_2000}), Norway (see \cite{opsahl_wolla_2009}), and the Netherlands (see \cite{marieke_meelen_v3_2020}). It is also documented in contexts where Germanic languages are spoken as minority languages, including German in Namibia and the US (see \cite{tracy_it_2010, sewell_sociolinguistic_2015, wiese_hidden_2018}), heritage Norwegian (see \cite{alexiadou_v3_2018, KinnKari2021Pdnp, KinnKari2022Pdih}), heritage Low German (see \cite{rocker_variation_2022}), heritage Swedish (\cite{KinnKari2021Pdnp, KinnKari2022Pdih}), heritage Danish (\cite{KühlKaroline2018WOiA}) and heritage Icelandic (\cite{Arnbjornsdottir2018}). V3 is also reported for monolingual German speakers (see \cite{schalowski_adverbial_2017, wiese_hidden_2018, bunk_aber_2020}). \citet{wiese_hidden_2018} find that multilingual speakers use more V3 structures than monolinguals, indicating that word order variation concerning the V2 constraint is particularly dynamic in this speaker group. 


\citet{WieseEtAl2022} looked at the distribution of V3 sentences across speaker groups in the RUEG corpus. Based on the corpus version 0.4.0 \parencite{RUEG-Corpus-0.4.0}, the study highlights that most V3 sentences were produced by bilingual speakers in Germany and the US. However, V3 also occurred in monolingual speakers. A closer look at the speakers in Germany revealed that bilingual speakers with Turkish as their heritage language produced the vast majority of the V3 sentences. \citet{wiese_heritage_2022} argue that these findings speak against contact\hyp linguistic transfer because Greek and Russian tend to have SVO, which might more easily allow for V3, while Turkish is predominantly SOV, making transfer to V3 in German less likely compared to the other two languages. However, for heritage German in the US, cross-linguistic effects might be at play, as these speakers produced V3 that also involves non-subjects (see \cite{wiese_heritage_2022}). In a follow-up study using the much larger database of the corpus version 1.0 \parencite{RUEGcorpus2024}, we found a similar distribution of V3 (see \cite{bunk_status_nodate}).


Information structure plays a crucial role in the emergence and use of German V3 patterns (see \cite{WieseHeike2009Giim, schalowski_adverbial_2017, wiese_hidden_2018, bunk_aber_2020}). The adverbial-subject-finite verb linearization is closely tied to a “frame-setter $>$ (aboutness) topic $>$ comment” order and this pattern appears to hold in non-verbal contexts, i.e., contexts in which participants are asked to retell a story using only toys and word cards (see \cite{wiese_language_2020}) as well as in second language speakers (\cite{bunk_v2_nodate}). \citet{WieseHeike2016Cinu} argue that bilinguals more frequently use V3 since they are exposed to more linguistic variation, leading to a less strict inventory of grammatical rules, which they apply more productively to form noncanonical patterns. However, V3 also occurs in monolingual speakers, and psycholinguistic evidence suggests that these speakers a) process V3 as an integral part of German grammar and b) judge adverbial-subject-finite verb orders as more grammatical than adverbial-object-finite verb, arguing for a representation of the former but not the latter structure (see \cite{bunk_aber_2020}). Taken together, these studies point to a cognitive preference for “frame-setter $>$ (aboutness) topic $>$ comment” orders as a way of information packaging. 


From these previous findings, it seems that V3 spotlights the interaction of information structure and word order. Concurrently, it sheds light on the external interface of syntax and discourse (see \cite{sorace_epistemological_2011}). The initial adverbial functions as a frame-setter at the information structural level, as a discourse marker at the discourse structuring level (see \cite{schalowski_adverbial_2017}), or takes both functions simultaneously (see \cite{bunk_aber_2020}). \citet{bunk_status_nodate} investigate the distribution of V3 types (frame-setting V3 vs. discourse-connecting V3 vs. ambiguous cases) in the RUEG corpus, the Kiezdeutsch-Korpus (KiDKo, \cite{kidko}), the DNam corpus (\cite{Dnam_corpus}), and a collection of V3 sentences from heritage Low-German in the US (\cite{rocker_variation_2022}), focusing on spoken language. While RUEG comprises data from the US and Germany, KiDKo contains data from multilingual and monolingual adolescents from Germany. DNam includes data from adult and adolescent speakers of the German-speaking community in Namibia. All corpora slightly differ in size and number of speakers, however normalized numbers allowed for a comparison of the data. The corpora permit investigating different contact scenarios with different statuses of German: 1) as a majority language (Germany), and 2) as a minority/heritage language (US and Namibia). While in the US, German is on the decline and influenced by majority English, the speech community in Namibia still uses German as a vital language. We were interested in potential differences in the impact of other (majority) languages due to these diverging statuses of German.

The data were annotated for the type of adverbial (frame-setter, discourse-connector, ambiguous) in all corpora. We found that frame-setting and discourse-connecting V3 structures were used at similar rates in the RUEG and DNam corpus and that only a few cases were ambiguous. KiDKo exhibited only a few discourse-connecting V3 sentences, while most were framesetting or ambiguous. We argued that differences between KiDKo and RUEG\slash DNam are due to contextual factors, such as the oppositions between dialogue vs. monologue and free conversation vs. narration. However, we found that (\textit{und}) \textit{dann} (‘(and) then’) is the preferred adverbial in V3 across all corpora (see \cite{wiese_hidden_2018} for a detailed discussion of \textit{dann} (‘then’) in KiDKo, see \cite{sewell_sociolinguistic_2015} for the prevalence of temporal adverbs licensing V3 in heritage German in Wisconsin German). We were also interested in the role of prosody in disambiguating frame-setting from discourse-connecting adverbials in V3, and found that prosodic boundaries might provide a disambiguating cue for the interlocutor (\cite{bunk_status_nodate, chapters/12} for a detailed summary).

Another case where information structure and discourse structure impact word order, particularly in the peripheries, is modal particles (henceforth MPs). MPs indicate “to the hearer the mood or attitude of a speaker" (\cite[183]{bross_german_2012}). \citet{bunk_sociolinguistic_nodate} focus on the MPs \textit{eben} and \textit{halt}, investigating their distribution across speaker groups, function, and syntactic structure.

\textit{Eben} and \textit{halt} are often considered synonyms (e.g., \cite{hentschel_funktion_1986, diewald_abtonungspartikel_2007}), marking a proposition as definite and irrevocable, indicating irreversibility and resignation (\cite{helbig_lexikon_1988}). However, others acknowledge subtle differences in meaning (see \cite{thurmair_modalpartikeln_1989, thielmann_halt_2015, bluhdorn_modalpartikeln_2019}). MPs are typically restricted to the so-called “middlefield", the position between a finite and non-finite verb or verbal parts (\cite{hohle_begriff_1986}). Both finite and non-finite verbs and verbal particles form the “left sentence bracket" and “right sentence bracket" surrounding the middlefield. \tabref{bunk:MP-middlefield} provides an example of MPs in their canonical position in the middlefield. 

\begin{table}[h]
\label{tab:bunk:MP_position_middlefield}
    \caption{MPs in their canonical position in the middlefield}
    \label{bunk:MP-middlefield}
    \begin{tabular}{lllll}
      \lsptoprule
      Prefield & L. verb bracket  & Middlefield & R. verb br. & Postfield \\
      \midrule
      {Der Mann} & {hat} & {\textbf{halt{/}eben}} {den Ball} & {in der Hand.} & {} \\
      the man & has & MP{/}MP the ball & in his hand & {} \\
      \addlinespace
      \multicolumn{5}{l}{‘The man has the ball in his hand.’} \\
      \lspbottomrule
    \end{tabular}
\end{table}

In this study, we investigated \textit{halt} and \textit{eben} in the RUEG corpus, KiDKo, and the DNam corpus, again, to tease apart the influence of the societal macro context and the status of German on the use of the two MPs. The corpora comprise data from Germany (RUEG, KiDKo), the US (RUEG), and Namibia (DNam).\largerpage

In line with previous studies on the distribution of \textit{halt} and \textit{eben} (e.g., \cite{elspas_zum_2005}), our data show that all speech communities prefer \textit{halt} over \textit{eben}, except for bilingual speakers of heritage German in the US. Here, we only encountered one occurrence of \textit{halt} and no occurrence of \textit{eben}. Interestingly, we found several cases where both MPs appear at the edge of the sentence and not in the middlefield. Peripheral \textit{halt} and \textit{eben} were more frequently used by multilingual than monolingual speakers, where we found only one instance of noncanonical particle placement in the right periphery. \citet{thurmair_zur_2020} argues that \textit{halt} can only appear in the right periphery but not in the left periphery, losing its function as a MP and rather functioning as a discourse particle, toning down the importance of the preceding information. \citet{imo_individuelle_2008} finds \textit{halt} in the left periphery, where it still functions as a MP. In our data, \textit{halt} was used as both discourse and modal particle in the left and the right periphery, predominantly frequent in the multilingual speaker groups. Thus, our data indicate that multilinguals not only use \textit{halt} more frequently in a noncanonical position but also with a wider functional spectrum, including the highlighting of important information. However, there might also be sociolinguistic factors at play. While Namibia considers multilingualism the norm, Germany and the US are characterized by monoglossic ideologies. The absence of MPs in the US might indicate a strong influence of English as majority language due to these monoglossic ideologies. Previous studies indicate that these differences in macro contexts lead to different types of contact linguistic varieties (\cite{wiese_heritage_2022}) and linguistic structures (\cite{bunk_sociolinguistic_nodate, bunk_bare_nodate}). Thus, an external factor of linguistic variation such as different language contact settings might lead to different linguistic structures as an external factor of linguistic variation. 


\subsection{Word order in Russian}\largerpage

Of the three languages considered here, Russian is the most flexible concerning word order. Russian is reported to have a basic SVO order in pragmatically-neutral contexts (i.e., in broad focus contexts). However, Russian allows remarkable word order variation that is highly governed by information structure/status (\cite{svedova_russkaja_1980, kovtunova_sovremennyj_2002, sirotinina_porjadok_2003, kallestinova_aspects_2007, slioussar_grammar_2007}, \citeyear{slioussar_processing_2011, bailyn_syntax_2012}).

Several studies investigate word order in heritage Russian in Germany. However, these studies do not explicitly investigate the impact of information structure. For example, \citet{brehmer_lets_2015} report that bilinguals differed from monolingually-raised speakers of Russian by producing significantly more V-final linearizations in both main and embedded clauses. The authors explain the results in embedded clauses by referring to the influence from the majority language German, which has finite V-final in embedded clauses, whereas the results in main clauses might be due to pragmatic unmarking, in other words, V-final orders being used in a wider range of contexts (for details see \cite{brehmer_lets_2015}).

In contrast to \citet{brehmer_lets_2015}, \citet{zuban_word_2021} and \citet{martynova_ovvo_nodate} found that bilinguals with heritage Russian in Germany were similar to monolingual speakers of Russian regarding the choice of different word orders in both main and embedded clauses, not taking information structural factors into consideration. \citet{zuban_word_2021} investigate word orders of subject, verb, and (direct or indirect) object (i.e., SVO, OVS, SOV, OSV, VOS, VSO) of 16 adolescent bilinguals residing in the US and Germany and 8 age-matched monolingual speakers of Russian in the Russian subcorpus of the RUEG corpus. The data were manually annotated for clause type (main/embedded), verb type (e.g., auxiliary, copula, finite, gerund, infinitive), and word order pattern. The word orders included in the analysis contained a nonoblique subject, a finite verb, and an object (either direct or oblique). The overall dataset consisted of 783 clauses. The study revealed that bilinguals with heritage Russian in Germany were similar to monolingual speakers regarding their word order repertoire and word order distribution in both main and embedded clauses.

\citet{martynova_ovvo_nodate} examined the choice of OV/VO orders by 24 adolescent bilingual speakers in the US and Germany and monolingual speakers of Russian. The overall dataset consisted of 1,010 clauses. The data were manually annotated for clause type (main/embedded), verb type (e.g., auxiliary, copula, finite, gerund, infinitive), word order pattern (either OV or VO), and object realization (nominal or pronominal). Contrary to \citet{zuban_word_2021} and \citet{brehmer_lets_2015}, the study focused on the position of the object in relation to the verb, such that the word orders included in the analysis were OV and VO with either a finite or a non-finite verb and at least one object (direct or oblique). In addition, the study by \citet{martynova_ovvo_nodate} explicitly considered the influence of object realization (noun/pronoun) on OV/VO choice. 

The study by \citet{martynova_ovvo_nodate} revealed that bilinguals in Germany were similar to monolingual speakers of Russian. Further, clause type and object role influenced the choice of OV vs. VO order in the bilingual and monolingual speakers similarly. Specifically, and in accordance with the literature on Standard Russian, the probability of OV over VO orders significantly decreased in embedded clauses compared to main clauses. Furthermore, and again in line with the literature on Standard Russian, objects realized by pronouns were associated with OV orders, while objects realized by nouns were associated with VO orders in narrations of both speaker groups.\largerpage

As already pointed out, the findings in \citet{zuban_word_2021} and \citet{martynova_ovvo_nodate} contradict those of \citet{brehmer_lets_2015}. These differences could be due to 1) different text types (semi-spontaneous narrations in different registers in \citet{zuban_word_2021} and \citet{martynova_ovvo_nodate}; vs. written narrations in \citet{brehmer_lets_2015}), 2) differences in data annotation, and/or 3) differences in the data grouping and analysis.

As for the influence of information packaging on word order of heritage speakers in Germany, some studies such as \citet{zuban_word_2021} and \citet{brehmer_lets_2015} acknowledge that word orders produced by heritage speakers are not always used to express the expected information structure or status. However, the above-mentioned studies do not explicitly focus on information packaging and do not provide any quantitative results to support their observations. In the following paragraphs, we discuss one study that has examined the expression of information status by heritage speakers of Russian in Germany (\cite{zuban_unexpected_2023}).

In formal speech in monolingual Russian, in accordance with the given-before-new principle, given referents canonically occur before new referents and new referents appear postverbally and clause-finally (\cite{slioussar_grammar_2007, slioussar_processing_2011}). In informal speech, this given-new order can be violated (e.g., \cite{sirotinina_porjadok_2003}), leading to a noncanonical new-given order. \citet{zuban_unexpected_2023} examined the syntactic and prosodic expression of referents with the order new-given produced in the RUEG corpus. The study analyzes the data of 120 speakers of the following groups: heritage speakers of Russian in the US ($N=40$: 20 adolescents and 20 adults), heritage speakers of Russian in Germany ($N=40$: 20 adolescents and 20 adults), and monolingual speakers of Russian ($N=40$: 20 adolescents and 20 adults). The study focuses (among other things) on two questions: 1) whether heritage speakers of both groups and monolingual speakers in Russia produce noncanonical orders of referents (i.e., new before given referents in one clause), and 2) whether the frequency of the order of referents new-given is similar across the three speaker groups. 

The data were manually annotated for word order based on \citet{bailyn_syntax_2012} and \citet{villavicencio_learning_2002}. Information status of the 23 most frequent referents was manually annotated according to the RefLex scheme (\cite{riester_reflex_2017}) and language-specific principles such as the position of a referent in a clause or the expression of a referent by lexical or morphosyntactic means. Four types of referents were annotated: new, bridging, unused, and given (see their definitions in \citealt[5]{riester_reflex_2017}). 
The study shows that heritage speakers in Germany and monolingual speakers produced referents with the order new-given, as shown in the “target” in (\ref{ex:bunk-russ-dog-ball}):\largerpage

\begin{exe} 
\ex \label{ex:bunk-russ-dog-ball}
 Context: \\
 {i: v tot moment idët čelovek s mjačom (-) gde (-) u nego iz ruki (--) otpuskaetsja mjač i vykatyvaetsja na dorogu} 
 \glt ‘and: at that moment there is a person walking with a ball where the ball is dropped from his hand and rolls out onto the road'\medskip\\
Target: \\
\# \glll {potom} {sobaka} {(-)} {uvidela} {mjač} \\
{} \textsc{s}\textsubscript{new} {} \textsc{v} \textsc{o}\textsubscript{given} \\
then dog {} saw ball\\
\glt \hskip 0.7 cm ‘then a dog saw the ball’ \medskip\\
Expected word order: \\
\gll {potom} {mjač} {uvidela} {sobaka} {}\\
 {} \textsc{o}\textsubscript{given} \textsc{v} \textsc{s}\textsubscript{new} (RUEG corpus, DEbi74MR\_isR) \\
\end{exe}

Overall, heritage speakers in Germany produced 52 instances of new-given combinations out of 408 clauses with different discourse-new referents (i.e., new, unused, and bridging), while monolinguals had 22 combinations of new-given out of 292 clauses with different discourse-new referents. The numerical difference of new-given combinations between the two speaker groups was statistically significant, i.e., heritage speakers in Germany produced significantly more combinations of new-given referents than monolingual speakers.

In the US, research generally reports on the increase of SVO and the reduction of word order flexibility in productions of Russian-English bilinguals (e.g., \cite{isurin_cross-linguistic_2005, kagan_russian_2006, polinsky_incomplete_2006, isurin_lost_2008, laleko_word_2018}). However, \citet{zuban_word_2021} found that other factors, such as clause type, may modulate the reduction of word order flexibility and the increase of SVO. They found that bilingual speakers in the US were similar to Russian monolinguals in main clauses but not in embedded ones, where they predominantly produced SVO orders. \citet{zuban_word_2021} argue that these results are not caused by transfer since transfer effects should emerge in both main and subordinate clauses. Rather, higher complexity of embedded clauses compared to main clauses led heritage speakers to the increased use of SVO word order (see \cite{zuban_word_2021} for a detailed discussion).

\citet{martynova_ovvo_nodate} found that, like bilingual speakers in Germany, bilinguals in the US and monolingual speakers of Russian are similar regarding their preference for OV and VO orders. Furthermore, the two groups of speakers were similar concerning clause type and object realization (noun vs. pronoun). Again, the different results in \citet{martynova_ovvo_nodate} and \citet{zuban_word_2021} are most likely due to the differences in the investigated phenomenon.

Regarding information packaging, some studies in the US report that word orders produced by heritage speakers are sometimes “contextually inappropriate”, i.e., they do not always express the intended information structure or information status (e.g., \cite{laleko_word_2018, kisselev_word_2019}). This terminology of “contextually inappropriate”  is highly problematic as it considers the monolingual formal language to be a norm for heritage speakers although the latter usually do not have exposure to formal instruction in their heritage language (see \cite{wiese_heritage_2022} for a discussion). Several studies suggest differences between bilingual and monolingual speakers and hint at different dynamics regarding information packaging. For instance, “contextually inappropriate” dislocation was found to comprise from around 12\% up to 30\% of all word orders with dislocation in the data of bilinguals while monolingual speakers were reported to always produce “contextually appropriate” dislocation (\cites[203]{laleko_word_2018}[164]{kisselev_word_2019}). Importantly, both bilinguals and monolinguals produced word orders with dislocation. With respect to new-given orders in heritage Russian in the US, we found that heritage speakers of Russian in the US produced referents with the order of constituents new-given (61 instances out of 437 clauses) significantly more frequently than monolingual speakers (see \cite{zuban_unexpected_2023, zuban_word_submitted}). The results were thus in line with the findings from heritage Russian in Germany as reported above.

Taken together, the results from our studies provide valuable insights into the issue of word order choices in heritage Russian. At first glance, bilinguals prefer the same word orders as monolingual speakers (\cite{martynova_ovvo_nodate, zuban_word_2021}). However, once information packaging is added to the equation, the differences between bilinguals and monolinguals become obvious (\cite{zuban_unexpected_2023, zuban_word_submitted}). Concurrently, the results of \citet{zuban_unexpected_2023, zuban_word_submitted} confirm the predictions of the Interface Hypothesis, according to which phenomena that lie at the external interface of syntax and discourse are predicted to show increased variability under language contact (\cite{sorace_epistemological_2011}). The exact reasons behind this variability are less clear. However, the constant exposure of bilingual speakers to the various linguistic structures of their majority language and heritage language as well as different quantity and quality of input in the heritage languages might lead to the outcomes of heritage speakers that differ from monolingual speakers (see \cite{SoraceSerratrice2009, zuban_different_2023}).

This section has summarized studies conducted within RUEG, showing that word order phenomena are modulated through information structure and status. These studies indicate that monolingual and bilingual speakers produce noncanonical patterns for specific information packaging purposes, for example regarding referent introduction in English. Some phenomena even highlighted that the relationship between information structure and word order seems particularly pronounced in bilingual speakers, such as V3 orders and modal particles in German and the order of new and given referents in Russian.

These findings provide valuable insights into the relationship between information packaging and word order in English, German and Russian, discussing isolated syntactic phenomena, with the major unifying factor being the influence of information packaging on word order. As will be shown below, focussing on a pattern occurring across these languages allows for a more systematic analysis of the interaction of information structure and word order under language contact. Taking this perspective, we might be able to pinpoint cross-linguistic effects as well as more general, language external factors that might impact the interaction of information packaging and word order. In the next section, we look at left dislocation (LD) constructions as a case in point for such an endeavor.  


\section{Left dislocation constructions: A comparative view} \label{sec:bunk-left-dislocation}
\subsection{LD in English, German, and Russian }

Left dislocation (henceforth LD) is defined as “a construction in which a constituent (e.g., a noun, a full pronoun, etc.) that appears before/to the left of its predicate has, within the same sentence, a (nonreflexive) coreferential pronoun” \citep[378]{duranti_left-dislocation_1979}.

The construction is reported for most documented languages and is thus considered a universal phenomenon (\cite{lambrecht_dislocation_2001, westbury_left_2016}). This makes LD constructions particularly revealing regarding the effects of language contact on syntax and information packaging, especially when the languages involved differ concerning word order flexibility and exhibit slight grammatical and functional differences in LD constructions (see \citealp{westbury_left_2016} for an extensive overview of grammatical and functional features of LD across different languages). Several studies in the RUEG group investigated LD constructions in Russian (\citealp{zerbian_leftdislocations}), English (\citealp{pashkova_left_nodate}), and German (\citealt{conti_german_2022, sluckin_noncanonical_2023}), both in monolingual speakers and bilingual speakers. (\ref{ex:bunk-LD-ERG}) exemplifies LD constructions in English, Russian, and German: 

\begin{exe}
\ex \label{ex:bunk-LD-ERG}
    \begin{xlist}
    \ex English \\ 			
    {\textbf{My father}, \textbf{he}’s Armenian.} \citep[2]{prince_functions_1997}
    \ex  Russian \\
    \gll {\textbf{Moskv-a},} {\textbf{ona}} {gorodam} {mat’.} \\
    Moscow-\textsc{nom} she cities mother \\
    \glt ‘Moscow is the mother of cities.’ \citep[103]{king_configuring_1995} 
    \ex German \\
\gll {Die Brigitte} {\textbf{die}} {kann} {ich} {schon} {gar} {nicht} {leiden.} \\
{the {} Brigitte} she can I \textsc{mp} \textsc{mp} not like \\
\glt ‘Brigitte, I don’t like her at all.’ \citep[48]{altmann_formen_1981}
    \end{xlist}
\end{exe}

LD constructions are a specific type of topic constructions where the dislocated constituent is topicalized through its initial position and resumption (see, among many others, \cite{altmann_formen_1981}).

LDs (re)introduce or (re)activate discourse referents (see \cite{westbury_left_2016}) or promote topics (see \cite{gregory_topicalization_2001, frey_pragmatic_2005}). On the level of discourse pragmatics, they are used as floor-seeking devices (see \cite{duranti_left-dislocation_1979}). Even though there is no systematic analysis of LD in different communicative situations or registers, several studies have pointed out that LD tends to occur in informal (\cite{geluykens_discourse_1992}) and spoken language (\cite{shaer_integrated_2004, guryev_left-dislocation_nodate}). In spoken discourse, LD is particularly frequent in narrations and discourses with present interlocutors (\cite{bousquette_competition_2021}).

Similar LD types have been categorized using different terminology in different languages, causing confusion from a cross-linguistic perspective. For example, \citet{frey_pragmatic_2005} indicates that the term “left dislocation” is misleading when comparing German with English. While in English, the resumptive occurs in the canonical position of the dislocated constituent in the following syntagma, in German, the resumptive appears right after the dislocated constituent. However, resumptives in German may also appear in other positions in the following syntagma. In this case, the construction is called “Hanging Topic (Left Dislocation) Construction” (HTLD; \cite{altmann_formen_1981, selting_voranstellungen_1993, frey_pragmatic_2005}). \citet{frey_pragmatic_2005} refers to constructions such as in (\ref{ex:bunk-Maria-c-HTLD}) as “Contrastive Left Dislocations” (CLD). (\ref{ex:bunk-Maria}) illustrates the different constructions. 

\begin{exe}
    \ex \label{ex:bunk-Maria}
    \begin{xlist}
        \ex {Maria, I know her.} (English LD) 
        \ex {Maria, die kenne ich.} (German CLD) 
        \ex \label{ex:bunk-Maria-c-HTLD}
        {Maria, ich kenne sie.} (German HTLD) 
    \end{xlist}
\end{exe}
 
Various subtypes of LD constructions differ regarding their grammatical structure. In German CLD, NPs, PPs, CPs, APs, and AdvPs can be dislocated (\cite{dewald_versetzungsstrukturen_2012}). These constituents are resumed by a \textit{d}-pronoun (\textit{der, die, das}) or an adverbial (see \cite{sluckin_noncanonical_2023} on adverbial resumption). HTLD in German is less restricted than CLD concerning the resumptive. It can occur as a \textit{d}-pronoun, personal pronoun, a phrase, or it can be absent (see \cite{altmann_formen_1981, dewald_versetzungsstrukturen_2012}). HTLDs are usually restricted to left dislocated NPs or PPs (\cite{selting_voranstellungen_1993}). Russian and English LD constructions typically involve NPs that are resumed by a subject or object pronoun (\cite{king_configuring_1995, kallestinova_aspects_2007}). LD constructions in the three languages also differ concerning prosodic realization. While the left dislocated element is usually separated from the rest of the clause by a pause in Russian (\cite{bailyn_syntax_2012}), English (\cite{frey_pragmatic_2005}), and German HTLD, German CLDs lack such a pause and are prosodically integrated in the following syntagma (\cite{altmann_formen_1981, selting_voranstellungen_1993, dewald_versetzungsstrukturen_2012}). In our analysis, we subsume all these different constructions under the term “left dislocation” (LD) for English, Russian, and German. In order to account for grammatical and functional differences between the constructions, we annotated LD constructions concerning different grammatical features using a joint annotation scheme (see \sectref{sec:bunk-data-annotation}).

Even though LD is a widespread construction in the languages of the world, only a few studies have systematically investigated LD from the perspective of language contact. Generally, studies indicate that bilingual speakers have “robust knowledge of LD constructions” and “behave like monolingual native controls as regards production, interpretation, and use” \citep[11]{bousquette_competition_2021}. In their investigation of heritage Norwegian and heritage German in the US, \citet[17]{bousquette_competition_2021} find that “specific constraints on LD appear to have been weakened”. Overall, however, bilinguals patterned with monolinguals. Similar results have been found for adverbial resumption in Germany in informal German and 
Kiezdeutsch (\cite{sluckin_noncanonical_2023}). Both varieties display similar patterns, even though Kiezdeutsch generally allows for more word order variation than the German of monolinguals, for example concerning verb placement (see \cite{WieseHeike2016Cinu}), which we already discussed above.

For Canadian French and English, \citet{nagy_second_2003} report that anglophones in Montreal use a distinct type of LD (“subject doubling”) in their French, but the construction is also known from the varieties of Canadian English spoken in Ontario, giving it a “distinctly French flavor” \citep[1]{tagliamonte_grammatical_2019}. Usually, this construction is rare in English but frequent in French. While at first glance, this observation might indicate cross-linguistic effects, the construction is also found in speech communities with less direct influence from French. Thus, \citet[13]{tagliamonte_grammatical_2019} conclude that the construction does not come from French but that “French is influencing the use of English”.

For Russian, \citet{laleko_word_2018} investigate different word order patterns in bilingual and monolingual speakers of Russian in the US. Their study looks at different word order patterns, though not specifically including the LD constructions described here. They found that bilinguals produced fewer noncanonical patterns than monolinguals, indicating decreased word order flexibility in Russian.

In the following, we present first results from our comparative study on LD in language contact and contextualize findings from ongoing or published studies on LD in English, German, and Russian in language contact to gain further insights into LD constructions from a cross-linguistic perspective.  


\subsection{Data annotation} \label{sec:bunk-data-annotation}

The empirical basis is the RUEG corpus as described above. LD constructions were identified in the RUEG corpus manually and annotated using a joint annotation scheme allowing for systematic, cross-linguistic analyses. We thoroughly examined the corpus to identify all instances of LDs according to the definition provided at the beginning of \sectref{sec:bunk-left-dislocation}. Subsequently, we proceeded to annotate each LD based on five features: 1. referent, 2. pronoun type (personal, possessive, partitive), 3. noun phrase type (simple noun phrase, noun phrase with a preposition, noun phrase with coordination, noun phrase with a relative clause), 4. presence of intervening material (present or absent), and 5. function (new introduction, reintroduction, set, clarification).


\subsection{Findings}
\largerpage
Overall, our data show that all speaker groups in Germany and the US use LDs, and the construction is almost always used in spoken contexts. However, the groups differ concerning the number of LDs used by particular groups in the respective countries. Differences also emerged for LD use in formal vs. informal communicative situations. \tabref{bunk:lds-cus-ru} gives a general overview of the number of LDs normalized against the number of CUs.\footnote{ We define a communication unit (CU) as an “independent clause with its modifiers" \citep[53]{hughes_guide_1997}.} 

 \begin{table}
 \label{tab:bunk:frequencies_of_LDs}
   \caption{Overall frequencies of LDs across speaker groups}
  \label{bunk:lds-cus-ru}
 \begin{tabular}{lrcc}
 \lsptoprule 
    &  Left & \\
    &  dislocations  & CUs & LDs per CU \\ 
   \midrule
   RUEG-RU & \\
   mo-Russian (Russia)  & 26   & 2780 & 0.009  \\
   h-Russian (Germany)  & 90   & 3374 & 0.030  \\
   h-Russian (US)       & 103  & 3222 & 0.030 \\
   \midrule
   RUEG-EN & \\
   mo-English (US)  & 24 & 2734 & 0.009   \\
   h-German (US)    & 15 & 1499 & 0.009   \\
   h-Greek (US)     & 19 & 2879 & 0.007  \\
   h-Russian (US)   & 41 & 3170 & 0.013 \\
   h-Turkish (US)   & 30 & 2864 & 0.010 \\
   \midrule
   RUEG-DE & \\
   mo-German (Germany)   & 19   & 3727   & 0.005   \\
   h-German (US)  & 2 & 1633    & 0.001  \\
   h-Turkish (Germany)    & 20 & 3627   & 0.006  \\
   h-Russian (Germany) & 12 & 3639  & 0.003  \\
   h-Greek (Germany)  & 4 & 2251  & 0.002  \\
\lspbottomrule
\end{tabular}
\end{table}

While the overall figures do not indicate major differences at first glance, our statistical analyses revealed interesting results. For Russian, bilinguals in both the US and Germany used a significantly higher number of LDs in total than monolinguals (see \citealp{zerbian_leftdislocations}). This is in contrast to the literature reporting a decreased word order flexibility in bilinguals (e.g., \cite{laleko_word_2018}). In addition to the number of LDs, more bilingual speakers (again in both the US and Germany) than monolinguals produced LDs at all. Formality was not a significant factor for the groups investigated, i.e., there was no difference between formal and informal contexts, pointing to register leveling in these groups (see \cite{ozsoy_shifting_2022}).

In German, monolinguals and bilinguals did not differ significantly from each other. LDs were produced at similar rates. However, formality had a minor influence when comparing monolinguals and bilinguals speaking Turkish as a heritage language. German monolinguals produced fewer LDs in the informal than in the formal context, while bilinguals of heritage Turkish produced fewer in the formal than in the informal setting. These results were only marginally significant; further investigations with a larger data set might thus enlighten this aspect. Interestingly, bilinguals used a structure not yet attested in the literature. They occasionally (9 times) produced an LD structure with a personal pronoun, where a \textit{d}-pronoun would be expected, as discussed in the literature. (\ref{ex:bunk-mann-ball}) provides an example: 

\begin{exe}
   \ex \label{ex:bunk-mann-ball}
   \gll {dieser} {mann} {er} {spielt} {mit} {äh} {ball} \\
   this man he plays with uhm ball \\  
   \glt ‘this man, he plays with the ball' (RUEG corpus, DEbi59MT\_isD)
\end{exe}

Like LD in German, the number of LDs in English was similar between monolinguals and bilinguals~-- slightly above or under 1\% out of all CUs. However, speaker groups differed in using LDs in formal vs. informal situations. We found different patterns of use between bilinguals with the heritage languages Greek and Turkish on the one hand and bilinguals with heritage German, heritage Russian, and monolinguals on the other hand. In the formal situations, speakers of the heritage languages Greek and Turkish used fewer LDs than monolinguals, while bilinguals with heritage German and heritage Russian did not differ from monolinguals. In the informal context, bilinguals with heritage Turkish produced more LDs than monolinguals, while bilingual speakers with heritage Greek, heritage German, and heritage Russian did not differ from monolinguals. Finally, bilinguals with heritage Greek and heritage Turkish approached the formality distinction differently from English monolinguals: these bilinguals had slightly more LDs in the informal than in the formal setting, while monolinguals had a reverse pattern: they had slightly more LDs in the formal setting than in the informal one. Bilinguals with heritage German and heritage Russian did not differ from monolinguals.


\subsection{Discussion}

The study on LD in the three languages leaves us with valuable insights and many questions. For one, cross-linguistic influence might explain some of the results, such as the more frequent use of LD in heritage Russian compared to monolinguals, and the new pattern in German of replacing the resumptive \textit{d}-pronoun with a personal pronoun. Bilingual heritage Russian speakers might use the noncanonical structure more frequently than monolingual English speakers since, in both their languages, LD is a plausible option for topic promotion or (re)introduction. The general flexibility of Russian word order compared to Turkish, Greek, and German might amplify the use of LD in such contexts. Statistical analyses are needed to further explore whether heritage speakers’ majority language aligns with their heritage language regarding LD, in other words, whether in this case speakers use LDs more regularly.

Cross-linguistic transfer, however, does not explain the distributional patterns across different communicative contexts: while formality was not a relevant factor for heritage Russian in Germany, bilinguals with heritage Turkish produced slightly more LDs in informal than in formal contexts, while monolinguals produced more LDs in formal than in informal situations. Interestingly, we found a similar pattern in the US. Again, bilinguals with heritage Turkish (and heritage Greek) produced fewer LDs in formal contexts than monolinguals or bilinguals with heritage German and heritage Russian, who, in turn, produced fewer LDs in informal than in formal situations. This finding is rather surprising, as it concerns particular speaker groups. While for Russian, linguistic factors might be at play, as Russian word order flexibility appears to provide speakers with a wider range of grammar patterns, in heritage Greek and heritage Turkish, potential reasons for the distribution of LDs in formal and informal settings might be extralinguistic factors. In the final section, we suggest how sociolinguistic factors might provide further insights into LD use in language contact and that these factors might influence linguistic structure.  


\section{Conclusion and outlook}\label{sec:bunk-conclusion}

This chapter aimed to provide an overview of studies conducted within RUEG exploring the relationship between information packaging and word order in language contact settings. These studies suggest that both monolinguals and bilinguals make use of noncanonical structures that are influenced by information structure and status. In English, monolinguals and bilinguals used non-SVO to introduce new subjects, while in Russian, both groups used noncanonical new-before-given word orders. In German, noncanonical V3 and peripheral modal particles are used by both bilinguals and monolinguals and in all of these languages, LDs occur in both monolingual and bilingual groups.

Many of our studies also imply that bilingual speakers are flexible regarding word order and its interaction with information structure and status. In Russian, bilinguals use LDs more frequently, contrary to the decrease in flexibility in bilinguals with heritage Russian, as often claimed in the literature. In addition, they apply noncanonical new-before-given word order more frequently than monolinguals. Even though some previous studies did not find any differences in word order between monolinguals and bilinguals (e.g., \cite{martynova_ovvo_nodate}), or these differences were not found across all the investigated conditions (e.g., \cite{zuban_word_2021}), the picture changes when considering how new vs. given referents are presented. In German, V3 is used more frequently by bilingual speakers, and bilinguals place the modal particles \textit{halt} and \textit{eben} more frequently in the sentence peripheries, taking on the functions of discourse particles and highlighting important information.

Our data also indicate that bilingual speakers are not only more dynamic but tend to be more sensitive to communicative situations, as we have seen for the preference of LD in informal situations in most bilingual groups, except for Russian. These effects cannot be explained by cross-linguistic transfer, as we would expect differences in all communicative situations for specific speaker groups. Considering sociolinguistic factors might further illuminate these findings. \citet{bunk_what_nodate} found that many German bilingual speakers with heritage Turkish and heritage Russian produced fewer noncanonical patterns and more features associated with formal, standard-like German on several linguistic levels (lexicon, phonetics, discourse-pragmatics) in the formal situations compared to monolinguals. For example, speakers tend to articulate non-morphemic -\textit{t} in the auxiliary \textit{ist} (‘is') more frequently than their monolingual peers. In spoken language, speakers often drop non-morphemic -\textit{t} due to a regular phonological process. This variant, however, deviates from written standard German and may not be associated with formal language. \citet{bunk_what_nodate, bunk_anxious_nodate} argues that a reason for such patterns may be linguistic pressure that multilingual speakers experience in a societal macro context that is characterized by widespread monoglossic ideologies. Multilingual speakers might feel the need to align with the majority society linguistically to be accepted as members of that society and to avoid Othering. This conclusion is further supported by qualitative sociolinguistic interviews, in which speakers reported the need to excel over monolinguals regarding standard language use in order to be considered valid members of the German society \citet{bunk_what_nodate, bunk_anxious_nodate}. An example for such a notion, uttered by a bilingual speaker with Russian as their heritage language in Germany,  is given in (\ref{ex:bunk-Deutsch-perfekt}):   

\begin{exe}
    \ex \label{ex:bunk-Deutsch-perfekt}
    sagen wir mal, man sieht ausländisch aus, [...] man wird einfach anders wahrgenommen, wenn das Deutsch perfekt ist \\
‘let's say you look foreign, [...] you are simply perceived differently if your German is perfect’ (CS\_hR, 3:56) 
\end{exe}

The interview data illustrate strong monoglossic ideologies, standard language ideology, and perceived linguicism as indicated in (\ref{ex:bunk-Deutsch-perfekt}). These ideologies are present in both the US and Germany, which is typical for countries of the Global North (\cite{Lippi-Green1997, Blackledge_2000}). Hence, they might affect the use of noncanonical patterns, such as LDs, in bilingual speakers in these countries. However, this interpretation of the data does not explain why speaker groups differ from each other. Two possible factors come to mind: a) different experiences of discrimination due to socialization, and b) different ideologies towards multilingualism and the standard variety of the majority language between the speaker groups.

If the societal macro context plays a role, we would not expect fewer LDs in formal contexts than in informal contexts in bilinguals and more LDs in formal contexts than in informal contexts by monolinguals in societies that perceive multilingualism as normalcy, e.g., Namibia. In an ongoing study, \citet{conti_left_nodate} compare the use of LDs in German in Germany, the US, and Namibia, investigating this question further. However, we found other patterns that support this idea in the use of modal particles (see \citeauthor{bunk_sociolinguistic_nodate} \citeyear{bunk_sociolinguistic_nodate} and \sectref{sec:bunk-ref-introduction-eng}), and bare NPs (\cite{bunk_bare_nodate}). Additionally, \citet{wiese_heritage_2022} show that the societal macro context seems to influence the overall structure of languages strongly. Even though this perspective does not explain all of our data, it might explain the fewer occurrences of LDs in formal situations in some bilingual speakers of heritage Turkish and heritage Greek in the US and heritage Turkish in Germany compared to monolinguals. While more in-depth studies are needed to determine language ideologies that are present in the societal macro-context, such as through systematic discourse analysis of public discussions or policy papers, these findings indicate that linguistic ideologies might be an important factor to consider when analyzing linguistic structures in language contact, in particular regarding noncanonical variation.

The case of left dislocation shows that in addition to looking at language-specific phenomena, systematic cross-linguistic analyses of similar patterns are important to illuminate the interaction of information packaging and word order variation. We further suggest that integrating sociolinguistic factors such as the societal macro contexts and ideologies towards multilingualism into our models of grammatical variation might further deepen our understanding of linguistic dynamics in heritage speakers’ language use.


\section*{Acknowledgements}

The research for this paper was funded by the DFG (Research Unit FOR 2537, P6 (394838878) and P8 (313607803)). We thank the editors of this volume, two anonymous external reviewers, and three internal reviewers (Cem Keskin, Luca Szucsich, and Rosemary Tracy) for their valuable feedback. For all remaining errors and shortcomings, the authors, of course, take full responsibility. We would also like to thank all research assistants involved in the projects: Daria Alkhimchenkova, Erica Conti, Amelie Ellerich, Isabell Furkert, Yuliia Ivashchyk, Alexander Lehmann, Birte Pravemann, Sharon Rauschenbach, Guendaline Reul, Myrto Rompaki, Alina Schöpf, Madeleine Spitzer, Chris Allison, Ricarda Bothe, Mert Can, Ryan Carroll, Franziska Cavar, Leah Doroski, Mary Elliott, Hannah Lee, Mark Murphy, Mariia Naumovets, Simge Sargın Kısacık, Jasmine Segarra, Golshan Shakeebaee, Selena Song, Shreya Srivastava, Fiona Wong and Charlott Thomas for their support. We additionally thank Lea Coy for help with pre-publication formatting. Finally, we thank the whole RUEG team and the Mercator fellows for numerous fruitful discussions, feedback, and great support.

{\sloppy\printbibliography[heading=subbibliography,notkeyword=this]}
\cleardoublepage
\end{document}
