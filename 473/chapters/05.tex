\documentclass[output=paper]{langscibook}
\ChapterDOI{10.5281/zenodo.15775165}
\author{Wintai Tsehaye \orcid{0000-0001-7963-1208} \affiliation{University of Mannheim} and  Rosemarie Tracy \orcid{0000-0002-6683-3481}\affiliation{University of Mannheim} and  Johanna Tausch \orcid{0009-0008-1945-3356} \affiliation{University of Mannheim; Leibniz Institute for the German Language}}

\title[Inter- and intra-individual variation]{Inter- and intra-individual variation: How it materializes in Heritage German and why it matters}
\abstract{\sloppy This contribution focuses on the clausal architecture of Heritage German in the United States. In particular, we identify structural properties of German clauses which prove to be robustly canonical even though there is inter- and intra-individual variation on other levels of the linguistic system. With respect to specific clausal features, group comparisons of heritage speakers in both their languages and with monolingually-raised speakers of German have been conducted. We also contribute to current discussions of the heterogeneity of heritage speaker competence and performance and the relevance of individual speaker profiles by qualitative analyses based on data from what we consider \emph{Tiny Language Islands}, i.e., acquisition scenarios with the heritage language only spoken within the immediate family. These findings matter because they speak to hypotheses concerning interfaces between subsystems of our overall linguistic knowledge and, moreover, to discussions on language learnability under conditions of reduced first language exposure. \keywords{variation, heritage language, word order, Tiny Language Islands}}


\IfFileExists{../localcommands.tex}{
  \addbibresource{../localbibliography.bib}
  % add all extra packages you need to load to this file

\usepackage{tabularx,multicol}
\usepackage{url}
\urlstyle{same}

\usepackage{listings}
\lstset{basicstyle=\ttfamily,tabsize=2,breaklines=true}

\usepackage{langsci-basic}
\usepackage{langsci-optional}
\usepackage{langsci-lgr}
\usepackage{langsci-osl}
% \usepackage{./langsci/styles/langsci-lgr}
% \usepackage{./langsci/styles/langsci-osl}
% \usepackage{langsci-gb4e}

\usepackage{tikz}
\usetikzlibrary{patterns,calc}
\pgfdeclarepatternformonly{south east lines}{\pgfqpoint{-0pt}{-0pt}}{\pgfqpoint{3pt}{3pt}}{\pgfqpoint{3pt}{3pt}}{
    \pgfsetlinewidth{0.6pt}
    \pgfpathmoveto{\pgfqpoint{0pt}{3pt}}
    \pgfpathlineto{\pgfqpoint{3pt}{0pt}}
    \pgfpathmoveto{\pgfqpoint{.2pt}{-.2pt}}
    \pgfpathlineto{\pgfqpoint{-.2pt}{.2pt}}
    \pgfpathmoveto{\pgfqpoint{3.2pt}{2.8pt}}
    \pgfpathlineto{\pgfqpoint{2.8pt}{3.2pt}}
    \pgfusepath{stroke}}
    
\usepackage{stmaryrd}
\usepackage{wasysym}
\usepackage{multirow}
\usepackage{caption}
\usepackage{subcaption}
\usepackage{mathrsfs}
\usepackage{qtree}

\usepackage{linguex}


  %pminos do not split footnotes
% \interfootnotelinepenalty=10000 %Footnote in Laporte chapters has to be split SN


%\DeclareIndexNameFormat{default}{%
%\nameparts{#1}%
%\usebibmacro{index:name}%
%{\index[names]}%
%{\namepartfamily}%
%{\namepartgiveni}%
% {}% L1
% {}% L2
%{\namepartprefix}% generates spurious space L3
%{\namepartsuffix}% generates spurious space L4
%}

%  {\DeclareIndexNameFormat{default}{%
%     \usebibmacro{index:name}{\index[names]}{#1}{#3}{#5}{#7}}}

%\DeclareIndexNameFormat{default}{%
%  \usebibmacro{index:name}{\sindex[nom]}{#1}{#3}{#5}{#7}}

%\DeclareIndexNameFormat{default}{%
%  \usebibmacro{index:name}{\sindex[person]}{#1}{#3}{#5}{#7}}
%\DeclareIndexNameFormat{default}{%
%\nameparts{#1} \usebibmacro{index:name}{\sindex[person]]}{\namepartfamily}{‌​\namepartgiven}{\nam‌​epartprefix}{\namepa‌​rtsuffix}}

%\newcommand{\smiley}{:)}

%\renewbibmacro*{index:name}[5]{%
%\usebibmacro{index:entry}{#1}%
%{\iffieldundef{usera}{}{\thefield{usera}\actualoperator}\mkbibindexname{#2}{#3}{#4}{#5}}}

% \newcommand{\noop}[1]{}

%remove for final
%\overfullrule=1mm

\newcommand{\tobi}[2]}}
\renewcommand{\S}[1]{\tobi{#1}{\textsc{*}}}

% this volume references
% puts: [this volume]
% already defined: \citetv
%\newcommand{\citepv}[1]{(\citeauthor{#1} \citeyear*{#1} [this volume])}
\newcommand{\citealtv}[1]{\citeauthor{#1} \citeyear*{#1} [this volume]}

%parentheses around example number
\newcommand{\pref}[1]{(\ref{#1})}

% in-text examples

\newcommand{\lnex}[1]{\textit{#1}} %target lang word
\newcommand{\lnlit}[1]{(lit.: `#1')} %literal reading
\newcommand{\lnlat}[1]{(#1)} % latinization
\newcommand{\lntrans}[1]{`#1'} %translation
\newcommand{\lnexl}[2]%
{\lnex{#1}{} \lnlat{#2}} % ex with latinization
\newcommand{\lnexlat}[3]{\lnex{#1}{} \lnlat{#2}{} \lntrans{#3}} % ex with latinization and tranl.

%ch01
\newcommand{\co}[1]{\mbox{\textbf{#1}}}

%ch09

\newcommand{\cyrbulg}[1]{\begin{otherlanguage*}{bulgarian}#1\end{otherlanguage*}}


%ch10
\newcommand{\nlp}{{\small NLP}}
\newcommand{\mwe}{{\small MWE}}
\newcommand{\rae}{{\small RAE}}
\newcommand{\lvc}{{\small LVC}}
\newcommand{\pos}{{\small P}o{\small S}}
%\newcommand{\todo}[1]{ \textcolor{red}{#1} }

%\renewcommand{\labelenumi}{\theenumi}
%\ainamefmt{{vv}{ll}{, ff}{, jj}} % fullname

\newcommand{\biberror}[1]{{\color{red}#1}}

\newcommand{\osenovaitem}{--~} 
  %% hyphenation points for line breaks
%% Normally, automatic hyphenation in LaTeX is very good
%% If a word is mis-hyphenated, add it to this file
%%
%% add information to TeX file before \begin{document} with:
%% %% hyphenation points for line breaks
%% Normally, automatic hyphenation in LaTeX is very good
%% If a word is mis-hyphenated, add it to this file
%%
%% add information to TeX file before \begin{document} with:
%% %% hyphenation points for line breaks
%% Normally, automatic hyphenation in LaTeX is very good
%% If a word is mis-hyphenated, add it to this file
%%
%% add information to TeX file before \begin{document} with:
%% \include{localhyphenation}
\hyphenation{
    Beck-man
    Ngu-yen
    back-chan-nel
    back-chan-nels
    mo-not-o-nous
    ste-reo-typ-i-cal
}

\hyphenation{
    Beck-man
    Ngu-yen
    back-chan-nel
    back-chan-nels
    mo-not-o-nous
    ste-reo-typ-i-cal
}

\hyphenation{
    Beck-man
    Ngu-yen
    back-chan-nel
    back-chan-nels
    mo-not-o-nous
    ste-reo-typ-i-cal
}
 
  \togglepaper[5]%%chapternumber
}{}

\begin{document}
\maketitle

\section{Introduction}

The question of what happens to grammars of German under intensive language contact has been addressed many times. Since German-speaking communities form diasporic enclaves in many parts of the world, there exists a rich natural laboratory for investigating the effects of internal and external variables impacting on language maintenance, attrition, cross-linguistic interaction, and change. In some cases, as in Namibia, German remains a vital minority language, enjoys considerable social prestige, and is supported within the educational system (cf. \citealt{ShahZappen-Thomson2017, WieseEtAl2017, WieseEtAl2022, Zimmer2019}). In other parts of the world, and given other status constellations, perspectives for the future of German look rather grim (with respect to German varieties in the US, see \citet{Boas2009b, Boas2009a} for the decline of Texas German; \citet{HoppPutnam2015} for a \textit{moribund} variety, Moundridge Schweitzer German in Kansas). At the same time, upward trends have been noted for Pennsylvania German (PG) communities, with PG gaining prestige among young generations who did not acquire it as their first language (L1) but who positively identify with their German legacy and feel motivated to learn the ancestral language as adolescents or adults (cf. \citealt{Stolberg2014}).

In addition to the wealth of information already available on heritage German in the context of majority English, we aim at providing further insight by investigating both spoken and written productions of additional heritage speaker populations in systematically varied elicitation contexts.

Our contribution targets a very specific subset of heritage German speakers: the adolescent offspring of first generation German immigrants. They were either born and raised in the US, with German as their L1 and English as a simultaneous first language (2L1), or, in the case of immigration by age two, as an early second language (L2). We therefore contribute to filling a data and description gap between research on first generation German immigrants in English-speaking countries on the one hand (e.g., \citealt{LatteyTracy2001, Schmid2011, Tracy2022, TracyStolberg2008, Keller2014}), and research on minority German language islands around the world, on the other (e.g., \citealt{Andersen2016,Boas2009b, Boas2009a,Boas2010,Clyne2003,EichingerEtAl2008,Földes2016, Fuller2001, HoppPutnam2015, Huffines1980, JohannessenSalmons2015, Louden2008, PlewniaRiehl2018, PutnamSalmons2013, Rosenberg2005,Stolberg2014,Stolberg2015}, among many others).

As part of the Research Unit \textit{Emerging Grammars in Language Contact Situations} (RUEG) and its shared methodology, all heritage speakers (HSs) took part in the same elicitation tasks (see \citetv{chapters/02}), both in speech and in writing, in formal and informal situations, and in both their heritage language (HL) and majority language (ML). This allowed us to tap into their linguistic repertoires, including register-specific resources and preferences for specific structural options according to formality and mode of communicative situations. Repeated elicitations of factually similar reports helped distinguish performance errors from systematically recurring, i.e., more stable phenomena. In various instances, comparing what speakers produced in their minority and in their majority language helped clarify how they interpreted a specific scene and which formulations in German likely resulted from language contact with English. Moreover, additional data in form of informal chitchats, written sentence corrections, and oral and written sentence completions provided insight into participants’ performance in the face of different task demands.

On the basis of biographical information available, we could take language exposure and maintenance conditions into consideration. As opposed to historical language islands, where the minority language may still be used outside the home, a good number of our participants grew up in what we consider a \textit{Tiny Language Island} scenario, with the HL only spoken within the realm of the immediate family, in some cases with only one family member. Potential consequences for HL productions will be addressed in our discussion of HS heterogeneity (\sectref{sec:tsehaye:6}). Since our HS group includes nine sets of siblings, we could take advantage of this more optimally controlled condition due to individuals sharing various linguistic and extra-linguistic background variables.

The main focus of this contribution lies on clausal syntax, specifically on the distribution of finite and non-finite verbs and verb components, on the morphology\babelhyphen{hard}syntax interface and on phonological exponents of morphological subsystems. In \sectref{sec:tsehaye:2}, we illustrate critical properties of German clause structure and point out relevant contrasts between German and English. \sectref{sec:tsehaye:3} turns to findings for historical German language island communities in majority English contexts and formulates research questions. \sectref{sec:tsehaye:4} provides background on our research methods, consultants, and the corpus, followed by results in \sectref{sec:tsehaye:5}. \sectref{sec:tsehaye:6}, with its focus on Tiny Language Islands, illustrates what individual learner profiles can contribute to our understanding of HS heterogeneity. Our conclusion in \sectref{sec:tsehaye:7} summarizes findings and relates the spectrum of variation identified in our data to language acquisition in childhood and general theoretical issues.

\section{Properties of German clause structure}\label{sec:tsehaye:2}

For the clausal architecture at issue in this chapter we draw on a linear model based on a \textit{topological fields} metaphor (\citealt{Drach1963, Höhle1986, Wöllstein2014}). \figref{fig:tsehaye:1} captures the distributional properties at issue and introduces the terminology employed henceforth, including the left-to-right orientation based on written clauses rather than talking about what is perceived first or last. 


\begin{figure}
\includegraphics[width=\textwidth]{figures/Ch5_Figure_1.jpeg}
\caption{Topological schema of German clauses}
\label{fig:tsehaye:1}
\end{figure}

In canonical{\interfootnotelinepenalty=10000\footnote{By canonical we here mean adherence to codified standard language norms (cf. \citetv{chapters/01}).}} main clauses, the finite verb appears in the left sentence bracket (LSB), resulting in verb second (V2) or verb first (V1) structures, while non-finite parts of the verb phrase (VP), such as infinitival verbs, participles, separable verbal particles, are verb final (VF), i.e., they occur in the right sentence bracket (RSB), as in \tabref{tab:tsehaye:new:1}a. \textit{X} and \textit{Y} in \figref{fig:tsehaye:1} indicate the place of pre- and post-clausal adjuncts. In subordinate clauses, the LSB is occupied by complementizers and relative pronouns.\footnote{We here forgo the discussion of whether relative constituents are placed in the forefield in subordinate clauses (cf. \citealt[214]{Imo2016}, \citealt[27ff]{Wöllstein2014}).} In this case, finite verbs have to join non-finite verbs in the RSB as in \tabref{tab:tsehaye:new:1}b. In main clauses the forefield is taken up by preposed, topicalized constituents, including subordinate clauses, as in \tabref{tab:tsehaye:new:1}c. In this case, subjects remain in the middlefield. In yes/no questions (\textit{Bist du hungrig?} ‘Are you hungry?’) and irrealis constructions (\textit{Hättest du das nur früher gesagt!} ‘If you had only said this sooner!’), the preverbal field remains empty, hence the finite verb occurs sentence-initially (V1).

\begin{sloppypar}
The postfield holds constituents (clauses and adjuncts) extraposed from the middlefield, as in \tabref{tab:tsehaye:new:1}b--d, or freely adjoined constituents. As a consequence, constituents belonging syntactically to a preceding clause and clarifying afterthoughts are not always easily distinguishable as such, especially when prosodic cues are missing (see various papers in \citealt{Vinckel-Roisin2015}). The postfield provides the canonical area for subordinate clauses and otherwise \textit{heavy} constituents. Shifting them out of the middlefield reduces the distance between discontinuous parts of verbs, thereby easing memory load (\citealt{Haider2010, Imo2016, Proske2015}). The example in \tabref{tab:tsehaye:new:1}d illustrates a \textit{heavy} middlefield since the adverbial clause has not been extraposed. At the same time, postposing “lightweights” (focus particles and adverbials, i.e., \textit{auch} ‘also’, \textit{immer} ‘always’, \textit{jetzt} ‘now’, e.g., \textit{Sie fährt nach Hause jetzt.} ‘She drives home now.’), is typically considered stylistically infelicitous or highly marked, at best, in written contexts. In informal conversational contexts, however, the right clausal edge is used extensively.
\end{sloppypar}

\begin{table}
\small
\begin{tabularx}{\textwidth}{lQlQlQ}
\lsptoprule
    & {Forefield} & {LSB} & {Middlefield} & {RSB} & {Postfield}\\\midrule
(a) & Ich & lief & die Straße & entlang. & \\
    & I & walked & the street & along & \\
    & \multicolumn{5}{l}{‘I was walking along the street.’}\\
\midrule
\end{tabularx}\\
\begin{tabularx}{\textwidth}{lQlQlQ}
(b) & Ich & lief & die Straße & entlang & als der Unfall passierte.\\
    & I & walked & the street & along & when the accident happened\\
    & \multicolumn{5}{l}{‘I was walking along the street when the accident happened.’}\\
\midrule
\end{tabularx}\\
\begin{tabularx}{\textwidth}{lQlQlQ}
(c) & Als der Unfall passierte & lief & ich die Straße & entlang. & \\
    & When the accident happened & walked & I the street & along. & \\
    & \multicolumn{5}{l}{‘When the accident happened I was walking down the street.’}\\
\midrule
\end{tabularx}\\
\begin{tabularx}{\textwidth}{lQlQlQ}
(d) & Ich & lief & als der Unfall passierte & die Straße entlang. & \\
	& I & walked & when the accident happened & the street along. & \\
    & \multicolumn{5}{l}{‘I was walking down the street when the accident happened.’}\\
\lspbottomrule
\end{tabularx}
\caption{German clause structure illustrated}
\label{tab:tsehaye:new:1}
\end{table}

Over the last two decades, numerous publications have shown the need for more fine-grained analyses of both clausal peripheries (for critical discussions, cf. \citealt{AntomoSteinbach2010,  Freywald2016, FreywaldEtAl2023, MolnarWinkler2010, Speyer2009, Vinckel-Roisin2015, Winkler2017}, see also \citetv{chapters/11}). Since we find similar ordering at the clausal peripheries in our HL data, taking into account what happens in non-heritage varieties of German is crucial. Including monolingually-raised speakers (MSs) in our investigations enables us to identify trends in homeland varieties independently of language contact and guards us against hastily attributing one or the other HL pattern to transfer from English or to some other “flaw” in HL competence or performance.

At the same time, language contact phenomena\footnote{For comprehensive studies on German-English language contact, see among others (\citealt{Clyne2003, Johanson2002, Johanson2008, LatteyTracy2001, TracyLattey2010, Stolberg2014, Keller2014}).} are fully expected. After all, German and English share typological features and a considerable number of cognates, the latter being congenial to the emergence of converging forms. Yet, there are also obvious differences between the two languages. Hence, both the extent of structural ‘overlap’ due to parallel clausal patterns in main clauses with simplex verbs (\textit{Bist du hungrig?} ‘Are you hungry?’, \textit{Er schreibt alle Briefe mit der Hand.} ‘He writes all letters by hand.’) and the specific contrasts selectively mentioned below make the two an intriguing but also quite challenging language pair to consider in language-contact scenarios.

The German VP is head-last, while English is head-first across all phrasal constituents. Whereas German allows a full range of V2 effects in main clauses, with almost any constituent able to access the forefield, English is an SVO language, restricting the occurrence of subjects to positions preceding the finite verb. Topicalized constituents then appear in front of subjects (\textit{Ice-cream I only eat in summer.} \textit{Only in summer I eat ice-cream}). Nevertheless, V2 effects productive in earlier stages of English survive in subject-auxiliary inversion in questions (\textit{Could/Did the man find his key? What have you told him?}) and are tied to special triggers, such as negation (\textit{Under no circumstances/Never again will I eat ice-cream, summer or winter.}). English also residually allows subject-main verb inversion with intransitive verbs and initial locative or directional adverbials (\textit{All of a sudden, across the street jumped a dog.}).

Topicalization and inversion aside, English word order is canonically SVO in both main and subordinate clauses. German, on the other hand, as pointed out above, requires V2 in finite main clauses but asymmetrically places finite verbs in final position (VF) in subordinate clauses. Neither English nor German license pro-drop, i.e., empty subjects in finite clauses, most easily recognized in finite subordinate clauses (*\textit{Ich würde behaupten, \_\_ kommt heute nicht}. ‘*I’d claim \_\_ won’t come today’). In specific contexts, English allows truncating main-clause preverbal subjects (\textit{diary drop}), as in  \_\_ \textit{Depends on the weather} or \_\_ \textit{Arrived late, \_\_missed my train}. In German, on the other hand, all sorts of topicalized constituents, not just subjects, can be dropped from the forefield, given appropriate discourse licensing contexts, such as topics already under discussion. The result is a V1 pattern as in \REF{ex:tsehaye:Buch}, with ${\varnothing}$ indicating the position of an inferrable constituent, in this case a topicalized direct object.

\begin{exe}%4
\ex[]{ \label{ex:tsehaye:Buch}
\gll    Willst du mein Buch lesen?\\
        want you my book read \\
\glt    ‘Do you want to read my book?\\}
\sn[]{
\gll    ${\varnothing}$ Kenn ich schon.\\
	    ${\varnothing}$ know I already\\
\glt    `I already know it.'}
\end{exe}

In addition to distributional differences, morphological contrasts abound. German possesses rich inflectional paradigms spelling out person, number, and tense in the verb. English, however, – the suppletive copula paradigm apart – only marks person and number in main verbs in present tense third-person contexts. Case surfaces in articles, pronouns, and adjectives within the determiner phrase in German, confounded by syncretism involving grammatical or natural gender and number, and complicated by weak/strong inflections depending on choice of determiner (definite vs. indefinite). As a general rule, only masculine and neuter are marked morphologically for case in the genitive singular and the dative plural. Dative marking in singular masculine nouns, can be considered outdated and appears in highly literary genres (“Dem Mann\textbf{e} kann geholfen werden.”, Schiller, Die Räuber, act 5, scence 2). In addition to configurationally determined case marking, some German verbs call for idiosyncratic cases, as the genitive in \textit{jemanden eine}\textbf{\textit{s} }\textit{Verbrechen}\textbf{\textit{s} }\textit{beschuldigen}, ‘to accuse someone of a crime’, where case is also marked on the noun.

In comparison, Modern English inflectional morphology is greatly reduced across word classes. Natural gender as well as nominative vs. non-nominative case distinctions are only visible in singular pronouns (e.g., he/him/his; she/her), with their plural unmarked for gender (they/them/their). Articles as well as adjectives do not require agreement inflections, while nouns are only marked for plural and the genitive. 

On the basis of the selected typological differences, the question arises, then, which features of German clausal architecture on the one hand and of morphological subsystems on the other can be identified in HS productions and whether findings pointing towards noncanonicity are due to influence from English or to developments independent of language contact (e.g., language-internal dynamics, register sensitivity, task type). We hypothesize that our second generation immigrants will be able to discover most of these properties, even under reduced input conditions, different from what we see in established language islands to which we briefly turn next.

\section{German from a language contact and \textit{Heritage Language Island} perspective} \label{sec:tsehaye:3}

Given the cross-linguistic similarities and differences mentioned, quite a number of contact phenomena can be expected in bilingual German-English speech and writing (\citealt{LatteyTracy2001, Schmid2011, Tracy2022}, for first generation immigrants; \citealt{VainikkaYoung-Scholten2011} for L2 acquisition of German by L1 English speakers). With respect to word order, one could, for instance, reasonably expect an increase in postposed constituents due to the stability of English [V\textsubscript{+/− finite} XP (YP)] across clauses. Intensive contact with English might even facilitate the highly noncanonical extraposition of direct objects into the postfield, as demonstrated by \citet{Clyne2003} for heritage German speakers in Australia, see \REF{ex:tsehaye:Wörter}.

\ea {\citep[137]{Clyne2003}\\\label{ex:tsehaye:Wörter} 
 \gll Mummy  hat    gesagt  die  Wörter  für   mich.\\
Mummy  has      said      the   words   for   me.\\
\glt ‘Mummy told me what to say.’}
\z

At the same time, Clyne’s data also underscore the robustness of V2 effects in main clauses and of VF in subordinate clauses, even though there may be complementizer-specific variation. Similarly, both \citet{Louden2008}, for Pennsylvania German, and \citet{Boas2009b, Boas2009a}, for German-speaking communities in Texas, found some complementizers in canonical VF clauses, others exclusively with V2 (hence noncanonical), while a third group of complementizers appeared in both positions. Likewise, \citet[29]{HoppPutnam2015} report that their consultants consistently produced \textit{dass}+V2-clauses in elicited production tasks (e.g., \textit{dass sie hot ei(n) Car gekauft gestern}\footnote{This example also illustrates the extraposition of a light-weight constituent, the adverb ‘yesterday’. While a common pattern in spoken German, in English-German bilinguals an increase of such structures enhanced by surface parallelism can be expected.}, ‘that she has bought a car yesterday’). Speakers also rated these patterns more acceptable than \textit{dass}+VF.\footnote{\citet{Stolberg2015}, who also found noncanonical complement clauses with \textit{dass} in her written PG data, points out that this clausal pattern could also be due to verb-projection raising, which is a canonical option in several modern and old Germanic languages and dialects, including standard patterns with infinitives (\textit{dass/weil er hat das nicht wissen können}, ‘that/because he couldn’t know that’).} On the other hand, the very same persons preferred VF in clauses introduced by \textit{wenn} ‘when’ and \textit{weil} ‘because’, both in their spontaneous productions and when they were asked for well-formedness judgments. Importantly, several researchers stressed that some noncanonical patterns were specific to individual speakers (cf. \citealt{Boas2009b, Boas2009a, Clyne2003, HoppPutnam2015}). These findings underscore that it is crucial not to lose sight of speaker idiosyncrasy.

\begin{sloppypar}
In German-speaking diaspora communities, specific consequences of contact with English surface in the choice of near-homophonous\footnote{See also \citet[292]{Johanson2002} for the selective copying of “relatively homophonous false friends”.} complementizers which partially overlap in meaning, such as German \textit{wenn} (conditional), corresponding to ‘if’, but also to ‘always when’, and English ‘when’ (temporal), corresponding to German \textit{als}, see \REF{ex:tsehaye:geheiratet} from one of the elderly participants in \citet{HoppPutnam2015}. At the same time, clausal syntax is canonical, with the finite verb following the participle in the RSB.
\end{sloppypar}

\ea%6
    \label{ex:tsehaye:geheiratet}(German in Kansas, \citealt[196]{HoppPutnam2015})\\
    \gll wenn mir  erscht  geheirat   henn\\
	when  we    first     marrried  have \\
    \glt ‘when we first got married’ 
\z

Evidence for the very same crosslinguistic effect involving \textit{when/wenn} can be found in first generation immigrants, both in conversations \REF{ex:tsehaye:Feld} and in writing \REF{ex:tsehaye:Glitzern}, (\citealt[421]{Tracy2022}, spelling of the handwritten original preserved).

\ea \label{ex:tsehaye:Feld}
\gllll {Die}   {mussten} {immer} {im} {im} {Feld} {helfen} {im} {Sommer}\\
       They   had-to always in-the in-the field help in-the summer\\
       \textbf{{wenn}} sie auch noch in der Schule war‘n ne.\\
       {when} they also still in the school were \textsc{particle.}\\
\glt ‘They always had to help in the fields in summer when they were still going to school, didn’t they?’
\ex\label{ex:tsehaye:Glitzern}
\gll {ein} {Glitzern} {eine} {Pracht} \textbf{{when}} {man} {von} {innen} {rausschauen} {kann}\\
     a glittering a splendor {when} one from inside outlook can\\
\glt ‘a glittering, such a splendid view if you can look out at it from inside (the house)’
\z

While choice of \textit{when} is semantically appropriate in \REF{ex:tsehaye:Glitzern}, there is interference on the orthographic level. Both the evidence from German islands with a long history of language maintenance and from first generation immigrants show how partial congruence of features from various levels (semantics, syntax, phonology, orthography) create “grey zones” \citep[755]{Clyne1987}, favoring convergence. It comes as no surprise, then, that the heritage data under discussion here from second generation immigrants (RUEG corpus\footnote{In the RUEG corpus, transcriptions of the spoken data are annotated for pauses “(-)”, prolongations “:”, hesitations \textit{äh/ähm} and other production\hyp related phenomena, such as tongue\hyp clicks. These were not included in our examples since we do not discuss them here. See \textcitetv{chapters/10} for spoken discourse phenomena, also \citealt{BöttcherZellers2023, TracyGibbon2023}).}, \citealt{RUEGcorpus2024}) points in the same direction \REF{ex:tsehaye:gehaltet}.

\ea%7
    \label{ex:tsehaye:gehaltet}
    \gll {so} \textit{{wenn}} {sie} \textbf{{hat}} {gehaltet} {sie} \textbf{{hat}} {die} {das} {erste} {Auto} {geschlagt}\\
         so {when} she {has} stopped she {has} the the first car hit \\
    \glt ‘so when she had stopped she hit the first car’ (Language Situations (LangSit) narration formal-spoken)
\z

Besides the occurrence of \textit{wenn} instead of \textit{als}, two more features in \REF{ex:tsehaye:gehaltet} are noteworthy, both involving the position of the auxiliary \textit{hat} ‘has’. Canonically, the linearly first \textit{hat} would occur in the right sentence bracket (VF). Since the follow-up clause is the actual matrix clause, the \textit{wenn}\hyp clause is expected in its forefield, with the subject – the linearly second token of \textit{sie} ‘she’ – in the middlefield. A partially parsed canonical version is shown in \REF{ex:tsehaye:geschlagen}, where we maintain the calque of ‘hit’, \textit{schlagen} instead of the expected \textit{getroffen} ‘driven into’, and ignore gender repair in the article (\textit{die das}).

\ea%8
    \label{ex:tsehaye:geschlagen}\relax
    [{\textsubscript{S}} [{\textsubscript{S}} {als sie gehalten} \textbf{{hat}}] \textbf{{hat}} {sie das erste Auto geschlagen}]
\z

In our discussion of heterogeneity (\sectref{sec:tsehaye:6}) we will return to the question of how prominent clausal patterns like \REF{ex:tsehaye:gehaltet} are within individual HSs.

Against this backcloth of research on German in close contact with US English, our contribution pursues the following questions to see whether both HSs and MSs, if faced with the same communicative challenges and task demands, perform differently in terms of register availability and structural choice. 

\begin{description}[font=\normalfont]
\item[RQ1:] Are there differences in the clausal syntax in MSs and HSs, and, if so, could this be due to influence from English?

\item[RQ2:] In contrast to early acquired core syntactic phenomena, how does (non)ca\-non\-i\-cal variation manifest itself in the domain of morpho-syntax?

\item[RQ3:] What can we conclude about heterogeneity and individual variation by taking a closer look at Tiny Language Islands where HL input is limited to the family (parents and siblings), and maybe to one parent only?
\end{description}

While the first question will be pursued on the basis of published studies (\citealt{PashkovaEtAl2022, Tsehaye2023, TsehayeEtAl2021}), questions two and three will be tackled on the basis of novel data in order to clarify what heterogeneity means along a scale ranging from the predominantly canonical to increases in variation to the emergence of novel grammars.

\section{Method}\label{sec:tsehaye:4}

Our adolescent HL participants ($N=29$, age 13--18)\footnote{These are the numbers and ages for all the heritage German participants. Throughout this chapter, we will provide the numbers for speakers included in the various analyses undertaken.}, all second generation speakers of German, were recruited in the US (Boston, Madison, St. Paul, Minnesota). They grew up with at least one German-speaking parent in the household. Only a few had temporary access to classes in German, attended Saturday Schools, or were engaged in leisure-time activities involving other HSs of German. Crucially, though, they did not grow up in a German-speaking community. Participant metadata was available through a questionnaire on language background, language use, and personality that our consultants filled in and in which they rated their own proficiency\footnote{The two visits to our temporary “labs” on part of our participants were time-consuming and required willingness to engage in different tasks, some of which, especially where they required writing, were not without stress. While many HSs enjoyed taking part (as evident from our informal conversations), they also felt, that they were being tested, causing some of them to do their best and produce highly elaborate clauses. The link for the adolescent participant questionnaires can be accessed via: \url{https://osf.io/qhupg/}} with respect to speaking, listening comprehension, reading, and writing. Parallel age groups of MSs of German and MSs of English were recruited in Germany and in the US, respectively. 

In addition to the RUEG-wide elicited narratives concerning a fictive accident (cf. \citetv{chapters/02}), we ran an on-site oral sentence completion task as well as a follow-up computer-based online sentence completion and sentence correction task. Both online tasks were conducted with a subset of speakers only. On-site meetings with participants also included a 15-minute informal, relaxed conversation, referred to here as \textit{chitchat}.\footnote{The data produced in the LangSit narrations are openly accessible via: \url{https://zenodo.org/record/5808870}. The data of the sentence completion and sentence correction tasks can be accessed via: \url{https://osf.io/28j57/}. The data of the chitchat is not openly accessible. We included it in our analyses as an additional cue to the options available to participants.}

Quite apart from narrowly linguistic, system-internal properties, the tasks designed differed along a scale of complexity. While there is an ongoing discussion on what counts as complex (\citealt{HousenKuikenVedder2012}), what matters in our case is \textit{task complexification} via constraints placed on choices left to the participant. Positively framed: We provided opportunities for speakers to display their linguistic resources, including choices associated with different formality settings (formal, informal) and modes (speech, writing). In our sentence completion task, we offered cues nudging our participants in the direction of specific clausal patterns. The following list provides a summary of tasks according to what we consider an increase in complexification.

\begin{itemize}
\item[A.]
Chitchat: The informal chitchat consisted of a 15-minute warm-up and familiarization conversation, relatively unconstrained in terms of topics (interests, favorite foods, neighborhood, etc.) and completely unconstrained with respect to syntactic choices. The only constraints were those imposed by general discourse-pragmatic principles governing conversational alignment (\citealt{PickeringGarrod2004}), first\hyp encounter topics, politeness, and relevance.
\item[B.] LangSit: Elicited narrations were based on a video shown – without intervening turns by elicitors – in different communicative situations (see \citetv{chapters/01}). Syntactic preferences were up to the speaker, while narration contents were restricted by the events observed. Specific contents were primed by task demands specified in the instructions of who to report to (police vs. friend). More detail in terms of cause and effect and protagonist identification was called for in spoken and written police reports.\footnote{If speakers had reported the same detail (for instance about the way people were dressed, etc.) in informal productions, it would have been at odds with conversational maxims.} Participation in the narration task also differed from the chitchat encounters in that it required willingness to play-act.
\item[C.] Written sentence completion based on the video: Sentence completions confronted the participants with structures provided in the stimuli that required specific continuation patterns and left them no choice, at least not with respect to canonical verb placement. During the written online completion task, there was no time pressure since participants had been sent a link to the stimuli and could take as much time as they wished.
\item[D.] Oral sentence completion of sentences read out to them: In this task, an immediate response in form of clausal continuation was called for, so speakers had very little time for planning of how to pick up the baton.
\item[E.]\sloppy Written sentence correction: Sentence corrections gathered in an additional online study requested reactions to stimuli including noncanonical verb placement, auxiliary selection, as well as tense, case, and gender inflection.~Again, no time constraint was imposed. 
\end{itemize}

These five task types differ in terms of relative complexity and the high vs. low production constraints placed on participants’ choices. The informal chitchat was not analyzed quantitatively, but, as we show below, qualitative properties provide insight into trends identified in the standardized elicitation of the narratives.\footnote{As our main goal was to elicit either German or English data, interlocutors remained in a monolingual mode and did not engage in codeswitching. With respect to German productions, this might not have corresponded to family language practice and was certainly costly in terms of the monitoring required in order to stick to German. For spontaneous borrowing, calquing, and convergence, see \textcitetv{chapters/10}.}

On the whole, speaker performance across different tasks with varying demands allowed us to arrive at a comprehensive picture of the spectrum of intra-individual variation as well as of both stable and newly emerging options.

\section{Results}\label{sec:tsehaye:5}

In the following section, results from quantitative and qualitative studies relating to the issues sketched above are reported. After laying out quantitative findings regarding the clausal architecture across HSs and MSs, we briefly turn to a very different picture emerging from investigations of morphological phenomena, especially with respect to exponents of case and gender. Contrary to what we see in the syntactic domain, there is considerable inter- and intra-individual variation in the realization of various morphological paradigms and with respect to auxiliary choice.

\subsection{ Clause-type optionality}
\label{sec:tsehaye:5.1}
In a study on clause type optionality, operationalized as choice between different clause structures in narrations of the same event, the production of three clause types (i--iii) was investigated in 20 adolescent HSs, 20 adolescent MSs of German, and 20 adolescent MSs of English \citep{PashkovaEtAl2022}. Thus, this study inquired into the syntactic (verb placement in main and subordinate clauses) and pragmatic (discourse structure according to register) choices available to MSs and HSs of German in both their HL, German, and their ML, English. In the following, we only focus on the German productions.

\begin{itemize}
\item[(i)] independent main clauses (IMCs):\\
		   Ich lief die Straße entlang. Ich sah einen Unfall.
		  \glt ‘I walked down the street. I saw an accident.’

\item[(ii)] coordinate main clauses (CMCs) with and without subject gapping:\\
		    Ich lief die Straße entlang und sah einen Unfall.
		    \glt ‘I walked down the street and saw an accident.’

\item[(iii)] finite subordinate clauses (SCs):\\
			Während ich die Straße entlanglief, sah ich einen Unfall.
			\glt  ‘While I was walking down the street, I saw an accident.’
\end{itemize}

While all participants produced clauses of the types (i--iii), sometimes more or less canonically, we were concerned with preferred choices and avoidance. Our results show that HSs and MSs of German behave similarly in their production of IMCs across communicative situations, with more IMCs in writing than in speaking, but that they differ in the frequency and distributional patterns of CMCs and SCs. HSs produced more CMCs than MSs, but both participant groups were sensitive to changes in communicative situations, with more CMCs in informal situations and in spoken productions. Additionally, the differences in CMC frequency between speaker groups was greater in formal communicative situations, regardless of the production mode. HSs produced significantly fewer SCs than MSs. However, both speaker groups were sensitive to formality with more SCs in formal settings. A closer look at SC productions across communicative situations showed that MSs distinguished (in)formality in both production modes while HSs only differentiated (in)formality in writing. Even though both speaker groups differed with respect to frequency and distribution of SCs across formality and mode, our results indicate that HSs had mastered the overall clausal architecture of German main and subordinate clauses, and that differences from MSs can be attributed to factors impinging on performance. This shows that HSs are, in fact, sensitive to register but an increase in cognitive load due to typological differences between English and German might make adherence to register sensitivity harder in spoken (online) productions than in written (offline) productions.

\subsection{Finite subordinate clause distribution}\label{sec:tsehaye:5.2}

As a follow-up to the previous study, \citet{TsehayeEtAl2021} investigated the distribution of finite subordinate clause types (complement clauses, CompCs; adverbial clauses, AdvCs; and relative clauses, RelCs) in 27 adolescent HSs, 32 adolescent MSs of German, and 32 adolescent MSs of English. Again, the reportings in this chapter focus only on the German productions. Quantitative analyses showed that both speaker groups behaved similarly regarding the frequencies of SC types across formality settings (more CompCs in informal situations and more RelCs and AdvCs in formal situations, \figref{fig:tsehaye:2}). This is interesting, as differences between speaker groups were expected especially in AdvCs and RelCs due to their relatively late acquisition (\citealt{ParadisEtAl2017, VasilyevaWaterfallHuttenlocher2008, AndreouTorregrossaBongartz2020}). With this study, we showed that, while we find differences in overall SC productions across speaker groups, the distribution of SC types is similar. 


\begin{figure}
\includegraphics[width=\textwidth]{figures/Ch5_Figure_2.jpeg}
\caption{Mean proportions of CompCs, AdvCs, and RelCs across formality levels and speaker group \citep[10]{TsehayeEtAl2021}}
\label{fig:tsehaye:2}
\end{figure}


\subsection{Clausal peripheries}
\label{sec:tsehaye:5.3}
While the first two studies addressed interface phenomena across clausal boundaries, we now turn to an interface phenomenon within clauses. Once researchers conducted in-depth investigations of the left periphery of German sentences in vernacular speech, a tendency towards V3 patterns in connection with some clausal linkers or framesetters was identified (\citealt{SluckinBunk2023, Wiese2013, WieseMüller2018, Bunk2020, Walkden2017, teVelde2017}). At the same time, there is still a recognizable middlefield framed by two sentential brackets, unlike what we see in SVO languages like English, a phenomenon we also find in our data. While V3 remains marginal in group comparisons – with a slight increase in bilinguals (cf. \citealt{WieseEtAl2022}) – it appears to be a strikingly prominent option for individual speakers, possibly enhanced by cross-linguistic impact from English. Nevertheless, even in these cases, evidence for the clausal bracket, i.e., for the discontinuous placement of finite and non-finite verbs in main clauses prevails, a point we return to in \sectref{sec:tsehaye:6}.

What about the right periphery of German sentences? Once researchers discovered the pervasiveness of \textit{light} constituents in the postfield in informal styles of MSs of German, it became clear that they were on the track of yet another neglected phenomenon located at a clausal edge and at the interface of syntax and discourse pragmatics (cf. \citealt{Vinckel-Roisin2015}), independently of language contact. The question then arises, whether these trends are visible, if not enhanced, in HSs, where another language is added to the picture. After all, extensive extraposing mentioned so far in research on established German language islands could indeed be a consequence of parallelism and convergence with English (\citealt{Clyne2003, WestphalFitch2011}).

In a study with 29 adolescent HSs and 32 adolescent MSs of German, \citet{Tsehaye2023} focused on non-sentential light-weight constituents (LWCs), i.e., prepositional phrases, adverbial phrases, determiner phrases, etc. that appeared after the clause-final predicate. Analyses showed a similar variational spectrum (i.e., the same constituent types) and overall frequency of LWCs in the postfield in HSs and MSs but different distributional patterns. Both speaker groups behaved similarly across production modes: more LWCs were produced in the spoken mode, indicating that the extraposition of LWCs is still a predominantly spoken phenomenon (\citealt{Imo2015, Zifonun2015, ZifonunHoffmannStrecker1997}). Productions across formality levels revealed that MSs furthermore distinguished between formal and informal situations; they produced more LWCs in the informal communicative situations while HSs did not draw this distinction (see \figref{fig:tsehaye:5}).

\begin{figure}
\includegraphics[width=.75\textwidth]{figures/Ch5_Figure_3.pdf}
\caption{Mean percentage of LWCs across formality levels and speaker group (\citealt[9]{Tsehaye2023}).}
\label{fig:tsehaye:5}
\end{figure}

It appears plausible that the missing formality differentiation in HSs can be traced to differences in input conditions between speaker groups and, thus, to diverging access to and awareness of register norms. In order to zoom in on potential language contact effects and transfer phenomena underway in HSs of German, an additional analysis of the extraposition of prepositional phrases (PPs) across speaker groups was performed. While PPs might be ideal extraposition candidates, due to surface parallelism between English and German \citep{HoppPutnam2015, WestphalFitch2011}, the HSs and MSs in this study did not differ with respect to the occurrence of PPs in the postfield.~This could be traced back to an increased tendency of PP extrapositions, also present in MSs, which, in turn, results in similarities across speaker groups. Most importantly, with respect to the overall spectrum of constituents extraposed by HSs, we found no extrapositions of direct objects, except for very few instances entirely due to individual speakers (to be discussed in \sectref{sec:tsehaye:6}). Hence, our consultants did not differ from MSs in their overall frequency of extrapositions, regardless of language contact with English. Rather, differences in the distribution of LWCs between speaker groups can be attributed to differences in register awareness.

\subsection{Multiple solutions to the same challenge}\label{sec:tsehaye:5.4}
\begin{sloppypar}
In contrast to the (relative) stability of clausal architecture, within and across clausal boundaries, there is considerably more divergence in morphological subsystems. Studies on established language islands typically report leveling of agreement, gender, and case paradigms (\citealt{Boas2009b, Boas2009a, Clyne2003, YagerEtAl2015, Zimmer2020}). However, \citet{Boas2009b} also stresses that the overall extent of variability in his Texas German corpus, for instance in gender assignment, is due to just some individuals among the group of participants. Clearly, it is important not to lose sight of these individual profiles because they will, in the end, deliver detailed information on the spectrum of choices available. Affected are formally non-transparent exponents due to massive syncretism and distribution requirements across different carriers (articles, quantifiers, wh-constituents, prenominal adjectives, some overtly on the noun as well). Previous research has shown that HSs have difficulty acquiring and/or retaining a canonical and complete paradigm of inflectional morphology (\citealt{Flores2020, Montrul2011, Polinsky2018Bilingual, Polinsky2018Heritage}). In contrast with basic syntax-internal regularities such as head placement within VPs, the detection and reconstruction of morphological subsystems depends on many – often idiosyncratic – properties of target systems.
\end{sloppypar}

Cross-linguistic differences show up in the way morphological paradigms are integrated in the course of acquisition, especially where various levels or subsystems have to be mapped onto each other. This sits well with the Interface Hypothesis \citep{Tsimpli2014}: Core syntactic, syntax-internal grammatical properties are early in L1, 2L1 and in early childhood L2 acquisition, hence thoroughly entrenched and remarkably stable in the long run, as opposed to other subsystems involving interfaces requiring more and differentiated exposure in order to reach target-like states. The latter undergo leveling more easily. The discovery of, for instance, non-transparent case or gender marking in German and the mapping at the syntax-morphology interface needed for case requires time. Case marking, for instance, is not fully acquired before school age. In addition, both case and gender marking presuppose the emergence of articles as functional categories and carriers of morphological exponents.

First, morphosyntactic aspects raise a number of acquisition hurdles even for MSs of German. With respect to HSs with English as the ML, we expect variation depending on transparency of the system and typological properties. Second, we expect inter-individual variation and differences depending on quantity and quality of exposure. Third, speakers may come up with intra-individual, unique and nevertheless systematic options.\largerpage[1.5]

In our heritage German data, individual speakers show an unexpected preference for dative forms and even overgeneralize them in accusative contexts. This differs from early case marking in L1 German (\citealt{Clahsen1984, Tracy1986}), where the accusative gets the better of the dative for several years. Overgeneralized datives from one HS participant across all registers can be seen in (8a-d), with masculine and neuter articles in the context of feminine nouns, as in \REF{ex:tsehaye:8b}, where a canonical accusative form (\textit{eine}) gets replaced by a noncanonical dative alternative (\textit{einem}).\footnote{To enhance readability, we restrict our glosses to the point under discussion, currently case marking.} This could be attributed to a tendency towards transparency and salience in HSs  (\cites[165]{Polinsky2018Bilingual}{Polinsky2018}, see also \citetv{chapters/10}). In the following oral and written set by the same speaker, only two of the marked datives are formally canonical (\textit{diesem Unfall} in \ref{ex:tsehaye:8a} and \textit{der Straße} in \ref{ex:tsehaye:8d}), with \REF{ex:tsehaye:8d} being semantically infelicitous.\footnote{Canonical alternatives would have been \textit{die Dame auf der Straße gegenüber} ‘the lady on the street across’, \textit{…auf der gegenüberliegenden Straßenseite} ‘…on the across lying street side’.}

\ea\label{ex:tsehaye:8}%8
\ea LangSit formal-spoken \label{ex:tsehaye:8a}\\
\gll \textbf{{mein-em}} {erfahrung} {mit} \textbf{{dies-em}} {Unfall}\\
	 my-\Dat{} experience with this-\Dat{} accident\\
\glt ‘my experience with this accident’

\ex LangSit formal-spoken\label{ex:tsehaye:8b}\\
\gll {es} {gab} {auch} \textbf{{ein-e}} \textbf{{ein-em}} {Frau}\\
	 it gave also {a}-\Acc{} {a}-\Dat{} woman\\
\glt ‘there was also a woman’

\ex LangSit formal-written\label{ex:tsehaye:8c}\\
\gll {der} {man} {hatte} {auch} \textbf{{ein-em}} {fussball}\\
	 the man had also {a}-\Dat{} {soccer ball}\\
\glt ‘the man also had a soccer ball’

\ex LangSit informal-written\label{ex:tsehaye:8d}\\
\gll {die} {dame} {uber} \textbf{{der}} {straße} {hatte} \textbf{{ein-em}} {Hund}\\
	 the lady over {the}.\Dat{} street had {a}-\Dat{} dog\\
\glt ‘the lady across the street had a dog’~
\z
\z

Despite the overall spectrum of noncanonical inter- and intra-individual variation in case marking, the data show: HSs regularly, albeit sometimes noncanonically, mark DPs for various features. They implicitly know that a functional category is needed as a carrier, and they grasp detail of formal inventory, while the actual mapping and accuracy of choice in various online and offline tasks is a totally different matter.

\section{Heritage speaker heterogeneity vs. individual systematicity}\label{sec:tsehaye:6}

Characterizations of HS groups point out their heterogeneity in terms of exposure, proficiency, and performance. While some of the HSs considered here are hardly distinguishable from MSs, especially highly literate ones, others produce strikingly different structures, both with respect to clause structure and morphology. Hence, quantitative analyses over such a highly diverse population blur relevant distinctions and may make us overlook the emergence of new grammatical systems. As a next step, we turn our attention to inter- and intra-individual variation and to qualitative analyses, first, for a subgroup of HSs and then for the productions of an individual speaker.

Our heritage German subcorpus includes nine sets of siblings, making it possible to compare participants who grew up under similar linguistic and extra-linguistic circumstances and were exposed to (presumably) similar input on part of the parents, at least initially (see \citealt{AalberseEtAl2019, BridgesHoff2014, Shin2002} for effects of birth order on HL input and proficiency). For younger siblings, the probability of older siblings and eventually parents speaking the ML at home as well certainly increases. We briefly consider the productions of three siblings (two brothers and one sister: the oldest brother being 18 years old, the sister 17 years, and the youngest brother 14 years\footnote{The speaker codes of the three siblings in the RUEG corpus, are: USbi74MD (oldest brother), USbi72FD (sister), USbi73MD (youngest brother).}). All three were born in the US and, at the time of elicitation, they still lived in the same household with their parents. Their mother was born in Germany and their father in the US, and they reported that their parents both spoke German and English at home. All three siblings rated themselves native speakers of German and English. In the following, we present selected results across several production tasks. While small, the number of instances of clausal structures is sufficient to infer basic patterns.

\begin{table}
\begin{tabular}{l *3{rr}}
\lsptoprule
               & \multicolumn{2}{c}{older}  &               &            & \multicolumn{2}{c}{younger} \\
               & \multicolumn{2}{c}{brother} & \multicolumn{2}{c}{sister} & \multicolumn{2}{c}{brother}\\\midrule
formal-written   & 25 & (9)         & 14 & (4) & 11 & (1)\\
informal-written & 7  & (1)         & 11 & (1) & 5  & (0)\\
formal-spoken    & 39 & (9)         & 12 & (1) & 16 & (2)\\
informal-spoken  & 15 & (6)         & 21 & (2) & 11 & (0)\\
total            & 86 & (25)        & 58 & (8) & 43 & (3)\\
                 & \multicolumn{2}{c}{29.1\%}  & \multicolumn{2}{c}{13.8\%} & \multicolumn{2}{c}{7.0\%}\\
\lspbottomrule
\end{tabular}
\caption{\label{tab:tsehaye:1}Total number of clauses, total number of finite SCs in brackets per sibling and LangSit narration, and mean percentages of SCs across productions.}
\end{table}

\tabref{tab:tsehaye:1} suggests two patterns: Firstly, we find that with decreasing age of the siblings, the narrations become shorter. Secondly, the overall proportion of finite SCs in comparison to matrix clauses decreases as well. Concerning the canonicity of the siblings’ clause productions across LangSit narrations, the data show that the older brother and the sister produce exclusively canonical matrix and subordinate clauses. The younger brother produces four noncanonical V3 structures in matrix clauses. Two of these are the result of a preposed SC \REF{ex:tsehaye:9a} which, if placed in the forefield, needs to be immediately followed by the finite verb to maintain canonical V2 structure in the matrix clause. The other two cases occur with an adverbial in the prefield \REF{ex:tsehaye:9b}.\footnote{In order to highlight our relevant point, we added the index V3 to the English gloss.}

In a next step, we analyzed the siblings’ syntax in the oral sentence completion task. The results show that the older brother and the sister again did not produce a single structurally noncanonical sentence, hence we do not list them here. In the younger brother’s data one noncanonical clause \REF{ex:tsehaye:9c} follows the same clausal pattern as in his LangSit narrations.

\ea%9
    \label{ex:tsehaye:9}
\ea LangSit formal-written\label{ex:tsehaye:9a}\\
\gll {als} {die} {Autos} {gestoppt} {war-en} {ein} {Hund} \textbf{{ist}} {weg-gerannt}\\
when the cars stopped were-\Tpl{} a dog \textbf{is}\textsubscript{V3} away-run\\
\glt ‘when the cars stopped, a dog ran away’

\ex LangSit informal-spoken\label{ex:tsehaye:9b}\\
\gll {und} {dann} {des} {auto} {hinter} {des} {erste} {auto} \textbf{{is}} {in} {des} {andere} {rein-gefahren}\\
and then the car behind the first car \textbf{is}\textsubscript{V3} in the other in-driven\\
\glt ‘and then the car behind the first car drove into the other one’

\ex {Als das erste Auto bremste, …} ‘When the first car braked, …’\label{ex:tsehaye:9c}\\
\gll {das} {andere} \textbf{{ist}} {hinten} {rein-gefahren.}\\
the other \textbf{is}\textsubscript{V3} behind in-driven.\\
\glt ‘…the other-one hit it from behind.’
\z
\z

The siblings also took part in the online written sentence completion task with the very same stimuli administered a few months later. Their written sentence completion mirrors the results of the corresponding oral sentence completion task. Here as well, the older brother and the sister showed no noncanonical verb placement. In the data of the younger brother, however, there were several noncanonical sequels, both with respect to case marking \REF{ex:tsehaye:10a} and word order (\ref{ex:tsehaye:10b}--c).

\ea%10
    \label{ex:tsehaye:10}
    
\ea {Aus der Tasche der Frau mit dem Auto…}\label{ex:tsehaye:10a}\\
	 ‘Out of the bag of the woman with the car…’\\
\gll ist \textbf{{ein-em}} {Apfel} {aus} {der} {Tasche} {gefallen.}\\
	 is {an}-\Dat{} apple out the bag fell.\\
\glt ‘…an apple fell out.’

\ex \textit{Nachdem der Ball über die Straße rollte, …} \label{ex:tsehaye:10b}\\
	 ‘After the ball rolled across the street, …’\\
\gll {der} {Hund} {\textbf{ist}\textsubscript{V3}} {zu} {dem} {Ball} {gesprungen.}\\
	 the dog {is} to the ball jumped\\
\glt ‘After the ball rolled across the street, the dog jumped to the ball.’

\ex \textit{Als das erste Auto bremste, …}\label{ex:tsehaye:10c}\\
	 ‘When the first car braked, …’\\\
\gll {das} {andere} {\textbf{is}\textsubscript{V3}} {hinten} {rein-gefahrn.}\\
	  the other {is} behind in-driven.\\
\glt ‘When the first car braked, the other one hit it from behind.’
\z
\z

Results across the three tasks show the smallest number of noncanonical productions in the older brother, followed by his younger sister. Most noncanonical structures are produced by the youngest brother, and all point in the expected direction: Noncanonical patterns predominantly result from placement of the finite verb not far enough towards the left periphery, i.e., V3 instead of V2. The same tendency and difference between the siblings also manifest themselves in the sentence correction task – not illustrated here for reasons of space. The crucial point is that even within a single family, i.e., a Tiny Language Island scenario, we see individual differences – in line with what previous research has found on birth order effects on HL productions – pointing in a direction relevant to our discussion on clause structure. Only one of the siblings exhibits a clear shift towards V3 clauses with a complex prefield. While the result is a parallel with English, the main verb and its particle are still separated by other constituents: the remnant of a shrunken middlefield.

In the following, we take the Tiny Island scenario one step further and consider data from a 17-year-old female adolescent HS\footnote{The speaker code of this participant is USbi77FD.}, whose German input only comes from the mother. In her LangSit narrations, her oral sentence completions, and all throughout the informal chitchat, she considerably – and well beyond what we saw above – diverges from canonical patterns, both in word order and morphological subsystems, including an idiosyncratic spell-out of subject-verb agreement. The data show noncanonical third person singular verbs in contexts calling for plurals (\ref{ex:tsehaye:11}a--b), with \REF{ex:tsehaye:11a} potentially strengthened by English \textit{were}. The utterances in (\ref{ex:tsehaye:11}c--d) also show noncanonical first person singular and plural inflections. We will turn to the “?” in the interlinear glosses in \REF{ex:tsehaye:11b} shortly.

\ea%11
    \label{ex:tsehaye:11}
\ea LangSit formal-written\label{ex:tsehaye:11a}\\
\gll {all-e} \textbf{war}  {okay}\\
	 all-\Pl{} was.\Sg{} okay\\
\glt ‘all were okay’

\ex LangSit formal-spoken\label{ex:tsehaye:11b}\\
\gll {ein} {mann} \textbf{{er}}\footnotemark {frau} {und} \textbf{{er}} {baby} {und} {die} \textbf{{hat}} {zu} {eine} {straße} {gelaufen}\\
	  a man       ? wife and ? baby and they.\Pl{} has.\Sg{} to a street walked\\
\glt ‘a man, his wife and his baby walked towards a street’

\ex Chitchat\label{ex:tsehaye:11c}\\
\gll {Ich} {gern} \textbf{{geh-t-st}} {draußen} {mit} {meine} {Freunde}\\
I gladly go-\Tsg-\Ssg{} outside with my friends\\
\glt ‘I like going out with my friends’

\ex Chitchat (topic: cooking chili)\label{ex:tsehaye:11d}\\
\gll {wir} \textbf{{nehm-st}} {Bier}\\
we.\Fpl{} take-\Ssg{} beer\\
\glt ‘we take beer’
\z
\z
\footnotetext{We return to the identification of \textbf{er} as a potential placeholder later on.}
Taking into account all auxiliaries\footnote{German exhibits auxiliary alternation between \textit{haben} ‘have’ and \textit{sein} ‘be’ depending on the property of the main verb. English, on the other hand, always uses ‘have’ in active clauses.} across elicitation tasks, a clear pattern emerges: \textit{haben} ‘have’ predominates, irrespective of contextual requirements of both standard German and dialects. The examples in (\ref{ex:tsehaye:10a}--b) show this kind of leveling of the auxiliary inventory, possibly under the influence of cross-lin\-guis\-tic transfer from English. The noncanonical suffixes in (\ref{ex:tsehaye:11c}+d) are based on noncanonical person markers: a 2\textsuperscript{nd} Ps.Sg. added to an already 3\textsuperscript{rd} Ps.Sg.-marked verb in \REF{ex:tsehaye:11c} and added to a noncanonical stem form in \REF{ex:tsehaye:11d}. This overgeneralization pattern is quite unexpected and uncommon for L1 acquisition \parencites[26]{Aalberse2009}[193]{AalberseStoop2015}.

In \REF{ex:tsehaye:12a}, a very fluently produced utterance, we see multiple departures from canonical German, including calques based on English words and collocations. Also, in the same example, word order in the relative clause follows an English pattern, as does the sequel after the complex PP stimulus in the oral sentence completion task in \REF{ex:tsehaye:12b}.

\ea%12
    \label{ex:tsehaye:12}
\ea Chitchat\label{ex:tsehaye:12a}\\
\gll {ich} {weiß} {nich} {alle} {diesen} {platz} {auf} {mein} {kopf} {wo} {ich} \textbf{habe} {gegangen}\\
I know not all this place on my head where I have gone\\
\glt  ‘I don’t remember all the places I have been on top of my head’

\ex {{Aus der Tasche der Frau mit dem Auto…}} ‘Out of the bag of the woman with the car…’\label{ex:tsehaye:12b}\\
\gll {Äpfel} {oder} {etwas} \textbf{{hat}} {gefallen} {und} {das} {Mann} \textbf{{hat}} {gerennen}\\
apples.\Pl{} or something {has}.\Sg{} fallen and the man {has} run\\
\glt ‘…apples or something fell out and the man ran.’
\z
\z

Despite many clearly noncanonical and not even dialectally licensed features of the young woman’s written and spoken productions, a number of relevant properties can be identified. Examples (\ref{ex:tsehaye:13a}--\ref{ex:tsehaye:13d}) illustrate the idiosyncratic but consistent creation of what we hypothesize to be a possessive placeholder. Crucial insight in support of this interpretation comes from her written narratives. Just on the basis of spoken versions alone, one could have easily dismissed the syllables preceding \textit{frau}, \textit{baby}, and \textit{ball} as hesitations or filler particles. But as the written version \REF{ex:tsehaye:13b} shows, there is a visible orthographic exponent. Moreover, when we checked the participant’s English narratives, we found our interpretation of the kind of concepts she wanted to express corroborated, as shown in (\ref{ex:tsehaye:13c}) and (\ref{ex:tsehaye:13d}).

\ea%13
    \label{ex:tsehaye:13}
\ea \label{ex:tsehaye:13a} LangSit formal-spoken\\
\gll {ein} {mann} \textbf{{er}} {frau} {und} \textbf{{er}} {baby}\\
a man ? wife and ? baby\\
\glt ‘a man, his wife, and his baby’

\ex \label{ex:tsehaye:13b} LangSit formal-written\\
\gll {es} {gibt} {ein} {par} {leute} {ein} {mann,} \textbf{{er}} {frau} {un} {kind}\\
it gives a few people a man ? woman and child\\
\glt ‘There were a few people. A man, his wife and child’

\ex \label{ex:tsehaye:13c} LangSit formal-spoken\\
{there was a dude} \textbf{{his}} {wife and} \textbf{{his}} {baby}

\ex \label{ex:tsehaye:13d} LangSit formal-written\\
{there was a man with} \textbf{{his}} {wife and baby}
\z
\z

The hypothesis that \textit{er} in German fulfills a syntactic dummy function aligns with what we know from research on L1, 2L1, and L2 acquisition in children and on L2 adults on other placeholder phenomena (\citealt{DaskalakiEtAl2019,Tracy2011}, see also various papers in \citealt{BlomCraatsVerhagen2013}).

More importantly with respect to our current focus on clause structure: In this HS’s productions, regardless of type of communicative context, of register, and of the type of task, word order is predominantly noncanonical, here selectively illustrated with \REF{ex:tsehaye:14a}, with only non-finite verbs (here infinitives and participles) placed in final position and at a distance from the finite verb, as shown in \REF{ex:tsehaye:14b}.

\ea%14
\label{ex:tsehaye:14}
\ea \label{ex:tsehaye:14a} Chitchat\\
\gll {aber} {das} {ist} {nicht} {was} {ich} \textbf{{will}} {tun} {wenn} {ich} \textbf{{bin}} {alt}\\
but this is not what I want do when I am old\\
\glt ‘but this is not what I want to do when I am old’

\ex \label{ex:tsehaye:14b} Chitchat\\
\gll {dass} {ich} \textbf{{habe}} {zu} {einundzwanzig} {countries} {gegangen}\\
that I have to twenty-one countries gone\\
\glt ‘that I have visited 21 countries’
\z
\z

Finally, this speaker is one of the very few participants who extrapose a direct object \REF{ex:tsehaye:15}.

\ea%15
\label{ex:tsehaye:15}
Chitchat\\
\gll {dass} {ich} {gerne} {esst} \textbf{{chili}}\\
that I gladly eat chili\\
\glt ‘that I like eating chilli’
\z

What makes the overall idiosyncratic but also internally systematic spectrum of noncanonical structures in this adolescent speaker particularly valuable is that it is like a fast forward into patterns familiar from established language islands. After all, many older speakers in what once used to be vibrant German speech communities are on their own way towards Tiny Language Islands since they may well be the only HSs left within their family.

\section{Discussion and conclusion}\label{sec:tsehaye:7}

We hope to have shown that HS research offers a promising testing ground for exploring various intriguing phenomena which have been puzzling language acquisition researchers for a long time. In this last section, we recapitulate our main points. 

Our first research question focused on the retention of clausal syntax in the face of language contact. Our findings show, with the exception of a few individuals, predominantly canonical productions across HSs. As we know from early phases of word combining in typically developing German-speaking children, OV\textsubscript{[-fin]} is acquired before age two (\citealt{SchulzTracy2018}). This may explain why, in the overwhelming number of cases, the direct objects produced by adolescent HSs occur in their canonical position, as predicted by the Interface Hypothesis. Thus, with respect to core syntactic, early acquired features, we find no major transfer effects from the ML.

\begin{sloppypar}
A different picture emerged when we focused on the morphological features of HSs’ productions, addressing research question two. Qualitative analyses showed noncanonical variation in case and gender marking, i.e., phenomena at the interface of morphology, syntax, and semantics.
\end{sloppypar}

Concerning our last research question, which focused on inter- and intra\babelhyphen{hard}individual variation, our findings fit the general picture emerging from research on traditional language islands and from current HL investigations with respect to the heterogeneity of speaker profiles (\citealt{AalberseEtAl2019, MontrulPolinsky2021, Polinsky2018Bilingual, Polinsky2018Heritage, WieseEtAl2022} for other languages represented in the RUEG project). In the majority of the heritage German speakers taking part in our study, core-syntactic properties related to finite and non-finite verb placement in main and subordinate clauses – especially the sentence bracket – proved relatively stable. At the same time, we also saw differences in speakers’ output even in cases where they grew up within the same family and were (most likely) exposed to qualitatively similar input. A subset of crucial properties of German clause structure was still detectable in the final case study we selected for illustration: a speaker whose grammatical system had considerably diverged from canonical German patterns, both in the syntax and in morphological spell-out. Importantly, variation proved intra-individually systematic.

We conclude that there are certainly many \textit{dots to connect}, as also formulated in \citet{Montrul2018}, between research on language acquisition in early childhood and attempts at reconstructing the grammatical systems available to adolescent and older HSs \citep{Polinsky2018Bilingual}. After all, typically developing monolingual and bilingual children also pass along trajectories where they produce noncanonical main and subordinate clauses and noncanonical inflectional morphology.\footnote{For transitional phases with noncanonical clauses and idiosyncratic patterns in monolingual and bilingual children see \citealt{Döpke2000, FritzenschaftEtAl1990, GawlitzekMaiwald1994, Müller1998, Rothweiler1993, Tracy2011}.}  Since these phases may be short or undocumented, they might easily go by unnoticed in cross-sectional investigations without access to individual longitudinal data.

It does not take much imagination to think of what might happen in typically developing L1 children for whom temporarily divergent paths have been documented in longitudinal research, had they been transported into a different ML setting before their grammars had converged on target states. In that case, their intermediate solutions might easily have been there to stay, for instance the regularization of irregular forms amply documented in L1 acquisition for many languages (\citealt{Yang2016, GawlitzekMaiwald1994How}).

Our research design provided various opportunities for speakers to activate and demonstrate their linguistic resources across tasks with varying complexity: from free conversations to the elicitation of contents to report and the elicitation of particular target patterns, as in the spoken and written sentence completion task. This provided us, in turn, with opportunities for triangulation and guarded us against misconceptions which might have arisen had we only relied on one kind of data or on just one type of situation, or had we not included the productions of MSs of German and productions in the ML of HSs.

Our findings and our conclusions are also relevant beyond research on HSs of German, as shown by other contributions in this volume. Given the heterogeneity of background situations in terms of quantity and quality of exposure, personal preferences, and attitudes, it is quite impressive to what extent HSs, including those raised on Tiny Language Islands, are able to discover and maintain crucial properties of their parental baseline, i.e. of first-generation immigrants. Despite reduced HL exposure and the increasing relevance of the ML, in our case English, core-syntactic features, especially those acquired early, remain stable, even in contexts with reduced exposure to the HL, while interface phenomena are subject to variation and change from one generation to the next. The theoretically most relevant outcome of our study may be that both inter- and intra-individual variation are systematic, and that like any other acquisition scenario, HL acquisition and use contribute to our understanding of natural languages.

\section*{Abbreviations}

\begin{multicols}{2}
\begin{tabbing}
MMMM \= Pennsylvania German\kill
2L1 \> Simultaneous acquisition of \\ \> two first languages\\
AdvC \> Adverbial clause\\
CMC \> Coordinate main clause\\
CompC \> Complement clause\\
HL \> Heritage language\\
HS \> Heritage speaker\\
IMC \> Independent main clause\\
L1 \> First language\\
L2 \> Second language\\
LSB \> Left sentence bracket\\
LWC \> Non-sentential light-weight \\ \> constituent\\
ML \> Majority language\\
MS \> Monolingually-raised speaker\\
PG \> Pennsylvania German\\
RelC \> Relative clause\\
RSB \> Right sentence bracket\\
% % % RUEG \> Research Unit Emerging Grammars in Language Contact Situations\\
SC \> (Finite) Subordinate clause\\
S \> Sentence\\
SVO \> Subject-Verb-Object\\
V1 \> Verb-first-position\\
V2 \> Verb-second-position\\
VL \> Verb-last-position\\
\end{tabbing}
\end{multicols}

\renewcommand{\exfont}{\upshape}
\examplesroman
\section*{Acknowledgements}

We would like to thank the DFG for funding the research for this chapter (grant number: 394995401). We also owe thanks to two anonymous external reviewers for their much appreciated and encouraging comments on previous version of this chapter, as well as to our internal reviewers Tatiana Pashkova, Vassiliki Rizou, and Annika Labrenz. For helpful criticism on the pre-final version of our chapter we would like to thank our cooperation partner Dafydd Gibbon (University of Bielefeld). Our research would not have been possible without the help of our enthusiastic student research assistants. We would therefore like to thank Stella Baumann, Lale Bilgili, Joshua Boivin, Ryan Caroll, Franziska Cavar, Anna Kuhn, Nils Picksak, Sam Schirm, Elena Unger, and Nadine Zürn. We additionally thank Lea Coy for help with pre-publication formatting.

\printbibliography[heading=subbibliography,notkeyword=this]
\end{document} 
