\documentclass[output=paper,colorlinks,citecolor=brown]{langscibook}
\ChapterDOI{10.5281/zenodo.15775183}
\author{Sabine Zerbian \orcid{0000-0002-4631-369X} \affiliation{University of Stuttgart} and         Yulia Zuban \orcid{0009-0009-3033-1760} \affiliation{University of Stuttgart} and        Marlene Böttcher \orcid{0009-0000-7538-2902} \affiliation{Kiel University} and        Oliver Bunk \orcid{0000-0003-4505-4873} \affiliation{Humboldt-Universität zu Berlin}}

\title{Intonation in heritage languages}

\abstract{Intonation in heritage languages is a much less researched linguistic domain than, for example, morphosyntax. This chapter reports on several studies investigating the intonation of Russian and German bilingual speakers, based on the data from the RUEG corpus, in their heritage language and/or in the majority language. \sectref{chapter12:section 3} addresses overall intonational features in (heritage) Russian, relating to phrasing and frequency of pitch accents. \sectref{chapter12:section 4} reports findings on intonation used for various phenomena at linguistic interfaces, such as intonation in (heritage) Russian yes/no questions (\sectref{chapter12:section 4.1}), for contrastive adjective focus in (heritage) Russian (\sectref{chapter12:section 4.2}) and in discourse linking in heritage and majority German (\sectref{chapter12:section 4.3}). \sectref{chapter12:section 5} addresses the perception of accents by (heritage) Russian listeners. Together, the studies reveal that heritage language speakers and monolingually-raised speakers show differences in prosodic realization and perception, although such differences are more quantitative than qualitative in nature.

\keywords{intonation, heritage language, Russian, German, information structure}
}

\IfFileExists{../localcommands.tex}{
   \addbibresource{../localbibliography.bib}
   \usepackage{langsci-optional}
\usepackage{langsci-gb4e}
\usepackage{langsci-lgr}

\usepackage{listings}
\lstset{basicstyle=\ttfamily,tabsize=2,breaklines=true}

%added by author
% \usepackage{tipa}
\usepackage{multirow}
\graphicspath{{figures/}}
\usepackage{langsci-branding}

   
\newcommand{\sent}{\enumsentence}
\newcommand{\sents}{\eenumsentence}
\let\citeasnoun\citet

\renewcommand{\lsCoverTitleFont}[1]{\sffamily\addfontfeatures{Scale=MatchUppercase}\fontsize{44pt}{16mm}\selectfont #1}
  
   %% hyphenation points for line breaks
%% Normally, automatic hyphenation in LaTeX is very good
%% If a word is mis-hyphenated, add it to this file
%%
%% add information to TeX file before \begin{document} with:
%% %% hyphenation points for line breaks
%% Normally, automatic hyphenation in LaTeX is very good
%% If a word is mis-hyphenated, add it to this file
%%
%% add information to TeX file before \begin{document} with:
%% %% hyphenation points for line breaks
%% Normally, automatic hyphenation in LaTeX is very good
%% If a word is mis-hyphenated, add it to this file
%%
%% add information to TeX file before \begin{document} with:
%% \include{localhyphenation}
\hyphenation{
affri-ca-te
affri-ca-tes
an-no-tated
com-ple-ments
com-po-si-tio-na-li-ty
non-com-po-si-tio-na-li-ty
Gon-zá-lez
out-side
Ri-chárd
se-man-tics
STREU-SLE
Tie-de-mann
}
\hyphenation{
affri-ca-te
affri-ca-tes
an-no-tated
com-ple-ments
com-po-si-tio-na-li-ty
non-com-po-si-tio-na-li-ty
Gon-zá-lez
out-side
Ri-chárd
se-man-tics
STREU-SLE
Tie-de-mann
}
\hyphenation{
affri-ca-te
affri-ca-tes
an-no-tated
com-ple-ments
com-po-si-tio-na-li-ty
non-com-po-si-tio-na-li-ty
Gon-zá-lez
out-side
Ri-chárd
se-man-tics
STREU-SLE
Tie-de-mann
}
   \boolfalse{bookcompile}
   \togglepaper[12] %%chapternumber
}{}

\begin{document}
\maketitle

\section{Introduction and background} \label{chapter12:section 1}

Heritage language speakers (HSs) are speakers who grew up in a bi- or multilingual home with a minority language in addition to the majority language (ML) dominant in the larger society (e.g., \cites{Rothman_2009}[116]{Polinsky_2018}). On the one hand, HSs are reported to be similar to monolingually-raised speakers with respect to the phonetics and phonology of their heritage language (HL), which has been termed “phonetic advantage” \citep[116]{Polinsky_2018}. On the other hand, studies have shown consistent differences in the phonetic realization of, for example, Voice Onset Time or vowels by HSs as compared to monolingually-raised speakers (see \cite{Polinsky_2018, Montrul_2016} for overviews). These fine phonetic differences might contribute to the clearly discernible accent or heritage accent 
\citep{Chang_2021} that has likewise been reported for HSs. It has been confirmed in various empirical studies that listeners rate the speech of HSs in their HL as more accented than monolinguals’ speech (and less accented than L2 speech) (e.g., \cite{Kupisch_Barton_Hailer_Klaschik_Stangen_Lein_vandeWeijer_2014} on Italian, French; \cite{Flores_Rato_2016} on European Portuguese) although it remains unclear what the perceived accent is due to.

In addition to differences in the segmental domain, it can be expected that there will also be differences in the suprasegmental domain relating to stress, rhythm and intonation, when comparing HSs with monolingually-raised speakers (see \cite{Montrul_2016, Polinsky_2018}). And indeed, previous studies find that HSs produce suprasegmental features differently from both monolingually\hyp raised speakers and L2 learners (e.g., \cite{Chang_Yao_2016} for Mandarin tone; \cite{Dehé_2018} for yes/no questions in North American heritage Icelandic). It is even suggested that suprasegmental features are more salient contributors to a heritage accent than segmental ones \citep[601]{Chang_2021}, thus further motivating the study of intonational aspects. 

There are several reasons why we expect to find differences in the domain of intonation. On the one hand, intonation has been found to be fully acquired late in first language acquisition; thus, for a child who grows up with two languages simultaneously at a young age (as the HSs in our studies did), there is considerable room for cross-linguistic interactions in prosody, both from the ML to the HL and from the HL to the ML. Cross-linguistic influence might show itself along different dimensions: it might refer to the phonetic realization of identical tunes, the intonational inventory of the languages, the semantic meaning associated with a certain intonational event, the distribution of tunes, or the frequency of use (as detailed in the L2 Intonation Learning Theory model, LILt, by \cite{Mennen_2015}).\largerpage

On the other hand, intonation in languages like English, German or Russian is determined both by structural constraints including a nuclear pitch accent on the last constituent, as well as by pragmatic constraints, e.g., nuclear accent on the focused constituent, deaccentuation of given constituents, or final tunes determined by sentence type. Therefore, intonation is an interface phenomenon. The Interface Hypothesis \citep{Sorace_2011} states that linguistic phenomena at the interface with external domains, such as pragmatics, are prone to change in language contact situations. A recent study by \citet{Feldhausen_Vanrell_forth} on the syntactic and prosodic realization of different focus types produced by HSs of Peninsular Spanish with German as ML gives support to the Interface Hypothesis by showing that HSs differ from monolingual speakers of Spanish at both the syntax-discourse and the phonology-discourse interface.

Finally, Polinsky describes somewhat opposing tendencies in heritage language phonology. On the one hand, various studies “converge on the observation that some contrasts present in the input undergo leveling” and, thus, a loss of distinction 
\citep[125]{Polinsky_2018}. On the other hand, based on data on final devoicing by HSs of Polish with ML English, \citet[134]{Polinsky_2018} describes a “tendency to overemphasize the sound properties of the heritage language that set it apart from the dominant language”. She further claims that this is “typical of heritage speakers who are more proficient in Polish”, as deduced from their higher rate of code-switching (cf. \cite{Poplack_1980} for code-switching as a sign of more balanced bilingualism). She suggests that it is “a general side effect of bilingualism” \citep[136]{Polinsky_2018} that the differences between the languages are exaggerated. However, the nature of such “side effects” is unclear. Sociolinguistic perspectives might help in uncovering such effects; for example, some speakers might produce specific features more distinctly to align with the more privileged variety of a language (see \cite{chapters/11}).

We thus expect to find considerable variation in intonation produced by HSs in both their HL and ML. The variation might reveal new patterns along various dimensions of the intonation system. These might be attributable to factors such as cross-linguistic influence or general effects of bilingualism, for example, related to processing, fluency or decreased language exposure or use.

This chapter presents several studies that illustrate the spectrum of intonational variation that we have found in the data of our corpus of spoken language, focusing primarily on Russian HSs but also including some information on German HSs (see also \cite{chapters/13} for intonational aspects of English as a majority language).
\footnote{\citet{Wandrei_2019} analysed features of segmental phonology in four HSs of Russian (ML German, female, 16--18 years old), using data from the RUEG corpus. The following phonological features were analyzed auditorily: palatalization, realization of <r>, allophonic realizations of /x/, final devoicing, voicing assimilation, and aspiration of voiceless plosives. The results show that salient differences emerge only in the realization of <r>, although not in all speakers \citep[32]{Wandrei_2019}. Another interesting observation was individual speakers producing single lexical items, such as “okej” with a German [oˈkeː] instead of a Russian pronunciation [aˈkej] \citep[34]{Wandrei_2019}.} It presents examinations of overall intonational features, specifically prosodic phrasing and the inventory and distribution of pitch accents in (heritage) Russian. It also reports the results of investigations of intonational features in specific linguistic structures, zooming in on the prosodic realization of selected sentences, phrases or words in heritage and monolingual Russian as well as bilingual and monolingual German. Intonational patterns are compared across HSs in different countries and across formality. By including and presenting results from productions of Russian by monolingually-raised speakers, our studies also make interesting empirical contributions for this speaker group. In the case of German as an ML, the chapter presents results from a study of phrasing in the left sentence periphery. 


\section{Methodology} \label{chapter12:section 2}
All research reported in this study is based on data in the RUEG corpus. The relevant subcorpora and/or additional data are further specified in their respective sections.

The RUEG corpus \citep{RUEGcorpus2024} was collected and annotated by the collaborative Research Unit \textit{Emerging Grammars} funded by the German Science Foundation from 2018--2024. Naturalistic data were elicited by means of presenting participants with a video clip depicting a fictional car accident. Participants were asked to describe the accident, imagining themselves in different communicative situations which covered different modes (spoken/written) and formality (formal/informal). In the formal communicative situation, participants were asked to give an account of what they saw to the police. In the informal communicative situation, participants were asked to leave a message for a friend using WhatsApp (for details on methodology, see \cite{Wiese_2020}). Data were elicited from HSs of Turkish, Greek, and Russian in Germany and the US, from HSs of German in the US, and from monolingually-raised speakers in Turkey, Greece, Russia, Germany, and the US. For the studies on intonation reported in this chapter, only the spoken Russian data are considered as well as data from heritage and monolingually-raised speakers of German.

\subsection{Russian: Data and participants} \label{chapter12:section 2.1}
The spoken Russian data can be accessed as the Russian Prosody subcorpus (RuPro).\footnote{\url{https://korpling.german.hu-berlin.de/annis/}} It contains spoken Russian data of about 25k word tokens by a total of 40 monolingually-raised Russian speakers and 53 bilingual Russian HSs in the United States. The speakers fall into two age groups: adolescents (14--18 years; 20 mono, 22 bilingual) and adults (22--35 years; 20 mono, 31 bilingual). The corpus data of each participant are enriched with extensive metadata on their language background in both HL and ML (e.g., beginning of language acquisition, media use, language use with parents, self-assessment of language skills such as understanding, speaking, reading, and writing), socio-economic status (e.g., highest degree for adult participants; parents’ profession for adolescent participants), and personality traits (along five dimensions: Extraversion, Agreeableness, Conscientiousness, Emotional Stability, and Openness to Experience; \cite{GoslingSwann2003}).

All data were transcribed, normalized, and annotated for communication units (CU; \cite{Hughes_McGillivray_Schmidek_1997}) and part of speech. Three prosodic layers were added to the data of the spoken formal and informal modes. The prosodic layers show annotations for pitch accent placement on the word level, pitch accent type, and Intonation Phrase (IP) boundaries. Pitch accent placement and pitch accent type were determined auditorily and phonetically, based on examination of the F0 (low and high turning points). IP boundaries were annotated following the guidelines stated in \citet{Himmelmann_Sandler_Strunk_Unterladstetter_2018}. Six types of pitch accents (henceforth PA; L*, H*, L*+H, L+H*, H+L* and H*+L) and two final boundary tones (henceforth BT; L\%, H\%) emerged as relevant in these data. Sample contours of the three most frequent PAs are provided in Figures~\ref{chapter12:fig:1} and \ref{chapter12:fig:2}. An illustration of all PA labels can be found in \citet{Zerbian_Zuban_Klotz_2024}.

\begin{figure}
  \centering
  \includegraphics[width=\textwidth]{figures/Ch12_Figure1_pitch_accents_example.png}
  \caption{Sample contours of the three most frequent pitch accents in Russian (H*, L+H*, H+L*)}
  \label{chapter12:fig:1}
\end{figure}

\begin{figure}
  \centering
  \includegraphics[width=\textwidth]{figures/Ch12_Figure2_boundary_tones_example.png}
  \caption{Sample contours of L\% and H\% in yes/no questions in (heritage) Russian (see \sectref{chapter12:section 4.1})}
  \label{chapter12:fig:2}
\end{figure}

All H tones could be additionally upstepped (pitch range expansion compared to a preceding high tone) or downstepped (pitch range compression). Praat was used for phonetic annotation of the audio files \citep{Boersma2001}.

Concerning exposure and active language use, HSs had a high involvement with Russian in their everyday life according to self-assessments (e.g., watching TV programs in Russian, writing in Russian-speaking social media; almost every HS could write and read Russian Cyrillic script) (for more details see \cite{Zerbian_Barabashova_Zuban_subm, Zuban}).

Studies consider proficiency in bilingual populations to be a multidimensional phenomenon that includes many different aspects (e.g., type-token ratio, ratio of main and subordinate clauses, frequency of word-internal code switching and repetitions, word length, speech rate, and number of empty and filled pauses; see \cite{Heegard_etal_2018, Montrul_2016} on HSs; see \cite{Iwashita_Brown_McNamara_O’Hagan_2008, Christoffersen_2017} on L2 learners). We concentrated on fluency and quantified HSs’ fluency by measuring their speech rate and the number of filled pauses. Speech rate was calculated as number of words per minute (as done in other studies on heritage Russian, e.g., \cite{Polinsky_2008, Laleko_Dubinina_2018}). Mean speech rate was comparable between HSs (114 words/min) and monolingually-raised speakers (119 words/min) \citep{Zuban}. This shows an overall high fluency of the HSs in their HL. (In)formality was an important factor, though: HSs showed a similar speech rate in both formal and informal communicative situations while monolingually-raised speakers spoke significantly faster in the informal situation compared to the formal one, maybe due to lower habitual ease in the latter. As a result, HSs and monolingually-raised speakers were similar to each other in the formal situations, but not in the informal ones.

Another indicator for fluency is the use of filled pauses like \textit{um} in English (e.g., \cite{deJong_2016}). \citet{Böttcher_Zellers_subm} looked at the frequency of filled pauses in the RUEG corpus (version 1.0\_SNAPSHOT) including both heritage Russian in the US and Germany and monolingual Russian, as well as majority English and majority German. Statistical analyses reveal significantly more filled pauses in formal narrations and in the speech of bilingual speakers irrespective of their language. This indicates similarities in another aspect of fluency between HSs and monolingually raised speakers in formal situations, but not in informal situations. 

The results concerning fluency, considering both speech rate and use of filled pauses, can be related to aspects of higher speech planning costs required in monitoring two languages and in formal situations. The monitoring of two languages and suppressing one in monolingual mode has been reported to be cognitively demanding \citep{Kroll_Gollan_2014}. Also, speech planning in formal contexts may take more effort, with higher pressure on language form and meaning, and might also include higher interpersonal insecurity compared to more intimate, informal situations (\cite{Tottie_2014, Staley_Jucker_2021} based on the use of filler particles).


\subsection{German: Data and participants} \label{chapter12:section 2.2}

The spoken German data come from the German subcorpus of the RUEG corpus (RUEG-DE), and from a corpus on heritage Low German in the US \citep{Rocker_2022}. RUEG-DE (version 1.0-SNAPSHOT) contains a total of 156k word tokens by monolingually-raised German speakers ($n=64$; 32 adults, 32 adolescents), bilingual German speakers with the HLs Turkish ($n=65$; 33 adults, 32 adolescents), Russian ($n=59$; 30 adults, 29 adolescents), or Greek ($n=44$; 26 adults, 18 adolescents), and heritage German speakers in the US ($n=36$; 7 adults, 29 adolescents). All data were transcribed, annotated morphologically for part of speech, lemma, and function, and annotated syntactically for topological fields and syntactic dependencies. The topological field model analyzes German sentence structure into different fields that are occupied by specific elements (see, e.g., \cite{Drach_1937, Höhle_1986}). Topological fields were annotated based on the annotation guidelines used for the Kiezdeutschkorpus \citep{Wiese_etal_2010}, which allow annotation of specific noncanonical phenomena in the left sentence periphery such as left dislocation constructions, hanging topics, and constituents such as adverbials that are followed by a non-verbal constituent. The corpus by \citet{Rocker_2022} on heritage low German in the US comprises data from 46 speakers and a total of 58 sociolinguistic interviews. The relevant data for the analyses were extracted from the corpora and prosodically annotated for phrase boundaries in order to analyze phrasing in the left periphery (see \sectref{chapter12:section 4.3} for a detailed description).


\section{Overall intonational features: Intonational phrases and pitch accents} \label{chapter12:section 3}

Heritage speakers and monolingually-raised speakers may differ with respect to overall intonational features; in other words, intonational features observed across entire narrations as opposed to specific linguistic constructions. Here we focus our research on two areas. First, we compare the intonation of HSs and monolingually-raised speakers on the frequency dimension \citep{Mennen_2015}, investigating whether intonational events such as pitch accents (PAs) are used equally frequently. Second, we asked whether prosodic units such as Intonational Phrases (IPs), which are defined by intonational contours, are similar in size in comparable narrations.

In an earlier case study, \citet{Comstock_2018} investigated intonation patterns of one HS of Russian in the US and two monolingual Russian-speaking journalists in two types of political interviews: affiliative and antagonistic. The affiliative interviews were those that did not involve any controversy, but instead included topics that helped an interviewer establish friendly relationships with an interviewee (e.g., personal opinions on whether it is easy for foreign journalists to work in Russia). \citet[268]{Comstock_2018} classifies affiliative interviews as informal in nature since they resemble a conversation rather than a strict question-answer format. The antagonistic interviews had an element of conflict (e.g., the HS was asked to comment on a US presidential candidate who claimed that Russia was America’s main enemy), and are therefore viewed as more formal by \citet[268]{Comstock_2018}.

\citet{Comstock_2018} found that the HS differed from the two monolingual speakers of Russian in both interviews regarding the frequency of some intonational patterns. For instance, the monolingual speakers of Russian did not produce any monotonal PAs, while the HS produced on average 5.9 monotonal accents per IP in the affiliative interview and 5.2 monotonal accents per IP in the antagonistic interview \citep[220; 231; 248; 256]{Comstock_2018}. Furthermore, some prosodic patterns of monolingual interviewers were found to differ depending on the interview (e.g., the number of single-word IPs or the number of falling H+L* nuclear accents) while prosodic patterns of the HS only differed in one aspect across the two interviews; namely, the number of fronted constituents was greater in the affiliative interview than in the antagonistic one \citep[267]{Comstock_2018}. Thus, this case study showed that the monolingual speakers and the HS differed in their intonation, and that type of interview mattered more for the intonation of monolingual speakers than for the intonation of the HS.

Similar to the study by \citet{Comstock_2018}, we explored intonational differences between monolingual speakers and HSs of Russian, using the RuPro subcorpus. We explored differences along the systemic (e.g., inventory of tonal events) and frequency dimensions by extracting labels and intervals pertaining to intonational events (pitch accents (PA) and boundary tones (BT)) and units (Intonation Phrases (IP)) from the prosodic annotation of the data. Detailed descriptions are available in \citet{Zerbian_Zuban_Klotz_2024} and \citet{Zuban}.  

Regarding the inventory of PAs and BTs, no systemic difference emerged between HSs and monolingually-raised speakers as in both speaker groups the same kinds of PAs and BTs were attested in overall similar frequency. Here we report the group results for size of IPs and number of PAs to illustrate the difference in patterns of overall intonational features between the two groups. 

To determine the size of intonational phrases, the overall number of IPs was extracted and divided by the total numbers of words to account for differences in the overall length of narrations. As shown in \tabref{chapter12:tab:1}, HSs produce overall shorter IPs (i.e., fewer words per IP) than monolingually-raised speakers. A linear mixed model was fit to the data and revealed a significant effect of formality ($z = 5.97$, $p < 0.001$) and a significant interaction of formality and speaker group ($z = -3.52$, $p < 0.001$). A post-hoc pairwise comparison showed that monolingual speakers produced shorter IPs in formal situations than in informal situations ($p < 0.001$), and monolinguals differed from HSs by producing longer IPs in informal situations ($p < 0.001$). HSs did not show any effect of formality on IP length. Thus, formality emerged as an important factor for IP length in monolingual speakers but not in HSs, in line with the results by \citet{Comstock_2018}.

The number of PAs was extracted and divided by the number of content words on which a PA could be placed. A binomial generalized linear mixed effects model revealed that on average HSs produced more PAs on content words than monolingual speakers ($z = 6.53$, $p< 0.001$), as also shown in \tabref{chapter12:tab:1}.

\begin{table}[h]
  \centering
  \begin{tabular}{ lcc }
    \lsptoprule
                 & words/IP & PAs/content words\\\midrule
    HSs ($N=53$)    & 2.46 & 0.85 \\
    \text{formal}   & 2.42 & 0.85 \\
    \text{informal} & 2.51 & 0.84 \\
    \addlinespace
    Monos ($N=40$)  & 2.77 & 0.77 \\
    \text{formal}   & 2.55 & 0.77 \\
    \text{informal} & 3.00 & 0.75 \\
    \lspbottomrule
  \end{tabular}
  \caption{Overall intonational features in RuPro}
  \label{chapter12:tab:1}
\end{table}

The architecture of the RUEG corpus also allows for testing of whether age and gender are relevant factors, as they are reported to be crucial in language variation (\cite{Tagliamonte_2016, Nagy_2017}). 
\citet{Zuban} explored whether PA frequency was influenced by the factors age or gender in a reduced prosodically-annotated data set (40 HSs, 40 mono) that was later included in the RuPro corpus. However, age and gender were not found to be relevant for the number of PAs. Interestingly, \citet{Dehé_Rommel_forth} also did not find age to be a relevant factor in Heritage Icelandic (question) intonation.

To sum up, the results show that formality influences PA frequency in productions of monolingually-raised speakers, but not HSs. Further, HSs of Russian do not differentiate IP length according to formality. In informal situations, however, they produce IPs containing fewer words than monolingual speakers. At the same time, they produce more PAs on content words than monolingual speakers. These results are in line with previous studies that report the lack of register differentiation in HSs, possibly caused by the absence of formal instruction in the HL (e.g., \cite{Schroeder_etal_forth, Wiese_etal._2022, Alexiadou_Rizou_Karkaletsou_2022, Comstock_2018}). More frequent PAs produced by HSs have also been reported in studies by \citet{Zuban_Rathcke_Zerbian_2023} and \citet{Zerbian_Böttcher_Zuban_2022} (presented in Sections~\ref{chapter12:section 4.1} and~\ref{chapter12:section 4.2}), which have been interpreted as a general effect of bilingualism. 

Together, the distribution of these intonational features might point to a different overall rhythm in the language. The result that the two speaker groups only differ in pitch accent frequency in the informal communicative situation is also in line with the fluency results by \citet{Böttcher_Zellers_subm}. Thus, this effect of bilingualism can be linked to increased processing costs in bilingual speakers when monitoring two languages. 

Similarly, the results for monolingual speakers can be connected to higher processing costs in formal situations with lower habitual ease, such as when speaking to a policeman (see \sectref{chapter12:section 2} on methodology). A police report might be more detailed, require more careful choice of words, and require a longer planning period than a story presented to a friend, all leading to slower speech and more frequent accentuation in monolingual speakers. 


\section{Intonation at the interfaces} \label{chapter12:section 4}

The intonation systems of HSs and monolingually-raised speakers can also be compared along the semantic dimension; in other words, how intonational events are used in different semantic/pragmatic contexts. Examples are intonational patterns used for the differentiation of sentence types (e.g., declarative versus question), or the expression of information structure or information status (e.g., PA for focus, deaccentuation for given information). Several studies investigate specific aspects of heritage language intonation, such as intonation in declaratives (e.g., \cite{Robles-Puente_2014, Colantoni_Cuza_Mazzaro_2016}), in questions (e.g., \cite{Dehé_2018}), for focus (e.g., \cite{Gries_Miglio_2014, RijswijkDijkstra2017, Kim_2019}), or for a combination of those (e.g., \cite{Rao_2016}).

In this section we will present the results of our studies on intonation used in yes/no questions (\sectref{chapter12:section 4.1}), contrastive focus (\sectref{chapter12:section 4.2}) and discourse linking (\sectref{chapter12:section 4.3}). \citet{Zuban} investigates the syntax and intonation of referents in the order new-given produced by Russian HSs and monolingual speakers of Russian from the same corpus.


\subsection{Intonation of yes/no questions} \label{chapter12:section 4.1}

\citet{Zuban_Rathcke_Zerbian_2023} looked at the prosodic realization of yes/no questions in Russian HSs in Germany and the US and monolingually-raised Russian speakers. Yes/no questions are questions that require as the answer either “yes” or “no” (henceforth abbreviated YNQ). In Standard monolingual Russian, a frequent way of forming a YNQ is through intonation; in other words, declarative clauses and YNQs do not differ in their morphosyntax, but only in their prosodic realization (\cite{Bryzgunova_1980, Svetozarova_1998, Rathcke_2006}). In a YNQ, the verb typically receives the main prominence, as opposed to the sentence-final argument in declaratives. 

Moreover, the main accent is usually falling in declaratives and rising in YNQs (bitonal L*+H or L+H*; \cites{Rathcke_2006}[1619]{Meyer_Mleinek_2006}[93]{Makarova_2003}). Constituents other than the verb are usually not accented \citep{Rathcke_2006}. The rising accent on the verb in Russian YNQs is followed by a low boundary tone at the end of the question if the nuclear accent is not realized on the utterance-final syllable.

In YNQs in English and German, however, the main prominence (monotonal L*) remains on the sentence-final argument and the YNQs end in a rising boundary tone. Investigating heritage Russian in the US and in Germany therefore provides an ideal situation to test the influence of the dominant language. More specifically, \citet{Zuban_Rathcke_Zerbian_2023} explored three questions: (1) how YNQs are produced by HSs of Russian (both in the US and in Germany) as compared to monolingual speakers in terms of location of accent placement (on subject, verb and object), pitch accent type (monotonal versus bitonal) and the choice of final boundary tone (L\% versus H\%); (2) whether speakers of heritage Russian show similar patterns concerning PA location, PA type and BT across different MLs; and (3) if the observed patterns can be attributed to a direct influence from the ML, such as through transfer, or if they should be characterized as innovations.

In their study, 20 adolescents in each of three groups took part: speakers of Russian in Russia (10 female, mean age=17.0, SD=0.58), HSs of Russian in the US (10 female, mean age=16.2, SD=1.53) and HSs of Russian in Germany (11 female, mean age=16.9, SD=0.87). These participants also provided data to the RUEG corpus, although the data for this study were collected in addition to the corpus data. The two populations of HSs share some characteristics: they are 2nd generation speakers who were either born in the US or Germany, respectively, or came there by the age of five. Furthermore, both countries have a large number of Russian-speaking immigrants (\cite{UnitedStatesCensusBureau_2009, StatistischesBundesamt_2022}). Both groups of HSs were comparable in their self-assessed involvement with Russian.

Their task was to read out a list of scripted questions consisting of ten YNQs in total, divided into five transitive questions (SVO) and five intransitive questions (SV). All questions referred to the events of the fictional car accident that was shown in the video that the participants had watched prior to the recording of the experimental sentences. Across the 60 participants, there were 600 YNQs.

All productions were annotated for PA location, PA type, and BT, using surface oriented phonetic labels. The results show that all speaker groups produce the most prominent PA on the verb (for a detailed description of the results including statistical analyses, see \cite{Zuban_Rathcke_Zerbian_2023}). Beyond that, HSs in both countries produce significantly more PAs on subjects and objects compared to monolingual speakers, although monolingual speakers do not fully deaccentuate other constituents as might have been expected based on the literature.

Regarding the PA type, all speaker groups have the same preference in using a monotonal H* on subjects. Whereas monolingually-raised speakers also frequently use a L* on subjects, HSs of Russian use H* nearly exclusively, thus showing less variation of PA type.

For the PA on the verb, it is noteworthy that speakers of all groups use rising bitonal accents (L*+H, L+H*), with the alignment being partly determined by the segmental material available for the realization of the tones.

As for the use of boundary tones, the results show that both monolingual speakers and HSs predominantly produce salient low BTs in Russian YNQs. This is particularly noteworthy for the groups of HSs, for whom we had expected to find more final rises due to a possible influence of language contact. However, HSs in Germany and monolingual speakers produce high BTs (H\%) mainly due to phonetic pressure; that is, when there is less segmental material available for the realization of a low BT following a rising nuclear pitch accent.

It is interesting to note that a group-level difference also emerges in another case of phonetic pressure, namely in the choice of the PA on the verb in SV sentences. Here, HSs in Germany group together with monolingual speakers, whereas HSs in the US show a slightly different pattern, as shown in \figref{chapter12:fig:3}.

\begin{figure}
  \centering
  \includegraphics[width=\textwidth]{figures/Ch12_Figure3_YNQs_montage.png}
  \caption{F0-contours of the SV sentence “Продукты упали?” (/prɐˈduktɨ ʊˈpalʲɪ/ ‘‘Did the groceries fall down?’), produced by a HS in the US (top panel); a HS in Germany (middle panel); and a monolingual speaker (bottom panel)}
  \label{chapter12:fig:3}
\end{figure}

Overall, the results show that HSs produce YNQs in a way comparable to monolinguals in terms of the location and type of a PA on the verb, and also of the use of BTs. Influence from the dominant ML does not manifest itself directly. Nevertheless, HSs of both groups differ from monolinguals in the number of additional PAs on subjects and objects. Furthermore, HSs in the US differ from HSs in Germany and monolingually-raised speakers regarding tonal preferences and the handling of phonetic pressure when little segmental material is available. One important implication of these results is the existence of group-specific strategies to handle phonetic pressure. The prosodic strategies used to resolve phonetic pressure used by HSs in the US cannot be easily attributed to transfer from their ML; rather, these strategies should be considered innovations.


\subsection{Contrastive focus} \label{chapter12:section 4.2}

\citet{Zerbian_Böttcher_Zuban_2022} investigated the prosodic marking of contrastively focused adjectives in heritage Russian spoken in the US, using data from the RUEG corpus. The prosodic expression of focus and contrast within a noun phrase is a particularly interesting domain because languages differ in their prosodic marking of information structure not only at the sentence level but also at the level of noun phrases (e.g., \cite{Ladd_2008}).

English and Russian share the option of prosodically marking contrastive adjectives through pitch accents. In a modified NP, such as \textit{a blue car}, a PA on the noun (indicated by small capitals), as in (\ref{chapter12:example 1a}), can indicate a focus (indicated by brackets and subscript $_F$) only on the noun or on the entire NP. A PA on both adjective and noun, as in (\ref{chapter12:example 1b}), can indicate separate focus on the noun and on the adjective, focus on the NP, or focus only on the noun (with a prenuclear accent on the adjective in the latter two cases; \cite[8]{Ladd_2008}). However, an accent on the adjective (and co-occurring deaccentuation of the noun), as in (\ref{chapter12:example 1c}), can only indicate narrow focus on the adjective. For Russian, the parallel has been reported as illustrated in \REF{chapter12:example 2} (\cites[57]{King_1993}[4--5]{Jasinskaja_2016}).

\begin{exe}
    \ex Prosody and interpretation in modified NPs in English
    \label{chapter12:example 1}
        \begin{xlist}
            \ex a white \textsc{CAR}= [a white car]$_F$\slash a white [car]$_F$
            \label{chapter12:example 1a}
            \ex a \textsc{WHITE CAR} = a [white]$_F$ [car]$_F$
            \label{chapter12:example 1b}
            \ex a \textsc{WHITE} car   = a [white]$_F$ car
            \label{chapter12:example 1c}
        \end{xlist}
    \ex Prosody and interpretation in modified NPs in Russian
    \label{chapter12:example 2}
        \begin{xlist}
            \ex 
            \gll krasnuju \textsc{ZVE}zdočku = [red star]$_F$\slash red [star]$_F$ \\
                red star\\
            \ex \textsc{KRAS}nuju \textsc{ZVE}zdočku =  [red]$_F$ [star]$_F$
            \ex \textsc{KRAS}nuju zvezdočku  = [red]$_F$ star
        \end{xlist}
\end{exe}
        
In addition to this similarity in prosodic focus marking shown in examples (\ref{chapter12:example 1}) and (\ref{chapter12:example 2}), English and Russian differ in the availability of additional syntactic means to indicate contrast. One example is the split construction in Russian, illustrated in (\ref{chapter12:example 3b}), in which a modifying adjective is separated from the head noun \citep[281]{Sekerina_Trueswell_2011}. 

\begin{exe}
    \ex Split construction in Russian (from \cite[282]{Sekerina_Trueswell_2011})
    \begin{xlist}
    \label{chapter12:example 3}
        \ex
        \gll Položite \textsc{KRAS}nuju       zvezdočku      v Poziciju 4. \\
         put         red.\textsc{acc.fem} star.\textsc{acc.fem} in position 4 \\
         \glt `Put the \emph{red} star in position 4.'
    \ex 
    \label{chapter12:example 3b}
    \gll \textsc{KRAS}nuju položite zvezdočku v Poziciju 4. \\
     red.\textsc{acc.fem} put star.\textsc{acc.fem} in position 4 \\
    \end{xlist}
\end{exe}

Thus, the research question was how bilingual speakers of heritage Russian in the US produce NPs with a contrastively focused adjective in their two languages, namely heritage Russian and majority English. The data in \citet{Zerbian_Böttcher_Zuban_2022} form a subset of the RUEG corpus \citep{RUEGcorpus2024}, accessible as “RUEG-EnPro”, which contains the data from the spoken tasks of the elicitation sessions provided by 36 English monolingual speakers (24 adolescents, 12 adults), 40 Russian monolingual speakers (20 adolescents, 20 adults), 60 HSs of Russian in their ML English (30 adolescents, 30 adults) and 53 HSs of Russian in their HL Russian (31 adolescents, 22 adults). All bilingual Russian speakers were either born in the US or moved there before the age of 5. The data were transcribed and annotated for PA location and type.\footnote{\url{https://korpling.german.hu-berlin.de/annis/}}

Because the car accident prompted a contrast between two differently coloured cars, a lexical search was performed on the corpus for the noun \textit{car} immediately preceded by an adjective. The resulting hits were annotated manually for contrastive focus \citep{Götze_Weskott_Endriss_Fiedler_Hinterwimmer_Petrova_Schwarz_Skopeteas_Stoel_2007}. The phrases containing contrastively focused adjectives (henceforth Adj\textsubscript{CF}+N) were extracted from the corpus and were manually annotated for presence/absence of PAs on their constituents.

The results concerning the frequency of occurrence revealed that fewer monolingual speakers of Russian produced Adj\textsubscript{CF}+N than HSs of Russian in their ML and HL and monolingual speakers English (for the detailed results, see \cite{Zerbian_Böttcher_Zuban_2022}). Also, those monolingual speakers of Russian who did produce it, produced fewer instances than HSs of Russian in their ML and HL and monolingual speakers of English. Moreover, there seems to be a language-specific difference in that monolingual speakers and HSs of Russian produced Adj\textsubscript{CF}+N less frequently than HSs in their ML and monolingual English speakers. Across all four speaker groups, Adj\textsubscript{CF}+N is produced more frequently in the formal communicative situation than in the informal situation.

As for accentuation patterns in Adj\textsubscript{CF}+N, the results show that in the English data, the expected preference for a single PA on the adjective prevails across communicative situations and speaker groups. It can also be noted that bilingual speakers produce double accents (accent on both adjective and noun) more often than monolingual speakers. Across the mono- and bilingual speaker groups, there is a higher percentage of double accents in the formal communicative situation as compared to the informal situation.

In the Russian data, the preference for a single PA on the adjective that would have been expected based on the literature only holds for monolingually-raised speakers in the informal situation (shown in \figref{chapter12:fig:5}). HSs show a preference for double accents in both formal and informal communicative situations, exemplified in \figref{chapter12:fig:4}. Double accents also seem to prevail in monolingually-raised speakers in the formal situation, although the number of monolingual speakers is too low for any strong claims. An increased use of double accents can be seen in the formal situation (only monolinguals) and overall in bilingual HSs, similar to the English data.

\begin{figure}
  \centering
  \includegraphics[width=\textwidth]{figures/Ch12_Figure4_ContAdj_HSs_montage.png}
  \caption{F0-contours of the contrastively focused adjectives “первая” (/ˈpʲervəjə/ “first.NOM”) and “белая” (/ˈbʲeləjə/ “white.NOM”), produced by HSs in formal (top panel) and informal situations (bottom panel)}
  \label{chapter12:fig:4}
\end{figure}

\begin{figure}
  \centering
  \includegraphics[width=\textwidth]{figures/Ch12_Figure5_ContAdj_monos_montage.png}
  \caption{F0-contours of the contrastively focused adjectives “вторая” (/ftɐˈrajə/ “second.NOM”) and “синюю” (/ˈsʲinʲʉjʉ/ “blue.ACC”), produced by monolingual speakers in formal (top panel) and informal situations (bottom panel)}
  \label{chapter12:fig:5}
\end{figure}

To sum up, the investigation of the corpus data reveals that Russian speakers (both heritage and monolingual) use the structure Adj\textsubscript{CF}+N less frequently than English speakers (both mono- and bilingual), despite the reported parallel in terms of semantics and prosody. Also, English and Russian speakers differ in their accentuation pattern in Adj\textsubscript{CF}+N. Russian HSs frequently use double accents in Adj\textsubscript{CF}+N, a pattern that indicates a different information structure (see example \ref{chapter12:example 1}b). Across English and Russian, double accents in Adj\textsubscript{CF}+N occur more frequently in formal than in informal situations, and more frequently in bilingual than in monolingual speakers.

One reason for why Adj\textsubscript{CF}+N are used less frequently in Russian than in English might lie in syntax: whereas English has a rigid SVO word order, Russian allows for word order changes for the expression of information structure, such as inversion, dislocation, and split constructions as in Example \REF{chapter12:example 3} (e.g., \cite{Jasinskaja_2016}). Although we did not explicitly check word order for these constructions, other studies on word order in the RUEG corpus have attested word order variations in HSs of Russian, such as left dislocation (\cite{Zerbian_Barabashova_Zuban_subm}; see also \cite{chapters/11}), inversion (OVS) and dislocation (SOV) \citep{Zuban_etal_2021}. Moreover, in a processing study, \citet[296]{Sekerina_Trueswell_2011} found prosody to be a weak cue to contrastiveness in Russian. In a production study, \citet{Kim_2019} also found prosody to be a weak cue to focus in the Spanish of monolingual speakers, but used more by HSs of Spanish in the US.

The increased use of double accents in bilinguals’ heritage Russian and in their English (though to a lesser extent) could again be interpreted as relating to increased processing costs when using two languages (see \sectref{chapter12:section 3}). However, interpreting double accents as a general feature of bilingualism makes the prediction that we would find an increased use across all bilingual speaker groups. A comparative look at the other language constellations in the RUEG corpus, however, shows that the increased use of (double) accents is not found in all bilingual groups. For example, \citet{Böttcher_2021, Böttcher_2022} investigated the prosody of contrastive adjectives in majority English by HSs of Russian and Greek. Whereas the use of double accents was frequently found in the English productions of Russian HSs, this was not the case for Greek HSs. Thus, the frequent production of double accents in our study cannot be related to transfer nor to bilingualism in general, but might be a group-specific pattern.


\subsection{Intonation and discourse markers}
\label{chapter12:section 4.3}

The data collected in the RUEG corpus also allowed for an explorative investigation of the interplay of prosody, syntax, and discourse linking. Intonation can play an important and disambiguating role in discourse linking. The data from the RUEG project provide some first insights from the realization of V3 structures in mono- and bilingual German.

V3 sentences display an adverbial-subject-finite verb order and thus violate the V2 constraint (for a more detailed description of the pattern, see \cite{chapters/11}). An example is given in example (\ref{chapter12:example 4}).

\begin{exe}
     \ex V3 in German (DEbi58MT\_is) \label{chapter12:example 4}\\
        \gll und dann er lässt sein Ball einfach fallen \\
        and then he lets his ball just fall \\
        \glt `and then he just drops his ball' 
\end{exe}

The structures have been described for various contact varieties in north western Europe (e.g., Germany, Sweden, Norway, Denmark, the Netherlands; see \cite{Wiese_2009, Kotsinas_1992, Opsahl_2009, Quist_2000}, and \cite{Meeleen_Mourigh_Cheng_2020}, respectively) and HLs across the globe (e.g., US, Namibia; see \cite{Tracy_Lattey_2009, Wiese_Müller_2018}). V3 is also reported for monolingual German; however, multilinguals appear to make use of V3 more frequently \citep{Wiese_Müller_2018}.

On the level of information structure, \citet{Wiese_Rehbein_2016} identify an order of frame-setter > topic > comment. \citet{Schalowski_2017} argues that the initial adverbial can be either frame-setting (as in \ref{chapter12:example 4}) or discourse connecting (as in \ref{chapter12:example 5}) while \citet{Bunk_2020} claims it can have both functions at the same time (as in \ref{chapter12:example 6}). The ambiguous nature of the initial adverbial makes V3 structures particularly interesting to look at from the perspective of prosody, as prosody might give the interlocutor additional cues as to the interpretation of the initial adverbial. In one of our studies we investigated the prosodic realization of V3 sentences in German and the role of prosody in the disambiguation of information structure \citep{Bunk_Rocker_inprep}.

\begin{exe}
  \ex Discourse connecting (DEbi18MT\_isD)\\\label{chapter12:example 5}
      \gll auf jeden Fall ich bin dann einfach weggegangen\\
      in any case I am then just away.walked \\
      \glt `Anyway I just walked away'


    \ex Frame-setting and discourse connecting (DEbi25MT\_is)\\\label{chapter12:example 6}
        \gll und denn der eine war ziemlich schnell\\
        and then the one was pretty quick \\
        \glt `And then one of them was pretty quick'
\end{exe}

Previous studies have extensively investigated prosodic properties of discourse markers (DMs) and come to different conclusions. While a number of studies highlight that DMs are prosodically and syntactically not integrated into the following utterance (\cite{Jucker_Ziv_1998, Auer_Günthner_2005, Brinton_2010}), others argue that there is no evidence that DMs have specific prosodic features, so they might be prosodically integrated or not \citep{Imo_2012}. Another strand of research indicates that prosodic integration differs between items and is related to the degree of grammaticalization of these items. Particular lexical items lose parts of their meaning as they grammaticalize\slash pragmaticalize (see \cite{Traugott_1995, Auer_Günthner_2005}). This development goes hand in hand with a loss of prosodic prominence, in particular stress (see \cite{Bybee_Pagliuca_Perkins_1991, Hirschberg_Litman_1993, Wichmann_2011, Dehé_Stathi_2016}). \citet{Wichmann_Simon-Vandenbergen_Aijmer_2010} show that \textit{of course} is syntactically more integrated into the following syntagma when it is used as a discourse marker, which is characterized by the lack of stress.

Only a few studies so far investigate the prosodic realization of V3 sentences. However, it seems that prosody can give us hints with respect to the status of the initial adverbial in V3 sentences. In addition, it might help us to better understand the mechanisms of grammaticalization in German in general, as V3 adverbials appear to spotlight this phenomenon. The preverbal area in V3 predominantly entails prosodically light material and does not bear accent (see \cite{Freywald_etal_2015, teVelde_2017}). \Citet{teVelde_2016} states that the preverbal area is realized in one prosodic phrase, while \citet{Breitbarth_2022} finds that the vast majority of V3 structures exhibit a prosodic boundary between the initial constituent and the subject. However, both studies look at very specific lexical items. \Citet{teVelde_2016} restricts his analysis to temporal adverbials and \citet{Breitbarth_2022} excludes discourse organizing \textit{dann} ‘then’, even though \textit{dann} has been described to be particularly frequent in V3 sentences \citep{Wiese_Müller_2018}. Hence, the studies to date do not distinguish between frame-setting and discourse linking adverbials in V3 with respect to prosodic realization.

The empirical basis of our study were two corpora as detailed in \sectref{chapter12:section 2.2}: the German subcorpus of the RUEG corpus \citep{Wiese_etal._2021}, i.e., monolingual and bilingual speakers in Germany and bilinguals in the US, and the corpus by \citet{Rocker_2022}, comprising data of heritage Low German in the US. The corpora served the purpose of including different contact scenarios in order to investigate V3 under different conditions of language contact.

The data were annotated for prosodic boundaries based on \citet{Himmelmann_Sandler_Strunk_Unterladstetter_2018} and \citet{Breitbarth_2022}. The presence of a boundary indicated prosodic separation while the absence indicated prosodic integration. Based on the observations concerning prosodic integration and function described above (i.e., the higher the level of prosodic integration, the more likely the discourse marker functions of the adverbial), we expected V3 adverbials to be more integrated if they were used as discourse markers (= no prosodic boundary) and less integrated if they were used as frame-setters (= prosodic boundary).

Even though figures were rather low (79 V3 sentences in RUEG, 180 V3 sentences in heritage Low German), \citet{Bunk_Rocker_inprep} found the following tendencies that need further investigation with a larger data set or experiments. Overall, they found that the German speakers produced V3 with and without boundaries in all corpora. However, the vast majority of the structures did not have a prosodic boundary. If boundaries occurred, they tended to occur after frame-setting adverbials, not discourse connective adverbials. There were only very few cases where discourse connective adverbials were followed by a prosodic boundary. This was particularly true for the most frequent adverbial \textit{dann} ‘then’. This result points to the development of \textit{dann} ‘then’ into a discourse marker, as highlighted by prosodic integration. The patterns were similar across all groups; however, monolingual speakers in Germany used V3 less frequently than the other groups. HSs of German in the US almost always used \textit{und dann} ‘and then’ instead of \textit{dann} ‘then’ in discourse marking function, indicating that these two elements form a unit that is schematized and conceptualized as a grammatical or discourse-pragmatic pattern (see \cite{Haspelmath_2002}). V3 structures in heritage Low German more often had a prosodic boundary. However, the majority of V3 sentences with initial \textit{(und) dann} ‘(and) then’ were again produced without a prosodic boundary. In sum, the study showed that the default case in our data seems to be prosodic integration (i.e., lack of a boundary) in V3 sentences, even though V3 is also produced with a boundary. These findings indicate that V3 adverbials are at different stages on a grammaticalization path. Particular items, such as \textit{(und) dann} ‘(and) then’ are more grammaticalized than others, such as \textit{auf einmal} ‘suddenly’. When boundaries were used, they marked frame-setting adverbials.


\section{Perception of intonation by heritage speakers} \label{chapter12:section 5}

Whereas production is often investigated in heritage language phonology, perception or comprehension, especially of prosody, is researched less often (\cite{Polinsky_Scontras_2019}, footnote 6; but see \cite{Sekerina_Trueswell_2011} and the short summary in \cite[158--162]{Polinsky_2018}).

HL listeners have been claimed to have a perceptual advantage \citep[11]{Polinsky_2018} due to the exposure to the HL in childhood which gives them an advantage in phonological categorization and perception of contrasts, particularly in segmental phonology. Prosodic distinctions have been claimed to be much more subject to change and variation.

Prior work on the processing of prosody and word order in mono- and bilingual Russian listeners has attested only a weak effect of contrastive accent \citep{Sekerina_Trueswell_2011}. In addition, \citet{Sekerina_Trueswell_2011} found that it is relevant for the processing of prosody whether a semantic contrast is explicitly expressed or not. \citet[161f]{Polinsky_2018} suggests that “HSs have a selective prosodic advantage over native speakers: [they] are sensitive to stronger, more salient prosodic cues” but “[t]he prosodic defaults in heritage speakers may be more general than the defaults established by native speakers”.

In an online perception experiment, \citet{Zerbian_Barabashova_2023} tested the acceptability of different accent patterns in adjective+noun phrases with semantic focus on the adjective (Adj\textsubscript{CF}+N) by mono- and bilingual listeners of Russian. The sentence structure was SVO with a verb in active voice. Both naturally produced and lab-recorded Adj\textsubscript{CF}+N phrases were presented auditorily, as part of a discourse context which evokes semantic contrastive focus on the adjective (in bold), as in (\ref{chapter12:example 7}).

\begin{exe}
    \ex
    \gll Èta belaja  mašina  vrezalas’ v      \textit{sinjuju} mašinu. \\
        this white car crashed into blue car \\
   \glt `This white car crashed into a blue\textsubscript{CF} car.'
\label{chapter12:example 7}
\end{exe}

Listeners had to evaluate how natural the presented Adj\textsubscript{CF}+N phrases sounded. The accent patterns varied with respect to accent on the noun, on the adjective or on both. Moreover, the Adj\textsubscript{CF}+N stimuli represented either the first or second noun phrase in the sentence (i.e., encoding the contrast explicitly only if occurring as the second noun phrase, as in \REF{chapter12:example 7}).

The aim was to investigate whether mono- and bilingual listeners rate different accent patterns in Adj\textsubscript{CF}+N differently with respect to naturalness. Furthermore, it was asked whether the position (i.e., whether the noun phrase is mentioned first or second in a sentence) leads to a difference in the naturalness ratings. The ratings of accent patterns on Adj\textsubscript{CF}+N were predicted to depend on three factors. First, they were predicted to depend on accent location such that adjective accents would be rated high on the naturalness scale by both mono- and bilingual listeners and clearly different in acceptability from accents on nouns. Second, based on production data (see Section  \ref{chapter12:section 4.2}) they were predicted to depend on the bilingualism of listeners such that double accents (i.e., on both adjective and noun) would be rated as more natural more often by bilingual than by monolingual listeners. And third, they were predicted to depend on the NP's position in the sentence such that double accents would be rated as more natural if they occurred on the first noun phrase of a sentence rather than on the second, because in the second noun phrase the contrast is linguistically explicit.

The online study consisted of two experimental parts. Experimental part 1 elicited naturalness ratings for Adj\textsubscript{CF}+N with differing accent patterns, namely an accent on the adjective, on the noun or on both (double accent), using a scale from 1 (= not at all natural) to 7 (= very natural). Experimental part 2 explored the role of explicit linguistic contrast for naturalness ratings of Adj\textsubscript{CF}+Ns as first and second occurrence in an utterance, respectively.

For both experimental parts, original productions of Adj\textsubscript{CF}+N from the RUEG corpus were used (see \sectref{chapter12:section 2.1} above). The narrations described an accident involving two differently coloured cars, thus providing the context for a semantic contrastive focus on the colour adjective. As a control condition, listeners were additionally presented with lab-produced recordings of Adj\textsubscript{CF}+N phrases, with the same three accent patterns: on the adjective, on the noun, and on both. 

Participants were recruited via the platforms Prolific and Mechanical Turk. Requirements specified via the platform included the US or Germany as location, Russian as L1, and Russian and English or German as fluent languages. Twelve participants were either born to Russian-speaking parents in the US or Germany or immigrated to the respective countries when they were younger than 5 years old and are thus considered Russian HSs. Sixteen monolingually-raised Russian speakers who lived in Russia and were recruited via social networks served as a comparison group. 

With the lab-recorded stimuli, both listener groups rated an accent on the adjective (HSs: mean=5.4, mono: mean=6.0) \footnote{Further details on the spread of the data are given in \citet{Zerbian_Barabashova_2023}.} as more natural than a double accent (HSs: mean=4.0, mono: mean=3.6) or an accent on the noun (HSs: mean=3.9, mono: mean=4.2). For the stimuli taken from the corpus, heritage listeners rated accents on adjectives on average higher (mean=5.3) than double accents (mean=4.9) or accents on nouns (mean=4.8), though less clearly so than in the lab-recorded stimuli. For monolingual listeners, all accent patterns were rated as similarly highly natural.

As for the role of explicit linguistic contrast (whether Adj\textsubscript{CF}+N was first or second occurrence), monolingual as well as heritage listeners judge accents on adjectives as more natural than double accents both in first and second position. However, heritage listeners rated double accents significantly more natural in first position (mean=4.7) than in second (mean=3.9). For monolingual listeners, the position did not matter.

The results for the lab-recorded stimuli were in line with the literature for both listener groups, namely that adjective accent is judged significantly more natural than an accent on the noun or double accent. Thus, both listener groups were shown to be sensitive to accent and its relation to information structure in Russian.

For the stimuli taken from natural productions, all accent patterns sounded equally (highly) natural, at least to monolingual listeners. The naturalness and greater variability of the data might have indirectly led to convergence of judgments in monolingual speakers. This might be even more so due to the generally weak load of accent in Russian \citep{Sekerina_Trueswell_2011}. Bilingual heritage listeners, on the other hand, might benefit from a perceptual advantage that leads to ``better-than-native perception'' (e.g., \cite{Chang_2016}). HSs have more familiarity with variation and accents and might thus adapt more quickly to varied speech.


\section{Discussion} \label{chapter12:section 6}

What emerges across our studies on intonation presented in this chapter is that we find evidence for various differences between monolingually-raised speakers and HSs. For intonation used at the linguistic interface with semantics and pragmatics, such as in contrastive focus or discourse linking, this is in line with the Interface Hypothesis (e.g., \cite{Sorace_2011}) which predicts phenomena at linguistic interfaces to show variation in language contact.

Variation is found along different dimensions of intonation (see the LILt model by \cite{Mennen_2015}). For example, the different phonetic implementation of boundary tones in YNQs by Russian HSs in the US as opposed to Germany can be considered variation along the realizational dimension. The increased frequency of pitch accents across the productions in the corpus (\sectref{chapter12:section 3}) as well as in the additional experimental items of YNQs (\sectref{chapter12:section 4.1}) point to a difference along the frequency dimension. At the same time, the latter can also be considered variation along the semantic dimension as pitch accents occur on otherwise deaccented constituents and thereby might take on a different semantic meaning. The same reasoning can be applied to the placement of accents in modified noun phrases with contrastively focused adjectives (\sectref{chapter12:section 4.2}). The use of phrasing (\sectref{chapter12:section 4.3}) to potentially distinguish frame-setting adverbials from discourse-linking adverbials is another instance of the semantic dimension of intonation.

It is noteworthy that the frequency difference concerning the occurrence of pitch accents emerges across different data sets (Sections \ref{chapter12:section 3}, \ref{chapter12:section 4.1}, \ref{chapter12:section 4.2}). Similarly, an effect of formality emerges across different data sets (Sections \ref{chapter12:section 3}, \ref{chapter12:section 4.2}). \citet{Zerbian_Böttcher_Zuban_2022} found that in the cases investigated HSs of Russian in the US accentuate both a contrastive adjective and noun. Monolingually-raised speakers of Russian were found to place a single pitch accent on the adjective in the informal situation, but not in the formal one where they produced double accents on both adjective and a noun. Thus, there is a possible formality effect for monolingually-raised speakers (note the few realizations though), but not for the HSs. The same trend was found for overall intonational features (\sectref{chapter12:section 3}): shorter IPs in formal than in informal settings in monolingually-raised speakers but a levelling of registers for this feature in HSs \citep{Zerbian_Zuban_Klotz_2024}. Our findings are in line with other studies that report register-levelling processes in HSs in other linguistic domains (e.g., \cite{Colantoni_Cuza_Mazzaro_2016, Alexiadou_Rizou_Karkaletsou_2022, Schroeder_etal_forth}). Register-levelling might be due to a lack of exposure to formal registers. The potential influence of exposure to formal registers on prosodic features, such as pitch and speech rate, is detailed in \citet{Rao_etal_2020}.

In the cases reported here, the observed differences cannot clearly be traced back to cross-linguistic influence from the majority language. For boundary tones in YNQs, HSs of Russian in the US seem to implement them closest to what has been reported for monolingually-raised speakers of Russian, although the latter (and HSs in Germany) were not found to strongly prefer that pattern themselves. Some HS groups might be more conservative in their linguistic features (e.g., \cite[129]{Polinsky_2018}), overemphasizing features of the HL.

What also emerges across all studies on intonation is that HSs do not behave completely differently compared to monolingually-raised speakers. Rather, they show certain patterns~-- these could be innovative or conservative patterns~-- more frequently than monolingually-raised speakers. This is in line with considering heritage and monolingually-raised speakers as part of the same native speaker continuum (see \cite{Wiese_etal._2022}), albeit situated at different places along this continuum.

Large-scale, multilingual heritage language corpora like the one of the Heritage Language Variation and Change Project (HLVC) \citep{Nagy_2011} or the RUEG corpus provide opportunities to address questions of heritage language grammar beyond individual language pairings. For example, the RUEG corpus allows differentiating between the influence of transfer, bilingualism, or group-specific patterns in the linguistic phenomena under investigation. Thus, one direction of future research is to investigate whether general prosodic patterns observed with Russian HSs are also attested in other HS groups. If, for example, smaller IPs and an increase in pitch accents is a general feature of bilingual speakers, then we would expect to find it also in Greek and Turkish HSs as compared to monolingually-raised speakers. A first comparison along these lines for the intonation of Adj\textsubscript{CF}+N sequences by mono- and bilingual speakers of English does not suggest a unified pattern of increased double accents across different groups of HSs (see \sectref{chapter12:section 4.2}). Thus, this seems to make it a group-specific pattern in this particular instance, even if more accents are often reported for bilingual speakers. The RUEG corpus affords the unique situation to have comparable data available to test such hypotheses across different language pairings. The findings can inform models on the role of transfer, bilingualism and group-specific innovations in heritage language grammar. 

Furthermore, the RUEG corpus also allows testing hypotheses about prosodic features across formal and informal communicative situations. Qualitative research will allow for a better understanding of some specific results of our studies. For instance, shorter IPs and more frequent accentuation in formal situations by monolingually-raised speakers of Russian might be connected to differences in the discourse structure in the two communicative situations.

The work reported in this chapter thus contributes to theories of language contact, such as the L2 Intonation Learning theory (LILt, \cite{Mennen_2015}) or the Interface Hypothesis \citep{Sorace_2011}. The results help to identify stable and dynamic features of intonation in HSs. Furthermore, when HSs and monolinguals showed differences in intonation, possible sources behind these differences were discussed.


\section*{Acknowledgments}
The research reported in this chapter was funded by the German Research Foundation (DFG) as part of the Research Unit \textit{Emerging Grammars in Language Contact Situations: A comparative approach} (FOR 2537) in project P8 (grant number: 313607803). We thank all research assistants who contributed to the work reported in this chapter either by data collection, data annotation and/or technical help. These are Yuliia Ivashchyk, Nash Whaley, Daria Alkhimchenkova, Erica Conti, Myrto Rompaki, Birte Pravemann. We further thank Lea Coy for help with pre-publication formatting. We are also grateful to the anonymous external reviewers and the RUEG-internal reviewers Johanna Tausch, Sofia Grigoriadou, and Martin Klotz for constructive feedback. Last but not least, a special thanks to the entire RUEG team and the Mercator fellows for fruitful discussions and guidance.


\printbibliography[heading=subbibliography, notkeyword=this]
\end{document}
