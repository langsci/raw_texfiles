\documentclass[output=paper,colorlinks,citecolor=brown]{langscibook}
\ChapterDOI{10.5281/zenodo.15775173}
%This is chapter 8

\author{Christoph Schroeder\orcid{0000-0003-1188-7746}\affiliation{University of Potsdam} and Kateryna Iefremenko\orcid{0000-0003-3711-0935}\affiliation{University of Potsdam; Leibniz-Centre General Linguistics} and Kalliopi Katsika\orcid{0000-0002-6736-4963}\affiliation{University of Kaiserslautern-Landau} and Annika Labrenz\orcid{0000-0002-6235-9321}\affiliation{Humboldt-Universität zu Berlin} and Shanley E. M. Allen\orcid{0000-0002-5421-6750}\affiliation{University of Kaiserslautern-Landau}
   }
   
\title[Clause combining in narrative discourse]
      {Clause combining in narrative discourse: A contrastive analysis across heritage and majority languages}

\abstract{In this chapter, we analyze how monolingual and bilingual speakers combine clauses in narratives. We investigate both languages of bilingual speakers, that is, heritage and majority languages of speakers in Germany and the US with Russian, Greek or Turkish as heritage languages and German or English as majority languages. We ask to what extent differences and similarities between the varieties can be related i) to the typological structures of the languages, ii) to communicative situations within which the narratives were produced, and iii) to language contact. Our findings indicate that the typological differences between the
languages in question regarding preferences for subordination strategies are preserved also in the language contact settings. Furthermore, we do not find much evidence of language contact
effects in either the majority or heritage languages. However, the findings suggest that communicative situations play a crucial role. In instances where differences between heritage and majority varieties of the investigated languages are observed, they arise mainly due to register levelling in the heritage languages of heritage speakers.
\keywords{clause combining, narratives, heritage language, majority language, register levelling}
}



\IfFileExists{../localcommands.tex}{
   \addbibresource{../localbibliography.bib}
   \usepackage{langsci-optional}
\usepackage{langsci-gb4e}
\usepackage{langsci-lgr}

\usepackage{listings}
\lstset{basicstyle=\ttfamily,tabsize=2,breaklines=true}

%added by author
% \usepackage{tipa}
\usepackage{multirow}
\graphicspath{{figures/}}
\usepackage{langsci-branding}

   
\newcommand{\sent}{\enumsentence}
\newcommand{\sents}{\eenumsentence}
\let\citeasnoun\citet

\renewcommand{\lsCoverTitleFont}[1]{\sffamily\addfontfeatures{Scale=MatchUppercase}\fontsize{44pt}{16mm}\selectfont #1}
  
   %% hyphenation points for line breaks
%% Normally, automatic hyphenation in LaTeX is very good
%% If a word is mis-hyphenated, add it to this file
%%
%% add information to TeX file before \begin{document} with:
%% %% hyphenation points for line breaks
%% Normally, automatic hyphenation in LaTeX is very good
%% If a word is mis-hyphenated, add it to this file
%%
%% add information to TeX file before \begin{document} with:
%% %% hyphenation points for line breaks
%% Normally, automatic hyphenation in LaTeX is very good
%% If a word is mis-hyphenated, add it to this file
%%
%% add information to TeX file before \begin{document} with:
%% \include{localhyphenation}
\hyphenation{
affri-ca-te
affri-ca-tes
an-no-tated
com-ple-ments
com-po-si-tio-na-li-ty
non-com-po-si-tio-na-li-ty
Gon-zá-lez
out-side
Ri-chárd
se-man-tics
STREU-SLE
Tie-de-mann
}
\hyphenation{
affri-ca-te
affri-ca-tes
an-no-tated
com-ple-ments
com-po-si-tio-na-li-ty
non-com-po-si-tio-na-li-ty
Gon-zá-lez
out-side
Ri-chárd
se-man-tics
STREU-SLE
Tie-de-mann
}
\hyphenation{
affri-ca-te
affri-ca-tes
an-no-tated
com-ple-ments
com-po-si-tio-na-li-ty
non-com-po-si-tio-na-li-ty
Gon-zá-lez
out-side
Ri-chárd
se-man-tics
STREU-SLE
Tie-de-mann
}
   \boolfalse{bookcompile}
   \togglepaper[8]%%chapternumber
}{}

\begin{document}
\lehead{Christoph Schroeder et al.}
\maketitle

\section{Introduction}
\label{sec:schroederetal:1}
In this chapter, we ask how monolingual and bilingual speakers, with different linguistic repertoires, combine clauses in narrations. We analyze both languages of bilingual speakers, that is, heritage and majority languages of speakers in Germany and the US with Russian, Greek or Turkish as heritage languages and German or English as majority languages. 
Generally speaking, distributions of clause combining strategies in a given language are co-determined by several parameters. On the one hand, typological properties of the language provide a predisposition in a certain direction; for example, a language prefers either finite or non-finite subordination based on its typological structure (see \sectref{sec:schroederetal:1.1} for typological discussion and \sectref{sec:schroederetal:1.2} for the strategies of the languages in question here). At the same time, preferences for clause combining strategies emerge in different varieties of the language, and this includes, quite centrally for us, varieties arising from situational parameters such as (in)formality and mode (spoken/written) (see \sectref{sec:schroederetal:1.3}). We furthermore assume that the choice of text type (argumentation, narration, description, report, ...) and possibly also speech acts within text types (descriptive, evaluative, text organizing, …) affect clause combining strategies. Therefore, we concentrate on one particular text type: narrations (see \sectref{sec:schroederetal:1.4}). And last but not least, in our data, the bilingualism of speakers has to be considered as a specific constellation which may have an impact on the strategies used, through language contact or otherwise (see \sectref{sec:schroederetal:1.5}). 

The data we analyze were collected using the same method across languages and speaker groups (see \sectref{sec:schroederetal:2}), and were controlled for (in)formality and mode. This allows us to investigate the factors listed above. Hence, in the chapter we explore whether there are similar dynamic patterns in clause combining strategies among the speakers of the five languages under research. Our research question is the following: To what extent are potential differences and similarities between the varieties related to i) the typological structures of the languages, ii) communicative situations (comm-sits) of the narratives, and iii) language contact.   Further elaboration on each of these points will be provided below. 

Our study is exploratory in nature. That is, we aim to discuss tendencies that we observe based on the analysis presented in figures. We believe that observations will offer a more holistic understanding of the issue under investigation and identified patterns can later be tested more rigorously with statistical methods.

\subsection{Structural and typological issues of clause combining} \label{sec:schroederetal:1.1}

In this chapter, we adopt a broad notion of clause in the sense of the use of the term by \textcite[182]{lehmann1988towards}; in other words, any syntagma containing a predication is a clause. The minimal requirement for this is a verbal form (except for nominal clauses).
Our terminology with respect to the structural aspects of clause combining is as follows: 

\begin{enumerate}

\item \textit{Independent clauses} are clauses that are syntactically independent. 

\item \textit{Matrix clauses} are clauses which embed a subordinate clause.

\item \textit{Subordinate clauses} are clauses that are dependent on a matrix clause in the sense that they are embedded in it, that is, they form a constituent within a matrix clause. Subordinate clauses can be finite or non-finite, and they can be reduced, meaning that arguments of the verb, further optional complements, and/or aspecto-temporal or other clausal properties are suppressed. 

\item \textit{A finite clause} is one with a finite verb (i.e., a verb that has verbal agreement morphology). A non-finite clause is a clause formed with a non-finite verb (i.e., a verb that has no verbal agreement morphology).\footnote{We are aware of the fact that definitions of (non-)finiteness are always disputable. This definition is meant to apply to the languages that are at issue here.}

\item \textit{Extended clauses} consist of a matrix clause and one or more subordinate clauses; simple clauses consist of only an independent clause.

\item Clause combining strategies can be either \textit{hypotactic }(subordinated) or \textit{paratactic} (coordinated).\footnote{We take as a starting point \posscitet[1]{haspelmath2007language} definition of coordination: “The term coordination refers to syntactic constructions in which two or more units of the same type are combined into a larger unit and still have the same semantic relations with other surrounding elements.”}

\item Paratactic clause combining can be \textit{syndetic} (i.e., by means of an overt connector) or \textit{asyndetic} (i.e., without an overt connector). 

\end{enumerate}

Languages typically make use of a spectrum of clause combining strategies (see, for example, the typological studies by \textcite{haspelmath2004}. In principle, we understand the distribution of clause combining strategies as ordered on continua, based on \textcite{lehmann1988towards}. Lehmann develops three continua of clause linkage (or, as we mostly call it here, clause combining). These are as follows: i) the continuum of integration of one clause into the other, ii) the continuum of the expansion vs. reduction of the clauses, and iii) the continuum of the mutual isolation vs. linkage of the two clauses. We concentrate here on the first and the second continuum. One pole is that of parataxis, where two clauses are coordinated and are syntactically independent from each other. The other is that of hypotaxis, where two clauses are in a dependency relation to each other in the sense that one of the clauses either lies within the temporal, aspectual, modal or negative scope of its matrix clause or  has a syntactic function, in that it is embedded in the matrix clause.\footnote{Some works distinguish “chained clauses” from “subordinate clauses”, where “chained clauses” are dependent only in the sense that they lie within the TMA or negation scope of the matrix clause but have no syntactic position in it, see \textcite{sarvasy2020acquisition}. In the languages at issue here, this distinction might apply to certain clause types in Turkish \parencite{ogel2020clause}, which in our approach are subordinate adverbial (converb) clauses, see also \textcite{iefremenko2021converbs}.} Languages may prefer finite subordination on the hypotactic pole, as is the case with English, German, Russian and Greek, or they may prefer non-finite subordination, as is the case with Turkish (see below). Whatever means the individual languages prefer, the relationship between finite and non-finite appears to be directed in the sense of the second continuum \textcite{lehmann1988towards} develops, namely, that of reduction. In the opposition between finite and non-finite subordination in a given language (at least of the five languages which are at issue here), non-finite clauses are typically more reduced than finite subordinated clauses, or at least allow more reduction, and the non-finite subordination type may move up further towards the pole of reduction by getting “stripped” of further characteristics of verbality (e.g., binding arguments, a subject other than that of the matrix clause, marking for tense or aspect). Distributions of clause combining strategies in linguistic practice can be ordered along these continua. 

\subsection{Clause combining strategies of each of the languages involved} \label{sec:schroederetal:1.2}
In German, English, Greek and Russian, subordination is realized mainly by finite right-branching clauses with a preposed subjunctor. Subjunctors in relative clauses have pronominal character, though non-pronominal subjunctors are also possible, as are combined subjunctors. Finite adverbial subordinated clauses are introduced by subjunctors that define the nature of the temporal, spatial or logical relation to the matrix clause. Finite complement clauses are headed by various complementizers; specific matrix clause predicates (e.g.\textit{verba dicendi}) allow complement clauses without an (overt) complementizer. 

All four languages also have non-finite means of subordination, albeit to different extents: 

English has reduced infinitival complement clauses, where the infinitive combines with the preposition \textit{to}. It also has non-finite relative clauses which are postposed to their head noun: \textit{the person to ask}. Furthermore, English has a non-finite verb-based form in \textit{-ing}. This can be an attributive participle and an adverbial gerund \parencite{nedjalkov1995some,kortmann1995adverbial}. Clauses formed with this non-finite verb allow the use of connectives (e.g.\textit{ while going to school, …}), they also allow the binding of direct objects and adverbial phrases, and can, in the form of “absolute” small clauses, also have subjects that are different from the matrix clause (e.g.\textit{ Dodo joined him, two laden bellboys following}, see \textcite [217]{kortmann1995adverbial} and the discussion therein). \textit{-ing-}forms have aspectual (perfective and continuous) differentiations. 

\begin{sloppypar}
German also has reduced infinitival clauses. These may be complement clauses, where the infinitival verb combines with the preposition \textit{zu}, but may, with \textit{verba dicendi} in the matrix clause, also stand without the complementizer. In addition, adverbial infinitive clauses may be formed with  \textit{um zu}. Furthermore, German has two participle forms, one in the present tense (\textit{Partizip I}), one perfective (\textit{Partizip II}). Both forms can form adverbial and relative clauses. When forming relative clauses, the participles are preposed to their head noun and inflect like adjectives. The German participles are not only much less frequent than their English counterparts, but the clauses they form are also more restricted in use; they can only be same-subject and usually consist only of the non-finite verbal form (cf. \cite{book}, 74f., 241f.).
\end{sloppypar}

Russian also has infinitives that form complement clauses \parencite{shagal2022multifunctionality}  and can form adverbial non-finite clauses \parencite{madariaga2011infinitive}. In the latter function they are usually used together with such connectors as (\textit{dlya togo}) \textit{čtoby} ‘(in order) to’, \textit{daby} ‘to’. Russian furthermore has adverbial participles \parencite{kuprii︠anova2002}, sometimes referred to as ``gerunds'' or ``verbal adverbs''. \textit{Deepričastije} is “a bare, non-finite verb phrase lacking a subject NP and headed by a particular type of verb form with adverbial features supplied by the adverbial participle suffix” \parencite[245]{babby1998syntax}. Non-finite relative clauses in Russian, also referred to as participles (in Russian linguistics usually called \textit{pričastije}), inflect for gender and number and also express aspect \parencite{kuprii︠anova2002}. 


Greek has no infinitival complements \parencite{joseph2020greek}. It does, however, have a complement type where a finite form of the verb in subjunctive forms a clause introduced by the particle \textit{na} (‘to’) and is inflected for person, number, tense and aspect \parencite{joseph2020greek}. Greek also has non-finite verbal forms, which are formed by attaching the affix \textit{-ondas} to the stem of the verb (e.g. \textit{perpat-ondas} ‘walk-ing’). Gerunds have no tense and agreement features and they cannot bind arguments, thus gerundial/converbal clauses in Greek are highly reduced \parencite [see][87]{roussou2000left}. They do have a temporal interpretation however, and can either be anterior or simultaneous to the main clause event, but never posterior \parencite[137]{tsimpli2000gerunds}. Contrary to English gerunds, Greek gerunds may not be introduced by any type of connective. 

As for Turkish, we already hinted at the fact that subordination is realized mainly by means of non-finite forms. Complement clauses are clausal nominalizations, relative clauses are formed with participles, and adverbial clauses are formed with converbs or other non-finite verbs which are combined with postpositions. Non-finite subordination is preposed to its governing structure, so Turkish hypotaxis is left-branching \parencite[267]{johanson1992strukturelle}. Non-finite subordinate clauses in Turkish can be fully expanded; there are certain adverbial clause forms, certain nominalizations and certain relative clauses which do allow independent subjects, and others which do not (see \cite{iefremenko2021converbs} for details). Turkish also has a few means of finite subordination with non-pronominal subjunctors; these form adverbial clauses and complement clauses. Finite complement clauses without an (overt) complementizer are also possible \parencite{kerslake2007alternative}. 

Summarizing this brief overview, we see that four of the languages involved use finite subordination as the dominant strategy for hypotactic linking; one language, Turkish, prefers non-finite subordination. At the same time, there seem to be differences in the breadth of means for non-finite subordination among the languages that prefer finite subordination. Greek appears to be most restricted, in that non-finite clauses are highly reduced in form, in function, and with regard to the expansion they allow. German is a little richer as it has also non-finite complementation, but it also is fairly strict with regard to the expansion of non-finite clauses. Russian and English are broader in their employment of non-finite forms, both in terms of the variety of forms and functions, as well as with regard to lesser restrictions imposed on their expansion. And Turkish clearly has the broadest portfolio of non-finite clauses, in all aspects, but is on the other hand more restricted with regard to finite subordinate clauses. Overall, degrees of intensity and breadth of means for non-finite subordination in the languages at issue can be ordered the following way: 

\begin{quote} Turkish > Russian / English > German > Greek.\end{quote}

\subsection{Situational parameters and clause combining} \label{sec:schroederetal:1.3}

The typological overview and the descriptions of the individual languages shows that clause combining strategies in individual languages are variable, in that speakers can use a variety of strategies in a given inter-propositional relation. We assume that the distribution of strategies used in each language is not random but that, within each language, different varieties may prefer different clause combining strategies. This includes also varieties arising from different comm-sits (“registers”). In this context, \textcite{chafe1976givenness}’s early work on “information packaging”, as well as investigations of discourse structure (e.g., \citeauthor{miller1998spontaneous}, \citeyear{miller1998spontaneous}) and register-oriented approaches \parencite{biber2011should,biber2019register}  show that in a given comm-sit, the choice between a strategy which tends more towards the paratactic and/or less reduced pole or one which tends more towards the hypotactic and/or more reduced pole depends on situational parameters. In ad-hoc comm-sits of higher context-dependence (e.g., informal situations), the frequency of paratactic linking structures increases, and in comm-sits of higher context-independence (e.g., formal situations), the frequency of hypotactic linking structures increases. This has also been shown, or at least discussed, for most of the languages we look at; for instance, see \textcite{kozina2011stilistika}  for Russian, \textcite{schroeder2002structure,schroeder2016clause} for Turkish, also under the heading of “academic language” for English in \textcite{schleppegrell2004language}  and under \textit{“Bildungssprache”} for German \parencite{gogolin2006bilingualitat,gogolin2013mehrsprachigkeit,haberzettl2016bildungssprache}. Related to this is the claim that embedded clauses indicate syntactic complexity \parencite{sanchez2017syntactic,housen2019multiple}, and they are proposed as diagnostic criteria to determine competence in first or second language acquisition in language testing,\footnote{For example the German pre-school language test “Lise-DaZ”, see \textcite{schulz2011}.}  or to determine complexity in written texts \parencite{biber2011should}. On the other hand, \textcite{martynova2024use} conclude that embedding rate in Russian is not an appropriate measure of complexity.

\subsection{Narration} \label{sec:schroederetal:1.4}

As highlighted in the introduction, in addition to typological and situational factors, the selection of text type and the specific speech acts within that text type also impact the strategies employed in clause combining. Therefore, this chapter delves into the examination of structures within narrations.
Narrations have three levels. The first and broadest level is that of text type. Narrations are a particular text type: A speaker conceptualizes a sequence of events that they experienced (e.g., participated in, saw, read, imagined) as meaningfully coherent and worth telling, and articulates this linguistically in the form of a coherent text \parencite{labov1997narrative}. The narrations we examine in the chapter have a particular macro structure, consisting of an opening, a narrative body, and a closing.{\interfootnotelinepenalty=10000\footnote{We do not claim this macro structure to be a universal ``story grammar''; see the respective critical discussion of \textcite{labov1997narrative}  in \textcite{lambrou2005story}, \textcite{andrews2013doing}, amongst others.}}  
The second level of narration is the narrative body, where the narration of a plotline (i.e., a meaningfully knitted sequence of events) is dominant. It must be made clear here that the narrative text body does not consist only of iconically sequentialized descriptions of events. Woven into the narrative are also evaluative propositions, propositions that organize the text itself, dialogic interchanges with an interlocutor, and direct or indirect speech, all of which create a particular stance reflecting the personal experience of the speaker \parencite{lehmann2023guidelines}. 
The specific composition of the linguistic forms within the narrative body create the narrative structure. In the outline of their crosslinguistic developmental study “Relating events in the narrative”, \textcite [19] {berman2013research}  distinguish here between five categories of functions that these forms may express: Temporality, Event Conflation, Perspective, Narrative Type and Connectivity. Connectivity, which is in the focus of this chapter, is described as ``knitting the fabric'' of narrative discourse by means of syntactic conjunction and subordination” (ibid.). 
We investigate connectivity in the narrative body of the texts. In doing so, we assume that event descriptions must far outweigh other speech acts in narrations in completed language acquisition of adolescents and adults, in order to create a “well-formed” narration \parencite [40] {berman2013research}. But we nevertheless assume that the degree to which other speech acts are interwoven bears a relation to mode (spoken, written), formality (informal, formal), the culture of story-telling in a particular language or society, and personal style \parencite{lambrou2005story,andrews2013doing}. 

\subsection{Language contact and bilingualism} \label{sec:schroederetal:1.5}

The question we are pursuing in this chapter requires taking into account yet another variable, namely that of the bilingualism of the speakers. This brings us, on the one hand, to the issue of change through language contact. Here the notion of “convergence” is important to us. We understand it as the structural approximation of languages to each other in language contact, which can express itself for example in the increased frequency of structures existent in both languages \parencite{matras2009,matras2020language,grant2020contact}.

However, convergence must not necessarily be understood as mutual approximation between the languages involved. The bilingualism in focus here is not balanced (if such a thing exists at all). Rather, one of the languages (the “heritage language”, i.e. Russian, Greek, Turkish) is acquired as a first language, but is used only in specific comm-sits – predominantly in the family and with peers, less often in informal public settings – and there is little access to literacy or to language used in more formal comm-sits. The other language (i.e. German, English) is acquired as a second first language or early second language (age of onset before age 4), and it is the majority language of the social context and the dominant language of the formal public sphere including school, work life, social institutions, etc. \parencite{maas2008sprache}. On the basis of this sociolinguistic imbalance between the languages involved, we might expect convergence to express itself in structures of the heritage language(s) rather than in both languages involved.

The specific sociolinguistic distribution of languages brings into play yet other possible language contact dynamics beyond convergence, and these are related to language acquisition. For a long time, research on multilingualism was dominated by a claim of “bilingual disadvantage”, meaning that bilingualism leads to higher processing costs in both languages (see \cite{higby2013multilingualism} for a neurolinguistic overview). Also, on the side of the heritage language, studies spoke of “incomplete acquisition”, “attrition” or “erosion” of structures (see \cite[ch. 7]{aalberse2019heritage} for a critical overview). Across RUEG projects, however, we assume that different varieties in the speech of monolingual vs. bilingual heritage speakers rather emerge from limited access of heritage speakers to the structures of the written standard of the language and to the registers of formality. This may lead to a levelling between the linguistic practices used in different comm-sits \parencite{wiese2022heritage}. Importantly, however, there may be differences in the relationship between majority and heritage language(s) in different societies and there are reasons to assume that this also has an impact on the development of the heritage language \parencite{iefremenko2021converbs}. The size of the speaker group may play a role here, and the vitality of the particular heritage language in the country (group size and vitality may or may not correlate). 

Generally speaking, clause combining is a grammatical area which is clearly sensitive to language contact and convergence. For example, the fact that Russian uses more non-finite subordination than any other Slavic language has been attributed to centuries of language contact of Russian with Uralic and Altaic languages, which rely more heavily on non-finite subordination \parencite[772]{grenoble2013syntax}. For Turkish, numerous works show shifts from non-finite to finite subordination in languages of the Turkic language family, due to contact with Indo-European languages that prefer finite subordination (\cite[566]{johanson2020turkic} for an overview,
\cite{keskin2023transient, chapters/09} for the Balkans). Also, expansions of syndetic paratactic clause combining strategies have been shown to be accelerated by language contact \parencite{mithun1988}, and coordinating conjunctions seem to be a part of speech that is frequently borrowed between languages \parencite{winford2003introduction}.

As for clause combining and bilingualism, studies on German as an early second language speak of delayed acquisition of subordination structures in bilingual children starting school (age of onset of L2 between 2 and 4 years; \cite{grimm2014sprachfahigkeiten}). Other studies investigating monolingual and bilingual German-speaking pupils do not find evidence of a “bilingual disadvantage”; see \textcite{goschler2017syntaxerwerb} for primary school pupils and \parencite{haberzettl2016bildungssprache}  for 13-year-olds. For clause combining in heritage languages on the other side, we find a claim that heritage speakers use less subordination than majority speakers of the same language; see \textcite[47]{polinsky2018heritage} for a generalization across heritage languages. \textcite{polinsly2008} also states this to be valid for Russian heritage speakers in the United States. For heritage Turkish, \textcite{treffers2006oral} , \textcite{bayram2013acquisition}  and \textcite{worfel2022adverbial} argue in a similar direction. This development has in the past sometimes been attributed to “incomplete acquisition” 
 \parencite{polinsly2008,bayram2013acquisition}, sometimes also to “attrition” \parencite{yagmur1997first} or “erosion” \parencite{dussias2004parsing}. However, \textcite{sanchez2017syntactic} do not find any evidence for a “decline” in subordination in the written narrative texts of Spanish heritage speakers in France and Germany. Several studies using the RUEG methodology also do not find such evidence for Russian heritage speakers in Germany and the US and German heritage speakers in the US \parencite{martynova2024use,pashkova2022syntactic,tsehaye2023light}. In a similar vein, a growing body of research concentrating on Turkish as a heritage language steps away from the more deficit-oriented perspective and shows that the dynamics in heritage Turkish should rather be understood as the emergence of a new system of clause combining, which expands both paratactic connectivity and finite subordination \parencite{herkenrath2003interrogative,karakocc2007connectivity,onar2015transformation,schroeder2016clause,ozsoy2022shifting,Keskinetal_ir}, and is related to limited access to the formal register of the language \parencite[cf.][57]{aalberse2019heritage}.

\section{Methodology} \label{sec:schroederetal:2}

\subsection{Data collection} \label{schroederetal:2.1}

The data used for the study come from the latest version of the RUEG corpora (RUEG-DE, EL, EN, RU, TR) \parencite{RUEGcorpus2024}. The data were collected using the “language situations” method described in \textcite{wiese2020language}. This method elicits naturalistic and ecologically valid narrative data across different formalities and modes (see \cite{chapters/02}). Participants were shown a short video with a staged minor car accident and asked to imagine that they witnessed it. Then participants were asked to retell this accident in four different comm-sits: to a friend via a WhatsApp voice message (informal-spoken), to a friend via a WhatsApp text message (informal\hyp written), to the police via a voicemail (formal-spoken), and to the police in the form of a written witness report (formal-written). Bilingual participants completed the task twice, with a break of at least three days between sessions: once in their heritage language and once in the majority language. Monolingual speakers’ data were elicited only in their native language. All participants filled out an extensive online language background questionnaire, including questions about their metalinguistic profiles. 

\subsection{Participants} \label{sec:schroederetal:2.2}

For the current study, we analyze data from five languages, namely Greek, Russian, Turkish, German and English. In the first three languages, we investigate heritage varieties in Germany and in the US and monolingual varieties in Greece, Russia and Turkey, respectively (see Table~\ref{table:schroederetal:1} for the number of speakers in each group). German and English are analyzed as majority languages of bi- and monolingual speakers in Germany and the US respectively (see Table~\ref{table:schroederetal:2} for the number of speakers in each group). All participants are teenagers (15--17 years old) who at the time of data collection were attending school.
As mentioned in \textcitetv{chapters/02}, the data were controlled for (in)formality (formal vs. informal) and mode (spoken vs. written). As a result, each speaker produced four productions in each of their languages. In total, we analyzed 1588 texts produced by a total of 277 participants. 

\begin{table}
\caption{Number of speakers and texts for Greek, Russian, Turkish speakers}
\begin{tabular}{lllrr}
    \lsptoprule
    Language & Status & Country & Number of speakers & Number of texts \\
    \midrule
    Greek & Heritage & Germany & 20 & 80 \\
    ~ & Heritage & USA & 24 & 96 \\
    ~ & Majority & Greece & 24 & 96 \\
    \midrule
    Russian & Heritage & Germany & 28 & 112 \\
    ~ & Heritage & USA & 34 & 136 \\
    ~ & Majority & Greece & 35 & 140 \\
    \midrule
    Turkish & Heritage & Germany & 24 & 96 \\
    ~ & Heritage & USA & 24 & 96 \\
    ~ & Majority & Greece & 24 & 96 \\
    \lspbottomrule
\end{tabular}
\label{table:schroederetal:1}
\end{table}


\begin{table}
\centering
\caption{Number of speakers and texts for German and English speakers}
\begin{tabular}{llllrr}
%\hspace{1cm}
%\begin{tabular}{ |p{1.5cm}|p{1.5cm}|p{1.5cm}|p{4 cm}|p{0.75cm}|p{0.75cm}|  }
 \lsptoprule
Language & Status & Country & Group & Number & Number\\
~ & ~ & ~ & ~ & of speakers & of texts\\
  \midrule
 German & Majority & Germany & German-Greek & ~ & ~ \\ 
 ~ & ~ & ~ & bilinguals & 16 & 64 \\
 ~ & ~ & ~ & German-Russian & ~ & ~ \\ 
  ~ & ~ & ~ & bilinguals & 16 & 64 \\
 ~ & ~ & ~ & German-Turkish & ~ & ~ \\ 
  ~ & ~ & ~ & bilinguals & 16 & 64 \\
 ~ & ~ & ~ & German & ~ & ~ \\
   ~ & ~ & ~ & monolinguals & 16 & 64 \\
 \midrule
 English & Majority & USA & English-Greek & ~ & ~ \\ 
   ~ & ~ & ~ & bilinguals & 24 & 96 \\
 ~ & ~ & ~ & English-Russian & ~ & ~ \\ 
  ~ & ~ & ~ & bilinguals & 24 & 96 \\
 ~ & ~ & ~ & English-Turkish & ~ & ~ \\ 
   ~ & ~ & ~ & bilinguals & 24 & 96 \\
 ~ & ~ & ~ & English & ~ & ~  \\ 
   ~ & ~ & ~ & monolinguals & 24 & 96 \\
 \lspbottomrule
\end{tabular}
\label{table:schroederetal:2}
\end{table}


\subsection{Annotation} \label{sec:schroederetal:2.3}

The main body of each narrative was annotated for connectivity. The main body comprises the text that excludes openings and closings, that is, it is the part of the text that contains the story line (see \cite{chapters/15}). 
Using EXMARaLDA \parencite{schmidt2014exmarlda}, we annotated the data on three tiers for the following features:

\begin{enumerate}[label=\Roman*.]
\item Speech act
\item Clause type
\item Connectivity elements
\end{enumerate}

\begin{enumerate}[label=\Roman*.]
\item
The \textit{Speech act tier} comprises the following categories, which only apply to clauses.

\begin{itemize}
\item
\textit{Situational}: Includes all clauses that directly relate to the description of the events displayed in the stimulus video. This tier also includes instances of negation and modality. An example of a situational speech act is given in \REF{ex:schroederetal:ball}.

\ea
\label{ex:schroederetal:ball}
\gll und der Mann  hat-te ein Ball in der Hand genau.\\
and the man have-\Pst.\Tsg{} a ball in the hand yeah\\
\glt `And the man had a ball in his hands, yeah.' (DEbi61MG\verb|_|fsD)\footnote{The participant codes provide the following information:
\begin{itemize}
    \item Country: DE—Germany; GR—Greece; RU—Russia; TU—Turkey; US—US
    \item Bi-/monolingual speaker: bi vs. mo
    \item Speaker number incl. age group: 01 to 49—adults; from 50 onwards—adolescents
    \item Gender: M vs. F (there were no speakers who identified as non-binary)
    \item Heritage language for bilingual speakers or only majority language for monolinguals: D for German; E for English; G for Greek; R for Russian; T for Turkish
    \item comm-sit: formal—f vs. informal—i; spoken—s vs. written—w
    \item Language of production: D for German; E for English; G for Greek; R for Russian; T for Turkish
\end{itemize}
} 
\z


\item
\textit{Subjective}: Includes all predications which explicitly express the speaker’s personal stance, such as evaluations, explanations, personal statements, e.g. example \REF{ex:schroederetal:mess}.

\ea
\label{ex:schroederetal:mess}
\gll It was a mess.\\
(-) \\ %Glossen??\\ 
\glt (DEbi61MG\_fsD)
\z

\item
\textit{Intersubjective}: Includes all dialogic clausal instances in which the speaker addresses the imaginary interlocutor (see example \ref{ex:schroederetal:extreme}).

\ea
\label{ex:schroederetal:extreme}
\gll aber jo (-) ah	weiß-t du was ich krass fand?\\
but	well {} ah	know-\Prs.\Ssg{} you	what I		extreme	find.\Pst.\Fsg{}\\
\glt  `But do you know what I found extreme?' (DEmo53FD\_isD)
\z

\item
\textit{Textual}: Includes predications that relate to the organization of the narrative (see example \ref{ex:schroederetal:end}).

\ea
\label{ex:schroederetal:end}
\gll bun-un-la da kal-ma-dı\\
this-\Gen-\Inst{} also stay-\Neg-\Pst.3sg{}\\
\glt  `And it didn’t end here.' (TUmo57FT\_fsT)
\z

\item
\textit{Direct speech}: Includes instances of direct speech within the narration, e.g. example \REF{ex:schroederetal:ruki}.

\ea
\label{ex:schroederetal:ruki}
\gll sta-l takoj (-) ruk-i na bok takoj oj da izvin-i-te\\
stand-\Pst.\Tsg{} such {} hand-\Pl{} on side such oh yes excuse-\IMP-\Spl{}\\
\glt `and he was like with his arms akimbo: “oh yeah, excuse me."' (USbi69FR\_isR)
\z

\end{itemize}

\item
The \textit{Clause type tier} comprises the following categories:

\begin{enumerate}
\item Independent clause or matrix clause
\item Elliptical clause
\item Complement clause: finite or non-finite
\item Adverbial clause: finite or non-finite
\item Relative clause: finite or non-finite
\end{enumerate}

Not all types of clauses are present in each language. Thus, for example, there are no instances of finite relative clauses in Turkish (as expected).

\item
The \textit{Connectivity tier} comprises the categories listed in \tabref{table:schroederetal:3}.

\begin{table}
\caption{Connectivity tier}
\label{table:schroederetal:3}
\begin{tabularx}{\textwidth}{lQ}
 \lsptoprule
Connectivity & Example (from English, for convenience) \\\midrule
 Neutral (NCON)  & \textit{And}   \\
 Event-related and logical (ELCON) & \textit{but, moreover, fortunately, luckily, either, or} \\
 Spatial (SAD) & \textit{behind (him), across (from him), (from) behind}\\
 Temporal (TAD) & \textit{then, afterwards, at that time, at that moment}\\
 Temporal adverbial clause, finite or & \textit{when they arrived}\\
 non-finite (which explicitly refers & ~ \\
 to the time of a previously narrated & ~ \\
 event) (TACA) & ~ \\
 Subordinating connector (SCON) & \textit{while, because, who (relative)}\\
 Linking discourse marker (LDM) & \textit{yeah, like, well}\\
 Filled pause (FP) & \textit{um, e, ee } \\
 (tagged only when there are no other & ~ \\
 connectivity devices in the clause) \\
\lspbottomrule
\end{tabularx}
\end{table}

\end{enumerate}

Asyndetic linking was interpreted as the absence of overt linking devices between independent extended or simple clauses.


\section{Results} \label{sec:schroederetal:3}
This subchapter displays the descriptive quantitative results of our investigation. We present figures and explain them briefly, focusing on four topics that were introduced in Sections~\ref{sec:schroederetal:1} and \ref{sec:schroederetal:2}. First, we look at the degree to which the texts, stripped of their openings and closings, are narrations on the second level, as discussed in \sectref{sec:schroederetal:1.4}. That is, we investigate whether a description of events dominates, as expected, and to what extent other speech acts are involved (\sectref{sec:schroederetal:3.1}). Second, we focus on subordination and investigate in more detail the frequency and distribution of subordination strategies (finite vs. non-finite, different clause types) (\sectref{sec:schroederetal:3.2}). Third, we look at the distribution of strategies for paratactic clause combining (\sectref{sec:schroederetal:3.3}). Fourth, throughout the Subsections \ref{sec:schroederetal:3.1} to \ref{sec:schroederetal:3.3}, we inquire into the linguistic practice in the different comm-sits distinguished here.

\subsection{Speech acts} \label{sec:schroederetal:3.1}

\begin{figure}
    \centering
    \includegraphics[width=\linewidth]{figures/Ch8_Figure_1.pdf}
    \caption{Percentage of different types of speech acts in the bilingual speakers’ heritage language and monolingual speakers’ majority language}
    \label{fig:schroederetal:1}
\end{figure}

\begin{figure}
    \centering
    \includegraphics[width=\linewidth]{figures/Ch8_Figure_2.pdf}
    \caption{Percentage of different types of speech acts in the bi- and monolingual speakers’ majority languages}
    \label{fig:schroederetal:2}
\end{figure}

Figures~\ref{fig:schroederetal:1}  and \ref{fig:schroederetal:2} present the frequency distribution of speech acts in heritage and majority languages of the speakers. Both figures clearly show that the situational speech act by far dominates. Across all speaker groups, it makes up between 89 and 98\% of all clauses. Thus, the text sections that we are investigating indeed qualify as narratives. Next to the descriptive speech act, only the speech act of evaluation (subjective) is worth mentioning, which has an overall percentage of 1 to almost 9\%. The percentage of evaluative speech acts is the highest in the Russian texts; in the monolingual Russian speakers from Russia they reach 8.8\%, while in the Russian heritage speakers in Germany evaluations are at 6\%. The lowest percentage of evaluations is found in the Turkish texts; the three Turkish speaker groups (monolinguals, heritage speakers in the US and in Germany) have between 1.4 and 1.9\% of evaluative comments in their texts. The next most frequent, at 3.7\%, is found in Greek-English bilinguals in their English productions. There is no evidence of cross-linguistic influence from the heritage languages to the majority languages in percentage of evaluative speech acts. In German, there are no notable differences between the speaker groups, and in English, the Russian-English bilinguals may have the highest ratio (5.5\%), but the monolingual speakers score on the same level (5.2\%), while the Greek-English bilinguals and the Turkish-English bilinguals have a considerably lower ratio (3.7\% and 3.7\%, respectively). 

\begin{figure}
    \centering
    \includegraphics[width=\linewidth]{figures/Ch8_Figure_3.pdf}
    \caption{Percentage of different types of speech acts in the bilingual speakers’ heritage language and monolingual speakers’ majority language across comm-sits.}
    \label{fig:schroederetal:3}
\end{figure}

\begin{figure}
    \centering
    \includegraphics[width=\linewidth]{figures/Ch8_Figure_4.pdf}
    \caption{Percentage of different types of speech acts in the bi- and monolingual speakers’ majority languages across comm-sits.}
    \label{fig:schroederetal:4}
\end{figure}
\largerpage
Figures \ref{fig:schroederetal:3} and \ref{fig:schroederetal:4} present frequency distribution of speech acts in heritage and majority languages of the speakers across the four comm-sits: informal-spoken, informal-written, formal-spoken and formal-written. From the figures we see that the situational speech act dominates in the four situations. At the same time, the figures also show that subjectivity is very much a characteristic of informal situations, while in the formal situations, particularly the written mode, subjective comments are scarce. Such differences between the comm-sits are most evident in the monolingual majority varieties of the investigated languages, less so in the bilingual majority varieties, and are least pronounced in the heritage varieties.

\subsection{Subordination and subordination strategies} \label{sec:schroederetal:3.2}
We start with a general overview on the quantities of subordinate clauses and then proceed to details with regard to subordination strategies.

\begin{figure}
    \centering
    \includegraphics[width=\linewidth]{figures/Ch8_Figure_5.pdf}
    \caption{Percentage of main clauses vs. subordinate clauses in the bilingual speakers’ heritage language and monolingual speakers’ majority language across comm-sits.}
    \label{fig:schroederetal:5}
\end{figure}

\begin{figure}
    \centering
    \includegraphics[width=\linewidth]{figures/Ch8_Figure_6.pdf}
    \caption{Percentage of main clauses vs. subordinate clauses in the bi- and monolingual speakers’ majority languages across comm-sits.}
    \label{fig:schroederetal:6}
\end{figure}

\figref{fig:schroederetal:5} and \ref{fig:schroederetal:6} show frequency distribution of main and subordinate clauses in heritage and majority languages of the speakers across four comm-sits. When we look at the ratio of subordinate clauses within all clauses in Russian, Greek and Turkish (\figref{fig:schroederetal:5}), we find that in Turkish and Greek, the intensity of the use of subordinate clauses drops considerably in the heritage varieties, while in Russian, it does not.

In the majority languages in Germany and the US, the percentage of subordinate clauses does not differ much between the speaker groups. However, German-Greek bilingual speakers tend to employ more subordinate clauses than German-Russian and German-Turkish bilinguals.

Furthermore, the figures show that formality leads to more subordination: the number of subordinate clauses in the informal-spoken and informal-written comm-sits is lower than in the formal comm-sits. On the other hand, mode (spoken vs. written) cannot be generalized to make a difference: In some speaker groups, formal-written texts have the highest percentage of subordinate clauses, while in other speaker groups the formal-spoken texts have the highest percentage, and vice versa for informal-written versus informal-spoken and (lower) percentages of subordinate clauses. 

In the heritage varieties, the difference in intensity of use of subordinate structures between formal and informal is smaller, compared to the corresponding monolingual language practice – lesser so for heritage speakers in the US than in Germany. Note, however, that the frequency of the use of subordinate structures in the informal-spoken comm-sit is similar between the monolingual and the heritage varieties (for Turkish also in the informal-written and for Russian also in informal-written and formal-spoken comm-sit). As for the monolingual and bilingual speakers of the majority languages English and German, we find differences between the groups, but no opposition between monolinguals on the one hand and bilinguals on the other.

Next, we zoom in on subordinate clauses and discuss frequency distribution of finite vs. non-finite structures in the languages under research here.

\begin{figure}
    \centering
    \includegraphics[width=\linewidth]{figures/Ch8_Figure_7.pdf}
    \caption{Percentage of finite vs. non-finite subordinate clauses in the bilingual speakers’ heritage language and monolingual speakers’ majority language across comm-sit.}
    \label{fig:schroederetal:7}
\end{figure}

\begin{figure}
    \centering
    \includegraphics[width=\linewidth]{figures/Ch8_Figure_8.pdf}
    \caption{Percentage of finite vs. non-finite subordinate clauses in the bi- and monolingual speakers’ majority languages across comm-sits.}
    \label{fig:schroederetal:8}
\end{figure}

Figures \ref{fig:schroederetal:7} and \ref{fig:schroederetal:8} demonstrate the frequency distribution of finite and non-finite subordinate clauses in heritage and majority languages of the speakers across four comm-sits. In general, figures show that the typological characteristics hold. That is, all Turkish speaker groups overwhelmingly use non-finite subordination strategies; the Greek speakers use the lowest percentage of non-finite clauses as compared to finite clauses, the Russian and the English speakers again use more non-finite structures than the German speakers. Thus, the intensity of the use of non-finite structures goes as shown in \sectref{sec:schroederetal:1.2}:

\begin{center}
  Turkish > Russian, English > German > Greek
\end{center}

Furthermore, Figures \ref{fig:schroederetal:7} and \ref{fig:schroederetal:8} show different distributions of finite vs. non-finite subordinate clauses across heritage and majority speaker groups in different comm-sits. In the monolingual varieties of Greek, Russian, and Turkish, formality leads to the use of more non-finite subordinate clauses. Heritage speakers of the three languages produce more finite subordinate clauses in contrast to non-finite subordinate clauses, compared to monolinguals (although the difference in percentage varies considerably across the languages). However, for Russian and Greek heritage speakers this is only observed in the formal comm-sits, while Turkish, it is observed across all four comm-sits. 

As for the majority languages, \figref{fig:schroederetal:8} shows that German monolinguals prefer finite subordinate clauses, particularly in the spoken mode. While German-Russian bilinguals closely resemble German monolinguals, German-Turkish and German-Greek bilinguals employ more finite subordinate clauses than German monolinguals, particularly in the written mode. As for English, all groups behave similarly in terms of frequency distribution of finite vs. non-finite subordinate clauses.

\subsection{Paratactic clause combining strategies} \label{sec:schroederetal:3.3}

A further observation concerns paratactic clause combining devices. Here we distinguish between different semantic kinds of connectors (see \sectref{sec:schroederetal:2}) and present the percentage of asyndetically connected clauses.

\begin{figure}
    \centering
    \includegraphics[width=\linewidth]{figures/Ch8_Figure_9.pdf}
    \caption{Percentage of different kinds of connecting devices in the bilingual speakers’ heritage language and monolingual speakers’ majority language across comm-sits.}
    \label{fig:schroederetal:9}
\end{figure}

\begin{figure}
    \centering
    \includegraphics[width=\linewidth]{figures/Ch8_Figure_10.pdf}
    \caption{Percentage of different kinds of connecting devices in the bi- and monolingual speakers’ majority languages across comm-sits.}
    \label{fig:schroederetal:10}
\end{figure}

Figures \ref{fig:schroederetal:9} and \ref{fig:schroederetal:10} present frequency distribution of different kinds of connectors across comm-sits in the five languages. First of all, we can go back to Figures~\ref{fig:schroederetal:5} and~\ref{fig:schroederetal:6} and generalize that the reverse of what was said for subordinate clauses counts for paratactic linking; in other words, the lesser the percentage of subordination, the more the use of paratactic combining of clauses. That means that the heritage varieties of Greek and Turkish use overall more parataxis than the monolingual varieties, while for Russian, this generalization does not apply. 

Turning now to the details of paratactic clause combining, we do not find notable differences between the different groups of the majority languages English and German. In Greek and Russian, on the other hand, the heritage speakers use the neutral connector more frequently than the corresponding monolingual speakers. However, what is interesting is that this does not seem to raise the frequency or variety of other connecting devices used by the monolingual speakers. Rather, the monolinguals use a higher percentage of asyndetic connections than the bilinguals, that is, no overt connecting device at all. 

We also find that in all speaker groups with the exception of Turkish, there is a considerable drop in (relative) frequency of neutral connectors in the formal-written texts, as opposed to the other comm-sits, and, vice versa, a rise of the use of semantically more specific paratactic connectors (event-related or temporal connectors). In Greek and Russian, this drop is less pronounced in the heritage varieties than in the monolingual varieties, that is, the heritage speakers use the neutral connector relatively more than the monolingual speakers in the formal comm-sits.

For Turkish, the picture looks quite different. First of all, compared to all other languages, the absence of overt paratactic connecting devices is most prominent here – over 50 per cent of all paratactically linked clauses have no connector (compared to 30--40\% in the other languages), and in monolingual speakers in Turkey this absence is more prevalent (60--80\% depending on the comm-sit) than in heritage speakers (55--65\% depending on the comm-sit). This shows a stronger tendency among heritage speakers towards the use of overt paratactic linking devices, and they also use a broader variety of overt linkers than the monolingual speakers, including new connectors based on discourse markers \parencite{chapters/14}.
\pagebreak

Furthermore, we also observe a high frequency of event-related and logical connectors in the Russian speakers, with the highest occurrence found in monolingual Russian speakers. Interestingly, in Russian-German bilinguals this pattern is exclusive to the Russian language and is not found in their German productions. Such an increase in the frequency of event-related and logical connectors may be the result of the Russian speakers using a high number of evaluative comments (as discussed in \sectref{sec:schroederetal:3.1}).

\section{Discussion} \label{sec:schroederetal:4}

In this chapter, we investigated clause combining in typologically different languages: four Indo-European languages, namely English, German, Greek and Russian, and an Altaic language, Turkish. We explored the majority as well as heritage varieties of these languages, thus investigating both languages of the bilinguals. The goal was to explore whether there are similar dynamic patterns in clause combining strategies among the speakers of the five languages under research, and to find out to what extent the potential differences and similarities between the varieties are related to the typological structures of the languages, to communicative situations, and to language contact.   

First of all, we can safely say that the typological differences between the languages at issue concerning preferences of subordination strategies are preserved also in the language contact settings addressed here. For instance, in terms of the finiteness of subordinate clauses, heritage varieties pattern with the corresponding majority varieties. Thus, Turkish predominantly relies on non-finite subordination; Greek speakers overwhelmingly use finite structures, while Russian, English and German speakers employ both strategies, although to different degrees. Bilingualism does not bring radical changes, neither to the majority languages nor to the heritage varieties we investigate here, which cling to the general typological characteristics of the corresponding monolingual varieties. This adherence exists in spite of intensive language contact which is, however, relatively recent.

Nevertheless, some phenomena raise further questions and some changes seem to be arising   in the heritage and majority varieties. 

To begin with the majority languages, we certainly cannot generalize a distinction between monolinguals on the one side and bilinguals on the other, as the claim of a “bilingual disadvantage” (see \sectref{sec:schroederetal:1.5}) would have it. What we do find are different preferences across the speaker groups, for example differing frequencies of the uses of subordination in general, and differing frequencies in the employment of non-finite structures. In very selected domains, these differing preferences may allow us to discuss issues of convergence. For example, we see that the Greek speakers in Germany have the highest percentage of subordination among the heritage speaker groups, and they are also the bilingual group which has the highest percentage of subordination in German. All in all, language contact does not provide a good argument for the different distributions of clause combining strategies between the speaker groups in the majority languages. At minimum, language contact cannot be the only factor. Thus, we have to leave this question to further research and discussion.

We did not find much evidence for effects of language contact in the majority languages, but do we find evidence in the structures of the heritage languages? The clearest phenomenon we find here is the “levelling of registers”. The linguistic practices (here: clause combining strategies) in the different comm-sits show fewer cross-linguistic differences in the heritage varieties than in the monolingual varieties. We already hinted at a possible reason here: Heritage speakers have less access to the variety of registers of the standard language, at least to formal registers. Consequently, structures that would be classified as belonging to informal registers in the monolingual varieties are generalized also to more formal comm-sits in the heritage varieties. Interestingly, this levelling seems more pronounced in the US than in Germany. Is this related to a higher vitality of Russian, Greek and Turkish in Germany as compared to the US, in the sense that speakers of the heritage languages can use or are exposed to a larger variety of comm-sits in these languages in Germany than in the US? For Turkish and Russian, this seems to be the case, as \textcite{iefremenko2021converbs} and \textcite{schroeder2022} argue for Turkish and \textcite{martynova2024use} for Russian, but it remains to be investigated whether the same argument holds also for Greek.

Furthermore, our research here is not consistent with a characterization of heritage languages in general as “incomplete acquisition” or “attrition” or “erosion”. All subordinating structures are used by all speaker groups, and at least in the informal-spoken comm-sit, there are no differences between monolingual and heritage speakers in terms of the frequency of uses of subordinate structures. Thus, the overall higher frequency of subordinate structures in the monolingual varieties is rather related to the speakers’ differing practices in more formal comm-sits, where they use more subordinating and less paratactic structuring, while the heritage speakers adhere somewhat more to the linguistic practices that they also use in the informal comm-sits. Thus, it is not that the heritage speakers do not “have” the respective subordinating structures, it is just that they do not employ them in similar frequencies. The same holds for the use of the neutral connector in Russian and Greek. And it also accounts for the very small percentage of non-finite structures in the language use of the Greek heritage speakers. These structures seem to belong to the formal register of standard Greek, to which the Greek speakers in Germany and the US have lesser access.

A further interesting point is the observation that asyndetic linking seems to be more frequent in the monolingual varieties of Greek, Russian and Turkish, as compared to the heritage varieties. Does this point to a higher “explicitness” and “transparency” in the language use of heritage speakers, as pointed out in \textcite{polinsky2018heritage}  – in this case a more intensive use of overt paratactic linking devices? For Turkish, the tendency to use overt paratactic linking devices goes hand in hand with an expansion of the system of these devices and a stronger use of finite subordination strategies, as opposed to non-finite subordination. Together with \textcite{ozsoy2022shifting} we argue that here, register levelling and convergence combine together. The speakers expand a particular subordination strategy that is already there in the language, and they expand it according to the example of their majority languages. 

All in all, then, just as we found for the majority languages, language contact does not seem to be the decisive factor initializing and/or driving change in the heritage languages. Rather, the strongest factor seems to be the particular sociolinguistic situation of heritage languages in the majority societies, narrowing the spectrum of language practice in different comm-sits, and leading to an expansion of the language practice from informal comm-sits.

The exploratory study presented in this chapter lays strong foundations for further work.   Although we have examined several aspects of subordination structures in this chapter, other aspects remain to be examined that will undoubtedly shed further light on tendencies to change in the system. These include more detailed analyses of degrees of expansion or reduction of subordinate clauses along the continua proposed in \textcite{lehmann1988towards}, as well as case hierarchies in relative clause formation \parencite{polinsly2008}. Both points relate less to the frequency and occurrence of subordinate structures than to the breadth of what the structures are used for. Furthermore, we believe that a closer look at individual variation would strengthen our argument in the direction of a sociolinguistic explanation \parencite{ozsoy2023exploring}.

Summarizing the main results from the chapter, we can say that the study contributes to a better understanding of the causes for changes in both languages of heritage speakers. Importantly, it is not the bilingualism of the speakers but rather their limited access to formal registers in their heritage language that is the reason for the observed differences. Hence, our study corroborates the findings of the several recent studies from our research unit that demonstrated that the cause for the differences in the linguistic structures between heritage and majority varieties can be attributed to register levelling in heritage speakers 
(\cite{alexiadou2023use,wiese2022heritage,tsehaye2023light,schroeder2022}, among others). Furthermore, our study supports the claim that vitality of a heritage language plays an important role in the development of the language, as has already been discussed for Turkish and Russian in recent works from our research group \parencite{iefremenko2021converbs,schroeder2022,martynova2024use}.

\section*{Acknowledgements}
\label{sec:schroederetal:acknowledgements}

We are indebted to our student assistants, who carried out the immense annotation tasks: Iliuza Akhmetzianova, Franziska Groth, Ksenia Idrisova, Oğuzan Kuyrukçu, Simge Türe, and Barbara Zeyer. Also, we want to extend special thanks to our student assistant, Özce Özceçelik, for handling the formatting\hyp related tasks in \LaTeX, as well as to Lea Coy for help with pre-publication formatting. We also thank to the anonymous reviewers for their valuable and highly constructive comments. The research was supported through funding by the Deutsche Forschungsgemeinschaft (DFG, German Research Foundation) for the Research Unit \textit{Emerging Grammars in Language Contact Situations}, project P9 (grant number: 313607803).

\section*{Abbreviations}
\begin{multicols}{3}
\begin{tabbing}
MMM \= 2nd person\kill
2    \> 2nd person\\
3    \> 3rd person\\
\textsc{gen}\> genitive\\
\textsc{imp}\> imperative\\
\textsc{inst}\> instrumental\\
\textsc{neg}\> negative\\
\textsc{pl}\> plural\\
\textsc{pres}\> present\\
\textsc{pret}\> preterite\\
\textsc{pst}\> past\\
\textsc{sg}\> singular
\end{tabbing}
\end{multicols}

{\sloppy\printbibliography[heading=subbibliography,notkeyword=this]}
\cleardoublepage
\end{document}
