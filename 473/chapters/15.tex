\documentclass[output=paper,colorlinks,citecolor=brown]{langscibook}
\ChapterDOI{10.5281/zenodo.15775189}
\author{Kalliopi Katsika\orcid{0000-0002-6736-4963}\affiliation{University Kaiserslautern-Landau} and         Annika Labrenz\orcid{0000-0002-6235-9321}\affiliation{Humboldt-Universität zu Berlin} and         Kateryna Iefremenko\orcid{0000-0003-3711-0935}\affiliation{Leibniz-Centre General Linguistics; University of Potsdam} and        Shanley E.M. Allen\orcid{0000-0002-5421-6750}\affiliation{University Kaiserslautern-Landau}}

\title{Discourse openings and closings across languages in contact}

\abstract{This chapter presents an exploratory study focused on the analysis of macro discourse strategies across various speaker groups using data from the Research Unit \textit{Emerging Grammars} (RUEG) corpus. Specifically, the present study examines openings and closings – the parts of a narration or conversation that precede and follow the main discourse phase – in adult and adolescent heritage speakers of Greek, German, Russian and Turkish and monolingual speakers of Greek, German, Russian and Turkish. The main aim of this study is to unveil the complex dynamics of discourse organization in multilingual contexts and to investigate the role of age, language status and communicative situation in the distribution of discourse functions in openings and closings across languages in contact. Our methodology includes the creation of an annotation scheme through identification and subsequent annotation of textual, subjective and intersubjective functions in Greek, German, Russian and Turkish openings and closings in the narrations of heritage and monolingual speakers. Statistical analyses revealed similarities and differences in the distribution of openings and closings, and the functions included in them, across speaker groups. The research and analyses included in this chapter aim to contribute to a better understanding of macro discourse organization in language contact situations.
\keywords {macro functions, discourse, openings, closings, heritage languages}
}

\IfFileExists{../localcommands.tex}{
   \addbibresource{../localbibliography.bib}
   % add all extra packages you need to load to this file

\usepackage{tabularx,multicol}
\usepackage{url}
\urlstyle{same}

\usepackage{listings}
\lstset{basicstyle=\ttfamily,tabsize=2,breaklines=true}

\usepackage{langsci-basic}
\usepackage{langsci-optional}
\usepackage{langsci-lgr}
\usepackage{langsci-osl}
% \usepackage{./langsci/styles/langsci-lgr}
% \usepackage{./langsci/styles/langsci-osl}
% \usepackage{langsci-gb4e}

\usepackage{tikz}
\usetikzlibrary{patterns,calc}
\pgfdeclarepatternformonly{south east lines}{\pgfqpoint{-0pt}{-0pt}}{\pgfqpoint{3pt}{3pt}}{\pgfqpoint{3pt}{3pt}}{
    \pgfsetlinewidth{0.6pt}
    \pgfpathmoveto{\pgfqpoint{0pt}{3pt}}
    \pgfpathlineto{\pgfqpoint{3pt}{0pt}}
    \pgfpathmoveto{\pgfqpoint{.2pt}{-.2pt}}
    \pgfpathlineto{\pgfqpoint{-.2pt}{.2pt}}
    \pgfpathmoveto{\pgfqpoint{3.2pt}{2.8pt}}
    \pgfpathlineto{\pgfqpoint{2.8pt}{3.2pt}}
    \pgfusepath{stroke}}
    
\usepackage{stmaryrd}
\usepackage{wasysym}
\usepackage{multirow}
\usepackage{caption}
\usepackage{subcaption}
\usepackage{mathrsfs}
\usepackage{qtree}

\usepackage{linguex}


   %pminos do not split footnotes
% \interfootnotelinepenalty=10000 %Footnote in Laporte chapters has to be split SN


%\DeclareIndexNameFormat{default}{%
%\nameparts{#1}%
%\usebibmacro{index:name}%
%{\index[names]}%
%{\namepartfamily}%
%{\namepartgiveni}%
% {}% L1
% {}% L2
%{\namepartprefix}% generates spurious space L3
%{\namepartsuffix}% generates spurious space L4
%}

%  {\DeclareIndexNameFormat{default}{%
%     \usebibmacro{index:name}{\index[names]}{#1}{#3}{#5}{#7}}}

%\DeclareIndexNameFormat{default}{%
%  \usebibmacro{index:name}{\sindex[nom]}{#1}{#3}{#5}{#7}}

%\DeclareIndexNameFormat{default}{%
%  \usebibmacro{index:name}{\sindex[person]}{#1}{#3}{#5}{#7}}
%\DeclareIndexNameFormat{default}{%
%\nameparts{#1} \usebibmacro{index:name}{\sindex[person]]}{\namepartfamily}{‌​\namepartgiven}{\nam‌​epartprefix}{\namepa‌​rtsuffix}}

%\newcommand{\smiley}{:)}

%\renewbibmacro*{index:name}[5]{%
%\usebibmacro{index:entry}{#1}%
%{\iffieldundef{usera}{}{\thefield{usera}\actualoperator}\mkbibindexname{#2}{#3}{#4}{#5}}}

% \newcommand{\noop}[1]{}

%remove for final
%\overfullrule=1mm

\newcommand{\tobi}[2]}}
\renewcommand{\S}[1]{\tobi{#1}{\textsc{*}}}

% this volume references
% puts: [this volume]
% already defined: \citetv
%\newcommand{\citepv}[1]{(\citeauthor{#1} \citeyear*{#1} [this volume])}
\newcommand{\citealtv}[1]{\citeauthor{#1} \citeyear*{#1} [this volume]}

%parentheses around example number
\newcommand{\pref}[1]{(\ref{#1})}

% in-text examples

\newcommand{\lnex}[1]{\textit{#1}} %target lang word
\newcommand{\lnlit}[1]{(lit.: `#1')} %literal reading
\newcommand{\lnlat}[1]{(#1)} % latinization
\newcommand{\lntrans}[1]{`#1'} %translation
\newcommand{\lnexl}[2]%
{\lnex{#1}{} \lnlat{#2}} % ex with latinization
\newcommand{\lnexlat}[3]{\lnex{#1}{} \lnlat{#2}{} \lntrans{#3}} % ex with latinization and tranl.

%ch01
\newcommand{\co}[1]{\mbox{\textbf{#1}}}

%ch09

\newcommand{\cyrbulg}[1]{\begin{otherlanguage*}{bulgarian}#1\end{otherlanguage*}}


%ch10
\newcommand{\nlp}{{\small NLP}}
\newcommand{\mwe}{{\small MWE}}
\newcommand{\rae}{{\small RAE}}
\newcommand{\lvc}{{\small LVC}}
\newcommand{\pos}{{\small P}o{\small S}}
%\newcommand{\todo}[1]{ \textcolor{red}{#1} }

%\renewcommand{\labelenumi}{\theenumi}
%\ainamefmt{{vv}{ll}{, ff}{, jj}} % fullname

\newcommand{\biberror}[1]{{\color{red}#1}}

\newcommand{\osenovaitem}{--~}
   %% hyphenation points for line breaks
%% Normally, automatic hyphenation in LaTeX is very good
%% If a word is mis-hyphenated, add it to this file
%%
%% add information to TeX file before \begin{document} with:
%% %% hyphenation points for line breaks
%% Normally, automatic hyphenation in LaTeX is very good
%% If a word is mis-hyphenated, add it to this file
%%
%% add information to TeX file before \begin{document} with:
%% %% hyphenation points for line breaks
%% Normally, automatic hyphenation in LaTeX is very good
%% If a word is mis-hyphenated, add it to this file
%%
%% add information to TeX file before \begin{document} with:
%% \include{localhyphenation}
\hyphenation{
    Beck-man
    Ngu-yen
    back-chan-nel
    back-chan-nels
    mo-not-o-nous
    ste-reo-typ-i-cal
}

\hyphenation{
    Beck-man
    Ngu-yen
    back-chan-nel
    back-chan-nels
    mo-not-o-nous
    ste-reo-typ-i-cal
}

\hyphenation{
    Beck-man
    Ngu-yen
    back-chan-nel
    back-chan-nels
    mo-not-o-nous
    ste-reo-typ-i-cal
}

   \boolfalse{bookcompile}
   \togglepaper[15]%%chapternumber
}{}

\begin{document}
\maketitle

\section{Introduction} \label{sec:katsikaetal:introduction}
\subsection{Openings and closings in monolingual and bilingual speakers}
Discourse openings and closings are defined as the linguistic material that precedes or follows the core phase of the discourse \parencite{schegloff_sequencing_1968, schegloff_sequence_2007, schegloff_opening_1973, luke_initiation_2002, pavlidou_phases_2014}. Their typical functions include defining textual boundaries that set the core of the text apart from the framing, introducing or closing of a new topic, setting the time frame and/or space of the event, and interpreting and evaluating the event \parencite{labov_transformation_1972, tolchinsky_text_2002, berman_form_2004}. The construction of discourse borders through openings and closings \parencite{schegloff_sequence_2007} has been shown to be an area of dynamic difference across languages \parencite[e.g.][] {luke_telephonebook_2002}, but has rarely been studied in situations of language contact \parencite[but see][] {dollnick_entwicklung_2013}. Our study fills this gap in the literature by exploring discourse opening and closing patterns in monolingually raised speakers in Germany, Greece, Russia, and Turkey (speakers who were raised in a monolingual environment), and heritage speakers of German, Greek, Russian and Turkish in the USA and Germany. All heritage speakers provided narrations in their heritage and majority languages (either English or German). Although we have annotated and analyzed majority language data too, we do not present them in the current chapter for space reasons. 

Heritage speakers (HSs) are speakers who grow up acquiring a heritage language in the home environment in addition to the language of the larger society which is different than the heritage language (see \cite{chapters/02}). In many cases, HSs gradually become dominant in the societal language. Thus, although HSs acquire their heritage language as a first language (L1), they may end up using their heritage language in only certain communicative situations, such as among family members or peers \parencite[see e.g.][] {rothman_understanding_2009,montrul_dominant_2012,polinsky_heritage_2018,tsehaye_deconstructing_2021}. The restricted use of a heritage language sometimes results in notable differences between the language that is spoken by the HSs and the homeland variety of the language. In contrast to speakers of the homeland variety of the language, HSs acquire their L1 in a language contact situation (heritage language and societal language). In addition, HSs may transfer from the majority language (which is also usually HSs’ dominant language) to the heritage language, or they may create unique forms and structures that are not found directly or transparently in either of their languages \parencite[see][] {polinsky_heritage_2018,ozsoy_shifting_2022}. Understanding the discourse abilities of HSs is crucial for comprehending the complexities of heritage language development and the maintenance of linguistic heritage. 

Openings and closings have been a central topic in the study of discourse since the early 1970s, often within the framework of Conversation Analysis \parencite[see review in][] {schegloff_sequence_2007} and also in the discourse analysis of narratives \parencite[e.g.][] {tolchinsky_text_2002}. Openings and closings have played a key role in demonstrating the formulaic nature of routines in both interpersonal interactions (e.g., telephone conversations, face-to-face communication) and monologic productions (e.g., narratives, expository texts). Since our data come from the Research Unit \textit{Emerging Grammars} (RUEG) corpus \parencite{wiese_heritage_2021}, which combines elements of both interpersonal interactions and narratives, we focus on that literature here.

 In interpersonal interactions, openings typically begin with an exchange of greetings (e.g., \textit{hi}), then initial inquiries (e.g., \textit{how are you}), and finally an anchor position where the reason for the interaction is introduced (e.g., \textit{I called to let you know that …}) \parencite {schegloff_sequencing_1968}. Emanuel Schegloff, one of the pioneers in the study of openings and closings in conversational environments, identified four ‘core sequences’ in openings of telephone conversations in English: (1) summons/response, (2) identification/recognition, (3) a greeting; and (4) a ``how are you'' sequence \parencite {schegloff_sequencing_1968,schegloff_identification_1979}. These patterns were initially observed in data from English, and although these basic sequences seem to hold across cultures and languages, many differences are also apparent \parencite {luke_initiation_2002,pavlidou_phases_2014}. For example, the “initial inquiries” phase of introductions is typically reduced or omitted in German and Finnish, while it is often extended in Greek and Spanish. The content of this phase is often different as well, with politeness routines playing a main role in Persian, humor in Greek, and health inquiries in Spanish. Closings in English interactions typically comprise a pre-closing (e.g., \textit{okay, I have to go}) and a closing greeting (e.g., \textit{goodbye, have a good day}) \parencite {schegloff_opening_1973}. A similar pattern holds in cross-linguistic studies so far, but again the content of the phases varies to some degree \parencite {luke_initiation_2002, pavlidou_phases_2014}. For example, the “pre-closing” tends to be more direct in Greek, Chinese, and Spanish, using phrases equivalent to \textit{that’s all for now} and \textit{that’s about it}. In addition, since closings in Greek, Chinese, and Spanish show a higher tendency towards repetition and a higher use of phrases of relationship affirmation or even thanking for something implying politeness/ positive attitude in communication settings (see \cite{gkouma__2023} for Greek), closings tend to be longer in those languages than in English and German. As pointed out by \textcite{Marquez_telephone_2010}, Schegloff’s sequences should not be taken as a ‘must’ in every phone conversation. In fact, after analyzing twenty-five openings in English, \textcite{hopper_speech_1989} showed that most of the interactions did not actually follow Schegloff’s sequences. What Schegloff’s work in the 70s~-- and also \textcite{sacks_harvey_1967}'s earlier work~-- has successfully achieved is to spark interest in the study of openings and closings in conversational settings across different languages, e.g. Arabic \parencite{hopper_telephone_1989}, Chinese \parencite{hopper_chia-hui_chen_languages_1996,luke_initiation_2002}, Dutch \parencite{houtkoop-steenstra_opening_1991,houtkoop-steenstra_questioning_2002}, French \parencite{godard_same_1977,hopper_telephone_1989}, German \parencite{pavlidou_contrasting_1994,pavlidou_last_1997}, Japanese \parencite{park_recognition_2002}, Persian \parencite{taleghani-nikazm_conversation_2002}, Samoan \parencite{soo_telephone_2000}, Spanish \parencite{coronel-molina_openings_1998}, Swedish \parencite{lindstrom_identification_1994}, and Vietnamese \parencite{do_vietnamese_2018} with researchers identifying different opening and closing functions cross-linguistically.
 
 In narratives, a somewhat different pattern is found since there is no need for the interpersonal exchange \parencite{berman_narrative_1997,labov_transformation_1972}. Narratives typically begin with a setting or orientation that identifies relevant aspects of the story to come (e.g., time, place, people involved, activity, situation) or provides initial interpretation and evaluation, followed by an initiating event that launches the sequence of events that constitute the plot line. Closings in narratives typically begin with some kind of summary, followed by an indication that the narrative is completed. Also relevant for narratives is the way in which the speaker establishes their stance towards the story, including what means or point of reference they select for framing their text and how they attract the attention of the interlocutor or reader \parencite{berman_narrative_1997,tolchinsky_text_2002}. Stance can be more general in situating the event in relation to other similar events in the world, or more specific in mentioning a concrete scene or an evaluation or resolution of the event itself. 
 
 Parallel to interpersonal interactions, the basic patterns for narrative construction seem to bear similarities across the cultures and languages that have been studied so far, but show clear differences as well. In their study of narratives in Spanish, English and Swedish, for example, \textcite{tolchinsky_text_2002} found that openings and closings were typically longer in Spanish than in English narratives, and also longer in English than in Swedish narratives. In addition, they found that English speakers tended to introduce the topic of the narrative directly while Spanish and Swedish speakers tended to introduce it by means of referring to an external event. In the only study, to our knowledge, on openings and closings in the narratives of bilinguals, \textcite{dollnick_entwicklung_2013} showed that Turkish HSs who grew up in Germany showed different approaches in each of their languages both to stance and to the functions conveyed by their openings and closings.\largerpage
 
 As noted earlier, research on openings and closings has focused predominantly on interpersonal interactions, with some research on monologic productions. However, very little work has been published to date on openings and closings in computer-mediated communication environments. One example is \textcite{raclaw_two_2008}'s study on closings in instant messaging (IM) of university students. Raclaw found that the IM environment led to an extension of the typical closing routines in spoken conversations. In particular, the pre-closing portion was typically elaborated with accounts for why the speaker initiated the closing (e.g., \textit{so like, i love you and all, but i should probably start my homework :/}) and arrangements to talk at a later date (e.g., \textit{I will talk to you tomorrow, jah?}). Within the accounts, it was common to include a hedge (e.g., \textit{probably, :/}) and/or a palliative or apology (e.g., \textit{I love you and all, but …}). Raclaw hypothesized that closings are extended in IM environments because of perceived difficulty or rudeness of removing oneself from the constant availability that IM offers.

 In addition to analyzing opening and closing strategies across languages, our study explores the extent to which adolescents may have different distributional patterns in openings and closings than adults. Previous studies have shown differences in openings and closings across age groups in monolingual and bilingual populations \parencite[e.g.][] {tolchinsky_text_2002,dollnick_entwicklung_2013} although differences are mostly found between child and adolescent groups. Studies in several languages show that skills in producing openings and closings in narratives develop with age, from early childhood through and beyond adolescence \parencite{berman_cross-linguistic_2002,berman_form_2004,dollnick_entwicklung_2013,ravid_developing_2002,tolchinsky_text_2002}. 
 
 In sum, openings and closings in discourse constitute a highly dynamic environment that offers potential for numerous emerging patterns in cases of language contact such as those in the RUEG corpus \parencite{wiese_heritage_2021}. Existing literature shows substantial differences in patterns of openings and closings across languages that could possibly lead to new patterns across the two languages of the speakers. Differences in patterns by age (adolescent vs. adult), by levels of formality (informal vs. formal), and by mode (spoken vs. written) are also evident in the literature. Given that very little research has been conducted on openings and closings in bilingual populations, our corpus study aims to add to our knowledge of discourse patterning in bilinguals. More specifically, the current study focuses on adolescent and adult HSs’ discourse organization strategies with the goal of exploring the means that HSs~-- as well as monolingual speakers (MSs) of the respective languages – use in order to introduce and close narrations extracted in different communicative situations, that is, informal-spoken, formal-spoken, informal-written, formal-written. For example, previous research has shown that speakers may use different opening functions in formal settings \parencite{Marquez_telephone_2010,zimmerman_interactional_1992,zimmerman_talk_1984}. Our study is exploratory in nature and aims to present distributional patterns of discourse openings and closings in monolingual and heritage speakers so as to set the foundations for more detailed analyses in the future.
 
 \subsection{Research questions and hypotheses}
 Our research questions are the following: 
 \begin{description}
     \item [Research Question 1:] Do monolingual and heritage speakers of German, Greek, Russian and Turkish exhibit a similar distribution of openings and closings in their narratives?\\
     On the basis of the research across different languages reported in \sectref{sec:katsikaetal:introduction}, we hypothesize that we may find partial evidence for Schegloff’s opening sequences (our data do not include actual telephone conversations), and that we may also find variability in opening and closing functions across speaker groups. 
     \item [Research Question 2:] Does age play a role in the distribution of openings and closings across monolingual and heritage speakers?\\ 
     Based on the previous studies by \textcite{tolchinsky_text_2002} and \textcite{dollnick_entwicklung_2013}, we expect to find differences in the distribution of openings and closings between the adolescent and adult groups.
     \item [Research Question 3:] Does the distribution of intersubjective, subjective and textual functions differ across speaker groups and communicative situations?\\
     We expect that formality (formal vs. informal communicative situations) and mode (spoken vs. written) will affect the functions used in openings and closings across language groups. One difference across communicative situations may be an increase of intersubjective functions in informal (and possibly also spoken) settings, and in contrast, a possible increase of textual functions in formal-written settings.  
 \end{description}
 In sum, given the results of previous research on openings and closings in conversations and narratives, we expect to find age-, contact-, formality- and mode-related differences in heritage speakers’ openings and closings.

 \section{Method}
 \subsection{Corpus database and data elicitation}
 Data for this study come from the corpus collected for the Research Unit \textit{Emerging Grammars} (RUEG; \cite{wiese_heritage_2021}). An exciting and pioneering aspect of the RUEG corpus is that it comprises majority and heritage language data in four different languages (German, Greek, Russian, and Turkish) in two contact environments, namely Germany and the USA. These data are complemented with data from monolingual speakers of all RUEG languages. In the present study, we analyzed heritage language data from adolescent and adult German, Greek, Russian and Turkish HSs as well as data from monolingual speakers of the aforementioned languages (\tabref{tab:katsikaetal:participants}). The age of the adolescents ranged from 13 to 19 years, and the age of the adults ranged from 20 to 37 years.

 The data were collected using the Language Situations method \parencite{asahi_language_2019}. Participants were shown a short video clip of a minor car accident and were asked to describe it, imagining that they had witnessed the event. Importantly, they narrated the video in four different communicative situations: informal-spoken (voice message to a friend), informal-written (WhatsApp message to a friend), formal-spoken (voice message of a police witness report) and formal-written (witness report for the police). Participants in all sites and situations were given similar instructions, that is, to narrate the incident that occurred in a video. The data were collected by two elicitors at each site: one for the informal situations and the other for the formal situations. Heritage speakers provided descriptions of the video event in both their heritage and their majority languages in two separate sessions with at least a three-day interval between the two sessions, although only data from their heritage language is considered in this chapter.
 
 This data collection approach allows us to easily compare openings and closings across different languages (German, Greek, Russian, and Turkish) and also different speaker groups for each (MSs and HSs). Thus, we can examine the extent to which MSs and HSs of different languages produce openings and closings in their narrations, as well as if they use similar (or distinct) functions in openings and closings across communicative situations. 
 
 %The database used in the present study is the German, Greek, Russian and Turkish sub-corpora local snapshots of version 1.0 of the RUEG corpus \parencite{wiese_heritage_2021}. <-- old version
 The present study is based on the German, Greek, Russian and Turkish subcorpora of the RUEG corpus \parencite{RUEGcorpus2024}, which we enriched with additional annotations for openings and closings. % <-- new version
 \tabref{tab:katsikaetal:participants} presents the total number of heritage speakers with US English and German as their majority language as well as the total number of monolingual speakers for each language. Note that every speaker produced four different narrations and thus the total number of texts is quadruple the total number of speakers.

\begin{table}
\caption{Distribution of adult and adolescent speakers per speaker group}
\label{tab:katsikaetal:participants}
 \begin{tabularx}{.9\textwidth}{lcccc}
   \lsptoprule
    Speakers & Majority language & Adolescents & Adults & Total N\\
    
 \midrule
  German HSs & English  &   29 & 7 & 36\\
  German MSs & German  &  32 & 32 & 64\\
  Greek HSs & English  &  32 & 32 & 64\\
  Greek HSs & German  &  18 & 26 & 44\\
  Greek MSs &  Greek  &  32 & 32 & 64\\
  Russian HSs & English  &  33 & 33 & 66\\
  Russian HSs & German  &  28 & 29 & 57\\
  Russian MSs &  Russian  & 34 & 33 & 67\\
  Turkish HSs & English  & 32 & 27 & 59\\
  Turkish HSs & German  &  32 & 33 & 65\\
  Turkish MSs &  Turkish  & 34 & 32 & 66\\
  \lspbottomrule
  \end{tabularx}
  \end{table}
 

\begin{table}
\caption{Distribution of analyzed texts per language}
\label{tab:katsikaetal:texts}
 \begin{tabularx}{.5\textwidth}{lc}
   \lsptoprule
    Language of texts & Total N of texts\\
    
 \midrule
  German & 400\\
  Greek  & 688\\
  Russian & 760\\
  Turkish  & 760\\
  \lspbottomrule
  \end{tabularx}
  \end{table}

Overall, we annotated and analyzed 2608 texts from the RUEG corpus (see \tabref{tab:katsikaetal:texts}). The annotations were extracted using the ANNIS corpus search tool \parencite{ANNIS3}. 

\subsection{Openings and closings annotation scheme}
Openings and closings were annotated using the Partitur\hyp Editor component of the annotation software Exmaralda \parencite{Exmaralda}. Annotations were completed in two rounds: annotating in the first round and proof checking annotations in the second round. Our study included three levels of annotation, each tagged on a different tier in Exmaralda. The first tier identified and spanned the complete area of openings and closings, while the second and third tiers included the functions inside each opening and closing~-- a detailed presentation of the functions can be found in \sectref{sec:katsikaetal:functionsopenclos} of this chapter. 

\begin{figure}
    \centering
    \includegraphics[width=1\linewidth]{figures/Ch15_Figure_1_openingsexample.png}
    \caption{Screenshot example of the three layers of annotation of an opening in a spoken narration in Exmaralda}
    \label{fig:katsikaetal:openingsexample}
\end{figure}

\figref{fig:katsikaetal:openingsexample} shows the three tiers of annotation for an opening. The first tier [macro-dstr] spans the whole opening including punctuation and filled or silent pauses (such as (-)). In the second tier [macro-function1], general functions in openings and closings (subjective, intersubjective, textual) are tagged. In the third tier [macro-function], each tag has the same length as in the second tier, and comprises the specific functions of openings, such as contextual, evaluation, greeting, etc. In the example of an opening in \figref{fig:katsikaetal:openingsexample}, the functions for the first phrase (\textit{so I just witnessed an accident}) are textual and contextual, while the functions for the second part (\textit{it seemed like a minor accident}) are subjective and evaluation.

\begin{figure}
    \centering
    \includegraphics[width=1\linewidth]{figures/Ch15_Figure_2_closingssexample.png}
    \caption{Screenshot example of the three layers of annotation of a closing in a spoken narration in Exmaralda}
    \label{fig:katsikaetal:closingsexample}
\end{figure}

Closing annotations are shown in \figref{fig:katsikaetal:closingsexample}. The first tier [macro-dstr] includes the whole span of the closing, while the second and third tiers include discourse functions in closings. The example in \figref{fig:katsikaetal:closingsexample} has three functions all of which are textual functions. The first two parts (\textit{no one got hurt, everything’s okay}) are tagged as resolutions, while the third part (\textit{but yeah that’s what happened}) is tagged as coda. 

\subsubsection{Annotation procedure}
The minimum size of an opening and/or closing is a Communication Unit (CU). As seen in the annotation layers in Figures \ref{fig:katsikaetal:openingsexample} and \ref{fig:katsikaetal:closingsexample}, CUs were already annotated in the RUEG corpus. CUs roughly correspond to clauses, and one CU may be a main clause plus all possible embedded clauses \parencite{topaj_grammatical_2020}. For long CUs containing multiple functions, we often overrode the pre-annotated CU division to assign multiple functions per CU. There were also cases of multifunctional openings that could not be divided (e.g., \textit{I just saw a stupid accident} (contextual + evaluation at the same time)). In such cases, we tagged as either contextual or evaluation on the basis of the amount of elements that pointed towards either one or the other function per CU. We did not insert combined tags (e.g. con-eva) so as to not end up with a very long list of tags and thus make the annotation process excessively complicated. There were at least two rounds of annotations per sub-corpus, and a sample of the annotations was checked for inter-annotator agreement. Annotation issues were discussed at weekly meetings across all languages at the same time so as to ensure similarity of annotations across sub-corpora to the highest possible extent. There were two to five annotators per sub-corpus, all of whom were native or near-native speakers of the language of the sub-corpus.

\subsubsection{Functions in openings and closings} \label{sec:katsikaetal:functionsopenclos}
Analyzing the discourse functions of openings and closings allows for an understanding of how linguistic choices structure discourse, guide the reader or listener, and convey meaning. They also provide insights as to the contribution of (inter-)subjectivity in monolingual and heritage speakers’ texts across communicative situations. To facilitate our analysis, we annotated the discourse functions at both a general and more specific level, as mentioned earlier. Following functional categorization of discourse markers \parencite[e.g.,][]{brinton_pragmatic_1996}, we identified three main functional categories of openings and closings: intersubjective, subjective, and textual. Each of these three general categories of functions can be elaborated into further, more detailed functions. 

\subsubsubsection{Intersubjective functions}
\largerpage
\begin{sloppypar}
Intersubjective functions can be understood as contributions to the overall dynamics of communication and social interaction. Essentially, intersubjective functions correspond to interpersonal interactions, for which conversational analysis research has shown cross-linguistic differences \parencite[e.g.,][]{luke_telephonebook_2002}. Within the general intersubjective functions, we annotated eleven more specific functions: greeting, summoning, initial inquiry, inquiry response, identification, justification, attention getter, common ground, reaction seeking, giving advice, and valediction. These are each elaborated below.
\end{sloppypar}

\begin{enumerate}[label=(\alph*)]
    \item Greeting: expression of a greeting. 
    \ea \label{katsikaetal:onegreeting}
\gll hallo Max\\
     hello Max\\
\glt `hello Max' (\textit{opening}, DEbi01MR\_isD\footnote{Explanation of participant codes: country where data was collected: ``DE'' for Germany, ``US'' for USA, ``TU'' for Turkey, ``RU'' for Russia, ``GR'' for Greece; ``bi'' for bilingual speakers, ``mo'' for monolingual speakers; speaker number incl. age group: 1 to 49: adults; 50 to 99: adolescents; gender: M(ale), F(emale) (there were no speakers who identified as non-binary); language of speakers: ``D'' for German, ``E" for English, ``G" for Greek, ``R" for Russian, ``T" for Turkish; communicative situation: ``f" formal vs. ``i" informal; ``s" spoken  vs. ``w" written; language of production: ``D" for German, ``E" for English, ``G" for Greek, ``R" for Russian, ``T" for Turkish. For example: ``DEbi01MR\_isD" represents an adult male majority German, Russian HS speaking German in an informal-spoken communicative situation.})
\ex \label{katsikaetal:twogreeting}
\gll hallo schönen   guten Tag\\
     hello beautiful good  day\\
\glt `hello good day' (\textit{opening}, DEbi02FT\_fsD)
\z

\item Summoning: calling the other person’s name or using some other form of address such as \textit{dude}, \textit{bro}, \textit{babe}, etc. 
\ea \label{katsikaetal:threegreeting}
\gll kanka merhaba\\
     dude  hello\\
\glt `dude hello' (\textit{opening}, TUmo10MT\_iwT) 
\ex \label{katsikaetal:fourgreeting}
\gll dicker\\
     dude\\
\glt `dude' (\textit{opening}, DEbi65MG\_isD) 
\z

    \item Initial inquiry: inquiry about how the other person is doing. 
\ea \label{katsikaetal:fivegreeting}
\gll nasılsın\\
     how.are.you\\
\glt `how are you?' (\textit{opening}, DEbi08FT\_isT) 
\z

    \item Inquiry response: response to an imaginary inquiry of an imaginary interlocutor, such as “how are you?”. This type of opening emerged from the communicative situations showing that participants were engaged in the imaginary scenario. 
\ea \label{katsikaetal:sixinqresponse}
\gll kala ki ego\\
     fine and I\\
\glt `I am also fine' (\textit{opening}, DEbi59MG\_isG) 
\z

    \item Identification: statement of identifying information, usually the participant’s name (which has been replaced by their participant code for anonymization purposes). In formal-spoken and written communicative situations, participants were asked to mention that their case number was “F16”. Participants sometimes used this number in an identification function, e.g. ‘I am F16’ or to introduce a textual function, such as ‘This is case F16’. 
\ea \label{katsikaetal:sevenidentification}
\gll DEbi06FT  ist mein name\\
     DEbi06FT  is  my   name\\
\glt `My name is DEbi06FT' (\textit{opening}, DEbi06FT\_fsD) 
\z

  \item Justification: mention of the reason why the participant was calling or texting. 
\ea \label{katsikaetal:eightjustification}
\gll tilefono na kano mia katathesi ja ena peristatiko pu ida\\
     I.call to make a statement about an incident that I.saw\\
\glt `I am calling to make a statement about an incident that I saw' (\textit{opening}, GRmo78MG\_fsG) 
\z
\pagebreak
\item Attention getter: use of certain phrases or words to get the imaginary interlocutor’s attention.  
\ea \label{katsikaetal:nineattention}
\gll simdi  ne    oldu     bi   bildeniz\\
     now    what  happened if   you.knew\\
\glt `If only you knew what happened now' (\textit{opening}, USbi66FT\_iwT) 
\ex \label{katsikaetal:tenattention}
\gll sana    bir  sey        anlatmam  lazim\\
     to.you  one  something  to.tell   I.need\\
\glt `I need to tell you something' (\textit{opening}, DEbi69FT\_iwT) 
\ex \label{katsikaetal:elevenattention}
\gll du  glaubst echt    nicht  was   mir   grad  passiert ist \\
     you believe really  not    what  to.me just  happened\\
\glt `You really won't believe what just happened to me' (\textit{opening}, DEbi54FR\_isD) 
\z

\item Common ground: reference to information that the participant shares in common with the imaginary interlocutor.   
\ea \label{katsikaetal:twelvecommon}
\gll du  weißt was  isch meine\\
     you know  what I    mean\\
\glt `You know what I mean' (\textit{opening}, DEbi53MG\_isD) 
\z

\item Reaction seeking: statement seeking reaction from imaginary interlocutor. 
\ea \label{katsikaetal:thirteenreactionseek}
\gll kakie nashi dal'nejshie dejstvija\\
     what our next actions\\
\glt `What are our next steps?' (\textit{closing}, RUmo56FR\_fsR) 
\z

\item Giving advice: statement offering advice to the imaginary interlocutor.
\ea \label{katsikaetal:fourteengadv}
\gll halbuki şehir içi    yavaş  gitsene\\
     however city  in     slow   go\\
\glt `However, go slower in the city' (\textit{closing}, TUmo32FT\_iwT) 
\z

\item Valediction: expression of goodbye\slash closing with a greeting.
\ea \label{katsikaetal:fifteenvalediction}
\gll do svidanija\\
     to dates\\
\glt `goodbye' (\textit{closing}, USbi77FR\_fsR) 
\z
\end{enumerate}

 \subsubsubsection{Subjective functions}
 Subjective functions focus on the subjective elements of discourse, such as opinions, beliefs, emotions, evaluations, emotional markers (which we have tagged as ‘reactions’), and personal statements. Subjective functions are important in understanding how individuals express their subjective stance, shape their identities, and influence the interpretation of discourse. Within the general subjective function, we annotated three more specific functions: reaction, evaluation, and personal statement. These are each elaborated below.

 \begin{enumerate}[label=(\alph*)]
    \item Reaction: all expressions of emotion, such as happiness, anger, surprise, or sadness, as well as reactions to the event that the participants were asked to describe including expressions such as \textit{wow}, laughter, and also emojis  in written texts \footnote{We tagged all emojis as ``reactions" because our annotation scheme did not include a detailed analysis of discourse functions in emojis. Such an analysis, which also conveys textual and intesubjuctive functions of emojis, can be found in \textcite{van_olmen_chapter_2021}.}.  
    
    \ea \label{katsikaetal:sixteenreaction}
\gll oh mein gott\\
     oh my   god\\
\glt `Oh my god' (\textit{opening}, DEmo61FD\_isD)
\z

\item Evaluation: Ethical, evaluative, or prescriptive statements relating to the events that are found in the situational narrative. Parts of the situational narrative may or may not be explicitly referred to inside the statements.   
    \ea \label{katsikaetal:seventeenevaluation}
\gll oder na ja  so krass ist es auch nicht\\
     or   oh yes so bad   is  it also not\\
\glt `or well it's not that bad either' (\textit{closing}, DEbi03FG\_isD)
\z

\item Personal statement: Personal information or personal statements not necessarily linked to the description of the incident.   
    \ea \label{katsikaetal:eighteenpersonal}
\gll ich hoffe ich muss hier nicht als Zeuge   dableiben\\
     I   hope  I   must here not   as  witness stay\\
\glt `I hope I won't have to stay here as witness' (\textit{closing}, DEbi62MG\_isD)
\z

\end{enumerate}

 \subsubsubsection{Textual functions}
Textual functions focus on the structural and cohesive aspects of texts, examining how linguistic features are used to create coherence, guide the flow of information, and shape the overall textual organization. Textual functions are crucial in understanding how discourse is constructed and how meaning is conveyed through linguistic choices. Within the general textual function, we annotated six more specific functions: initialize narrative, contextual, orientation, encapsulation, resolution, and coda. These are each elaborated below.

\begin{enumerate}[label=(\alph*)]
    \item Initialize narrative: Phrase or statement that indicates the beginning of the situational narration.  
    \ea \label{katsikaetal:nineteeninitialize}
\gll yasanan kazanin gidisatı su yöndedir\\
     that.happened of.accident its.course this in.direction\\
\glt `the course of the accident is as follows' (\textit{opening}, TUmo80FT\_fwT)
\ex \label{katsikaetal:twentyinitialize}
\gll und es ereignete sich folgendermaßen:\\
     and it happened  REFL     as.follows:\\
\glt `and it happened as follows' (\textit{opening}, DEbi07FG\_fsD)
\z

\item Contextual: Explicit reference to the events depicted in the video as a starting point for introducing the main body of the narrative (events in the video/accident). The contextual function differs from the initialize narrative function in that the core narration starts immediately after the initialize narrative function, but this is not necessarily the case in contextual parts of the opening.   
    \ea \label{katsikaetal:twentyonecontextual}
\gll demin     bi     kaza      oldu  burada\\
     just.now one accident happened  here\\
\glt `there was just an accident here' (\textit{opening}, TUmo23MT\_iwt)
\ex \label{katsikaetal:twentytwocontextual}
\gll Zeugenaussage  zum      Fall F16\\
     witness.report for.the   case F16\\
\glt `witness report for the case F16' (\textit{opening}, DEmo85FD\_fwD)
\z

\item Orientation: Expression of a particular event or circumstance located specifically in terms of time and space, thematically related to the video.  
    \ea \label{katsikaetal:twentythreeorientation}
\gll park yerinde duruyordum.\\
     in.the.parking place   I.was.standing\\
\glt `I was standing in the parking lot' (\textit{opening}, DEbi51MT\_fwt)
\z

\item Encapsulation: Statement that summarizes the events that have been previously described in the main narration including explicit mention of the main event, i.e., the accident.  
    \ea \label{katsikaetal:twentyfourencapsulation}
\gll und so kam  es zum    unfall\\
     and so came it to.the accident\\
\glt `and this is how the accident happened' (\textit{closing}, DEbi15MG\_fsD)
\ex \label{katsikaetal:twentyfiveencapsulation}
\gll ve  kaza      öyle          oluştu\\
     and accident  in.such.a.way happened\\
\glt `and that's how the accident happened' (\textit{closing}, TUmo16FT\_ fwT)
\z

\item Resolution: Resolution of the plot developed in the narrative. 
    \ea \label{katsikaetal:twentysixresolution}
\gll ja   aber niemandem ist was       passiert alles      gut \\
     yeah but  nobody    is  something happened everything good\\
\glt `but yeah nothing happened to anybody, all good' (\textit{closing}, DEbi06FG\_isD)
\z

\item Coda: Conclusion of the text with a formulaic or non-formulaic expression, relating the chain of events to the state of affairs at the time of telling the story. Codas differ from encapsulation instances in that the encapsulation instances include specific mention of the “accident” that the participants saw in the video. 
    \ea \label{katsikaetal:twentysevencoda}
\gll ich bedanke mich\\
     I   thank   me\\
\glt `thank you' (\textit{closing}, DEbi24FR\_fsD)
\ex \label{katsikaetal:twentyeightcoda}
\gll jeto vsjo chto sluchilos\\
     that's all that happened\\
\glt `that’s all (that happened)' (\textit{closing}, USbi67MR\_isR)
\z
\end{enumerate}

\section{Results}
\subsection{Levels of analysis}
Our analysis includes the following levels of analysis:

\begin{enumerate}[label=(\alph*)]
    \item Distribution of openings and closings across speaker groups. 
    \item Distribution of discourse functions inside openings and closings per communicative situation.
    \item Distribution of openings and closings per age group (adolescents vs. adults).
\end{enumerate}

\subsection{Distribution of openings and closings across speaker groups} \label{sec:katsikaetal:openingsclosingsgeneral}
We first analyzed the distribution of openings and closings that monolingual and heritage speakers used in their narratives across the four different communicative situations in the RUEG corpus. To this end, we calculated the percentage of texts including openings and closings across the total number of texts. The results are presented per speaker group. 

\subsubsection{Openings and closings in German texts}
German MSs and HSs had the same pattern of distribution of openings and closings in their texts. For both groups, the overwhelming majority of texts included an opening, and more than half the texts included a closing (\figref{fig:katsikaetal:Germanopeningsclosings}). We conducted a Kruskal-Wallis Test in R \parencite{rstudio_team_rstudio_2020} to examine possible differences in opening and closing distribution in the two groups of speakers. No significant differences were found between MSs and HSs ($\chi^2 = 0.33$, $p = 0.56$, $\text{df} = 1$).

\begin{figure}
    \centering
    \includegraphics[width=.85\textwidth]{figures/Ch15_Figure_3_openings and closings in German narrations.pdf}
    \caption{Distribution of openings and closings per speaker group in German narrations}
    \label{fig:katsikaetal:Germanopeningsclosings}
\end{figure}

\subsubsection{Openings and closings in Greek texts}
The Greek texts in the RUEG corpus were produced by three groups of Greek-speaking participants: monolingually-raised Greek speakers in Greece (Greek MSs), Greek HSs in Germany and Greek HSs in the US. As can be seen in \figref{fig:katsikaetal:Greekopeningsclosings}, all groups had a similar pattern of opening and closing distribution. This was also reflected in the statistical analysis checking for possible differences in the distribution of openings and closings, which revealed no significant differences between MSs and HSs ($\chi^2 = 0.50$, $p = 0.77$, $\text{df} = 2$). All three groups produced significantly more openings than closings ($\chi^2= 10.38$, $p = 0.001$, $\text{df} = 3$). 

\begin{figure}
    \centering
    \includegraphics[width=1\linewidth]{figures/Ch15_Figure_4_openings and closings in Greek narrations.pdf}
    \caption{Distribution of openings and closings per speaker group in Greek narrations.}
    \label{fig:katsikaetal:Greekopeningsclosings}
\end{figure}

\subsubsection{Openings and closings in Russian texts}
We also analyzed the distribution of openings and closings in the texts of Russian MSs, HSs in Germany, and HSs in the US (\figref{fig:katsikaetal:Russianopeningsclosings}). We compared frequency distribution and the analysis did not reveal any statistical difference across the three Russian-speaking groups ($\chi^2 = 0.26$, $p = 0.87$, $\text{df} = 2$). All three Russian-speaking groups produced significantly more openings than closings ($\chi^2 = 9.66$, $p = 0.02$, $\text{df} = 3$).

\begin{figure}
    \includegraphics[width=1\linewidth]{figures/Ch15_Figure_5_openings and closings in Russian narrations.pdf}
    \caption{Distribution of openings and closings per speaker group in Russian narrations}
    \label{fig:katsikaetal:Russianopeningsclosings}
\end{figure}

\subsubsection{Openings and closings in Turkish texts}
In addition, we statistically compared the distribution of openings and closings in Turkish MSs, HSs in Germany and HSs in the US, and found no statistically significant differences in the distribution of openings and closings in the Turkish texts ($\chi^2= 0.03$, $p = 0.98$, $\text{df} = 2$). As can also be seen in \figref{fig:katsikaetal:Turkishopeningsclosings}, all three Turkish-speaking groups produced significantly more openings than closings ($\chi^2 = 9.66$, $p = 0.02$, $\text{df} = 3$).

\begin{figure}
    \includegraphics[width=1\linewidth]{figures/Ch15_Figure_6_openings and closings in Turkish narrations.pdf}
    \caption{Distribution of openings and closings per speaker group in Turkish narrations}
    \label{fig:katsikaetal:Turkishopeningsclosings}
\end{figure}

\subsubsection{Summary}
In sum, the analysis of openings and closings across MSs and HSs in the RUEG corpus texts reveals that HSs of German, Greek, Turkish and Russian did not differ significantly in the distribution of openings and closings from the MSs of the respective languages. Speakers produced far more openings than closings cross-linguistically. We believe that this pattern of results may relate to the fact that participants were required to produce narrations in four different communicative situations, and may have thus used their openings to establish each of the four different communicative situations. Overall, the percentages of openings are more or less similar cross-linguistically, but the percentages of closings seem to be lower in HSs compared to MSs in German speakers. In addition, the distribution of closings follows the pattern Russian/Turkish MSs > HSs in Germany > HSs in the US in the Russian and Turkish-speaking groups. We discuss this point further in \sectref{sec:katsikaetal:discussionconclusions}. 

\subsection{Distribution of openings and closings across age groups}
The RUEG corpus includes data from adolescents and adults. Based on previous studies in openings and closings including different age groups \parencite{tolchinsky_text_2002, dollnick_entwicklung_2013}, we analyzed the different age groups per language in order to examine possible differences in the distribution of openings and closings across adolescent and adult HSs and MSs for each language. 

\begin{table}
\caption{Distribution of openings and closings per speaker and age group in each majority and heritage language}
\label{tab:katsikaetal:ageopenclose}
 \begin{tabular}{lcccc}
   \lsptoprule
            &\multicolumn{2}{c}{Openings} & \multicolumn{2}{c}{Closings}\\\cmidrule(lr){2-3}\cmidrule(lr){4-5}
    Speakers & Adults & Adolescents & Adults & Adolescents\\
    
 \midrule
  German HSs & 76\%  &   82\% & 55\% & 53\%\\
  German MSs & 97\%  &  86\% & 70\% & 70\%\\
  Greek HSs\_US & 69\%  &  75\% & 34\% & 29\%\\
  Greek HSs\_Germany & 86\%  &  85\% & 39\% & 37\%\\
  Greek MSs &  81\%  &  88\% & 39\% & 37\%\\
  Russian HSs\_US & 79\%  &  85\% & 55\% & 42\%\\
  Russian HSs\_Germany & 87\%  &  92\% & 64\% & 43\%\\
  Russian MSs &  86\%  & 92\% & 66\% & 71\%\\
  Turkish HSs\_US & 83\%  & 82\% & 41\% & 25\%\\
  Turkish HSs\_Germany & 92\%  &  85\% & 55\% & 32\%\\
  Turkish MSs &  91\%  & 88\% & 65\% & 44\%\\
  \lspbottomrule
  \end{tabular}
  \end{table}

We statistically compared adolescents and adults’ openings and closings per speaker group using the Kruskal-Wallis Test. None of the analyses revealed statistically significant results between adolescents and adults neither in the distribution of openings nor in the distribution of closings. There were small differences across groups, which however did not reach statistical significance. For example, adult German MSs produced more openings than adolescents whereas adolescent HSs produced more openings than adult HSs. We can also see in \tabref{tab:katsikaetal:ageopenclose} that the distributional difference between MSs and HSs in Russian and Turkish closings was actually because of the very low percentages of closings in adolescents. We thus see again that openings were more similarly distributed across age and language groups, whereas there was more variability in closings, which were, however, less frequent than openings.  

\subsection{Openings: distribution of discourse functions across communicative situations}
This section includes the analysis of discourse functions across communicative situations in openings. We conducted this analysis in order to investigate the type of discourse material that MSs and HSs chose to include in their openings, and the extent to which the functions in openings followed the expected norms for each communicative situation. We present the analysis per language group, starting with MSs and HSs of German. 

\subsubsection{Opening discourse functions in German MSs and HSs}
We analyzed the types of functions (intersubjective, subjective, textual) in German narrations across speaker groups and communicative situations (\figref{fig:katsikaetal:Germanfunctionsopenings}). The Kruskal-Wallis test did not yield any significant differences across speakers ($\chi^2= 1.19$, $\text{df} = 1$, $p = 0.27$), which means that the two groups of speakers showed a similar pattern.

\begin{figure}
    \centering
    \includegraphics[width=1\linewidth]{figures/Ch15_Figure_7_Function in German narrations in openings.pdf}
    \caption{Distribution of functions per speaker group in German openings}
    \label{fig:katsikaetal:Germanfunctionsopenings}
\end{figure}

As we can see in \figref{fig:katsikaetal:Germanfunctionsopenings}, both groups of speakers produced more intersubjective and textual functions than subjective functions in their openings. Thus, speakers introduced their texts with intersubjective functions such as greetings, justifications and identifications as well as textual functions such as initializing of narrative, contextual and episodic functions and not so often with subjective functions such as evaluations and personal statements. In total, German MSs produced 361 intersubjective, 96 subjective, and 297 textual functions. German HSs produced 99 intersubjetive, 32 subjective, and 153 textual functions. Therefore, intersubjective were the most frequent functions in MSs whereas textual were the most frequent functions HSs. The distribution of functions per communicative situation in the two speaker groups can be seen in \tabref{tab:katsikaetal:Germanfunctiosopen}. 

\begin{table}
\caption{Percentages of functions in openings in each communicative situation and speaker group in German openings}
\label{tab:katsikaetal:Germanfunctiosopen}
 \begin{tabular}{llrr}
   \lsptoprule
    Situation & Function & German MSs & HSs\_US\\
    
 \midrule
  Informal-spoken & Intersubjective & 44\% & 33\%\\
                  & Subjective & 22\% & 20\%\\
                  & Textual & 33\% & 48\%\\
  Informal-written & Intersubjective & 45\% & 38\%\\
                  & Subjective & 17\% & 8\%\\
                  & Textual & 38\% & 54\%\\
  Formal-spoken   & Intersubjective & 62\% & 41\%\\
                  & Subjective & 7\% & 7\%\\
                  & Textual & 31\% & 52\%\\
  Formal-written  & Intersubjective & 28\% & 14\%\\
                  & Subjective & 4\% & 7\%\\
                  & Textual & 68\% & 79\%\\
  \lspbottomrule
  \end{tabular}
  \end{table}

As already mentioned, intersubjective functions were the most frequent functions in MSs and  second most frequent in HSs. A closer look at intersubjective functions showed that MSs and HSs mostly used greetings, identifications and justifications in formal-spoken communicative situations, as illustrated in examples \ref{katsikaetal:twentyninegreeting}-\ref{thirtyonejustification}. 
\pagebreak
\ea \label{katsikaetal:twentyninegreeting} Greeting\\
\gll ähm ja   schönen   guten tag\\
     um  jeah beautiful good  day\\
\glt `um yeah good day' (DEmo21MD\_fsD)
\z

\ea Identification\\ \label{thirtyidentification}
\gll mein name ist DEmo19FD\\
     my   name is  DEmo19FD\\
\glt `my name is' (DEmo19FD\_fsD)
\z

\ea Justification\\ \label{thirtyonejustification}
\gll ich wollte ein tatvorfall berichten den   ich eben beobachtet habe\\
     I   wanted an  incident   report    which I   just watched    have\\
\glt `I wanted to report an incident which I just watched' (DEmo79MD\_fsD)
\z

Greetings were the standard way for MSs to start an opening in formal-spoken communicative situations (out of the 63 total openings, 54 start with a greeting, i.e. 86\%). Half of these greetings (27 greetings) were of the more “formal” \emph{Guten Tag} (‘good day’) type as in examples \REF{katsikaetal:twogreeting} and \REF{katsikaetal:twentyninegreeting}, almost half of the greetings (25 greetings) involved the neutral greeting \emph{Hallo} (‘hello’), and two greetings were a combination of \emph{Hallo} and \emph{Guten Tag} (see also \cite{chapters/02} for an analysis of discourse organization and lexical choices in openings with a focus on register distinctions). Interestingly, in HSs, only half of the openings included a greeting (16 greetings out of 32 openings), and most of these greetings were more informal (13 out of 16 greetings, 81\%). Therefore, HSs did not follow the prevalent monolingual strategy of starting a formal-spoken opening with a greeting. And even when HSs started an opening with a greeting, they consisted of the more neutral greeting ``hallo". This result should be treated with caution, however, as the HS German data only include 7 adults and 29 adolescents, and the lack of formal greeting may be a characteristic of age rather than speaker group. When checking the adolescent MS greetings, we saw that half of them (13/26) were informal or neutral, and half of them formal. 
Adult MSs used a slightly higher number of formal greetings (16/28, 57\%) but neither adolescent nor adult MSs reached 81\% of informal/neutral greetings found in HSs. This leads us to the conlcusion that HSs' preferred discourse strategy is to use informal\slash neutral greetings in both informal and formal communicative situations. 

Turning to identifications in formal-spoken settings, MSs used identification sequences such as the one in example \REF{thirtyidentification}, that is, they provided their name (replaced with their code number for anonymity reasons as stated previously) most of the time (46/53, 87\%), and sometimes instead of their name, they provided the case number F16, which was given to them as part of the experimental instructions. HSs provided only 8 identifications, and out of these eight, five referred to the case number and not their name. We thus see here that HSs were reluctant to provide their own name, and they were more likely to use the case number instead. Finally, justifications (\ref{thirtyonejustification}) did not differ much across the two speaker groups except for the fact that HSs provided justifications to a much lesser extent than MSs.  

In terms of textual functions in German openings, these were the most frequent in HSs and second most frequent in MSs. In both groups, the two most common textual functions across all communicative situations were contextual (specifically mentioning the incident) and episodic (mentioning time and/or place of incident) functions, as in examples \REF{katsikaetal:thirtytwocontextual} and \REF{katsikaetal:thirtythreeepisodic}. 

\ea Contextual\\ \label{katsikaetal:thirtytwocontextual}
\gll Gerade habe ich einen autounfahl   draussen vor die wohnung  gesehen\\
     Just   have I   a     car.accident outside  of  the building saw\\
\glt `I just saw a car accident outside of the building' (USbi01FD\_iwD)
\newpage
\ex Episodic\\ \label{katsikaetal:thirtythreeepisodic}
\gll Um Mittag am 7.5.19 war am     Parkplatz   hinter dem    Wohngebäude an der Unteren Hauptbahnstrasse ein Autounfall\\
     at noon   on 7.5.19 was at.the parking.lot behind the building on the Unteren Hauptbahnstrasse an accident\\
\glt `At noon on 7.5.19 at the parking lot behind the residential building on Unteren Hapbahnstrasse there was an accident' (USbi63MD\_fwD)
\z

Although HSs overall produced fewer textual functions than MSs, the type and the distribution of textual functions across communicative situations patterned with MSs (see \tabref{tab:katsikaetal:Germanfunctiosopen}).  

\subsubsection{Opening discourse functions in Greek MSs and HSs}
Greek MSs and HSs had a similar pattern of intersubjective, subjective and textual functions in their openings (\figref{fig:katsikaetal:Greekfunctionsopenings}). Consequently, the statistical analysis showed no significant difference across speaker groups ($\chi^2= 1.06$, $\text{df} = 2$, $p = 0.58$) but a statistically significant effect in terms of differences across functions ($\chi^2 = 6.48$, $\text{df} = 2$, $p = 0.03$). This difference stems from the fact that subjective functions were used very sparingly across all speaker groups. 

\begin{figure}
    \includegraphics[width=1\linewidth]{figures/Ch15_Figure_8_Function in Greek narrations in openings.pdf}
    \caption{Distribution of functions per speaker group in Greek openings}
    \label{fig:katsikaetal:Greekfunctionsopenings}
\end{figure}

\begin{table}
\caption{Percentages of functions in openings in each communicative situation and speaker group in Greek openings}
\label{tab:katsikaetal:Greekfunctiosopen}
 \begin{tabular}{llrrr}
   \lsptoprule
    Situation & Function & Greek MSs & HSs\_Germany & HSs\_US\\
 \midrule
  Informal-spoken & Intersubjective & 57\% & 64\% & 57\%\\
                  & Subjective & 10\% & 8\% & 11\%\\
                  & Textual & 33\% & 28\% & 32\% \\
  Informal-written & Intersubjective & 61\% & 60\% & 60\%\\
                  & Subjective & 6\% & 9\% & 6\%\\
                  & Textual & 34\% & 32\% & 34\%\\
  Formal-spoken   & Intersubjective & 44\% & 61\% & 61\%\\
                  & Subjective & 2\% & 3\% & 2\%\\
                  & Textual & 54\% & 36\% & 37\%\\
  Formal-written  & Intersubjective & 39\% & 50\% & 55\%\\
                  & Subjective & 3\% & 6\% & 0\%\\
                  & Textual & 59\% & 44\% & 45\%\\
  \lspbottomrule
  \end{tabular}
  \end{table}
  
The distribution of subjective, intersubjective and textual functions in Greek MSs and HSs per communicative situation is presented in \tabref{tab:katsikaetal:Greekfunctiosopen}. 

As we can see in \tabref{tab:katsikaetal:Greekfunctiosopen}, the distribution of intersubjective functions across communicative situations in informal communicative situations is similar across speaker groups. In formal communicative situations, MSs produced fewer intersubjective functions than the two HS groups. This means that Greek HSs in Germany and the US inserted more conversational elements in their formal openings than Greek MSs. In addition, HSs in Germany produced a relatively large number of identifications in formal-spoken situations (19/78: 24\%) in comparison to MSs (1/56: 2\%) and HSs in the US (8/58: 13\%). All of these cases involved “my name is…” instances as in example \REF{katsikaetal:thirtyfouridentification}:

\ea Identification\\ \label{katsikaetal:thirtyfouridentification}
\gll legome DEbi17FG; onomazome Debi26MG; t=onoma mu ine DEbi14MG\\
     I.am.named DEbi17FG; I.am.named DEbi26MG; the.name of.me is DEbi14MG\\
\glt `my name is DEbi17FG\_P' (DEbi17FG\_fsG); `DEbi26MG' (DEbi26MG\_fsG); `DEbi14MG’(DEbi14MG\_fsG)
\z
\newpage
\ea Identification\\ \label{katsikaetal:thirtyfiveidentification}
\gll ime o USbi17MG\\
     I.am the USbi17MG\\
\glt `I am USbi17MG’ (USbi17MG\_fsG)
\z

All instances of identification in Greek HSs in Germany look like example \ref{katsikaetal:thirtyfouridentification} and constitute formal ways to introduce oneself in Greek. In contrast, all of the 8 instances used by Greek HSs in the US look like example (\ref{katsikaetal:thirtyfiveidentification}) and are primarily informal ways of introducing oneself in Greek. These results lead us to two conclusions; firstly, identifying oneself when making a phone call to the police to leave a voice message is not a discourse strategy that Greek MSs use. Secondly, the two HS groups differ in that the HSs in Germany used the “appropriate” or expected way to introduce themselves, whereas HSs in the US used an informal way to introduce themselves in a formal setting. We thus see that both groups of HSs used a discourse strategy that is not common across Greek MSs, but the two HS groups differ in that only the HSs in Germany identified themselves in a formal way in a formal environment. 
The second noticeable result in the analysis of intersubjective functions in Greek was the relatively low number of intersubjective functions in informal-written texts in HSs in the US which was due to the fact that only one HS in the US used a justification sequence, whereas for Greek MSs justifications (such as \REF{katsikaetal:thirtysixjustification}) were the primary way to introduce informal-written settings (24/65: 37\%).

\ea Justification\\ \label{katsikaetal:thirtysixjustification}
\gll Tha  kathisteriso na ertho, giati   opos erhomun      egine    ena atihima!\\
     Will delay        to come,  because as   I.was.coming happened an  accident\\
\glt ‘I will come later because there was an accident as I was coming’(GRmo02FG\_iwG)
\z

In informal-written texts, the distribution of discourse functions in Greek MSs and HSs was the following: 

\begin{itemize}
    \item Greek MSs: justifications > attention getters > greetings
    \item Greek HSs in Germany: greetings > justifications > attention getters
    \item Greek HSs in the US: greetings > attention getters
\end{itemize}

Although HSs primarily used greetings in their instant messages, a greeting was not the preferred strategy to begin a text message in Greek MSs. The HS groups were similar in that both groups placed priority on greetings, and also both of them used at least one discourse strategy that was also found in MSs. Textual and subjective functions had a similar distribution across groups, and therefore we do not discuss them further. 

\subsubsection{Opening discourse functions in Russian MSs and HSs} \label{sec:katsikaetal:Russianfunctionsopenings}
The data analysis of Russian MSs and HSs revealed a similar distribution of discourse functions across the three speaker groups (\figref{fig:katsikaetal:Russianfunctionsopenings}). There was no statistical significance across groups ($\chi^2 = 0.62$, $\text{df} = 2$, $p = 0.73$) but there was a statistically significant effect of functions ($\chi^2 = 6.48$, $\text{df} = 2$, $p = 0.03$), which reflected the relatively high distribution of intersubjective and textual functions compared to the low distribution of subjective functions.

\begin{figure}
    \centering
    \includegraphics[width=\textwidth]{figures/Ch15_Figure_9_Functions in Russian narrations in openings.pdf}
    \caption{Distribution of functions per speaker group in Russian openings}
    \label{fig:katsikaetal:Russianfunctionsopenings}
\end{figure}

The most frequent type of function in Russian MSs and HSs openings was intersubjective. Intersubjective functions were most frequent in Russian HSs in Germany who produced a total of 401 intersubjective functions, followed by MSs and HSs in the US who produced almost a similar amount of intersubjective functions (320 and 300 respectively). Although the distribution discourse functions per communicative situations did not differ across the three speaker groups as can be seen in \tabref{tab:katsikaetal:Russianfunctiosopen}, Russian HSs in Germany produced a lot more intersubjective functions in informal than in formal settings (HSs in Germany: 212, HSs in US: 160, MSs: 154 intersubjective functions in informal settings). 

\begin{table}
\caption{Percentages of functions in openings in each communicative situation and speaker group in Russian openings}
\label{tab:katsikaetal:Russianfunctiosopen}
 \begin{tabularx}{\textwidth}{llYYY}
   \lsptoprule
    Situation & Function & Russian MSs & HSs\_Germany & HSs\_US\\
    
 \midrule
  Informal-spoken & Intersubjective & 46\% & 55\% & 52\%\\
                  & Subjective & 9\% & 16\% & 9\%\\
                  & Textual & 45\% & 30\% & 39\% \\
  Informal-written & Intersubjective & 45\% & 61\% & 46\%\\
                  & Subjective & 7\% & 6\% & 10\%\\
                  & Textual & 48\% & 33\% & 45\%\\
  Formal-spoken   & Intersubjective & 60\% & 63\% & 59\%\\
                  & Subjective & 2\% & 8\% & 5\%\\
                  & Textual & 38\% & 29\% & 37\%\\
  Formal-written  & Intersubjective & 35\% & 45\% & 43\%\\
                  & Subjective & 3\% & 2\% & 2\%\\
                  & Textual & 62\% & 53\% & 55\%\\
  \lspbottomrule
  \end{tabularx}
  \end{table}

 A closer examination of intersubjective functions revealed that HSs in Germany used more justifications (see example \REF{katsikaetal:thirtysevenjustification}) than Russian MSs across all communicative situations, and they also had a higher variety of functions in informal communicative situations. Russian MSs used only one justification in the informal-spoken situation, 8 (24\%) in formal-written, and 42 (32\%) justifications in formal-spoken. In contrast, HSs in Germany used 13 instances of justifications (6\%) in informal settings, 11 (28\%) in formal-written and 55 (37\%) in formal-spoken situations.

\ea Justification\\ \label{katsikaetal:thirtysevenjustification}
\gll i ja zvonju potomu chto segodnja ja uvidela\\
     and I am calling because of which today I saw\\
\glt ‘and I am calling because of what I saw today’ (DEbi03FR\_fsR)
\z
\newpage
Another characteristic of Russian openings was that the two HS groups produced more summonings such as \REF{katsikaetal:thirtyeightsummoning}\footnote{Please note that summonings differ from greetings in that in summonings the name occurs first and then the greeting, whereas in greetings the greeting precedes the name.}, and they also used a wider variety of intersubjective functions in informal communicative situations (such as common ground (\ref{katsikaetal:thirtyninecommon}) and reaction seeking (\ref{katsikaetal:fourtyreactionseek})) than MSs.

\ea Summoning\\ \label{katsikaetal:thirtyeightsummoning}
\gll DEbi26FR privet\\
     DEbi26FR hello\\
\glt ‘DEbi26FR hello’ (DEbi26FR\_isR)
\ex Common ground\\ \label{katsikaetal:thirtyninecommon}
\gll ty znaesh' zdes' v centre gorode da\\
     you know here in center downtown jeah\\
\glt ‘You know, here in the center of the town, yeah’ (DEbi64MR\_isR)
\ex Reaction seeking\\ \label{katsikaetal:fourtyreactionseek}
\gll pozhalujsta priezzhajte\\
     please come.over\\
\glt ‘Please come over’ (USbi12MR\_fsR)
\z

We also checked Russian HSs in the US to see if we would find the same pattern in identifications as in Greek HSs in the US, that is, using informal means to identify themselves in formal-spoken communicative situations. Our search did not show a similar pattern in the Russian data. Russian HSs in the US used similar ways of identifying themselves in formal-spoken communicative situations as MSs (\ref{katsikaetal:fourtyoneidentification}). 

\ea Identification\\ \label{katsikaetal:fourtyoneidentification}
\gll menja zovut USbi20MR\\
     my name USbi20MR\\
\glt ‘my name is USbi20MRr’ (USbi20MR\_fsR)
\z

In terms of subjective functions, Russian HSs in Germany produced more subjective functions~-- such as personal statements (\ref{katsikaetal:fourtythreepersonal}) and evaluations (\ref{katsikaetal:fourtytwoevaluation})~-- than MSs and HSs in the US, especially in spoken communicative situations: HSs in Germany: 53, HSs in US: 25, MSs: 22. This pattern may reflect a majority (German) language pattern. 

\ea Evaluation\\ \label{katsikaetal:fourtytwoevaluation}
\gll i vygljadelo jeto vneshne chisto obiektivno kak chistyj sluchaj\\
     and looked it outwardly purely objectively like pure accident\\
\glt ‘and it looked outwardly purely objectively like a pure accident’ (DEbi01MR\_isR)
\ex Personal statement\\ \label{katsikaetal:fourtythreepersonal}
\gll ja zhe hotela v kino pojti\\
     I also wanted to cinema go\\
\glt ‘I also wanted to go to the cinema’ (DEbi63FR\_isR)
\z

Regarding textual functions (\tabref{tab:katsikaetal:Russianfunctiosopen}), all three speaker groups showed the same pattern in that they all produced the highest percentages of subjective functions in formal-written situations. At the same time, the three groups differed slightly in that MSs produced a lot more textual functions (62\%) than intersubjective functions (35\%) in formal-written situations, whereas this difference was less pronounced in HSs in Germany and the US. 

\subsubsection{Opening discourse functions in Turkish MSs and HSs}
We also examined intersubjective, subjective and textual functions in the openings of Turkish speakers (\figref{fig:katsikaetal:Turkishfunctionsopenings}). The statistical analyses revealed no effect of group ($\chi^2= 2.22$, $\text{df} = 2$, $p = 0.32$) and a marginally significant effect of functions ($\chi^2 = 5.42$, $\text{df} = 2$, $p = 0.06$).

\begin{figure}
    \centering
    \includegraphics[width=1\linewidth]{figures/Ch15_Figure_10_Functions in Turkish narrations in openings.pdf}
    \caption{Distribution of functions per speaker group in Turkish openings}
    \label{fig:katsikaetal:Turkishfunctionsopenings}
\end{figure}

The distribution of intersubjective, subjective and textual functions in the Turkish texts per communicative situation and speaker group can be seen in \tabref{tab:katsikaetal:Turkishfunctiosopen}. 

\begin{table}
\caption{Percentages of functions in openings in each communicative situation and speaker group in Turkish openings}
\label{tab:katsikaetal:Turkishfunctiosopen}
 \begin{tabularx}{\textwidth}{llYYY}
   \lsptoprule
    Situation & Function & Turkish MSs & HSs\_Germany & HSs\_US\\
    
 \midrule
  Informal-spoken & Intersubjective & 48\% & 46\% & 40\%\\
                  & Subjective & 14\% & 15\% & 27\%\\
                  & Textual & 38\% & 39\% & 33\% \\
  Informal-written & Intersubjective & 44\% & 44\% & 37\%\\
                  & Subjective & 12\% & 19\% & 22\%\\
                  & Textual & 44\% & 36\% & 41\%\\
  Formal-spoken   & Intersubjective & 58\% & 51\% & 57\%\\
                  & Subjective & 4\% & 5\% & 4\%\\
                  & Textual & 38\% & 44\% & 40\%\\
  Formal-written  & Intersubjective & 48\% & 51\% & 56\%\\
                  & Subjective & 4\% & 2\% & 2\%\\
                  & Textual & 48\% & 47\% & 42\%\\
  \lspbottomrule
  \end{tabularx}
  \end{table}

Overall, Turkish speakers produced high numbers of intersubjective and textual functions in their openings. HSs in Germany produced the highest number of intersubjective functions, followed by MSs, and HSs in the US. Further analysis showed that the two HS groups produced more intersubjective functions in formal than in informal communicative situations. For example, HSs used a very high number of identifications in formal communicative situations. All identification instances, however, included the case number F16, which was the number that participants were explicitly asked to provide (see example \ref{katsikaetal:fourtyfouridentification}). We thus did not explore this result further. 

\ea Identification\\ \label{katsikaetal:fourtyfouridentification}
\gll ben ef on altı\\
     I am F sixteen\\
\glt ‘I am F16’ (USbi68FT\_fsT)
\z

In addition, our analysis showed that all three groups mostly used attention getters (\ref{katsikaetal:fourtyfiveattention}), summonings (\ref{katsikaetal:fourtysixsummonings}), greetings (\ref{katsikaetal:fourtyeightgreetings}) and initial inquiries (\ref{katsikaetal:fourtyseveninitial}) in informal communicative situations. In contrast, they primarily used greetings, identifications and justifications in formal situations. The distribution of these functions across spoken and written situations was similar for all groups. 

\ea Attention getters \label{katsikaetal:fourtyfiveattention}
  \ea[]{
  \gll bugün n=oldu biliyo musunuz\\
       today what.happened you.know\\
   \glt ‘Do you know what happened today’ (TUmo23FT\_isT)
}
\ex []{
\gll sana çok önemli bi\hspace{0,2cm}şey anlatmam gerek\\
     to.you very important something I.tell need\\
 \glt ‘I have something very important to tell you’ (DEbi11FT\_isT)
}
      \ex []{
    \gll bugün noldu inanamıcaksın\\
         today what.happened you.won't.believe\\
     \glt ‘You won’t believe what happened today’ (USbi15FT\_isT)
}
    \z
\z

\ea Summonings \label{katsikaetal:fourtysixsummonings}
  \ea[]{
  \gll kanka\\
       dude\\
   \glt ‘Dude’ (TUmo60MT\_isT)
   }
    \ex []{
    \gll hayatım\\
         honey\\
     \glt ‘Honey’ (DEbi04MT\_isT)
     }
      \ex []{
    \gll kızla:r\\
         girls\\
     \glt ‘Girls’ (USbi66FT\_isT)
}
    \z
\z

\ea Greetings \label{katsikaetal:fourtyeightgreetings}
  \ea[]{
  \gll ii günler\\
       good day\\
   \glt ‘Good day’ (TUmo24FT\_fsT)
   }
    \ex []{
    \gll kolay gelsin memur bey\\
         easy come.in officer\\
     \glt ‘Good day officer’ or `Let your work be easy officer' (TUmo25MT\_fsT)
     }
      \ex []{
    \gll merhabalar\\
         hello.there\\
     \glt ‘Hello there’ (TUmo61FT\_fsT)
}
    \z
\z

\ea Initial inquiry\\ \label{katsikaetal:fourtyseveninitial}
\gll canım n=apıyon nasılsın\\
     dear how.doing.you how.are.you\\
\glt ‘Dear, how are you doing, how are you?’ (DEbi52FT\_isT)
\z


The fact that we found no attention getters and initial inquiries and only a very small number of summonings in formal communicative situations indicates that these discourse functions are characteristic of informal situations in Turkish. Similarly, there were almost no justifications and a much lower number of greetings (informal: 52, formal: 128 in total) in informal communicative situations, and this implies that justifications (\ref{katsikaetal:fourtyninejustification}) and greetings (\ref{katsikaetal:fourtyeightgreetings}) are characteristics of formal communicative situations for Turkish speakers. 

\ea Justification\\ \label{katsikaetal:fourtyninejustification}
\gll ben bi kazayı bildirmek istiyordum da\\
     I an accident report have.been.wanting well\\
\glt ‘I wanted to report an accident’ (TUmo60MT\_fsT)
\z


Regarding subjective and textual functions, the distribution of functions was overall similar across groups and we did not analyze them further.

\subsubsection{Summary}
In sum, our analysis revealed both similarities and differences in the distribution of discourse functions in openings across speaker groups and languages. In general, HSs and MSs had a similar distribution of discourse functions in openings. In terms of differences, the fact that German and Greek HSs in the US did not use formal greetings in formal situations may be interpreted as evidence for register leveling in HSs. In addition, the relatively high percentages of subjective functions in Russian HSs in Germany may reflect majority language influence as German MSs also produced a relatively high percentage of subjective functions in informal situations (see \tabref{tab:katsikaetal:Germanfunctiosopen}).

\subsection{Closings: distribution of discourse functions across communicative situations}
All speaker groups produced fewer closings than openings (see \sectref{sec:katsikaetal:openingsclosingsgeneral}). The current section includes a basic summary of the main results without statistical analyses because of the low number of closing instances. As in openings, we present the main results of our analysis starting with the German speakers, then the Greek, Russian and Turkish speakers. Overall, the general pattern across all speaker groups was a high number of subjective functions in informal compared to formal narrations. 

\subsubsection{Closing discourse functions in German MSs and HSs} \label{sec:katsikaetal:Germanclosings}
As can be seen in \figref{fig:katsikaetal:Germanfunctionsclosings}, German MSs produced primarily subjective functions, such as evaluations and personal statements, in their closings. In contrast, German HSs in the US produced textual functions, such as codas and resolutions, more frequently than subjective and intersubjective functions. This difference indicates that HSs were more reluctant than MSs to provide their personal opinions at the end of their narrations. 

\begin{figure}
    \centering
    \includegraphics[width=1\linewidth]{figures/Ch15_Figure_11_Functions in German closings.pdf}
    \caption{Distribution of functions per speaker group in German closings}
    \label{fig:katsikaetal:Germanfunctionsclosings}
\end{figure}

We further analyzed the distribution of functions per communicative situation. In informal situations, German MSs produced 203, and HSs produced 46 subjective functions (MSs: 203/322: 63\%, HSs:46/106: 43\%). This shows that German MSs’ main strategy to end their spoken and written informal narrations was by an evaluation or personal comment. HSs also did so but to a lesser extent.  In formal situations, MSs produced a relatively large number of valedictions (45/72: 62\%) as in example \REF{katsikaetal:fiftyvaledictions}, but this was not the case for HSs who actually produced no valedictions at all. HSs mostly used textual functions to end their formal texts, such as codas and resolutions (\ref{katsikaetal:fiftyoneresolution}). In fact, HSs produced more resolutions (65\%) than MSs (46\%) whereas MSs produced more codas than HSs (MSs: 48\%, HSs: 25\%) across all communicative situations. Therefore, although MSs mostly used codas to close their narrations, HSs primarily used resolutions.

\ea Valedictions \label{katsikaetal:fiftyvaledictions}
    \ea []{
    \gll auf wiederhören\\
         to listen.again\\
     \glt ‘Goodbye’ (DEmo70MD\_fsD)
     }
      \ex []{
    \gll einen wunderschönen tag noch\\
         a wonderful day still\\
     \glt ‘Have a wonderful day’ (DEmo12MD\_fsD)
}
    \z
\z

\ea Resolution\\ \label{katsikaetal:fiftyoneresolution}
\gll es ist aber keiner verletzt worden\\
     it is but nobody hurt gotten\\
\glt ‘But nobody got hurt’ (USbi01FD\_fsD)
\z

Overall, German MSs and HSs mostly produced evaluations, personal statements and reactions in informal situations. In formal situations, the two groups differed in that HSs used mostly textual functions in their closings. This difference indicates that HSs behaved in a more “formal” way than MSs in formal communicative situations. 

\subsubsection{Closing discourse functions in Greek MSs and HSs}
The distribution of functions in Greek closings was similar across MSs and HSs in Germany and the US (see \figref{fig:katsikaetal:Greekfunctionsclosings}). 

\begin{figure}
    \centering
    \includegraphics[width=1\linewidth]{figures/Ch15_Figure_12_Functions in Greek closings.pdf}
    \caption{Distribution of functions per speaker group in Greek closings}
    \label{fig:katsikaetal:Greekfunctionsclosings}
\end{figure}

The analysis of functions per communicative situation showed that all three speaker groups mostly produced textual functions (codas and resolutions) in formal communicative situations, and the distribution of codas and resolutions was similar in the three groups. In contrast, distributional differences were found in informal situations. Specifically, MSs produced 57 (39\%) textual, 52 (36\%)  subjective, and 37 (25\%)  intersubjective functions. In HSs in Germany, informal discourse functions were equally distributed (intersubjective: 32/96 (33\%), subjective: 33/96 (34\%), textual: 31/96 (32\%)). HSs in the US had a similar distribution to MSs in that they produced more textual functions (31/58: 53\%) than subjective (20/58: 34\%) and intersubjective (7/58: 12\%), but they differed from MSs and HSs in Germany in that they did not produce any justifications in their informal closings. 

\ea Justification\\ \label{katsikaetal:fiftytwojustification}
\gll that argiso ligo tha kathisteriso\\
     will delay a.bit will delay\\
\glt ‘I will be a bit late, I will be delayed’ (DEbi15MG\_isG)
\z

The justifications in closings for MSs and HSs in Germany were mostly statements that they would be late, as shown in (\ref{katsikaetal:fiftytwojustification}, and often a repetition of the same justification function in the opening. In these cases, speakers opened and ended their narrations in a similar way. 

\subsubsection{Closing discourse functions in Russian MSs and HSs}
The analysis of the Russian closings showed that all groups mostly used subjective and textual functions in their closings (see \figref{fig:katsikaetal:Russianfunctionsclosings}).

\begin{figure}
    \centering
    \includegraphics[width=1\linewidth]{figures/Ch15_Figure_13_Functions in Russian closings.pdf}
    \caption{Distribution of functions per speaker group in Russian closings}
    \label{fig:katsikaetal:Russianfunctionsclosings}
\end{figure}

The analysis of functions across communicative situations showed that all groups used codas and resolutions to a similar extent in formal situations. In informal situations, HSs used more intersubjective functions such as reaction seeking (example \REF{katsikaetal:fiftythreereactionseek}) and valedictions than MSs (reaction seeking: MSs: 4/14 (29\%), HSs in Germany: 14/37 (38\%), HSs in the US: 18/42 (42\%); valediction: MSs: 2/14 (4\%), HSs in Germany: 15/37 (41\%), HSs in the US: 15/42 (36\%)). This shows that in informal situations, HSs tended to be more interactive than MSs.

\ea Reaction seeking\\ \label{katsikaetal:fiftythreereactionseek}
\gll a perezvoni mne\\
     and call.back me\\
\glt ‘And call me back’ (USbi14MR\_isR)
\ex Resolution\\ \label{katsikaetal:fiftyfourresolution}
\gll nu k.schast'ju nikto ne postradal\\
     well luckily no one was hurt\\
\glt ‘Well, luckily noone got hurt’ (RUmo07FR\_isR)
\ex Coda\\ \label{katsikaetal:fiftyfivecoda}
\gll vot vsjo chto ja videla\\
     here all that I seen\\
\glt ‘That's all I saw' (USbi18FR\_fsR)
\z

Regarding textual functions, MSs produced more resolutions (\ref{katsikaetal:fiftyfourresolution}) than HSs (MSs: 62/99 (63\%), HSs in Germany: 18/72 (25\%), HSs in the US: 26/74 (35\%)), whereas HSs produced more codas (\ref{katsikaetal:fiftyfivecoda}) than MSs (MSs: 29/99 (29\%), HS in Germany: 46/72 (64\%), HSs in the US: 39/74 (53\%)). Therefore, Russian HSs did not follow the Russian MS strategy of closing their narrations with resolutions as they produced more codas than resolutions in their closings. Future examination of the majority language texts (German, English) of Russian HSs will shed more light onto these results. The current results showed that German MSs produced slightly more codas than resolutions and thus the overall distribution of codas and resolutions was quite balanced in German MSs (see \sectref{sec:katsikaetal:Germanclosings}). Thus, the strategy of the Russian HSs in Germany to produce more codas than resolutions does not seem to stem from their majority language. A closer look into the distribution of codas and resolution across communicative situations showed that although Russian MSs overall produced more resolutions than codas, they actually produced more codas than resolutions in formal situations (codas: 57\%, resolutions: 20\%). Russian HSs in Germany produced more codas than resolutions both in formal (codas: 70\%, resolutions: 21\%) and informal situations (codas: 64\%, resolutions: 25\%). It may thus be the case that Russian HSs in Germany applied a formal situation strategy to informal situations. 

\subsubsection{Closing discourse functions in Turkish MSs and HSs}
The analysis of functions in the Turkish closings showed that all groups mostly produced textual and subjective functions to a similar extent. In addition, Turkish HSs produced fewer intersubjective functions than MSs. The distribution of functions in Turkish closings per speaker group can be seen in \figref{fig:katsikaetal:Turkishfunctionsclosings}. 

\begin{figure}
    \centering
    \includegraphics[width=.9\textwidth]{figures/Ch15_Figure_14_Functions in Turkish closings.pdf}
    \caption{Distribution of functions per speaker group in Turkish closings}
    \label{fig:katsikaetal:Turkishfunctionsclosings}
\end{figure}

The analysis of closing functions per communicative situation showed that textual functions had the highest frequency across all three groups in formal situations. There was only a small difference between MSs and HSs, as MSs produced a slightly higher amount of resolutions (MSs: 25/57 (44\%), HSs in Germany: 17/49 (35\%), HSs in the US: 12/34 (35\%)) whereas HSs produced more codas (MSs: 18/57 (32\%), HSs in Germany: 24/49 (49\%), HSs in the US: 15/34 (44\%)). In informal situations, all Turkish groups produced a large number of subjective functions, which was mostly evaluations (\ref{katsikaetal:fiftysevenevaluation}) and personal statements (\ref{katsikaetal:fiftysixpersonal}). The distribution of these functions was similar in all three groups: all produced more personal statements than evaluations (personal statements: MSs: 51/97 (52\%), HSs in Germany: 39/86 (45\%), HSs in the US: 32/65 (57\%); evaluations: MSs: 39/97 (40\%), HSs in Germany: 31/86 (36\%), HSs in the US: 17/56 (30\%)). Intersubjective functions were quite limited in informal communicative situations for all groups with reaction seeking being the relatively most frequent function. Regarding textual functions, all groups had a similar distribution of resolutions, codas and encapsulations, with resolutions being slightly more than codas.  

\ea Personal statement\\ \label{katsikaetal:fiftysixpersonal}
\gll film izlemiş gibi oldum bi dakikada\\
     movie watched like have.been in one.minute\\
\glt ‘I felt like watching a movie in a minute' (TUmo76MT\_isT)
\ex Evaluation\\ \label{katsikaetal:fiftysevenevaluation}
\gll adamın yüzünden bence kesinlikle o topla oynayan adamın yüzünden yani\\
     man's because.of I.think definitely that ball.with  playing man's because.of well\\
\glt ‘I think it's the guy's fault, it's definitely the guy playing with the ball' (DEbi05FT\_isT)
\z

In formal communicative situations, textual functions had the highest frequency, and we found only a small difference between MSs and HSs as MSs produced a slightly higher amount of resolutions (MSs: 25/57 (44\%), HSs in Germany: 17/49 (35\%), HSs in the US: 12/34 (35\%)) whereas HSs produced more codas (MSs: 18/57 (32\%), HSs in Germany: 24/49 (49\%), HSs in the US: 15/34 (44\%)). In sum, all three Turkish-speaking groups had a similar discourse function distribution in informal situations.

\subsubsection{Summary}
Overall, the analysis of closings showed a higher use of subjective functions in relation to openings, and very few differences across speaker groups. In formal communicative situations, no major differences were found between MSs and HSs in Greek, Russian and Turkish narrations. In German formal situations, HSs differed from MSs in that HSs did not produce any valedictions in spoken and written formal situations. In informal situations, German and Greek HSs in the US produced a high number of textual functions but only German HSs differed from MSs. Russian HSs differed from MSs in that they produced more intersubjective functions than MSs, and Turkish HSs did not show differences from MSs in the distribution of functions in informal situations. Finally, we did not find a consistent pattern across HS groups of different languages concerning the production of resolutions and codas. In German narrations, HSs produced more resolutions than MSs, in Russian narrations MSs produced more resolutions than HSs, and there were no differences between MSs and HSs in Greek and Turkish narrations. 

\section{Discussion and conclusions} \label{sec:katsikaetal:discussionconclusions}
The present study explored the distribution of openings and closings in narratives produced by monolingually raised speakers (MSs) and heritage speakers (HSs) of German, Greek, Russian, and Turkish across different communicative situations. Our main aim was to investigate the universality of macro discourse structures and explore the use of discourse functions in openings and closings across speaker groups and communicative situations. Research Question 1 aimed to explore whether MSs and HSs of German, Greek, Russian, and Turkish use openings and closings in their narrations with a similar frequency. Our analysis partially confirmed the results of previous research on openings and closings arguing for the optionality of openings and closings, especially in electronic interactions \parencite{baron_letters_1998, crystal_language_2001, herring_two_1996}. Our data analysis showed that although not all texts included an opening and a closing, openings occurred in our data with a frequency of 80\% and above. This result is similar to that of \textcite{bou-franch_openings_2011} who found 85\% of openings in Spanish e-mail conversations of various types. At the same time, other studies examining e-mail interactions such as the one by \textcite{waldvogel_greetings_2007} found a much lower occurrence of greetings in openings (59\%) (see also \cite{chapters/02}). We believe that the very high percentages of openings in our study are most possibly linked to the experimental method. Participants were asked to describe an incident in four different communicative situations (voice message to friend, instant message to friend, voice message to police, written report to police). It is therefore very possible that participants used the opening sequences to establish each communicative situation. 

We then further analyzed our data in order to investigate the discourse functions that are found in openings and closings (Research Question 2). We identified three types of function categories, intersubjective, subjective and textual~-- with each function category including several more specific discourse functions. This analysis revealed interesting patterns across speaker groups. In the German-speaking group, a greeting was the most common way to introduce an opening sequence and, similarly, a valediction was the most common way to end a closing sequence in MSs. HSs showed a similar pattern as MSs in openings, but not in closings where they produced more codas and resolutions than valedictions. Informal greetings in German spoken and written narrations were used to express closeness and familiarity (see \textcite{bou-franch_openings_2011} for an analysis of informal greetings in e-mail interactions), and formal greetings were used in formal communicative situations by MSs. In contrast, HSs did not use formal greetings in formal communicative situations, and this may constitute evidence for register leveling. A similar strategy was observed in Greek HSs in the US who used informal self\hyp identifications in formal situations. Our data suggest that HSs may use informal language in formal settings due to the familial context of heritage language acquisition. Heritage speakers acquire the language in the family and thus are exposed to more informal language use, and this can be the reason why they may use informal language in formal settings.

As pointed out by \textcite{wiese_heritage_2022}, setting (or “register”) differences between HSs and MSs should not be taken to mean that HSs are less “native” speakers than MSs because formality is not a determining factor for HSs’ “nativeness” \parencite[cf.][]{montrul_dominant_2012, polinsky_heritage_2018}. According to \textcite{wiese_heritage_2022}, formality distinctions are closely tied to social and communicative needs, which may vary among different social groups. This aspect is independent of whether someone is bilingual or monolingual, and in the case of monolinguals, it typically does not affect our perception of individuals as native speakers \parencite{wiese_heritage_2022}. The results of the present study show that in openings and closings, HSs follow formality conventions of the standard language to a large extent, and even in cases that they exhibit distinct patterns, this does not mean that their German language ability is more restricted in relation to German MSs.  

Evidence from the analysis of the Russian and the Turkish data did not show any register leveling effects in HSs. A main strategy found in Russian HSs in Germany was to use justifications in their opening sequences, that is, to provide a reason why they are calling or writing. This strategy was not prevalent in either the German or the Russian MSs, and thus may be a discourse strategy specific to HSs in Germany. In addition, Russian HSs in Germany and in the US use more intersubjective functions in their openings than MSs, and this may indicate that HSs are more interactive than MSs, or HSs feel much more confident using intersubjective functions in informal settings compared to formal settings and thus make extensive use of them. A large variety of intersubjective functions was also found in the openings of the Turkish-speaking group with a similar distribution across MSs and HSs. 

Turning to closings, a general founding was that closings were less frequent than openings in our data. In addition, there was more variation in the distribution of closings than in the distribution of openings across speaker groups. This result contrasts with previous research that showed a high frequency of closings (over 75\%) in e-mail interactions \parencite{bou-franch_openings_2011}. This difference is not surprising as e-mails have very different characteristics in general. Being quasi-synchronous in nature, IM always leaves the communication possibility open. In addition, following our argument that speakers in our study may have used the opening sequences to establish the communicative situation, it can be argued here that closings were not so frequent because the communicative situation had already been established through the opening. Previous studies have also found minimal occurrence of farewell closings present \parencite{herring_two_1996, waldvogel_features_2002}. Although valediction sequences were quite frequent in the German and Russian texts, they were not so frequent in Greek and Turkish texts. The main functions across all closings in the present study were evaluations of the incident that the participants described and personal statements, that is, switching the topic to something that focuses on the individual and not the situation that participants were asked to describe. In addition, closing sequences included many instances of textual functions such as resolutions and codas. In contrast to opening sequences, closing sequences included much fewer intersubjective functions across all speaker groups.

We further examined the role of age in the frequency of openings and closings (Research Question 3). Previous research has indicated age-related differences in openings and closings, particularly between child and adolescent groups \parencite{dollnick_entwicklung_2013, tolchinsky_text_2002}. Contrary to previous findings, our study did not show any statistically significant differences between adolescents and adults in the use of openings and closings. It can thus be that there are fewer differences between adolescents and adults than children and adolescents when it comes to macro discourse structures. Future analysis of discourse functions in openings and closings across adolescents and adults would shed more light on possible age differences in our data.  

In summary, the annotation and analysis of openings and closings in the narrative texts of German, Greek, Russian and Turkish MSs and HSs across four different communicative situations provided insights into the macro discourse strategies that MSs and HSs of different languages use to narrate events. Our results showed similar patterns as well as variation relating to cultural and language status differences across different speaker groups, and support the view that although core discourse functions may occur cross-linguistically, discourse functions are also language-specific \parencite[see e.g.][]{Marquez_telephone_2010}. We hope that the findings of this exploratory study will contribute to a better understanding of macro discourse structures in cross-cultural and language contact contexts. 

\section*{Acknowledgements}
This work would not have been possible without the contribution of the brilliant student assistants who spent many hours annotating, discussing issues, and providing valuable feedback. We would thus like to thank: Barbara Zeyer (German and English annotations), Aban Alehojat, Sadaf Rezai, Mateo Vargas-Nuñez (English annotations), Iliuza Akhmetzianova,  Ksenia Idrisova, Nadezhda Kushina (Russian annotations), Hannah Plückebaum, Josefine Hundelt, Franziska Groth (German annotations), Aikaterini Tsaroucha (Greek annotations and co\hyp presenting in AMLaP 2023). Special thanks to the Turkish annotation team: Mert Can, Oğuzhan Kuyrukçu, and Simge Türe for taking up the extra task to present in the PiF and HL@Cross conferences, and to Lea Coy for help with pre-publication formatting.  
We would also like to thank Nadine Zürn, Vicky Rizou and Anke Lüdeling for their feedback on a previous version of this chapter, as well as two anonymous reviewers and the audiences of PiF 2023, HL@cross 2023, and AMLaP 2023 conferences for their helpful and constructive comments and suggestions. 
This research was supported financially by the Deutsche Forschungsgemeinschaft (DFG, German Research Foundation) for the Research Unit \textit{Emerging Grammars in Language Contact Situations}, project P9 (grant number: 313607803).

\sloppy
\printbibliography[heading=subbibliography,notkeyword=this]
\end{document}
