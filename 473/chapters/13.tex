\documentclass[output=paper,colorlinks,citecolor=brown]{langscibook}
\ChapterDOI{10.5281/zenodo.15775185}


\author{Tatiana Pashkova\orcid{0000-0002-6676-9555}\affiliation{University of Kaiserslautern-Landau}  and  Marlene Böttcher\orcid{0009-0000-7538-2902}\affiliation{Kiel University}  and  Kalliopi Katsika\orcid{0000-0002-6736-4963}\affiliation{University of Kaiserslautern-Landau}  and  Sabine Zerbian\orcid{0000-0002-4631-369X}\affiliation{University of Stuttgart} and Shanley E. M. Allen\orcid{0000-0002-5421-6750}\affiliation{University of Kaiserslautern-Landau}
}


\title{Majority English of heritage speakers}

\abstract{Research on heritage speakers to date has focused largely on their heritage languages. In contrast, while their majority languages have been often examined in young children, less is known about adolescents and adults. Filling this gap, the current chapter presents our research on majority English of adolescent and adult heritage speakers of German, Greek, Russian, and Turkish compared to monolingually-raised English speakers. Using the RUEG corpus, we investigated four external interface areas prone to dynamicity in language contact – prosody, article use, reference, and clausal syntax in relation to information and discourse structure. In addition, we present findings on discourse organization. Based on our results, we discuss implications for majority language research.
\keywords{heritage speakers, majority English, prosody, article use and reference, syntax and discourse}
}

\IfFileExists{../localcommands.tex}{
   \addbibresource{../localbibliography.bib}
  \usepackage{langsci-optional}
\usepackage{langsci-gb4e}
\usepackage{langsci-lgr}

\usepackage{listings}
\lstset{basicstyle=\ttfamily,tabsize=2,breaklines=true}

%added by author
% \usepackage{tipa}
\usepackage{multirow}
\graphicspath{{figures/}}
\usepackage{langsci-branding}

   
\newcommand{\sent}{\enumsentence}
\newcommand{\sents}{\eenumsentence}
\let\citeasnoun\citet

\renewcommand{\lsCoverTitleFont}[1]{\sffamily\addfontfeatures{Scale=MatchUppercase}\fontsize{44pt}{16mm}\selectfont #1}
  
   %% hyphenation points for line breaks
%% Normally, automatic hyphenation in LaTeX is very good
%% If a word is mis-hyphenated, add it to this file
%%
%% add information to TeX file before \begin{document} with:
%% %% hyphenation points for line breaks
%% Normally, automatic hyphenation in LaTeX is very good
%% If a word is mis-hyphenated, add it to this file
%%
%% add information to TeX file before \begin{document} with:
%% %% hyphenation points for line breaks
%% Normally, automatic hyphenation in LaTeX is very good
%% If a word is mis-hyphenated, add it to this file
%%
%% add information to TeX file before \begin{document} with:
%% \include{localhyphenation}
\hyphenation{
affri-ca-te
affri-ca-tes
an-no-tated
com-ple-ments
com-po-si-tio-na-li-ty
non-com-po-si-tio-na-li-ty
Gon-zá-lez
out-side
Ri-chárd
se-man-tics
STREU-SLE
Tie-de-mann
}
\hyphenation{
affri-ca-te
affri-ca-tes
an-no-tated
com-ple-ments
com-po-si-tio-na-li-ty
non-com-po-si-tio-na-li-ty
Gon-zá-lez
out-side
Ri-chárd
se-man-tics
STREU-SLE
Tie-de-mann
}
\hyphenation{
affri-ca-te
affri-ca-tes
an-no-tated
com-ple-ments
com-po-si-tio-na-li-ty
non-com-po-si-tio-na-li-ty
Gon-zá-lez
out-side
Ri-chárd
se-man-tics
STREU-SLE
Tie-de-mann
}
   \boolfalse{bookcompile}
   \togglepaper[13]%%chapternumber
}{}


\begin{document}
\lehead{Tatiana Pashkova et al.}
\maketitle

\section{Introduction} \label{sec:pashkovaetal:Intro}

Most research on heritage speakers (HSs) to date focuses on their heritage language. In this chapter we focus on their other language – the majority language in the society – and ask whether the bilingualism resulting from heritage language (HL) maintenance leads to differences in the way heritage and monolingually\hyp raised speakers use the majority language (ML).

HSs  are bilinguals who typically speak one language within the family and other personal spheres (e.g., cultural or religious settings) – the HL – and another language that is the language of the larger society in which they live – the ML \parencite{pascualycabo2012, Montrul2016, MontrulPolinsky2021}. They typically begin acquiring the HL at birth, and begin learning the ML sometime between birth and the onset of schooling. Through formal schooling they often shift in dominance from the HL to the ML, such that they become dominant in the ML by adulthood. There is typically a wide variation across speakers in adult proficiency in the HL: they may retain full fluency, may gradually reduce dominance in HL so that they only comprehend but do not speak the HL, or anything in between.

The linguistic study of HSs has blossomed over the last two decades. However, as already noted, most research on HSs to date focuses on their HL \parencite{Montrul2008, Kupisch2013, Montrul2016, Polinsky2018book, MontrulPolinsky2021}. This is understandable from the cultural standpoint since there is considerable interest in supporting HL retention, and also from the linguistic research standpoint since it is both theoretically and practically relevant to explore what the HL looks like after dominance is lost. 

However, research on the ML is also valuable for at least two reasons. First, it contributes to our understanding of cross-linguistic influence and effects of bilingualism. Many studies on HSs have focused on the influence of the ML on the HL~-- in other words, the more dominant language on the less dominant one. 
However, research on L2 learners has highlighted that the influence also occurs in the opposite direction, from the less dominant language to a more dominant one (see, for example, \cite{Schoonbaert2007}, \cite{Chen2013} for late L2-to-L1 structural priming; \cite{Pavlenko2002}, \cite{Hohenstein2006}, \cite{Gorba2019} for late L2-to-L1 cross-linguistic influence).

Thus, we expect that research on the ML of HSs can provide insights into whether this type of cross-linguistic influence can also be attested in HL-to-ML situations. Moreover, we might observe general effects in the ML that are not traceable to any particular HL, which would add evidence that HL bilingualism in itself can lead to dynamic language patterns.

Second, ML research on adolescent and adult HSs is crucial to inform language education policies, which often argue against HL maintenance due to its harmful effects on ML ultimate attainment (e.g., \cite{mccardle2015}, \cite[30--33]{tracy2023}). By comparing the ML of adult and adolescent HSs to their MS counterparts, we aim to provide practical evidence that HL retention does not negatively affect the ML in the long-term perspective.

Most research on the ML focuses on children given the goal that children become competent in the ML to facilitate school achievement, with primary focus on the lexicon and morphology (e.g., \cite{Bialystok2010, Marinis2010, Paradis2017}). Research on the ML in adolescents and adults is also evident in the literature, but it is relatively less common. Most of it shows some differences between the ML as produced or comprehended by HSs compared to MSs. In the area of \textit{semantics}, for example, both \citet{Scontras2017} and \citet{Lee2011} showed differences in scope marking patterns in majority English between English MSs and Korean and Mandarin HSs, which the authors attribute to cross-linguistic influence from the scope patterns in Korean and Mandarin. With regard to \textit{phonology}, HSs are typically found to not have a detectably different accent from MSs of the ML (e.g., \cite{Kupisch2014a}, \cite{Lloyd-Smith2020}), although there is some evidence that they articulate more explicitly \parencite[142--144]{Polinsky2018book}. Findings for perception are mixed: \citet{Chang2016} finds that Korean HSs perceive unreleased stops more accurately in majority English than English MSs due to cross-linguistic influence from Korean, while \citet{Lee-Ellis2012} shows that HSs with the same language pair perform like English MSs in perceiving an English-specific contrast in nonce words. For prosodic aspects of the ML of HSs, cross-linguistic influence is more consistently evidenced. This was shown for phonetic and phonological aspects of rises in ML German for Turkish HSs \parencite{queen2012}, for broader prosodic features in the intonation of ML Dutch by Turkish HSs \parencite{VanRijswijk2017}, and for intonational aspects of polar questions in ML English by Icelandic HSs \parencite{Dehe2018}. 
 
In the domain of \textit{morphosyntax}, most but not all studies have found effects of bilingualism in the ML of HSs. For example, three studies on referent tracking found differences between MSs and HSs in the ML. \citet{azar2020reference} found that Turkish HSs produced more pronouns as compared to null forms for referent maintenance in majority Dutch compared to Dutch MSs, which can be viewed as higher explicitness of HSs than MSs in the ML. The authors attribute this explicitness to differences in likely input to MSs vs. HSs in the ML: HSs may have many more interlocutors who are L2 speakers of Dutch and who themselves are more explicit in their pronoun use. Contemori and colleagues \parencite{Contemori2021a, Contemori2023} found a somewhat opposite pattern -- that Spanish HSs often produced pronouns for referent tracking in majority English in contexts where English MSs would typically produce full NPs. They attributed this to cross-linguistic influence from Spanish, since pronouns would have been appropriate in these contexts in Spanish. Other differences between HSs and MSs were found in the use of definite articles in ML German by Turkish HSs \parencite{Felser2019}, acceptability judgements of definite and indefinite articles in ML English by Spanish HSs \parencite{MontrulIonin2010}, and use of subject-verb agreement in ML English by HSs with various HLs \parencite{Paradis2019}. However, \citet{MontrulIonin2010} found that Spanish HSs patterned like English MSs in some aspects of interpretation of definiteness and possession in majority English, while \citet{Lee-Ellis2012} found that Korean HSs patterned mostly like English MSs in their interpretation of locative alternation structures in majority English, and \citet{Kupisch2017} found that Turkish HSs performed like German MSs in their comprehension of subtle definiteness effects in majority German. Finally, two studies on a variety of morphosyntactic phenomena found that Spanish HSs patterned more closely to Swedish MSs in majority Swedish the higher their language aptitude \parencite{Bylund2012} and the earlier their age of onset of Swedish \parencite{Bylund2021}.

In sum, most of the existing research on the ML use of HSs across linguistic domains shows some cross-linguistic influence from their HL or from language contact in general, although some research also shows no differences between HSs and MSs in ML use. Possible reasons for divergent results of various studies are the different language areas and different languages examined, different methods employed and different speakers tested. Our work reduces this methodological variation to different language areas only, while keeping the languages (majority English in contact with heritage German, Greek, Russian, and Turkish), the method (examination of elicited narratives)\footnote{ {Except for one study on article use reported in \sectref{sec:pashkovaetal:Articles}, which used a written fill-in-the-blank test.}} and the speakers the same. This approach will show us whether there exists a more consistent pattern of difference/similarity between HSs and MSs in the ML.

Furthermore, our research addresses several open theoretical questions in the majority language literature. First, do we see consistent influence on majority English from typologically different HLs if speakers of these HLs  are tested in the same methodological  paradigm? Only a few studies investigate more than one HS group \parencite{Kupisch2014a, Polinsky2018book, Paradis2019, labrenz2023}. This is important because it is only through comparing across HS groups that we can determine if a pattern really derives from cross-linguistic influence (different effect across groups depending on properties of HL) or from effects of bilingualism or language contact in general (similar effect across groups regardless of HL). 

Second, is the effect of HS bilingualism equally evident in different registers of the ML and do HSs separate the registers in the ML the same way as MSs? The role of register has been examined in the HL, with some indications of register leveling (e.g., \cite{alexiadou2022, alexiadou2023, tsehaye2023}). However, few if any studies have investigated register differentiation in the ML (see \cite{labrenz2023} as an example of a recent RUEG-based study), so it is unclear if registers are dynamic not only in the HL but also in the ML. 

Third, what differences between HSs and MSs can we discover outside of a strict experimental setting and in a more ecologically valid set up? Many of the previous ML studies that found a difference between HSs and MSs used carefully designed experiments to test specific language phenomena (e.g., \cite{Lee2011} and \cite{Scontras2017} on scope; \cite{Contemori2021a} on referring expressions; \cite{Felser2019} on definiteness; \cite{Paradis2019} on subject-verb agreement; \cite{Bylund2012, Bylund2021} on various phenomena; \cite{MontrulIonin2010} on article choice; \cite{VanRijswijk2017} on prosody). Some authors highlighted that the experiments were designed to be very demanding even for MSs \parencite{Bylund2012, Paradis2019}. In contrast, we are aware of only a few studies that discovered a difference between HSs and MSs in the ML in an elicited narrative set up (\cite{azar2020reference, Contemori2023} on referring expressions). Other studies that used naturalistic data from tasks not specifically targeting a certain linguistic phenomenon (elicited narratives, interviews, picture descriptions) either did not find a difference between HSs and MSs in the ML (\cite{Kupisch2014a} and \cite{Lloyd-Smith2020} on perceived foreign accent) or did not provide a conclusive comparison of HSs and MSs (\cite{Kupisch2014} on adjective placement; \cite{lein2016} on VOT production; \cite{queen2012} on intonation). 

In the present chapter, we address these gaps by reviewing several studies examining the production patterns in majority English of German, Greek, Russian and Turkish HSs living in the USA. These studies were completed in the context of the six-year research project Research Unit \textit{ Emerging Grammars} (RUEG) funded by the German Research Foundation (FOR2357). Each of these studies examines a structure at the interface between a core area of language (e.g., phonology, morphology, syntax) and a non-core area of language (e.g., pragmatics, discourse). This follows the Interface Hypothesis \parencite{Sorace_2011, tsimpli2014}, which states that phenomena at the interface between core and non-core areas of language are particularly open to the development of dynamic patterns in bilinguals. Interface structures crucially involve the integration of knowledge between different areas of language, and thus engender a higher cognitive load for processing than structures involving only one area of language. Each study looks for dynamic patterns in the ML productions of HSs and examines them in the light of potential sources of the pattern, including cross-linguistic influence or transfer from the HL to the ML as well as the presence of a language contact situation regardless of the particular languages involved (e.g., increased explicitness compared to MSs). Crucially, the data were collected via elicited narrative in four different communicative situations leading to production of different registers, thus allowing us to see productions in a more ecologically valid context than the experimental setups used in many of the studies to date. Finally, data from the same participants were analyzed in each of the studies on different linguistic phenomena, thus reducing inter-individual variation between studies. 

The remainder of the chapter is structured as follows. \sectref{sec:pashkovaetal:Corpus} describes the RUEG corpus \parencite{RUEGcorpus2024}~-- the source of most of the ML data used in the studies presented in this chapter. \sectref{sec:pashkovaetal:Prosody} looks at HSs’ prosody, \sectref{sec:pashkovaetal:Reference} examines reference, \sectref{sec:pashkovaetal:Articles} covers article use, \sectref{sec:pashkovaetal:Syntax} explores syntax, and \sectref{sec:pashkovaetal:Discourse} turns to openings and closings in discourse. \sectref{sec:pashkovaetal:Discussion} critically discusses the reported findings and draws conclusions for heritage language and bilingualism research.

\section{Corpus} \label{sec:pashkovaetal:Corpus}
  The studies reported in this chapter are based on the data from the English subcorpus of the RUEG corpus \parencite{wiese2021}. The subcorpus contains the English data of 223 HSs with English as their majority language and 64 English MSs, all of whom were raised in the USA and resided there at the time of testing. The HS group includes 34 German, 65 Greek, 65 Russian, and 59 Turkish HSs. A total of 91\% of HSs were first exposed to English at the age of 5 or earlier, with 46\% having the first contact with English from birth. All speakers -- HSs and MSs -- comprise two age groups, adolescents (13--18 years old) and adults (20--37 years old); see \tabref{tab:pashkovaetal:ages}.

Speakers were interviewed in large cities in the Eastern United States, or in nearby areas. These included New York City; Washington, D.C.; Chicago, Illinois; Boston, Massachusetts; Madison, Wisconsin; St. Paul and Minneapolis, Minnesota; Long Island, New York; Fort Lee and Bloomfield, New Jersey; and New Haven, Connecticut.

 
\begin{table}
\small
  \fittable{\begin{tabularx}{\textwidth}{l*{6}{C}}
    \lsptoprule
    & \multicolumn{3}{c}{Adolescents} & \multicolumn{3}{c}{Adults} \\
    \cmidrule(lr){2-4} \cmidrule(lr){5-7}
    Group & $N$ (Male) & Age Mean (sd) & Eng. AoO Mean (sd) & $N$ (Male) & Age Mean (sd) & Eng. AoO Mean (sd) \\
    \midrule
    English MSs & 32 (13) & 16.1 (1.4) & - & 32 (13) & 28.5 (3.9) & - \\
    German HSs & 27 (15) & 5.5 (1.5) & 0.3 (0.7) & 7 (2) & 25.3 (4.1) & 0.9 (1.5) \\
    Greek HSs & 33 (16) & 16.3 (1.4) & 1.1 (1.8) & 32 (13) & 29.1 (3.4) & 1.0 (1.8) \\
    Russian HSs & 32 (13) & 15.8 (1.4) & 2.5 (2.0) & 33 (11) & 27.5 (3.3) & 3.7 (2.0) \\
    Turkish HSs & 32 (10) & 16.0 (1.6) & 2.6 (2.1) & 27 (9) & 26.2 (4.1) & 2.2 (2.2) \\
    \midrule
    Total & 156 (67) & 15.9 (1.5) & - & 131 (48) & 27.7 (3.8) & - \\
    \lspbottomrule
  \end{tabularx}}
  \caption{Characteristics of speakers from the English subcorpus of the RUEG corpus}
    \label{tab:pashkovaetal:ages}
\end{table}

The data were collected using the Language Situations methodology (\cite{chapters/02, wiese_2020_langsit}), which allows eliciting comparable and naturalistic productions across communicative situations. Participants were shown a brief non-verbal video of a minor car accident and were asked to recount what they saw. The procedure consisted of two formality settings. In the formal setting, the participants were met by a formally dressed elicitor in an office-like room, whereas in the informal setting, the elicitor dressed casually and met the participant in a more relaxed environment, offering snacks and drinks. Prior to the informal session, the participant and the elicitor engaged in 10--15 minutes of conversation to create an easy-going atmosphere. The participant watched the video three times in total, twice in the first setting, once in the second setting, and recounted the event in spoken and written modes.

In the formal setting, participants left a voice message to a police hotline (spoken mode) and provided a written witness report to the police (written mode). In the informal setting, participants recorded a WhatsApp voice message to a friend (spoken mode) and typed a WhatsApp text message to a friend (written mode). The order of settings (formal and informal) and modes (spoken and written) was balanced among participants. English MSs accomplished all the tasks during one session. HSs completed the tasks in two sessions, one in English, their ML, and one in their HL, with a 3--5 day interval between sessions to reduce priming effects. The order of the language sessions was counterbalanced among HSs. After completing the narrative tasks, all participants filled out a language background questionnaire, which included a self-assessment of their language skills. The self-assessments showed that HSs and English MSs rated their English skills comparably high. In addition, HSs assessed their skills higher in their majority English than in their HL (\tabref{tab:pashkovaetal:ratings}).

The spoken data were transcribed in Praat \parencite{Boersma2001} or EXMARaLDA \parencite{schmidt2014}. Subsequently, spoken and written data were annotated in EXMARaLDA for various phenomena of interest. The annotated data were accessed through the corpus tool ANNIS (\cite{krause20160}; see \cite{chapters/03} and \citet{Klotzetal2024} for further detail on the RUEG corpus).

To assess speakers’ fluency and proficiency in English, we calculated their speech rate and lexical diversity (\tabref{tab:pashkovaetal:prof}). Speech rate (in syllables per second) was calculated using a Praat script \parencite{DeJong2021} based on all spoken narratives in the English subcorpus. Lexical diversity was evaluated with two measures -- the moving-average type-token ratio (MATTR) and measure of textual lexical diversity (MTLD), demonstrated to produce stable results for short texts \parencite{Zenker2021}. MATTR and MTLD were calculated based on all narratives in the English subcorpus using the lexical diversity package in Python \parencite{kyle_lex_div_2020}


\begin{table}
\small
\begin{tabularx}{\textwidth}{lYYYY}
\lsptoprule
Group & 
{Spoken understanding Mean (sd)} & {Spoken production Mean (sd)} & {Written understanding Mean (sd)} & {Written production Mean (sd)} \\
\midrule
\multicolumn{5}{l}{\textbf{In majority English}} \\
\midrule
English MSs & 4.92 (0.27) & 4.81 (0.47) & 4.80 (0.48) & 4.77 (0.46) \\
German HSs  & 5.00 (0.00) & 5.00 (0.00) & 5.00 (0.00) & 4.97 (0.17) \\
Greek HSs   & 4.97 (0.18) & 4.95 (0.21) & 4.88 (0.33) & 4.89 (0.31) \\
Russian HSs & 4.92 (0.32) & 4.92 (0.27) & 4.91 (0.29) & 4.89 (0.36) \\
Turkish HSs & 4.97 (0.18) & 4.93 (0.25) & 4.92 (0.38) & 4.92 (0.34) \\\tablevspace

\multicolumn{5}{l}{\textbf{In heritage languages}} \\
\midrule
German HSs & 4.47 (0.56) & 3.65 (0.85) & 3.71 (1.06) & 2.88 (1.27) \\
Greek HSs   & 4.16 (0.78) & 3.72 (1.05) & 3.46 (1.20) & 3.13 (1.29) \\
Russian HSs & 4.38 (0.75) & 3.87 (0.85) & 3.25 (1.05) & 2.83 (1.22) \\
Turkish HSs & 4.28 (0.83) & 3.90 (0.95) & 3.37 (1.05) & 3.30 (1.16) \\
\lspbottomrule
\end{tabularx}
\caption{Language skill self-ratings of speakers from the English subcorpus of the RUEG corpus}
    \label{tab:pashkovaetal:ratings}
\end{table}


\begin{table}
\begin{tabularx}{\textwidth}{l YYY}
\lsptoprule
Group & Speech rate Mean (sd) & MATTR Mean (sd) & MTLD Mean (sd) \\
\midrule
English MSs & 3.26 (0.52) & 0.678 (0.056) & 38.80 (11.19) \\
German HSs & 3.34 (0.46) & 0.678 (0.049) & 40.43 (10.45) \\
Greek HSs & 3.23 (0.44) & 0.670 (0.053) & 37.66 \phantom{1}(9.85) \\
Russian HSs & 3.20 (0.47) & 0.672 (0.049) & 38.54 (10.06) \\
Turkish HSs & 3.22 (0.45) & 0.659 (0.050) & 36.10 (10.63) \\
\lspbottomrule
\end{tabularx}
\caption{Speech rate (syll/sec), MATTR, and MTLD of speakers from the English subcorpus of the RUEG corpus (in majority English)}
\label{tab:pashkovaetal:prof}
\end{table}

Most studies in this chapter examined the data of various subsets of speakers from the English subcorpus. The exact subset is specified in the description of each study. Furthermore, the studies used different corpus versions (e.g., 0.3.0 or 0.4.0), which is detailed in the cited papers. If a study did not employ the English subcorpus, the characteristics of the data set are provided in the study’s description.


\section{Prosody} \label{sec:pashkovaetal:Prosody}

As noted earlier, previous studies have found that HSs tend to pattern with MSs in rating studies assessing their perceived foreign accent in the ML (e.g., \cite{Kupisch2014a}, \cite{Lloyd-Smith2020}, but see \cite[138--153]{Polinsky2018book} for a review of segmental and suprasegmental differences). However, they often show effects of cross-linguistic influence in ML prosody \parencite{queen2012, VanRijswijk2017, Dehe2018}. In the RUEG project we contributed to research on prosodic interference in the ML of HSs (here majority English), investigating two phenomena~-- stressed pronouns and contrastive adjectives -- at the interface of prosody and information structure. 

One line of research looked at the prosodic realization of stressed pronouns in English MSs and HSs speaking majority English (for details, see \cite{bottcher2021}, \cite{bottcher2020}). In English, pronouns as discourse-given constituents are generally unaccented; thus, prominence on pronouns (stress) indicates focus or contrast \parencite{krifka2008, Selkirk1996}, as illustrated in (\ref{ex:pashkovaetal:key:1}) (small caps indicate prosodic prominence in the form of a pitch accent).

\ea
\label{ex:pashkovaetal:key:1}
There was this couple walking down the sidewalk.
    \begin{enumerate}
        \item[a.] They\textsubscript{\textsc{given}} were about to cross the street.
        \item[b.] [\textsc{he}]\textsc{\textsubscript{contrast}} had a ball while [\textsc{she}]\textsc{\textsubscript{contrast}} was pushing a baby carriage.
    \end{enumerate}
\z 

In a pilot study, \citet{bottcher2020} investigated the realization of pronouns in majority English by six English MSs and 12 HSs (three Turkish, four Russian,  and five Greek). The focus of this investigation was to analyze and explain the occurrence of stressed pronouns.  Pronouns are not expected to be frequently stressed as given constituents \parencite{krifka2008, Selkirk1996}. However, the prosodic realization of given constituents is reportedly different in spontaneous compared to read speech \parencite{DeRuiter2015}. The spontaneous narrations from the RUEG corpus (two by each speaker, formal and informal) were, therefore, analyzed for the occurrence of stressed pronouns, prosodic phrasing, and information structure (following \cite{krifka2008}, \cite{BaumannRiester2013}, \cite{Himmelmann2018}). The analysis revealed that stressed pronouns were produced frequently in spontaneous narrations by HSs and English MSs. Contrast was the most frequent factor for stressed pronoun realization ($n = 52$, 38\%), while prosodic phrasing was also important (separate or small phrase: $n = 30$, 22\%; phrase finally: $n = 16$, 12\%; and phrase initially: $n = 33$, 25\%). 

In a subsequent study, \citet{bottcher2021} explored whether HSs and English MSs differed in their prosodic realization of contrastively focused and not focused pronouns in majority English. This was done based on 40 narrations by 20 speakers from the RUEG corpus 0.3.0. -- four English MSs and 16 HSs, with four speakers each of heritage Greek, Turkish, Russian, and German. Pronouns were analyzed for their information structure \parencite{gotze2007} and their prosodic realization \parencite{beckaman_hirschberg2006, kuglerbaumann2019}. Overall, speakers show similar tendencies of realizing pronouns: unstressed if unfocused and stressed if contrastively focused (\figref{fig:pashkovaetal:1}). Overall, there is a tendency for fewer stressed pronouns in the narrations of HSs, independently of contrastive focus (see \cite{bottcher2021} for further detail). In the case of contrastive focus, only one pronoun in a monolingual production was unstressed, while heritage speakers leave these pronouns unstressed in about half of the cases. Since especially Turkish HSs produce contrastively focused pronouns frequently without stress, it is a topic for further research whether there is a relation between the absence of prosodic prominence and the pro-drop features of the languages in question (see \cite{chapters/06}).

In sum, English MSs and HSs speaking majority English do not show qualitatively different patterns in the realization of stressed pronouns: both speaker groups stress pronouns in spontaneous speech, dependent on both information structure and prosodic phrasing (for majority German, see \cite{zerbian_boettcher_2019, bottcherschubo2021}). However, there is a difference in distribution of unstressed pronouns: HSs do not always stress contrastively-focused pronouns in majority English and thus do not follow the pragmatic constraints that hold in English in all cases.


\begin{figure}
    \centering
    \includegraphics[width=\textwidth]{figures/Ch13_figure_1_wide.pdf}
    \caption{The occurrence of unstressed and stressed pronouns (ratio and in numbers) in the case of presence or absence of contrastive focus in English across different speaker groups in RUEG corpus 0.3.0}
    \label{fig:pashkovaetal:1}
\end{figure}

The second line of research investigated the relationship between prosody and contrastive focus of adjectives. The prosodic realization of modified NPs, such as \textit{a blue car}, is dependent on focus and contrast, as shown in examples (\ref{ex:pashkovaetal:key:2}a-d). The placement of a pitch accent on the noun (indicated by small caps) allows for various focus interpretations while a pitch accent on the adjective (as in \ref{ex:pashkovaetal:key:2}c) can only indicate narrow or contrastive focus on the adjective \parencite{Ladd2008}.

\ea%2
    \label{ex:pashkovaetal:key:2}
Pitch accent placement in modified noun phrases and its interpretation
\begin{enumerate}
    \item [a.] wide/broad focus: Along came [a white \textsc{car}]\textsubscript{FOCUS}.
    \item[b.] contrastive focus on noun: There was a white [\textsc{van}]\textsubscript{FOCUS} and a white [\textsc{car}]\textsubscript{FOCUS.}
    \item[c.] contrastive focus on adjective: The [\textsc{blue}]\textsubscript{FOCUS} car crashed into the [\textsc{white}]\textsubscript{FOCUS} car.
    \item[d.] contrastive focus on adjective and noun: The [\textsc{blue}]\textsubscript{FOCUS} [\textsc{car}]\textsubscript{FOCUS} rammed the [\textsc{white}]\textsubscript{FOCUS} [\textsc{van}]\textsubscript{FOCUS}.
\end{enumerate}
\z 

\citet{bottcher2022} investigated the accent placement in noun phrases containing adjectives in English. Data analysis included 144 narrations from the RUEG corpus (0.4.0) by adolescent English MSs and Greek and Russian HSs speaking majority English. Each group comprised 24 speakers, each producing two narratives. Narratives were searched for modified noun phrases including the lexeme \textit{car} and were analyzed for semantic contrast \parencite{gotze2007} and for the location of pitch accents and pitch accent type (ToBI guidelines, \cite{beckaman_hirschberg2006}). As expected, the contrastive adjectives were accentuated in most cases (\figref{fig:pashkovaetal:2}). Surprisingly, adjectives were also accented when not explicitly expressing a contrast, which might still reflect the speaker’s contrasting perspective \parencite{Kaland2014}. Additionally, contrastively accented adjectives were also frequently followed by an accented noun. Such a double accent pattern was found more frequently in formal speech and in the speech of Russian HSs. The double accent also occurred frequently in the Russian heritage language of these HSs indicating a possible transfer (see \cite{chapters/12, zerbian2022}). The analysis of pitch accent types as high (H*) or rising tone accents (LH*) did not show differences across speaker groups (see \cite{bottcher2022} for further detail).

The results of this study reveal that English MSs and HSs speaking majority English do not show qualitatively different patterns in the prosodic realization of contrastively focused adjectives: the adjectives generally carry a pitch accent, although at the same time double accent patterns are more frequently produced by Russian HSs. In the light of the higher use of double accents in formal narrations by MSs and Greek HSs, this can possibly be interpreted as an effort towards being linguistically -- in this case prosodically -- more explicit in the context of formal situations (see also next section, for HSs see \cite[144]{Polinsky2018book}).


\begin{figure}
    \centering
    \includegraphics[width=\textwidth]{figures/Ch13_figure_2_norm.pdf}
    \caption{Distribution of accent patterns within modified noun phrases containing a contrastively focused adjective in English across speaker groups and communicative situations in the RUEG corpus 0.4.0}
    \label{fig:pashkovaetal:2}
\end{figure}


Overall, our results suggest that English MSs and HSs speaking majority English use similar prosodic strategies in realizing given pronouns and contrastive focus on adjectives, thus using available linguistic features along a continuum (cf. \cite{wiese2022}). Given that both structures are at the interface of prosody and semantics/pragmatics, the Interface Hypothesis \parencite{Sorace_2011} predicts these areas to be vulnerable and thus prone to change in language contact. However, our results do not lend strong support for this, at least in the ML.


\section{Reference} \label{sec:pashkovaetal:Reference}

Previous research has indicated that bilinguals might be more explicit than monolinguals in the domain of reference. In referent introduction, \citet{Barbosa2017} have found that simultaneous bilingual children lexicalize more concepts in their narratives than monolinguals. In referent tracking, adult late L2 speakers tend to use full NPs instead of pronouns, compared to their L1 \parencite{Hendriks2003, Gullberg2006} and compared to L1 speakers of their L2 \parencite{Hendriks2003, Yoshioka2008}. In addition, adult late L2 speakers and simultaneous bilingual children tend to overproduce and over-accept overt pronouns instead of null forms in pro-drop languages, compared to monolinguals \parencite{Serratrice2004, Sorace2006, Sorace2009a}. Finally, adult Turkish HSs use more overt pronouns in their majority Dutch (a non-pro-drop language) compared to Dutch MSs \parencite{azar2020reference}. However, not all previous findings support this trend. \citet{Contemori2023} and \citet{Contemori2021a} found that Spanish HSs produced fewer nouns and more pronouns in their majority English compared to English MSs, and hence were less explicit.

Despite the disagreement in the literature, Pashkova and colleagues hypothesized that HSs might be more explicit either due to their frequent communication with L2 English speakers, who might benefit from extra detail \parencite[144]{Polinsky2018book} or due to more frequent input from L2 English speakers, who themselves might be more explicit (for results in majority Dutch see \cite{azar2020reference}). We conducted two studies to test this hypothesis: on referent introduction \parencite{pashkova2020bucld} and on referring expressions \parencite{pashkovainprep_a}.

In the referent introduction study, \citet{pashkova2020bucld} compared the number of introduced referents in a subset of the English subcorpus: 32 adolescent English MSs and 120 HSs (27 German, 31 Greek, 32 Russian, 30 Turkish). We identified 19 frequent referents, and coded each referent as introduced (lexicalized) or not in each narrative. Even though referent introduction is affected by various extra-linguistic factors (e.g., attention and memory play a role in which characters/objects will be mentioned in a narrative), it is an important first step of establishing reference \parencite[1]{Vogels2019}. A higher proportion of introduced referents would be a sign of higher explicitness since the narrative would provide more detail about the events. Contrary to the hypothesis, a linear mixed model analysis showed no significant difference between HSs and English MSs in the number of introduced referents in majority English. In addition, we observed that HSs and MSs had the same effect of formality: both speaker groups introduced more referents in the formal setting than in the informal one.

The study of referring expressions \parencite{pashkovainprep_a} consisted of two parts: modification of referring expressions and their form. In the modification part, we were interested whether HSs speaking majority English would modify referring expressions more often than English MSs, for instance, with a prepositional phrase (\textit{the woman with the dog}) or a relative clause (\textit{the woman who had a dog}). A higher proportion of modified referring expressions would be an indication of higher explicitness because it would provide more information about the referent of the NP. We conducted a linear mixed effects model analysis based on the full English subcorpus. Contrary to our expectations, there was no significant difference between HSs and English MSs in the proportion of modified referring expressions. In addition, we observed that HSs and MSs had the same effect of formality: both speaker groups modified referring expressions more often in the formal setting than in the informal one.

In the part concerning the form of referring expressions, we hypothesized that HSs speaking majority English would be more explicit than English MSs in their use of expressions for referents that have been previously introduced; new referents are typically introduced by noun-headed NPs, so they do not exhibit much variation in referent form. We examined three types of referring expressions: noun-headed NPs (\textit{the woman had a dog}), pronouns (\textit{she also had some groceries}), and null anaphora (\textit{she got scared and ${\emptyset}$ dropped her groceries}). Each referent from the list of 19 frequent referents was annotated for its type of referring expression, irrespective of its syntactic position (i.e., subject, object, object of a preposition, etc.). Following \citet{azar2020reference}, we conducted two comparisons: (1) noun-headed NPs vs. pronouns and (2) pronouns vs. null anaphora. We expected that HSs would use more noun-headed NPs than MSs in the first comparison and more pronouns in the second comparison.

The first comparison was performed on the full English subcorpus, where each referent was annotated for its type of referring expression. The second comparison was done on a subset of the English subcorpus that included 40 English MSs, 42 Turkish and 40 Russian HSs (each speaker group equally split between adolescents and adults). The subsetting was done due to the necessity to annotate each referent for the type of clause it was used in -- main or subordinate~-- since null anaphora can occur only in coordinate main clauses. 

The confirmatory analyses with linear mixed effects models did not indicate a significant difference between MSs and HSs combined, neither in the noun-headed NP vs. pronoun comparison nor in the pronoun vs. null anaphora one. This shows that HSs in our sample are not more explicit than MSs across the board. 

However, an exploratory analysis of noun-headed NP vs. pronoun referring expressions showed an interesting pattern that we had not previously hypothesized. This pattern has to do with formality and with a distinction that speakers make between two types of referent tracking -- maintenance and reintroduction, which in our study are defined based on the syntactic roles of the referent in the current and previous finite clauses (following \cite{Hickmann1999, Serratrice2007, Perniss2015}). The definitions of maintenance and reintroduction are presented in \tabref{tab:pashkovaetal:maint}, along with examples.

\begin{table}
\begin{tabularx}{\textwidth}{p{0.3\textwidth} p{0.3\textwidth} p{0.3\textwidth}}

\lsptoprule 
Previous clause & Current clause & Referent tracking type \\

\midrule 
\makecell[l]{Subject\\\textit{\textbf{The dog} ran}.} & 
\makecell[l]{Any syntactic role\\\textit{The driver saw \textbf{it}}.} & 
\makecell[l]{Maintenance:\\Referent \textit{dog} in current \\clause is maintained} \\\tablevspace

\makecell[l]{Non-subject\\\textit{The driver saw \textbf{the dog}}.} & 
\makecell[l]{Non-subject\\\textit{He didn't want to hit \textbf{it}}.} & 
Maintenance \\\tablevspace

\makecell[l]{Non-subject\\\textit{The driver saw \textbf{the dog}}.} & 
\makecell[l]{Subject\\\textit{\textbf{The dog} ran}.} & 
\makecell[l]{Reintroduction:\\Referent \textit{dog} in current \\clause is reintroduced} \\\tablevspace

\makecell[l]{Absent\\\textit{The driver turned right}.} & 
\makecell[l]{Any syntactic role\\\textit{He saw \textbf{the dog}}.} & 
Reintroduction \\
\lspbottomrule 
\end{tabularx}
\caption{Definitions and examples of maintenance and reintroduction}
\label{tab:pashkovaetal:maint}
\end{table}

The exploratory analysis individually compared German, Greek, Russian, and Turkish HSs to English MSs, as opposed to comparing all HSs together to MSs in the confirmatory analysis. Although there was considerable variability across speakers, the results (\figref{fig:pashkovaetal:3}) indicated two trends: first, in the formal setting, there was a larger difference between the proportions of full NPs in maintenance and reintroduction in narratives by German, Russian and Turkish HSs compared to English MSs (difference of 0.54 for German HSs, 0.48 for Russian HSs, 0.53 for Turkish HSs vs. 0.41 for English MSs). Second, in the informal setting, Russian and Turkish HSs produced significantly more NPs both in maintenance and reintroduction compared to English MSs. German and Greek HSs followed the same pattern but the difference did not reach significance.

\begin{figure}
    \includegraphics[width=\textwidth]{figures/Ch13_figure_3.pdf}
    \caption{Predicted probabilities and individual proportions of noun-headed NPs by speaker group, formality and referent tracking. Predicted means and CIs are based on a linear model, individual proportions on raw data.}
    \label{fig:pashkovaetal:3}
\end{figure}
 
These findings suggest that HSs taken together as one group are not more explicit than MSs across all contexts. The situation is more nuanced: when speaking or writing formally, some HS groups tend to separate maintenance and reintroduction more sharply than MSs. This is similar to a trend that we will present in \sectref{sec:pashkovaetal:Syntax} -- there, heritage speakers show a tendency of a sharper distinction between formal and informal settings in their use of subordinate clauses and left dislocations, in comparison to MSs.

However, HSs also show a different tendency when speaking or writing informally: some HSs are more explicit in that they use more full NPs than MSs both in maintenance and reintroduction. Overall, we see that HSs employ different strategies in the formal and informal settings: in a formal situation, they highlight the difference between maintenance and reintroduction, and in an informal situation they prefer to be more explicit regardless of the referent tracking status. We will come back to this trend in the general discussion.


\section{Article use} \label{sec:pashkovaetal:Articles}

We investigated article use by HSs speaking majority English and English MSs in narrative data from the English subcorpus and in an additional experimental study focusing specifically on article choice. The study on article use in narratives compared the proportions of unexpected articles in new and given referents based on the full English subcorpus \parencite[10--14]{wiese2022}. We extracted all strictly new referents, that is, referents that were mentioned for the first time and did not have any related anchor in the previous discourse (such as \textit{the car} would be for \textit{the driver}). We also extracted all given referents that were already mentioned in the previous discourse. Next, we identified the referents that featured unexpected article choice, that is, “\textit{the} + new” and “a + \textit{given}” referents. All other referents, including the most expected (“\textit{a} + new” and “\textit{the} + given”), were marked as “other.” Comparing HSs to MSs, we observed no effect of bilingualism: the two groups produced similar proportions of unexpected “\textit{the} + new” referents out of all new referents and of unexpected “\textit{a} + given” referents out of all given referents. In addition, we found that HSs and MSs were influenced by spoken/written mode in a similar fashion: irrespective of the speaker group, there were significantly more unexpected “\textit{the} + new” and “\textit{a} + given” referents in the spoken mode than in the written mode. Overall, this study indicated no evidence of differences between HSs and English MSs in majority English, with both groups producing a small proportion of unexpected articles with new and given referents, especially in the spoken mode.

The second study on article use \parencite{pashkovainprep_b} featured more constrained contexts and is the only study mentioned in this chapter that did not use the RUEG corpus data. Speakers were given a fill-in-the-blank task eliciting determiners, where each item was a short dialogue between two people (adopted from \cite{Ionin2008}). This study tested the Fluctuation Hypothesis (originally developed for late L2 speakers; \cite{Ionin2006}) on adult and adolescent HSs speaking English as their ML. The Fluctuation Hypothesis posits that L2 English speakers whose L1 does not have definiteness-based articles (e.g., Korean) fluctuate between definiteness and specificity in their article choice in English: sometimes they attend to definiteness of the NP, and sometimes to specificity, resulting in productions that are different from those of L1 English speakers. On the other hand, L2 English speakers whose L1 has definiteness-based articles (e.g., Spanish) are expected to perform similarly to L1 English speakers and not exhibit any fluctuation.

HSs are similar to late L2 speakers because both speaker groups learn English while having experience of using an additional language (HL for HSs, and the only L1 for L2 speakers), so similar transfer effects could be observed in HSs as well. We hypothesized that Russian HSs in our sample ($n=63$) would exhibit the most fluctuation since their HL does not have articles, followed by Turkish HSs ($n=58$), whose HL has a complex definiteness- and specificity-based differential object marking system. Finally, we predicted that Greek HSs ($n=78$) would be the most similar to English MSs ($n=46$) since Greek has definiteness-based articles.

The results did not support the Fluctuation Hypothesis: none of the HS groups exhibited the fluctuation pattern attested in late L2 speakers. As a sole result, this would be quite unsurprising since exposure to English by the HS begins much earlier than that of typical late L2 speakers. However, we also observed two unpredicted patterns: first, all speaker groups, including MSs, produced more unexpected articles (\textit{the}) in the indefinite contexts (slightly above or under 1\% in indefinite contexts, close to zero in definite ones). Second, Turkish HSs were the only bilingual group significantly different from English MSs: they produced more unexpected articles in both definite and indefinite contexts (\textit{the} in definite and \textit{a} in indefinite).

The first unpredicted pattern is similar to the finding from the narrative study described above: there the percentage of unexpected articles was also numerically higher for the new referents (= indefinite contexts) than for given ones (= definite contexts), even though this comparison was not statistically tested. However, it is hard to attribute these two results to the same cause since the narrative task and the fill-in-the-blank task were quite different. In the narrative task, it is plausible that speakers fell back to their own perspective of being able to uniquely identify the referents (since they were familiar with the characters and objects in the story after watching the video) and did not consider the addressee’s perspective. This is especially likely given the fact that unexpected articles were produced more often in the spoken than in the written mode because the spoken mode does not offer a possibility to plan ahead and revise. In the fill-in-the-blank task, this explanation does not seem to apply: first, the task was written and untimed, removing any performance pressure, and second, the speakers did not have a perspective of being familiar with the referents: they were reading the items for the first time. In this case, a possible explanation would be that speakers found the task too easy and engaged in creating background contexts that would justify the use of definite articles in indefinite contexts (e.g., prior shared knowledge or experience between the interlocutors in the dialogues). This would explain why indefinite articles were rare in definite contexts: once the uniqueness of the referent has been established, it seems almost impossible to create a context where it would disappear.

The second unpredicted pattern, the difference between Turkish HSs and English MSs, is not straightforward to explain. One possibility is that Turkish HSs engaged in creating the background contexts that justify definite articles more often than any other group. Another potential account is that acquiring a definiteness\hyp based system in English along with a definiteness- and specificity-based system in Turkish led to the formation of a unique system that is different from both. This finding requires further investigation to see if the pattern persists in other speaker samples, and if so, to understand its underlying cause.

Summing up, in the area of article use we observed no difference between the HSs and English MSs in majority English when examining the narrative data, and more unexpected articles produced by Turkish HSs than English MSs in an experimental task (with no difference between other HS groups and MSs). At the same time, all speakers, including English MSs, produced a small proportion of unexpected articles in indefinite contexts, which might point to task effects.


\section{Syntax} \label{sec:pashkovaetal:Syntax}

In the domain of syntax, we compared HSs speaking majority English and English MSs for three phenomena – the use of various clause types across registers, the use of left dislocations across registers, and word order variation caused by information status of referents.

We conducted two studies on clause types across registers. \citet{pashkova2022} compared the use of independent main clauses (\textit{I was walking down the street. I saw a couple.)} and coordinate main clauses (\textit{I was walking down the street, and I saw a couple.)} in a subset of the English subcorpus: 20 adolescent English MSs and 20 German HSs. Since previous research has focused on clause-type variation mainly in the HL of HSs (\cite{chapters/09, Ozsoy2022, schleppegrell1997, sanchez_abchi2017}), we aimed to create a more comprehensive picture of HSs’ repertoires and examine their ML as well.

We did not observe a significant difference between German HSs and English MSs: both groups used independent main clauses and coordinate main clauses with similar frequencies. In addition, HSs and MSs alike used more independent main clauses in written narratives than in spoken ones, and more coordinate main clauses in spoken narratives than in written ones. Finally, both speaker groups showed the same effect of formality: they used more coordinate main clauses in informal settings than in formal ones.

The second study on the use of clause types \parencite{tsehaye2021} looked into the use of subordinate clauses in a larger subset of the English subcorpus: 32 adolescent English MSs and 27 adolescent German HSs. We found that bilinguals and monolinguals used subordinate clauses differently depending on the formality and mode (\figref{fig:pashkovaetal:4}). German HSs made a distinction between the formal and informal setting in both spoken and written modes, while English MSs made a distinction between the settings only in the written mode.

\begin{figure}
    \includegraphics[width=\textwidth]{figures/Ch13_figure_4.pdf}
    \caption{Mean proportions of subordinate clauses in English by speaker group and communicative situation. Means and bootstrapped CIs are based on raw data.}
    \label{fig:pashkovaetal:4}
\end{figure}

Our second syntactic structure of interest is left dislocation (LD), in which a noun phrase precedes its core clause and the argument position within the core clause is filled with a pronoun that is co-referential with the dislocated noun phrase (e.g., \textit{The woman}{\textsc{\textsubscript{NP}}} \textit{, she} \textsc{\textsubscript{pronoun}} {\textit{spilled her groceries}}). LDs are indicative of an informal register (\cite{keenan1977, Geluykens1992}; see \cite{chapters/11} for a comparative review of LD in English, German, and Russian).

\citet{pashkovainprep_d} hypothesized that the use of LD in the majority English of HSs might be different than that of English MSs because previous studies have found differences in the frequency of LD use by French-English bilinguals and English MSs \parencite{hervé2016, Tagliamonte2019}. We also hypothesized that HSs may approach the formality constraint on LD usage differently from English MSs, given that divergent approaches to formality have been observed in other register-dependent phenomena, such as subordinate clauses (\cite{tsehaye2021}, see above).

We investigated the use of LD in the full English subcorpus. The findings reveal that in the formal setting, Greek and Turkish HSs use fewer LDs than English MSs, while German and Russian HSs do not show any significant difference from the MSs. In the informal setting, Turkish HSs use more LDs than English MSs, while Greek, German, and Russian HSs do not demonstrate any significant difference from the MSs.

As to the difference between the formal and informal settings, Greek and Turkish HSs have a dissimilar approach to the formality distinction in comparison to English MSs. These HSs use slightly more LDs in informal than formal settings, whereas English monolinguals use slightly fewer LDs in informal than formal settings. That is, English monolinguals have a reverse formality effect compared to Greek and Turkish HSs. In fact, Greek and Turkish HSs adhere more strictly to the expected pattern of using LDs predominantly in the informal context. German and Russian HSs do not differ from English MSs statistically, even though numerically they pattern with Greek and Turkish HSs because they have slightly more LDs in the informal setting than in the formal one (\figref{fig:pashkovaetal:5}). Also noticeable in \figref{fig:pashkovaetal:5} is that Turkish HSs differentiate the formalities quite strictly, while English MSs and other HS groups do not show such a strict differentiation.

\begin{figure}
    \includegraphics[width=\textwidth]{figures/Ch13_figure_5.pdf}
    \caption{Predicted probabilities and individual mean proportions of LDs by speaker group and formality. Predicted means and CIs are based on a linear model, individual proportions on raw data. The y-axis is zoomed from the original size of 0--30\% to the size of 0--15\% to make the model predictions more visible. When zooming, 10 data points that were above the 15\% mark were removed: two in English MSs, three in German HSs, two in Russian HSs, and three in Turkish HSs.}
    \label{fig:pashkovaetal:5}
\end{figure}

In summary, the LD study shows that some HS groups use LDs differently in their majority English compared to English MSs, while other groups do not. The difference does not lie in the overall frequency of LDs, but in the way speakers distribute LDs in formal and informal settings: some HSs are actually more in line with the expected patterns reported in the literature than English MSs. It is a matter of future research to determine the source of this difference: while we cannot exclude cross-linguistic influence from the Greek and Turkish HLs, a general HS pattern seems more likely for two reasons. First, it is quite surprising that cross-linguistic influence from two HLs would make HSs’ productions more aligned with the expected LD pattern in majority English. Second, all four HS groups trend in the same direction numerically, which is also somewhat unlikely if the HL influence was the source of difference.

The third phenomenon in the domain of syntax that we examined is the variability in word order caused by the information status of referents (see \textcitetv{chapters/11} for more on word order dynamics in language contact and \textcitetv{chapters/05} for word order phenomena in heritage German). English follows a fixed SVO word order, which can be changed to accommodate pragmatic factors, leading to various information packaging constructions \parencite[67--68]{Huddleston2002}. One such factor is the given-before-new principle, which suggests that a clause should begin with given/background information and end with new information \parencite[888--889]{Biber2021grammar}. However, this principle creates a conflict when it comes to new subjects in English. While the SVO word order demands that a new subject be placed at the beginning of a clause, it would be more helpful to position it closer to the end because the referent is new.

To address this conflict, English speakers may use different constructions that deviate from the SVO order, such as existential \textit{there} {(e.g.,} {\textit{There was a woman unloading her groceries on the other side of the street}}), presentational \textit{there} (e.g., {\textit{There stood a woman unloading her groceries on the other side of the street}}), locative inversion (e.g., {\textit{On the other side of the street was a woman unloading her groceries}}), or passivization (e.g., \textit{Some groceries were being unloaded by a woman}).

{Research suggests that bilingual English speakers may use information\hyp packaging constructions differently than English MSs, potentially due to cross-linguistic influence. For instance, speakers of Singapore or Jamaican English use fewer existential} \textit{there} constructions than speakers of monolingual varieties \parencite{Winkle2015}.

{Building on these previous findings, \citet{pashkovainprep_c} investigated the syntactic structures employed for the introduction of new subjects in majority English. We asked whether HSs of Russian and Turkish, languages with more flexible word orders, utilize more non-SVO structures than English MSs, potentially due to cross-linguistic influence from their heritage languages. To test this hypothesis, we compared the syntactic patterns employed for the introduction of new subjects, based on a subset of the English subcorpus. It included 82 HSs (40 Russian and 42 Turkish) and 40 English MSs. Each subject was annotated for its information status (new or given) as well as the syntactic structure it appeared in. The syntactic structures were classified into two categories: SVO-type structures (SVO and non-inverted copular clauses), and non-SVO-type structures, encompassing existential and presentational} \textit{there}, locative and non-locative inversions, right and left dislocations, passives, questions, and \textit{it-} and pseudo-clefts.

The results revealed no significant difference in the frequency of non-SVO structures employed by Russian and Turkish HSs and English MSs. However, a distinction emerged in the frequency of non-SVO structures in new and given subjects: new subjects appeared in non-SVO structures more often compared to given subjects. There was no interaction between the subject information status and the speaker group (HSs vs. MSs). These findings indicate that new subjects are more commonly associated with information packaging constructions (i.e., non-SVO structures) than given subjects, and both HS and MSs in our sample follow this trend in majority English in a similar way.

Summing up, in the domain of syntax we have not observed qualitatively different patterns in HSs’ and MSs' productions. If there are dissimilarities between bilinguals and monolinguals, they are connected to different use of syntactic structures in formal and informal contexts (subordinate clauses in German HSs, LDs in Greek and Turkish HSs).


\section{Discourse} \label{sec:pashkovaetal:Discourse}

Discourse organization refers to the way in which speakers organize their ideas in a coherent way to convey a meaningful message. This is typically achieved through the use of various linguistic devices such as discourse markers, as well as through macro-level textual components such as openings and closings. Discourse organization can vary across countries and cultures, social groups, age groups, genre, speech communities, and registers, with sometimes very subtle differences (e.g., \cite{FoxTree2010}). Previous studies have shown that HSs who are highly proficient in both their HL and ML exhibit similar patterns of discourse organization as MSs \parencite{Silva-Corvalán2014}, while those who are less proficient in one or both languages often display different patterns \parencite{MONTRUL2010}.~

To examine this further, we investigated whether the Greek, Turkish, German and Russian HSs use similar organization strategies as English MSs in their majority English, and whether any differences are a result of cross-linguistic influence. We focus on both openings and closings and discourse markers in the English subcorpus of the RUEG corpus (see \textcitetv{chapters/15} for an analysis of openings and closings in various languages and \textcitetv{chapters/14} for an analysis of discourse markers).

Discourse openings and closings are defined as any material preceding or following the core phase of the discourse (for English, see \cite{schlegloff1973, schleggloff1986, Schegloff2007}; for other languages, see \cite{Pavlidou2014, luke2002}). By core phase, we mean the part of the text that constitutes the narrative, which in our case is the description of the events presented in the video. They are used to define textual boundaries that set the main text apart from the framing, introduce or close off a new topic, orient the addressee to what is coming next, make the speaker’s stance clear, and interpret and evaluate linguistic material (\cite{labov1972, tolchinsky2002, berman2004}).

We identified the functions of openings and closings within three main categories: intersubjective (oriented towards the imaginary interlocutor), subjective (oriented towards the speaker), and textual (focusing on the organization of the text). Intersubjective, subjective and textual functions and their respective examples are presented in \tabref{tab:pashkovaetal:discourse}.

\begin{table}
\begin{tabular}{l >{\itshape} l}
\lsptoprule 
Functions & {\normalfont Examples} \\
\midrule \multicolumn{2}{l}{\textit{Intersubjective}} \\
Greeting & {hello} \\
Summoning & {bro} \\
Initial inquiry & {How are you?} \\
Identification & {My name is...} \\
Justification & {I am calling to report an accident} \\
Attention getter & {You wouldn't believe what happened!} \\
Reaction seeking & {Can you call me back?} \\
Inquiry response & {I'm fine too!} \\
Give advice & {Be careful when you're driving} \\
Common ground & {Remember that day...} \\
Valediction & {bye} \\\tablevspace

\multicolumn{2}{l}{\textit{Subjective}} \\
Reaction & {whoa} \\
Evaluation & {Totally crazy} \\
Personal statement & {I was walking as I always do in the mornings} \\\tablevspace

\multicolumn{2}{l}{\textit{Textual}} \\
Initialize narrative & {Here's what I saw:} \\
Contextual & {I just saw an accident} \\
Episodic & {It happened at $12 \mathrm{pm}$ today} \\
Resolution & {Nobody got hurt} \\
Encapsulation & {It was all an accident} \\
Coda & {That's the end} \\
\lspbottomrule
\end{tabular}
\caption{List of intersubjective, subjective and textual functions with examples }
\label{tab:pashkovaetal:discourse}
\end{table}


A descriptive frequency analysis revealed that all groups of speakers in our sample used more openings ({\textasciitilde}80\%) than closings ({\textasciitilde}50\%) across all communicative situations in the majority English texts. The overwhelming use of an opening in speakers’ narrations indicates that openings were used as a means to establish the communicative situation (e.g., formal vs. informal). In contrast, the relatively low percentage of closings shows that closings have a more optional character in speakers’ narratives.

German and Russian HSs showed the highest frequency of openings and closings, while Turkish HSs showed the lowest frequency including many texts in which they only described the video events with no introductory or closing material (\figref{fig:pashkovaetal:6}). This indicates that although English is the dominant language for all speaker groups, there are differences in macro discourse strategies across groups which may originate from contact with the HL (see \textcitetv{chapters/15} for analyses of openings and closings in Turkish MSs and HSs) or from other sources that may be uncovered in further research. In terms of functions, we find that intersubjective functions are mostly used in informal settings whereas textual functions are primarily used in formal settings.

\begin{figure}
    \includegraphics[width=\textwidth]{figures/Ch13_figure_6.pdf}
    \caption{Frequency distribution of openings and closings in the English texts. Error bars represent standard errors. Means and error bars are based on raw data.}
    \label{fig:pashkovaetal:6}
\end{figure}

Discourse markers also play a crucial role in the organization of discourse. We analyzed the frequency and discourse functions of two discourse markers in the full English corpus: {\textit{so}} {and} {\textit{yeah}}{. The literature reports that} {\textit{so}} {as a discourse marker is mostly used with an inferential function to indicate “result” or “inference” \parencite{Schiffrin1987}. The MSs in our corpus used} {\textit{so}} {with a higher frequency than all groups of HSs in majority English, although the functions of} {\textit{so}} {are distributed similarly across all speaker groups (see \textcitetv{chapters/14} for a more detailed analysis of the German} {\textit{so}} {and other discourse markers). In terms of function, \textit{so} was primarily used across groups as a connector (\ref{ex:pashkovaetal:key:3}a), then to initialize the narrative (\ref{ex:pashkovaetal:key:3}b), and third most frequently to switch the theme (\ref{ex:pashkovaetal:key:3}c).}

\ea%3
\label{ex:pashkovaetal:key:3}
\begin{enumerate}
    \item[a.] it was cause this guy like lost control of his soccer ball and I know you love soccer \textit{so} thought you’d be interested  (USmo11FE\_isE)
    \item[b.] \textit{so} i just witnessed an accident it seemed like a minor accident there was a woman on the right side of the street opening her trunk  (USmo05FE\_isE)
    \item[c.] and then a dog starts barking and he runs in the street \textit{so} this guy is driving down the street and he sees a dog  (USmo10ME\_isE) 
\end{enumerate}
\z 

\textrm{The analysis of the discourse marker} \textrm{\textit{yeah} }\textrm{showed that its frequency is much more restricted than that of} \textrm{\textit{so}}\textrm{. Interestingly,} \textrm{\textit{yeah} }\textrm{is used most frequently by German HSs, followed by MSs, and then Greek and Turkish HSs. The elevated use of} \textrm{\textit{yeah} }\textrm{by German HSs may stem from the high frequency of} \textrm{\textit{ja}} \textrm{in the German language, as well as the}\textrm{ }\textrm{functional similarity between} \textrm{\textit{yeah} }\textrm{and the German} \textrm{\textit{ja}} \textrm{(‘yeah’) \parencite{trotzke2020}.\footnote{Although there is phonological similarity between the English \textit{yeah} and the Turkish discourse marker \textit{ya}, we did not find inflated use of \textit{yeah} in Turkish HSs. This possibly stems from the fact that the English \textit{yeah} and the Turkish \textit{ya} do not have full functional overlap (as is the case between the German \textit{ja} and the English \textit{yeah}) \parencite{Simsek2012}.} The most frequent discourse functions of both} \textrm{\textit{ja}} \textrm{and} \textrm{\textit{yeah}} \textrm{across groups are initializing and ending the narrative, evaluation, and hesitation. (\ref{ex:pashkovaetal:key:4}) shows an example of \textit{yeah} ending a narrative.}

\ea%4
\label{ex:pashkovaetal:key:4}
the car stop and hit the other car so it’s really all that happened um \textit{yeah} (USmo69ME\_isE)
\z 

\textrm{In sum, we see that the patterns of use of discourse openings and closings in majority English are similar across HS and MS groups, although the frequency of use differs somewhat. This extends to the discourse markers} \textrm{\textit{so}} \textrm{and} \textrm{\textit{yeah}} \textrm{as well; both are used with similar functions across groups but with somewhat different frequencies. In the case of} \textrm{\textit{yeah}}\textrm{, this appears to be driven by cross-linguistic influence.~}


\section{Discussion} \label{sec:pashkovaetal:Discussion}

We began this chapter by asking whether the majority English of adolescent and adult HSs showed any dynamic patterns due to bilingualism or whether it looked similar to patterns found in English MSs. In short, we found that the four groups of HSs patterned remarkably similarly to each other and to the English MSs in most respects, with some quantitative differences. We elucidate these similarities and differences in the following paragraphs.


\subsection{Similarities} \label{sec:pashkovaetal:Similarities}

We begin with the similarities between HSs and MSs in majority English. In the area of phonology, HSs performed like MSs in their use of stressed pronouns based on information structure and phrasing, and in their use of pitch accents marking contrastive adjectives prosodically.

In terms of reference, they introduced and modified new referents with the same frequency as MSs, and, similarly to MSs, introduced and modified more referents in formal than informal settings. They also patterned like MSs in the proportion of NPs vs. pronouns and pronouns vs. null forms that they used to realize already-introduced referents, as well as in using more NPs for referent re-introduction than for referent maintenance.

In the domain of article use, HSs produced unexpected structures at the same rate as MSs (regarding use of both \textit{a} for definite and \textit{the} for indefinite), and produced more unexpected articles in indefinite than definite contexts and in spoken than written modes (both with similar frequency to MSs). For syntax, they produced independent vs. coordinate main clauses at similar frequencies to MSs, and patterned like MSs in preferring independent main clauses for the spoken mode vs. coordinated main clauses for the written mode, and in producing more coordinated main clauses in informal than formal settings. Additionally, HSs used a similar frequency of non-SVO structures to introduce new subjects compared to MSs. Finally, HSs structured their discourse similarly to MSs: all of the speaker groups produced more openings than closings, and preferred intersubjective functions for openings and closings in the informal setting vs. textual functions in the formal setting. The functions of the discourse markers \textit{so} and \textit{yeah} were also distributed similarly across the HS and MS groups. In sum, we found no differences between our four HS groups and English MSs in many patterns of majority English.

It may seem surprising that our results show so many similarities in majority English between HSs and MSs, given a large number of differences found in the literature reviewed in the Introduction. We offer several possible reasons for this. One potential reason is that our studies focus on a particular type of heritage speaker for whom many demographic factors favor a positive language learning situation. They were all fluent and literate in the HL (which is not the case in several studies), typically from mid to high SES families (50--89\% of mothers in each speaker group had a postsecondary degree), and without indication of traumatic immigration history in their recent background (e.g., refugee from a war zone). In addition, 91\% of them started acquiring English from the age of 5 or earlier, and nearly half from birth, so they had very early exposure to the ML in a society where the ML is highly dominant.

Another main factor relates to the methods of our study. Most of our data derive from naturalistic narrative productions, for which the cognitive demands are relatively low. The speaker is also in control of the language they produce, so they can easily keep to language that is comfortable for them. It may be that experiments or other tasks with higher cognitive demands, or focusing more narrowly on a particular structure, would provide more possibility to see dynamic patterns resulting from bilingualism. For example, \citet{Bylund2012} used materials designed to be particularly challenging for the HS participants, and \citet{Lee2011} and \citet{Scontras2017} focused narrowly on differences in scope phenomena in Korean/Mandarin vs. English that appear relatively rarely in narrative contexts such as those we elicited. Further, we examined only a subset of possible structures; examining other structures may lead to different results.

In general, however, we have added evidence to the literature that several structures in bilinguals’ more dominant language (here, ML) are not influenced by cross-linguistic influence or general bilingualism effects. In addition, we have shown that for at least our sample of HSs, bilingualism is not an impediment to performance in the majority language that is consistent with that of MSs \parencite{flege1999, Ventureyra2004, macwhinney2005}. This suggests that there is no reason to prohibit heritage speakers from maintaining their HL, and supports the view that heritage speakers are indeed native speakers of the ML rather than “deficient versions” of speakers of the ML (as already expressed in \cite{rothman2014, Montrul2016, kupisch2018, tsehaye2021, wiese2022}).


\subsection{Differences} \label{sec:pashkovaetal:Differences}

However, the productions of HSs in majority English also differed significantly in several respects from those of English MSs. Many of the differences are consistent with the idea put forth in the literature that HSs might be more explicit than MSs. In the domain of phonology, Russian HSs used double-accented structures in contrastive accent situations -- accenting both the adjective and the noun -- where MSs and Greek HSs typically accented only the adjective. The occasional use of double accent by MSs and HSs of Greek in formal situations gives an indication that this phenomenon could be related to an effort to be prosodically more explicit by making the structure more salient (e.g., producing an accent on more lexical constituents than expected). Both patterns suggest that they were trying to make the structures more salient than MSs did. In referent use, Russian and Turkish HSs used more NPs in the informal setting than MSs -- again a sign of greater explicitness. In one area, however, we found that HSs were less explicit than MSs: they stressed fewer contrastively-focused pronouns than MSs.

In other cases, the HS bilinguals seemed to make a stricter distinction between different linguistic and extra-linguistic contexts than MSs did. In the area of reference, for example, German, Russian, and Turkish HSs had a larger difference in the proportion of NPs between maintenance and reintroduction of referents in the formal setting than MSs did, signaling a stricter distinction between these functions. We also found an exaggerated distinction in HSs vs. MSs in their use of subordinate clauses depending on the formality and mode: German HSs distinguished between the formal and informal setting in both spoken and written modes, while MSs only made a distinction between the settings in the written mode. Finally, when producing left dislocation structures, Turkish HSs differentiated between the formal and informal settings more than MSs did.

Related to the stricter distinction between extra-linguistic contexts, in some situations HSs appeared to follow the differentiation between the contexts that is expected from the literature more strictly than MSs. In the case of left dislocations, all HS groups used more LDs in the informal setting than in the formal one (in line with the literature), while MSs had the reverse pattern~-- they produced slightly more LDs in the formal setting than the informal one. While the reversal of the pattern is visible in all HSs, it reached significance only in the Greek and Turkish HSs (see similar results for left dislocations in majority German in \textcitetv{chapters/11}).

The finding in our data that HSs are more explicit in some contexts and tend to differentiate contexts more strictly is striking, and also consistent with other reports in the literature noted earlier (e.g., \cite{Serratrice2004, Barbosa2017, Polinsky2018book, azar2020reference}). It suggests that the dominant language of bilinguals can show signs of general bilingualism effects, in addition to potential cross-linguistic influence from the less dominant language.

Why might this be the case for our data? It may well be that HSs want to show that they can speak the ML “properly”, especially in a test-like situation. It could also be an effect of their typical environment -- that HSs are more exposed to input containing explicit forms from L2 speakers than MSs are, as suggested by \citet{azar2020reference}, or that they are more explicit to make their speech clearer to their L2 interlocutors, as suggested by \citet{Polinsky2018book}. Finally, it may be that HSs choose to be explicit in a more nuanced way, such as signaling different contextual factors to the fullest degree, rather than in the straightforward way of simply being more explicit at all costs. This may explain why we see effects of explicitness or greater differentiation in some contexts but not in others where we might have expected them.

Cross-linguistic influence does not seem to be a main driver of dynamic patterns in our data, but may account for two of our findings. First, Turkish HSs produced more unexpected articles in our fill-in-the-blank study than did MSs, which may be related to the fact that Turkish has a very different article system than English. Second, German HSs used the discourse marker \textit{yeah} more frequently than MSs, which we hypothesize results from its functional similarity with the German \textit{ja} "yeah". However, the reduced use of the discourse marker \textit{so} by HSs compared to MSs does not seem to stem from cross-linguistic influence.

The fact that we did not see more cross-linguistic influence in our data overall may be for at least two reasons. First, while we selected structures that differ broadly across the languages in our study (e.g., Russian and Turkish have more free word order than English so we expected more use of non-SVO utterances to introduce new subjects), we did not specifically target a narrow point on which there were clear differences. Second, the type of data we collected -- elicited narratives -- allows speakers to avoid structures they are uncomfortable with. While elicited narratives provide ecologically valid data in various registers, they are likely not the ideal type of data to see effects of cross-linguistic influence that are necessarily subtle due to the high proficiency of the speakers.

Another perplexing point is that we sometimes found dynamic patterns in one or two of the four groups of HSs but not all of them, and the groups that differed changed from study to study. For example, the German and Russian HSs patterned like the MSs in production of left dislocations while the Greek and Turkish HSs did not. Similarly, the Greek HSs patterned like the MSs in their use of noun phrases vs. pronouns to refer to given referents, but the other three groups of HSs patterned differently. On the surface, it does not seem like cross-linguistic influence is a likely explanation because there are no apparent salient differences between the languages. However, future research should carefully examine the relevant structures in data from monolingually-raised speakers of those languages. Another possible explanation for differences across groups is non-linguistic factors such as subtle SES differences or psychological factors. Nonetheless, we highlight the importance of comparison between different groups of HSs using the same task as the performance across groups allows us to see more clearly if a particular phenomenon is related to bilingualism in general or to some feature of the language(s) in question.


\section{Conclusion} \label{sec:pashkovaetal:Conclusion}

In sum, we examined the production of majority English by heritage speakers of German, Greek, Russian and Turkish in a variety of studies focusing on different interface phenomena covering several areas of language. The studies revealed many phenomena where heritage speakers and monolingually\hyp raised speakers showed identical patterns, underlining the growing perception in the field that heritage speakers can reasonably be considered native speakers of their majority language (as well as their heritage language). We also found several differences between the productions of (certain groups of) heritage speakers and monolingually-raised speakers, most of which could be attributed to either cross-linguistic influence or some form of explicitness or greater differentiation between contexts than is present in monolingually-raised speakers. Overall, the studies reported here reveal the richness and complexity of patterns in the majority language of heritage speakers.


\section*{Acknowledgements}

We are very grateful to all the research assistants who contributed to data collection, annotation, and analysis of the majority English data reported in this chapter, as well as the preparation of this chapter: 
Iliuza Akhmezianova,
Amy Amoakuh,
Aban Alehojat,
Chris Allison, 
Ricarda Bothe, 
Mert Can,
Ryan Carroll, 
Franziska Cavar, 
Leah Doroski, 
Mary Elliott, 
Harshaa Gopalakrishnan, 
Gajaneh Hartz, 
Luc Henriquez, 
Abigail Hodge, 
Yuliia Ivashchyk, 
Janie Fink, 
Hanna Kim, 
Hannah Lee,
Mark Murphy, 
Mariia Naumovets, 
Gizem Öskürci,
Sadaf Rezai, 
Simge Sargın Kısacık, 
Jasmine Segarra, 
Golshan Shakeebaee,
Selena Song, 
Shreya Srivastava,
Rhiannon Stewart,
Mateo Vargas-Nunez,
Fiona Wong and Barbara Zeyer.
We further thank Lea Coy for help with pre-publication formatting.

We also thank the members on the RUEG project and Mercator Fellows for their many contributions to the RUEG method and for very helpful discussions of our studies over the years of the project. We would also like to thank two anonymous reviewers and three reviewers from the RUEG project (Annika Labrenz, Onur Özsoy, and Wintai Tsehaye) for their constructive feedback and insightful comments. 

The studies in this chapter were funded by the Deutsche Forschungsgemeinschaft (DFG, German Research Foundation); the grant was awarded to the Research Unit \textit{Emerging Grammars in Language Contact Situations} (FOR 2537), grant numbers 394837597 (P2), 394995401 (P5), 394844953 (P7), 313607803 (P8 and P9).

\sloppy
\printbibliography[heading=subbibliography,notkeyword=this]
\end{document}
