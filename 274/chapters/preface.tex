\addchap{Preface}
\begin{refsection}
In their 1963 paper “The structure of a semantic theory”, Jerrold Katz and Jerry Fodor argued that a characterisation of the abstract form of a semantic theory is given by a meta-theory that answers questions such as \textit{What is the domain of a semantic theory? What are the descriptive and explanatory goals of a semantic theory? What mechanisms are employed in pursuit of these goals? What are the empirical and methodological constraints upon a semantic theory?} Even though the Katz and Fodor paper was an early attempt to develop a semantic theory that would be compatible with a Chomskyan syntax, their introductory comments are applicable to semantic theories in general. That is, in order to be taken seriously, any semantic theory must be able to answer such meta-theoretical questions. Moreover, the extent to which competing semantic theories give similar answers to these questions is the extent to which such theories can be compared, for different answers will result in different explanatory aims and perhaps in incommensurable domains of inquiry. In regard to linguistic science and the way in which linguists think and work, sorting out what the domain of a semantic theory is and what explanatory goals it has are paramount in assessing the success or otherwise of the theory.

This book discusses the two main construals of the explanatory goals of semantic theories. These two construals, I argue below, are not so much in opposition as they are orthogonal. The first understands semantic theories in terms of an interpretive (or hermeneutic) explanatory project, this is often referred to in philosophy of language as externalism. As I detail in the second half of the book, this construal sees the task of a semantic theory as specifying how expressions are to be interpreted. For example, in their two volume study of truth-theoretic semantics, Lepore and Ludwig remark that “there is no question of a standpoint for understanding meaning that is outside of language altogether”. That is, they argue that “the most fundamental and powerful devices for representation can obviously not be explicated without the use of just those devices. We can then at best show how they work by showing how they systematically contribute to how we understand sentences in which they appear” \citep[9]{LeporeLudwig2007}. This construal, often implicit, is the standard one in philosophy and in formal semantics, but it is far from being the only one.

The second construal understands semantic theories in terms of the internalist study of the psychological mechanisms in virtue of which meaning production and comprehension are made possible. There is a sense in which there is no competition between the internalist and externalist understanding of semantics, for each approach asks different questions and has different explanatory aims. Unfortunately, this is not the way in which the debate has often been couched, for it is often assumed that both sides are engaged in the same research project. This has led to much misunderstanding and ill-founded criticism from both sides. The internalist side is often criticised for not doing semantics in the way in which the externalist and hermeneutic side assumes semantics should be done. In other words, psychological theories of semantics are often criticised for eschewing the interpretive aspects of semantics that form the basis of the hermeneutic approach to meaning. But these critics fail to see the force and difference in the internalist approach. Regardless of what one thinks of the internalist approach to semantics, its explanatory project both in theory and practice is not hermeneutic but rather scientific in the sense to be spelled out below.

This book argues that a fruitful scientific explanation is one that aims to uncover the underlying mechanisms in virtue of which the observable phenomena are made possible, and that a scientific semantics should be doing just that. I should note at the outset that nothing follows about approaches that are not scientific in this particular sense. There is clearly a great deal to learn from the hermeneutic approach and much good work has been done that takes this approach, but we should not confuse ourselves by claiming that this approach is scientific. Another way to put the matter is as follows. Until recently (perhaps until the mid twentieth century) it was not possible to do semantics \textit{qua} science, and so it was done in a hermeneutic fashion with much success and offering many insights into the nature of language and mind. However, if (as I detail in Chapter 4) we understand scientific explanations to be unearthing the underlying mechanisms in virtue of which the observable phenomena are made possible, then the hermeneutic approach does not offer scientific explanations (and most of its practitioners do not claim to be doing so). The externalist project is one that often aims to provide meta-linguistic semantic descriptions that are essentially interpretive and hermeneutic. Nothing follows about the validity or fecundity of this hermeneutic approach by showing that it is not scientific, except clarifying that it does not aim to unearth the psychological mechanisms in virtue of which meaning comprehension and production are made possible. Showing that this is the case is important in the context of any field that studies meaning, whether it be linguistics, philosophy, psychology, or cognitive science. To see why this is the case, let me offer a few remarks about the current status of semantics, both within linguistics and in other fields that study the phenomenon of meaningfulness.

The introduction to the recent \textit{Routledge Handbook of Semantics} is titled “Semantics – a theory in search of an object”. The editor of the handbook argues that current linguistic semantics is “a subfield whose object – meaning and reference – could hardly be more ambiguous or protean, and which is studied by a highly various scatter of often incompatible theoretical approaches, each of which makes truth-claims, at least implicitly, in favour of its own kind of analysis” \citep[1]{Riemer2015}. The editor of the handbook is also the author of a semantics textbook in which he notes that there is a “lack of disciplinary agreement over the basic theoretical questions” at the core of semantics \citep[xiii]{Riemer2010}. Such a diagnosis (and indeed philosophical self reflection of this kind) is rare in linguistics, yet it is accurate, and Riemer remarks that due to this theoretical heterogeneity “it is no surprise that consensus is almost wholly absent about any of the key questions semantics sets out to answer”. These questions include, “what meaning as an object of study might, in detail, amount to; how it – whatever ‘it’ is – should be theoretically approached; how – even pretheoretically – it should be characterized on the level of individual expressions, constructions, and utterances; and what relation semantics should entertain with other fields of enquiry within and outside linguistics” \citep[1--2]{Riemer2015}. Indeed, as Riemer remarks, “it’s striking how little explicit theory-evaluation is undertaken by semantics researchers, and how rarely theoretical bridges between different research programmes are even sought, let alone found” \citep[2]{Riemer2015}. This lack of explicit theory evaluation is a primary reason for the lack of consensus on fundamental linguistic phenomena. This is a major hurdle faced by linguistic semantics, and the fact that it is rarely noticed or acknowledged calls for a remedy.

This book aims to provide the beginning of such a remedy by discussing the two major construals of the nature of meaning. By investigating the debate between internalist semanticists and those who advocate for a hermeneutic and interpretive semantics, I hope to clarify the theoretical landscape and provide a rigorous characterisation of what meaning is according to these two schools of thought. I should note that, historically and to this day, linguistics has attracted the interest of many philosophers of language that seek to understand the nature of meaning. However, as Riemer correctly remarks, this interest has not been reciprocated. Too few linguists have investigated the philosophical approach to meaning or compared their own semantic theories to those offered by philosophers. This is unfortunate, for the divide between philosophy and linguistics is artificial. In the same way as people working on the nature of time or the interpretation of quantum physics are often in philosophy (and not physics) departments, there are people working on the nature of meaning that are in philosophy (and not in linguistics) departments (or they may be in psychology, literature, anthropology, or sociology departments). Linguists, then, and semanticists in particular, have as much to learn from philosophers of language as philosophers of language have to learn from linguists. But again, this divide is artificial. 

This book compares the internalist and externalist approach to semantics, describing their different motivations and theoretical assumptions. I do this from the point of view of explanatory scientific theories. This is an important issue to sort out, for the way in which we construe the nature of meaning is essential for a fecund explanatory language science. I argue that a science of semantics is unlikely to be an externalist one, for reasons having to do with the subject matter and form of externalist and hermeneutic theories. Unlike the internalist approach to semantics, the externalist approach is not usually discussed in terms of scientific explanations, and so my argument might be open to the charge that externalists do not see their enterprise as scientific and thus it is a moot point to compare them to other scientific pursuits. However, as will become evident, there are leading externalists and formal semanticists who explicitly state that their theory is a scientific one. Thus, it is both possible and illuminating to look at the externalist research programme from the perspective of scientific explanatory strategies and to ask whether it is a promising avenue in regard to constructing an explanatory scientific theory.

I argue that externalist explanations of meaning are concerned with \textit{ascription and description }of meaning rather than the \textit{mechanisms} of meaning. That is, externalism is not concerned with the mental mechanisms in virtue of which humans produce and comprehend meaning. Therefore, it is not part of the psychological explanation of the mechanisms in virtue of which meaning is made possible. Rather, externalist explanations are a hermeneutic explanatory project in that they are an inherently interpretive project. Works in favour of the internalist approach are currently in the minority, and thus this book also meets the need of describing and helping in advancing a particular understanding of meaning that has been used in the philosophical and linguistics literature for a long time. I provide a critical examination of externalism and present the internalist alternative that, I argue, is better placed to provide the foundation upon which to build a fruitful explanatory science of semantics.

Lastly, I should note that in addition to discussing recent debates, I will also be discussing many of the classic references in the field because the latter still represent mainstream positions in the discipline. Much can be learned by considering the classic references in light of current debates.
\end{refsection}