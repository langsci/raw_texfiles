\documentclass[output=paper]{langsci/langscibook}
\ChapterDOI{10.5281/zenodo.4680320}
\author{Dimitris Michelioudakis\affiliation{Aristotle University of Thessaloniki}}
\title{Rethinking implicit agents: Syntax cares but not always}


\abstract{In this paper, I examine implicit control in Greek passives, both
    verbal passives and a passive-like construction in the nominal domain,
    namely complex event nominals with an agentive interpretation but a
    genitive theme DP as the only argument which is realised overtly. The
    availability of implicit control into temporal gerundival clauses depends
    on the features of the internal argument and the varying interpretation of
    the implicit argument. I argue that the implicit agent is only represented
    syntactically as a covert arbitrary pronoun and is thus able to exert
    implicit control as long as that pronoun does not
    trigger relativised minimality effects, blocking promotion of/Agree with
the internal argument. The very existence of relativised minimality effects is
a purely syntactic argument in favour of the syntactic reality of implicit
arguments.}


\begin{document}\glsresetall
\maketitle

\section{Introduction}\label{sec:22.1}

The syntactic status of implicit arguments, especially in short
passives\is{passive}, has been a controversial issue for decades (see
\citealt{Roberts1985,Roberts:1987a,Jaeggli1986,Roeper1984,Williams1985,Williams1987,BhattPancheva2006}
and references therein). Recent approaches to passives
\parencite[e.g.][]{Bruening2014,Schafer2012b,AlexiadouEtAl2015} seem to converge
in assigning no syntactic representation to the \gls{IAg} and cast doubt on the
syntactic nature of most of its alleged effects, reanalysing them as mainly
semantic effects. In this light, an unequivocally syntactic diagnostic is
needed and in this paper I will discuss such a potential diagnostic, namely the
presence/absence of minimality effects in Agree/Move triggered by a
demoted/unpronounced external argument. Such effects must be attributed to the
varying, as it turns out, feature specification of implicit arguments. The
implications of these findings are twofold: (i) the syntactic, rather than
merely semantic, identity/representation of implicit arguments which can
control into non-finite subordinate clauses is reinforced, while at the same
time (ii) not all non-active constructions with agentive readings have
syntactically realised \glspl{IAg}\is{implicit arguments}.

In \Cref{sec:22.2}, I summarise the reasons why the arguments proposed so
far regarding the syntactic representation of implicit argument\is{implicit
arguments} can all be recast as purely semantic phenomena, including possibly
even implicit control\is{control!implicit control} into infinitives. In
\Cref{sec:22.3} I outline the argument from \ili{Greek} gerundival clauses
and draw a distinction between manner and absolute/temporal gerunds, of which
only the latter really involve syntactic control. In \Cref{sec:22.4} I
present the data from verbal and nominal passives\is{passive}, episodic and
generic, and a featural relativised minimality-based analysis. In
\Cref{sec:22.5}, I conclude and present some implications and
cross-linguistic considerations that emerge.

\section{Questioning the syntactic status of implicit
agents}\label{sec:22.2}

The role, the presence and the position of the \gls{IAg}\is{implicit arguments}
in short passives\is{passive} is often thought to become evident in two types of paradigms:
(i) when a certain bit of structure is licensed, if that bit of structure
cannot be licensed in non-agentive constructions, and/or (ii) when the implicit
argument itself is part of a referential dependency. On different occasions,
all types of evidence have been disputed, either through counterexamples or by
suggesting that the mechanism involved does not have to be syntactic. To name
four such cases, (a) unpronounced agents have been thought to license secondary
predicates \REF{ex:22.1}, (b) passives\is{passive}, but not
unaccusatives\is{unaccusativity} or middles,
license non-finite purpose clauses in which PRO is controlled by the IA
\REF{ex:22.2}, (c) the \gls{IAg}\is{implicit arguments} can be the
antecedent of reflexive pronouns (with arbitrary reference) \REF{ex:22.3},
and (d) internal arguments in passives\is{passive} cannot be coreferential with the
implicit external argument \REF{ex:22.4}, a restriction which can be
analysed as a principle B or C effect \parencite{Kratzer1994,Kratzer2000},
depending on the category of the covert element, or as a crossover violation,
as in \citet{BakJohRob1989}.

\ea\label{ex:22.1}
    The game was played nude.
\ex\label{ex:22.2} \citeauthor{BhattPancheva2006}
(\citeyear{BhattPancheva2006}, their grammaticality
judgements/diacritics, adapted from \citealt{Manzini1983})\\
    \ea[]{The ship was sunk [ PRO to collect the insurance ].}
    \ex[\#]{The ship sank [ PRO to collect the insurance ].}
    \ex[*]{The ship sinks easily [ PRO to collect the insurance ].}
    \z
\ex\label{ex:22.3}\textcite[228]{BakJohRob1989}\\
    Such privileges should be kept to oneself.
\ex\label{ex:22.4}
    The children\tss{i} were being washed IMP\tss{k/*i}.
\z

The licensing of secondary predicates in \ili{English} passives\is{passive} is very limited, in
fact restricted to adjectives such as \textit{nude} and \textit{drunk}. \citet{Landau2010}
provides more examples of adjectives which often function as secondary
predicates but fail to do so when a co-indexation with the \gls{IAg}\is{implicit arguments} is intended
\REF{ex:22.5}.

\ea\label{ex:22.5}
    \ea[]{\textcite[3]{Landau2010}, adapted from \textcite[120--121]{Chomsky1986}\\
        The room was left (*angry).}
    \ex[*]{The issue was decided unassisted.}
    \ex[*]{The game was played shoeless.}
    \z
\z

Similarly, \citet{Williams1985} dismisses \REF{ex:22.1} on the grounds that
“one may call a game nude if it is played by nude people”, therefore
\emph{nude} might in fact be (derivatively) predicated of \emph{the}
\emph{game} (or the playing of the game).  If one “modif[ies] the adjunct
predicate suitably to make such a predication unreasonable, the sentence
becomes unacceptable” (\citealt{BhattPancheva2006}: 16).  However, while these
observations do suggest that \ili{English} passives\is{passive} do not license secondary
predicates predicated of the unpronounced agent, \citet{AlexiadouEtAl2015}
suggest that such secondary predicates are possible in other languages, a
necessary condition being that they are not required to \isi{Agree} with their
subject in phi-features. For instance, the \ili{German} counterpart of
(\ref{ex:22.5}a) is grammatical. \citet{PitteroffSchafer2017} propose that
the semantics of depictives in \citet{Pylkkanen2008}, combined with
\citegen{Bruening2014} theory of passives\is{passive}, can account for this possibility.

The apparent \isi{binding} effects illustrated in \eqref{ex:22.3}
and~\eqref{ex:22.4} have also been claimed to be analysable without
resorting to binding-theoretic syntactic explanations.  According to
\citet[219]{AlexiadouEtAl2015}, examples such as \REF{ex:22.3} could “find
[\dots{}] a different explanation as they could arguably involve a logophor
instead of an ordinary reflexive pronoun”. They further argue that anaphors
bound by \emph{by}-phrases, e.g.\ in impersonal passives\is{passive} in \ili{German},
are default, invariable 3\textsuperscript{rd} person forms, even when the
antecedent is first person, unlike cases involving real syntactic \isi{binding},
which requires person/number agreement between the anaphor and its antecedent
(ibid.).\footnote{As an anonymous reviewer points out, “a reflexivity
based account also needs no syntactically realised \gls{IAg}\is{implicit
arguments} to predict the facts”.}  As for the disjointness effect in
\REF{ex:22.4}, this could be made to simply follow directly from the
semantics of the \isi{passive} Voice head.  \citet{SpathasEtAl2015}, partly
following \citet{Bruening2014}, assume the \emph{Pass} is merged with a
Spec-less VoiceP and imposes existential quantification over the open argument
of VoiceP, while they treat the disjointness as a presupposition in the
denotation of \emph{Pass}, not to be found in other types of non-active/middle
Voice heads attested cross-linguistically \REF{ex:22.6}.

\ea\label{ex:22.6}
    ⟦Pass⟧ = λf\tss{es,t} λe${\exists}$x.f(x)(e)\\
    Presupposition: ${\forall}$f\tss{es,t}.f(x)(e) $\to$ f${\neq}$theme
\z

Control into infinitival purpose clauses is not uncontroversial either.
\citet{Williams1985} proposed that in examples such as \REF{ex:22.2} it is
the whole matrix clause that controls the subject of the infinitival adjunct,
i.e.  the sinking of the boat causes the collection of the insurance and can
even be referred to by the subject in sentences like \emph{That will collect\slash
earn you some insurance.} (\citealt{Williams1985}, via
\citealt[573]{BhattPancheva2006}). When such a semantic relationship between
the event in the matrix clause the one in the adjunct cannot be established,
then control fails \REF{ex:22.7}; likewise, similar S-control phenomena can
be obtained even with unaccusative\is{unaccusativity} predicates, given
appropriate additional context \REF{ex:22.8}, or even with events
disallowing the participation of an agent \REF{ex:22.9}.

\ea[*]{The boat was sunk [ PRO to become a hero ].}\label{ex:22.7}
\ex[]{\label{ex:22.8}The boat sank in order to impress the queen and move her
to murder her husband by the end of Act III.}
\ex[]{\label{ex:22.9}\textcite{Williams1985}\\
    Grass is green [ to promote photosynthesis ].}
\z

Nonetheless, this kind of argumentation does not easily carry over to implicit
control into infinitival complements of (passivised) control predicates such as
\emph{decide}/\emph{agree}/\emph{promise} \eqref{ex:22.10}.

\ea\label{ex:22.10} \textcite[4]{Landau2010}\\
    It was decided [ PRO to leave ].
\z

Among such predicates, ditransitives like \emph{promise} are particularly
interesting in that they disallow implicit control\is{control!implicit control} in goal passives
(\ref{ex:22.11}a), as per \citegen{Visser1973} generalization, while the
corresponding impersonal passives\is{passive} are licit in e.g.\ \ili{Norwegian}, as
\citet{vanUrk2013} notes, but also in \ili{English}
(\ref{ex:22.11}b).\largerpage

\ea\label{ex:22.11} \textcite{PitteroffSchafer2017}
    \ea[*]{Maggie was \emph{e}\tss{i} promised [ PRO\tss{i}
    to do the shopping ].}
    \ex[]{It was \emph{e}\tss{i} promised [ PRO\tss{i} to do
    the shopping ].}
    \z
\z

In the light of contrasts like this, \textcite{vanUrk2013} revises Visser’s
generalisation, suggesting that implicit control\is{control!implicit control}
is only licit if no overt DP establishes an \isi{Agree} relation with T,
assuming that the expletive\is{expletives} in impersonal passives does not
enter such a relationship. Such a proposal is indeed akin to the idea pursued
in this paper that the \gls{IAg}\is{implicit arguments}, if realised
syntactically, should lead to minimality effects when intervening between T and
an overtly agreeing DP. \Textcite{vanUrk2013} does not quite analyse the
ungrammaticality of (\ref{ex:22.11}a) as a minimality violation, but
proposes that implicit control is a case of subject control, which is always
mediated by agreement of T with both the controller and PRO. Thus, if T overtly
agrees with an argument which is not the controller, as in
(\ref{ex:22.11}a), control fails.

However, recall \citegen{Landau2015} generalisation that only attitude
predicates allow implicit control\is{control!implicit control}. Landau suggests that control with attitude
predicates involves what he calls \emph{logophoric} control, while control with
non-attitude predicates involves \emph{predicative} control, therefore only
logophoric control can be exerted by an implicit controller. Based on
\citegen{Landau2015} idea that logophoric control does not directly involve
predication between the controller and a clausal constituent, which would
require syntactic representation of the controller, then perhaps implicit
control with attitude predicates is no argument for the syntactic realisation
of the \gls{IAg}\is{implicit arguments}.

Furthermore, \citet{PitteroffSchafer2017} dispute Landau’s generalisation and
argue that there is a split between languages that disallow implicit control\is{control!implicit control}
with non-attitude predicates and languages that do. Interestingly, they
attribute this split to the availability and the nature of
\enquote{associative} expletive\is{expletives} pronouns that can satisfy the
\glsunset{EPP}\gls{EPP}. Thus, given that their explanation relies on the
associative pronoun functioning as the subject and valuing T’s phi-features,
van Urk’s revision of Visser’s generalization has to “find a different
explanation from the one [\dots{}] where T in implicit control\is{control!implicit control} structures is
valued by a syntactically projected (weak) implicit argument”
\parencite[38--39]{PitteroffSchafer2017}. Casting doubt on the IA’s
participation in \isi{Agree} relationships also undermines the hypothesis that it has
to be syntactically realised.

In the following sections, I will argue that \glspl{IAg}\is{implicit arguments} controlling into non-finite
subordinate clauses may not themselves be able to enter any Agree
relationships, however they can variably act as defective or transparent
interveners in \isi{Agree} relationships between a functional head\is{functional items} and the overt DP
that head licenses, depending on the feature specification of the functional
head but also the covert pronominal element realising the demoted argument.

\section{Towards a new diagnostic: Control into gerundival adverbial clauses in
Greek}\label{sec:22.3}

In the following sections, I put forward an argument that implicit control\is{control!implicit control} into
absolute/temporal gerundival clauses is subject to syntactic restrictions,
namely (featural) relativised minimality. In relation to the discussion above
this means that, even if we cannot be sure about implicit control\is{control!implicit control} into
infinitives, implicit control\is{control!implicit control} into absolute/temporal gerundival clauses has to
be established in narrow syntax. The core tenet of the argument is that
implicit control sometimes is successful and sometimes is not. All cases under
discussion involve an A-dependency across the presumed position of an implicit
argument. Those A-dependencies are obligatory: (a) promotion (to subject) of
the internal argument in verbal passives\is{passive}, episodic and generic; (b) promotion
(to a unique Case position) of the internal argument in \isi{passive} nominals.
Successful implicit control\is{control!implicit control} is in principle compatible with two explanations:
(i) either the implicit argument\is{implicit arguments} is not syntactically represented and implicit
control is semantic anyway; or (ii) implicit control\is{control!implicit control} is syntactic and therefore
the implicit argument\is{implicit arguments} is indeed projected syntactically, but its features are
such that they cannot give rise to minimality effects in Agree/Move
dependencies across the implicit argument. The fact that implicit control\is{control!implicit control}
is not successful in some other cases points towards the latter explanation: in
such Agree/Move dependencies the features of the probe are such that the
potential intervention of an implicit argument\is{implicit arguments} would trigger a minimality
violation.

Therefore, the existence of such A-dependencies and the absence of implicit
control in the latter cases is incompatible with the idea that implicit control\is{control!implicit control} is merely
semantic, if \enquote{semantic} is to be understood as \enquote{possible in the
    absence of syntactic representation}. Syntactic representation of the
    implicit argument\is{implicit arguments} is indeed needed for implicit control\is{control!implicit control} and the failure of
    implicit control\is{control!implicit control} is simply due to the absence of a syntactically
    represented implicit argument\is{implicit arguments} in such cases. The fact that certain
    non-overt thematic relationships are achieved through syntactically
    projected covert pronominals does not preclude the satisfaction of certain
    relationships. In other words, we cannot categorically rule out as a
    possibility the existence of constructions in which the relevant thematic
    entailments follow from the denotation of the functional (Voice) heads
    involved, as in \textcite{SpathasEtAl2015}. Anticipating somewhat the discussion in
    later sections, it turns out that the implicit agent is not projected
    syntactically in \ili{Greek} episodic verbal passives\is{passive}. In such
    cases, the agentive interpretation, i.e.\ the existentially-bound reading,
    has to come from the semantics of the Voice head, as in (6$'$) below
    (p.~\pageref{ex:22.6prime}).

Before moving to the argument itself, a crucial distinction needs to be drawn
first, regarding the control properties of gerunds in \ili{Greek}, a rather murky
area.  I will adopt and adapt a broad bipartite classification of \ili{Greek} gerunds
(see e.g.\ \citealt{Tsimpli2000}), which recognises absolute/temporal gerunds as
one category and manner gerunds as the other relevant type. The former can
usually be rephrased as an adverbial clause introduced by (the equivalent(s)
of) ‘while’, whereas the latter can be rephrased as adjuncts introduced by
phrases such as `by means/virtue of'. With the exception of gerunds with overt
nominative subjects (see \citealt{Tzartzanos1946,Kotzoglou2016}),
absolute gerunds license \isi{null subjects} which are obligatorily controlled by
some argument of the matrix clause, usually the subject but not necessarily.
According to \citet{Kotzoglou2016}, “[r]eferential \isi{null subjects} that are
totally thematically unrelated to the event denoted by the main clause
predicate are hardly licit as subjects of gerunds”.  In fact, absolute gerunds
can be controlled by any core or non-core argument of the matrix predicate. In
\REF{ex:22.12}, the null subject of the gerund can be co-indexed with either
the null subject of the matrix clause or the (cliticised) object. In
\REF{ex:22.13}, it is co-indexed with the indirect object of the matrix, and
in \REF{ex:22.14} it is shown that it can be co-indexed with object
experiencers of any type, i.e.\ both dative\is{dative case} and accusative experiencers are
licit antecedents. Cliticisation of non-subject antecedents may be preferred or
even required but I will put this aside for now, as well as the issue of gerund
placement (but see \citealt{HaidouSitaridou2002}).\largerpage[2]

\ea%12
    \label{ex:22.12} \ili{Greek}\\
    \gll    \emph{pro}\tss{i} ton\tss{j}   pirovolisan, e\tss{i/(?)j} vjenondas   apo to      peripoliko\\
            {}       him  shot.\Tpl{} {}  getting-out  of    the    patrol car\\
    \glt    \enquote*{They shot him, as he was / they were getting off the police car.}
\ex%13
    \label{ex:22.13} \ili{Greek}\\
    \gll    e\tss{i} telionondas   ti   thitia   tu, \emph{pro} tu\tss{i}     edhosan  vravio ja   tis   ipiresies tu\\
            {} ending    the   term  his {} him.\Dat{}  gave.\Tpl{} prize    for  the services  his\\
    \glt    \enquote*{As he was ending his term, they gave him a prize in recognition of his work.}
\ex%14
    \label{ex:22.14} \ili{Greek} (adapted from \citealt{Anagnostopoulou1999})\\
    \gll    e\tss{i}  akugondas   afta, archise  na   mi  mu\tss{i}  aresi / na   me\tss{i}  enochli        afti     i       istoria\\
    {}   hearing    these started    to  not  me  appeal {} to   me  annoy this     the  story\\
    \glt    \enquote*{As I was hearing those things, that story started to bother/annoy me.}
\z

Crucially, there is clear evidence that nothing prevents \isi{null subjects} of such
gerunds from taking \glspl{IAg}\is{implicit arguments} as their antecedents. In \REF{ex:22.15}, the subject
of the adjunct clause is obligatorily coreferential with the understood
experiencer of the evaluative adjective of the matrix clause.

\ea%15
    \label{ex:22.15} \ili{Greek} (adapted from \citealt{Kotzoglou2016})\\
    \gll    e\tss{i} grafondas to   vivlio, itan [ enoxlitiko  EXP\tss{i}] pu      i   {aftoptes martires} dhen milusan  ja   ta  mavra   chronia tis     hundas\\
            {} writing    the   book   was {} annoying {} that the eye-witnesses not     talked     about  the   black   years of-the   dictatorship\\
    \glt    \enquote*{While writing the book, it was annoying that the eye-witnesses did not talk about the dark period of the dictatorship.}
\z

These examples suggest that absolute gerunds can indeed be controlled by any
type of argument, regardless of its theta-role, and putting aside irrelevant
considerations regarding the feature makeup/size of overt antecedents. If this
is so, then the fact that existentially bound understood agents of episodic
verbal passives, as well as
overt \emph{by}-phrases, cannot be the antecedent of gerundival subjects is a
noteworthy exception \REF{ex:22.16}.

\ea%16
    \label{ex:22.16} \ili{Greek}\\
    \gll    \emph{pro}\tss{i} pirovolithike (apo tus astinomikus\tss{k}/ARB\tss{m}), \emph{e}\tss{i/*k/*m} vjenondas  apo  to peripoliko.\\
    {}  was-shot          \hphantom{(}by   the policemen {} getting-out from the {patrol car}\\
    \glt    \enquote*{He was shot as he was getting out of the police car'}
\z

\citet{Kotzoglou2016} provides a number of examples which appear to threaten
this neat picture, as they feature understood subjects of gerunds of all types
controlled by understood participants of the matrix event. His conclusion then
is that “felicitous \isi{null subjects} of \ili{Greek} gerunds might in fact be controlled
by an (implicit) argument of the matrix middle [\eqref{ex:22.17}],
\isi{passive} [\eqref{ex:22.18}, \eqref{ex:22.19}], ergative
[\eqref{ex:22.20}], or psych predicate [\eqref{ex:22.15}]”.

\ea%17
\label{ex:22.17} \ili{Greek}\\
    \gll    To    portokali  katharizete  kratondas   macheri ke      pirouni.\\
    the  orange    is-cleaned/cut  holding  knife    and   fork\\
    \glt    \enquote*{Oranges peel / are peeled using knife and fork.}
\ex%18
    \label{ex:22.18} \ili{Greek}\\
    \gll    Kaliptondas tis    {thesis ergasias} me     ikano             prosopiko afksanete      i   paragogikotita.\\
            covering     the  vacancies            with competent  staff is-increased the   productivity\\
    \glt    \enquote*{Productivity is increased by covering the vacancies with competent staff.}
\ex%19
    \label{ex:22.19} \ili{Greek}\\
    \gll    Epichirithike  perigrafi  tis  glosas    prosegizondas  tin sinolika   os  fenomeno.\\
            was-attempted  description  of-the  language  approaching  it holistically   as  phenomenon\\
    \glt    \enquote*{A description of the language as a whole was attempted.}
\ex%20
    \label{ex:22.20} \ili{Greek}\\
    \gll    I       porta   tu  banju    aniji  jirnondas  afto to  klidi.\\
            the   door    of-the  bathroom  opens  turning      this the  key\\
    \glt    \enquote*{The door to the bathroom opens by turning this key.}
\z

Crucially, with the exception of the implicit experiencer in \REF{ex:22.15},
the examples involving “implicit” external arguments are all examples of manner
gerunds.  So, we either have to assume that there is some level of
representation in which even unaccusatives\is{unaccusativity} take implicit agent arguments or to
draw a distinction between manner and temporal/absolute gerunds and show that
apparent control into clauses of the former type is not a syntactic dependency.

\hspace*{-0.98575pt}The first argument that manner gerunds may not allow syntactic control comes
from partial control\is{control!partial control}. \citet{Landau2010} argues in detail that partial control\is{control!partial control}
cannot be reduced to analyses compliant with “the locality of lexical
relations” \citep[361]{Landau2010}, hence controllers in partial
control\is{control!partial control} constructions
have to be syntactically realised and control dependencies that also allow for
partial control have to be syntactic. As shown below, if possible at all,
partial/split control is marginally possible with absolute gerunds
\eqref{ex:22.21} and~\eqref{ex:22.22}, but completely ruled out with
manner gerunds \eqref{ex:22.23} and~\eqref{ex:22.24}.

\ea%21
    \label{ex:22.21} \ili{Greek}\\
    \gll    \llap{?}Proigumenos, (vjenondas\tss{j+m} apo   to   ksenodochio) o Janis\tss{j} tis\tss{m}   kratise  (tis Marias\tss{m})\tss{} tin  porta (vjenondas\tss{j+m} apo   to   ksenodochio).\\
            earlier \hphantom{(}getting-out     from the hotel the  John her.\Cl{}  held \hphantom{(}the Mary.\Dat{}  the  door  \hphantom{(}getting-out    from the hotel\\
    \glt    \enquote*{Earlier, when leaving the hotel, John held the door for Mary.}
\ex%22
\label{ex:22.22} \ili{Greek}\\
    \gll    (Ksekinondas\tss{j+m} tin karjera tus\tss{j+m} os glosoloji),  o  Janis\tss{j}  sinergastike  poli   me   ti Maria\tss{m} (?ksekinondas\tss{j+m} tin karjera tus\tss{j+m} os glosoloji).\\
            \hphantom{(}starting the career their as linguists  the  John collaborated a-lot with the Mary \hphantom{(?}starting the career their as linguists\\
    \glt    \enquote*{When starting their careers as linguists, John collaborated with Mary a lot.}
\ex%23
    \label{ex:22.23} \ili{Greek}\\
    \gll    O  Janis\tss{j}   ke   i  Maria\tss{m}   sinergastikan sto   pirama     isoropias, kratondas\tss{j+m}   tis   dio akres  tu skinju.\\
            the John    and  the  Mary  collaborated  at-the  experiment  of-balance holding       the   two ends   of-the  rope\\
    \glt    \enquote*{John and Mary collaborated for the balance experiment, holding the two ends of the  rope.}
\ex%24
    \label{ex:22.24} \ili{Greek}\\
    \gll    O      Janis\tss{j} sinergastike   me   ti Maria\tss{m} sto   pirama   isoropias, kratondas\tss{j+m}   tis   dio  akres  tu   skinju.\\
            the John  collaborated  with  the  Mary  at-the  experiment of-balance holding        the  two  ends  of-the  rope\\
    \glt    \enquote*{John collaborated with Mary for the balance experiment, by holding the two ends of the rope.}
\z

Second, if we take the temporal/manner distinction into consideration, then it
turns out that the null subject of an absolute gerund can only pick out as its
antecedent arguments which are independently known to be syntactic objects.
\citet{Tsimpli2000} observes that manner gerunds are obligatorily
subject-oriented and, despite the exceptions noted above \eqref{ex:22.17}
and~\eqref{ex:22.20} that Kotzoglou observes, Tsimpli’s observation is
still correct in that manner gerunds can never be controlled by (overt)
non-subjects \REF{ex:22.25}.

\ea%25
    \label{ex:22.25} \ili{Greek}\\
    \gll    \emph{pro}\tss{i} ton\tss{k}     enochlusan   akugondas\tss{i/*k} dinata musiki, tin     opia   evazan     mes     sta    mesanixta\\
            {} him.\Cl{}  bothered.\Tpl{}  listening/hearing  loud     music the which   put.\Tpl{}  during   the  (mid)night\\
    \glt    \enquote*{They annoyed him, listening to music at top volume in the middle of the night.}
\z

\glsunset{EA}
This restriction brings manner gerunds closer to subject-oriented manner
adverbials rather than real clausal elements. Similarly to manner gerunds, and
unlike absolute ones, manner adverbials are never “controlled” by non-\gls{EA}
subjects, their controller can only be an external argument, either overt or
understood, and they do not allow this control to be partial.  Thus, in e.g.
\REF{ex:22.26}, there must be complete and not partial overlap between the
culprit(s) and the person(s) who wanted the event to take place.

\ea%26
    \label{ex:22.26}
    \gll    Ta      stichia    parapiithikan ithelimena\\
            the evidence  {was forged} {purposefully / willfully}\\
\z

Therefore, manner gerunds are just \gls{EA}-oriented adverbials, potentially
taking overt internal arguments, i.e.\ with some \emph{v}P structure, rather
than elements with clausal structure. Compared to absolute gerunds, they are
known to be truncated (cf.\ \citealt{Tsimpli2000}), lacking an inflectional
layer (hence they cannot be negated). They probably lack Voice too, or whatever
licenses external arguments syntactically. We can assume that they are
interpreted as predicated of some external argument at a post-syntactic level.
If an external argument is not provided by the syntax/LF, then it must be
inferred/provided by the context, as in the case of anticausatives (\ref{ex:22.18}, \ref{ex:22.20}). To
conclude this section, there is enough evidence that control into manner
gerundival clauses does not have to be syntactic, which leaves us with absolute
gerunds as the only construction in which control may indeed be established
syntactically.

\section{Different types of \gls{IAg} in different types of
passive}\label{sec:22.4}

The data from control into absolute/temporal gerunds seem to suggest that a
crucial variable is the interpretation of the implicit pronominal element.
Covert pronominal elements of the sort discussed here have arbitrary reference
and it appears that \citegen{Cinque:1988} broad distinction between two types of
arbitrary pronominal elements is reflected in the facts under discussion. Thus,
the success of implicit control\is{control!implicit control} often depends on
the extent to which the interpretation of
the presumed implicit argument\is{implicit arguments} falls under each of the
two interpretations that Cinque distinguishes: (i) \emph{quasi-existential
ARB}, which is compatible with the existence of a unique referent (cf.\ the
interpretation of \emph{they} in \emph{They have called for you; I think it was
your brother}) or (ii) \emph{quasi-universal ARB}, the interpretation of
generic arbitrary arguments that necessarily includes more than one individual,
potentially every relevant individual (cf.\ the interpretation of \emph{you} in
\emph{When you eat in Spain, you eat well}).

Existentially bound agents in (short) episodic verbal passives\is{passive} have
the properties of \citegen{Cinque:1988} \enquote{quasi-existential}
arbitrary pronominal elements (ARB): (i) they are compatible with specific time
reference (\ref{ex:22.27}a), (ii) they are compatible with the existence of
a single individual satisfying the description (\ref{ex:22.27}b), (iii)
they are incompatible (on the existential interpretation) with generic time
reference, (iv) they are restricted to external argument roles, and (v) they
are necessarily [+human] (\ref{ex:22.27}c).

\ea%27
    \label{ex:22.27} Adapted from \textcite{Roberts2014b}
    \ea This question was answered yesterday afternoon.
    \ex This question was answered rudely (I think it was Fred).
    \ex Strangers were barked at for fun.
    \z
\z

These properties are all present in the agentive readings of non-active
constructions of transitive predicates in \ili{Greek}. But, as shown in
\REF{ex:22.16} above, such understood agents fail to control into absolute
gerunds. To make sure that they are not syntactically realised in such
constructions and that there is no mysterious/independent ban on control by
this specific type of implicit argument\is{implicit arguments} in \ili{Greek}, it would suffice to find some
other construction with de\-mot\-ed/un\-pro\-nounced agents that does allow them to
control into a non-finite clause. Indeed, event nominalizations with objects
occupying a (unique) functional genitive\is{genitive case} position can license absolute gerunds
whose null subject is successfully controlled by the understood agent
\REF{ex:22.28}.

\ea%28
    \label{ex:22.28}\ili{Greek}\\
    \gll    Etia  tu  xtesinu  distiximatos  itan \dots{} i  katanalosi megalon  posotiton  alkool    [ PRO  odigondas ]\\
            cause   of-the  yesterday’s  {car accident}  was {} the  consumption of-large  amounts  of-alcohol {} {} driving\\
    \glt    \enquote*{The cause of yesterday's car accident was the consumption
    of large amounts of alcohol while driving.}
\z

\citet{AlexiadouEtAl2015}, who concede that implicit agents of nominals need to
be syntactically projected, note that “nominals differ from [episodic] passives
in that the implicit argument\is{implicit arguments} cannot be existentially bound”
\parencite[238]{AlexiadouEtAl2015}. \glspl{IAg} in nominals seem to behave more
like principle B pronouns, they can be bound by a referring expression outside
their \isi{binding} domain or they can serve as variables bound by a quantifier
\REF{ex:22.29}.

\ea%29
    \label{ex:22.29} \textcite{Bruening2014}, via
    \textcite[238]{AlexiadouEtAl2015}\\
    Every journalist\tss{i} hopes that a conversation
    \gls{IAg}\tss{i} with the president will be forthcoming.
\z

Notwithstanding \citeauthor{AlexiadouEtAl2015}'s observation regarding \isi{binding}, we
can establish a certain striking similarity between quasi-existential ARB in
episodic verbal passives\is{passive} and syntactically projected null pronominal \glspl{IAg}\is{implicit arguments} in
Greek nominals: they are both restricted to external theta-roles. As we show in
\REF{ex:22.30}, the internal argument of an unaccusative\is{unaccusativity} predicate is not a
licit controller.

\ea\label{ex:22.30} \ili{Greek}\\
    \gll    Pliroforithika      enan thanato [ PRO diefthinontas orchistra]\\
            {learnt / heard-of.\Fsg} a  death {} {}     conducting
            orchestra\\
    \glt    \enquote*{I heard of a death while conducting the orchestra.} (PRO=hearer/*the deceased)
\z

Crucially, non-agents can control only as long as the interpretation is generic
rather than episodic \REF{ex:22.31}.

\ea%31
    \label{ex:22.31} \ili{Greek}\\
    \gll    O     thanatos   [ PRO diefthinontas (tin) orchistra ] \dots{} ine to   kalytero  telos  ja  enan / ton  maestro\\
            the  death {} {} conducting the  orchestra {} {} is  the best  end for  a  {} the  conductor\\
    \glt    \enquote*{The best death for a conductor is while conducting the orchestra.}
\z

In fact, in generic nominals, PRO can be controlled by agent and non-agent
implicit arguments alike.

\ea%32new
    \label{ex:22.32n} \ili{Greek}\\
    \gll    To prosektiko klidhoma tis portas PRO vjenondas apo to ktirio ine aparetito.\\
            the	careful locking of-the door {} getting-out from the building is necessary\\
    \glt    \enquote*{The careful locking of the door/carefully locking the
    door when getting out of the building is necessary.}
\z

The contrast between generic and episodic nominals points to the different
categorial/featural status of implicit arguments in the former. Arguably, the
controller in \REF{ex:22.31} is an arbitrary, non-referential element, and
more specifically a \emph{quasi-universal} ARB, following \citegen{Cinque:1988}
dichotomy. Such ARB elements are known to be (i) compatible with all
theta-roles/not restricted to external arguments, (ii) compatible with generic
time reference, and (iii) incompatible with specific time reference. All of
these properties are manifested in \REF{ex:22.31}. \citet{Roberts2014b}
derives the thematic restrictions (and the absence thereof) on arbitrary
arguments from potential intervention effects between ARB and its licenser.
Specifically, he proposes that quasi-existential ARB elements (e.g.\ \glspl{IAg}\is{implicit arguments} in
episodic verbal passives) are licensed by T, while quasi-universal ARB is
licensed by a generic operator (GEN) in C. Thus, GEN can license the closest
ARB in its domain, i.e.  anything that ends up in subject position, Spec-TP,
whereas T can only license elements in Spec-\emph{v}P (\ref{ex:22.32}a);
according to Roberts, there can be no dependency between T and ARB if the
latter is (i) in an internal argument position of the \isi{passive}, as the external
argument in Spec-\emph{v}P would intervene (\ref{ex:22.32}b); (ii) in an
internal argument position of a non-stative unaccusative,\is{unaccusativity} as an Event argument
would intervene (\ref{ex:22.32}c), or (iii) in an internal argument position of
a stative unaccusative, as a Loc argument would intervene (\ref{ex:22.32}d).

\ea%32
    \label{ex:22.32} \textcite[5]{Roberts2014b}
    \ea[]{T\tss{i} [\tss{\emph{v}P}  arb\tss{i}  [VP \dots}
    \ex[*]{T\tss{i} [\tss{\emph{v}P} \gls{EA} [\tss{VP} \dots{} arb\tss{i} \dots{}}
    \ex[*]{T\tss{i} \dots{} Ev \dots{} [\tss{VP} \dots{} arb\tss{i} \dots{}}
    \ex[*]{T\tss{i} \dots{} Loc \dots{}  [\tss{VP} \dots{} arb\tss{i} \dots{}}
    \z
\z

That \REF{ex:22.31} is no exception to Roberts’ licensing principle is
shown by the fact that such nominals, containing an ARB internal argument,
would be illicit in object position. Such a dependency between GEN in C and ARB
within DP would violate the \isi{phase impenetrability condition} (which version of
the \glsunset{PIC}\gls{PIC} is operative here, i.e.\ \citegen{Chomsky2000} \enquote{strong} or his
(\citeyear{Chomsky2001}) \enquote{weak} formulation depends on whether
DP/\emph{n}P is a phase\is{phases}). In \REF{ex:22.33}, PRO cannot be interpreted as
bound by a quasi-universal ARB; in fact, in this context the gerund cannot be
part of the object nominal at all and PRO can only be bound by the matrix
subject.\is{PIC|see{phase impenetrability condition}}

\ea%33
\label{ex:22.33} \ili{Greek}\\
    \gll    O  Mitropulos\tss{m}  fovotan / ksorkize / imnuse  to  thanato (*ARB\tss{i}) PRO\tss{m/*i}    diefhtynondas     tin  orchistra\\
            the  Mitropulos  feared {} exorcised {} extolled  the  death {} {} conducting      the  orchestra\\
    \glt    \enquote*{Mitropulos feared / exorcised / extolled death when conducting the orchestra.}
\z

The fact that non-generic \glspl{IAg}\is{implicit arguments} in nominals are subject to the same restriction
as quasi-existential \glspl{IAg}\is{implicit arguments} of episodic verbal passives\is{passive} suggests that a similar
licensing mechanism is at play. I propose that the relevant licensing head is
the lowest functional projection c-commanding the agent in event nominals,
arguably \emph{n} \eqref{ex:22.34}.  Then the same intervention effects arising in the
possible verbal configurations in \REF{ex:22.32} will have to arise within
nominals. Also, if T as a licenser is responsible for some of the interpretive
effects of the \gls{IAg}\is{implicit arguments} in episodic verbal passives\is{passive} (e.g.\ existential \isi{binding}), the
absence of T in the DP also explains the lack of such readings for \glspl{IAg}\is{implicit arguments} in
passive nominals.

\ea%34
    \label{ex:22.34}
    {}[\tss{\emph{n}P} (R-argument) \emph{n} [\tss{\emph{v}P}  \gls{EA}  \emph{v} \dots{} ]]
\z

To sum up our findings so far, in \ili{Greek} nominals both generic and non-generic
IAs can be licensed and both can control into temporal gerunds. On the
contrary, in episodic verbal passives\is{passive}, existentially bound \glspl{IAg}\is{implicit arguments} cannot be
controllers of \isi{null subjects} in temporal gerunds. We have not explored the
status of generic/quasi-universal \glspl{IAg}\is{implicit arguments} in verbal passives\is{passive} yet. Interestingly,
generic verbal passives\is{passive} are \emph{not} incompatible with an \gls{IAg}\is{implicit arguments} controlling into
absolute gerunds. Such \gls{IAg}\is{implicit arguments} arbitrary elements are
clearly quasi-universal:

\ea%35
    \label{ex:22.35}\ili{Greek}\\
    \gll    (?Didaskontas),   I  antidrasis ton     mathiton  prepi  {na lamvanonde} ipopsi     (?didaskontas)\\
            \hphantom{(?}teaching  the  reactions of-the students  must  be-taken into-account  \hphantom{(?}teaching\\
    \glt    ‘When teaching, the students’ reactions must be taken into account’
\z

Even more interestingly, notwithstanding the ban on existentially bound IA
controllers, episodic sentences like \REF{ex:22.36} below the following, are
also possible.

\ea%36
    \label{ex:22.36} \ili{Greek}\\
    \gll    Afti  i   fotografia travixtike [ PRO fevgontas apo    tin  poli ] \\
            this the     picture      was-taken {} {}    leaving     from  the town\\
    \glt    \enquote*{This picture was taken when leaving the town.}
\z

For most speakers, if there is an obligatory control relation there, then the
unpronounced arguments that get co-indexed both refer to an unspecified set of
people \emph{including} \emph{the} \emph{speaker}.   Even \REF{ex:22.16}
paraphrased below as \REF{ex:22.37} can have a similar reading for some
speakers, if actually uttered by the policeman who shot the suspect or someone
who was with him:

\ea%37
\label{ex:22.37}\ili{Greek}\\
    \gll    \llap{\%}O    ipoptos   pirovolithike   [ PRO pijenondas   na   ton silavume ]\\
            the suspect  was-shot {} {}      going    to  him.\Cl{} arrest.\Fpl{}\\
    \glt    \enquote*{The suspect was shot as we were approaching him to arrest him.}
\z

This surprising effect is reminiscent of so-called non-argumental impersonal
\emph{si} in Italian. Non-argumental \emph{si}, being compatible with
non-external theta-roles is necessarily quasi-universal \citep{Cinque:1988}.
However, in the context of specific temporal reference, a paradoxical,
first plural, interpretation arises (\ref{ex:22.38}b).

\ea%38
    \label{ex:22.38}\ili{Italian}
    \ea[]{Oggi, a Beirut, si nasce senza assistenza medica.\\
            \enquote*{Today, in Beirut, one/babies can be born with no medical assistance.}}
    \ex[\#]{Oggi, a Beirut, si è nati senza assistenza medica.\\
            \enquote*{Today, in Beirut, we were born with no medical assistance.}}
    \z
\z

So, this \Fpl{} interpretation arises when the arbitrary argument typically
receives a quasi-universal interpretation but this is blocked by factors such
as specific time reference (see \citealt{Cinque:1988} and \citealt{Roberts2014b}
for explanations of this phenomenon). Thus, combining our two variables, i.e.
verbal vs nominal \isi{passive} and generic vs.\ non-generic, we get the four-way
typology illustrated in \tabref{tab:22.1}.\largerpage

\begin{table}
    \begin{tabular}{lcc}
    \lsptoprule
    ARB & Verbal passives\is{passive} & “Passive” event nominals\\
    \midrule
    Quasi-existential/non-generic & * & Yes\\
    Quasi-universal & Yes & Yes\\
    \lspbottomrule
    \end{tabular}
    \caption{Control into absolute gerunds\label{tab:22.1}}
\end{table}

Nevertheless, looking more closely at the properties of genitive/possessivised
themes in \ili{Greek}, it turns out that they are not always possible in the
presence of an \gls{IAg}\is{implicit arguments}. Implicit control is licit when
the genitivised theme is a full lexical DP (\ref{ex:22.39}a,
\ref{ex:22.40}a), but this kind of co-indexation is impossible when the
theme is realised by a clitic attaching to an adjective within the DP,
typically the leftmost one (\ref{ex:22.39}b,
\ref{ex:22.40}b).\footnote{An anonymous reviewer takes issue with the
    judgements reported in this section regarding control from the implicit
    argument of nominals into such absolute gerunds, which she finds
ungrammatical (regardless of the realisation of the internal argument of the
nominal, I suppose). Apart from myself, 6 other native speakers were consulted,
who all agree with the judgements reported here.}

\ea%39
    \label{ex:22.39}\ili{Greek}
    \ea[]{
    \gll    I        sixni  xrisi  narkotikon  IA\tss{i}  tote  PRO\tss{i}  telionondas  ti diatrivi\\
            the  frequent  use  drugs.\Gen{} {} then {} writing-up  the thesis\\
    \glt    \enquote*{The frequent use of drugs back then, when writing up the thesis \dots{}}}
    \ex[*]{
    \gll    I        sixni  tus                xrisi  IA\tss{i}  tote PRO\tss{i}  telionondas  ti    diatrivi\\
            the  frequent  \Tpl{}.\Cl{}.\Gen{}  use {} then {} writing-up  the thesis\\
    \glt    \enquote*{Their frequent use back then, when writing up the thesis \dots}}
    \z
\ex%40
    \label{ex:22.40}\ili{Greek}
    \ea[]{
        \gll    To  aprosekto  klisimo    tis  portas\tss{p} IA\tss{i},  PRO\tss{i} vjenondas apo  to  spiti,  epetrepse  stus  kleftes  na  bun  anenoxliti\\
                the  mindless   shutting  of-the   door {} {} leaving from  the  house allowed   to-the  thieves  to  enter  easily\\
        \glt    \enquote*{The mindless shutting (e.g.\ without locking) of the door, when leaving the house, let the thieves enter easily.}}
    \ex[*]{
        \gll    to  prosektiko / dhiko  tis\tss{p}  klisimo      IA\tss{i},  PRO\tss{i} vjenondas apo  to   spiti   kratise tus   kleftes  makria\\
                the  careful {} own    her.\Cl{}  shutting {} {} leaving from   the  house  kept  the thieves  away\\
        \glt    \enquote*{Its careful / own locking when leaving the house prevented the thieves from entering.}}
    \z
\z

On the other hand, in generic contexts, implicit control by the implicit
(quasi-arbitrary) agent is possible in the presence of both genitive DP themes
(see~\ref{ex:22.32n}) and themes realised as genitive clitics:

\ea%41n
    \label{ex:22.41n} \ili{Greek}\\
    \gll    To prosektiko tis klidhoma IA\tss{i} PRO\tss{i} vjenondas apo to ktirio ine aparetito.\\
            the careful its locking {} {} getting-out from the building is necessary\\
	\glt    \enquote*{Its careful locking (=of the door) is necessary when getting out of the building.}
\z

In \ili{Greek} process nominals, only one argument can be realised as a genitive\is{genitive case} DP,
unlike e.g.\ in \ili{German} or \ili{Latin}. This suggests that there is a
unique functional projection licensing such genitives (see
\citealt{AleHaeSta2007} and references therein) and therefore a unique probe
for DPs above the thematic domain. Attraction of a genitive\is{genitive case} argument to the
relevant functional projection is followed by movement of the head noun (or
\emph{n}P) immediately above the genitive\is{genitive case}.

\ea%41
    \label{ex:22.41}
    {}[ \dots{}\tss{} \emph{n} F\tss{\Gen{}}\textsuperscript{0}
        [\tss{\emph{n}P} \sout{\emph{n}}
            [  ext.argument [ int.argument \dots{} \sout{N} \dots{} ]]]]
\z

Apart from the genitive\is{genitive case} realisation of one of the arguments, \ili{Greek} also allows
for the realisation of adnominal arguments as possessive \isi{clitics}. In fact, a
(unique) genitive\is{genitive case} DP, which realises one of the arguments, can co-occur with a
possessive clitic, realising an additional argument. Such co-occurrence
obligatorily obeys Superiority, such that the higher argument is realised as a
clitic, while the genitive\is{genitive case} DP necessarily realises a lower, internal argument
\REF{ex:22.42}.

\ea%42
    \label{ex:22.42}\ili{Greek}\\
    \gll    I  proti  mu  perigrafi  tis    Marias\\
            the  first  my   description  the.\Gen{}  Mary.\Gen{}\\
    \glt    \enquote*{my first description of Mary / *Mary’s first description of me}
\z

When two overt arguments co-occur, the clitic is realised higher than the head
noun. Therefore, the probe for possessive \isi{clitics} is higher than the landing
site of the moved head noun \REF{ex:22.43}.

\ea%43
    \label{ex:22.43}
    {}[ \dots{} F\tss{\Poss\Cl{}}\textsuperscript{0}
        [ \emph{n} F\tss{\Gen{}}\textsuperscript{0}
            [\tss{\emph{n}P} \sout{\emph{n}} [  ext.argument
                [ int.argument  \dots{}  \sout{N} \dots{} ]]]]]
\z

Movement of an internal argument genitive\is{genitive case} DP to F\tss{\Gen{}} across the
external thematic position (\ref{ex:22.39}a, \ref{ex:22.40}a) seems
to be fine, but movement of a clitic (\ref{ex:22.39}b,
\ref{ex:22.40}b) is out. This indicates that the intervention of the
implicit agent gives rise to minimality effects relativised to the features of
the probe. F\tss{\Gen{}} can attract full lexical DPs, so its probe consists of
both phi-features, i.e.  number and gender, and some additional feature,
probably [+D] or [+NP].  F\tss{\Poss\Cl{}}\textsuperscript{0} instead, which
can at most attract \isi{clitics}, comprises no more than a bundle of phi-features.
Following featural relativised minimality
\parencite{Starke2001,Rizzi2001,Rizzi2013}, summarised in \REF{ex:22.44}
below, the features of the \gls{IAg}\is{implicit arguments} must be such that
they make it an offending intervener when the probe is
F\tss{\Poss\Cl{}}\textsuperscript{0}, but not when the probe is
F\tss{\Gen{}}\textsuperscript{0} \REF{ex:22.45}. In other words, the
feature makeup of a non-generic \gls{IAg}\is{implicit arguments} is that of a
(possessive) pronominal clitic.\largerpage

\ea%44
    \label{ex:22.44}
    Featural relativised minimality:\\
    A local relation cannot hold between X and Y when Z intervenes, and Z is
    somehow a potential candidate for the local relation. The features of X
    should not be a subset of the features of Z.\smallskip\\
    \begin{tabular}{lllll}
    & X \dots{} & Z \dots{} & Y & \\
    a. & +A \dots{} & +A \dots{} & \tuple{+A} & * \\
    b. & +A+B \dots{} & +A \dots{} & \tuple{+A+B} & ok \\
    \end{tabular}
\ex%45
    \label{ex:22.45}%
    \begin{tabular}[t]{lllll}
    F\tss{\Poss\Cl{}}\textsuperscript{0} & F\tss{\Gen{}}\textsuperscript{0} & ext.arg.   & int.arg.        & \\
    +φ                               &                               & IA\tss{+φ} & clitic\tss{+φ}  & *\\
    +φ                               &                               &            & clitic\tss{+φ}  & ok\\
                                     & +φ, +D/+NP                    & IA\tss{+φ} & DP\tss{+φ, +NP} & ok\\
    \end{tabular}
    \z

Turning to verbal passives\is{passive}, it is necessary to explain the contrast
between quasi-existential and quasi-universal arbitrary \glspl{IAg}\is{implicit
arguments}. The feature makeup of existentially bound \glspl{IAg}\is{implicit
arguments} is arguably the same as that of non-generic \glspl{IAg}\is{implicit
arguments} in nominal passives\is{passive}, namely a simple bundle of
phi-features. This is in line with the fact that \ili{Greek} is a null subject
language and, thus, its T should be able to attract non-lexical/weak pronominal
elements such as \emph{pro}. It appears then that quasi-existential ARB fully
matches T’s uninterpretable features,\footnote{In fact I am assuming that the
    only kind of goal that matches T's features is \emph{pro}. Thus, in line
    with \textcite{AleAna1998}, it follows that any overt DP subjects are
    either \glsunset{CLLD}\gls{CLLD}ed\is{clitic left dislocation} topics\is{topicalization} (when
    preverbal), with \emph{pro} serving as a clitic in the relevant sense, or
    the result of \glsunset{CLRD}\gls{CLRD}\is{clitic right dislocation}/\isi{clitic doubling} (when
postverbal).} thus blocking further probing downwards
(\ref{ex:22.46}a)\footnote{Recall that, unlike other \isi{null subject languages} (e.g.\ \ili{Italian}/\ili{Spanish}), Modern \ili{Greek} lacks participial
passives\is{passive}, which may provide a mechanism of circumventing the
intervention of the \gls{IAg}\is{implicit arguments}, i.e.\
\citegen{Collins2005} \enquote{smuggling}.}. Quasi-universal probes on the
other hand must have a reduced/defective feature makeup (\ref{ex:22.46}b).
Indeed, unlike episodic passives\is{passive}, generic passives\is{passive} do
not allow the \gls{IAg}\is{implicit arguments} to be co-indexed with a
\emph{by}-phrase. Also ARB in such (generic) passives\is{passive} can marginally bind
an anaphor, but that has to be (generic) second person singular (which is also
its default person when realised overtly) or first person plural
(\ref{ex:22.47}a), as opposed to non-generic \glspl{IAg}\is{implicit
arguments} which are compatible with any [Person] value (\ref{ex:22.47}b).
Thus, it can be argued that quasi-universal ARB lacks an
interpretable/lexically valued person feature\is{person features} (and possible also gender), as
its person is valued by default. This makes its feature specification a proper
subset of T’s probing features and its intervention is not enough to block T
from probing and matching the internal argument.\largerpage

\ea%46
    \label{ex:22.46}\leavevmode\\[-1\baselineskip]
    \begin{tabular}{lllll}
       & T  & Spec\emph{v}P                        & Object                          & \\
    a. & +φ & qu-$\exists$ IA\tss{+φ}              & pro\tss{+φ}/DP\tss{+φ, +D, +NP} & *\\
    b. & +φ & qu-$\forall$ IA\tss{+number,uPerson} & pro\tss{+φ}/DP\tss{+φ, +D, +NP} & ok\\
    \end{tabular}
\ex%47
    \label{ex:22.47}\ili{Greek}
    \ea[?]{
        \gll    i  antidrasis  ton  allon  prepi  na  lamvanonde ipopsi    milondas  ja  ton eafto  su / mas / *tu / *tus \\
                the  reactions  of-the  others  must  to  be-taken into-account  talking about the   self  your {} our {} \hphantom{*}his {} \hphantom{*}their \\
        \glt    \enquote*{The reactions of the others must be taken into consideration when talking about yourself/ourselves/himself/themselves.}}
    \ex[]{
        \gll    i   efarmoji   tis   therapias IA\tss{i/j/k/l/m/n/p}   ston  eafto mu\tss{i} / su\tss{j} / tis\tss{k} / tu\tss{l} / mas\tss{m} / sas\tss{n} / tus\tss{p}  itan terastio  lathos.\\
        the  application  of-the  therapy  {}  to-the  self my {} your {} her {} his {} our {} your.\Pl{} {} their  was  huge    mistake\\
        \glt    \enquote*{Applying the therapy to myself / yourself / herself / himself / ourselves / yourselves / themselves was a huge mistake.}}
    \z
\z

To conclude this section, when manipulating a number of variables concerning
the behaviour or implicit arguments intervening in an \isi{Agree} relationship,
namely their generic/non-generic interpretation and the nature of the probe, it
turns out that \glspl{IAg}\is{implicit arguments} do cause relativised minimality effects, thus providing a
clear argument that they are syntactically projected whenever \isi{Agree} goes
through. \Cref{tab:22.2} presents all the conceivable combinations of the
different states of the variables discussed in this section and their relativized
minimality-based analysis.

\begin{table}
\begin{tabular}{llllc}
\lsptoprule
\multicolumn{5}{c}{Passive nominals}\\
\midrule
{F\tss{\Poss\Cl{}}} & {F\tss{Gen}} & {External argument} & {Int. argument} & \\
\midrule
+φ &  & non-generic, +φ & clitic\tss{+φ} & *\\
+φ &  & generic/qu-${\forall}$, iNumber, 0Person & clitic\tss{+φ} & OK\\
+φ &  & not projected & clitic\tss{+φ} & OK\\
& +φ, +D & non-generic, +φ & DP\tss{+φ,+D} & OK\\
& +φ, +D & generic/qu-${\forall}$, iNumber, 0Person & DP\tss{+φ,+D} & OK\\
& +φ, +D & not projected & DP\tss{+φ,+D} & OK\\
\midrule
\multicolumn{5}{c}{Verbal passives}\\
\midrule
\multicolumn{2}{c}{{T}} & {External argument} & {Int. argument} & \\
\midrule
\multicolumn{2}{c}{+φ} & Qu-$\exists$, +φ & pro\tss{+φ} & *\\
\multicolumn{2}{c}{+φ} & Qu-$\forall$, iNumber, 0Person & pro\tss{+φ} & OK\\
\multicolumn{2}{c}{+φ} & not projected & pro\tss{+φ} & OK\\
\lspbottomrule
\end{tabular}
\caption{Possible and impossible combinations of probes and covert ARB pronouns\label{tab:22.2}}
\end{table}

\section{Conclusions, implications for the theory of passives, open
questions}\label{sec:22.5}

In this paper, a new argument was put forward for the syntactic realisation of
some implicit agents, based on relativised minimality effects in \isi{Agree} which
can only be explained if an \gls{IAg}\is{implicit arguments} is indeed projected. Given the patterns
observed, \glspl{IAg}\is{implicit arguments} that control into non-finite (adjunct) clauses are real syntactic
objects, and at the same time constructions with \isi{passive} readings may in fact
not contain syntactically represented \glspl{IAg}\is{implicit arguments}, given that their presence would
cause an irreparable minimality violation and block licensing of the promoted
internal argument.

The latter scenario is exactly what happens with existentially bound agents in
Greek short episodic verbal passives\is{passive}. This has certain implications for the
theory of passives\is{passive}. A truly \isi{passive}, i.e.\ agentive, interpretation is possible
even when the language lacks a dedicated \isi{passive} Voice. Generalising a bit, it
can be argued that demoted theta-roles must be represented if the grammar
allows them to be represented. For instance, there can be no agentive reading
for a construction lacking both an external argument subject and passive
morphology, if \isi{passive} morphology is independently available in the
language. However, if the grammar does not provide a syntactic slot for an
understood argument, another related operation/construction (e.g.  the
homophonous middle/reflexive in Greek) is employed as some sort of last resort
and the demoted theta-role can, in fact has, to be inferred. \ili{Greek} does
not lack agentive readings, as shown by the felicitous use of agent-oriented
adverbials \REF{ex:22.48} -- which is therefore not to be taken as a safe
diagnostic for syntactically realised agents).

\ea%48
    \label{ex:22.48}\ili{Greek}\\
    \gll    To     plio  vithistike   epitides.\\
            the ship   was-sunk   deliberately\\
\z

Nevertheless, in the absence of such an adverb or a related expression specific
to agentive readings, the \ili{Greek} construction is ambiguous between the passive
and other intransitive readings (e.g.\ anticausative or reflexive). Therefore,
in the absence of mechanisms that would allow verbal constructions in which a
quasi-existential \gls{IAg}\is{implicit arguments} can survive, \ili{Greek} has to make do with a middle Voice, as proposed by
\citet{SpathasEtAl2015}, which allows the understood agent to be anyone,
including the individual referred to by the internal argument (see also
\citealt{AlexiadouDoron2012}). In other words, the denotation of the relevant
Voice head in \ili{Greek} is the one proposed in \REF{ex:22.6} above, without the
presupposition that derives disjointness – repeated (and adapted) here as (6$'$).
(\ref{ex:22.49}a,b) illustrates the relevant contrast between \ili{English}
and \ili{Greek}. It remains to be seen if natural languages do this more
widely, i.e.\ whether in the absence of a syntactic mechanism that can be used
for the grammaticalisation,\is{grammaticalization} i.e.\ the obligatory
expression, of a meaning, related constructions are employed and the otherwise
grammaticalised\is{grammaticalization} meaning is only an inferred meaning.

\begin{exe}
    \exi{(6$'$)}
    ⟦Middle⟧ = λf\tss{es,t} λe${\exists}$x.f(x)(e)\label{ex:22.6prime}
\ex%49
    \label{ex:22.49}
    \ea They\tss{i} were being killed e\tss{*i} t\tss{i}.
    \ex Skotonondusan\\
        \enquote*{They were being killed\slash they were killing themselves\slash they
        were killing each other.}
    \z
\z

The unavailability of an English-like syntax for existentially-bound agents is
due to the feature specification of \isi{null subjects} and of intervening implicit
arguments, as well as the absence of other mechanisms that can circumvent the
intervention of the external argument (e.g.\ participial passives\is{passive}
may allow \citegen{Collins2005} smuggling). As opposed to quasi-existential
covert pronouns, quasi-universal ones can be projected causing no minimality
effects, therefore \ili{Greek} also has an agentive \isi{passive} Voice which
may only host a (reduced) φ-bundle in its Spec \citep{Legate2014}. This
configuration gives rise to generic passives\is{passive} or to episodic
passives\is{passive} with a paradoxical first plural interpretation of the
understood agent. Generic passives also subsume dispositional middles in
\ili{Greek}, which have independently been argued by \citet{Lekakou2005} to
involve syntactically projected agents.

It would also be interesting to explore whether in some languages the
possibility for syntactically expressed implicit arguments is suppressed in a
subset of argument-demoting constructions only, thus forcing such argument
relationships to be inferred. If extended to examples such as
(\ref{ex:22.11}a,b), then the present account would also reduce Visser’s
generalization to relativised minimality: passivisation of the indirect object
is impossible exactly because Agree with T is blocked by an intervening
\gls{IAg}\is{implicit arguments} which controls into a non-finite complement
clause. Such an explanation would have wider implications for the analysis of
goal passives\is{passive} more generally, but I will leave this issue open for
future research. Finally, another set of predictions of the present account
that needs to be tested concerns languages with partial pro drop, especially
subject drop which is available only for some person values but not others; the
prediction is that the same arbitrary element should exhibit variable
minimality effects, depending on the person feature\is{person features} of the
promoted/agreeing internal argument. This is also something that I will put
aside now and hope to address in future work.

\printchapterglossary{}

\section*{Acknowledgements}

I am indebted to Giorgos Spathas for his criticism and lots of fruitful
suggestions, as well as the audience of the RUESHeL group seminar, at Humboldt
University, Berlin, in December 2017, especially Artemis Alexiadou and Florian
Schäfer. For discussions on data and judgements, I would like to thank George
Tsoulas, Nikos Angelopoulos and Margarita Makri, among others, although none of
them bear any responsibility for the judgements reported here. For their
continued patience I wish to sincerely thank the editors of the volume,
especially Theresa Biberauer and András Bárány, as well as two anonymous
reviewers for comments and useful remarks. All errors are my own. Last but
certainly not least, I am grateful to Ian Roberts for sparking my interest in
implicit arguments.

{\sloppy\printbibliography[heading=subbibliography,notkeyword=this]}
\end{document}
