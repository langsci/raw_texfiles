\addchap{Preface}
%\chapter*{Preface}

%\begin{quotation}
%An especially powerful form for theory is a body of underlying 
%mechanisms, whose interactions and compositions provide the answers to 
%all the questions we have. \citep{newell_unified_1990}
%\end{quotation}


This essay explores some conceptual foundations for understanding the natural causes of linguistic systems. At the core of it are three ideas. 

%\begin{enumerate}
%\item[] 
The first is that causal processes in linguistic reality apply in multiple frames\is{causal frames} or ``time scales''\is{scales} simultaneously, and we need to understand and address each and all of these frames in our work. This is the topic of Chapter \ref{causaldynamics}. 

%\item[] 
This leads to the second idea. For language and the rest of culture to exist, its constituent parts must have been successfully diffused\is{diffusion} and kept in circulation in the social histories of communities. This relies on convergent processes in multiple causal frames, and depends especially on the micro-level behavior of people in social interaction\is{social interaction}. This is the topic of Chapter \ref{Transmission biases}. 

%\item[] 
The third idea, building on this, is that the socially-diffusing parts of language and culture are not just floating around, but are firmly integrated within larger systems\is{systems}. We need to understand the link between the parts and the higher-level systems they belong to. This point is underappreciated. Inferences made from facts about \textit{items} are often presented without reflection as being facts about the whole \textit{systems} they fit into. Tree diagrams\is{tree diagrams} help to perpetuate this problem. It is difficult to assess work on the history of languages if that work does not offer a solution to the item/system problem\is{item/system problem}. Facts about items need to be linked to facts about systems. We need a causal account of how it is that mobile bits of knowledge and behavior become structured cultural systems such as languages. This is the topic of Chapter \ref{itemsystemproblem} (where the problem is articulated) and Chapter \ref{micromacrosolution} (where a solution is offered). 

%\end{enumerate}
In exploring these ideas, this book suggests a conceptual framework for explaining, in causal terms, what language is like and why it is like that. It does not attempt to explain specifics, for example why one language has verbal agreement involving noun class markers and another language does not. But the basic elements of causal frames and transmission biases, and the item/system dynamics that arise, are argued to be adequate for ultimately answering specific questions like these. Any detailed explanation will work -- explicitly or implicitly -- in these terms. Here is another thing this book does not do: It does not give detailed or lengthy case studies. Instead, the examples are illustrative, and many can be found in the literature referred to. The \textit{Conceptual Foundations of Language Science} book series is intended for short and readable studies that address and provoke conceptual questions. While methods of research on language keep changing, and often provide much-needed drive to a line of work, the underlying conceptual work -- always independent from the methods being applied -- must provide the foundation.
