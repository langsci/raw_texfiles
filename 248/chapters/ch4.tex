\addtocontents{toc}{\protect\enlargethispage{\baselineskip}}\chapter{Language and structure} \label{ch:language and structure}
In \chapref{ch:information and agents}, I described informational spaces as well as agents and their interactions. Among other things, I introduced the infon space $({\cal I}, \odot)$ which will supply the semantic contents of utterances in Equilibrium Semantics. Assessing utterances requires certain operations on words and parse trees analogous to the operation $\odot$ on infons and so I now define two systems similar to $({\cal I}, \odot)$. One serves as the language to be analyzed and the other provides the syntactic contents.

\section{Language} \label{sec:language definition}
Assume a finite set of words $\cal W$. For example, $\cal W$ might include $\{\Expression{Bill}, \Expression{ran}\}$. A string on $\cal W$ of length $n$ is a function from $\{0, 1, 2, \ldots, n\}$ to $\cal W$. We usually write just a list of words such as $\Expression{Bill}$, $\Expression{Bill ran}$, or $\Expression{ran Bill}$. The unique string of length $0$, the empty string, is denoted by $e$. The unique zero string $0$ can be thought of as an illegitimate string. Both $e$ and $0$ are always members of $\cal W$. The concatenation $\cdot$ of two nonzero strings is defined in the usual way\footnote{See \citet[164--166]{wall:ml} for example.} and $w \cdot w'$ is written $ww'$. For example, $\Expression{Bill} \cdot \Expression{Bill} = \Expression{Bill Bill}$ and $\Expression{ran} \cdot \Expression{Bill} = \Expression{ran Bill}$. The concatenation of any string with $0$ is $0$. ${\cal W}^{\ast}$ is the set of all strings on ${\cal W}$ and is called the free monoid on $\cal W$. It is an infinite set and has an identity $e$ and a zero $0$.

\is{grammar}

Assume $G$ is a context-free grammar (CFG).\footnote{A context-free grammar is a grammar whose rules all have the form $\mathrm{A} \to w$. That is, there is no (linguistic) context surrounding $\mathrm{A}$ and $w$ so that the rule can be freely applied. See \citet[Chapter~9]{wall:ml}.}  Here is a simple example:
\[\mathrm{S} \to  \hbox{NP VP};\ \ \ \ \hbox{NP} \to \hbox{N};\ \ \ \ \hbox{VP} \to \hbox{V};\ \ \ \ \hbox{N} \to \hbox{Bill};\ \ \ \  \hbox{V} \to \hbox{ran}\]
Here, ``S'' stands for sentence, ``NP'' for noun phrase, ``VP'' for verb phrase, ``N'' for noun, and ``V'' for verb. The only sentence generated by this CFG is $\Expression{Bill ran}$. I will use ``parse'' and ``parse tree'' to refer also to the subtrees of phrases with the root node being any relevant nonterminal symbol. A tree such as $[_{\mathrm{NP}} [_{\mathrm{N}}\, \mathrm{Bill}]]$ would count as a parse or parse tree of the noun phrase \Expression{Bill}.

I now use concatenation to define an operation that yields exactly the subsentential expressions and sentences of the language 
$\cal L$ by forming an appropriate proper subset of ${\cal W}^{\ast}$. Define the following modification of concatenation: $w \circ_G w' = w \cdot w' = ww'$ when $ww'$ is a substring of some string in ${\cal W}^{\ast}$ that has at least one parse by $G$ and $w \circ_G w' = 0$ otherwise. The set formed by freely generating all strings from ${\cal W}$ by this special concatenation operation is the language $\cal L$. In our little example, ${\cal L} = \{e, 0, \Expression{Bill}, \Expression{ran}, \Expression{Bill ran}\}$. For instance, the string \Expression{ran Bill} has been dropped from ${\cal W}^{\ast}$ because it is not a substring of any string parseable by $G$.\footnote{Consider the sentence \Expression{I handed you the salt}. Then the string \Expression{you the} is a member of the corresponding $\cal L$ even though it is not a legitimate phrase because it is a substring of the whole sentence which would be parseable by the corresponding $G$. I owe this example to Tom Wasow.\ia{Wasow, Thomas}} This operation is called \emph{grammatical concatenation} and is abbreviated to $\circ$. It is associative but not commutative and has $e$ as its identity and the zero element $0$. In other words, $({\cal L}, \circ)$ is a monoid with a zero. A sentence $\varphi$ of $\cal L$ is made up of individual words $\varphi_1 \circ \varphi_2 \circ \ldots \circ \varphi_n = \varphi_1 \varphi_2 \ldots \varphi_n$ for some natural number $n$. For the sentence $\varphi = \Expression{Bill ran}$, $\varphi_1 = \Expression{Bill}$, $\varphi_2 = \Expression{ran}$, and $\varphi = \varphi_1 \circ \varphi_2 = \varphi_1\varphi_2$. 



\section{Algebraic system of trees} \label{sec:algebraic system of trees}
So far, I have defined two algebraic systems $({\cal I}, \odot)$ and $({\cal L}, \circ)$ that capture the structure of infons and linguistic expressions. One describes the world and the other language. The third system involves a new way to describe the grammar $G$ as a system of trees with a product operation.

Each of the five rules of the CFG above can be re-described as a tree. For example, the tree corresponding to the first rule is $[_{\mathrm{S}}[_{\mathrm{NP}}\, ][_{\mathrm{VP}}\, ]]$ and the tree corresponding to the fourth rule is $[_{\mathrm{N}}\, \mathrm{Bill}]$. Thus, $G$ can be expressed either as a set of rules or as a set of trees. However, we cannot define the desired operation on these trees directly and a little work is required to get them into the appropriate form.

The product operation is defined in two stages. First, an intuitive substitution or merging operation on parse trees is specified as follows. A tree such as $t' = [_{\mathrm{X}}\, \ldots ]$ can be substituted into $t = [_{\mathrm{Z}} [_{\mathrm{X}}\, ] \ldots ]$ to form $t'' = [_{\mathrm{Z}} [_{\mathrm{X}}\, \ldots ] \ldots ]$ where the $\ldots$ from $t'$ have been entered into $t$ because the outer category X of $t'$ matched an inner category X of $t$. If there is more than one X in $t$ that matches the X in $t'$, then the leftmost one is substituted into. This operation is denoted $\lhd$ and we write $t \lhd t' = t''$. It is neither associative nor commutative and is identical to the substitution operation defined in \citet{joshi:tag} and \citet{joshi:tag2}.

Consider now the following merging of the second and fourth tree above: $[_{\mathrm{NP}} [_{\mathrm{N}}\, ]] \lhd [_{\mathrm{N}}\, \mathrm{Bill}] = [_{\mathrm{NP}} [_{\mathrm{N}}\, \mathrm{Bill}]]$. This can be informally described by saying that we have merged the simple tree $[_{\mathrm{N}}\, \mathrm{Bill}]$ as far as it could go up the parse tree of the whole sentence without encountering another branch. It cannot go any further because the tree encountered is $[_{\mathrm{S}}[_{\mathrm{NP}}\, ][_{\mathrm{VP}}\, ]]$ which has another branch involving the verb phrase. I call this procedure \emph{chaining}, so we say that simple lexical trees are chained as far up as possible. It is possible to chain the third and fifth trees in the same way to get $[_{\mathrm{VP}}[_{\mathrm{V}}\, \mathrm{ran}]]$. Thus, we are left with just two maximally chained trees from the original five trees and we also have the first tree -- $[_{\mathrm{S}}[_{\mathrm{NP}}\, ][_{\mathrm{VP}}\, ]]$. These are:
\[[_{\mathrm{S}}[_{\mathrm{NP}}\, ][_{\mathrm{VP}}\, ]];\ \ \ \ [_{\mathrm{NP}} [_{\mathrm{N}}\, \mathrm{Bill}];\ \ \ \ [_{\mathrm{VP}}[_{\mathrm{V}}\, \mathrm{ran}]]\]
%After we have obtained such maximally chained trees and unchained trees, we divide them into two groups of chained and unchained trees. In our example, the two groups contain $[_{\mathrm{NP}} [_{\mathrm{N}}\, \mathrm{Bill}]$ and $[_{\mathrm{VP}}[_{\mathrm{V}}\, \mathrm{ran}]]$ on the one hand and $[_{\mathrm{S}}[_{\mathrm{NP}}\, ][_{\mathrm{VP}}\, ]]$ on the other. 
The chained trees, also called elementary trees, are collected and given names as follows:
\[t_1 = [_{\mathrm{NP}} [_{\mathrm{N}}\, \mathrm{Bill}]]\]
\[t_2 = [_{\mathrm{VP}}[_{\mathrm{V}}\, \mathrm{ran}]]\]
The subscripts of $t_1$ and $t_2$ are determined by the sentence being considered, namely, \Expression{Bill ran}, so that the tree involving the first word \Expression{Bill} is indexed by 1 and the tree involving the second word \Expression{ran} is indexed by 2. In more complex sentences, there will be more than nine elementary trees and then the indexes will be written (10), (11), (12), and so on. Keeping track of the indexes in this way makes it easier to describe the operation below. The unchained trees -- just $[_{\mathrm{S}}[_{\mathrm{NP}}\, ][_{\mathrm{VP}}\, ]]$ in our case -- are left in $G$. 

Now, I describe a more complex product operation $\star_{G, u}$ on these trees. It is parametrized by $G$ as $\circ_G$ was and by $u$ as $\odot_{u}$ was. The utterance situation is needed because the sentence being parsed enters via $u$ and it is on the basis of the sentence that the trees can be properly indexed as explained above.

Because the chosen CFG is so simple, there is just one nontrivial product:
\begin{eqnarray*}
t_1 \star_{G, u} t_2 & = & [_{\mathrm{NP}} [_{\mathrm{N}}\, \mathrm{Bill}]] \star_{G, u} [_{\mathrm{VP}}[_{\mathrm{V}}\, \mathrm{ran}]] \\
& = & ([_{\mathrm{S}}[_{\mathrm{NP}}\, ][_{\mathrm{VP}}\, ]] \lhd [_{\mathrm{NP}} [_{\mathrm{N}}\, \mathrm{Bill}]]) \lhd  
[_{\mathrm{VP}}[_{\mathrm{V}}\, \mathrm{ran}]] \\
& = & [_{\mathrm{S}}[_{\mathrm{NP}}\, \mathrm{Bill}][_{\mathrm{VP}}\, ]] \lhd [_{\mathrm{VP}}[_{\mathrm{V}}\, \mathrm{ran}]] \\
& = & [_{\mathrm{S}}[_{\mathrm{NP}}\, \mathrm{Bill}][_{\mathrm{VP}}\, \mathrm{ran}]] \\
& = & t_{12}
\end{eqnarray*}

In other words, the tree product draws upon relevant trees in $G$ such as $[_{\mathrm{S}}[_{\mathrm{NP}}\, ]\allowbreak\relax[_{\mathrm{VP}}\, ]]$ to enable them to be merged or substituted. In this product, only one such tree in $G$ was introduced but in general there could be more, both to the left of $t_1$ or $t_2$. This is why the operation is parametrized by $G$.

We now add two special trees. The first is the empty tree, $t_e$, and the second is a  tree $t_0$. The former serves as the identity of the set of trees obtained via this product and the latter as the zero of the set. That is, $t \star_{G,u} t_e = t_e \star_{G,u} t = t$ for all $t$ and $t \star_{G,u} t_0 = t_0 \star_{G,u} t = t_0$ for all $t$, the latter being true by definition. When two incompatible trees are multiplied, for example $t_1 \star_{G,u} t_{12}$, the result is stipulated to be $t_0$.

Note the special vector index ``12'' of the product $t_{12}$. First, this index should not be confused with an elementary tree index such as (12) which would be expressed as $t_{(12)}$. In the case of $t_{12}$, the vector index has two components; in the case of $t_{(12)}$ the vector index has just one component. Second, it may seem at first sight that we could have obtained the product $t_2 \star_{G,u} t_1 = t_{12}$ in the same way, by first substituting the verb and then the noun. But instead the product $t_2 \star_{G,u} t_1 = t_0$ by stipulation. The basic rule is that the subscript of the first multiplicand must be strictly lower than the subscript of the second multiplicand to potentially yield a nonzero tree. When higher-level trees are multiplied in a more complex setting, the rule is that the first component of the first vector index must be strictly less than the first component of the second vector index. For example, a tree labeled $t_{34}$ would have a lower vector index than one labeled $t_{5}$ because $3 < 5$. So $t_{34} \star_{G,u} t_5$ could potentially be nonzero. In the reverse order, the product is always $t_0$.

There are thus five trees in the operation table for this sentence with respect to this CFG, the three above and $t_e$ and $t_0$. All combinations of these five trees yield one of these five trees. This gives us closure for the operation. The  multiplication table for $\star_{G,u}$ is shown in Figure~\ref{fig:operation table}. Observe that $t_1 \star_{G,u} t_2 = t _{12} \neq t_2 \star_{G,u} t_1$.

\begin{figure}
\renewcommand{\arraystretch}{1.5}
\begin{tabular}{c|c|c|c|c|c|c}
\multicolumn{1}{c}{$\star_{G,u}$} & \multicolumn{1}{c}{$t_e$} & \multicolumn{1}{c}{$t_1$} & \multicolumn{1}{c}{$t_2$} & \multicolumn{1}{c}{$t_{12}$} & \multicolumn{1}{c}{$t_0$} \\[.5ex] \cline{2-6}
$t_e$ & $t_e$ & $t_1$ & $t_2$ & $t_{12}$ & $t_0$ & \qquad \\[.5ex] \cline{2-6}
$t_1$ & $t_1$ & $t_0$ & $t_{12}$ & $t_0$ & $t_0$ & \qquad \\[.5ex] \cline{2-6}
$t_2$ & $t_2$ & $t_0$ & $t_0$ & $t_0$ & $t_0$ & \qquad \\[.5ex] \cline{2-6}
$t_{12}$ & $t_{12}$ & $t_0$ & $t_0$ & $t_0$ & $t_0$ & \qquad \\[.5ex] \cline{2-6}
$t_0$ & $t_0$ & $t_0$ & $t_0$ & $t_0$ & $t_0$ & \qquad \\[.5ex] \cline{2-6}
\end{tabular}
\caption{The operation table for $\star_{G,u}$} \label{fig:operation table}
\end{figure}


The basic rules for forming the product of trees are as follows:

\begin{enumerate}
\item Merge or substitute via $\lhd$ if possible.
\item Otherwise, successively introduce trees from $G$ to the left of either or both multiplicands until one or more substitutions via $\lhd$ are possible. For example, if two trees $t'$ and $t''$ have to be introduced in that order to the left of $t_i$ in a product $t_i \star_{G,u} t_j$, it would be as follows: $(t'' \lhd (t' \lhd t_i)) \lhd t_j$.
\item If the above fails, the result is $t_0$.
\end{enumerate}

This procedure always gives a unique result as only trees such as Z $\to$ X Y from $G$ can left multiply a product like $t_i \star_{G,u} t_j$ where $t_i = [_{\mathrm{X}}\, \ldots ]$ and $t_j = [_{\mathrm{Y}}\, \ldots ]$ and we stipulate that there cannot be another rule $\mathrm{Z}' \to$ X Y with $\mathrm{Z}' \neq$ Z in the CFG. Like $\lhd$, the operation $\star_{G,u}$ is neither associative nor commutative.

Full parsing may be done in either of two ways: by successive application of compatible rules in the CFG to yield S or by successive application of the $\star_{G,u}$ operation to yield a tree like $t_{12}$. All other combinations will ultimately result in $t_0$. In more complex CFGs, there will be multiple trees like $t_{12}$ that may be the end result of combining subtrees, which corresponds to multiple parses for the sentence. The word order of the sentence is automatically taken into account by the product operation owing to the indexing procedure so the yield of successive applications of $\star_{G,u}$ is guaranteed to match it.

Consider the algebraic system $({\cal T},\star_{G,u})$ where $\cal T$ is the set of five trees. This captures the relevant subset of the CFG for this sentence and so is an equivalent way to express a context-free grammar sentence by sentence. Each sentence corresponds to a separate algebraic system, all of which can in principle be combined into a larger system but this is unnecessary in practice. If we start with the elementary trees $t_e$, $t_1$, $t_2$, and $t_0$, we can freely generate the whole set $\cal T$. 

%In general, there will be an infinite number of trees in $\cal T$ but only a finite number of elementary trees and one may work with just these elements.

I will assume the grammar $G$ for $\cal L$ can be rewritten as a system of trees $({\cal T},\star_{G,u})$ for each sentence in the manner described above. For convenience, I will drop the parameters from the notation and write just $\star$ henceforth and also frequently write $t_it_j$ instead of $t_i \star t_j$.


\section{Summary of assumptions}\label{sec:4.3}

Three algebraic systems have been constructed: $({\cal I}, \odot_{u})$, $({\cal L}, \circ_G)$, $({\cal T}, \star_{G,u})$ or, more simply, $({\cal I}, \odot)$, $({\cal L}, \circ)$, $({\cal T}, \star)$. The second system is a monoid with a zero but the  first and third have just an identity and a zero. As the parameter $u$ is fixed at the outset, each sentence in it is identifiable, and so the corresponding subsystem $({\cal T}, \star)$ is also identifiable.

In \chapref{ch:picture of communication}, I sketched what communication looks like in the small and in the large and briefly mentioned the games of partial information that arise. The interaction between speaker and addressee can be partly described by these partial information games and they lead to two more monoidal systems $({\cal G}, \otimes)$ and $({\cal G}', \otimes')$, the first involving semantic games and the second involving syntactic games. These are introduced in the context of the sentence $\varphi = \Expression{Bill ran}$ that we will consider in detail in \chapref{ch:defining communication games}.

A peculiarity of all these systems is that every element of each system that is not an identity element is what is called a zero divisor, that is, a nonzero element for which another nonzero element exists such that their product is zero. Also, there is just an operation of multiplication and the zeros are stipulated rather than arising as identities of a second operation of addition as happens in rings.
