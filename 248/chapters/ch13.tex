\chapter{Relevance} \label{ch:relevance}



Much of this Part falls under the topic of the global game $G_u(\varphi) = LG_u(\varphi) \cup IG_u(\varphi)$. I deferred considering this aspect of meaning as it requires some new insights that are best studied separately. Illocutionary meaning includes free enrichment, implicature, modulation, illocutionary force, and perhaps other things, all very interesting phenomena that allow speakers to convey meanings beyond what is conveyed \emph{directly} by the words in an utterance, that is, beyond locutionary meaning. The same sorts of Constraints operate in this realm. There is no need for a new Phonetic or Syntactic Constraint; they are the same as with locutionary meaning as it is the same sentence $\varphi$ that gives rise to illocutionary meaning. The Semantic Constraint, which generates the possible illocutionary meanings of an utterance, is no longer based on the conventional meaning of the words in the sentence and so does not split into two sub-Constraints, Conventional and Referential, as happened with locutionary meaning. It is based instead on two different sub-Constraints of Relevance and Distance that are needed only for illocutionary meaning, not for locutionary meaning. They perform the same function as the earlier sub-Constraints of generating the \emph{possible} (illocutionary) meanings. The Flow Constraint then disambiguates these possibilities via partial information games as before. Thus, there are many similarities between locutionary and illocutionary meaning, the fundamental one being that the possibilities are first generated and then disambiguated. This is one reason to take all of meaning as belonging to the single realm of \emph{semantics} as argued in \sectref{sec:semantics and pragmatics}.

After describing how the illocutionary meaning of an utterance is computed, I will look at further kinds of meaning that lie beyond illocutionary meaning. At the end of the Part, we will classify all these meanings and discuss the issue of what literal meaning is. But, as was pointed out in \sectref{sec:semantics and pragmatics}, all these distinctions are somewhat indeterminate, even the so-called primary one between literal meaning and implicature.

I will start by discussing the two sub-Constraints of Relevance and Distance in this chapter and the next.

I had defined relevance as the \emph{value} of information, an idea from choice theory, in \citet{parikh:gtai, parikh:rs, parikh:le}. In the later two references, I had extended the standard decision-theoretic notion to game theory, though this extension remained incomplete there for technical reasons. In this chapter, I complete it.

I first examine the general notion more closely, starting with some of the limitations of the way the Relevance Theorists have defined it, and then broadening the discussion to my own idea. 

%After this examination, I go on to consider whether relevance is needed to understand communication at all. Through this, I introduce the need for thinking of semantics in terms of practical action, interaction, and choice as opposed to the mainstream epistemic and cognitive view of agents. I then describe the qualitative structure of communication using my framework of equilibrium semantics and address the pervasive influence of intensionality.
%
%Throughout, my discussion is informal and conceptual with relatively few technical details.


\section{The Relevance-Theoretic concept of relevance} \label{sec:Relevance Theory}

\is{Relevance Theory|(}\ia{Sperber, Dan@Sperber, Dan|(}\ia{Wilson, Deirdre@Wilson, Deirdre|(}
Relevance Theory\footnote{There are many accounts of the theory in the literature. I have referred primarily to \citet{sw:r, sw:pmm} and \citet{sw:odr, sw:rt}.} has been around now for over thirty years and has contributed many rich insights to the fields of semantics, pragmatics, and the psychology of language comprehension. Yet, the formulation of its foundational concept remains weak. This is partly because the idea of relevance is extraordinarily difficult to pin down. But it is also partly because the Relevance-Theoretic approach  is flawed in certain key respects. This weakness threatens the integrity and operation of the theory because both of its primary pillars -- how gaps between sentence meaning\footnote{Relevance theorists take sentence meaning to be a kind of logical form that is an incomplete proposition as opposed to the mainstream Gricean view that it more or less coincides with literal meaning modulo certain linguistically mandated items that need to be filled in contextually. My own view, described in \citet{parikh:le} and in this book, is more radical in that I believe there is nothing like sentence meaning involved in interpretation, just conventionally given word meanings and utterance meaning based on a generalized compositionality, as discussed in \partref{part:III}. That is, the difference between the Relevance-Theoretic view and the Gricean view is one of degree, not one of kind -- as mine is.} and speaker meaning are to be filled and how, more broadly, utterance comprehension occurs -- rest upon computations based on relevance. For this reason, a close examination of its eponymous concept is warranted. Here is their own description.

\begin{quote}

When is an input relevant? Intuitively, an input (a sight, a sound, an utterance, a memory) is relevant to an individual when it connects with background information he has available to yield conclusions that matter to him: say, by answering a question he had in mind, improving his knowledge on a certain topic, settling a doubt, confirming a suspicion, or correcting a mistaken impression. According to relevance theory, an input is relevant to an individual when its processing in a context of available assumptions yields a POSITIVE COGNITIVE EFFECT. A positive cognitive effect is a worthwhile difference to the individual's representation of the world: a true conclusion, for example. False conclusions are not worth having; they are cognitive effects, but not positive ones.

The most important type of cognitive effect is a CONTEXTUAL IMPLICATION, a conclusion deducible from the input and the context together, but from neither input nor context alone. For example, on seeing my train arriving, I might look at my watch, access my knowledge of the train timetable, and derive the contextual implication that my train is late (which may itself achieve relevance by combining with further contextual assumptions to yield further implications). Other types of cognitive effect include the strengthening, revision or abandonment of available assumptions. For example, the sight of my train arriving late might confirm my impression that the service is deteriorating, or make me alter my plans to do some shopping on the way to work. According to relevance theory, an input is RELEVANT to an individual when, and only when, its processing yields such positive cognitive effects.

Relevance is not just an all-or-none matter but a matter of degree. There are potentially relevant inputs all around us, but we cannot attend to them all. What makes an input worth picking out from the mass of competing stimuli is not just that it is relevant, but that it is MORE relevant than any alternative input available to us at that time. Intuitively, other things being equal, the more worthwhile conclusions achieved by processing an input, the more relevant it will be. According to relevance theory, other things being equal, the greater the positive cognitive effects achieved by processing an input, the greater its relevance will be. Thus, the sight of my train arriving one minute late may make little worthwhile difference to my representation of the world, while the sight of it arriving half an hour late may lead to a radical reorganization of my day, and the relevance of the two inputs will vary accordingly.

What makes an input worth attending to is not just the cognitive effects it achieves. In different circumstances, the same stimulus may be more or less salient, the same contextual assumptions more or less \emph{accessible} [my emphasis], and the same cognitive effects easier or harder to derive. Intuitively, the greater the effort of perception, memory and inference required, the less rewarding the input will be to process, and hence the less deserving of attention. According to relevance theory, other things being equal, the greater the PROCESSING EFFORT required, the less relevant the input will be. Thus, RELEVANCE may be assessed in terms of cognitive effects and processing effort:

(1) \textbf{Relevance of an input to an individual}:

\begin{itemize}
\item[a.] Other things being equal, the greater the positive cognitive effects achieved by processing an input, the greater the relevance of the input to the individual at that time.
\item[b.] Other things being equal, the greater the processing effort expended, the lower the relevance of the input to the individual at that time. 
\end{itemize}\vspace{-\baselineskip}
\hbox{}\hfill{\hbox{\citep[608--609]{sw:rt}}}
\end{quote}


\noindent The key word in the foregoing is \emph{worthwhile} difference to the individual's representation of the world. Sperber and Wilson develop this idea through the ``maximization'' of positive cognitive effects relative to processing effort. Cognitive effects are taken to be \emph{contextual implications} or the strengthening, revision, or abandonment of assumptions. This last equation -- of cognitive effects with contextual implications and other kinds of inference -- is primarily where the Relevance-Theoretic idea falls short in a number of ways.

%(and the phrase in the second sentence of the quote: ``conclusions that \emph{matter} to him'')

\subsection{The first difficulty}

To start with, it seems that there is no practical way for the individual to assess which of many cognitive effects is greater. In most of their examples, the magnitude of these effects is taken to be the \emph{number} of contextual implications or inferences each input generates. This raises the obvious difficulty that if $p$ and $q$ are contextual implications of an input, are these to be taken as two propositions $\{p, q\}$ or as just the single conjunction $\{p \wedge q\}$ or as the triple $\{p,\ q,\ p \wedge q\}$, and so on? Is $p$ itself just one proposition or more than one?\footnote{For example, for some input an implication may be that the ball is round. Now, is that one implication or two or more? It could be said that the implications are that the ball has a shape and that this shape is round, and so on.} There is no unique way to individuate implications and each way is equally arbitrary. While the authors appeared to take this exclusively numerical route in \citet{sw:r}, they try to avoid it in their more recent \citet[609--610]{sw:rt} by expanding the scope of cognitive effects to other less quantitative aspects: ``In the first place, only some aspects of effect and effort (e.g.\ processing time, number of contextual implications) are likely to be be measurable in absolute numerical terms, while others (e.g.\ strength of implications, level of attention) are not.''

%They describe their more recent approach as follows:

%\begin{quote}
%
%Here is a brief and artificial illustration of how the relevance of alternative inputs might be compared. Mary, who dislikes most meat and is allergic to chicken, rings her host to find out what is on the menu. He could truly tell her any of three things:
%
%(2)	We are serving meat.\\
%(3)	We are serving chicken.\\
%(4)	Either we are serving chicken or $(7^2 - 3)$ is not 46.
%
%According to the characterization in (1), all three utterances would be relevant to Mary, but (3) would be more relevant than either (2) or (4). It would be more relevant than (2) for reasons of cognitive effect: (3) entails (2), and therefore yields all the conclusions derivable from (2), and more besides. It would be more relevant than (4) for reasons of processing effort: although (3) and (4) are logically equivalent, and therefore yield exactly the same cognitive effects,\footnote{My footnote: As discussed in \sectref{sec:value of information}, logical equivalence does \emph{not} guarantee the same cognitive effects. \label{foot:intensionality}} these effects are easier to derive from (3) than from (4), which requires an additional effort of parsing and inference (in order to work out that the second disjunct is false and the first is therefore true). More generally, when similar amounts of effort are required, the effect factor is decisive in determining degrees of relevance, and when similar amounts of effect are achievable, the effort factor is decisive.
%
%This characterization of relevance is comparative rather than quantitative: it makes clear comparisons possible in some cases, but not in all. While quantitative notions of relevance might be worth exploring from a formal point of view, the comparative notion provides a better starting point for constructing a psychologically plausible theory. In the first place, only some aspects of effect and effort (e.g.\ processing time, number of contextual implications) are likely to be be measurable in absolute numerical terms, while others (e.g.\ strength of implications, level of attention) are not. In the second place, even when absolute measures exist (for weight or distance, for example), we generally have access to more intuitive methods of assessment which are comparative rather than quantitative, and which are in some sense more basic. It therefore seems preferable to treat effort and effect (and relevance, which is a function of effort and effect) as \textbf{non-representational} dimensions of mental processes: they exist and play a role in cognition whether or not they are mentally represented; and when they are mentally represented, it is in the form of intuitive comparative judgments rather than absolute numerical ones. \citep[609--610]{sw:rt}
%
%\end{quote}

Unfortunately, their comparative approach is applicable only to the very simplest of comparisons involving entailment of one input by another. If two inputs were picked randomly from a large pool of inputs, the likelihood of their comparability would be zero, making the determination of optimal relevance a rarity.

%While I am sympathetic to this nonrepresentational stance and, more importantly in the present context, to their comparative approach, as should be clear from the various comparisons of probabilities discussed in \partref{part:III}, they don't give any indication that

%So, while they initially bring us closer to a characterization of relevance by elaborating it in terms of effect and effort, they then move away from getting a firm handle on the concept by expanding effect to number and strength of contextual implications and effort to processing time and level of attention among other things. Not only does this not solve the original problem of counting the number of implications, it also adds new problems of how to include ideas like the strength of implications and so on, making relevance a somewhat vague concept that is open to the charge I made at the outset that there is no practical way for the subject to actually assess  the relevance of multiple inputs \emph{even} in a comparative way. If one input has more implications and another has fewer but stronger implications, which of these should be taken as being more relevant? Questions of this sort cannot be left open in a theory of relevance.


\subsection{The second difficulty}

The problem of comparing the relevance of different inputs is in practice sidestepped by bringing in the idea of accessibility mentioned in the first quote. Based on the ``definition'' of relevance in (1) from the quote above, the following comprehension procedure is described in \citet[613]{sw:rt}:

\begin{quote}
\noindent Relevance-theoretic comprehension procedure:
\begin{itemize}
\item[a.] Follow a path of least effort in computing cognitive effects: Test interpretive hypotheses (disambiguations, reference resolutions, implicatures, etc.) in order of accessibility.
\item[b.] Stop when your expectations of relevance are satisfied (or abandoned).
\end{itemize}
\end{quote}

But it is not clear what is accessible and there is a grave danger of circularity because what is regarded as optimally relevant is usually what is most accessible \emph{and} vice versa. There is no explanation for why things always seem to turn out this way. Why isn't a less accessible interpretation ever optimally relevant? 

%Presumably, the specialized mental architecture the authors posit makes certain interpretations more accessible than others. But that amounts to begging the question because it then appears that optimal relevance = greatest cognitive effects modulo effort = most accessible interpretation = something posited! Sperber and Wilson don't actually \emph{say} this but many of their examples seem to support such a conclusion.

Besides, nothing of consequence is said about an interlocutor's expectations of relevance. How much relevance is enough? This question is never addressed though it plays a crucial role in stopping the process before a comparison of the relevance of different interpretations can occur.

%For instance, consider the following exchange from the paper cited above (615):
%
%\begin{itemize}
%
%\item[a.] Peter:	 Did John pay back the money he owed you?
%\item[b.] Mary:	No. He forgot to go to the bank.
%
%\end{itemize}
%
%\noindent The interpretation of Mary's response -- which is ambiguous and also requires identifying the referent of the pronoun -- is meant to be guided by the comprehension procedure above as follows:\footnote{I have reformatted the table in the original article for convenience.}
%
%\begin{quote}
%
%\begin{itemize}
%
%\item[(a) ] Mary has said to Peter, ``He$_x$ forgot to go to the BANK$_1$ / BANK$_2$.''
%	
%[He$_x$ = uninterpreted pronoun]
%
%[BANK$_1$ = financial institution]
%
%[BANK$_2$ = river bank]			
%
%(Embedding of the decoded (incomplete) logical form of Mary's utterance into a description of Mary's ostensive behavior.)
%
%
%\item[(b) ] Mary's utterance will be optimally relevant to Peter.
%
%(Expectation raised by recognition of Mary's ostensive behavior and acceptance of the presumption of relevance it conveys.)
%	
%\item[(c) ] Mary's utterance will achieve relevance by explaining why John has not repaid the money he owed her.
%
%(Expectation raised by (b), together with the fact that such an explanation would be most relevant to Peter at this point.)
%	
%\item[(d) ] Forgetting to go to the BANK$_1$ may make one unable to repay the money one owes.
%
%(First assumption to occur to Peter which, together with other appropriate premises, might satisfy expectation (c). Accepted as an implicit premise of Mary's utterance.)
%	
%\item[(e) ] John forgot to go to the BANK$_1$.			
%
%(First enrichment of the logical form of Mary's utterance to occur to Peter which might combine with (d) to lead to the satisfaction of (c). Accepted as an explicature of Mary's utterance.)
%	
%\item[(f) ] John was unable to repay Mary the money he owes because he forgot to go to the BANK$_1$.		
%
%(Inferred from (d) and (e), satisfying (c) and accepted as an implicit conclusion of Mary's utterance.) 
%	
%\item[(g) ] John may repay Mary the money he owes when he next goes to the BANK$_1$.		 
%
%(From (f) plus background knowledge. One of several possible weak implicatures of Mary's utterance which, together with (f), satisfy expectation (b).) (Wilson and Sperber~\citeyear[616]{sw:rt})
%
%\end{itemize}
%
%\end{quote}
%
%\noindent They explain some of the motivation behind the example as follows:
%
%\begin{quote}
%
%For ease of exposition, we have used an example where preceding discourse creates a specific expectation of relevance, so that the interpretation process is strongly driven by expectations of effect. In an indirect answer such as (ib), where there are two possible implicatures (positive or negative), considerations of effort, and in particular the accessibility of contextual assumptions, play a more important role. In a discourse-initial utterance such as (ii), or in a questionnaire situation, considerations of effort are likely to play a decisive role in choosing among possible interpretations:
%
%\begin{itemize}
%
%\item[(ia) ] Peter: Did John pay back the money he owed?
%\item[(ib) ] Mary: He forgot to go to the bank.
%\item[(ii) ] He forgot to go to the bank. (Wilson and Sperber~\citeyear[footnote~14, 629]{sw:rt})
%
%\end{itemize}
%
%\end{quote}

Their examples (e.g.\ \citealt{sw:rt}, \citealt{sw:pmm}) are fram\-ed in a way that encourages the reader to conflate their own \emph{intuitive} concept of relevance with the Relevance-Theoretic one. They are lulled into thinking that there is a definite procedure being followed when in fact a highly incomplete and vague series of steps is described with crucial gaps filled in with one's own intuitions. 

%Why is (d) above the first assumption to occur to Peter that might satisfy expectation (c)? What is the space of assumptions being searched? It is certainly the first assumption to occur to \emph{us} consciously but how do we know that Peter (and the theory) might not find some more recondite assumption that satisfies expectation (c)? River banks may be used to hide cash. In this particular example, it could be argued that since financial banks are linked with money whereas river banks are generally not, it is the \emph{only} assumption that can meet the expectation. Even so, how does Peter know this in advance before considering both options? And can we be sure that this kind of reasoning will work for every example? Essentially, this step is based on commonsense knowledge of the world, but such knowledge may not always yield such an unambiguous result.
%
%Moreover, how do we know that Peter's ``expectations of relevance'' have been satisfied as required by step (b) of the procedure? The reader's intuition operates implicitly: \emph{we} know that our \emph{intuitive} expectations have been met, so we assume Peter's have too. This allows Peter to sidestep the issue of the comparative relevance of BANK$_2$. As I said above, how can he know for sure in advance that there isn't an even better explanation available with the latter? This is again where the reader of the paper fills in the gap with his or her own commonsense intuition of the greater relevance of BANK$_1$ in the context as framed by Peter's question. This work is done by the reader and not by Relevance Theory. But if we bring in our intuitive sense of the greater relevance of BANK$_1$ to start with, then what work is the theory doing? 

The problem is that no comparison between the relevance of competing inputs is ever required because the comprehension procedure stops operating after the first and most accessible input's relevance invariably meets the addressee's expectations of relevance.

%Here is another example -- from Sperber and Wilson~\cite{sw:pmm}:
%
%\begin{quote}
%
%\begin{itemize}
%
%\item[(5) ] Peter: Can we trust John to do as we tell him and defend the interests of the Linguistics Department in the University Council?
%\item[    ] Mary: John is a soldier!
%
%\end{itemize}
%
%Peter's mentally represented concept of a soldier includes many attributes (e.g.\ patriotism, sense of duty, discipline) which are all activated to some extent by Mary's use of the word `soldier'. However, they are not all activated to the same degree. Certain attributes also receive some activation from the context (and in particular from Peter's immediately preceding allusions to trust, doing as one is told, and defending interests), and these become the most accessible ones. These differences in accessibility of the various attributes of `soldier' create corresponding differences in the accessibility of various possible implications of Mary's utterance, as shown in (6):
%
%\begin{itemize}
%
%\item[(6a) ] John is devoted to his duty
%\item[(6b) ] John willingly follows orders
%\item[(6c) ] John does not question authority
%\item[(6d) ] John identifies with the goals of his team
%\item[(6e) ] John is a patriot
%\item[(6f) ] John earns a soldier's pay
%\item[(6g) ] John is a member of the military
%
%\end{itemize}
%
%Following the relevance-theoretic comprehension procedure, Peter considers these implications in order of accessibility, arrives at an interpretation which satisfies his expectations of relevance at (6d), and stops there. He does not even consider further possible implications such as (6e)-(6g), let alone evaluate and reject them. In particular, he does not consider (6g), i.e.\ the literal interpretation of Mary's utterance (contrary to what is predicted by most pragmatic accounts, e.g.\ Grice~\citeyear{grice:sitwow}, p. 34).
% 
%Now consider dialogue (7):
%
%\begin{itemize}
%
%\item[(7) ] Peter: What does John do for a living?
%\item[] Mary: John is a soldier!
%
%\end{itemize}
%
%Again, Mary's use of the word `soldier' adds some degree of activation to all the attributes of Peter's mental concept of a soldier, but in this context, the degree of activation, and the order of accessibility of the corresponding implications, may be the reverse of what we found in (5): that is, (g) may now be the most accessible implication and (a) the least accessible one. Again following the relevance-theoretic comprehension procedure, Peter now accesses implications (g) and (f) and, with his expectations of relevance satisfied, stops there. Thus, by applying exactly the same comprehension procedure (i.e.\ following a path of least effort and stopping when his expectations of relevance are satisfied), Peter arrives in the one case at a metaphorical interpretation, and in the other at a literal one.\footnote{I have changed the numbering scheme slightly to conform to the numbers in the section as a whole.}
%
%\end{quote}

Basically, we are simply \emph{told} that certain interpretations are more accessible than others based on commonsense reasoning that \emph{we} employ, not the protagonists in their examples. By pointing out that the correct interpretation is the one that is the most accessible, no real work is done by the theory because there is no \emph{theory} of accessibility. What starts off as an account of a relevance-theoretic procedure for comprehension ends up as an accessibility-theoretic procedure but there is, in fact, no such procedure made available by the theory because there is no such theory to begin with. 

Unfortunately, even if an empirical theory of accessibility became available, more serious difficulties with following the Relevance-Theoretic comprehension procedure and evaluating relevance would still remain.

%In addition, in the case of example (5), how does Peter know that (6d) is relevant enough and how does he know that (6g) will not yield even greater relevance? Again, \emph{we} know it is relevant enough but that is not enough for a \emph{theory} of relevance. And how would the comparison between (6d) and (6g) be made in practice if one cannot count implications reliably?


%One of their examples involves an utterance of ``John is a soldier!'' in a certain context. Here, equal weight is given to the conventional (or ``literal'') attribute of ``soldier'' (i.e.\ being a member of the military) and the other attributes (i.e.\ patriotism, sense of duty, discipline). As Sperber and Wilson~\cite{sw:pmm} say, psycholinguistic research seems to indicate that our mental representations for words contain ``encyclopedic'' entries. If this is right, then they should all be treated as aspects of the conventional meaning of a word, rather than certain senses being called literal and others metaphorical. Or there should be some differentiation between conventional and metaphorical meanings so that they are made accessible in different ways. If either of these is not done, and these different types of meaning are just lumped together, then it will follow \emph{by definition} that the literal meaning is not always accessed if its accessibility is allowed to be lower than that of metaphorical meanings. Such a renaming of the problem obscures the issue of whether literal meanings are always or only sometimes accessed.

%This slough of problems may be ascribed to the general imprecision of terms like ``context'' and ``contextual implication,'' ``accessibility,'' ``greater cognitive effect,'' ``expectations of relevance,'' and ``representation of the world.''  

%A purely quantitative notion may not be psychologically plausible. But it won't do to have a qualitative notion that is so vague that it cannot do the work reliably in every case, or worse, that it can mislead one into thinking that it is in fact doing so.

%It is not that there is a need for a quantitative notion of relevance; the need is for a notion and for a procedure that are precisely spelled out.

%However, this is not just a difficulty that Relevance Theory faces. It is a problem for any theory of relevance. 


%It is unlikely we can have a general theory of accessibility in the near future because mapping the encyclopedic entries of close to a hundred thousand words is a huge task.


\subsection{The third difficulty}

The problem with the Relevance-Theoretic notion is graver still. Even its qualitative notions appear, on closer inspection, to be on the wrong track \emph{conceptually}.

As I said earlier, the key word in the first quote above was \emph{worthwhile} differences to an individual's representation of the world. In those paragraphs, relevance is related to positive cognitive effects and positive cognitive effects are related to \emph{worthwhile} differences in an individual's representation of the world. These differences are cashed out initially in epistemic terms: ``a true conclusion, for example.'' It is asserted that false conclusions are not worth having; they are not positive cognitive effects. Later, such effects are elaborated in terms of the number and strength of contextual implications, as we have seen.

Both of these dimensions of positive cognitive effects are misplaced. It is just not true that false conclusions do not have positive cognitive effects. Indeed, most if not all human beings entertain their share of false beliefs; it is precisely these false beliefs that enable them to get by. The truth may be too unbearable and so people invent all sorts of polite behaviors and euphemisms and other circumlocutory modes of being that border on or are falsehoods. Not only that, even entire systems of belief -- like various religions and ideologies -- may help alleviate human suffering to a considerable degree while being completely false. People typically choose hope over despair even when such a choice is not warranted.\footnote{As Woody Allen\ia{Allen, Woody} has said: ``More than any time in history mankind faces a crossroads. One path leads to despair and utter hopelessness, the other to total extinction. Let us pray that we have the wisdom to choose correctly.''} Often, believing something false will get a person into trouble but, depending on how marginal the false belief is to the practical realities of life, the person may in many cases be better off holding onto the false belief.\footnote{``She loves me, she loves me not \ldots\ .''} This psychological phenomenon is well known enough to not require further argument.

More importantly, though, it is the identification of cognitive effects with the number and strength of contextual implications, the heart of the Relevance-Theoretic view, that is the real problem. Implicit in this identification is the equation of effects with the \emph{amount} of information extractable from an input rather than with the \emph{value} of that information to the individual. That is, a \emph{worthwhile} difference or effect is one that is \emph{valuable}, not one that involves large amounts of information. Once one has this clear insight -- first expressed in \citet{parikh:gtai} and later in \citet{parikh:rs, parikh:le} -- then the \emph{conceptual} flaw in the Relevance-Theoretic notion becomes obvious. Simply put, more information is often but not always better for the individual than less information because that additional information may simply not \emph{matter}. Would you prefer a gift of the Encyclopedia Brittanica or the winning lottery number?


%Maxim of Quantity:
%
%(a) Make your contribution as informative as is required (for the current purposes of the exchange).
%
%(b) Do not make your contribution more informative than is required.

%Here is an example from the Wikipedia article on the Gricean maxims that illustrates both parts of this maxim.
%
%\begin{quote}
%
%Background: A man stops his vehicle in the middle of the road to briefly ask for directions.
%
%A: Where is the post office?
%
%B (Improper): There are two in town, but the closest one is brand new. Down the road, about 50 meters past the second left. Also, you shouldn't stop your car in the middle of the road anymore.
%
%B (Improper): Not far.
%
%B (Proper): Continue on, and make the second left up there. You'll see it.
%
%\end{quote}
%
%By most ordinary construals of the context, the first improper response by B will have many more contextual implications than the other two responses, potentially outweighing the greater effort involved. 

%Of course, it may always be possible to restrict the context in various ways to equalize the number of implications. But the point should be clear enough that the amount of information as measured by the number and strength of contextual implications is just not the right kind of metric for relevance.

It is to address such issues that Grice's (\citeyear{grice:lc, grice:sitwow}) maxim of quantity referred to the \emph{purposes} of the exchange. Suppose two bank robbers had robbed a bank some years ago and one of them had hidden the loot without the knowledge of the other at a river bank they frequented. Now, the first robber is dying and the second asks him where the loot is, and he says:

\ea It is at the bank.\z

\noindent It appears prima facie that the financial bank should be more accessible and should yield sufficient relevance to meet the second robber's expectations because, as Willie Sutton reminded us, ``that is where the money is.'' But it is equally arguable that the river bank is the right choice because both the robbers know the location and it is possible to stash loot in such places. Indeed, with some more reasoning it is clear that the dying robber meant the river bank because he would otherwise have  provided some more information about how to retrieve the money from the financial bank (e.g.\ what the account number was, etc.). On the other hand, the financial bank reading may have many more contextual implications (e.g.\ how long the loot had been accumulating interest and how much the balance might now be, whether it was in a savings account or certificate of deposit, and so on). But, intuitively, all these implications are \emph{irrelevant} to the matter at hand. 

%Indeed, if the second robber got carried away dreaming of how the amount might have grown, someone could tell him to focus on what was \emph{relevant}.

It might be said that the context could be construed differently but such maneuvers begin to beg the question. It is as if the context is being tailored to fit our pre-theoretic notion of relevance which is then shown to yield an outcome based on the Relevance-Theoretic notion resulting in an identity of the two.

%The same problem afflicts the earlier example of John's being a soldier. The ``wrong'' interpretation (e.g.\ (6g) for example (5) of the quote involving Peter and Mary) may well have many more contextual implications, however they are counted, than the ``correct'' interpretation (i.e.\ (6d) in the example) and this would yield the wrong result if the theory were scrupulously followed. But it isn't because, as I wrote above, the burden has been shifted to the concept of accessibility instead. And, besides, tailoring the context in order to get the right \emph{amount} of information isn't allowed. The fact is that all the implications that may flow from (6g), the incorrect interpretation, are simply \emph{irrelevant}. But this means that Relevance Theory has failed to capture the concept of relevance, even \emph{conceptually}.

It is the purposes of the exchange that are missing from the Relevance-Theo\-retic account. By confining worthwhile differences to the individual's representation of the world, Relevance Theory treats relevance in a purely \emph{epistemic} way. There is no room for purposes or goals or preferences in this view. A better intuition is that an input is relevant \emph{if it warrants a change in the individual's intended actions} because this idea is intimately bound up in a concrete way with the purposes and goals of the individual. Here, the term ``action'' includes both external behavior as well as internal, that is, ``mental'' actions like assessing a statement as true and forming a corresponding belief. As I point out in \citet{parikh:gtai, parikh:rs, parikh:le}, this is just the choice-theoretic notion of the \emph{value of information}. It readily admits of degrees of relevance and allows one to compare the relevance of two inputs. And it more accurately captures the idea of \emph{worthwhile} differences because worth and value are one and the same thing.

%\footnote{In his first paper, \emph{Relevance of Communicative Acts}, 2001, available at \url{http://staff.science.uva.nl/~vanrooij/Tark3.pdf}, van Rooij makes use of my notion of relevance. A quote from this paper: ``For an early use of this notion of relevance to account for particularized conversational implicatures as described in the introduction, see Parikh (1992).'' (page~3) And a second quote: ``For a related analysis of particularized conversational implicatures using game theory, see Parikh (1992).'' (page~6) It is interesting that when this decision-theoretic notion of the value of information is mentioned in the context of relevance, many researchers, including Sperber and Wilson, mention van Rooij's papers but not mine even though I introduced the idea almost a decade earlier and van Rooij merely followed my idea.}


\subsection{The fourth difficulty}

Lastly, the point made in the last paragraph above is worth making more broadly. Relevance Theory, influenced by Grice,\ia{Grice, Paul@Grice, Paul} Chomsky,\ia{Chomsky, Noam@Chomsky, Noam} and Fodor\ia{Fodor, Jerry@Fodor, Jerry} as it is, faces an insurmountable meta-difficulty with the very \emph{way} in which it approaches semantics -- what it calls pragmatics -- involving a narrow focus on the \emph{cognitive} aspects of agency and ignoring its practical aspects. This leads to prematurely conflating the related domains of semantics and psycholinguistics by directly importing psychological notions like \emph{accessibility} without any theoretical underpinning via speculations about specialized pragmatic modules and the like.\footnote{See \citet{sw:pmm}.} Semantics and psycholinguistics ought to be studied at somewhat different levels of abstraction, as the discussion in \sectref{sec:psycholinguistics} makes clear.

For example, memory certainly plays a key role in utterance comprehension but a semantic theory of communication would not typically attempt to incorporate its structure and would abstract from this constraint. In cases of disambiguation, it is true that we are not aware of all the senses of all the words we encounter in a sentence; our memory plays a key role in enabling the mechanism of \emph{priming}, for instance, which leads to the suppression of some alternative senses of words in a sentence. But the semantic task is to explain how disambiguation occurs. If such a theory is right, its main ideas will be transferable to the psychological level. A communicative theory of priming will abstract from the details of memory and use just its broad contours.\footnote{While Equilibrium Semantics as described in \citet{parikh:le} and in this book does not consider the phenomenon of priming in any detail, its framework is set up in a way that readily allows explicit disambiguation to occur when required and allows priming to eliminate alternative possibilities directly when required. It is then up to a psychological theory to specify when different mechanisms operate based on how memory functions. The precise details of how priming works will have to be described at the level of memory. This is analogous to the way in which, say, the levels of physics and chemistry are separated.} By seeing linguistics as a branch of cognitive psychology -- as Chomsky first proposed -- Relevance Theory falters methodologically because the proper domain of semantics is the domain of actions and interactions, not just the cognitive processes of the mind. And this requires a more abstract level of inquiry.

Such a level is, indeed, provided by decision and game theory, as I have argued in my publications. This abstract level does \emph{not} mean that the framework is not intended to be an empirical one -- it is. But it will still be a more general framework compatible with multiple psychological frameworks. It is interesting to observe this empiricism in its own domain. Earlier, rational choice theory was regarded by many as the right theory to employ when modeling action and interaction because it was sufficiently abstracted from the details of psychology and sociology and it appeared to give the right predictions. But, as \citet{allais:risk}, \citet{simon:bmrc, simon:rcse}, and \citet{kt:pt, kst:juu} among many others in a large literature have argued, the utility-theoretic axioms of rationality do not in fact always give the right results and this has led to a major revolution in the theory of choice. So far, no \emph{single} theoretical alternative has emerged as the clear successor although there are many competing approaches. The number of experiments has also mushroomed and the field of behavioral decision theory has begun to partially merge with the even more nascent and experimental field of neuroeconomics.\footnote{See \citet{glimcher:dub}.} If a clear theoretical winner does emerge eventually, it will also be relatively abstract and devoid of the details of neuroscience and psychology and sociology, and just like utility theory in its broad form, but will include new features abstracted from such details.

Incidentally, Grice\ia{Grice, Paul@Grice, Paul} himself is also always clear that he is pursuing \emph{philosophical} psychology, not psychology per se, even though he, too, remains confined to an epistemic view of agency\is{agency!epistemic} as I have argued in \sectref{sec:communication as rational activity}. This collapsing of the distinction between semantics and psychology creates insuperable methodological problems for Relevance Theory. Overall, Relevance Theory is an advance on traditional theories because it does try to countenance the pervasiveness of context and offer a method to derive content. It unfortunately errs by remaining trapped within the same \emph{epistemic} view of agency that also ensnared the tradition and by conflating semantics with psychology. And its key underlying concept of relevance is deeply flawed.
\is{Relevance Theory|)}\ia{Sperber, Dan@Sperber, Dan|)}\ia{Wilson, Deirdre@Wilson, Deirdre|)}

\section{The value of information} \label{sec:value of information}

I will not develop the idea of the value of information in any mathematical detail here; I have done this in the references cited at the start of this chapter, although, as I said, the extension of the concept to games as opposed to single-person decisions remained incomplete, something I rectify here. 

%It is also possible to move away from a quantitative approach to a more comparative approach by using algebraic stand-ins for numbers and corresponding inequalities as I have done in \emph{Language and Equilibrium}~\cite{parikh:le}, or by using fuzzy numbers or interval arithmetic, but I do not pursue these options here. Indeed, the standard approach itself is only partially quantitative because qualitative and comparative preferences are expressed in terms of utilities that represent a \emph{relative} comparative scale like the temperature scale rather than absolute numbers. So, while the approach looks fully numerical, it is in fact only quasi-numerical.

The basic idea is that an agent may be faced with some choices and, based on his goals and preferences as well as on the uncertain consequences that may ensue if he were to pursue each particular choice, he identifies an optimal course of action. Now, he may also be in a position to get some information that could reduce his uncertainty about the events following his actions. If he acquires this information by asking an informed person a question, he may find that, with this additional information, he may be able to pursue either the same action or a different action with better payoffs. In either case, he can assess the value of the information he has received by computing the difference in payoffs. If his optimal choice does not change, we may say that the information was not relevant or that it was not relevant enough. But if his optimal action changes, then we can not only say that the information was relevant but we can also say \emph{how} relevant it was depending on how much better off he is after receiving the information. In other words, information is relevant to an agent when it makes him better off. The virtue of decision theory is that it allows us to express this commonsense reasoning in a precise way. An excellent informal introduction can be found in the classic by \citet{raiffa:da}.

%The reader may want to tarry a bit to see how such an approach could be applied to the examples considered in \sectref{sec:Relevance Theory}.

This is very different from counting the number or strength of implications; the latter notion is logical and epistemic and is concerned solely with relations among propositions. Practical agency does not enter into this calculus at all and so it is no wonder that it fails to get a grip on how information might \emph{matter} to an agent. Relevance Theory falls short because it, like much of linguistics and the philosophy of language, has only a conception of a reasoning agent but none of practical agency as discussed in \sectref{sec:communication as rational activity}.

Now, to make the ideas above a little more concrete, recall that $\cal A$ and $\cal B$ are involved at the outset in a Setting Game. This Setting Game may in fact be just a decision problem for either agent which I am calling a game to make the terminology more convenient. In this event, the value of information as dictated by decision theory would apply directly. A possible implicature, for example, would be tested for its value in the context of this decision problem. If its value is sufficiently positive relative to some threshold, it might become a candidate for further processing by the Flow Constraint. In other words, this criterion of Relevance or the value of information would be one sub-Constraint of the Semantic Constraint as mentioned at the start of this chapter.

The difficulty arises when there is no apparent isolated decision problem that either agent faces in the Setting Game. The choice problem is really a two-person game (or even a multi-person game). How then do we evaluate the value of information? The idea, which eluded me in \citet{parikh:rs, parikh:le}, is straightforward. One can always extract a single-person decision problem from a game because a game is just an interactive decision problem between two or more persons. All that has to be done is to set up some choices of action for a player based on his available strategies in the game and uncertainty about what the other player will do based on her strategies and then use the first player's payoffs in the game to determine his payoffs in the decision problem so constructed. And this allows one to compute the value of information of a possible implicature in the usual manner of decision theory. So all that has to be done is to \emph{reduce} a game to a component decision problem to compute the relevance of possible illocutionary meanings.

However, there are two formidable difficulties with the value-theoretic concept as well.

\subsection{The first difficulty}
\is{intensionality|(}
%To go back to the first example with Mary and the dinner host: assume now that she is a strict vegetarian and an animal rights activist who finds the cruel treatment of animals in providing food to people abhorrent and who cannot even bear to hear the names of specific types of meat mentioned. The dinner host has the same choices as before:
%
%(9) We are serving meat. \\
%\indent (10) We are serving chicken.
%
%Suppose the host knows Mary's preferences. Now it is clear intuitively that the host would find (9) better to say than (10) even though both statements have either the same number of contextual implications or (10) has more than (9), depending on how the context is construed, \emph{and} both statements have the same value. That is, even though (10) may be at least as relevant as (9) if not more relevant by the standards of Relevance Theory, and even though both have the same relevance by the standards of choice theory, the host should opt for (9).
%
%This is one type of instance of the intensionality of relevance. It isn't just the information that is conveyed that has more or less relevance, it is also \emph{how} it is conveyed that matters. Mentioning chicken is upsetting to Mary so, even though her choice of action remains the same in either case, (10) is the less preferred option. In this example, maybe it is possible to say that both statements are equally relevant but the second is less desirable for ``extraneous'' reasons. That is, relevance really is a notion that depends only on the content and not the form of information.

%But here is a related kind of example. 

Suppose John has tentatively decided to opt for a surgical procedure but decides to get some more information from his doctor. His doctor may say either of the following:

\ea
\ea\label{ex:13:2a} 90\% of the patients survive the procedure.
\ex\label{ex:13:2b} 10\% of the patients do not survive the procedure.
\z
\z

It is well known that an agent's response to each of these two statements is often not the same.\footnote{See \citet{arrow:rppe}. The experiment Arrow discusses concerns clinicians' recommendations for surgery based on information about the survival probabilities. When told that the chance of survival was 90\%, some 80\% of surgeons who were asked recommended surgery. However, of those told that the same course of action was associated with a 10\% chance of death, only 50\% recommended surgery. Since the contents of both statements were equivalent, this implies that the preferences of some surgeons must have been intensional. I have changed the example a little. \label{foot:arrow}} With \REF{ex:13:2a} John is likely to want to go ahead with the procedure but with \REF{ex:13:2b} he might, like a number of other imperfectly rational agents, balk at the choice. This is so even though the information conveyed by both statements is equivalent. That is, if $\sigma$ represents the information conveyed by \REF{ex:13:2a} and $\sigma'$ the information conveyed by \REF{ex:13:2b}, we have $\sigma \Longleftrightarrow \sigma'$. This is clearly a situation where the form in which information is couched matters and so this phenomenon is called \emph{framing}, a result of the intensionality of choice.

In other words, relevance cannot simply be taken as the value of information without assuming that agents are perfectly rational in the sense of utility theory. If they are not, the concept has to be defined in terms of \emph{behavioral} choice theory rather than rational choice theory and this kind of behavioral relevance may not yield the right results. If behavioral relevance can be defined in a general way that accounts for intensionality, that is, if it encompasses the value of \emph{framed} information, then it may predict an agent's behavior correctly but give an incorrect view of relevance from a normative standpoint. This would create problems in explaining how communication occurs between imperfectly rational agents because each agent would have to consider the other agent's possible imperfections. As I argued in \sectref{sec:theory of conversation}, all utterances involve framing, whether by design or not, and this poses problems for a theory of relevance. 

One way to address the intensionality of choice is to use the prospect theory of \citet{kt:pt}, though I will not pursue it here. Another way to begin considering the value of framed information is to first notice that situation theory as opposed to competing frameworks like possible worlds theory that game theorists and philosophers tend to favor \emph{does} differentiate between equivalent infons $\sigma$ and $\sigma'$ because it is more fine-grained and can potentially avoid the problem of logical omniscience.\footnote{There is a large literature on this problem in epistemic logic. It occurs because with certain standard closure axioms, whenever an agent knows all of the formulas in a set and a formula follows logically from this set, the agent also knows this formula. See \citet{HendricksSymons2015}, \citet[Chapter~9]{fhmv:rk}, and \citet{stalnaker:ploI}, for example. If \citegen[Chapter~9]{dretske:kfi} argument is right, then his account does not have this problem.} So it is possible to represent an agent's knowing $\sigma$ without knowing $\sigma'$. Then an agent who is partially rational in a certain situation can be modeled as knowing some (framed) information such as $\sigma$ but not knowing other equivalent information such as $\sigma'$. With these assumptions, the value of this partial information can be computed. This procedure would, in principle, give the behavioral relevance of some information. But it is not entirely clear how such a computation might be carried out in practice and when such partial rationality would have to be assumed.

The intensionality of choice rests on framing or the intensionality of language but, interestingly, the intensionality of language itself rests on what might be called the intensionality of metaphysics, the phenomenon that I briefly alluded to in \sectref{sec:information} as that of the glass being half full or half empty. If the infon $\tau = \soa{\hbox{half full; glass}}$ and $\tau' = \soa{\hbox{half empty; glass}}$ then we have the further fact that $\tau \Longleftrightarrow \tau'$. These are two equivalent \emph{ways} of individuating the world. This kind of situational framing is basic to how we interact with the world and with each other and has profound effects on us even though, often, the different facts or infons we individuate are equivalent. Whether we see $\tau$ or $\tau'$ may result in the difference between hope and despair. Even surgeons, experts in their fields, behaved differently when the survival rates of surgery were differently but equivalently described.

Why is it that human beings respond differently to $\tau$ and $\tau'$ even though $\tau \Longleftrightarrow \tau'$? It is because they do not always realize that $\tau \Longleftrightarrow \tau'$ and, in my view, this lies at the heart of the problem of rationality. This is why an intensional ontological framework like situation theory that distinguishes between $\tau$ and $\tau'$ is required. Only the underlying \emph{reality} is the same, almost everything else -- situations, infons (and facts), properties and relations, and perhaps other objects -- are intensional. It is not just that human beings \emph{perceive} the same fact in different ways; the facts themselves -- of the glass being half full and half empty -- are distinct but equivalent. (I believe this situation is different from \citegen{wittgenstein:pi} duck-rabbit, which \emph{is}, as he says, a case of \emph{seeing} different aspects.) What happens is that people interact with reality to individuate it and produce the ontologies we then see as real. This is very similar to the Buddhist idea of dependent origination as well as Quine's ontological relativity mentioned in \sectref{sec:information}. Where both fall short is in missing the further idea that only a subclass of ontologies is \emph{optimal} (as argued in  \citet[Section~7.6]{parikh:le}, where I call the codeterminative process \emph{equilibrium metaphysics}) and so cannot be dismissed either as merely conventional as the Buddhist \ia{Buddhism} does or as merely relative as Quine\ia{Quine, W. V. O.@Quine, W. V. O.} does. This optimal subset has a certain \emph{immovability} and this makes the ``transcendental'' aims of Buddhism all the more difficult to realize and the radical indeterminacy claims of Quine suspect.

I am not suggesting some kind of unbridled relativism (with optimality). Reality does offer resistance and does constrain the class of possible ontologies. The glass can only be half full or half empty or both; it cannot be one-third full in the situation under consideration. Frameworks like possible worlds theory that do not make such distinctions are ineffectual because they cannot hope to explain how human beings act. It is also revealing that possible worlds theory describes this indistinguishability as the problem of \emph{logical omniscience} which indicates its logicist origins as it sees the problem as concerning just the \emph{epistemic} dimension of agency rather than being much more widespread.\is{agency!epistemic}\is{logicism}

In any case, as I just said, such finer ontologies lie at the heart of the problem of rationality. Even though a half-full glass is the ``same'' as a half-empty glass, we may not realize this and so our responses to them may be different. Sometimes, even when their equivalence is pointed out, we may persist in giving them different weights and thereby respond differently even when we know they are the same. (Also, a glass being one or the other is usually just a metaphor for more complex situations, and then the posited equivalence can be somewhat more difficult to see.)

%The reader may wish to refer back to footnote~\ref{foot:intensionality} in \sectref{sec:Relevance Theory}.

%\citegen{kripke:pab} Pierre, who believed both that \emph{Londres est joile} and that \emph{London is ugly} and later learned that Londres \emph{is} London, raises a different issue that can be explained by revising his beliefs to make the synecdoche explicit: parts of London are pretty and parts are not. There is no need to invoke any kind of intensionality here. It is the same with \emph{A Tale of Two Cities} where Dickens famously wrote, ``It was the best of times, it was the worst of times; \ldots\ .''
%
% BP semantic innocence Martinich; Batman and Bruce Wayne;


Of course, the intensionality of metaphysics is not confined to different ways of individuating \emph{equivalent} infons. The problem is far more pervasive. In \sectref{sec:expanded content selection game}, $\cal A$'s goal is to have $\cal B$ accept his statement that Bill Smith ran in the local election and he has two fundamentally different and also nonequivalent ways of realizing this goal: either to issue a bald statement or to give $\cal B$ a reason to accept the information by adding that he read the information in a newspaper. While these two ways may not be informationally equivalent, they do possess what we may call \emph{act-equivalence}, that is, they are meant to induce the same act by $\cal B$. This ties in how we individuate our choices in a situation very closely to human psychology, relationships, and perception, and the way we individuate our choices depends on how we can individuate the world. We will discuss an example of this act-equivalence in some detail later in \sectref{sec:a complete example}.

While the behavioral revolution in decision-making of the last three decades covers a wider ground than the intensionality of metaphysics and language and choice, I believe the latter lies at its core. Because of this, a certain level of \emph{creativity} or \emph{idiosyncrasy} or \emph{indeterminacy} may be ineradicable from human action and choice and, even though we may be able to explain certain actions partially after the fact, a truly predictive science that is also explanatory and not just data-driven, especially for large-scale phenomena like economy and society, may lie beyond our means. Intensionality also makes the problem of natural language generation much harder than the reciprocal problem of natural language understanding.

In \sectref{sec:information}, I mentioned the presence and general indeterminacy of context as a barrier to developing a complete science of communication; now the intensionality of metaphysics and language and choice and action can be seen as a further impediment.
\is{intensionality|)}

\subsection{The second difficulty}

A different sort of difficulty arises when we recognize that communication does not always involve statements. What if a speaker says something that changes an addressee's goals or preferences? For example, the doctor in the example above may recommend the procedure to the patient by saying something like:

\ea I recommend the procedure.\z

How, now, should the addressee evaluate the relevance of the exhortation? Or, a mother may command her child:

\ea Drink your milk.\z

In such cases, it could be argued that the addressee changes his current goals and preferences in light of his larger goals and preferences; the patient could reason that the doctor's recommendation would be good for him in the long term and the child could reason in the same sort of way. But this complicates the simpler theory of relevance as the value of information where a goal and decision problem are simply \emph{given}. For example, the patient may earlier have decided against the procedure on the basis of his initial preferences; now, he changes these preferences by taking account of his longer-term preferences and then decides to opt for the procedure. In the course of this evaluation, the immediate preferences have changed, so how exactly should the value of information be computed? This raises thorny questions of intertemporal choice among others. \citegen{portner:si} proposal that imperatives result in a change in the addressee's To-Do List does not help because he seems to lack the crucial idea of evaluating actions (i.e.\ the members of To-Do Lists) in terms of preferences. The value of information involves calculating the change in payoff and the problem is which payoffs to use, the prior ones or the posterior ones after the utterance.


%\subsubsection{The Third Difficulty}
%
%As I have said above, my extension of relevance from background decision problems to background games (i.e.\ Setting Games) has remained incomplete owing to certain technical difficulties. I cannot go into these here but these too muddy the waters because it is not always clear how to evaluate relevance when the background situation is a two or multi-person game rather than a single-person decision problem.

%~\\
Both these and perhaps other such issues pose hurdles to developing a satisfactory theory of relevance, even one that is able to incorporate ideas of the value and worth of information. This behooves us to look at what role relevance is required to play in a theory of communication.


\section{The role of relevance} \label{sec:role of relevance}

In Relevance Theory, relevance is meant, roughly speaking, to replace the four Gricean maxims \citep{grice:lc, grice:landc} with a single principle that can serve to explain human interpretive behavior. But the natural question that arises is whether \emph{any} maxim or principle is required to explain human behavior once one has the apparatus of rationality and rational action (and of possible deviations from rationality in the sense of utility theory). If this rhetorical observation is correct, then of the three choices -- four maxims (and their variants\footnote{See the neo-Gricean work of \citet{horn:qr} and \citet{levinson:pm}.}), one principle, or just rational action with no further layer -- it is clear that the last option is to be preferred. This is not only because of an obvious scientific principle of economy but also because it would bring the field of communication in line with the rest of the social sciences by recognizing a full-fledged notion of \emph{agency} that is conspicuously absent from linguistics and the philosophy of language. 

%To be sure, both relevance-theoretic relevance and the Gricean maxims would retain a role as ceteris paribus \emph{descriptive generalizations},\footnote{That is, other things being equal, human agents behave as if they were observing the generalizations.} but would nevertheless be otiose when it comes to the actual mechanisms that underlie communication.

As I have argued in several places and as we have seen in detail in \partref{part:III}, rationality by itself suffices to explain the transmission of locutionary meaning. It is only when \emph{illocutionary} meaning comes into play that we need to rely on relevance viewed as the value of information. The key reason for having to introduce some external Constraints like Relevance (and Distance) is that for illocutionary contents like implicature we do not have any counterpart of linguistically sourced entities like conventional meanings (e.g.\ the conventional meanings of \Expression{bank} enable us to identify its range of possible referential meanings) to generate the \emph{possibilities} which the Flow Constraint would disambiguate. Illocutionary meanings are more or less purely contextual and emerge through a kind of bootstrapping process which is facilitated by having access to the notion of the value of information. In other words, the sub-Constraints of Relevance and Distance together identify the illocutionary possibilities from the entire universe of contents; then the games of partial information disambiguate among these and determine one or more contents as those communicated. This is very different from the role of relevance in Relevance Theory where it determines all contents directly, both locutionary and illocutionary.

To the extent relevance is required, perhaps the main obstacle to using it is intensionality. As described in footnote~\ref{foot:arrow}, \emph{all} the surgeons did not deviate from rationality, only some did. This makes things somewhat unpredictable since different agents in identical situations behave quite differently. I will accept these limitations of the decision-theoretic notion of the value of information to serve as our criterion of relevance and use it in what follows.

Because illocutionary meanings are not constrained by conventional meanings as locutionary meanings are, they may bear no very direct relation to locutionary meanings as is familiar from many examples of implicature. Almost anything can be illocutionarily implied given the appropriate circumstances. This means the potential candidates from which the Flow Constraint enables a choice of one or a few optimal meanings can come from anywhere in the infon space $\cal I$ mentioned in \sectref{sec:information}. In the main example $\varphi = \varphi_1\varphi_2 = \Expression{Bill ran}$ of \partref{part:III}, an enrichment or completion such as \emph{Bill Smith ran for President of the US} might be very relevant by almost any criterion of relevance, whether the \emph{amount} of information notion of Relevance Theory or my less problematic \emph{value} of information notion. By both these criteria, this content would be far more relevant than the more plausible and intuitively acceptable content \emph{Bill Smith ran in the local election} because it would either generate many more contextual implications (i.e.\ positive cognitive effects) as required by Relevance Theory or would deliver a greater change of payoff as required by my account. This observation likely applies even to any intuitive criterion of relevance such as Grice's.\ia{Grice, Paul@Grice, Paul}

This ought to make it clear that relevance by itself is not sufficient for identifying the potential candidates or possibilities that enter into the Flow Constraint. Some other constraint is also required and this turns out to be the new idea of \emph{distance}. Incidentally, this idea has nothing to do with the measure of weighted psychological distance introduced in the discussion of \isi{vagueness} in \chapref{ch:vagueness}. As we will see, it may even be possible to manage just with this new idea and do without the idea of relevance altogether.

