\addchap{Praise for \textit{Communication and content}}

\noindent This book is the culmination of Prashant Parikh's long and deep work on fundamental questions of language and how they can be illuminated by game-theoretic analysis.\medskip\\
— Roger Myerson, 2007 Nobel Laureate in Economics, University of Chicago\vfill

\noindent Prashant Parikh has, over the years, accumulated a substantial and impressive body of work on the nature of language, deploying the resources of game theory. \textit{Communication and content} is a vastly ambitious culmination of this lifelong pursuit. It covers a tremendously wide range of themes and critically discusses an enormous range of writing on those themes from diverse intellectual traditions, as it systematically develops a game-theoretic account of content in the communicative contexts in which human linguistic capacities are employed, eschewing standard distinctions between semantics and pragmatics, and offering instead a highly integrated elaboration of the slogan ``meaning is use''. It is a work that is at once creative yet conscientious, bold yet rigorously technical, systematic yet sensitive to contingency and context.  It will abundantly reward close study.\medskip\\
— \parbox[t]{\textwidth-10.8624pt}{Akeel Bilgrami, Sidney Morgenbesser Professor of Philosophy, Columbia University}\vfill

\noindent Prashant Parikh has made fundamental contributions to the game-theoretic analysis of linguistic meaning. \textit{Communication and content} summarizes and extends this important work, offering a truly novel approach to the strategic foundations of meaning. This approach finds a way out of the prison of methodological solipsism and opens up the study of linguistic meaning to scientific study.\medskip\\
— Robin Clark, Linguistics, University of Pennsylvania\vfill\pagebreak

\noindent A pioneering attempt to work out things like literal meaning, modulation, enrichment, implicature, etc. in mathematical detail within a game-theoretic framework.\medskip\\
— François Recanati, Chair, Philosophy of Language and Mind, Collège de France\bigskip\\

\noindent\textit{Communication and content} is the crowning achievement of a long line of research pioneered by Prashant Parikh. In this groundbreaking work Parikh introduces a fresh perspective on natural language pragmatics, by making a creative tie with game theory. Clearly written, \textit{Communication and content} weaves together semantics, game theory, and situation theory to create a thought-pro\-vok\-ing picture of natural language pragmatics. Every modern AI researcher interested in the foundations of natural language pragmatics owes it to him- or herself to become familiar with this picture.\medskip\\
— Yoav Shoham, Computer Science Department, Stanford University\bigskip\\
