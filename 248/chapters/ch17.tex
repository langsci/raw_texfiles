\chapter{Modulation} \label{ch:modulation}

Modulation is possibly the most complex aspect of meaning. It straddles both locutionary and illocutionary meaning, \emph{unlike} both free enrichment and implicature, and involves a more complex Semantic Constraint consisting not only of the Conventional and Referential sub-Constraints of \partref{part:III} but also of the Relevance and Distance sub-Constraints of \partref{part:IV}. Moreover, it requires for the first time a back-and-forth interaction between the Semantic and Flow Constraints.

Modulation was perhaps first identified by \citet{cohen:pa, cohen:ci}, who cites \citet{ross:pa} as an influence.\footnote{\citet{barsalou:ahc, barsalou:igs} investigated the related idea of ad hoc categories in a different psychological context.} It was later picked up by several writers including \citet{recanati:lm, recanati:tcp} and \citet{wc:lp} as an essential component of literal meaning. Cohen's own example is the use of a phrase like \Expression{the stone lion} where the content of \Expression{lion} has to be modulated to be made compatible with its adjective \Expression{stone}. This fascinating aspect of meaning, possibly responsible for much in \emph{meaning change}\is{meaning!change in} as one dimension of language change, can only be tackled by applying a more complex combination of the locutionary and illocutionary Semantic Constraints and a Flow Constraint that interacts with them.

It is best to understand the phenomenon through \citegen{cohen:ci} own words. He contrasts ``insulationism'' and ``interactionism'' as two different ways in which the meaning of the utterance of a sentence depends on the meanings of its component words.

\begin{quote}

According to the insulationist account the meaning of any one word that occurs in a particular sentence is insulated against interference from the meaning of any other word in the same sentence. On this view the composition of a sentence resembles the construction of a wall from bricks of different shapes. The result depends on the properties of the parts and the pattern of their combination. But just as each brick has exactly the same shape in every wall or part of a wall to which it is moved, so too each standard sense of a word or phrase is exactly the same in every sentence or part of a sentence in which it occurs. We may sometimes need to look at neighboring words in order to discover the sense in which a word is functioning in the sentence in question, as we might infer the concavity of one brick from the convexity of its neighbor. But even then the meanings that we discover are not made what they are by one another, any more than the presence of a convex brick alters the shape of its neighbor. Rather, the words in the sentence have been given these meanings by diachronic facts of etymology. (page~223)
\[\vdots\]
Interactionism makes the contradictory assertion: in some sentences in\linebreak some languages the meaning of a word in a sentence may be determined by the word's verbal context in that sentence, though the extent and nature of this determination shows a wide range of variation. On this view the composition of a sentence is more like the construction of a wall from sand-bags of different kinds. Though the size, structure, texture and contents of a sand-bag restrict the range of shapes it can take on, the actual shape it adopts in a particular situation depends to a greater or lesser extent on the shapes adopted by other sand-bags in the wall, and the same sand-bag might take on a somewhat different shape in another wall or in a different position in the same wall. By exploiting local context in this way a language can be much more prolific of semantic variety than insulationism can give it credit for being. Moreover these sense-refinements or sense-modifications are generated by verbal interaction with a particular synchronic state of a language. Many of them, especially metaphors, are not common enough to be recorded in dictionaries or assigned dates and places of origin. (page~223)
\[\vdots\]
Once the difference between insulationism and interactionism has been recognized it becomes clear that we cannot construct a semantics for any natural language along the same lines as a semantics for a formal system of any currently familiar kind. Projects like Davidson's or Montague's cannot succeed. And one can see the temptation for philosophers of language to relapse into a later-Wittgensteinian emphasis on anomalism. (page~230)
\[\vdots\]
The interactionist must certainly grant that the borderline between the two types of meaning-determination is a fluid and flexible one. \ldots\ Moreover, just as it is possible to shift the borderline between the two kinds of meaning-determination so as to make the insulationist story more extensively applicable, so too one could shift the borderline in the other direction. \ldots\ In short, while homonyms like `pen' and `case' are definitely on the side of the borderline where insulationism holds sway, and indefinitely pliable verbs like `drop' and `make' are definitely on the interactionist side, there is a lot of \isi{polysemy} in natural language that can apparently be treated in either way because it is just not known whether all the word-meanings concerned have actually occurred in human utterances. In such ambivalent cases considerations of theoretical simplicity are the only factors that can determine the issue. (232--233)
\[\vdots\]
It has to be emphasized, of course, that the details of verbal interaction in natural language are as yet little understood and that very many problems remain as yet unresolved. For example, when one word dominates another, as ``stone'' dominates ``lion'' in (1), or ``geography'' dominates ``drop'' in (2), what ensures that the domination proceeds in one direction rather than the other?\footnote{(1) Four stone lions occupy the corners of Trafalgar Square. (2) Most students here drop geography in their final year (where ``drop'' means ``drop studying'').} Does the less ambiguous dominate over the more ambiguous, or topic over comment, or the relatively abstract over the relatively concrete, or the inanimate over the animate? But whatever be the correct solution of these problems, it seems highly unlikely that we shall obtain any guidance towards finding it from the ideas of Tarski, Montague, etc., about the semantics of artificial languages. Recognition of the difference between insulationist and interactionist conceptions forces us to treat the semantics of natural language as a largely autonomous discipline, rather than as a topic for Davidsonian or Montagueian theory. (page~234)

\end{quote}

This book, and my earlier work, has tried to tread precisely the fine line between the Scylla of ``formalism'' or, more precisely, ``logicism''\is{logicism} (Russell,\ia{Russell, Bertrand@Russell, Bertrand} Tarski,\ia{Tarski, Alfred} Montague,\ia{Montague, Richard} Davidson)\ia{Davidson, Donald} and the Charybdis of ``anomalism'' (the later Wittgenstein's\ia{Wittgenstein, Ludwig@Wittgenstein, Ludwig} lack of system as described in the quote from Dummett\ia{Dummett, Michael@Dummett, Michael} in \sectref{sec:classic example} as well as the similar outlooks of Austin,\ia{Austin, J. L.@Austin, J. L.} Strawson,\ia{Strawson, P. F.@Strawson, P. F.} and Searle).\ia{Searle, John R.@Searle, John R.} But the modulation of meanings shows us how formidable the task is. To date, there are no adequate theories of this phenomenon except for descriptive attempts by Recanati\ia{Recanati, François@Recanati, François} and the Relevance theorists\is{Relevance Theory} and a few others.

The kind of disambiguation \partref{part:III} looked at dealt with homonyms (e.g.\ words like \Expression{bank} which can actually be thought of as two or more distinct words with the same spelling). It does not show specifically how such meanings can be \emph{modulated} in the context of a sentence. Cohen also seems to focus exclusively on the verbal context and leaves out modulations that occur on account of the situational context.

There is also sometimes real indeterminacy\is{indeterminacy} about whether a particular indirect meaning is a modulation or free enrichment or implicature. Both of the latter can often be viewed simply as modulations as I mentioned with Grice's example of the \emph{possibly open} garage and as could also be said of the example of Smith weighing 150 lbs. \emph{on Earth}. It is nevertheless right to treat the three phenomena as distinct and allow for indeterminacies from time to time. In such events, one agent may treat the matter one way and the other agent may treat it the other way.

The classic -- and perhaps obvious -- first approach to such a problem would be to mimic Grice's approach to implicature as might be suggested, for example, by \citet{grice:lc} and \citet{searle:m}. This involves computing the whole locutionary meaning of the utterance first, finding it inadequate in one or another way, and then recomputing a related meaning such as an implicature or a modulated meaning to make up for the inadequacy. Unfortunately, the experimental evidence (e.g.\ \citealt{frisson:sulp}) makes this kind of computation highly implausible as the time delays implied by the model are incompatible with those found empirically.

The alternative is to undertake ``local'' computations as opposed to ``global'' ones but use roughly the same kind of reasoning. In the case of \Expression{the stone lion}, both the determiner \Expression{the} and the adjective \Expression{stone} play a role in constraining the possibilities for the content of \Expression{lion}, the first by guiding the addressee to some resource situation or anaphoric anchor and the second by eliminating the possibility of a real animate lion in most ordinary contexts as inadequate. Modulations are generally activated by other words in the sentence (e.g.\ \Expression{stone}) whose referential meanings seem incompatible with those of the word in question (e.g.\ \Expression{lion}).

Any theory that is developed must be compatible with the psycholinguistic evidence. But, at this stage, such experiments themselves are still in their infancy and so it is not clear what the final findings will be. So it is essential to build a framework that can itself be modulated in different directions and can offer different theories based on the data. I will offer one such theory in what follows. But the framework of Equilibrium Linguistics is flexible enough to house alternative accounts of the phenomena.

As I said in \sectref{sec:psycholinguistics}, \citet{plm:rp} and \citet{frisson:sulp} suggest that the conventional meanings of polysemous words are likely to be relatively abstract and impoverished underspecified cores relative to the full meanings they are given during processing. As Frisson says, the comprehension of the underspecified core sense of a polysemous word occurs instantaneously but the full meaning is realized only at the end of the sentence if at all. Such delays are likely to be present even when conventional meanings are modulated. This empirical result gives us a way into the problem. Essentially, it is the Semantic and Flow Constraints that will need to be modified.

To make things concrete, consider the following situation $u$. Batman and Robin are perched above a tall building in Gotham City in the dead of night and they have the following exchange:

\begin{quote}
Robin: People were busy today.\\
Batman: The city is asleep. ($\varphi = \varphi_1\varphi_2\varphi_3\varphi_4$)\footnote{\citet[34-36]{recanati:lm} discusses this example.}
\end{quote}

Here, the Setting Game is just one involving the making of commonplace observations. But all four words $\varphi_i$ in $\varphi$ cannot be given their conventional meanings because cities do not sleep. Intuitively, its  three possible meanings could be \emph{the residents are asleep} or \emph{the city is quiet} or \emph{the residents are quiet}, all of which are modulated contents. It is reasonably clear from $u$ that Batman could have meant any of these three meanings. How might such meanings be derived?

\section{The Semantic Constraint}

%Recall from \chapref{ch:vagueness} that we started with the locutionary Semantic Constraint (i.e.\ the Conventional and Referential sub-Constraints):
%
%\[ \hbox{word} \longrightarrow \hbox{conventional meaning(s)} \stackrel{u}\longrightarrow \hbox{referential meaning(s)} \]
%
%\noindent or:
%
%\[ \omega \longrightarrow C^{\omega} \stackrel{u}\longrightarrow P^\omega \]
%
%\noindent where the concept $C^{\omega}$ is converted into the corresponding property $P^\omega$ in $u$. This schema allows agents to compute possible referential meanings based on the conventional meanings in their heads and to convey abstract propositions that are not in their heads.
%
%In \sectref{sec:back to communication},\footnote{See especially  page~\pageref{page:vague communication}.} we encountered an utterance by $\cal A$ to $\cal B$ in $u$ of the sentence \Expression{Bill is very bald}. There the two maps above applied to each of the four words in the utterance. For \Expression{bald} this meant:
%
%
%\[ \Expression{bald} \longrightarrow C^{\Expression{bald}} \stackrel{u}\longrightarrow P^\Expression{bald} \]

%\noindent For the reader's convenience, I quote the relevant passage from that section:
%
%\begin{quote}
%
%Each concept and property can be understood as marked by the relevant agent $\cal A$ or $\cal B$ implying that there are actually \emph{two} such arrow diagrams for the word, one for each agent. The vague concept $C^{\Expression{bald}}$ can be more or less any of the many situated concepts that $\cal A$ (or $\cal B$) has used in the past or it can be some average of these. Further, in the current situation $u$, it gets transformed via a new $u$-relative concept into its corresponding vague property. That is, the concept $C^{\Expression{bald}}$ that is the conventional meaning \emph{shifts} to a related concept $C'^{\Expression{bald}}$ relative to $u$ and thence to the corresponding property. This shift is required to accommodate the different kinds of uses of \Expression{bald} that may occur -- for example, in the sentences \Expression{She won't date Bill -- he's bald} and \Expression{Bill isn't bald -- he needs a haircut.} Here, the penumbra of the same agent's concept shifts in accord with the use by accessing different exemplars or attributes or weights. The transformation of $C^{\Expression{bald}}$ to the shifted concept $C'^{\Expression{bald}}$ also depends on truth as discussed in \sectref{sec:meaning and truth}. Working out the details is likely to be a bit involved but the broad contours of penumbral shift, the change of truth value of the same sentence in different situations, do not seem to raise any special problems once the general context-sensitivity of language is accounted for. This can be displayed as an extended Semantic Constraint as follows:
%
%
%\[ \Expression{bald} \longrightarrow C^{\Expression{bald}} \stackrel{u}\longrightarrow C'^{\Expression{bald}} \stackrel{u}\longrightarrow P^\Expression{bald} \]
%
%
%The property that is the referential meaning can be taken to be either the \emph{intersubjectively derived} average property or the subjective property and, in the former case, it can be assumed that each agent has just a partial and ``vague'' understanding of the content of the utterance. Also, with an actual utterance of the sentence, there would be more than two diagrams as the word \Expression{bald} is \emph{ambiguous} besides being vague. For example, it can also mean \emph{plain} or \emph{blunt} as in \Expression{a bald statement}. So there will be two or more conventional meanings, each of which will be mapped into their respective referential meanings for each agent.
%
%\end{quote}

As indicated in \sectref{sec:psycholinguistics}, there are homonyms like \Expression{bank} with unrelated conventional meanings and polysemes like \Expression{eye} or \Expression{school} with related conventional meanings, the former taking more time to process than the latter. Indeed, one sense of \Expression{bank} is itself also polysemous because there can be many different kinds of financial institutions that are all called \Expression{bank}. These more specific senses can also be conveyed in an utterance. The different related senses of a polyseme can all be thought of as conventionalized modulations, that is, modulations that become conventionalized with use.

Recall the extended Semantic Constraint from \sectref{sec:back to communication},\footnote{See especially page~\pageref{page:vague communication}.}

%the extended Semantic Constraint above can be expressed abstractly as:

\[ \omega \longrightarrow C^{\omega} \stackrel{u}\longrightarrow C'^{\omega} \stackrel{u}\longrightarrow P^\omega \]

Any modification of this schema must preserve the possibility of penumbral shift that occurs with vague concepts. Remember that \isi{vagueness} is ubiquitous and it is very likely that it is a vague property (subjective or intersubjective) that undergoes modulation. 

Earlier, in \partref{part:III}, $C^{\omega}$ was assumed to be the full conventional meaning of $\omega$ even when it was polysemous. Now, based on \citegen{frisson:sulp} experimental findings, it can be said to be the common, underspecified core of all the related senses when such related senses exist. For example, in the sense of \Expression{bank} which is a financial institution, just this core, that is, just \emph{financial institution}, will be designated as one homonymic conventional meaning $C^{\Expression{bank}}_1$, where the subscript $1$ is used to distinguish this sense from the other homonymic sense of \textit{river bank}.\footnote{There are several other conventional meanings of \Expression{bank} but I will deal with just these two.} Refinements of this core such as a retail bank or a savings and loan institution or a credit union or a shadow bank or whatever will not be included in this core. This core is a concept in the agent's head. So are the refinements all concepts in the agent's head.

If \Expression{bank} is used in an utterance with this first sense rather than \emph{river bank} then the shifted concept $C'^{\Expression{bank}}_1$ may be such that $C'^{\Expression{bank}}_1 = C^{\Expression{bank}}_1$, that is, no penumbral shift may occur. There is likely to be a great deal of indeterminacy here as it may not be clear whether a shift occurs owing to, say, a different set of exemplars being taken into account for the vague concept \emph{financial institution} or whether its meaning is being modulated. But let us suspend the possibility of a penumbral shift for now and take $C'^{\Expression{bank}}_1 = C^{\Expression{bank}}_1$.

The last step in the extended Semantic Constraint is from $C'^{\Expression{bank}}_1$ to $P^{\Expression{bank}}_1$, the corresponding subjective or intersubjective property that is a \emph{possible} constituent of the proposition conveyed. There will be as many such properties as there are conventionalized refinements of the core concept available either in the agent's head or intersubjectively. For example, the concept \emph{financial institution} may get refined to the property \emph{shadow bank} in some particular situation. The ``null'' possibility of \emph{no} refinement, that is, just the property \emph{financial institution} can also occur because in many utterances this core is all that may be intended. Which one of these is selected will depend on the Flow Constraint. This game-theoretic process of handling \isi{polysemy} is identical to that of handling \isi{homonymy} except that there are usually many more polysemous senses.\largerpage[1.5]

Let us take it, then, that in cases of both \isi{homonymy} and \isi{polysemy}, there is a possibly underspecified core concept that is the conventional meaning, which may or may not undergo penumbral shift, that is then transformed by the referential map into a possibly refined subjective or intersubjective property. There will be as many such possibilities as there are conventional meanings and their refinements.{\interfootnotelinepenalty=10000\footnote{This actually depends on the kind of word being considered. There are different things that happen when, for example, the word is a quantifier like \Expression{the} or \Expression{every} as discussed in \citet[Chapter~6]{parikh:le}, but I am not concerned with such words here.}} These possibilities then enter the Flow Constraint where they get disambiguated by interdependent partial information games as discussed in \partref{part:III}. All I have done so far is to change the conventional meaning from its full meaning to an underspecified core in the case of \isi{polysemy}. The rest of the process is the same as before.\largerpage

To understand modulation, return to the example of Batman and Robin and focus, for the moment, on the word $\varphi_2 = \Expression{city}$. How might its sense get modulated to \emph{residents}? Assume that the latter content is \emph{not} a conventionalized refinement of the core concept even though it may well be. If it were, it would just be handled as described above. But, for the sake of the argument, assume it is not.

The locutionary Semantic Constraint applies as follows:

\[ \Expression{city} \longrightarrow C^{\Expression{city}} \stackrel{u}\longrightarrow C'^{\Expression{city}} \stackrel{u}\longrightarrow P^\Expression{city} \]

\noindent I now make three assumptions to keep things simple. First, assume that \Expression{city} has just one conventional meaning. Second, assume that it is not polysemous so that $C^{\Expression{city}}$ is fully specified rather than being just a common core. (The reason for tackling \isi{polysemy} above is that modulation can apply to either a fully specified or partially specified concept so the analysis must apply to both cases.) Third, assume there is no penumbral shift. Notice that \emph{city} is a vague concept because it just stands for a ``large'' town so penumbral shifts are certainly possible, but in the current utterance situation $u$, they can be ignored. In such a situation, then, the property $P^\Expression{city}$ will be just the usual vague property of being a city. This is one \emph{possible} content of the word that will enter the Flow Constraint.

%However, for simplicity, I will assume there is no polysemy here.

That is, the (locutionary) Flow Constraint would be activated with the unmodulated locutionary meaning $P^\Expression{city}$ as one interpretation. I will look at this presently. Before I do so, I want to look at what enables the modulation of $P^\Expression{city}$ to \emph{residents}.

In order to get to \emph{residents}, it becomes necessary to avail of the \emph{illocutionary} Semantic Constraint involving the sub-Constraints of Relevance and Distance. Recall that Robin had earlier uttered, ``People were busy today.'' This means the concept \emph{residents} is derivationally \emph{near} the baseline concept $C^{\Expression{city}}$ (actually $C'^{\Expression{city}}$, but the two are the same here) and the former is also \emph{relevant}. Why can this be asserted? It is because Batman's goal, which can be assumed to be cooperative, is to respond to Robin's observation about the people or residents of the city. So, the property \emph{residents}\footnote{We deal with the corresponding properties rather than concepts as there is no guarantee that the final destination infon will be in the agent's head from the outset. As he reasons, he is led to discover the result and this result may lead him outside his head, that is, outside his initial memory.} can be \emph{derived} from the corresponding property $P^{\Expression{city}}$ within the open ball $\hbox{Ball}^{\cal B}_u(P^{\Expression{city}}, \epsilon_{d,u}) = \{ x \in {\cal I} \mid d^{\cal B}_u(P^{\Expression{city}}, x) < \epsilon_{d,u} \}$. Such synecdochic derivations from whole to part are just routinely available reasoning strategies. For example, the agent may first observe that Robin has just talked about the residents being busy. Then he may suppose that if Batman were also talking about the residents the conversational goal of discussing more or less the same topic would be fulfilled. And, so, partly by abduction, he would conclude that Batman is in fact talking about the residents. The length of such a derivational chain might be just two or three steps. Moreover, \emph{residents} is also relevant because it has positive value for the interlocutors on account of adding  information to the dialogue. So \emph{residents} is both relevant and sufficiently close and could be admitted as a \emph{possibility}. 

But why would the local search for this content be triggered at all? Why would the agent not be satisfied with just $P^\Expression{city}$ which is available from the locutionary Semantic Constraint? This is where the back and forth between the Semantic and Flow Constraints comes in. In other words, the missing first step in the short derivation above is that $P^\Expression{city}$ is not adequate, as I will demonstrate below. That is why the agent has to search for something that is derivationally related to $P^\Expression{city}$ within the ball. This something will turn out to be \emph{residents} because of the rest of the derivation suggested above. 

Before we get to the back and forth, I want to make a couple of observations.

First, a city has many parts and has many aspects, not just its residents. So why is \emph{residents} near but not some other related aspect of a city? Why is only \emph{residents} within the open ball? This is because it is the only aspect of a city that would, in $u$, contribute to the conversational goal as made clear by the abductive step above. In some other setting, a speaker may wish to refer to a city's buildings as that meaning may contribute to the goal there. And so on. So only \emph{residents} is found as a possibility after a search is triggered by the inadequacy of $P^\Expression{city}$.

%Alternatively, it could be argued that while such other aspects may also be accessible within the ball, they may not be relevant. Remember that \emph{both} Relevance and Distance have to be satisfied for a meaning to count as a possibility. Either way,

Second, consider $\varphi_4 = \Expression{asleep}$. Again, a similar argument can be employed to show that the property \emph{quiet} or \emph{inactive} is close to the baseline property \emph{asleep} because these properties contrast with Robin's use of the word \Expression{busy}. These meanings also have relevance because Batman could be seen as drawing a contrast between the residents being quiet or inactive at night and busy or active during the day, and thereby contributing to his dialogue with Robin. Intuitively, the physical city's being quiet or inactive does not seem to fulfill any ``direct'' conversational goal but it could also be a possibility if Robin's earlier utterance implied the physical city was noisy and full of movement during the day. In such loose conversations especially, just like the sentence \Expression{Bill ran} in \partref{part:III}, relevance or the value of information is itself measured quite liberally.

I will therefore assume that the illocutionary Semantic Constraint can, if triggered, make meanings such as \emph{residents} and \emph{quiet} available as possibilities for the Flow Constraint. Thus, both the locutionary and illocutionary Semantic Constraint may play a role in generating the possibilities, the latter only when triggered. When this happens, the resulting meaning may no longer be clearly classifiable as locutionary or illocutionary, reinforcing the fuzziness of all these categories that attempt to partition meaning.


\section{The Flow Constraint}

As I said above, the locutionary Semantic Constraint is activated upon encountering $\varphi_2 = \Expression{city}$ and it results in $P^\Expression{city}$, which is just the vague subjective or intersubjective property of being a city. This assumes there was just one conventional meaning $C^{\Expression{city}}$ and there is no \isi{polysemy} and no penumbral shift so that $C^{\Expression{city}} = C'^{\Expression{city}}$. This generates the game in Figure~\ref{fig:game corresponding to P^city} where $\sigma_2 = P^\Expression{city}$.

\begin{figure}[h] 
\input{figures/pix5varphi2algebraic.tex} 
\caption{The locutionary partial information game for $P^\Expression{city}$}
\label{fig:game corresponding to P^city}
\end{figure}

Likewise, the locutionary Semantic Constraint operates on $\varphi_4 = \Expression{asleep}$ and results in $P^\Expression{asleep}$ and generates the game in Figure~~\ref{fig:game corresponding to P^asleep} where $\sigma_4 = P^\Expression{asleep}$.

\begin{figure}[h] 
\input{figures/pix5varphi4algebraic.tex} 
\caption{The locutionary partial information game for $P^\Expression{asleep}$}
\label{fig:game corresponding to P^asleep}
\end{figure}

Recall from \partref{part:III} that $\sigma_4$ is a conditioning variable in $\rho_2$ and $\sigma_2$ is a conditioning variable in $\rho_4$. (I am ignoring the other conditioning variables that arise from the games corresponding to the words $\varphi_1 = \Expression{the}$ and $\varphi_3 = \Expression{is}$ as they do not play any significant role in this example.\footnote{For more on such games relating to \Expression{the} and \Expression{is}, see \citet[Chapter~6]{parikh:le}.}) The prior probability $\rho_2$ is the probability that ${\cal A} = \hbox{Batman}$ is conveying $\sigma_2 = P^\Expression{city}$ \emph{given} that he is conveying $\sigma_4 = P^\Expression{asleep}$ (and the other semantic and syntactic contents). Similarly, the prior probability $\rho_4$ is the probability that $\cal A$ is conveying $\sigma_4 = P^\Expression{asleep}$ \emph{given} that he is conveying $\sigma_2 = P^\Expression{city}$ (and the other semantic and syntactic contents). But, obviously, $\cal A$ could not possibly be conveying either content given that he is conveying the other(s) because cities and sleep are incompatible and even a partially rational agent would not so contradict himself. 

This mutual incompatibility of $\sigma_2 = P^\Expression{city}$ and $\sigma_4 = P^\Expression{asleep}$ is what provides the desired trigger that activates the local search for a modulated meaning for either or both contents. Observe that I have used Grice's and Searle's idea that there should be a trigger for modulation based on some kind of inadequacy together with the idea that this inadequacy should be detected without computing the entire locutionary content as they believed. In the mechanism I have described, the mutual contradictoriness of $\sigma_2 = P^\Expression{city}$ and $\sigma_4 = P^\Expression{asleep}$ is realized locally, without computing the full locutionary proposition.

\emph{Some} trigger is required because, otherwise, agents would be trying to modulate every word they encountered and would take unconscionably long to process every utterance. This is where \citet{sw:dam} and \citet{wilson:pd} err because in their system, vaguely specified as it is, there is no trigger and what they call ``mutual adjustments'' occur ubiquitously even when an unmodulated meaning is perfectly acceptable. The term ``mutual adjustment'' is fine as an informal description but a \emph{theory} needs to spell out what this back and forth consists in. That is precisely what Equilibrium Linguistics offers. In addition, this agrees with \citegen{frisson:sulp} empirical assertion that the core conventional meaning is instantaneously accessed but the full, refined -- or modulated -- meaning is computed at the end of the sentence if at all.

Because the interpretive process as it occurs within the speaker's Generation Game and the addressee's Interpretation Game runs up against this block of contradictoriness, it backtracks to the Semantic Constraint from the Flow Constraint. This time the illocutionary Semantic Constraint is activated and the Distance and Relevance criterion is put to use. Now that $P^\Expression{city}$ and $P^\Expression{asleep}$ have been found to be mutually inadequate, two local searches ensue based on derivational proximity and relevance relative to the goals of the agents as inferred from the Setting Game. The result of the two searches is the \emph{discovery} of the property \emph{residents} starting from $P^\Expression{city}$ and \emph{quiet} or \emph{inactive} starting from $P^\Expression{asleep}$. Thus, these two properties become new possibilities as the computation returns to the Flow Constraint from the Semantic Constraint.

In the new games that emerge, $P^\Expression{city}$ and $P^\Expression{asleep}$ are still viable contents because they are only mutually contradictory and so cannot \emph{both} be present in the equilibrium content. Either of them singly may well persist. The next round of games is shown in Figures~\ref{fig:game corresponding to the modulation of P^city} and \ref{fig:game corresponding to the modulation of P^asleep} with $\sigma'_2 = \hbox{\emph{residents}}$ and $\sigma'_4 = \hbox{\emph{quiet}}$ or $\sigma'_4 = \hbox{\emph{inactive}}$. (Figure~\ref{fig:game corresponding to the modulation of P^city} and Figure~\ref{fig:semantic lexical game g2 - new example} look the same but stand for completely different things.)

%\footnote{Incidentally, the first of these games should not be confused with the game in Figure~\ref{fig:semantic lexical game g2 - new example} as the symbols $\varphi_2$, $\sigma_2$, and $\sigma'_2$ stand for completely different things.}

\begin{figure}[h] 
\input{figures/pix4varphi2algebraic.tex} 
\caption{The partial information game for the modulation of $P^\Expression{city}$}
\label{fig:game corresponding to the modulation of P^city}
\end{figure}

\begin{figure}[h] 
\input{figures/pix4varphi4algebraic.tex} 
\caption{The partial information game for the modulation of $P^\Expression{asleep}$}
\label{fig:game corresponding to the modulation of P^asleep}
\end{figure}

Interestingly, these games are neither wholly locutionary or wholly illocutionary because one content in each is locutionary and the other is illocutionary. By now the reader is likely an expert in solving such games and it should be easy to see that the joint solution can be either $(\sigma'_2, \sigma_4) = (\hbox{\emph{residents}}, P^\Expression{asleep})$ or $(\sigma_2, \sigma'_4) = (P^\Expression{city}, \hbox{\emph{quiet}})$ or $(\sigma'_2, \sigma'_4) = (\hbox{\emph{residents}}, \hbox{\emph{quiet}})$. It cannot be $(\sigma_2, \sigma_4) = (P^\Expression{city}, P^\Expression{asleep})$ as the prior probabilities will not allow it.\footnote{See \sectref{sec:solving locutionary global games} to see how such games are solved. I have dropped the locutionary contents $\sigma_1$ and $\sigma_3$ of \Expression{the} and \Expression{is} for convenience. Also, I have dropped the meaning \emph{inactive}, although it should be evident that such indeterminacies among possible modulations will be rife.}

So we started with the locutionary Semantic Constraint and then went to the locutionary Flow Constraint. The latter involved a mutually contradictory pair of contents $P^\Expression{city}$ and $P^\Expression{asleep}$ and so we backtracked to the illocutionary Semantic Constraint, found possible modulations of these two meanings through local searches of the infon space, and then returned to a mixed locutionary and illocutionary Flow Constraint where the games could be solved with mutually compatible solutions. This is the back and forth or mutual adjustment between the Semantic and Flow Constraints mentioned earlier. All three possible meanings that I referred to as meanings Batman could have conveyed emerged naturally as solutions. If desired, the equilibria can be augmented with probabilities to show that each of the solutions has some range of probability based on $u$ and on how close and relevant the modulations are. The equilibrium content of the whole utterance $\varphi = \Expression{The city is asleep}$ is then partly locutionary and partly illocutionary because the words \Expression{the} and \Expression{is} have locutionary equilibria and one or both of \Expression{city} and \Expression{asleep} have illocutionary contents.

Just to make this process crystal clear, I set it out more graphically:

\begin{enumerate}\setlength{\itemsep}{0pt}
\item[] \underline{Utterance Situation} 
%& \hbox{Setting Game} \\
\item[\functionarrow] Individual words of sentence become available 
\item[\functionarrow] Locutionary Semantic Constraint activated 
\item[\functionarrow] Possible referential contents become available 
\item[\functionarrow] Locutionary Flow Constraint activated 
\item[\functionarrow] Mutual incompatibility among locutionary contents detected 
\item[\functionarrow] Need for modulation triggered 
\item[\functionarrow] Illocutionary Semantic Constraint activated 
\item[\functionarrow] Possible modulated contents found and made available 
\item[\functionarrow] Mixed Flow Constraint with locutionary and illocutionary contents activated 
\item[\functionarrow] Mixed equilibrium contents
\end{enumerate}

What is the range of possibilities for modulations of locutionary meanings? Consider the example of metaphor from \citet[27]{sw:dam} where a woman says to an uncouth suitor, ``Keep your paws off me.'' The authors claim that ``the ad hoc concepts constructed to carry these implications will then at least overlap with the concepts encoded by the utterance (otherwise we would be dealing with purely associationist rather than inferential relations). Since the concepts PAW and HAND have disjoint extensions, we claim that `paw' could not be used to convey the meaning HAND.'' But this is patently false and shows that their conception of inferential relation is too narrow. Indeed, as I discussed in the case of \Expression{He is a fine friend} in \chapref{ch:distance}, human reasoning is sophisticated enough to \emph{revise} and \emph{reject} one or more of its premises. In their example, it is quite clear that the utterance has the rough meaning \emph{keep your clumsy, grasping hands off me} and the locutionary starting point \emph{paw} has been discarded. It is true, as Sperber and Wilson claim, that the exact meaning conveyed may not be paraphrasable but this is because the \emph{number} of properties constituting the modulated meaning is indefinite and because these properties are also vague, making it hard to reproduce the same effects with a paraphrase.

Returning to Cohen's\ia{Cohen, L. Jonathan@Cohen, L. Jonathan} example of \Expression{the stone lion} and his concern about how which word dominates which is to be determined, consider the examples of \Expression{the large mouse} and \Expression{the large elephant}. In the latter two phrases, it is clear that the adjective is modulated whereas in the former the noun is modulated. So one can say that it is unlikely that any syntactic determination of what is modulated is at play. In \Expression{the stone lion}, in most situations, no modulation of \emph{stone} will be found in the set of possible modulations $\{ x \in \hbox{Ball}_u(\hbox{\emph{stone}}, \epsilon_{d,u}) \mid  \hbox{Relevance}_u(x) > \epsilon_{R,u} \}$. That is, this set will be empty. But \emph{lion} will get modulated successfully to something like \emph{representation of a lion}. The opposite will happen with \Expression{the large mouse} in most contexts and \emph{large} will be modulated, not \emph{mouse}. But, given the highly situated nature of language, one should always expect that some rare context will invert these possible modulations.

\section{Recapitulation}

I have now shown in detail how modulation works. When it occurs, unlike free enrichment and implicature, the partial information games in the Flow Constraint are part of the locutionary global game $LG_u(\varphi)$ even though they contain illocutionary modulated contents uncovered by the illocutionary Semantic Constraint. As such, they interact with the games corresponding to all the other semantic, syntactic, and phonetic possible contents thrown up by the sentence uttered as described in \partref{part:III}. The only reason modulation was analyzed in this Part is that we needed the illocutionary Semantic Constraint.

%with its sub-Constraints of Relevance and Distance.

%Free enrichment and implicature are different from modulation because they are wholly illocutionary and the partial information games corresponding to them are part of the illocutionary global game $IG_U(\varphi)$. Together, they form the global game $G_u(\varphi) = LG_u(\varphi) \cup IG_u(\varphi)$.
%
%%, which, as I remarked at the start of this Part in \chapref{ch:relevance}, is what much of the discussion in this Part is about. 
%
%In free enrichment and implicature, too, the illocutionary partial information games contain either the whole locutionary content or one or more of its proper parts as one possibility but such games are not seen as mixed locutionary and illocutionary as the games involving modulation were. This is because in free enrichment and implicature, the locutionary content \emph{reappears} either as a ``null'' completion or a ``null'' implicature indicating that there is no enrichment and no implicature. These null completions and null implicatures -- that is, the locutionary contents -- will often enter probabilistically into the full meaning of the utterance as we saw in \chapref{ch:free enrichment}. This illocutionary probabilistic presence is separate from its locutionary presence as can be seen in a content such as
%
%\[\{\ \{(\tau_0, 1)\},\ \{(\tau_0, \pi_0), (\tau_1, \pi_1), (\tau_2, \pi_2)\},\ \{(\tau_0, \pi'_0), (\tau_6, \pi_6)\}\ \}\]
%
%\noindent where $\tau_0$ has both a locutionary and two illocutionary occurrences in the content. In modulation, on the other hand, at least one sense (e.g. $P^\Expression{city}$ or $P^\Expression{asleep}$) will be rejected in favor of a modulated meaning. Such a modulated meaning may or may not have an overlap with the original locutionary meaning as indicated above with the example of \emph{paw} and \emph{hand}.

Earlier, I discussed \isi{polysemy} and its associated conventional meaning which is an underspecified core that gets shifted and refined if required. In \Expression{The city is asleep}, this issue did not come up explicitly as I assumed no \isi{polysemy} was involved. But, in practice, all three kinds of lexical meaning -- \isi{homonymy}, \isi{polysemy}, and \isi{vagueness} -- can occur simultaneously, as evinced by a word like \Expression{bank}, which has two homonymic meanings, of which one is polysemous, and of which both could also be vague. So, in attempting a full account of the mechanism of modulation, it is necessary to ensure that none of these possibilities are neglected. When there is \isi{polysemy} and \isi{vagueness}, the underspecified core $C^{\omega}$ gets shifted to $C'^{\omega}$ and it is the latter concept from which a corresponding subjective or intersubjective property $P^{\omega}$ is formed. This property then provides the starting point for a local search for a modulated meaning if such a need arises.

Additionally, as \citet{frisson:sulp} says, the comprehension of the underspecified core sense of a polysemous word occurs instantaneously but the full meaning is realized only at the end of the sentence if at all. Modulations behave similarly to polysemous refinements in this respect except that they are not conventionalized but arrived at through inferential search. Indeed, \isi{polysemy} can be viewed as conventionalized modulation as I remarked earlier. If it turns out that Frisson's result cannot be so extended to modulation, then a revision of the theory I have presented might be required. But Equilibrium Linguistics is a framework which allows multiple theories to be developed within it and I have developed just one option. 

%It is a tribute to the power of its concepts that the very same foundational ideas of partial rationality and ontology (and grammar) can generate theories of all aspects of meaning.

I have also shown how all three classical aspects of lexical meaning -- \isi{homonymy}, \isi{polysemy}, and \isi{vagueness} -- would be handled by Equilibrium Linguistics.\footnote{See \citet[Chapter~5]{murphy:lm} for a discussion of these three aspects of lexical meaning. Also, look back to Footnote~\ref{foot:conventional meanings} in \sectref{sec:micro-semantics} to see how \Expression{Can you pass the salt?} would be handled as an instance of \isi{polysemy} or conventionalized modulation.} The first and third of these were analyzed in \partref{part:III} and \isi{polysemy} required just a small modification of the conventional meaning. Modulation, what might be called the fourth aspect though it applies not just to words but also to phrases, builds on the first three.

Figures of speech like metaphor, metonymy, synecdoche, hyperbole, and others can all be seen as instances of modulation. For example, an utterance of \Expression{The pen is mightier than the sword} would result in a modulation of both nouns. In this case, the whole sentence would have to be evaluated locutionarily as there is no direct incompatibility between \emph{pen} and \emph{sword}. As discussed in \sectref{sec:meaning and truth}, the truth of the utterance may also play a role in this determination. The meaning of this particular aphorism is now obviously conventionalized and so agents do not actually need to infer its meaning any more. Other tropes might result from implicature as happens with some examples of irony. So by giving an account of all these types of indirect meaning, I have also provided accounts of such phenomena.

More complex examples such as \Expression{The city which is polluted is unhappy} can also be tackled. Here, the word \Expression{polluted} is about the city itself and the word \Expression{unhappy} possibly makes a reference to its residents. I leave it to the reader to apply the theory I have described to this example.

As always, we have to contend with indeterminacy. I have already alluded to the indeterminate modulation of $P^{\Expression{asleep}}$ to either \emph{quiet} or \emph{inactive} or both as well as to other possible properties that may not be easy to make verbally explicit. All such modulations will lie within the set of possible modulations $\{ x \in \hbox{Ball}_u(P^{\Expression{asleep}}, \epsilon_{d,u}) \mid  \hbox{Relevance}_u(x) > \epsilon_{R,u} \}$ though an agent may not always be able to articulate them. The other indeterminacy, also discussed earlier, is between modulation on the one hand and free enrichment and implicature on the other. As the example of the garage being open shows, one agent may handle it as a modulation, another as an implicature.

A psycholinguistic issue is whether all these indirect meanings are stored in the agent's memory or lie outside and are discovered afresh. I have tried to emphasize that the process of local search will sometimes lead an agent outside his head, that is to say, it will lead him to entertain an entirely new idea that he has never contemplated before. This is also presumably how creativity occurs and an understanding of modulation-like processes may lead to deep insights into creativity.

A computational issue is how the complex and flexible human reasoning involved in search can be modeled. The usual tack is to model some subset of the infon space and some subset of possible reasoning strategies and thereby succeed with a subset of indirect meanings. It seems unlikely that the full range of possibilities can be mapped although, with machine learning techniques, one may be able to annotate training data sufficiently to allow such algorithms to work fairly well. In any case, I have at least shown how, in principle, the realm of illocutionary or indirect meaning can be brought within the ambit of science via a model that is philosophically sound, mathematically solid, computationally tractable, and empirically adequate.

Having made these observations, I now point out perhaps the most important one. Does the account of the \emph{fixed point principle} which generalizes Fregean\ia{Frege, Gottlob@Frege, Gottlob} compositionality and the context principle as discussed in \sectref{sec:compositionality/context} need to be altered in any way to accommodate modulation? As modulation involves just the locutionary global game $LG_u(\varphi)$, and as it therefore involves interdependence among the various local partial information games and among all the possible semantic, syntactic, and phonetic contents of the sentence uttered, no change needs to be made to it. All the results of \sectref{sec:solving locutionary global games} continue to apply including Theorem~\ref{thm:simple equation} and Equation~\ref{eq:simple} from \sectref{sec:maintheorem} in particular.

This completes my discussion of modulation.


