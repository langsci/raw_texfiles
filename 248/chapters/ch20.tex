\chapter{Classifying meaning} \label{ch:classifying meaning}

I have now laid out more or less the entire field of meaning. Picturing this takes a few diagrams. The first basic division is of utterance meaning into intended meaning and extracted but unintended meaning. Intentions may be explicit or implicit. This is shown in Figure~\ref{fig:utterance meaning}.


\begin{figure}[h]\small
\begin{forest} for tree={align=center,base=top}
[utterance meaning
    [intended meaning] [extracted but unintended meaning]
]
\end{forest}
\caption{Classification of utterance meaning}
\label{fig:utterance meaning}
\end{figure}


I could have added a third branch showing syntactic and phonetic contents. Dividing the left branch further, we get the tree in Figure~\ref{fig:intended meaning}.

\begin{figure}[h]\small
\begin{forest} for tree={align=center,base=top}
[intended meaning
    [communicated\\meaning
        [locutionary\\meaning
                [lexical\slash structural\\disambiguation] [saturation] [modulation]
        ] [illocutionary\\meaning
                [free\\enrichment] [implicature] [force]
        ]
    ] [intended weaker flows\\(e.g.\ suggestion)]
]
\end{forest}
\caption{Classification of intended meaning}
\label{fig:intended meaning}
\end{figure} 


I have placed modulation under locutionary meaning even though it would in fact straddle locutionary and illocutionary meaning. Implicature includes higher-order implicatures as well. Many if not all figures of speech fall under modulation or implicature.

The category of intended weaker flows represents transfers of information that do not require full common knowledge. These were studied in some detail in my previous two books and mentioned in \sectref{sec:interpretation game}. There is an entire infinite lattice of flows of information with communicative flows involving common knowledge being the strongest. When something is suggested, for example, we get a flow that is weaker than communication. Much art and literature involve suggestion and similar weaker flows.

I have touched upon Grice's distinction between literal meaning (or \emph{what is said}) and implicature now and then and this is just a different way to carve up direct and indirect contents. Literal meaning includes locutionary meaning and  enrichments. The direct/indirect distinction has more to do with \emph{how} these contents are derived and Grice's distinction has more to do with \emph{what} contents we lump together psychologically. There has been a great deal of discussion about the latter division, especially by \citet{recanati:lm}, and he tries to argue that literal meanings are in some sense \emph{available} to our conscious perception in a way that locutionary or direct meanings are not because the latter have to be teased apart from modulations and enrichments after the fact. So an alternative way to classify intended meaning is shown in Figure~\ref{fig:intended meaning 2}.


\begin{figure}[h]\small
\begin{forest} for tree={align=center,base=top}
[intended\\meaning
    [communicated\\meaning
        [literal\\meaning
            [lexical/structural\\disambiguation] [saturation] [modulation] [free\\enrichment] 
        ]
        [implicature] [force]
    ]
    [intended weaker flows\\(e.g.\ suggestion)]
]
\end{forest}
\caption{Alternative classification of intended meaning}
\label{fig:intended meaning 2}
\end{figure} 


I have emphasized throughout that many of these categories are not separated by strict black-and-white divisions but rather that it is often indeterminate where a particular content is placed by an agent. Such a classification is useful nonetheless because it tells one what mechanisms might have produced the content or whether it was consciously available to the agent or not.

The category of extracted but unintended meaning can be divided as shown in Figure~\ref{fig:extracted meaning}.


\begin{figure}[h]\small
\begin{forest} for tree={align=center,base=top}
[extracted but unintended meaning
    [significance] [association and\\extended meaning] [inverse information] [latent meaning]
]
\end{forest}
\caption{Classification of extracted but unintended meaning}
\label{fig:extracted meaning}
\end{figure}


Recall that some parts of significance, especially logical significance, are intended so again these classes are not watertight. The same is true of associations, extended meanings, and inverse information. Latent meanings, by definition, cannot be intended. I have tried to show in the previous chapter how the two basic aspects of utterance meaning, intended and unintended meaning, fit together into a natural whole. If we wish, we can draw a final diagram for discourse meaning shown in Figure~\ref{fig:discourse meaning}. The category of inter-utterance meanings that transcend individual utterance meanings was discussed in the previous chapter.

\begin{figure}[h]\small
\begin{forest} for tree={align=center,base=top}
[discourse meaning [utterance meanings] [inter-utterance meanings]]
\end{forest}
\caption{Classification of discourse meaning}
\label{fig:discourse meaning}
\end{figure}

As a test case, consider \citegen{stalnaker:smctc} discussion of the following remark of US Treasury Secretary John Snow in May 2003: ``When the dollar is at a lower level it helps exports, and I think exports are getting stronger as a result.'' The announcement caused the dollar to drop precipitously in value even though only a commonplace economic insight had been stated. What meaning issuing from this utterance created this effect? There might be more than one plausible way to analyze this example but it seems to me that it belongs to possibly unintended inverse information of the same kind as the inverse information about the mother's not wanting to fuss.

At the very start of the book, I said communication involves the relaying of content and has an \emph{effect} and that while the former is cognitive, the latter can involve the whole person. I have so far not said much about this effect although it is related to the emotive meaning of an utterance discussed earlier. Effects accompany most utterances though they are not part of their meaning. Happy or sad news may affect the addressee differently, possibly even creating palpable bodily effects. These may be intended or unintended. Much communication is colored in this way even if there are no actual linguistic markers of the emotion transmitted. When understanding an utterance, it is very important to capture the textures of such effects as well. These are nonpropositional but nevertheless part of semantics. This is possibly where it is best to resort to concrete descriptive techniques if such detail is required but, as I said in \sectref{sec:cl}, sentiment analysis in natural language processing deals with this phenomenon as a classification problem. I have already discussed in that section how Equilibrium Linguistics can be extended to such classification tasks so I will not say more about it here.

Finally, I repeat my definition of communication, Definition~\ref{def:communicates}, from \sectref{sec:speaker meaning and word meaning}, as we have come full circle. Recall that communication is the right notion to reduce and not speaker meaning as Grice believed. Also keep in mind that the proposition $p$ contains probabilistic contents as described in \sectref{sec:how to think about content}.

\begin{definition}

$\cal A$ communicates $p$ to $\cal B$ by uttering $\varphi$ in $u$ if and only if $\cal A$ intends (possibly partly implicitly) to convey $p$ to $\cal B$ in $u$, $\cal B$ intends (possibly partly implicitly) to interpret $\cal A$'s utterance of 
$\varphi$ in $u$, and the games $G^{\cal A}_u(\varphi)$, $G^{\cal B}_u(\varphi)$, $G_u(\varphi)$ induced thereby are equal and common knowledge and their solution is $p$.

%$\cal A$ communicates $p$ to $\cal B$ by uttering $\varphi$ in $u$ if and only if $\cal A$ intends (possibly partly implicitly) to convey $p$ to $\cal B$ in $u$, $\cal B$ intends (possibly partly implicitly) to interpret $\cal A$'s utterance of 
%$\varphi$ in $u$, and the three games $G^{\cal A}_u(\varphi)$, $G^{\cal B}_u(\varphi)$, $G_u(\varphi)$ induced thereby are equal and common knowledge and their solution is $p$.
\label{def:communicates 2}
\end{definition}

I have now developed the full details of the global games in the definition as well as of the larger picture of micro-semantics sketched in \chapref{ch:picture of communication}. This should make it possible to see how the definition naturalizes meaning \emph{modulo} the category of conventional meaning which has so far been taken as externally given. In \partref{part:V} on macro-semantics, I provide an internal account of conventional meaning after which the naturalization of meaning will be fully realized assuming the framework I have described is correct.

Before I turn to that task, we take a brief detour to see what Equilibrium Linguistics has to offer the problem of translation.
