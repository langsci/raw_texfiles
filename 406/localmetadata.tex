\author{Eleonore Schmitt}
\title{Frequenz. Prototyp. Schema.}
\subtitle{Ein gebrauchsbasierter Ansatz zur Entstehung grammatischer Varianten}

\BackBody{Die Arbeit entwickelt ein gebrauchsbasiertes Modell zur Entstehung grammatischer Varianten. Dieses wird auf drei Variationsphänomene angewandt: Variation in der Konjugation (\textit{geglimmt}/\textit{geglommen}), Variation in der Deklination (\textit{des Bären}/\textit{Bärs}) und Variation in der Selektion zwischen \textit{haben} und \textit{sein} im Perfekt (\textit{ich bin/habe Auto gefahren}). Zudem wird das Modell psycholinguistisch überprüft. Das Modell greift auf den gebrauchsbasierten Ansatz der Kognitionslinguistik zurück und erarbeitet Frequenz, Prototyp und Schema als grundlegende Einflussfaktoren darauf, wie wahrscheinlich Variation und Stabilität in einem Sprachsystem sind: Bei allen Variationsphänomenen sind neben der Variation auch stabile Verwendungen zu beobachten (\textit{geflogen}/*\textit{gefliegt}, \textit{des Matrosen}/*\textit{des Matroses}, \textit{ich bin gegangen}/*\textit{ich habe gegangen}).

Im theoretischen Teil der Arbeit werden Frequenz, Protoyp und Schema als kognitive Einflussfaktoren auf Variation und Stabilität modelliert und anschließend ihr Einfluss auf die drei Variationsphänomene theoretisch beleuchtet.  Im empirischen Teil der Arbeit wird der Einfluss der Faktoren Frequenz, Prototyp und Schema anhand von Reaktionszeitmessungen überprüft.

Das in der Arbeit entwickelte Modell fasst Variation und Stabilität von Sprache probabilistisch und prognostiziert auf diese Weise Variation. Der Rückgriff auf Reaktionszeiten erlaubt es, in der Sprachverarbeitung Variationspotential zu erkennen, das noch nicht im Sprachgebrauch sichtbar ist. Die Arbeit verdeutlicht damit den zentralen Stellenwert, den Variation in der Sprache einnimmt, erweitert mit der Verbindung aus Kognitions- und Psycholinguistik bestehende Forschung und ermöglicht einen systematischen, empirisch überprüfbaren Zugang zu Variation.}

\renewcommand{\lsImpressumExtra}{\vskip\smallskipamount{}
Diese Arbeit hat der Fakultät Geistes- und Kulturwissenschaften der Otto-Friedrich-Universität Bamberg als Dissertation vorgelegen. Gutachter\_innen: Prof. Dr. Renata Szczepaniak; Prof. Dr. Alexander Werth; \\ Prof. Dr. Klaus-Michael Köpcke \\Tag der mündlichen Prüfung: 21.06.2021.\smallskip\\}

\renewcommand{\lsSeries}{ogl}
\renewcommand{\lsSeriesNumber}{7}
\renewcommand{\lsID}{406}

\BookDOI{10.5281/zenodo.8434909}
\renewcommand{\lsISBNdigital}{978-3-96110-423-9}
\renewcommand{\lsISBNhardcover}{978-3-98554-081-5}

\proofreader{Hella Olbertz,
             Katja Politt,
             Lea Schäfer,
             Nicole Benker,
             Rebecca Madlener,
             Sophie Ellsaeßer,
             Tom Bossuyt,
             Yvonne Treis}
