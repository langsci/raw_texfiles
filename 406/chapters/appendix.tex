\chapter{Digitaler Anhang}

Der digitale Anhang der Dissertation kann über \url{https://osf.io/6yxzt/?view\_only=bdd5d42c44b8459e83c40af807c32e60} aufgerufen werden.

\chapter{Ablautreihen des Deutschen}
\label{ablautreihe}

\begin{table}
\begin{tabular}{llll}
\lsptoprule
\textbf{1} &\textit{reiten}& \textit{ritt}& \textit{geritten} \\
/aɪ̯/ + C & /aɪ̯/  & /ɪ/ & /ɪ/ \\
\midrule
\textbf{2} &\textit{bieten} &\textit{bot}& \textit{geboten}\\
/iː/ + C & /iː/ & /oː/ & /oː/ \\
\midrule
\textbf{3a} & \textit{sinken}& \textit{sank}& \textit{gesunken}\\
/ɪ/ + Nasal + C & /ɪ/ & /a/ & /ʊ/ \\
\textbf{3b} &\textit{helfen}& \textit{half}& \textit{geholfen}\\
/ɛ/ + Liquid + C & /ɛ/ & /a/ & /ɔ/ \\  
\midrule
\textbf{4} & \textit{nehmen}& \textit{nahm} &\textit{genommen}\\
/eː/ + Nasal/Liquid & /eː/ & /ɑː/ & /ɔ/\\
\midrule
\textbf{5} &\textit{geben}& \textit{gab}& \textit{gegeben}\\
/eː/ + C & /eː/ & /ɑː/ & /eː/ \\
\midrule
\textbf{6} &\textit{fahren}& \textit{fuhr}& \textit{gefahren}\\
/ɑː/ + C & /ɑː/ & /uː/ & /ɑː/ \\
\midrule
\textbf{7} &\textit{schlafen} &\textit{schlief} &\textit{geschlafen}\\
Vokal außer /eː/ & x {\tiny{(im Bsp. /ɑː/)}} & /iː/ & x {\tiny{(im Bsp. /ɑː/)}}\\
\lspbottomrule
\end{tabular}
\captionof{table}{Überblick über die Ablautreihen im Deutschen nach \textcite[209]{Bergmann.2016}}
\label{ARov}
\end{table}

\noindent Tabelle \ref{ARov} orientiert sich an \textcite[209]{Bergmann.2016}. Die Ablautreihe~1~und~2 werden dort in zwei Subgruppen geteilt. Da die Subgruppen für die Dissertation nicht relevant sind, wurde in Tabelle \ref{ARov} auf die Trennung  verzichtet. Die hier vorgestellten Ablautreihen lassen sich für das Neuhochdeutsche nicht mehr ansetzen, hier ist von ca. 40 Ablautalternanzen auszugehen (\cite[285--286]{Nubling.2017}). In der Dissertation werden die Ablautreihen dennoch als Orientierung genutzt.

\chapter{Frequenz der Präteritalformen in der Frequenzstudie}
\label{matfreqanh}

\begin{table}
\begin{tabular}{lrlr}
\lsptoprule
schwache Form & Belege & starke Form & Belege \\
\midrule
\textit{flocht} &  	1.327 & \textit{flechtete} & 15 \\
\textit{sponn}  & 139 & \textit{spinnte} & 12 \\
\textit{schwoll an} & 516 & \textit{schwellte an} & 0 \\
\textit{schmolz} & 8.281 & \textit{schmelzte} & 20\\
\textit{kniff} & 1.704 & \textit{kneifte} & 15\\
\textit{focht} & 4.729 & \textit{fechtete} & 14\\
\textit{molk} & 161 & \textit{melkte} & 110\\
\textit{drosch} & 7.980 & \textit{dreschte} & 18\\
\textit{hieb} & 5.844 & \textit{haute} & 5.526\\
\textit{quoll} & 3.605 & \textit{quellte} & 8\\
\textit{sog ein} & 17 & \textit{saugte ein} & 20\\
\textit{wob} & 1.080 & \textit{webte} & 378\\
\textit{glomm} & 322 & \textit{glimmte} & 291\\
\textit{es gor} &  2 & \textit{es gärte} & 83\\
\textit{sann} & 1.677 & \textit{sinnte} & 12 \\
\textit{sonn} & 1.809 & & \\
\lspbottomrule
\end{tabular}
\captionof{table}{Frequenz von schwachen und starken Präteritalformen der infrequenten Testverben}
\label{prätfrequ}
\end{table}

\noindent Das Verb \textit{salzen} ist nicht in Tabelle \ref{prätfrequ} aufgeführt, da es das Präteritum ausschließlich schwach bildet. Für die Präteritalformen von \textit{gären} wurde nach \textit{es gor} gesucht, da die Suche nach \textit{gor} zu viele Fehltreffer (z.~B. \textit{gor} als Variante zu \textit{gar} wie in \textit{gor ned}) enthält. 

\chapter{Lexical-decision-Studie}
\section{Aufbau der Blöcke in der lexical-decision-Studie}
\label{blocklex}

Die Abfolge der Blöcke A bis D ist in der Studie randomisiert. In Tabelle \ref{ablauflexta} sind sie in alphabetischer Reihenfolge aufgeführt. \textit{o.S.} steht für \textit{ohne Schwankung}, \textit{m.S.} für \textit{mit Schwankung}. Die Tabelle gibt zudem Auskunft über die Position der Stimuli innerhalb der Blöcke. Die Spalte \textit{Art} unterscheidet Testsubstantive und Testverben, zudem sind einige Items mit \textit{Evaluation} und \textit{Designtest} markiert. \textit{Evaluation} benennt Items, die zur Evaluation der Konzentriertheit der Proband\_innen genutzt wurden, sie stellen allesamt Filler dar. Auch die Test\-items mit \textit{Designtest} wurden hierfür genutzt, sie hätten aber zusätzlich als Items genutzt werden können, die testen, ob das Versuchdesign funktioniert. Dies war jedoch nicht nötig. Die Spalte \textit{Ausprägung} benennt die Frequenz- bzw. Form-Schematizitätsausprägung, der die Testverben bzw. -substantive angehören.

\begin{small}
\begin{longtable}{lllll}
\caption{Blockdesign in der lexical-decision-Studie\label{ablauflexta}}\\
\lsptoprule Position & Block & Stimulus & Art & Ausprägung \\ im Block \\ \midrule\endfirsthead
\midrule Position & Block & Stimulus & Art & Ausprägung \\ im Block \\ \midrule\endhead
\endfoot\lspbottomrule\endlastfoot
 1 & Beispiel & die Tische & Evaluation & \\ 
 2 & Beispiel & du gebst & Evaluation &  \\ 
 3 & Beispiel & die Autoe & Evaluation &  \\ 
 4 & Beispiel & du siehst & Evaluation &  \\ 
	\midrule
 1 & Einführung & des Laptopes & Evaluation &  \\ 
  2 & Einführung & des Transporteres & Evaluation &  \\ 
 3 & Einführung & des Bodens & Evaluation &  \\ 
 4 & Einführung & des Spiels & Evaluation &  \\ 
   5 & Einführung & des Praktikumes & Evaluation &  \\ 
		\midrule
   1 & A & des Kontoes & Evaluation &  \\ 
   2 & A & getragen & Testverb & frequent \\ 
  3 & A & des Helds & Testsubstantiv & Peripherie \\ 
 4 & A & gefahrt & Testverb & frequent \\ 
  5 & A & geflechtet & Testverb & infrequent o. S. \\ 
   6 & A & des Flügeles & Evaluation &  \\ 
   7 & A & gezogen & Testverb & frequent \\ 
   8 & A & des Neffen & Testsubstantiv & Form-Schema \\ 
   9 & A & gesinnt & Testverb & infrequent m. S. \\ 
 10 & A & des Feinden & Testsubstantiv & stark \\ 
   11 & A & gegoren & Testverb & infrequent m. S. \\ 
12 & A & des Zaren & Testsubstantiv & Peripherie \\ 
 13 & A & gebringt & Designtest &  \\ 
 14 & A & gequollen & Testverb & infrequent m. S. \\ 
   15 & A & des Kollegen & Testsubstantiv & Form-Schema \\ 
  16 & A & gedroschen & Testverb & infrequent o. S. \\ 
 17 & A & des Franzoses & Testsubstantiv & Form-Schema \\ 
  18 & A & gemelkt & Testverb & infrequent o. S. \\ 
   19 & A & des Fürsts & Testsubstantiv & Peripherie \\ 
  20 & A & des Angleres & Evaluation &  \\ 
 21 & A & angeschwollen & Testverb & infrequent o. S. \\ 
  22 & A & des Ultimatumes & Designtest &  \\ 
  23 & A & geschreibt & Testverb & frequent \\ 
  24 & A & des Kerls & Testsubstantiv & stark \\ 
   25 & A & gewebt & Testverb & infrequent m. S. \\ 
\midrule
   1 & B & des Frühstückes & Evaluation &  \\ 
 2 & B & gebracht & Designtest &  \\ 
  3 & B & des Helden & Testsubstantiv & Peripherie \\ 
  4 & B & gefahren & Testverb & frequent \\ 
  5 & B & des Manteles & Evaluation &  \\ 
  6 & B & geflochten & Testverb & infrequent o. S. \\ 
   7 & B & des Feindes & Testsubstantiv & stark \\ 
8 & B & gesonnen & Testverb & infrequent m. S. \\ 
  9 & B & des Kerlen & Testsubstantiv & stark \\ 
 10 & B & gequellt & Testverb & infrequent m. S. \\ 
  11 & B & des Zars & Testsubstantiv & Peripherie \\ 
  12 & B & gedrescht & Testverb & infrequent o. S. \\ 
 13 & B & des Internetes & Designtest &  \\ 
 14 & B & gemolken & Testverb & infrequent o. S. \\ 
 15 & B & gezieht & Testverb & frequent \\ 
 16 & B & des Kolleges & Testsubstantiv & Form-Schema \\ 
 17 & B & gegärt & Testverb & infrequent m. S. \\ 
  18 & B & des Neffes & Testsubstantiv & Form-Schema \\ 
  19 & B & gewoben & Testverb & infrequent m. S. \\ 
   20 & B & des Franzosen & Testsubstantiv & Form-Schema \\ 
   21 & B & angeschwellt & Testverb & infrequent o. S. \\ 
  22 & B & des Fürsten & Testsubstantiv & Peripherie \\ 
   23 & B & geschrieben & Testverb & frequent \\ 
  24 & B & getragt & Testverb & frequent \\ 
  25 & B & des Wetteres & Evaluation &  \\ 
	\midrule
  1 & C & des Abendes & Evaluation &  \\ 
 2 & C & gespinnt & Testverb & infrequent o. S. \\ 
   3 & C & des Schützen & Testsubstantiv & Form-Schema \\ 
   4 & C & gesprochen & Testverb & frequent \\ 
  5 & C & des Orchesteres & Evaluation &  \\ 
  6 & C & geflogen & Testverb & frequent \\ 
 7 & C & gegangen & Designtest &  \\ 
  8 & C & des Geselles & Testsubstantiv & Form-Schema \\ 
 9 & C & geschmolzen & Testverb & infrequent o. S. \\ 
  10 & C & des Grafs & Testsubstantiv & Peripherie \\ 
  11 & C & gefechtet & Testverb & infrequent o. S. \\ 
  12 & C & des Vogten & Testsubstantiv & stark \\ 
   13 & C & eingesogen & Testverb & infrequent m. S. \\ 
\midrule
14 & C & des Löffeles & Evaluation &  \\ 
  15 & C & gesalzt & Testverb & infrequent m. S. \\ 
 16 & C & des Ultimatums & Designtest &  \\ 
   17 & C & gesinkt & Testverb & frequent \\ 
   18 & C & des Dieben & Testsubstantiv & stark \\ 
  19 & C & gekniffen & Testverb & infrequent o. S. \\ 
 20 & C & des Nachbarn & Testsubstantiv & Peripherie \\ 
   21 & C & gehaut & Testverb & infrequent m. S. \\ 
  22 & C & des Regenes & Evaluation &  \\ 
  23 & C & gehaltet & Testverb & frequent \\ 
 24 & C & geglommen & Testverb & infrequent m. S. \\ 
  25 & C & des Freundes & Testsubstantiv & stark \\ 
	\midrule
   1 & D & des Hebeles & Evaluation &  \\ 
   2 & D & gesponnen & Testverb & infrequent o. S. \\ 
 3 & D & des Grafen & Testsubstantiv & Peripherie \\ 
   4 & D & gegeht & Designtest &  \\ 
5 & D & des Internets & Designtest &  \\ 
 6 & D & gesunken & Testverb & frequent \\ 
 7 & D & des Vogts & Testsubstantiv & stark \\ 
 8 & D & gesprecht & Testverb & frequent \\ 
   9 & D & gesalzen & Testverb & infrequent m. S. \\ 
 10 & D & des Nebeles & Evaluation &  \\ 
  11 & D & geschmelzt & Testverb & infrequent o. S. \\ 
 12 & D & des Messeres & Evaluation &  \\ 
    13 & D & eingesaugt & Testverb & infrequent m. S. \\ 
 14 & D & des Diebes & Testsubstantiv & stark \\ 
  15 & D & des Wagenes & Evaluation &  \\ 
  16 & D & gekneift & Testverb & infrequent o. S. \\ 
   17 & D & des Schützes & Testsubstantiv & Form-Schema \\ 
  18 & D & gehalten & Testverb & frequent \\ 
   19 & D & des Gesellen & Testsubstantiv & Form-Schema \\ 
   20 & D & gefliegt & Testverb & frequent \\ 
 21 & D & des Nachbars & Testsubstantiv & Peripherie \\ 
 22 & D & gehauen & Testverb & infrequent m. S. \\ 
 23 & D & des Freunden & Testsubstantiv & stark \\ 
  24 & D & geglimmt & Testverb & infrequent m. S. \\ 
 25 & D & gefochten & Testverb & infrequent o. S. \\ 
\end{longtable}
\end{small}

\section{Überblick über die Filleritems}
\label{fillerlex}

Folgende Filleritems wurden genutzt: 

\begin{multicols}{3}
\ea des Abendes
\ex des Angleres
\ex des Bodens
\ex des Flügeles
\ex des Frühstückes
\ex des Hebeles
\ex des Internetes
\ex des Internets
\ex des Kontoes
\ex des Laptopes
\ex des Löffeles
\ex des Manteles
\ex des Messeres
\ex des Nebeles
\ex des Orchesteres
\ex des Praktikumes
\ex des Regenes
\ex des Spiels
\ex des Transporteres
\ex des Ultimatumes
\ex des Ultimatums
\ex des Wagenes
\ex des Wetteres
\ex die Autoe
\ex die Tische
\ex du gebst
\ex du siehst
\ex gebracht
\ex gebringt
\ex gegangen
\ex gegeht
\z
\end{multicols}

\chapter{Sentence-maze-Studie}
\label{Anhangmatsent}


\section{Aufbau der Blöcke in der sentence-maze-Studie}
\label{blocksent}

Die Experimentblöcke A bis D waren in der Studie in randomisiert. In Tabelle~\ref{ablaufsentta} (S.~\pageref{ablaufsentta}f.) werden sie in tabellarischer Reihenfolge aufgeführt. Zudem wird der Beispiel- und der Einführungsblock gezeigt (in der Tabelle werden sie mit \textit{Bsp.} und \textit{Einf.} abgekürzt). Die Tabelle enthält neben den Testsätzen die Position der Testsätze im Block und benennt die Testitems sowie die Transitivitäts- bzw. Form-Sche\-ma\-ti\-zi\-täts\-aus\-prä\-gung der Testitems. Die senkrechten Striche (|) markieren die Abschnitte, in die die Sätze geteilt sind. Der obere Satz in der Tabelle besteht jeweils aus den zu erwartenden Möglichkeiten, der untere aus den unerwarteten. Bei Varianten sind selbstverständlich beide Möglichkeiten erwartbar, hier ist die Einteilung nur zufällig. 


\begin{sidewaystable}
\begin{tabular8}{lll}
\lsptoprule
  	Block &	Satz &	Testitem	 \\
\midrule
1 	Bsp.	& Erst wähle ich den Satzanfang, | dann | wähle | ich | die weiteren Wörter | Stück für Stück | aus, | &\\
& die | den Satz | am besten | fortsetzen&  \\		
1	 Bsp. &	Er wäle ich den Satzanfang, | trotz | wehle | du | das weiteren Wörter | Apfel für Apfel | vor, |& \\ 
& dass | den Zeichen | am Besten | vortsetzen &\\
\midrule	
1 Einf. &Ich wusste nicht, | dass | meine Tante | durch die USA | reisen | will  &\\		
1	Einf. & Ich wuste nicht, | weil | meine Onkel | durch den USA | schwimmen | sollen 	&\\	
\midrule
2	Einf.	&Ich weiß, | dass | der Vater | die Kinder | zur Schule | gebracht | hat &	\\
2	Einf.	& Ich weis, | das | der Fater | die Junge | zur Tennisplatz | geklettert | ist	&\\
\midrule
3	Einf.	&Er geht spazieren, | weil | die Sonne | scheint		&\\
3	Einf.	&Er get spazieren, | dass | die Wolke | regnet &\\
\midrule		
1	A&	Sie konnte nicht fassen, | dass | das Anwesen | des Grafen | so groß | ist | wie ein Schloss &Graf \\
1	A	& Sie konnte nicht fasen, | das | das Burg | des Grafs | so glitzernd | hat | als | ein Schloß	&Peripherie\\	
\midrule
2	A	& Er sah, | wie | die Mutter | die Kinder | zur Schule | gefahren hat &	fahren	\\
2	A	& Er sa, | obwohl | die Vater | das Junge | zur Schuhle | geschwommen | ist		&transitiv\\
\midrule
3	A	& Ich nehme an, | dass | die Arbeit | meines Kollegen | wertgeschätzt wird&	Kollege	\\
3	A	& Ich neme an, | das | der Leistung | meines Kolleges | wertgeschetzt | muss	&Form-Schema\\
\midrule
4	A	& In der Zeitung stand, | dass | der Pilot | die Distanz | in Rekordzeit | geflogen | ist&	fliegen \\
4	A	& In das Zeitung stand, | weil | mein Nachbarin | der Rennen | in Beszeit | geschwommen | hat	& ambig II\\
\midrule
5	A	& Er fährt nochmal zurück, | weil | er | die Jacke | seines Freundes | aus Versehen | eingesteckt | hat | & Freund\\
& und | dieser | sie | auf der Arbeit | benötigt & stark	\\
5	A	& Er färt nochmal zurück, | dass | sie | die Anorak | seines Freunden | aus Absicht | eingestekt | ist |& \\ 
& oder | diese | ihn | auf das Arbeit | benöhtigt &\\	
\midrule
6 A	& Meine Oma erzählte, | dass | sie | in den Siebzigern | regelmäßig | durch den Schwarzwald | geritten | ist &	reiten	\\
6	A	& Meine Ohma erzählte, | trotz | er | in den 70igern | Regel mäßig | durch den Schwartswald | geschwommen | hat	&intransitiv\\
\midrule
1	B	& Mein Freund war nicht da, | weil | er | zum Geburtstag | seines Neffen | gegangen | ist	&Neffe\\
1	B	&  Meine Freund war nicht da, | dass | sie | zum Geburstag | seines Neffes | geschwommen | hat	&Form-Schema\\
\midrule
2 B	& Ich nehme an, | dass | der Pilot | die Passagiere | zum nächstgelegenen | Flughafen | geflogen | hat&	fliegen\\
2	B	& Ich neme an, | ob | der Stewardess | das Ware | zum nächst gelegenen | Bahnhof | geschwommen | ist	&transitiv\\
\midrule
\end{tabular8}
\captionof{table}{Blockdesign in der sentence-maze-Studie}
\label{ablaufsentta}
\end{sidewaystable}
\begin{sidewaystable}
\begin{tabular8}{lll}
\midrule
  	Block &	Satz &	Testitem	\\
   \midrule
3 B	&Mein Neffe mag Superman nicht, | weil | ihm | der Anzug | des Helden | nicht gefällt	&Held	\\
3 B	& Mein Nefe mag Superman nicht, | dass | ihre Tante | das Aufzug | des Helds | nicht gefellt &Peripherie\\		
\midrule
4 B	&Mir wurde erzählt, | dass | mein Onkel | früher | ein schwarzes Pferd | geritten | hat	&reiten\\
4 B	 &Mir wurde erzehlt, | trotzdem | mein Tante | früer | ein schwarzes Stute | geschwommen | ist&ambig I\\ 		
\midrule
5 B	&Ich habe gehört, dass | mein Neffe | seit gestern | Fahrradfahren | kann		&\\
5 B&	 Ich habe gehöhrt, | trotz | meine Neffe | seid gestern | Fahrrad fahren | müssen&\\  		
\midrule
6 B&	Ich ging davon aus, | dass | die Oma | am Sonntag | zur Kur | gefahren | ist	&fahren\\
6 B	& Ich gehte davon aus, | das | der Oma | am Sontag | zum Kur | geklettert | hat 		&intransitiv\\
\midrule
1 C	&Ich habe gesehen, | wie | der Pfeil | des Schützen | ins Schwarze | traf  	& Schütze\\
1 C	&Ich habe gesen, | trotzdem | der Feil | des Schützes | ins Blaue | treffte 	&Form-Schema\\	
\midrule
2 C	&Ich wusste nicht, | dass | mein Bruder | am Mittwoch | in die USA | geflogen | ist &	fliegen\\
2 C	&Ich wuste nicht, | weil | mein Schwester | an das Mittwoch | in das USA | geschwommen | hat 		&intransitiv\\
\midrule
3 C	&Er erzählt, | dass | sein Bruder | in den USA | herum reisen | will 		&\\
3 C	&Er erzehlt, | trotz | seine Bruder | auf die USA | herumreisen | müssen		&\\
\midrule
4 C	&Im Radio hieß es, | dass | der Reiter | den Parcours | fehlerlos | geritten | ist& 	reiten\\
4 C	&Im Radio hies es, | trotzdem | die Jockey | die Hindernis | fehlerfrei | geflogen | hat		&ambig II\\
\midrule
5 C	&Er macht einen Ausflug, | um | die Burg | des Vogts | zu betrachten & Vogt \\
5 C	&Er macht eine Ausflug, | dass | die Haus | des Vogten | zu singen		&stark\\
\midrule
6 C	&Er erzählte mir, | dass | seine Tante | eine Zeit lang | ein Cabrio | gefahren | ist &	fahren	\\
6 C	&Er erzälte mir, | das | ihre Tante | eine zeitlang | eine Cabrio | geflogen | hat 		&ambig I\\
\midrule
1 D	&Er teilte mir mit, | dass | der Kurier | die Ware | gestern morgen | liefern | sollte		&\\
1 D	&Er tailte mir mit, | obwohl | die Lieferant | die Wahre | gestern Morgen | essen | durften		&\\
\midrule
2 D	&Ich kann kaum glauben, | dass | meine Schwester | neulich | einen Helikopter | geflogen | hat&	fliegen\\
2 D&	Ich kan kaum glauben, | das | meine Bruder | neuhlich | einen Flugzeug | gefahren | ist		&ambig I\\
\midrule
3 D	&Ihm war nicht bewusst, | dass | die Macht | des Zaren | einmal | derart | groß | gewesen | war& Zar \\
3 D	&Ihm war nicht bewust, | ob | die Einfluss | des Zars | ein Mal | der Art | gross | gestaltet | wurde		&Peripherie\\
\midrule
4 D	&In den Nachrichten sagten sie, | dass | der neue Fahrer | die Strecke | in Rekordzeit | gefahren | ist	& fahren \\
4 D	&In die Nachrichten sagten sie, | das | der neue Farer | das Runde | in Beszeit | geklettert | hat		&ambig II\\
\midrule
5 D	&Der Krimi-Fan ahnt schon früh, | dass | der Plan | des Diebes | zu simpel | ist | und | nicht klappt	& Dieb \\
5 D	&Der Krimi-Fan ant schon früh, | das | der Tatik | des Dieben | zu blau | hat | oder | nicht klapt		&stark\\
\midrule
6 D	&Ich gehe davon aus, | dass | das Mädchen | das Pferd | zur Koppel | geritten | hat	& reiten	\\
6 D	&Ich gehe dahvon aus, | trotzdem | die Junge | das Hengst | zur Kopel | geflogen | ist 		&transitiv\\
		\lspbottomrule
   \end{tabular8}
\end{sidewaystable}




\section{Überblick über die Füllsätze}
\label{anhangfüll}

Folgende Sätze wurden genutzt:

\ea Ich wusste nicht, dass meine Tante durch die USA reisen will.  		
\ex Ich weiß, dass der Vater die Kinder zur Schule gebracht hat. 		
\ex Er geht spazieren, weil die Sonne scheint.	
\ex Ich habe gehört, dass mein Neffe seit gestern Fahrradfahren kann.		
\ex Er erzählt, dass sein Bruder in den USA herum reisen will. 		
\ex Er teilte mir mit, dass der Kurier die Ware gestern morgen liefern sollte.
\z

\chapter{Plausibilitätstest zur sentence-maze-Studie}
\label{plausent}

Für den Plausibilitätstest zur sentence-maze-Studie wurden alle Sätze aus der Prototypizitäts- und Form-Schematizitätsstudie inklusive der Füllsätze abgefragt. Zusätzlich dazu wurden Distraktorsätze genutzt. Diese dieten einerseits zur Ablenkung, andererseits als Vergleichspunkt: Da die Proband\_innen die Sätze nach ihrer semantischen Plausibilität beurteilen sollten, war der Inhalt der Distraktorsätze unwahrscheinlich oder unmöglich (siehe \sectref{distraktoran}), um einen Kontrast zu den Testsätzen zu generieren.
Die Daten wurden mithilfe von SoSciSurvey (\cite{Leiner.2019}), Version 3.1.06, erhoben. Die Stichprobengröße für den Plausibilitätstest beträgt nach Bereinigung der Daten 58 Proband\_innen. Weitere Informationen zum Plausibilitätstest sind im digitalen Anhang.

\section{Ergebnisse des Plausibilitätstests}
In diesem Abschnitt werden die Ergebnisse des Plausibilitätstests für die Testsätze der Studien zu Prototypizität und Form-Schematizität samt den Ergebnissen für die transitiven Fillersätze in der Prototypizitätsstudie präsentiert. Die Tabellen \ref{plaushaben} und \ref{plausubst} zeigen die Ergebnisse für die Testsätze in der Prototypizitätsstudie und der Form-Schematizitätsstudie. Die Ergebnisse zu den restlichen Fillersätzen und den Distraktorsätzen sind im digitalen Anhang.  Die einzelnen Testsätze werden in den Abschnitten \ref{protmat} und \ref{schemadesignsent} aufgeschlüsselt. 

\begin{table}
\begin{tabular}{lrrrrrcc}
\lsptoprule
Testsatz &	1	&2&	3&	4&	5& gesamt&	Durchschnitt\\
\midrule
fliegen$_{transitiv}$ &	9	&7&	12&	16&	14&58	&3,33\\
fliegen$_{ambig I}$	&4	&10&	11	&12	&21&58	&3,62\\
fliegen$_{ambig II}$	&3	&5	&5&	18	&27&58&	4,05\\
fliegen$_{intransitiv}$&7&	10	&9	&10&	22&58	&3,52\\
\midrule
reiten$_{transitiv}$&	5	&10	&19	&24	&0&58&	3,07\\
reiten$_{ambig I}$&	18	&11	&9&	9	&11&58&	2,72\\
reiten$_{ambig II}$&	3	&6	&4	&10&	35&58	&4,17\\
reiten$_{intransitiv}$&	5	&10	&14	&14	&15&58	&3,41\\
\midrule
fahren$_{transitiv}$&	1	&2	&0	&10	&45&58	&4,66\\
fahren$_{ambig\_I}$	&2	&5	&8&	16	&27&58&	4,05\\
fahren$_{ambig\_II}$	&2	&2	&17&	37&	0&58	&3,53\\
fahren$_{intransitiv}$&	6	&6	&7	&17&	22&58	&3,74\\
\midrule
Filler$_{Jacke}$\footnote{Filler$_{Jacke}$: \textit{Er fährt nochmal zurück, weil er die Jacke seines Freundes aus Versehen eingesteckt hat und dieser sie auf der Arbeit benötigt} } &	2	&7&	9	&10&	30&58	&4,02\\
Filler$_{Vater}$\footnote{Filler$_{Vater}$: \textit{Ich weiß, dass der Vater die Kinder zur Schule gebracht hat}} &2	&0	&3&	18&	35&58&	4,45\\
\lspbottomrule 
	\end{tabular}
	\captionof{table}{Ergebnisse des Plausbilitätstests für die Testsätze der Prototypizitätsstudie}
	\label{plaushaben}		
\end{table}


\begin{table}
\begin{tabular}{lrrrrrcc}
\lsptoprule
Testsubstantiv& 1 &	2	&3&	4&	5	& gesamt& Durchschnitt\\
\midrule
\textit{Kollege}&	3&	6&	13&	18&	18&	58&3,72\\
\textit{Neffe}&	1&	4&	6&	10&	37&58&	4,53\\
\textit{Schütze}&	1&	2&	4&	9&	42&58&	4,53\\
\textit{Graf}&	5&	8&	9&	8&	28&	58&3,79\\
\textit{Held}	&1&	1&	3&	19&	34&	58&4,45\\
\textit{Zar}	&3&	8&	16&	31&	0&	58&3,29\\
\textit{Vogt}	&5&	5&	6&	17&	25&58&	3,90\\
\textit{Freund}	&2&	7&	9&	10&	30&	58&4,02\\
\textit{Dieb}	&1&	8&	19&	30&	0&58&	3,34\\
\lspbottomrule
\end{tabular}
\captionof{table}{Ergebnisse des Plausbilitätstests für die Testsubstantive}
\label{plausubst}		
\end{table}

In den Tabellen \ref{plaushaben} und \ref{plausubst} reichte die Skala bei \textit{reiten\_transitiv}, \textit{fahren\_ambig II}, \textit{Zar} und \textit{Dieb} nur von 1 bis 4. Aufgrund der geringen Akzeptanz wurde \textit{\mbox{reiten\_}\mbox{ambig}~I} geändert. Ursprünglich lautete der Satz  \textit{Mir wurde erzählt, dass mein Onkel während seines Studiums ein schwarzes Pferd geritten hat/ist}. Er wurde zu \textit{Mir wurde erzählt, dass mein Onkel früher ein schwarzes Pferd geritten hat/ist} geändert und so den anderen Sätzen mit \textit{reiten} ähnlicher gemacht. Im Nachhinein wurde außerdem \textit{Parkour} in \textit{reiten\_ambig II} zu \textit{Parcours} geändert. Zudem war bei \textit{reiten\_ambig II} ursprünglich \textit{Bestzeit} statt \textit{Rekordzeit} vorgesehen. Die Änderung entstand durch einen Übertragungsfehler. In Tabelle \ref{plausubst} wurde im Nachhinein nur \textit{gewertschätzt} im Testsatz zum Substantiv \textit{Kollege} zu \textit{wertgeschätzt} geändert.


\section{Distraktorsätze}
\label{distraktoran}
Folgende Sätze wurden genutzt:

\ea Ich habe in der Zeitung gelesen, dass meine Nichte gestern im Schwimmbad war.
\ex Ich habe gehört, dass mein Bruder den Marathon in einer halben Stunde gelaufen ist.
\ex In den Nachrichten hieß es, dass Hamburg die sonnigste Stadt Deutschlands ist.
\ex Im Radio hieß es, dass du vorhin falsch geparkt hast.
\ex Der Polizist ahnt schon früh, dass der Plan des Täters wegklettert.
\ex In der Zeitung steht, dass Deutschland keine Hauptstadt hat.
\ex Ihm war nicht bewusst, dass der Hund des Grafen sprechen konnte.
\ex Mir hat er gesagt, dass er früher gerne Äpfel geritten ist.
\ex Mein Freund war nicht da, weil er bei der Queen zum Kaffee eingeladen war.
\ex Mir hat er gesagt, dass er früher eine schwarze Kuh geritten ist.
\ex Ich nehme an, dass der Pilot die Passagiere zum Mars geflogen hat.
\ex Er fährt den weiten Weg nochmal zurück, weil er die Zahnpasta seines Freunds vergessen hat.
\ex Der Polizist ahnt schon früh, dass der Plan des Täters ein fliegendes Taxi involviert.
\ex Ich weiß, dass die Kinder ihren Vater zur Schule gebracht haben.
\ex Meine Oma erzählte, dass sie früher im Sommer Skifahren war.
\z

\chapter{Self-paced-reading-Studie}
\label{ablaufpaced}
\section{Texte zu den Testitems}
\label{testitemself}

Die Abfolge der Endungen bei den Testitems ist pseudorandomisiert. Eine Gruppe liest zuerst die erwartbare Form (-\textit{n} für \textit{Schettose} und \textit{Truntake}, -\textit{s} für \textit{Grettel} und \textit{Knatt}), dann die unerwartete (-\textit{s} für \textit{Schettose}, endungslos bei \textit{Truntake}, -\textit{(e)n} für \textit{Grettel} und \textit{Knatt}). Bei \textit{Knatt} sind beide Formen erwartbar, aber -\textit{s} aufgrund der Typenfrequenz wahrscheinlicher. Auch bei den Fragen zu den Testitems wird in dieser Gruppe zuerst die erwartete und dann die unerwartete genannt (Haben Sie \textit{des Schettosen} oder \textit{des Schettoses} gelesen?). Diese Verteilung wird auch hier präsentiert. Die andere Gruppe liest die gegenteilige Verteilung. Die senkrechten Striche (|) markieren die Wortgruppen, in die die Sätze geteilt wurden.\largerpage[-1]

\subsection{Prototyp des Form-Schemas: \textit{Schettose} und \textit{Truntake}}

\subsubsection*{Schettose, der}
(Betonung: Schettóse) \\
Amtsbezeichnung am ungarischen Hof, Amtsschreiber/Diener
\begin{SchmittAppendixList}
\item Der Schettose | war ähnlich gestellt | wie ein Amtsschreiber, | zusätzlich erfüllte er | jedoch niedere Dienste. | Frage$_1$ 
\item Der Schettose | wurde vom König | direkt ernannt. | Somit war | das Amt | des Schettosen | offensichtlich | ein verantwortungsvoller Posten. | Frage$_2$
\item Es ist | davon auszugehen, | dass die Übertragung dieses Amts | als Ehre galt. | Frage$_3$ 
\item Interessanterweise | ließ sich | außer dem Amtsschreiber | kein vergleichbares Amt | an anderen Höfen | beobachten. | Frage$_4$ 
\item Dementsprechend war | das Amt | des Schettoses | offensichtlich | ein Spezifikum | des ungarischen Könighofs. | Frage$_5$ 
\end{SchmittAppendixList}

\begin{enumerate}
\item Haben Sie \textit{Amtschreiber} oder \textit{Amtsschreiber} gelesen?   
\item Haben Sie \textit{des Schettosen} oder \textit{des Schettoses} gelesen?   
\item Haben Sie \textit{des Amtes} oder \textit{des Amts} gelesen?  
\item Haben Sie \textit{Amtschreiber} oder \textit{Amtsschreiber} gelesen?   
\item Haben Sie \textit{des Schettosen} oder \textit{des Schettoses} gelesen?
\end{enumerate}

\subsubsection*{Truntake, der}
(Betonung: Truntáke)\\
Vorsitzender bei Gerichtshöfen
\begin{SchmittAppendixList}
\item Der Truntake war | im 17. Jahrhundert | in englischen Gerichtshöfen | weit verbreitet. | Frage$_1$ 
\item Der Truntake | wurde dabei | formal wie ein Richter | des Hofs | behandelt. | Frage$_2$ 
\item Jedoch kam | dem Truntaken | üblicherweise | keine Urteilskraft zu. | Frage$_3$ 
\item Der Truntake | stand dem Gericht | jedoch vor | und achtete | auf die Einhaltung | der Gerichtsordnung. | Frage$_4$ 
\item Häufig wurden | angehende Richter | des Hofes | mit diesem Amt betraut. | Frage$_5$  
\item Sie galten |  als vielversprechende Aspiranten | auf verantwortungsvolle Posten. | Frage$_6$ 
\item Dementsprechend begehrt war | der Posten. | Jedoch kam | dem Truntake | üblicherweise | auch eine unangenehme Aufgabe zu: | der Schuldspruch. | Frage$_7$ 
\end{SchmittAppendixList}

\begin{enumerate}
\item Haben Sie \textit{Gerichthöfen} oder \textit{Gerichtshöfen} gelesen?   
\item Haben Sie \textit{des Hofs} oder \textit{des Hofes} gelesen? 
\item Haben Sie \textit{dem Truntaken} oder \textit{dem Truntake} gelesen?  
\item Haben Sie \textit{Gerichtsordnung} oder \textit{Gerichtordnung} gelesen?  
\item Haben Sie \textit{des Hofs} oder \textit{des Hofes} gelesen?  
\item Haben Sie \textit{verantwortungvoll} oder \textit{verantwortungsvoll} gelesen?   
\item Haben Sie \textit{dem Truntaken} oder \textit{dem Truntake} gelesen?
\end{enumerate}

\subsection{Peripherie des Form-Schemas: \textit{Knatt}}

\subsubsection*{Knatt, der}
Veraltete Berufsbezeichnung, Feuerhüter
\begin{SchmittAppendixList}
\item Dieser Beruf | war über Jahrhunderte | zentraler Bestandteil | einer funktionierenden Dorfgemeinschaft.| Frage$_1$ 
\item Er war zunächst |  vor allem | in Franken und Bayern üblich. | Mit der Zeit | setzte sich | der Beruf | des Knatts | anscheinend | auch in anderen Regionen durch. | Frage$_2$ 
\item Der Knatt | wachte über große Feuerstellen | und läutete | im Brandfall |  die Feuerglocken. | Frage$_3$   
\item Mit der Verbreitung der Feuerwehren | wurde der Knatt | für die Dorfgemeinschaft | zunehmend bedeutungslos. | Frage$_4$ 
\item Auch in der Stadt | setzte sich |  der Beruf | des Knatten | anscheinend | nie durch. | Frage$_5$  
\end{SchmittAppendixList}

\begin{enumerate}
\item Haben Sie \textit{Dorfgemeinschaft} oder \textit{Dorfsgemeinschaft} gelesen?
\item Haben Sie \textit{des Knatts} oder \textit{des Knatten} gelesen?   
\item Haben Sie \textit{Brandsfall} oder \textit{Brandfall} gelesen?   
\item Haben Sie \textit{Dorfgemeinschaft} oder \textit{Dorfsgemeinschaft} gelesen? 
\item Haben Sie \textit{des Knatts} oder \textit{des Knatten} gelesen?
\end{enumerate}

\subsection{Form-Schema der starken Flexion: \textit{Grettel}}

\subsubsection*{Grettel, der}
(Betonung: Gréttel) \\ Funktionsträger in Heimatvereinen, Trachtenwart
\begin{SchmittAppendixList}
\item Der Grettel | ist für die Instandhaltung | der Trachten | von Heimatvereinen zuständig. | Frage$_1$   
\item Die Trachten  | sind oft | hochwertig und empfindlich. | Deswegen ist | das Amt | des Grettels | in vielen Vereinen | nur für langjährige Mitglieder | vorgesehen. | Frage$_2$  
\item Außerdem übernimmt | der Grettel | repräsentative Aufgaben. | Die Ausführung dieses Ehrenamts | nimmt deshalb | viel Zeit | in Anspruch. | Frage$_3$ 
\item Dementsprechend ist | das Amt | des Gretteln | in vielen Vereinen | nicht besetzt. | Frage$_4$ 
\end{SchmittAppendixList}

\begin{enumerate}
\item Haben Sie \textit{Instandshaltung} oder \textit{Instandhaltung} gelesen?  
\item Haben Sie \textit{des Grettels} oder \textit{des Gretteln} gelesen?
\item Haben Sie \textit{des Ehrenamtes} oder \textit{des Ehrenamts} gelesen? 
\item Haben Sie \textit{des Grettels} oder \textit{des Gretteln} gelesen? 
\end{enumerate}

\section{Texte zu den Distraktoren}
\label{distraktorself}

\begin{sloppypar}
Hinsichtlich der Distraktoren wurde die Reihenfolge der Pluralformen nicht pseudorandomisiert, sodass beide Gruppen identische Texte lasen. Die senkrechten Striche (|) markieren die Wortgruppen, in die die Sätze geteilt wurden.
\end{sloppypar}

\subsubsection*{Taff, die}
Veraltete Berufsbezeichnung, eine Art Hellseherin
\begin{SchmittAppendixList}
\item Der Beruf | war | innerhalb des osmanischen Reichs | weit verbreitet. | Frage$_1$ 
\item Nahezu | in jedem Dorf | arbeiteten Taffs. | Frage$_2$ 
\item Zu den Aufgaben der Täffe | zählte die Begleitung | von Geburten | durch Gebete. | Frage$_3$     
\item Zudem sagten sie | das Schicksal | des neugeborenen Kindes | vorher. | Frage$_4$ 
\item In der Gesellschaft |  waren Taffe |  hoch angesehen. | Frage$_5$ 
\end{SchmittAppendixList}

\begin{enumerate}
\item Haben Sie \textit{des Reichs} oder \textit{des Reiches} gelesen?   
\item Haben Sie \textit{Taffe} oder \textit{Taffs} gelesen?   
\item Haben Sie \textit{Täffe} oder \textit{Taffe} gelesen?  
\item Haben Sie \textit{des Kinds} oder \textit{des Kindes} gelesen?   
\item Haben Sie \textit{Täffe} oder \textit{Taffe} gelesen?   
\end{enumerate}


\subsubsection*{Schirr, die}
Adelstitel, ähnlich einer Gräfin
\begin{SchmittAppendixList}
\item Der Adelstitel der Schirr | wurde | in Sachsen eingeführt. | Frage$_1$ 
\item Schirrs waren | Gräfinnen gleichgestellt, | hatten aber | in die gräfliche Familie | eingeheiratet. | Frage$_2$ 
\item Die Schirre wurden | inoffiziell als Gräfin | angesprochen, | weswegen sich | der Titel | nie durchsetzte. | Frage$_3$
\end{SchmittAppendixList}

\begin{enumerate}
\item Haben Sie \textit{Adeltitel} oder \textit{Adelstitel} gelesen?
\item Haben Sie \textit{Schirrs} oder \textit{Schirre} gelesen?
\item Haben Sie \textit{Schirrs} oder \textit{Schirre} gelesen?
\end{enumerate}

\subsubsection*{Bracht, die}
Veraltete Berufsbezeichnung, eine Art Schneiderin
\begin{SchmittAppendixList}
\item Zu den Aufgaben von Brächten |  zählte das Ausbessern | von Kleidungsstücken. | Frage$_1$ 
\item Dabei waren Brachts | oft in Schneidereien | beschäftigt | und ähnlich gestellt | wie Lehrlinge. | Frage$_2$ 
\item Zusätzlich zu den Ausbesserungsarbeiten | waren Brachten | damit betraut, |für das leibliche Wohl | der Schneiderfamilie | zu sorgen. | Frage$_3$
\end{SchmittAppendixList}

\begin{enumerate}
\item Haben Sie \textit{von Brächten} oder \textit{von Brachten} gelesen?
\item Haben Sie \textit{Brachte} oder \textit{Brachts} gelesen? 
\item Haben Sie \textit{Ausbesserungsarbeiten} oder \textit{Ausbesserungarbeiten} gelesen?
\end{enumerate}

\largerpage[-2]
\section{Produktionsexperiment}
\label{prodpaced}
Folgende Sätze wurden genutzt. Weitere Informationen zum Produktionsexperiment sind im digitalen Anhang.

\ea Viele \_\_\_ (Taff) arbeiteten in Dörfern. 
\ex Der Beruf des \_\_\_ (Knatt) diente dem Feuerschutz.
\ex Die meisten  \_\_\_ (Schirr) wurden als Gräfin angesprochen.
\ex Das Amt des \_\_\_(Schettose) war am ungarischen Königshof üblich.
\ex Die meisten \_\_\_ (Bracht) arbeiteten in Schneidereien.
\ex Das Amt des \_\_\_ (Grettel) ist in Heimatvereinen üblich.
\ex Zunächst kam \_\_\_ (Truntake) üblicherweise keine Urteilskraft zu.
\z

\chapter{Assoziationstest zur self-paced-reading-Studie}
\label{asso}

Im Assozationstest wurden neben den in der Studie genutzten Testsubstantiven \textit{Schettose}, \textit{Truntake}, \textit{Knatt} und \textit{Grettel} als Alternativen \textit{Trilch} und \textit{Fletter} abgefragt, die ebenfalls \textcite{Kopcke.2000b} entnommen wurden.  Die Pseudosubstantive wurden mit Artikel (z.~B. \textit{der Grettel}) präsentiert, sodass das Genus der Wörter vorgegeben war. Die Daten wurden mithilfe von ausgedruckten Fragebögen erhoben. Die Stichprobe umfasst 90 Proband\_innen, Metadaten wurden nicht erhoben. Weitere Informationen sind im digitalen Anhang zu finden.

Für Tabelle \ref{kld} wurden alle Assoziationen berücksichtigt, die mehr als zehnmal genannt wurden. Hierbei wurde auch gezählt, wie häufig keine Assoziation aufgeschrieben wurde. Den Prozentzahlen liegt als Gesamtanzahl 270 (90 Proband\_innen * 3 Assoziationen) zugrunde, da jede\_r Proband\_in maximal drei Angaben machen konnte. Assoziierte Substantive wurden lexemweise einbezogen. Alle anderen Wortarten wurden unter der jeweiligen Wortart zusammengefasst, da diese die Deklinationsklasse nicht beeinflussen sollten. Die Anzahl der Assoziationen wurde ermittelt, indem die Nennungen einzelner Lexeme ausgezählt wurden. Die einzigen Ausnahmen hiervon bilden die Assoziationskategorien \textit{Betrunkener} sowie \textit{Gretel}: Für \textit{Betrunkener} wurden verschiedene Bezeichnungen zusammengefasst (\textit{Trunkenbold}, \textit{jemand, der getrunken hat}, \textit{Alkoholiker}), im Fall von \textit{Gretel} wurde auch \textit{Hänsel und Gretel} als Teil der Assoziationskategorie angesehen, jedoch nicht die Assoziationen wie \textit{Grimm} und \textit{Märchen}, da diese auf einer abstrakteren Ebene agieren. Bei \textit{Vetter} wurden vier Schreibungen mit \textit{F} (\textit{Fetter}) als diesem Lexem zugehörig interpretiert, da diese Lesart wahrscheinlicher ist als eine Konversion des Adjektivs \textit{fett}. 

\begin{table}
\begin{tabular}{ll ll}
\lsptoprule
Assoziation & \textit{Schettose}	& Assoziation	&	\textit{Truntake}\\\cmidrule(lr){1-2}\cmidrule(lr){3-4}
n.a. &	103	(38 \%	) & n.a. & 105	(62 \%) \cr
\textit{Steckdose} &	18	(7 \%)	& ADJ &	20	(12 \%) \cr
\textit{Matrose}	& 13	(5 \%) &	VV &	20	(12 \%) \cr
ADJ	& 12	(4 \%) &	\textit{Truthahn} &	16	(9 \%) \cr
\textit{Hose} &	12	(4 \%) &	\textit{Tentakel} &	12	(7 \%) \cr
\textit{Schotte} &	11	(4 \%) &	\textit{Betrunkener} &	12	(7 \%)\\\cmidrule(lr){1-2}\cmidrule(lr){3-4}
Assoziation & \textit{Trilch} &	Assoziation &		\textit{Knatt}\\\cmidrule(lr){1-2}\cmidrule(lr){3-4}
n.a.	& 81	(30 \%) &	n.a.	& 95	(35 \%)\cr	
\textit{Milch}	& 33	(12 \%) &	ADJ	& 35	(13 \%)\cr	
\textit{Knilch}	& 24	(9 \%) &	VV &	26	(10 \%)\cr	
\textit{Trichter}	& 20	(7 \%) &	\textit{Knast}	& 23	(9 \%) \cr	
ADJ	& 15	(6 \%)			& &\cr	
VV	& 11	(4 \%)			& &\\\cmidrule(lr){1-2}\cmidrule(lr){3-4}
Assoziation & \textit{Fletter} & Assoziation &			\textit{Grettel}\\\cmidrule(lr){1-2}\cmidrule(lr){3-4}
n.a. &	94	(35 \%) &	n.a. &	130	(48 \%) \cr
VV &	34	(13 \%) &	\textit{Gretel}	& 45	(17 \%)\cr
\textit{Vetter} &	18	(7 \%) &	\textit{Zettel} &	15	(6 \%)\cr
ADJ	& 14	(5 \%)	& ADJ	& 10	(4 \%)\cr
\textit{Fledermaus} &	13	(5 \%)	& & \cr		
\lspbottomrule
\end{tabular}
\captionof{table}{Überblick über die häufigsten Assoziationen mit den Pseudosubstantiven}
\label{kld}
\end{table}

Es ist festzuhalten, dass bei allen Pseudosubstantiven \textit{n.a.} die häufigste Antwortkategorie darstellt. Somit scheint keines der Pseudosubstantive starke intersubjektive Assoziationen auszulösen. \textit{Schettose} wird nur mit schwach flektierenden Substantiven (\textit{Matrose}, \textit{Schotte}) assoziiert. Da \textit{Schettose} dem Form-Schema I entspricht (siehe \sectref{schemamask}), sind die Assoziationen mit schwachen Maskulina nicht verwunderlich und für das Experiment eher zuträglich denn schädigend, da die Assoziationen die schwache Flexion unterstützen können. Die restlichen Assoziationen entfallen auf Feminina (\textit{Steckdose}, \textit{Hose}), welche die Deklination nicht beeinflussen sollten, da das Testitem als Maskulinum eingeführt wird. Die häufigste Assoziation (\textit{Steckdose}) rührt vermutlich daher, dass die Proband\_innen das Substantiv initial betonten (\textit{Schéttose}). Aufgrund dessen wurde in der self-paced-reading-Studie ein Hinweis auf die Betonung des Substantivs ergänzt. \textit{Truntake} scheint auf den ersten Blick nicht problematisch zu sein, allein die Assoziation \textit{Truthahn} könnte Probleme bereiten, da dies ein starkes Maskulinum ist. Zudem scheint \textit{Truntake} negativ konnotiert zu sein, wie die Assoziationen \textit{Betrunkener, Trunkenbold} zeigen. Aufgrund dieser negativen Assoziation wurde \textit{Schettose} als Testitem für die Genitivformen gewählt. Da für Substantive, die dem Form-Schema~I komplett entsprechen, zusätzlich zum Genitiv der Dativ  getestet wurde, wurde hierfür \textit{Truntake} als Testitem eingeflochten. Wie bei \textit{Schettose} wurde ein Hinweis zur Betonung ergänzt (\textit{Truntáke}), um die Assoziation mit schwachen Flexionsverhalten zu erleichtern.

Bei dem Pseudowortpaar \textit{Trilch} und \textit{Knatt} lassen sich zwei problematische Assoziationen feststellen: \textit{Trilch} wurde mit dem starken Maskulinum \textit{Knilch} assoziiert, \textit{Knatt} mit dem starken Maskulinum \textit{Knast}. Beide Assoziationen wurden ähnlich oft genannt, sodass die Wahl zwischen \textit{Trilch} und \textit{Knatt} nach semantischen Faktoren getroffen wurde. Da das Pseudowort im Experiment als Mensch eingeführt wird, ist bei \textit{Knatt} eine Assoziation mit \textit{Knast} unwahrscheinlich, während bei \textit{Trilch} eine Assoziation mit \textit{Knilch} durchaus möglich ist, da \textit{Knilch} auf eine menschliche Entität referiert. Aufgrund dessen wurde \textit{Knatt} als Testitem gewählt. Bei \textit{Fletter} ist die Assoziation mit \textit{Vetter} problematisch, da dieses Substantiv gemischt flektiert (\cite{Duden.2020}). Zwar werden im Experiment keine Pluralformen abgefragt, jedoch ist es möglich, dass sich bei \textit{Vetter} ähnlich wie bei \textit{Autor} aufgrund der Nähe zum Form-Schema schwache Formen im Singular finden lassen (siehe \sectref{schemamask}). Aufgrund dessen wurde \textit{Grettel} als Versuchitem gewählt. Die häufige Assoziation mit \textit{Gretel} ist unproblematisch, da es sich bei \textit{Grettel} um ein Femininum handelt. 


\chapter{Ergebnisse der Studien}
\label{ergebnissestudien}


\section{Ergebnisse der Frequenzstudie}
\label{ergebnissefreq}

In der Abbildung \ref{antwfrequentev} steht ein Rechteck für eine Antwort.

\begin{figure}[H]
\includegraphics[width=\textwidth]{figures/Anhang/antwortfrequent.png} 
\caption{Antwortverhalten bei den Testverben der Ausprägung \textsc{frequent}}
\label{antwfrequentev}
\end{figure}\pagebreak\largerpage[2]

\begin{figure}[H]
\includegraphics[width=.85\textwidth]{figures/Anhang/lengthfrequ.png} 
\caption{Reaktionszeiten im Abhängigkeit von der Testverblänge in der Frequenzstudie}
\label{lengthfrequ}
\end{figure}

\begin{figure}[H]
\includegraphics[width=.85\textwidth]{figures/Anhang/blockfrequ.png} 
\caption{Reaktionszeiten in Abhängigkeit vom Testblock in der Frequenzstudie}
\label{blockfrequ}
\end{figure}
\pagebreak
\begin{table}[h]
\begin{tabular}{lrr}
  \lsptoprule
	& schwach & stark \\
	\midrule
	infrequent o.S. & $ \mathbf{0{,}42}                $ & $ 0{,}42 - 0{,}33 = \mathbf{0{,}09} $\\
	frequent                   & $ 0{,}42 - 0{,}19 = \mathbf{0{,}23} $ & $ 0{,}42 - 0{,}33 - 0{,}19 + 0{,}06 = \mathbf{-0{,}04}$\\
	infrequent m.S.  & $ 0{,}42 - 0{,}04 = \mathbf{0{,}38} $ & $ 0{,}42 - 0{,}33 - 0{,}04 + 0{,}17 = \mathbf{0{,}22} $\\
	\lspbottomrule
\end{tabular}
\captionof{table}{Kreuztabelle der Werte des Modells für Reaktionszeiten mit Interaktion zwischen Frequenz und Flexion. o.S.: ohne Schwankung; m.S.: mit Schwankung.}
\label{rtfreqinterkr}
\end{table}

\section{Ergebnisse der Prototypizitätsstudie}
\label{ergebnisseproto}

\vfill
\begin{table}[H]
\begin{tabular}{lrrr}
  \lsptoprule
Kombinationsmöglichkeit & \textit{fahren} & \textit{fliegen} & \textit{reiten} \\ 
  \midrule
  hat\_ist\_ist\_ist & 16 (80 \%) & 18 (90 \%) & 3 (15 \%) \\
	hat\_hat\_ist\_ist & 4 (20 \%) & 1 (5 \%) & 7 (35 \%) \\
	ist\_ist\_ist\_ist & 0 (0 \%) & 1 (5 \%) & 3 (15 \%)\\
	ist\_hat\_ist\_ist &  0 (0 \%) & 0 (0 \%) & 3 (15 \%) \\ 
	ist\_hat\_hat\_ist &  0 (0 \%) & 0 (0 \%) & 2 (10 \%) \\ 
	hat\_hat\_hat\_ist &   0 (0 \%) & 0 (0 \%) & 1 (5 \%) \\  
 ist\_ist\_hat\_ist &  0 (0 \%) & 0 (0 \%) & 1 (5 \%) \\ 
\midrule
gesamt & 20 (100 \%) & 20 (100 \%) & 20 (100 \%)\\
   \lspbottomrule
\end{tabular}
\captionof{table}{Häufigkeit der Auxiliarkombinationen für die Abfolge \textsc{transitiv}, \textsc{ambig~I}, \textsc{ambig~II}, \textsc{intransitiv} bei den von der Stichprobe ausgeschlossenen Proband\_innen}
\label{antwortmusteraus}
\end{table}
\vfill\pagebreak

\begin{figure}
\includegraphics[width= 0.85 \textwidth]{figures/Anhang/blockprot.png} 
\captionof{figure}{Reaktionszeiten in Abhängigkeit vom Testblock in der Prototypizitätsstudie}
\label{blockprot}
\end{figure}

\section{Ergebnisse der Form-Schematizitätsstudien}
\label{ergebnisseschema}

\subsection{Lexical-decision-Studie}
\vfill
\begin{table}[H]
\begin{tabular}{lrr}
\lsptoprule
		& schwach & stark \\ 
\midrule
Form-Schema & $ \mathbf{-3{,}28}                 $ & $-3{,}28 + 5{,}77 = \mathbf{2{,}49} $\\
Peripherie  & $ -3{,}28 - 0{,}38 = \mathbf{-3{,}66}  $ & $-3{,}28 + 5{,}77 - 0{,}38 - 2{,}29 = \mathbf{-0{,}18}$\\
stark       & $ -3{,}28 + 6{,}19 = \mathbf{2{,}91}   $ & $-3{,}28 + 5{,}77 + 6{,}19 - 11{,}23 = \mathbf{-2{,}55}$\\
\lspbottomrule
\end{tabular} 
\captionof{table}{Kreuztabelle der Werte des Modells für die Unbekanntheit starker und schwacher Formen in der lexical-decision-Studie zu Form-Schematizität}
\label{schemadecergkreuz}
\end{table}
\vfill\pagebreak\largerpage

\begin{figure}[H]
\includegraphics[width=0.85\textwidth]{figures/Anhang/lengthschemadec.png} 
\captionof{figure}{Reaktionszeiten im Abhängigkeit von der Testitemlänge in der lexical-decision-Studie zu Form-Schematizität}
\label{lengthschemadec}
\end{figure}

\begin{figure}[H]
\includegraphics[width=0.85\textwidth]{figures/Anhang/blockschemadec.png} 
\captionof{figure}{Reaktionszeiten in Abhängigkeit vom Testblock in der lexical-decision-Studie zu Form-Schematizität}
\label{blockschemadec}
\end{figure}
\pagebreak
\hbox{}\vfill
\begin{table}[H]
\begin{tabular}{lrr}
\lsptoprule
&				schwach & stark \\
\midrule
Form-Schema & $ \mathbf{0{,}28}               $   & $0{,}28 + 0{,}2 = \mathbf{0{,}48} $\\
Peripherie  & $ 0{,}28  - 0{,}07 = \mathbf{0{,}21}$   & $0{,}28 + 0{,}2 - 0{,}07 + 0{,}09 = \mathbf{0{,}50} $\\
stark       & $ 0{,}28  + 0{,}25 = \mathbf{0{,}53}$   & $0{,}28 + 0{,}2 + 0{,}25 - 0{,}46 = \mathbf{0{,}27}$ \\
\lspbottomrule
 \midrule
\end{tabular}
\captionof{table}{Kreuztabelle der Werte des Modells für die Reaktionszeiten der lexical-decision-Studie zu Form-Schematizität}
\label{kreuzergschemadecrt}
\end{table}
\vfill
\begin{table}[H]
 \begin{tabular}{lrr}
\lsptoprule
  &schwach	& stark\\
  \midrule
frequent         & $ \mathbf{0{,}26}                   $ & $ 0{,}26 + 0{,}23 = \mathbf{0{,}49} $\\
mittelfrequent   & $ 0{,}26 + 0{,}12 = \mathbf{0{,}38} $ & $ 0{,}26 + 0{,}23 + 0{,}12 - 0{,}19 = \mathbf{0{,}42}$\\
infrequent       & $ 0{,}26 + 0{,}12 = \mathbf{0{,}38} $ & $ 0{,}26 + 0{,}23 + 0{,}12 - 0{,}27 = \mathbf{0{,}34}$\\
\lspbottomrule
\end{tabular}
\captionof{table}{Kreuztabelle der Werte des Modells für die Reaktionszeiten in der lexical-decision-Studie zu Form-Schematizität in Abhängigkeit von Frequenz}
\label{schemadecfreqkreuz}
\end{table}
\vfill\pagebreak

\subsection{Sentence-maze-Studie}
In den Abbildungen \ref{antwortschemasentm} und \ref{antwortstarksentm} steht ein Rechteck für zwei Antworten.

\begin{figure}[h]
\includegraphics[width=\textwidth]{figures/Anhang/antwortschemasent.png} 
\captionof{figure}{Antwortverhalten bei den Testsubstantiven der Ausprägung \textsc{Form-Schema} in der sentence-maze-Studie zu Form-Schematizität}
\label{antwortschemasentm}
\end{figure}


\begin{figure}[p]
\includegraphics[width=\textwidth]{figures/Anhang/antwortstarksent.png} 
\captionof{figure}{Antwortverhalten bei den Testsubstantiven der Ausprägung \textsc{stark} in der sentence-maze-Studie zu Form-Schematizität}
\label{antwortstarksentm}
\end{figure}


\begin{figure}[p]
\includegraphics[width=\textwidth]{figures/Anhang/blockschemasent.png} 
\captionof{figure}{Reaktionszeiten in Abhängigkeit vom Testblock in der sentence-maze-Studie zu Form-Schematizität}
\label{blockschemasent}
\end{figure}


\clearpage
\subsection{Self-paced-reading-Studie}\largerpage[2]

\begin{figure}[H]
\includegraphics[width=.75\textwidth]{figures/Anhang/loglesnach.png} 
\captionof{figure}{Lesezeiten in Abhängigkeit von dem Element nach dem Testitem in der self-paced-reading-Studie (\textit{stark} und \textit{schwach} beziehen sich auf die Flexionsform des Testitems)}
\label{loglesnach}
\end{figure}

\begin{figure}[H]
\includegraphics[width=.75\textwidth]{figures/Anhang/loglesfrage.png} 
\captionof{figure}{Reaktionszeiten in Abhängigkeit von den Fragen zu den Testitems in der self-paced-reading-Studie}
\label{loglesfrage}
\end{figure}\pagebreak
\hbox{}
\vfill
\begin{figure}[H]
\includegraphics[width=.75\textwidth]{figures/Anhang/loglescondition.png} 
\captionof{figure}{Lesezeiten in Abhängigkeit von der Reihenfolge der Testitems in der self-paced-reading-Studie}
\label{loglescondition}
\end{figure}

\begin{figure}[H]
\includegraphics[width=.75\textwidth]{figures/Anhang/knattles.png} 
\captionof{figure}{Lesezeiten in Abhängigkeit von starken und schwachen Formen bei \textit{Knatt}}
\label{knattles}
\end{figure}
\pagebreak

Wie Abbildung \ref{knattles} zeigt, evozieren beide Formen vergleichbare Lesezeiten, sodass ein Einfluss der starken und schwachen Form sowie der Länge des Testsubstantivs ausgeschlossen werden kann. Würde die Länge der Formen eine Rolle spielen, müsste die starke Form (\textit{des Knatts}) zu niedrigeren Lesezeiten neigen, da sie um einen Buchstaben kürzer sind als die schwache (\textit{des Knatten}).

\begin{figure}[H]
\includegraphics[width=.75\textwidth]{figures/Anhang/truntakeles.png} 
\captionof{figure}{Lesezeiten in Abhängigkeit von starken und schwachen Formen bei \textit{Truntake}}
\label{truntakeles}
\end{figure}

Wie Abbildung \ref{truntakeles} zeigt, streut die starke Form stärker zwischen niedrigeren und höheren Lesezeiten als die schwache Form. Die kürzeren Lesezeiten bei der starken Form könnten einen Längeneffekt darstellen, da die starke Form (\textit{dem Truntake}) um einen Buchstaben kürzer ist als die schwache (\textit{dem Truntaken}). Umgekehrt könnten die erhöhten Lesezeiten bei der starken Form einen Effekt der Flexionsform darstellen, da die starke Form nicht erwartbar ist. 

\begin{table}[H]
\begin{tabular}{lrr}
  \lsptoprule
			&		schwach	&	stark \\
			\midrule
Schettose & $ \mathbf{0{,}12} $               & $ 0{,}12 +0{,}18 = \mathbf{0{,}40} $ \\
Truntake  & $ 0{,}12 + 0{,}17 = \mathbf{0{,}29} $ & $ 0{,}12 + 0{,}18 +0{,}17 -0{,}19 = \mathbf{0{,}28} $\\
Knatt     & $ 0{,}12 -0{,}08 = \mathbf{0{,}04}  $ & $ 0{,}12 +0{,}18 -0{,}08  -0{,}25 = \mathbf{-0{,}03}$\\
Grettel   & $ 0{,}12 -0{,}11 = \mathbf{0{,}01}  $ & $ 0{,}12 +0{,}18 -0{,}11  - 0{,}22 = \mathbf{-0{,}03}$ \\
\lspbottomrule
\end{tabular}
\captionof{table}{Kreuztabelle der Werte des Modells für die Lesezeiten in der self-paced-reading-Studie}
\label{ergselfkreuz}
\end{table}

\section[Vergleich zwischen den Reaktionszeiten der Verben und der Substantive]{Vergleich zwischen den Reaktionszeiten der Verben und der Substantive in der lexical-decision-Studie}
\label{vgl}

\begin{table}\small
\begin{tabular}{lrrrr}
\lsptoprule
					&				\multicolumn{2}{c}{Substantiv}&  \multicolumn{2}{c}{Verb}\\\cmidrule(lr){2-3}\cmidrule(lr){4-5}
						&			schwach  & stark									& schwach  & stark  \\
\midrule
&&&& \\
frequent/   & \textbf{0,24}   		& 0,24 + 0,22 & 0,24 − 0,01 & 0,24 − 0,01 + 0,22  \\
Form-Schema&& = \textbf{0,46}	 & = \textbf{0,23}	 & − 0,49  \\
&&&& = \textbf{-0,04} \\
&&&& \\
ohne/  & 0,24 − 0,05  & 0,24 + 0,22   & 0,24 − 0,01  	& 0,24 − 0,01 + 0,22   \\
Peripherie& = \textbf{0,19} & − 0,05 + 0,06 & − 0,05 + 0,25& − 0,05 + 0,25 − 0,49 \\
&&  = \textbf{0,47} &  = \textbf{0,43} &  + 0,06 − 0,12 \\
&&&& = \textbf{0,1}  \\
&&&& \\
mit/stark	& 0,24 + 0,26  &  0,24 + 0,22  	& 0,24 − 0,01  		& 0,24 − 0,01 + 0,22  \\
& = \textbf{0,5} & + 0,26  − 0,48  & + 0,26 − 0,10  & + 0,26 − 0,10 − 0,49\\
&& = \textbf{0,24} & = \textbf{0,39} &   − 0,48 + 0,59 \\
&&&& = \textbf{0,23}\\
\lspbottomrule
\end{tabular}
\captionof{table}{Kreuztabelle der Werte des Modells für die Reaktionszeiten bei Substantiven und Verben in den Studien zu Form-Sche\-ma\-ti\-zi\-tät und Frequenz}
\label{vglwapredkreuz}
\end{table}
