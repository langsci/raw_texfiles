\chapter{Psycholinguistische Betrachtung der Einflussfaktoren: Methode}\label{psych}

Der Einfluss von Frequenz, Prototypizität und Form-Schematizität auf Variation wird anhand von Reaktionszeiten untersucht, indem überprüft wird, inwiefern eine Manipulation der Einflussfaktoren Reaktionszeiten verändert. In diesem Kapitel wird das methodische Vorgehen für die einzelnen Reaktionszeitstudien vorgestellt. Hierfür wird in \sectref{reaktion} zunächst allgemein auf die Verfahren eingegangen, mithilfe derer die Reaktionszeiten in den Studien erhoben wurden. Im Anschluss daran wird das methodische Vorgehen in den Studien zum Einfluss der Frequenz (\sectref{metfreq}), der Prototypizität (\sectref{metproto}) und der Form-Schematizität (\sectref{methschema} und \ref{schemalex}) auf Reaktionszeiten beschrieben. Der Einfluss der Frequenz wird mithilfe von starken Verben untersucht, der Einfluss der Prototypizität mithilfe der Selektion von \textit{haben} und \textit{sein} und der Einfluss der Form-Schematizität mithilfe des Form-Schemas schwacher Maskulina. Mit dem Einfluss der Form-Schematizität wird auch der Einfluss der Prototypizität überprüft, da Form-Schemata prototypisch organisiert sind (siehe \sectref{konstruktion}). 

\section{Reaktionsmessverfahren} 
\label{reaktion}

In den hier vorgestellten Studien wird mit drei verschiedenen Verfahren gearbeitet, um Reaktionszeiten zu evozieren: \textit{lexical decision}, \textit{self-paced reading} und \textit{sentence maze tasks}. Der Einfluss der Frequenz auf Reaktionszeiten wird anhand einer \textit{lexical decision task} überprüft, der Einfluss der Prototypizität anhand einer \textit{sentence maze task}. Der Einfluss der Form-Schematizität wird mithilfe aller hier vorgestellten Verfahren überprüft. Im Folgenden werden die Verfahren näher erläutert.

\subsection{\textit{Lexical decision tasks}}
\label{lexdectask}

In \textit{lexical decision tasks} werden den Proband\_innen Wörter und Pseudowörter präsentiert. Sie erhalten die Aufgabe, möglichst schnell und präzise zu entscheiden, ob der Stimulus ein Wort ist oder nicht (\cite[95]{Baayen.2014}). Die Stimuli werden visuell an einem Bildschirm oder auditiv über Lautsprecher präsentiert. Vor den visuellen Stimuli erscheint i.~d.~R. ein Fixationskreuz, um die Aufmerksamkeit der Proband\_innen auf die Mitte des Bildschirms zu lenken. Dabei wird die Reaktionszeit zwischen Stimuluspräsentation und Antwort gemessen und die Antworten der Proband\_innen werden gespeichert. Auf diese Weise kann ermittelt werden, wie präzise die Proband\_innen antworten, d.~h. wie groß der Anteil der Stimuli ist, die korrekt als Pseudowort bzw. korrekt als existierendes Wort eingeschätzt wurden. Die Reaktionszeiten geben bspw. Aufschluss über Worterkennungseffekte: So hängt die Schnelligkeit, mit der Pseudowörter abgelehnt werden, davon ab, wie ähnlich sie real existierenden Wörtern sind (\cite{Barca.2012}).  



Die Reaktionszeiten aus \textit{lexical decision tasks} wurden meines Wissens noch nicht zur Untersuchung von Variation genutzt. Sie eignen sich aber dafür, da Wortformen wie bspw. *\textit{gezieht} vs. \textit{gezogen} und \textit{geglimmt} vs. \textit{geglommen} gezielt kontrastiert werden können. Zudem weisen bisherige Studien darauf hin, dass der Einfluss von Frequenz und Form-Schematizität auf Reaktionszeiten durch \textit{lexical decision tasks} erfasst werden kann. Frequente Wörter werden in \textit{lexical decision tasks} schneller erkannt als infrequente (\cite[489--491]{Rubenstein.1970}, \cite[150--152]{Whaley.1978}, siehe hierzu auch \sectref{korrelation}). Zudem sind Proband\_innen in \textit{lexical decision tasks} für phonologische Eigenschaften der Stimuli sensibel (\cite[6--7]{Keuleers.2010}), sodass auch Form-Schematizität\footnote{Der Einfluss der Prototypizität wird mithilfe eines syntaktischen Variationsphänomens überprüft, daher eignet sich die \textit{lexical decision task} hierfür nicht. Da Prototypizität aber auch innerhalb der Form-Schematizität eine Rolle spielt, ist der Einflussfaktor nicht völlig ausgeklammert.} mithilfe der Methode erfassbar zu sein scheint. 

Die \textit{lexical decision task} ist eine der am häufigsten genutzten Methoden in der Psycho\-linguistik (\cite[1]{Diependaele.2012}, \cite[95]{Baayen.2014}). Dies lässt sich vorrangig mit der Praktikabilität der Methode erklären: Sie ist leicht zu implementieren und durchzuführen (\cite[96]{Baayen.2014}). Die erhobenen Daten sind zudem gut zu interpretieren und die Aufgabe kann leicht vermittelt werden.\footnote{Außerdem bieten sich \textit{lexical decision tasks} für Priming-Experimente an, weil vor den Stimulus ein Prime geschaltet werden kann (\cite[46]{Jones.2012}).}  


\textcite[96]{Baayen.2014} weist darauf hin, dass \textit{lexical decision tasks} keinen Einblick in den zeitlichen Verlauf von Prozessierung geben können, da sie nur die späten Phasen der Prozessierung messen. Ein weiterer Nachteil ist, dass Stimuli kontextlos präsentiert werden (\cite[96]{Baayen.2014}). Zudem wird die Beurteilungskomponente, also die Beurteilung der Bekanntheit des Stimulus,  als problematisch gesehen (\cite[30--31]{Haberlandt.1994}). Die Kritik fußt darauf, dass sowohl die Kontextlosigkeit als auch die Beurteilung einzelner Wörter kein Pendant im alltäglichen Sprachgebrauch haben. Dies führt laut \textcite[96]{Baayen.2014} bspw. dazu, dass die Reaktionszeiten in lexical-decision-Studien und die Länge der Fixationszeiten in Eye\-tracking-Studien zwar korrelieren, aber diese Korrelation weniger stark ausfällt als die Korrelation der Reaktionszeiten zwischen lexical-decision- und word-naming-Studien, die ebenfalls kontextlos sind (\cite[21]{Kuperman.2013}).\footnote{
\ Ein weiteres Problem kann sich durch die Konstruktion der Pseudowortstimuli ergeben: Pseudowörter, die mit einem komplexen Verfahren erstellt wurden (bspw. durch Rekombinieren von Silben) erzielen in einer Studie von \textcite[6--7]{Keuleers.2010} geringere Reaktionszeiten als Pseudowörter, die durch Änderung eines Buchstabens erstellt wurden, da das Rekombinieren von Silben zu Pseudowörtern führt, die real existierenden Wörtern ähnlich sind. Ähnliche Effekte wurden bereits in \sectref{korrelation} diskutiert: Sie weisen auf exemplarbasierte Cluster und damit auf eine generelle Eigenschaft der Prozessierung von Sprache hin und können daher m.~E. nicht als spezielles Problem der Methode gesehen werden. Zudem kann der Einfluss durch ein gutes Design der Stimuli minimiert werden.} 



Ein weiterer Kritikpunkt an \textit{lexical decision tasks} ist die Annahme eines konsistenten Antwortverhaltens einzelner Personen. Ein Stimulus, der abgelehnt wurde, sollte also auch bei mehrmaligem Bewerten abgelehnt werden. Dies ist jedoch laut \textcite{Diependaele.2012} nicht der Fall: Bewerten Proband\_innen die Stimuli in einer Aufgabe doppelt, ist die Bewertung nicht zwingend konsistent (\cite[5--8]{Diependaele.2012}). Dies gilt insbesondere für Substantive mit geringer Frequenz. \textcite[8]{Diependaele.2012} konkludieren, dass \textit{lexical decision} eine "`noisy task"' ist, wenn man von einem konstanten Antwortverhalten ausgeht. Sie empfehlen, sicherzustellen, dass die Proband\_innen die real existierenden Wörter kennen und Antworten mit Vorsicht zu behandeln, die aus einer Kondition stammen, bei der mehr als 10 \% der Stimuli falsch beurteilt wurden.  

 \textcite[117]{Baayen.2014} sieht wegen der genannten Kritikpunkte \textit{lexical decision tasks} eher als metalinguistische Beurteilung denn als Einblick in die Prozessierung. Wie er selbst schreibt, erlauben sie aber durchaus Rückschlüsse auf Prozessierung (\cite[96]{Baayen.2014}). Die Schnelligkeit, mit  der die Aufgabe erfüllt wird, gibt Aufschluss darüber, wie schnell die mentale Repräsentation von Wörtern aktiviert werden kann. Da die Reaktionszeit durch isoliert präsentierte Stimuli evoziert wird, ist der gewonnene Einblick in die Prozessierung zwar weniger detailliert als bei anderen Methoden, dennoch kann er genutzt werden, um Rückschlüsse auf Prozessierungunterschiede zwischen Stimuli zu ziehen. Man sollte sich bei der Datenauswertung über die eingeschränkte Interpretierbarkeit der Reaktionszeiten in \textit{lexical decision tasks} bewusst sein. Allerdings ist diese Notwendigkeit nicht auf \textit{lexical decision tasks} beschränkt, sondern gilt für jede Methode. 



Generell kritisiert \textcite[99]{Baayen.2014} die Auswahl der Stimuli für psycholinguistische Experimente: Wenn bspw. Testwörter allein anhand ihrer Tokenfrequenz in verschiedene Gruppen geteilt werden, kann dies problematisch sein, da Einflussfaktoren ignoriert werden, die mit Frequenz einhergehen. So haben frequente Stimuli i.~d.~R. breitere Verwendungskontexte und eine breitere Semantik als infrequente (siehe \sectref{korrelation} für weitere Erläuterungen). Zudem ist es aus statistischer Sicht problematisch, dass durch die Auswahl der Stimuli keine zufällige Stichprobe gezogen wird (\cite[99]{Baayen.2014}). Dieses Problem stellt sich jedoch nicht nur für \textit{lexical decision tasks}, sondern generell für Designs von Experimenten. 


Der Einfluss von Frequenz und Form-Schematizität wird mithilfe einer \textit{lexical decision task} überprüft (siehe \sectref{metfreq} und \ref{schemalex} für Erläuterungen zum Studiendesign). Die beschriebenen Probleme der \textit{lexical decision task} gelten nur eingeschränkt für die Studien. In den Studien werden jeweils zwei korrespondierende Formen gegenübergestellt (siehe \sectref{metfreq} sowie \ref{methschema}), bspw. starke und schwache Partizip-II-Formen von \textit{ziehen}. Zwar wird nicht ein und derselbe Stimulus wiederholt, dennoch lässt sich überprüfen, inwiefern die Antworten konsistent sind (siehe \sectref{freqant} und \ref{ergschemadecant}). Dabei wird angenommen, dass die Antworten bei Varianten weniger konsistent sind als bei Wörtern, bei denen nur eine Form mental gefestigt ist (siehe \sectref{freqhyp} sowie \ref{hyposchema}). Zudem werden Proband\_innen ausgeschlossen, die mehr als 10~\% der Stimuli, die die Konzentration der Proband\_innen evaluieren, falsch beantworten. Die Kontextlosigkeit bleibt eine Einschränkung des Studiendesigns. Um die Validität der Ergebnisse überprüfen zu können, wird der Einfluss der Form-Schematizität nicht nur durch eine \textit{lexical decision task}, sondern auch durch eine \textit{self-paced reading}- und eine \textit{sentence maze task} überprüft.

\subsection{\textit{Self-paced reading tasks}}
\label{selfpaced}

\textit{Self-paced reading tasks} werden genutzt, um Lesezeiten zu evozieren: Bei dieser Methode ist der zu lesende Text zunächst verdeckt (bspw. durch Rauten repräsentiert) und wird erst per Tastendruck von Proband\_innen Stück für Stück aufgedeckt (\cite[118]{McDonough.2012}). Wie \textit{lexical decision tasks} wurde auch \textit{self-paced reading} meines Wissens noch nicht genutzt, um Variation zu untersuchen. Durch die Segmentierung erlaubt es die Methode, die Lesezeiten von Wortformen (z.~B. \textit{des Kollegen} und *\textit{des Kolleges}) zu vergleichen. Auf diese Weise kann überprüft werden, ob Reaktionszeitunterschiede zwischen Varianten und Formen, die keine Varianten darstellen, bestehen: Überraschende Formen werden langsamer gelesen als im Kontext erwartbare, siehe bspw. \textcite{Tabor.2000} mit einer Studie zu Holzwegsätzen. Bei einer nicht zu erwartenden Form wie *\textit{des Kolleges} sind also höhere Reaktionszeiten zu erwarten als für \textit{des Kollegen}. Stellen die beiden Wortformen hingegen Varianten dar (z.B. Genitivvarianten wie bei \textit{des Grafs/des Grafen}), ist davon auszugehen, dass sich keine Unterschiede in der Lesezeit ergeben: Hier koexistieren zwei Formen mit gleicher Funktion, somit sind beide Formen im Kontext erwartbar. 

\textit{Self-paced reading tasks} wurden in den 1970er Jahren entwickelt und sind inzwischen in der Psycholinguistik etabliert (\cite[18--19]{Mitchel.2013}, \cite[20--21]{Jegerski.2014}). Sie können auf verschiedene Weisen gestaltet sein (für einen Überblick siehe \cite[118--119]{McDonough.2012}, \cite[18--19]{Mitchel.2013}). In der vorliegenden Studie wird mit einem \textit{moving window}, d.~h. mit einem sich bewegenden Sichtfenster, gearbeitet: Der zu lesende Text ist zunächst durch Rauten verdeckt und wird durch Tastendruck Stück für Stück aufgedeckt. Ein aufgedecktes Element bleibt aber nicht die ganze Zeit sichtbar, sondern wird beim nächsten Tastendruck wieder verdeckt. Beispiel~\ref{beispiel1} gibt einen Eindruck in das Lesen von \textit{self-paced reading} mit \textit{moving window}:


\begin{exe} 
\ex \label{beispiel1} 
\ \ \ \ \ \ \ \ \ \ \ \ \ \ \ \ \ \ \ \ \ \ \ \ \ {\footnotesize{\#\#\#\#\#\#\#\#\#\#\#\#\#\#\#\#\#\#\#\#\#\#\#\#\#\#\#\#\#\#}} \\
1. Tastendruck \                          {\footnotesize{Die Katzen \#\#\#\#\#\#\#\#\#\#\#\#\#\#\#\#\#\#\# }}\\
2. Tastendruck \                          {\footnotesize{\#\#\#\#\#\#\#\#\# sitzen \#\#\#\#\#\#\#\#\#\#\#\#\#\# }} \\
3. Tastendruck \  												{\footnotesize{\#\#\#\#\#\#\#\#\#\#\#\#\#\#\# auf dem Baum.}} 
\end{exe}

Die Zeit zwischen zwei Tastaturanschlägen wird gemessen. Da während des Lesens jeweils nur ein Element aufgedeckt ist, wird die Zeit zwischen zwei Tastaturanschlägen mit der Lesezeit des jeweiligen Elements gleichgesetzt (\cite[118]{McDonough.2012}, \cite[21--22]{Jegerski.2014}).\footnote{Diese Annahme ist problematisch (\cite[349]{Forster.2010}), wie sich auch in den späteren Ausführungen dieses Abschnitts zeigen wird.} Um sicherzustellen, dass Proband\_innen den Text konzentriert lesen, werden in regelmäßigen Abständen Verständnisfragen gestellt (\cite[349]{Forster.2010}). 

Die Methode erlaubt es, die Lesezeiten verschiedener sprachlicher Strukturen zu messen und miteinander zu vergleichen. Sie ist zudem kostengünstig, effizient, leicht zu implementieren und nicht an einen bestimmten Raum gebunden, da außer einem Computer (und ggf. einem transportablen Response Pad, das eine genauere Reaktionszeitauflösung ermöglicht als herkömmliche Tastaturen) keine technischen Anforderungen gestellt werden (\cite[24]{Mitchel.2013}, \cite[43]{Jegerski.2014}). Wie \textcite{Keller.2009} zeigen, kann \textit{self-paced reading} auch ohne Verzerrungen der Ergebnisse online implementiert werden. 



Der Nachteil der Methode ist, dass sie stark in die Lesegewohnheiten von Proband\_innen eingreift: Anders als bei Eyetracking-Studien müssen Proband\_innen vorgefertigte Einheiten lesen. Zudem erlauben \textit{self-paced reading tasks} keine Regressionen (\cite[44]{Jegerski.2014}).\footnote{Neben diesen Eigenschaften wurde das konstante Tasturanschlagen der Proband\_innen als unnatürlicher Faktor diskutiert (\cite[44]{Jegerski.2014}). Dieser Faktor wurde insbesondere in Kontrast zu gedruckten Texten stark gemacht und ist nach \textcite[44]{Jegerski.2014} heute zu vernachlässigen, da Lesen an Bildschirmen an Normalität gewonnen hat.}  Aufgrund dieser Unnatürlichkeit könnte \textit{self-paced reading} einen zusätzlichen Prozessierungsaufwand generieren (\cite[44]{Jegerski.2014}). Für diese Hypothese spricht, dass die gemessenen Lesezeiten bei \textit{self-paced reading} verlangsamt sind (\cite[25]{Mitchel.2013}). Die Methode ist daher nur für geübte Leser\_innen geeignet (\cite[43]{Jegerski.2014}).

 \textcite[107--108]{Witzel.2012} sehen einen Vorteil in der Segmentierung in \textit{self-paced reading tasks}: Hierdurch werden individuelle Lesestrategien weitgehend blockiert, weswegen standardisiertere Ergebnisse als bei Eyetracking-Studien erzielt werden können. Allerdings weisen \textcite[107--108]{Witzel.2012} darauf hin, dass auch bei \textit{self-paced reading} individuelle Lesestrategien zum Tragen kommen können wie bspw. \textit{tapping}, bei dem Proband\_innen mit konstant geringer Reaktionszeit durch die Elemente klicken (\cite[349]{Forster.2010}). Die Segmentierung selbst kann aber auch Einfluss auf die Lesezeit nehmen: So ist davon auszugehen, dass Segmente, die keine Phrasen darstellen wie bspw. \textit{am Eingang der}, schwieriger zu prozessieren sind und somit höhere Lesezeiten evozieren als Segmente, die Phrasen darstellen wie \textit{am Eingang der Universität}. \textcite[25--26]{Mitchel.2013} konstatiert allerdings, dass ihm keine Fälle bekannt seien, in denen die gemessenen Effekte eindeutig allein auf die Segmentierung innerhalb des Experiments zurückzuführen sind. 



\textcite[349]{Forster.2010} problematisiert die Annahme, dass Tastaturanschlag und Verarbeitung synchron verlaufen: Proband\_innen von \textit{self-paced reading tasks} berichten bspw., dass sie in einem konstanten Tempo durch das Experiment klicken und bei Prozessierungsproblemen das Tempo verringern. Die Anpassung des Tempos könnte somit erst nach dem Element passieren, das die Probleme ausgelöst hat. Aber auch unabhängig vom individuellen Verhalten von Pro\-\mbox{band\_in}\-nen können  spill-over-Effekte entstehen: Die Prozessierung eines Element ist nicht zwingend abgeschlossen, wenn das nächste Element aufgedeckt wird. Somit kann es sein, dass erhöhte Lesezeiten erst nach dem Element auftreten, das die erhöhten Lesezeiten verursacht hat (\cite[24--25]{Mitchel.2013}). Spill-over-Effekte sind jedoch nicht nur in self-paced-reading-Studien zu beobachten, sondern auch in Eyetracking- und EKP-Studien. \textcite[25]{Mitchel.2013} sieht sie daher als Teil der Satzprozessierung. In self-paced-reading-Studien können spill-over-Effekte verringert werden, indem die Länge der Elemente vor den Testitems konstant gehalten wird und Wortgruppen statt einzelner Wörter angezeigt werden (\cite[25]{Mitchel.2013}).   

\textcite[25]{Mitchel.2013} weist darauf hin, dass die Ergebnisse von self-paced-reading-Studien in Eyetracking-Studien repliziert werden können. Die Methode scheint also generell geeignet zu sein, um Unterschiede in Lesezeiten festzustellen (\cite[25]{Mitchel.2013}). Auch \textcite[123--126]{Witzel.2012} zeigen, dass \textit{self-paced reading tasks} und Eyetracking-Studien generell ähnliche Ergebnisse erzielen:  In ihrer Studie vergleichen sie unter anderem die Ergebnisse von \textit{self-paced reading tasks} und \textit{eyetracking} anhand von drei verschiedenen syntaktischen Phänomene, bei denen Prozessierungsschwierigkeiten zu erwarten sind. Dies waren bspw. Ambiguitäten, die durch Koordination ausgelöst und erst im Verlauf des Satzes aufgelöst werden: Im Satz \textit{The robber shot the jeweler and the salesman reported the crime to the police} könnte \textit{the salesmann} ein koordiniertes Objekt oder Subjekt eines koordinierten Satzes sein. Die Ambiguität wird erst bei \textit{reported} aufgelöst (\cite[111]{Witzel.2012}). Dieser Effekt wird vermieden, wenn ein Komma vor \textit{and} gesetzt wird (\textit{The robber shot the jeweler, and the salesman reported the crime to the police}). Es ist daher zu erwarten, dass bei der ambigen Struktur längere Reaktions- bzw. Fixationszeiten zu messen sind als bei dem desambiguierten Satz. Dies war nur bei \textit{eyetracking} zu beobachten, die Sätze mit und ohne Ambiguität lösten bei \textit{self-paced reading} aber keine Unterschiede in der Lesezeit aus. Die anderen in der Studie untersuchten syntaktischen Phänomene\footnote{Dies waren Sätze, die Relativsätze und Adverbiale mit ambigem Skopus enthielten. Der Relativsatz kann sich, wenn er eingeleitet wird, entweder auf das Subjekt oder auf das Attribut zum Subjekt beziehen (\textit{The son of the actress who shot herself/himself on the set was under investigation}), die Ambiguität wird dann durch das Reflexivpronomen \textit{himself/herself} aufgelöst. Das Adverbial kann sich auf zwei Verbalphrasen im Satz beziehen, hierbei desambiguiert das Adverb durch den Zeitbezug (\textit{Susan bought the wine she will drink next week/this week, but she didn't buy any cheese}). Geringere Reaktions- und Fixationszeiten werden dabei von dem Relativsatz \textit{who shot himself} mit Bezug auf das Subjekt \textit{the son of the actress} bzw. von dem Adverbial \textit{next week} mit Bezug auf die zweite Verbalphrase \textit{she will drink} ausgelöst (\cite[115--120]{Witzel.2012}).} zeigten hingegen sowohl bei \textit{eyetracking} als auch bei \textit{self-paced reading} dieselben Ergebnisse.    

Der Einfluss der  Form-Schematizität  auf Reaktionszeiten soll mithilfe einer \textit{self-paced reading task} überprüft werden (siehe \sectref{methschema}). Hierfür sind die Nachteile von \textit{self-paced reading} m.~E. nicht schwerwiegend: Nur Lesezeiten einzelner Items sind relevant, sodass davon auszugehen ist, dass der Eingriff in die Lesegewohnheiten keine großen Auswirkungen auf die Ergebnisse hat. Zudem werden Lesezeiten miteinander verglichen, die in demselben unnatürlichen Setting gemessen wurden. Eventuelle Einflüsse des Settings lassen sich also bei jedem gemessenen Item finden, sodass Unterschiede zwischen den Items nicht auf das Untersuchungsdesign zurückzuführen sein sollten. Dies bedeutet, dass die Lesezeiten nur in Bezug auf eine relativ unnatürliche Lesesituation interpretiert werden können. Spill-over-Effekte werden vermieden, indem der Text in Wortgruppen statt einzelne Wörter segmentiert wird. Der Gefahr einer Verzerrung der Lesezeiten durch Segmentierung wird begegnet, indem eine möglichst unauffällige Einteilung der Sätze vorgenommen wird, die sich an Phrasenstrukturen orientiert. Die Aufmerksamkeit der Proband\_innen wird über Fragen zum Aufbau der Wörter sichergestellt (z.~B. Haben Sie \textit{des Schettosen} oder \textit{des Schettoses} gelesen?, siehe hierzu genauer \sectref{methschema}). Um die Ergebnisse der \textit{self-paced reading task} evaluieren zu können, wird der Effekt der  Form-Schematizität  zusätzlich mithilfe einer \textit{lexical decision} und einer \textit{sentence maze task} überprüft.

\subsection{\textit{Sentence maze tasks}}
\label{sentmazetask}

\textit{Sentence maze tasks} lassen sich als eine Mischung aus \textit{self-paced reading} und \textit{lexical decision tasks} betrachten: Wie in einer \textit{self-paced reading task} lesen Proband\_innen Sätze Stück für Stück. Wie in einer \textit{lexical decision task} bewerten Proband\_innen Stimuli (\cite[347]{Forster.2010}).  Der Anfang der Sätze ist dabei vorgegeben und die Proband\_innen haben im Anschluss daran die Wahl zwischen zwei Möglichkeiten, siehe Beispiel \ref{mazebsp}, das an ein Beispiel von \textcite[163--164]{Forster.2009} angelehnt ist. 

\begin{exe}

\item Kein ... \label{mazebsp}
\begin{xlist}
\item Mensch	\ \ \ \ \ geht
\item Tür \ \ \ \ \ \ \ \ \ \ \ \ ist
\item illegal \ \ \ \ \ \ \ \ wackelt
\end{xlist}
\end{exe}

Nur eine der Kombinationsmöglichkeiten in Beispiel \ref{mazebsp} führt zu einem grammatischen Satz: \textit{Kein Mensch ist illegal}. Die anderen Kombinationsmöglichkeiten sind hingegen ungrammatisch (z.~B. *\textit{Kein geht Tür illegal}/*\textit{Kein Mensch ist wackelt}). Es handelt sich somit um eine \textit{grammaticality maze (G-maze) task}, da die Alternativen zu den präferierten Möglichkeiten jeweils zu einem ungrammatischen Satz führen. 


Wie die anderen Verfahren zur Reaktionszeitmessung wurden auch \textit{sentence maze tasks} meines Wissens bislang noch nicht zur Erforschung von Variation genutzt. Dabei bietet sich die Methode dafür an, da Proband\_innen zwischen zwei Formen wählen müssen. So können die Reaktionszeiten für die Wahl zwischen Formen, die Varianten darstellen (\textit{Auto gefahren sein/haben}), und Formen, bei denen dies nicht der Fall ist (\textit{zur Kur gefahren sein/*haben}), gegenüber gestellt werden.

Neben \textit{G-maze tasks} existieren auch \textit{lexicality maze (L-maze) tasks}, bei denen die Wahl zwischen real existierenden und Pseudowörtern besteht (\cite[108]{Witzel.2012}). Die \textit{G-maze task} wird dabei als die komplexere gesehen, da Pro\-\mbox{band\_in}\-nen den Satz inkrementell prozessieren müssen, um sie zu lösen. Das ist bei einer \textit{L-maze task} nicht der Fall, da Proband\_innen jeweils unabhängig von den anderen Stimuli entscheiden können, ob sie den zu bewertenden Stimulus kennen (\cite[109]{Witzel.2012}). 



Anders als in \textit{self-paced reading tasks} wird in \textit{sentence maze tasks} i.~d.~R. mit einzelnen Wörtern und nicht mit Wortgruppen gearbeitet (\cite[163]{Forster.2009}). Ähnlich wie bei \textit{self-paced reading} mit \textit{moving window} sehen die Proband\_innen immer nur die Möglichkeiten, zwischen denen sie sich entscheiden sollen. Sie müssen sich daher die Satzteile merken, die sie gewählt haben. Das Abfragen des Textverständnisses entfällt bei \textit{sentence maze tasks}, da die Aufmerksamkeit der Proband\_innen über die Wahl der Wörter evaluiert werden kann. Dabei kann der aktuelle Versuch abgebrochen und zum nächsten Satz gesprungen werden, wenn Proband\_innen einen Fehler machen (\cite[163]{Forster.2009}). Eine Wiederholung des fehlerhaften Versuchs ist dabei nicht vorgesehen.  

Ein Vorteil der \textit{sentence maze task} ist, dass Proband\_innen zu einem inkrementellen Lesen gezwungen werden (\cite[164]{Forster.2009}). Zudem wird der Leseprozess verlangsamt, sodass die Synchronisierung von Tastendruck und Lesegeschwindigkeit erleichtert wird und spill-over-Effekte vermieden werden (\cite[164]{Forster.2009}). Ein weiterer Vorteil besteht darin, dass die Aufgabe der Pro\-\mbox{band\_in}\-nen klar definiert ist, da sie anders als beim \textit{self-paced reading} nicht weiter klicken, wenn sie meinen, ein Wort verstanden zu haben, sondern aktiv zwischen zwei Möglichkeiten wählen.\footnote{Inwiefern die Reihenfolge der Möglichkeiten (grammatisch-intendierte Form links oder rechts von der ungrammatisch-intendierten Form) eine Rolle in der Auswahl spielt, wurde meines Wissens noch nicht erforscht.}  


\begin{sloppypar}
An der \textit{sentence maze task} wird kritisiert, dass zwei anstelle von einem Wort prozessiert werden müssen und Reaktionszeiten somit verzerrt sein könnten (\cite[353]{Forster.2010}). Zudem spiegelt die Methode offensichtlich keine natürliche Lesesituation wider (\cite[350]{Forster.2009}, \cite[164]{Forster.2010}, \cite[107]{Witzel.2012}). \textcite[108--109]{Witzel.2012} sehen darin aber auch Vorteile: Dadurch, dass die Methode die Proband\_innen zwingt, Sätze Wort für Wort zu prozessieren, werden individuelle Lesestrategien verhindert, sodass gezielt gemessen werden kann, welchen Prozessierungsaufwand das Integrieren von Wörtern in einen Satz hat. Zudem berichten Proband\_innen von \textit{sentence maze tasks}, dass sie das Gefühl haben, die Sätze relativ natürlich zu lesen (\cite[164]{Forster.2009}).
\end{sloppypar}

\textcite{Witzel.2012} vergleichen die Resultate von \textit{eyetracking}, \textit{self-paced reading} und \textit{sentenze maze tasks} anhand drei verschiedener syntaktischer Phänomene (siehe \sectref{selfpaced} für weitere Erläuterungen). Die \textit{sentence maze task} zeigt genauso wie die \textit{self-paced reading task} bei zwei der drei Phänomene die erwarteten Ergebnisse: Wie bei der \textit{self-paced reading task} konnten auch in der \textit{sentence maze task} keine Effekte von Ambiguität in der Koordination zwischen Objekten und Sätzen festgestellt werden (\cite[121--124]{Witzel.2012}). \textcite[167--170]{Forster.2010} zeigt, dass die \textit{sentence maze task} sensitiv für Holzwegsatz- und Frequenzeffekte ist (siehe \sectref{Statistik} für eine bayesianische Modellierung der Effekte).

Der Einfluss der Prototypizität und der Form-Schematizität auf Reaktionszeiten wird mithilfe einer \textit{sentence maze task} überprüft (siehe \sectref{metproto} und \ref{schemalex}). Da das Interesse der Studie nicht das Lesen an sich, sondern die Prozessierung einzelner Formen ist, bietet sich diese Methode an, da sie es ermöglicht, die Reaktionszeit für die Wahl verschiedener Formen gezielt zu vergleichen. Dabei wird die \textit{sentence maze task} leicht abgewandelt: Die vorliegende Studie arbeitet nicht mit einzelnen Wörtern, sondern mit Wortgruppen. Bereits für den Satzanfang wählen die Proband\_innen aus zwei Möglichkeiten. Die alternativen Möglichkeiten zur im Design präferierten Möglichkeit stellen grammatische und orthographische Abweichungen sowie Anschlüsse dar, die im Kontext semantisch unsinnig sind. Dabei wird neben klaren Abweichungen (*\textit{Schuhle}) auch mit Varianten (\textit{Fahrradfahren/Fahrrad fahren}) gearbeitet. Daher führen Fehler nicht zu einem Abbruch des Versuchs. Die Konzentration der Proband\_innen wird anhand des Antwortverhaltens bei Filleritems evaluiert. Wie in der lexical-decision-Studie werden Proband\_innen ab 10 \% falscher Antworten aus der Stichprobe ausgeschlossen. Der Einfluss der Form-Schematizität wird nicht nur in der \textit{sentence maze task}, sondern auch mithilfe einer \textit{lexical decision} und einer \textit{self-paced reading task} überprüft, sodass die verschiedenen Verfahren zur Reaktionszeitmessung miteinander verglichen und evaluiert werden können. In den folgenden Abschnitten wird das methodische Design der einzelnen Studien erläutert.

\section{Lexical-decision-Studie zu Frequenzeffekten}\label{metfreq}\largerpage

Der Einfluss der Frequenz auf Variation wird anhand der Variation in der Konjugation mithilfe einer \textit{lexical decision task} untersucht. In die Aufgabe sind sowohl starke als auch schwache Formen starker Verben mit unterschiedlicher Tokenfrequenz eingeflochten. Die Proband\_innen beurteilen die Formen stets danach, ob sie sie kennen. Im Folgenden werden zunächst die Fragestellungen und Hypothesen vorgestellt. Anschließend wird das methodische Vorgehen (Materialien, Versuchsdesign und -ablauf, Metadaten, Probandenakquise und Proband\_innen) näher erläutert.

\subsection{Fragestellung und Hypothesen}
\label{freqhyp}

Die Untersuchung geht der Frage nach, welchen Einfluss Tokenfrequenz auf die Verarbeitung und die Bekanntheit starker und schwacher Partizip-II-Formen von starken Verben hat (zum Einfluss der Tokenfrequenz auf die Konjugation siehe \sectref{freqverb}). Zudem wird untersucht, ob schwache Partizip-II-Formen starker Verben anders wahrgenommen und verarbeitet werden, wenn sie bereits nachgewiesenermaßen im Sprachgebrauch genutzt werden. Dies wird überprüft, indem Verben genutzt werden, bei denen Schwankungen in Korpora attestiert bzw. nicht attestiert sind. In der Studie wird daher mit der Unterscheidung zwischen frequenten und infrequenten Verben gearbeitet und innerhalb der infrequenten Verben zwischen attestierter und nicht-attestierter Schwankung unterschieden. Die frequenten Verben weisen aufgrund der hohen Tokenfrequenz allesamt keine attestierte Variation auf (siehe \sectref{freqmat} für Erläuterungen zu den Testverben). 


\begin{sloppypar}
In Hinblick auf Reaktionszeiten und Antwortverhalten der Proband\_innen werden folgende Hypothesen aufgestellt:
\end{sloppypar}

\subsubsection{Reaktionszeiten} 

\begin{enumerate} 
\item Die Reaktionszeiten für frequente Verben sind niedriger als für infrequente Verben mit und ohne Schwankung in Korpora.
\begin{enumerate}
\item Die Unterschiede in den Reaktionszeiten lassen sich sowohl für starke als auch für schwache Formen feststellen.
\item Die Unterschiede in den Reaktionszeiten lassen sich sowohl für als bekannt bewertete Formen als auch für als unbekannt bewertete Formen feststellen.
\end{enumerate}

Sollten sich die Hypothesen bestätigen, können die Unterschiede in den Reaktionszeiten als ein Hinweis auf Variationspotential gedeutet werden. Zwar ist auch aus rein frequentieller Sicht zu erwarten, dass die Reaktionszeiten bei frequenten Verben kürzer sind als bei infrequenten (\cite{Clahsen.2001}), jedoch kann über die schwachen Formen der generelle Einfluss der Frequenz eingegrenzt werden: Die schwachen Partizip-II-Formen von tokenfrequenten Verben (bspw. *\textit{gezieht} statt \textit{gezogen}) sind zwar theoretisch mögliche Partizip-II-Formen, die Formen an sich sind aber nicht frequent. Somit kann ein schnelles Ablehnen dieser Formen nicht mit der Frequenz der Formen erklärt werden, da diese praktisch inexistent sind. Ein Unterschied in den Reaktionszeiten zwischen frequenten und infrequenten Verben kann daher als Hinweis auf Variationspotential interpretiert werden: Nicht nur starke Formen, sondern auch schwache Formen von frequenten starken Verben sollten schnell beurteilt werden, weil eine der Formen stark gefestigt ist. Bei infrequenten Verben sind dagegen unabhängig von Schwankungen in Korpora erhöhte Reaktionszeiten für beide Formen zu erwarten, da die starken Formen nicht so stark gefestigt sind und daher deren statistisches Vorkaufsrecht (\cite[74--94]{Goldberg.2019}; siehe \sectref{Statistik} für weitere Erläuterungen) nur noch bedingt greifen kann. Erhöhte Reaktionszeiten würden also darauf hindeuten, dass das statistische Vorkaufsrecht bei infrequenten starken Verben nicht mehr vollständig greift, und damit auf Variationspotential hinweisen, da mit dem schwindenden Einfluss des statistischen Vorkaufsrechts für starke Formen schwache Formen möglich werden.

\item Zwischen den infrequenten Verben mit attestierter Schwankung und ohne attestierte Schwankung in Korpora lassen sich keine systematischen Unterschiede in den Reaktionszeiten messen.
\begin{enumerate}
\item Die Reaktionszeiten bei infrequenten Verben mit und ohne Schwankung werden nicht durch starke oder schwache Verbformen beeinflusst.
\item Die Reaktionszeiten bei infrequenten Verben mit und ohne Schwankung werden nicht durch die Bewertung der Formen als bekannt oder unbekannt beeinflusst.
\end{enumerate}\largerpage

\begin{sloppypar}
Da die Verben mit und ohne attestierte Schwankung infrequent sind, sind auch die starken Formen dieser Verben seltener und sie können die schwachen Formen daher im Vorkommen nicht so stark dominieren. Das statistische Vorkaufsrecht frequenter Formen kann daher nicht mehr vollständig greifen, weshalb sowohl Verben mit attestierter Schwankung als auch Verben ohne attestierte Schwankung vergleichbare Reaktionszeiten hervorrufen sollten. Diese Hypothese hat sich in der Prästudie bestätigt, die Prästudie ist im digitalen Anhang zu finden. Neben dieser Hypothese lässt sich eine Alternativhypothese aufstellen: Kleine Unterschiede in der Reaktionszeit sind zwischen Verben mit attestierter und nicht-attestierter Schwankung erwartbar, weil schwache Formen bei Verben ohne attestierte Schwankung weniger frequent sind als bei Verben mit attestierter Schwankung. Diese Unterschiede sollten jedoch geringer sein als der Unterschied in den Reaktionszeiten zwischen frequenten und infrequenten Verben. Das Studiendesign erlaubt es, beide Hypothesen zu überprüfen.  Sollten keine Unterschiede zwischen Verben mit attestierter und ohne attestierte Schwankung in Korpora gemessen werden, kann das als Hinweis darauf gedeutet werden, dass die Formen von Verben ohne Schwankung in Korpora ähnlich gefestigt sind wie die Formen von Verben mit Schwankung in Korpora. Vergleichbare Reaktionszeiten würden daher auf ein Variationspotential hinweisen, auch wenn Variation noch nicht in Korpora zu beobachten ist.
\end{sloppypar}
\end{enumerate}

\subsubsection{Antwortverhalten}

\begin{enumerate}
\item Starke Formen werden bei infrequenten Verben eher als unbekannt bewertet als bei den frequenten. Trotz dieser Unterschiede ist die Bekanntheit der starken Formen bei allen Testverben auf einem hohen Niveau.
\item Die Bekanntheit der schwachen Formen ist von der Tokenfrequenz abhängig: Schwache Formen frequenter Verben sind unbekannt, schwache Formen infrequenter Verben weisen hingegen einen höheren Bekanntheitsgrad auf.
\item Schwache Formen von infrequenten Verben mit attestierter Schwankung in Korpora sind eher bekannt als schwache Formen von Verben ohne attestierte Schwankung in Korpora.

Schwache Formen haben eine größere Wahrscheinlichkeit bekannt zu sein, wenn sie bereits in Gebrauch, d.~h. in einem Korpus attestiert sind. Auch wenn Korpora nur einen Ausschnitt der Sprachwirklichkeit abbilden, die Sprecher\_innen rezipieren, ist bei Formen, die bereits in einem Korpus attestiert sind, die Wahrscheinlichkeit groß, dass Proband\_innen die Form schon einmal rezipiert oder sogar selbst gebildet haben. Bei Verben, bei denen die schwache Form (noch) nicht attestiert ist, ist die Wahrscheinlichkeit hingegen geringer, dass die schwache Form bereits rezipiert wurde. Die Ergebnisse der Prästudie stützen diese Hypothese. Die Prästudie ist im digitalen Anhang zu finden. Wie bei den Reaktionszeiten lässt sich auch in Bezug auf das Antwortverhalten eine alternative Hypothese aufstellen: Da die Verben gleichermaßen infrequent sind, könnten die starken und schwachen Partizip-II-Formen der Verben unabhängig von ihrem attestierten Vorkommen in Korpora ähnlich bewertet werden. Das Design erlaubt es, beide Hypothesen zu überprüfen.\largerpage

\item Für frequente Verben ist ein konsistentes Antwortverhalten zu beobachten (überwiegend Unbekanntheit für schwache Formen, überwiegend Bekanntheit für starke Formen), bei infrequenten Verben sind auch inkonsistente Antworten zu finden (Bekanntheit/Unbekanntheit für beide Formen).
\end{enumerate}

Die hier vorgestellten Hypothesen sowie das methodische Vorgehen inklusive der geplanten statistischen Auswertung wurden vor der Datenerhebung registriert. Die Registrierung lässt sich unter \url{https://osf.io/ynvb5} aufrufen. Die statistische Auswertung wird in \sectref{freqerg} erläutert, in dem auch die Ergebnisse vorgestellt werden.

\subsection{Materialien}
\label{freqmat}

Für die Studie werden Partizip-II-Formen starker Verben mit unterschiedlicher Frequenz genutzt. Der Einfluss der Frequenz auf starke Verben wird anhand von Partizip-II-Formen getestet, da alle starken Verben -- anders als bei der Wechselflexion im Präsens und der Imperativhebung -- starke Partizip-II-Formen aufweisen und das Partizip~II frequenter ist als das Präteritum (\cite[161]{Fischer.2018}). 

\begin{sloppypar}
Die Ermittlung der Frequenz erfolgte anhand der Wortverlaufskurven im Digitalen Wörterbuch der Deutschen Sprache (DWDS, \cite{BerlinBrandenburgischeAkademiederWissenschaften.2019}). Basis der Wortverlaufskurven ist das DWDS-Zei\-tungs\-kor\-pus. Um die Unterscheidung zwischen Verben mit attestierter und ohne attestierte Schwankung zu operationalisieren, wurde die Ratio zwischen starken und schwachen Partizip-II-Formen im Deutschen Referenzkorpus ermittelt (DeReKo, \cite{LeibnizInstitutfurDeutscheSprache.2019}). Die Abfragen im DWDS und im DeReKo wurden am 5.11.2019 durchgeführt. 
\end{sloppypar}

Tabelle \ref{freqverbtabelle}\footnote{Die Ergebnisse wurden nur in Hinblick auf schwache Formen für frequente und infrequente Verben ohne Schwankung bereinigt. Dabei wurden dialektale und metasprachliche Belege sowie Fehlschreibungen (\textit{getragt} statt \textit{gefragt}) ausgeschlossen. Belege, die darauf schließen lassen, dass ein Kind spricht oder nachgeahmt wird, wurden nicht ausgeschlossen, da nicht immer eindeutig ersichtlich war, ob dies der Fall ist. Bei \textit{gefahren} wurden groß geschriebene Belege ausgeschlossen, um Fehltreffer aufgrund des homographen Substantivs (\textit{die Gefahren}) zu vermeiden.} gibt einen Überblick über die Testverben. Darin wird die relative Frequenz des starken Partizips II der Verben (Vorkommen pro Million Token) in den Wortverlaufskurven im Jahr 2018 des DWDS\footnote{Die relative Frequenz der schwachen Partizip-II-Formen ist zu vernachlässigen: Die höchste relative Frequenz im Jahr 2018 weisen \textit{gesinnt} (0,47) und \textit{gewebt} (0,31) auf. Es folgen \textit{eingesaugt} (0,06) und \textit{gehaut} (0,03). Die Verben entstammen alle der Ausprägung \textsc{infrequent mit Schwankung}. Bei allen anderen Testverben liegt die relative Frequenz der schwachen Partizip-II-Formen bei 0.}, die absolute Frequenz der starken und schwachen Formen im DeReKo sowie die Ratio zwischen starken und schwachen Formen im DeReKo angegeben. Zwischen Verwendungen der Partizip-II-Formen als Perfekt- bzw. Plusquamperfektform und Prädikatsnomen wird nicht unterschieden. Die Verben sind absteigend nach ihrer Ratio zwischen starken und schwachen Formen sortiert. 

\begin{table}
\fittable{\begin{tabular}{lrlrlrr}
\lsptoprule
 Verb  & RF & stark & absolut & schwach & absolut & Ratio  \cr
\midrule
\multicolumn{7}{l}{{frequent}}\\
\textit{sprechen} & 53 & \textit{gesprochen} & 506.078 & \textit{gesprecht}& 1 & 506.000/1 \cr
\textit{fahren} & 26 & \textit{gefahren}& 308.377 & \textit{gefahrt}& 0 & 308.00/0 \cr 
\textit{tragen} & 21 & \textit{getragen} &283.025 & \textit{getragt}& 1 & 283.000/1 \cr
\textit{ziehen} & 44 & \textit{gezogen} &481.702 & \textit{gezieht}& 2 & 240.000/1 \cr
\textit{halten} & 46 & \textit{gehalten} & 537.019 & \textit{gehaltet} & 5 & 107.000/1 \cr
\textit{schreiben} & 61 & \textit{geschrieben}& 664.701 & \textit{geschreibt} &10 & 66.000/1 \cr	
\textit{sinken} & 22 & \textit{gesunken} & 178.562 & \textit{gesinkt}& 3 & 60.000/1 \cr
\textit{fliegen} & 13 & \textit{geflogen} & 113.074 & \textit{gefliegt}& 4 & 28.000/1 \cr
\midrule
\multicolumn{7}{l}{{infrequent ohne Schwankung}}\\
\textit{flechten}& 0,4 & \textit{geflochten}& 4.847  & \textit{geflechtet} &6			& 807/1 \cr
\textit{spinnen} & 0,38 & \textit{gesponnen}& 5.712& \textit{gespinnt}& 11	&		519/1 \cr
\textit{schmelzen}& 0,93 & \textit{geschmolzen} &9.278& \textit{geschmelzt}& 19	&	488/1 \cr
\textit{anschwellen} & 0,69 & \textit{angeschwollen} &4.910& \textit{angeschwellt} &11&	446/1 \cr
\textit{kneifen} & 0,16 & \textit{gekniffen}& 2.070 & \textit{gekneift} & 6 	&	345/1 \cr
\textit{fechten}& 0,23 &\textit{gefochten}& 3.449& \textit{gefechtet} &13	&265/1 \cr
\textit{melken} & 0,34 &\textit{gemolken}& 4.441 & \textit{gemelkt}& 25 &		177/1 \cr
\textit{dreschen} & 0,16 &\textit{gedroschen} & 2.967 & \textit{gedrescht} & 24		& 123/1 \cr
\midrule
\multicolumn{7}{l}{{infrequent mit Schwankung}}\\
\textit{salzen} & 0,31 & \textit{gesalzen} & 2.955 & \textit{gesalzt} &49			& 60/1 \cr
\textit{hauen} & 1,45 &\textit{gehauen}& 14.999 & \textit{gehaut} & 593 & 25/1 \cr
\textit{quellen}& 0,03 & \textit{gequollen} &268 & \textit{gequellt}& 29 & 9/1 \cr
\textit{einsaugen} & 0,11 & \textit{eingesogen} &1.012 & \textit{eingesaugt} &446 & 2,2/1 \cr
\textit{weben} & 0,16 & \textit{gewoben} &2.242 & \textit{gewebt} &2.946				& 0,8/1 \cr
\textit{glimmen}& 0 & \textit{geglommen}& 25 & \textit{geglimmt} &49			& 0,5/1 \cr
\textit{gären}& 0 & \textit{gegoren}& 130 & \textit{gegärt} &280				& 0,46/1 \cr
\textit{sinnen}& 0,15 & \textit{gesonnen} & 3.285 & \textit{gesinnt} & 5.611	&			0,59/1 \cr
\lspbottomrule
\end{tabular}}
\caption{Überblick über die Testverben (RF: Relative Frequenz)}
\label{freqverbtabelle}
\end{table}

Ausschlaggebend für die Einteilung der Testverben in Tokenfrequenzausprägungen ist die Frequenz der starken Partizip-II-Form (z.~B. \textit{@gezogen}) innerhalb der Wortverlaufskurve im Jahr 2018: Frequent sind Verben mit einer Partizip-II-Frequenz ab 13 Belegen pro Million Token, infrequent sind alle Verben mit bis zu zwei Belegen pro Million Token. 



Neben der Einteilung in Frequenzausprägungen wurden die infrequenten Verben in Verben mit attestierter Schwankung und ohne attestierte Schwankung in Korpora geteilt. Um die Unterscheidung zwischen Verben mit und ohne Schwankungen in Korpora zu operationalisieren, wurde nach schwachen und starken Partizip-II-Formen der Verben im Archiv W der geschriebenen Sprache (Korpus W-öffentlich) im DeReKo über Cosmas II (\cite{LeibnizInstitutfurDeutscheSprache.2019b}) gesucht und die Ratio zwischen starken und schwachen Formen bestimmt. Ab einer Ratio von über 100 starken Formen zu einer schwachen gilt die Schwankung als nicht attestiert, bei weniger oder genau 100 starken Formen zu einer schwachen Form gilt sie als attestiert. 

Die Testverben gehören unterschiedlichen Ablautreihen an. Dies hat zur Folge, dass bspw. Effekte von Form-Schemata mit der Tokenfrequenz interagieren und Verben wie \textit{spinnen} sich eher in der starken Flexion halten könnten als ähnlich frequente Verben, die nicht dem Form-Schema [\#\_ɪ + ŋ + (C)] angehören (zu Form-Schemata bei starken Verben siehe \sectref{schemaverb}). Allerdings ist \textit{spinnen} aufgrund des alveolaren Nasals in der Peripherie des Form-Schemas. Daher ist es nicht überraschend, dass \textit{spinnen} im Präteritum zwischen dem ursprünglichen Ablaut /a/, dem neuen Ablaut /ɔ/ und der schwachen Form schwankt (\cite[172--175]{Nowak.2013}). Neben Form-Schemata kann die Ablautalternanz Einfluss auf die Reaktionszeit nehmen: Bei der Ablautalternanz ABA ist bspw. zu erwarten, dass der Ablaut im Partizip~II keine große kognitive Herausforderung darstellt, da er identisch mit dem Präsensvokal ist (zu Ablautalternanzen und Frequenz siehe \sectref{freqverb}). Daher könnten Verben mit der  Ablautalternanz ABA niedrigere Reaktionszeiten evozieren als Verben mit den Ablautalternanzen ABC oder ABB. In der Auswertung werden die Lemmata der Testverben als zufällige Effekte in das statistische Modell eingeflochten, sodass einzelne Lemmata das Ergebnis nicht verzerren.\largerpage

Die Auswahl der infrequenten Verben stellte eine Herausforderung dar, da die meisten starken Verben tokenfrequent sind (siehe \sectref{freqverb}). Für die Auswahl musste zusätzlich zur geringen Tokenfrequenz darauf geachtet werden, dass die schwachen Partizip-II-Formen der Testverben nicht mit anderen Formen im Paradigma zusammenfallen. Daher können Präfixverben wie \textit{erklimmen} nicht genutzt werden, da \textit{erklimmt}{\interfootnotelinepenalty=10000\footnote{Aufgrund der Ähnlichkeit von \textit{geklommen} zu \textit{beklommen} wurde \textit{klimmen} nicht als Simplex in die Untersuchung eingeflochten.}} auch die 3. Ps. Sg. Präs. Ind. sein könnte (\textit{Sie erklimmt den Berg}). Zudem musste darauf geachtet werden, dass die schwache Form des starken Verbs nicht mit der Form eines anderen schwachen Verbs zusammenfällt, wie bei \textit{gebackt} im Sinne von \SchmittSingleQuot{festkleben} (\cite{Duden.2020}), oder als orthographische Abweichung des Partizips~II eines schwachen Verbs wahrgenommen werden könnte wie bei \textit{gemahlt}, das als eine Abweichung von \textit{gemalt} angesehen werden könnte.

 
Bis auf \textit{anschwellen} und \textit{einsaugen} sind alle Testverben Simplizia. \textit{Anschwellen} wurde statt \textit{schwellen} genutzt, um eine transitive Lesart auszuschließen, die nur schwache Formen zulässt (\textit{Der Affe schwellte die Brust vor Stolz}) (\cite{Duden.2020}). \textit{Einsaugen} wurde genutzt, um die Assoziation mit \textit{staubsaugen} zu stören, da \textit{staubsaugen} nur schwache Formen zeigt, während \textit{saugen} im Sinne von \SchmittSingleQuot{eine Flüssigkeit aufnehmen} weiterhin zwischen stark und schwach schwankt (\cite[316]{Nowak.2011}). Bei \textit{anschwellen} und \textit{einsaugen} handelt es sich um Partikelverben, weshalb eine Verwechslung der schwachen Partizipform (\textit{eingesaugt/angeschwellt}) mit der 3. Ps. Sg. Ind. (\textit{er/sie/es saugt ein/schwellt an}) ausgeschlossen ist.

Wenig überraschend gehören bis auf \textit{salzen}, \textit{kneifen} und \textit{hauen} alle infrequenten Verben dem Ablautmuster \textit{x-o-o} an, haben also den ursprünglichen Ablaut im Präteritum gewechselt und /oː/ bzw. /ɔ/ angenommen.\footnote{Dies gilt nicht für \textit{saugen}, das bereits im Mhd. zur Ablautreihe 2 zählte und daher bereits ursprünglich auf /oː/ ablautete (\cite[233]{Nowak.2015}). Wie oben erwähnt zeigen sich für \textit{saugen} im Partizip~II Schwankungen.}  \textit{Salzen} bildet das Präteritum schwach (\cite{Duden.2020}). \textit{Kneifen} zeigt im Präteritum Schwankungen zwischen stark und schwach: Im Archiv W des DeReKo stehen 15 schwache Formen 1704 starken Formen gegenüber (Ratio 75/1). Bei \textit{hauen} ist die Schwankung im Präteritum deutlich zu beobachten: 5.844 starke Belege stehen~5.526~schwachen Belegen gegenüber (Ratio 1,05/1).\footnote{Bei der Suche nach \textit{kniff}, \textit{hieb} und \textit{haute} wurden großgeschriebene Substantive ausgeschlossen, um Fehltreffer aufgrund von Konversion (\textit{der Kniff}, \textit{der Hieb}) und dem homographen französischen Adjektiv \textit{haute} (wie in \textit{Haute Couture}) zu vermeiden. Für die kleingeschriebenen Belege von \textit{haute} wurde eine Stichprobe von 1.000 Belegen durchgesehen: Darin waren 30 Fehltreffer enthalten, weshalb die Trefferanzahl um 3~\% nach unten korrigiert wurde, unbereinigt ergeben sich 5.697 Belege.} Somit zeigen alle infrequenten Testverben Schwankungen im Präteritum zu \textit{x-o-o} und/oder schwachen Formen, was eine Schwankung im Partizip~II wahrscheinlicher macht (siehe \tabref{prätfrequ} in Anhang~\ref{matfreqanh} für einen Überblick über die Frequenz starker und schwacher Präteritalformen der infrequenten Testverben).

\subsection{Versuchsdesign}
\label{studiefreqdesign}

 Die Studie nutzt ein within-subject-Design mit zwei Faktoren: Frequenz  (mit drei Ausprägungen: \textsc{frequent}, \textsc{infrequent ohne Schwankung}, \textsc{infrequent mit Schwankung}) und Partizip-II-Form (mit zwei Ausprägungen: stark, schwach). Die Ausprägungen enthalten jeweils acht Testwörter. Die Proband\_innen müssen entscheiden, ob sie das angezeigte Wort kennen. Dabei können sie \textit{ja} oder \textit{nein} antworten. Die Tastaturbelegung für \textit{ja} und \textit{nein} ist \textit{f} für \textit{nein} und \textit{j} für \textit{ja}. Um keine Unruhe in das Experiment zu bringen, wird diese Tastaturbelegung das gesamte Experiment beibehalten. Vor dem Versuchswort erscheint jeweils zwei Sekunden lang eine Fixation (\textit{\#\#\#}), um spill-over-Effekte zu vermeiden. Die Antwort sowie die Reaktionszeit werden gespeichert. 

 

Als Filler dienen schwache und starke Maskulina mit schwacher bzw. starker Flexion (\textit{des Kollegen/des Kolleges}). Diese stellen die Testwörter der lexical-decision-Studie zu Form-Schematizitätseffekten bei schwachen Maskulina dar (siehe \sectref{schemalex}). Zudem werden starke Maskulina und Neutra mit langer Genitivform als Filler genutzt, die lange Endungen phonotaktisch nicht zulassen (z.~B. \textit{des *Bodenes}) (\cite[107--108]{Szczepaniak.2010}) (siehe Anhang \ref{fillerlex} für eine Aufschlüsselung der Filleritems). 



Als Test, ob das Design funktioniert, dient der Kontrast zwischen langer und kurzer Genitiv\-endung bei zwei Substantiven, die nur die kurze Endung zulassen (\textit{des Ultimatums/*Ultima\-tumes}, \textit{des Internets/*Internetes}), sowie zwischen unregelmäßiger und regelmäßiger Konjugation bei zwei unregelmäßigen Verben (\textit{gegangen/*gegeht}, \textit{gebracht/*gebringt}). Diese Items sind nur relevant, wenn für die Testitems keine Unterschiede in den Reaktionszeiten festzustellen sind. Für die kurzen Endungen bzw. unregelmäßigen Formen sind geringere Reak\-tionszeiten zu erwarten als für die langen bzw. regelmäßigen Formen. Zudem ist ein klares Antwortverhalten zu erwarten (kurze bzw. unregelmäßige Formen sind bekannt, lange bzw. regelmäßige unbekannt). Die Items eigenen sich daher, um zu prüfen, ob das Versuchsdesign generell Unterschiede in Reaktionszeiten hervorruft. 

Insgesamt werden 109 Versuchitems präsentiert. Dabei entfallen 48 auf die Testitems, 49~auf Filler (30 starke/schwache Maskulina, 19 lange Genitivformen), acht auf Items, die das Design testen, sowie vier auf Items im Beispielblock. Bei 34 Items sind affirmierende Antworten auf die Frage nach der Bekanntheit der Formen zu erwarten (bspw. bei der Form \textit{gegangen}), bei 49 verneinende Antworten (bspw. bei der Form \textit{gegeht}). Es verbleiben 26 Items, bei denen theoretisch beide Formen bekannt sein können. Der vermeintliche Überschuss an verneinenden Antworten ergibt sich aus Fillern für die Experimentblöcke mit langen Genitivformen. Hier wird nur die lange Genitivform, nicht aber die kurze abgefragt. Der vermeintliche Überschuss ist dabei beabsichtigt, um ein Gegengewicht zu den 26~Items zu erhalten, bei denen beide Formen bekannt sein können. Dies ist bspw. bei den infrequenten Verben mit attestierter Schwankung (\textit{gewebt} und \textit{gewoben}) der Fall. Berechnet man diese Items als affirmierend in das Verhältnis von affirmierenden und verneinenden Antworten ein, ist davon auszugehen, dass mehr Items affirmierend als verneinend beantwortet werden. Deswegen werden durch die Filler zusätzlich verneinende Antworten evoziert, um ein Ungleichgewicht zwischen affirmierenden und verneinenden Antworten zu vermeiden. Bei den frequenten Testverben sowie den infrequenten Testverben ohne attestierte Schwankung wird im Gegensatz zu den infrequenten Verben mit attestierter Schwankung erwartet, dass die starke Form jeweils bekannt, die schwache hingegen nicht bekannt ist.  

 

Die Testitems sind in vier Experimentblöcken organisiert, die jeweils 25 Items enthalten. Die Blöcke erscheinen in randomisierter Reihenfolge, aber die Reihenfolge der Items innerhalb der Blöcke variiert nicht. Zudem existiert ein Beispielblock mit vier Items und ein Eingewöhnungsblock mit fünf Items (siehe Tabelle~\ref{ablauflexta} in Anhang \ref{blocklex}). In den Experimentblöcken sind jeweils zwölf Testverben (vier Testverben pro Frequenzausprägung) enthalten. Dabei sind die starken und schwachen Formen eines Testitems stets auf zwei Blöcke verteilt. Die restlichen Items sind Filler. Jeder Block beginnt mit einem Filler, um das zufällige Aufeinandertreffen von zwei Testwörtern zwischen zwei Blöcken zu verhindern. Innerhalb der Blöcke konnte nicht vermieden werden, dass zwei Testverben hintereinander präsentiert werden. Es wurde jedoch darauf geachtet, dass dies maximal zweimal pro Experimentblock geschieht und dass die Testverben jeweils aus unterschiedlichen Frequenzausprägungen stammen. Pro Experimentblock sind elf Items enthalten, bei denen eine verneinende Antwort antizipiert wird, und 14, bei denen eine affirmierende Antwort antizipiert wird. Bei sieben bis acht der Items mit affirmierender Antwort (bspw. \textit{gezogen}) ist die Wahrscheinlichkeit hoch, dass die korrespondierende Form unbekannt ist und daher verneinende Antworten evoziert (bspw. *\textit{gezieht}). Die restlichen Items gehören zu den Items, bei denen beide Formen bekannt sein könnten (bspw. bei \textit{gewebt/gewoben}). Innerhalb der Blöcke wurde darauf geachtet, dass keine Muster in den zu erwartenden Antworten entstehen.

\subsection{Versuchsablauf}
\label{ablauffreq}

  Die Studie wurde online mithilfe der Plattform Pavlovia (\cite{PsychoPyTeam.2019}) durchgeführt. In der Studie erhalten die Teilnehmer\_innen alle Anweisungen per Bildschirm. Zunächst werden die Teilnehmer\_innen in einem gesonderten Browser-Fenster begrüßt und darauf hingewiesen, dass Ihre Metadaten anonymisiert gespeichert werden. In diesem Brow\-ser-Fenster werden auch die Metadaten abgefragt. Dann werden im ursprünglichen Browser-Fenster allgemeine Informationen gegeben mit dem Hinweis, dass die Proband\_innen mit der Teilnahme der anonymisierten Speicherung ihrer Daten zustimmen. Zudem werden die Instruktionen zum Experiment angezeigt. Die Proband\_innen werden anhand eines Beispiels an die Aufgabe gewöhnt. Die Proband\_innen werden dazu aufgefordert, die angezeigten Wörter bzw. Wortabfolgen danach zu bewerten, ob sie sie kennen (Kennen Sie dieses Wort bzw. diese Wortabfolge?). Im Beispielblock erscheinen sowohl Formen, die wahrscheinlich bekannt sind (\textit{die Tische}, \textit{du siehst}), als auch Formen, die wahrscheinlich unbekannt sind (\textit{die *Autoe}, \textit{du *gebst}). Im Anschluss an das Beispiel folgt zunächst ein Eingewöhnungsblock, der fließend und somit von den Proband\_innen unbemerkt in die Experimentblöcke übergeht. 
  Für die Erstellung des Experiments wurde PsychoPy3 genutzt (\cite{Peirce.2019}), das eine präzise Messung von Reaktionszeiten ermöglicht. Die Skripte, die für die Datenerhebung verwendet wurden, sind im digitalen Anhang dokumentiert.

\subsection{Metadaten}
\label{studiefreqmeta}

Folgende Metadaten wurden abgefragt. Die Abfrage erfolgte jeweils über Freitextfelder.

\begin{enumerate}
\item Geburtsjahr
\item Geschlecht
\item Legasthenie
\item höchster Abschluss
\item L1
\item Händigkeit 
\item Vornehmlich genutzte Varietät 
\item Dialektkenntnis 
\item Herkunft (Ort, Bundesland)
\item Wohnort
\item Leseintensität (täglich, wöchentlich, monatlich).
\end{enumerate}

Die Metadaten dienen vornehmlich dazu, sicherzustellen, dass die untersuchten Gruppen möglichst homogen sind. Die Händigkeit der Proband\_innen wird abgefragt, da diese Einfluss auf das Antwortverhalten der Personen haben kann: Je nachdem, ob für die Beantwortung von Fragen die rechte oder linke Hand genutzt wird, sind Rechts- bzw. Linkshänder schneller (\cite[215]{Hohle.2010}). Da es sich um ein within-subject-Design handelt und somit alle Proband\_innen Daten zu jedem Versuchitem liefern, sollte die Händigkeit jedoch nur bedingt Einfluss auf die Ergebnisse nehmen. Zudem werden die Proband\_innen als zufälliger Effekt in die statistische Auswertung aufgenommen, sodass individuelle Präferenzen die Ergebnisse nicht verzerren sollten.  


Die L1 ist als die Sprache(n) definiert, die vor dem 12. Lebensjahr außerhalb der Schule erworben wurde(n). Die Frage wird genutzt, um Proband\_innen auszuschließen, die nicht Deutsch als L1 sprechen oder bilingual aufgewachen sind. In der Studie wird nicht der Terminus \textit{L1}, sondern \textit{Muttersprache} genutzt. Die Altersgrenze wurde gewählt, da die Pubertät als obere Grenze der \textit{critical period} angesetzt wird, innerhalb derer ein Vorteil beim Erlernen einer Sprache gegenüber Sprecher\_innen besteht, deren Spracherwerb erst später einsetzt (\cite[434]{Fromkin.2014}).


Die vornehmlich genutzte Varietät wurde abgefragt, um besser einschätzen zu können, inwiefern die Proband\_innen einen ähnlichen Sprachgebrauch aufweisen. In der Frage wurden Standard und Dialekt gegenübergestellt (Sprechen Sie in Ihrem familiären Umfeld vornehmlich Hochdeutsch oder Dialekt?). Dabei wurde in der Studie der Terminus \textit{Hochdeutsch} genutzt, da davon auszugehen ist, dass dieser Terminus den Proband\_innen geläufiger ist als \textit{Standard}. Die Spezifizierung auf das familiäre Umfeld wurde genutzt, um in den Daten eher eine Verzerrung hin zum Dialekt als zum Standard zu erreichen. Auf diese Weise ist es unwahrscheinlicher, dass die Homogenität zwischen den Sprecher\_innen überschätzt wird. Im Anschluss an die Frage nach der präferierten Varietät wurde die Dialektkenntnis der Proband\_innen erhoben. Dabei wurde gefragt, ob die Proband\_innen einen Dialekt sprechen und wenn ja, welchen. Die Dialektkenntnis wird somit als reine Selbsteinschätzung abgefragt. Alle Fragen zu den Metadaten sind im digitalen Anhang aufgelistet.

\subsection{Probandenakquise und Proband\_innen}\label{studiefreqprob}\largerpage

Die Proband\_innen wurden akquiriert, indem der Link zur Studie an Studierende in Germanistikveranstaltungen an der Universität Bamberg sowie der Univerität Mainz\footnote{Ich danke Jessica Nowak, Renata Szczepaniak, Anette Kremer und Jonas Fehn für das Weiterleiten des Studienaufrufs.} geschickt wurde. Zusätzlich habe ich den Link in meinem persönlichen Umfeld verbreitet. Die Germanistikveranstaltungen, in denen Proband\_innen akquiriert wurden, werden üblicherweise am Anfang des Studiums besucht, sodass die linguistische Vorbildung der Proband\_innen begrenzt sein sollte. Es wurde eine Stichprobengröße von 50 Proband\_innen angestrebt. Um diese zu erreichen, wurden insgesamt 100~Datensätze erhoben. Für die Studie werden nur Pro\-\mbox{band\_in}\-nen berücksichtigt, die Deutsch als L1 sprechen, monolingual aufgewachsen sind und keine Legasthenie haben. Proband\_innen, die keine Angaben zu L1 oder Legasthenie machten, wurden ausgeschlossen. Mit der Art der Probandenakquise geht einher, dass vornehmlich Menschen im Alter zwischen 20 und 30 getestet wurden. Diese Limitation gilt für alle Studien, die innerhalb des Dissertationsvorhabens durchgeführt wurden.



Zudem wurden Proband\_innen ausgeschlossen, deren Antwortverhalten auf Unkonzentriertheit schließen lässt. Als Entscheidungskriterium hierfür dient das Antwortverhalten bei Fillern und Items, die im Beispielblock genutzt wurden bzw. als Test für das Design dienen. Bei~31~dieser Items ist von einem eindeutigen Antwortverhalten auszugehen. Proband\_innen, die drei oder mehr der 31 Items anders beantworten als erwartet, werden von der Studie ausgeschlossen. Dies entspricht einem Ausschluss bei einer Fehlerquote ab 10~\%. Außerdem werden Proband\_innen ausgeschlossen, deren Reaktionszeiten mehr als drei Standardabweichungen über oder unter dem Mittelwert eines Testitems liegen. 

\begin{sloppypar}
Die Stichprobengröße von 50 wurde basierend auf einer Prästudie\footnote{Die Stichprobengröße in der Prästudie betrug nach Ausschluss von Proband\_innen anhand der oben genannten Kriterien 19 Proband\_innen. Die Prästudie hat denselben Versuchsablauf wie die Hauptstudie. Nur der aktuelle Wohnort wurde in der Prästudie nicht abgefragt. Die Proband\_innen der Prästudie erfüllen dieselben Kriterien wie in der Hauptstudie. Die Probandenakquise lief über mein persönliches Umfeld. Die Metadaten der Proband\_innen und die Ergebnisse der Prästudie sind im digitalen Anhang aufgeschlüsselt.} bestimmt. Die Ergebnisse der Prästudie deuten darauf hin, dass frequente Testverben schneller prozessiert werden als infrequente. Die Werte aus der Prästudie wurden als Grundlage für eine Datensimulation genutzt, um die Stichprobengröße zu bestimmen, die für eine Teststärke von über 0,9 (für den Schätzwert der Ausprägung \textsc{infrequent mit Schwankung} mit schwachen Formen im Vergleich zu den Schätzwerten für dieselbe Ausprägung mit starken Formen sowie für Formen der Ausprägung \textsc{frequent}) bei einem $\alpha$-Level von 0,01 benötigt wird. Um die Zielgröße von 50 Proband\_innen zu erreichen, wurden doppelt so viele Daten erhoben, da in der Prästudie mit 13 von 32 Proband\_innen 40,6 \% der Pro\-\mbox{band\_in}\-nen ausgeschlossen werden mussten.
\end{sloppypar}

Die Daten wurden vom 9.1.2020 bis zum 27.1.2020 erhoben. Nach Bereinigung der Daten blieb eine Stichprobe von 53 Proband\_innen. Zu einigen Metadaten fehlen Angaben, daher ergeben die nun folgenden Angaben zu den Pro\-\mbox{band\_in}\-nen aufaddiert nicht immer 53. Alle Metadaten sind im digitalen Anhang auch tabellarisch aufgeschlüsselt. 


Die Proband\_innen sind im Mittel 26,5 Jahre alt mit einer Standardabweichung von sechs Jahren. Zwölf identifizieren sich als männlich, 38 als weiblich; niemand gab ein anderes Geschlecht an. 18 der Proband\_innen kommen aus Bayern, sechs aus Rheinland-Pfalz, jeweils vier aus Baden-Württemberg und Niedersachsen. Die anderen Bundesländer sind maximal mit zwei Proband\_innen vertreten. 38 Proband\_innen leben derzeit in Bayern, acht in Rheinland-Pfalz, zwei in Hessen und jeweils ein\_e in Baden-Württemberg und Niedersachsen. 45 Proband\_innen gaben an, täglich zu lesen, fünf lesen wöchentlich, eine Person gab an, gar nicht zu lesen, verhielt sich aber in der Aufgabe ähnlich wie die anderen Proband\_innen. Die restlichen Proband\_innen machten keine Angabe zu ihrem Leseverhalten. 36 Proband\_innen gaben an, in ihrem familiären Umfeld Standard zu sprechen, elf sprechen dialektal, zwei nach eigenen Angaben eine Mischform. Dabei erläutern sie jedoch nicht, wie sie zu der Einschätzung als Mischform kommen. Die angegebenen Dialekte sind im digitalen Anhang  aufgeschlüsselt. 31~der~Proband\_innen haben Abitur, vier einen Bachelor- und zehn einen Masterabschluss. 45~Proband\_innen verwenden vornehmlich die rechte Hand, fünf die linke. 


Die Ergebnisse der Studie zu Frequenzeffekten werden in \sectref{freqerg} vorgestellt. Im folgenden Abschnitt wird das methodische Vorgehen in der sentence-maze-Studie zu Prototypizitätseffekten erläutert.

\section{Sentence-maze-Studie zu Prototypizitätseffekten}
\label{metproto}

Der Einfluss der Prototypizität auf Variation wird anhand der Auxiliarselektion von \textit{haben} und \textit{sein} mithilfe einer \textit{sentence maze task} untersucht. Darin wählen Proband\_innen in Sätzen mit unterschiedlichem Transitivitätsgrad zwischen \textit{hat} und \textit{ist}. Im Abschnitt werden zunächst die der Studie zugrunde liegenden Fragestellungen und Hypothesen erläutert und anschließend das methodische Vorgehen  (Materialien, Versuchsdesign und -ablauf, Metadaten, Probandenakquise und Proband\_innen). 

\subsection{Fragestellung und Hypothesen}
\label{prothypo}

In der Studie wird der Frage nachgegangen, inwiefern Prototypizität sich in den Reak\-tionszeiten bei der Selektion von \textit{haben} und \textit{sein} zeigt und wie sie die Selektion beeinflusst. Dabei wird mit verschiedenen Transitivitätsgraden gearbeitet: Prototypisch transitive sowie intransitive Sätze werden gegenübergestellt. Transitive Sätze selegieren \textit{haben}. Bei intransitiven Sätzen beeinflussen Telizität und Bewegungssemantik, ob \textit{sein} selegiert wird (\cite[316--319]{Gillmann.2016}; siehe \sectref{trans} für Erläuterungen zu Transitivität). Da die Testverben Bewegungsverben sind, können zwei Prototypen kontrastiert werden: Transitivität mit \textit{haben}-Selektion und Intransitivität mit \textit{sein}-Selektion. In den prototypisch transitiven Sätzen ist das Objekt ein Patiens, in den prototypisch intransitiven Sätzen ist kein Objekt vorhanden. Im Übergangsbereich zwischen Transitivität und Intransitivität werden Sätze getestet, deren Objekt ambig zwischen Patiens und Instrument ist (\textsc{ambig I}). Zudem werden Sätze mit Satzgliedern getestet, die ein Adverbial darstellen (\textsc{ambig II}), siehe \sectref{protmat} für genauere Erläuterungen zu den Testsätzen. Es werden folgende Hypothesen überprüft:

\subsubsection{Reaktionszeiten}

\begin{enumerate}
\item Die Reaktionszeiten sind bei intransitiven Sätzen niedriger als bei den anderen Sätzen. 
\item Die transitiven Sätze rufen vergleichbare Reaktionszeiten hervor wie \textsc{ambig~II}.
\item \textsc{ambig~I} evoziert die höchsten Reaktionszeiten.
\end{enumerate}
 
Diese Annahmen basieren auf den Ergebnissen der Prästudie. Diese sind im digitalen Anhang dokumentiert.  Ursprünglich wurde angenommen, dass die Reaktionszeiten für \textsc{ambig~I} und \textsc{II}  höher sind als für die Prototypen der beiden Schemata (eindeutige Transitivität und eindeutige Intransitivität), denn durch die Ambiguität sind theoretisch beide Funktionen (Transitivität und Intransitivtät) möglich, weshalb die Entscheidung zwischen \textit{haben} und \textit{sein} länger dauern sollte als bei den prototypisch intransitiven und transitiven Sätzen. Bei den Prototypen der Schemata existiert hingegen nur eine gefestigte Form, sodass die Wahl zwischen \textit{haben} und \textit{sein} schnell erfolgen sollte. Allerdings sind für transitive und ambige Sätze auch etwas höhere Reaktionszeiten möglich als für intransitive Sätze, da diese komplexer sind als die intransitiven Sätze. In der Prästudie hat \textsc{intransitiv} die niedrigsten Reaktionszeiten evoziert, \textsc{ambig~II} hat vergleichbare Reaktionszeiten hervorgerufen wie die transitiven Sätze, während \textsc{ambig~I} die höchsten Reaktionszeiten evoziert hat. Sollte sich dies in der Hauptstudie bestätigen, wären die Reaktionszeiten auch durch die Komplexität der Sätze beeinflusst. Zudem könnten die geringen Reaktionszeiten von \textsc{ambig~II} im Vergleich zu \textsc{ambig~I} einen Hinweis darauf geben, dass \textsc{ambig~II} nur theoretisch ambig ist, sich aber tatsächlich nur eine Form mental gefestigt hat und daher vergleichbare Reaktionszeiten wie bei der Ausprägung \textsc{transitiv} evoziert werden, bei der nur eine Form gefestigt ist.

\subsubsection{Antwortverhalten}
\begin{enumerate}
\item Bei intransitiven Sätzen wird \textit{sein} bevorzugt, bei transitiven \textit{haben}.
\item In den ambigen Sätzen zeigen Proband\_innen eine klare Tendenz zu \textit{sein}, dennoch wird hier mit mehr \textit{haben}-Antworten gerechnet als für die intransitiven Sätze.
\item \textsc{ambig~II} zeigt einen höheren Anteil an \textit{sein}-Antworten als \textsc{ambig~I}.
\item Für alle Sätze ist mit itemspezifischen Präferenzen zu rechnen (bspw. bei \textsc{ambig~I} \textit{sein}-Antworten für \textit{reiten}, aber \textit{haben}-Antworten für \textit{fahren}).
\end{enumerate}


Die Hypothesen zu den ambigen Sätzen basieren auf den Ergebnissen der Prästudie. Ursprünglich wurde Variation zwischen \textit{haben} und \textit{sein} für die ambigen Sätze angenommen. Für \textsc{ambig~I} wurde eine kleine Tendenz zu \textit{haben} erwartet, da die \textit{haben}-Selektion für \textit{fahren} in Sätzen mit determinierten Konkreta, die auch in \textsc{ambig~I} enthalten sind, mit 80 \% deutlich frequenter ist als die \textit{sein}-Selektion und für \textit{fliegen} beide Auxiliare immerhin etwa gleich häufig vorkommen (\cite[288--290]{Gillmann.2016}). Die \textit{sein}-Präferenz für \textsc{ambig~II} war vorab erwartet worden: Die verwendeten Abstrakta stellen Pfad-Akkusative dar (siehe \sectref{protmat}). Für diese stellt \textcite[299--301]{Gillmann.2016} korpuslinguistisch fest, dass sie mit über 90 \% stark zur \textit{sein}-Selektion tendieren. Die itemspezifischen Präferenzen sind vor allem für das Testverb \textit{reiten} zu erwarten, das in der Prästudie ein anderes Antwortverhalten aufwies als \textit{fahren} und \textit{fliegen}.

Die hier vorgestellten Hypothesen sowie das methodische Vorgehen inklusive der geplanten statistischen Auswertung wurden vor der Datenerhebung registriert. Die Registrierung lässt sich unter \url{https://osf.io/6vy8k} aufrufen. Die statistische Auswertung wird zusammen mit den Ergebnissen in \sectref{ergprot} vorgestellt.

\subsection{Materialien} 
\label{protmat}

Als Testitems wurden die Bewegungsverben \textit{fahren}, \textit{reiten} und \textit{fliegen} gewählt, da diese unterschiedliche Transitivitätsgrade zulassen. Sie wurden in folgende Sätze integriert:

\begin{exe}
\ex transitiv (Objekt dient als Patiens)
\begin{xlist}
\ex Er sah, wie die Mutter die Kinder zur Schule gefahren \textbf{hat/*ist}.
\ex Ich nehme an, dass der Pilot die Passagiere zum nächstgelegenen Flughafen geflogen \textbf{hat/*ist}.
 \ex Ich gehe davon aus, dass das Mädchen das Pferd zur Koppel geritten \textbf{hat/*ist}.
\end{xlist}
\ex ambig~I (Objekt könnte Patiens oder ein Instrument sein) 
\begin{xlist}
\ex Er erzählte mir, dass seine Tante eine Zeit lang ein Cabrio gefahren \textbf{hat/ist}.
\ex Ich kann kaum glauben, dass meine Schwester neulich einen Helikopter geflogen \textbf{hat/ist}.
\ex Mir wurde erzählt, dass mein Onkel früher ein schwarzes Pferd geritten \textbf{hat/ist}.
\end{xlist}
\ex ambig~II (Objekt vs. Adverbial)
\begin{xlist}
\ex In den Nachrichten sagten sie, dass der neue Fahrer die Strecke in Rekordzeit gefahren \textbf{ist/hat}.
\ex In der Zeitung stand, dass der Pilot die Distanz in Rekordzeit geflogen \textbf{ist/hat}.
\ex Im Radio hieß es, dass der Reiter den Parcours fehlerlos geritten \textbf{ist/hat}.
\end{xlist}
\ex intransitiv (kein Objekt)
\begin{xlist}
\ex Ich ging davon aus, dass die Oma am Sonntag zur Kur gefahren \textbf{ist/*hat}.
\ex Ich wusste nicht, dass mein Bruder am Mittwoch in die USA geflogen \textbf{ist/*hat}.
 \ex Meine Oma erzählte, dass sie in den Siebzigern regelmäßig durch den Schwarzwald geritten \textbf{ist/*hat}.
\end{xlist}
\end{exe}\largerpage

Die Sätze wurden als Nebensätze konzipiert, damit das Auxiliar das letzte Element im Satz darstellt und somit die Transitivität des Satzes beim Prozessieren des Auxiliars eindeutig ist. Die Testsätze wurden vorab auf ihre semantische Plausibilität geprüft, indem sie 58~Proband\_innen vorgelegt wurden. Alle Sätze wurden dabei als semantisch plausibel bewertet.\footnote{Die Sätze erzielten durchschnittlich mindestens eine Bewertung von 3 auf einer Skala von 1 (nicht plausibel) bis 5 (plausibel). Nur ein Satz musste leicht modifiziert werden, da er mit 2,7 unter dem Grenzwert lag. Ursprünglich lautete der Satz \textit{Mir wurde erzählt, dass mein Onkel während des Studiums ein schwarzes Pferd geritten hat}. \textit{Während des Studiums} wurde durch \textit {früher} ersetzt, sodass der Satz mehr den anderen Sätzen mit \textit{reiten} ähnelt. Zusätzlich musste eine Kleinigkeit korrigiert werden: \textit{Parkour} wurde zu \textit{Parcours} geändert, da \textit{Parkour} normalerweise nicht in Bezug auf Hindernisreiten genutzt wird. Zudem war bei \textsc{ambig II} mit dem Testverb \textit{fliegen} ursprünglich \textit{Bestzeit} vorgesehen, was in \textit{Rekordzeit} geändert wurde. Dies war ein Übertragungsfehler, der die Plausibilität des Satzes jedoch nicht verändern sollte.} Erläuterungen zum Design des Plausibilitätstests und die Plausibilitätsbewertungen für die einzelnen Testsätze sind in Anhang \ref{plausent} zu finden.

\subsection{Versuchsdesign}\label{studieprotodesign}

Die Studie nutzt ein within-subject-Design mit zwei Faktoren: Transitivität (mit vier Ausprägungen: \textsc{transitiv}, \textsc{ambig I}, \textsc{ambig II}, \textsc{intransitiv}) und Auxiliarwahl (mit zwei Ausprägungen: \textit{hat}, \textit{ist}). Die Ausprägungen enthalten jeweils drei Testwörter (\textit{fahren}, \textit{fliegen}, \textit{reiten}). Die Aufgabe der Proband\_innen ist es, Sätze Stück für Stück zusammenzusetzen. Dabei werden den Proband\_innen zwei Möglichkeiten gegeben, die links und rechts auf dem Bildschirm angezeigt werden. Die Möglichkeiten können ausgewählt werden, indem die Taste \textit{f} für das linke Element oder die Taste \textit{j} für das rechte gedrückt wird. Die Aufgabe der Pro\-\mbox{band\_in}\-nen besteht darin, die Möglichkeit zu wählen, die den bisher gelesenen Satz am besten fortsetzt. Beim letzten Element der Testsätze wählen die Proband\_innen jeweils zwischen \textit{hat} und \textit{ist}. Dabei wird die Antwort der Proband\_innen sowie die Reaktionszeit gespeichert.


Zwischen der Tastatureingabe der Proband\_innen und dem Erscheinen des nächsten Satz\-elements ist eine Verzögerung von 0,5 Sekunden vorgesehen. Dies soll spill-over-Effekte vermeiden. Um die Prozessierung des Satzes nicht durch eine lange Pause zu stören und da die Gefahr von spill-over-Effekten bei \textit{sentence maze tasks} relativ gering ist (\cite[164]{Forster.2009}), ist die Pause mit 0,5 Sekunden sehr kurz gehalten. 

Der Matrixsatz der Testsätze wird stets als Ganzes präsentiert. Dabei werden zwei Möglichkeiten angeboten. Die Alternative zum Matrixsatz, der in den Testsätzen vorgesehen ist, wird durch orthographische und grammatische Abweichungen eindeutig als nicht zu favorisieren gekennzeichnet. Die Nebensätze sind in Wörter und Wortgruppen geteilt. Als Alternativen zu den beabsichtigten Wörtern und Wortgruppen werden orthographische (*\textit{Schuhle} statt \textit{Schule}) und grammatische (*\textit{meine Neffe} statt \textit{mein Neffe}) Abweichungen angeboten sowie Wörter, die im Kontext semantisch unpassend sind (\textit{singen} statt \textit{betrachten}). Tabelle \ref{ablaufsentta} in Anhang \ref{blocksent} führt alle Sätze in der Studie samt der Einteilung in Wörter und Wortgruppen sowie den Alternativen zu den intendierten Wörtern und Wortgruppen auf. Die Sätze in der Tabelle sind nach ihrem Vorkommen in den Blöcken sortiert. 

 
Das Ende des Matrixsatzes wird jeweils durch Kommata angezeigt, das Ende des Satzgefüges jedoch nicht durch Punkte. Auf diese Weise können wrap-up-Effekte am Ende eines Satzes (\cite{Rayner.2000}) minimiert und außerdem verhindert werden, dass die Proband\_innen das letzte Element des Satzes unkonzentriert wählen, da die Proband\_innen beim Prozessieren des letzten Items in einem Satz noch nicht wissen, ob der Satz noch fortgeführt wird oder nicht.



In den Testsätzen haben Proband\_innen bei dem Partizip~II vor dem Auxiliar stets die Wahl zwischen dem Testverb und einem anderen, semantisch unplausiblen Bewegungsverb (bspw. \textit{gefahren} oder \textit{geschwommen} im Satz \textit{Er sah, wie die Mutter die Kinder zur Schule gefahren hat}). So kann ausgeschlossen werden, dass die Alternative zum intendierten Vollverb die Wahl des  Auxiliars beeinflusst. Dies könnte passieren, wenn statt \textit{geschwommen} als semantisch unplausible Alternative \textit{gelacht} zur Auswahl stünde, das immer \textit{haben} selegiert.



Als Filler dienen Sätze, die Maskulina mit schwacher bzw. starker Flexion enthalten, sowie Sätze mit ähnlichem Inhalt wie die Testsätze. Die Maskulina sind Versuch\-items der sentence-maze-Studie zum Einfluss der Form-Schematizität auf Deklination (siehe \sectref{schemalex}). Die Füllsätze sind wie die Testsätze Satzgefüge. Zusätzlich zu den Füllsätzen lenkt das gesamte Untersuchungsdesign vom Zweck der Studie ab, da die Proband\_innen semantische, grammatische sowie orthographische Abweichungen vermeiden müssen. Als Test, ob das Design funktioniert, dienen Genusfehler (\textit{der Kurier/*die Lieferant}), Rechtschreibfehler (\textit{Schule}/\textit{*Schuhle}) sowie Varianten (\textit{Fahrrad fahren}/\textit{fahrradfahren}; \textit{fehlerlos}/\textit{fehlerfrei}). Diese Items sind nur relevant, wenn für die Testitems kein Unterschied in den Reaktionszeiten festzustellen ist. Bei Genus- und Rechtschreibfehlern ist ein klares und schnelles Antwortverhalten zu erwarten, bei den Varianten jedoch nicht. Somit eignen sich die Items, um zu überprüfen, ob das Design generell Unterschiede in Reaktionszeiten hervorruft.

Die zu bevorzugende Möglichkeit erscheint  gleich häufig links und rechts (jeweils 86-mal), um keine Verzerrung der Reaktionszeiten durch die Antwortverteilung hervorzurufen. Bei~28~der Items sind die Möglichkeiten Varianten, sodass hier kein Antwortverhalten antizipiert werden kann. Dabei sind auch Möglichkeiten als Variante kodiert, die keine Variante darstellen, aber häufig verwechselt werden wie bspw. \textit{dass} und \textit{das} sowie \textit{seit} und \textit{seid}. 


Auch innerhalb der Blöcke, in die die Test- und Fillersätze geteilt sind, ist die zu bevorzugende Möglichkeit gleich häufig links und rechts mit maximaler Abweichung um eins. Davon abgesehen ist die Verteilung der Antwortmöglichkeiten innerhalb der Blöcke zufällig, sodass keine Muster zu antizipieren sind. Die Proband\_innen müssen nicht  nur in den Testsätzen zwischen  \textit{hat} und \textit{ist} wählen, sondern auch in vier Füllsätzen, sodass die Auxiliarwahl nicht mit den Testsätzen assoziiert wird. Damit die Auxiliare nicht immer nur am Ende der Sätze erscheinen, werden zwei der Füllsätze nach den Auxiliaren fortgeführt. 



Insgesamt werden vier Experimentblöcke mit jeweils sechs Sätzen präsentiert. Die Experimentblöcke werden in randomisierter Reihenfolge angezeigt, aber die Reihenfolge der Sätze innerhalb der Blöcke ist fest. In den Blöcken werden jeweils drei Testsätze genutzt. Diese entstammen jeweils unterschiedlichen Prototypizitätsausprägungen und enthalten unterschiedliche Testverben. Die anderen drei Sätze sind Füllsätze. In den Blöcken wechseln sich Füllsätze und Testsätze regelmäßig ab (siehe Tabelle \ref{ablaufsentta} in Anhang \ref{blocksent} für einen Überblick über den Aufbau der Blöcke). Um Priming-Effekte zu vermeiden, wird in den Sätzen vor den Testsätzen der Ausprägungen \textsc{ambig~I} und \textsc{ambig~II} jeweils kein Auxiliar verwendet. In Sätzen vor den transitiven Testsätzen kommt stets \textit{ist} vor, während in Sätzen vor den intransitiven Testsätzen stets \textit{hat} oder \textit{habe} steht. So kann ein möglicher Priming-Effekt die Auxiliarselektion nicht überlagern, da sich der Effekt auf das gegenteilige Auxiliar auswirkt. 


Zusätzlich zu den Experimentblöcken lesen die Proband\_innen einen Beispielblock mit einem Beispielsatz sowie einen Eingewöhnungsblock mit drei Füllsätzen, der den Einstieg in das Experiment erleichtern soll.  
 
\subsection{Versuchsablauf}

Der Versuchsablauf ist identisch mit dem Ablauf in der lexical-decision-Studie zu Frequenzeffekten (siehe \sectref{ablauffreq}). Nur die Aufgabe der Proband\_innen weicht logischerweise von der lexical-decision-Studie ab: Wählen Sie die Möglichkeit, die für Sie am besten zu dem bisher gelesenen Satz passt. Genau wie die lexical-decision-Studie wurde auch die sentence-maze-Studie mit PsychoPy3 (\cite{Peirce.2019}) erstellt und  online mithilfe der Plattform Pavlovia (\cite{PsychoPyTeam.2019}) durchgeführt.

\subsection{Metadaten, Probandenakquise und Proband\_innen}
\label{probprototy}

Es wurden dieselben Metadaten erhoben wie für die lexical-decision-Studie zu Frequenzeffekten (siehe \sectref{studiefreqmeta}) und dieselben Ausschlusskriterien angewandt.\footnote{Die Konzentriertheit der Proband\_innen wurde anhand von 132 Filleritems bestimmt. Für diese~132~Items ist die dispräferierte Antwortmöglichkeit leicht als solche zu erkennen. Dies ist bspw. der Fall bei orthographischen und grammatischen Abweichungen. Bei den orthographischen Abweichungen wurden nur auffällige Abweichungen (*\textit{Schuhle}) für die Evaluation des Antwortverhaltens genutzt. Zusammen- und Getrenntschreibung wurde daher nicht berücksichtigt. Auch eindeutig geregelte Fälle wie \textit{dass/das} wurden nicht genutzt, da der Diskurs um \textit{dass/das} darauf schließen lässt, dass die Unterscheidung für Schreiber\_innen problematisch sein kann. Ab 10 \% (13) falsch beantworteter Items wurden Proband\_innen ausgeschlossen.} Die angestrebte Stichprobengröße von 50 Proband\_innen basiert auf einer Datensimulation, die mithilfe der Daten aus der Prästudie\footnote{Die Stichprobengröße der Prästudie betrug 26 Proband\_innen nach Ausschluss von Proband\_innen, die die Kriterien nicht erfüllten. Wie in der lexical-decision-Studie zu Frequenzeffekten ist die Prästudie äquivalent zur Hauptstudie. Nur der aktuelle Wohnort wurde in der Prästudie nicht abgefragt. Die einzige Abweichung in den Ausschlusskriterien ist, dass in der Prästudie keine Proband\_innen ausgeschlossen wurden, die aus bestimmten Bundesländern kommen. Die Metadaten der Proband\_innen und die Ergebnisse der Prästudie sind im digitalen Anhang aufgeschlüsselt.} durchgeführt wurde. Die Prästudie deutet darauf hin, dass das Auxiliar in intransitiven Sätzen schneller gewählt wird als in den anderen Konditionen und dass die Ausprägungen \textsc{transitiv} sowie \textsc{ambig~II} niedrigere Reaktionszeiten hervorrufen als \textsc{ambig~I}. Unter der Annahme, dass diese Schätzung robust ist, reichen 50 Pro\-\mbox{band\_in}\-nen bei einem $\alpha$-Level von 0,01 für eine Teststärke von mindestens 0,89 (für den Schätzwert der Ausprägung \textsc{ambig~I} im Vergleich zu den Schätzwerten der anderen Ausprägungen) aus. Eine Dokumentation des Codes für die Datensimulation ist im digitalen Anhang zu finden.

 

Um die Stichprobengröße von 50 Proband\_innen zu erreichen, wurden insgesamt 170 Personen akquiriert. Zwar mussten in der Prästudie mit sechs von 32 Proband\_innen nur 19~\% ausgeschlossen werden, da mit der Prototypizitätsstudie jedoch auch die Daten für die sentence-maze-Studie zu Form-Schematizität (siehe \sectref{schemalex}) erhoben wurden, in der 125~Proband\_innen benötigt werden, mussten deutlich mehr Proband\_innen akquiriert werden als für die Prototypizitätsstudie nötig sind.

\begin{sloppypar}
Abweichend von der Frequenzstudie wurden für die Prototypizitätsstudie nur Proband\_in\-nen berücksichtigt, die nicht in Bayern, Baden-Württemberg, Rhein\-land-Pfalz oder Saarland aufgewachsen sind oder dort derzeit leben. In diesen Bundesländern werden Positionsverben mit \textit{sein} genutzt (\cite{VariantengrammatikdesStandarddeutschen.2018}). Um eine Interferenz von Positions- auf Bewegungsverben auszuschließen, werden Proband\_innen aus diesen Bundesländern daher ausgeschlossen. Die Datenerhebung lief vom 9.1.2020 bis zum 2.2.2020. Um Proband\_innen zu akquirieren, wurde der Link zur Studie an Dozent\_innen sowie Vertreter\_innen der Studierendenschaft\footnote{Ich danke Alexander Werth, Heike Zinsmeister, Tanja Stevanovic, Lena Schnee, Oliver Czulo, Susanne Flach, Alan Lena van Beek, Patrick Grommes, Konstanze Marx, Ilka Lemke und Katharina Böhnert für das Weiterleiten des Studienaufrufs.} an deutschen Universitäten versendet mit der Bitte, Studierende, die sich am Anfang und in der Mitte des Bachelorstudium befinden, zur Teilnahme aufzurufen. Zudem wurden Teilnehmer\_innen über mein persönliches Umfeld akquiriert.
\end{sloppypar}

Nach Bereinigung der Daten bleiben 107 Proband\_innen. Wie in der lexical-decision-Studie zu Frequenzeffekten war die Angabe von Metadaten freiwillig, weshalb nicht zu allen Metadaten Angaben vorliegen. Eine tabellarische Aufschlüsselung aller Metadaten ist im digitalen Anhang zu finden.  

\begin{sloppypar}
Die Proband\_innen sind durchschnittlich 24 Jahre alt mit einer Standardabweichung von fünf~Jahren. 15 identifizieren sich als männlich, 91 als weiblich. 29 Proband\_innen kommen aus Niedersachsen, 26 aus Nordrhein-Westfalen, 14 aus Schleswig-Holstein. Die restlichen Bundesländer (bis auf die ausgeschlossenen Bundesländer) entfallen auf maximal sechs Personen, viele Proband\_innen haben in mehreren Bundesländern gelebt. Derzeit wohnen 26 der Pro\-\mbox{band\_in}\-nen in Nordrhein-Westfalen, 21 in Niedersachsen, 17 in Hamburg und 13 in Mecklenburg-Vorpommern und Schleswig-Holstein. Andere Bundesländer sind maximal mit sechs Personen vertreten, eine Probandin lebt im Ausland. 95 der Proband\_innen gaben an, täglich zu lesen, elf wöchentlich und eine Person monatlich. 104 Proband\_innen sprechen in ihrem persönlichen Umfeld Standard, zwei Dialekt und eine Person eine Mischform. 74 Proband\_innen haben Abitur, 18 einen Bachelor- und acht einen Masterabschluss. 94 nutzen die rechte Hand, elf die linke, eine Person beide Hände.
\end{sloppypar}

Die Ergebnisse der Studie zu Prototypizitätseffekten werden in \sectref{ergprot} erläutert. Im nächsten Abschnitt wird die self-paced-reading-Studie zu Form-Sche\-ma\-ti\-zi\-täts\-ef\-fek\-ten vorgestellt.

\section{Self-paced-reading-Studie zu Form-Schematizitätseffekten}
\label{methschema}

Der Einfluss von Form-Schematizität auf Variation wird anhand des Form-Sche\-mas~I\footnote{Im Folgenden wird verkürzend \textit{Form-Schema} genutzt, da das Form-Schema~II für die empirische Untersuchung nicht berücksichtigt wurde.} schwacher Maskulina untersucht. Dafür wird neben anderen Studien eine self-paced-reading-Studie durchgeführt. In der self-paced-reading-Studie lesen Proband\_innen Pseudosubstantive, die dem Form-Schema schwacher Maskulina zu unterschiedlichen Graden entsprechen. Diese werden mit schwachen und starken Flexionsformen präsentiert. Die Studie ist aus einer vorherigen Studie (\cite{Schmitt.2019}) entstanden, in der konfligierende Ergebnisse zum Einfluss des Form-Schemas schwacher Maskulina auf Lesezeiten festgestellt wurden (siehe \sectref{schemamask}). Daher wird der fragliche Einfluss des Form-Schemas auf Lesezeiten in der vorliegenden Studie mit einem ähnlichem Design überprüft. Um die generelle Aufmerksamkeit der Proband\_innen gezielt auf Wortendungen zu lenken, wurde gefragt, welche Wortform gelesen wurde (Haben Sie \textit{des Schettosen} oder \textit{des Schettoses} gelesen?). Auf diese Weise sollte die Wahrscheinlichkeit verringert werden, dass Proband\_innen Unterschiede in der Flexion überlesen. 


In diesem Abschnitt werden zunächst die der Studie zugrunde liegenden Fragestellungen und Hypothesen erläutert. Anschließend wird das methodische Vorgehen (Materialien, Versuchsdesign und -ablauf, Metadaten, Probandenakquise und Proband\_innen) für die Studie beschrieben.  


Der Einfluss von Form-Schemata wird nicht nur durch \textit{self-paced reading} evaluiert: Schwache Maskulina mit unterschiedlichem Form-Schematizitätsgrad werden auch als Filler in den Studien zu Frequenz- und Prototypizitätseffekten  (siehe \sectref{metfreq} und \ref{metproto}) genutzt, sodass bezüglich des Einflusses von Form-Sche\-ma\-ti\-zi\-tät auf Variation die Ergebnisse mehrere Stu\-dien verglichen werden können.

\subsection{Fragestellung und Hypothesen}
\label{lesehypo}

Die Studie geht der Frage nach, welchen Einfluss das Form-Schema schwacher Maskulina auf die Prozessierung und Produktion schwacher und starker Formen von Maskulina nimmt. Dabei wird mit Pseudosubstantiven gearbeitet, die dem Form-Schema schwacher Maskulina zu unterschiedlichen Graden entsprechen (siehe \sectref{schemamask} für ausführliche Erläuterungen zu Form-Schemata bei Maskulina): Es werden Pseudosubstantive getestet, die dem Prototyp des Form-Schemas entsprechen, in der Peripherie des Form-Schemas sind, und Substantive, deren phonologische Eigenschaften schwache Deklination ausschließen, da sie zum Form-Schema starker Maskulina gehören (siehe \sectref{methschemamaterial} für genauere Erläute\-rungen zu den verwendeten Materialien). Im Anschluss an die \textit{self-paced reading task} erhalten die Teilnehmer\_innen einen Lückentext, in dem sie die vorgegebenen Testwörter deklinieren müssen. Folgende Hypothesen werden überprüft:

\subsubsection{Lesezeit}

\begin{enumerate} 
\item Schwache Formen von Substantiven, die prototypische Vertreter des Form-Schemas schwacher Maskulina sind, lösen geringere Lesezeiten aus als starke Formen. Im Gegensatz dazu lösen schwache Formen von Substantiven, die dem Form-Schema starker Maskulina angehören, höhere Lesezeiten aus als starke Formen.
\item Bei Testsubstantiven, die in der Peripherie des Form-Schemas schwacher Maskulina sind, sind vergleichbare Lesezeiten für beide Formen messbar.
\end{enumerate}

Bestätigen sich die Hypothesen, können die nicht existenten Unterschiede in der Lesezeit bei Substantiven in der Peripherie des Form-Schemas als ein Hinweis auf Varianten interpretiert werden: Da nicht nur eine, sondern zwei Formen mental gefestigt sind, stellt keine der beiden Formen eine Herausforderung für die Prozessierung dar.

\subsubsection{Produktion}
\begin{enumerate} 
\item Substantive, deren Eigenschaften prototypisch zum Form-Schema schwacher Maskulina passen, werden vornehmlich schwach dekliniert. Im Gegensatz dazu werden Substantive, die der starken Flexion angehören, vorrangig stark dekliniert.
\item Bei Substantiven, die in der Peripherie des Form-Schemas sind, werden schwache sowie starke Formen gewählt.
\end{enumerate}

\subsection{Materialien}
\label{methschemamaterial}

Folgende Pseudowörter werden für die self-paced-reading-Studie genutzt:

\begin{enumerate}
\item Prototyp des Form-Schemas schwacher Maskulina: \textit{der Schettose}; \textit{der Truntake} 
\item Peripherie des Form-Schemas schwacher Maskulina: \textit{der Knatt} 
\item Form-Schema starker Maskulina: \textit{der Grettel}
\end{enumerate}

Alle Pseudowörter werden über menschliche Referenten eingeführt, um den Einfluss der Semantik auf das Form-Schema und somit auf die Flexion konstant zu halten. Die genutzten Testitems sind \textcite{Kopcke.2000b} entnommen, der in seinem Produktionsexperiment mit diesen Pseudowörtern den Einfluss von Form-Schemata auf das Flexionsverhalten nachweisen konnte. Um zu verhindern, dass Assoziationen mit tatsächlich existierenden Lexemen Einfluss auf das Leseverhalten nehmen, wurden 90 Student\_innen\footnote{Es handelt sich um Student\_innen der Heinrich-Heine-Universität Düsseldorf sowie der Johannes Gutenberg-Universität Mainz. Ich bedanke mich bei Johanna Flick und Jessica Nowak für die Hilfe bei der Datenerhebung.} nach den ersten drei Assoziationen mit dem jeweiligen Pseudosubstantiv gefragt. Dabei wurden neben den Testwörtern Alternativwörter (\textit{Trilch} und \textit{Fletter}) abgefragt. Einen ähnlichen Assoziationstest führt auch \textcite[160]{Kopcke.2000b} durch. Die ausgewählten Testitems lösten dabei keine problematischen Assoziationen aus, während \textit{Fletter} mit \textit{Vetter} und \textit{Trilch} mit \textit{Knilch} assoziiert wurde. Sowohl \textit{Vetter} als gemischt flektierendes Substantiv als auch \textit{Knilch} als starkes Substantiv könnten Einfluss auf die Deklination nehmen, weswegen sie nicht ausgewählt wurden. Siehe Anhang \ref{asso} für weitere Erläuterungen zum Assoziationstest und einen Überblick über die Assoziationen.

\subsection{Versuchsdesign}
\label{versuchspr}

Die Versuchitems werden in Texten präsentiert, die wie Lexikonartikel aufgebaut sind, siehe Beispiel \ref{beispielschettose}.\footnote{Im Beispiel werden zunächst schwache und dann starke Genitivformen von \textit{Schettose} genutzt. Diese Reihenfolge war jedoch nicht fest, stattdessen wurde pseudorandomisiert zwischen dieser und der gegensätzlichen Reihenfolge gewechselt, wie unten erläutert wird.} Die Proband\_innen haben die Aufgabe, die Texte zu lesen. Nach jedem Satz\footnote{Vereinzelt werden zwei Sätze gelesen, bevor nach dem Aufbau eines Worts in den Sätzen gefragt wird. Dies ist vor allem bei sehr kurzen Sätzen der Fall.} werden sie nach einem Wort gefragt, das im Satz vorkam (Haben Sie \textit{Amtsschreiber} oder \textit{Amtschreiber} gelesen?). Da nach jedem Satz nach dem Aufbau eines darin enthaltenen Worts gefragt wird, wird die Aufmerksamkeit der Proband\_innen auf die Endungen der Testitems gelenkt, ohne den Untersuchungsgegenstand zu verraten. Zudem werden nicht nur Wortendungen abgefragt, sondern auch Fugenelemente (\textit{Amtsschreiber} vs. \textit{Amtschreiber}) und Umlaute in den Distraktortexten, die ebenfalls Pseudosubstantive enthalten (\textit{Brachte} vs. \textit{Brächte}).\largerpage


\begin{exe}
\ex \label{beispielschettose} 
\textbf{Schettose, der}\\
Betonung: Schettóse\\
Amtsbezeichnung am ungarischen Hof, Amtsschreiber/Diener\medskip\\
Der Schettose war ähnlich gestellt wie ein Amtsschreiber, zusätzlich erfüllte er jedoch niedere Dienste. Der Schettose wurde vom König direkt ernannt.  Somit war das Amt \textbf{des Schettosen} offensichtlich ein verantwortungsvoller Posten. Es ist davon auszugehen, dass die Übertragung dieses Amts als Ehre galt. Interessanterweise ließ sich außer dem Amtsschreiber kein vergleichbares Amt an anderen Höfen beobachten. Dementsprechend war das Amt \textbf{des Schettoses} offensichtlich ein Spezifikum des ungarischen Könighofs.
\end{exe}

Die Texte sind als \textit{self-paced reading task} mit \textit{moving window} gestaltet. Um spill-over-Effekte zu minimieren, wird auf eine Wort-für-Wort-Präsentation verzichtet, stattdessen werden den Proband\_innen Wortgruppen präsentiert (\cite[121--122]{McDonough.2012}). Die Wortgruppen entsprechen meist Präpositional- und Nominalphrasen. Verben werden nicht einzeln gelesen, sondern gemeinsam mit einer weiteren Konstituente. Adverbien werden zu anderen Phrasen hinzu genommen oder einzeln präsentiert. Die Texte sind inklusive der Wortgruppeneinteilung und der Fragen zu den Testwörtern und Distraktoren (z.~B. Haben Sie \textit{des Schettoses} oder \textit{des Schettosen} gelesen?) in Anhang \ref{testitemself} zu finden.



Die Testitems werden in einer vergleichbaren syntaktischen Umgebung präsentiert. So sind sie stets Teil der dritten Konstituente im Satz und stellen ein Genitivattribut dar. Genitivformen wurden gewählt, da hier die starken und schwachen Formen gleich lang sind. Die einzige Ausnahme davon bildet das Testsubstantiv \textit{Truntake}, das im Dativ präsentiert wird. Auf diese Weise ist ein Vergleich von Genitiv- und Dativformen bei zwei prototypischen Vertretern des Form-Schemas schwacher Maskulina (\textit{Schettose}, \textit{Truntake}) möglich. Die Beispiele~\ref{satzschettose} bis~\ref{satztruntake} geben einen Überblick über die Sätze, in die die Testsubstantive eingeflochten sind:

\begin{exe} 
\ex \label{satzschettose} \textit{Schettose} 
\begin{xlist}
\ex Somit war \textbf{das Amt des Schettosen/*Schettoses} offensichtlich ein verantwortungsvoller Posten.
\ex\sloppy Dementsprechend war \textbf{das Amt des Schettosen/*Schettoses} offensichtlich ein Spezifikum des ungarischen Könighofs.
\end{xlist}
\ex \textit{Knatt} \begin{xlist}
\ex Mit der Zeit setzte sich \textbf{der Beruf des Knatts/Knatten} anscheinend auch in anderen Regionen durch.
\ex Auch in der Stadt setzte sich \textbf{der Beruf des Knatts/Knatten} anscheinend nie durch.  
\end{xlist} 
\ex \label{satzgrettel}  \textit{Grettel} 
\begin{xlist}
\ex Deswegen ist \textbf{das Amt des Grettels/*Gretteln} in vielen Vereinen nur für langjährige Mitglieder vorgesehen. 
\ex Dementsprechend ist \textbf{das Amt des Grettels/*Gretteln} in vielen Vereinen nicht besetzt.   
\end{xlist}
\ex \label{satztruntake} \textit{Truntake}
\begin{xlist}
\ex Jedoch kam \textbf{dem Truntaken/*TruntakeØ} üblicherweise keine Urteilskraft zu.
\ex Jedoch kam \textbf{dem Truntaken/*TruntakeØ} üblicherweise auch eine unangenehme Aufgabe zu: der Schuldspruch.
\end{xlist}
\end{exe}\largerpage

Die Testitems werden jeweils als Bigram (bspw. \textit{des Gretteln} bzw. \textit{des Grettels}) präsentiert. Da spill-over-Effekte möglich sind, wird das Element nach dem Versuchitem jeweils konstant gehalten (\textit{offensichtlich}, \textit{anscheinend}, \textit{in vielen Vereinen}, \textit{üblicherweise}), sodass die Lesezeiten der Elemente verglichen werden können. Nach den Sätzen, die Testitems enthalten, wird jeweils nach den Endungen der Testsubstantive gefragt (bspw. Haben Sie \textit{des Knatten} oder \textit{des Knatts} gelesen?). 

Die Reihenfolge der Genitivformen ist pseudo-randomisiert: Ein Teil der Proband\_innen liest zunächst die zu erwartenden Formen (\textit{des Schettosen}, \textit{des Knatts}, \textit{des Grettels}, \textit{dem Truntaken}) und dann nicht-erwartbare Formen (\textit{des Schettoses}, \textit{des Knatten}, \textit{des Gretteln}, \textit{den Truntake}). \textit{Des Knatten} wurde dabei als weniger erwartbar als \textit{des Knatts} eingestuft, weil -\textit{n} die typeninfrequente Flexion darstellt. Der andere Teil der Proband\_innen liest die Testitems in umgekehrter Reihenfolge. Die Reihenfolgenrandomisierung dient dazu, Reihenfolge-Effekte zu vermeiden: Proband\_innen neigen bei \textit{self-paced reading tasks} dazu, schneller zu werden (\cite[166--168]{Ackermann.2017}), somit könnte die Lesezeit der Items durch die Reihenfolge beeinflusst werden. Auch die Reihenfolge der Formen in den Fragen ist auf die Gruppen verteilt: In der ersten Gruppe wird in der Frage nach dem Aufbau der Testitems stets die erwartbare Form zuerst genannt (Haben Sie \textit{des Schettosen} oder \textit{des Schettoses} gelesen?), in der zweiten Gruppe die nicht-erwartbare Form (Haben Sie \textit{des Schettoses} oder \textit{des Schettosen} gelesen?).


  

 

Im Gegensatz zu den Genitivformen der Testitems sind die Texte selbst nicht randomisiert, sondern in konstanter Reihenfolge gehalten. Dabei sind die Testitems nach potentieller Sa\-lienz sortiert: \textit{Knatt} wird als erstes Testitem genutzt, da hier aufgrund der Peripherie des Form-Schemas keine Flexion präferiert werden sollte. \textit{Schettose} wird als zweites Testitem präsentiert, darauf folgt \textit{Grettel} und schließlich \textit{Truntake}. Da schwache Maskulina weniger typenfrequent sind als starke, wird davon ausgegangen, dass die starke Deklination von schwachen Maskulina (\textit{des Schettoses}) weniger salient ist als die schwache Deklination von starken Maskulina (\textit{des Gretteln}) (siehe \sectref{freqmask} zu Typen- und Tokenfrequenz von Maskulina). \textit{Truntake} wurde nicht aufgrund seiner Salienz als letztes Item genutzt, sondern um zu verhindern, dass die dreisilbigen Testitems dicht aufeinander folgen.  

Neben den Testtexten werden Distraktortexte genutzt. Hierfür wird auf Pseudosubstantive aus \textcite{Kopcke.2000b} zurückgegriffen. Die Distraktoren (\textit{die Taff}, \textit{die Schirr}, \textit{die Bracht}) werden mit variierenden Pluralformen (\textit{die Täffe}/\textit{Taffe}/\textit{Taffs}) eingeflochten. Die Fragen zum Wortaufbau lenken die Aufmerksamkeit der Proband\_innen auf die Flexionsformen der Distraktoren (\textit{Brachte}/\textit{Brächte}/\textit{Brächten}) (siehe Anhang \ref{distraktorself} für einen Überblick über die Distraktortexte). Distraktor- und Testsätze wechseln sich im Experiment regelmäßig ab, nur der Text zum Testwort \textit{Truntake} folgt direkt auf den Text zu \textit{Grettel} (siehe Anhang \ref{ablaufpaced}).  Vor den Test- und Distraktortexten wird ein Beispielsatz präsentiert, um die Proband\_innen an die Aufgabe zu gewöhnen. Auch im Beispiel wird nach dem Wortaufbau (\textit{Haben Sie Pizzas oder Pizzen gelesen?}) gefragt, hierbei wird die Aufmerksamkeit bereits allgemein auf Wortendungen gelenkt und die Proband\_innen mit der Aufgabe vertraut gemacht. 



Nach der \textit{self-paced reading task} werden den Proband\_innen Lückensätze vorgelegt, in die Genitivformen der Pseudosubstantive ergänzt werden sollen. Die Sätze entsprechen dabei jeweils den Sätzen aus der \textit{self-paced reading task} (siehe Anhang \ref{prodpaced} für eine Auflistung der Lückensätze). Auch die Distraktoren sind in das Produktionsexperiment eingeflochten. Im Anschluss an das Produktionsexperiment werden die Proband\_innen gefragt, ob ihnen Varianten in den zu lesenden Texten aufgefallen sind, um überprüfen zu können, ob die Varianten in der \textit{self-paced reading task} die Antworten im Produktionsexperiment beeinflusst haben.


\subsection{Versuchsablauf}

Anders als die lexical-decision- und sentence-maze-Studie wurde die self-paced-reading-Studie nicht online durchgeführt. Daher wurden die Proband\_innen von mir sowie von studentischen Hilfskräften betreut, die vorab in den Versuchsablauf eingewiesen wurden. Die Betreuungspersonen hielten sich bei der Durchführung an einen schriftlichen Ablaufplan, der neben den Instruktionen an die Hilfskräfte\footnote{Ich bedanke mich bei Jonas Fehn und  Patricia Pfeffer für die Unterstützung bei der Datenerhebung.} auch generelle Anweisungen zur Studie enthielt, die den Pro\-\mbox{band\_in}\-nen mündlich gegeben wurden. Diese wurden abgelesen, um die Studie für alle Proband\_innen möglichst konstant zu halten (\cite[215]{Hohle.2010}). Die exakten Anweisungen zur \textit{self-paced reading task} erhielten die Proband\_innen per Bildschirm.

Vor den Distraktor- und Testtexten werden ein Begrüßungstext und ein kurzes Beispiel gezeigt (siehe Anhang \ref{ablaufpaced} für Begrüßung und Beispiel), um die Pro\-\mbox{band\_in}\-nen an das \textit{self-paced reading} zu gewöhnen. Im Begrüßungstext werden die Proband\_innen gebeten, die Texte in einer komfortablen Geschwindigkeit zu lesen und auf die Schreibung der Wörter zu achten. Nach Begrüßung und Beispiel werden die Proband\_innen per Bildschirm aufgefordert, Verständnisfragen zu stellen, falls etwas unklar geblieben sein sollte. Die Metadaten wurden nach dem Lesen der Texte mündlich abgefragt und direkt in eine Excel-Tabelle eingetragen. Abschließend wurde den Proband\_innen das Produktionsexperiment vorgelegt, das sie handschriftlich ausfüllten.


Die Studie wurde mithilfe eines Laptops (Windows 7) und eines Response Pads (RB-540) durchgeführt, dabei wurde das Programm DMDX genutzt. Dieses Programm erlaubt die Bestimmung von Lesezeiten mit einer Genauigkeit von Millisekunden (\cite{Forster.2003}). Das Response Pad wurde hierfür im Keyboardmodus genutzt, sodass die Messgenauigkeit bei zehn bis zwölf ms liegt (Herstellerangabe unter \url{https://cedrus.com/rb\_series/}). Das Response Pad hat den Vorteil, dass die Proband\_innen nur aus vier Tasten auswählen müssen und nicht aus über 100 bei einer normalen Tastatur. 


Die Messungen für alle Studien fanden in einem ruhigen Raum statt. Um Störgeräusche von außen zu minimieren, erhielten die Proband\_innen zusätzlich einen Gehörschutz. Den Proband\_innen wurde am Ende der Studie ein Aufklärungsformular vorgelegt und das Einverständnis eingeholt, die Daten anonymisiert zu verwenden.

\subsection{Metadaten}
\label{metadaten}

Für die Studie wurden dieselben Metadaten abgefragt wie für die lexical-decision-Studie zum Einfluss der Frequenz auf Variation (siehe \sectref{metfreq}). Nur der aktuelle Wohnort wurde nicht abgefragt, da die Proband\_innen Bamberger Student\_innen darstellen und somit ein Wohnort in Bamberg und Umgebung angenommen werden kann. Zudem wurden die Fragen nach der präferierten sprechsprachlichen Varietät und nach Dialektkenntnissen als eine Frage zusammengefasst und anders als in den anderen Studien nach dem vermuteten Untersuchungsgegenstand sowie nach Auffälligkeiten im Experiment gefragt, da die Metadaten am Ende des Experiments erhoben wurden. 


Wie in den anderen Studien dienen die Metadaten vornehmlich dazu, sicherzustellen, dass die untersuchte Gruppe möglichst homogen ist. Die Frage nach der Leseintensität ist für \textit{self-paced reading}  relevant, da davon auszugehen ist, dass ungeübte Leser\_innen längere Lesezeiten erzielen als geübte (\cite[43]{Jegerski.2014}). Generell ist jedoch anzunehmen, dass es sich bei den Proband\_innen um geübte Leser\_innen handelt, da Student\_innen getestet werden. Aufgrund des within-subject-Designs sollten allerdings auch ungeübte Leser\_innen das Ergebnis kaum verzerren. Zudem sind Proband\_innen als zufällige Effekte in das statistische Modell eingebunden. 



Der vermutete Untersuchungsgegenstand und Strukturen, die Pro\-\mbox{band\_in}\-nen im Experiment aufgefallen sind, werden abgefragt, um Pro\-\mbox{band\_in}\-nen, die das Untersuchungsinteresse durchschaut haben, ausschließen zu können. Zusätzlich zu den Metadaten werden generelle Kommentare festgehalten, um Pro\-\mbox{band\_in}\-nen (z.~B. aufgrund von missverstandenen Anweisungen) ggf. ausschließen zu können. 

\subsection{Probandenakquise und Proband\_innen}\label{selfpacedprob}\largerpage
\begin{sloppypar}
Als Proband\_innen wurden Studierende der Germanistik an der Otto-Friedrich-Universität Bamberg akquiriert. Zusätzlich nahmen Proband\_innen aus meinem persönlichen Umfeld teil. Um weitgehend auszuschließen, dass Fachwissen die Studienergebnisse beeinflusst, wurden vornehmlich Studierende am Anfang des Bachelorstudiums für die Studie geworben, allerdings nahmen auch einige Masterstudent\_innen an der Studie teil. Die Akquise lief über Seminare ab, die im Sommersemester 2018 an der Otto-Friedrich-Universität Bamberg angeboten wurden. Alle Proband\_innen wurden darauf hingewiesen, dass die Teilnahme an den Studien freiwillig ist, und ihr Einverständnis, dass die Daten anonymisiert gespeichert und ausgewertet werden, wurde schriftlich eingeholt. Das Einverständnis konnte bis zum Ende der Datenerhebung zurückgezogen werden. 
\end{sloppypar}

Es werden nur Teilnehmer\_innen berücksichtigt, die Deutsch als L1 sprechen, monolingual sind und keine Legasthenie haben. Um sicherzustellen, dass die Teilnehmer\_innen die Texte aufmerksam gelesen haben, wird das Antwortverhalten bei den Füllsätzen evaluiert. In den Füllsätzen sollten insgesamt sieben Fragen beantwortet werden. Proband\_innen, die drei oder mehr dieser Sätze falsch beantwortet haben, werden ausgeschlossen. Außerdem werden Proband\_innen ausgeschlossen, die bei den korrespondierenden Testitems (bspw. \textit{Gretteln} und \textit{Grettels}) beide Fragen falsch beantwortet haben, d.~h. jeweils angaben, dass sie die gegensätzliche Flexion gelesen haben. Wie in den anderen Studien werden Proband\_innen ausgeschlossen, deren Reaktionszeiten mehr als drei Standardabweichungen
über oder unter dem Mittelwert eines Testitems liegen. Da in der self-paced-reading-Studie der direkte Vergleich zwischen den Flexionsformen relevant ist, werden hierbei starke und schwache Formen separat betrachtet.

Die Daten für die Studie wurden vom 04.6.2018 bis 29.06.2018 an der Otto-Friedrich-Universität Bamberg erhoben. Insgesamt wurden 61 Proband\_innen akquiriert, von denen 54 Proband\_innen in die Analyse aufgenommen wurden.\largerpage


Alle Proband\_innen machten zu allen Metadaten Angaben. Eine tabellarische Aufschlüsselung der Metadaten sind im digitalen Anhang. Das Durchschnittsalter beträgt 25 Jahre mit einer Standardabweichung von acht Jahren. 45 der Proband\_innen identifizieren sich als weiblich, acht als männlich, eine Person machte keine Angabe. Die Proband\_innen stammen vornehmlich aus Bayern (28). Die meisten Proband\_innen sprechen in ihrem familiären Umfeld vornehmlich Standard (21), weitere 17 Proband\_innen geben an, Standard mit dialektalem Einschlag zu sprechen. Die restlichen Proband\_innen nannten einen Dialekt als bevorzugte sprechsprachliche Varietät. Die meisten Proband\_innen nannten Abitur als höchsten Bildungsabschluss (40), eine Person Fachabitur oder mittlere Reife und die restlichen 12~höhere Abschlüsse. Die meisten Proband\_innen lesen täglich (49), nur fünf geben an, gelegentlich zu lesen. 48~Proband\_innen sind Rechtshänder\_innen, davon wurde eine Person auf rechts umgeschult. Fünf Proband\_innen sind Linkshänder\_innen, eine Person nutzt beide Hände gleichwertig.


Die Ergebnisse der self-paced-reading-Studie zu Form-Schematizitätseffekten werden in \sectref{ergschema} vorgestellt. Im folgenden Abschnitt wird das methodische Vorgehen in der lexical-decision- und der sentence-maze-Studie zu Form-Sche\-ma\-ti\-zi\-täts\-ef\-fek\-ten erläutert.

\section[Lexical-decision- und sentence-maze-Studie zu Form-Schematizität]{Lexical-decision- und sentence-maze-Studie zu Form-Schematizitätseffekten}
\label{schemalex}

In diesem Abschnitt werden die lexical-decision- sowie die sentence-maze-Studie zum Einfluss von Form-Schematizität auf Variation beschrieben. Da die beiden Studien viele Ähnlichkeiten aufweisen, werden sie gemeinsam vorgestellt. Beide Studien arbeiten mit starken und schwachen Formen real existierender Substantive, die dem Form-Schema schwacher Maskulina zu unterschiedlichem Graden entsprechen. In der lexical-decision-Studie bewerten die Proband\_innen die starken und schwachen Formen nach ihrer Bekanntheit, in der sentence-maze-Studie wählen die Proband\_innen zwischen starken und schwachen Formen. Im Folgenden werden die Fragestellungen und Hypothesen sowie das methodische Vorgehen (Materialien, Versuchsdesign und -ablauf, Metadaten, Probandenakquise und Proband\_innen) vorgestellt.

\subsection{Fragestellung und Hypothesen}
\label{hyposchema}

Die Studien gehen der Frage nach, welchen Einfluss das Form-Schema schwacher Maskulina auf Reaktionszeiten und die Wahl von Flexionsformen nimmt. Anders als in der self-paced-reading-Studie werden real existierende Substantive getestet. Die Testsubstantive entsprechen entweder dem Prototyp des Form-Schemas, sind in der Peripherie des Form-Schemas oder stellen starke Substantive dar. Dabei werden keine starken Maskulina auf -\textit{el} genutzt, sondern monosyllabische wie \textit{der Vogt} (siehe \sectref{schemalexmaterial} für ausführliche Erläuterungen zu den Materialien). Folgende Hypothesen werden überprüft:

\subsubsection{Reaktionszeiten}
\begin{enumerate}
  \item In der \textit{lexical decision task} ist eine Interaktion zwischen Deklinationsklasse und Flexion zu beobachten: Schwache Formen lösen bei schwachen Maskulina unabhängig von deren Passgenauigkeit zum Form-Schema niedrige Reaktionszeiten aus, bei starken Maskulina hingegen hohe. Das Gegenteil wird für starke Formen angenommen. Es wird also nur ein Effekt der Deklinationsklasse, aber kein Form-Schematizitätseffekt erwartet.

  \item In der \textit{sentence maze task} evozieren Substantive, die dem Form-Schema peripher angehören, geringere Reaktionszeiten  als Substantive, die prototypisch zum Form-Schema gehören oder stark sind.
\end{enumerate}

Diese Hypothesen basieren auf den Ergebnissen der Prästudien. Die Prästudie ist im digitalen Anhang dokumentiert. Ursprünglich wurde für beide Studien erwartet, dass die Reaktionszeiten bei Substantiven niedrig sind, die eindeutig zum Form-Schema passen oder der starken Flexion angehören, aber erhöht bei Substantiven, die nur peripher dem Form-Schema angehören, da hier zwei Formen möglich sind und die Entscheidung daher verlangsamt sein könnte. Auch wenn in den Studien auf Tokenfrequenz kontrolliert wird (siehe \sectref{schemalexmaterial}), könnten sich die niedrigen Reaktionszeiten für die peripheren Substantive in der sentence-maze-Studie durch unterschiedliche Tokenfrequenzen erklären lassen, dies wird in der explorativen Analyse der Daten getestet. Ist dies der Fall, könnte es sein, dass Tokenfrequenz die zu erwartenden Form-Schematizitätseffekte überdeckt. Sind hingegen höhere Reaktionszeiten für die Substantive in der Peripherie des Form-Schemas messbar, kann dies als ein Hinweis auf Varianten interpretiert werden: Es existiert nicht nur eine einzige, stark gefestigte Form, stattdessen sind es zwei Formen. Hierdurch verlangsamt sich die Beurteilung der Formen.

\subsubsection{Antwortverhalten}
\begin{enumerate}\sloppy
  \item Passen die phonologischen Eigenschaften prototypisch zum Form-Schema schwacher Maskulina, wird die schwache Form -\textit{(e)n} in der lex\-i\-cal-de\-ci\-sion-Stu\-die als bekannt bewertet bzw. in der sentence-maze-Studie gewählt. Bei Substantiven der starken Deklinationsklasse wird -\textit{(e)n} hingegen als unbekannt bewertet bzw. nicht gewählt. Das Gegenteil wird für die Bekanntheit/Wahl der starken Form -\textit{(e)s} angenommen.
  \item In der lexical-decision-Studie weisen schwache Formen von Substantiven aus der Peripherie des Form-Schemas ähnliche Bekanntheitswerte auf wie schwache Formen von prototypischen Vertretern des Form-Schemas. Die starken Formen werden aber bei Substantiven in der Peripherie ebenfalls als bekannt bewertet. Dabei werden für die schwachen Formen von Substantiven in der Peripherie des Form-Schemas hohe Bekanntheitswerte erwartet (über 90~\%), für starke hingegen nur eine Bekanntheit von ca. 50~\%.
  \item In der lexical-decision-Studie antworten Teilnehmer\_innen bei Substantiven, die zum Form-Schema gehören, und Substantiven, die der starken Deklination angehören, konsistent (Bekanntheit einer Form und Unbekanntheit der anderen). In der Peripherie des Form-Schemas sind hingegen auch inkonsistente Antworten möglich (Bekanntheit/Unbekanntheit beider Formen).
  \item In der sentence-maze-Studie evozieren die Testsubstantive in der Peripherie des Form-Schemas ein ähnliches Antwortverhalten wie die Testsubstantive, die dem Form-Schema prototypisch entsprechen.
\end{enumerate}

Die Hypothesen zu den Substantiven in der Peripherie des Form-Schemas basieren auf den Ergebnissen der Prästudien. Ursprünglich wurde erwartet, dass sowohl hinsichtlich schwacher als auch hinsichtlich starker Formen Variation in den Antworten messbar ist. Die auf Basis der sentence-maze-Prästudie erwartete ausbleibende Variation bei Substantiven in der Peripherie des Form-Schemas  lässt sich durch das Studiendesign der sentence-maze-Studie erklären:  Da die mental gefestigte Form (-\textit{(e)n}) neben der mental noch nicht so stark gefestigten Alternative (-\textit{(e)s}) präsentiert wird, kann die gefestigte Form die weniger gefestigte leichter dominieren.

Die hier vorgestellten Hypothesen sowie das methodische Vorgehen inklusive der geplanten statistischen Auswertung wurden vor der Datenerhebung registriert. Die Registrierungen lassen sich unter \url{https://osf.io/yx3tv} (\textit{lexical decision task}) und \url {https://osf.io/bhxpz} (\textit{sentence maze task}) aufrufen. Auf die statistische Auswertung wird zusammen mit den Ergebnissen in \sectref{ergschema} eingegangen.

\subsection{Materialien}
\label{schemalexmaterial}

Für die Studien werden Testitems genutzt, die jeweils im Genitiv präsentiert werden, sodass die starke und die schwache Form gleich lang sind. Die Zahlen in Klammern geben die relative Frequenz der Genitivformen der Substantive im Jahr 2018 an. Die Frequenz der Genitivformen wurde anhand der Wortverlaufskurven im DWDS (\cite{BerlinBrandenburgischeAkademiederWissenschaften.2019}) bestimmt.

\begin{enumerate}
\item Prototyp des Form-Schemas:\\ 
\textit{des Kollegen} (0,58), \textit{des Neffen} (0,05), \textit{des Schützen} (0,41), \textit{des Franzosen} (1,66), \textit{des Gesellen} (0)
\item Peripherie:\\ 
\textit{des Grafen} (0,36), \textit{des Helden} (0,53), \textit{des Zaren} (0,48), \textit{des Fürsten} (0,29), \textit{des Nachbarn} (0,86)
\item stark:\\ 
\textit{des Diebes} (0,04), \textit{des Freundes} (0,31), \textit{des Vogts} (0), \textit{des Kerls} (0), \textit{des Feindes} (0,46)
\end{enumerate}

In der sentence-maze-Studie werden jeweils nur die ersten drei Testwörter pro Ausprägung genutzt. 


Für die Bestimmung der Frequenz der Substantive wurde in den Wortverlaufskurven des DWDS nach dem Genitiv der Wörter gesucht.  Dabei war jeweils die frequentere Genitivform (-\textit{(e)n}, -\textit{es} oder -\textit{s}) ausschlaggebend.\footnote{Die infrequenteren Formen lagen für \textit{Nachbar} bei einer relativen Frequenz von 0,02 (\textit{des Nachbars}), für \textit{Freund}, \textit{Dieb} und \textit{Feind} bei 0,01 (\textit{des Freunds, Diebs, Feinds}). Bei den anderen Testsubstantiven haben die alternativen Genitivformen eine relative Frequenz von 0.} Bei der Suche wurde ein Determinierer vorangestellt (bspw. \textit{@des @Kollegen}), um Fehltreffer aufgrund von Homonymien zu anderen Kasus- und Pluralformen zu vermeiden. Die Abfrage im DWDS wurde am 28.10.2019 durchgeführt. 

 
Wie in der self-paced-reading-Studie (siehe \sectref{methschema}) wird die Belebtheit konstant gehalten, indem nur Substantive genutzt werden, die auf Menschen referieren. Anders als in der self-paced-reading-Studie schließen die starken Substantive die schwache Flexion jedoch durch ihre phonologischen Eigenschaften nicht aus, denn sie ähneln Substantiven in der Peripherie des Form-Schemas. 


Alle Substantive haben eine Frequenz von bis zu zwei Belegen pro Million Token im Jahr 2018. Auf diese Weise können Tokenfrequenzeffekte minimiert werden. Zwischen den Ausprägungen existieren jedoch leichte Frequenzunterschiede: Drei Substantive, die dem Prototyp des Form-Schemas entsprechen (\textit{Schütze, Kollege, Franzose}), weisen eine relative Frequenz von über 0,4 auf. Die anderen zwei Substantive dieser Ausprägung (\textit{Neffe, Geselle})  bewegen sich im Frequenzbereich von bis zu 0,05. Die Substantive in der Peripherie des Form-Schemas zeigen eine Frequenz von 0,3 bis 2,5. Ähnlich frequent sind zwei der starken Substantive (\textit{Freund, Feind}). Die restlichen drei Substantive dieser Ausprägung (\textit{Vogt, Kerl, Dieb}) sind mit einer Frequenz von bis zu 0,04 deutlich weniger frequent als die Substantive der anderen Ausprägungen. \textit{Geselle} und \textit{Neffe} sind jedoch ähnlich frequent wie die starken Substantive. 

\subsection{Versuchsdesign}
\label{schemadesignsent}

Die Studien nutzen ein within-subject-Design mit zwei Faktoren: Form-Sche\-ma\-ti\-zi\-tät (mit drei Ausprägungen: Prototyp des Form-Schemas, Peripherie des Form-Schemas, stark) und Flexion (mit zwei Ausprägungen: stark, schwach). Pro Ausprägung werden fünf bzw. drei Testwörter genutzt. Die Proband\_innen müssen in der lexical-decision-Studie die Wortformen danach bewerten, ob sie sie kennen. In der sentence-maze-Studie haben die Proband\_innen die Wahl zwischen schwacher und starker Flexion (siehe \sectref{studiefreqdesign} und \ref{studieprotodesign} zum generellen Versuchsdesign der Studien). 



Die Testitems sind jeweils im Genitiv Singular. In der lexical-decision-Studie stehen sie jeweils mit Definitartikel. Als starke Endung wird i.~d.~R. die kurze Endung (-\textit{s}) angezeigt. Wenn die lange Endung (\textit{-es}) frequenter ist als die kurze, wie bei \textit{Dieb}, \textit{Freund} und \textit{Feind}, wird diese genutzt. In der sentence-maze-Studie sind die Testitems in folgende Sätze eingeflochten:

\begin{exe}
\ex Prototyp des Form-Schemas
\begin{xlist}
\ex Ich nehme an, dass die Arbeit \textbf{meines Kollegen/meines *Kolleges} wertgeschätzt wird.
\ex Ich habe gesehen, wie der Pfeil \textbf{des Schützen/des *Schützes} ins Schwarze traf.
\ex\sloppy Mein Freund war nicht da, weil er zum Geburtstag \textbf{seines Neffen\slash seines *Neffes} gegangen ist.
\end{xlist} 
\ex Peripherie des Form-Schemas
\begin{xlist}
\ex Mein Neffe mag Superman nicht, weil ihm der Anzug \textbf{des Helden/des Helds} nicht gefällt.
\ex Sie konnte nicht fassen, dass das Anwesen \textbf{des Grafen/des Grafs} so groß ist wie ein Schloss.
\ex Ihm war nicht bewusst, dass die Macht \textbf{des Zaren/des Zars} einmal derart groß gewesen war.
\end{xlist}
\ex stark
\begin{xlist}
\ex Er fährt nochmal zurück, weil er die Jacke \textbf{seines Freundes/seines *Freunden} aus Versehen eingesteckt hat und dieser sie auf der Arbeit benötigt.
\ex Der Krimi-Fan ahnt schon früh, dass der Plan \textbf{des Diebes/des *Dieben} zu simpel ist und nicht klappt.
\ex Er macht einen Ausflug, um die Burg \textbf{des Vogts/des *Vogten} zu betrachten.
\end{xlist}
\end{exe}

Um die Testsätze ähnlich zu gestalten wie die Sätze zur Auxiliarselektion von \textit{haben} und \textit{sein}, werden Satzgefüge genutzt. Die Testitems sind jeweils als Genitivattribut in die Testsätze eingeflochten (\textit{..., weil er zum Geburtstag \textbf{seines Neffen} gegangen ist}). Sie stehen allesamt im Singular und werden mit einem bestimmten Artikel bzw. mit einem Possessivpronomen (\textit{ihres Neffen/*s}) determiniert. Die Proband\_innen haben jeweils die Wahl zwischen starker und schwacher Endung (\textit{des Helden/des Helds}).  Der vorab durchgeführte Plausibilitätstest mit 58 Proband\_innen war für alle Testsätze zu schwachen Maskulina unauffällig.\footnote{Alle Sätze wurden mit mindestens 3 auf einer Skala von 1 (nicht plausibel) bis 5 (plausibel) bewertet. Eine kleine Änderung wurde nach dem Test vorgenommen: \textit{Gewertschätzt} wurde zu \textit{wertgeschätzt}. Weitere Erläuterungen zum Plausiblitätstest sowie die Bewertungen der Proband\_innen sind in Anhang \ref{plausent}.} 

In der lexical-decision-Studie sind pro Block sieben bis acht Testsubstantive eingeflochten, dementsprechend finden sich pro Ausprägung zwei bis drei Testsubstantive in einem Block. Starke und schwache Formen eines Testsubstantivs kommen nicht in einem Block vor. In der sentence-maze-Studie sind pro Experimentblock zwei Testsätze enthalten. Diese entstammten verschiedenen Form-Schematizitätsausprägungen.  Für nähere Erläuterungen zum Aufbau der Experimentblöcke in den Studien siehe \sectref{studiefreqdesign} und \ref{studieprotodesign}. Die Tabellen~\ref{ablauflexta} und \ref{ablaufsentta} in Anhang \ref{blocklex} und \ref{blocksent} schlüsseln den Aufbau der Blöcke in den Studien auf, Tabelle \ref{ablaufsentta} bietet zudem einen Überblick über die Einteilung der Testsätze in die zu lesenden Abschnitte.

\subsection{Versuchsablauf, Metadaten, Probandenakquise, Proband\_innen}
\label{probschema}
\begin{sloppypar}
Der Versuchsablauf, die erhobenen Metadaten und die Probandenakquise entsprechen den Erläuterungen zur lexical-decision-Studie zu Frequenzeffekten (\sectref{metfreq}) bzw. zur sentence-maze-Studie zu Prototypizitätseffekten (\sectref{metproto}). 
Die Stichprobengrößen von 50 (\textit{lexical decision}) und 125 (\textit{sentence maze}) wurde wiederum anhand einer Datensimulation berechnet, die auf Prästudien\footnote{Die Stichprobengröße der Prästudien betrug 20 Proband\_innen in der lexical-decision-Studie und 27 Proband\_innen in der sentence-maze-studie nach Ausschluss von Pro\-\mbox{band\_in}\-nen, die die Kriterien nicht erfüllten. Die Prästudien haben denselben Versuchsablauf wie die Hauptstudien. Nur der aktuelle Wohnort wurde in den Prästudien nicht abgefragt. Die Proband\_innen wurden anhand derselben Kriterien wie in den Hauptstudien aus der Stichprobe ausgeschlossen.} basiert.
\end{sloppypar}
 
In der Prästudie zur lexical-decision-Studie zeigte sich eine Interaktion zwischen Deklina\-tionsklasse und Flexionsform: Schwache Formen von schwachen Substantiven werden schneller beurteilt als starke Formen. Das Gegenteil ist bei starken Substantiven der Fall. Unterschiede zwischen prototypischen und peripheren Vertretern des Form-Schemas schwacher Maskulina waren nicht zu erkennen. Die angestrebte Stichprobengröße sollte also groß genug sein, um die Interaktion zwischen Deklinationsklasse und Flexionsform messen zu können. Unter der Annahme, dass die Schätzung aus der Prästudie robust ist, sollten  50 Proband\_innen bei einem $\alpha$-Level von 0,01 für eine Teststärke von mindestens 0,8 ausreichend sein (für den Schätzwert der Ausprägung \textsc{Form-Schema} mit schwachen Formen im Vergleich zu den Schätzwerten der Ausprägung \textsc{Form-Schema} und \textsc{Peripherie} mit starken Formen sowie der Ausprägung \textsc{stark} mit schwachen und starken Formen). Insgesamt wurden 100 Proband\_innen akquiriert, da in der Prästudie mit 12 von 32 Proband\_innen 38~\% der Teilnehmer\_innen ausgeschlossen werden mussten. 


In der Prästudie zur sentence-maze-Studie war erkennbar, dass die Substantive aus der Peripherie des Form-Schemas niedrigere Reaktionszeiten evozierten als Substantive der anderen Ausprägungen. Um hierfür eine Teststärke von 0,8 (für den Schätzwert der Ausprägung \textsc{Form-Schema} im Vergleich zum Schätzwert der Ausprägung \textsc{Peripherie}) bei einem $\alpha$-Level von 0,01 zu erreichen, sollte die Stichprobengröße 125 Proband\_innen umfassen. Insgesamt werden 170 Proband\_innen akquiriert, da in der Prästudie mit fünf von 32 Proband\_innen 16 \% ausgeschlossen wurden.  

Insgesamt umfasst die Stichprobe der lexical-decision-Studie nach Aus\-schluss von Proband\_innen 57 Teilnehmer\_innen. Die Proband\_innen der lexical-de\-ci\-sion-Studie unterscheiden sich dabei nur minimal von denen der Frequenzstudie. Der Unterschied zur Frequenzstudie ergibt sich dadurch, dass Proband\_innen jeweils neben den anderen Ausschlusskriterien auch anhand ihrer Reaktionszeit (mehr als drei Standardabweichungen über bzw. unter dem Mittelwert der jeweiligen Test\-items) ausgeschlossen wurden. In der Stichprobe für die lexical-decision-Studie zu Form-Schematizitätseffekten sind daher mit 57 Proband\_innen vier Proband\_innen mehr als in der Frequenzstudie. Aufgrund der hohen Deckungsgleichheit mit der Frequenzstudie werden die Proband\_innen an dieser Stelle nicht gesondert vorgestellt, siehe \sectref{studiefreqprob} für einen Überblick über die Proband\_innen. 

 

Die Stichprobe der sentence-maze-Studie umfasst nach Bereinigung der Daten 132 Proband\_innen. Hinsichtlich der sentence-maze-Studie zu Form-Sche\-ma\-ti\-zi\-tät ergibt sich eine Abweichung von der Prototypizitätsstudie, da bei der Prototypizitätsstudie Proband\_innen aus Bayern, Baden-Württemberg, Rheinland-Pfalz und dem Saarland ausgeschlossen wurden (siehe \sectref{probprototy}). Dies war für die Form-Schematizitätstudie nicht nötig.\footnote{Zwar diskutiert \cite{RonnebergerSibold.2020} einen Einfluss von mittel- und nordbayerischen Dialekten auf die Wahrnehmung des Genitivs bei schwachen Maskulina, sie untersucht jedoch Studierende mit stark dialektalem Sprachverhalten. Daher wurde für diese Studie davon abgesehen, pauschal Proband\_innen aus Bayern auszuschließen.} Aufgrund dessen sind in der Stichprobe mit 132 Proband\_innen~25~Proband\_innen mehr als in der Prototypizitätsstudie. Daher wird die Stichprobe hier noch einmal gesondert vorgestellt. Die Angaben zu Metadaten waren freiwillig, daher fehlen für einige Metadaten Angaben. Eine ausführliche Aufschlüsselung der Metadaten ist im digitalen Anhang zu finden. 



Die Proband\_innen der sentence-maze-Studie zu Form-Schematizität sind im Schnitt 25~Jahre alt mit einer Standardabweichung von 7,6 Jahren. 110 identifizieren sich als weiblich, 21~als männlich. 30 Proband\_innen kommen aus Niedersachsen, 26 aus Nordrhein-Westfalen, 16~aus Schleswig Holstein. Die restlichen Bundesländer sind mit bis zu acht Proband\_innen vertreten, einige Proband\_innen gaben auch an, in mehreren Bundesländern aufgewachsen zu sein. 27 Pro\-\mbox{band\_in}\-nen leben derzeit in Nordrhein-Westfalen bzw. Hamburg, 20 in Niedersachsen, 15 in Schleswig-Holstein und 14 in Mecklen\-burg-Vorpommern. Die anderen Bundesländer entfallen auf maximal acht Proband\_innen, drei Pro\-\mbox{band\_in}\-nen leben im Ausland. 117 geben an, täglich zu lesen, zwölf lesen wöchentlich und zwei monatlich. 123 Proband\_innen sprechen in familiären Kontexten Standard, acht Dialekt und eine Person eine Mischform. 84 Proband\_innen haben Abitur,~25~einen Bachelor- und zwölf einen Masterabschluss. 117 der Proband\_innen nutzen vornehmlich die rechte Hand, 14 die linke, eine Person gab an, beide Hände gleichermaßen zu nutzen. Im nächsten Kapitel werden die Ergebnisse der Studien vorgestellt und diskutiert.
