\addchap{\lsAcknowledgementTitle} 

Mein Dank gilt allen, die mir während meiner Promotion zur Seite gestanden haben, egal ob mit Rat, Literaturhinweisen oder einer Umarmung und aufmunternden Worten. Ich konnte mich auf eine sehr hohe Frequenz an Rückhalt und Zuspruch stützen. Dabei war sowohl die Typen- als auch die Tokenfrequenz extrem hoch, was aus frequenztheoretischer Sicht ungewöhnlich ist.  Gemessen an dieser hohen Frequenz ist das Lemma \textit{(be-)danken} in meiner Dissertation relativ selten. Eine Danksagung mit einer möglichst hohen Dichte an Dank soll diesen Missstand ausgleichen.

Besonderer Dank gilt Renata Szczepaniak, die meine Dissertation betreut hat. In ihren Seminaren habe ich früh meine Faszination für Sprachwissenschaft entdeckt. Während der Dissertation konnte ich stets auf ihren fachlichen Rat und ihre Unterstützung zählen. Auch einer thematischen Verschiebung hin zu einer Dissertation mit dezidiert gebrauchsbasiertem Ansatz und ohne Fehler(-diskussion) stimmte sie ohne Bedenken zu.

Besonders möchte ich auch Alexander Werth danken, der meine Arbeit zweitbetreut hat. Alexander fragte mich im Februar 2018, wer eigentlich mein Zweitbetreuer sei. Ich antwortete: „Wenn du magst, du.“ Damit konnte ich mich bereits früh in meiner Promotion auf Rat von zwei Seiten stützen. Ich danke Alexander insbesondere für wertvolle methodische Hinweise und viele Diskussionen zum Aufbau der Reaktionszeitstudien.

Für fachlichen Input und offene Ohren danke ich meinen Hamburger und Bamberger Kolleg\_innen. Weiterhin bedanke ich mich bei der Studienstiftung des deutschen Volkes, von deren finanzieller und ideeller Förderung ich während meines Studiums und meiner Promotion profitieren durfte.

Für unendliche emotionale Unterstützung danke ich meinem Freundeskreis und ganz besonders dem Promowendland.  Meinen Eltern Jürgen und Margret danke ich für ihr Vertrauen in mich und den unbedingten Rückhalt, den sie mir geben. Genauso danke ich meinen Geschwistern Verena und Maximilian: Es ist schön zu wissen, dass ich immer auf euch zählen kann. Zum Schluss danke ich David~–~für alles.

Eure Unterstützung hat die Wahrscheinlichkeit für das Gelingen der Dissertation massiv zum Positiven beeinflusst. 
