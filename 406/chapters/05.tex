\chapter{Psycholinguistische Betrachtung der Einflussfaktoren: Ergebnisse}\label{Ergebnisse}

In diesem Kapitel werden die Ergebnisse der einzelnen Studien vorgestellt. Für die Datenaufbereitung wurden Python (\cite{vanRossum.2009})\footnote{Folgende Pakete wurden genutzt: glob (\cite{vanRossum.2009}); os (\cite{vanRossum.2009}); pandas (\cite{McKinney.2010}).}, Version 3.8,  sowie R (\cite{RCoreTeam.2019}), Version 3.6.0, und RStudio (\cite{RStudioTeam.2016}), Version 1.1.453, genutzt. R~und RStudio wurden auch für die Datenauswertung inklusive der Visualisierung der Daten verwendet.\footnote{Folgende Pakete wurden für die Datenaufbereitung, -auswertung und Visualisierung der Daten in R genutzt: afex (\cite{Singmann.2019}); broom.mixed (\cite{Bolker.2019}); data.table (\cite{Dowle.2019}); ggbee\-swarm (\cite{Clarke.2017}); ggeffects (\cite{Ludecke.2018}); Hmisc (\cite{Harrell.2019}); lme4 (\cite{Bates.2015}); MuMIn (\cite{Barton.2019}); optimx (\cite{Nash.2011}); party (\cite{Hothorn.2006, Strobl.2008}); patchwork (\cite{Pedersen.2020}); readr (\cite{Wickham.2018}); readxl (\cite{Wickham.2019}); tidyverse (\cite{Wickham.2017}); waffle (\cite{Rudis.2019}); xtable (\cite{Dahl.2019}).}  



Die in diesem Kapitel vorgestellten Ergebnisse beziehen sich auf die Daten aus den Hauptstudien. Eine analoge Auswertung der Daten aus den Prästudien\footnote{Für die self-paced-reading-Studie wurde keine Prästudie durchgeführt.} befindet sich im digitalen Anhang. Erläuterungen zur Zusammensetzung der Stichprobe und zur Bereinigung der Daten sind in \chapref{psych} zu finden, in dem auch das methodische Vorgehen vorgestellt wurde. Bevor die Ergebnisse diskutiert werden, werden die in der Analyse genutzten statistischen Modelle, (generalisierte) gemischte lineare Modelle und \textit{random forests}, vorgestellt. Im Anschluss daran werden zunächst die Ergebnisse der Studie zu Frequenzeffekten analysiert, darauf folgen die Ergebnisse der Studie zu Prototypizitätseffekten und schließlich die der Studien zu Form-Schematizitätseffekten. Das Kapitel schließt mit einer Diskussion, in der die Ergebnisse der einzelnen Studien kontrastiert werden und der Einfluss von Frequenz, Prototypizität und Form-Schematizität auf Variation diskutiert und evaluiert wird. 



Das Hauptinteresse der Studien liegt in den Reaktionszeiten, das Antwortverhalten wird als zusätzliche Evidenz für Variation analysiert. Dabei wird für alle Studien zuerst das Antwortverhalten der Proband\_innen betrachtet und erst im Anschluss daran werden die Reaktionszeiten diskutiert, da das Antwortverhalten potentiell Einfluss auf die Reaktionszeiten nehmen kann. Daher ist es sinnvoll, die Muster im Antwortverhalten für die Analyse der Reaktionszeiten zu kennen, um sie ggf. berücksichtigen zu können.

 
Das Antwortverhalten und die Reaktionszeiten werden zunächst deskriptiv ausgewertet und visualisiert. Anschließend folgt die konfirmatorische statistische Auswertung mithilfe (generalisierter) gemischter linearer Modelle. Zusätzlich werden \textit{random forests} hinzugezogen, wenn sich Konvergierungsprobleme\footnote{Bei Konvergierungsproblemen kann ein Modell keine stabilen Schätzwerte für die unabhängigen Variablen berechnen. Dieses Problem besteht häufig, wenn komplexe Modelle auf einen zu kleinen Datensatz angewendet werden. Konvergierungsprobleme sind nicht zwingend mit einem unzuverlässigen Schätzwert gleichzusetzen, jedoch können sie darauf hindeuten (\cite[265--266]{Winter.2020}).} bei den gemischten linearen Modellen ergeben. Die konfirmatorische Auswertung überprüft die Hypothesen, die mit Ausnahme der self-paced-reading-Studie samt statistischer Auswertung bereits vorab registriert wurden. Auf die konfirmatorische Analyse folgt eine explorative statistische Auswertung, wenn sich interessante Muster in den Daten zeigen. Die explorative Auswertung wurde teilweise ebenfalls vorab registriert, wenn die Muster bereits in der Prästudie zu erkennen waren.  Die explorative Auswertung hat zum Ziel, zu überprüfen, ob sich Muster, die sich in den Daten erkennen lassen, auch in den statistischen Modellen zeigen. Aufgrund ihres explorativen Charakters sind $p$-Werte für diese Analysen nicht aussagekräftig, da die Analysen sich nicht auf vorab registrierte Hypothesen beziehen. Daher werden die $p$-Werte für die explorativen Analysen nicht berichtet.  Im folgenden Abschnitt wird auf die Modelle, die den statistischen Analysen zugrundeliegen, näher eingegangen.

\section{(Generalisierte) gemischte lineare Modelle und \textit{random forests}}

In diesem Abschnitt werden zunächst (generalisierte) gemischte lineare Modelle vorgestellt und im Anschluss daran \textit{random forests}.

\subsection{(Generalisierte) gemischte lineare Modelle} 

Für die statistische Analyse der Reaktionszeiten und des Antwortverhaltens werden vornehmlich (generalisierte) gemischte lineare Modelle genutzt. Um die Logik (generalisierter) gemischter linearer Modelle zu verstehen, ist es sinnvoll, zunächst lineare sowie  gemischte lineare Modelle zu betrachten. Diese Modelle werden für die Analyse der Reaktionszeiten genutzt. Im Anschluss daran werden generalisierte gemischte lineare Modelle erläutert, da diese auf gemischten linearen Modellen basieren. Die generalisierten Modelle werden für die Analyse des Antwortverhaltens gerechnet.

Lineare Modelle basieren auf Regression: Hierbei wird eine kontinuierliche Variable (zum Beispiel Reaktionszeit) als Funktion einer anderen Variable modelliert (\cite[69--72]{Winter.2020}):
\[ \text{Reaktionszeit} = b{_0} + b{_1} * \text{Frequenz} + e \]

$b{_0}$ ist der y-Achsenabschnitt (\textit{intercept}), also der Punkt, an dem die Regressionslinie die y-Achse schneidet. $b{_1}$ ist die Steigung (\textit{slope}) der Regressionslinie, die in diesem Beispiel durch die Frequenzausprägung (z.~B. \textsc{frequent}/\textsc{infrequent}) beeinflusst wird. $e$ gibt den Fehler an, also in der Beispielgleichung alles, was nicht von der Variable Frequenz abgedeckt wird (\cite[73--74]{Winter.2020}). Der Fehler entspricht in diesem Fall den Residuen: Residuen geben an, wie weit der tatsächliche gemessene Wert von dem Wert abweicht, den das Modell vorhersagt. Sie werden berechnet, indem die tatsächlich gemessenen Werte von den vom Modell vorhergesagten Werten abgezogen werden (\cite[73--74]{Winter.2020}). Die Gleichung lässt sich beliebig erweitern:
\[ \text{Reaktionszeit} = b{_0} + b{_1} * \text{Frequenz} + b{_2} * \text{Flexion} + e \]

In dieser Gleichung nimmt zusätzlich zu Frequenz Flexion (\textsc{stark}/\textsc{schwach}) Einfluss auf die Reaktionszeiten. In der Gleichung wird der Einfluss der Faktoren auf die Reaktionszeiten separat berechnet. Genauso kann eine Interaktion zwischen den Einflussfaktoren überprüft werden:
\[ \text{Reaktionszeit} = b{_0} + b{_1} * \text{Frequenz} + b{_2} * \text{Flexion} + b{_3} * (\text{Frequenz} * \text{Flexion}) + e \]

Bei einer Interaktion liegt die Erwartung zugrunde, dass die Einflussfaktoren einzeln nur bedingt Einfluss auf Reaktionszeiten nehmen, aber beide Faktoren gemeinsam die Reaktions\-zeit beeinflussen (\cite[133--134]{Winter.2020}). Ein gutes Beispiel für Interaktion ist der Einfluss von Wasser und Sonne auf Pflanzenwachstum: Viel Wasser und viel Sonne allein fördern das Pflanzenwachstum nicht, erst beides in Kombination.

Während lineare Modelle nur feste Effekte (z.~B. Frequenz und Flexion) modellieren, werden in einem gemischten linearen Modell neben den festen Effekten auch zufällige Effekte berücksichtigt (\cite[236]{Winter.2020}). Teilnehmer\_innen und Testitems sind zufällige Effekte. Sie bringen Varianz in die Daten, die sich nicht durch die festen Effekte erklären lässt (\cite[236]{Winter.2020}). Zufällige Effekte haben einen unvorhersagbaren, idiosynkratischen Einfluss auf die Daten (\cite[39]{Winter.2013}), während feste Effekte über viele Experimente hinweg konstant sind (\cite[236]{Winter.2020}). Man sollte also in vielen Experimenten bspw.  einen Einfluss von Frequenz messen können, auch wenn die Teilnehmer\_innen variieren.

In den Beispielen oben wird angenommen, dass die gemessenen Daten unabhängig voneinander sind. Wenn aber Proband\_innen mehrere Versuchitems pro Kondition (bspw. pro Frequenzausprägung) beurteilen, ist dies nicht der Fall: Es kann sein, dass einige Proband\_innen generell schneller sind oder auf bestimmte Versuchitems langsamer oder schneller reagieren als andere Proband\_innen. Daher werden Proband\_innen als zufälliger Effekt in linearen Modellen berücksichtigt, wenn mehrere Messpunkte pro Proband\_in vorliegen (\cite[236]{Winter.2020}). Dabei können \textit{random intercepts} (zufällige y-Achsenabschnitte) und zusätzlich dazu \textit{random slopes} (zufällige Steigungen) genutzt werden.\footnote{Es ist nicht immer leicht zu entscheiden, ob sowohl \textit{random intercepts} als auch \textit{random slopes} für ein Modell genutzt werden sollten (siehe \cite[241--244]{Winter.2020} für eine Diskussion). In den vorliegenden Studien ist es sinnvoll, für Proband\_innen sowohl \textit{random intercepts} als auch \textit{random slopes} zu berücksichtigen. Für Versuchitems sind jedoch in fast allen Studien nur \textit{random intercepts} möglich, da die Versuchitems über Konditionen hinweg variieren (\cite[243]{Winter.2020}). Nur bei der Prototypizitätsstudie ist das nicht der Fall, weswegen hier auch \textit{random slopes} für Versuchitems vorgesehen waren. In der Anwendung mussten \textit{random slopes} jedoch aufgrund von \textit{singular fits}, die anzeigen, dass die Effektstruktur für die vorliegenden Daten zu komplex ist, sowie Konvergierungsproblemen weggelassen werden.} \textit{Random intercepts} ermöglichen es, unterschiedliche Reaktionszeiten für schnelle und langsame Proband\_innen zu berechnen, mit \textit{random slopes} können unterschiedliche Entwicklungen in den Reak\-tionszeiten modelliert werden: Zum Beispiel könnten Proband\_innen mit niedrigen Reak\-tions\-zeiten mit der Zeit langsamer werden, aber Proband\_innen mit hohen Reaktionszeiten konstant bleiben (\cite[237--240; 267--268]{Winter.2020}). Wie Proband\_innen werden auch Versuchitems als zufällige Effekte berücksichtigt, wenn mehrere Items pro Kondition (bspw. pro Frequenzausprägung) genutzt werden. 

Linearen Modellen liegen Vorannahmen zugrunde. Der Annahme der Unabhängigkeit von Datenpunkten kann durch ein gemischtes lineares Modell begegnet werden. Weiterhin wird davon ausgegangen, dass eine Normalverteilung und Homoskedastizität (konstante Varianz) der Residuen vorliegt (\cite[74--75]{Winter.2020}). Ist dies nicht der Fall, sind die Vorhersagen des Modells unzuverlässig. Für die gemischten linearen Modelle wurden die Vorannahmen der Normalverteilung und Homoskedastizität der Residuen jeweils wie von \textcite[109]{Winter.2020} empfohlen visuell überprüft. Die enstprechenden Grafiken sind im digitalen Anhang. Die Vorannahmen werden in den Modellen, die Reaktionszeiten in Millisekunden berücksichtigen, verletzt. Daher werden die Reaktionszeiten logarithmiert. Mit logarithmierten Reaktionszeiten werden die Annahmen erfüllt.  

Für die Analyse des Antwortverhaltens werden generalisierte gemischte lineare Modelle genutzt, um logistische Regressionen rechnen zu können. Das Antwortverhalten ist im Gegensatz zu Reaktionszeiten nicht kontinuierlich, sondern binär: Die Proband\_innen können die Frage zur Bekanntheit der Formen in der lexical-decision-Studie nur bejahen oder negieren und in der sentence-maze-Studie nur zwischen zwei Formen wählen. Auch bei dem Produktions\-experiment im Rahmen der self-paced-reading-Studie lassen sich die Antworten auf schwache und andere Formen (stark auf -\textit{(e)s}, ohne Endung, Doppelformen)  herunterbrechen.

 

In einem gemischten generalisierten Modell wird nicht von einer Normalverteilung der Daten ausgegangen, sondern von einer anderen, im Fall der logistischen Regression von einer binomialen (\cite[200]{Winter.2020}). Anders als bei kontinuierlichen Variablen versucht das Modell nicht, Werte der kontinuierlichen Variable vorherzusagen, sondern die Wahrscheinlichkeit für eine Antwort zu bestimmen, etwa für eine affirmierende Antwort unter einem bestimmten Einflussfaktor (z.~B. Frequenz) (\cite[200--201]{Winter.2020}).


Für binomiale Verteilungen gilt:
\[ y \sim \text{binomial} (N=1,p)\]

$N$ beschreibt die Anzahl an Vorkommnissen (\textit{trials}) und kann auf 1 beschränkt werden: Hierdurch wird geschätzt, wie wahrscheinlich es für eine einzelne Antwort ist, affirmierend oder negierend zu sein (\cite[200]{Winter.2020}). Binominale Verteilungen mit $N=1$ sind Bernoulli-Verteilungen. $p$ beschreibt die Wahrscheinlichkeit für eine affirmierende oder negierende Antwort bzw. für die Wahl einer schwachen oder starken Flexionsendung. Hier wird deutlich, warum die Daten binär sein müssen: Die Wahrscheinlichkeit ($p$) für eine Antwort ist auf diese Weise $1-p$ für die andere Antwort. Wie bei den linearen Modellen kann auch bei den generalisierten linearen Modellen der Einfluss mehrerer Variablen auf die abhängige Variable berechnet werden. In der Gleichung ist dies am Beispiel der Wahrscheinlichkeit für eine affirmierende Antwort dargestellt: 
\[
\begin{split}
p_{\text{(Affirmation)}} & =  b{_0} + b{_1} * \text{Einflussfaktor}{_1} + b{_2} * \text{Einflussfaktor}{_2} \\
                         & \quad (+ b{_3}* \text{Einflussfaktor}{_3})\\
                         & \quad (+ ...)
\end{split}
\]

Problematisch ist, dass die Gleichung alle möglichen Werte annehmen kann. Wahrscheinlichkeiten bewegen sich aber zwischen 0 und 1 (\cite[201]{Winter.2020}). Daher muss das Ergebnis der Gleichung transformiert werden: Durch die logistische Transformation werden die Daten auf Werte zwischen 0 und 1 beschränkt (\cite[201--202]{Winter.2020}). Bei einem Wert nahe 0 ist die Wahrscheinlichkeit für affirmierende Antworten bspw. gering, bei einem Wert nahe 1 hoch. Das Gegenteil ist dann für negierende Antworten der Fall.  Mithilfe der logistischen Transformation kann die Wahrscheinlichkeit für affirmierende und negierende Antworten daher in Abhängigkeit von bestimmten Einflussfaktoren (z.~B. Frequenz) modelliert werden. 
 
Für die Berechnung der (generalisierten) gemischten linearen Modelle wird das Paket lme4 (\cite{Bates.2015}), Version 1.1-23, genutzt. Zusätzlich wird auf die Pakete MuMIn (\cite{Barton.2019}), Version 1.43.17, und afex (\cite{Singmann.2019}), Version 0.27-2, zurückgegriffen. 

\subsection{\textit{Random forests}}

Zusätzlich zu den generalisierten gemischten linearen Modellen werden \textit{random forests} genutzt, um das Antwortverhalten zu analysieren. Sie werden wie von \textcite[275]{Levshina.2015} empfohlen dann verwendet, wenn die zufälligen Effekte in den generalisierten gemischten linearen Modellen aufgrund von Konvergierungsproblemen nur teilweise oder nicht berücksichtigt werden können und die Aussagekraft der Modelle daher eingeschränkt ist. \textit{Random forests} bestehen aus vielen \textit{conditional inference trees}. \textit{Conditional inference trees} werden mithilfe eines Permutationsverfahrens berechnet. Durch die Permutation werden zufällige Muster erstellt, deren Basis die real existierenden Daten sind. Die permutierten Daten werden mit den tatsächlichen verglichen, sodass geprüft werden kann, ob das zufällige Muster die Daten besser erklären kann als das tatsächliche Muster (\cite[159]{Tagliamonte.2012}, \cite[291--292]{Levshina.2015}). Zudem wird getestet, welcher Einflussfaktor die Verteilung der Daten am besten vorhersagen kann. Die Daten werden anhand dieses Faktors geteilt. Dann wird innerhalb der geteilten Daten getestet, welcher Einflussfaktor die Verteilung am besten vorhersagt. Dieser Vorgang wird so lange wiederholt, bis die tatsächlich gemessenen Daten keine signifikant besseren Voraussagen treffen als die permutierten Daten (\cite[159]{Tagliamonte.2012}, \cite[291--292]{Levshina.2015}). \textit{Random forests} entstehen, indem viele \textit{conditional inference trees} berechnet werden. Diese basieren jeweils auf zufälligen Stichproben, die aus den Daten mit Zurücklegen gezogen wurden. Auf diese Weise können die Vorhersagen jedes \textit{conditional inference trees} mit den Daten, die nicht Teil der Stichprobe waren, verglichen und evaluiert werden (\cite[179--180]{Tagliamonte.2012}). Für die Berechnung der \textit{random forests} wird das Paket party (\cite{Hothorn.2006, Strobl.2008}), Version 1.3-5, genutzt. Für die Berechnung der Passgenauigkeit der \textit{random forests} wird zusätzlich das Paket Hmisc (\cite{Harrell.2019}), Version 4.4-0, verwendet. Im folgenden Abschnitt werden die Ergebnisse der lexical-decision-Studie zu Frequenzeffekten vorgestellt.

\section{Frequenz}
\label{freqerg}

In diesem Abschnitt werden das Antwortverhalten und die Reaktionszeiten in der lexical-decision-Studie diskutiert. Die Studie hat zum Ziel, anhand von starken Verben mit unterschiedlicher Tokenfrequenz den Einfluss von Frequenz auf Variation zu untersuchen (siehe \sectref{konjugation} für ausführliche Erläuterungen zur Variation in der Konjugation). Alle in diesem Abschnitt diskutierten Daten beziehen sich auf die in \sectref{studiefreqprob} vorgestellten Proband\_innen. Die Stichprobengröße beträgt 53 Personen. In der Studie wurden jeweils starke Testverben genutzt, die entweder frequent (relative Tokenfrequenz ab 13 Token pro Million Wortformen) oder infrequent sind (relative Tokenfrequenz bis zu zwei Token pro Million Wortformen). Die infrequenten Verben teilen sich zusätzlich in Verben ohne Schwankung in Korpora (Ratio von starken zu schwachen Formen liegt über 100 starken Formen zu einer schwachen Form) und Verben mit Schwankung in Korpora (Ratio von starken zu schwachen Formen liegt bei unter oder gleich 100 starken Formen zu einer schwachen Form). Pro Frequenzsausprägung (\textsc{frequent}, \textsc{infrequent ohne Schwankung} und \textsc{infrequent mit Schwankung}) wurden acht Testverben\footnote{\textsc{frequent}: \textit{sprechen}, \textit{fahren}, \textit{tragen}, \textit{ziehen}, \textit{halten}, \textit{schreiben}, \textit{sinken}, \textit{fliegen}; \textsc{infrequent ohne Schwankung}: \textit{flechten}, \textit{spinnen}, \textit{schmelzen}, \textit{anschwellen}, \textit{kneifen}, \textit{fechten}, \textit{melken}, \textit{dreschen}; \textsc{infrequent mit Schwankung}: \textit{salzen}, \textit{hauen}, \textit{quellen}, \textit{einsaugen}, \textit{weben}, \textit{glimmen}, \textit{gären}, \textit{sinnen} (siehe \sectref{freqmat} für Erläuterungen zu den Testverben).} genutzt.

\begin{sloppypar}
Innerhalb einer \textit{lexical decision task} haben Proband\_innen die starken und schwachen Par\-ti\-zip-II-Formen jedes Testverbs nach ihrer Bekanntheit bewertet (siehe \sectref{metfreq} für ausführliche Erläuterungen). Es wird erwartet, dass starke Formen der Testverben unabhängig von ihrer Frequenz bekannt sind. Hinsichtlich der Bekanntheit schwacher Formen wird ein Frequenzeffekt angenommen: Für die Ausprägung \textsc{frequent} wird erwartet, dass schwache Formen unbekannt sind (z.~B. *\textit{gefahrt}). Für die infrequenten Verben wird antizipiert, dass schwache Formen bekannter sind, dabei wird davon ausgegangen, dass die schwachen Verbformen der Ausprägung \textsc{infrequent ohne Schwankung} (z.~B. \textit{gefechtet}) weniger bekannt sind als schwache Verbformen der Ausprägung \textsc{infrequent mit Schwankung} (z.~B. \textit{geglimmt}). Hinsichtlich der Reaktionszeiten wird unabhängig von der beurteilten Flexionsform (stark/schwach) für Verben der Ausprägung \textsc{frequent} ein schnelleres Antwortverhalten erwartet als für die anderen Ausprägungen. Zwischen den Ausprägungen \textsc{infrequent ohne Schwankung} und \textsc{infrequent mit Schwankung} wird kein Unterschied in den Reak\-tions\-zeiten antizipiert (siehe \sectref{freqhyp} für genauere Erläuterungen).
\end{sloppypar}

\subsection{Antwortverhalten}\label{freqant}\largerpage

Für jede Frequenzausprägung (\textsc{frequent}, \textsc{infrequent ohne Schwankung}, \textsc{infrequent mit Schwankung}) wurden starke und schwache Partizip-II-Formen von jeweils acht Testverben beurteilt (bspw. \textit{gezieht} und \textit{gezogen}). Insgesamt wurden dabei pro Frequenzausprägung 424 starke und schwache Formen auf Ihre Bekanntheit hin bewertet.\footnote{Die 424 Antworten pro Frequenzausprägung und Flexion ergeben sich aus den acht Testverben und den 53 Proband\_innen.} Abbildung~\ref{freqantwortalle} zeigt einen Waffleplot mit den Antworten. In dem Waffleplot wird die Verteilung der Antworten zu der Frage \textit{Kennen Sie diese Wortform?} für die einzelnen Frequenzausprägungen und Flexionsformen dargestellt. Jedes Rechteck entspricht einer Antwort. Dunkelgrau eingefärbt sind Rechtecke, die eine affirmierende Antwort abbilden, hellgrau Rechtecke, die eine verneinende Antwort darstellen. 

\begin{figure}
\includegraphics[width=\textwidth]{figures/Kap6/freqantwort.png} 
\caption{Antwortverhalten in der Frequenzstudie}
\label{freqantwortalle}
\end{figure}

Es zeigt sich ein klar verteiltes Antwortverhalten: Starke Formen werden unabhängig von der Frequenzausprägung als bekannt bewertet. Lediglich bei der Ausprägung \textsc{infrequent mit Schwankung} werden starke Formen vereinzelt (7,1 \%) als unbekannt gekennzeichnet. Im Gegensatz zu den starken Formen hängt bei den schwachen Formen die Bekanntheit deutlich von der Frequenzausprägung ab: Während die schwachen Formen bei Testverben der Ausprägung \textsc{frequent} (z.~B. *\textit{gefahrt}) als unbekannt bewertet werden,\footnote{Die klare Antwortverteilung in der Ausprägung \textsc{frequent} zwischen starken und schwachen Formen weist darauf hin, dass die Proband\_innen die Stimuli einheitlich beantworten. Daher wären auch bei einer mehrmaligen Präsentation derselben Stimuli keine abweichenden Antworten zu erwarten (siehe \sectref{lexdectask} für weitere Erläuterungen zu diesem Problem bei \textit{lexical decision tasks}), auch das Antwortverhalten in der lexical-decision-Studie zu Form-Schematizität weist auf ein konsistentes Antwortverhalten für die Stimuli der Ausprägung \textsc{Form-Schema} und \textsc{stark} hin (siehe \sectref{ergschemadecant}),} zeigen sich bei den Verben der Ausprägung \textsc{infrequent ohne Schwankung} (z.~B. \textit{gefechtet}) Tendenzen zur Bekanntheit (24,7 \%). Bei den Testverben der Ausprägung \textsc{infrequent mit Schwankung} (z.~B. \textit{geglimmt}) werden die schwachen Formen zu 51,4 \% als bekannt bewertet.

Ein näherer Blick auf die Ausprägungen \textsc{infrequent mit Schwankung} und \textsc{infrequent ohne Schwankung}\footnote{Ein Überblick über das Antwortverhalten bei den Testverben der Ausprägung \textsc{frequent} befindet sich in Abbildung \ref{antwfrequentev}.} zeigt, dass die Testverben innerhalb dieser Ausprägungen unterschiedliche Antwortverteilungen evozieren. Abbildung \ref{freqantwortohne} gibt einen Überblick über das Antwortverhalten bei den Testverben der Ausprägung \textsc{infrequent ohne Schwankung}.\footnote{Das Testverb \textit{spinnen} könnte eine Verzerrung der Daten bei den Verben der Ausprägung \textsc{infrequent ohne Schwankung} auslösen: Die schwache Form \textit{gespinnt} könnte auch auf das homographe Verb \textit{spinnen} im Sinne von \SchmittSingleQuot{drehen} zurückgehen, das aus dem Englischen (\textit{to spin}) entlehnt wurde und daher mit /s/ statt mit /ʃ/ ausgesprochen wird. Diese Bedeutung sollte jedoch im deutschen Sprachgebrauch weit weniger frequent sein als die Bedeutung \SchmittSingleQuot{Wolle zu einem Faden drehen} bzw. \SchmittSingleQuot{verrückt sein}. Eine Beeinflussung der Ergebnisse durch das homographe Verb auf das Antwortverhalten kann jedoch nicht ausgeschlossen werden.} Die Testverben sind absteigend nach ihrer Ratio von starken zu schwachen Formen geordnet. Oben ist also mit \textit{flechten} das Verb mit der höchsten Ratio von starken zu schwachen Formen (807/1) und unten mit \textit{dreschen} das Verb mit der geringsten Ratio von starken zu schwachen Formen (123/1) (siehe \sectref{freqmat} für nähere Ausführungen zur Berechnung der Ratio zwischen starken und schwachen Formen). Insgesamt liegen 53 Antworten pro starker und schwacher Form jedes Testverbs vor, da 53 Proband\_innen die Testverben beurteilten. Ein Rechteck steht für die Antwort einer Person.

\begin{figure}
\includegraphics[width=\textwidth]{figures/Kap6/antwortinfrequentohne.png} 
\caption{Antwortverhalten bei den Testverben der Ausprägung \textsc{infrequent ohne Schwankung}}
\label{freqantwortohne}
\end{figure}

Wie bereits aus Abbildung \ref{freqantwortalle} deutlich wurde, werden die starken Formen durchweg als bekannt bewertet. Die schwachen Formen werden dagegen nur selten als bekannt gekennzeichnet. Der Anteil der Bekanntheit macht dabei maximal 35,8 \% (\textit{geflechtet}) aus. \textit{Geschmelzt} und \textit{angeschwellt} kommen nur auf einen Anteil von 9 \% bzw. 11 \% Bekanntheit, bei den restlichen Verben ist das Minimum an Bekanntheit schwacher Formen mit 24,5 \% deutlich höher. 

Abbildung \ref{freqantwortmit} zeigt das Antwortverhalten für die Testverben der Ausprägung \textsc{infrequent mit Schwankung}. Auch hier wurden insgesamt 53 Antworten pro starker und schwacher Form der Testverben abgegeben. Die Testverben sind wieder nach ihrer Ratio geordnet: Oben ist mit \textit{salzen} das Verb mit der größten Ratio von starken zu schwachen Formen (60/1) unten mit \textit{sinnen} das mit der kleinsten (0,59/1) (zur Berechnung der Ratios siehe \sectref{metfreq}).  Auch hier steht ein Rechteck für die Antwort einer Person.

\begin{figure}
\includegraphics[width=\textwidth]{figures/Kap6/antwortinfrequentm.png} 
\caption{Antwortverhalten bei den Testverben der Ausprägung \textsc{infrequent mit Schwankung}}
\label{freqantwortmit}
\end{figure}

Die starken Formen werden bei infrequenten Verben mit Schwankung nicht durchweg als bekannt angegeben. Dies ist nur bei \textit{hauen} und \textit{einsaugen} zu beobachten. Der Anteil der Bekanntheit starker Formen ist bei allen Verben bis auf \textit{glimmen} jedoch hoch: Nur in einer (1,8~\%) bis fünf Antworten  (9,4~\%) wird die Form als unbekannt bewertet. Im Gegensatz dazu fällt \textit{geglommen} auf, dass nur zu 60,9~\% als bekannt bewertet wurde. 



Die schwachen Formen werden sehr heterogen bewertet: \textit{Gehaut} wird nur zu 18,8 \% als bekannt gekennzeichnet. Dies könnte an der arealen Verteilung der schwachen Formen liegen: Diese sind laut der \textcite{VariantengrammatikdesStandarddeutschen.2018b} vorrangig in Österreich zu finden und machen dort in einigen Regionen 78 \% der Verwendungen aus, aber in Deutschland ist \textit{gehaut} sehr selten (2 \% in Bayern). Dieser Umstand wurde bei der Konzeption der Studie nicht berücksichtigt. Die Teilnehmer\_innen stammen allesamt aus Deutschland und nur 18 der 53 Proband\_innen aus Bayern, was die geringe Bekanntheit erklärt. Deutlich höher als bei \textit{gehaut} liegt der Anteil der Bekanntheit bei \textit{geglimmt} (28,3~\%), \textit{gequellt} (35,8~\%) und \textit{gesalzt} (37,7~\%). Dennoch sind die schwachen Formen bei diesen Verben vergleichsweise wenig bekannt: Die schwachen Formen anderer Testverben dieser Ausprägung werden zu über 50~\% als bekannt bewertet, dabei fällt besonders \textit{eingesaugt} mit 96,2~\% Bekanntheit auf.

Bei \textit{hauen}, \textit{salzen} und \textit{quellen} scheinen die schwachen Formen noch vergleichsweise wenig mental gefestigt zu sein, da die Bekanntheitswerte der schwachen Formen den höchsten Bekanntheitswerten der schwachen Formen von Verben der Ausprägung \textsc{infrequent ohne Schwankung} ähneln. Auch bei \textit{glimmen} ist die Bekanntheit für schwache Formen gering. Jedoch weisen bei \textit{glimmen} auch die starken Formen geringe Bekanntheitswerte auf. Sie scheinen somit nicht mehr stark gefestigt zu sein, aber die schwachen Formen sind noch nicht gefestigt, sodass beide Formen niedrige Bekanntheitswerte zeigen. Dies lässt darauf schließen, dass die Partizip-II-Formen des Verbs generell so infrequent sind, dass sich keine der Formen festigen kann. Diese Schlussfolgerung wird gestützt durch \textcite{Cappellaro.2013} und \textcite{Thornton.2019}, die vergleichbare Effekte bei infrequenten Strukturen beobachten.



Während bei \textit{glimmen} sowohl die starke als auch die schwache Form vergleichsweise hohe Anteile an Unbekanntheit aufweisen, ist bei \textit{einsaugen} das Gegenteil der Fall: Hier liegt die Bekanntheit für die schwache Form bei 96,2 \%  und für die starke bei 100 \%. Der Schluss liegt nahe, dass die starke und die schwache Form mit unterschiedlichen Funktionen assoziiert sind. Die schwache Form von \textit{saugen} ist mit dem Einziehen von Staub oder Flüssigkeit in technische Geräte assoziiert, sodass bspw. \textit{staubsaugen} nur schwache Formen aufweist (\cite[316--322]{Nowak.2013, Duden.2020}). Um in der Studie die Assoziation der schwachen Form mit Technik zu stören, wurde das Präfixverb \textit{einsaugen} genutzt (siehe \sectref{freqmat}), allerdings ist eine technische Bedeutung auch für \textit{einsaugen} nicht ausgeschlossen. Die schwachen Formen von \textit{saugen} könnten so stark mit technischen Geräten assoziiert sein, dass auch \textit{eingesaugt} und \textit{eingesogen} als Verbformen mit unterschiedlicher Konnotation wahrgenommen werden (\textit{Der Saugbagger hat das Öl eingesaugt} vs. \textit{Er hat den Zigarettenrauch eingesogen}). Diese Hypothese müsste in einer weiterführenden Studie überprüft werden: Sie ließe sich zunächst in Korpora und darauf aufbauend bspw. mithilfe einer \textit{sentence maze task} untersuchen, in der starke und schwache Formen von \textit{saugen} und \textit{einsaugen} in unterschiedlichen Kontexten genutzt werden. In der vorliegenden Studie zeigt sich hinsichtlich der logarithmierten Reaktionszeiten kein Unterschied zwischen starken (Mittelwert = 0,24; Standardabweichung = 0,41) und schwachen Formen (Mittelwert = 0,23; Standardabweichung = 0,36) von \textit{einsaugen} (ausführlich zur Analyse der Reaktionszeiten siehe \sectref{freqrt}). 

Für die konfirmatorische statistische Analyse der Daten wird ein generalisiertes gemischtes lineares Modell{\interfootnotelinepenalty=10000\footnote{Das Modell hat mit verschiedenen Optimierungsmethoden nicht konvergiert. Da die verschiedenen Konvergierungsmethoden jedoch vergleichbare Werte ergaben, scheinen die geschätzten Werte der Modelle dennoch verlässlich zu sein.}} genutzt, das Frequenzausprägung sowie Flexion als feste Effekte enthält: Antwortverhalten \~{} Frequenz * Flexion + (1|Proband\_in) + (1|Item). Die beiden Effekte können interagieren: Der Einfluss von Frequenz und Flexion wird also nicht separat betrachtet, indem der Einfluss der Frequenzausprägungen (bspw. \textsc{frequent} vs. \textsc{infrequent ohne Schwankung}) unabhängig vom Einfluss der Flexion getestet wird und dann separat der Einfluss der Flexion (schwach vs. stark) unabhängig von der Frequenzausprägung. Stattdessen wird getestet, wie Frequenzausprägung und Flexion zusammen das Antwortverhalten ändern (bspw. die Ausprägung \textsc{frequent} und schwache Flexion vs. \textsc{frequent} und starke Flexion). Als zufälliger Effekt werden \textit{random intercepts} für Pro\-\mbox{band\_in}\-nen und Versuchitems (Lemmata) angesetzt. \textit{Random slopes} für Proband\_innen waren in der Präregistrierung vorgesehen, werden aber aufgrund von \textit{singular fits} weggelassen, die anzeigen, dass die zufällige Effektstruktur für die vorliegenden Daten zu komplex ist. Da die Versuch\-items über die Frequenzausprägungen nicht konstant sind, sind \textit{random slopes} hierfür nicht sinnvoll (\cite[243]{Winter.2020}).\largerpage

 
Das Modell gibt Standardabweichungen um den y-Achsenabschnitt (\textit{intercept}) für die zufälligen Effekte aus: Die Standardabweichung der Testitems ist mit 0,89 geringfügig größer als die der Teilnehmer\_innen (0,85). Tabelle~\ref{antwortfreq} zeigt die geschätzten Werte für die festen Effekte. Die Werte sind auf zwei Nachkomma\-stellen gerundet. Der Wert für den y-Achsenabschnitt bezieht sich auf das Referenzlevel, für das die Ausprägung \textsc{infrequent ohne Schwankung} und schwache Formen gewählt wurden. Die drei nächsten Zeilen benennen den Effekt, wenn nur die Form (stark) oder nur die Frequenzausprägung (\textsc{frequent}, \textsc{infrequent mit Schwankung}) geändert werden. Die letzten beiden Zeilen geben den Effekt für eine Änderung in der Frequenzausprägung und in der Form an. Aufgrund der Interaktion ist eine direkte Interpretation der Werte nur eingeschränkt möglich.

\begin{table}
\begin{tabularx}{\textwidth}{Q S[table-format=-2.2] S[table-format=1.2] S[table-format=-4.2] S[table-format=<1.2]}
\lsptoprule
                                                    & {Wert} & {SE} & {$z$} & {$p$} \\\midrule
  (Intercept = infrequent ohne Schwankung, schwach) & 1,36 & 0,00 & 423,67 & < 0,01 \\ 
  Flexion: stark & -20,90 & 0,00 & -6522,19 & < 0,01  \\ 
  Frequenz: frequent & 2,85 & 0,00 & 890,86 & < 0,01  \\ 
  Frequenz: infrequent mit Schwankung & -1,50 & 0,00 & -468,60 & < 0,01  \\ 
  Flexion: stark \& Frequenz: frequent & 11,25 & 0,00 & 3513,81 & < 0,01  \\ 
  Flexion: stark \& Frequenz: infrequent mit Schwankung & 17,75 & 0,00 & 5526,05 & < 0,01  \\ 
  \lspbottomrule
\end{tabularx}
\caption{Werte des Modells für die Unbekanntheit starker und schwacher Formen in der Frequenzstudie}
\label{antwortfreq}
\end{table}

Festhalten lässt sich an dieser Stelle bereits, dass der Standardfehler (SE) für alle Werte minimal ist, sodass nur von einer sehr geringen Schwankung um die geschätzten Werte auszugehen ist. Um die Werte interpretieren zu können, bietet es sich an, diese in einer Kreuztabelle darzustellen (\cite[142--143]{Winter.2020}).

\begin{table}
\begin{tabular9}{lrr}
\lsptoprule
& schwach & stark \\\midrule
\textsc{infrequent ohne Schwankung} & $\mathbf{1{,}36}$ & $1{,}36 - 20{,}9 = \mathbf{-19{,}54}$\\
\textsc{frequent}                   & $1{,}36 + 2{,}85 = \mathbf{4{,}21}$ & $1{,}36 - 20{,}9 + 2{,}85 + 11{,}25 = \mathbf{-5{,}44}$ \\
\textsc{infrequent mit Schwankung} & $1{,}36 - 1{,}5 = \mathbf{-0{,}14}$ & $1{,}36 - 20{,}9 - 1{,}5 + 17{,}75 = \mathbf{-3{,}29}$\\
\lspbottomrule
\end{tabular9}
\caption{Kreuztabelle der Werte des Modells für die Unbekanntheit starker und schwacher Formen in der Frequenzstudie}
\label{kreuzfreq}
\end{table}

\begin{sloppypar}
Der y-Achsenabschnitt von 1,36 wird für alle Frequenzausprägungen und Flexionsformen als Basis genutzt.  Für die Berechnung der starken Formen der Ausprägung \textsc{infrequent ohne Schwankung} wird zusätzlich die Steigung von −20,9 berücksichtigt. Die anderen Zeilen folgen derselben Logik: Für die Frequenzausprägungen \textsc{frequent} und \textsc{infrequent mit Schwankung} wird 2,85 bzw. −1,5 addiert und für die starken Formen dieser Ausprägung zusätzlich 11,25 und 17,75. Die Werte 11,25 und 17,75 ergeben sich aus der Interaktion aus Frequenzausprägung und Flexionsform. 
\end{sloppypar}

  
Die  so berechneten Werte lassen sich interpretieren. Die Rohwerte sowie die berechneten Werte geben \textit{log odds} an. Aus den \textit{log odds} lässt sich auf Wahrscheinlichkeiten schließen: \textit{log odds} von 0 entsprechen einer Wahrscheinlichkeit von 0,5; \textit{log odds} von unter bzw. über 0 korrespondieren dementsprechend mit  Wahrscheinlichkeiten von unter bzw. über 0,5 (\cite[203--204]{Winter.2020}). Nur für schwache Formen der Ausprägung \textsc{infrequent mit Schwankung} liegen die \textit{log odds} mit −0,14 nahe 0. Die anderen \textit{logg odds} sind klar ausgeprägt. Für schwache Formen sind die \textit{log odds} positiv, für starke negativ. Dies weist auf ein unterschiedliches Antwortverhalten für schwache und starke Formen hin. 

Um die Wahrscheinlichkeit für negierende und affirmierende Antworten zu erhalten, müssen die Daten logistisch transformiert werden (\cite[206]{Winter.2020}). Die Wahrscheinlichkeit für die Unbekanntheit von Flexionsformen in Abhängigkeit von Frequenzausprägungen ist in Abbildung \ref{freqantwortestimates} dargestellt. Schwache Formen werden durch Kreise dargestellt, starke durch Dreiecke. 

\begin{figure}
\includegraphics[width=\textwidth]{figures/Kap6/predictfrequenzantw.png} 
\caption{Wahrscheinlichkeit für die Unbekanntheit starker und schwacher Formen in der Frequenzstudie}
\label{freqantwortestimates}
\end{figure}

\begin{sloppypar}
Hinsichtlich der Flexion sagt das Modell eine klare Verteilung vorher: Für starke Flexion sind frequenzunabhängig Werte um 0 zu erwarten, die Wahrscheinlichkeit für Unbekanntheit liegt also bei 0. Dementsprechend ist Bekanntheit sehr wahrscheinlich. Es zeigt sich aber, dass bei der Frequenzausprägung \textsc{infrequent mit Schwankung} die Unbekanntheit etwas wahrscheinlicher ist als für die anderen Ausprägungen. Hinsichtlich der schwachen Flexion hängt die Wahrscheinlichkeit für Unbekanntheit von der Frequenzausprägung ab: Bei der Ausprägung \textsc{frequent} liegt der Wert bei 1, Unbekanntheit ist also sehr wahrscheinlich. Bei der Ausprägung \textsc{infrequent ohne Schwankung} ist Unbekanntheit mit ungefähr 0,8 etwas weniger wahrscheinlich. Für die Ausprägung \textsc{infrequent mit Schwankung} liegt die Wahrscheinlichkeit für Unbekanntheit bei ungefähr 0,5, sodass Unbekanntheit und Bekanntheit gleich wahrscheinlich sind. 
\end{sloppypar}

Die 95 \%-Konfidenzintervalle für die Wahrscheinlichkeitswerte sind in der Darstellung nicht zu sehen, da sie kaum um die Werte schwanken. Das 95 \%-Konfidenzintervall gibt den Bereich an, in dem 95 \% der Stichproben liegen, die den wahren Mittelwert einer Population enthalten (\cite[45]{Field.2012}). Zieht man 100 Stichproben aus der gleichen Population, liegen 95 der Stichproben innerhalb des 95~\%-Konfidenzintervalls. Für alle Frequenzausprägungen ist von systematischen Effekten auszugehen, da die 95 \%-Konfidenzintervalle sich nicht überlappen.



Dies bestätigt auch ein Blick auf die $p$-Werte aus der Zusammenfassung des Modells: Alle Ausprägungen liegen unter dem $\alpha$-Level von 0,01. Es ist somit davon auszugehen, dass die vorliegenden Daten unter der Nullhypothese (die Flexion und die Frequenzausprägung beeinflussen das Antwortverhalten nicht) sehr unwahrscheinlich sind. Die Effektstärke des Modells lässt sich mithilfe des $R^2$-Werts\footnote{Die Berechnung eines $R^2$-Werts stellt für generalisierte gemischte lineare Modelle im Gegensatz zu gemischten linearen Modellen eine Herausforderung dar (\cite[1--2]{Nakagawa.2017}). Für generalisierte Modelle wird sowohl die theoretische als auch die observierte Varianz berechnet. Die theoretische Varianz bezieht sich auf die mögliche Varianz in der Verteilung, die observierte auf Varianz in den beobachteten Werten (\cite[4--6]{Nakagawa.2017}). Zur Berechnung der observierten Varianz wird die delta-Methode angewandt, weshalb der Wert delta genannt wird.} einschätzen. Dieser Wert gibt an, wie viel Varianz in den Daten durch das Modell abgedeckt wird (\cite[89--90]{Winter.2020}). Für das Modell ergibt sich ein Wert von 0,92  (theoretisch) bzw. 0,91 (delta). Fast die gesamte Varianz in den Daten kann also mithilfe der festen Effekte erklärt werden. Bezieht man die zufälligen Effekte mit ein, erhöht sich der Wert nur leicht auf 0,95 bzw. 0,93.

Aufgrund der Konvergierungsprobleme bei dem generalisierten linearen Modell wird zusätzlich ein \textit{random forest} mit Frequenzausprägung und Flexion als Einflussfaktoren gerechnet: Antwortverhalten \~{} Frequenz + Flexion. Für den \textit{random forest} wurden 1.000 \textit{conditional inference trees} gerechnet, für die jeweils beide Einflussfaktoren genutzt wurden (\cite[297]{Levshina.2015}).  Abbildung \ref{treef} zeigt den \textit{conditional inference tree} für die Einflussfaktoren.

\begin{figure}
\includegraphics[height = 0.26 \textheight, angle=90]{figures/Kap6/treef.png} 
\caption{\textit{Conditional inference tree} für die Bekanntheit bzw. Unbekanntheit starker und schwacher Formen in der Frequenzstudie}
\label{treef}
\end{figure}

\begin{sloppypar}
Der wichtigste Einflussfaktor ist die Flexion: Für starke Formen liegt die Wahrscheinlichkeit für Unbekanntheit unabhängig von der Frequenzausprägung maximal bei 0,05. Diesen maximalen Wert weist die Ausprägung \textsc{infrequent mit Schwankung} auf. Bei schwachen Formen hängt die Bekanntheit stark von der Frequenzausprägung ab: Bei \textsc{frequent} ist die Wahrscheinlichkeit für Unbekanntheit mit 0,95 sehr hoch, bei \textsc{infrequent ohne Schwankung} ist Unbekanntheit mit 0,75 ebenfalls recht wahrscheinlich. Bei \textsc{infrequent mit Schwankung} ist hinsichtlich schwacher Formen Bekanntheit und Unbekanntheit ungefähr gleich wahrscheinlich. Die Ergebnisse des generalisierten gemischten linearen Modells werden also bestätigt. Für den \textit{random forest} kann die \textit{conditional variance importance} berechnet werden, die angibt, wie sehr die einzelnen Faktoren Einfluss auf das Modell nehmen (\cite[298--299]{Levshina.2015}). Der Wert liegt für die Flexion bei 0,3 und für die Frequenzausprägung bei 0,06. Flexion (stark vs. schwach) beeinflusst das Antwortverhalten also deutlich stärker als die Frequenzausprägung (\textsc{frequent}, \textsc{infrequent ohne Schwankung}, \textsc{infrequent mit Schwankung}. Da es sich bei den Testverben um starke Verben handelt, ist dieser Befund zu erwarten. Für den \textit{random forest} wurde geprüft, wie gut er auf die Daten passt, hierfür wurde der C-Wert berechnet (\cite[299]{Levshina.2015}). Der Wert ist mit 0,94 hoch, \textcite[156]{Tagliamonte.2012} schlagen 0,8 als Minimum für eine gute Passung vor.
\end{sloppypar}

Die in \sectref{freqhyp} aufgestellten Hypothesen haben sich bestätigt: Starke Formen der Testverben sind unabhängig von der Frequenzausprägung bekannt. Nur bei den infrequenten Verben mit Schwankung (z.~B. \textit{geglommen}) zeigen sich erste Tendenzen zur Unbekanntheit. Die Bekanntheit der schwachen Formen hängt hingegen von der Frequenzausprägung ab: Während schwache Formen bei frequenten Verben (z.~B. \textit{gezieht}) unbekannt sind, zeigen sich erste Tendenzen zur Bekanntheit bei infrequenten Verben ohne Schwankung (z.~B. \textit{geflechtet}). Bei infrequenten Verben mit Schwankung (z.~B. \textit{gesinnt}) halten sich Unbekanntheit und Bekanntheit für schwache Formen die Waage.  

\subsection{Reaktionszeiten}\label{freqrt}

Der Blick auf die Reaktionszeiten zeigt, dass Testverben der Frequenzausprägung \textsc{frequent} kürzere Reaktionszeiten hervorrufen als Testverben der anderen Frequenzausprägungen. Abbildung \ref{freqrtbee} zeigt die logarithmierten Reaktionszeiten\footnote{Da im gemischten linearen Modell mit logarithmierten Reaktionszeiten gearbeitet wird, werden auch in der visuellen Betrachtung logarithmierte Reaktionszeiten berücksichtigt. Dies gilt für alle Studien.}  in einem Beeswarmplot. Jeder Punkt im Plot steht für die Reaktionszeit einer Person für einen einzelnen Stimulus. Die Reak\-tions\-zeiten für starke Formen sind als hellgraue Dreiecke dargestellt, die Reak\-tionszeiten für schwache Formen als dunkelgraue Kreise. Auf den Beeswarmplots ist jeweils das arithmetische Mittel (M) abgebildet, die Fehlerbalken geben die Standardabweichung an. Zusätzlich zum arithmetischen Mittel  sowie der Standardabweichung (\textit{standard devia\-tion} SD) wird der Standardfehler (\textit{standard error} SE) berichtet. Dieser fällt sehr klein aus, weshalb er nicht graphisch dargestellt wird. Die Standardabweichung gibt an, wie gut der Mittelwert die Daten repräsentiert (\cite[39--40]{Field.2012}). Der Standardfehler misst, wie gut der Mittelwert der Stichprobe dem Mittelwert der Population entspricht (\cite[42--43]{Field.2012}). Für beide Maße ist die Entsprechung höher, je kleiner der Wert ist.

\begin{figure}
\includegraphics[width= 1 \textwidth]{figures/Kap6/freqlog.png} 
\caption{Reaktionszeiten in der Frequenzstudie}
\label{freqrtbee}
\end{figure}

Alle Frequenzausprägungen zeigen eine ähnliche Verteilung der Reaktionszeiten. Allerdings ist deutlich zu erkennen, dass die Ausprägung \textsc{frequent} insgesamt geringere Reaktionszeiten hervorruft als die anderen Frequenzausprägungen. Dies zeigt sich daran, dass die Reaktionszeiten nicht so stark streuen (von −0,65 bis 1,4) wie die Reaktionszeiten der anderen Ausprägungen (von −0,56 bis 1,93 bei \textsc{infrequent ohne Schwankung} bzw. von −0,58 bis 2,02 bei \textsc{infrequent mit Schwankung}). Dies wird durch den Mittelwert gespiegelt, der mit 0,09 und einer Standardabweichung von 0,39 deutlich unter den Mittelwerten der Ausprägungen \textsc{infrequent ohne Schwankung} (0,31, Standardabweichung 0,49) und \textsc{infrequent mit Schwankung} (0,25, Standardabweichung 0,49) liegt. 

Die Verben der Ausprägungen \textsc{infrequent ohne Schwankung} und \textsc{infrequent mit Schwankung} scheinen vergleichbare Reaktionszeiten hervorgerufen zu haben: Die Reak\-tionszeiten sind ähnlich verteilt und auch die Differenz zwischen den Mittelwerten deutet mit 0,06 allenfalls auf einen vernachlässigbaren Unterschied hin.

Vergleicht man die Reaktionszeiten für starke und schwache Formen, so fällt auf, dass bei der Ausprägung \textsc{frequent} die starken Formen schneller beurteilt wurden als die schwachen Formen. Dies ist zu erwarten, da die schwachen Formen im Gegensatz zu den starken nicht mental gefestigt sind. Eine schnelle Ablehnung kann daher nur durch den Rückgriff auf starke Formen stattfinden, die die schwachen Formen aufgrund der hohen Tokenfrequenz statistisch ausstechen (zum statistischen Vorkaufsrecht siehe \cite[74--94]{Goldberg.2019} sowie \sectref{Statistik}). Für die Ausprägung \textsc{infrequent ohne Schwankung} lässt sich ebenfalls eine Tendenz zu niedrigeren Reaktionszeiten bei starken Formen erkennen, bei \textsc{infrequent mit Schwankung} scheint die Flexion keinen Einfluss auf die Reaktionszeiten zu haben. 

Der Blick auf die Mittelwerte der Reaktionszeiten für starke und schwache Formen bestätigt diesen Eindruck, wie Abbildung \ref{starkschwach} zeigt. Die Reaktionszeiten für schwache Formen sind als Kreise dargestellt, die Reaktionszeiten für starke Formen als Dreiecke. Die Fehlerbalken zeigen die Standardabweichung.

\begin{figure}
\includegraphics[width= 1 \textwidth]{figures/Kap6/starkschwachsep.png} 
\caption{Reaktionszeiten in der Frequenzstudie getrennt nach starken und schwachen Formen}
\label{starkschwach}
\end{figure}

Die Reaktionszeiten für schwache Formen liegen für alle Frequenzausprägungen über den Reaktionszeiten für starke Formen. Dies ist deutlich bei den Ausprägungen \textsc{frequent} und \textsc{infrequent ohne Schwankung} zu erkennen. Bei \textsc{infrequent mit Schwankkung} ist nur ein kleiner Vorteil für starke Formen zu sehen: Da schwache Formen bereits in Korpora attestiert sind, ist davon auszugehen, dass sie durch den Gebrauch mental gefestigt sind, sodass sie schneller verarbeitet werden können. 

Auch in dieser Darstellung zeigt sich, dass die starken und schwachen Formen der Ausprägung \textsc{frequent} jeweils geringere Reaktionszeiten aufweisen als die starken und schwachen Formen der anderen Ausprägungen. Die Reaktionszeiten für \textsc{infrequent ohne Schwankung} und \textsc{infrequent mit Schwankung} sind hingegen vergleichbar.

Neben dem oben diskutierten Einfluss des \textit{entrenchments} könnte ein weiterer Faktor die Unterschiede in den Reaktionszeiten zwischen starken und schwachen Formen hervorrufen: Die Ablehnung der Frage nach der Bekanntheit könnte längere Reaktionszeiten hervorrufen als die Zustimmung zur Frage. Die schwachen Formen würden in diesem Fall aufgrund der Ablehnung höhere Reaktionszeiten hervorrufen. 


Für schwache Formen von Verben der Ausprägung \textsc{infrequent mit Schwankung}, die ungefähr zur Hälfte als bekannt und unbekannt bewertet wurden, ist im arithmetischen Mittel ein kleiner Unterschied in den Reaktionszeiten zwischen der Zustimmung zur Frage nach der Bekanntheit (0,34; Standardabweichung 0,48) und der Ablehnung (0,43; Standardabweichung 0,48) zu beobachten, siehe Abbildung \ref{bekanntvsunbekannt}, für die nur schwache Formen der Ausprägung \textsc{infrequent mit Schwankung} berücksichtigt wurden. 

\begin{figure}[ph]
\includegraphics[width= 1 \textwidth]{figures/Kap6/bekanntheit.png} 
\caption{Reaktionszeiten für schwache Formen von Verben der Ausprägung \textsc{infrequent mit Schwankung} nach Bekanntheit}
\label{bekanntvsunbekannt}
\end{figure}

\begin{figure}[ph]
\includegraphics[width= 1 \textwidth]{figures/Kap6/stvssw.png} 
\caption{Scatterplots der Reaktionszeiten für starke und schwache Formen der Testverben}
\label{freqrtscatter}
\end{figure}

Als bekannt bewertete Formen scheinen etwas schneller beurteilt zu werden als Formen, die als unbekannt bewertet wurden: Die Reaktionszeiten bei als bekannt bewerteten Formen clustern stärker um den Mittelwert und streuen weniger. Ein Einfluss der Bekanntheit von Formen auf Reaktionszeiten kann daher nicht ausgeschlossen werden, scheint aber nur gering zu sein. Auch in anderen Studien weisen affirmierende Antworten niedrigere Reaktionszeiten auf als negierende (\cite[282--283]{Wegge.2001}). 

Der Unterschied in den Reaktionszeiten zwischen starken und schwachen Formen lässt sich anhand der Scatterplots in Abbildung \ref{freqrtscatter} näher betrachten. Die Abbildung vergleicht die Reaktionszeiten für starke und schwache Formen pro Frequenzausprägung. Ein Punkt gibt die Reaktionszeit einer Person für ein Test\-item an. Die grauen Linien in den Plots dienen als Orientierung: Sie zeigen den Bereich an, in dem starke und schwache Formen dieselbe Reaktionszeit aufweisen würden. Wenn starke und schwache Formen keinen Effekt auf die Reaktionszeiten haben, ist also zu erwarten, dass diese sich gleichmäßig um die Linie verteilen. Ist hingegen eine Verzerrung nach oben oder unten zu erkennen, deutet dies auf Unterschiede in den Reaktionszeiten hin.\clearpage

Aus der Abbildung lässt sich deutlich erkennen, dass die Verben der Ausprägung \textsc{frequent} und \textsc{infrequent ohne Schwankung} höhere Reaktionszeiten bei schwachen Formen hervorrufen als bei starken: Der Großteil der Punkte liegt oberhalb der grauen Linie. Für die Ausprägung \textsc{infrequent mit Schwankung} ist dies nicht mehr so deutlich zu erkennen: Zwar scheinen einige schwache Formen langsamer beurteilt worden zu sein als starke Formen, jedoch finden sich auch viele Punkte unterhalb der grauen Linie, die anzeigen, dass die starke Form langsamer bewertet wurde als die schwache. Zudem finden sich viele Punkte nahe der grauen Linie, sodass hier von vergleichbaren Reaktionszeiten für die Formen ausgegangen werden kann. Dies ist bspw. bei \textit{glimmen} der Fall: Starke und schwache Formen weisen bei diesem Verb mit Mittelwerten von 0,23 für die schwache Form und 0,24 für die starke mit Standardabweichungen von 0,36 und 0,4 im Schnitt vergleichbare Reaktionszeiten auf. Auch in dieser Darstellung zeigt sich der Effekt der Frequenzausprägungen: Während die Verben der Ausprägung \textsc{frequent} nur bis 1 schwanken, sind bei den anderen beiden Ausprägungen Schwankungen bis 2 zu erkennen. 

In Bezug auf die Reaktionszeitunterschiede zwischen starken und schwachen Formen ist ein Blick auf die Differenz in den Reaktionszeiten zwischen starken und schwachen Formen pro Testverb interessant. Tabelle \ref{diff} zeigt die Differenz in den Reaktionszeiten zwischen starken und schwachen Formen: Bei negativen Zahlen wurde die starke Form schneller bewertet, bei positiven die schwache. Die Tabelle ist nach aufsteigender Differenz sortiert. Somit stehen die Verben oben, bei denen die starke Form schneller bewertet wurde als die schwache, und die Verben unten, bei denen die schwache Form schneller bewertet wurde als die starke.

\begin{table}
\begin{tabular}{llS[table-format=-1.2]}
\lsptoprule
 Verb & Frequenzausprägung & {Differenz} \\\midrule
\textit{flechten} & infrequent ohne Schwankung & -0,53 \\ 
\textit{salzen} & infrequent mit Schwankung & -0,51 \\ 
\textit{fahren} & frequent & -0,43 \\ 
\textit{melken} & infrequent ohne Schwankung & -0,40 \\ 
\textit{halten} & frequent & -0,39 \\ 
\textit{kneifen} & infrequent ohne Schwankung & -0,36 \\ 
\textit{ziehen} & frequent & -0,31 \\ 
\textit{sinken} & frequent & -0,31 \\ 
\textit{spinnen} & infrequent ohne Schwankung & -0,29 \\ 
\textit{fechten} & infrequent ohne Schwankung & -0,27 \\ 
\textit{schmelzen} & infrequent ohne Schwankung & -0,27 \\ 
\textit{dreschen} & infrequent ohne Schwankung & -0,25 \\ 
\textit{anschwellen} & infrequent ohne Schwankung & -0,25 \\ 
\textit{fliegen} & frequent & -0,20 \\ 
\textit{schreiben} & frequent & -0,19 \\ 
\textit{sprechen} & frequent & -0,19 \\ 
\textit{hauen} & infrequent mit Schwankung & -0,19 \\ 
\textit{sinnen} & infrequent mit Schwankung & -0,16 \\ 
\textit{tragen} & frequent & -0,14 \\ 
\textit{gären} & infrequent mit Schwankung & -0,14 \\ 
\textit{quellen} & infrequent mit Schwankung & -0,12 \\ 
\textit{weben} & infrequent mit Schwankung & -0,12 \\ 
\textit{glimmen} & infrequent mit Schwankung & -0,04 \\ 
\textit{einsaugen} & infrequent mit Schwankung & 0,01 \\ 
\lspbottomrule
\end{tabular}
\caption{Differenz zwischen den Reaktionszeiten für starke und schwache Formen pro Testverb\label{diff}}
\end{table}

Es zeigt sich zunächst erneut, dass die starken Formen generell schneller bewertet wurden als die schwachen. Nur bei \textit{einsaugen} (0,01) und \textit{glimmen} (−0,04) ist der Unterschied praktisch inexistent. Interessanterweise sind dies die Verben, die beim Antwortverhalten andere Bewertungen evozierten als die restlichen Testverben: Für \textit{einsaugen} wurden beide Formen als bekannt angesehen, für \textit{glimmen} beide Formen eher als unbekannt bewertet (siehe hierzu ausführlich \sectref{freqant}).


Weiterhin fällt auf, dass die Verben, bei denen die Differenz zwischen starken und schwachen Formen −0,25 oder größer war, bis auf eine Ausnahme (\textit{salzen}) den Ausprägungen \textsc{frequent} und \textsc{infrequent ohne Schwankung} entstammen. Die Verben der Ausprägung \textsc{infrequent mit Schwankung} weisen dagegen (bis auf \textit{salzen}) maximal einen Unterschied von −0,19 zwischen den Formen auf. Diese Differenz ist auch für einige  Verben der Ausprägung \textsc{frequent} zu finden. Die kleinste Differenz für frequente Verben liegt bei −0,14 (\textit{tragen}). Für Verben der Ausprägung \textsc{infrequent ohne Schwankung} ist die kleinste Differenz  mit −0,25 hingegen deutlich höher. Der geringe Unterschied zwischen starken und schwachen Formen bei \textit{tragen} könnte auf die Identität im Stammvokal zwischen Infinitiv und Partizip~II zurückgehen, allerdings weisen auch andere frequente Verben (\textit{halten}, \textit{fahren}) Vokalidentität auf, bei denen eine höhere Differenz zwischen den Reaktionszeiten der starken und schwachen Form gemessen wurde. Der Blick auf die Differenzen zeigt insgesamt, dass Verben der Ausprägungen \textsc{frequent} und \textsc{infrequent ohne Schwankung} vergleichbare Unterschiede in den Reaktionszeiten zwischen starken und schwachen Formen hervorrufen, während Verben der Ausprägung \textsc{infrequent mit Schwankung} deutlich geringere Reaktionszeitunterschiede als die anderen Ausprägungen aufweisen. Die einzige Ausnahme hiervon ist \textit{salzen} (−0,51), das jedoch im Antwortverhalten eher Verben der Ausprägung \textsc{infrequent ohne Schwankung} gleicht als Verben der Ausprägung \textsc{infrequent mit Schwankung} (siehe \sectref{freqant}).

\begin{sloppypar}
Neben \textit{salzen} sind in der Analyse des Antwortverhaltens \textit{hauen} und \textit{quellen} aufgefallen (siehe \sectref{freqant}): Diese wurden aufgrund ihrer Ratios von starken zu schwachen Formen als \textsc{infrequent mit Schwankung} in die Studie integriert, allerdings war das Antwortverhalten für diese Verben vergleichbar mit dem Antwortverhalten für die Verben der Ausprägung \textsc{infrequent ohne Schwankung}. Um auszuschließen, dass das Antwortverhalten bei diesen Verben die Reaktionszeiten für die Ausprägung \textsc{infrequent mit Schwankung} verzerrt hat, wurden die Verben \textit{salzen}, \textit{hauen} und \textit{quellen} zu Verben der Ausprägung \textsc{infrequent ohne Schwankung} umkodiert.\footnote{\textit{Glimmen} wurde nicht umkodiert, da für dieses Verb nicht nur die schwachen, sondern auch die starken Formen häufig als unbekannt bewertet wurden.} Abbildung \ref{freqrtumkat} zeigt die Beeswarmplots mit der Umkodierung. Zusätzlich werden das arithmeritsche Mittel (M), die Standardabweichung (SD) und der Standardfehler berichtet. Aufgrund der Umkodierung ist der Beeswarmplot der Ausprägung \textsc{infrequent mit Schwankung} schmaler und der Beeswarmplot der Ausprägung \textsc{infrequent ohne Schwankung} breiter als in Abbildung \ref{freqrtbee}.
\end{sloppypar}

\begin{figure}
\includegraphics[width= 1 \textwidth]{figures/Kap6/freqlogumkategorisiert.png} 
\caption{Reaktionszeiten für die umkodierten Frequenzausprägungen}
\label{freqrtumkat}
\end{figure}


Es zeigt sich nach wie vor dieselbe Verteilung. Das abweichende Antwortverhalten bei \textit{salzen}, \textit{hauen} und \textit{quellen} scheint die Reaktionszeiten also nicht verzerrt zu haben. Zudem wurde ein möglicher Einfluss der Stimulusreihenfolge und der Wortlänge, die bedingt durch die starken und schwachen Formen leicht schwankt, visuell überprüft. Hierbei wurde jedoch kein Hinweis auf einen Effekt gefunden, siehe Anhang \ref{ergebnissefreq} für die entsprechenden Grafiken.

Der Einfluss von Frequenz und Flexion auf Reaktionszeiten wird mithilfe eines gemischten linearen Modells überprüft. Als feste Effekte werden die Frequenzausprägung sowie die Flexion genutzt, als zufällige Effekte Teilnehmer\_in und Versuchitem (Lemma): Reaktionszeit \~{} Frequenz + Flexion + (1|Proband\_in) + (1|Item). Für beide zufälligen Effekte werden nur \textit{random intercepts} berechnet. \textit{Random slopes} werden für Teilnehmer\_innen aufgrund von \textit{singular fits} nicht genutzt, bei Versuchitems sind \textit{random slopes} nicht sinnvoll, weil die Versuchitems über die Frequenzausprägungen hinweg nicht konstant sind.



Die Standardabweichung der Versuchitems beträgt nur 0,06. Bezüglich der Teilnehmer\_innen ist mit 0,25 eine deutlich größere Standardabweichung um den y-Achsenabschnitt zu be\-obachten. Das ist zu erwarten, da Reaktionszeiten individuell stark variieren. Trotz der zufälligen Effekte bleibt mit 0,34 eine relativ hohe Residuenvarianz. Tabelle \ref{rtfreq} zeigt die geschätzten Werte für die festen Effekte.\footnote{Die Abkürzung \textit{df} in der Tabelle steht für Freiheitsgrade (\textit{degrees of freedom}). Freiheitsgrade geben die Anzahl an Werten in der Berechnung an, die frei variieren können.}

\begin{table}
\begin{tabularx}{\textwidth}{Q S[table-format=-1.2] S[table-format=-1.2] S[table-format=-2.2] S[table-format=4.2] S[table-format=<1.2]}
\lsptoprule
& {Wert} & {SE} & {$t$} & {df} & {$p$}\\\midrule
(Intercept = frequent ohne Schwankung, schwach) & 0,38 & 0,04 & 8,91 & 75,88 & < 0,01 \\ 
Frequenz: frequent & -0,16 & 0,03 & -4,75 & 23,26 & < 0,01 \\ 
Frequenz: frequent mit Schwankung & 0,05 & 0,03 & 1,42 & 23,26 & 0,17 \\ 
Flexion: stark & -0,25 & 0,01 & -18,77 & 2467,93 & < 0,01 \\  
\lspbottomrule
\end{tabularx}
\caption{Werte des Modells für die Reaktionszeiten in der Frequenzstudie\label{rtfreq}}
\end{table}

Das Referenzlevel ist die Ausprägung \textsc{infrequent ohne Schwankung} mit schwachen Flexionsformen. Hierfür wird ein y-Achsenabschnitt von 0,38 mit einem Standardfehler von~0,04 geschätzt. Für die Ausprägung \textsc{frequent} ändert sich der Wert um −0,16 mit einem Standardfehler von 0,03 auf 0,22. Die Reaktionszeiten werden für die Ausprägung \textsc{frequent} also niedriger geschätzt als für \textsc{infrequent ohne Schwankung}. Für die Ausprägung \textsc{infrequent mit Schwankung} ändert sich der Wert mit 0,05 und einem Standardfehler von 0,03 im Vergleich zur Ausprägung \textsc{infrequent ohne Schwankung} hingegen kaum. 


Für starke Formen ändert sich der Wert im Vergleich zum Referenzlevel deutlich um −0,25 mit einem Standardfehler von 0,01 auf 0,23. Starke Formen werden also deutlich schneller bewertet als schwache. Für die Ausprägung \textsc{frequent} und für starke Flexion liegen die $p$-Werte im Vergleich zum Referenzlevel (\textsc{in\-fre\-quent ohne Schwankung}, schwache Formen) unter dem $\alpha$-Level von 0,01. Die Daten sind also unter der Nullhypothese, nach der die Frequenzausprägungen sowie die Flexion keinen Einfluss auf Reaktionszeiten haben, unwahrscheinlich. Für die Ausprägung \textsc{infrequent mit Schwankung} liegt der $p$-Wert mit 0,17 hingegen deutlich über dem $\alpha$-Level.

Wenn nur die festen Effekte berücksichtigt werden, ist die Effektstärke mit $R^2m = 0{,}117$ recht gering. Inklusive der zufälligen Effekte ist sie mit $R^2c = 0{,}443$ deutlich höher.\footnote{$R^2m$ und $R^2c$ geben an, wie viel Varianz durch das Modell erklärt wird. Hierfür wird die Residuenquadratsumme berechnet, indem die Residuen aufsummiert und quadriert werden. Die Residuenquadratsumme des Modells wird durch die Residuenquadratsumme des Nullmodells geteilt, das lediglich den Mittelwert aller Reaktionszeiten als Vorhersage für die Datenpunkte nutzt. Je kleiner die Residuenquadratsumme des Modells im Vergleich zum Nullmodell ist, desto geringer ist der berechnete Wert. Dieser Wert wird von 1 abgezogen, um R zu erhalten: Ist R hoch, ist die erklärte Varianz hoch, ist R hingegen niedrig, ist auch die erklärte Varianz niedrig (\cite[75--77]{Winter.2020}). Beim $R^2m$ werden nur die festen Effekte berücksichtigt, beim $R^2c$ zusätzlich zu den festen Effekten auch die zufälligen (\cite[264]{Winter.2020}).}  Dennoch bleibt ein Anteil von 0,557 der Varianz nicht erklärt. Da Reaktionszeiten hochvariabel sind und durch viele Faktoren beeinflusst werden, ist ein hoher Anteil nicht erklärter Varianz jedoch erwartbar.

Abbildung \ref{predrtfreq1} zeigt die vorhergesagten Reaktionszeiten. Schwache Formen sind als Kreise dargestellt, starke als Dreiecke. 

\begin{figure}
\includegraphics[width=\textwidth]{figures/Kap6/predrt.png} 
\caption{Vorhergesagte Reaktionszeiten in der Frequenzstudie}
\label{predrtfreq1}
\end{figure}

Die vorhergesagten Reaktionszeiten unterscheiden sich deutlich pro Frequenzausprägung: Die starken und schwachen Formen neigen für die Ausprägung \textsc{frequent} mit −0,03 (stark) und 0,22 (schwach) jeweils zu niedrigeren Reaktionszeiten als die starken und schwachen Formen der anderen Frequenzausprägungen. Die vorhergesagten Reaktionszeiten für die Ausprä\-gungen \textsc{infrequent ohne Schwankung} und \textsc{infrequent mit Schwankung} unterscheiden sich mit 0,13 bzw. 0,38 und 0,18 bzw. 0,43 zwar leicht, die 95 \%-Konfidenzintervalle überlappen sich jedoch deutlich. Aus Abbildung \ref{predrtfreq1} ist außerdem zu erkennen, dass für starke und schwache Formen unterschiedliche Reaktionszeiten vorhergesagt werden. Dies ist besonders deutlich bei den schwachen Formen der Ausprägung \textsc{frequent} zu erkennen, für die ähnlich hohe Reaktionszeiten wie für die starken Formen der anderen Frequenzausprägungen vorhergesagt werden.

Die visuelle Betrachtung der Daten deutet darauf hin, dass Frequenz und Flexion nicht unabhängig vonei\-nander Einfluss auf die Reaktionszeiten nehmen, sondern interagieren. So scheinen starke Formen von Verben der Ausprägung \textsc{frequent} schneller bewertet worden zu sein als schwache Formen von Verben derselben Ausprägung. Allerdings deuten die Daten nicht auf einen Unterschied in den Reaktionszeiten zwischen starken und schwachen Verbformen der Ausprägung \textsc{infrequent mit Schwankung} hin. Das obige Modell berücksichtigt dies nicht, da es keine Interaktion vorsieht und daher den Einfluss der Frequenz und der Flexion unabhängig voneinander betrachtet. Daher wird als explorative Analyse ein zweites gemischtes lineares Modell gerechnet, das eine Interaktion zwischen Form und Frequenzausprägung vorsieht: Reaktionszeit \~{} Frequenz * Flexion + (1|Proband\_in) + (1|Item). Dieses Modell wurde nicht als konfirmatorische Analyse gerechnet, da für die vorliegende Untersuchung in erster Linie Effekte der Frequenzausprägung und weniger Interaktionen mit starken und schwachen Formen von Interesse sind. Da es sich um eine explorative Analyse handelt, werden keine $p$-Werte berichtet. Die zufälligen Effekte sind identisch mit dem Modell ohne Interaktion. Tabelle \ref{rtfreqinter} zeigt die vom Modell berechneten Werte für das Modell,  Abbildung~\ref{predrtfreq} die vorhergesagten Reaktionszeiten inklusive der 95 \%-Konfidenzintervalle. Die Kreuztabelle der Werte (Tabelle \ref{rtfreqinterkr}) befindet sich in Anhang~\ref{ergebnissefreq}. Die Standardfehler sind für alle Werte mit bis zu 0,04 recht gering.

\begin{table}
\begin{tabularx}{\textwidth}{Q S[table-format=-1.2] S[table-format=1.2] S[table-format=-2.2] S[table-format=4.2]}
  \lsptoprule
   & {Wert} & {SE} & {$t$} & {df} \\\midrule
   (Intercept = infrequent ohne Schwankung, schwach) & 0,42 & 0,04 & 9,57 & 83,41 \\
   Frequenz: frequent & -0,19 & 0,04 & -5,04 & 34,91 \\
   Frequenz: infrequent mit Schwankung & -0,04 & 0,04 & -0,94 & 34,91 \\
   Flexion: stark & -0,33 & 0,02 & -14,17 & 2467,94\\
   Frequenz: frequent \& Flexion: stark & 0,06 & 0,03 & 1,74 & 2467,94 \\
   Frequenz: infrequent mit Schwankung \& Flexion: stark & 0,17 & 0,03 & 5,20 & 2467,94 \\
\lspbottomrule
\end{tabularx}
\caption{Werte des Modells für die Reaktionszeiten mit Interaktion zwischen Frequenz und Flexion}
\label{rtfreqinter}
\end{table}


\begin{figure}
\includegraphics[width=\textwidth]{figures/Kap6/predrti.png} 
\caption{Vorhergesagte Reaktionszeiten mit Interaktion zwischen Frequenz und Flexion}
\label{predrtfreq}
\end{figure}

Wie bei dem Modell ohne Interaktion lässt sich ein klarer Effekt der Frequenzausprägungen erkennen: Die starken Verbformen der Ausprägung \textsc{frequent} werden durchweg schneller beurteilt als die starken Verbformen der anderen Frequenzausprägungen. Dasselbe ist für die schwachen Verbformen festzuhalten. Durch die Interaktion zeigt sich, dass die Reaktionszeiten bei Verben der Ausprägung \textsc{infrequent mit Schwankung} weniger stark durch die Formen beeinflusst werden als die Reaktionszeiten bei Verben der anderen Frequenzausprägungen: Bei \textsc{infrequent mit Schwankung} ist der Wert für schwache Formen bei 0,38 und für starke bei 0,23, während bei den Verben der Ausprägung \textsc{infrequent ohne Schwankung} schwache Formen bei 0,42 und starke nur bei 0,09 liegen. Bei Verben der Ausprägung \textsc{frequent} ist der Unterschied mit 0,23 zu −0,04 ähnlich deutlich wie bei \textsc{infrequent ohne Schwankung}. Wieder zeigt sich eine Überlappung der 95 \%-Konfidenzintervalle für die Ausprägungen \textsc{infrequent ohne Schwankung} und \textsc{infrequent mit Schwankung}. Zudem überlappen sich bei der Ausprägung \textsc{infrequent mit Schwankung} die 95~\%-Konfidenzintervalle für die starken und schwachen Formen, sodass ein systematischer Einfluss der Flexion auf die Reaktionszeiten für diese Ausprägung fraglich ist. Dies bestätigt die visuelle Betrachtung der Daten, in der kein Einfluss starker und schwacher Formen zu erkennen war. Die Effektstärke des Modells hat sich durch die Interaktion mit einem $R^2m$ von 0,123 und einem $R^2c$ von 0,444 kaum geändert (vorherige Werte: $R^2m = 0{,}117$; $R^2c = 0{,}443$). Die festen Effekte erklären mit der Interaktion jedoch etwas mehr Varianz.

Insgesamt zeigt die Analyse der Reaktionszeiten, dass die Frequenzausprägungen Einfluss auf Reaktionszeiten nehmen: Frequente Testverben werden schneller bewertet als infrequente. Zwischen infrequenten Testverben mit und ohne Schwankung lässt sich hingegen kein Unterschied in den Reaktionszeiten messen. Zudem zeigt sich, dass starke Formen schneller bewertet werden als schwache. Die Analyse bestätigt damit insgesamt die Hypothesen aus \sectref{freqhyp}. Anders als vorab angenommen scheint die Bewertung als bekannt oder unbekannt die Reaktionszeiten zu beeinflussen, zudem zeigen sich Reaktionszeitunterschiede zwischen starken und schwachen Formen. Diese überdecken den Vorteil für frequente Verben jedoch nicht: Starke und schwache Formen der frequenten Verben wurden jeweils schneller beurteilt als starke und schwache Verbformen der anderen Ausprägungen. Die visuelle sowie die explorative Analyse legen nahe, dass starke Formen nur bei frequenten und infrequenten Testverben ohne Schwankung schneller bewertet werden als schwache Formen und sich die Reaktionszeiten für starke und schwache Formen bei infrequenten Verben mit Schwankung einander annähern. 
Im folgenden Abschnitt werden die Ergebnisse der sentence-maze-Studie zu Prototypizitätseffekten vorgestellt.

\section{Prototypizität}
\label{ergprot}

In diesem Abschnitt werden das Antwortverhalten und die Reaktionszeiten in der sentence-maze-Studie ausgewertet. Die Studie hat zum Ziel, anhand der Auxiliarselektion von \textit{haben} und \textit{sein} in Sätzen mit verschiedenen Transitivitätsgraden den Einfluss der Prototypizität auf Varia\-tion zu untersuchen. Die in diesem Abschnitt diskutierten Ergebnisse beziehen sich auf die in \sectref{probprototy} vorgestellten Proband\_innen. Die Stichprobengröße beträgt 107~Personen. In der Studie wählten die Proband\_innen zwischen \textit{hat} und \textit{ist} in Sätzen mit unterschiedlichen Transitivitätsgraden. Prototypisch transitive Testsätze enthalten ein belebtes Objekt und ein Direktionaladverbial (\textit{Er sah, wie die Mutter die Kinder zur Schule gefahren hat/*ist}). Prototypisch intransitive Sätze enthalten kein Objekt (\textit{Ich ging davon aus, dass die Oma am Sonntag zur Kur gefahren ist/*hat}). Zudem werden zwei ambige Testsätze genutzt: \textsc{Ambig~I} enthält ein Objekt, das ambig zwischen Objekt und Instrument ist (\textit{Er erzählte mir, dass seine Tante eine Zeit lang ein Cabrio gefahren hat/ist}), \textsc{ambig~II} enthält ein Akkusativobjekt mit Referenz auf eine unbelebte Entität, das als Adverbial aufgefasst werden kann (\textit{In den Nachrichten sagten sie, dass der neue Fahrer die Strecke in Rekordzeit gefahren ist/hat}). Die vier Transitivitätsausprägungen (\textsc{transitiv}, \textsc{ambig~I}, \textsc{ambig~II} und \textsc{intransitiv}) werden jeweils mit drei Testverben präsentiert (\textit{fahren}, \textit{fliegen}, \textit{reiten}) (siehe \sectref{metproto} für weitere Erläuterungen).

 
Für transitive Sätze werden \textit{haben}-Antworten erwartet, für intransitive \textit{sein}-Antworten. Für die ambigen Sätze wird basierend auf der Prästudie eine Tendenz zu \textit{sein}-Antworten antizipiert, allerdings keine so klare wie bei den intransitiven Sätzen (siehe \sectref{selektion} für weitere Erläuterungen zur Selektion von \textit{haben} und \textit{sein}). Hinsichtlich der Reaktionszeiten wurden ursprünglich kürzere Reaktionszeiten für die Prototypen (\textsc{transitiv} und \textsc{intransitiv}) und längere für die Peripherie (\textsc{ambig~I} und \textsc{ambig~II}) erwartet. Basierend auf der Prästudie werden allerdings kürzere Reaktionszeiten für die intransitiven Sätze erwartet als für die anderen Ausprägungen. Für die Ausprägungen \textsc{transitiv} und \textsc{ambig~II} werden vergleichbare Reaktionszeiten angenommen und für \textsc{ambig~I} werden die höchsten Reaktionszeiten erwartet. Die erhöhten Reaktionszeiten der transitiven und ambigen Sätze im Vergleich zu den intransitiven könnte durch die Komplexität der Sätze bedingt sein. Sollten sich  geringere Reaktionszeiten für die Ausprägung \textsc{ambig~II} im Vergleich zur Ausprägung \textsc{ambig~I} zeigen, könnte dies darauf hindeuten, dass \textsc{ambig~II} von den Proband\_innen nicht als ambig aufgefasst wird (siehe \sectref{prothypo} für genauere Erläuterungen zu den Hypothesen).      

\subsection{Antwortverhalten}
\label{antproto}
Die 107 Proband\_innen wählten für jedes Versuchitem (\textit{fahren}, \textit{fliegen}, \textit{reiten}) in jeder Transitivitätsausprägung zwischen \textit{hat} und \textit{ist}, sodass 107 Antworten pro Testverb und Transitivitätsausprägung vorliegen. Abbildung \ref{transantwortalle} zeigt das Antwortverhalten in einem Waffleplot. Ein Rechteck bildet jeweils eine Antwort ab. Rechtecke in dunkelgrau geben die Antworten wieder, bei denen sich die Proband\_innen für \textit{hat} entschieden, hellgraue Rechtecke dementsprechend Antworten für \textit{ist}. 

\begin{figure}
\includegraphics[width=\textwidth]{figures/Kap6/transantwort.png} 
\caption{Antwortverhalten in der Prototypizitätsstudie}
\label{transantwortalle}
\end{figure}

Es zeigt sich ein klarer Einfluss der Transitivität: Für die Ausprägung \textsc{transitiv} wird \textit{hat} gewählt, für \textsc{intransitiv} \textit{ist}. Während für \textsc{transitiv} auch vereinzelt \textit{sein}-Antworten gegeben werden, sind \textit{haben}-Antworten für \textsc{intransitiv} bis auf eine Ausnahme nicht zu finden. Zudem fällt das Testverb \textit{reiten} auf, bei dem auch für \textsc{transitiv} 44 \% der Proband\_innen \textit{ist} wählen. 



Der Blick auf \textsc{ambig~I} und \textsc{ambig~II} zeigt, dass hier vorwiegend \textit{ist} gewählt wird: Bei \textsc{ambig~I} liegt der Anteil von \textit{hat} für die Testverben \textit{fliegen} und \textit{fahren} bei ca. 14 \%, bei \textsc{ambig~II} nur bei 0,03 \%. Die Antwortverteilung für \textsc{ambig~I} weicht von den Ergebnissen von \textcite[286--291]{Gillmann.2016} ab. In den Sätzen der Ausprägung \textsc{ambig~I} wird ein Konkretum genutzt, das als Fortbewegungsmittel interpretiert werden kann (\textit{Er erzählte mir, dass seine Tante eine Zeit lang \textbf{ein Cabrio} gefahren hat/ist}). In ihrem Korpus stellt \textcite[286--291]{Gillmann.2016} für \textit{fahren} in solchen Sätzen eine klare Tendenz zur \textit{haben}-Selektion fest und für \textit{fliegen} Variation zwischen beiden Auxiliaren. Hinsichtlich der Ausprägung \textsc{ambig~II} bestätigen sich die Ergebnisse von \textcite[299--301]{Gillmann.2016} dagegen. Die Sätze der Ausprägung \textsc{ambig~II} enthalten als Akkusativergänzung Abstrakta, die Pfad-Akkusative darstellen (\textit{In den Nachrichten sagten sie, dass der neue Fahrer \textbf{die Strecke} in Rekordzeit gefahren ist/hat}). Für solche Sätze stellt \textcite[299--301]{Gillmann.2016} ebenfalls eine hohe \textit{sein}-Selektion fest. 



Auch in Bezug auf die ambigen Sätze verhält sich \textit{reiten} anders als \textit{fliegen} und \textit{fahren}: Mit 43,9~\% wählten weit mehr Proband\_innen für \textsc{ambig~I} \textit{hat} als für \textit{fliegen} und \textit{fahren}. Auch für \textsc{ambig~II} ist der Anteil an \textit{haben}-Antworten bei \textit{reiten} mit 16,8 \% im Vergleich zu den anderen Testverben hoch.\footnote{Die 20 Proband\_innen, die ausgeschlossen wurden, da sie in Bayern, Baden-Württemberg, Rheinland-Pfalz oder im Saarland aufgewachsen sind bzw. leben (siehe \sectref{probprototy}), weisen ein ähnliches Antwortverhalten auf: Für intransitive Sätze wird bei \textit{fahren} und \textit{fliegen} ausnahmslos \textit{ist} gewählt, für transitive \textit{hat}. Auch \textsc{ambig~II} ruft ausnahmslos \textit{sein}-Antworten hervor. Bei \textsc{ambig~I} finden sich nur vereinzelte \textit{haben}-Antworten (vier für \textit{fahren}, eine für \textit{fliegen}). Die Tendenz zu \textit{sein} in den ambigen Sätzen könnte bei diesen Proband\_innen  daher noch deutlicher sein als bei den Proband\_innen in der Stichprobe. \textit{Reiten} evoziert ein von den anderen Versuchsverben abweichendes Antwortverhalten: Die intransitiven Sätze rufen ausnahmslos \textit{sein} hervor, aber \textit{hat} ist in den anderen Ausprägungen häufiger als bei \textit{fahren} und \textit{fliegen}: Bei \textsc{ambig~II} wird es in vier Antworten gewählt, bei \textsc{ambig~I} in 13 Antworten. Der transitive Satz fällt mit relativ wenig \textit{haben}-Antworten (11) auf. Insgesamt scheinen die Antworten denen der Proband\_innen in der Stichprobe zu ähneln, allerdings könnte das Antwortverhalten aufgrund der wenigen Proband\_innen verzerrt sein. Für Reaktionszeiten wird aufgrund der kleinen Stichprobengröße kein Vergleich zwischen den Proband\_innen der Hauptstudie und den aus der Studie ausgeschlossenen Proband\_innen angestellt.} In den Daten von \textcite{Gillmann.2016} tendiert \textit{reiten} ebenfalls etwas stärker zu \textit{haben} als andere Fortbewegungsverben. Als Erklärung hierfür können die geringere Tokenfrequenz sowie die Semantik von \textit{reiten} angeführt werden, da es nicht nur auf eine Fortbewegung, sondern auch auf eine Aktivität referiert (\cite[273--280]{Gillmann.2016}). 

Um das Antwortverhalten für die einzelnen Testverben näher zu untersuchen, werden die Kombinationsmöglichkeiten der Auxiliare betrachtet, z.~B. könnte eine Person folgende Kombination für ein Testverb gewählt haben: \textsc{transitiv}: \textit{hat}, \textsc{ambig~I}: \textit{sein}, \textsc{ambig~II}: \textit{sein}, \textsc{intransitiv}: \textit{sein}. Für Tabelle~\ref{antwortmuster} wurde gezählt, wie häufig welche der möglichen Kombinationen pro Testverb gewählt wurde. Die Tabelle ist nach der Vorkommenshäufigkeit der Kombinationen für das Testverb \textit{fahren} sortiert.  Die Kombinationen beziehen sich auf die Abfolge \textsc{transitiv}, \textsc{ambig~I}, \textsc{ambig~II}, \textsc{intransitiv}.


\begin{table}
\begin{tabular}{l *3{r@{~}r}}
\lsptoprule
Kombinationsmöglichkeit & \multicolumn{2}{c}{\textit{fahren}} & \multicolumn{2}{c}{\textit{fliegen}} & \multicolumn{2}{c}{\textit{reiten}} \\\midrule
hat\_ist\_ist\_ist & 86 & (80,4 \%) & 78 & (72,9 \%) & 23 & (21,5 \%) \\ 
hat\_hat\_ist\_ist & 13 & (12,2 \%) & 14 & (13,1 \%) & 21 & (19,6 \%) \\
ist\_ist\_ist\_ist & 4  & (3,7 \%)  & 9  & (8,4 \%)  & 31 & (29,0 \%) \\ 
hat\_ist\_hat\_ist & 2  & (1,9 \%)  & 4  & (3,7 \%)  &  4 & (3,7 \%) \\ 
hat\_hat\_hat\_ist & 1  & (0,9 \%)  & 0  & (0,0 \%)  & 11 & (10,3 \%) \\  
ist\_hat\_ist\_ist & 1  & (0,9 \%)  & 2  & (1,9 \%)  & 13 & (12,2 \%) \\ 
ist\_hat\_hat\_ist & 0  & (0,0 \%)  & 0  & (0,0 \%)  &  2 & (1,9 \%) \\ 
ist\_ist\_hat\_ist & 0  & (0,0 \%)  & 0  & (0,0 \%)  &  1 & (0,9 \%) \\ 
ist\_ist\_ist\_hat & 0  & (0,0 \%)  & 0  & (0,0 \%)  &  1 & (0,9 \%) \\ 
\midrule
gesamt & 107 & (100 \%) & 107 & (100 \%) & 107 & (100 \%) \\
\lspbottomrule
\end{tabular}
\caption{Häufigkeit der Auxiliarkombinationen für die Abfolge \textsc{transitiv}, \textsc{ambig~I}, \textsc{ambig~II}, \textsc{intransitiv}}
\label{antwortmuster}
\end{table}

Bei \textit{fahren} und \textit{fliegen} haben mit 80,4~\% und 72,9~\% die meisten Proband\_innen nur für die Ausprägung \textsc{transitiv} \textit{hat} gewählt und für die anderen Ausprägungen \textit{ist}. Mit 12,2~\% bzw. 13,1~\% folgt weit abgeschlagen eine Kombination, bei der \textit{hat} für die Ausprägungen \textsc{transitiv} sowie \textsc{ambig~I} und \textit{ist} für die Ausprägungen \textsc{ambig~II} sowie \textsc{intransitiv} gewählt wurde. Nur wenige Proband\_innen wählen für alle Sätze \textit{ist} (3,7 \% bzw. 8,4 \%). Andere Kombinationen liegen unter 4 \%.\footnote{Ein Überblick über die Auxiliarkombinationen bei den Proband\_innen, die aufgrund des Bundeslands aus der Stichprobe ausgeschlossen wurden, befindet sich in Tabelle \ref{antwortmusteraus} in Anhang~\ref{ergebnisseproto}. Wie bei den Proband\_innen in der Hauptstudie tendieren \textit{fahren} und \textit{fliegen} hier eindeutig zu \textit{hat} für transitive Sätze und \textit{ist} für alle anderen Ausprägungen, während bei \textit{reiten} mehr Variation in den Auxiliarkombinationen vorherrscht.} 

 
Bei \textit{reiten} sind die Auxiliarkombinationen weniger eindeutig verteilt: Mit 29 \% wählen die meisten Proband\_innen für alle Ausprägungen \textit{ist}. Ein etwas geringerer Anteil (21,5 \%) wählt nur für die Ausprägung \textsc{transitiv} \textit{hat} und für die restlichen Sätze \textit{ist}. Mit 19,6 \% ebenfalls recht häufig sind \textit{haben}-Antworten für die Ausprägungen \textsc{transitiv} sowie \textsc{ambig~I} und \textit{sein}-Antworten für die Ausprägungen \textsc{ambig~II} sowie \textsc{intransitiv}. Deutlich seltener ist die Kombination \textit{hat} für \textsc{ambig~I} und \textit{ist} für alle anderen Ausprägungen (12,2 \%) sowie die Kombination \textit{hat} für alle Ausprägungen bis auf \textsc{intransitiv} (10,3 \%). Die anderen in der Tabelle aufgeführten Kombinationen liegen unter 4 \%. 



Während die Auxiliarkombinationen bei \textit{fahren} und \textit{fliegen} auf eine klare Präferenz für eine Trennung zwischen \textsc{transitiv} (\textit{haben}) und den anderen Ausprägungen (\textit{sein}) hindeuten und einige Proband\_innen die Auxiliargrenze zwischen \textsc{transitiv} sowie \textsc{ambig~I} (\textit{haben}) und \textsc{ambig~II} sowie \textsc{intransitiv} (\textit{sein}) ziehen, zeigt sich bei \textit{reiten} zusätzlich zu diesen Antwortkombinationen eine Tendenz dazu, unabhängig von der Transitivität \textit{sein} zu nutzen. Diese Antwortkombination ist bei \textit{fahren} und \textit{fliegen} nur vereinzelt zu finden. Zudem ist für \textit{reiten} bei weitem keine Präferenz zu einem Wechsel des Auxiliars zwischen \textsc{transitiv} und anderen Ausprägungen zu erkennen: Stattdessen sind drei Kombinationen (durchgängige \textit{sein}-Antworten; \textit{sein}-Antworten für alle Ausprägungen bis auf \textsc{transitiv}; \textit{hat} für \textsc{transitiv} und \textsc{ambig~I}, aber \textit{ist} für \textsc{ambig~II} und \textsc{intransitiv}) ähnlich häufig. Dass \textit{reiten} generell mehr Variation in der Auxiliarwahl zulässt als die anderen Testverben, verdeutlichen außerdem Kombinationen, die bei \textit{reiten} zu ca. 10~\% gewählt wurden, aber für \textit{fahren} und \textit{fliegen} maximal zweimal (\textit{hat} für alle Ausprägungen bis auf \textsc{intransitiv}, \textit{ist} für alle Ausprägungen bis auf \textsc{ambig~I}).  

Die häufige Wahl von \textit{sein} in allen Transitivitätsausprägungen bei \textit{reiten} lässt den Schluss zu, dass Transitivität für einige Proband\_innen keinen Einfluss auf die Auxiliarwahl bei \textit{reiten} hat. In Bezug auf die Auxiliarwahl bei transitiv gebrauchtem \textit{reiten} ist ein Blick auf die Korpusstudie von \textcite{Gillmann.2016} interessant: In ihrem Korpus sind nur 19 \textit{reiten}-Belege, die ein Objekt enthalten, das auf eine belebte Entität verweist (\textit{Ich bin/habe ein Pferd geritten}).\footnote{Diese entsprechen dem Satz der Ausprägung \textsc{ambig~I} in der Studie (\textit{Mir wurde erzählt, dass mein Onkel früher ein schwarzes Pferd geritten hat/ist}).} In zwei der 19 Belege ist das Auxiliar \textit{sein} zu finden (\cite[282--283]{Gillmann.2016}). \textit{sein}-Selektion ist daher im Korpus nicht ausgeschlossen, aber weitaus weniger prominent als in der vorliegenden Studie. Sätze mit Direktionaladverbial wie der Satz der Ausprägung \textsc{transitiv} (\textit{Ich gehe davon aus, dass das Mädchen das Pferd zur Koppel geritten hat/ist}) kommen im Korpus nicht vor, sodass diesbezüglich kein Vergleich angestellt werden kann. 


Die durchgängigen \textit{sein}-Antworten bei \textit{reiten} könnten auch durch die Transitivität des Testsatzes bedingt sein: Der transitive Satz könnte nur eingeschränkt als transitiv wahrgenommen worden sein. Pferde sind zwar belebt, aber nicht menschlich, sodass sie nicht prototypisch in der Patiensrolle vorkommen. Zudem lässt die Studie von \textcite{Gillmann.2016} darauf schließen, dass \textit{reiten} kaum mit Direktionaladverbial genutzt wird, sodass der transitive Satz eventuell als ungewöhnlich wahrgenommen wurde. Dies könnte auch eine Erklärung für die auffällige Auxiliarkombination bei \textit{reiten} sein, bei der nur für \textsc{ambig~I} \textit{hat} und ansonsten \textit{ist} gewählt wurde. 


Neben dem Einfluss der Prototypizität könnte auch Tokenfrequenz eine Rolle spielen: \textit{reiten} ist weniger tokenfrequent als \textit{fliegen} und \textit{fahren}. Es ist daher möglich, dass die Auxiliare weniger stark gefestigt sind als bei den frequenteren Verben.

Der Einfluss der Transitivität auf das Antwortverhalten wird für die konfirmatorische statistische Analyse mithilfe eines generalisierten gemischten linearen Modells überprüft.\footnote{\textit{Reiten} evozierte bereits in der Prästudie ein anderes Antwortverhalten als die anderen Testverben. Aufgrund dessen wurde in der Präregistrierung ein Modell angegeben, das \textit{reiten} nicht in der statistischen Analyse berücksichtigt. Das Modell ohne \textit{reiten} konvergierte jedoch auch mit verschiedenen Optimierungsmethoden nicht, weshalb alle Testverben in das Modell einbezogen wurden.} Als fester Effekt, der Einfluss auf die Wahl zwischen \textit{haben} und \textit{sein} nimmt, wird die Transitivitätsausprägung der Sätze eingeflochten: Antwortverhalten \~{} Transitivität + (1|Proband\_in). Für Teilnehmer\_innen werden \textit{random intercepts} genutzt, \textit{random slopes} werden aufgrund von Konvergierungsproblemen und \textit{singular fits} nicht berücksichtigt. Aus demselben Grund werden die Versuchitems (\textit{reiten}, \textit{fahren}, \textit{fliegen}) nicht als zufälliger Effekt einbezogen. Das Modell ist daher in seiner Aussagekraft eingeschränkt. Die Teilnehmer\_innen variieren mit einer Standardabweichung von 0,8 um den y-Achsenabschnitt. Tabelle \ref{protmodell} zeigt die \textit{log odds} für \textit{sein}-Antworten, Abbildung \ref{transpredict} die auf den \textit{log odds} basierenden Wahrscheinlichkeiten für \textit{sein}-Antworten in Abhängigkeit von der Transitivitätsausprägung. 

\begin{table}
\begin{tabular}{l S[table-format=-1.2] S[table-format=1.2] S[table-format=-2.2] S[table-format=<1.2]}
\lsptoprule
& {Wert} & {SE} & {$z$} & {$p$} \\\midrule
(Intercept = ambig~I) & 1,29 & 0,17 & 7,83 & < 0,01 \\ 
Transitivität: ambig~II & 1,44 & 0,26 & 5,62 & < 0,01 \\ 
Transitivität: intransitiv & 4,78 & 1,01 & 4,72 & < 0,01 \\ 
Transitivität: transitiv & -2,87 & 0,23 & -12,57 & < 0,01 \\ 
\lspbottomrule
\end{tabular}
\caption{Werte des Modells für \textit{sein}-Antworten in der Prototypizitätsstudie}
\label{protmodell}
\end{table}

\begin{figure}
\includegraphics[width=\textwidth]{figures/Kap6/protantpred.png} 
\caption{Wahrscheinlichkeit für \textit{sein}-Antworten in der Prototypizitätsstudie}
\label{transpredict}
\end{figure}

Für die Ausprägung \textsc{transitiv} wird ein Wert von 0,17 vorhergesagt, d.~h. die Wahrscheinlichkeit für \textit{sein}-Antworten ist gering. Dementsprechend ist die Wahrscheinlichkeit für \textit{haben}-Antworten hoch. Das Gegenteil ist bei \textsc{intransitiv} und \textsc{ambig~II} der Fall: Hier werden Werte von über 0,9 vorhergesagt. Auch \textsc{ambig~I} zeigt mit 0,78 eine klare Tendenz dazu auf, \textit{sein} zu evozieren. Die 95~\%-Kon\-fi\-denz\-in\-ter\-val\-le für die einzelnen Werte sind jeweils sehr klein und keine Überlappungen zu erkennen. Dies deutet auf ein systematisches Antwortverhalten je nach Transitivitätsausprägung hin. Für alle Ausprägungen liegt der $p$-Wert unter dem $\alpha$-Level von 0,01, die Daten sind also unter der Nullhypothese (die Transitivitätsausprägungen beeinflussen das Antwortverhalten nicht) unwahrscheinlich.

Die Effektstärke ist mit $R^2m = 0{,}66$ (theoretisch) bzw. 0,58 (delta) recht hoch, deckt aber bei Weitem nicht so viel Varianz ab wie das Modell zum Antwortverhalten in der Frequenzstudie (siehe \sectref{freqant}). Da \textit{reiten} sich anders verhält als die anderen Testverben und die Testverben nicht als zufälliger Effekt einbezogen werden konnten, ist dies nicht verwunderlich. Der $R^2c$, der zusätzlich zufällige Effekte einbezieht, ist mit 0,71 (theoretisch) und 0,62 (delta) etwas höher. 

Wie in der Präregistrierung vorgesehen, wird aufgrund der Konvergierungsprobleme zusätzlich ein \textit{random forest} für das Antwortverhalten gerechnet (Antwortverhalten \~{} Transitivität). Für den \textit{random forest} wurden 1.000 \textit{conditional inference trees} gerechnet, die den Einfluss der Transitivität überprüfen. Um die Ergebnisse des linearen Models und des \textit{random forests} besser vergleichen zu können, wurden auch hier, anders als in der Präregistrierung angegeben, alle Testverben einbezogen. 

Der \textit{conditional inference tree}  deutet auf einen klaren Einfluss der Transitivitätsausprägungen hin, wie Abbildung \ref{treetrans} zeigt.

\begin{figure}
\includegraphics[width= 1 \textwidth]{figures/Kap6/tree.png} 
\caption{\textit{Conditional inference tree} für die Wahl zwischen \textit{haben} und \textit{sein} in der Prototypizitätsstudie}
\label{treetrans}
\end{figure}

Während die Wahrscheinlichkeit für \textit{sein} für die Ausprägung \textsc{transitiv} bei 0,2 liegt, ist sie für alle anderen Ausprägungen deutlich höher: Für \textsc{ambig~I} liegt sie bei 0,8, für \textsc{ambig~II} bei 0,9 und für \textsc{intransitiv} bei 1. Die Ergebnisse decken sich also mit denen des generalisierten gemischten linearen Models. Der C-Wert des \textit{random forest} ist mit 0,89 hoch, es liegt somit eine gute Passung vor. 

Als explorative statistische Auswertung der Daten wird ein generalisiertes gemischtes lineares Modell gerechnet, in dem Transitivitätsausprägung und Versuchsverb als feste Effekte vorgesehen sind: Antwortverhalten \~{} Transitivität + Versuchsverb + (1|Proband\_in). Auf diese Weise kann der Einfluss der einzelnen Versuchsverben auf das Antwortverhalten gemessen werden. Dies ist von Interesse, da \textit{reiten} ein anderes Verhalten zu evozieren scheint als die anderen Verben. Das abweichende Antwortverhalten scheint dabei kein Zufall zu sein, sondern systematisch, da \textit{reiten} bereits in der Prästudie andere Antworten evozierte als \textit{fahren} und \textit{fliegen}. Proband\_innen werden als zufällige Effekte mit \textit{random intercepts}  einbezogen, \textit{random slopes} werden aufgrund von \textit{singular fits} weggelassen. Die Standardabweichung für die Proband\_innen um den y-Achsenabschnitt beträgt wie im Modell ohne Versuchsverb als festen Effekt 0,8. Die \textit{log odds} für \textit{sein}-Antworten finden sich in Tabelle~\ref{protexpl}. Abbildung~\ref{predantexpl} zeigt die auf den \textit{log odds} basierenden Wahrscheinlichkeiten für \textit{sein}-Antworten. Dabei stehen Kreise für \textit{fahren}, Dreiecke für \textit{fliegen} und Vierecke für \textit{reiten}.

\begin{table}
\begin{tabular}{l S[table-format=-1.2] S[table-format=1.2] S[table-format=-2.2]}
\lsptoprule
& {Wert} & {SE} & {$z$} \\\midrule
(Intercept, ambig~I \& fahren) & 1,30 & 0,21 & 6,23 \\ 
Transitivität: ambig~II & 1,44 & 0,26 & 5,62 \\ 
Transitivität: intransitiv & 4,79 & 1,01 & 4,72 \\ 
Transitivität: transitiv & -2,87 & 0,23 & -12,57 \\ 
Item: fliegen & 0,10 & 0,22 & 0,44 \\ 
Item: reiten & -0,12 & 0,22 & -0,55 \\  
\lspbottomrule
\end{tabular}
\caption{Werte des Modells für \textit{sein}-Antworten in Abhängigkeit von Transitivität und Testverb}
\label{protexpl}
\end{table}

Die Wahrscheinlichkeiten für \textit{sein}- bzw. \textit{haben}-Antworten haben sich hinsichtlich der Transitivitätsausprägungen nicht verändert. Auch hinsichtlich der Testverben unterscheiden sich die Wahrscheinlichkeiten kaum: Für alle Testverben ist eine klare Überlappung der 95 \%-Konfidenzintervalle für die jeweiligen Ausprägungen zu sehen. Die Effektstärke des Modells hat sich durch die Berücksichtigung des Testverbs als festen Effekt nicht verändert. 

Da \textit{reiten} in Abhängigkeit von der Transitivitätsausprägung Unterschiede im Antwortverhalten hervorzurufen scheint, ist ein Blick auf die Interaktion zwischen Testverb und Transitivitätsausprägung interessant. Um die Interaktion näher zu betrachten, wird ein \textit{random forest} mit Transitivitätsausprägung und Versuchsverb als Prädiktoren gerechnet (Antwortverhalten \~{} Transitivität + Versuchsverb).\footnote{Ein entsprechendes generalisiertes gemischtes lineares Modell mit Interaktion zwischen Transitivitätsausprägung und Versuchsverb und \textit{random intercepts} für Teilnehmer\_innen konvergierte auch mit verschiedenen Optimierungsmethoden nicht. Da die Optimierungsmethoden unterschiedliche Werte schätzen, wird das Modell nicht berücksichtigt. Der digitale Anhang enthält eine Dokumentation des Skripts zur Berechnung des
Modells.} Wieder wurden für den \textit{random forest} 1.000 \textit{conditional inference trees} gerechnet, die den Einfluss von Transitivität und Testitem überprüfen. Der \textit{conditional inference tree} zeigt wie in der konfirmatorischen Analyse einen klaren Einfluss der Transitivitätsausprägungen, wie Abbildung \ref{treetransitem} zeigt.


\begin{figure}[p]
\includegraphics[width=.9\textwidth]{figures/Kap6/protantpredexpl.png} 
\caption{Wahrscheinlichkeit für \textit{sein}-Antworten in Abhängigkeit von Transitivität und Testverb}
\label{predantexpl}
\end{figure}

\begin{figure}[p]
\includegraphics[width=.9\textwidth]{figures/Kap6/treeitem.png} 
\caption{\textit{Conditional inference tree} für die Wahl zwischen \textit{haben} und \textit{sein} in Abhängigkeit von Transitivität und Testverb}
\label{treetransitem}
\end{figure}

Zusätzlich zeigt sich der Einfluss der Testverben: \textit{Reiten} verhält sich anders als \textit{fahren} und \textit{fliegen}. So liegt die Wahrscheinlichkeit für \textit{sein}-Antworten bei der Ausprägung \textsc{transitiv} für das Testverb \textit{reiten} bei 0,45, während sie bei den anderen Versuchsverben nur bei 0,1 liegt. Auch bei \textsc{ambig~I} und \textsc{ambig~II} ist dies zu erkennen: Hier liegt die Wahrscheinlichkeit für \textit{sein}-Antworten in Bezug auf \textit{reiten} bei 0,55 bzw. 0,85, während sie in Bezug auf \textit{fahren} und \textit{fliegen} mit 0,85 bzw. 0,95 jeweils höher liegt. Nur bei der Ausprägung \textsc{intransitiv} hat das Testverb keinen Einfluss. Für die Transitivitätsausprägung ergibt sich eine \textit{con\-di\-tional variance importance} von 0,26, für das Testverb von 0,003. Die Transitivitätsausprägung nimmt also mehr Einfluss auf das Modell als das Testverb, was sich auch darin spiegelt, dass \textit{reiten} sich nur bei bestimmten Ausprägungen anders verhält als die anderen Testverben. Der C-Wert für das Modell liegt mit 0,93 höher als beim Modell ohne Testitem als gesonderten Einflussfaktor.

Insgesamt zeigt sich ein deutlicher Einfluss der Transitivität auf die Auxiliarwahl: Transitive Sätze evozieren erwartbarerweise \textit{haben}, intransitive \textit{sein}. Auch bei den ambigen Sätzen herrschen \textit{sein}-Antworten vor. Es ist jedoch ein Unterschied in den Antworten zwischen den ambigen Sätzen zu erkennen: \textsc{Ambig~I} evoziert etwas mehr \textit{haben}-Antworten als \textsc{ambig~II}. Der Einfluss der Prototypizität ist daher bei den Prototypen (transitive und intransitive Sätze) eindeutig festzustellen und in der Peripherie (ambige Sätze) zumindest zu erahnen. Der Einfluss der Transitivitätsausprägungen zeigt sich auch in der statistischen Analyse. Die Hypothesen aus \sectref{prothypo} bestätigen sich somit insgesamt. Zudem fällt \textit{reiten} durch ein von den anderen Testverben abweichendes Antwortverhalten auf: Dies wird vor allem in den häufigen \textit{sein}-Antworten im transitiven Satz und den vergleichsweise häufigen \textit{haben}-Antworten bei \textsc{ambig}~\textsc{I} und \textsc{II} deutlich. 




\subsection{Reaktionszeiten}\label{rtprotokap}\largerpage[-1]

Die Reaktionszeiten beziehen sich auf die Wahl zwischen \textit{hat} und \textit{ist} in den Testsätzen. Es zeigt sich ein deutlicher Unterschied zwischen intransitiven Sätzen und den anderen Ausprägungen. Diese evozieren höhere Reaktionszeiten als die intransitiven Sätze. Abbildung~\ref{rtproto} zeigt einen Beeswarmplot der logarithmierten Reaktionszeiten für die einzelnen Transitivitätsausprägungen inklusive arithmetischem Mittel (M) und Standardabweichung (SD) als Fehlerbalken. Zudem wird der Standardfehler (SE) berichtet. Die Reaktionszeiten für \textit{fahren} sind als Kreise dargestellt, die Reaktionszeiten für \textit{fliegen} als Dreiecke und die für \textit{reiten} als Vierecke.

\begin{figure}
\includegraphics[width= 1 \textwidth]{figures/Kap6/translogRT.png} 
\caption{Reaktionszeiten in der Prototypizitätsstudie}
\label{rtproto}
\end{figure}

Die Reaktionszeiten für die Ausprägung \textsc{intransitiv} streuen weniger zu hohen Werten als die der anderen Transitivitätsausprägungen: Die intransitiven Sätze weisen maximal Reaktionszeiten von 0,66 auf, die anderen Ausprägungen von bis zu über 1,5. Dies zeigt sich auch im Mittelwert der Reaktionszeiten für die intransitive Sätze, der bei −0,38 mit einer Standardabweichung von 0,3 liegt. Die anderen Ausprägungen zeigen höhere Mittelwerte und Standardabweichungen: Bei \textsc{ambig~I} und \textsc{ambig~II} liegen die Mittelwerte bei −0,12 bzw. −0,18 und die Standardabweichungen bei 0,6 bzw. 0,4. Der Mittelwert für die Ausprägung \textsc{transitiv} ist mit −0,17 und einer Standardabweichung von 0,5 vergleichbar mit dem Mittelwert für die Ausprägung \textsc{ambig II}. Insgesamt deuten die Beeswarmplots somit auf eine schnellere Verarbeitung der intransitiven Sätze im Vergleich zu den anderen Transitivitätsausprägungen hin. 


Hinsichtlich der Testverben zeigt sich keine auffällige Verteilung in den Daten. Da \textit{reiten} jedoch in der Prästudie andere Reaktionszeiten hervorrief als die anderen Testverben und \textit{reiten} zudem ein anderes Antwortverhalten evoziert, wird in Abbildung \ref{rtprotooR} die Verteilung der Reaktionszeiten ohne das Testverb \textit{reiten} betrachtet.  

\begin{figure}
\includegraphics[width= 1 \textwidth]{figures/Kap6/RTtranslogor.png} 
\caption{Reaktionszeiten in der Prototypizitätsstudie ohne \textit{reiten}}
\label{rtprotooR}
\end{figure}

Es sind vergleichbare Reaktionszeiten zu beobachten wie in den Beeswarmplots mit \textit{reiten}, sodass ausgeschlossen werden kann, dass \textit{reiten} einen gesonderten Einfluss auf die Reaktionszeiten nimmt. Ein Reihenfolge-Effekt konnte ebenfalls ausgeschlossen werden (siehe Abbildung \ref{blockprot} in Anhang \ref{ergebnisseproto}). Da die Länge der kritischen Segmente bei \textit{hat} und \textit{ist} nicht variiert und beide Stimuli gleichzeitig zu sehen sind, wird der Einfluss der Länge nicht überprüft.\largerpage

Um den Unterschied in den Reaktionszeiten zwischen den Transitivitätsausprägungen genauer in den Blick zu nehmen, werden Scatterplots betrachtet. Abbildung \ref{scatterintr} vergleicht die Reaktionszeiten für die Ausprägung \textsc{intransitiv} mit denen der anderen Ausprägungen. Ein Punkt stellt die Reaktionszeit einer Person für ein Testitem dar. Die Reaktionszeiten sind bei \textit{fahren} als Kreise, bei \textit{fliegen} als Dreiecke und bei \textit{reiten} als Vierecke dargestellt. Die graue Linie ist als Orientierung gedacht, sie zeigt den Bereich an, in dem die jeweils miteinander verglichenen Ausprägungen die gleiche Reaktionszeit hervorrufen würden.

Die Ausprägungen \textsc{transitiv}, \textsc{ambig~I} und \textsc{ambig~II} rufen höhere Reaktionszeiten hervor als die Ausprägung \textsc{intransitiv}, was sich deutlich an den Reaktionszeiten oberhalb der grauen Linie zeigt. Nur bei den niedrigen Reaktionszeiten (bis 0) wurden vergleichbare Zeiten gemessen. Ein Einfluss der Testverben auf die Reaktionszeiten ist wie in den Beeswarmplots nicht zu erkennen.

\begin{figure}[h]
\includegraphics[width=\textwidth]{figures/Kap6/vglintr.png} 
\caption{Scatterplots für \textsc{intransitiv} in Kontrast zu den anderen Transitivitätsausprägungen}
\label{scatterintr}
\end{figure}

Abbildung \ref{scatterrest} kontrastiert die Reaktionszeiten für die Ausprägung \textsc{transitiv} mit denen der Ausprägungen \textsc{ambig~I} und \textsc{ambig~II}. Zudem werden die Reaktionszeiten der Ausprägungen \textsc{ambig~I} und \textsc{ambig~II} miteinander verglichen. 

\begin{figure}
\includegraphics[width=\textwidth]{figures/Kap6/vglrest.png} 
\caption{Scatterplots für \textsc{transitiv}, \textsc{ambig~I} und \textsc{ambig~II}}
\label{scatterrest}
\end{figure}

Im Vergleich zu der Ausprägung \textsc{transitiv} streuen \textsc{ambig~I} und \textsc{ambig~II} stärker zu hohen Reaktionszeiten, was sich an den Punkten oberhalb der grauen Linie erkennen lässt. Anders als die Ausprägung \textsc{intransitiv} weist \textsc{transitiv} aber teilweise hohe Reaktionszeiten auf, wenn \textsc{ambig~I} und \textsc{ambig~II} geringe Reaktionszeiten zeigen, wie an den Punkten unterhalb der grauen Linie erkennbar ist. \textsc{ambig~I} und \textsc{ambig~II} evozieren vergleichbare Re\-ak\-tions\-zei\-ten: Die Reaktionszeiten schwanken um die graue Linie. Es sind zudem sowohl hohe Reaktionszeiten für \textsc{ambig~I} bei geringen Reaktionszeiten für \textsc{ambig~II} als auch geringe Reaktionszeiten für \textsc{ambig~I} bei hohen Reaktionszeiten für \textsc{ambig~II} zu sehen. Hinsichtlich der Testverben ist wiederum kein Einfluss auf die Reaktionszeiten zu erkennen. 


Die Reaktionszeiten deuten somit auf einen klaren Unterschied zwischen \textsc{intransitiv} und den anderen Ausprägungen hin. Zudem scheint einen kleiner Unterschied zwischen der Ausprägung \textsc{transitiv} und den Ausprägungen \textsc{ambig~I} und \textsc{ambig~II} zu bestehen, aber kein Unterschied zwischen \textsc{ambig~I} und \textsc{ambig~II}. 

Für die konfirmatorische statistische Analyse wird ein gemischtes lineares Modell mit der Transitivitätsausprägung der Testsätze als festem Effekt genutzt: Reaktionszeit \~{} Transitivität + (1|Proband\_in) + (1|Item).\footnote{Die Präregistrierung sieht vor, dass nur \textit{fahren} und \textit{fliegen} im Modell berücksichtigt werden. Da \textit{reiten} aber anders als in der Prästudie mit den anderen Versuchsverben vergleichbare Reaktionszeiten hervorruft, wird auch \textit{reiten} einbezogen.} Für Proband\_in und Testverb werden jeweils \textit{random intercepts} berücksichtigt, auf \textit{random slopes} muss aufgrund von \textit{singular fits} und Konvergierungsproblemen verzichtet werden.


Die Standardabweichung für die Versuchitems ist mit 0,04 sehr gering, für die Proband\_innen ist sie  erwartbarerweise mit 0,22 deutlich höher, da Reaktionszeiten individuell stark schwanken. Es bleibt eine Residuenvarianz von 0,44. Die vom Modell geschätzten Werte für die Reaktionszeiten sind in Tabelle \ref{ergprotrt} aufgeführt.  Abbildung \ref{protortpred} zeigt die Reaktionszeiten, die das Modell vorhersagt.

\begin{table}
\begin{tabular}{l S[table-format=-1.2] S[table-format=1.2] S[table-format=-1.2] S[table-format=4.2] S[table-format=<1.2]}
\lsptoprule
& {Wert} & {SE} & {$t$} & {df} & {$p$} \\\midrule
(Intercept = ambig~I) & -0,12 & 0,04 & 3,19 & 16,54 & < 0,01 \\ 
Transitivität: ambig~II & -0,06 & 0,03 & -1,60 & 1174,70 & 0,11 \\ 
Transitivität: intransitiv & -0,26 & 0,03 & -7,50 & 1174,70 & < 0,01 \\ 
Transitivität: transitiv & -0,05 & 0,03 & -1,49 & 1174,70 & 0,14 \\
\lspbottomrule
\end{tabular}
\caption{Werte des Modells für die Reaktionszeiten in der Prototypizitätsstudie}
\label{ergprotrt}
\end{table}

\begin{figure}
\includegraphics[width=\textwidth]{figures/Kap6/predrtprot.png} 
\caption{Vorhergesagte Reaktionszeiten in der Prototypizitätsstudie}
\label{protortpred}
\end{figure}

Für \textsc{transitiv}, \textsc{ambig~I} und \textsc{ambig~II} werden ähnliche Werte (−0,12 bis 0,17) vorhergesagt, deren 95~\%-Konfidenzintervalle sich überlappen. Die vorhergesagten Reaktionszeiten für \textsc{intransitiv} (−0,38) sind deutlich geringer und das 95~\%-Konfidenzintervall überlappt nicht mit den 95~\%-Konfidenzintervallen der Werte für die anderen Ausprägungen. 


Nur für \textsc{intransitiv} wird im Vergleich zum Referenzlevel \textsc{ambig~II} das $\alpha$-Level von 0,01 unterschritten. Die Ausprägung \textsc{intransitiv} scheint daher systematisch geringere Reaktions\-zeiten hervorzurufen als die anderen Ausprägungen. Dies ist für die Ausprägung \textsc{transitiv} nicht der Fall. Die Effektstärke des Modells ist mit einem $R^2m$ von 0,039 sehr gering, mit den zufälligen Effekten ist sie mit einem $R^2c$ von 0,24 deutlich höher.\footnote{Als explorative Analyse der Daten war in der Präregistrierung ein Modell geplant, das Transitivität und Versuchsverb als feste Effekte enthält, die interagieren. Dies war vorgesehen, weil \textit{reiten} in der Prästudie andere Reaktionszeiten evozierte als die anderen Testverben. Da die visuelle Betrachtung der Daten keinen Hinweis auf eine Interaktion gibt, wird darauf verzichtet.} 

Der geschätzte Wert für die Reaktionszeiten im transitiven Satz war in der Prästudie mit −0,2 deutlich niedriger als in der Hauptstudie (−0,05; siehe Tabelle \ref{ergprotrt}). Da die Stichprobengrößenberechnung für eine Teststärke von 0,87 auf diesem deutlich niedrigeren Wert beruhte, ist es möglich, dass die Teststärke zu gering war, um einen Unterschied zu entdecken. Allerdings ist der Schätzwert von −0,05 sehr klein, sodass ein Effekt zwar möglich, aber vernachlässigbar zu sein scheint. Ähnliches ist für \textsc{ambig~II} festzuhalten: Auch hier liegt der Schätzwert aus der Prästudie mit −0,2 deutlich niedriger als in der Hauptstudie (−0,06 siehe Tabelle \ref{ergprotrt}).

In den Hypothesen in \sectref{prothypo} wurden geringere Reaktionszeiten für die Ausprägung \textsc{intransitiv} erwartet als für die anderen Ausprägungen. Jedoch wurde auch erwartet, dass die Ausprägung \textsc{transitiv} geringere Reaktionszeiten hervorruft als die ambigen Sätze, da für transitive Sätze nur ein Auxiliar (\textit{haben}) mental stark gefestigt ist, während bei den ambigen Sätzen theoretisch beide Auxiliare möglich sind und in Korpora vorkommen. Aufgrund der Ergebnisse der Prästudie wurde zudem die Möglichkeit in Betracht gezogen, dass die Ausprägung \textsc{ambig~II} vergleichbare Reaktionszeiten hervorruft wie die Ausprägung \textsc{transitiv} und beide geringere Reaktionszeiten evozieren als \textsc{ambig~I}. Die Daten der Hauptstudie deuten hingegen darauf hin, dass die transitiven Sätze genauso schnell verarbeitet werden wie die ambigen Sätze und dass zwischen den ambigen Sätzen kein Unterschied in den Reaktionszeiten besteht. 



 Es stellt sich die Frage, warum die Ausprägung \textsc{transitiv} höhere Reaktionszeiten evoziert als die Ausprägung \textsc{intransitiv}. Eine mögliche Erklärung für die erhöhten Reaktionszeiten ist, dass die transitiven Sätze aufgrund des Objekts komplexer sind als die intransitiven Sätze. Diese Erklärung greift auch für die ambigen Sätze, die ebenfalls ein Objekt bzw. ein Adverbial aufweisen. Um zu überprüfen, ob die Reaktionszeitunterschiede allein durch die unterschiedliche Komplexität von intransitiven und transitiven Sätzen zu erklären sind, werden die Reaktionszeiten für die Auxiliarwahl in zwei transitiven Fillersätzen\footnote{Die Fillersätze lauten \textit{Ich weiß, dass der Vater die Kinder zur Schule gebracht hat} und \textit{Er fährt nochmal zurück, weil er die Jacke seines Freundes aus Versehen eingesteckt hat und dieser sie auf der Arbeit benötigt}. Beide wurden vorab von Proband\_innen als semantisch plausibel bewertet (4,4 und 4 auf einer Skala von 1 (nicht plausibel) bis 5 (plausibel), siehe Tabelle \ref{plaushaben}.} mit den bereits analysierten Reaktionszeiten der Ausprägungen \textsc{intransitiv} und \textsc{transitiv} verglichen. Abbildung \ref{transvgllogRT} zeigt Beeswarmplots für die intransitiven Sätze, die transitiven Fillersätze und die transitiven Testsätze.

\begin{figure}
\includegraphics[width= 1 \textwidth]{figures/Kap6/transvgllogRT.png} 
\caption{Reaktionszeiten für intransitive Sätze sowie transitive Filler- und Testsätze }
\label{transvgllogRT}
\end{figure}

Die Reaktionszeiten für die transitiven Fillersätze streuen weniger zu hohen Reaktionszeiten als die Reaktionszeiten für die transitiven Testsätze, was sich auch in der geringeren Standardabweichung zeigt. Zudem ähneln die Reaktionszeiten für die Fillersätze in ihrer Verteilung den Reaktionszeiten für die intransitiven Sätze, auch wenn die Verteilung etwas höhere Reaktionszeiten zeigt. Dies wird auch durch die Mittelwerte deutlich: Die Reaktions\-zeiten der transitiven Fillersätze haben einen Mittelwert von −0,28 mit einer Standardabweichung von 0,29. Der Mittelwert liegt damit zwischen dem Mittelwert der intransitiven Sätze (−0,38, Standardabweichung = 0,3) und dem Mittelwert der transitiven Testsätze (−0,18, Standardabweichung = 0,51). Das Komplexitäts\-argument scheint somit nur eingeschränkt zu greifen: Zwar sind die Reaktionszeiten für die transitiven Fillersätze höher als für die intransitiven Sätze, jedoch müssten sie ähnlich hoch sein wie die Reaktionszeiten für die transitiven Testsätze, wenn die Unterschiede in den Reaktionszeiten allein auf die Komplexität der Sätze zurückzuführen wären.


Die Daten legen daher eine andere Erklärung nahe: Wie \textcite[267--268]{Gillmann.2016} in ihrer Korpusstudie zeigt, werden Bewegungsverben vornehmlich in intransitiven Sätzen genutzt. Daher ist die \textit{sein}-Selektion bei Bewegungsverben viel häufiger als die \textit{haben}-Selektion, weshalb Bewegungsverben mit \textit{sein} assoziiert sind. In transitiven Sätzen muss entgegen dieser Assoziation \textit{haben} gewählt werden. Der Wechsel von \textit{sein} zu \textit{haben} ist daher mit kogni\-tivem Aufwand verbunden und könnte zu erhöhten Reaktionszeiten in den transitiven Testsätzen geführt haben. Diese Erklärung greift nicht für die ambigen Sätze, da hier vorrangig \textit{sein} gewählt wurde. Aufgrund der Ambiguität sind hier beide Auxiliare ähnlich wahrscheinlich. Dies spiegelt sich auch ansatzweise im Antwortverhalten wider, sodass sich die erhöhten Reaktionszeiten hier mit der Wahlmöglichkeit erklären lassen.  

Insgesamt zeigt sich, dass die intransitiven Sätze niedrigere Reaktionszeiten hervorrufen als die anderen Transitivitätsausprägungen. Für diese lassen sich keine Unterschiede in den Reaktionszeiten feststellen. Abweichend von diesem Befund waren für transitive Sätze basierend auf der Prästudie leicht erhöhte Reaktionszeiten im Vergleich zu intransitiven Sätzen erwartet worden, die vergleichbar mit den Reaktionszeiten der Ausprägung \textsc{ambig~II} sind, aber unter den Reaktionszeiten der Ausprägung \textsc{ambig~I} liegen (siehe \sectref{prothypo} für nähere Erläuterungen). Ein visueller Vergleich der Reaktionszeiten für transitive Filler- und Testsätze deutet darauf hin, dass die erhöhten Reaktionszeiten bei den transitiven Sätzen nicht allein auf die erhöhte Komplexität der transitiven Sätze zurückzuführen sind. Dies müsste in einer wei\-teren Studie statistisch geprüft werden. Der Schluss liegt nahe, dass Bewegungsverben generell mit \textit{sein} assoziiert sind und die Selektion von \textit{haben} in transitiven Sätzen daher einen kognitiven Aufwand darstellt, der sich in höheren Reaktionszeiten spiegelt. Somit lässt sich zumindest für einen der Prototypen (intransitive Sätze) ein schnelleres Antwortverhalten beobachten, für die Peripherie (ambige Sätze) hingegen ein langsameres. Die erhöhten Reaktionszeiten beim zweiten Prototyp (transitive Sätze) lassen sich durch das Wirken des anderen Prototyps erklären. 
Im folgenden Abschnitt werden die Ergebnisse der Studien zu Form-Schematizitätseffekten vorgestellt.

\section{Form-Schematizität}
\label{ergschema}

In diesem Abschnitt werden das Antwortverhalten und die Reaktionszeiten in den Studien zu Form-Schematizitätseffekten vorgestellt. Diese überprüfen anhand des Form-Schemas schwacher Maskulina den Einfluss von Form-Sche\-ma\-ti\-zi\-tät auf Variation. Alle  Daten beziehen sich auf die in \sectref{selfpacedprob} und \ref{probschema} vorgestellten Proband\_innen. Die Stichprobengröße beträgt 54 Personen für die self-paced-reading-Studie, 56 für die lexical-decision-Studie und 132 für die sentence-maze-Studie. 

\subsection{Antwortverhalten}

Das Antwortverhalten wurde innerhalb der Studien auf verschiedene Weisen evoziert: Im Rahmen der self-paced-reading-Studie wurde ein Produktionsexperiment durchgeführt, in dem die Pro\-band\_\-innen Pseudosubstantive (\textit{Schettose, Knatt, Grettel, Truntake}) in den Genitiv bzw. bei \textit{Truntake} in den Dativ setzten (siehe \sectref{methschema} für weitere Erläuterungen).\footnote{Die Antworten der Proband\_innen auf die Frage, ob ihnen innerhalb der \textit{self-paced reading task} Varianten aufgefallen sind, lassen nicht darauf schließen, dass die vorab gelesenen Varianten das Antwortverhalten im Produktionsexperiment beeinflusst haben. Die Antworten sind in den Rohdaten enthalten und im digitalen Anhang dokumentiert. Dennoch ist ein Einfluss der vorab durchgeführten \textit{self-paced-reading task} auf das Antwortverhalten nicht komplett auszuschließen.} In der lexical-decision-Studie wurden jeweils starke und schwache Formen von real existierenden Substantiven nach ihrer Bekanntheit bewertet und in der sentence-maze-Studie wurde zwischen starken und schwachen Formen von real existierenden Substantiven gewählt (siehe \sectref{schemalex} für weitere Erläuterungen). Die Testsubstantive entsprechen jeweils dem Prototyp des Form-Schemas schwacher Maskulina (z.~B. \textit{Kollege}, \textit{Neffe}, \textit{Schettose}), der Peripherie des Form-Schemas (z.~B. \textit{Graf}, \textit{Held}, \textit{Knatt}) oder zählen zur starken Flexion (z.~B. \textit{Vogt}, \textit{Freund}, \textit{Grettel}) (siehe \sectref{deklination} für Erläuterungen zur Variation in der Deklination von Maskulina). Als starke Genitivform wurde in der lexical-decision- und der sentence-maze-Studie meist die kurze Endung genutzt (\textit{des Grafs}), nur bei den Substantiven \textit{Dieb}, \textit{Freund} und \textit{Feind} wurde die lange Endung präsentiert (\textit{des Diebes}), da diese in den Wortverlaufskurven des DWDS (\cite{BerlinBrandenburgischeAkademiederWissenschaften.2019}) häufiger ist als die kurze Endung. 

 
Für Substantive, die dem Prototyp des Form-Schemas schwacher Maskulina komplett entsprechen, wird erwartet, dass schwache Formen im Produktionsexperiment genutzt werden, in der lexical-decision-Studie als bekannt bewertet werden und in der sentence-maze-Studie gegenüber den starken Formen bevorzugt werden. Starke Formen sollten hingegen nur selten produziert, als bekannt bewertet und gegenüber schwachen Formen bevorzugt werden. Für starke Maskulina wird das gegenteilige Antwortverhalten und damit eine Tendenz dazu erwartet, starke Formen zu nutzen, als bekannt zu bewerten und gegenüber schwachen Formen zu bevorzugen. Hinsichtlich der Testitems in der Peripherie des Form-Schemas wird ein schwankendes Antwortverhalten antizipiert.

\subsubsection{Produktionsexperiment in der self-paced-reading-Studie}\largerpage
\begin{sloppypar}
In diesem Abschnitt werden die Ergebnisse des Produktionsexperiments vorgestellt, das Teil der self-paced-reading-Studie war. Abbildung \ref{antwortueberblick} zeigt für die Aus\-prä\-gungen \textsc{Form-Schema} (\textit{Schettose}/\textit{Truntake}), \textsc{Peripherie} (\textit{Knatt}) und \textsc{stark} (\textit{Grettel}) den Anteil schwacher Formen (-\textit{(e)n}) im Vergleich zu anderen gewählten Formen (-\textit{(e)s}, Nullendungen, doppelte Formen). Bis auf \textit{Truntake} sollten alle Substantive in den Genitiv gesetzt werden, bei \textit{Truntake} wurde der Dativ elizitiert. Ein Rechteck entspricht jeweils einer Antwort. Dunkelgraue Rechtecke stehen für schwache Formen, hellgraue für andere Flexionsformen. Insgesamt wurden 54 Antworten gegeben. 
\end{sloppypar}

\begin{figure}
\includegraphics[width=\textwidth]{figures/Kap6/antwortengenerell.png} 
\caption{Antwortverhalten im Produktionsexperiment}
\label{antwortueberblick}
\end{figure}

Für \textit{Schettose} und \textit{Truntake} überwiegen erwartungsgemäß die schwachen Formen mit 79,6~\% bzw. 90,7~\% deutlich. Bei \textit{Grettel} ist das Gegenteil der Fall: Hier entfallen nur 11,1 \% der Antworten auf die schwache Form. \textit{Knatt} zeigt Variation in den Antworten: Mit 55,6 \%  entfallen nur gut die Hälfte der Antworten auf die schwache Form. Die Ergebnisse decken sich mit den Ergebnissen der Produktionsexperimente von \textcite{Kopcke.2000b} und \textcite{Schmitt.2019} (siehe \sectref{schemamask}) und bestätigen daher das Wirken des Form-Schemas in der Sprachproduktion.

Um die Formen, die in Abbildung \ref{antwortueberblick} unter \textsc{andere} zusammengefasst wurden, näher in den Blick zu nehmen, wird die Verteilung aller Formen für die einzelnen Test\-items betrachtet. Tabelle \ref{schettose} zeigt das Antwortverhalten für \textit{Schettose} und \textit{Truntake} gemeinsam, da beide Substantive zum Prototyp des Form-Schemas schwacher Maskulina gehören. \textit{Schettose} wurde im Genitiv abgefragt, \textit{Truntake} im Dativ. Schrägstriche werden in der Tabelle genutzt, wenn eine Form für ein Substantiv aufgrund des Kasusunterschieds nicht möglich ist. In der Tabelle ist die Form auf -\textit{n} hervorgehoben, da diese die zu erwartenden Deklinationsform darstellt. 

\begin{table}
\begin{tabular}{lrr}
\lsptoprule
Endung & \textit{Schettose} (im Genitiv) & \textit{Truntake} (im Dativ) \\\midrule
\textbf{-\textit{n}} &   \textbf{43} (79,6 \%) & \textbf{49} (90,7 \%) \\
-\textit{s} &    5 (9,3 \%) & / \\ 
 ohne &  4 (7,4 \%) & 4 (7,4 \%)\\ 
  -\textit{ns} & 1 (1,9 \%) & / \\
-\textit{n}/-\textit{s} &    1 (1,9 \%) & /\\ 
-\textit{n}/ohne & 0 (0 \%) & 1 (1,9 \%)\\
\midrule
gesamt & 54 (100 \%) & 54 (100 \%) \\
\lspbottomrule
\end{tabular}
\caption{Antwortverhalten bei den Testsubstantiven \textit{Schettose} und \textit{Truntake}}
\label{schettose}
\end{table}

Die wenigen Antworten, die nicht auf -\textit{n} entfallen, machen bei \textit{Schettose} Genitivformen auf -\textit{s} (9,3 \%)  und Nullendungen (7,4 \%) aus. Zudem wählte eine Person bei \textit{Schettose} -\textit{ns} und eine weitere sowohl -\textit{n} als auch -\textit{s}. Obwohl Nullendungen mit 7,4 \% bei \textit{Schettose} nur sehr selten gewählt wurden, ist ein näherer Blick auf sie interessant: Nullendungen sind auch in den Produktionsexperimenten von \textcite{Kopcke.2000b} und \textcite{Schmitt.2019} zu beobachten, die mit denselben Pseudosubstantiven arbeiten. \textcite[161]{Kopcke.2000b} berichtet bei einer Stichprobengröße von 31 Proband\_innen nur für 3 \% der Testsubstantive, die dem Prototyp des Form-Schemas entsprechen, Nullendungen. \textcite[167]{Schmitt.2019} nennt mit 5,5 \% einen ähnlich hohen Anteil an Nullendungen für \textit{Schettose} wie in der vorliegenden Studie. Auch die Stichprobengrößen der Studien sind mit 55 und 54~Proband\_innen ähnlich groß.

 
Nullendungen sind zu erwarten, da die Testsubstantive Pseudowörter darstellen und daher den Proband\_innen unbekannt sind (\cite[167]{Schmitt.2019}). Deshalb liegt es nahe, dass das Prinzip der Morphemkonstanz greift: Seltene (und damit auch unbekannte) Wörter sollten aufgrund ihrer geringen Tokenfrequenz möglichst wenig verändert werden, um sicherzustellen, dass sie wiedererkannt werden (\cite[150]{Ackermann.2017}). Dies kann am besten erreicht werden, indem die Form der Wörter überhaupt nicht verändert wird.  Das Morphemkonstanzprinzip lässt sich bei Wörtern in der Peripherie der deutschen Substantive beobachten, bspw. bei Eigennamen, Kurzwörtern und tokeninfrequenten nicht-nativen Wörtern (\cite[151--154]{Ackermann.2017}). Da die Pseudowörter in Texte eingeflochten waren, die Lexikoneinträgen ähneln, ist es möglich, dass die Proband\_innen die Pseudowörter als real existierende Wörter angesehen und sie aufgrund der geringen Bekanntheit wie periphere Substantive wahrgenommen haben. 

 
Bei \textit{Truntake} entfallen die wenigen von -\textit{n} abweichenden Antworten auf die starke und somit endungslose Dativform (7,4 \%) und eine Doppelnennung der starken endungslosen und der schwachen Form auf -\textit{n}.



Auch für \textit{Knatt} ist ein Blick auf die genauere Aufschlüsselung der gewählten Formen in\-teres\-sant, die Tabelle \ref{knatt} zeigt. Für \textit{Knatt} sind sowohl -\textit{n} als auch -\textit{s} erwartbar, weswegen beide Formen graphisch hervorgehoben sind. 

\begin{table}
\begin{tabular}{lr}
\lsptoprule
Endung  & Häufigkeit \\ 
\midrule
\textbf{-\textit{en}}  &  \textbf{30 (55,6 \%)} \\
\textbf{-\textit{(e)s}}  &  \textbf{16 (29,6 \%)} \\ 
-\textit{en}/-\textit{(e)s}  &   5 (9,3 \%) \\ 
-\textit{en}/\textit{ens} & 1 (1,9 \%)\\
-\textit{ers}  &   1 (1,9 \%) \\ 
ohne  &   1 (1,9 \%)\\
\midrule
gesamt  & 54 (100 \%) \\
\lspbottomrule
\end{tabular}
\caption{Antwortverhalten bei dem Testsubstantiv \textit{Knatt}}
\label{knatt}
\end{table}

Wie zu erwarten ist, entfallen bei \textit{Knatt} die meisten Formen, die nicht auf -\textit{n} enden, auf -\textit{(e)s} (29,6 \%).\footnote{Von den 16 Formen auf -\textit{(e)s} sind fünf lang (-\textit{es}) und elf kurz (-\textit{s}).} Zudem sind bei \textit{Knatt} mit 9,3~\% Doppelnennungen von -\textit{en} und -\textit{(e)s} vergleichsweise häufig: Bei \textit{Schettose} und \textit{Truntake} wurden jeweils nur einmal Formen doppelt genannt. Zudem wird bei \textit{Knatt} einmal -\textit{en} und -\textit{ens} doppelt genannt. Die Nullendung kommt bei \textit{Knatt} einmal vor, genauso wie die Endung auf -\textit{ers}. Auch diese Endung hat \textcite[167--168]{Schmitt.2019} bereits beobachtet. Sie kann als eine Anpassung von \textit{Knatt} an das Form-Schema starker Maskulina gewertet werden: Formen auf -\textit{er} werden immer stark dekliniert. Indem -\textit{er} an \textit{Knatt} angehängt wird, wird das Variationspotential von \textit{Knatt} aufgelöst (\cite[168]{Schmitt.2019}). 



\textit{Grettel} weist erwartungsgemäß den größten Anteil starker Formen auf. Diese entfallen zu 58,8 \% auf -\textit{s}, wie Tabelle \ref{grettel} zeigt.  -\textit{s} ist in der Tabelle hervorgehoben, da -\textit{s} die zu erwartende Form darstellt.

\begin{table}
\begin{tabular}{lr}
\lsptoprule
Endung  & Häufigkeit \\ 
\midrule
\textbf{-\textit{s}}  &  \textbf{32 (59,3 \%)} \\
ohne  &   9 (16,7 \%) \\ 
-\textit{s}/ohne &   7 (13 \%) \\ 
-\textit{n} &   6 (11,1 \%) \\ 
\midrule
gesamt  & 54 (100 \%) \\
\lspbottomrule
\end{tabular}
\caption{Antwortverhalten bei dem Testsubstantiv \textit{Grettel}}
\label{grettel}
\end{table}

Die Form auf -\textit{n} ist bei \textit{Grettel} auch in dieser detaillierten Betrachtung mit 11,1 \% am seltensten. Der Anteil an Nullendungen ist dagegen mit 16,7 \% im Vergleich zu den anderen Testsubstantiven am höchsten. Dies ist auch in der Studie von \textcite[167]{Schmitt.2019} der Fall, bei der die Nullendungen 32,7 \% der gewählten Formen für \textit{Grettel} ausmachen. Ähnlich häufig wie Nullendungen sind in der vorliegenden Studie Doppelnennungen von -\textit{s} und Nullendung (13 \%). In der Studie von \textcite[167]{Schmitt.2019} kommen für \textit{Grettel} hingegen gar keine Doppelnennungen dieser Art vor. Der hohe Anteil an Nullendungen bei \textit{Grettel} im Vergleich zu \textit{Schettose} und \textit{Knatt} könnte durch das Form-Schema schwacher Maskulina bedingt sein: Bei \textit{Schettose} kann der naheliegende Anschluss an das Form-Schema die Möglichkeit der Nullendung ausstechen, bei \textit{Knatt} könnte dies trotz der Peripherie des Form-Schemas ebenfalls der Fall sein. Dies ist für \textit{Grettel} ausgeschlossen, daher zeigt \textit{Grettel} eine klare Tendenz zu starken Formen (-\textit{s} oder Nullendung).   

Für die konfirmatorische statistische Analyse des Antwortverhaltens wird ein generalisiertes lineares Modell mit Form-Schematizitätsausprägungen als festem Effekt genutzt: Antwortverhalten \~{} Form-Sche\-ma\-ti\-zi\-tät. Dabei wird das Antwortverhalten binär betrachtet: Den schwachen Formen werden nicht-schwache Formen (-\textit{(e)s}, -\textit{ers}, Nullendung, doppelt genannte Formen) gegenübergestellt. Die Form-Schematizitätsausprägungen werden durch die Testitems (\textit{Schettose}, \textit{Truntake}, \textit{Knatt} und \textit{Grettel}) repräsentiert. Prototypische Vertreter des Form-Schemas sind mit \textit{Schettose} und \textit{Truntake} zweimal vertreten, jedoch einmal im Genitiv und einmal im Dativ, weshalb sie als zwei Ausprägungen betrachtet werden. Aufgrund von \textit{singular fits} können Proband\_innen nicht als zufälliger Effekt berücksichtigt werden. Ursprünglich waren \textit{random intercepts} für Proband\_innen vorgesehen, \textit{random slopes} sind nicht möglich, da pro Form-Schematizitätsausprägung nur ein Testitem genutzt wird. Die \textit{singular fits} sind vermutlich der singulären Itemstruktur geschuldet: Hierdurch liegen zu wenig Beobachtungen vor, um die zufällige Effektstruktur zu modellieren. Daher ist die Annahme der unabhängigen Datenpunkte verletzt, weshalb zusätzlich zu diesem Modell ein \textit{random forest} gerechnet wird. Die \textit{log odds} für die Produktion schwacher Formen sind in Tabelle \ref{antwortlesemod} gelistet. Abbildung~\ref{predantles} zeigt die auf den \textit{log odds} basierende Wahrscheinlichkeit für die Wahl schwacher Formen im Produktionsexperiment.\largerpage[-1]

\begin{table}
\begin{tabular}{l S[table-format=-1.2] S[table-format=1.2] S[table-format=-1.2] S[table-format=<1.2]}
\lsptoprule
& {Wert} & {SE} & {$z$} & {$p$} \\\midrule
(Intercept = Schettose) & 1,36 & 0,34 & 4,03 & < 0,01 \\  
Item: Grettel & -3,44 & 0,55 & -6,27 & < 0,01 \\  
Item: Knatt & -1,14 & 0,43 & -2,62 & < 0,01 \\  
Item: Truntake & 0,92 & 0,58 & 1,59 & 0,11 \\ 
\lspbottomrule
\end{tabular}
\caption{Werte des Modells für die Wahl schwacher Formen im Produktionsexperiment}
\label{antwortlesemod}
\end{table}

\begin{figure}
\includegraphics[width=\textwidth]{figures/Kap6/predant.png} 
\caption{Wahrscheinlichkeit für die Wahl schwacher Formen im Produktionsexperiment}
\label{predantles}
\end{figure}

Für \textit{Schettose} und \textit{Truntake} werden Werte von 0,8 bzw. 0,9 angesetzt, d.~h. schwache Antworten sind sehr wahrscheinlich. Das Gegenteil ist für \textit{Grettel} der Fall, das einen Wert von 0,1 aufweist. Für \textit{Knatt} sind mit 0,56 beide Formen gleich wahrscheinlich. Die 95~\%-Konfidenzintervalle sind jeweils relativ lang, aber überlappen sich bis auf die Intervalle von \textit{Schettose} und \textit{Truntake} nicht. Die Überlappung bei \textit{Schettose} und \textit{Truntake} ist erwartbar, da beide Substantive dem Prototyp des Form-Schemas schwacher Maskulina entsprechen. Abseits von \textit{Schettose} und \textit{Truntake} ist daher von systematischen Effekten im Antwortverhalten auszugehen, die durch die Form-Schematizitätsausprägungen ausgelöst werden. Dies bestätigt auch ein Blick auf die $p$-Werte, die für alle Ausprägungen bis auf \textit{Truntake} im Vergleich zum Referenzlevel \textit{Schettose} unter dem $\alpha$-Level von 0,01 liegen. Die Effektstärke weist mit einem $R^2m$ von 0,45 (theoretisch) und 0,39 (delta) auf einen mittleren Effekt hin. Sie ist jedoch im Vergleich zu den Modellen zum Antwortverhalten in den Studien zu Frequenz und Prototypizität relativ gering. Dies könnte daran liegen, dass das Antwortverhalten durch die Aufteilung in schwache und andere Formen stark vereinfacht wurde.

Zusätzlich zum generalisierten linearen Modell wird ein \textit{random forest} mit Form-Schemati\-zi\-tät als Einflussfaktor gerechnet: Antwortverhalten \~{} Form-Sche\-ma\-ti\-zi\-tät. Für den \textit{random forest} wurden 1.000 \textit{conditional inference trees} gerechnet, die den Einfluss der Form-Schematizität überprüfen. Abbildung \ref{treeles} zeigt den \textit{conditional inference tree} für die Wahl schwacher und anderer Formen. 

\begin{figure}
\includegraphics[width= 1 \textwidth]{figures/Kap6/treelese.png} 
\caption{\textit{Conditional inference tree} für die Wahl schwacher und anderer Formen im Produktionsexperiment}
\label{treeles}
\end{figure}

Der \textit{conditional inference tree} bestätigt die Ergebnisse des generalisierten linearen Modells: Für \textit{Truntake} und \textit{Schettose} liegt die Wahrscheinlichkeit für schwache Formen bei 0,85, für \textit{Grettel} hingegen lediglich bei 0,1. Für \textit{Knatt} sind die schwache Form und andere Formen etwa gleich wahrscheinlich. Die Passung des \textit{random forests} auf die Daten ist mit einem C-Wert von 0,84 gut. 

\begin{sloppypar}
Insgesamt zeigt sich das erwartete Antwortverhalten (siehe \sectref{lesehypo}): \textit{Truntake} und \textit{Schettose} weisen vornehmlich schwache Formen auf, \textit{Grettel} vornehmlich starke. Das Testsubstantiv \textit{Knatt} schwankt zwischen stark und schwach und weist auch die meisten Doppelnennungen zwischen starker und schwacher Form auf. Die Studie repliziert damit die Ergebnisse von \textcite{Kopcke.2000b} und \textcite{Schmitt.2019}. Interessant sind die Nullendungen, die sich vor allem bei \textit{Grettel} zeigen. Diese geben einen Hinweis auf das Wirken des Morphemkonstanzprinzips bei unbekannten Substantiven. Bei \textit{Knatt} ist zudem einmal zu beobachten, dass das Testsubstantiv durch die Endung -\textit{er} dem Form-Schema starker Maskulina (Substantive auf -\textit{er} und -\textit{el}) angepasst wird (\cite[167--168]{Schmitt.2019}).
\end{sloppypar}

\subsubsection{Lexical-decision-Studie}\label{ergschemadecant}

In der lexical-decision-Studie bewerteten Proband\_innen schwache und starke Flexionsformen der Testsubstantive nach ihrer Bekanntheit. Die Testsubstantive weisen unterschiedliche Form-Schematizitätsausprägungen auf. Dabei gehören die Substantive der Ausprägung \textsc{Form-Schema} (Testsubstantive: \textit{Kollege}, \textit{Neffe}, \textit{Schütze}, \textit{Franzose}, \textit{Geselle}) und der Ausprägung \textsc{Peripherie} (Testsubstantive: \textit{Graf}, \textit{Held}, \textit{Zar}, \textit{Fürst}, \textit{Nachbar}) der schwachen Flexion an, Substantive der Ausprägung \textsc{stark} (Testsubstantive: \textit{Dieb}, \textit{Freund}, \textit{Vogt}, \textit{Kerl}, \textit{Feind}) gehören offensichtlich der starken Flexion an. Um die Ausprägung \textsc{stark} und die starken Flexionsformen graphisch voneinander zu unterscheiden, wird mit Kapitälchen gearbeitet. 


Das Antwortverhalten ergibt sich aus 280 Antworten, die pro Form-Sche\-ma\-ti\-zi\-täts\-aus\-prä\-gung und Flexionsform gegeben wurden.\footnote{Die 280 Antworten ergeben sich daraus, dass 56 Proband\_innen jeweils fünf Testsubstantive pro Form-Schematizitätsausprägung und Flexionsform bewertet haben.} Abbildung \ref{schemaantalle} zeigt einen Waffleplot mit den Antworten für die einzelnen Form-Sche\-ma\-ti\-zi\-täts\-aus\-prä\-gun\-gen und Flexionsformen. Dunkelgraue Rechtecke zeigen die Antworten, bei denen die Proband\_innen angaben, die Form zu kennen; hellgraue Rechtecke die Antworten, bei denen die Form als unbekannt bewertet wurde. Ein Rechteck steht jeweils für eine Antwort.

\begin{figure}
\includegraphics[width=\textwidth]{figures/Kap6/antwschema.png} 
\caption{Antwortverhalten in der lexical-decision-Studie zu Form-Schematizität}
\label{schemaantalle}
\end{figure}

Für die Ausprägung \textsc{Form-Schema} zeigt sich ein deutlicher Unterschied zwischen starken und schwachen Formen: Die schwachen Formen sind zu 95 \% bekannt, während die starken zu 90 \% unbekannt sind. Das Gegenteil lässt sich für die Ausprägung \textsc{stark} festhalten: Hier sind schwache Formen zu 6,8 \% und starke Formen zu 90,7 \% bekannt. Da die Ausprägungen \textsc{Form-Schema} und \textsc{stark} verschiedenen Deklinationsklassen angehören, ist dieses Ergebnis zu erwarten. Spannend ist ein Blick auf die Ausprägung \textsc{Peripherie}. Die schwachen Formen sind mit 96,4~\% ähnlich bekannt wie die schwachen Formen der Ausprägung \textsc{Form-Schema}. Die starken Formen der Ausprägung \textsc{Peripherie} sind mit 53,9 \%  jedoch deutlich bekannter als die starken Formen der Ausprägung \textsc{Form-Schema}. Die Variation zeigt sich somit nicht in den schwachen Formen der Ausprägung \textsc{Peripherie}, sondern in den starken, die bereits mental gefestigt zu sein scheinen. Ein ähnliches Antwortverhalten wurde in der Studie zu Frequenzeffekten bei den infrequenten Verben mit Schwankung in Korpora beobachtet (siehe \sectref{freqant}).  

Um herauszufinden, ob die Bekanntheit starker und schwacher Formen von den Versuchssubstantiven abhängt, werden diese näher in den Blick genommen. Pro Substantiv und Flexionsform wurden 56 Antworten gegeben, weil 56 Pro\-\mbox{band\_in}\-nen an der Studie teilnahmen. Abbildung \ref{schemaantschema} zeigt die Bewertung für die Substantive der Ausprägung \textsc{Form-Schema} (\textit{Neffe}, \textit{Kollege}).

\begin{figure}
\includegraphics[width=\textwidth]{figures/Kap6/antwortschema.png} 
\caption{Antwortverhalten bei den Testsubstantiven der Ausprägung \textsc{Form-Schema} in der lexical-decision-Studie zu Form-Schematizität}
\label{schemaantschema}
\end{figure}

Die schwachen Formen befinden sich für alle Substantive auf einem ähnlichen Bekanntheitslevel von mindestens 92 \%. Bei den starken Formen zeigen sich leichte Unterschiede: Während \textit{des Franzoses} als völlig unbekannt eingestuft wird und \textit{des Neffes} mit 94,6 \% ebenfalls hohe Unbekanntheitswerte aufweist, liegen die Werte für \textit{des Geselles} und \textit{des Kolleges} niedriger (89,3 \% Unbekanntheit). Mit 76,8 \%  weist \textit{des Schützes} einen deutlich geringeren Anteil an Unbekanntheit auf als die anderen Testsubstantive. Dies könnte eventuell daran liegen, dass \textit{Schütze} auch als Familienname verbreitet ist und ein Sternbild benennt. Zudem existiert das Wort \textit{Schütz}, das auf einen elektromagnetischen Schalter referiert und den Genitiv auf -\textit{es} bildet (\cite{Duden.2020}). Allerdings ergibt die Suche nach \textit{Schützes} im Archiv~W der geschriebenen Sprache nur 210 Belege (DeReKo, \cite{LeibnizInstitutfurDeutscheSprache.2019}). Diese setzen sich vornehmlich aus dem Familiennamen \textit{Schütze} im Genitiv oder im Plural zusammen. Trotz der kleinen Unterschiede zwischen den Testsubstantiven der Ausprägung \textsc{Form-Schema} deutet die visuelle Analyse auf eine systematische Bekanntheit der schwachen und eine systematische Unbekanntheit der starken Formen hin. 

Ein entgegengesetztes Bild mit ähnlicher Eindeutigkeit ergibt sich für die Substantive der Ausprägung~\textsc{stark} (\textit{Freund}, \textit{Dieb}), wie Abbildung \ref{schemaantstark} zeigt.
 
\begin{figure}
\includegraphics[width=\textwidth]{figures/Kap6/antwortstark.png} 
\caption{Antwortverhalten bei den Testsubstantiven der Ausprägung \textsc{stark} in der lexical-decision-Studie zu Form-Schematizität}
\label{schemaantstark}
\end{figure}

Starke Formen sind weitgehend bekannt, die schwachen unbekannt. Allerdings zeigen sich auch hier kleine Unterschiede zwischen den Testsubstantiven. Hierbei fällt vor allem \textit{des Vogts} mit vergleichsweise geringen Bekanntheitswerten auf (71,4 \%). Auch hinsichtlich der schwachen Form fällt \textit{Vogt} auf, diesmal mit etwas höheren Bekanntheitswerten als die restlichen Substantive (14,3 \%). \textit{Vogt} zählt zu den Testsubstantiven mit der geringsten Tokenfrequenz, sodass die Proband\_innen das Substantiv eventuell nicht kannten und daher in der Bewertung unsicher waren.


Auch die Antwortverteilung bei den Substantiven der Ausprägung \textsc{Peripherie} (\textit{Vogt}, \textit{Fürst}) scheint nicht stark von einzelnen Testsubstantiven beeinflusst worden zu sein, wie Abbildung \ref{schemaantperi} zeigt.

\begin{figure}
\includegraphics[width=\textwidth]{figures/Kap6/antwortperipherie.png} 
\caption{Antwortverhalten bei den Testsubstantiven der Ausprägung \textsc{Peripherie} in der lexical-decision-Studie zu Form-Schematizität}
\label{schemaantperi}
\end{figure}

Die schwachen Formen sind für alle Substantive bekannt, nur bei \textit{des Zaren} sind die Bekanntheitswerte mit 91,1 \% etwas geringer als bei den anderen Substantiven. Dafür weist \textit{des Zars} mit 73,2 \% vergleichsweise hohe Bekanntheitswerte auf. \textit{Des Nachbars} und \textit{des Grafs} sind mit 66,1~\% und 62,5~\% ähnlich bekannt. Deutlich weniger bekannt sind \textit{des Helds} (41,1~\%) und \textit{des Fürsts} (26,8 \%). Die Bekanntheit von \textit{des Fürsts} ist damit ähnlich hoch wie die von \textit{des Schützes}, das zum Form-Schema schwacher Maskulina zählt.

 
Die Unterschiede im Antwortverhalten zwischen den Testsubstantiven der Abstufung \textsc{Peripherie} könnten durch Unterschiede in der Sonorität\footnote{Das Prinzip der Sonorität unterscheidet Laute nach ihrer Lautstärke (\cite[107]{Szczepaniak.2010}). Dabei stellen Vokale die sonorsten Laute dar, es folgen Liquide, Nasale und schießlich Frikative und Plosive als unsonore Laute (\cite[167]{Vennemann.1987}).} der Stammauslaute bedingt sein: Die Formen \textit{Helds} und \textit{Fürsts} weisen aufgrund des Plosivs als Stammauslaut ein extrasilbisches \textit{s} auf (\cite[114]{Szczepaniak.2010}). Genitivformen auf -\textit{s} sind bei Substantiven, deren Stamm auf Plosiv endet, daher generell selten. Substantive auf Plosiv tendieren stattdessen zu langen Genitivfor\-\mbox{men~(-\textit{es}}): Lange Genitivformen machen bei Simplizia auf Plosiv (\textit{des Kruges}) 79 \%  der Genitivformen aus und bei Simplizia auf -\textit{st} (\textit{des Dienstes}) sogar 89~\%  (\cite[113]{Szczepaniak.2010}). Bei Frikativen im Auslaut ist die relative Frequenz der langen Genitivform hingegen deutlich geringer (66 \%).

 
Die vergleichsweise geringe Frequenz von langen Genitivformen bei Frikativen im Stammauslaut könnte den höheren Bekanntheitswert von \textit{des Grafs} im Vergleich zu \textit{des Helds} und \textit{des Fürsts} erklären: Die Beschränkungen in der Phonotaktik blockieren die kurze Endung hier nicht so stark wie bei den Plosiven. Die lange Form -\textit{es} scheint für die ursprünglichen schwachen Maskulina dabei keine Option zu sein (*\textit{des Heldes}, *\textit{des Fürstes}).  Bei \textit{Zar} und \textit{Nachbar} stellt die kurze Form aufgrund des vokalischen Auslauts keine phonotaktischen Probleme dar, was sich ebenfalls in der hohen Bekanntheit spiegelt. Insgesamt lässt sich daher festhalten, dass sich die globale Verteilung der Antworten innerhalb der einzelnen Testsubstantive widerspiegelt. Die wenigen Unterschiede im Antwortverhalten zwischen den Substantiven der Ausprägung \textsc{Peripherie} lassen sich durch den Einfluss von Phonotaktik erklären.

Für die konfirmatorische Analyse des Antwortverhaltens wird ein generalisiertes gemischtes lineares Modell genutzt, das Form-Schematizität und Flexion als feste Effekte enthält, die interagieren: Antwortverhalten \~{} Form-Schematizität * Flexion + (1|Proband\_in) + (1|Item). Als zufällige Effekte werden \textit{random intercepts} für Proband\_in und Versuch\-item (Lemma) berücksichtigt. \textit{Random slopes} werden aufgrund von Konvergierungsproblemen und \textit{singular fits} für Pro\-\mbox{band\_in}\-nen nicht genutzt, für Versuchitems sind \textit{random slopes} nicht sinnvoll, da für die Form-Schematizitätsausprägungen verschiedene Items genutzt werden. Die Proband\_innen schwanken mit einer Standardabweichung von 0,71 um den y-Achsenabschnitt, die Versuchitems variieren geringer mit einer Standardabweichung von 0,51. In Tabelle \ref{schemadecerg} sind die \textit{log odds} des Modells für die Unbekanntheit starker und schwacher Formen aufgelistet, die entsprechende Kreuztabelle (Tabelle \ref{schemadecergkreuz}) der \textit{log odds} befindet sich in Anhang~\ref{ergebnisseschema}. Abbildung~\ref{schemaantpred} zeigt die auf den \textit{log odds} basierende Wahrscheinlichkeit für die Unbekanntheit starker und schwacher Formen pro Form-Schematizitätsausprägung. Kreise stehen für schwache Formen, Dreiecke für starke.

\begin{table}
\begin{tabular10}{l S[table-format=-2.2] S[table-format=1.2] S[table-format=-2.2] S[table-format=<1.2]}
\lsptoprule
& {Wert} & {SE} & {$z$} & {$p$} \\\midrule
(Intercept = Form-Schema \& schwach) & -3,28 & 0,39 & -8,47 & < 0,01 \\ 
Form-Schematizität: Peripherie & -0,38 & 0,54 & -0,69 & 0,49 \\ 
Form-Schematizität: stark & 6,19 & 0,52 & 11,94 & < 0,01 \\ 
Flexion: stark & 5,77 & 0,39 & 14,76 & < 0,01 \\ 
Form-Schematizität: Peripherie \& Flexion: stark & -2,29 & 0,51 & -4,46 & < 0,01 \\ 
Form-Schematizität: stark \& Flexion: stark & -11,23 & 0,56 & -19,94 & < 0,01 \\  
\lspbottomrule
\end{tabular10} 
\caption{Werte des Modells für die Unbekanntheit starker und schwacher Formen in der lexical-decision-Studie zu Form-Schematizität}
\label{schemadecerg} 
\end{table}

\begin{figure}
\includegraphics[width=\textwidth]{figures/Kap6/predantschemadec.png} 
\caption{Wahrscheinlichkeit für die Unbekanntheit starker und schwacher Formen in der lexical-decision-Studie zu Form-Schema\-ti\-zi\-tät}
\label{schemaantpred}
\end{figure}

Bei Substantiven der Ausprägung \textsc{Form-Schema} liegt die Wahrscheinlichkeit für Unbekanntheit für starke Formen bei 0,92 und für schwache Formen bei 0,04. Starke Formen sind also mit großer Wahrscheinlichkeit unbekannt und schwache bekannt. Das Gegenteil ist für die Ausprägung \textsc{stark} der Fall: Starke Formen haben eine Wahrscheinlichkeit für Unbekanntheit von 0,07, schwache von 0,95. Die 95 \%-Konfidenzintervalle sind recht klein und überlappen sich nicht. Bei der Ausprägung \textsc{Peripherie} zeigt sich eine Wahrscheinlichkeit von 0,03 für schwache Formen, die Wahrscheinlichkeit für Unbekanntheit ist also ungefähr so gering wie für schwache Formen der Ausprägung \textsc{Form-Schema}. Bei starken Formen der Ausprägung \textsc{Peripherie} liegt die Wahrscheinlichkeit bei 0,46, Bekanntheit und Unbekanntheit sind für starke Formen also in etwa gleich wahrscheinlich. Die starken Formen der Ausprägung \textsc{Peripherie} haben auch ein langes 95~\%-Konfidenzintervall, das jedoch nicht mit anderen 95~\%-Konfidenzintervallen überlappt. Bis auf den $p$-Wert für schwache Formen der Ausprägung \textsc{Peripherie} sind alle $p$-Werte für alle Formen und Ausprägungen im Vergleich zum Referenzlevel \textsc{Form-Schema} mit schwachen Formen jeweils unter dem $\alpha$-Level von 0,01. Es ist daher von systematischen Effekten auszugehen. Der ausbleibende Unterschied in der Bewertung von schwachen Formen der Ausprägungen \textsc{Form-Schema} und \textsc{Peripherie} war bereits in der visuellen Analyse zu erkennen. Die Effektstärke ist mit einem $R^2m$ von 0,63 (theoretisch) bzw. 0,59 (delta) recht hoch. Unter Einbezug der zufälligen Effekte erhöht sie sich mit einem $R^2c$ von 0,7 (theoretisch) und 0,65 (delta) leicht.

Die Analyse des Antwortverhaltens in der lexical-decision-Studie zeigt, dass starke und schwache Formen für starke und schwache Maskulina gegenteilige Bekanntheitsgrade aufweisen. Das ist aufgrund der Flexionklassenzugehörigkeit auch zu erwarten. Schwache Substantive in der Peripherie des Form-Schemas weisen hohe Bekanntheitsgrade für schwache Formen auf, bei starken Formen halten sich Bekanntheit und Unbekanntheit die Waage. Diese Verteilung lässt sich für alle Testsubstantive erkennen. Die Hypothesen aus \sectref{hyposchema} werden somit bestätigt.


\subsubsection{Sentence-maze-Studie}

In der sentence-maze-Studie wählten die Proband\_innen zwischen starken -\textit{(e)s}  und schwachen -\textit{(e)n} Genitivformen der Testsubstantive. Dabei wurden Testsubstantive der Ausprägungen \textsc{Form-Schema} (Testsubstantive: \textit{Kollege}, \textit{Neffe}, \textit{Schütze}), \textsc{Peripherie} (Testsubstantive: \textit{Graf}, \textit{Held}, \textit{Zar}) und \textsc{stark} (Testsubstantive: \textit{Dieb}, \textit{Freund}, \textit{Vogt}) präsentiert. Pro Ausprägung liegen 396 Antworten vor.\footnote{Diese ergeben sich daraus, dass 132 Proband\_innen Flexionsformen für jeweils drei Testsubstantive wählten.}  Abbildung \ref{schemasentwaf} zeigt das Antwortverhalten pro Form-Schema\-ti\-zi\-täts\-aus\-prä\-gung in einem Waffleplot. Rechtecke in dunkelgrau stehen für die Wahl der schwachen Form, Rechtecke in hellgrau für die Wahl der starken. Ein Rechteck steht für zwei Antworten.


\begin{figure}
\includegraphics[width= 0.68 \textwidth]{figures/Kap6/antwsentschema.png} 
\caption{Antwortverhalten in der sentence-maze-Studie zu Form-Schematizität}
\label{schemasentwaf}
\end{figure}


Für Substantive der Ausprägung \textsc{Form-Schema} wählten die Proband\_innen mit 94,2~\% vorrangig schwache Formen, für Substantive der Ausprägung \textsc{stark} mit 96,2 \% vorrangig starke. Interessant ist ein Blick auf die Peripherie: Auch hier herrschen mit 91,7~\% schwache Formen vor. Es zeigt sich somit ein klar verteiltes Antwortverhalten. Die generelle Tendenz zu schwachen Formen in der Peripherie mag zunächst überraschen, da die starken Formen in der lexical-decision-Studie immerhin in ungefähr der Hälfte der Antworten als bekannt angesehen wurden. Diese Diskrepanz ist durch das Versuchsdesign bedingt: Anders als in der lexical-decision-Studie werden die Formen in der sentence-maze-Studie nicht einzeln bewertet, sondern beide stehen gleichzeitig zur Auswahl. Die mental gefestigte schwache Form kann die weniger gefestigte starke Form so leicht statistisch ausstechen (\cite[74--94]{Goldberg.2019}, zum statistischen Vorkaufsrecht siehe \sectref{Statistik}).    

 

Die globale Verteilung der Antworten zeigt sich auch im Antwortverhalten für die einzelnen Testsubstantive: Die starke Endung wurde bei den Substantiven der Ausprägung \textsc{Form-Schema} maximal zu 6,8 \% gewählt und die schwache Endung für Substantive der Ausprägung \textsc{stark} maximal zu 6,1 \%, siehe die Abbildungen \ref{antwortschemasentm} und \ref{antwortstarksentm} in Anhang \ref{ergebnisseschema}. Nur bei der Ausprägung \textsc{Peripherie} zeigen sich leichte Unterschiede zwischen den Testsubstantiven, wie Abbildung \ref{schemasentwafper} zeigt. Pro Substantiv wurden 132 Antworten gegeben. Ein Rechteck steht für zwei Antworten.  

\begin{figure}
\includegraphics[width= 0.68 \textwidth]{figures/Kap6/antwortperipheriesent.png} 
\caption{Antwortverhalten bei den Testsubstantiven der Ausprägung \textsc{Peripherie} in der sentence-maze-Studie zu Form-Schematizität}
\label{schemasentwafper}
\end{figure}

Das Testsubstantiv \textit{Held} weist mit 1,5 \% deutlich weniger Antworten für starke Formen auf als die anderen Testsubstantive, bei denen die Antworten für starke Formen jeweils um die 10 \% ausmachen. In der lexical-decision-Studie wies \textit{Held} im Vergleich zu \textit{Graf} und \textit{Zar} ebenfalls weniger Zustimmung für starke Formen auf (41,1~\% Zustimmung für \textit{Held}, 62,5~\% für \textit{Graf} und 73,2~\% für \textit{Zar}). Dies könnte wiederum an den Unterschieden in der Sonorität des Stammauslauts liegen: Da \textit{Held} auf einen Plosiv auslautet, ist die kurze Genitivform -\textit{s} phonotaktisch ungünstig, da sie zu einem extrasilbischen Element führt (\cite[114]{Szczepaniak.2010}). Die Kombination aus -\textit{s} und einem Frikativ (\textit{Grafs}) oder vokalischem Auslaut (\textit{Zars}) ist hingegen phonotaktisch gesehen weniger bzw. überhaupt nicht problematisch (\cite{Szczepaniak.2010}, siehe hierzu genauer \sectref{ergschemadecant}). Unabhängig von den kleinen Unterschieden zwischen den Testsubstantiven der Ausprägung \textsc{Peripherie} ist festzuhalten, dass die schwachen Formen in der direkten Wahl zwischen stark und schwach bevorzugt werden. 

Für die konfirmatorische statistische Analyse wird ein generalisiertes gemischtes lineares Modell mit Form-Schematizitätsausprägung als festem Effekt gerechnet: Antwortverhalten \~{} Form-Sche\-ma\-ti\-zi\-tät + (1|Proband\_in) + (1|Item). Als zufällige Effekte sind \textit{random intercepts} für Proband\_in und Versuchitem (Lemma) vorgesehen. \textit{Random slopes} werden für Proband\_innen aufgrund von \textit{singular fits} nicht genutzt, für Versuchitems sind sie in diesem Modell nicht sinnvoll, da verschiedene Items für die Form-Schematizitätsausprägungen genutzt werden.


Die Proband\_innen variieren mit einer Standardabweichung von 0,48 um den y-Ach\-sen\-ab\-schnitt, die Versuchitems etwas weniger mit einer Standardabweichung von 0,37. Die \textit{log odds} für die Wahl starker Formen sind in Tabelle \ref{ergsentmazeschema} aufgelistet. Abbildung \ref{predschemasent} zeigt die auf den \textit{log odds} basierende Wahrscheinlichkeit für die Wahl starker Formen.

\begin{table}
\begin{tabular}{l S[table-format=-1.2] S[table-format=1.2] S[table-format=-1.2] S[table-format=<1.2]}
\lsptoprule
& {Wert} & {SE} & {$z$} & {$p$} \\\midrule
(Intercept = Form-Schema) & -2,92 & 0,34 & -8,67 & < 0,01 \\ 
Form-Schematizität: Peripherie & 0,34 & 0,42 & 0,82 & 0,42 \\ 
Form-Schematizität: stark & 6,31 & 0,54 & 11,75 & < 0,01 \\  
\lspbottomrule
\end{tabular}
\caption{Werte des Modells für die Wahl starker Formen in der sentence-maze-Studie zu Form-Schematizität}
\label{ergsentmazeschema}
\end{table}


\begin{figure}
\includegraphics[width= 0.8 \textwidth]{figures/Kap6/predantschemasent.png} 
\caption{Wahrscheinlichkeit für die Wahl starker Formen in der sentence-maze-Studie zu Form-Schema\-ti\-zi\-tät}
\label{predschemasent}
\end{figure}

Für Substantive der Ausprägungen \textsc{Form-Schema} und \textsc{Peripherie} ist die Wahl starker Formen mit Werten von 0,05 und 0,07 sehr unwahrscheinlich, das Gegenteil ist für Substantive der Ausprägung \textsc{stark} der Fall, bei denen die Wahrscheinlichkeit bei 0,97 liegt. Die 95~\%-Konfidenzintervalle sind relativ klein und zeigen nur für die Ausprägungen \textsc{Form-Schema} und \textsc{Peripherie} Überlappungen. Die $p$-Werte unterschreiten nur für die Ausprägung \textsc{stark} im Vergleich zum Referenzlevel \textsc{Form-Schema} das $\alpha$-Level von 0,01. Für die Ausprägungen \textsc{Form-Schema} und \textsc{Peripherie} ist dagegen nicht von einem systematischen Unterschied im Antwortverhalten auszugehen.  Die Effektstärke des Modells ist mit einem $R^2m$ von 0,7 (theoretisch) und 0,66 (delta) hoch. Mit den zufälligen Effekten erhöht sie sich mit einem $R^2c$ von 0,73 (theoretisch) und 0,69 (delta) nur leicht. Die Effektstärke des Modells ist vergleichbar mit der Effektstärke des Modells für die lexical-decision-Studie.

Insgesamt zeigt sich im Antwortverhalten eine klare Präferenz für schwache Formen bei den schwachen Maskulina unabhängig davon, ob sie zum Prototyp oder zur Peripherie des Form-Schemas gehören. Für starke Substantive zeigt sich erwartbarerweise eine klare Präferenz für starke Formen. Die Verteilung bestätigt somit die Hypothesen in \sectref{hyposchema}. Die Bevorzugung für die schwachen Formen in der Peripherie des Form-Schemas lässt sich durch das Versuchsdesign erklären, in dem schwache und starke Formen nebeneinander präsentiert werden. Daher kann die mental gefestigte Form (schwach) die weniger gefestigte (stark) leichter statistisch ausstechen.

\subsubsection{Zusammenfassung}

In allen Studien ist der Einfluss der Form-Schematizität auf die Wahl der Flexionsendung zu erkennen. Die Unterschiede im Antwortverhalten zwischen den Studien lassen sich jeweils durch das Versuchsdesign erklären: Im Produktionsexperiment mussten die Proband\_innen die Formen selbst bilden, in der lexical-decision-Studie wurden starke und schwache Formen separat beurteilt und in der sentence-maze-Studie jeweils zwischen der starken und schwachen Form eines Testsubstantivs gewählt.


Der Einfluss der Form-Schematizität zeigt sich am deutlichsten im Produktionsexperiment zu der self-paced-reading-Studie und in der lexical-decision-Stu\-die. Wenn ein Substantiv prototypischer Vertreter des Form-Schemas schwacher Maskulina ist, werden schwache Formen im Produktionsexperiment gebildet, während starke Formen gemieden werden. Dasselbe zeigt sich in der lexical-decision-Studie: Schwache Formen prototypischer Vertreter des Form-Sche\-mas werden als bekannt und starke Formen als unbekannt bewertet. Bei starken Maskulina ist das Gegenteil der Fall: Hier werden starke Formen im Produktionsexperiment gebildet und schwache gemieden. Genauso werden in der lexical-decision-Studie starke Formen von starken Maskulina als bekannt bewertet und schwache als unbekannt. Die Substantive in der Peripherie des Form-Schemas weisen im Produktionsexperiment und in der lexical-decision-Studie Schwankungen auf: Im Produktionsexperiment wird die schwache Form -\textit{en} zu 55~\% gewählt und die starke Form -\textit{(e)s} zu 30 \%. Zudem werden Doppelformen genannt. Auch in der lexical-decision-Studie ist in Hinblick auf starke Formen Variation zu erkennen: Nur  50~\% der starken Formen von Substantiven in der Peripherie des Form-Schemas werden als bekannt bewertet, bei den schwachen liegt der Bekanntheitsgrad hingegen bei ca. 90~\%. Der Einfluss des Form-Schemas schwacher Maskulina lässt sich somit sowohl in der Produktion als auch in der Bewertung von Formen nach Bekanntheit beobachten und wirkt bei Pseudosubstantiven genauso wie bei real existierenden. Die Einblicke in die Sprachproduktion zeigen zudem, dass bei Pseudosubstantiven weitere Formen (Nullendungen, Formen auf -\textit{ers}) möglich sind.

 

Die Ergebnisse der sentence-maze-Studie weichen etwas von den Ergebnissen der anderen Studien ab. Zwar werden in der sentence-maze-Studie ebenfalls schwache und starke Formen je nach Flexionsklasse gewählt, jedoch ist in der Peripherie des Form-Schemas keine Variation zu beobachten. Der Einfluss des Form-Schemas ist daher in der sentence-maze-Studie nur insofern zu erkennen, als dass die Mitglieder des Form-Schemas schwacher Maskulina als solche erkannt werden, unabhängig davon, ob sie prototypische oder periphere Vertreter darstellen. Das Ausbleiben der Schwankungen in der Peripherie geht auf das Studiendesign zurück: Die schwache und die starke Form wurden direkt nebeneinander als Antwortmöglichkeiten präsentiert, sodass die stärker gefestigte schwache Form die weniger gefestigte starke leichter statistisch ausstechen konnte. Diese These stützen die anderen Studien, in denen Schwankungen im Antwortverhalten für die Testsubstantive in der Peripherie des Form-Schemas zu beobachten sind. Die \textit{sentence maze task} scheint sich daher nur bedingt zu eignen, um Variation zu untersuchen.

Insgesamt bestätigt die Analyse des Antwortverhaltens den Einfluss der Form-Schematizität auf die Deklination von Maskulina. Unabhängig von dem Un\-ter\-su\-chungs\-de\-sign werden starke und schwache Formen je nach Flexionsklassenzugehörigkeit bevorzugt. Im Produktionsex\-periment zu der self-paced-reading-Studie und in der lexical-decision-Studie lässt sich zudem Variation in der Peripherie des Form-Schemas schwacher Maskulina beobachten. Generell ist dabei festzuhalten, dass die schwachen Formen auch in der Peripherie des Form-Schemas schwacher Maskulina stark gefestigt sind, da sie hohe Bekanntheitswerte in der lexical-decision-Studie aufweisen und die starken Formen in der sentence-maze-Studie ausstechen. Umgekehrt scheinen die starken Formen von Substantiven in der Peripherie des Form-Schemas nicht so stark gefestigt zu sein, da sie in der lexical-decision-Studie nur zur Hälfte als bekannt bewertet werden. Das Produktionsexperiment zeigt jedoch, dass Proband\_innen in der Flexion von Pseudosubstantiven aus der Peripherie des Form-Schemas zwischen starken und schwachen Formen schwanken. Dies verdeutlicht, dass die starken Formen von Substantiven in der Peripherie des Form-Schemas durchaus die schwachen in der Sprachproduktion statistisch ausstechen können. Es ist jedoch möglich, dass dies bei real existierenden Substantiven in der Peripherie des Form-Schemas aufgrund der Tokenfrequenzunterschiede zwischen schwachen und starken Formen unwahrscheinlicher ist.

\subsection{Reaktionszeiten}

In diesem Abschnitt werden die Reaktionszeiten der Form-Schematizitätsstudien analysiert und die Ergebnisse anschließend miteinander verglichen. Dabei ist zu bedenken, dass die Reaktionszeiten in den Studien nicht direkt vergleichbar sind, da sie Lesezeiten (\textit{self-paced reading}) sowie kontextlose (\textit{lexical decision}) und kontextbedingte (\textit{sentence maze}) Reaktionszeiten  widerspiegeln.

 
Für die einzelnen Studien werden verschiedene Reaktionszeitmuster erwartet: Für die self-paced-reading-Studie wird angenommen, dass schwache Formen bei Testsubstantiven, die dem Prototyp des Form-Schemas entsprechen, schnell gelesen werden, starke hingegen langsam. Das Gegenteil wird für starke Testsubstantive erwartet. Für die Formen in der Peripherie des Form-Schemas wird kein Unterschied in den Lesezeiten antizipiert (siehe hierzu genauer \sectref{methschema}). Für die lexical-decision- und die sentence-maze-Studie wurden ursprünglich geringe Reaktionszeiten für den Prototyp des Form-Schemas und starke Maskulina erwartet und höhere für die Peripherie. In der Peripherie des Form-Schemas sind schwache und starke Formen ähnlich wahrscheinlich, weshalb die Beurteilung der Formen verlangsamt sein könnte. Bei den anderen Ausprägungen kann eine Form die andere hingegen klar statistisch ausstechen. Dies wurde jedoch in der Prästudie nicht bestätigt, sodass die Hypothesen angepasst wurden (siehe hierzu ausführlich \sectref{schemalex}). Für die lexical-decision-Studie wird auf Basis der Prästudie eine Interaktion zwischen Form-Schematizitätsausprägung und Flexionsform erwartet: Schwache Formen werden für schwache Substantive (sowohl bei prototypischen als auch bei peripheren Vertretern des Form-Schemas) schnell bewertet und starke langsam, das Gegenteil ist für starke Substantive der Fall. In der sentence-maze-Studie wird auf Basis der Prästudie angenommen, dass die Wahl zwischen starken und schwachen Formen bei Substantiven in der Peripherie des Form-Schemas schneller abläuft als bei den Substantiven der anderen Form-Schematizitätsausprägungen (siehe hierzu \sectref{schemalex}). 

\subsubsection{Self-paced-reading-Studie}\largerpage[-1]

In diesem Abschnitt werden die Lesezeiten innerhalb der self-paced-reading-Stu\-die analysiert. Abbildung \ref{leslogRT} zeigt die logarithmierten Lesezeiten in Beeswarmplots für die einzelnen Testsubstantive samt arithmetischem Mittel und Standardabweichung der starken und schwachen Formen. Das arithmetische Mittel (M), die Standardabweichung (SD) und der Standardfehler (SE) sind zudem gesondert aufgeführt. Dunkelgraue Kreise zeigen Lesezeiten für schwache Formen, hellgraue Dreiecke Lesezeiten für starke Formen. 

\begin{figure}
\includegraphics[width= 1 \textwidth]{figures/Kap6/logles.png} 
\caption{Lesezeiten in der self-paced-reading-Studie}
\label{leslogRT}
\end{figure}

Die Testsubstantive weisen vergleichbare Lesezeiten auf, wobei \textit{Knatt} und \textit{Grettel} vermutlich aufgrund ihrer Kürze etwas geringere Lesezeiten evozieren. Starke und schwache Formen sind pro Testsubstantiv recht gleichmäßig auf die Lesezeiten verteilt. Bei \textit{Grettel} fällt ein Datenpunkt mit deutlich geringer Lesezeit als die anderen Datenpunkte ins Auge. Dieser weicht jedoch nicht mehr als drei Standardabweichungen vom Mittelwert ab und wird daher nicht als extremer Wert betrachtet (siehe \sectref{selfpacedprob}). Abseits davon sind die Lesezeiten für starke und schwache Formen ähnlich. Nur bei \textit{Schettose} weist die starke Form höhere Lesezeiten auf als die schwache. Dies zeigt sich in der Verteilung der Lesezeiten und in den Mittelwerten: Die Lesezeiten für die starke Form von \textit{Schettose} haben einen Mittelwert von 0,3 mit einer Standardabweichung von 0,4, während der Mittelwert für die schwache Form mit 0,12 und einer Standardabweichung von 0,36 deutlich niedriger liegt. Die anderen Testsubstantive weisen maximale Differenzen von 0,07 zwischen den Mittelwerten auf.

 
Die visuelle Analyse der direkt nach den Testitems gelesenen Elemente lässt keine spill-over-Effekte erkennen,\footnote{Die Elemente nach den Testitems wurden jeweils konstant gehalten, auf \textit{des Schettoses} bzw. \textit{des Schettosen} folgte also dasselbe Wort bzw. dieselbe Wortgruppe (siehe \sectref{versuchspr}). Unterschiede in den Lesezeiten zwischen den Elementen nach den Testitems würden damit auf spill-over-Effekte hindeuten.} auch hinsichtlich der Fragen zu den Test\-items (Haben Sie \textit{des Schettoses} oder \textit{des Schettosen} gelesen?) ergab sich kein Unterschied in den Reaktionszeiten (siehe Abbildungen \ref{loglesnach} und \ref{loglesfrage} in Anhang \ref{ergebnisseschema}). Zudem weisen die Daten nicht auf Reihenfolgeeffekte hin (siehe Abbildung \ref{loglescondition} in Anhang \ref{ergebnisseschema}). Längeneffekte können für \textit{des Knatten} im Vergleich zu dem um einen Buchstaben kürzeren starken Genitiv  (\textit{des Knatts}) ausgeschlossen werden (siehe Abbildung \ref{knattles} in Anhang \ref{ergebnisseschema}), für \textit{Truntake} ist es hingegen möglich, dass die im Dativ kürzere starke Form (\textit{dem Truntake}) vereinzelt kürzere Lesezeiten hervorruft als die schwache (siehe Abbildung \ref{truntakeles} in Anhang \ref{ergebnisseschema}). Bei den anderen Testsubstantiven sind starke und schwache Formen gleich lang.\largerpage[-2]

Um den Unterschied in den logarithmierten Lesezeiten zwischen starken und schwachen Formen näher bestimmen zu können, werden Scatterplots der Test\-items betrachtet, siehe  Abbildung \ref{scatterleslogRT}. Ein Punkt steht für die Lesezeit einer Person für die starke bzw. schwache Form. Die graue Linie markiert den Bereich, in dem die starke und die schwache Form dieselben Lesezeiten hervorrufen würden.

\begin{figure}
\includegraphics[width=\textwidth]{figures/Kap6/patchles.png} 
\caption{Scatterplots der Lesezeiten für starke und schwache Formen der Testsubstantive}
\label{scatterleslogRT}
\end{figure}

Bei \textit{Schettose} scheinen die Lesezeiten für die starke Endung etwas erhöht zu sein, wie die Punkte oberhalb der grauen Linie zeigen. Bei \textit{Truntake} ist hingegen kein Effekt der schwachen und starken Endung zu sehen, die Punkte verteilen sich regelmäßig um die graue Linie. Der ausbleibende Effekt bei \textit{Truntake} könnte der unterschiedlichen Länge der schwachen (\textit{dem Truntaken}) und starken (\textit{dem Truntake}) Form geschuldet sein, wie oben bereits erwähnt wurde.  Bei \textit{Knatt} und \textit{Grettel} scheinen schwache Endungen etwas erhöhte Lesezeiten zu evozieren als starke, wie die Punkte unterhalb der grauen Linie zeigen. Dieser Trend ist jedoch weniger ausgeprägt als der Trend zu höheren Lesezeiten für die starke Form bei \textit{Schettose}. 

Für die statistische Analyse der Reaktionszeiten wird ein gemischtes lineares Modell mit Form-Schematizitäts\-aus\-prägung und Flexionsform als feste Effekte gerechnet, die interagieren: Lesezeit \~{} Form-Schematizität * Flexion + (1|Proband\_in). Die Form-Schematizitäts\-aus\-prä\-gungen werden jeweils durch die Test\-items repräsentiert. Teilnehmer\_innen werden als zufällige Effekte mit \textit{random intercepts} berücksichtigt, \textit{random slopes} sind nicht möglich, da pro Ausprägung nur ein Testitem genutzt wird. Die Standardabweichung der Proband\_innen um den y-Achsenabschnitt beträgt 0,23 mit einer restlichen Residuenvarianz von 0,32. Die Struktur der zufälligen Effekte ist damit vergleichbar mit der der zufälligen Effekte in den Modellen zu Reaktionszeiten in der Frequenz- und Prototypizitätsstudie. Die geschätzten Werte des Modells sind in Tabelle \ref{ergself} aufgeschlüsselt, eine entsprechende Kreuztabelle (Tabelle~\ref{ergselfkreuz}) befindet sich in Anhang \ref{ergebnisseschema}. Abbildung \ref{schemalespredrt} zeigt die vom Modell vorhergesagten Lesezeiten. Kreise stellen schwache Formen dar, Dreiecke starke.

\begin{table}
\begin{tabular}{l S[table-format=-1.2] S[table-format=1.2] S[table-format=-1.2] S[table-format=3.2] S[table-format=<1.2]}
\lsptoprule
& {Wert} & {SE} & {$t$} & {df} & {$p$} \\\midrule
(Intercept = Schettose \& schwach) & 0,12 & 0,05 & 2,24 & 236,10 & 0,03 \\ 
Endung: stark & 0,18 & 0,06 & 2,85 & 378,00 & < 0,01 \\  
Item: Grettel & -0,11 & 0,06 & -1,78 & 378,00 & 0,08 \\  
Item: Knatt & -0,08 & 0,06 & -1,31 & 378,00 & 0,19 \\ 
Item: Truntake  & 0,17 & 0,06 & 2,67 & 378,00 & < 0,01 \\  
Endung: stark \& Item: Grettel & -0,22 & 0,09 & -2,53 & 378,00 & 0,01 \\  
Endung: stark \& Item: Knatt  & -0,25 & 0,09 & -2,86 & 378,00 & < 0,01 \\  
Endung: stark \& Item: Truntake & -0,19 & 0,09 & -2,13 & 378,00 & 0,03 \\  
\lspbottomrule
\end{tabular}
\caption{Werte des Modells für die Lesezeiten in der self-paced-reading-Studie}
\label{ergself}
\end{table}

\begin{figure}
\includegraphics[width=\textwidth]{figures/Kap6/loglespred.png} 
\caption{Vorhergesagte Lesezeiten in der self-paced-reading-Studie}
\label{schemalespredrt}
\end{figure}

Für \textit{Schettose} ist ein Unterschied zwischen starker und schwacher Form zu erkennen, während für die anderen Testsubstantive starke und schwache Formen nah beieinander liegen: Bei \textit{Truntake} lässt sich kein Unterschied zwischen der starken und schwachen Form feststellen. Bei \textit{Knatt} und \textit{Grettel} tendieren starke Formen dazu, etwas schneller gelesen zu werden als schwache, die 95~\%-Konfidenzintervalle überlappen sich jedoch deutlich. Es ist somit nicht davon auszugehen, dass die Unterschiede systematisch sind. Auch bei \textit{Schettose} ist eine Überlappung der 95~\%-Konfidenzintervalle festzustellen, jedoch liegen die Mittelwerte jeweils außerhalb der Überlappung. Dennoch ist es möglich, dass die Unterschiede nicht systematisch sind. Interessant ist der Unterschied zwischen \textit{Schettose} und \textit{Truntake}: Während die schwache Form von \textit{Schettose} bei 0,12 liegt und die starke bei 0,3, ist bei \textit{Truntake} sowohl die schwache als auch die starke Form nah bei 0,3.   


Die $p$-Werte sind für dieses Modell nur bedingt interessant, da sie immer im Vergleich zum Referenzlevel \textit{Schettose} mit schwacher Form stehen und daher keine Aussage über die Unterschiede zwischen starken und schwachen Formen der anderen Ausprägungen machen. Die Lesezeiten für die starke Form von \textit{Schettose} liegen im Vergleich zum Referenzlevel unter dem $\alpha$-Level von 0,01, was auf einen sys\-te\-ma\-tisch\-en Effekt für die starke und die schwache Form von \textit{Schettose} hindeutet. Da sich die Konfidenzintervalle jedoch überlappen, ist ein Zufallsbefund dennoch möglich. 


Auch \textit{Truntake} mit schwacher Form und \textit{Knatt} mit starker Form unterschreiten das $\alpha$-Level. Bei \textit{Grettel} mit starker Form wird das $\alpha$-Level hingegen knapp verfehlt. Die Konfidenzintervalle für die schwache Form von \textit{Schettose} und die starken Formen von \textit{Knatt} und \textit{Grettel} überlappen sich jeweils nicht. Zumindest für \textit{Knatt} kann somit von systematischen Unterschieden ausgegangen werden, bei \textit{Grettel} kann der Unterschied auch zufällig sein. Es ist also möglich, dass die schwache Form von \textit{Schettose} systematisch langsamer gelesen wird als die starke Form von \textit{Knatt}. Dies ließe sich als ein genereller Vorteil der starken Formen bei Substantiven deuten, die die starke Flexion nicht qua Form-Schema komplett ausschließen. Allerdings ist es in diesem Zusammenhang verwunderlich, dass die starke Form von \textit{Grettel} keinen systematischen Unterschied zu der schwachen von \textit{Schettose} aufweist. 


Hinsichtlich möglicher Lesezeitunterschiede zwischen den starken und schwachen Formen der Test\-items deutet das Modell lediglich für \textit{Schettose} auf einen sys\-te\-ma\-tisch\-en Effekt hin. Aufgrund der Überlappung der 95~\%-Konfidenzintervalle ist ein Zufallsbefund aber nicht auszuschließen. Wenig überraschend ist die Effektstärke des Modells mit einem $R^2m$ von 0,11 klein, erst unter Einbezug der zufälligen Effekte ist sie mit einem $R^2c$ von 0,41 höher.  

Die Lesezeiten scheinen kaum von den Form-Schematizitätsausprägungen beeinflusst worden zu sein. Dies steht diametral zu den Hypothesen (siehe \sectref{lesehypo}) und dem Antwortverhalten im Produktionsexperiment, das eindeutig von Form-Schematizität beeinflusst ist. Nur bei starken und schwachen Formen von \textit{Schettose} lässt sich ein Effekt erahnen, der jedoch nicht groß ist und aufgrund der Überlappung der 95~\%-Konfidenzintervalle auch nicht zwingend sys\-te\-ma\-tisch ist. Auch in der Studie von \textcite[169--171]{Schmitt.2019} waren Form-Schematizitätseffekte in den Lesezeiten kaum zu greifen: Hier wurde ebenfalls mit den Testsubstantiven \textit{Schettose}, \textit{Knatt} und \textit{Grettel} gearbeitet. Nur für \textit{Grettel} waren Unterschiede in der Lesezeit zwischen starker und schwacher Form festzustellen, für \textit{Schettose} hingegen nicht. Um zuverlässige Aussagen über den Einfluss der Form-Schematizität auf Lesezeiten treffen zu können, müsste eine weitere Studie mit mehr Versuchitems und teststärkenbasierter Stichprobengröße durchgeführt werden. Sollten die Unterschiede zwischen starken und schwachen Formen bei \textit{Schettose} und \textit{Grettel} systematisch sein und der ausbleibende Unterschied bei \textit{Grettel} in dieser Studie einen $\beta$-Fehler darstellen, würde dies auf einen Einfluss der Form-Schematizität hinweisen: Die jeweils unwahrscheinlichen Formen würden dann Probleme in der Prozessierung auslösen, die sich in erhöhten Lesezeiten niederschlagen würden. Nur für die Peripherie wäre dies nicht zu beobachten, da hier beide Formen ähnlich wahrscheinlich sind. In diesem Fall wäre ein genauerer Blick auf den anscheinend ausbleibenden Effekt bei \textit{Truntake} interessant: Da die schwache Form im Dativ um einen Buchstaben länger ist als die starke, könnte dies die Lesezeit der starken Form verkürzen und so einen möglichen Form-Schematizitätseffekt überdecken. 

\subsubsection{Lexical-decision-Studie}

In diesem Abschnitt werden die Reaktionszeiten in der lexical-decision-Studie betrachtet. Diese sind aufgrund der Aufgabenstellung (Bewertung der Bekanntheit statt Lesen) und aufgrund der Nutzung von real existierenden Testsubstantiven anders gelagert als in der self-paced-reading-Studie. Abbildung \ref{schemadeclogRT} zeigt die logarithmierten Reaktionszeiten in Beeswarmplots für die einzelnen Form-Schema\-ti\-zi\-täts\-aus\-prägungen samt arithmetischem Mittel (M) und Standardabweichung (SD), diese werden samt Standardfehler (SE) auch gesondert berichtet. Reaktionszeiten für starke Formen sind als hellgraue Dreiecke dargestellt, Reaktionszeiten für schwache Formen als dunkelgraue Kreise. 

\begin{figure}
\includegraphics[width= 1 \textwidth]{figures/Kap6/schemadeclogRT.png} 
\caption{{\footnotesize{Reaktionszeiten in der lexical-decision-Studie zu Form-Schematizität}}}
\label{schemadeclogRT}
\end{figure}

Eine deutliche Interaktion zwischen Form-Schematizitätsausprägung und Flexionsform ist zu erkennen: Während Substantive der Ausprägungen \textsc{Form-Sche\-ma} und \textsc{Peripherie} hohe Reaktionszeiten für starke Formen aufweisen und niedrige für schwache, ist das Gegenteil für Substantive der Ausprägung \textsc{stark} der Fall. Die starken Formen sind bei schwachen Maskulina mental nicht gefestigt und lösen daher höhere Reaktionszeiten aus als die mental gefestigten schwachen Formen. Dies scheint unabhängig von der Passgenauigkeit der Substantive zum Form-Schema der Fall zu sein, da der Vorteil für schwache Formen sowohl bei prototypischen als auch bei peripheren Vertretern des Form-Schemas zu erkennen ist. Umgekehrt besteht für die starken Substantive ein Vorteil für starke Formen. 


Die Reaktionszeiten spiegeln damit das Antwortverhalten. Nur bei Substantiven aus der Peripherie des Form-Schemas ist dies in Hinblick auf starke Formen nicht der Fall: Da die starken Formen bei Substantiven der Ausprägung \textsc{Peripherie} in 50 \% der Antworten als bekannt bewertet wurden, scheinen sie stärker gefestigt zu sein als starke Formen bei Substantiven der Ausprägung \textsc{Form-Schema}, die nur zu 10 \% als bekannt bewertet wurden (siehe \sectref{ergschemadecant} zum Antwortverhalten in der lexical-decision-Studie). Daher wäre zu erwarten, dass die Reaktionszeiten für starke und schwache Formen bei Substantiven der Ausprägung \textsc{Peripherie} näher beieinander liegen als bei Substantiven der Ausprägung \textsc{Form-Schema}. Dies ist aber in den Reaktionszeiten nicht zu erkennen.   


Abseits der Interaktion ist kein Unterschied in den Reaktionszeiten zwischen den Ausprägungen zu erkennen: Die Verteilungen decken dieselbe Reaktionszeitenspanne ab und auch die Mittelwerte ähneln sich stark: Sie liegen zwischen 0,35 und 0,4 mit Standardabweichungen von ca. 0,5. Die Daten wurden auf Längen- und Reihenfolge-Effekte überprüft (siehe Abbildungen \ref{lengthschemadec} und \ref{blockschemadec} in Anhang \ref{ergebnisseschema}), die anhand der visuellen Analyse ausgeschlossen werden konnten.

Ein Einfluss der starken und schwachen Formen auf Reaktionszeiten war bereits in der lexical-decision-Studie zu Frequenzeffekten bei starken Verben zu erkennen (siehe \sectref{freqrt}). Wie in der Frequenzstudie könnte auch in dieser Studie die Ablehnung der Frage nach der Bekanntheit der Form höhere Reaktionszeiten hervorgerufen haben als die Zustimmung zur Frage. Die starken Formen von Substantiven in der Peripherie des Form-Schemas wurden etwa zur Hälfte als bekannt oder unbekannt angesehen, daher können sie einen Hinweis auf einen möglichen Einfluss der Bewertung geben. Abbildung \ref{bekanntheitlex} zeigt die Reaktionszeiten für starke Formen von Substantiven der Ausprägung \textsc{Peripherie} nach ihrer Bekanntheit.

\begin{figure}
\includegraphics[width= 1 \textwidth]{figures/Kap6/bekanntheitschema.png} 
\caption{Reaktionszeiten für starke Formen von Substantiven der Ausprägung \textsc{Peripherie} nach Bekanntheit}
\label{bekanntheitlex}
\end{figure}

Es zeigt sich eine ähnliche Verteilung für beide Bewertungen, allerdings tendieren die Reak\-tionszeiten bei als bekannt bewerteten Substantiven eher zum Mittelwert und streuen weniger. Daher ergeben sich auch leichte Unterschiede im Mittelwert: Der Mittelwert für die Antworten, die der Bekanntheit der Form zustimmten, liegt mit 0,43 und einer Standardabweichung von 0,3 unter dem Mittelwert für ablehnende Antworten (0,59; Standardabweichung 0,48). Ein Einfluss der Bewertung als bekannt oder unbekannt auf Reaktionszeiten ist daher wie in der Frequenzstudie nicht auszuschließen.
  

Um die Interaktion zwischen Form-Schematizitätsausprägung und Flexionsform näher betrachten zu können, werden Scatterplots in den Blick genommen. Abbildung \ref{schemadecscatter} kontrastiert die logarithmierten Reaktionszeiten für starke und schwache Formen von Substantiven der einzelnen Form-Sche\-ma\-ti\-zi\-täts\-aus\-prä\-gun\-gen. Ein Punkt steht für die Reaktionszeiten einer Person für ein Testitem. Die graue Linie dient als Orientierung: Sie zeigt den Bereich an, in dem starke und schwache Formen gleich schnell bewertet wurden. 

\begin{figure}
\includegraphics[width= 1 \textwidth]{figures/Kap6/starkvsschwachRTalle.png} 
\caption{Scatterplots der Reaktionszeiten für starke und schwache Formen der Testsubstantive}
\label{schemadecscatter}
\end{figure}

Die Interaktion ist auch in dieser Darstellung deutlich zu erkennen: Während die Ausprägungen \textsc{Form-Schema} und \textsc{Peripherie} vornehmlich bei starken Formen höhere Reaktionszeiten hervorrufen, was an den Punkten unterhalb der grauen Linie zu erkennen ist, evoziert die Ausprägung \textsc{stark} höhere Reaktionszeiten für schwache Formen, was anhand der Punkte oberhalb der grauen Linie deutlich wird.

Im Versuchsdesign wurde weitgehend auf Tokenfrequenz kontrolliert (siehe \sectref{schemalexmaterial}). Da die Testsubstantive in ihrer Tokenfrequenz aber dennoch leicht variieren, wird überprüft, ob Tokenfrequenz einen Einfluss auf die Reaktionszeiten nimmt. Hierfür wurden die Testsubstantive anhand ihrer Tokenfrequenz in drei Gruppen geteilt: \textsc{Frequent} für Substantive mit einer relativen Tokenfrequenz von über 0,47 Belegen pro Million Token (\textit{Franzose}, \textit{Nachbar}, \textit{Kollege}, \textit{Held}, \textit{Zar}), \textsc{mittelfrequent} für Substantive mit einer Tokenfrequenz von 0,28 bis 0,47 (\textit{Feind}, \textit{Schütze}, \textit{Graf}, \textit{Freund}, \textit{Fürst}) und \textsc{infrequent} für Substantive mit einer Tokenfrequenz von unter 0,05 (\textit{Neffe}, \textit{Dieb}, \textit{Geselle}, \textit{Kerl}, \textit{Vogt}).\footnote{Dieses Vorgehen wurde vorab registriert, siehe \url{https://osf.io/yx3tv.}} Die drei Frequenzausprägungen wurden genutzt, um die Vergleichbarkeit mit den Form-Schematizitätsausprägungen aufrecht zu erhalten, daher wurden auch pro Frequenzausprägung fünf Substantive genutzt. Die Frequenzunterschiede zwischen den Frequenzausprägungen ergeben sich daher allein durch die Anordnung der Testsubstantive nach Frequenz und durch die Einteilung in fünf Substantive pro Frequenzausprägung. Da die Testsubstantive nicht auf Unterschiede in ihrer Tokenfrequenz hin ausgewählt wurden, sind die Abstufungen zwischen den Frequenzausprägungen z.~T. subtil. Dennoch fällt der Frequenzunterschied zwischen \textsc{frequent} (ab 0,47) und \textsc{infrequent} (unter 0,05) recht deutlich aus.  


Abbildung \ref{schemadecfreqlogRT} zeigt Beeswarmplots für die Reaktionszeiten nach Frequenzausprägung samt arithmetischem Mittel (M) und Standardabweichung (SD), zusätzlich wird der Standardfehler (SE) berichtet.

\begin{figure}
\includegraphics[width= 1 \textwidth]{figures/Kap6/schemadecfreqlogRT.png} 
\caption{Reaktionszeiten nach Frequenzausprägung in der lexical-decision-Studie zu Form-Schematizität}
\label{schemadecfreqlogRT}
\end{figure}

Der Einfluss der starken und schwachen Formen deutet sich nur noch in der Ausprägung \textsc{frequent} an: Die schwachen Formen haben eher niedrige Reaktionszeiten hervorgerufen, die starken eher hohe. Bei den anderen Ausprägungen lässt sich kein Unterschied zwischen starken und schwachen Formen erkennen. Die Ausprägung \textsc{frequent} ist die einzige, die nur schwache Maskulina enthält, während die anderen beiden Ausprägungen durchmischt sind. Ein Frequenzeffekt ist nicht zu erkennen: Die Verteilung der Reaktionszeiten hat sich nicht geändert, die Mittelwerte liegen bei ca. 0,4 mit Standardabweichungen von 0,5. 

Für die konfirmatorische statistische Analyse der Daten wird ein gemischtes lineares Modell mit Form-Schematizitätsausprägung und Flexionsform als feste Effekte genutzt, die interagieren können: Reaktionszeit \~{} Form-Schematizität * Flexion + (1|Proband\_in) + (1|Item). Für Teilnehmer\_innen und Versuchitems (Lemma) werden \textit{random intercepts} berücksichtigt. \textit{Random slopes} werden für Teilnehmer\_innen wegen \textit{singular fits} nicht genutzt. Bei den Versuchitems sind keine \textit{random slopes} möglich, da sie über die Form-Schematizitätsausprägungen hinweg variieren. Die Standardabweichung um den y-Ach\-sen\-ab\-schnitt ist für Versuchitems mit 0,05 sehr gering, Proband\_innen weisen mit 0,27  erwartbarerweise eine höhere Standardabweichung auf. Die restliche Residuenvarianz beträgt~0,35. Die geschätzten Werte des Modells sind in Tabelle \ref{ergschemadecrt}, eine entsprechende Kreuztabelle (Tabelle \ref{kreuzergschemadecrt}) befindet sich in Anhang \ref{ergebnisseschema}. Abbildung \ref{predrtschemadec} zeigt die vom Modell vorhergesagten Reaktionszeiten. Kreise stehen für schwache Formen, Dreiecke für starke.

\begin{table}
\begin{tabular9}{l S[table-format=-1.2] S[table-format=1.2] S[table-format=-2.2] S[table-format=4.2] S[table-format=<1.2]}
\lsptoprule
& {Wert} & {SE} & {$t$} & {df} & {$p$} \\\midrule
(Intercept = Form-Schema \& schwach) & 0,28 & 0,05 & 5,87 & 71,43 & < 0,01 \\ 
Form-Schematizität: Peripherie & -0,07 & 0,04 & -1,59 & 24,64 & 0,12 \\ 
Form-Schematizität: stark & 0,25 & 0,04 & 5,84 & 24,64 & < 0,01 \\ 
Flexion: stark & 0,20 & 0,03 & 6,78 & 1609,90 & < 0,01 \\ 
Form-Schematizität: Peripherie \& Flexion: stark & 0,09 & 0,04 & 2,14 & 1609,90 & 0,03 \\ 
Form-Schematizität: stark \& Flexion: stark & -0,46 & 0,04 & -10,94 & 1609,90 & < 0,01 \\ 
\lspbottomrule
\end{tabular9}
\caption{Werte des Modells für die Reaktionszeiten in der lexical-decision-Studie zu Form-Sche\-ma\-ti\-zi\-tät}
\label{ergschemadecrt}
\end{table}

\begin{figure}
\includegraphics[width=\textwidth]{figures/Kap6/predrtschemadec.png} 
\caption{Vorhergesagte Reaktionszeiten in der lexical-decision-Studie zu Form-Schematizität}
\label{predrtschemadec}
\end{figure}

\begin{sloppypar}
Auch bei den vorhergesagten Reaktionszeiten ist die Interaktion zwischen Form-Schema\-ti\-zi\-täts\-aus\-prä\-gung und starken und schwachen Flexionsformen klar zu erkennen. Die 95~\%-Konfidenzintervalle der Ausprägungen überlappen sich jeweils hinsichtlich der schneller und langsamer bewerteten Formen, sodass abseits der Interaktion kein Einfluss der Form-Schematizitätsausprägung auf Reaktionszeiten zu erkennen ist. Es scheint somit vorrangig relevant zu sein, welche Formen beurteilt werden: Die mental gefestigten Formen werden schneller bewertet als die mental weniger bzw. nicht gefestigten.
Starke Formen der Ausprägung \textsc{Form-Schema} liegen im Vergleich zum Referenzlevel \textsc{Form-Schema} mit schwachen Formen unter dem $\alpha$-Level von 0,01. Dasselbe gilt für die Ausprägung \textsc{stark} mit starken und schwachen Formen. Hier ist also jeweils von systematischen Effekten auszugehen. Der Unterschied zur Ausprägung \textsc{Peripherie} ist hingegen über $\alpha$-Level. Somit ist nicht von einem Unterschied in den Reaktionszeiten zwischen \textsc{Form-Schema} und \textsc{Peripherie} auszugehen, was bereits in der visuellen Analyse sichtbar war. Die Effektstärke des Modells ist mit einem $R^2m$ von 0,08 sehr klein, mit den zufälligen Effekten ist sie mit einem $R^2c$ von 0,4 deutlich höher.
\end{sloppypar}

Für die explorative Analyse der Daten werden die Frequenzausprägungen der Substantive in einem gemischten linearen Modell betrachtet. Auch hier ist zusätzlich die Flexionsform als fester Effekt vorgesehen, die beiden Effekte können interagieren: Reaktionszeit \~{} Frequenz * Flexion + (1|Proband\_in) + (1|Item). Für die zufälligen Effekte Proband\_in und Test\-item sind \textit{random intercepts} vorgesehen, \textit{random slopes} werden wegen \textit{singular fits} nicht genutzt bzw. sind für die Test\-items nicht möglich, da die Testitems über die Frequenzausprägungen hinweg variieren. Die zufälligen Effekte ähneln stark dem oben diskutierten Modell zum Einfluss der Form-Schematizitätsausprägungen: Items variieren mit 0,05 und Proband\_innen mit 0,27 um den y-Achsenabschnitt. Die restliche Residuenvarianz beträgt 0,37. Die Ergebnisse des Modells sind in Tabelle \ref{schemadecfreq}, die entsprechende Kreuztabelle (Tabelle \ref{schemadecfreqkreuz}) befindet sich in Anhang \ref{ergebnisseschema}. Abbildung \ref{predrtschemadecfreq} zeigt die vorhergesagten Reaktionszeiten. 

\begin{table}
\begin{tabular}{l S[table-format=-1.2] S[table-format=1.2] S[table-format=-1.2] S[table-format=4.2]}
\lsptoprule
& {Wert} & {SE} & {$t$} & {df}\\ \midrule
(Intercept = frequent \& schwach) & 0,26 & 0,05 & 5,42 & 72,08 \\ 
Frequenz: infrequent & 0,12 & 0,04 & 2,79 & 25,66 \\ 
Frequenz: mittelfrequent & 0,12 & 0,04 & 2,77 & 25,66 \\ 
Flexion: stark & 0,23 & 0,03 & 7,40 & 1609,89 \\ 
Frequenz: infrequent \& Flexion: stark & -0,27 & 0,04 & -6,16 & 1609,89 \\ 
Frequenz: mittelfrequent \& Flexion: stark & -0,19 & 0,04 & -4,21 & 1609,89 \\
\lspbottomrule
\end{tabular}
\caption{Werte des Modells für die Reaktionszeiten in der lexical-decision-Studie zu Form-Schematizität in Abhängigkeit von Frequenz}
\label{schemadecfreq}
\end{table}


\begin{figure}
\includegraphics[width=\textwidth]{figures/Kap6/predrtschemadecfreq.png} 
\caption{Vorhergesagte Reaktionszeiten in der lexical-decision-Studie zu Form-Schematizität in Abhängigkeit von Frequenz }
\label{predrtschemadecfreq}
\end{figure}

Der Einfluss von starken und schwachen Formen ist wieder zu erkennen: Da in der Ausprägung \textsc{frequent} nur schwache Maskulina enthalten sind, werden starke Formen langsamer bewertet als schwache. Bei \textsc{mittelfrequent} und \textsc{infrequent} ist dieser Effekt nicht zu sehen, da hier Substantive beider Klassen vorkommen. Ein Frequenzeffekt lässt sich ebenfalls nicht erkennen. Zwar weisen die Substantive der Ausprägung \textsc{frequent} für schwache Formen niedrigere Werte auf als die anderen Ausprägungen, dies lässt sich aber dadurch erklären, dass die anderen Ausprägungen auch starke Maskulina enthalten und die schwachen Formen daher nicht so stark gefestigt sind. Diese Erklärung wird von den starken Formen der Ausprägung \textsc{frequent} gestützt, die die höchsten Reaktionszeiten evozieren: Da in der Ausprägung \textsc{frequent} nur schwache Maskulina enthalten sind, sind hier die starken Formen nicht gefestigt. 


Die 95~\%-Konfidenzintervalle sind für fast alle Werte recht lang, nur für die starken und schwachen Formen der Ausprägung \textsc{frequent} zeigen sich keine Überlappungen. Ein systematischer Einfluss der Frequenzausprägungen scheint somit nicht gegeben zu sein. Die Effektstärke ist mit einem $R^2m$ von 0,02 und einem $R^2c$ von 0,36 deutlich unterhalb der Effektstärke des Modells zum Einfluss der Form-Schematizitätsausprägungen.

Um die Reaktionszeiten innerhalb der lexical-decision-Studie zu Form-Sche\-ma\-ti\-zi\-tät besser einordnen zu können, ist ein Vergleich der Reaktionszeiten mit den Reaktionszeiten in der Studie zu Frequenzeffekten bei starken Verben (siehe \sectref{freqerg}) hilfreich, da in beiden Fällen zwischen zwei Fle\-xionsformen gewählt wurde. Dabei ist zu bedenken, dass starke und schwache Formen von Verben und Substantiven nicht direkt vergleichbar sind. Aber dennoch ist eine Gegenüberstellung interessant, da auf diese Weise Unterschiede in den Reaktionszeitmustern zwischen der Frequenz- und der Form-Schematizitätsstudie herausgearbeitet werden können. Für den Vergleich wurden die Frequenz- und Form-Schematizitätsausprägungen zusammengefasst: \textsc{frequent/Form-Schema}, \textsc{infrequent ohne Schwankung/Pe\-ri\-phe\-rie} und \textsc{infrequent mit Schwankung/stark} sind die Kategorien, die miteinander verglichen werden. Abbildung \ref{wavgl} zeigt die Mittelwerte für die logarithmierten Reaktionszeiten samt Standardabweichung. Auf Beeswarmplots wurde aufgrund der Komplexität der Daten verzichtet. Substantive sind als Kreise dargestellt, Verben als Dreiecke. Reaktionszeiten für schwache Formen sind dunkelgrau eingefärbt, Reaktionszeiten für starke Formen hellgrau.

\begin{figure}
\includegraphics[width= 0.9 \textwidth]{figures/Kap6/wavgl.png} 
\caption{Vergleich der Reaktionszeiten für Substantive und Verben in den Studien zu Form-Schematizität und Frequenz}
\label{wavgl}
\end{figure}

Während die Mittelwerte der Reaktionszeiten bei den Substantiven auf einem Niveau bleiben und man lediglich die Interaktion zwischen Flexionsklasse und Flexionsform beobachten kann, sind die Mittelwerte der Reaktionszeiten bei den Verben zusätzlich zum Einfluss der Flexionsform frequenzabhängig gestaffelt: Die Reaktionszeiten für frequente Verben sind geringer als für infrequente Verben mit und ohne Schwankung. 


Innerhalb der Kategorieausprägungen weisen die Reaktionszeiten für starke und schwache Flexionsformen interessante Unterschiede zwischen Verben und Substantiven auf: Der Mittelwert der Reaktionszeiten für starke Formen von Verben der Ausprägung \textsc{frequent} ist deutlich niedriger als der Mittelwert der Reaktionszeiten für starke und schwache Formen von Substantiven der Ausprägung \textsc{Form-Schema}. Der Mittelwert für schwache Formen von frequenten Verben gleicht hingegen dem Mittelwert der schwachen Formen von Substantiven der Ausprägung \textsc{Form-Schema}. Die Unterschiede in den Mittelwerten verdeutlichen die Tokenfrequenzunterschiede zwischen den Testverben und -sub\-stan\-ti\-ven der Ausprägung \textsc{frequent/Form-Schema}, die zu einem unterschiedlichen Grad an \textit{entrenchment} führen: Die schwachen Substantivformen sind aufgrund der geringeren Tokenfrequenz weniger stark gefestigt als die starken Verbformen. Daher lösen die starken Verbformen geringere Reaktionszeiten aus als die Substantivformen. Die schwachen Verbformen sind selbst nicht gefestigt und können daher nur schnell abgelehnt werden, da sie von den frequenten starken Verbformen statistisch ausgestochen werden. Deshalb sind die Reaktionszeiten der schwachen Verbformen im Vergleich zu den starken Verbformen leicht erhöht und auf dem Niveau der schwachen Substantivformen.  

Bei den Verben der Ausprägung \textsc{infrequent ohne Schwankung} verschieben sich die Mittelwerte der Reaktionszeiten im Vergleich zu den frequenten Verben deutlich nach oben: Nun sind die starken Verbformen auf einem Niveau mit den schwachen Substantivformen der Ausprägung \textsc{Peripherie} und die schwachen Verbformen auf einem Niveau mit den starken Substantivformen derselben Ausprägung. Ähnlich verhalten sich die Verben der Ausprägung \textsc{infrequent mit Schwankung}: Auch hier ist die jeweils schneller beurteilte Verbform auf einem Niveau mit der schneller beurteilten Substantivform. Bei den infrequenten Verben sind die Reaktionszeiten der starken Formen somit unabhängig von Schwankungen in Korpora jeweils vergleichbar mit den Reaktionszeiten der mental gefestigteren Form der Testsubstantive (schwache Formen bei der Ausprägung \textsc{Peripherie}, starke Formen bei der Ausprägung \textsc{stark}). Vergleichbare Reaktionszeiten für die jeweils mental gefestigten bzw. nicht gefestigten Formen sind erwartbar, da die infrequenten Testverben wie die Testsubstantive eine Tokenfrequenz von unter zwei Belegen pro Million Token aufweisen.

Um den Vergleich zwischen Frequenz- und Form-Schematizitätseffekten statistisch zu fassen, wird als explorative Analyse ein gemischtes lineares Modell gerechnet, das Wortart, Kategorie (\textsc{\textsc{frequent/Form-Schema}}, \textsc{infrequent ohne Schwankung/Peri\-phe\-rie} und \textsc{infrequent mit Schwankung/stark}) und Flexionsform beinhaltet, die festen Effekte können interagieren: Reaktionszeit \~{} Wortart * Kategorie * Flexion + (1|Proband\_in) + (1|Item). Als zufällige Effekte werden \textit{random intercepts} für Teilnehmer\_innen und Versuch\-items genutzt. Die zufälligen Effekte sind vergleichbar mit den vorherigen Modellen. Die Ergebnisse des Modells sind in Tabelle \ref{vglwamodell} aufgeführt, hierfür wurde \textsc{infrequent ohne Schwankung} durch \textit{ohne} und \textsc{infrequent mit Schwankung} durch \textit{mit} abgekürzt. Die darauf aufbauende Kreuztabelle (Tabelle \ref{vglwapredkreuz}) befindet sich in Anhang \ref{vgl}. Abbildung~\ref{wavglpred} (S.~\pageref{wavglpred}) zeigt die vorhergesagten Reaktionszeiten. Sie sind für schwache Formen dunkelgrau und für starke hellgrau eingefärbt. Kreise stehen für Substantive, Dreiecke für Verben. 

\begin{table}
\begin{tabularx}{\textwidth}{Q S[table-format=-1.2] S[table-format=1.2] S[table-format=-2.2] S[table-format=4.2]}
\lsptoprule
& {Wert} & {SE} & {$t$} & {df}\\ \midrule
(Intercept = frequent/Form-Schema \& Substantiv \& schwach) & 0,24 & 0,05 & 4,99 & 97,47 \\ 
Kategorie: mit/stark & 0,26 & 0,04 & 5,79 & 65,85 \\ 
Kategorie: ohne/Peripherie & -0,05 & 0,04 & -1,23 & 65,85 \\ 
Wortart: Verb & -0,01 & 0,04 & -0,37 & 65,85 \\ 
Flexion: stark & 0,22 & 0,03 & 7,04 & 3580,96 \\ 
Kategorie: mit/stark \& Wortart: Verb & -0,10 & 0,06 & -1,73 & 65,85 \\ 
Kategorie: ohne/Peripherie \& Wortart: Verb & 0,25 & 0,06 & 4,36 & 65,85 \\ 
Kategorie: mit/stark \& Flexion: stark & -0,48 & 0,04 & -10,95 & 3580,96 \\ 
Kategorie: ohne/Peripherie \& Flexion: stark & 0,06 & 0,04 & 1,49 & 3580,96 \\ 
Wortart: Verb \& Flexion: stark & -0,49 & 0,04 & -12,38 & 3580,96 \\ 
Kategorie: mit/stark \& Wortart: Verb \& Flexion: stark & 0,59 & 0,06 & 10,58 & 3580,96 \\ 
Kategorie: ohne/Peripherie \& Wortart: Verb \& Flexion: stark & -0,12 & 0,06 & -2,16 & 3580,96 \\
\lspbottomrule
\end{tabularx}
\caption{Werte des Modells für die Reaktionszeiten bei Substantiven und Verben in den Studien zu Form-Sche\-ma\-ti\-zi\-tät und Frequenz}
\label{vglwamodell}
\end{table}

\begin{figure}
\includegraphics[width=\textwidth]{figures/Kap6/wavglpred.png} 
\caption{Vorhergesagte Reaktionszeiten für Substantive und Verben in den Studien zu Form-Schematizität und Frequenz}
\label{wavglpred}
\end{figure}

Die vorhergesagten Reaktionszeiten unterscheiden sich sowohl für Substantive als auch für Verben deutlich je nach Flexionsform. Die schwachen Formen weisen für Substantive der Ausprägungen \textsc{Form-Schema} und \textsc{Peripherie} geringere Reaktionszeiten auf als die starken, das Gegenteil ist für Substantive der Ausprägung \textsc{stark} der Fall. Die Verben zeigen unabhängig von der Frequenzausprägung niedrigere Reaktionszeiten für starke Formen. Die 95~\%-Konfidenzintervalle überlappen sich in dieser Hinsicht nur bei starken und schwachen Formen der infrequenten Verben mit Schwankung. Bei Verben ist zusätzlich zum Einfluss der Flexionsformen ein Frequenzeffekt zu sehen: Die Ausprägung \textsc{frequent} weist geringere Reaktionszeiten auf als die anderen Frequenzausprägungen. Eine Überlappung des 95~\%-Konfidenzintervalls der Ausprägung \textsc{frequent} mit den 95~\%-Konfidenzintervallen der anderen Ausprägung ist dabei nicht zu erkennen. Für die Form-Schematizitäts\-aus\-prä\-gungen lässt sich hingegen kein systematischer Unterschied in den Reaktionszeiten beobachten. Die Effektstärke ist mit einem $R^2m$ von 0,13 und einem $R^2c$ von 0,46 vergleichbar mit der Effektstärke des Modells der Frequenzstudie.

Insgesamt zeigt die Betrachtung der Reaktionszeiten in der lexical-decision-Studie vorrangig Effekte der schwachen und starken Formen: Je nach Flexionsklasse der Substantive werden schwache Formen schneller oder langsamer beurteilt als starke, da sie je nach Flexionklasse stärker bzw. schwächer mental gefestigt sind. Zudem könnte die Beurteilung der Formen (Bekanntheit\slash Un\-be\-kannt\-heit) die Reaktionszeiten beeinflussen. Ein Effekt der Form-Schematizität lässt sich hingegen nicht erkennen. Die Substantive der einzelnen Form-Sche\-ma\-ti\-zi\-täts\-aus\-prä\-gun\-gen evozieren ähnliche Reaktionszeiten, auch Tokenfrequenz scheint keinen Einfluss auf die Reaktionszeiten zu haben. Anders als in der Frequenzstudie, in der neben dem Einfluss der Flexionsformen auch Frequenzeffekte beobachtet wurden, scheint in der lexical-decision-Studie zu Form-Schematizität somit nur die Flexionsform die Reaktionszeiten zu beeinflussen. Die Ergebnisse bestätigen damit die in Abschnitt~\ref{hyposchema} vorgestellten Hypothesen, die anhand der Ergebnisse der Prästudie aufgestellt wurden.\largerpage

Ursprünglich wurde erwartet, dass Substantive, die dem Form-Schema nur peripher angehören, für starke und schwache Formen höhere Reaktionszeiten evozieren, da die Wahl zwischen zwei mental gefestigten Formen besteht. Dabei weist auch das Antwortverhalten in der Studie darauf hin, dass in der Peripherie des Form-Schemas beide Formen mental gefestigt sind: Schwache Formen von Substantiven der Ausprägung \textsc{Peripherie} wurden zu 90 \% und starke Formen zu 50 \% als bekannt bewertet. In den Reaktionszeiten zeigt sich jedoch kein Unterschied zwischen Peripherie und den anderen Ausprägungen. Starke und schwache Formen von Substantiven der Ausprägung \textsc{Peripherie} werden jeweils ähnlich schnell bewertet wie starke und schwache Formen von Substantiven der Ausprägung \textsc{Form-Schema}. Die Existenz einer alternativen Form scheint die Reaktionszeiten in der lexical-decision-Studie daher nicht zu beeinflussen.



Neben generell erhöhten Reaktionszeiten bei Substantiven in der Peripherie im Vergleich zu den anderen Form-Sche\-ma\-ti\-zi\-täts\-aus\-prä\-gun\-gen wäre auch eine Annäherung der Reaktionszeiten von starken und schwachen Formen bei Substantiven der Peripherie denkbar. Beide Formen sind mental gefestigt und sollten daher ähnliche Reaktionszeiten hervorrufen. Genau das ist bei den infrequenten Verben mit Schwankung, deren neue schwache Formen in der Frequenzstudie ebenfalls zu 50 \% als bekannt bewertet wurden, zu beobachten, aber nicht bei den Substantiven in der Peripherie des Form-Schemas in der lexical-decision-Studie: Bei Substantiven in der Peripherie des Form-Schemas weisen die starken Formen höhere Reaktionszeiten auf als schwache Formen. Daher ist davon auszugehen, dass die starken Formen in der Peripherie des Form-Schemas im Vergleich zu schwachen nur wenig gefestigt sind, sodass sie von den schwachen Formen dominiert werden. 

\subsubsection{Sentence-maze-Studie}\largerpage

In diesem Abschnitt werden die Reaktionszeiten in der sentence-maze-Studie analysiert. Diese sind anders strukturiert als die Reaktionszeiten in der lex\-i\-cal-de\-ci\-sion-Stu\-die, da hier zwischen zwei Formen gewählt wurde und nicht beide Formen einzeln auf ihre Bekanntheit hin bewertet wurden. 


Abbildung \ref{rtschemasent} zeigt die logarithmierten Reaktionszeiten für die Form-Sche\-ma\-ti\-zi\-täts\-aus\-prä\-gun\-gen in Beeswarmplots samt arithmetischem Mittel (M) und Standardabweichung (SD). Diese werden samt Standardfehler (SE) auch gesondert berichtet. 

\begin{figure}
\includegraphics[width= 1 \textwidth]{figures/Kap6/schemasentlogRT.png} 
\caption{Reaktionszeiten in der sentence-maze-Studie zu Form-Schematizität}
\label{rtschemasent}
\end{figure}


Die Verteilungen der Reaktionszeiten sehen pro Form-Sche\-ma\-ti\-zi\-täts\-aus\-prä\-gung unterschiedlich aus: Während bei der Ausprägung \textsc{Form-Schema} viele Datenpunkte um den Mittelwert herum clustern, finden sich bei den anderen Ausprägungen viele Datenpunkte unterhalb des Mittelwerts. Auch in den Mittelwerten sind Unterschiede zu erkennen: Die Ausprägung \textsc{Form-Schema} weist mit 0,58 bei einer Standardabweichung von 0,42 den höchsten Mittelwert auf, die Ausprägung \textsc{stark} liegt knapp darunter mit einem Mittelwert von 0,53 und einer Standardabweichung von 0,4. Geringere Reaktionszeiten weisen die Testsubstantive der Ausprägung \textsc{Peripherie} mit einem Mittelwert von 0,4 und einer Standardabweichung von 0,42 auf. Ein Effekt durch Form-Schematizität ist somit nicht zu erkennen, da die Testsubstantive der Peripherie geringere statt höhere Reaktionszeiten aufweisen als die Substantive der anderen Form-Sche\-ma\-ti\-zi\-täts\-aus\-prä\-gun\-gen. Höhere Reaktionszeiten wären bei einem Form-Sche\-ma\-ti\-zi\-täts\-ef\-fekt für die Peripherie zu erwarten, da in der Peripherie beide Formen, die zur Wahl stehen, mental gefestigt sind und nicht nur jeweils eine wie bei den anderen Form-Sche\-ma\-ti\-zi\-täts\-aus\-prä\-gun\-gen. Die visuelle Überprüfung auf Reihenfolge-Effekte ergab keine Auffälligkeiten (siehe Abbildung \ref{blockschemasent} in Anhang \ref{ergebnisseschema}). Auf Längeneffekte muss nicht geprüft werden, da beide Stimuli jeweils gleichzeitig sichtbar sind.\largerpage

Um zu überprüfen, ob Tokenfrequenz Einfluss auf die Reaktionszeiten nimmt, wurden die Testsubstantive in verschiedene Frequenzklassen geteilt: Dabei wird wie in der lexical-decision-Studie zu Form-Schematizität zwischen \textsc{frequent} (ab 0,48 Belegen pro Million Token; \textit{Kollege}, \textit{Held}, \textit{Zar}), \textsc{mittelfrequent} (von 0,3 bis 0,41; \textit{Schütze}, \textit{Graf}, \textit{Freund}) und \textsc{infrequent} (bis 0,05; \textit{Neffe}, \textit{Dieb}, \textit{Vogt}) unterschieden.\footnote{Die Aufteilung in Frequenzklassen wurde vorab registriert, siehe \url{https://osf.io/bhxpz.}} Wie in der lexical-decision-Studie wurden auch in dieser Studie drei Frequenzausprägungen gewählt, um die Vergleichbarkeit mit den Form-Schematizitätsausprägungen sicherzustellen. Abbildung \ref{rtschemasentfreq} zeigt die Beeswarmplots mit den Frequenzausprägungen.

\begin{figure}
\includegraphics[width= 1 \textwidth]{figures/Kap6/schemasentfreqlogRT.png} 
\caption{Reaktionszeiten nach Frequenzausprä\-gung in der sen\-ten\-ce-maze-Stu\-die zu Form-Sche\-ma\-ti\-zi\-tät}
\label{rtschemasentfreq}
\end{figure}

Die Reaktionszeiten der Ausprägung \textsc{frequent} scheinen um niedrigere Reaktionszeiten zu clustern als die der anderen Ausprägungen. Die Ausprägungen \textsc{mittelfrequent} und \textsc{infrequent} weisen dagegen eine ähnliche Verteilung der Reaktionszeiten auf. Die Spanne der Reaktionszeiten ist für alle Ausprägungen vergleichbar (\textsc{frequent}: −0,55 bis 2,02; \textsc{mittelfrequent}: −0,47 bis 1,96; \textsc{infrequent}: −0,41 bis 1,97). In den Mittelwerten zeigen sich Unterschiede zwischen den Ausprägungen: Die Reaktionszeiten der Ausprägung \textsc{frequent} haben einen Mittelwert von 0,4 mit einer Standardabweichung von 0,4. Die Mittelwerte der anderen Ausprägungen liegen mit 0,5 und Standardabweichungen von 0,4 etwas höher. Es ist somit ein kleiner Frequenzeffekt zu erkennen.\largerpage

Für die konfirmatorische statistische Analyse wird ein gemischtes lineares Modell mit Form-Schema\-ti\-zi\-tät als festem Effekt gerechnet: Reaktionszeit \~{} Form-Schematizität + (1|Proband\_in) + (1|Item). Teilnehmer\_innen und Versuchitems werden mit \textit{random intercepts} berücksichtigt. Für Teilnehmer\_innen werden wegen \textit{singular fits} keine \textit{random slopes} genutzt. Für Versuchitems sind \textit{random slopes} nicht sinnvoll, da unterschiedliche Versuchitems pro Form-Sche\-ma\-ti\-zi\-täts\-aus\-prä\-gung genutzt werden.  Die Versuchitems variieren lediglich mit einer Standardabweichung von 0,09 um den y-Achsenabschnitt, die Proband\_innen  deutlich stärker mit 0,26. Die restliche Residuenvarianz liegt bei 0,31. Die geschätzten Werte des Modells sind in Tabelle \ref{sentmazeschemamodell} aufgelistet, Abbildung \ref{schemasentlogRTpred} zeigt die vorhergesagten Re\-ak\-tions\-zei\-ten. 

\begin{table}[H]
\begin{tabular}{l S[table-format=-1.2] S[table-format=1.2] S[table-format=-1.2] S[table-format=2.2] S[table-format=<1.2]}
\lsptoprule
& {Wert} & {SE} & {$t$} & {df} & {$p$} \\\midrule
(Intercept = Form-Schema) & 0,58 & 0,06 & 9,87 & 12,14 & < 0,01 \\ 
Form-Schematizität: Peripherie & -0,17 & 0,08 & -2,26 & 8,88 & 0,05 \\ 
Form-Schematizität: stark & -0,04 & 0,08 & -0,58 & 8,88 & 0,58 \\ 
\lspbottomrule
\end{tabular}
\caption{Werte des Modells für die Reaktionszeiten in der sentence-maze-Studie zu Form-Schematizität}
\label{sentmazeschemamodell}
\end{table}
\vfill
\begin{figure}[H]
\includegraphics[width=\textwidth]{figures/Kap6/schemasentlogRTpred.png} 
\caption{Vorhergesagte Reaktionszeiten in der sentence-maze-Studie zu Form-Schematizität}
\label{schemasentlogRTpred}
\end{figure}
\pagebreak

Für die Ausprägungen \textsc{Form-Schema} und \textsc{stark} werden mit 0,58 und 0,53 vergleichbare Reaktionszeiten vorhergesagt. Die vorhergesagten Reaktionszeiten für die Ausprägung \textsc{Peripherie} sind mit 0,4 geringer. Die 95~\%-Konfidenzin\-ter\-val\-le der Ausprägungen sind jedoch recht lang und überlappen sich deutlich. Die $p$-Werte für die Unterschiede zwischen den Ausprägungen sind jeweils über dem $\alpha$-Level von 0,01, sodass die Daten unter der Nullhypothese (die Form-Sche\-ma\-ti\-zi\-täts\-aus\-prä\-gun\-gen beeinflussen die Reaktionszeiten nicht) nicht überraschend wären. Es ist daher nicht von systematischen Effekten auszugehen.  Die Effektstärke des Modells ist mit einem $R^2m$ von 0,03 dementsprechend gering, bezieht man die zufälligen Effekte mit ein, ist sie mit einem $R^2c$ von 0,45 deutlich höher.



Die Schätzwerte für die Reaktionszeiten aus der Prästudie, anhand derer die benötigte Stichprobengröße für die Hauptstudie berechnet wurde, sind mit den Werten im aktuellen Modell vergleichbar. Es ist daher davon auszugehen, dass die Stichprobengröße von 132 Proband\_innen es ermöglicht, einen Effekt mit einer Wahrscheinlichkeit von 0,8 zu entdecken. Die Wahrscheinlichkeit, fälschlicherweise nicht von einem systematischen Effekt auszugehen, liegt daher nur bei 0,2. Es scheint somit kein Effekt vorzuliegen.

Um zu überprüfen, ob die Unterschiede in der Tokenfrequenz der Testsubstantive die Reak\-tionszeiten beeinflussen, wird als explorative Analyse dasselbe Modell mit Frequenzausprägungen statt Form-Schematizitätsausprägungen gerechnet: Reaktionszeit \~{} Frequenz + (1| Proband\_in) + (1|Item). Dafür werden dieselben zufälligen Effekte wie für das Modell mit Form-Sche\-ma\-ti\-zi\-täts\-aus\-prä\-gung\-en genutzt. Die zufälligen Effekte sind vergleichbar mit dem vorherigen Modell: Versuchitems variieren mit einer Standardabweichung von 0,1 um den y-Achsenabschnitt, Proband\_innen mit 0,26. Die restliche Residuenvarianz beträgt 0,3.  Die geschätzten Werte des Modells sind in Tabelle \ref{sentmazeschemamodellfr} aufgelistet. Abbildung \ref{schemasentlogRTpredf} zeigt die vorhergesagten Reaktionszeiten.

\vfill
\begin{table}[H]
\begin{tabular}{l S[table-format=1.2] S[table-format=1.2] S[table-format=1.2] S[table-format=2.2]}
\lsptoprule
& {Wert} & {SE} & {$t$} & {df}\\\midrule
(Intercept = frequent) & 0,43 & 0,07 & 6,56 & 11,38 \\ 
Frequenz: infrequent & 0,11 & 0,09 & 1,24 & 8,92 \\ 
Frequenz: mittelfrequent & 0,11 & 0,09 & 1,26 & 8,92 \\ 
\lspbottomrule
\end{tabular}
\caption{Werte des Modells für die Reaktionszeiten in der sentence-maze-Studie zu Form-Schematizität in Abhängigkeit von Frequenz}
\label{sentmazeschemamodellfr}
\end{table}
\vfill\pagebreak

\begin{figure}
\includegraphics[width=\textwidth]{figures/Kap6/schemasentlogRTpredf.png} 
\caption{Vorhergesagte Reaktionszeiten in der sentence-maze-Studie zu Form-Schematizität in Abhängigkeit von Frequenz}
\label{schemasentlogRTpredf}
\end{figure}

Die Substantive der Ausprägung \textsc{frequent} zeigen mit 0,43 leicht niedrigere Werte als die anderen Ausprägungen mit 0,54. Die 95~\%-Konfidenzintervalle überlappen sich jedoch deutlich, sodass nicht von einem systematischen Effekt ausgegangen werden kann. Die Effektstärke ist mit einem $R^2m$ von 0,015 noch niedriger als die Effektstärke für das Modell mit Form-Schematizitäts\-aus\-prä\-gungen, mit zufälligen Effekten ist sie mit einem $R^2c$ von 0,45 höher.

In den Reaktionszeiten ist kein Einfluss der Form-Schematizität zu erkennen. Die Hypothesen aus \sectref{hyposchema} bestätigen sich daher nicht. Für Substantive aus der Peripherie des Form-Schemas waren Reaktionszeiten erwartet worden, die von den anderen Form-Schematizitätsausprägungen abweichen. Dabei wurden einerseits höhere Reaktionszeiten in Betracht gezogen, da bei Substantiven in der Peripherie des Form-Schemas im Gegensatz zu den Substantiven der anderen Ausprägungen zwischen zwei (allerdings unterschiedlich stark) mental gefestigten Formen gewählt wird. Andererseits wären auch niedrigere Re\-ak\-tions\-zei\-ten erwartbar, weil die Testsubstantive aus der Peripherie des Form-Schemas zu den tokenfrequenten Substantiven\footnote{Mit \textit{Held} und \textit{Zar} zählen zwei der drei tokenfrequenten Testsubstantive zur Peripherie des Form-Schemas.} zählen. In den Daten zeigen sich allenfalls leicht geringere Reaktionszeiten für Substantive in der Peripherie des Form-Schemas im Vergleich zu den anderen Ausprägungen, die jedoch nicht systematisch zu sein scheinen. Ein Frequenzeffekt ist in den Daten zu erahnen, jedoch deutet die explorative Analyse allenfalls auf einen kleinen Effekt hin. Ein möglicher Frequenzeffekt müsste daher in einer weiteren Studie überprüft werden. Auch das Antwortverhalten in der sentence-maze-Studie zeigt lediglich einen klaren und erwartbaren Unterschied zwischen den Substantivklassen, jedoch nicht zwischen prototypischen und peripheren Vertretern des Form-Schemas. Der Grund für das Nullergebnis hinsichtlich der Reaktionszeiten und des Antwortverhaltens könnte darin liegen, dass das Versuchsdesign eine direkte Wahl zwischen stark gefestigten und weniger bzw. nicht gefestigten Formen vorsieht, die sich nur wenig voneinander unterscheiden. 

\subsubsection{Zusammenfassung}

Für die einzelnen Studien ließ sich kein Form-Schematiziätseffekt in den Reaktionszeiten erkennen. In der self-paced-reading-Studie werden starke und schwache Formen gleich schnell gelesen, nur bei \textit{Schettose} sind leichte Unterschiede in der Lesezeit zu beobachten. Ob diese systematisch sind, bleibt aufgrund der Überlappung der 95~\%-Konfidenzintervalle jedoch fraglich. Auch in der sentence-maze-Studie konnte kein Effekt der Form-Schematizität festgestellt werden. Die Versuchssubstantive aus der Peripherie des Form-Schemas evozieren leicht geringere Reaktionszeiten als die anderen Ausprägungen, dieser Unterschied scheint jedoch nicht systematisch zu sein. Falls der Unterschied systematisch sein sollte, muss näher untersucht werden, wie er mit Tokenfrequenz zusammenhängt, da die frequenten Testsubstantive ebenfalls geringere Reaktionszeiten aufweisen als mittel- und infrequente. Auch hier ist fraglich, ob es sich um einen systematischen Zusammenhang handelt. 


In den Reaktionszeiten der lexical-decision-Studie zeigen sich weder Einflüsse der Form-Schematizität noch der Frequenz, allerdings eine Interaktion zwischen Flexionsformen und Deklinationsklasse: Schwache Formen werden bei schwachen Maskulina unabhängig von ihrer Passgenauigkeit zum Form-Schema schnell beurteilt und starke langsam. Das Gegenteil ist für starke Maskulina der Fall. Unabhängig davon, ob der Unterschied in den starken und schwachen Formen an sich oder in der Bewertung als bekannt oder unbekannt begründet ist, zeigt dies, dass Proband\_innen für starke und schwache Formen bei Maskulina sensibel sind. Zudem deuten die Reaktionszeiten darauf hin, dass schwache Flexionsformen bei Substantiven in der Peripherie des Form-Schemas stark gefestigt sind und daher starke Formen dominieren, obwohl die starken Formen bereits zu 50~\% als bekannt bewertet werden. 
Im folgenden Abschnitt werden die Ergebnisse der Studien zu Frequenz-, Prototypizität- und Form-Schematizitätseffekten miteinander verglichen und diskutiert.

\section{Diskussion der Ergebnisse}

Die in diesem Abschnitt vorgestellten Untersuchungen hatten zum Ziel, den Einfluss von Frequenz, Prototypizität und Form-Schematizität auf Variation psycholinguistisch zu überprüfen, indem Reaktionszeiten von Stimuli verglichen wurden, die nach Frequenz, Prototypizität sowie Form-Schematizität gestaffelt sind. Zusätzlich zu den Reaktionszeiten wurde das Antwortverhalten der Proband\_innen hinsichtlich der Einflussfaktoren untersucht, um Reaktionszeiten und Antwortverhalten in Kontrast setzen zu können. 


In diesem Abschnitt werden zunächst die Ergebnisse zu Frequenz, dann zu Prototypizität und schließlich zu Form-Schematizität diskutiert und miteinander verglichen. Hierbei wird jeweils anhand der Reaktionszeiten und des Antwortverhaltens überprüft, inwiefern Frequenz, Prototypizität und Form-Schematizität Variation beeinflussen. Der Fokus liegt dabei auf den Reaktionszeiten, da sie das Kerninteresse der Studien spiegeln. Der Abschnitt schließt mit einer Zusammenfassung samt einer Evaluation der Eignung von \textit{lexical decision}, \textit{sentence maze} und \textit{self-paced reading tasks} zur Elizitation von Reaktionszeiten. 

\subsection{Frequenz}\largerpage

Der Einfluss von Frequenz wurde in einer \textit{lexical decision task} mithilfe von starken Verben untersucht, die unterschiedliche Tokenfrequenzen aufweisen. Hierbei wurde zwischen tokenfrequent (\textit{ziehen}, \textit{sprechen})  und tokeninfrequent  (\textit{flechten}, \textit{glimmen}) unterschieden. Die tokeninfrequenten Verben wurden zusätzlich nach attestierter Schwankung in Korpora in zwei Gruppen geteilt: Es wurden infrequente Verben genutzt, die in Korpora noch nicht zwischen starken und schwachen Formen schwanken (\textit{flechten}, \textit{dreschen}), und infrequente Verben, die dies bereits tun (\textit{glimmen}, \textit{sinnen}). Die Proband\_innen beurteilten die starke und die schwache Partizip-II-Form der Verben danach, ob sie die Form kennen (siehe \sectref{metfreq} für ausführliche Erläuterungen zum Design der Studie). Mit diesem Versuchsdesign ließ sich ein deutlicher Einfluss von Frequenz auf Reaktionszeiten feststellen. Frequenz spiegelte sich dabei deutlich stärker als die Einflussfaktoren Prototypizität und Form-Schematizität in den Reaktionszeiten wider. 

\begin{sloppypar}
Starke und schwache Formen tokenfrequenter starker Verben evozieren jeweils geringere Reaktionszeiten als starke und schwache Formen tokeninfrequenter starker Verben, unabhängig davon, ob diese bereits in Korpora schwanken oder nicht. Dieser Unterschied verdeutlicht das \textit{entrenchment} tokenfrequenter Elemente: Die starken Formen frequenter starker Verben sind mental stark gefestigt und daher schnell zu verarbeiten. Der Vorteil für schwache Formen frequenter Verben gegenüber den schwachen Formen infrequenter Verben kann jedoch nicht allein durch Frequenz erklärt werden, denn schwache Formen tokenfrequenter starker Verben werden selten genutzt und sind daher nicht mental gefestigt. Hier greift das statistische Vorkaufsrecht (\cite[74--94]{Goldberg.2019}; siehe \sectref{Statistik}): Da eine tokenfrequente Alternative (starke Form) existiert, kann die selten genutzte Alternative (schwache Form) schnell als unbekannt bewertet werden. Aus demselben Grund werden die schwachen Formen der frequenten Verben langsamer beurteilt als die starken, denn erst der Rückgriff auf die starken Formen erlaubt die Beurteilung der schwachen. Die Reaktionszeiten für die schwachen Formen der frequenten Testverben sind dabei vergleichbar mit den Reaktionszeiten für die starken Formen der anderen Ausprägungen. 
\end{sloppypar}
 
Die schnelle Bewertung von schwachen Formen tokenfrequenter starker Verben im Vergleich zu den schwachen Formen infrequenter Verben verdeutlicht die Relevanz des statistischen Vorkaufsrechts: Fokussiert man nur auf die Frequenz einzelner Formen, wäre zu erwarten, dass kein Vorteil für die schwachen Formen der tokenfrequenten starken Verben besteht, da diese so gut wie nie genutzt werden. Dies ist offensichtlich nicht der Fall. Stattdessen wurden die schwachen Formen ebenfalls schnell beurteilt, da sie aufgrund des statistischen Vorkaufsrechts schnell abgelehnt werden können. 

Die Häufigkeit, mit der schwache Verbformen in Korpora vorkommen, nimmt keinen Einfluss auf die Schnelligkeit der Beurteilung in der Frequenzstudie. Allerdings weisen infrequente starke Verben ohne Schwankung  Unterschiede in den Reaktionszeiten zwischen starken und schwachen Formen auf: Wie bei den tokenfrequenten starken Verben werden auch bei den infrequenten starken Verben ohne Schwankung starke Formen schneller beurteilt als schwache. Dies ist bei den infrequenten starken Verben mit Schwankung nur für wenige Testverben der Fall. Daher ergibt sich bei den infrequenten Verben mit Schwankung kein Unterschied zwischen starken und schwachen Formen.\largerpage

Der Unterschied in den Reaktionszeiten für starke und schwache Formen bei den infrequenten starken Verben ohne Schwankung könnte im unterschiedlichen Grad an \textit{entrenchment} der Formen begründet sein: Die starken Formen sind zwar aufgrund der geringen Tokenfrequenz bereits nicht mehr so stark gefestigt, sodass das statistische Vorkaufsrecht nur noch eingeschränkt greift, aber die schwachen Formen werden noch nicht verwendet, sodass sie mental nicht gefestigt sind und daher höhere Reaktionszeiten hervorrufen als die starken Formen. Bei infrequenten Verben mit Schwankung sind die schwachen Formen hingegen bereits durch Gebrauch gefestigt. Daher ist der Vorteil für starke Formen nicht mehr zu erkennen. Neben dem \textit{entrenchment} könnte die Bewertung der Formen die Reaktionszeiten beeinflussen: Die Bewertung als unbekannte Form scheint längere Reaktionszeiten hervorzurufen als die Bewertung als bekannte Form. Dies war sowohl bei schwachen Formen der infrequenten Verben mit Schwankung als auch bei den starken Formen der Substantive aus der Peripherie des Form-Schemas schwacher Maskulina in der Form-Schematizitätsstudie zu erkennen. Für diese Formen wurde die Frage nach der Bekanntheit zu ungefähr 50 \% bejaht bzw. verneint, wobei die Bewertung als bekannt im Mittel jeweils geringere Reak\-tionszeiten aufwies als die Bewertung als unbekannt. Dass affirmierende Antworten geringere Reak\-tionszeiten auslösen als negierende, ist auch in anderen Studien zu beobachten (\cite[282--283]{Wegge.2001}).

Neben den Reaktionszeiten verdeutlicht das Antwortverhalten in der Studie den Frequenzeffekt: Während bei frequenten Verben schwache Formen als unbekannt bewertet werden, werden sie bei den infrequenten Verben ohne Schwankung vereinzelt als bekannt angesehen und bei Verben mit Schwankung zur Hälfte. Die Beurteilung der starken Formen wird dagegen nicht von der Frequenzausprägung beeinflusst.

Wie die Studie zeigt, sind Variation und Frequenz eng verbunden: Starke Formen von tokenfrequenten Verben sind mental stark gefestigt, was sich einerseits in den niedrigen Re\-ak\-tions\-zei\-ten und andererseits in den hohen Bekanntheitswerten der Formen zeigt. Zudem gibt die geringe Bekanntheit schwacher Formen einen indirekten Hinweis auf das \textit{entrenchment} der starken Formen. Hier ist Variation sehr unwahrscheinlich. Die starken Formen tokeninfrequenter starker Verben sind hingegen weniger stark gefestigt, was sich in den höheren Reaktionszeiten zeigt. Zwar weisen die starken Formen auch bei den infrequenten starken Verben einen hohen Grad an Bekanntheit auf, jedoch sind die schwachen Formen deutlich häufiger bekannt als bei den frequenten starken Verben. Die Studie verdeutlicht somit, dass Variation für tokeninfrequente Verben generell möglich ist. Insgesamt lässt sich der Einfluss der Frequenz somit als ein Zweischritt modellieren: Im ersten Schritt ist relevant, wie frequent ein Verb ist. Die Tokenfrequenz definiert, ob überhaupt Variation möglich ist. Bei geringer Tokenfrequenz können starke Formen die schwachen nicht mehr so leicht dominieren. Dies spiegelt sich in erhöhten Reaktionszeiten für starke Formen bei tokeninfrequenten Verben. Variation ist somit wahrscheinlich. Im zweiten Schritt werden schwache Formen durch Gebrauch gefestigt. Erst hier wird Variation in Korpora sichtbar.



Zwischen den infrequenten Verben mit und ohne Schwankung in Korpora lassen sich Unterschiede im Antwortverhalten feststellen, aber keine Unterschiede in Hinblick auf die Re\-ak\-tions\-zeiten. Die ausbleibenden Unterschiede in den Reaktionszeiten lassen darauf schließen, dass die mentale Repräsentation der starken Formen unabhängig vom Vorkommen in Korpora nicht mehr so stark gefestigt ist wie bei den frequenten starken Verben. Variation ist daher auch bei den Verben ohne attestierte Schwankung bereits angelegt, da das statistische Vorkaufsrecht nur noch eingeschränkt greifen kann. Dennoch scheinen die starken Formen bei den infrequenten Verben ohne Schwankung noch einen kleinen Vorteil gegenüber den schwachen zu haben, der bei den infrequenten Verben mit Schwankung nicht mehr besteht. Der Wegfall des Vorteils für starke Formen deutet darauf hin, dass sich die mentale Repräsentation der schwachen Formen bei den infrequenten Verben mit Schwankung festigt und sich der Repräsentation der starken Formen annähert. Die Variation in der Konjugation ist damit zweifach von Frequenz beeinflusst: Zunächst wirkt das statistische Vorkaufsrecht starker Formen. Wenn dieses durch den Verlust der Tokenfrequenz an Einfluss verliert, wird die Nutzung schwacher Formen ermöglicht. Im nächsten Schritt werden die schwachen Formen durch Gebrauch tokenfrequenter und mental gefestigt, sodass sich die mentalen Repräsentationen beider Formen anpassen. Im folgenden Abschnitt werden die Ergebnisse zum Einfluss der Prototypizität diskutiert.

\subsection{Prototypizität}

 Um den Einfluss der Prototypizität auf Variation zu untersuchen, wurde die Auxiliarselektion von \textit{haben} und \textit{sein} mithilfe einer \textit{sentence maze task} in Sätzen mit unterschiedlichen Transitivitätsgraden untersucht: Transitive Sätze mit einem Objekt, das auf eine belebte Entität referiert (\textit{die Kinder nach Hause fahren}), und intransitive Sätze ohne Akkusativobjekt (\textit{zur Kur fahren}) wurden kontrastiert. Zusätzlich wurde zwischen zwei ambigen Konstruktionen unterschieden. Bei der Ausprägung \textsc{ambig~I} kann das Objekt als Patiens oder Instrument aufgefasst werden (\textit{ein Cabrio fahren}), die Ausprägung \textsc{ambig~II} enthält eine Akkusativergänzung, die ein Adverbial darstellt (\textit{die Strecke fahren}) (siehe \sectref{metproto} für ausführliche Erläuterungen zum methodischen Vorgehen). Die Sätze enthielten jeweils Bewegungsverben (\textit{fahren}, \textit{fliegen}, \textit{reiten}) und waren als Nebensätze konzipiert, sodass das zu wählende Auxiliar direkt auf das Partizip~II der Bewegungsverben folgte (z.~B. \textit{Er sah, wie die Mutter die Kinder zur Schule gefahren hat/*ist}). Im Gegensatz zu den Ergebnissen der Frequenzstudie sind die Ergebnisse der Prototypizitätsstudie weniger eindeutig.

Die intransitiven Sätze evozieren deutlich geringere Reaktionszeiten als die anderen Transitivitätsausprägungen (\textsc{transitiv}, \textsc{ambig~I}, \textsc{ambig~II}). Diese rufen jeweils miteinander vergleichbare Reaktionszeiten hervor. Der visuelle Vergleich zwischen den transitiven Testsätzen und transitiven Fillersätzen ohne Bewegungsverb (\textit{die Kinder zur Schule bringen}) deutet darauf hin, dass die transitiven Testsätze höhere Reaktionszeiten evozieren als die Fillersätze. Dieser Zusammenhang müsste in einer weiteren Studie statistisch überprüft werden. Sollten transitive Sätze mit Bewegungsverben höhere Reaktionszeiten evozieren als transitive Sätze ohne Bewegungsverb, kann dies auf zwei Ursachen zurückgeführt werden: Bewegungsverben werden vorrangig in intransitiven Sätzen genutzt (\cite[267--268]{Gillmann.2016}), sodass die Verwendung von Bewegungsverben in transitiven Sätzen einen höheren Prozessierungsaufwand darstellen könnte. Zudem lassen sich die erhöhten Reaktionszeiten als Einfluss des \textit{sein}-Schemas interpretieren. Bewegungsverben sind aufgrund der häufigen Verwendung in intransitiven Sätzen mit \textit{sein} assoziiert. Nur in transitiven Sätzen ist diese Assoziation gestört und \textit{haben} wird selegiert. Der Wechsel von generell überwiegendem \textit{sein} bei Bewegungsverben zu in transitiven Sätzen genutztem \textit{haben} könnte daher als Erklärung für die erhöhten Reaktionszeiten dienen und auf die Wirksamkeit der beiden Schemata hindeuten. Während sich die Wirksamkeit des \textit{haben}- und des \textit{sein}-Schemas in den Reak\-tionszeiten nur ansatzweise zeigt, ist es im Antwortverhalten deutlich zu erkennen: Transitive Sätze mit Bewegungsverben evozieren \textit{haben}, intransitive \textit{sein}.

Die prototypisch organisierte Verbindung zwischen den Funktionen des \textit{haben}- und \textit{sein}-Schemas lässt sich im Antwortverhalten nur ansatzweise feststellen. Die ambigen Sätze evozieren vornehmlich \textit{sein}-Antworten. Jedoch ist bei \textit{fliegen} und \textit{fahren} ein Unterschied zwischen \textsc{ambig~I} (Objekt ambig zwischen Patiens und Instrument) mit ca. 14 \% \textit{haben}-Antworten  und \textsc{ambig~II} (Akkusativergänzung ist ein Adverbial) mit nur 0,03 \% \textit{haben}-Antworten zu erkennen. Damit tendiert die Ausprägung \textsc{ambig~II}, die nah am intransitiven Pol ist,  mehr zu \textit{sein}-Antworten als die Ausprägung \textsc{ambig~I}, die nah am transitiven Pol ist. 


Zusammen mit dem Antwortverhalten lassen sich die im Vergleich zu den intransitiven Sätzen erhöhten Reaktionszeiten in den ambigen Sätzen interpretieren: Aufgrund der überwiegenden \textit{sein}-Antworten können die Reaktionszeiten nicht durch die Umstellung von \textit{sein} auf \textit{haben} erklärt werden. Zudem kann die Komplexität der Sätze nur eingeschränkt als Auslöser der erhöhten Zeiten gesehen werden, da die transitiven Fillersätze, die eine vergleichbare Komplexität wie die ambigen Sätze aufweisen, geringere Reaktionszeiten evozieren als die ambigen Sätze. Die erhöhten Reaktionszeiten könnten daher auf die Ambiguität der Sätze zurückgehen. Somit lässt sich ein Prototypizitätseffekt zumindest erahnen: Prototypische Vertreter werden schnell verarbeitet, periphere langsamer, da beide Auxiliare möglich sind. Der Prototypizitätseffekt eines Prototyps (Transitivität) wird jedoch durch den anderen Prototyp (Intransitivität und Bewegungsverb) und dessen starke Assoziation mit \textit{sein} überlagert. Da der Unterschied zwischen Test- und Fillersätzen nur visuell beobachtet wurde, muss dies jedoch eine Hypothese bleiben, die in weiteren Studien überprüft werden muss.

In der Studie fällt das Testverb \textit{reiten} auf, da es ein deutlich anderes Antwortverhalten aufweist als \textit{fahren} und \textit{fliegen}. Vor allem die durchgängigen \textit{sein}-Antworten unabhängig vom Transitivitätsgrad der Sätze stechen dabei ins Auge. Die \textit{sein}-Antworten im transitiven Satz könnten ebenfalls ein Prototypizitätseffekt sein, da Pferde keine Menschen sind und somit aufgrund der geringeren Belebtheit nicht prototypisch in der Patiensrolle vorkommen. Allerdings scheint sich \textit{reiten} insgesamt anders zu verhalten als die anderen Bewegungsverben, wie die vergleichsweise vielen \textit{haben}-Antworten in den ambigen Sätzen und die von den anderen Bewegungsverben abweichenden Auxiliarkombinationen zeigen: Anders als bei \textit{fahren} und \textit{fliegen}, bei denen die Kombination aus \textit{haben}-Wahl für transitive Sätze und \textit{sein}-Wahl für alle anderen Ausprägungen deutlich überwiegt, kommt diese Kombination bei \textit{reiten} ähnlich häufig vor wie die durchgängige \textit{sein}-Wahl für alle Ausprägungen und die Kombination aus \textit{haben}-Wahl für \textsc{transitiv} und \textsc{ambig~I} und \textit{sein}-Wahl für \textsc{intransitiv} und \textsc{ambig~II}. 

Der Einfluss von Prototypizität auf Variation lässt sich in der sentence-maze-Studie trotz der oben genannten Einschränkungen erkennen. Die Auxiliare sind bei den prototypischen Funktionen der Schemata (transitiv; intransitiv und Bewegungsverb) mental gefestigt, was sich in den niedrigen Reaktionszeiten und den klar verteilten Antworten zeigt. Die erhöhten Reaktionszeiten für die transitiven Sätze lassen sich mit dem Wechsel von einem zum anderen Prototyp erklären. Im Übergangsbereich zwischen transitiv und intransitiv sind die Auxiliare nicht so stark gefestigt, was sich in höheren Reaktionszeiten und einer leicht erhöhten Variation in den Antworten zeigt. Die Variation in den Antworten ist allerdings weniger stark als angenommen, da in den ambigen Sätzen \textit{sein} als Defaultauxiliar für Bewegungsverben zu greifen scheint. Die durchgängigen \textit{sein}-Antworten bei \textit{reiten} sowie die erhöhten Rea\-ktionszeiten in den transitiven Sätzen könnten darauf hindeuten, dass sich die \textit{sein}-Selektion bei Bewegungsverben auf transitive Kontexte ausweiten wird. Dies muss zum gegenwärtigen Zeitpunkt jedoch eine kaum überprüfbare Hypothese bleiben. Im folgenden Abschnitt werden die Ergebnisse zum Einfluss der Form-Schematizität diskutiert.  

\subsection{Form-Schematizität}

Der Einfluss der Form-Schematizität auf Variation wurde innerhalb einer \textit{self-paced reading task} mit anschließendem Produktionsexperiment, einer \textit{lexical decision task} und einer \textit{sentence maze task} mithilfe von schwachen und starken Maskulina untersucht. Dabei wurde in der self-paced-reading-Studie mit Pseudosubstantiven gearbeitet, in den anderen Studien wurden real existierende Substantive genutzt. Die Versuchitems gehörten entweder zum Prototyp des Form-Schemas schwacher Maskulina (\textit{Kollege}, \textit{Schettose}), waren in der Peripherie des Form-Schemas (\textit{Graf}, \textit{Knatt}) oder gehörten der starken Flexion an (\textit{Vogt}, \textit{Grettel}). In der self-paced-reading-Studie lasen die Proband\_innen starke und schwache Formen der Testsubstantive, in der lexical-decision-Studie bewerteten sie starke und schwache Formen der Testsubstantive danach, ob ihnen die Form bekannt ist, und in der sentence-maze-Studie wählten sie zwischen der starken und der schwachen Form der Testsubstantive (siehe \sectref{methschema} und \ref{schemalex} für ausführliche Erläuterungen zum methodischen Vorgehen). Anders als in der Frequenz- und der Prototypizitätsstudie lässt sich der Einfluss der Form-Schematizität in den Reaktionszeiten allenfalls erahnen. 

Lediglich in der self-paced-reading-Studie zeigt sich ein kleiner Form-Sche\-ma\-ti\-zi\-täts\-effekt für das Testitem \textit{Schettose}: Hier sind die Lesezeiten für die schwache Form etwas niedriger als die Lesezeiten für die starke Form. Dies ist aufgrund der hohen Passgenauigkeit von \textit{Schettose} zum Form-Schema schwacher Maskulina auch zu erwarten. Es ist aber fraglich, ob dieser Effekt systematisch ist. Wenn er systematisch ist, stellt sich die Frage, warum kein gegenteiliger Effekt bei \textit{Grettel} zu erkennen ist, bei dem aufgrund des Form-Schemas starker Maskulina nur starke Formen zu erwarten sind. In den anderen Studien zeigte sich gar kein Einfluss der Form-Schematizität auf Reaktionszeiten. Das Antwortverhalten wurde dagegen in allen Studien deutlich durch Form-Schematizität beeinflusst: Für die prototypischen Vertreter des Form-Schemas schwacher Maskulina wurden jeweils schwache Formen bevorzugt, für starke Maskulina jeweils starke. Variation war für die Peripherie festzustellen. Nur für die sentence-maze-Studie war das aufgrund der Gegenüberstellung starker und schwacher Formen nicht der Fall: Die stärker gefestigte Form (schwach) kann durch die Gegenüberstellung die weniger gefestigte (stark) leichter statistisch ausstechen und wird daher bevorzugt.\footnote{Dieses Problem stellte sich für die Prototypizitätsstudie nicht, da \textit{haben} und \textit{sein} jeweils nach den Bewegungsverben präsentiert wurden und die Auxiliare daher erst mit den Bewegungsverben verknüpft werden mussten, sodass die stark gefestigte Form nicht der weniger bzw. nicht gefestigten gegenüberstand.}  



Form-Schematizität nimmt somit Einfluss auf Variation: Die prototypischen Vertreter des Form-Schemas schwacher Maskulina sowie Substantive, die dem Form-Schema nicht angehören, evozieren jeweils ein klar verteiltes Antwortverhalten. Variation ist in der Peripherie des Form-Schemas zu beobachten. Das Antwortverhalten in den Form-Schematizitätsstudien verdeutlicht damit nicht nur den Einfluss von Form-Schemata, sondern auch von Prototypizität auf Variation. Es stellt sich daher die Frage, warum sich das Antwortverhalten nicht in den Reaktionszeiten spiegelt. 

Eine Erklärung für die Diskrepanz zwischen Antwortverhalten und Reaktionszeiten, die auch \textcite[171--172]{Schmitt.2019} diskutiert, ist, dass die Pro\-\mbox{band\_in}\-nen die Endungen der Substantive nicht wahrnehmen. Dies ist für die vorliegenden Studien jedoch auszuschließen, da in der self-paced-reading-Studie der Aufbau der zu lesenden Wörter abgefragt wurde (Haben Sie \textit{des Schettosen} oder \textit{des Schettoses} gelesen?) und in der sentence-maze-Studie zwischen beiden Formen gewählt wurde. In der lexical-decision-Studie zeigt sich die Sensibilität für starke und schwache Formen einerseits im Antwortverhalten und andererseits hinsichtlich der Reaktionszeiten. Die Reaktionszeiten interagieren in Hinblick auf Deklinationsklasse und die Flexionsform: Schwache Formen von schwachen Maskulina werden schnell beurteilt und starke langsam, das Gegenteil ist für starke Maskulina der Fall.\footnote{Auch wenn die Interaktion vorrangig an der Ablehnung oder Zustimmung hinsichtlich der Bekanntheit von Formen hängen sollte, würde sie implizit auf die Sensibilität für Formen hinweisen, da das Antwortverhalten von der Deklinationsklassenzugehörigkeit abhängt.} Dies deutet auf Sensibilität für Formen hin und zeigt, dass Form-Schematizität nicht nur für die Produktion von Formen relevant ist, sondern auch für die Prozessierung, zumindest in dem Umfang, in dem die Methode Einblick in Prozessierung gewährt. 



Hinsichtlich der self-paced-reading-Studie ist es zudem möglich, dass die Pseudosubstantive das Ergebnis beeinflusst haben, da hier keine der Formen durch Gebrauch mental gefestigt ist. Ein Unterschied in den Reaktionszeiten kann nur durch Rückgriff auf die Form-Schemata schwacher und starker Maskulina zustande kommen und ist daher eventuell sehr subtil. Wie anhand der Antworten im Produktionsexperiment zu erkennen ist, wählten viele Proband\_innen Nullendungen für das Substantiv \textit{Grettel}, sodass es möglich ist, dass eine Nullendung in diesem Fall die am leichtesten zu prozessierende Option gewesen wäre. Diese Erklärung greift nicht für die anderen Studien, da hier mit real existierenden Substantiven gearbeitet wurde und die Formen der jeweiligen Deklinationsklasse daher durch Gebrauch gefestigt sind. Dies wird auch durch das Antwortverhalten in den Studien sowie durch die Interaktion in den Reaktionszeiten zwischen Flexionsform und Deklinationsklasse in der lexical-decision-Studie deutlich. Die Reaktionszeiten spiegeln das Antwortverhalten zumindest in dieser Hinsicht.

Die Diskrepanz zwischen Reaktionszeiten und Antwortverhalten lässt abseits der methodischen Überlegungen auch theoretische Schlüsse zu. So legt die lex\-i\-cal-de\-ci\-sion-Stu\-die nahe, dass die starken Formen von Substantiven in der Peripherie des Form-Schemas trotz der Variation im Antwortverhalten nicht stark gefestigt sind: Zwar liegt die Bekanntheit der starken Formen für Testsubstantive in der Peripherie des Form-Schemas bei 50~\%. Dies spiegelt sich aber nicht in den Reaktionszeiten.\footnote{Im Fließtext wird hinsichtlich der lexical-decision-Studie nur eine mögliche Angleichung der Reaktionszeiten für starke und schwache Formen bei Substantiven aus der Peripherie des Form-Schemas diskutiert. Neben dieser Möglichkeit wäre auch ein generell langsameres Antwortverhalten für starke und schwache Formen der Substantive in der Peripherie im Vergleich zu den anderen Ausprägungen denkbar. Dies wurde auch ursprünglich erwartet. Im Gegensatz zu den anderen Ausprägungen existieren in der Peripherie zwei Formen, die (zwar unterschiedlich stark, aber dennoch beide) mental gefestigt sind und daher beide bekannt sein können. Die Schnelligkeit, mit der eine Form beurteilt wird, scheint jedoch nicht durch die theoretische Existenz einer anderen Form beeinflusst worden zu sein.} Das statistische Vorkaufsrecht für die schwachen Formen scheint bereits eingeschränkt zu sein, wie das Antwortverhalten zeigt, die neuen starken Formen scheinen jedoch noch nicht genug gefestigt zu sein, um  Reaktionszeiten hervorzurufen, die mit den schwachen Formen vergleichbar sind. In der Frequenzstudie weisen die infrequenten Verben mit Schwankung in Korpora ein vergleichbares Antwortverhalten wie die Substantive aus der Peripherie des Form-Schemas in der lexical-decision-Studie zu Form-Schematizität auf. Allerdings ist für die infrequenten Verben mit Schwankung eine Angleichung der Reaktionszeiten zwischen starken und schwachen Formen zu erkennen, was für die Substantive in der Peripherie nicht der Fall ist. Auch dies deutet darauf hin, dass die starken Formen von Substantiven in der Peripherie des Form-Schemas vergleichsweise wenig gefestigt sind. 



Da die Testsubstantive in der Peripherie anders als die infrequenten Verben mit Schwankung in Korpora nicht danach ausgewählt wurden, ob sie bereits Schwankungen in Korpora aufweisen, könnte es sein, dass die starken Formen von Substantiven in der Peripherie tatsächlich im Sprachgebrauch zu selten vorkommen, um ähnlich gefestigt zu sein wie die schwachen Formen von infrequenten starken Verben mit Schwankung in Korpora. Die dennoch zu beobachtende hohe Bekanntheit der starken Formen von Substantiven aus der Peripherie des Form-Schemas lässt sich vor diesem Hintergrund mit Rückgriff auf die Relevanz der Kategorien \textsc{Tempus} und \textsc{Kasus} erklären: Kasus ist weniger relevant als Tempus, da das Tempus die Bedeutung eines Verbs stark verändert, während der Kasus nur anzeigt, wie Entitäten zueinander stehen und daher die Semantik eines Substantivs nicht beeinflusst \parencites[179--180]{Nubling.2004}[158--159]{Nubling.2016}. Aufgrund der geringeren Relevanz ist Variation für die Kategorie \textsc{Kasus} wahrscheinlicher, sodass die starken Formen von Substantiven in der Peripherie des Form-Schemas trotz des geringen \textit{entrenchments} bereits als bekannt bewertet werden. Zudem ist die Genitivmarkierung mit -\textit{(e)s} vs. -\textit{(e)n} sehr viel subtiler als der Ablaut. 

 
Um das Zusammenspiel aus \textit{entrenchment} und Relevanz näher zu untersuchen, wäre es für eine weitere Studie interessant, analog zu den Verben mit und ohne Schwankung in Korpora die Testsubstantive in der Peripherie des Form-Schemas nach ihrer Ratio von schwachen zu starken Formen in Substantive mit und ohne Schwankung in Korpora zu teilen. Hierbei wäre aufgrund der geringen Relevanz der Kategorie \textsc{Kasus} unabhängig von der attestierten Schwankung in Korpora eine Akzeptanz der starken Formen von ca. 50 \% zu erwarten und für Substantive mit Schwankung in Korpora aufgrund des höheren \textit{entrenchments} starker Formen eine Angleichung der Reaktionszeiten für starke und schwache Formen.

In der sentence-maze-Studie könnte Tokenfrequenz eine Erklärung für die Diskrepanz zwischen Reaktionszeit und Antwortverhalten sein. Ursprünglich wurde angenommen, dass periphere Vertreter des Form-Schemas in der sentence-maze-Studie höhere Reaktionszeiten hervorrufen als prototypische Vertreter des Form-Schemas sowie starke Maskulina, weil in der Peripherie zwischen zwei mental gefestigten Formen gewählt werden muss und die Wahl daher verlangsamt sein könnte. Dies ist in den Reaktionszeiten jedoch nicht zu erkennen. Die Substantive in der Peripherie neigen in der sentence-maze-Studie sogar zu geringeren Reaktionszeiten als die Substantive der anderen Ausprägungen, wobei dieser Unterschied zufällig sein kann. Die unterschiedlichen Tokenfrequenzen der Testsubstantive könnten als Erklärung für die niedrigeren Reaktionszeiten der Testsubstantive aus der Peripherie des Form-Schemas dienen. Zwei der drei Testsubstantive aus der Peripherie des Form-Schemas zählen zu den tokenfrequenten Testsubstantiven in der Studie. Die tokenfrequenten Testsubstantive evozieren geringere Reaktionszeiten als die mittel- und infrequenten, jedoch kann auch dieser Unterschied zufällig sein. Sollte der Effekt systematisch sein, wäre es möglich, dass Tokenfrequenz den Form-Schematizitätseffekt überschreibt und Substantive in der Peripherie aufgrund der höheren Tokenfrequenz  etwas schneller verarbeitet werden als andere Substantive. Ist dies der Fall, könnte man den Schluss ziehen, dass Form-Schematizität die Deklinationsklassenzugehörigkeit grundlegend steuert und periphere Vertreter durch Tokenfrequenz in der schwachen Klasse gehalten werden: Da tokenfrequente Formen immer wieder aktiviert und somit mental gefestigt werden, bleibt die Flexion trotz der Peripherie des Form-Schemas schwach. Dies muss jedoch eine Hypothese bleiben, da die vergleichsweise niedrigen Reaktionszeiten sowohl bei den frequenten Substantiven als auch bei den Substantiven aus der Peripherie zufällig sein könnten. 

Insgesamt zeigt sich hinsichtlich der Form-Schematizität ein zweigeteiltes Ergebnis: Das Form-Schema beeinflusst das Antwortverhalten und in der Peripherie des Form-Schemas ist Variation im Antwortverhalten zu erkennen. Die Reaktionszeiten spiegeln den Einfluss des Form-Schemas hingegen nicht. Es kann sein, dass Tokenfrequenzeffekte die Form-Schematizität überlagern. Dies müsste jedoch in weiteren Studien überprüft werden. Zudem ist es möglich, dass die in den Studien genutzten Verfahren zur Elizitation von Reaktionszeitunterschieden sich nicht für die Messung von Form-Schematizitätseffekten eignen. Im nächsten Abschnitt wird daher neben einer Zusammenfassung der Ergebnisse eine Evaluation der Reaktionszeitmessverfahren in Bezug auf alle Studien vorgenommen.

\subsection{Zusammenfassung und Evaluation der Reaktionszeitmessverfahren}

Der Einfluss von Frequenz, Prototypizität und (Form-)Schematizität auf Variation lässt sich im Antwortverhalten klar beobachten. In den Reaktionszeiten zeigt sich der Einfluss hingegen nur teilweise: Der Einfluss der Frequenz ist sehr deutlich zu erkennen, ein Einfluss von Schemata ist für \textit{haben} und \textit{sein} ebenfalls festzustellen. Die erhöhten Reaktionszeiten in den ambigen Sätzen lassen sich zudem als Effekt der Prototypizität sehen, der jedoch nur teilweise im Antwortverhalten gespiegelt wird, da auch die ambigen Sätze stark zur Selektion von \textit{sein} neigen. Form-Schematizität zeigt sich hingegen überhaupt nicht in den Reaktionszeiten. 

Es ist möglich, dass die gemessenen Reaktionszeiten zu ungenau sind, um Effekte von Form-Schematizität in der Prozessierung beobachten zu können. Generell zeigen die Ergebnisse der Frequenz- und der Prototypizitätsstudie, dass \textit{lexical decision} und \textit{sentence maze tasks} geeignet sind, um Unterschiede in Reaktionszeiten herauszuarbeiten.  Es stellt sich aber die Frage, ob \textit{self-paced reading} zweckmäßig ist, um  Prozessierungsunterschiede zwischen Deklinationsformen zu messen. In Hinblick auf salientere Variationsfälle scheint \textit{self-paced reading} geeignet zu sein, wie \textcite{Schmitt.2019b} anhand der Kasusselektion von \textit{wegen} zeigt, jedoch könnten die Unterschiede zwischen den Deklinationsklassen zu subtil für die Methode sein. Dies könnte auch darin begründet sein, dass der Kasus ein Kongruenzphänomen ist: In den Form-Schematizitätsstudien war die Funktion \textsc{Genitiv} immer schon durch den Determinierer markiert, sodass die starken und schwachen Endungen der Substantive keine so große Rolle für die Prozessierung spielen könnten. Zwar sind die Proband\_innen offensichtlich sensitiv für die Genitivendungen der Substantive, wie das Antwortverhalten zeigt, dennoch könnten die Endungen in der Prozessierung nur eine geringe Rolle spielen. Die durch die Endungen hervorgerufenen Unterschiede in der Prozessierung könnten daher sehr subtil sein, da die Endungen nicht eigenständig die Funktion \textsc{Genitiv} anzeigen. 



Zusätzlich dazu ist für die \textit{sentence maze task} festzuhalten, dass diese für die Erforschung von Variation in der Flexion weniger gut geeignet zu sein scheint, da mental stark gefestigte und weniger stark gefestigte Formen gleichzeitig präsentiert werden. Für alle Messmethoden lassen die ausbleibenden Reaktionszeitunterschiede in Hinblick auf Form-Schematizität bei gleichzeitiger Beeinflussung des Antwortverhaltens darauf schließen, dass Unterschiede in der Prozessierung existieren, die gemessenen Reaktionszeiten jedoch zu wenig sensitiv für diese sind. Hier wäre es sinnvoll, mit feineren Messmethoden wie bspw. EEG zu arbeiten. 

\begin{sloppypar}
Die Ergebnisse der Frequenz- und der Prototypizitätsstudie verdeutlichen trotz der genannten Einschränkungen das Potential von Reaktionszeitmessungen für die Erforschung grammatischer Varianten. So zeigt die Prototypizitätsstudie, dass Unterschiede in der Verarbeitung von Prototypen existieren und die Frequenzstudie verdeutlicht, dass Reaktionszeiten Hinweise auf Variation liefern können, die noch nicht in der Sprachproduktion sichtbar ist. Im nächsten Kapitel wird anhand der Kernergebnisse der Dissertation ein Fazit gezogen und ein Ausblick gegeben.
\end{sloppypar}
