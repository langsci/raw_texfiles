\chapter{Einleitung und Zielsetzung}

Sprache und Variation sind eng miteinander verwoben -- bzw. verwebt. Dies verdeutlichen die verschiedenen sprachlichen Ebenen, auf denen Variation zu beobachten ist: Varianten existieren in der Phonologie ([\c{c}eː miː] vs. [ʃ eː miː] vs. [keː miː]), in der Morphologie (\textit{geglommen} vs. \textit{geglimmt}; \textit{des Grafen} vs. \textit{des Grafs}), in der Syntax (\textit{des Vaters Haus} vs. \textit{das Haus des Vaters}), an der Syntax-Semantik-Schnittstelle (\textit{ein Auto gefahren sein} vs. \textit{ein Auto gefahren haben}) und der Morphosyntax (\textit{wegen des} vs. \textit{wegen dem}). Wo Sprache ist, ist auch Variation.

Aber nicht alles variiert, was variieren könnte: Zwar schwankt \textit{glimmen} zwischen \textit{geglommen} und \textit{geglimmt}, aber \textit{ziehen} nicht zwischen \textit{gezogen} und *\textit{gezieht}. Genauso schwankt \textit{Graf} zwischen \textit{des Grafen} und \textit{des Grafs}, aber \textit{Matrose} nicht zwischen \textit{des Matrosen} und \textit{des *Matroses}. Und während \textit{Auto fahren} sowohl  \textit{sein} als auch \textit{haben} im Perfekt erlaubt, ist bei \textit{die Kinder nach Hause fahren} nur \textit{haben} möglich und bei \textit{zur Kur fahren} nur \textit{sein}.
Die vorliegende Arbeit entwickelt ein Modell, das Vorhersagen darüber ermöglicht, unter welchen Umständen Variation auftritt und unter welchen sie unwahrscheinlich ist. 

Zu Variation im Deutschen existieren zahlreiche Studien, die den Sprachgebrauch in Hinblick auf Variation bspw. in der Flexion und Auxiliarselektion synchron und diachron in Korpora untersuchen (z.~B. \cite{Nowak.2015}, \cite{Gillmann.2016}, \cite{Schafer.2019}) oder mithilfe von Produktionsexperimenten überprüfen, welche Faktoren die Variation beeinflussen (z.~B. \cite{Kopcke.2000b, Kopcke.2005}). Diese fokussieren jedoch jeweils auf einzelne Variationsphänomene und erklären diese post-hoc. Es fehlt daher ein umfassendes Modell, das Variation und deren Entstehung übergreifend modelliert und es damit ermöglicht, Variation zu prognostizieren. Zudem sind bisherige Studien zu Variation derzeit stark auf Korpora fokussiert. Zwar wurde zur Prozessierung von ambigen Strukturen im Vergleich zu eindeutigen Strukturen vereinzelt geforscht, wie bspw. \textcite{Roehm.2013} zur Prozessierung der Auxiliare \textit{haben} und \textit{sein} bei \textit{degree achievements} (\textit{Der Pfirsich ist/hat geschimmelt}) und telischen Verben (\textit{Der Pfirsch ist/*hat verschimmelt}), um die Entstehung von Variation umfassend beleuchten zu können, sollte aber die Verarbeitung von Varianten und von Formen, die nicht im Gebrauch schwanken, stärker in den Fokus gerückt werden. Dies hat die vorliegende Arbeit zum Ziel.

In der Arbeit wird ein gebrauchsbasiertes Modell entwickelt, das psycholinguistisch überprüft wird. Die Arbeit verfolgt damit einen dezidiert gebrauchsbasierten Ansatz. In dem gebrauchsbasierten Modell zu Variation werden drei Faktoren ausgearbeitet, die Einfluss darauf nehmen, ob Variation (z.~B. das Verb \textit{ziehen} oder die Konstruktion \textit{Auto fahren}) auftritt oder nicht. Der Einfluss der Faktoren wird psycholinguistisch mithilfe von Reaktionszeitmessungen überprüft. 


Ein solches Modell ermöglicht es, Variation unabhängig von einzelnen Variationsphänomenen zu betrachten und Variation nicht nur post hoc zu erklären, sondern zu prognostizieren, wie wahrscheinlich es ist, dass Variation auftreten wird. Die empirische Überprüfung mithilfe von Reaktionszeiten ist ein wichtiger Bestandteil des Modells, da Reaktionszeiten  Hinweise auf Variationspotential liefern können, das noch nicht im Sprachgebrauch sichtbar ist. Dies ist bspw. der Fall, wenn im Sprachgebrauch existierende Varianten ähnlich schnell verarbeitet werden wie Formen, die im Sprachgebrauch (noch) keine Variation aufweisen. Die Betrachtung der Prozessierung von Varianten zusätzlich zu ihrer Verwendung ermöglicht somit einen umfassenden Blick auf Variation. Die vorliegende Arbeit ergänzt und erweitert die bestehende Forschung durch den Fokus auf Faktoren, die variationsphänomenübergreifend Einfluss auf Variation nehmen, und die systematische psycholinguistische Überprüfung der Faktoren. Im Rahmen der psycholinguistischen Überprüfung der Faktoren wird ein direkter Vergleich von Varianten im Sprachgebrauch (\textit{geglimmt} vs. \textit{geglommen}) und Formen vorgenommen, die keine Variation aufweisen (\textit{gezogen} vs. *\textit{gezieht}). Hierdurch ermöglicht die Arbeit es, die Prozessierung von Varianten besser zu verstehen. Die Arbeit greift auf drei verschiedene Verfahren zur Reaktionszeitszeitmessung zurück (\textit{self-paced reading}, \textit{lexical decision} und \textit{sentence maze tasks}) und bietet somit einen vielfältigen Zugriff auf Reaktionszeiten.


Die Arbeit nutzt den gebrauchsbasierten Ansatz als Grundlage für die Modellierung von Variation. Der gebrauchsbasierte Ansatz sieht Sprache als ein wandelbares und kontextabhängiges Geflecht aus Strukturen, das durch Nutzung entsteht und gefestigt wird. Die kognitive Repräsentation sprachlichen Wissens ist in dem Ansatz daher stets auf Sprachgebrauch zurückzuführen: "`The basic premise of Usage-based Theory is that experience with language creates and impacts the cognitive representations for language"' (\cite[49]{Bybee.2013}). Die Struktur von Sprachsystemen entsteht also durch Gebrauch und ist nicht a priori festgelegt (\cite[714]{Bybee.2006b}). Grammatik lässt sich dabei mit mentalen Repräsentationen im Gehirn und Prozessierung gleichsetzen (\cites[2--3]{Bybee.2001}[xii]{Kemmer.2000}). Mentale Repräsentationen -- und damit auch Grammatik~-- verändern und stabilisieren sich ständig durch Gebrauch \parencites[2]{Bybee.2001}[68]{Bybee.2013} und stellen somit "`provisional and temporary states of affair"' (\cite[2]{Bybee.2001}) dar.  In der gebrauchsbasierten Perspektive kann ein Sprachsystem daher nur existieren, wenn es immer wieder genutzt wird, und durch die Nutzung wird es ständig verändert. Der gebrauchsbasierte Ansatz eignet sich zur Modellierung von Variation, da er mit seinem Fokus auf die Veränderung von mentalen Repräsentationen durch Nutzung Sprache eng an Variation koppelt. Zudem verdeutlicht der Ansatz die Relevanz, die Prozessierung von Varianten zu erforschen, da sie Rückschlüsse auf die mentalen Repräsentationen der Varianten liefern kann. 

Frequenz, Prototyp und Schema werden innerhalb des Modells als grundlegende kognitive Faktoren entwickelt, die Einfluss auf Variation in einer Sprache nehmen. Sie werden anhand von drei Variationsphänomenen in den Blick genommen, die sich zur Untersuchung der Einflussfaktoren eignen: Variation in der Konjugation zwischen starken und schwachen Verben (\textit{geglommen}/\textit{geglimmt}),  Variation in der Deklination zwischen schwachen und starken Maskulina (\textit{des Bären}/\textit{Bärs}) und Variation in der Auxiliarselektion von \textit{haben} und \textit{sein} (\textit{das Auto gefahren haben}/\textit{sein}).  Mit diesem Zuschnitt fokussiert die Arbeit auf morphologische und semantisch-syntaktische Variation. Die Faktoren, ihr Einfluss auf Variation und ihre Verknüpfung mit den Variationsphänomenen werden in \sectref{faktoren} näher erläutert. 

Auf empirischer Ebene schließt sich die Arbeit dem \textit{open science movement} an (\cite[257--292]{Cumming.2017}): Für alle empirischen Studien bis auf eine\footnote{Die self-paced-reading-Studie zu Form-Schematizitätseffekten wurde nicht registriert. Sie wurde vor den anderen Studien durchgeführt. In der self-paced-reading-Studie wurde die Notwendigkeit für eine Präregistrierung inklusive einer Prästudie für die weiteren Studien deutlich.} wurden die Hypothesen, das Untersuchungsdesign und die statistische Auswertung vorab online registriert. Im Rahmen der Präregistrierung wurden Prästudien durchgeführt und anhand der Prästudien eine Datensimulation vorgenommen, um sicherzustellen, dass die für die Hauptstudie angestrebte Stichprobengröße ausreichend ist, um die Wahrscheinlichkeit für ein zufällig negatives Ergebnis möglichst gering zu halten. Außerdem werden die erhobenen Daten und die Skripte, mit denen die Daten aufbereitet und ausgewertet wurden, zur Verfügung gestellt. Die Nutzung von frei zugänglichen Programmen für die Datenerhebung und -auswertung erleichtert zudem eine Replikation der Studien und führt dazu, dass die Datenauswertung im Detail nachvollzogen werden kann. Im Folgenden wird auf die in dieser Arbeit modellierten kognitiven Einflussfaktoren auf Variation -- Frequenz, Prototyp und Schema -- eingegangen und die Gliederung der Arbeit erläutert.

\section{Frequenz, Prototyp und Schema als Einflussfaktoren auf Variation}
\label{faktoren}\largerpage[-1]

Frequenz, Prototyp und Schema sind zentrale Einflussfaktoren auf Variation und Stabilität in einem Sprachsystem. In Hinblick auf Frequenz ist zunächst festzustellen, dass hohe Tokenfrequenz zu gefestigten und leicht aktivierbaren Formen und damit zu Stabilität führt (\cite[380]{Bybee.1997}). Der Einfluss von Frequenz auf Variation ist aber nicht nur auf Tokenfrequenz beschränkt. Typenfrequenz beeinflusst Variation ebenfalls. Dabei ist bei Flexionsklassen ein Zusammenspiel aus Typen- und Tokenfrequenz zu beobachten, das sich anhand der Variation in der Konjugation gut veranschaulichen lässt: Schwache Verben weisen als typenfrequente Flexionsklasse viele und sehr unterschiedliche Mitglieder auf. Die Merkmale der Flexionsklasse treffen deswegen auf viele Verben zu und können daher abstrahiert werden (\cites[384]{Bybee.1997}[63--67]{Goldberg.2019}): Wenn -\textit{t} an einen Infinitivstamm gehängt wird, setzt man ein Verb in die Vergangenheit (\textit{lachte}). Bei typeninfrequenten Klassen wie den starken Verben kann das Verhalten dagegen nur eingeschränkt abstrahiert werden, da es nur wenige Mitglieder gibt, die sich zudem nicht uniform verhalten. \textit{Geben} wird bspw. durch den Ablaut /aː/ in die Vergangenheit gesetzt (\textit{gab}), aber \textit{heben} durch /oː/ (\textit{hob}). Theoretisch sind bei \textit{geben} und \textit{heben} auch schwache Formen möglich (*\textit{gebte}, *\textit{hebte}). Dies verdeutlichen bspw. Übergeneralisierungen der schwachen Formen bei Kindern (\cite[219--220]{Rumelhart.1986}). Die Option der schwachen Formen (*\textit{gebte}, *\textit{hebte}) wird aber abseits von Übergeneralisierungen nicht genutzt (\cite[75--77]{Goldberg.2019}), da mit \textit{gab} und \textit{hob} andere, häufigere Formen existieren. Hier zeigt sich der Einfluss von Tokenfrequenz: Nur wenn ein Verb aus einer typeninfrequenten Klasse tokenfrequent ist (wie bspw. \textit{geben}), sind auch seine starken Formen (z.~B. \textit{gab}, \textit{gegeben}) häufig. Es existieren somit sehr oft genutzte Formen, sodass die schwachen Formen (*\textit{gebte}, *\textit{gegebt}) statistisch ausgestochen werden und deren Verwendung daher unwahrscheinlich ist (\cite[83--84]{Goldberg.2019}). Wird ein Verb dagegen selten genutzt wie bspw. \textit{glimmen}, sind auch seine starken Formen (\textit{glomm}, \textit{geglommen}) nicht häufig. Die schwachen Formen (\textit{glimmte}, \textit{geglimmt}) werden daher weniger stark dominiert. In der Folge weist das tokeninfrequente Verb starke sowie schwache Formen (\textit{geglommen}, \textit{geglimmt}) auf und wechselt schließlich die Konjugationsklasse (\cite[267--268]{Augst.1975}). Typenfrequenz beeinflusst daher die Schwankungsrichtung (von typeninfrequent zu typenfrequent). Hohe Tokenfrequenz führt zu Stabilität in der typeninfrequenten Klasse und niedrige Tokenfrequenz zu Variation.

Um den Einfluss der Frequenz auf Variation psycholinguistisch zu überprüfen, werden im empirischen Teil der Arbeit starke Verben mit unterschiedlicher Tokenfrequenz in starker (\textit{geglommen}, \textit{geflogen}) und schwacher Flexion (\textit{geglimmt}, *\textit{gefliegt}) präsentiert und die Reaktionszeiten auf diese Formen gemessen. Die Verben unterteilen sich dabei in drei Gruppen: tokenfrequente Verben (\textit{fliegen}), tokeninfrequente Verben mit Schwankung in Korpora (\textit{glimmen}) und tokeninfrequente Verben ohne Schwankung in Korpora (\textit{fechten}). Auf diese Weise kann der Einfluss von Tokenfrequenz auf die Verarbeitung gemessen werden und gleichzeitig überprüft werden, ob sich Variationspotential in der Prozessierung niederschlägt, bevor Schwankungen beobachtet werden können.

Der Einfluss von Prototypen auf Variation lässt sich anhand der Auxiliarselektion von \textit{haben} und \textit{sein} gut umreißen. Sätze mit einem Bewegungsverb (z.~B. \textit{fahren}) weisen je nach Transitivitätsgrad unterschiedliche Auxiliare auf (\cite[316--319]{Gillmann.2016}): Prototypisch transitive Sätze mit einem Objekt, das als Patiens fungiert (\textit{die Kinder zur Schule fahren}), selegieren \textit{haben} als Auxiliar (\textit{Ich habe/*bin die Kinder zur Schule gefahren}). Prototypisch intransitive Sätze ohne Objekt (\textit{zur Kur fahren}) weisen dagegen \textit{sein} auf (\textit{Ich bin/*habe zur Kur gefahren}). Schwankungen zwischen den Auxiliaren treten nur in Sätzen auf, die im Übergangsbereich zwischen Transitivität und Intransitivität sind. Im Satz \textit{Ich bin/habe das Auto gefahren} kann das Objekt als Patiens gesehen werden, da das Auto der Handlung ausgesetzt ist, aber auch als Instrument, da das Auto die Handlung erst ermöglicht (\cite[251]{Gillmann.2016}). Da prototypisch transitive Sätze ein Patiens enthalten, führt die Interpretation als Patiens zu hoher Transitivität, die Interpretation als Instrument dagegen zu geringerer Transitivität. Durch die Ambiguität zwischen Patiens und Instrument ist \textit{das Auto fahren} im Übergangsbereich zwischen Transitivität und Intransitivität, weswegen beide Auxiliare (\textit{Ich bin/habe das Auto gefahren}) möglich sind. Prototypische Vertreter zeigen somit ein stabiles Verhalten, während im Übergangsbereich und damit jeweils in der Peripherie der Funktionen Variation auftreten kann.

In Bezug auf Variation in der Flexion ermöglicht Prototypizität außerdem, Flexionsklassen als prototypisch organisiert zu betrachten. Anhand von Flexionseigenschaften können prototypisch starke und prototypisch schwache Vertreter einer Flexionsklasse ermittelt werden. Dies lässt sich anhand der starken und schwachen Verben verdeutlichen: \textit{Geben} ist prototypisch stark, da es alle Eigenschaften starker Verben (Imperativhebung \textit{gib}, Wechselflexion im Präsens \textit{er/sie gibt}, Ablaut \textit{gab}, \textit{gegeben}) vereint (\cites[57]{Bittner.1985}[157]{Nowak.2013}). \textit{Sagen} ist prototypisch schwach, da es keine Imperativhebung und keine Wechselflexion aufweist und Präterital- sowie Partizip-II-Formen mit Dentalsuffix bildet (\textit{sagte}, \textit{gesagt}). Ein Verb wie \textit{mahlen} ist ein peripheres Mitglied der starken Flexion, da es nur das Partizip~II stark bildet (\textit{gemahlen}). Nicht nur die Mitglieder der starken Konjugationsklasse lassen sich als prototypisch und peripher beschreiben, sondern auch die Flexionseigenschaften der starken Konjugation. Der Ablaut im Perfekt ist eine prototypische Eigenschaft starker Flexion, da alle starken Verben sie aufweisen, die Imperativhebung hingegen eine periphere, da nur wenige starke Verben diese Flexionseigenschaft zeigen.  Diese Prototypizitätsskala ermöglicht Vorhersagen über Variation: Periphere Eigenschaften wie die Imperativhebung werden zuerst aufgegeben, dann folgen prototypische wie der Ablaut (\cite[57]{Bittner.1985}).

Der Einfluss der Prototypizität auf Variation wird im empirischen Teil der Arbeit anhand der Prozessierung der Auxiliarselektion von \textit{haben} und \textit{sein} überprüft. Dafür werden Sätze mit Bewegungsverben mit unterschiedlichem Transitivitätsgrad genutzt und die Reaktionszeiten auf die Auxiliare gemessen. Prototypisch transitive Sätze (\textit{die Kinder zur Schule gefahren haben/*sein}) werden prototypisch intransitiven Sätzen (\textit{zur Kur gefahren sein/*haben}) gegenüber gestellt. Zusätzlich werden zwei ambige Strukturen präsentiert: Bewegungsverben mit einem Fortbewegungsmittel als Objekt (\textit{ein Cabrio gefahren haben/sein}) und Bewegungsverben mit einem Abstraktum als Akkusativergänzung, das zu einer Ambiguität zwischen Objekt und Adverbial führt (\textit{die Strecke in Rekordzeit gefahren sein/haben}). Mithilfe dieses Designs kann die Reaktionszeit auf Auxiliare in prototypischen und damit eindeutigen Kontexten mit Reaktionszeiten in peripheren und damit ambigen Kontexten gegenübergestellt werden.

Schemata im Sinne von abstrakten Konstruktionen (Form-Funktions-Paaren) (\cite[80]{Bybee.2010}) und ihr Einfluss auf Variation wurden bereits anhand der Prototypizität diskutiert: Die Form \textit{haben} und die Funktion Transitivität bilden ein Schema, genauso wie die Form \textit{sein} und die Funktion Intransitivität bei Bewegungsverben (\cite[316--319]{Gillmann.2016}). Die Funktionen der beiden Schemata stehen in einer prototypischen Beziehung zueinander, wodurch Schwankungen im Übergangsbereich der Funktionen möglich sind.

Neben Schemata in diesem Sinn spielt für Variation eine zweite Form von Schemata eine Rolle. Diese werden in der vorliegenden Arbeit erstmals modelliert. Während bei der ersten Form von Schemata eine Form mit einer Funktion assoziiert ist, trifft bei der zweiten Form eine Funktion auf zwei (oder mehr) Formen, mit denen sie assoziiert sein kann. Dieses Prinzip lässt sich gut anhand der Variation in der Deklination veranschaulichen: So kann die Funktion [+Genitiv] durch starke (-\textit{(e)s}; \textit{des Anglers}) und schwache Formen (-\textit{(e)n}; \textit{des Matrosen}) ausgedrückt werden. Dies wird in der vorliegenden Arbeit als Form-Schematizität bezeichnet. Die Formen -\textit{(e)s} und -\textit{(e)n} verbinden sich mit verschiedenen Maskulina: Während Formen auf -\textit{(e)s} mit einer Vielzahl von Maskulina kompatibel sind, sind Formen auf -\textit{(e)n} nur bei bestimmten Maskulina möglich: Sie sind prototypischerweise dreisilbig, auf der Pänultima betont, enden auf Schwa und referieren auf menschliche Entitäten (\textit{Matróse}, \textit{Gesélle}) (\cite[168--170]{Kopcke.1995}). Es steht somit eine abstrakte und damit hochschematische Form (X-\textit{(e)s}) einer weniger abstrakten und damit teilschematischen Form (X\textsubscript{[dreisilbig, Pänultima, Schwa, menschlich]}-\textit{(e)n}) gegenüber. Die Assoziation aus teilschematischer Form und Funktion wird in der vorliegenden Arbeit als Form-Schema bezeichnet. Form-Schematizität führt zu Stabilität und zu Variation: Maskulina, die dem Form-Schema komplett entsprechen (z.~B. \textit{Matrose}) werden stabil schwach flektiert (\cite{Kopcke.2000, Kopcke.2000b}). Maskulina in der Peripherie des Form-Schemas schwanken hingegen zwischen starken und schwachen Formen (\textit{des Grafs/Grafen}; \textit{des Willen/Willens}): \textit{Der Graf} referiert zwar auf eine menschliche Entität, erfüllt aber nicht die prosodisch-pho\-no\-lo\-gi\-schen Eigenschaften des Form-Schemas; \textit{der Wille} entspricht zwar den prosodisch-phonologischen Eigenschaften, aber nicht der semantischen, da er auf ein Abstraktum referiert (\cite[111]{Kopcke.2000}). Zudem kann Variation entstehen, wenn Substantive, die nach der hochschematischen Form flektieren, Ähnlichkeiten mit der teilschematischen Form des Form-Schemas aufweisen, wie bspw. bei \textit{Autor}. Hier kann es zu Schwankungen zur teilschematischen Form kommen (\textit{des Autoren} statt \textit{des Autors}) (\cite[78--79]{Kopcke.2005}).
 
Der Einfluss von Form-Schematizität auf Variation wird im empirischen Teil der Arbeit anhand der Verarbeitung von schwachen Maskulina überprüft. Es werden Maskulina mit schwacher (\textit{des Matrosen}/\textit{des Grafen}) und schwacher Flexion (\textit{des *Matroses}/\textit{des Grafs}) präsentiert und die Reaktionszeiten auf die präsentierten Formen gemessen. Dabei werden starke (\textit{Vogt}) oder schwache Maskulina (\textit{Matrose}/\textit{Graf}) genutzt. Die schwachen Maskulina entsprechen dem Form-Schema (X\textsubscript{[dreisilbig, Pänultima, Schwa, menschlich]}) entweder prototypisch (\textit{Matrose}) oder nur peripher (\textit{Graf}). Die Gegenüberstellung dieser Formen ermöglicht es, den Einfluss des Form-Schemas auf Reaktionszeiten zu messen und gleichzeitig zu prüfen, inwiefern sich der prototypische Aufbau des Form-Schemas auf Reaktionszeiten niederschlägt. 

Aus den bisherigen Ausführungen wird bereits deutlich, dass die Einflussfaktoren Frequenz, Prototyp und Schema nicht einzeln betrachtet werden können, da sie sich gegenseitig bedingen und somit in hohem Maße miteinander interagieren. Um den Einfluss der einzelnen Faktoren dennoch klar benennen zu können, werden sie in der Arbeit getrennt voneinander modelliert und ihr Zusammenspiel im Anschluss an die getrennte Betrachtung erläutert. Aus demselben Grund umfassen die psycholinguistischen Experimente jeweils nur einen Einflussfaktor. Die Arbeit möchte damit den Grundstein für weitere Forschung zum Zusammenspiel der Faktoren legen.

\section{Gliederung der Arbeit}

Die Arbeit gliedert sich in vier große Kapitel, zwei theoretische und zwei empirische: In \chapref{variationallg} werden Frequenz, Prototypizität und (Form-)Sche\-ma\-ti\-zi\-tät theoretisch fundiert und modelliert. Daran anschließend wird in \chapref{Fallstudien} der Einfluss von Frequenz, Prototypizität und (Form-)Schematizität in Bezug auf Variation in der Konjugation, Deklination und Auxiliarselektion anhand der bestehenden Forschung zu den Variationsphänomenen diskutiert. Die letzten beiden Kapitel überprüfen den Einfluss der vorgeschlagenen Faktoren psycholinguistisch, indem Reaktionszeiten gemessen werden. Dabei werden in \chapref{psych} zunächst die in den Studien genutzten Verfahren zur Elizitation von Reaktionszeiten sowie das Untersuchungsdesign der Studien vorgestellt, anschließend werden in \chapref{Ergebnisse}  die Ergebnisse der Studien präsentiert und diskutiert.
 
Die vier Kapitel werden durch das abschließende \chapref{fazit} abgerundet, das die theoretischen und methodischen Überlegungen zusammenfasst und einen Ausblick auf weitere Forschungsfragen gibt. 


Mit diesem Zuschnitt bietet die Arbeit einen neuen theoretischen und methodischen Zugriff auf Variation:  Durch den  gebrauchsbasierten Ansatz in dieser Arbeit lassen sich Verbindungen zwischen einzelnen Va\-ri\-a\-tions\-phänomenen aufzeigen und die Mechanismen hinter der Variation besser begreifen.  Der Einbezug von Sprachprozessierung erlaubt neue Einblicke in Variation, die die Betrachtung von Sprachproduktion allein nicht ermöglicht. Daher ergänzt die vorliegende Arbeit bestehende Forschung zu Variation um die Perspektive der Sprachprozessierung. Dabei wird mit drei Verfahren zur Reaktionszeitmessung gearbeitet, um einen breiten Zugriff auf Reaktionszeiten zu ermöglichen. Die Arbeit leistet damit einen wichtigen Beitrag zur Theorie und Empirie der variations\-lingu\-ist\-ischen Forschung.


