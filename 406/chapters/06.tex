\chapter{Zusammenfassung und Ausblick}\label{fazit}

In diesem Kapitel soll ein abschließender Blick auf den Einfluss von Frequenz, Prototypizität und (Form-)Schematizität auf sprachliche Variation geworfen werden. Dabei werden die Einflussfaktoren zunächst theoretisch modelliert. Zudem wird ihr Potential, Variation und Stabilität zu beeinflussen, skizziert. Darüber hinaus wird beschrieben, wie die Einflussfaktoren die Wahrscheinlichkeit für Variation steigen bzw. sinken lassen. Die Vorhersagen des Modells werden im Anschluss mit den empirischen Befunden in dieser Arbeit verglichen und es wird diskutiert, inwiefern sich der Einfluss der Faktoren in Reaktionszeiten spiegelt. Für jeden Einflussfaktor wird ein Ausblick auf offene Forschungsfragen und weiteren Forschungsbedarf gegeben. Abschließend wird diskutiert, wie die Einflussfaktoren auf Variation gewichtet werden können.

\section{Frequenz}

Frequenz beeinflusst Sprache und Variation grundlegend, da eine hohe Frequenz sowohl \textit{entrenchment} als auch Kategorisierung und Schematisierung ermöglicht (\cite[220]{Langacker.2008}). Somit lassen sich Prototypizität als Kategorisierungsprinzip und \mbox{(Form}-)Sche\-ma\-ti\-zi\-tät als Ergebnis von Abstraktion und damit von Schematisierung auch als Konsequenzen von Frequenz sehen. 

Hinsichtlich der Frequenz ist zwischen Typen- und  Tokenfrequenz zu unterscheiden, wie in \sectref{typ} herausgearbeitet wurde. Typenfrequenz hängt eng mit der Generalisierbarkeit eines Flexionsmusters und dessen Produktivität zusammen (\cite[384]{Bybee.1997}). Gemeinsam mit der Variabilität der bereits in einer Kategorie existierenden Exemplare und der Ähnlichkeit eines neuen potentiellen Mitglieds der Kategorie zu bereits bestehenden Exemplaren nimmt Typenfrequenz Einfluss auf die Abdeckung (\textit{coverage}) einer Kategorie (\cite[62--65]{Goldberg.2019}). Das Konzept der Abdeckung fragt danach, wie gut eine Kategorie attestiert ist, wenn zu den bereits bestehenden Mitgliedern ein neues hinzutritt. Eine Kategorie wird dabei als Cluster in einem hyperdimensionalen Raum betrachtet. Hat eine Kategorie viele Mitglieder, die variable Eigenschaften aufweisen, ist das Cluster sehr vielfältig. Ein potentielles neues Mitglied führt daher zu einer Kategorie mit guter Abdeckung, denn es ist wahrscheinlich, dass Ähnlichkeiten zu bereits existierenden Mitgliedern bestehen. Anhand dieser exemplarbasierten Kategorisierung kann aus einem vielfältigen Cluster ein abstraktes Schema generalisiert werden, wie bspw. das Dentalsuffix -\textit{t} und die Funktion [+Vergangenheit] (\cite[136]{Goldberg.2019}). Dieses abstrakte Schema kann als Default für die Konjugation genutzt werden, da Neuzugänge aufgrund der Abstraktheit des Schemas immer zu einer gut abgedeckten Kategorie führen.  


Ein typeninfrequentes Cluster mit Mitgliedern, die in ihren Eigenschaften wenig variabel sind, kann hingegen weniger gut abstrahiert werden: Ein potentielles neues Mitglied muss den bereits bestehenden Exemplaren sehr ähnlich sein, um eine gute Abdeckung zu ermöglichen (\cite[62--65]{Goldberg.2019}). Ist dies nicht der Fall, entsteht durch den Neuzugang eine Kategorie mit schlechter Abdeckung, da der Neuzugang zu einem brüchigen Cluster führt. Dies kann leicht an einem Beispiel verdeutlicht werden: Verben der Ablautreihe 3a sind eine wenig variable Kategorie mit guter Abdeckung, da sie sich in ihren phonologischen Eigenschaften stark ähneln und so das klar konturierte phonologische Cluster [\#\_ɪ + ŋ +~(C)]  bilden. Ein Verb wie \textit{googeln} passt nicht zu diesem Cluster, da es /uː/ statt /ɪ/ als Infinitivstammvokal aufweist und einen Plosiv statt einem Nasal. \textit{Googeln} hat in der ersten Silbe somit keine phonologische Verbindung zu den restlichen Mitgliedern der Ablautreihe~3a. Wenn \textit{googeln} wie Verben der Ablautreihe 3a konjugiert würde, wäre die Abdeckung der Kategorie (Verben der Ablautreihe~3a und \textit{googeln}) daher schlecht. Deswegen wird \textit{googeln} auch nicht wie die Verben der Ablautreihe 3a konjugiert, sondern nach dem abstrakten Schema mit Dentalsuffix (\textit{googelte}/\textit{gegoogelt}).  

 
Das Beispiel zeigt, dass die Möglichkeit zur Generalisierung bei typeninfrequenten Clustern mit Mitgliedern, die wenig variabel sind, eingeschränkt ist (\cite[65]{Goldberg.2019}). Die phonologische Form der Verben ist daher Teil der Generalisierung. In diesem Fall bildet eine Kategorie mit geringer Typenfrequenz somit ein Form-Schema aus, in dem die Funktion [+Vergangenheit] mit der Form [\#\_ɪ + ŋ + (C)]  assoziiert ist. An dieser Stelle zeigt sich bereits der grundlegende Einfluss, den Frequenz auf Form-Schematizität hat.
   
Die enge Verknüpfung von Typenfrequenz mit Generalisierbarkeit führt zu einem grundlegenden Einfluss der Typenfrequenz auf Variation. Sie gibt bei Variation zwischen Flexionsklassen die Schwankungsrichtung vor: Mitglieder der typeninfrequenten Klasse können zu Formen der typenfrequenten Klasse schwanken (\textit{geglommen} > \textit{geglimmt}). Eine umgekehrte Schwankungsrichtung ist hingegen generell unwahrscheinlich, sie kann jedoch bspw. durch Form-Schemata ausgelöst werden (\textit{gewinkt} > \textit{gewunken}).

 
Wichtig ist hierbei, dass eine Funktion (z.~B. [+Vergangenheit]) mit mehreren Formen assoziiert ist, von denen eine typenfrequent ist, während die anderen typeninfrequent sind. Die typenfrequentere Form (Dentalsuffix) wäre theoretisch immer eine mögliche Option, weswegen sie die typeninfrequenten Formen (z.~B. Ablaut auf /oː/, /iː/) ablösen kann. Tokenfrequenz beeinflusst, ob Variation tatsächlich stattfindet: Tokenfrequente Elemente sind mental stark gefestigt und somit leicht und schnell zu verarbeiten (\cite{Rubenstein.1970}, \cite[380]{Bybee.1997}, \cite[13]{Schneider.2014}). Es ist daher für die Prozessierung von Vorteil, wenn tokenfrequente Elemente einer typeninfrequenten Flexionsklasse angehören: Da sie häufig genutzt werden, müssen sie nicht durch ein generalisierbares Muster gestützt werden. Hierdurch sind kurze und  gut unterscheidbare, da differente Flexionsformen möglich (\cite[42--43]{Werner.1989}, \cite[226--227]{Nubling.2000}). Zudem können die Formen der typeninfrequenten Klasse aufgrund ihrer hohen Tokenfrequenz  die theoretisch immer möglichen regelmäßigen Formen statistisch ausstechen (\cite[83--84]{Goldberg.2019}). Für tokeninfrequente Formen der typeninfrequenten Klasse besteht dieser Vorteil nicht mehr, da sie weniger stark gefestigt sind. Der Anschluss an ein generalisierbares Muster bietet sich daher an. Dies lässt sich anhand der starken Verben gut verdeutlichen: Die starke Konjugation ist typeninfrequent. Wenn ein starkes Verb tokeninfrequent wird, können die starken Formen die schwachen aufgrund des geringeren \textit{entrenchments} nicht mehr so stark dominieren und statistisch ausstechen, wie dies bei tokenfrequenten und daher stark gefestigten Formen der Fall wäre. Daher werden bei Verben, die ursprünglich der starken Klasse angehören, schwache Formen möglich und durch Nutzung gefestigt. Zudem ist beim Wechsel von starken Verben zur schwachen Konjugation ein Zwischenschritt möglich: Tokeninfrequente starke Verben, die nach einem typeninfrequenten Ablautmuster flektieren, können sich dem typenfrequenten Muster \textit{x-o-o} anschließen. Auf diese Weise kann es zu einem doppelten Anschluss an Typenfrequenz kommen: zunächst an das typenfrequente Flexionsverhalten innerhalb der typeninfrequenten Klasse (\textit{x-o-o}) und bei zu geringer Tokenfrequenz schließlich an die typenfrequente schwache Flexionsklasse (\textit{der Hund ball} > \textit{boll} > \textit{bellte}) (\cite[178--183]{Nowak.2013}). 


\begin{sloppypar}
Das typen- und tokenfrequenzbasierte statistische Vorkaufsrecht kann in der Variation zwischen starken und schwachen Verben und in der Variation zwischen schwachen und starken Maskulina beobachtet werden (siehe \sectref{freqverb} und \ref{freqmask}). Bei der Variation zwischen \textit{haben} und \textit{sein} ist dies nicht der Fall, da es sich hierbei um zwei Schemata handelt, die dem Makroschema [X\textsubscript{NP} X\textsubscript{VAFIN} \textit{ge}-V-\textit{t}/\textit{en}] und [+Perfekt] angehören: Die Form \textit{sein} ist mit der Funktion [-- Transitivität, + Telizität oder + Bewegungssemantik] assoziiert, die Form \textit{haben} dagegen mit der Funktion [+Transitivität, -- Telizität, -- Bewegungssemantik] (\cite[316--319]{Gillmann.2016}, siehe \sectref{selektion}). Hier ist also nicht eine Funktion mit Formen verknüpft, die sich in der Typenfrequenz unterscheiden, sondern zwei separate Schemata liegen vor, die durch prototypisch organisierte Funktionen verbunden sind. Daher greift anders als bei der Variation in der Flexion die Kombination aus typen- und tokenfrequenzbasiertem statistischem Vorkaufsrecht nicht (\cite[74--94]{Goldberg.2019}). Es ist allenfalls möglich, dass Konstruktionen, die sich im Übergangsbereich zwischen den Funktionen befinden,  aufgrund der Typenfrequenz zu \textit{haben} neigen. Das könnte bspw. bei \textit{das Pferd geritten haben/sein} der Fall sein, denn hier kann das Objekt als Patiens oder Instrument interpretiert werden. Das war im Antwortverhalten der Prototypizitätsstudie jedoch nicht zu beobachten. Stattdessen scheinen Bewegungsverben so stark mit der \textit{sein}-Selektion assoziiert zu sein, dass diese Assoziation auch in ambigen Fällen greift. Unabhängig von der Typenfrequenz spielt Tokenfrequenz aber eine Rolle in der Auxiliarselektion: So weisen bspw. tokenfrequente Bewegungsverben eine höhere Stabilität in der Auxiliarselektion auf als tokeninfrequente (\cite[272--274]{Gillmann.2016}, siehe \sectref{selfreq}). Auch das Antwortverhalten in der Prototypizitätsstudie weist auf einen Tokenfrequenzeffekt hin, da das vergleichsweise wenig frequente Bewegungsverb \textit{reiten} mit auffällig vielen \textit{sein}-Antworten in transitiven Sätzen und einem hohen Anteil von \textit{haben}-Antworten bei den ambigen Sätzen im Vergleich zu den anderen Testverben die höchste Variation im Antwortverhalten zeigt. 
\end{sloppypar}

Tokenfrequenz wurde in \sectref{korrelation} als ein guter Prädiktor für die Schnelligkeit und Genauigkeit vorgestellt, mit der Proband\_innen auf Stimuli in psycholinguistischen Studien reagieren (\cite[55]{Divjak.2015}). Allerdings stellt Tokenfrequenz als pure Wortwiederholung eine Vereinfachung dar, da bspw. der Kontext in dieser Operationalisierung nicht berücksichtigt wird. Zudem haben viele Faktoren, die mit Frequenz korrelieren, wie bspw. Abfolgewahrscheinlichkeiten und Kontextdiversität, ebenfalls Einfluss auf die Prozessierung (\cite{McDonald.2003, Baayen.2012}). Es stellt sich daher die Frage, ob Frequenz selbst einen umfassenden Einfluss auf Prozessierung und das Verhalten von Proband\_innen nimmt oder ob der Einfluss besser durch einen Faktor erklärt werden kann, der mit Frequenz korreliert. Die Berücksichtigung von Kontexten, die bspw. durch den Blick auf Kontextdiversität möglich ist, scheint die Varianz im Verhalten von Proband\_innen dabei besser zu erklären als Frequenz im Sinne einer puren Wortwiederholung (\cite[181--191]{Baayen.2012}). Tokenfrequenz bietet dennoch einen schnellen und einfachen Zugriff auf die Varianz im Verhalten von Proband\_innen. Man sollte sich aber über die Korrelation mit anderen Faktoren bewusst sein, die möglicherweise die Varianz besser erklären als Tokenfrequenz. Daher wurden Typen- und Tokenfrequenz in \sectref{Statistik}  als Teil des statistischen Lernens in ein bayesianisches Modell integriert.

\begin{sloppypar}
Bayesianische Modelle arbeiten mit bedingten Wahrscheinlichkeiten und berechnen die Wahrscheinlichkeit für ein Ereignis (\cite[330]{Norris.2006}), z.~B.: Wenn ein bestimmter Input gegeben ist, welche mentale Repräsentation passt zum Input? Diese Frage stellen sich Sprecher\_innen in der Sprachverarbeitung. Wenn Sprecher\_innen ein Wort lesen oder hören, gleichen sie das Wort mit mentalen Repräsentationen ab und überprüfen, welche Repräsentation am besten zum Input passt. Ist der Input nicht gut zu verstehen, noch nicht ganz gesprochen bzw. gebärdet oder unleserlich geschrieben, sind mehrere mentale Repräsentationen möglich. Um herauszufinden, welche Repräsentation zum Input passt, greifen Sprecher\_innen auf die priore Wahrscheinlichkeit zurück. Frequenz ist Teil der prioren Wahrscheinlichkeit: Hört man [bɑː], ist die Wahrscheinlichkeit höher, dass [lt] folgt als [jɛzjɑːnɪʃ], weil \textit{bald} frequenter ist als die Wortform \textit{bayesianisch}. Im Kontext dieser Dissertation ist das Gegenteil der Fall: \textit{Bayesianisch} wird häufiger genutzt als \textit{bald}, somit ist beim Wortanfang <ba> die Wahrscheinlichkeit für <yesianisch> höher als für <ld>. Auch der Kontext kann somit Teil der prioren Wahrscheinlichkeit sein. Frequenz nimmt in diesem Modell daher zusammen mit anderen Faktoren (bspw. Kontext und Abfolgewahrscheinlichkeiten) Einfluss auf die priore Wahrscheinlichkeit (\cite[330--331]{Norris.2006}). Genauso beeinflusst Frequenz das statistische Vorkaufsrecht (\cite[75--77]{Goldberg.2019}): Existieren zwei Formen mit derselben Funktion, sind zunächst beide Formen möglich. Ist eine der Formen tokenfrequenter als die andere, hat diese aufgrund der höheren Frequenz eine größere priore Wahrscheinlichkeit genutzt zu werden und sticht die andere Form daher statistisch aus.
\end{sloppypar}

Im empirischen Teil der Arbeit ist der Einfluss von Tokenfrequenz deutlich zu erkennen (siehe \sectref{freqerg}). Die frequenten starken Verben evozieren deutlich geringere Reaktionszeiten als die infrequenten starken Verben. Dies war jeweils für die starken (\textit{gezogen}) als auch für die schwachen Formen (\textit{gezieht}) zu beobachten. Die Ergebnisse stützen das Prinzip des statistischen Vorkaufsrechts: Schwache Formen sind bei frequenten starken Verben kaum gefestigt. Da aber eine tokenfrequente Alternative existiert, können die Formen schnell als unbekannt bewertet werden. Daher werden die schwachen Formen von frequenten Verben schneller beurteilt als die schwachen Formen von infrequenten Verben. Die schwachen Formen werden bei den frequenten Verben erwartbarerweise langsamer beurteilt als die starken, da die starken Formen selbst mental gefestigt sind und Proband\_innen anders als bei den schwachen Formen in der Bewertung nicht auf die mentale Repräsentation einer anderen stark gefestigten Form zurückgreifen müssen. Die schwachen Formen der frequenten Verben weisen dabei vergleichbare Reaktionszeiten wie die starken Formen der infrequenten Verben auf.


Innerhalb der infrequenten starken Verben zeigt sich, dass das Vorkommen von schwachen Formen im Sprachgebrauch keinen Einfluss auf die Verarbeitungsgeschwindigkeit hat. Sowohl die Verben ohne attestierte Schwankung in Korpora als auch die Verben mit attestierter Schwankung weisen vergleichbare Reaktionszeiten auf (zur Operationalisierung von attestierter und nicht-at\-tes\-tier\-ter Schwankung siehe \sectref{metfreq}). Dies gibt einen Hinweis darauf, dass die starken Formen infrequenter Verben bereits nicht mehr so stark gefestigt sind wie bei den frequenten Verben. Daher können sie die schwachen Formen weniger stark dominieren. Die ausbleibenden Reaktionszeitunterschiede zwischen Verben mit und ohne Schwankung weisen damit auf eine mögliche Variation bei Verben ohne attestierte Schwankung hin, die im Sprachgebrauch noch nicht sichtbar ist.

 
Nur ein kleiner Unterschied in den Reaktionszeiten lässt sich zwischen den Verben mit und ohne Schwankung erkennen: Die starken Formen von Verben ohne Schwankung werden wie bei den frequenten Verben schneller bewertet als schwache Formen. Bei den Verben mit Schwankung besteht hingegen kein Vorteil für starke Formen. Die schwachen Formen scheinen durch den Gebrauch bereits gefestigt zu sein und können daher schneller bewertet werden.

 
Es zeigen sich zwei aufeinander aufbauende Schritte: Im ersten Schritt beeinflusst das statistische Vorkaufsrecht, wie sehr die starken Formen die schwachen dominieren. Wenn die starken Formen die schwachen stark dominieren, ist Variation unwahrscheinlich. Wenn die starken Formen aufgrund geringer Tokenfrequenz die schwachen nicht mehr so stark dominieren können, ermöglicht dies den Gebrauch schwacher Formen, sodass Variation wahrscheinlich wird. Die schwachen Formen werden im zweiten Schritt durch Gebrauch gefestigt und daher schneller verarbeitet. Der erste Schritt zu Variation lässt sich bereits in Re\-ak\-tions\-zei\-ten fassen, in Korpora wird Variation erst im zweiten Schritt sichtbar.

Für anschließende Studien wäre es sinnvoll, mit weiteren starken Verben zu arbeiten und die Operationalisierung von in Korpora attestierter und nicht-at\-tes\-tier\-ter Schwankung zu verfeinern, indem bspw. mit feingliedrigeren Abstufungen in der Ratio zwischen starken und schwachen Formen gearbeitet wird und eine mögliche regionale Schwankung der Verben vorab systematischer überprüft wird. Hierbei wäre es auch wichtig, Faktoren wie Kontextdiversität zu berücksichtigen, die mit Frequenz korrelieren. Zudem wäre es interessant, mit feineren Frequenzabstufungen zu arbeiten, um auszuloten, ab welcher Tokenfrequenz das statistische Vorkaufsrecht für starke Formen nur noch bedingt greift.   



\section{Prototypizität}

Prototypizität erwächst aus exemplarbasiertem Lernen und ist daher eng mit Frequenz verknüpft: Prototypische Vertreter einer Kategorie sind tokenfrequent und außerdem teilen sie viele Eigenschaften mit anderen Vertretern der Kategorie, daher sind ihre Eigenschaften typenfrequent (\cite[85]{Ellis.2014}, \cite[564]{Taylor.2015}). Zudem weisen  prototypische Vertreter nur wenige Eigenschaften auf, die Vertreter anderer Kategorien zeigen, sodass den Eigenschaften von prototypischen Vertretern eine hohe \textit{cue validity} zukommt (siehe \sectref{standard}). Prototypizität ergibt sich daher aus Token- und Typenfrequenz. Aufgrund dieser engen Verknüpfung von Prototyp und Frequenz werden prototypisch organisierte Kategorien anders als in der Standardversion der Prototypensemantik, die in Abschnitt~\ref{standard} vorgestellt wird, in Abschnitt~\ref{exemplar} als Ergebnis exemplarbasierten Lernens modelliert: Elemente mit ähnlichen Eigenschaften werden als Cluster gefasst, dabei bilden die häufigen Elemente einen Ankerpunkt, um den sich die anderen Elemente gruppieren (\cites[89]{Goldberg.2006}[242]{Ellis.2016}[51--73]{Goldberg.2019}). Die Kategorie wird über das Prinzip der Abdeckung erweitert. Das Resultat dieses exemplarbasierten Lernens ist eine Kategorie mit prototypischen und peripheren Mitgliedern (\cite[216--222]{Ross.1999}). 

  
Dass frequente prototypische Vertreter als Ankerpunkt von Kategorien fungieren, zeigt sich in Abschnitt~\ref{proteffekt} bspw. darin, dass prototypische Vertreter einer Konstruktion als Erstes erlernt werden und Kategorien mit frequenten prototypischen Vertretern  schneller erlernt werden als Kategorien mit weniger frequenten prototypischen Vertretern (\cites[198--200]{Goldberg.2006}{Ellis.2009}[58]{Ellis.2014}). Prototypische Vertreter einer Kategorie werden aufgrund ihrer Tokenfrequenz und der vielen Eigenschaften, die sie mit anderen Mitgliedern der Kategorie teilen, immer wieder gefestigt (\cite[85]{Ellis.2014}). Sie sind daher in ihrer Verwendung stabil, sodass Variation hier unwahrscheinlich ist. Bei peripheren Vertretern ist Variation hingegen wahrscheinlich (\cite[67--68]{Agel.2008}). Dies liegt auch daran, dass die Peripherie den Übergangsbereich zwischen zwei Clustern darstellen kann. Das ist bspw. bei der Auxiliarselektion von \textit{haben} und \textit{sein} der Fall: Die beiden Schemata sind durch prototypisch organisierte Funktionen miteinander verbunden, dabei sind im Übergangsbereich zwischen den Funktionen beide Auxiliare möglich (siehe \sectref{selproto}).  


Bei Flexionsklassen lässt sich das Prinzip der Prototypizität nutzen, um das Flexionsverhalten prototypisch zu modellieren: Verben und Substantive lassen sich anhand ihres Verhaltens in prototypisch stark und prototypisch schwach teilen (siehe \sectref{protverb} und \ref{protmask}). Die gemischte Deklination der Maskulina erscheint in dieser Betrachtung als Übergangsbereich zwischen der schwachen und starken Deklinationsklasse. Zudem lassen sich auch prototypische und periphere Eigenschaften der typeninfrequenten Klasse (starke Verben/schwache Maskulina) ableiten. Die Eigenschaften, die nur die prototypischen Vertreter der typeninfrequenten Klasse aufweisen (z.~B. Imperativhebung, Genitivprofilierung), sind periphere Eigenschaften, prototypische Eigenschaften zeigen hingegen auch die peripheren Mitglieder der infrequenten Klasse (z.~B. Ablaut, Plural auf -\textit{n}) (\cite[57]{Bittner.1985}, \cite[157]{Nowak.2013}). Daraus ergibt sich, dass viele Mitglieder einer Flexionsklasse die prototypischen Merkmale der Klasse aufweisen und nur wenige die peripheren Eigenschaften. Den peripheren Flexionseigenschaften kommt außerdem i.~d.~R. eine geringe intraparadigmatische Frequenz und eine geringere Relevanz nach \textcite{Bybee.1985} zu als den prototypischen Eigenschaften. Bei einem Klassenwechsel werden die peripheren Eigenschaften daher zuerst aufgegeben und die prototypischen Eigenschaften zuletzt. Die prototypische Modellierung der Flexionsklassen lässt sich somit nutzen, um Schwankungen von der typeninfrequenten zur typenfrequenten Klasse zu prognostizieren. So ist eine Aufgabe der prototypischen Flexionseigenschaften unwahrscheinlich, so lange die peripheren Eigenschaften noch stabil im Sprachgebrauch zu beobachten sind.

Der Einfluss der Prototypizität auf Variation zeigt sich im empirischen Teil der Arbeit nur bedingt (siehe \sectref{ergprot}). Die prototypischen Vertreter des \textit{haben}- und \textit{sein}-Schemas evozieren jeweils das erwartbare Auxiliar, aber in der Peripherie ist kaum Schwankung zu beobachten, denn in beiden ambigen Konditionen\footnote{Die ambigen Sätze enthielten ein Objekt, das entweder Patiens oder Instrument sein kann (\textsc{ambig~I}: \textit{ein Cabrio fahren}), bzw. eine Akkusativergänzung, die als Adverbial interpretiert werden kann (\textsc{ambig~II}: \textit{ein Rennen fahren}).} überwiegen \textit{sein}-Antworten (siehe \sectref{metproto} für ausführliche Erläuterungen zum Untersuchungsdesign). Im Antwortverhalten der Studie zu Form-Schematizität zeigt sich jedoch Variation in der Peripherie des Form-Schemas. Hinsichtlich der Reaktionszeiten lässt sich der Einfluss von Prototypizität ansatzweise feststellen: In der Prototypizitätsstudie wurde mit den intransitiven Sätzen zumindest einer der Prototypen schneller verarbeitet als die anderen Ausprägungen. Dabei deuten die erhöhten Lesezeiten bei den transitiven Sätzen auch auf das Wirken der Prototypizität hin: Durch die häufige Verwendung in intransitiven Sätzen sind Bewegungsverben so stark mit \textit{sein} assoziiert, dass der Wechsel zu \textit{haben} in transitiven Sätzen einen Prozessierungsaufwand darstellt, der sich in höheren Reaktionszeiten niederschlagen könnte. Die generelle Tendenz von Bewegungsverben zu \textit{sein} könnte auch die klare \textit{sein}-Präferenz der Proband\_innen in den Sätzen im Übergangsbereich zwischen den beiden Schemata erklären. 


 Wie die transitiven Sätze weisen die Sätze in der Peripherie, \textsc{ambig~I} und II, erhöhte Reaktionszeiten auf. Die erhöhten Reaktionszeiten sind aufgrund der vorwiegenden \textit{sein}-Wahl nicht mit dem Wechsel zu \textit{haben} zu erklären und könnten auf Variationspotential hindeuten, da in den ambigen Sätzen theoretisch beide Auxiliare möglich sind.

In zukünftigen Studien sollte das Wirken der Schemata dahingehend überprüft werden, ob transitive Sätze ohne Bewegungsverb tatsächlich geringere Reaktionszeiten hervorrufen als transitive Sätze mit Bewegungsverb. Die vorliegende Studie gibt einen Hinweis darauf, der jedoch systematisch überprüft werden müsste. Zudem wäre in Bezug auf einzelne Bewegungsverben eine genauere Analyse der Auxiliarselektion bei \textit{reiten} lohnend, das sich im Antwortverhalten von den anderen Testverben abhebt. Neben der prototypisch organisierten Transitivität wäre es lohnenswert, die ebenfalls prototypisch organisierten Einflussfaktoren Telizität und Bewegungssemantik auf die Auxiliarselektion in den Blick zu nehmen. In Bezug auf Telizität wäre es sinnvoll, \textit{degree achievements} näher zu untersuchen. Hierbei könnte man telische Verben (\textit{die Hefe ist/*hat verschimmelt}) mit \textit{degree achievements} (\textit{die Hefe ist/hat geschimmelt}) vergleichen sowie \textit{degree achievements} in Sätzen mit telischer Lesart (\textit{Das Messer ist/*hat von oben bis unten gerostet}) und atelischer Lesart (\textit{Das Messer hat/*ist jahrelang vor sich hin gerostet}) gegenüberstellen. In Bezug auf Bewegungssemantik wäre ein Blick auf kontextuell konstruierte Bewegungssemantik (\textit{Ich bin/*habe durch den Saal getanzt} vs. \textit{Ich habe/*bin immer gern getanzt}) lohnenswert.  

\begin{sloppypar}
Zudem sollte die Prototypizitätsskala von starken und schwachen Verben sowie Maskulina psycholinguistisch überprüft werden: Hierfür könnte man bspw. Verben mit Imperativhebung nutzen (z.~B. \textit{geben}) und diese mit und ohne Imperativhebung (\textit{gib}\slash\textit{geb}) sowie mit starken und schwachen Präteritalformen (\textit{gab}/*\textit{geb\-te}) präsentieren und überprüfen, ob die Schwankung im Imperativ geringere Reaktionszeitunterschiede hervorruft als die Variation in der Präteritalform.
\end{sloppypar}

\section{(Form-)Schematizität}

Wie Prototypizität ergibt sich (Form-)Schematizität aus Frequenz: Wenn eine Kon\-struk\-tion typenfrequente und hochvariable Slots aufweist, können diese abstrahiert und damit schematisiert werden. Das Ergebnis ist ein Schema (\cite[168--170]{Ellis.2002b}, \cite[80]{Bybee.2010}). Die vorliegende Arbeit führt die Termini \textit{Form-Schematizität} und \textit{Form-Schema} ein, um die Variation in der Flexion modellieren zu können: In Hinblick auf Flexionsklassen liegt Form-Schematizität vor, da eine Funktion nicht nur mit einer, sondern mit mehreren Formen assoziiert ist (siehe Abschnitt~\ref{konstruktion}). Dabei weist eine der Formen eine hohe Typenfrequenz und eine hohe Variabilität zwischen den Mitgliedern auf, weswegen sie generalisierbar ist. Die anderen Formen haben dagegen eine geringere Typenfrequenz und geringe Variabilität zwischen den Mitgliedern und sind daher nur eingeschränkt generalisierbar (\cite[69]{Bybee.2010}, \cite[64--65]{Goldberg.2019}). In der Folge sind die typeninfrequenten Formen nicht-schematisch oder teilschematisch.  In der vorliegenden Arbeit werden Assoziationen aus einer Funktion und einer teilschematischen Form als Form-Schemata bezeichnet, wenn Form-Schematizität vorliegt, also eine Funktion und mehrere Formen mit unterschiedlichen Schematizitätsgraden miteinander verknüpft sind. Form-Schemata nehmen Einfluss auf Variation, wie im weiteren Verlauf des Abschnitts erläutert wird.

Das Prinzip der Form-Schematizität ist in der Deklination und in der Konjugation zu beobachten und lässt sich anhand der Konjugation am besten veranschaulichen, da hier der typenfrequenten Form viele typeninfrequente Formen gegenüberstehen. Eine typeninfrequente Form kann nur für ein Verb greifen (z.~B. \textit{sein}) oder für mehrere Verben mit ähnlichen phonologischen Eigenschaften (z.~B. Verben der Ablautreihe 3a). Aufgrund der Ähnlichkeit in der Phonologie bilden die Verben ein Cluster, sodass die ähnlichen phonologischen Eigenschaften abstrahiert werden können (z.~B. [\#\_ɪ + ŋ + (C)]), da sie häufig vorkommen. Ein Form-Schema entsteht. Hierbei handelt es sich streng genommen nicht um ein einzelnes, sondern um mehrere ineinander verschachtelte Form-Schema: Die Form \textit{trank} ist mit [+Vergangenheit] verbunden, aufgrund dieser Verbindung und der Verbindung bei anderen Form-Funktions-Paaren (z.~B. \textit{getrunken} und [+Partizip~II]) ist das Verb \textit{trinken} mit starker Flexion assoziiert. Auf einer abstrakteren Ebene ist die Form [\#\_ɪ + ŋ + (C)] mit starkem Flexionsverhalten assoziiert, da mehrere Verben mit ähnlichen Eigenschaften existieren, die [+Vergangenheit] durch den Ablaut auf /a/ ausdrücken.

Prototypizität ist Form-Schemata eingeschrieben, da die Assoziation zwischen teilschematischer Form und Funktion probabilistisch ist (siehe \sectref{kognition}): Je mehr Eigenschaften der Form des Form-Schemas erfüllt werden, des\-to wahrscheinlicher ist eine Assoziation mit der Funktion (\cite[39]{Rumelhart.1980}, \cite[262--264]{Bybee.1983}). Es existieren daher prototypische und periphere Mitglieder eines Form-Schemas: Prototypische Vertreter des Form-Schemas [\#\_ɪ + ŋ + (C)] sind bspw. \textit{trinken} und \textit{sinken}, periphere hingegen \textit{sinnen} und \textit{rinnen}, da der Nasal vorne und nicht hinten im Mundraum gebildet wird. Form-Schematizität ist somit sowohl mit Frequenz als auch mit Prototypizität eng verbunden. 

Form-Schemata sind typisch für eine typeninfrequente Flexionsklasse, aber auch in der variablen typenfrequenten Klasse können teilschematische Subcluster entstehen, die dann ein Form-Schema bilden, wie bspw. Maskulina auf -\textit{er} (\textit{Angler}), die mit starker Deklination assoziiert sind (\cite[160--161]{Kopcke.2000b}). Für Variation sind vorrangig Form-Schemata in typeninfrequenten Klassen relevant: Durch Form-Schemata besteht eine Typenfrequenz innerhalb von typeninfrequenten Klassen, denn bestimmte Eigenschaften kommen häufig vor und lassen sich daher, wenn auch nur bedingt, abstrahieren. Als typenfrequentes Cluster führen Form-Schemata zu Stabilität (siehe \sectref{schemaeffekt}). So variieren bspw. Verben, die einem Form-Schema angehören, weniger, auch wenn sie genauso tokeninfrequent sind wie Verben, die keinem Form-Schema angehören (\cite[57--58]{Kopcke.1999}). Die Flexion wird gestützt, da die Verben einem teilschematischen Flexionsmuster folgen. Für Verben wie \textit{trinken}, die einem Form-Schema angehören, ist Variation also unwahrscheinlich. Variation ist bei Verben möglich, die Form-Schemata nur teilweise entsprechen wie bspw. \textit{sinnen}. Form-Schemata führen somit nicht nur zu Stabilität, sondern auch zu Variation. Sie können außerdem Variation hin zur tokeninfrequenten Klasse auslösen: Verben aus der schwachen Flexionsklasse, die in ihren phonologischen Eigenschaften einem Form-Schema ähneln, schwanken zur starken Flexion (\textit{gewinkt}~>~\textit{gewunken}) (\cite[56]{Kopcke.1999}). 


Form-Schemata können unterschiedliche Schematizitätsgrade aufweisen: Das Form-Schema der Verben der Ablaut\-reihe 3a ist mit der Form [\#\_ɪ + ŋ + (C)] relativ konkret, das Form-Schema \textit{x-o-o} dagegen weitaus abstrakter (siehe hierzu \sectref{schemaverb}). Das Form-Schema \textit{x-o-o} springt bei geringer Tokenfrequenz und geringer Passgenauigkeit zu konkreteren Form-Schemata ein und stützt so die starke Konjugation.  

Wie bereits deutlich wurde, können innerhalb einer typeninfrequenten Flexionsklasse mehrere Form-Schemata als typenfrequente Cluster existieren, dies ist bspw. bei starken Verben der Fall. Form-Schemata können aber auch eine gesamte Flexionsklasse strukturieren, wie bei den schwachen Maskulina.\footnote{Es existieren zwar zwei Form-Schemata, die die schwachen Maskulina prägen, jedoch sind sie in ihren Eigenschaften sehr ähnlich. Dies ist bei den starken Verben nicht der Fall.} Das Form-Schema schwacher Maskulina ist weit komplexer als die Form-Schemata starker Verben, da es neben prosodisch-phonologischen Eigenschaften (Dreisilbligkeit, Pänultimabetonung, Schwa) auch eine semantische Eigenschaft (hoher Belebtheitsgrad) aufweist. Substantive wie \textit{Kollege}, die dem prosodisch-phonologischen und dem semantischen Prototyp des Form-Schemas entsprechen, flektieren stabil schwach, für sie ist Variation unwahrscheinlich. Dasselbe gilt für Substantive wie \textit{Stein}, die weder den prosodisch-phonologischen noch den semantischen Eigenschaften des Form-Schemas entsprechen: Aufgrund der fehlenden prosodisch-phono\-logischen und semantischen Eigenschaften ähneln sie den schwachen Substantiven so wenig, dass sie nicht als Teil der schwachen Fle\-xions\-klas\-se wahrgenommen werden. Variation ist jeweils in der Kombination aus Prototyp und Peripherie wahrscheinlich: Monosyllabische Substantive mit hohem Belebtheitsgrad (formale Peripherie, semantischer Prototyp) und zwei- und dreisilbige Substantive mit niedrigem Belebtheitsgrad (formaler Prototyp, semantische Peripherie) schwanken zwischen schwachen und starken Formen oder wechseln das Genus (\textit{des Grafen/Grafs}, \textit{des Willen/Willens}, \textit{der Krake/die Krake}) (siehe \sectref{schemamask}). Das Form-Schema nimmt dabei auch Einfluss auf die Art der Variation: Zwei- und dreisilbige Substantive auf Schwa mit geringem Belebtheitsgrad schließen sich der starken Flexion an, indem sie -\textit{ns} im Genitiv annehmen (\textit{des Willen} > \textit{des Willens}) und schließlich -\textit{n} im Nominativ (\textit{der Wille}~>~der \textit{Willen}). Zweisilbigen Substantiven steht zudem die Möglichkeit offen, das Genus zu wechseln (\textit{der Krake} > \textit{die Krake}) (\cite[108--109]{Kopcke.2000}). Einsilbige Substantive mit hohem Belebtheitsgrad weisen hingegen zunächst endungslose Dativ- und Akkusativformen auf (\textit{den Grafen} > \textit{den GrafØ}) und wechseln dann das Genitivsuffix zu -\textit{s} (\textit{des Grafen}~>~\textit{des Grafs}) (\cite[108--109]{Kopcke.2000}). 

Im empirischen Teil der Arbeit zeigt sich der Einfluss von Form-Schemata nur im Antwortverhalten (siehe \sectref{ergschema}). Für Substantive, die dem Prototyp des Form-Schemas entsprechen, werden schwache Formen klar bevorzugt: Im Produktionsexperiment zur self-paced-reading-Studie werden für diese Substantive vorrangig schwache Formen gebildet, in der lexical-decision-Studie werden schwache Formen als bekannt und starke als unbekannt bewertet und in der sentence-maze-Studie vorrangig schwache Formen gewählt. Dies gilt für real existierende, aber auch für Pseudosubstantive. Die schwachen Formen bei Pseudosubstantiven verdeutlichen, dass das Form-Schema produktiv ist. Für Substantive, die zur starken Flexion gehören, wurden erwartbarerweise starke Formen genutzt. Substantive in der Peripherie des Form-Schemas weisen mit Ausnahme der sentence-maze-Studie Variation auf: Im Produktionsexperiment werden sowohl schwache als auch starke Formen gewählt, in der lexical-decision-Studie werden starke Formen zu 50 \% als bekannt bewertet. Die ausbleibende Variation in der sentence-maze-Studie kann mit dem Versuchsdesign erklärt werden, da die häufigere Form direkt neben der selteneren präsentiert wurde und die häufige die seltene Form somit leichter statistisch ausstechen kann. In den Reaktionszeiten ist hingegen kein systematischer Einfluss der Form-Schematizität zu erkennen. In der self-paced-reading-Studie liegen die Unterschiede zwischen den Reaktionszeiten für die schwache und die starke Form von \textit{Schettose} zwar unter dem $\alpha$-Level, die 95~\%-Konfidenzintervalle überlappen jedoch. In der visuellen Analyse der Reaktionszeiten der sentence-maze-Studie sind Einflüsse von Tokenfrequenz nicht auszuschließen, obwohl die Tokenfrequenz der Testsubstantive in der Studie (siehe \sectref{schemalex}) möglichst konstant gehalten wurde. Die Unterschiede in der Reaktionszeit scheinen aber auch in Hinblick auf Tokenfrequenz nicht systematisch zu sein.

In weiteren Untersuchungen sollte der Diskrepanz zwischen Antwortverhalten und Reak\-tionszeiten nachgegangen werden. Dabei ist es möglich, dass die in den Studien verwendeten Verfahren zu wenig sensitiv für die Messung von Prozessierungsunterschieden sind und daher auf feinere Methoden wie EEG zurückgegriffen werden sollte. Da der Einfluss von Form-Schematizität im Antwortverhalten zu erkennen ist, ist es unwahrscheinlich, dass er sich überhaupt nicht auf die Prozessierung auswirkt. Neben der Verwendung von feineren Messmethoden wäre es in\-te\-res\-sant, die Interaktion von Tokenfrequenz und Form-Schemata näher zu betrachten und Form-Schemata bei starken Verben psycholinguistisch zu untersuchen. Viele Unterschiede in den Flexionseigenschaften von starken und schwachen Verben sind salienter als die Unterschiede in den Flexionseigenschaften von starken und schwachen Maskulina, sodass mögliche Prozessierungsunterschiede auch in Reaktionszeiten sichtbar sein könnten. Es wäre sinnvoll, mit Verben der Ablaut\-reihe 3a zu arbeiten, da diese eine klar umrissene phonologische Form aufweisen. Hierbei bietet sich an, Reaktionszeiten von Verben der Ablaut\-reihe 3a mit unterschiedlicher Tokenfrequenz (z.~B. \textit{singen} vs. \textit{wringen}) zu messen und diese mit Reaktionszeiten von Verben zu vergleichen, die keinem Form-Schema angehören und jeweils eine ähnliche Tokenfrequenz aufweisen. 


Ein weiterer Faktor, der in der Arbeit ausgeklammert wurde, ist der Einflussfaktor des Mediums. Alle Experimente wurden visuell durchgeführt. Es wäre interessant, die Ergebnisse mit auditiven Stimuli zu kontrastieren. Dies gilt für alle Einflussfaktoren und Variationsphänomene, aber insbesondere für die Form-Schemata. Da diese phonologisch konditioniert sind, ist es möglich, dass sie einen größeren Einfluss auf Variation haben, wenn sie auditiv präsentiert werden, als nur mittelbar durch Schrift.

\section{Gewichtung der Einflussfaktoren}

Da Typen- und Tokenfrequenz die Existenz von Prototypizität und \mbox{(Form-)}Sche\-ma\-ti\-zi\-tät bedingen, ist der Einfluss von Frequenz auf Variation am stärksten zu gewichten. Gleichzeitig korreliert Frequenz am stärksten mit anderen Einflussfaktoren, sodass der Blick auf Frequenz allein eine Vereinfachung darstellen kann. Um diesem Problem zu begegnen, wird der Einfluss von Frequenz als Teil der prioren Wahrscheinlichkeit für das Vorkommen einer Struktur in einem bayesianischen Modell gefasst. Diese Modellierung schließt den Einfluss weiterer Faktoren nicht aus und kann den Einfluss der Frequenz auf Prozessierung erklären.

 

Prototypizität ergibt sich aus dem Einfluss von Frequenz: Prototypische Vertreter einer Kategorie haben eine hohe Tokenfrequenz und teilen viele Eigenschaften miteinander. Hieraus ergibt sich eine Typenfrequenz. Beides stützt die prototypischen Vertreter. (Form-)Sche\-ma\-ti\-zi\-tät ergibt sich aus Frequenz, da Typenfrequenz die Abstraktion von Mustern ermöglicht. Prototypizität und Form-Schematizität sind dabei eng miteinander verwandt. Prototypizität fokussiert auf die Struktur einer Kategorie und betrachtet, wie die Kategorie über periphere Mitglieder der Kategorie mit anderen Kategorien verbunden ist. Kategorien ergeben sich aus wiederkehrenden Eigenschaften von Konstruktionen und lassen sich somit wie Konstruktionen als Assoziationen aus Form und Funktion fassen. Bei Form-Schematizität ist eine Funktion nicht nur mit einer, sondern mit mehreren Formen verbunden. Hier wird ebenfalls die prototypisch organisierte Verbindung zwischen Form und Funktion betrachtet, nur in Bezug auf die Assoziation einer Funktion mit mehreren Formen. Trotz dieser engen Verknüpfung ist die Unterscheidung zwischen separaten Schemata, deren Funktionen prototypisch miteinander verbunden sind, und der in dieser Arbeit ausgearbeiteten Form-Schematizität, bei der eine Funktion mit mehreren Formen verbunden ist, essentiell. Bei Form-Schematizität spielt Typenfrequenz eine viel größere Rolle als bei separaten Schemata, da das statistische Vorkaufsrecht nur dann greift, wenn eine Funktion mit mehreren Formen assoziiert ist. Separate Schemata können zudem Elemente in ihre Slots zwingen: Auch wenn \textit{tanzen} normalerweise keine Bewegung benennt, sondern eine Aktivität, selegiert \textit{tanzen} das Auxiliar \textit{sein} statt \textit{haben}, wenn der Kontext eine Bewegung impliziert (\textit{Ich bin durch den Raum getanzt}). Dieses Prinzip ist bei Form-Schemata ausgeschlossen, da keine separaten Funktionen vorliegen.   


Trotz dieser Gewichtung ist nicht davon auszugehen, dass Frequenz die anderen Einflussfaktoren immer überlagert. Stattdessen handelt es sich um ein Zusammenspiel. Dies zeigt sich in Hinblick auf Typenfrequenz, Form-Schemata und Tokenfrequenz bei der Variation in der Deklination: Zunächst bedingt die Typenfrequenz die Schwankungsrichtung hin zu starken Maskulina, das Form-Schema bedingt die schwachen Substantive, die zur starken Flexion schwanken, und innerhalb der Peripherie des Form-Schemas bedingt die Tokenfrequenz, ob tatsächlich Variation stattfindet.  

Der Einfluss von Frequenz, Prototypizität und (Form-)Schematizität auf Variation wird in allen Variationsphänomenen (Variation in der Konjugation, Deklination und Auxiliarselektion) sichtbar. Daher wäre es lohnenswert, den Einfluss der Faktoren auf andere Variationsphänomene zu untersuchen, z.~B. auf Variation in der Kasusrektion von Präpositionen oder auf Variation zwischen langen und kurzen Genitivendungen. 

Empirisch ließ sich der Einfluss der Faktoren jeweils klar im Antwortverhalten der Proband\_innen fassen, aber nur bedingt in den Reaktionszeitmessungen. Hierdurch werden die Herausforderungen sichtbar, die Reaktionszeitmessungen mit sich bringen. Der Einfluss von Frequenz auf Reaktionszeiten ist messbar, der Einfluss von Prototypizität jedoch aufgrund der starken Assoziation von Bewegungssemantik und \textit{sein}-Selektion nur bedingt. Die Form-Schematizität ließ sich gar nicht in Reaktionszeitmessungen fassen. 

Gleichzeitig verdeutlichen die Studien aber auch das Potential von Reaktionszeitmessungen: Im Fall der Verben ohne attestierte Schwankung in Korpora können Reaktionszeiten Hinweise auf Variation geben, auch wenn diese noch nicht bzw. kaum in der Sprachproduktion zu beobachten ist. Hinsichtlich der Selektion von \textit{haben} und \textit{sein} zeigen die erhöhten Reak\-tionszeiten bei transitiven Sätzen die feste Assoziation von \textit{sein} mit Bewegungssemantik. Der empirische Teil der Dissertation unterstreicht, wie fundamental der Blick auf Prozessierung für die Modellierung von Variation und Grammatik aus einer gebrauchsbasierten Sicht ist. Erst mit Blick auf die Prozessierung zeigt sich das Variationspotential, das noch nicht im Sprachgebrauch sichtbar ist. Zudem zeigt Prozessierung, dass Prototypeneffekte sich nicht unbedingt in Stabilität äußern müssen, sondern komplexer sind, wie im Fall der erhöhten Reaktionszeiten bei transitiven Sätzen mit Bewegungsverben. Diese Erkenntnisse lassen sich nicht mit dem alleinigen Blick auf den Sprachgebrauch gewinnen. Die Hypothesen zu Variation, die auf Basis von Korpusdaten gemacht werden, sollten daher anhand von psycholinguistischen Studien systematisch überprüft werden. Reaktionszeitstudien bieten Einblicke in Variation, die eine bloße Betrachtung der Sprachproduktion nicht leisten kann. Auf diese Weise ermöglichen es Reaktionszeiten, Variation im Allgemeinen und Variationsphänomene im Speziellen besser zu verstehen.   
