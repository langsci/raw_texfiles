\documentclass[output=paper]{langsci/langscibook}
\ChapterDOI{10.5281/zenodo.4280649}

\author{Željko Bošković\affiliation{University of Connecticut}}
\title{On the coordinate structure constraint and the adjunct condition}

% \chapterDOI{} %will be filled in at production

\abstract{The paper argues for a unification of the ban on extraction out of
    conjuncts and the ban on extraction out of adjuncts\is{adjunction} based on
    the semantics of traditional \isi{adjunction} modification on which such
    modification actually involves \isi{coordination}, with ConjP present in
    the syntax of traditional adjunct modification. It is shown that there are
    a number of similarities in the islandhood of conjuncts and the
    islandhood\is{islands} of adjuncts\is{adjunction}. Thus, extraction out of
    conjuncts and extraction out of adjuncts\is{adjunction} are shown to be
    exceptionally possible in exactly the same environments, which can be
    captured if the two involve the same syntactic configuration. The proposed
    analysis is also shown to capture in a principled way a number of
    differences in the strength of the violation with extraction out of
    conjuncts and adjuncts\is{adjunction} in various languages/contexts, the
emphasis regarding the former being on \ili{Galician}, \ili{English},
\ili{Japanese}, and \ili{Serbo-Croatian}.}

\maketitle

\begin{document}\glsresetall

%\noindent\textbf{Keywords.} \isi{adjunction}, \isi{coordination}, \isi{islands}, the Adjunct
%Condition, the coordinate structure constraint

\section{Introduction}\label{sec:35.1}

The goal of this paper is to explore the possibility of a unification of two
rather ill-understood \isi{islands}, namely the \gls{CSC}\is{coordinate structure
constraint} and the \isi{adjunct condition} (AC). The \gls{CSC}\is{coordinate
structure constraint} is standardly assumed to have two parts, given in \eqref{ex-26:1} and
\eqref{ex-26:2} below.  However, recent research has shown that the two parts of the
traditional \gls{CSC}\is{coordinate structure constraint} need to be separated,
since there are languages which are sensitive to only one of the constraints in
(\ref{ex-26:1}--\ref{ex-26:2}). \textcite{Oda:2017} in fact explicitly argues for their separation,
providing strong arguments to this effect based on a number of languages. Thus,
he notes that \ili{Japanese} observes \eqref{ex-26:1}, but not \eqref{ex-26:2}, allowing extraction of
conjuncts but not extraction out of conjuncts. The same holds for \gls{SC}, as
discussed in \textcite{Stjepanovic2014} (see \citealt{Oda:2017} for a list of
languages that obey \eqref{ex-26:1} but not \eqref{ex-26:2}). In light of their arguments, I will also
separate the two parts of the traditional \gls{CSC}\is{coordinate structure
constraint},\footnote{On separating the two parts of the
\gls{CSC}\is{coordinate structure constraint}, see also \citet{Grosu1973} and
\citet{Postal1998}.} focusing on \eqref{ex-26:1} (though I will also make some remarks
regarding \eqref{ex-26:2} below).  As a result, for ease of exposition I will use the term
\gls{CSC}\is{coordinate structure constraint} to refer only to \eqref{ex-26:1}. (Where it
is necessary to make a distinction between \eqref{ex-26:1} and \eqref{ex-26:2} I will use the terms
CSC-1\is{coordinate structure constraint} and CSC-2\is{coordinate structure
constraint} respectively.)

\ea\label{ex-26:1}The \glsdesc{CSC} -- extraction out of conjuncts\hfill\hbox{(CSC-1)}\\
	Extraction out of conjuncts is disallowed.
\ex\label{ex-26:2}The \glsdesc{CSC} – extraction of conjuncts\hfill\hbox{(CSC-2)}\\
	Extraction of conjuncts is disallowed.
\z

Turning to adjuncts\is{adjunction}, the traditional ban on extraction out of adjuncts\is{adjunction} is given
in \eqref{ex-26:3}.

\ea\label{ex-26:3}The \gls{AC}\\
	Extraction out of adjuncts\is{adjunction} is disallowed.
\z

The paper will explore the possibility of a unification of \eqref{ex-26:1} and \eqref{ex-26:3}, which
are illustrated by \eqref{ex-26:4} and \eqref{ex-26:5} respectively.\footnote{The slight difference in
the grammaticality status of \eqref{ex-26:4} and \eqref{ex-26:5} will be accounted for under the
unified analysis proposed below.}

\begin{exe}
\judgewidth{?*}
\ex[*]{%
    What\tss{i} did you see [a picture of t\tss{i}] and a painting of Storrs?}%
    \label{ex-26:4}
\ex[?*]{%
    What\tss{i} did you fall asleep [after John had fixed t\tss{i}]?}%
    \label{ex-26:5}
\end{exe}

Before getting into the issue of islandhood\is{islands} of conjuncts and adjuncts\is{adjunction}, a brief
note is in order regarding extraction of conjuncts and adjuncts\is{adjunction}. It is
standardly assumed that conjuncts and adjuncts\is{adjunction} differ in this respect,
conjuncts being unmovable and adjuncts\is{adjunction} movable. It is actually not clear that
this is indeed the case. Thus, as noted above, many languages allow extraction
of conjuncts. Furthermore, a number of authors have argued that what looks like
adjunct extraction actually involves base-generation of adjuncts\is{adjunction} in their
surface position \parencite[e.g.][]{Uriagereka1988,Law1993,Stepanov2001a}. The
standard assumptions in this respect are thus incorrect, at least with respect
to conjuncts. At any rate, as noted above, the goal of this paper is not to
examine extraction of conjuncts and adjuncts\is{adjunction}, but islandhood\is{islands} of conjuncts and
adjuncts themselves (i.e.\ extraction out of conjuncts and adjuncts), though
some remarks regarding extraction of conjuncts and adjuncts\is{adjunction} will be made below
from the perspective of a unified analysis of \eqref{ex-26:1} and \eqref{ex-26:3} (more precisely, it
will be shown that \eqref{ex-26:2} is not an impediment to such an analysis).

The starting point in the discussion will be the semantics for adjuncts\is{adjunction} given
in \citet{Higginbotham1985}. \citeauthor{Higginbotham1985} argues that
traditional \isi{adjunction} modification (henceforth traditional adjuncts) actually
involves \isi{coordination} semantically.\footnote{There is a long line of research
    in this tradition, see e.g.\
    \citet{Davidson1967,Parsons1980,Parsons1990,Dowty1989,%
    Takahashi1994,Progovac1998,Progovac1999,Hunter2011}. I refer to
    \citet{Higginbotham1985} as the representative of this line of research
because \citet{Takahashi1994} bases his account of the \glsdesc{AC}\is{adjunct condition} on it, as
discussed below (following \citeauthor{Takahashi1994}, I also generalize this
approach to adjunct modification in general).} For example, the rough semantics
of (\ref{ex-26:6}a) is something like (\ref{ex-26:6}b), which can be paraphrased as \emph{There is an
event which is walking by John and it is slow}.

\ea\label{ex-26:6}
	\ea John walked slowly.
    \ex $\exists e$[Walk(John, $e$) and Slow($e$)]
	\z
\z

\citet{Takahashi1994} made an important observation that under
\citeauthor{Higginbotham1985}’s semantics of adjuncts\is{adjunction}, where adjuncts
essentially involve \isi{coordination}, it may be possible to unify the ban on
extraction out of conjuncts and the ban on extraction out of adjuncts\is{adjunction} by
reducing the latter to the former.\footnote{It is worth noting here that
    \citet{Ross1974} suggested a unification of the \gls{CSC}\is{coordinate structure constraint} with the complex
    NP constraint (clausal complements of nouns are also sometimes treated as
adjuncts, see e.g.\ \citealt{Stowell1981,Takahashi1994}).} Under
\citeauthor{Higginbotham1985}’s semantics, where adjuncts\is{adjunction} are in fact
conjuncts, extraction out of an adjunct does involve extraction out of a
conjunct, which makes the unification plausible and appealing. The unification,
however, raises an issue. In \citeauthor{Takahashi1994}’s analysis, while
conjuncts and adjuncts\is{adjunction} are treated in the same way semantically (following
\citeauthor{Higginbotham1985}), they are treated very differently
syntactically, since \citeauthor{Takahashi1994} follows standard assumptions in
the syntactic literature where \isi{coordination} involves the presence of a
conjunction phrase (ConjP), while adjuncts\is{adjunction} involve \isi{adjunction}, with no ConjP
present. Thus, the direct object in \eqref{ex-26:4} is a ConjP, with the conjuncts located
in the Spec and the complement position of ConjP ((\ref{ex-26:7}); the issue of where
exactly the conjuncts are located within ConjP is debated in the literature
\parencite[see e.g.][]{Munn1993,Progovac1999}, the details of their placement
will not matter for our purposes). On the other hand, there is no ConjP in \eqref{ex-26:5}.
Semantically, the VP and the traditional adjunct are conjoined here.  However,
this is not reflected in the structure, since Takahashi assumes, following
standard assumptions, that the adjunct is adjoined\is{adjunction} to VP, as in \eqref{ex-26:8}.

\begin{exe}
\judgewidth{?*}
\ex[*]{%
    Who\tss{i} did you see [\tss{ConjP} [a picture of
    t\tss{i}] and [a painting of Storrs]]?}\label{ex-26:7}

\ex[?*]{%
    What\tss{i} did you [\tss{VP} [\tss{VP} fall
    asleep] [after John had fixed t\tss{i}]]?}\label{ex-26:8}
\end{exe}

A serious issue then arises: locality of \isi{movement} is standardly assumed to be a
syntactic effect. However, under the above analysis, conjuncts and adjuncts\is{adjunction} are
unified only semantically, they are not unified syntactically in that they
involve very different syntactic configurations. It is then not clear that
\citeauthor{Higginbotham1985}’s conjunction semantics of adjuncts\is{adjunction} can help us
here.

While this paper will also take the conjunct semantics of adjuncts\is{adjunction} seriously,
taking it in fact as the point of departure, it will also take seriously the
issue of the syntax-semantics mapping here. An obvious question arises in this
respect: What would be the syntax that would most straightforwardly correspond
to the conjunct semantics of adjuncts? The answer is quite obvious in fact. It
is a syntax that involves a ConjP, where e.g.\ VP and the adjunct in \eqref{ex-26:6} are
conjoined. The only difference with true \isi{coordination} would then be that the
conjunction head is phonologically null.\footnote{This is in fact what
    \textcite{Progovac1998,Progovac1999} argues for. Thus, \citet{Progovac1998}
    adopts the structure in (i), where VP is the Spec of ConjP and the
    adverbial is a complement of a null conjunction (the structure is slightly
    richer in \citealt{Progovac1999}).

    \begin{exe}
        \exi{(i)} {}[\tss{ConjP} VP [\tss{Conj$'$} Conj AdvP]]
    \end{exe}

    \noindent In this respect, \textcite{Progovac1998,Progovac1999} is an important
    predecessor of the current work.

    It should also be noted that the discussion in this paper raises an issue
    of whether phrases are ever generated as adjuncts\is{adjunction} (in the traditional
    understanding of the term).  While the discussion in this paper falls in
    line with attempts to abandon \isi{adjunction} as a distinct structure-building
mechanism, showing that \isi{adjunction} can indeed be eliminated goes beyond the
scope of this paper.}

This paper will then take the conjunct semantics of adjuncts\is{adjunction} seriously,
assuming that it is also reflected in the syntax. From this perspective, it is
easy to see how \eqref{ex-26:1} and \eqref{ex-26:3} can be unified. Since they involve the same
configuration, whatever rules out extraction out of conjuncts will also rule
out extraction out of adjuncts\is{adjunction}.\footnote{There is an important issue that
    arises here. Under the analysis outlined above, not just the adjunct, but
    also the VP is a conjunct in constructions that involve traditional
    VP-adjunction.  It appears that extraction out of the VP should then also
    be ruled out here.  This is a serious issue that any unification of the
    \gls{CSC}\is{coordinate structure constraint} and the \glsdesc{AC}\is{adjunct condition} based on \citeauthor{Higginbotham1985}’s
    semantics of adjuncts\is{adjunction} needs to address. I will provide an account of this
    issue in~\Cref{sec:35.4} below (see \citealt{Takahashi1994} for an
alternative account which is however based on the assumption that conjuncts and
adjuncts\is{adjunction} have a different syntax).}

An important remark is, however, in order here. It seems fair to say that the
\gls{CSC} and the \glsdesc{AC}\is{adjunct condition} (\gls{AC}) are the least understood of the
traditional \isi{islands}. The suggestion made above reduces two mysteries to one.
Resolving this mystery, which would involve providing an actual account of the
\gls{CSC}, however, goes beyond the scope of this paper. Any attempt to do that
would involve a detailed discussion of the structure of \isi{coordination}, as well
as the theories of the locality of \isi{movement}, which is currently based on the
theory of \isi{phases}. A number of issues would arise in this respect: the precise
definition of \isi{phases}, the precise statement of the \gls{PIC}\is{phase
impenetrability condition}
and the notion of \emph{edge}, the issue of the generalized
\gls{EPP}\is{extended projection principle} effect as
it applies to successive-cyclic \isi{movement}, the theory of labeling\is{labelling}, which has
been argued to interact with the theory of \isi{phases} in the locality of \isi{movement}
effects \parencite[see][]{Boskovic2015,Boskovic2018}, etc; the list certainly
does not end here. Addressing all of this would go way beyond the scope of this
paper.\footnote{See, however, \citet{Boskovic2017,Boskovicinprep}.} The
scope of the paper is more modest: to point out a number of similarities
between extraction out of conjuncts and extraction out of adjuncts\is{adjunction} which can be
taken to justify unifying the two. \citeauthor{Higginbotham1985}’s semantics of
adjuncts, when taken seriously from the syntactic point of view, provides a
basis for such a unification since the two then have essentially the same
structure. Determining the precise source of islandhood\is{islands} of that structure is
beyond the scope of this paper (as a result, a number of phenomena noted below
will only be discussed at a descriptive level). I will therefore simply use the
term islandhood\is{islands} informally below. In several places, the discussion will become
more detailed structurally and theoretically when it comes to islandhood\is{islands} – in
fact, the paper will provide a principled account of a number of differences in
the strength of the violation with extraction out of various conjuncts and
adjuncts (as well as the voiding of their islandhood\is{islands} in certain cases);
however, the exact reason for the islandhood\is{islands} of conjuncts will not be provided
below. In this respect, the paper can be considered to be programmatic,
providing a foundation for future work that will account for the islandhood\is{islands} of
the syntactic configuration under consideration here (see
\citealt{Boskovicinprep}).

Having laid down the necessary background, the general line of argumentation,
and the limits of the current work, I now turn to making a case for unifying
\eqref{ex-26:1} and \eqref{ex-26:3}. In that vein, in \Cref{sec:35.2,sec:35.3}
I note a number of similarities between the \gls{CSC}\is{coordinate structure
constraint} and the \glsdesc{AC}\is{adjunct condition}. \Cref{sec:35.4} discusses and
resolves some potential impediments to the unification of the
islandhood\is{islands} of conjuncts and adjuncts\is{adjunction}.
\Cref{sec:35.5} discusses extraction of conjuncts and adjuncts\is{adjunction}.
\Cref{sec:35.6} concludes the paper.

\section{The stubbornness of the \glsentrytext{CSC} and the
\glsentrytext{AC}}\label{sec:35.2}

As discussed above, a unification of the traditional \isi{coordination} and the
traditional \isi{adjunction} has plausible semantic grounds, which can be taken to be
reflected in the syntax. From that perspective, it is not surprising that the
traditional \isi{coordination} and the traditional \isi{adjunction} share some syntactic
properties, in particular islandhood\is{islands}. The unification reduces two \isi{islands} to
one, which is already conceptually appealing, especially in light of the fact
that we are dealing here with a rather mysterious issue. (Admittedly, we still
have a mystery, but reducing two mysteries to one does leave us in a less
mysterious state).

One point that has generally been overlooked in the literature on islandhood\is{islands} is
worth emphasizing here. For pretty much all \isi{islands}, it has been noted that
there are languages that do not obey them. Thus, there are languages that do
not obey the subject condition (e.g.\ \ili{Japanese}; see \citealt{Stepanov2001b} for
a more exhaustive list), there are languages that do not obey the
\emph{wh}-island constraint (e.g.\ \ili{Swedish}, see \citealt{Engdahl1986}), there
are languages that do not obey the \isi{complex NP constraint} (e.g.\ \ili{Bantu}
languages, see \citealt{Boskovic2015}). The \gls{CSC}\is{coordinate structure constraint} and the \gls{AC} stand
out rather prominently in this respect. I am not aware of any language that
does not obey the \gls{CSC}\is{coordinate structure constraint} and the \gls{AC}.\footnote{As is well-known and as
    we will see below, there are particular coordinations and adjunctions that
    allow extraction (in fact likely universally). What I am referring to here
    is different, namely I am not aware of any language that would allow
extraction out of all coordinations and all adjuncts\is{adjunction}, where conjuncts and
adjuncts simply would not be \isi{islands} at all.} From the current perspective,
that the \gls{CSC}\is{coordinate structure constraint} and the \gls{AC} behave in the same way in this respect is
not surprising: we are after all dealing with one and the same constraint here
-- that the two behave in the same way in the relevant respect is then
expected.\is{CNPC|see{complex NP constraint}}

\section{Some exceptions to the \glsentrytext{CSC} and the
\glsentrytext{AC}}\label{sec:35.3}

\subsection{A semantically-based exception}\label{sub:35.3.1}

It is well-known that there are exceptions to both the \glsentrytext{AC}\is{adjunct condition} and the \glsentrytext{CSC}
(see \citealt{Truswell2011} and references therein for the former and
\citealt{Postal1998} and
references therein for the latter). Interestingly, some of these exceptions are
rather similar in nature. Thus, extraction from an adjunct is possible in some
cases where there is a contingent relationship between the relevant events.
Importantly, the same kind of exception is found with the \glsentrytext{CSC}. The former is
illustrated by \eqref{ex-26:9} and the latter by \eqref{ex-26:10}.\largerpage[2]

\ea\label{ex-26:9}
	\ea What\tss{i} did you come around [ to work on t\tss{i} ]?
	\ex What\tss{i} did Christ die [ to save us from t\tss{i} ]?%
        \hfill\parencite[131]{Truswell2011}
	\z
\ex\label{ex-26:10}
	\ea This is the drug which\tss{i} athletes [ take t\tss{i} ] and become quite strong.
	\ex the stuff which\tss{i} Arthur sneaked in and [stole t\tss{i}]%
        \hfill\parencite[53]{Postal1998}
	\z
\z

There are no good explanations for why under the semantic condition noted above
the \glsdesc{AC}\is{adjunct condition} effect and the \glsentrytext{CSC} effect are voided,
and I will not provide one in this work. What is important for our purposes is
that the two behave in the same way here. A unified approach to the two in this
respect has not been attempted before even at a descriptive level; what
complicates the situation even further when it comes to providing an actual
account is that only argument (both DP and PP) extraction is allowed in the
exceptional context in question, non-argument extraction is still unacceptable,
as illustrated below.

\ea[*]{How\tss{i} did you come around [to work on that car t\tss{i}]?}
\ex[*]{How\tss{i} should athletes [ take that drug t\tss{i} ] and
become strong?}
\z

This, however, further confirms that the \glsentrytext{CSC} and the \glsentrytext{AC}\is{adjunct condition} behave in
the same way here, which can be interpreted as calling for a unified analysis
of the two. The suggestion made here achieves this trivially, by treating the
\glsentrytext{CSC} and the \glsentrytext{AC}\is{adjunct condition} as one and the same phenomenon.

\subsection{Across-the-board \isi{movement} and parasitic gaps}\label{sub:35.3.2}\is{parasitic
gaps}

There is another well-known exception to the \glsentrytext{CSC} which is not
semantically based (i.e.\ it is not semantically restricted like the one noted
directly above). The exception, noted already in \citet{Ross1967}, concerns
\gls{ATB} movement\is{across-the-board movement}. As is well-known, an
unacceptable extraction out of a conjunct can be made acceptable if the
extraction takes place out of each conjunct in the
\isi{coordination}.\is{ATB|see{across-the-board movement}}

\ea\label{ex-26:13} Who did you see enemies of and friends of?
\ex\label{ex-26:14} cf.\ *Who did you see John and enemies of?
\z

There is an obvious counterpart of this with the \glsentrytext{AC}\is{adjunct condition}, which is the traditional
parasitic gap\is{parasitic gaps} construction \parencite[see
also][]{Haik1985,HuyvanRiem1985,Williams1990,Franks1993,Progovac1998,Nunes2004}.

\ea\label{ex-26:15} What did you file without reading?
\ex\label{ex-26:16} cf.\ *What did you file the book without reading?
\z

From the current perspective, (\ref{ex-26:15}--\ref{ex-26:16}) can be looked at on a par with
(\ref{ex-26:13}--\ref{ex-26:14}). Just like the unacceptable case of extraction out of a conjunct in
\eqref{ex-26:14} becomes acceptable if extraction takes place out of both conjuncts, as in
\eqref{ex-26:13}, so does the unacceptable case of extraction out of a conjunct in \eqref{ex-26:16}
(the traditional adjunct being a conjunct under the current analysis) become
acceptable if extraction takes place out of both conjuncts, as in \eqref{ex-26:15} (VP
being a conjunct under the current analysis; see below for extraction out of
the VP here).

There have in fact been many attempts to unify the
\gls{ATB}\is{across-the-board movement} and the parasitic\is{parasitic gaps}
gap construction (see the references cited above); the current perspective can
be taken to provide motivation for those attempts (\citealt{Takahashi1994} in
fact also argues for a unification of the two from the perspective of
\citeauthor{Higginbotham1985}’s semantic treatment of adjuncts\is{adjunction}
(recall, however, that \citeauthor{Takahashi1994} treats conjuncts and adjuncts
differently syntactically).

\subsection{The edge exception}\label{sub:35.3.3}

\textcite{Boskovic2018} notes another exception to the
\glsentrytext{AC}\is{adjunct condition}.  \textcite{Boskovic2018} shows that
the \glsentrytext{AC}\is{adjunct condition} effect is quite generally voided
for elements that are base-generated at the adjunct edge, also providing an
account of this state of affairs where the problem with extraction out of
adjuncts arises with \isi{movement} to the adjunct edge (which is required by
the \glsentrytext{PIC}); elements that are base-generated at the adjunct edge
can then extract.  The details of the account are not important for our
purposes; what is important is that elements base-generated at the edge of an
adjunct can extract out of it.

One illustration of this effect is provided by the different behavior of
agreeing \isi{possessors} and adnominal genitive\is{genitive case} complements
with respect to extraction out of adjuncts\is{adjunction} in \glsdesc{SC}
(\gls{SC}).  Consider first the former.  Agreeing \isi{possessors} in
\glsentrytext{SC} have been argued to be base-generated at the edge of the
\glsentrytext{TNP}.\footnote{The term \glsentrytext{TNP} is used neutrally, for
whatever the categorial status of the relevant element is.} As one argument to
that effect, consider the following \isi{binding} contrast between
\ili{English} and \glsentrytext{SC}, noted in \textcite{Despic2011,Despic2013}.

\ea\label{ex:35.17}
	\ea His\tss{i} latest movie really disappointed Kusturica\tss{i}.
	\ex Kusturica\tss{i}’s latest movie really disappointed him\tss{i}.
    \ex\label{ex-26:17c} \glsdesc{SC}\il{Serbo-Croatian} \parencites[31]{Despic2011}[245]{Despic2013}\\
        \gll    \llap{*}Kusturicin\tss{i} najnoviji  film ga\tss{i} je zaista razočarao.\\
                Kusturica’s  latest  movie  him  is  really  disappointed\\
	\ex
        \gll    \llap{*}Njegov\tss{i} najnoviji film je zaista razočarao Kusturicu\tss{i}.\\
                his latest movie is really disappointed Kusturica\\
	\z
\z

Under the assumption that traditional Specs c-command out of the phrase where
they are located, \citet{Kayne1994} takes the acceptability of (\ref{ex:35.17}a,b) to
indicate that \ili{English} \isi{possessors} are not located in SpecDP, but in the Spec of
a lower phrase, SpecPossP, with the DP confining the c-command domain of the
possessor. \textcite{Despic2011,Despic2013} observes that in \glsentrytext{SC}, a
language without articles which has been argued by a number of authors to lack
DP (e.g.\
\citealt{Corver1992,Zlatic1997,Trenkic2004,Boskovic2005,Boskovic2012,%
    Boskovic2014,Marelj2011,Despic2011,Despic2013,Runic2014a,Runic2014b,%
Takahashi2012,Talic2014,Talic2015}), \isi{possessors} do c-command out, as indicated
by the \isi{binding} violations in (\ref{ex:35.17}c,d) (condition B is at
issue in \ref{ex:35.17}c and condition C in \ref{ex:35.17}d), which contrast with \ili{English}
(\ref{ex:35.17}a,b). Despić takes the contrast in question as indicating
that DP is missing in \glsentrytext{SC}, with the possessor located in the
highest projection of the traditional NP.

Turning now to adjuncts\is{adjunction}, \glsentrytext{SC} is rather productive
regarding the possibility of \glspl{TNP} functioning as
adjuncts\is{adjunction}. One such case is given below, where an instrumental
nominal functions as an adjunct (see \citealt{Boskovic2018} for discussion of
such adjuncts).

\ea\label{ex-26:18}\glsdesc{SC}\il{Serbo-Croatian}\\
    \gll Trčao  je  šumom.\\
            run is forest.\Ins{}\\
	\glt    \enquote*{He ran through a/the forest.}
\z

\glsunset{ECP}
That the instrumental nominal in (\ref{ex-26:18}) is indeed an adjunct is
confirmed by extraction. First, its extraction out of \isi{islands} yields an
\gls{ECP}-strength, not a sub\-ja\-cency-strength violation (compare
\ref{ex-26:19}a,b).

\ea\label{ex-26:19} \ili{Serbo-Croatian}\\\judgewidth{??}
    \ea[*]{
        \gll    Šumom\tss{i} se pitaš [ kad je trčao t\tss{i} ].\\
                forest.\Ins{} \Refl{} wonder {} when is run\\
        \glt    \enquote*{You wonder when he ran through a/the forest.}}
	\ex[??]{
        \gll    Šumu\tss{i} se pitaš [ kad je posjekao t\tss{i} ].\\
                forest.\Acc{} \Refl{} wonder {} when is cut-down\\
        \glt    \enquote*{You wonder when he cut down a/the forest.}}
	\z
\z

In addition to agreeing \isi{possessors}, which roughly correspond to English
\emph{'s}-genitives, nominal arguments in \glsentrytext{SC} can be expressed through
adnominal genitive\is{genitive case}, which roughly corresponds to \ili{English} \emph{of}-genitives;
the element bearing adnominal genitive\is{genitive case} occurs in the complement position of the
noun. Returning now to the instrumental adjunct under discussion, notice that
while extraction of genitive\is{genitive case} complements of nouns is in general somewhat
degraded in \glsentrytext{SC}, (\ref{ex-26:20}a), which involves extraction
out of the nominal under consideration, is clearly worse than
(\ref{ex-26:20}b), which involves extraction out of an object. This
confirms the adjunct status of the instrumental \glsentrytext{TNP}
(\ref{ex-26:20}a is worse than \ref{ex-26:20}b because it involves
extraction out of an adjunct).

\ea\label{ex-26:20} \glsdesc{SC}\il{Serbo-Croatian}\\\judgewidth{??}
    \ea[*]{
        \gll    Moga djeda\tss{i} je trčao [ šumom t\tss{i} ].\\
                my.\Gen{} grandfather.\Gen{} is run {} forest.\textsc{instr}\\
        \glt    \enquote*{He ran through the forest of my grandfather.}}
    \ex[??]{
        \gll    Moga djeda\tss{i} je volio [ šumu t\tss{i} ].\\
                my.\Gen{} grandfather.\Gen{} is loved {} forest.\Acc{}\\
        \glt    \enquote*{He loved the forest of my grandfather.}}
	\z
\z

As noted above, \textcite{Boskovic2018} shows that in contrast to elements
that are not base-generated at an adjunct edge, elements that are
base-generated at an adjunct edge can be moved out of adjuncts\is{adjunction}. The adnominal
genitive \enquote*{my grandfather} in (\ref{ex-26:20}a) is base-generated
in the N-complement position.  Recall, however, that an agreeing possessor that
precedes the nominal is generated at the \glsentrytext{TNP} edge. Importantly,
such possessors can move out of the adjunct under consideration.\is{possessors}

\ea\label{ex-26:21}\glsdesc{SC}\il{Serbo-Croatian}\\
    \gll Ivanovom\tss{i} je on trčao [ t\tss{i} šumom ].\\
            Ivan’s.\Ins{} is he run {} {} forest.\Ins{}\\
	\glt    \enquote*{He ran through Ivan’s forest.}
\z

\textcite{Boskovic2018} provides a number of additional cases which also
show that elements that are base-generated at an adjunct edge can move out of
adjuncts, in contrast to those that are not generated at an adjunct
edge.\footnote{One such case is given in (i) (see \citealt{Boskovic2018} for
    an account why (i) is unacceptable in English).

\begin{exe}
    \exi{(i)}
    \gll    Izuzetno\tss{i} se on [ t\tss{i} loše ] ponašao?\\
            extremely is he {} {} badly {} behaved\\
    \glt    \enquote*{He behaved extremely badly.}
\end{exe}}

What is important for our purposes is that the \glsentrytext{CSC} behaves just like the
\glsentrytext{AC} in this respect. Recall that an agreeing possessor can extract out of
a \glsentrytext{TNP} adjunct, while an adnominal genitive\is{genitive case} cannot. Coordinations behave
in exactly the same way: an agreeing possessor\is{possessors} can extract out of a conjunct
\eqref{ex-26:22}, but an adnominal genitive\is{genitive case} cannot \eqref{ex-26:23}.\footnote{Left-branch extractions
    in \glsentrytext{SC} are best when the remnant precedes the verb, but the relevant
    contrast is also there when the \isi{coordination} follows the verb. Notice that
    there is an interfering factor when such extraction is attempted out of the
    second conjunct. As noted in \textcite{Stjepanovic2014} and discussed below,
    \emph{i} ‘and’ is a proclitic,\is{clitics} which procliticizes to the element following
it. A problem then arises if the element following it is a trace.}

\ea\label{ex-26:22}\glsdesc{SC}\il{Serbo-Croatian}\\
	\gll    Markovog\tss{i} je on [ t\tss{i} prijatelja ] i [ Ivanovu sestru ] vidio.\\
            Marko’s.\Acc{} is he {} {} friend.\Acc{} {} and {} Ivan’s.\Acc{}
            sister.\Acc{} {} seen\\
	\glt    \enquote*{He saw Marko’s friend and Ivan’s sister.}
\ex\label{ex-26:23}\glsdesc{SC}\il{Serbo-Croatian}\\
    \gll \llap{*}Fizike\tss{i} je on [ studenta t\tss{i} ] i [ Ivana ] vidio.\\
            physics.\Gen{} is he {} student.\Acc{} {} {} and {} Ivan.\Acc{} {} seen\\
    \glt    \enquote*{He saw a student of physics and Ivan.}%}
\z

What is important for our purposes is that both traditional adjuncts\is{adjunction} and
traditional conjuncts exceptionally allow extraction of elements that are
base-ge\-ne\-rat\-ed at their edge.

To sum up the discussion in this section, we have seen that in a number of
environments extraction is exceptionally possible out of conjuncts and
adjuncts. Significantly, the enviroments where extraction is exceptionally
possible out of conjuncts and adjuncts\is{adjunction} are the same – all the contexts
discussed in this section exceptionally allow extraction out of both conjuncts
and adjuncts\is{adjunction} (see below for an additional case). That the two behave in the
same way in this respect then provides an argument that they should be unified,
which is straightforwardly accomplished if they involve the same syntactic
configuration.

\section{Some differences between the \glsentrytext{CSC} and the
    \glsentrytext{AC} and rescue by PF deletion}\label{sec:35.4}

Above, I have discussed a number of similarities between \glsentrytext{CSC}
effects and \glsentrytext{AC}\is{adjunct condition} effects which can be captured under the analysis
on which traditional \isi{adjunction} actually involves \isi{coordination}, which is
motivated by Higginbotham’s semantics of \isi{adjunction}. There are, however, also
some differences between the two, which will be discussed in this section,
starting with an obvious difference.\footnote{A reviewer notes that
    \isi{coordination} and traditional \isi{adjunction} differ regarding gapping, compare
    \emph{John ate an apple and Mary a pear} with *\emph{John ate an apple
    after Mary a pear}. The difference can be accounted for under \citegen{Johnson2009} analysis of gapping (gapping is actually quite generally disallowed
in embedded clauses, even with \isi{coordination}).} Consider (\ref{ex-26:24}--\ref{ex-26:25}), which are
intended to represent a case of traditional \isi{coordination} \eqref{ex-26:24} and a case of
traditional \isi{adjunction} \eqref{ex-26:25}, which is also treated as involving coordination
under the current analysis.

\ea\label{ex-26:24} DP \& DP
\ex\label{ex-26:25} VP \& Adjunct
\z

The conjuncts in the traditional \isi{coordination} in \eqref{ex-26:24} are symmetric regarding
islandhood in that extraction is banned out of each conjunct (putting aside the
\glsentrytext{ATB}\is{across-the-board movement} case).

\ea\label{ex-26:26}
    \ea[*]{Who\tss{i} did you see [ a friend of t\tss{i} ] and John?}
    \ex[*]{Who\tss{i} did you see John and [ a friend of t\tss{i} ]?}
	\z
\z

However, this is not the case with \eqref{ex-26:25}, where extraction is not banned out of
the first conjunct, i.e.\ VP.

\ea\label{ex-26:27} What\tss{i} did you [ buy t\tss{i} ] slowly?
\z

A question then arises under the current analysis regarding the source of this
difference. In particular, what raises the issue here is the grammaticality of
\eqref{ex-26:27}, which appears to be unexpected.

As noted above, providing an account of the unacceptability of extraction out
of conjuncts goes beyond the scope of this paper. I simply assume here that
conjuncts are \isi{islands} (as explicitly also argued in \citealt{Oda:2017}). The
islandhood of conjuncts is apparently voided for the VP conjunct in \eqref{ex-26:27}. The
question is why. There is actually a rather straightforward answer to this
question.

\textcite{Boskovic2011,Boskovic2013b} discusses a variety of \isi{islands} from a
number of languages and observes that \isi{movement} of the head of an island voids
islandhood (for additional arguments to that effect,
see~\citealt{Boskovic2015}). Based on this, \citeauthor{Boskovic2015}
establishes the generalization in \eqref{ex-26:28}.

\ea\label{ex-26:28} Traces do not head \isi{islands}.
\z

\textcite{Boskovic2013b} provides a number of arguments for \eqref{ex-26:28}. As an
illustration, consider the saving effect of article \isi{incorporation} on islandhood
in \ili{Galician}, also discussed in \textcite{Uriagereka1988,Uriagereka1996}.
Galician has a rather interesting phenomenon of D-to-V \isi{incorporation}, which
quite generally voids islandhood\is{islands} of the DP from which the \isi{incorporation} takes
place \parencite[see][]{Uriagereka1988,Uriagereka1996,Boskovic2013b}. Thus,
Galician disallows \isi{movement} from definite DPs, as in \eqref{ex-26:29}. However, the
violation is voided when D incorporates into the verb, as shown by
\eqref{ex-26:30}.\footnote{As discussed in \citet{Uriagereka1988}, when the article
incorporates the final \emph{s} of the verb is truncated.} Further confirmation
of the islandhood-voiding effect of article \isi{incorporation} is provided by \eqref{ex-26:31}.
Extraction from adjuncts\is{adjunction} is banned in \ili{Galician}, as in \eqref{ex-26:31}.  However, the ban
is voided under D-incorporation, as in \eqref{ex-26:32} (the same holds for the subject
condition effect, which is also voided under article \isi{incorporation}).

\ea\label{ex-26:29} \ili{Galician} \parencite[81]{Uriagereka1988}\\
    \gll * e de quén\tss{i} viches [\tss{DP} o [\tss{NP} retrato t\tss{i}]]?\\
        {} and of who saw(you) {} the {} portrait\\
\ex\label{ex-26:30} \ili{Galician} \parencite[81]{Uriagereka1988}\\
    \gll e de quén\tss{j} viche-lo\tss{i} [\tss{DP} [\tss{D$'$} t\tss{i} [\tss{NP} retrato t\tss{j}]]]?\\
	        and of whom saw(you)-the {} {} {} {} portrait\\
    \glt \enquote*{So, who have you seen the portrait of?}
\ex\label{ex-26:31} \ili{Galician} \parencite[58]{Boskovic2016}\\
    \gll ?? de que semana\tss{j} traballastedes [\tss{DP} o Luns t\tss{j}]\\
		  {}  of which week worked(you) {} the Monday\\
    \glt \hphantom{??}\enquote*{Of which week did you guys work the Monday?}
\ex\label{ex-26:32} \ili{Galician} \parencite[58]{Boskovic2016}\\
	\gll    de  que  semana\tss{j}  traballastede-lo\tss{i} [\tss{DP} [\tss{D$'$} t\tss{i}  Luns t\tss{j}]] \\
            of  which week worked(you)-the {} {} {} Monday\\
\z

These cases illustrate the generalization in \eqref{ex-26:28}. The islandhood\is{islands}
of the DPs from \eqref{ex-26:29} and \eqref{ex-26:31} is voided in \eqref{ex-26:30} and \eqref{ex-26:32}, where the relevant
DPs are headed by a trace, due to the \isi{movement} of the head of the DP in
question.  \citeauthor{Boskovic2013b} (\citeyear{Boskovic2013b};\,\citeyear{Boskovic2015})
provides a number of other cases from a wide range of languages that illustrate
the same effect (thus, \citealt{Boskovic2013b} shows that, among other things,
\citegen{Baker1988} government transparency corollary
effects are also subsumed under (\ref{ex-26:28}); i.e.\ they also involve \isi{islands}
that are headed by a trace.) Under \eqref{ex-26:28}, if the head of an island α undergoes
movement, the islandhood\is{islands} of α is voided, making \isi{movement} out of α
possible.

\textcite{Boskovic2011,Boskovic2013b} also provides an account of the effect in
question, which unifies it with the rescuing effect that \isi{ellipsis} has on
islandhood, noted by \citet{Ross1969} and illustrated by \eqref{ex-26:33}.\footnote{See,
    however, \textcite{Abels2011,BarEllTho2014}.}

\ea\label{ex-26:33}
    \ea[*]{She kissed a man who bit one of my friends, but Tom does not realize
        [ which one of my friends ]\tss{i}  she kissed [ \textbf{a man who bit
        t\tss{i}} ].}
    \ex[]{She kissed a man who bit one of my friends, but Tom does not realize
        [ which one of my friends ]\tss{i} \sout{she kissed [ a man who bit
        t\tss{i} ]}.}\hfill\parencite[276]{Ross1969}
    \z
\z

The effect from \eqref{ex-26:33} is standardly treated in terms of rescue by \glsentrytext{PF} deletion
(\citealt{Chomsky:1972,Merchant2001,Lasnik2001,FoxLas2003,HorLasUr2003,%
BoeLas2006,Boskovic2011} among others): a * is assigned to an island when
movement crosses it. If the * remains in the final \glsentrytext{PF} representation, a
violation incurs. If a later operation like \isi{ellipsis} deletes the category that
contains the *-marked element, the derivation is rescued.  Under the standard
analysis, then, when \emph{wh}-movement crosses the island in \eqref{ex-26:33} the island
is *-marked in both (\ref{ex-26:33}a) and (\ref{ex-26:33}b). Since the *-marked element is deleted in
(\ref{ex-26:33}b) the islandhood\is{islands} effect disappears in this example.

\textcite{Boskovic2011,Boskovic2013b} also provides a
rescue-by-\glsentrytext{PF} deletion account of the generalization in \eqref{ex-26:28},
unifying \eqref{ex-26:28} with the rescuing effect of \isi{ellipsis} on islandhood\is{islands}.
\citeauthor{Boskovic2011} argues that what is *-marked is not the whole island,
but the head of the island. This means that in e.g.\ \eqref{ex-26:29}, what is *-marked is
the head of the object DP. The reason for the rescuing effect of \isi{head movement}
in \eqref{ex-26:30} is that the *-marked element in the head position of the object DP is
actually a copy that is deleted under \isi{copy deletion} in \glsentrytext{PF}. The
offending *-marked element is thus deleted in \glsentrytext{PF} in \eqref{ex-26:30}, just
as it is in \eqref{ex-26:33}. The analysis quite generally captures the generalization in
\eqref{ex-26:28}.\footnote{The analysis predicts that \isi{head movement} is not sensitive to
    (non-relativized minimality) \isi{islands}, more precisely, that the head
    of an island can move out of the island since the locality violation will
    be rescued by deleting the copy of the moved head (the prediction holds
    only for the head of the island and does not hold for relativized
    minimality -- i.e head-movement constraint -- violations; see
    \citealt{Boskovic2013b}). \textcite{Boskovic2013b} provides a number of
    cases from a variety of languages that this is indeed the case (in fact,
    \ili{Galician} article \isi{incorporation} -- cf.\ \REF{ex-26:32}~--, which is also
    acceptable without \emph{wh}-movement, is one such case; see also
\citealt{Boskovic2013b} on noun \isi{incorporation} in \ili{Kinyarwanda},
\ili{Chichewa}, and \ili{Southern Tiwa}).} (\citealt{Boskovic2011} also extends
the analysis to the generalization that traces do not count as interveners
\parencite{Chomsky1995}.  In the relevant cases, the *-marked intervener is
also removed under \glsentrytext{PF} \isi{copy deletion}, see the discussion below).

At any rate, what is important for our purposes is that \isi{head movement} voids
islandhood: if the head of an island undergoes \isi{movement}, the islandhood\is{islands} effect
disappears, making \isi{movement} out of the island possible.

Returning to the potentially problematic case in \eqref{ex-26:27}, we now have a
straightforward explanation why \isi{movement} out of the VP, which is a conjunct
hence an island under the current analysis, is allowed in this case. The reason
is V-to-v \isi{movement}.\footnote{ There are various proposals in the literature
    regarding the exact identity of the relevant head and the height of
    V-movement (e.g.\ we could be dealing here with a vP conjunct, with the
    verb moving to VoiceP above vP, see \citealt{Collins2005}); I simply use v
    for ease of exposition.} Being a conjunct, the VP (i.e.\ the bracketed
    element) in \eqref{ex-26:27} is an island. However, V-to-v \isi{movement}, i.e.\ \isi{movement} of
    the head of the VP, voids the islandhood\is{islands} of the VP, allowing \isi{movement} out
    of this VP, as in \eqref{ex-26:27}. The grammaticality of \eqref{ex-26:27} is then just another
    instance of the general rescuing effect of \isi{head movement} on islandhood\is{islands},
    given in \eqref{ex-26:28}. The potential obstacle to the unification of the \glsentrytext{CSC}
    and the \glsentrytext{AC}\is{adjunct condition} that was raised by \eqref{ex-26:27} is thus rather straightforwardly
    resolved; the reason for the grammaticality of \eqref{ex-26:27} is an independent and
    more general effect regarding locality of \isi{movement}.

The analysis does not only remove a potential problem for the unification of
the \glsentrytext{CSC} and the \glsentrytext{AC}\is{adjunct condition} raised by \eqref{ex-26:27} but it also makes a prediction.
Consider again (\ref{ex-26:24}--\ref{ex-26:25}). Just like in \eqref{ex-26:25} \isi{movement} of the head of the VP
conjunct makes \isi{movement} out of the VP possible so should \isi{movement} of the head
of the corresponding conjunct in \eqref{ex-26:24} make \isi{movement} out of this conjunct
possible. The prediction can in fact be tested with respect to \ili{Galician}. The
issue here is whether article \isi{incorporation} in \ili{Galician} also improves
extraction out of a conjunct. It turns out that it does. Consider (\ref{ex-26:34}--\ref{ex-26:35})
(the \ili{Galician} data below are due to Juan Uriagereka, p.c.; \emph{a} in
(\ref{ex-26:34}--\ref{ex-26:35}) is a differential object
marker).\is{differential object marking}\is{differential object marking}

\ea\label{ex-26:34} \ili{Galician}\\
    \gll    * De quén\tss{i} vistedes [ o amigo t\tss{i} ] e-mais [ a Xan ] onte?\\
            {} of who (you)saw {} the friend {} {} and {} \Dom{} Xan {} yesterday\\
    \glt    \hphantom{*}intended: \enquote*{You saw [[the friend of who] and [Juan]] yesterday?}
\ex\label{ex-26:35} \ili{Galician}\\
    \gll ?? De quén\tss{i} vistede-lo\tss{j} [ t\tss{j} amigo t\tss{i} ] e-mais [ a Xan ] onte?\\
         {} of who (you)saw-the {} {} friend {} {} and {} \Dom{} Xan {} yesterday\\
\z

\eqref{ex-26:34} shows that extraction out of a conjunct is not possible in \ili{Galician}, i.e.\
conjuncts are \isi{islands}. Importantly, \eqref{ex-26:35}, which involves article \isi{incorporation}
from the conjunct from which \emph{wh}-movement takes place, is clearly better
than \eqref{ex-26:34}, which does not involve article \isi{incorporation}. Article \isi{incorporation}
thus also improves extraction out of conjuncts.

Putting for the moment the residual awkwardness of \eqref{ex-26:35} aside, and focusing on
the fact that \eqref{ex-26:35} is better than \eqref{ex-26:34}, the current analysis unifies the
grammaticality of \eqref{ex-26:27} with the improvements that article \isi{incorporation} causes
for \emph{wh}-movement in (\ref{ex-26:31}--\ref{ex-26:32}) and (\ref{ex-26:34}--\ref{ex-26:35}). All the relevant cases
involve extraction out of a conjunct where the head of the conjunct undergoes
movement.

Consider now why, in contrast to \eqref{ex-26:27} and \eqref{ex-26:32}, \eqref{ex-26:35} is still degraded
(although better than \eqref{ex-26:34}, which is what is crucial here for our
purposes).\footnote{\eqref{ex-26:32} is actually slightly awkward (meriting at most ?).
    The proposal below will not explain the residual awkwardness of \eqref{ex-26:32}, which
    I leave open here (also putting it aside below), merely noting that there
    may be a weak intervention effect associated with phrasal \isi{movement} from the
    second conjunct crossing the first conjunct, also a phrase (\ref{ex-26:32} is in fact
    fully acceptable if it involves only head-movement/article \isi{incorporation},
    see \citealt{Boskovic2013b}); in this respect compare also \eqref{ex-26:35} with \eqref{ex-26:39}
    below and note that (\ref{ex-26:26}b) is worse than (\ref{ex-26:26}a); for discussion of the effect
in question, which I put aside here, see \textcite{Boskovicinprep}, who also
shows that the effect is selective in that it depends on labeling\is{labelling} (so it does
not arise in all relevant contexts).} \textcite{Oda:2017} captures the two
parts of the \glsentrytext{CSC}, i.e.\ (\ref{ex-26:1}--\ref{ex-26:2}), by proposing that both individual
conjuncts and ConjP are \isi{islands}. What this entails for our purposes is that
with extraction out of a conjunct, what is *-marked is the head of the conjunct
itself, as well as the head of ConjP (given that what is *-marked is the head
of an island). In \eqref{ex-26:34}, both *-marked heads survive into
\glsentrytext{PF}, hence the strong unacceptability of the construction. On the
other hand, in \eqref{ex-26:35}, the *-marked head of the conjunct is
removed in \glsentrytext{PF} through copy-deletion. However, the *-marked head
of ConjP is still present in \glsentrytext{PF}. I suggest that this is the
reason for the residual awkwardness of \eqref{ex-26:35}.
Article-incorporation voids the islandhood of the conjunct itself, by turning
its head into a trace (i.e.\ a copy that is deleted in \glsentrytext{PF}).
However, it does not affect the islandhood\is{islands} of ConjP. The analysis
thus captures the contrast between \eqref{ex-26:34} and
\eqref{ex-26:35}, as well as the fact that \eqref{ex-26:35} itself is
still degraded.

What about \eqref{ex-26:27} and \eqref{ex-26:32}, which involve traditional adjunction? I suggest that
what is important here is that the ConjP head in these examples is
phonologically null. In this respect, the head of ConjP in \eqref{ex-26:27} and \eqref{ex-26:32} in
fact does not differ from the head of the first conjunct in \eqref{ex-26:27} and the second
conjunct in \eqref{ex-26:32} – in all these cases the relevant head is phonologically null.
Now, it is standardly assumed that intervening heads block \isi{head movement} (see
e.g.\ \citealt{Roberts2010}). There is an additional implicit assumption here:
in all the cases that are traditionally given as an illustration of this effect
the blocking head is overt. This is in fact reminiscent of another standard
assumption, noted briefly above, that traces do not count as
interveners.\footnote{ Notice that there is no conflict between the assumption
    that traces do not count as interveners for extraction and the blocking
    effect of \emph{wh}-traces on \emph{wanna}-contraction. Under \isi{multiple
    spell-out} (see \citealt{Uriagereka1999,Epstein1999,Chomsky2000,Chomsky2001}
    among many others), it is not a \emph{wh}-trace but the \emph{wh}-phrase
    itself that blocks \emph{wanna}-contraction (see \citealt{Boskovic2013c},
    where it is shown that this kind of approach also captures the traditional
    claim that NP-traces do not block contraction; traces actually never block
contraction, only heads of chains do under a \isi{multiple spell-out} analysis).}
What traces and null heads have in common is that they are both phonologically
null; this means that null elements do not count as interveners.
\textcite{Boskovic2011} in fact provides a rescue by \glsentrytext{PF} deletion account of the
trace case that can be generalized to the null head case.
\textcite{Boskovic2011} argues that with intervention effects, what is *-marked
is the intervener itself. With traces, the intervener is deleted in \glsentrytext{PF}, which
voids the intervention effect.  Another way to look at this is that the
locality effect is voided if the *-marked element is not realized (i.e.\
pronounced) in \glsentrytext{PF}, i.e.\ a * induces a violation in \glsentrytext{PF} only if it is \glsentrytext{PF}
realized, i.e.\ if it is present on a PF-realized element.\footnote{Though see
below for a potential alternative.}

There is independent evidence for the above account of \eqref{ex-26:27},
where the reason why \eqref{ex-26:27} does not display the
\glsentrytext{CSC} effect, although \isi{adjunction} is treated as
\isi{coordination}, is that the ConjP head is phonologically null here.
\textcite{Progovac1998,Progovac1999}, who also argues for a unified analysis of
\isi{coordination} and traditional \isi{adjunction} based on the
\isi{coordination} analysis of the latter, observes that in some cases the
ConjP head can in fact be overt with traditional \isi{adjunction} based on
examples like \eqref{ex-26:36}. Importantly, extraction out of the VP
conjunct is degraded in such cases: (\ref{ex-26:37}a,b) are worse than
\eqref{ex-26:27}. This is exactly what is expected: since the *-marked head
of ConjP is phonologically realized in (\ref{ex-26:37}a,b), in contrast to
\eqref{ex-26:27}, examples (\ref{ex-26:37}a,b) are degraded, in
contrast to \eqref{ex-26:27}.

\begin{exe}
\judgewidth{??}
\ex\label{ex-26:36}
	\ea Mary read his paper, and quickly.
	\ex John read the book, and avidly.
	\z
\ex\label{ex-26:37}
    \ea[??]{What did Mary read, and quickly?}
    \ex[??]{What did John read, and avidly?}
	\z
\end{exe}

We now have all we need to account for the full paradigm under consideration.
In \eqref{ex-26:27} and \eqref{ex-26:32}, both the islandhood\is{islands} of the relevant individual conjuncts and
the islandhood\is{islands} of ConjP is voided since both the head of the relevant conjuncts
and the head of ConjP are phonologically null. On the other hand, in \eqref{ex-26:35}, only
the head of the conjunct is null, which means that the islandhood\is{islands} of the
conjunct, but not the islandhood\is{islands} of ConjP, is voided here. Notice also that
\eqref{ex-26:34} is worse than \eqref{ex-26:31}, which is also captured under the current analysis.
\eqref{ex-26:34} in a sense involves two violations, since the heads of both \isi{islands}, the
relevant conjunct and ConjP, are phonologically overt. On the other hand, in
\eqref{ex-26:31} only the former is phonologically overt: the islandhood\is{islands} of ConjP is voided
here since the head of ConjP itself is phonologically null. Furthermore, notice
that standard \glsentrytext{CSC} violations like (\ref{ex-26:26}a) are worse than traditional
adjunction cases with an overt conjunction like \eqref{ex-26:37}. This is also expected and
can be accounted for on a par with the contrast between \eqref{ex-26:31} and \eqref{ex-26:34}: (\ref{ex-26:26}a)
involves two island violations since both the head of the conjunct island and
the head of ConjP are overt while in \eqref{ex-26:37} only the head of ConjP is overt. The
proposed analysis thus captures the full paradigm in (\ref{ex-26:26}--\ref{ex-26:27}, \ref{ex-26:31}--\ref{ex-26:32},
\ref{ex-26:34}--\ref{ex-26:35}, and \ref{ex-26:37}): it captures the fact that \eqref{ex-26:27} and \eqref{ex-26:32} are better than
the rest of this paradigm, the contrast between \eqref{ex-26:34} and \eqref{ex-26:35} as well as the
fact that \eqref{ex-26:35} is still degraded, and the fact that \eqref{ex-26:34} is more strongly
degraded than \eqref{ex-26:31} and that \eqref{ex-26:26} is more strongly degraded than
\eqref{ex-26:37}.\footnote{One issue that I will put aside here is whether extraction out
    of all conjuncts can be saved by \isi{movement} of the conjunct head. What is
    important for us is that this is in principle possible, hence needs to be
    allowed. Whether there are factors that constrain the effect in question
will be left for future research (see \citealt{Boskovic2017}, where it is
argued that the status of a conjunct with respect to phasehood\is{phases} matters here;
for relevant discussion see also \citealt{Boskovicinprep}).}

What is particularly important for our purposes is that the current analysis
unifies the grammaticality of \eqref{ex-26:27} and the improvement that article
incorporation causes in (\ref{ex-26:34}--\ref{ex-26:35}). In both cases we are dealing with
extraction out of a conjunct where the head of the conjunct undergoes \isi{movement},
voiding the islandhood\is{islands} of the conjunct. The grammaticality of \eqref{ex-26:27} then turns
out not only not to be a problem for the unified \glsentrytext{CSC}/\glsentrytext{AC} analysis,
but it in fact has its counterpart with the traditional \glsentrytext{CSC}, thus
providing an argument for the unified analysis. In other words, we are dealing
here with another case where \isi{movement} out of a conjunct is exceptionally
allowed, which also extends to traditional \isi{adjunction}. In fact, the effect
holds not only for what under the traditional view would be considered to be
the \enquote{host} of \isi{adjunction}, i.e.\ the VP in \eqref{ex-26:25}, but also for the
traditional adjunct itself.  As shown in (\ref{ex-26:31}--\ref{ex-26:32}), the islandhood\is{islands}
of extraction out of adjuncts\is{adjunction} is also voided under \isi{movement} of
the adjunct head. I conclude therefore that what appeared here to be a
difference between the \glsentrytext{CSC} and the \glsentrytext{AC}\is{adjunct
condition} is in fact another case where the two behave in the same way, which
can be added to the cases discussed in \Cref{sec:35.3}: both the
\glsentrytext{CSC} and the \glsentrytext{AC}\is{adjunct condition} effect are
voided under \isi{head movement} of the head of the conjunct/adjunct.

There is still one missing piece needed to complete the paradigm regarding the
rescuing effect of \isi{head movement} on extraction from conjuncts. Returning to
(\ref{ex-26:24}--\ref{ex-26:25}), we have seen that \isi{head movement} rescues extraction out of both
conjuncts in the traditional \isi{adjunction} case in \eqref{ex-26:25}, i.e.\ it makes extraction
out of both VP and the traditional adjunct possible. Regarding \eqref{ex-26:24}, we have
seen that \isi{head movement} of the head of the conjunct makes extraction out of the
first conjunct possible. The remaining piece of the puzzle concerns extraction
out of the second conjunct in \eqref{ex-26:24}. Does \isi{head movement} of the head of that
conjunct make extraction out of it possible? We have confirmed the rescuing
effect of \isi{head movement} on extraction out of a conjunct regarding the first
conjunct in \eqref{ex-26:24} with article \isi{incorporation} in \ili{Galician}. Does the effect also
hold for extraction from the second conjunct? In fact, it does. Conjunction
\emph{e mais} in \ili{Galician} can host article \isi{incorporation}. Crucially, extraction
out of the second conjunct is worse in \eqref{ex-26:38} than in \eqref{ex-26:39}, the difference here
being that the article head of the second conjunct, from which
\emph{wh}-extraction takes place, undergoes \isi{incorporation} only in \eqref{ex-26:39}. (Not
surprisingly given the above discussion, while better than \eqref{ex-26:38}, \eqref{ex-26:39} is still
degraded.)

\ea\label{ex-26:38}Galician\\
    * \parbox[t]{10cm}{\gll De qué cidade\tss{i} vistedes um retrato de Diego e mais [ a  paisaxe t\tss{i}]?\\
     of what city (you)saw a portrait of Diego and {} {} the landscape\\}
\ex\label{ex-26:39}Galician\\
   ??? \parbox[t]{10cm}{\gll De qué cidade\tss{i} vistedes um retrato de Diego e-mai-la\tss{j} [t\tss{j} paisaxe t\tss{i}]?\\
      of what city (you)saw a portrait of Diego and-the {}  landscape\\}
\z

I will conclude the discussion in this section with an example which can be
analyzed in several ways within the approach argued for here. The example is
given in \eqref{ex-26:40}.

\ea[*]{What\tss{i} did you see [pictures of t\tss{i}] and paintings of
Storrs?}\label{ex-26:40}
\z

The conjunct from which extraction takes place in \eqref{ex-26:40} is most often assumed to
be a DP, headed by a null D. Given the grammaticality status of \eqref{ex-26:40}, here we
do want the *-marking on the head of the conjunct to contribute to the
ungrammaticality of the example.

There are several possibilities here. One possibility is that the conjunct is
actually smaller than DP, with the noun located in (possibly moving to) the
head position of the conjunct. Nothing special would then need to be said about
such cases.

If the conjunct is a DP, with the noun located lower than D, we could assume
that this is actually a D that is deleted in \glsentrytext{PF}, with \glsentrytext{PF} D-deletion either not
yet having taken place at the point when *-marking is checked, or with
*-marking interfering with the required D deletion here. However, what may be
relevant here is that DP is a phase\is{phases}, in contrast to ConjP (see
\citealt{Boskovic2017} for relevant discussion). In light of this, it is
possible that, as suggested above, *-marking on null heads never matters (i.e.\
it does not induce a \glsentrytext{PF} violation) but that \mbox{*-marked} heads are unable to send
their complement to spell-out. The standard assumption is that phasal heads
send their complement to spell-out \emph{after} all their uninterpretable
features are checked; under the suggestion made here *-marking has a similar
effect to uninterpretable features in that it prevents spell-out. As a result,
the *-marked null D in \eqref{ex-26:40} would not be able to send its complement to
spell-out.\footnote{I assume that spell-out must take place for each phasal
    level, which means that we do have a violation here.  Notice also that
    there is still a difference here with the \ili{Galician} case in \eqref{ex-26:30}, where the
    *-marked element in D is deleted under \isi{copy deletion}. Under the analysis
    under consideration, the spell-out for the DP phase\is{phases} in \eqref{ex-26:30} would be
    triggered only after D-incorporation\is{incorporation} (with \isi{copy deletion} appropriately
    ordered), which is in fact in line with \citegen{Chomsky2001} proposal that
    the spell-out for phase\is{phases} XP is triggered by a higher phase\is{phases} head. (Note also
    that, as argued in \citealt{Boskovic2015},
    D-incorporation\is{incorporation} is driven by an
    uninterpretable feature of D, which means that D anyway could not trigger
    spell-out before it moves.) It should, however, be noted that under the
    approach to \isi{phases} in \textcite{Boskovic2015},
    D-incorporation\is{incorporation} voids the
phasehood of the DP from which it takes place, so that the issue of DP-phase
spell-out would not even arise in this case.}

There is another possibility here. Assume a framework like \isi{Distributed
Morphology}, where phonological features are inserted in \glsentrytext{PF} to
essentially lexicalize appropriate feature matrices. As argued in
\textcite{Progovac1998,Progovac1999} and discussed briefly in \Cref{sec:35.6}
(see~\cref{fn:27}), the reason why Conj\textsuperscript{0} is typically not
lexicalized with traditional \isi{adjunction} is the \emph{avoid overt conjunction
principle}, which works in a similar way as \citegen{Chomsky:81} \isi{avoid pronoun
principle}. We can then assume that in the relevant situations (see
\Cref{sec:35.6} for why this happens with traditional \isi{adjunction}), the
feature matrix of the conjunction head (or the pronoun in the cases where the
\isi{avoid pronoun principle} is relevant, see \citealt{Holmberg2005}) is
deleted, as a result of which phonological features cannot be inserted. This is
not the case with the null D in \eqref{ex-26:40}. The feature matrix of this
null D simply does not correspond to any phonological features (in contrast to
the conjunction head, where, unless the relevant feature matrix is deleted,
phonological features would be inserted): there is no deletion of the feature
matrix here that would prevent phonological feature insertion.  Under this
analysis, the difference between the null Conj head in examples like
\eqref{ex-26:27} and the null D in examples like \eqref{ex-26:40} with respect
to *-marking is treated in the same way as the difference between the article
and its trace in \ili{Galician} examples like (\ref{ex-26:29}--\ref{ex-26:30}):
In all these cases the relevant head is *-marked due to extraction out of a
conjunct, conjuncts being \isi{islands}.  The *-marked head is then deleted in
\eqref{ex-26:30} (due to copy deletion) and \eqref{ex-26:27} (due to the avoid
overt conjunction principle, which works on a par with the \isi{avoid pronoun
principle}).  On the other hand, the *-marked head is not deleted in examples
like \eqref{ex-26:29} and \eqref{ex-26:40}.  Notice that under this analysis,
*-marking on elements which are not realized (i.e.\ pronounced) in
\glsentrytext{PF} would not actually be ignored.\footnote{For an argument that
it should not be, see \textcite{Boskovic2011}.}

At any rate, I leave teasing apart the analyses of \eqref{ex-26:40} suggested above for
future research and continue to assume below that a * induces a violation in
\glsentrytext{PF} only if it is present on a \glsentrytext{PF} realized
element.\footnote{The discussion below can be easily adjusted to the last
account of \eqref{ex-26:40} suggested above, if it turns out to be the most appropriate
one.}

\section{On extraction of conjuncts/adjuncts}\label{sec:35.5}

\glsunset{PF}
As noted at the outset, the discussion in this paper is limited to islandhood
of conjuncts and adjuncts\is{adjunction}, i.e.\ extraction out of
conjuncts/adjuncts; it does not deal with extraction of conjuncts/adjuncts. As
discussed in~\Cref{sec:35.1}, while the \gls{CSC} was traditionally assumed to
hold both for extraction out of conjuncts and for extraction of conjuncts, this
view is quite clearly wrong, since there are languages that productively allow
extraction of conjuncts but still disallow extraction out of conjuncts. This is
the reason why I have put the discussion of extraction of conjuncts, i.e.\
\eqref{ex-26:2}, aside above. In this section, I will, however, make some brief
remarks on extraction of conjuncts, i.e.\ the status of \eqref{ex-26:2}, the
reason being that the rescue-by-\gls{PF} deletion mechanism, which I have
appealed to above, turns out to be relevant to \eqref{ex-26:2}, as was in fact
explicitly argued in \textcite{Stjepanovic2014} and \textcite{Oda:2017}.

Notice first that the \gls{CSC}\is{coordinate structure constraint} is not
completely divorced from the \gls{AC} even when it comes to \eqref{ex-26:2}, i.e.\
extraction of the conjunct/adjunct. Both are in principle possible, but there
is a productivity difference here in that extraction of adjuncts\is{adjunction}
is more readily available crosslinguistically than extraction of conjuncts. In
this respect, we have the following situation: there are languages like
\ili{Japanese} and \gls{SC} that in principle allow both extraction of
conjuncts and extraction of adjuncts; there are languages like \ili{English}
that allow extraction of adjuncts\is{adjunction} but not extraction of
conjuncts.  I am, however, not aware of any languages that would allow
extraction of conjuncts but not extraction of adjuncts\is{adjunction}. In other
words, we have a small implicational hierarchy\is{implicational relations} here, where the possibility of
extraction of adjuncts entails the possibility of extraction of conjuncts. It
turns out that there is a way of making sense of this state of affairs under
the rescue-by-\gls{PF} deletion approach discussed above.

Recall that \textcite{Oda:2017} argues that both individual conjuncts and ConjP
are \isi{islands}. When it comes to extraction of conjuncts themselves, i.e.\
\eqref{ex-26:2}, what is relevant is the islandhood\is{islands} of ConjP: the
island that is crossed when a conjunct is extracted is ConjP. This means that
what is *-marked when a conjunct is extracted is the head of ConjP (given that
what is *-marked is the head of an island).

Importantly, in languages where extraction of a conjunct is allowed, it has
been shown that the ConjP head is a clitic\is{clitics} that undergoes \isi{movement}. In other
words, the head of ConjP is a trace. This immediately makes \eqref{ex-26:28} relevant here:
the cliticization\is{clitics} voids the islandhood\is{islands} of ConjP, making extraction of a
conjunct possible. In fact, \textcite{Oda:2017} and
\textcite{Stjepanovic2014} argue for exactly this account of the exceptional
possibility of extraction of conjuncts in \ili{Japanese} and \gls{SC}. In both languages
the conjunction head is a clitic,\is{clitics} which \citeauthor{Oda:2017} and
\citeauthor{Stjepanovic2014} argue undergoes \isi{movement}. In \ili{Japanese}, the
conjunction is an enclitic\is{clitics} and in \gls{SC} it is a proclitic.\is{clitics} In \ili{Japanese} \eqref{ex-26:41},
the conjunction cliticizes to the first conjunct and is in fact carried along
under the \isi{movement} of the first conjunct, which quite conclusively shows that
the conjunction head does not remain in its in situ position.

\ea\label{ex-26:41} \ili{Japanese} \parencite{Oda:2017}
    \judgewidth{(?)}
    \ea[?]{
    \gll Kyoodai\tss{i}-to kanojo-wa [ t\tss{i} Toodai~]-ni akogareteiru.\\
            Kyoto.University-and she-\Topic{} {} {} Tokyo.University-\Dat{} admire\\
    \glt    \enquote*{She admires Kyoto University and Tokyo University.}}
    \ex[(?)]{
    \gll Nani\tss{i}-to Taro-ga [ t\tss{i} mizu~]-o katta no?\\
            what-and Taro-\Nom{} {} {}  water-\Acc{} bought \glossQ{}?\\
    \glt    literally \enquote*{What did Taro buy and water?}}
    \z
\z

\largerpage
In fact, as discussed in \textcite{Oda:2017}, in all languages where extraction
of a conjunct is possible the conjunction head is a clitic\is{clitics} that
undergoes \isi{movement}.\footnote{As discussed in \textcite{Stjepanovic2014},
    in \gls{SC} the conjunction procliticizes\is{clitics} to the second
    conjunct, which makes \isi{movement} of the first conjunct, as in (i-a),
    possible. (See \citealt{Stjepanovic2014} for details of the derivation,
    which also involves ConjP-internal \isi{movement} of the second conjunct
    prior to the procliticization of the conjunction to it.
    \citeauthor{Stjepanovic2014} shows that the process in question quite
    generally applies to \gls{SC} proclitics;\is{clitics} thus, she shows,
    following \citealt{Boskovic2013b} and \citealt{Talic2014}, that the
    proclitic\is{clitics} preposition in (i-b) procliticizes to the AP (and is
    carried along under further \isi{movement} of the AP, as in (i-c)), with
    \citegen{Talic2014} prosodic arguments for procliticization\is{clitics} in
    terms of syntactic \isi{movement} of the preposition in (i-b) extending to
    the conjunction in (i-a).)

\begin{exe}
\exi{(i)} \glsdesc{SC}\il{Serbo-Croatian}
    \begin{xlist}
	\ex\gll \llap{?}Knjige\tss{i} je Marko [ t\tss{i} i filmove ] kupio.\\
            books is Marko {} {} and movies {} bought\\
	\glt    \enquote*{Marko bought books and movies.}
	\ex\gll On  je  ušao  u  veliku  sobu.\\
			he is entered in big room\\
	\glt \enquote*{He entered a big room.}
	\ex U veliku je ušao sobu.
    \end{xlist}
\end{exe}

It may also be worth noting here that the clitichood\is{clitics} of the conjunction may not
be the only requirement for the possibility of a CSC-2\is{coordinate structure constraint} violation.
\citeauthor{Oda:2017} notes that all the languages that he observes can
violate CSC-2\is{coordinate structure constraint} lack articles, which may suggest that such violations may be
possible only in NP languages under \citeauthor{Boskovic2008}’s
(\citeyear{Boskovic2008,Boskovic2012}) analysis, where languages without
articles lack DP (for an account along these lines, see
\citealt{Boskovic2017}).\label{fn:24}} The possibility of conjunct extraction
can then be rather straightforwardly accounted for under \eqref{ex-26:28}, i.e.\ in terms
of a rescue-by-\gls{PF} deletion analysis
\parencite[see][]{Oda:2017,Stjepanovic2014}.

As discussed above, with extraction of conjuncts, ConjP functions as an island.
This means that what is *-marked when such extraction takes place is the head
of ConjP. In \ili{Japanese}, where the conjunction head undergoes \isi{movement}, the
islandhood effect is voided since the *-marked element is deleted in \gls{PF} (under
\isi{copy deletion}). The analysis thus unifies acceptable CSC-2\is{coordinate structure constraint} violations like \eqref{ex-26:41}
with other acceptable island violations in \eqref{ex-26:30} and \eqref{ex-26:32}, all of which are
instances of the generalization in \eqref{ex-26:28}, which is, as discussed above, unified
with the rescuing effect of \isi{ellipsis} on locality violations, i.e.\ cases like
\eqref{ex-26:33}, in terms of the rescue-by-\gls{PF} deletion mechanism.

Recall now the observation made above regarding the availability of extraction
of traditional conjuncts and traditional adjuncts\is{adjunction}, both of which involve
extraction of conjuncts under the current analysis: extraction of traditional
adjuncts is much more generally available than extraction of traditional
conjuncts. The mechanism of rescue-by-\gls{PF} deletion provides a straightforward
account of why this is the case. The above discussion has indicated that
extraction of a traditional conjunct is possible only if the head of ConjP is
phonologically null, which we have seen can be captured by the mechanism of
rescue-by-PF deletion. Turning to adjunct extraction, under the current
analysis adjuncts\is{adjunction} are conjuncts, with ConjP headed by a null head present in
the structure. But this is exactly when extraction of a conjunct is possible
even with traditional \isi{coordination}: when the head of ConjP is phonologically
null. True, the reason for this is different (in one case the head is
phonologically null as a result of \gls{PF} \isi{copy deletion} and in the other case it is
null to start with), but that does not matter under the approach to rescue by
\gls{PF} deletion discussed above. The reason why the conjunct (a traditional
adjunct) in \eqref{ex-26:42} is then able to undergo \isi{movement} is the same as the reason why
the conjunct in \eqref{ex-26:41} (a traditional conjunct) is able to undergo
movement.\footnote{As discussed in \textcite{Oda:2017}, extraction of the
    second conjunct in traditional coordinations is not possible in \ili{Japanese}
for an independent \gls{PF} reason that does not arise in \eqref{ex-26:42} (the reason also does
not arise with \emph{wh}-in-situ in \ili{Japanese}, which Oda notes is possible as
both the first and the second conjunct).} What we see here is that
a ConjP that is headed by a trace behaves like traditional \isi{adjunction}
modification, which under the current analysis involves a ConjP with a null
head, in that both cases void islandhood\is{islands}, a state of affairs that can be
captured by the rescue-by-PF-deletion mechanism.

\ea\label{ex-26:42} How did John walk?
\z

The analysis thus unifies the possibility of extraction out of the VP conjunct
in \eqref{ex-26:27} and the improvement with extraction out of a traditional conjunct in
(\ref{ex-26:34}--\ref{ex-26:35}) with the possibility of extraction of a traditional conjunct in \eqref{ex-26:41}
and the traditional adjunct in \eqref{ex-26:42}; what matters in all these cases is that
the head of the island, the conjunct and ConjP in the former case and ConjP in
the latter case, is phonologically null, which is captured under the
rescue-by-\gls{PF} deletion analysis.

There is an interesting prediction made by the current analysis that is worth
noting at this point. Recall that, as argued in \textcite{Oda:2017}, both conjuncts
and ConjP are \isi{islands}. In cases like \ili{Galician} \eqref{ex-26:34}, both of these \isi{islands} are
“violated”. In \eqref{ex-26:35}, on other hand, the islandhood\is{islands} of the conjunct island is
voided since the head of the conjunct is phonologically null as a result of
article \isi{incorporation}. Recall now that in languages like \ili{Japanese} and \gls{SC}, the
head of ConjP (in traditional coordinations) is actually phonologically null
(due to conjunction \isi{incorporation}). This means that extraction out of a
conjunct in \ili{Japanese} and \gls{SC} involves extraction out of only one island,
the conjunct. As a result, we would expect it to be better than extraction out
of a conjunct in \ili{English} and \ili{Galician} \eqref{ex-26:34} – it should be more on a par with
Galician \eqref{ex-26:35} than \ili{Galician} \eqref{ex-26:34}. The prediction is in fact more general, it
holds for all languages where extraction of a conjunct is possible; more
precisely, in languages where CSC-2\is{coordinate structure constraint} can be voided by incorporating the
conjunction head CSC-1\is{coordinate structure constraint} violations should be somewhat weaker than in languages
where this is not the case (unless such languages have a way of incorporating
the conjunct head, like Galician). It is obviously difficult to compare the
strength of island violations across different languages, but
impressionistically, CSC-1\is{coordinate structure constraint} violations do seem to be slightly weaker in \ili{Japanese}
and \gls{SC} than in \ili{English} (one bilingual Japanese/English speaker consulted
did find that CSC-1\is{coordinate structure constraint} violations with \ili{Japanese} scrambling are weaker than CSC-1
violations with \ili{English} topicalization). Obviously, a more careful
investigation is needed here, which I leave for future research.\footnote{It
    is worth noting here that \textcite{Oda:2017} observes a construction in
    \gls{SC} where both the conjunct and ConjP are headed by a trace, namely
    (i).

\ea[(?) ]{
    \gll \llap{[}U veliku]\tss{i} je Ivan ušao [[t\tss{i} sobu] i u malu kuhinju].\\
    in big is Ivan entered {} room and in small kitchen\\}
\z

As noted in~\cref{fn:24}, the conjunction undergoes procliticization\is{clitics} in
\gls{SC}, which means ConjP is headed by a trace in (i). Moreover, as also
discussed in~\cref{fn:24}, the head of the first conjunct, which is a PP,
undergoes procliticization\is{clitics} to the AP, and is carried along under \isi{movement} of
the AP. As a result of P-procliticization, the conjunct from which the AP is
extracted is also headed by a trace. Both the islandhood\is{islands} of ConjP and the first
conjunct are then voided in (i) through the rescue-by-\gls{PF} deletion mechanism,
hence the acceptability of (i).}

The proposed analysis makes a similar prediction regarding the strength of
CSC-1 violations and the \isi{adjunct condition} violation. Consider cases
where no islandhood is voided through \isi{movement} of island heads (cf.\
\ref{ex-26:28}). As discussed above, both conjuncts and ConjP are
\isi{islands}. Extraction out of a conjunct then involves two island
violations. Since adjuncts\is{adjunction} are treated as conjuncts, extraction
out of an adjunct also involves extraction out of a conjunct island and a ConjP
island. However, since with adjuncts\is{adjunction} the head of ConjP is
phonologically null, the islandhood\is{islands} effect of ConjP is voided, as
discussed above. Extraction out of an adjunct then involves one island
violation. We may then expect that CSC-1\is{coordinate structure constraint}
violations should be stronger than adjunct condition violations in a language
like \ili{English}. That indeed seems to be the case: CSC-1 violations like
\eqref{ex-26:4} seem to be worse than \isi{adjunct condition} violations like
\eqref{ex-26:5} (as noted above, the prediction is also borne out with
\ili{Galician} \eqref{ex-26:31} and \eqref{ex-26:34}, \eqref{ex-26:34} being
worse than \eqref{ex-26:31}). On the other hand, in a language like \gls{SC}
where the head of ConjP is also phonologically null due to the
cliticization\is{clitics} of the conjunction, extraction out of both conjuncts
and adjuncts\is{adjunction} involves extraction out of a single island.
CSC-1\is{coordinate structure constraint} violations and the adjunct condition
violations indeed seem to have more or less the same status in \gls{SC}. Of
course, all the predictions noted in this passage still need to be confirmed
with more careful data elicitation.\is{CSC|see{coordinate structure
constraint}}

\section{Conclusion}\label{sec:35.6}

This paper has argued for a unified approach to the islandhood\is{islands} of conjuncts and
adjuncts, both of which disallow extraction out of them. The unification was
made possible by adopting \citeauthor{Higginbotham1985}’s semantics of
traditional \isi{adjunction}, on which traditional \isi{adjunction} actually involves
coordination. This paper took this to be reflected in the syntax, with ConjP
present in the syntax of traditional \isi{adjunction} \parencite[see
also][]{Progovac1998,Progovac1999}. Not only did this position achieve
straightforward syntax-semantics mapping in the case at hand, but it also made
possible a unification of the islandhood\is{islands} of conjuncts and traditional adjuncts
since the two then involve the same syntactic configuration.

I have shown that there are a number of similarities in the islandhood\is{islands} of
conjuncts and adjuncts\is{adjunction}, including the general resistance of their islandhood\is{islands} to
crosslinguistic variation (in contrast to other traditional \isi{islands}, which are
subject to crosslinguistic variation). We have also seen that in a number of
environments extraction is exceptionally possible out of conjuncts and
adjuncts. Significantly, the environments where extraction is exceptionally
possible are the same for conjuncts and adjuncts\is{adjunction}, which can be captured if the
two involve the same syntactic configuration. A number of important issues,
however, still remain to be addressed in future research, including the
question why conjunctions are typically null with traditional adjuncts\is{adjunction} and
overt with traditional \isi{coordination}, as well as providing an actual account of
the islandhood\is{islands} of conjuncts/adjuncts.

\largerpage[-2]
The intuition regarding the former issue seems clear: there are choices when it
comes to what heads ConjP in traditional coordinations. Even if we put aside the
obvious major distinction here, conjunction vs disjunction, languages often
have more than one coordinator, which come with different flavors syntactically
and/or semantically (note e.g.\ that the coordinator that hosts article
incorporation in \ili{Galician} is not simple \emph{e} ‘and’ but \emph{e mais}); in
other words, phonological realization of conjunction is a way of making a
choice of which coordinator to use. Traditional \isi{adjunction}, on the other
hand, involves the most neutral, straight \isi{coordination} which does not add
anything else – this is the null Conj\textsuperscript{0}.\footnote{This does
    not mean that null Conj\textsuperscript{0} can never be used with
    traditional \isi{coordination} (see \citealt{Progovac1999} for some such
    cases) or that an overt Conj\textsuperscript{0} cannot be used in
    traditional adjunct modification.  Regarding the latter, as noted
    in~\Cref{sec:35.4}, \textcite{Progovac1998,Progovac1999} discusses examples
    like \emph{I read his paper, and quickly} and \emph{John read the book and
    avidly}. Also relevant in the context of the current discussion is
    \citegen{Progovac1999} economy of pronunciation which works in a similar
way as \citegen{Chomsky:81} \emph{\isi{avoid pronoun principle}}, choosing the
null conjunction head when possible (\citealt{Progovac1998} in fact adopts
\emph{avoid overt conjunction}).\label{fn:27}}

Some preliminary remarks were also made regarding the islandhood\is{islands} of
conjuncts/adjuncts (an issue that is discussed in more detail from the
perspective taken in this paper in \citealt{Oda:2017} and
\citealt{Boskovic2017}; see also \citealt{Boskovicinprep}). Importantly, it was
shown that in several cases where the islandhood\is{islands} of traditional conjunction
configurations is voided (for both individual conjuncts and the conjunction
phrase itself), where traditional \isi{adjunction} configurations also do not show
islandhood (in both respects), the head of the conjunction (and individual
conjuncts) is phonologically null, with the parallel situation holding for the
traditional \isi{adjunction} configuration, a state of affairs which was
captured by appealing to the rescue-by-\gls{PF} deletion mechanism. We have
also seen that the rescue-by-PF deletion analysis can account in a principled
way for a number of differences in the strength of the violation with
extraction out of conjuncts and adjuncts\is{adjunction} in various
languages/contexts.

\printchapterglossary{}

\section*{Acknowledgements}

It is a pleasure and privilege to be able to dedicate this paper to Ian
Roberts, for his invaluable and lasting contributions to the field of
linguistics.

For helpful comments on this work I thank two anonymous reviewers and the
participants of my 2016 seminar at the University of Connecticut.

{\sloppy\printbibliography[heading=subbibliography,notkeyword=this]}

\end{document}
