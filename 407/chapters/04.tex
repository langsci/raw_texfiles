\documentclass[output=paper,colorlinks,citecolor=brown]{langscibook}
\ChapterDOI{10.5281/zenodo.10103080}

\author{R. Sandra Evans \affiliation{The University of the West Indies, St. Augustine}}
\title[Language(-related) rights in the criminal justice system in St. Lucia]{Assessing language(-related) rights in the criminal justice system in St. Lucia}
\abstract{The creole languages of the Commonwealth Caribbean have seldom been the subject of language rights i.e. laws that directly or indirectly confer rights regarding the use of languages (Brown-Blake 2014). A possible reason for this lack of attention is that language-rights discourse has been mostly centered around the rights of persons belonging to distinguishable minority groups such as ethnic minority groups, indigenous minorities, and im/migrant minorities. However, in contrast to these minority groups, creole languages are spoken by the mass of the population in Commonwealth Caribbean territories. Yet, despite their wide currency, none of them has been the explicit beneficiary of language rights legislation (Browne-Blake 2014). 

This chapter is concerned with the question of language rights legislation in criminal proceedings in St. Lucia. More specifically, it examines the language-(related) rights in St. Lucia’s constitution and the ways in which they are implemented in police and court procedures. The findings revealed that the use of unqualified and untrained interpreters in the implementation of these rights to Kwéyòl-dominant speakers often compromises the protection that they afford, particularly in the magistrates’ courts. The chapter ultimately calls for a constitutionally protected language rights regime and its proper implementation, which will enable Kwéyòl-dominant speakers to use their language freely in all criminal proceedings. The data used in this chapter were taken from the data pool of a larger study on the language use patterns and practices of the criminal justice system in St. Lucia.

\keywords{language-related rights, implementation, Kwéyòl speakers, criminal proceedings, St. Lucia}
}

\IfFileExists{../localcommands.tex}{
   \addbibresource{../localbibliography.bib}
   % add all extra packages you need to load to this file

\usepackage{tabularx,multicol}
\usepackage{url}
\urlstyle{same}

\usepackage{listings}
\lstset{basicstyle=\ttfamily,tabsize=2,breaklines=true}

\usepackage{langsci-basic}
\usepackage{langsci-optional}
\usepackage{langsci-lgr}
\usepackage{langsci-osl}
% \usepackage{./langsci/styles/langsci-lgr}
% \usepackage{./langsci/styles/langsci-osl}
% \usepackage{langsci-gb4e}

\usepackage{tikz}
\usetikzlibrary{patterns,calc}
\pgfdeclarepatternformonly{south east lines}{\pgfqpoint{-0pt}{-0pt}}{\pgfqpoint{3pt}{3pt}}{\pgfqpoint{3pt}{3pt}}{
    \pgfsetlinewidth{0.6pt}
    \pgfpathmoveto{\pgfqpoint{0pt}{3pt}}
    \pgfpathlineto{\pgfqpoint{3pt}{0pt}}
    \pgfpathmoveto{\pgfqpoint{.2pt}{-.2pt}}
    \pgfpathlineto{\pgfqpoint{-.2pt}{.2pt}}
    \pgfpathmoveto{\pgfqpoint{3.2pt}{2.8pt}}
    \pgfpathlineto{\pgfqpoint{2.8pt}{3.2pt}}
    \pgfusepath{stroke}}
    
\usepackage{stmaryrd}
\usepackage{wasysym}
\usepackage{multirow}
\usepackage{caption}
\usepackage{subcaption}
\usepackage{mathrsfs}
\usepackage{qtree}

\usepackage{linguex}


   %pminos do not split footnotes
% \interfootnotelinepenalty=10000 %Footnote in Laporte chapters has to be split SN


%\DeclareIndexNameFormat{default}{%
%\nameparts{#1}%
%\usebibmacro{index:name}%
%{\index[names]}%
%{\namepartfamily}%
%{\namepartgiveni}%
% {}% L1
% {}% L2
%{\namepartprefix}% generates spurious space L3
%{\namepartsuffix}% generates spurious space L4
%}

%  {\DeclareIndexNameFormat{default}{%
%     \usebibmacro{index:name}{\index[names]}{#1}{#3}{#5}{#7}}}

%\DeclareIndexNameFormat{default}{%
%  \usebibmacro{index:name}{\sindex[nom]}{#1}{#3}{#5}{#7}}

%\DeclareIndexNameFormat{default}{%
%  \usebibmacro{index:name}{\sindex[person]}{#1}{#3}{#5}{#7}}
%\DeclareIndexNameFormat{default}{%
%\nameparts{#1} \usebibmacro{index:name}{\sindex[person]]}{\namepartfamily}{‌​\namepartgiven}{\nam‌​epartprefix}{\namepa‌​rtsuffix}}

%\newcommand{\smiley}{:)}

%\renewbibmacro*{index:name}[5]{%
%\usebibmacro{index:entry}{#1}%
%{\iffieldundef{usera}{}{\thefield{usera}\actualoperator}\mkbibindexname{#2}{#3}{#4}{#5}}}

% \newcommand{\noop}[1]{}

%remove for final
%\overfullrule=1mm

\newcommand{\tobi}[2]}}
\renewcommand{\S}[1]{\tobi{#1}{\textsc{*}}}

% this volume references
% puts: [this volume]
% already defined: \citetv
%\newcommand{\citepv}[1]{(\citeauthor{#1} \citeyear*{#1} [this volume])}
\newcommand{\citealtv}[1]{\citeauthor{#1} \citeyear*{#1} [this volume]}

%parentheses around example number
\newcommand{\pref}[1]{(\ref{#1})}

% in-text examples

\newcommand{\lnex}[1]{\textit{#1}} %target lang word
\newcommand{\lnlit}[1]{(lit.: `#1')} %literal reading
\newcommand{\lnlat}[1]{(#1)} % latinization
\newcommand{\lntrans}[1]{`#1'} %translation
\newcommand{\lnexl}[2]%
{\lnex{#1}{} \lnlat{#2}} % ex with latinization
\newcommand{\lnexlat}[3]{\lnex{#1}{} \lnlat{#2}{} \lntrans{#3}} % ex with latinization and tranl.

%ch01
\newcommand{\co}[1]{\mbox{\textbf{#1}}}

%ch09

\newcommand{\cyrbulg}[1]{\begin{otherlanguage*}{bulgarian}#1\end{otherlanguage*}}


%ch10
\newcommand{\nlp}{{\small NLP}}
\newcommand{\mwe}{{\small MWE}}
\newcommand{\rae}{{\small RAE}}
\newcommand{\lvc}{{\small LVC}}
\newcommand{\pos}{{\small P}o{\small S}}
%\newcommand{\todo}[1]{ \textcolor{red}{#1} }

%\renewcommand{\labelenumi}{\theenumi}
%\ainamefmt{{vv}{ll}{, ff}{, jj}} % fullname

\newcommand{\biberror}[1]{{\color{red}#1}}

\newcommand{\osenovaitem}{--~}
   %% hyphenation points for line breaks
%% Normally, automatic hyphenation in LaTeX is very good
%% If a word is mis-hyphenated, add it to this file
%%
%% add information to TeX file before \begin{document} with:
%% %% hyphenation points for line breaks
%% Normally, automatic hyphenation in LaTeX is very good
%% If a word is mis-hyphenated, add it to this file
%%
%% add information to TeX file before \begin{document} with:
%% %% hyphenation points for line breaks
%% Normally, automatic hyphenation in LaTeX is very good
%% If a word is mis-hyphenated, add it to this file
%%
%% add information to TeX file before \begin{document} with:
%% \include{localhyphenation}
\hyphenation{
    Beck-man
    Ngu-yen
    back-chan-nel
    back-chan-nels
    mo-not-o-nous
    ste-reo-typ-i-cal
}

\hyphenation{
    Beck-man
    Ngu-yen
    back-chan-nel
    back-chan-nels
    mo-not-o-nous
    ste-reo-typ-i-cal
}

\hyphenation{
    Beck-man
    Ngu-yen
    back-chan-nel
    back-chan-nels
    mo-not-o-nous
    ste-reo-typ-i-cal
}

   \boolfalse{bookcompile}
   \togglepaper[23]%%chapternumber
}{}

\begin{document}
\maketitle

\section{Introduction}

The literature on language rights i.e. laws that directly or indirectly confer rights regarding the use of languages \citep[52]{BrownBlake2014}, has been, for the most part, largely centred around the language rights of persons belonging to distinguishable minority groups such as ethnic minority groups, religious minority groups, linguistic minorities, indigenous minorities, and immigrant minorities \citep{Capotorti1991,Thornberry1991,TovePhillipson1995,Edwards2003,Arzoz2007,May2014,May2012,May2011}. \citet[552]{Edwards2003} contends that it is hardly surprising that discussions of language rights often focus on minority groups since examining risks and rights naturally has more poignancy when the former are immediate, and the latter are in question. However, he further states that while issues of culture and language are of obvious or heightened salience in minority settings, they are important in \emph{all} groups \citep[552]{Edwards2003}. 

In other words, the question of language rights is also applicable to other language groups that are situated outside of this concept of a minority. \citet{Aguilar-Amat_Santamaria2000} point out that since the word “minority” implies a small number, minority languages are languages spoken by a relatively small number of speakers. Therefore, the category is based on a statistical notion of the number of speakers \citep{Aguilar-Amat_Santamaria2000} or refers only to the demographic weight of the relevant speech communities \citep{MoeketsiWallmach2005}. However, \citet{MoeketsiWallmach2005} contend that this interpretation of the concept is also inadequate to describe or label other types of linguistic situations e.g. in South Africa, where languages, which are the languages of the majority in their region, become minoritized and suffer from functional difficulties, not because of a lack of numbers or demographic weight, but rather as the result of historical events and socioeconomic conditions such as colonialism or a reorganization of territorial borders. \citet{Aguilar-Amat_Santamaria2000} refer to such languages as “minoritized” languages. \citet[78]{MoeketsiWallmach2005} note that this concept is more useful to describe and define these linguistic situations, for example in Africa, where African languages became minoritized through apartheid.

Another notable group of languages, that falls squarely within this bracket, is the Creole languages of the Commonwealth Caribbean, which have the widest local currency in their territories but continue to suffer from functional difficulties as a result of colonialism. In fact, an examination of the linguistic situation in Commonwealth Caribbean Creole-speaking territories would reveal that Creole-dominant and monolingual Creole speakers together comprise a majority of the respective populations in contrast to English speakers and/or functional English/Creole bilinguals \citep{BrownBlake2014}. However, despite the wide currency of these languages, and the fact that their speakers are the demographic majority in their territories, they are marginalized from effective participation and involvement in the domains in which English is the principal language of communication \citep{BrownBlake2014}. Furthermore, there is no official protection for these speakers against linguistic discrimination given that no Bill of Rights in the Commonwealth Caribbean includes language as a basis upon which discriminatory treatment is proscribed \citep{BrownBlake2008}.

It is also noteworthy that language rights have rarely been applied to Creole languages, and they have only received scant coverage in language rights discourse to date. This is strongly supported by \citet{Eades2010} who points out that there is little coverage in the literature on the language rights question in relation to Creole language speakers. The first and only notable attempt to address the issue of language rights for Creole speakers in the Commonwealth Caribbean was made almost a decade ago. In 2011, linguists, language planners, language rights advocates, and activists, put forward and agreed on the terms of a Charter of Language Policy and Language Rights in the Creole-speaking Caribbean at a conference held in Jamaica. This Charter sets out general principles and enumerates specific rights relating to language and language use in public formal domains such as education, public administration, and the courts \citep{BrownBlake2014}. It is a first attempt to confront the language rights issue at the regional level as well as a first step towards securing a raft of language rights across the region \citep{Brown-Blake2011}. It was meant to be used to lobby Caribbean governments to endorse language laws in accordance with its concepts and general principles. However, as it stands, this plan has not materialized, and none of the Creole languages of the Commonwealth Caribbean have been accorded language rights through the law.  

This chapter is concerned with the question of language rights for Creole (Kwéyòl) speakers in public domains in St. Lucia. More specifically, it examines the language-related laws in St. Lucia’s constitution and their implementation in the criminal justice process. It also assesses the ways in which these laws are implemented to determine whether they provide Kwéyòl speakers adequate access to the rights to which they are entitled.


\section{Language use in St. Lucia}

Kwéyòl is one of three languages spoken in St. Lucia. The other two are English, the island’s legacy from the British colonial period \citep{StHilaire_Aonghas2011}, and an English-lexicon vernacular that embodies features of English and Kwéyòl from which it emerged. It has marked structural differences from English \citep{Simmons-McDonald2014}. Regarding the social status, functions, and daily use of these languages in St. Lucia, English has always been the only official language and the language that is generally required for use within institutions of the state and in formal situations. It is largely acquired by St. Lucians through formal education and very few St. Lucians speak it in daily life. The other two languages, which have no official status and are widely used in private and informal domains, both have wider currency than English in daily communication among St. Lucians. Many St. Lucians are not proficient speakers of English although many who speak the English-lexicon vernacular, which is lexically similar to English, do not regard it as a distinct language from English and often believe that when they speak it, they are speaking English. However, there is no such confusion about Kwéyòl, which derives the bulk of its vocabulary from French and is not mutually intelligible with English. 


\section{Language(-related) rights}

The concept of language rights is defined in various ways in the literature. In legal and philosophical literature, they are defined as rights that protect the use of particular languages, namely one’s mother tongue or native language \citep[233]{Pinto2014}. For the purposes of this chapter, language rights are also regarded as being concerned with the rules that public institutions adopt with respect to language use in a variety of different domains including public services, courts and legislatures, and education \citep{Arzoz2007}. These rules serve to regulate language conduct and procedure in these domains, particularly in bilingual or multilingual language situations where there is a single, dominant official language. Although in some states, these rules are explicitly stated in the constitution, in many cases, there is no constitutional formal recognition of an official language. However, \citet[18]{Arzoz2007} contends that in these cases, there is no doubt about the existence of a \emph{de facto} official language and public monolingualism in this language is simply taken for granted by citizens \citep{May2014}. Therefore, whether the use of one official language is constitutionally guaranteed or not, a problem arises for speakers of other languages who lack proficiency in this language and do not have the right to access public services or to receive all or some of these services in their language. 

\hspace*{-4.3pt}Although language rights appear in constitutional documents around the world, they are commonly perceived as special rights that are distinctly different from fundamental human rights \citep[231]{Pinto2014}. However, in cases where there are no constitutional language rights, the common practice in many states is for special accommodations to be made for persons who do not speak the official language of the state such as the provision of interpreters. \citet[8-9]{KymlickaPatten2003} make a distinction between two different ways in which speakers of non-official languages can be accommodated in public institutions. The first one is referred to as the “norm-and-accommodation” approach, which involves the predominance of some normal language in public communication. Special accommodations are then made for people who lack proficiency in this normal language and can take a variety of forms including the provision of interpreters, the hiring of bilingual staff, and the use of transitional bilingual and/or intensive immersion education programmes to encourage the rapid acquisition of the state language. The key priority is to establish communication between the public and individuals with limited proficiency in the state language so that the latter can access the rights and benefits to which they are entitled \citep{KymlickaPatten2003}. The second approach is to designate certain languages as “official” and then to accord a series of rights to speakers of those languages \citep[9]{KymlickaPatten2003}. \citet{KymlickaPatten2003} further point out that in contrast to the first approach, this approach typically involves a degree of equality between the different languages that are selected for equal status. As a result, public services are received in both official languages, and the use of the second official language is not dependent upon or determined by a lack of proficiency in the first official language.

In addition to these two approaches, there is a third approach, which may be referred to as a “laissez-faire” approach that was not considered by citet{KymlickaPatten2003}. This approach is typically found in situations where English predominates in public institutions and there are speakers of nonstandard or non-mainstream English vernaculars or dialects (e.g. African American Vernacular English, Australian English) and speakers of English-lexicon Creoles (e.g. the English-lexicon Creoles of the Commonwealth Caribbean). These languages, which exist alongside English, are local or native languages of common communication which include “structures that are not mainstream or standard” \citep[16]{WolframSchilling2015}. They are spoken most frequently and fluently by ethnic minorities and/or by less educated working-class people, or poor people worldwide \citep{RickfordKing2016}. However, since their vocabulary is largely derived from English, they are often dismissed as nothing more than “bad English” or “badly pronounced English” and are not generally perceived as being distinct languages from English. As a result of this misguided perception of linguistic homogeneity, in contrast to the norm-and-accommodation approach, there are hardly any measures in place to accommodate speakers of these languages. As such, their participation in public formal domains is left to take its own course. For instance, \citet[951]{RickfordKing2016} note that interpreters are not generally provided for “dialects” of a language, only for foreign “languages.” In the case of English-lexicon Creoles, their speakers are normally presumed to be speakers of English. For example, \citet{BrownBlake2008} notes that the view that Jamaican English-lexicon Creole is not a distinct language from English persists to the current day, and relatively few Jamaicans believe they do not speak English \citep{Brown-BlakeChambers2007}. As a result, even though the Jamaican constitution (Chapter III Section 20 (6a)) guarantees individuals with little or no comprehension of English to solicit the services of an interpreter during the court trial, it does not appear to be enforced by the officers of the court nor requested by monolingual Jamaican Creole speakers \citep{Forrester2014}. \citet[228]{Forrester2014} further states that this constitutional provision gets clouded in a situation where either the court or the defendant does not see Jamaican Creole as different from Standard Jamaican English, and as such there is no need for an interpreter. This is in keeping with the situation in other Commonwealth Caribbean jurisdictions, where the language-related fair trial or due process rights are seldom if at all, invoked in relation to Creole-speaking nationals of a given territory \citep[4]{Brown-Blake_Devonish2012}. They continue to opt for public monolingualism in English, requiring its use in all public/civic communication. 


\section{Data and methods}

This chapter draws on data from a larger study, which explored the language use patterns and protocols in the criminal justice system in St. Lucia. These data comprise semi-structured interviews with major stakeholders at both levels of the criminal justice process including police officers, lawyers, magistrates, and clerks of the court. Direct systematic observations in the magistrates’ courts and field notes also contributed to the data pool. Since this paper focuses on the implementation of constitutional language-related rights in police procedures and in the courts, the analysis is centred around interviews with the police and the clerk of the courts, who are ultimately their “implementers”. 

A total of ($n=20$) face-to-face interviews were conducted with police officers who work at police stations and ($n=7$) of 8 clerks of the court who were working in courts in the different magisterial districts in St. Lucia. Prior to conducting the interviews, I met with the inspector at each police station. I explained the purpose of my research and sought permission to conduct interviews with police officers. Once permission was granted, I proceeded to interview the police officers who were willing to participate and to be recorded. In the case of the clerks of the court, I contacted them directly after court sessions and explained the purpose of my research to them. Once they agreed to participate in the research, I conducted face-to-face interviews with each of them about their experiences as court interpreters for Kwéyòl speakers in the magistrates’ court. All the interviews were conducted on a one-to-one basis and each one ranged from 10-15 minutes with each participant. The interviews were recorded with the permission of the participants and they were later transcribed verbatim. Excerpts from these transcripts are used in this chapter. The analysis of this kind of qualitative data was based on the general procedures commonly used in qualitative research. To anonymise the participants, they were assigned a random identification number, for example, PO-1 (police officer 1, etc.), CC-1 (court clerk 1, etc.), and L-1 (lawyer 1, etc.).


\section{Language-related rights in St. Lucia’s constitution}

Similar to the other constitutions in the Commonwealth Caribbean, language rights or any subjective rights regarding the use of languages are not expressly recognized in the constitution of St. Lucia. According to \citet{Arzoz2007} there can be two constitutional sources of language rights. The first one is a provision proclaiming the official status of some languages and the second one is a provision awarding protection to other languages. In the case of St. Lucia, there is neither a provision that proclaims the official status of any language nor a provision that protects any language. In fact, no creole language in any Commonwealth Caribbean territory has been the explicit beneficiary of language rights legislation \citep{BrownBlake2014}. In a similar vein,  \citet[4]{Brown-Blake_Devonish2012} posit that if language rights are regarded as legal entitlements relating to language use, then very few Caribbean Creole-speaking territories can boast a language rights regime outside of the universal language-related fair trial or due process right.

However, although there are no constitutional sources of language rights in St. Lucia’s constitution,  like the constitutions of many states, it includes a number of constitutional rights with an explicit linguistic dimension, which are largely confined to fair trial or due process rights in the criminal justice system. They are found in three statutes: the first two pertain directly to law enforcement procedures and the third one concerns court procedures. The first one, found in  Chapter 1 (3) (2), states,  

\begin{quote}
    (2) Any person who is arrested or detained shall with reasonable promptitude and in any case no later than 24 hours after such arrest or detention be informed in a language that he or she understands of the reasons for his or her arrest or detention...
\end{quote}

The second one, found in Chapter 1 (8) (2) (b), stipulates that,

\begin{quote}
    (2) Every person who is charged with a criminal offence 
    
    (b) shall be informed as soon as reasonably practicable, in a language that he or she understands and in detail, the nature of the offence charged; 
\end{quote}

However, since the act of arresting and charging persons falls within the remit of police officers, the enforcement of these statutes is automatically their responsibility. In other words, they are the ones who must inform persons of the reasons for their arrest and the nature of the offence charged “in a language that they understand”. This provision gives primacy to the language of the arrestee or the accused. 

The third language-related statute, which is found in Chapter 1 (8) (2) (f) in St. Lucia’s constitution, pertains to court procedures. It states,

\begin{quote}
    (2) Every person who is charged with a criminal offence
    
    (f) shall be permitted to have without payment the assistance of an interpreter if he or she cannot understand the language used at the trial. 
\end{quote}

The right to a fair trial requires that the accused persons understand the accusation. If they do not understand the language of the court, they must be accommodated through the assistance of an interpreter. Yet, researchers continue to warn that the right to an interpreter is not a language right, but a well-established human right, to which anyone facing a criminal charge is entitled. For instance, \citet[385]{Lubbe2009} warns that language rights should not be confused with the right to a fair trial, a universal right. Similarly, \citet{GonzalesGonzalesMikkelson1991} assert that the right to an interpreter is not a language right but simply guarantees the right to equal access the legal system. In addition, \citet[5]{Arzoz2007} contends that this right does not aim to afford tolerance, protection, or promotion for any language or any linguistic identity. Its rationale lies somewhere else: in securing trial fairness. The sole objective of the right is effective communication; it does not independently value \citep[5]{Arzoz2007} or give primacy to the language of the accused. 


\section{The implementation of language-related laws in the criminal justice system in St. Lucia}

As mentioned previously, police officers in St. Lucia are responsible for the implementation of language-related laws that pertain to law enforcement procedures. The law requires them to address language barriers by informing arrestees and suspects with limited English proficiency of the reasons for the arrest and the nature of the offence charged, in a language that they understand. However, although the provisions stipulate what must be done, they provide no information on the required method of delivery (whether orally or in writing), by whom (the police or an interpreter), or the consequences of failure to uphold these rights (protection of the rights). In other words, as it stands, the law allows police officers to exercise rational choice in its enactment \citep{Evans2019}. Therefore, not only are they at liberty to take on the responsibility themselves, but they can also seek the assistance of an independent party of their choice, to provide interpreting assistance. 

In St. Lucia, there are two possible groups of non/limited-English speakers, namely Kwéyòl-dominant speakers, and speakers of foreign languages such as French and Spanish, who would have to be provided with the requisite information in a language that they understand. In the case of Kwéyòl speakers, the data revealed that the general practice is for them to be informed of the reasons for their arrest, in Kwéyòl, by a Kwéyòl-speaking police officer, in the same way that an English-speaking person would receive the relevant information in English, by an English-speaking police officer. This is explained by police officers in the following excerpts:

\begin{quote}
  % 
  \begin{enumerate}
    \item PO-18: We handle them [Kwéyòl speakers] the same way we would handle English-speaking persons. The same way they speak Patwa you have to speak Patwa to them.

    \item PO-1: \\
    INT: Have you ever arrested a person who only spoke Kwéyòl \\
    PO: Yeah \\
    INT: So, pretend that you are arresting me and I can only speak Kwéyòl \\
    PO: Ok first I will identify myself to you and I would tell you “so and so moun sa la fè an wapò kont-ou èk ou pa oblijé di ayen si ou pa vlé mé ayen ou di n’y matjé’y" (that person made a report and you do not have to say anything if you do not want to but anything you say I write it).
  \end{enumerate}
  % 
\end{quote}

Therefore, when police officers speak to Kwéyòl speakers in Kwéyòl, it is not regarded as a special accommodation; it is regarded as the normal or natural thing to do. This practice is rooted in the commonly held ideology that since Kwéyòl is the local language of St. Lucia, police officers, who are St. Lucian, are or should be, bilingual in English and Kwéyòl. This was expressed by the following police officers:

\begin{quote}
    %
    \begin{enumerate}
    \setcounter{enumi}{2}
        \item PO-6: Well, you see most police officers speak both languages…where Patwa is  concerned we can all speak.
        
        \item PO-7: \\
        INT: So what happens when you have to deal with a Kwéyòl speaker? \\
        PO: It's very simple you see all our police officers are St. Lucian. \\
        INT: OK all? \\
        PO: Yes and they all speak Creole they were all born and raised in St. Lucia so  we speak the Creole language very fluently.

        \item PO-11: \\
        INT: So if you go to arrest somebody who is a Creole speaker what would you do?\\
        PO: OK so if this person is a Creole speaker I would speak to them in Creole. I would tell them x, y, and z made a report. You say for example “Misyé Harrow mwen vini oti’w am Mafa fè an wapò an station-an a koté ou kwashé an fidjay-li èk pou sa mwen vini mwen kay awété’w pou lapéti sa (Mr. Harrow I came to see you and Martha made a report in the station that you spat in her face and for that I came and I will arrest you because of that).
    \end{enumerate}
    %
\end{quote}

These comments confirm the general practice for dealing with Kwéyòl speakers. The main issues with this practice, which may have adverse implications for Kwéyòl speakers, is that not all police officers, some by their own admission, are competent in Kwéyòl, and for the ones who may be competent, their actual proficiency in spoken Kwéyòl is not ascertained. In fact, there is no basis for determining levels of proficiency for speech in Kwéyòl or English in St. Lucia. Therefore, in the absence of testing or proof of competence, one cannot say with any certainty that they are doing a good job of providing Kwéyòl speakers with the rights to which they are entitled. In addition, although police officers may be fully bilingual and may have a good command of English and Kwéyòl, researchers argue that despite the pervasive myth that if a person is bilingual they can interpret, bilingualism or the ability to speak two languages is not synonymous with interpreting ability \citep{GonzalesGonzalesMikkelson1991}. In fact, De Jongh posits that bilingualism, or fluency in two languages, is only the starting point in interpreter training (285). 

The data revealed that police officers use a different approach to dealing with foreign language speakers. Unlike in the case of Kwéyòl speakers, they employ a norm-and-accommodation approach and outsource persons to interpret and provide them with the requisite information in a language that they understand. This is explained by police officers (2) and (4) below: 

\begin{quote}
    %
    \begin{enumerate}
    \setcounter{enumi}{5}    
        \item PO-2: ...if you arrest a foreigner and you cannot speak the language you can get somebody from an embassy to translate in cases like that.

        \item PO-4: If the person does not speak English, let’s say French you cannot speak to the person if there is not a translator a French translator.
    \end{enumerate}
    %
\end{quote}

Note that the translator to whom PO-4 is referring is not a trained translator/interpreter, but someone who speaks French such as a teacher \citep{Evans2019} or as was mentioned by PO-2, an embassy employee.

In contrast to the first two provisions, the one which pertains to the court does not mandate that a person be tried “in a language that he or she understands.” Instead, it gives primacy to the language of the court and requires persons who cannot understand this language to be provided with an interpreter, free of charge. However, there is no safeguard included in the provision about the kind or quality of interpreter that should be used and as a result, virtually anyone can serve as an interpreter without violating the law \citep{BerkSeligson2000,Berk-Seligson2002,HerraezJuanFoulquie2008}. In the context of St. Lucia, there are two groups of persons, namely Kwéyòl-dominant speakers and foreign language speakers who would need interpreting assistance in court. The data revealed that in the case of Kwéyòl speakers, the clerk of court is the “resident” interpreter \citep{Evans2012} in the magistrates’ courts. They are not trained interpreters, but they are chosen by both local and foreign magistrates to perform the task because according to one local magistrate, “of their knowledge of Kwéyòl.” However, the actual nature of this knowledge and how it qualifies them to interpret remains unclear, especially since they were selected in vastly different ways and specialized training was not a prerequisite for their appointment as is evident in the following excerpts from the data:

\begin{quote}
    %
    \begin{enumerate}
    \setcounter{enumi}{7}      
        \item CC-1: I was called by the senior magistrate to hold on for three months and was encouraged to stay after that.

        \item CC-2: I sent an application and was interviewed by the senior magistrate and had to translate a charge form into creole. 

        \item CC-3: I did computer studies, went on job training and started doing clerk work.

        \item CC-4: I got promoted from office clerk to clerk of the courts.

        \item CC-5: I was an ex-police officer and became a clerk of the courts.

        \item CC-6: I was a bailiff and began to the work of the clerk of the courts.

        \item CC-7: I sent an application through the public service. I was interviewed in English and was asked to speak about myself in Kwéyòl.
    \end{enumerate}
    %
\end{quote}

The data clearly show that there is no systematic, institutionalized procedure for becoming a clerk of the court/interpreter in St. Lucia. Therefore, for the interpreting requirement of their job, the clerks in St. Lucia are very much left to their own devices and initiatives since their role as interpreter is largely undefined. This is due to the fact that they receive no training in interpreting and they must function in the absence of standardized procedures. Nonetheless they must bear the responsibility of interpreting well in the face of numerous difficulties. Some of them expressed some of the difficulties that they experience as court interpreter.

\begin{quote}
    %
    \begin{enumerate}
    \setcounter{enumi}{14}    
        \item CC-1 \\
         I: So in which area would you say you have encountered the most people who needed  you to interpret for them? \\
         CC: More likely in Choiseul \\
         I: Ok \\
         CC: I think that’s where I had the most difficulty interpreting \\
         I: Why would you say it was difficult? \\
         CC: Because some of it you see it does be difficult I mean like when the person speaks to pick it up one time and just interpret it to the magistrate. Sometimes you have to look for the right words in English \\
        CC: In which cases would you say you encounter the most problems? \\
        CC: When you have these like sexual carnal knowledge matters especially if there have a doctor and then the defendant does not understand the English and I have to interpret it from English to Patwa that is very difficult. \\
        I: Ok \\
        CC: There have been times when I just I actually had to tell the magistrate I don’t know. For example “interest” paid by the bank or “interest” paid on hire purchase. I just did not know it.
    \end{enumerate}
    %
\end{quote}

Another clerk expressed difficulties translating technical jargon in the following excerpt:

\begin{quote}
    %
    \begin{enumerate}
    \setcounter{enumi}{15}     
        \item CC-5 \\
         I: What is the most difficult part of interpreting? \\
         CC: It’s a bit difficult to do on the spot translations; sometimes I don’t understand words or phrases in Creole. \\ 
         I: What do you do when that happens? \\
         CC: I look around I seek help from lawyers and police officers. \\
         I: Do you interpret sentence by sentence? \\
         CC: One, two or three sentences hardly ever one sentence. \\
         I: Which cases you find are the most difficult to interpret? \\
         CC: Rape cases it is a little embarrassing and too raw at times especially when doctors are giving evidence. When I do not know the word in Creole I put \textit{la} after the English word for example \textit{swab-la}.
    \end{enumerate}
    %
\end{quote}

Note that \textit{la} is just a form of the definite article ‘the’ and it would not help a Kwéyòl speaker to understand the meaning of \textit{swab} in any way. In both of the excerpts above, the clerks expressed difficulties with medical and technical jargon and any misinterpretations could have implications for Kwéyòl speakers, particularly in cases in which a foreign, non-Kwéyòl speaking magistrate is presiding. This was expressed by one lawyer in the following excerpt:

\begin{quote}
    %
    \begin{enumerate}
\setcounter{enumi}{16}    
        \item L-1: ...and so you have had situations where the court clerk the interpreter did not interpret exactly what the person said and what compounds the problem is if the magistrate is not a Creole speaker and so you have miscarriages of justice which would be fundamental, they could be found guilty of an offense which they really ought not to have been found guilty about.
    \end{enumerate}
    %
\end{quote}

Another area that the clerks generally had difficulty interpreting is swear words. For instance, in a matter involving two women, the complainant said that the defendant called her \textit{manman salop}, which the clerk interpreted as ‘mother scunt’ instead of ‘whore’ or ‘slut, trollop’ \citep{Mondesir1992}. However, the St. Lucian magistrate who was presiding at the time interjected and said \textit{salop is not scunt, salop is nasty}, which would produce ‘mother nasty’. Although \textit{salop} also means ‘nasty’ \citep{Frank2001}, this meaning is not appropriate in the context. Therefore, both are inaccurate interpretations of \textit{manman salop}, which means ‘mother whore’ or ‘original whore or the whore of all whores’. Both misinterpretations are instructive because they underscore the fact that neither the clerk nor the magistrate seemed to understand the illocutionary force of the insult in Kwéyòl. This could have a negative impact on the complainant’s case since neither one captured the intended meaning and force of the insult. It is crucial that the intended force of the insult be maintained in the interpretation since the use of insults is an offence in St. Lucia. This is stated in St. Lucia’s Criminal Code as follows:

Insulting, abusive, or profane language, 

\begin{quote}
    \textbf{534}. Any person who, in any place utters any abusive, insulting, obscene, or profane language, to any other person, is liable on summary conviction to a fine of one thousand dollars.
\end{quote}

Therefore, the misinterpretation of an insult could ultimately result in the dismissal of a legitimate case. This example is consistent with several studies that have found that interpreters sometimes make the mistake of interpreting the semantic meaning only, the “fixed context-free meaning” \citep[29]{Cooke1996}, ignoring, misunderstanding or simply not conveying the pragmatic meaning of utterances \citep{Hale2004}. This is also supported by \citet[4]{Eades1999} who state that more attention is required concerning the transference of pragmatic meaning rather than merely semantic content, in the interpreting process. In other words, an expletive in one language can be interpreted semantically in another, but the intended meaning (that of an insult) and the intended force (the seriousness of the insult) may not be equivalent. According to \citet[76]{HatimMason1990}, equivalence is to be achieved not only of propositional content but also of illocutionary force. Thus, interpreting at the semantic level and not the pragmatic level will inevitably lead to misunderstandings, especially in cases where the magistrate is not competent in one language, as in the case of foreign magistrates. In sum, court interpreting is a complex interpersonal communication activity that entails far more than replacements of words, phrases and sentences in one language with equivalents in another \citep[133]{Moeketsi2001}.

Although the court interpreter is meant to allow the non-English individual to enjoy due process and equal protection under the law \citep{Arzoz2007}, these examples show that an ad hoc or untrained interpreter who does not always provide accurate interpretations could have the opposite effect. \citet{Moeketsi2001} asserts that accuracy in court interpreting is one of the most important requirements for court interpreters and given their pivotal role of facilitating communication in court proceedings, as well as the potentially adverse repercussions on the rights of the accused, they must be thoroughly prepared for their assignment. Yet, formal training is rarely a requirement for employment as a court interpreter \citep{Hale2004}.


\section{Conclusion}
\largerpage
This study sought to examine the language-related provisions in St. Lucia’s constitution and the ways in which they are implemented in practice. The responsibility of implementing these provisions, which are largely confined to fair trial and due process rights, falls on stakeholders in the criminal justice system. In the pre-trial phase, police officers are responsible for providing non-/limited English speakers with their right to information, that is, the right to be informed of the reasons for arrest and the nature and cause of any charge or accusation in a language that they understand. The data showed that in practice, the police officers adhere to the law by speaking to Kwéyòl speakers in Kwéyòl and for foreign language speakers, they seek the assistance of persons who speak their language, such as an embassy employee, to administer their rights in their language. The main issues that arise in the case of police practice in relation to Kwéyòl speakers is one of unknown competence and lack of formal training in interpreting. Although most of them claim to be bilingual or competent in Kwéyòl, this is not ascertained. \citet{Evans2019} found that not all police officers are competent in Kwéyòl and they sometimes have to seek assistance from another police officer. Therefore, in order to guarantee that the rights of Kwéyòl speakers are not jeopardized by a lack of, or inadequate competence, and different degrees of bilingualism, the oral competence of police officers who work at police stations in Kwéyòl should be tested. In addition, researchers continue to raise concerns about the use of police officers as interpreters in police procedures for reasons such as impartiality and conflict of interest \citep{BerkSeligson2000}, so if they must be used, they should at least receive professional training in interpreting in order to provide Kwéyòl speakers with the best chance of accessing justice. Another, perhaps more effective, alternative would be to hire independent qualified interpreters to interpret police procedures. This would help to ensure that Kwéyòl speakers’ access to justice in these procedures does not lie completely in the hands of the police, who are essentially facilitators of the justice process.

%%\todo{Please check these quotation marks}%Reinstert after proofreading!
There is an even bigger issue at the court phase of the system where the clerks of the court are appointed as “resident interpreters for Kwéyòl speakers”. The data showed some of the ways in which misinterpretations could have adverse effects on a Kwéyòl-speaking suspect, particularly in cases where foreign magistrates preside. Since the clerks are not trained or qualified in interpreting, they encounter various challenges on the job. However, all of these issues could be circumvented through a constitutional language rights regime that guarantees Kwéyòl speakers the right to use their language freely in court. If this is not practical, as in cases where foreign magistrates are presiding, then proper interpretation services should be provided, by persons who have the requisite training and qualifications in interpreting. A comprehensive language rights regime, which must be constitutionally secured, would no doubt serve to separate language rights from other rights in which they are merely implicated and provide Kwéyòl speakers with a better chance of enjoying the fundamental rights to which they are entitled.


% -- References:
{\sloppy\printbibliography[heading=subbibliography,notkeyword=this]}


\end{document}
