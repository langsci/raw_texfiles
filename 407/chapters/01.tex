\documentclass[output=paper,colorlinks,citecolor=brown]{langscibook}
\ChapterDOI{10.5281/zenodo.10103074}
\title[\#problematic] {{\#problematic}: Using English for social justice advocacy in Creole-speaking societies}
\author{Clive Forrester\affiliation{University of Waterloo}}

\abstract{Language advocacy in the Caribbean arguably has a fairly extensive history dating back to the colonial era when poets, storytellers, singers, and theatre practitioners started to disrupt the status quo and dared to create art using the local Creole languages in the region. This unwitting act of advocacy was bolstered by the fact that these same creatives managed to gain the approval of their communities in calling for the respect and recognition of Creole languages as “real” languages alongside their European counterparts. Once linguists took up the mantle and started to lobby the government for formal recognition of language rights, the support started to dissipate. Caribbean academics who engaged in language advocacy became seen as “elites”, who were already proficient in a European language and were interested in “imposing” the local Creole languages on marginalized speakers. 

This chapter investigates the dominance of the English language in matters of social justice even among societies where a Creole language is the national language. The data in this study comes from a corpus of reader responses in an online forum to newspaper articles dealing with language rights. Shielded by the veil of anonymity, and bolstered by social media style "up-votes", forum users are emboldened to be combative in their online commentary.  I argue that in its attempt to seek equality and inclusion, social justice discourse instead fosters inequality and exclusion by alienating large, and sometimes vulnerable, portions of society who lack the dexterity in English to engage in social justice dialogue. I assess the \textit{\#problematic} implications of this paradigm for language advocacy in the Caribbean and propose a shift towards a social justice dialectic grounded in local Creole languages.}


\IfFileExists{../localcommands.tex}{
   \addbibresource{../localbibliography.bib}
   \usepackage{langsci-optional}
\usepackage{langsci-gb4e}
\usepackage{langsci-lgr}

\usepackage{listings}
\lstset{basicstyle=\ttfamily,tabsize=2,breaklines=true}

%added by author
% \usepackage{tipa}
\usepackage{multirow}
\graphicspath{{figures/}}
\usepackage{langsci-branding}

   
\newcommand{\sent}{\enumsentence}
\newcommand{\sents}{\eenumsentence}
\let\citeasnoun\citet

\renewcommand{\lsCoverTitleFont}[1]{\sffamily\addfontfeatures{Scale=MatchUppercase}\fontsize{44pt}{16mm}\selectfont #1}
  
   %% hyphenation points for line breaks
%% Normally, automatic hyphenation in LaTeX is very good
%% If a word is mis-hyphenated, add it to this file
%%
%% add information to TeX file before \begin{document} with:
%% %% hyphenation points for line breaks
%% Normally, automatic hyphenation in LaTeX is very good
%% If a word is mis-hyphenated, add it to this file
%%
%% add information to TeX file before \begin{document} with:
%% %% hyphenation points for line breaks
%% Normally, automatic hyphenation in LaTeX is very good
%% If a word is mis-hyphenated, add it to this file
%%
%% add information to TeX file before \begin{document} with:
%% \include{localhyphenation}
\hyphenation{
affri-ca-te
affri-ca-tes
an-no-tated
com-ple-ments
com-po-si-tio-na-li-ty
non-com-po-si-tio-na-li-ty
Gon-zá-lez
out-side
Ri-chárd
se-man-tics
STREU-SLE
Tie-de-mann
}
\hyphenation{
affri-ca-te
affri-ca-tes
an-no-tated
com-ple-ments
com-po-si-tio-na-li-ty
non-com-po-si-tio-na-li-ty
Gon-zá-lez
out-side
Ri-chárd
se-man-tics
STREU-SLE
Tie-de-mann
}
\hyphenation{
affri-ca-te
affri-ca-tes
an-no-tated
com-ple-ments
com-po-si-tio-na-li-ty
non-com-po-si-tio-na-li-ty
Gon-zá-lez
out-side
Ri-chárd
se-man-tics
STREU-SLE
Tie-de-mann
}
   \boolfalse{bookcompile}
   \togglepaper[23]%%chapternumber
}{}

\begin{document}
\maketitle

\section{Introduction}

Social justice discourse, for better or worse, has taken on heightened importance in recent years and has been at the forefront of various advocacy movements all over the world in the last decade. A cursory glance at social justice movements between the Arab Spring of 2010 \citep{TheEditorsEncyclopediaBritannica2015} and the George Floyd protests in 2020 \citep{Taylor2021}, reveals how easily a protest that starts in one location can then emerge in several different spots globally. Social media is an ever-present arsenal in advocacy work and when combined with the twenty-four-hour news cycle it is not unusual that protests in far-flung regions of the world could be united around a singular concept, as expressed by a singular hashtag, even when advocates hail from different linguistic backgrounds. Jamaica, given its proximity to North America, is of course not excluded from these surges of social justice advocacy and it is not uncommon to see active and prolonged engagement with these issues on social media platforms. Indeed, Creole-speaking societies are all too familiar with tackling oppressive and discriminatory ideas, not least of which are concerned with language use. But how are imported social justice discoursive techniques, largely articulated in English, handled in predominantly Creole-speaking societies? How does this paradigm shape the attitudes of Creole speakers toward the very idea of social justice? 

Given the multitude of ways in which Caribbean life and identity have been characterized by injustices, it would not seem necessary to go and \emph{import} one, let alone one that did not already exist in the Caribbean in some shape or form. In fact, it is not the social injustice that is imported, but rather the discursive apparatus used to communicate and interrogate these issues. One such language-related social issue is linguistic human rights, which has entered the Caribbean discursive consciousness either through foreign media, but more likely through the work of local academics. Deliberations about language rights issues rarely, if ever, occur in the Creole language of the majority and are usually situated in contexts such as newspaper columns or editorials and in restrictive academic forums. The deliberations are meant to benefit the majority but do not incorporate their modes of communication nor sites of interaction. \citet{Devonish2014} addresses this issue in an article published by the \textit{Jamaica Gleaner} where he suggests that unless the government starts to communicate in the local language of the citizens, there will always be a communicative divide between the governed and the government. It is thus not accidental that discussions about human rights, social justice, intersectional privilege, and the like, are met mostly with ambivalence from the public at large. In worst-case scenarios, these same discussions are ironically seen as an external imposition, whether from foreign operatives or local elites, meant to further oppress poor Jamaican citizens. This chapter argues that this occurs partly because there is no widely accepted, grassroots discursive apparatus to deal with aspects of human rights such as language rights. What presently exists is perceived as contrived, external, and an imposition. 


\section{What is “social justice language”?}

Generally speaking, social justice language refers to the range of terms and phrases used to describe all the different areas of social justice. This could incorporate terms used to describe different forms of discrimination such as racism, ableism, and sexism; terms for platforms of advocacy like feminism; and theoretical paradigms associated with social justice like \emph{intersectionality} \citep{HumanRightsEquityService2021}. Only a few of these terms would pop up regularly in casual conversation -- most notably those related to discrimination on the grounds of sexual orientation or race -- and the few that do would feature prominently on social media. The fact is, the use of social justice terminology is primarily restricted to niche academic audiences, who may or may not be actively engaged in advocacy, and who are usually operating in a linguistic context dominated by the standardized variety of a European language \citep{Coppola2021}. Even if the concept of social justice is not foreign to Creole-dominant Caribbean communities, the modern discursive apparatus certainly is. 
Jamaica of course is no stranger to social justice concerns given its centuries-long legacy of colonialism. Every plantation revolt during the time of slavery, as well as the organized civil unrests shortly after slavery had ended, could all be considered acts of resistance to oppression. But it was not until the early 1930s with the emergence of Rastafari that one of the earliest, systematic, and indigenous social justice movements took root in Jamaica. The Rastafari arose as a decidedly Pan-Africanist movement with ideological influences drawn heavily from the teachings of Marcus Garvey and shaped by the harsh and oppressive socio-economic realities that working-class Jamaicans were facing at the time \citep{Edmonds2012}. One of the main philosophical ideas of the movement is that all oppressed African descendants should actively seek liberation through the rejection of all things “Babylon”\footnote{“Babylon” is broadly used to describe anything associated with former colonial powers. Institutions such as the government, school system, and church are treated as coming under the heavy influence of Babylon, the direct antithesis to the “Livity” (lifestyle) of the Rastafari.} and a return to the motherland of Africa \citep{Chevannes1994}. What is most fascinating about the Rastafari is the development of a language -- specifically, a discursive apparatus -- that is used to articulate and affirm the ideological stance of the movement. 

\citet{Pollard2009} treats the language of the Rastafari (or “Dread Talk”) as a lexical expansion of the existing language of local Jamaicans, Jamaican Creole. \citet[5]{Pollard2009} suggests, “What seems to be emerging is a certain lexical expansion to accommodate a particular, and for some people, a more accurate way of seeing life in Jamaican society.”

The same phenomenon is described by \citet{Schrenk2015} as a “reanalysis” of Jamaican Creole rather than merely a lexical expansion since “Rasta talk is predominantly based on calculated adjustments to perceived English lexical items.” Whether expansion or reanalysis, what is certain is that the linguistic ingenuity of the Rastafari was prompted by two primary factors: (1) the Rastafari needed a deliberate style of discourse to simultaneously challenge oppression and uplift the consciousness of the individual, and (2) neither Jamaican Creole nor English could satisfactorily fulfill this role. A new code had to be developed to frame and negotiate social justice matters through the lens of the Rastafari themselves, and this gave rise to Rasta Talk. 

The specifics of Rasta Talk, however, are not integral to the discussion here. What matters is the fact that the Rastafarian community noticed the linguistic vacuum for social justice terminology that was relevant to their perspectives, and went about filling that vacuum. Admittedly, though Rasta Talk has withstood the test of time, it has not evolved to a point where  it could serve as the discursive apparatus for social justice on a national level within Jamaica. This is neither a criticism of the Rastafari nor their linguistic ingenuity, but instead a statement about the new linguistic vacuum created by the complex ways in which social justice reasoning has changed and has started to infiltrate Creole-dominant societies, specifically as it relates to language rights. Rasta Talk might have successfully equipped its speech community to talk about issues of socio-economic and racial oppression, and a need for consciousness-raising for all Black persons, but how would it deal with, for example, intersectional privilege? What discursive apparatus exists in Creole dominant speech communities to explain the privilege differential between a poor cisgendered heterosexual dark-skinned basilectal speaker, and a poor gender-nonconforming light-skinned acrolectal speaker? These are complex questions that are made all the more difficult to answer by the fact that they are laden with technical English terms.

In many ways, Creole-speaking societies have been haphazardly navigating these very issues from time immemorial. Intersectional identities and privilege dynamics are nothing new to Jamaica and the wider Caribbean—they did not suddenly appear with the advent of social media. What is different since the explosion of social media and the mass consumption of North American content is that Jamaicans are now exposed to social justice concerns from the perspectives of the global north, in particular the USA and Europe \emph{in the discursive apparatus} of those territories. In an ironic acceptance of the strong version of the linguistic relativity hypothesis, some Jamaicans seem to think that the social justice ideas that exist in the global north are non-existent in Jamaica, presumably because an indigenous discursive apparatus is not seen. This is not the case. Social justice issues related to various types of discrimination, intersectional privilege, and human rights have always been a part of Jamaican life. What has not been as prominent is a method of discussing these issues using the Jamaican language as opposed to English. This phenomenon, I argue, has resulted in attitudes ranging from ambivalence to hostility whenever these same issues are broached in society. 


\section{Language rights in the right language}

Language advocacy has a fairly long, and checkered, history in Jamaica. Most Jamaicans would agree that the first bonafide language advocate for the recognition and appreciation of Jamaican Creole was the late Hon. Louise Bennett-Coverley, or “Miss Lou.” Miss Lou was writing and performing original poems, stories, and folk songs in Jamaican Creole from as early as the 1930s up until the time she died in 2006. What set her apart from her contemporaries in music and poetry at the time was that for her, the language was not merely a vehicle to deliver her artistry, but in her eyes, an emblem of national pride and an artifact worthy of both study and promotion \citep{Morris2014}. Her influence was as wide as the Jamaican diaspora, and she had the attention of all sectors of Jamaican society, from the head of state to the working-class woman in the city markets. She was simultaneously celebrated for elevating the status of the language and chided for moving attention away from Standard English, yet, through it all, her message remained constant and her position resolute -- the Jamaican language is a legitimate language. Miss Lou was doing language advocacy work before it had a name \citep{Forrester2022}. 

There is however one area of language advocacy that Miss Lou’s work did not explore in great detail, and that is the rights of the speakers of Jamaican Creole. Indeed, the idea that individuals could receive certain kinds of human rights provisions based purely on the particular language they used is a novel idea in Jamaican public consciousness and one which, even in the twenty-first century, has yet to fully catch on. Miss Lou used Jamaican folklore, comedy, and literature as her tools in her campaign for language advocacy, and while her advocacy no doubt laid the groundwork for what later followed, it simply was not enough to break into the frontier of language rights. For that, it would take a more targeted lobbying of the government, using the language of the government, and supported by research-based evidence. Miss Lou’s work was, and still is, paramount in this endeavour, but the next leg on the journey to language rights recognition was led by the academic community, most notably linguistic and literary scholars such as Hubert Devonish and Carolyn Cooper. 

This is where the public support started to run dry. Even when the guardians of the Standard English status quo in Jamaica disagreed with Miss Lou’s position on the validity of Jamaican Creole, she was at least humorous and therefore tolerable. From a political standpoint, Miss Lou was entirely non-threatening, and her primary domain of influence was from the stage. Academics like Devonish, however, meant serious business. He is an internationally respected full professor in the linguistics department who made a submission to the parliament to amend the Charter of Rights to include freedom from linguistic discrimination \citep{JamaicanLanguageUnit2011}. Cooper, though she did not directly lobby the parliament, proved no less an annoyance to the establishment, with her weekly bi-lingual column, “(W)uman Tong(ue)” published in \textit{The Observer} national newspaper, as well as delivering her inaugural lecture in Creole on the occasion of her promotion to full professor of literature. Both Devonish and Cooper, who have each published in Jamaican Creole, have consistently faced a level of public backlash and vitriol that would not normally be directed at Miss Lou.\footnote{Devonish and Cooper are now retired professors but have left a legacy of work calling for the acceptance of Jamaican Creole in more formal domains of usage such as government and education. Additionally, they’ve supervised and mentored scores of graduate students and junior academics who have continued this advocacy in various forms of linguistic and literary research.} This is in large part due to the fact that while Miss Lou only encouraged national pride in the language, academics like Devonish and Cooper wanted to take Miss Lou’s advocacy to its logical conclusion and were calling for having the language constitutionally recognized, made official alongside English, and used as the language of instruction in primary schools. This was a pill that proved too difficult to swallow without the sugarcoating of Miss Lou’s humour. 

This brings us to the present state of affairs as it relates to language advocacy in Jamaica. It is an undertaking almost completely concentrated in the voices of a small group of academics, no more than ten, all of whom were mentored, trained, or influenced by Devonish. The public sentiment, based on comments posted on newspaper messaging boards and on social media, is that the present generation of language advocates (inclusive of Devonish and Cooper), comprise an elite academic cabal, themselves already adept at Standard English, who want to keep ordinary folk down by “forcing” Jamaican Creole into public formal domains. In an unusual turn of events, tangible realizations of language rights advocacy such as constitutional recognition and Creole use in education are seen as a pointless academic preoccupation and a “foreign” imposition that will disadvantage the very people it seeks to assist. 

Devonish himself has spoken extensively about the derisive views of the educated elite in Jamaica as it relates to language rights. \citet[55]{DevonishCarpenter2020} state:

\begin{quote}
    The mass media in Jamaica, notably radio, television, and newspapers, over the last three decades at least, has been the arena for the “chaterrati” and their views on what has come to be labeled the “Patwa-English Debate.” The educated elite, the chattering classes or the “chaterrati” whose views dominate the traditional mass media, have treated this topic as a form of blood sport, as a target for literate and literary jibe.
\end{quote}

An interesting conundrum now presents itself. Devonish and Carpenter suggest that the views of the educated elite, or the chaterrati as they call them, have permeated the mass media in Jamaica and have transformed the discussions around language recognition and rights into a pointless combat of opposing opinions. Yet, one of the most persistent public criticisms of the new language advocates, in particular Devonish and Cooper, is that \emph{they} are the educated elite hell-bent on using their academic machinations to force the Creole issue against the wishes of the masses. Both sides—the language advocates and their vocal public opposition—are accusing each other of elitism! 

In the next section, I present a discourse analysis of an article written in one of the national newspapers, \textit{The Gleaner}, that quotes a former head of state arguing that teaching Jamaican Creole in schools is a waste of time. The article is typical of the cantankerous language issue and how the media plays the role of the arena, referee, and fight promoter. The views of the “elites” from both sides are presented (albeit not very accurately from the academic perspective) and the public is left to judge which of the two has emerged as the victor. At the end of the analysis, I include excerpts from message board comments left on the newspaper’s website to demonstrate how members of the public view the issues. 

\subsection{The data}
\largerpage
The data for this chapter comes from a corpus of twenty-four newspaper articles dealing with the “Patwa-English Debate” published in \textit{The Gleaner} between 2010-2017. I selected articles for the data set because of the topic it dealt with and the fact that it had at least three message board comments at the end. Some articles had as few as 3 or 4 comments, and others had as many as 95 or 144, with most articles having about 15 comments. There were 738 comments from the twenty-four articles in the data pool. The topics for the articles dealt with various aspects of the Creole debate including bilingual literacy, language standardization, making the language official, and language rights. The breakdown of authors includes editorial writers (2), politicians (1), teacher\slash education consultants (4), regular contributors (5), and professors\slash academics (6) (professors\slash academics were the only category of authors that had multiple articles from the same authors). 

Message board comments from readers at the end of newspaper articles, while still largely understudied, present a fruitful site for analysis. With the advent of Web 2.0, “participatory journalism”, as it has been coined, has grown in popularity internationally \citep{Reich2011,HermidaThurman2008,Ornebring2008} to the point where most audiences expect to be able to participate \citep{Jenkins2008}. For the media house, this type of engagement is an indication of what kinds of issues the reading public is interested in, and the marketing team at the newspaper can find innovative ways to monetize this interest. For the social scientist, this engagement is as good as the kind of qualitative data that one would find in a survey with open-ended questions. The two language attitude surveys conducted by the Jamaican Language Unit (JLU) tell a story of overall positive language attitudes from the 1000 participants sampled in each study, but it is not the only story. Reader feedback to newspaper articles dealing with the language issue in Jamaica presents a different perspective and it is pertinent to at least be aware of it.

For this sample analysis, the article “A Waste of Time to Teach Patois -- Seaga” published in \textit{The Jamaica Gleaner} on April 11, 2011, will be used. This article, written by Keisha Hill, a staff writer at \textit{The Gleaner}, was published in the Lead Stories section (online) and had the highest level of engagement of all the articles with 144 comments. Just below the headline, it features a prominent picture of Hon. Edward Seaga, former Prime Minister of Jamaica turned Distinguished Research Fellow at the University of the West Indies Mona, and Chancellor at the University of Technology Jamaica. The article is analyzed in terms of its recognizable journalistic components in the following order (i) headline, (ii) lede, and (iii) body of the story.

\subsection{Headline analysis}

The headline for the article, “A Waste of Time to Teach Patois”, uses a direct quotation of Hon. Edward Seaga. The quote establishes the position of the former Prime Minister without presenting any specific details about whose time is being wasted, to whom Patois is being taught, and who is doing the teaching. His picture right below the headline captures Seaga focusing his attention to his right, as if responding to an interviewer or engaged in debate, with his fingers flexed in the familiar position of someone delivering a premise to support their idea. The details of the headline are not necessary for the cultural understanding of the specifics to which Seaga refers; all that matters is that teaching the language is a wasted effort regardless of context.
 
There is, however, interesting background to the idea of teaching Patois that can trace its roots all the way back to 2001 (and perhaps before) when the Ministry of Education issued its first language and education position paper where it outlined several potential options for responding to Jamaica’s “language problem”, one of which was the adoption of a bilingual model of education at the primary and secondary levels, where both English and Jamaican would be used as the language of instruction \citep{MOEYC2001}. The Ministry of Education did not pursue this option, stating that the resources were not in place for such an undertaking, but Seaga here reiterates one of the most prominent public sentiments as it relates to this issue—it is a waste of time. Specifically, it is a waste of time (and by extension resources) for the Ministry of Education to develop a bilingual education curriculum for the primary and high school levels. 

As a former Prime Minister \emph{and} Minister of Finance himself, parliamentary positions with the greatest oversight of the national budget, Seaga would have a sense of what kinds of budgetary undertakings would be too cumbersome for Jamaica and bordering on wasteful. Seaga, however, isn’t merely a distinguished politician but also a researcher of Jamaican culture who has published in the areas of Jamaican music and folklore\footnote{Two of Seaga’s most popular publications on Jamaican culture were “Revival spirit cults” \citep{SeagaEdward1968} and “Reggae Golden Jubilee: Origins of Jamaican Music” \citep{SeagaEdward-album-2012}, a 100-track Reggae compilation to commemorate the 50\textsuperscript{th} anniversary of Jamaica’s independence.}, so he wields the authority to opine about what aspects of culture even deserve budgetary itemization in the first place. His considerable influence in the Jamaican political landscape has seen him presiding over significant constitutional changes in the form of the Charter of Rights, as well as important cultural milestones not least of which is the naming of Jamaica’s first national hero, Marcus Garvey. Of all the stakeholders who could influence this discourse on teaching Patois, Seaga ranks high among them.


\subsubsection{The lede}

The lede for the article departs somewhat from the canonical journalistic model of answering the five Ws— who, what, when, where, and why—and instead starts off by presenting the views of another prominent public figure who weighs in from time to time on the language debate, Prof. Hubert Devonish, professor of linguistics and then head of the linguistics department at the University of the West Indies campus in Jamaica.
 
The lede starts off by framing the context within which Prof. Devonish’s comment is made: 

\begin{quote}
    “In a renewed debate for recognition to be given to Jamaican Creole...”
\end{quote}

It highlights the fact that this debate about the place of Jamaican Creole both in Jamaica’s constitutional framework as well as the education system, has not only been ongoing but suggests that the debate somehow subsided and is now being reignited by Devonish with the new caveat that this time the recognition should come in the form of a constitutional provision. Of special note in the first few words of the lede is the fact that the label Jamaican Creole is used for the first and only time in the article, all other references to the language make use of the word Patois. This is worth commenting on since, while both terms refer to the same language, Patois is the label used by all speakers of the language in everyday dialogue, while Jamaican Creole is reserved primarily for academic usage. By using Jamaican Creole only in reference to the suggestion from Devonish, Devonish’s ideas are presented as more of an academic endeavour. 

This first portion of the lede is very important in establishing the ideological framework through which the discourse of language issues in Jamaica can be viewed. The idea of a “renewed debate” presents imagery of two opposing sides locked in a timeless battle punctuated by periods of heated exchange and prolonged silence. Indeed, the very presence of the word \emph{debate} frames the issue as having only two sides polarized by divergent viewpoints when this might not in fact be an accurate description of the situation. The media, however, specializes in presenting matters as dichotomous, and as such, it is no surprise that the first section following the headline with Seaga’s declaration that teaching Patois is a waste of time is Devonish’s call for Jamaican Creole to be given constitutional recognition.  The two most polarizing positions are highlighted very early in the article to establish clear demarcations on the issue. 

The lede continues: 

\begin{quote}\relax
[Devonish] has proposed that “language rights” should be recognised in the Charter of Rights.
\end{quote}

There is a deliberate attempt to bring attention to the term language rights by way of inverted commas, not because it is a direct quotation from Devonish, but because the concept is novel and perhaps unusual. In reality, the concept is at least ten years old at the time of the publication of that news article, seeing that Devonish himself was the first to propose the initiative in 2001 when the Jamaican parliament met to amend Chapter 3 of the constitution which dealt with fundamental rights and freedoms. What Devonish proposed in 2001 was not a blanket “language rights” provision, but instead, the inclusion of freedom from discrimination on the grounds of language, as a means of ensuring speakers of Jamaican receive equal treatment when dealing with state agencies. On the one hand, this might have been an attempt by the newspaper to abbreviate Devonish’s actual parliamentary submission, but their reductionist approach instead lumped the proposal along with the wave of other post-modern rights that have made their way into the Jamaican political landscape, not least among them LGBT rights. And just as how many Jamaicans think LGBT rights are an attempt by one sector to seek \emph{special} rights for themselves, so it is that language rights might be viewed as speakers of a particular language accruing special rights for themselves. This is the ideological frame through which various branches of human rights advocacy are viewed — any qualification other than human, be it women, LGBT, disabled, or language is really an attempt at seeking preferential treatment rather than equality. Whether intentional or not, framing Devonish’s parliamentary proposal in this fashion already assures disdain by some. 

The lede concludes by pointing out:

\begin{quote}
    There have also been proposals for a Patois Dictionary, a Patois Bible and for the language to be used at the primary level in schools.
\end{quote}


Of note is the fact that these three suggestions are without attribution—the report simply states that proposals have been made. But the mere fact that they occur adjacent to Devonish’s \emph{renewed} debate for language recognition, presents them as a continuation of his ideas. As if language rights were not enough, it appears that Devonish is pushing forward a Patois trifecta by proposing a dictionary, a bible, and a primary-level curriculum all in Patois (not the scientific label, \emph{Jamaican Creole}, this time). In actuality, only one of these proposals—the use of the Jamaican language as the language of instruction—has been put forward by Devonish, after having done a pilot study no less. The other two proposals have nothing to do with his work; the Dictionary of Jamaican English was published in 1961, a decade before Devonish was an academic, let alone a Jamaican language advocate and the Patois bible was a private initiative by the Bible Society of the West Indies. Devonish stands in as the default public face of language advocacy: he is presented as renewed and armed with a bevy of fresh language proposals. We now have the proposing team in the debate.

\subsubsection{The body}

Not surprisingly, the section immediately following the lead paragraph starts with a rhetorical shift: 

\begin{quote}
    However, former prime minister and chancellor of the University of Technology, Edward Seaga, weighing in on the issue, says it would be a waste of the country’s educational resources to teach Patois in schools.
\end{quote}

It is crucial to point out here that this is entirely a constructed debate between Devonish and Seaga. There is no indication that both of these debaters were either in the same location or directly addressing each other’s views when these comments were made. In fact, that could not have been possible given the timeline of proposals attributed to Devonish:

\begin{enumerate}[label=\Alph*.]
    \item Constitutional recognition of the Jamaican language or “language rights” happened in 2001 when Devonish made a submission to a committee in parliament.

    \item The Dictionary of Jamaican English, or the Patois dictionary as the newspaper puts it, was published in 1967 by a different linguist, Fredrick Cassidy. 

    \item The Patois Bible\footnote{The correct name for this publication is the Jamaican Diglot New Testament with KJV Bible, the “Patois Bible” is the term that most often appears in the media primarily because it is a more transparent label for the public \nocite{BSWI2012}} was an initiative funded by the Bible Society of the West Indies which was first announced in 2008.  

    \item The Bilingual Education Project (B.E.P), or a move to make Patois the main vehicle of communication in primary schools\footnote{This too is another mischaracterization of the B.E.P—the aim of the project was to provide bilingual education in both Jamaican \emph{and} English, giving both languages equal time in the classroom across all subject areas.} as the newspaper frames it, was piloted in 2004 by Devonish through the Jamaican Language Unit. 
\end{enumerate}

Given that the current newspaper report is happening in 2011, all of the items in A-D above have been stitched together and attributed to Devonish as a coherent, \emph{and} formidable position, ready to be challenged by Seaga and the other debaters present in the news report.

Returning to the opening rebuttal from Seaga, we notice that the reporter has included his status as the former Prime Minister of Jamaica and then current Chancellor of the University of Technology, thereby lending credibility to his contribution. Seaga’s opening here is paraphrased by the reporter who presents it as \emph{a waste of the country’s educational resources}, indicating that the move is wasteful both in the form of human and capital resources. Last but not least, is the mischaracterization of the language education proposal:  specifically, Devonish is proposing the use of Jamaican Creole as the language of instruction, but the newspaper presents it as teaching Patois in schools. 

From the standpoint of the lay public, this difference is merely a semantic one—to use the language as the means of instruction is no different from teaching the language. The difference from a linguistic standpoint,  however, is crucial. What Devonish and several other Caribbean linguists are proposing is rooted in decades-old research done by the United Nations Education and Scientific Council (UNESCO). It was documented by UNESCO that children who are instructed in their vernacular in the early years of primary education stand a better chance of successfully transitioning to a second language of instruction, usually a European language, during the latter part of primary education (Global Education Monitoring Report Team 2016). In this kind of bilingual education, all the different subject areas are taught in the vernacular, which is dramatically different from having the vernacular as a discrete subject area, which is how the issue is typically framed by the media and subsequently in the public perception. Either way, given the mixed public reception to the idea of even having Jamaican as a stand-alone subject, there is little doubt the objection would hold if \emph{all} subjects were to be taught in the Jamaican language. 

Seaga gives us the most common explanation for the public objection when he continues: 

\begin{quote}
    “There is no standard way of spelling a particular word in Patois,” Seaga said. “If you want people to be able to talk to one another in Jamaica and outside of Jamaica, it does not make any sense.”
\end{quote}

The idea that no standard spelling exists for the Jamaican language is a commonly perpetuated misunderstanding. It is correct to say that no standard spelling is widely \emph{used or known}, but it is not in fact correct to say that no standard spelling exists. Fredrick Cassidy, the Jamaican lexicographer who did the groundwork for the Dictionary of Jamaican English (DJE), developed a standardized phonemic writing system for the language in the 1960s. However, the one place that could ensure the widest public knowledge of the writing system, is the one place where the standard writing system is not introduced—at the primary school level. When influential figures like Seaga repeat this misconception, it further legitimizes the idea that incorporating the language into education is a futile project, since it lacks even the basic pre-requisite of a medium of instruction; a standard spelling and writing system. 

The second part of this statement, “if you want people to be able to talk to one another in Jamaica and outside Jamaica,” addresses the common, notwithstanding reasonable, concern of perpetual insularity. Seaga, like so many who discuss this issue, manages to link education in one’s mother tongue with an inability to communicate in anything other than said mother tongue. It invokes the rhetorical strategy of suggesting that adopting one course of action (teaching Patois to children) will ultimately exclude all other courses (teaching any other language to children) since they are in competition. A cursory look at educational systems globally would reveal that this is not true—children can be educated in the vernacular languages of their speech communities and go on to be multi-lingual in several other languages. Children can learn more than one language at a time: a completely obvious and common sense axiom that somehow is not widely accepted in the Jamaican context when one of the languages is Patois. 

Seaga then concludes by giving his opinion of the diglossic divide between English and Patois in the Jamaican situation: 

\begin{quote}
    He added: “If you look at it, government and commercial papers are all in English. Newspapers are mostly in English with a few Patois articles and Patois quotations in English articles. Television and radio are mixed with English and Patois and popular culture such as songs, DJ lyrics, and roots plays are mostly in Patois.” 
\end{quote}


He starts off by saying “if you look at it” which could mean if you are observant, you will notice the obvious. He goes on to list the domains of usage for English and Patois starting with the most prestigious and \emph{purest}, government and commercial papers, since this domain is untouched by Jamaican Creole, then to the intermediate domains which have some overlap of both languages, and finally to the domains which are Jamaican Creole dominant -- DJ lyrics and roots plays, domains holding the least prestige. Seaga urges us to “look at it” and observe the \emph{natural} order of things as it relates to domains of use for languages in Jamaica and to further preserve this reality. Interestingly, even in Seaga’s own schema, Jamaican Creole is only reduced or absent in domains that are predominantly written. For Seaga, any attempt to disrupt the observed reality of Jamaica’s diglossia is a waste of time, effort, and resources. 

Dr. Ralph Thompson is introduced into the debate following Seaga’s comments: 

\begin{quote}
    Education advocate Dr. Ralph Thompson says it is important for teachers, especially at the early childhood level, to understand the language as many students speak Patois fluently, even though some are unable to read or write it.
    
    If the early childhood teachers speak standard English, of course first they have to be able to speak Patois as well because if you go into a classroom and can’t speak Patois, you cannot connect to the kids,” Thompson said.
\end{quote}
 

Thompson, though possessing some sense of the nuance involved in language education at the early childhood level, is not as forthright as Devonish who is presented as arguing to teach Patois. Thompson, whose comments are legitimized by his status as an education advocate, merely wants teachers to use Patois as a means of \emph{connecting} with students. Far from the actual recommendation of the use of the language as a means of instruction, or the Gleaner’s framing of the issue as teaching Patois as a separate subject, Thompson favours the milder approach of using the language merely as connective tissue between teachers and students. Students would first be primed in Patois, and once they have settled down, \emph{actual learning}, the sort that is done in English, can begin. This sentiment is reflected in the last two lines of the article: 

\begin{quote}
    Said Thompson: The good thing for children between zero and six is their ability to learn and grasp information quickly. The teacher can get their attention speaking in Patois, but reinforce English in the same sentence and you will see how quickly they understand.
\end{quote}

Though Thompson sees Patois as having a place in the education system, he treats it as a strategy for facilitating learning, rather than the vehicle or focus of learning (which would require the proper institutional framework to execute). The final voice in the debate channels Seaga’s perspective and comes from then Prime Minister of Jamaica, Hon. Bruce Golding: 

\begin{quote}\sloppy
    Recently, Prime Minister Bruce Golding also weighed in on the debate. Speaking at a graduation ceremony at Kingsway High School, Golding said the debate about teaching Patois as a second language and translating the Bible into Patois signify an admission of failure. According to Golding, teaching Patois would be akin to saying, “We have failed to impart our accepted language of English, so we are giving up. This one can't work, so let us find another one that can work.”
\end{quote}


For the first time, an actual context within which the utterances are made is given. Golding’s comments take place at the graduation ceremony of a private school in the country’s corporate area, where he is addressing the gathering as a guest speaker. Like Seaga, Golding disapproves of any move to introduce Patois into the educational framework and further sees such an initiative as an admission of failure. He too adopts the idea that an introduction of Patois into the educational system means the exit of Standard English: 

\begin{quote}
We have failed to impart our accepted language of English, so we are giving up. This one can't work, so let us find another one that can work.
\end{quote}

Of interest is how Golding frames his ideas as a dialogue among stakeholders involved in the decision-making process of language and education. No such dialogue could have happened without the oversight of his government. Golding essentially mocks those engaged in their defeatist dialogue—having failed at teaching English without real effort, they have to resort to something easier. The referent for the first person plural pronoun “we” at the beginning of Golding’s statement is those supportive of the initiative to teach Patois -- Devonish, and crew -- but the referent in “our accepted language of English” is the entire nation of Jamaica. The “we” who want to teach Patois, is different from the “we” who have the good sense to accept that English is \emph{our} language. Golding uses this strategy to great effect—the language advocates promoting the teaching of Patois appear to do so quite flippantly: English does not seem to be working, so let us discard it and try again with an easier language.


\subsection{Summary}

The article used in this analysis is typical of the kinds of articles published in \textit{The Gleaner} which deal with language issues in Jamaica. There are usually two or more “talking heads”, who may or may not be in the same location when giving their views, and who may or may not even be aware of the views being expressed by the \emph{opposing} debater within the article. \textit{The Gleaner}, and media at large in Jamaica, simply arrange these ideas at polar ends of the discussion, even when the same talking heads do not see their ideas as being at odds with each other. The media has thus managed to consistently overlook nuance or middle grounds on this issue even when they exist: polarized positions are \emph{easier} to understand and make for better reading material. 

The situation is not helped by the fact that the media is not as precise as it could be when relaying the views of the linguists in this debate. No linguist, least of all Devonish, who proposes that Jamaican Creole should be used as the medium of instruction, thinks that children should not also be proficient in English. In fact, linguists who support the initiative see it as a means of achieving this English proficiency. This is rarely if ever expressed clearly in the media reports unless one of these linguists manages to publish a column in one of the major newspapers or does an interview on a public television or radio station. Perhaps linguists should shoulder some of the blame here for not being able to deliver their ideas in a way that the general public can easily comprehend, but the media, whose responsibility it is to do exactly that, also does a poor job of the communication by simple mischaracterizations and exclusion of important details. 

Several themes emerge from this article that embody the public opposition to attempts at moving the Jamaican language into public official domains such as education and the constitution. They are: 

\begin{enumerate}
    \item The resources required to incorporate Jamaican Creole into the educational system would simply be a strain on Jamaican taxpayers and essentially a waste of said resources. 

    \item Jamaican Creole lacks the basic entry requirements to be considered as a medium of instruction (that is, a standard spelling/writing system and scientific lexicon). 

    \item The status quo already favours English and the languages already have their proper domains of usage.  

    \item Teaching Jamaican means abandoning English — Jamaican children would be distracted and possibly confused by having both languages in the educational system.
    
    \item Jamaicans run the risk of political and economic isolation (from North America and the UK) if they learn their language in a formal educational setting.
\end{enumerate}	 

These are just five of several themes which emerge whenever the language issue is presented in the media; the list above is not exhaustive, but merely reveals five of the more prominent themes. All of the themes listed above were extracted from the utterances of two former Prime Ministers of Jamaica, both of whom had considerable influence during their terms as government leaders (and even afterward at least in the case of Seaga).  


\section{Reader feedback}

A sampling of the reader feedback to the article analyzed above, reveals that readers are mirroring some of the themes highlighted in the article. A few of the comments are presented below with the user name of each commenter as well as the number of “up-votes” (the equivalent of likes) that each comment received: 

\begin{quote}
    Sample 1: Patois being taught in school is a total waste of time and money as it has no place in commerical (sic) business and it is of no use to someone who wishes to go to an overseas university. Some people are advocating patois because it is a means of stalling the advancement of our people. [RDL, 53 up-votes]
\end{quote}

This comment above by user RDL neatly summarizes the most common objection to the use of Patois in formal domains—it simply has no place in those contexts, and worse, it is a useless addition for those who have their eyes set on overseas study. RDL also explicitly states that there is a more nefarious plot behind advocating for the language in these formal contexts, and it is to derail the social and perhaps economic advancement of Jamaicans. 

\begin{quote}
    Sample 2: ... People like Carolyn Cooper was (sic) educated using the English Language. Many of those professors at the University of the West Indies, were at tax payers’ expense, that is, your poor mother and father paid taxes so that they could be educated and now their contribution to society is to encourage further degradation of our children. [KarenFed]
\end{quote}

Prof. Cooper, who was not mentioned in the article, is nonetheless dragged into the conversation via the comments section from the user KarenFed. This user raises the issue that impoverished taxpayers bankroll the education of professors, provide them with proficiency and mastery of English, and then these same professors, in an ungrateful about-turn, decide they will further “degrade” the nation’s children with the Jamaican language.

\begin{quote}
    Sample 3: I wonder why we continue to have this continued thrust by Professor Hubert Devonish to force the Jamaican Creole upon the people... We need to resist every effort by Mr. Devonish and his group to drag Jamaica back to such an awful era... These people with too much time on their hands, too much public money to waste. Too much paid to them, too little fortitude to fight the good fight (to teach English...) [St. Marian, 27 up-votes]
\end{quote}

St. Marian believes that Devonish and company are trying to drag Jamaica back to an “awful era” which, one can presume, is a reference to Jamaica’s colonial past that gave rise to the emergence of Jamaican Creole. There is a caution to resist the efforts of this group who, despite an abundance of time and money, still lack the strength to “fight the good fight” and teach English. 

\begin{quote}
    Sample 4: Thank you Mr. Seaga. I have been preaching the same thing for years. The only way a child can learn to read is to read, read, read. In China all children are now required to learn English and all govt workers in China are required to learn 10,000 english (sic) words or phrases. If the Chinese can do this then why can’t our children who live in the third largest english (sic) speaking country in North America behind the US and Canada.  [Keith, 31 up-votes]
\end{quote}

Seaga receives a ringing endorsement from user Keith who is baffled that the largest English-speaking country in the Americas behind USA and Canada is unable to teach children English when China manages to do so effortlessly. Keith simplifies it for the academics who are forcing the bilingual education method and suggests instead that all children would need to do is to “read, read, read.”

\begin{quote}
    Sample 5: There should be a law against this professor talking and proposing such nonsense. He really should be arrested for wanting to commit such crime against the English language and ultimately against poor people who are trying to speak so that they can be a part of the world. [Joe, 28 upvotes]
\end{quote}

Devonish’s advocacy, according to user Joe, should be punishable as a criminal offense. The victims, in this case, are the poor people of Jamaica and the English language itself. Poor people are merely trying to better themselves through the use of English and Devonish is determined to rob them of even that. 

As is common in online platforms, commenters will occasionally engage in heated debate and the Gleaner newspaper’s virtual message board is no different. Indeed, a small portion of the comments on this article was directly challenging some of the other users and showing support for the advocacy work of the linguists. Here is a response to Sample 5: 

\begin{quote}
    Sample 6: My poor misguided Joe, it is beyond me how someone can commit a crime against a language, a system of communication... The same people of whom you speak are not allowed access to certain services because of how they speak. If you look back in history this was the case for the English language as well. If you never spoke French or Latin etc. but lowly English you were ostracized. They [sic] English has to fight for their status and one way they did it was to fight for their language. Why do we as a nation refuse to follow suite? Is it, Joe, that you will have us clinging to our colonial mother’s breast until we commit homicide on our mother tongue? [ACS, 4 upvotes]
\end{quote}

Linguists may well have the intellectual authority to articulate the practical steps that lead to linguistic rights recognition, but based on public sentiments expressed in the newspaper and elsewhere, we seem to have lost the moral authority. These same linguists, born in the Caribbean and themselves native speakers of Creole languages, are seen as separate from the communities they wish to serve and are actually invested in retarding the social and economic development of the members of these communities. It is not difficult to see how this accusation might get formulated: all the linguists involved in advocating for the rights of Creole speakers have already gained mastery of English, are comfortably insulated in high-paying university positions, and have the luxury of seeking employment on the international market. What is worse, academics receive merit-based promotions and salary increases even if our advocacy falls on deaf ears, as long as it gets published. There is no downside to engaging in this work -- if it improves the lives of Creole speakers, good; if it does not, at least it resulted in a journal article. Such is the perception of the academic as advocate. 


\section{Discussion}

Despite the precarious position the academic advocates find themselves in, there is still a role for linguistic expertise to address the issues outlined in the preceding discussion. Social justice concerns are increasingly a part of everyday Caribbean life, and in the areas where language and law intersect, there is evidence that linguistic rights are a potential area for which legislation will have to be drafted at some point in the future. But if the public at large sees the advocacy of linguists as working against the public interest, is there a path forward where academic expertise could benefit from public approval? I believe there is. 

\citet{Devonish2015-LanguageLiberation} makes an insightful observation that echoes the earlier sentiment in this chapter that the original language advocates were, from the standpoint of the political establishment, largely non-threatening. He uses the example of Miss Lou from Jamaica, and Wordsworth McAndrew from Guyana, who, despite their vigorous and lifelong work in the promotion of the language and folklore of their respective territories, accepted the status quo which favours English as the language of “serious” business and the Creole as the language reserved for informal private communication. Their goal was more about consciousness-raising than disrupting or overhauling the system. Arguably, this made it easier for successive post-independence governments in Jamaica and Guyana to draw on the work of Miss Lou and McAndrew for promoting cultural heritage. It was also easier to confer national awards and accolades (sometimes posthumously) on these advocates. Folklorists like Miss Lou and McAndrew, even though they spent their lives promoting low-status Creole languages, have enjoyed a level of public approval and support that academic advocates of the same Creole languages have never received. Undoubtedly, a part of this is rooted in the fact that these original advocates were not perceived as challenging the status of English. But the other, more salient feature of these original advocates was that they \emph{belonged} to the people. For all the awards they received later in life, and the training in British media they both received, Miss Lou and McAndrews were the embodiment of the grassroots. Any working-class Jamaican or Guyanese could see themselves in these two figures, because they talked with the people, about the issues of the people, and in the language of the people. Anyone who had the pleasure of meeting Miss Lou for example would recount how they came away from the experience feeling a newfound pride and appreciation for using their Creole language. The new academic advocates of Creole languages struggle to inspire this kind of pride in the grassroots communities. 

Admittedly, as an academic now living in the Jamaican diaspora, I have the benefit of being one of the foremost resource persons on the Jamaican language simply by virtue of being one of only a handful of persons with expertise in the area. Whenever I have a public lecture for members of the Jamaican diaspora, it is almost expected that a portion of the presentation would be done in Creole, if not the entire presentation\footnote{This was the case at a recent panel discussion to honour Miss Lou on the centenary of her birthday. My presentation on the journey of the Jamaican language from the plantation to lobbying the parliament for official language status was done entirely in Jamaican Creole.}, to the delight of the audience members. There is a real and sustained interest in the diaspora communities to learn about the evolution and usage of the Jamaican language and members are usually willing to invest time and money in acquiring proficiency in the language\footnote{I have several personal examples where this is concerned, such as the establishment of a Creole Heritage language program in Brampton, Ontario, a six weeks course in Basic Jamaican Creole run by the Jamaica Association of Montreal, and several other paid consultancies by lawyers who are trying to defend a speaker of Jamaican Creole in the Ontario court system.}. Such is the paradox of linguistic insecurity; the desire to use and promote the language appears strongest when the effort will have the least political impact. This is a large part of the reason why cultural practitioners are more successful at language advocacy than academics. Cultural practitioners can measure their success in hearts changed, but academics measure their success in policies implemented. It is no surprise then that Jamaican language advocacy would receive a warm reception from members of the diaspora since the impact on policy is negligible. Indeed, Jamaican Creole is not only a lower-status language in the various diasporas, but it is also an immigrant language of a minority, racialized group. And this, ironically, is why the academic advocate (or any language advocate for that matter) stands a good chance of garnering community support and approval in the diaspora. 

This is what the late Mervyn Alleyne has to offer as a way forward for the language advocate:

\begin{quote}
In all our work, I would stress one overriding need: we need an applied focus, with a partnership between linguist and community, not a divorce and separation. This may seem idealistic or unrealistic, but it should mean a serious purposeful campaign to train persons to study their own languages. \citep[13]{Alleyne2004}
\end{quote}

Alleyne was delivering this recommendation to an audience of linguists at the opening plenary of the Society for Caribbean Linguistics in 2002 at the UWI, St. Augustine campus. This was years before the explosion of social media, and even before it was common to have an internet-capable device in your pocket at all times. Yet, Alleyne’s own suggestion underscored a situation that has persisted until today, that is, the linguist is seen as separate and apart from the community they wish to advocate on behalf of. The comments at the end of the article in the sample analysis above, reflect this idea—language advocates attached to the university are seen as working against the best interest of communities.  

It is important to add some nuance to these user comments. While I agree with \citet[605]{Canter2013} who states “The nature of this type of comment participation is varied and some studies suggest that readers are mostly interested in discussing matters of personal interest or making abusive comments”, it would be hard to dismiss all the negative comments on the news articles as simply outbursts from the ignorant. Some of these comments do indeed emanate from bonafide internet trolls, but when the same comments keep recurring over several years and from different readers it may be time to ask whether there is a kernel of truth. And, more importantly, what, if anything, can be done about it? 

To its credit, the Jamaican Language Unit (first under the directorship of Prof. Hubert Devonish, and now led by Dr. Joseph Farquharson) has achieved significant strides in at least ensuring the “Patwa-English Debate” has a strong and data-driven perspective from the academic side. Devonish, and now more recently Farquharson, has spent a considerable amount of time engaged in public education both in forums that make use of traditional media, such as newspaper columns and TV interviews, but also in community meetings and online discussions. Even Prof. Cooper too, who has always frustrated the establishment with her use of literary subversion, and at times has shouldered the worst of the online battering (she is, after all, a woman), has also advanced the cause of promoting the Jamaican language. Language attitudes are not \emph{as bad} as they once were, the government has gradually become more receptive to the idea that Jamaican Creole should play some role in education, and far more members of the public today are able to reference the work of language advocates in the ongoing debate. The language advocates, who have largely (and perhaps rightly) ignored comments on their newspaper articles, seem to be doing something right. 

Despite the advances, there remains significant work to do. Advocacy work is soul-draining work; those who are engaged in it are few and usually a hair away from burnout, yet detractors are legion and are sustained by their ignorance. In an ideal situation, the language advocate has the expertise of Devonish and Cooper, the grassroots authenticity of Miss Lou and McAndrews, the community support and approval like what exists in the diaspora, but resides in the Caribbean where their advocacy is likely to have its greatest political impact. This is a tall order that has yet to be achieved. One of the important first steps in creating this ideal situation is for academic language advocates to devote some of their work to fostering community pride in Creole languages. And this needs to be done in the language of the community. Far from being an elaborate public relations campaign, the goal here is to ensure that the academic advocate is seen as a member of the community and is interested in the concerns of the community, specifically those which intersect with language, but also with those that do not. If, as the comments in online message boards suggest, the public sees the academic language advocate as out-of-touch elites, then advocacy will always remain an uphill battle. The linguist must strike a balance between being an agent of the academy and a member of the language community for which the advocacy work is carried out. This, I submit, is the best strategy for sustained and impactful language advocacy. 

% bibliography:
{\sloppy\printbibliography[heading=subbibliography,notkeyword=this]}


\end{document}
