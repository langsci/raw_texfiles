\documentclass[output=paper,colorlinks,citecolor=brown]{langscibook}
\ChapterDOI{10.5281/zenodo.10103082}

\title[Swearing-in]{Swearing-in: Language, class, and access to justice in Jamaica}
\author{Glenroy Murray\affiliation{Equality for All Jamaica} and Suelle Anglin\affiliation{Equality for All Jamaica}}
\abstract{Access to Justice as a cornerstone of the rule of law is complicated by the language situation in Jamaica. In a country where the striking majority of persons are either basilect-dominant or bilingual speakers \citep{JamaicaLanguageUnit2006} and the language of the courts remains Standard Jamaican English \citep{Smith2017}, communicative problems arise for those who are basilect-dominant as they seek to navigate courtrooms either to profess their innocence or seek to obtain redress for wrongs against them. Both \citet{Eades2003} and \citet{BrownBlake2017} have documented the challenges associated with second-dialect and basilect-dominant speakers in courtrooms, a situation complicated by the existence of legalese as a specialized lexicon of law and the tendency of lawyers to use “deceptive ambiguity” when interrogating witnesses \citep{Shuy2017}. 

This chapter briefly examines the Jamaican Language situation and underscores the challenges that basilect-dominant and monolingual speakers are likely to face when navigating courtrooms. It discusses the nexus between Language and class a precursor to the legal discussion of what is possible within the 2011 Charter of Rights. While the researchers acknowledge that as worded, linguistic discrimination may conceptually be justiciable under sections 13(3)(g), (h) and (i) of the Charter, they are doubtful of whether this will obtain in courts due to the difficulties in proving indirect discrimination and current judicial approaches to responding to other forms of discrimination which seemingly places a heavy burden on claimants. The researchers, therefore, argue for the officialization of the Jamaican Language among other strategies to preempt recourse to the Charter.}

\IfFileExists{../localcommands.tex}{
   \addbibresource{../localbibliography.bib}
   \usepackage{langsci-optional}
\usepackage{langsci-gb4e}
\usepackage{langsci-lgr}

\usepackage{listings}
\lstset{basicstyle=\ttfamily,tabsize=2,breaklines=true}

%added by author
% \usepackage{tipa}
\usepackage{multirow}
\graphicspath{{figures/}}
\usepackage{langsci-branding}

   
\newcommand{\sent}{\enumsentence}
\newcommand{\sents}{\eenumsentence}
\let\citeasnoun\citet

\renewcommand{\lsCoverTitleFont}[1]{\sffamily\addfontfeatures{Scale=MatchUppercase}\fontsize{44pt}{16mm}\selectfont #1}
  
   %% hyphenation points for line breaks
%% Normally, automatic hyphenation in LaTeX is very good
%% If a word is mis-hyphenated, add it to this file
%%
%% add information to TeX file before \begin{document} with:
%% %% hyphenation points for line breaks
%% Normally, automatic hyphenation in LaTeX is very good
%% If a word is mis-hyphenated, add it to this file
%%
%% add information to TeX file before \begin{document} with:
%% %% hyphenation points for line breaks
%% Normally, automatic hyphenation in LaTeX is very good
%% If a word is mis-hyphenated, add it to this file
%%
%% add information to TeX file before \begin{document} with:
%% \include{localhyphenation}
\hyphenation{
affri-ca-te
affri-ca-tes
an-no-tated
com-ple-ments
com-po-si-tio-na-li-ty
non-com-po-si-tio-na-li-ty
Gon-zá-lez
out-side
Ri-chárd
se-man-tics
STREU-SLE
Tie-de-mann
}
\hyphenation{
affri-ca-te
affri-ca-tes
an-no-tated
com-ple-ments
com-po-si-tio-na-li-ty
non-com-po-si-tio-na-li-ty
Gon-zá-lez
out-side
Ri-chárd
se-man-tics
STREU-SLE
Tie-de-mann
}
\hyphenation{
affri-ca-te
affri-ca-tes
an-no-tated
com-ple-ments
com-po-si-tio-na-li-ty
non-com-po-si-tio-na-li-ty
Gon-zá-lez
out-side
Ri-chárd
se-man-tics
STREU-SLE
Tie-de-mann
}
   \boolfalse{bookcompile}
   \togglepaper[23]%%chapternumber
}{}


\begin{document}
\maketitle


\begin{quote}
    When the voice of authority speaks back to the “masses” using the language of authority, English, much, if not all, of that message is lost. We truly have a dialogue of the deaf. This, however, is societal and self-induced deafness since there is none so deaf as (s)he who will not hear. \citep{Devonish2014}
\end{quote}

Conceptually, access to justice is generally treated as a constituent element of the rule of law with its promise of equal treatment of all by state institutions, particularly the courts \citep{Agrast2008}. \citet{Rhode2004} traces access to justice back to the Magna Carta with its promise not to sell, refuse or delay right or justice in clause 40. \citet{GhaiCottrell2010} discuss access to justice as having two possible meanings, the traditional narrow meaning and the more contemporary broader meaning. The traditional conception is exemplified by UNDP’s (2005) formulation as “the ability of people to seek and obtain a remedy through formal or informal institutions of justice, and in conformity with human rights standards.” This formulation focuses on what happens procedurally within courts and similar institutions, and the individual’s knowledge and experiences navigating those procedures. \citet[5]{Dias2011} critiques this conception as being limited, noting that “the mere fact that a formal judicial system exists and that people have a right to access it, may not translate to individuals having access to justice in reality.”

The more expansive conception of access to justice is defined by \citet[5]{Dias2011} as encompassing “a much broader bundle of issues which may impact or affect the ability of individuals or communities to seek redress for perceived wrongs through legitimate means and these may transcend the formal judicial system.” This more expansive conception therefore concerns itself with the establishment of institutions and procedural rules for granting access to all as well as the substantive laws themselves, and the empowerment of individuals to obtain justice (ibid).

Notwithstanding the strength of the latter, this chapter will be focused on a state-centred narrow conception of access given the nature of its key inquiry, i.e. what happens to Jamaican language dominant/monolingual speakers in Jamaican courts and what response is provided by the 2011 Jamaica Charter of Fundamental Rights and Freedoms. Formulations of access to justice within its broader conception will not be considered given the limitations of this chapter as well as the absence of legal and policy strictures which mandate the use of Standard Jamaican English (SJE) in any of these activities. The chapter will use the traditional understanding of access to justice as its frame – centring around due process rights as protected within international, regional, and national human rights legal instruments.


\section{Access to justice, language and the Jamaican situation}

Within international, regional, and national laws, a clear relationship is established between access to justice and language. \citet{Brown-Blake2006}, in analyzing British common law, identified the nexus between access to justice and language as being rooted in principles of natural justice. As a consequence of natural justice requiring “that a person be given prior notice of the charge against him and an opportunity to meet that charge,” a person who does not understand the case being brought against him, by virtue of a language barrier, cannot mount a proper defence to said charge \citep[393]{Brown-Blake2006}. Put differently by \citet[98]{Ng2009} the integrity of the court process would be compromised if “litigants were unable to communicate with or understand the judge, witnesses or opposing parties or counsel.” This is why the Jamaica Charter of Fundamental Rights and Freedoms requires that when a person is being arrested or charged, it is communicated in a language they understand and that an interpreter be provided by the state in criminal trials where there is a language barrier.

In Jamaica, the language situation has been characterized by \citet{Winford1985} as being a diglossic Creole continuum in which a hierarchy is produced with SJE being considered prestigious and fit for formal occasions, with Jamaican Language being understood as socio-culturally and linguistically inferior, and only fit for informal situations. This is exemplified by the fact that although there is no \textit{de jure} official language in Jamaica \citep{BrownBlake2017}, the Supreme Court of Judicature of Jamaica Criminal Bench Book establishes [Standard Jamaican] English as the language of the court (Judicial Education Institute of Jamaica 2017). The fact that 36.5\% of the population are monolingual Jamaican Language speakers, with only 17.1\% being monolingual SJE speakers and the rest being bilingual \citep{JamaicanLanguageUnit2007} means that this policy of the language of the court being English has implications for whether basilect-dominant and Jamaican Language monolingual speakers are able to adequately navigate the court system. 

It is critical to acknowledge that the specialized lexicon of law (legalese) poses difficulty for the layperson as noted by \citet{Shuy2017}. \citet{SolanTiersma2005} illustrate the challenges for plain language speakers when they engage the criminal justice system. They discuss how the nuances of plain language are not often taken into account by the court, and as such, statements by officers like “May I search your car” are treated as requests, using a literal interpretation, which a civilian may legally refuse, making the subsequent search illegal; rather than seeing them as commands given the power differential and the frequent use of commands veiled as requests in ordinary everyday speech \citep[32--42]{SolanTiersma2005}. They also note the unwillingness of some courts to see expressions such as “I think I need a lawyer” in the context of police interrogations as an invocation of the right to counsel, thereby making every subsequent question illegal \citep[58]{SolanTiersma2005}.

The gulf between legalese and plain (English) language is further complicated by the use of what \citet{Shuy2017} calls “deceptive ambiguity”. \citet{Shuy2017} indicates that within a courtroom setting, lawyers, police and judges have a disproportionate level of power compared to the layperson in using language to shape particular narratives. He notes that “[a]ttorneys can request, warn, threaten, complain, and give directives, but their hearers are limited to reporting their answers to the questions that the powerful speaker asks" \citep[44]{Shuy2017}. With this power in mind, Shuy argues that lawyers and police are often intentionally ambiguous to deceive defendants and witnesses to validate the narrative lawyers construct, which often implicates witnesses and/or defendants in criminal activity \citep[59--60]{Shuy2017}. Courts tacitly accept the use of these strategies that often place laypersons at a disadvantage because of their limited command of legalese, which only increases distrust in the system.

If the challenges are so significant for plain English speakers, this reality is doubly so for Jamaican speakers. This is illustrated in the Australian context by Diana \citegen{Eades2003} analysis of the treatment of Aboriginal English in Australian courts which points out that, as a result of the non-recognition of and prejudicial attitudes towards Aboriginal English, the pragmatic features of their communicative style are often misinterpreted or go unacknowledged—greatly impacting their dealings with the law. \citet{Brown-BlakeChambers2007} illustrate this challenge for Jamaican speakers by looking at the challenges they face in the UK criminal justice system. The most poignant example they use is the transcript of an interrogation of a witness by the police, in which the Jamaican states that after hearing gunshot noises, \textit{mi drap a groun} -- meaning `I fell to the ground'. The written transcript had “I drop the gun” \citep[276--277]{Brown-BlakeChambers2007}. While this mistake was caught before any action was taken, the translation error could have turned the witness to a crime into a perpetrator by way of a purported confession.

Importantly, the approximation of SJE to British English means that the barriers that are being discussed by Brown-Blake and Chambers may very possibly arise in the Jamaican courtroom context. However, this is often mitigated by the bilingual judicial officer or police understanding the language being used by the layperson. The challenge occurs when the former addresses the latter in SJE, particularly using legalese. This is exemplified in a case analysed by \citeauthor{Brown-BlakeChambers2007} where the Jamaican dominant speaker struggles to understand the caution being given by a customs officer before an interpreter is provided \citep[280--285]{Brown-BlakeChambers2007}. 

\begin{sloppypar}
Similar challenges in understanding courtroom communication in Jamaica have been documented by \citet{BrownBlake2017}. In her review, she considers the accused and witnesses being questioned by defence counsel and by the judge, and she notes that where there is difficulty understanding the language being used by the Court, there is a practice of code-switching to Jamaican Language on an ad hoc basis. This should not be taken as her approval of the situation, however, as she remains critical of the lack of formalization of this practice \citep[200]{BrownBlake2017}. The need for code-switching practices is critical because of the diglossic situation. Brown-Blake goes further to point out that a Jamaican language dominant/monolingual accused may not always be able to follow the dialogue between witnesses, attorneys and judges (ibid). This goes to the very heart of the need for interpretive services so that the accused can be considered present for material evidence being given, allowing them to prepare a response.
\end{sloppypar}

In addition to the identified barriers is the deployment of linguistic correctness in courtroom spaces. Linguistic correctness is more of an issue in a language acquisition environment where speakers must work to fulfil the requirements of the official language and is one of the ways in which holders of power in Jamaican society display language discrimination. Correctness is used as a tool of power that allows acrolectal speakers to show dominance and, it can be argued, manipulate information -- as seen in the Commission of Inquiry into Tivoli Incursion in Jamaica. Linguistic correctness occurs when speaker 1 makes an utterance that is deemed grammatically incorrect and speaker 2 responds with the corrected version of said utterance. \citet{Urciuoli2008} talks about the tension between linguistic correctness and cultural identity. In a Jamaican context, utterances made in the Jamaican language often differ from the SJE counterparts in both syntax and semantics. Consequently, the use of linguistic correctness by English language speakers can, arguably, manipulate the meaning and intent of said utterances by Jamaican language speakers. Language discrimination negatively impacts speakers of the Jamaican language as it limits authentic forms of expression and participation in national and critical forms of discourse. Similar to deceptive ambiguity, linguistic correctness can be used to discredit witnesses or otherwise make them look unreliable. This is particularly evident as it relates to written statements that witnesses “make,” which are discussed further below.

However, two other immediate issues arise. The first is the ability of jurors to understand clearly all courtroom proceedings because of their own language limitations. The legalese used in a Judge’s directions would pose a challenge for acrolectal jurors, and even more for those dominant in the basilect and mesolect. Additionally, there is the question of how a juror will regard a witness or accused who is unable to answer questions being asked because of the language barrier that exists in Jamaica’s diglossic context. It is possible that said jurors could either find the witness or accused to be unreliable because of perceived evasiveness or prejudices in favour of or against Jamaican speakers.

The second issue is the practice by court officials, attorneys and police officers of either translating Jamaican Language into English for court transcripts, depositions, witness statements and police reports or “legalesing” (used here to mean converting either English or Jamaican into a specialized lexicon of law) these documents. This practice has been noted by \citet{Eades2003} and \citet{BrownBlake2017}. \citet{Nelson2019}, in his personal account of reporting an incident, described the practice as “worryingly problematic and inhibitive of securing justice”. This also has implications for how a defendant or witness is received by the jury, particularly where the fluency and literacy in English of the witness/defendant is limited if they are denying the use of words they sign to.

Given the complicated situation this potentially poses for Jamaican speakers, this chapter now seeks to explore the potential for the relatively new equality and non-discrimination protections in the Charter of Rights to provide a solution. Before the legal analysis, however, the chapter will consider the relevance of this analysis of the relationship between language and class.


\section{Language use and social class in Jamaica}


One of the most frequently used descriptions of the Jamaican language is that it is a broken language -- one that has no syntactic structure or lexicon. This ideology does not represent a present-day belief, but one that has been intentionally, and carefully curated through interconnected agents of socialization such as the family and the school. The English Language gained official status in Jamaica in 1655 at the onset of British colonialism and was followed by a rigid process by plantation owners to enforce English monolingualism to diminish African dialects and cultural ties. This introduction of English to Jamaican society by the British was the first step in establishing a hierarchical language system among speakers in the country. Jamaica, which is described as a diglossic Creole continuum environment, has speakers who can fulfil the syntactic functions of the English language as acrolectal speakers; those who use language forms which combine English and Jamaican as mesolectal speakers; and those who use language forms farthest from English and closest to Jamaican as basilectal speakers \citep{Irvine-Sobers2018}. This hierarchy creates negative attitudes to and perceptions of the Jamaican language and the association of prestige, wealth and access to more beneficial opportunities with acrolectal speakers and associated poverty and a lack of education with basilectal speakers. Notably, using a Matched Guise experiment, \citet{Rickford_1983_reprint1985} conducted quantitative and qualitative analyses of the language attitudes of working-class field workers on a sugar estate and lower middle-class sedentary workers in Guyana. The results showed that both groups associated the use of Standard English with higher-paid jobs and social mobility (1985: 149-152). This aligns with \citeauthor{Justus1978}’ (\citeyear{Justus1978}) analysis which notes that in the Jamaican context, SJE is a mark of social class, education, economic standing, and urbanization.

The ongoing conversations about the Jamaican language and appropriateness continue, as the use of the language in particular contexts is considered acceptable but challenged in others. August 6, 2022 marks sixty years post-indepen\-den\-ce for Jamaica, and there are still deliberate acts to uphold the prestige associated with the English Language. Consequently, there is an enforced separation of the physical spaces in which SJE on the one hand and Jamaican on the other are considered socially acceptable. Professor Hubert Devonish of the University of the West Indies made an excellent point about the language divide when highlighting the difference between the Jamaican Language and the SJE used in the local media. In an editorial piece, \citet{Devonish2014} noted:

\begin{quote}
    For those who have power, as a result of their education, their wealth or by virtue of having been elected to exercise power, English is the only legitimate language in which to address power. Thus, a year or two ago, in the Senate, Mark Golding was upbraided by the then president of the Senate for using the Jamaican-language phrase, \textit{Rispek dyuu}, on the grounds that THAT language was not allowed in Parliament.
\end{quote}

In the same article, Professor Devonish argued that broadcasters and publishers only used excerpts of people using the Jamaican language for comedy and/or sensationalism of specific issues. This deliberate practice of language divide reinforces the negative colonially-derived attitudes toward the Jamaican language: that it is inferior to SJE and only to be used in less formal settings. Critically, the associations of SJE with a particular social class, though a matter of perception, means that there exists a link in the minds of Jamaicans about what a person’s language suggests about their socio-economic status. The Jamaican Language Unit, in its 2005 survey, noted that 61.7\% of respondents felt that English speakers were more educated and 44.7\% felt that English Speakers were richer. This at least creates the possibility that linguistic discrimination and class discrimination run parallel, given that, to quote a colleague, Tracy Robinson (personal communication) “there is no greater proxy for social class than language”. This nexus between language and class creates an avenue for what the Jamaica Charter of Rights prohibition of discrimination could mean for basilect speakers.


\section{The Charter of Rights and Linguistic Discrimination}

Within the 2011 Charter of Fundamental Rights and Freedoms of Jamaica, there are guarantees of equality and the prohibition of discrimination. Specifically, section 13(3)(g) guarantees to all Jamaicans the right to equality before the law; section 13(3)(h) guarantees the right to equitable and humane treatment by a public authority, and section 13(3)(i) prohibits discrimination on the basis of being male or female, race, place of origin, social class, colour, religion or political opinions. However, these provisions have largely been untested in Jamaican courts and as such the full scope of their protection is yet to be determined.

Notwithstanding this, section 13(3)(g) has been confirmed in case law to be focused on equality in the content of the law and its administration in the Courts. In section 13(3)(h), there is little case law, given that the terms, “equitable and humane treatment” have not been given much judicial attention. “Equal treatment” was decidedly avoided by the Joint Select Committee of Parliament of Jamaica recommending changes to the then Bill of Rights which the Charter of Rights replaced on the basis that equal treatment may not be appropriate, as treatment would differ according to the circumstances. Moreover, the direction of case law on this right is unclear. The dicta of the three judges in \textit{Rural Transit Association Limited v Jamaica Urban Transit Company Ltd and Others} make it clear that “equitable and humane treatment” does not mean equal treatment. Justice Beswick, in assessing whether the claimant’s rights under section 13(3)(h) were breached, found that “the evidence displays unequal treatment of RTA by the police, not inhumane and inequitable treatment.” Justice McDonald offered this distinction between the words equal and equitable -- “I find that the words equitable and inhumane are to be read conjunctively. Guided by the dictionary, I interpreted the word equitable to mean 'fair'/'just'. It does not mean equal.” By contrast, two of the judges on the Full Court in \textit{Sean W Harvey v Board of Management of Moneague College} \textit{and Others} have endorsed the test for the equality of treatment guarantee in the Trinidadian Constitution used in the case of \textit{Bhagwandeen v Attorney General of Trinidad \& Tobago} as the appropriate test for whether section 13(3)(h) has been violated. The potential for section 13(3)(h) to address discrimination at the hands of public actors remains unclear. 

Important to this discussion, sections 13(3)(g) and (h) have no status-based list while section 13(3)(i) has a closed list. Petrova (2013) explains the closed list as “narrowly constru[ing] the right to equality to apply to a limited range of protected grounds, or classes, and respective personal characteristics, while an open list usually explicitly lists grounds of discrimination but, in addition, opens up the list through the expressions ‘such as’, ‘other status’, or ‘any status such as …’ which enable new grounds of discrimination to be prohibited by law,” (2013: 494-495). Neither section 13(3)(g) nor (h) has a list and thus falls within what \citet{Fredman2011} considers the second model of formulating prohibited grounds of discrimination, “a broad open-textured equality protection.” This is the reality in Trinidad where similar provisions have no prescribed list and as such discrimination on a specific status-based ground is not required for the sections to apply. The situation is different for 13(3)(i) which has a closed list. The following sections will analyze the potential for addressing linguistic discrimination under section 13(3)(g) and (h), with a look at the analogous ground approach, as well as under section 13(3)(i) using the concept of indirect/disparate impact discrimination as developed in the United States, Trinidad \& Tobago and Canada.


\section{Linguistic discrimination as an “analogous ground”}

In the United States, where there is no enumerated list, the focus is placed on the level of scrutiny given by the Courts when assessing a claim of discrimination \citep[118]{Fredman2011}. Where the ground of discrimination falls within a ‘suspect category’ (race, alienage and ancestry) then strict scrutiny is applied, meaning there must be a compelling state interest to justify any differentiations made \citep[120]{Fredman2011}. For other grounds, all that is required is that the differentiation is “rationally related to a legitimate state interest” \citep[118]{Fredman2011}. In Trinidad, section 4 of the Constitution guarantees constitutional rights to all regardless of race, origin, colour, religion or sex. The suggestion has been made to treat discrimination on one of these grounds as, a fortiori, violative of the guarantee of equality before the law. If applied in Jamaica, this could mean that the enumerated sub-sections in sections 13(3)(i) could be regarded as already prohibited grounds of discrimination under sections 13(3)(g) and (h), and other grounds would be capable of being added. In Canada, there is a practice of recognizing additional grounds of discrimination that are analogous to those that have been explicitly protected. This has seen the addition of sexual orientation to the provision’s non-exhaustive open list of grounds which prohibits sex discrimination.

On the matter of linguistic discrimination, it is completely open to Jamaican courts to add this as a protected category of discrimination under section 13(3)(g) and (h), whether treating it as an analogous ground to social class given the connections between language and class or using the US approach of not requiring specific grounds. It should be noted that in the Trinidadian case of \textit{Paponette and Others v Attorney General}, differential treatment between separate categories of taxi drivers was found to be captured by the guarantee of equal treatment by a public authority. This, therefore, means that sections 13(3)(g) and (h) do not have to be read as exclusively contemplating immutable statuses, but interpreted as considering any type of unjustifiable differentiation as being within the rubric of those sections. This raises questions about the practical implications of the sections, which are analyzed more closely in the section on the possibilities within the Charter. 


\section{Linguistic discrimination as indirect discrimination}

Across human rights law, discrimination is generally regarded as being capable of being both direct and indirect. Hatzis explains the difference as follows:

\begin{quote}
    In its direct form, discrimination occurs when a person is singled out and targeted for negative or less favourable treatment because he has a particular characteristic. Indirect discrimination, on the other hand, is about neutral, or even benign, measures, [the] effect of which on people having a particular characteristic is more burdensome than their effect on people who don't have it. The focus of indirect discrimination law is on the disparate impact a policy may have on certain people in comparison to the impact on others. (2011: 287)
\end{quote}

In the repealed 1962 Bill of Rights of Jamaica, section 24(1) prohibited a law which was “discriminatory either of itself or in its effect”. The terminology “in its effect” suggests that indirect discrimination, which looks at the impact of laws that are neutral on their face, was being contemplated and prohibited. In the new dispensation of the 2011 Charter of Rights, freedom from discrimination is broadly guaranteed without any references to whether the discrimination needs to be direct or indirect. Given the approach of the constitutional of using a generous interpretation to human rights protections in the Charter, it is likely that the Charter will be interpreted as prohibiting indirect discrimination. Indirect discrimination provisions consider how persons in a particular group are disproportionately impacted by neutrally framed laws. A useful example is the Belizean case of \textit{Wade v Roches} in which a rule ostensibly prohibiting staff of a Catholic school from fornicating was considered to be discriminatory on the basis of sex since unmarried women who show proof of fornication via pregnancy would be disproportionately affected while unmarried men would not. 

Section 13(3)(i) only prohibits discrimination on seven grounds, one of which is social class (which is a new addition to the Charter). There have been no Jamaican cases claiming discrimination on the basis of social class. The question of how one proves membership in a social class remains to be determined by the Court. However, as discussed in earlier sections of this chapter, there is a socially understood link between perceptions of social class and language. The use of Standard Jamaican English or Jamaican Language serves to construct an identity in the public domain (Campbell. 2007). As such, the perceptions of the usage of SJE as being emblematic of one’s wealth, education and background -- ostensibly one’s class -- means social class discrimination is very likely in Jamaica’s diglossic context to involve value judgments and perceptions wholly or partly based on the individual’s use of language. Additionally, differential treatment on the basis of language used may, at least some of the time, disproportionately affect those who are more likely to use a particular language. 

Research into perceptions of bias in Trinidadian courts already reveals a correlation between wealth and language in terms of how people perceive they are being treated \citep{Kerrigan-etal-2019}. In this quantitative study involving 160 members of the public, 110 judicial staff, 22 judicial officers (judges/magistrates etc.) and 68 attorneys, 66.6\% of the public and 60.3\% of attorneys indicated a belief that the use of English will lead to better treatment in Court. Additionally, 87.3\% of the public, 79.4\% of attorneys and 65.1\% of judicial staff indicated a belief that a person who appears wealthy will receive better treatment (ibid). The nexus between wealth, class and language must be borne in mind when reading these statistics as language helps to shape perceptions of wealth. 

It is important to note that in the American context, structural linguistic discrimination has already been found to be a form of indirect racial discrimination. In the case of \textit{Martin Luther King Junior Elementary School Children v Ann Arbor School District Board}, a class action suit was successfully brought against the school district board for failing to provide professional development opportunities that would help teachers understand the challenges faced by Black students who spoke “Black English” (or African American Vernacular English) as their first language and equip teachers with the language and skills to remove barriers in their teaching practices. The consequence was that Black students in the district experienced significant difficulty in developing literacy skills in Standard English. The District Court recognized the linkage between the failure to see and treat Black English as a legitimate dialect and racial inequality. The aforementioned case, at the very least, establishes a precedent that a Jamaican court can follow in coming to a determination that a rule requiring communication in English caters only to SJE-dominant and bilingual speakers and excludes, by implication, basilect-dominant and monolingual Jamaican Language speakers, the latter being 36.5\% of the population. The question then becomes whether this constitutional possibility would bear actual fruit. 


\section{Productive possibilities in the charter}
 
Starting first with the case of indirect discrimination, \citet{Mercat-BurnsElaine2016} has noted that the nature of redress for indirect discrimination is redistributive, i.e. it is centred around eliminating institutional mechanisms, rather than simply awarding damages to victims. In the case of access to justice for Jamaican  speakers, this means taking positive steps to remove language barriers that exist in the courtroom. The possibilities include Jamaican interpreters for both civil and criminal trials, recognizing Jamaican Language as the second (or radically, the first) language of the court or the establishment of Jamaican-only courtrooms. 

The feasibility of these forms of redress is discussed below; however, the first barrier is proving indirect discrimination. In the United States, as well as in the United Kingdom and other jurisdictions, there is heavy reliance on statistical evidence to prove the “disparate impact” required to show that neutrally framed laws are indirectly discriminatory \citep{BarnardHepple1999}. Against the background of the language challenges discussed in this chapter, proving disparate impact would involve demonstrating that Jamaican speakers who access courts are disadvantaged at a disproportionate rate compared to SJE-dominant and bilingual speakers. Proving the disadvantage becomes even more complicated when we consider that perceptions of acquiring justice may differ from the procedural guarantees of justice constitutionally enshrined. Put differently, the general inaccessibility of legalese means that Jamaican speakers are not able, without assistance from an attorney, to fully appreciate whether the procedures which are themselves written in SJE are being followed, and so their perceptions of justice are inevitably affected by the fact that rules of justice are not spoken or written in a language they generally understand. There is also the question of whether courts will be willing to accept the linkages between language use and class, given the ubiquitous nature of Jamaican Language in Jamaica. Though conceptually possible, the practicality of bringing a claim of linguistic discrimination as indirect class discrimination is complicated. Moreover, nothing has yet been said of the dearth of local legal authority on how one establishes one’s social class, and whether the claimant, though a Jamaican speaker, would have to be regarded as a member of the poorer classes—those regularly associated with the dominant or exclusive use of the Jamaican Language—in order to benefit from this prohibition of indirect class discrimination.

Even with the possibility of linguistic discrimination being considered justiciable under the broadly framed sections 13(3)(g) and (h), there are other significant challenges to its adjudication. The first question is whether courts are prepared to treat the Jamaican language as a discrete language capable of being discriminated against. \citet{Eades2003} notes this very problem as existing for Indigenous communities in Australia who are second dialect speakers. In her analysis, the right to an interpreter is not enforced in favour of second dialect speakers because of bias against the language as well as non-recognition of same. Critically, the Jamaican language has still not been legally recognized by the government as a language despite growing public support \citep{JamaicanLanguageUnit2005}.

Even if the courts are minded to consider Jamaican as a language capable of being discriminated against, there is the second question of whether linguistic discrimination will be recognised given the refusal of the Parliament to include language as a basis of discrimination, despite substantial submissions made requesting same \citep{JointSelectCommittee_of_Parliament2002}. As noted by \citet{Wheatle2012}, Jamaican courts do not generally defer to what the framers intended but rather what is meant by their express words. In this case, however, there are no status-based grounds to guide the court, and while the Trinidadian precedent requires no status, the possibility exists that the court could consider the section as not including linguistic discrimination. 

An additional barrier that must be considered is the approach taken by the courts in Sean Harvey where the court placed a high burden on the claimant under section 13(3)(h) to show that the body breaching the right was a public authority acting in the exercise of its function. The court not only required the claimant to go beyond establishing that the body was established by law but also required that the claimant go further in establishing its status as a public authority by considering the nature of its functions, its receipt of government funding and other matters. While this may not be a matter for the courts, which are generally covered under section 13(3)(g) when they are administering law, this has broad implications for linguistic discrimination at the hands of schools, police, hospitals and other ministries, departments and agencies of government.

Taken together, whether there is a claim of linguistic discrimination as indirect class discrimination or linguistic discrimination under broad equality provisions, there will be an uphill task in establishing the justiciability and applicability of the right, particularly in the context of access to justice. But even if the prohibition of linguistic discrimination is recognised and the non-recognition of Jamaican is deemed as a breach of that prohibition, there is still the question of whether that non-recognition will be considered demonstrably justified in a free and democratic society per section 13(2) of the Charter. 

This section has been interpreted as necessitating the test of rationality and proportionality laid out in the case of R v Oakes. Put simply, there must be a good reason for limiting the right, the legislative measure taken must be rationally connected to that reason and the measure must limit the right no more than is necessary. Within the balancing exercise that takes place here, external considerations such as the financial cost of securing the right come into play. Since the established orthography of Jamaican has not been widely disseminated or taught in schools, there would be considerable costs in complying with a prohibition of linguistic discrimination. This involves the potential cost of recording Jamaican Language interpretations of legislation for public dissemination to increase the accessibility of legislation or otherwise formally establishing “Jamaican Language only” courtrooms with stenographers trained in properly translating spoken Jamaican Language into written English. The related costs may result in the prohibition of linguistic discrimination to be held as not including a right to access law in the Jamaican language. The unwillingness of courts to supervise costly initiatives has led to Brown-Blake’s skepticism of the prohibition of linguistic discrimination being able to address the challenges highlighted \citep{BrownBlake2008}.


\section{Conclusion: So what then?}

Instead of resorting to constitutional redress, the intervention of a judge in the case of Jamaican speakers becomes the singular most important act in ensuring a trial is at least procedurally fair. However, as noted by \citet{BrownBlake2017}, this is not mandated and only happens on an ad hoc basis. The \citeauthor{JamaicanJusticeSystemReformTaskForce2007} (JJSRTF) has acknowledged that there are language-based challenges within the justice system (\citeyear{JamaicanJusticeSystemReformTaskForce2007}). The recommendations have largely been to \textit{de-legalese} or to simplify English which does nothing to address the needs of the aforementioned speakers. Furthermore, the likelihood of a courtroom interpreter being employed is low because of the high levels of bilingualism in Jamaica and the prohibitive costs of doing the same. Even if they were to be used, both \citet{Eades2003} and \citet{Ng2009} have noted several challenges with the use of interpreters which will not be explored here.

Even with the potential prohibition of linguistic discrimination, whether using existing provisions, or through the process of constitutional reform to include this explicitly, there are limitations. The justiciability of linguistic discrimination within the existing provisions reveals a complicated reality requiring (currently unavailable) statistical evidence in the case of indirect discrimination and general hurdles to establishing that the Jamaican Language amounts to a language that can be discriminated against or that the provisions should even extend to including linguistic discrimination. \citet[53-54]{BrownBlake2008} has noted the unwillingness of courts to award remedies that would involve monitoring state action. In Jamaica’s case, the court would be monitoring itself by either ensuring the presence of interpreters or the existence of dedicated Jamaican language courts with trained officials. Both measures require institutional change and/or an influx of resources, which the court may object that providing any such remedy would be too burdensome and therefore not doing so would be “demonstrably justified” within the meaning of section 13(2).

One way to address the hurdle of Jamaican language not being seen as a language is through officialization. But officialization by itself is not enough, as noted by \citet[65]{BrownBlake2014}. She indicates that officialization in the Seychelles and Vanuatu of non-dominant languages did not lead to greater use in court proceedings as there is a tendency to legislate the primacy of the use of dominant languages in that arena \citep[148]{BrownBlake2014}. It stands to reason, therefore, that to produce the outcome that is best suited to Jamaican speakers, there would have to be a combination of officialization and political will to mobilize the necessary resources that would preempt the need for a constitutional claim. Anti-discrimination litigation is notoriously challenging and as such, a proactive rather than a remedial response from the State would be more impactful.


% -- References:
{\sloppy\printbibliography[heading=subbibliography,notkeyword=this]}


\section*{Conventions, declarations, treaties, statutes}
  Government of Jamaica. 1962. \emph{Constitution of Jamaica} \\
  Government of Jamaica. 2011.  \emph{Jamaica Charter of Fundamental Rights and Freedoms} \\
  Government of Trinidad \& Tobago. 1976.  \emph{Constitution of the Republic of Trinidad \& Tobago}  \\
  Organisation of American States. 1969.  \emph{American Convention on Human Rights} \\
  Organisation of American States. 1948. \emph{American Declaration of the Rights and Duties of Man} \\
  United Nations. 1966.  \emph{International Covenant on Civil and Political Rights}
  

\section*{Cases \& Communications}
   \emph{Attorney General v Orozco and Others}. 2019.  \\
   \emph{Bhagwandeen v Attorney General of Trinidad and Tobago}. 2004. \\
   \emph{Egan v Canada}. 1995. \\
   \emph{Griggs v Duke Power Company}. 1971. \\
   \emph{Martin Luther King Junior Elementary School Children v Ann Arbor School District Board}. 1979.  \\
   \emph{Maya Leaders Alliance and others v Attorney General of Belize}. 2015. \\
   \emph{Minister of Home Affairs v Fisher}. 1979. \\ 
   \emph{Paponette and others v Attorney General}. 2010. \\
   \emph{R v Oakes}. 1986. \\
   \emph{Rural Transit Association Limited v Jamaica Urban Transit Company Ltd and Others}. 2016. \\
   \emph{Sean W Harvey v Board of Management of Moneague College and Others}. 2018. \\
   \emph{Smith and AG v LJ Williams}. 1980. \\
   \emph{Toonen v Australia}. 1994. \\
   \emph{Virgo and ZV v Board of Management of Kensington Primary School and Others}. 2020. \\
   \emph{Wade v Roches}. 2005


\end{document}
