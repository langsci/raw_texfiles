\documentclass[output=paper,colorlinks,citecolor=brown]{langscibook}
\ChapterDOI{10.5281/zenodo.10103078}

\author{Celia Brown-Blake\affiliation{The University of the West Indies, Mona}}
\title[Giving expert evidence in Caribbean English vernacular languages]      {Giving expert evidence in connection with Caribbean English vernacular languages: Lessons from \emph{US v Kwame Richardson}}
\abstract{This chapter discusses the legal rules governing expert evidence and how they may interact with the provision of linguistic evidence, particularly relating to speakers of Caribbean English vernacular languages, sometimes called Caribbean English creole languages.  The case of \textit{United States v Kwame Richardson}, in which the defence had initially planned to rely on expert linguistic evidence concerning a speaker of Guyanese, is deployed as a launch pad to the discussion.  Although the expert’s report was not ultimately relied upon by the defence in the court proceedings, the discussion indicates the legal roadblocks that may defeat the use of potential testimony by a language expert.  The article stresses that it is important for linguists offering their expertise in forensic contexts to be acutely aware of the legal rules in order to meet, as far as possible, likely challenges to their methods and expert report or testimony.  As the article shows, these challenges may include, in some instances, (mis)conceptions on the part of legal professionals about language in general, and the nature of Caribbean English vernaculars in particular.}

\IfFileExists{../localcommands.tex}{
   \addbibresource{../localbibliography.bib}
   % add all extra packages you need to load to this file

\usepackage{tabularx,multicol}
\usepackage{url}
\urlstyle{same}

\usepackage{listings}
\lstset{basicstyle=\ttfamily,tabsize=2,breaklines=true}

\usepackage{langsci-basic}
\usepackage{langsci-optional}
\usepackage{langsci-lgr}
\usepackage{langsci-osl}
% \usepackage{./langsci/styles/langsci-lgr}
% \usepackage{./langsci/styles/langsci-osl}
% \usepackage{langsci-gb4e}

\usepackage{tikz}
\usetikzlibrary{patterns,calc}
\pgfdeclarepatternformonly{south east lines}{\pgfqpoint{-0pt}{-0pt}}{\pgfqpoint{3pt}{3pt}}{\pgfqpoint{3pt}{3pt}}{
    \pgfsetlinewidth{0.6pt}
    \pgfpathmoveto{\pgfqpoint{0pt}{3pt}}
    \pgfpathlineto{\pgfqpoint{3pt}{0pt}}
    \pgfpathmoveto{\pgfqpoint{.2pt}{-.2pt}}
    \pgfpathlineto{\pgfqpoint{-.2pt}{.2pt}}
    \pgfpathmoveto{\pgfqpoint{3.2pt}{2.8pt}}
    \pgfpathlineto{\pgfqpoint{2.8pt}{3.2pt}}
    \pgfusepath{stroke}}
    
\usepackage{stmaryrd}
\usepackage{wasysym}
\usepackage{multirow}
\usepackage{caption}
\usepackage{subcaption}
\usepackage{mathrsfs}
\usepackage{qtree}

\usepackage{linguex}


   %pminos do not split footnotes
% \interfootnotelinepenalty=10000 %Footnote in Laporte chapters has to be split SN


%\DeclareIndexNameFormat{default}{%
%\nameparts{#1}%
%\usebibmacro{index:name}%
%{\index[names]}%
%{\namepartfamily}%
%{\namepartgiveni}%
% {}% L1
% {}% L2
%{\namepartprefix}% generates spurious space L3
%{\namepartsuffix}% generates spurious space L4
%}

%  {\DeclareIndexNameFormat{default}{%
%     \usebibmacro{index:name}{\index[names]}{#1}{#3}{#5}{#7}}}

%\DeclareIndexNameFormat{default}{%
%  \usebibmacro{index:name}{\sindex[nom]}{#1}{#3}{#5}{#7}}

%\DeclareIndexNameFormat{default}{%
%  \usebibmacro{index:name}{\sindex[person]}{#1}{#3}{#5}{#7}}
%\DeclareIndexNameFormat{default}{%
%\nameparts{#1} \usebibmacro{index:name}{\sindex[person]]}{\namepartfamily}{‌​\namepartgiven}{\nam‌​epartprefix}{\namepa‌​rtsuffix}}

%\newcommand{\smiley}{:)}

%\renewbibmacro*{index:name}[5]{%
%\usebibmacro{index:entry}{#1}%
%{\iffieldundef{usera}{}{\thefield{usera}\actualoperator}\mkbibindexname{#2}{#3}{#4}{#5}}}

% \newcommand{\noop}[1]{}

%remove for final
%\overfullrule=1mm

\newcommand{\tobi}[2]}}
\renewcommand{\S}[1]{\tobi{#1}{\textsc{*}}}

% this volume references
% puts: [this volume]
% already defined: \citetv
%\newcommand{\citepv}[1]{(\citeauthor{#1} \citeyear*{#1} [this volume])}
\newcommand{\citealtv}[1]{\citeauthor{#1} \citeyear*{#1} [this volume]}

%parentheses around example number
\newcommand{\pref}[1]{(\ref{#1})}

% in-text examples

\newcommand{\lnex}[1]{\textit{#1}} %target lang word
\newcommand{\lnlit}[1]{(lit.: `#1')} %literal reading
\newcommand{\lnlat}[1]{(#1)} % latinization
\newcommand{\lntrans}[1]{`#1'} %translation
\newcommand{\lnexl}[2]%
{\lnex{#1}{} \lnlat{#2}} % ex with latinization
\newcommand{\lnexlat}[3]{\lnex{#1}{} \lnlat{#2}{} \lntrans{#3}} % ex with latinization and tranl.

%ch01
\newcommand{\co}[1]{\mbox{\textbf{#1}}}

%ch09

\newcommand{\cyrbulg}[1]{\begin{otherlanguage*}{bulgarian}#1\end{otherlanguage*}}


%ch10
\newcommand{\nlp}{{\small NLP}}
\newcommand{\mwe}{{\small MWE}}
\newcommand{\rae}{{\small RAE}}
\newcommand{\lvc}{{\small LVC}}
\newcommand{\pos}{{\small P}o{\small S}}
%\newcommand{\todo}[1]{ \textcolor{red}{#1} }

%\renewcommand{\labelenumi}{\theenumi}
%\ainamefmt{{vv}{ll}{, ff}{, jj}} % fullname

\newcommand{\biberror}[1]{{\color{red}#1}}

\newcommand{\osenovaitem}{--~}
   %% hyphenation points for line breaks
%% Normally, automatic hyphenation in LaTeX is very good
%% If a word is mis-hyphenated, add it to this file
%%
%% add information to TeX file before \begin{document} with:
%% %% hyphenation points for line breaks
%% Normally, automatic hyphenation in LaTeX is very good
%% If a word is mis-hyphenated, add it to this file
%%
%% add information to TeX file before \begin{document} with:
%% %% hyphenation points for line breaks
%% Normally, automatic hyphenation in LaTeX is very good
%% If a word is mis-hyphenated, add it to this file
%%
%% add information to TeX file before \begin{document} with:
%% \include{localhyphenation}
\hyphenation{
    Beck-man
    Ngu-yen
    back-chan-nel
    back-chan-nels
    mo-not-o-nous
    ste-reo-typ-i-cal
}

\hyphenation{
    Beck-man
    Ngu-yen
    back-chan-nel
    back-chan-nels
    mo-not-o-nous
    ste-reo-typ-i-cal
}

\hyphenation{
    Beck-man
    Ngu-yen
    back-chan-nel
    back-chan-nels
    mo-not-o-nous
    ste-reo-typ-i-cal
}

   \boolfalse{bookcompile}
   \togglepaper[23]%%chapternumber
}{}


\begin{document}
\maketitle

\section{Introduction}

The US criminal case of \emph{Kwame Richardson}\footnote{\emph{United States v Kwame Richardson}, No. 09-CR-874 (JFB), 2010 WL 5437206 (EDNY Dec. 23, 2010).} raises issues surrounding the provision of expert evidence in the context of Caribbean English vernacular languages. In this case, an expert in Caribbean linguistics, Hubert Devonish, was asked by the defendant’s attorneys, the office of the Federal Defenders of New York,\footnote{The Federal Defenders of New York is an organisation in the USA devoted to representing persons accused of federal crimes who cannot personally meet the expenses associated with hiring a lawyer.} to provide an opinion that bore on the defendant’s language competence. The defendant, Kwame Richardson, who had been charged with drug-related offences, was a speaker of Guyanese Creole, also called Creolese (hereafter referred to as Guyanese). It appears that doubts were raised in the minds of his attorneys as to whether Richardson had understood the \emph{Miranda} warning which had been told to him in English by a police officer, without the assistance of an interpreter, prior to interrogating him.

The \emph{Miranda} warning is essentially the US version of the police caution.\footnote{In jurisdictions of the Anglophone Caribbean, police officers are obliged to administer the caution to suspects and arrestees whom they intend to question or who wish to give a statement to the police. The caution informs a suspect or an arrestee of his or her right to silence and warns them that anything they say may be recorded in writing and used as evidence in a trial against them for the contemplated offence. Failure on the part of police officers to caution a suspect or arrestee when required has implications for the admissibility of the statements made by the suspect or arrestee as evidence at trial.  In addition to the right to silence and a warning that what a suspect says may be used against them in court, Miranda warnings include the right on the part of the suspect to confer with an attorney and to have an attorney present while he or she is being questioned by the police.} In keeping with the US Supreme Court decision in \emph{Miranda v Arizona,}\footnote{384 US 436 (1966).} police officers who are about to question a suspect in their custody are required by law to inform the suspect of certain rights. This bundle of rights comprises the \emph{Miranda} warning or \emph{Miranda} rights, designed as a safety measure against self-in\-crim\-i\-na\-tion. Although it is open to a suspect to waive his \textit{Miranda} rights, US law also provides that such a waiver must be done knowingly, voluntarily, and intelligently. This means that a suspect must comprehend and appreciate the nature of the rights in order to validly waive the rights.  Where it is established that a suspect did not understand and appreciate the \emph{Miranda} warning administered to him or her by the police officer, then whatever the suspect said upon questioning by the police, including confessions or other implicating statements, cannot be admitted in evidence at trial. Defence lawyers seeking to exclude such statements from a trial will file a motion to suppress on the basis that their client had not understood the \emph{Miranda} warning, and consequently could not have properly waived the rights.

Such a motion to suppress the post-arrest statements made by Richardson was filed by his attorneys. The court was asked to exclude those statements from his trial because the defendant, a speaker of Guyanese, did not understand the \emph{Miranda} rights when the police administered them in English without the intervention of a Guyanese language interpreter. The initial intention on the part of his attorneys was to support the motion by adducing expert opinion evidence from Devonish, then Professor of Linguistics in the Department of Language, Linguistics and Philosophy at The University of the West Indies, Mona Campus in Kingston, Jamaica. 

Devonish was asked to tender an opinion as to whether the defendant, Richardson, could have understood the \emph{Miranda} warning as told to him by the police officer. Although Devonish did prepare an expert witness report with a view to appearing at the hearing of the motion, ultimately he did not appear, and the defence did not use the report in support of the motion, which was heard without reliance on expert linguistic evidence. Despite this, Devonish’s production of an expert witness report, the reasons advanced by the prosecutors for opposing the motion to suppress, and the judge’s grounds for refusing the motion are arguably instructive for linguists. This may be particularly so for linguists engaged in providing expert opinions for English-medium judicial systems on comprehension by speakers whose dominant language is a Caribbean English vernacular. The fact that expert linguistic evidence was not ultimately used in Kwame Richardson's case, however, is perhaps a missed opportunity for the clarification of some key issues concerning Caribbean English vernaculars in a judicial context.

Against the background of the \emph{Richardson} case, this chapter discusses the legal rules governing expert evidence and how they may affect the admissibility of, or the weight ascribed to, expert linguistic opinion of the kind submitted by Devonish. Before embarking on this discussion, I briefly examine the sociolinguistic context surrounding speakers of Caribbean English vernaculars in overseas justice systems in which English is officially used.  


\section{Caribbean English vernacular speakers in “farin” justice systems}

Caribbean people have a tradition of journeying to “farin” -- places beyond their home country shores, particularly the UK and North America, where relatively large Caribbean diasporic communities have developed. Many of these people originate from jurisdictions in the Anglophone Caribbean -- territories in which English is the official language and where, invariably, English-lexicon vernacular languages, sometimes referred to as creole languages, are also widely used. Many migrants from the Anglophone Caribbean are vernacular-dominant speakers with restricted competence in Standard English.  Such speakers are therefore obliged to use their native vernaculars in their interaction with state institutions in the host country. \citet{Blackwell1996} work is a recorded example of the use of Jamaican vernacular forms by an accused person in his statement to the police in a case arising in the UK.

Communication difficulties are likely to arise when Caribbean English vernacular speakers interact with host country institutions that officially operate in Standard English. These difficulties occur because, while the vernacular languages and their superstrate are phenotypically alike, i.e., their lexica are related, they differ considerably in their underlying grammatical structures. Although some of these communication difficulties have been documented -- \citet{Brown-BlakeChambers2007} -- the potential degree of the problem seems to be disguised, partly perhaps because of the shared lexicon.\footnote{It should be noted that though English-lexicon Creoles may share lexemes with their superstrate, the meaning of a particular lexeme in a creole language may vary or differ from the meaning ascribed to the cognate lexeme in English. Such meaning differences may, and indeed have, raised communication questions in court proceedings. By way of examples, see the Canadian case of \textit{R v Douglas} 2014 ONSC 2573, para 34, regarding Jamaican, and the discussion in \citet[118]{Eades1994} regarding Torres Strait Creole.} Another probable factor in the communication difficulty is that these vernaculars are accorded little or no official recognition as languages in their respective home territories. The upshot of this, where vernacular-dominant speakers interface with “farin” criminal justice systems operating in English, is that there is an assumption on the part of these justice systems that they speak and understand English. Accordingly, they are usually not provided with the legal safeguard in such situations -- an interpreter which is afforded to non-English speaking suspects and defendants. Anecdotal evidence in the form a letter to the editor of a long-established daily newspaper \citep{Martin2002} circulating in Jamaica lends support to the existence of language-related misunderstandings involving speakers of Jamaican in the US justice system and the possible attendant legal danger which may arise.

\largerpage[-1]
It is likely that, in the \emph{Richardson} case, questions about the English proficiency of the accused arose as his defence team began taking instructions and themselves encountered communication problems. This would have led them to deduce that it was probable that he would not have understood the \emph{Miranda} warning, itself a text, as studies have shown, that is likely to present comprehension difficulties for both native and non-native speakers of English \citep{Rogers2007, Rogers2008, Roy1990, Pavlenko2008, Pavlenko2019}.\footnote{Research shows that police cautions used in other jurisdictions are also likely to pose language comprehension difficulties for both native and non-native speakers. See, for example, \citet{InnesErlam2018} regarding New Zealand; \citet{ChaulkEastwoodSnook2014} regarding Canada; \citet{Fenner2002}, \citet{Cotterill2000} and \citet[especially Ch.11]{Rock2007} regarding the UK; \citet{CookePhilip1998} regarding Scotland specifically; \citet[Ch 5]{Heydon2019} regarding Australia.} Despite a body of literature on the degree of comprehension of the \emph{Miranda} warning and other jurisdictional versions of the police caution by native and non-native speakers of English, there seems to be little work carried out in the context of Caribbean English vernacular speakers. Communication and comprehension problems involving these speakers may not be very apparent for a number of reasons, some of which have already been alluded to here. It is clear, though, that Richardson’s attorneys, at least initially, believed that it would be useful to support the motion to suppress with expert evidence as to the linguistic abilities of their client, as well as expert opinion on the likelihood of him understanding, and thus validly waiving, his \emph{Miranda} rights. The degree of evidential value that a court may attach to such an expert opinion is governed by rules that prescribe certain conditions that should be satisfied by the specialist and the opinion he or she provides.


\section{Law on expert evidence and its potential interplay with linguistic evidence}
\largerpage
Increasingly, linguists are being called upon to apply their expertise in criminal cases and legal disputes (\cite{Levi1994_2013}, \emph{State of Western Australia v Gibson}, 2014\footnote{\textit{State of Western Australia v Gibson} [2014] WASC 240 in which linguistic evidence provided by Eades is reported in the judgment delivered by Hall J. See paras. 117--124.}; \cite{Coulthard1997-2013}; \cite{TiersmaSolan2002}; \cite{Shuy1993,Shuy2005}; \cite{vanNaerssen2009}; \cite{EggingtonCox2013}). As the literature indicates, expert linguistic evidence has been provided concerning a range of issues, such as authorship analysis, including speaker identification, the degree of similarity between competing trademarks, analysis of conversation to assess criminal intent/knowledge, interpretation, and meaning of texts, including legal texts, comprehensibility of texts, language proficiency and national origin questions.   

Expert evidence includes, but is not limited to, opinions inferred from data by someone with specialist knowledge and experience.\footnote{The need for these opinions to be given by someone with specialist knowledge and experience distinguishes it from lay opinions, which are permissible in certain circumstances. (In relation to US federal cases, these circumstances are specified in the Federal Rules of Evidence, Rule 701.)} In law, opinion evidence is exceptional in the sense that, generally, evidence in court proceedings must be confined to facts, not opinions; and only those facts of which a witness has personal knowledge, i.e. facts personally observed or perceived by him or her. Given the exceptional nature of expert opinion evidence, there are a number of rules governing its use in judicial proceedings.  In the US, the Federal Rules of Evidence (FRE) constitute the statutory basis governing the provision of expert evidence in federal cases such as the \emph{Richardson} case under discussion.

\subsection{The exclusionary rule of common knowledge: Not expert evidence if merely common sense}
\largerpage[]
A fundamental principle is that the expert evidence must assist the trier of fact (the jury; or the court in judge-alone trials) to ascertain the facts in issue.  FRE Rule 702(a) provides that an expert may give opinion evidence only if “the expert’s scientific, technical, or other specialized knowledge will help the trier of fact to understand the evidence or to determine a fact in issue”.  This has been interpreted to mean that an expert’s evidence may not be directed to “lay matters which a jury is capable of understanding and deciding without the expert's help.”\footnote{\emph{Andrews v. Metro North Commuter Railroad Co}. 882 F.2d 705, 708 (1989).}  Thus, the proffered expert opinion must be beyond the common sense capacity of a lay person to be capable of admission into evidence. This rule has its counterpart in other jurisdictions, such as those in the Anglophone Caribbean, for example, Jamaica, where the law in the UK, particularly the English common law, has been influential. The law applicable in Jamaica is that the evidence must be necessary, in the sense that it must provide information beyond the scope of the “ordinary human experience,”\footnote{\emph{R v Turner} [1975] Q B 834, 841--842. In this case, expert psychiatric opinion regarding how the average person would likely react upon discovering spousal infidelity was ruled to be inadmissible to help establish that the defendant was likely to have been provoked in such circumstances.} i.e beyond the common knowledge and experience of the trier of fact, and “be such that a judge or jury without instruction or advice in the particular area of knowledge or experience would be \textit{unable} to reach a sound conclusion without the help of a witness who had such specialised knowledge or experience”\footnote{\emph{Wilson and Murray v Her Majesty’s Advocate} [2009] HCJAC 58, para. 58} (emphasis added). Expert evidence therefore becomes unnecessary if the question to be determined is within the knowledge and experience of the tribunal of fact.\footnote{See the judgment of the Court of Appeal of Jamaica in \emph{Bernal and Moore v R} (1996) 50 WIR 296, 361--364 which adopts the position in Canada and New Zealand.  This judgment by the Court of Appeal was appealed to Jamaica’s final appellate court, the Judicial Committee of the Privy Council (JCPC), which affirmed the Court of Appeal’s statements of law on expert evidence. See judgment of the JCPC, \emph{Bernal and Moore v R} (1997) 51 WIR 241, 252--253.} The principle is perhaps related to the notion that an expert should not encroach on the domain of the trier of fact.\footnote{H v R (2014) EWCA Crim 1555, para. 42, and \emph{Bernal and Moore v R} (1997) 51 WIR 241, 253.} Where, then, a trier of fact, applying common intelligence and understanding, is competent on their own to figure out the issue, expert evidence will be precluded. It is the duty of the judge to decide whether the expert evidence being offered should be excluded. Judges thus perform this function of gatekeeping in respect of expert evidence.

This exclusionary rule assumes some importance in the context of the case of \emph{Kwame Richardson}. Devonish’s expert report addressed the question of the defendant’s proficiency in English. The report opined that the defendant exhibited a “limited understanding of English”. Documents\footnote{The Government’s Memorandum of Law in Opposition to the Defendant’s Motion to Suppress, 9--10, \emph{United States v Kwame Richardson}, No. 1:09-cr-00874 (JFB), Document 51, Filed 09/30/10.} filed in court indicate that, at the hearing of the Motion to Suppress, the prosecution intended to rely on the argument that the defendant’s ability to speak English was not an issue for expert testimony. This argument was grounded in the exclusionary rule -- that the extent to which the defendant understands English was “an issue that the Court is capable of resolving without an expert’s help”\footnote{\emph{Ibid}.}. This issue, the prosecutors proposed, could be determined by the court itself on the basis of the testimonies of the agents who had interrogated the defendant, and on the basis of evidence from the defendant’s family, friends, and acquaintances as well as the defendant’s own evidence, should he elect to give it, as to his ability to comprehend English. The suggestion is that the court, as the trier of fact at the hearing of the Motion to Suppress, was capable of determining the issue without expert help,\footnote{It is worth noting that \citet[133]{Jensen1995}, in discussing an Australian case in which expert linguistic evidence was presented for a non-native speaker of English, reports that the prosecutor, in objecting to the evidence, submitted that the question as to the defendant’s English proficiency was “a matter within anybody’s capacity”.} by considering the nature of the agents’ testimonies about the defendant’s conduct during interrogation, and, possibly, evidence from the defendant as to his linguistic capabilities.  

It is worth noting that the prosecution also intended to rely on this common knowledge exclusionary rule at the trial of the offence charged\footnote{See Letter from US Attorney for EDNY, 6--9, \emph{US v Kwame Richardson}, No. 1:09-cr-00874 (JFB), Document 66, Filed 02/14/11 (the government’s letter in support of its motion \emph{in limine} to exclude the testimony of the defendant’s proposed expert in which the prosecution outlines arguments against the admission of the expert’s testimony \emph{at trial}).} at which the defence had also initially intended to call expert linguistic testimony. Such testimony would have been given in connection with the issue of whether the defendant voluntarily gave the confession statements\footnote{The law requires that confessions by suspects and accused persons be made voluntarily.  In US federal law, this rule is governed by 18 US Code § 3501.  The principle of voluntary confessions is equally applicable in other common law jurisdictions such as those in the Anglophone Caribbean, for example, Jamaica. See \emph{Peart v R }[2006] UKPC 5 (on appeal from Jamaica).} he made to law enforcement officers. This issue differs in law from the question of the waiver of one’s \emph{Miranda} rights, which is heard pre-trial. If, during trial, the confession statements are ruled by the judge in a \emph{voir dire}\footnote{This is a trial within a trial and is conducted in the absence of the jury.} as having been voluntarily made, and thus admissible into evidence, the evidence bearing on the question of voluntariness may also be presented to the fact triers. These would have been the empanelled jurors in the \textit{Richardson} trial. When any evidence going to the question of confession voluntariness is heard by a jury, the jury makes a determination as to the weight that they should ascribe to it in arriving at their verdict.  If expert testimony is to be part of the evidence bearing on voluntariness, the judge must first, as in all situations in which expert evidence is proffered, apply all the legal rules governing the admissibility of expert evidence. 

The prosecution’s suggestion that the intended expert linguistic evidence was dispensable on the basis of the common knowledge rule seems to be rooted in certain misapprehensions about language proficiency. Often, lay people misapprehend the fact that an individual who speaks English as a second or non-native language may display fairly strong competence levels in certain types of speech events. This, however, may belie the ability of such an individual to function equally competently in other types of speech events demanding higher levels of proficiency. Embedded in the concept of registers is the distinction emanating from the language education field between basic interpersonal communicative skills (BICS) and cognitive academic language proficiency (CALP, \citealt{Cummins1979}). The former refers to linguistic proficiency in everyday social interactions, while the latter concerns the ability to articulate and understand abstract, specialised, more cognitively demanding notions typical of academic pursuits. \citet{Cummins2008} states that BICS is often conflated with CALP so there is an assumption that speakers of a second language who display fluency and competence in everyday conversations possess comparable academic proficiency in the language.

Much of Cummins’ research on BICS/CALP has been carried out within the domain of education, but \citeauthor{Pavlenko2008}'s \citeyear{Pavlenko2008} study extends the application of the concepts to the legal domain, specifically to the \emph{Miranda} warning. Pavlenko analysed an actual interrogation by the police of a non-native speaker\footnote{The speaker’s first or native language was Russian.} of English who had received some of her education, including at the tertiary level, in US schools. The analysis led to the conclusion that although the speaker’s English proficiency was “sufficient to maintain social conversations and minimal academic performance [it was not sufficient] to process complex texts in an unfamiliar domain” (2008: 26). This study substantiates earlier research by \citet{Brière1978}  involving a Thai native speaker which also indicates the superior levels of language proficiency that are necessary to understand the \emph{Miranda} warning. 

A more recent study \citep{Hulstijn2011} advances the notions of basic language cognition (BLC) and higher language cognition (HLC) in attempting to account for language proficiency among native (L1) speakers and non-native (L2) speakers, as well as between these two groups. Although BLC and HLC may approximate Cummins’ notions of BICS and CALP respectively, \citet{Hulstijn2011} suggests that L2 speakers are likely to have deficiencies in the skills relating to BLC.  BLC essentially covers commonly used language forms at all levels -- phonology, morphology, syntax, prosody, and semantics -- in conjunction with the rate at which speakers process these forms. This rate will be so even in cases where speakers of L2, because of their academic and professional exposure, master forms associated with HLC which relate to uncommon morphosyntactic forms and lexical items, typically combining with topics which are not commonplace. This has implications for Caribbean English vernacular-dominant speakers, many of whom are not highly educated, which suggests low levels of HLC in their L2 (English), and, at the same time, indicates that their BLC in L2 is likely to be below the average BLC level of native speakers of English.

It is arguable that the misconception that personal interactive linguistic competence is directly correlative with academic language proficiency was at play in the \emph{Kwame Richardson} case. As already indicated in this discussion, this misconception is signalled by the prosecution’s suggestion that the court could rely on the interrogating agents’ account of the defendant’s conduct while he was being questioned. Although the expert’s report did not come before the court for consideration as to admissibility, the basis for the judge’s decision on the motion to suppress also suggests a failure by the court to appreciate the BICS/CALP distinction. The court’s decision on this motion, contained in the Memorandum and Order, was that the defendant had “sufficiently strong English skills to enable him to have voluntarily, knowingly, and intelligently waive[d] his Miranda rights.”\footnote{\emph{United States v Kwame Richardson}, No. 09-CR-874 (JFB), 2010 WL 5437206, at *6 (EDNY Dec. 23, 2010).} The decision was based on several factors, including the evidence provided by the interrogating agent that during the interrogation he spoke to the defendant in English. The nature of the questions\footnote{\emph{Ibid}. at *1-*2. The questions include whether the defendant had come to the location in question by himself; whether he knew the type of drugs in the suitcase he had received; whether he knew the people who sent him to pick up the suitcase; whether other people were involved; whether he knew who in Guyana was supplying the drugs.} put to the defendant in the interrogation, as reported in the Memorandum and Order, does not appear to demand the higher language proficiency levels associated with CALP or HLC and is rather in keeping with proficiency levels necessary for everyday social interaction.\footnote{The distinction in the nature of language proficiency levels is effectively demonstrated by \citet[142--145]{EggingtonCox2013} in their discussion of an actual case. In this case, the first author was asked to provide an expert opinion on whether the Spanish-dominant respondent, a candidate for elected office in the relevant city council, possessed sufficient English language proficiency, as required by statute, so as to be eligible for election.} The court’s reliance on this evidence, then, seems to provide support for the claim that there seems to be a tendency on the part of laypersons, including judges, to conflate BICS with CALP. This is arguably in keeping with a general perception, including on the part of judges, that language is not a specialised field so laypersons are typically competent to decide on questions regarding language. This mindset is demonstrated in the US copyright case of \emph{Mowry v Viacom International Inc,}\footnote{No. 03 Civ 3090(AJP), 2005 WL 1793773 (SDNY July 29, 2005).} in which expert linguistic evidence was proffered with a view to supporting striking similarity between the works in question. The court stated:

\begin{quote}
    The Court has read The Crew and read and viewed versions of The Truman Show. \textit{Unlike specialized areas like music}, the trier of fact can compare the works without the need of expert evidence.\footnote{\emph{Ibid}. at *13. } (emphasis added)
\end{quote}


This tendency is arguably compounded by the fact that law professionals tend to regard themselves as experts on language. This self-perception leads to a devaluing or facile rejection of evidence offered by linguists, as reported, for example, by \citet[300]{Coulthard1997-2013} concerning disputed text. \citet[223]{TiersmaSolan2002} comment that “courts shy away from linguistic testimony when it conflicts with certain beliefs about language and cognition deeply entrenched in the legal system”. The implication is that there is a latent judicial resistance to expert linguistic evidence. This is a factor that may lead to the exclusion of such evidence, or its rejection where it has been admitted, or to flawed bases for making a judicial determination.  

The overt similarity between Caribbean English vernaculars and English may also cause lay persons to believe that the speech of a Guyanese-dominant speaker, for example, is English and that the speaker is proficient in English. There are, arguably, hints of this confusion in the court’s Memorandum and Order on the motion to suppress. The judge notes that “[a]lthough Richardson spoke with a thick Guyanese accent, throughout the course of the two interviews [with the Special Agent] he spoke in English” and that a “Pretrial Services’ interview sheet for Richardson... indicates that Richardson is fluent in English as a secondary language”\footnote{\emph{United States v Kwame Richardson}, No. 09-CR-874(JFB), 2010 WL 5437206, at *2 (EDNY Dec. 23, 2010). It is worth mentioning that \citet{English_Fiona2010} shows how police assessments of English proficiency of non-native speakers of English are sometimes exaggerated.}. These assertions about the defendant’s language skills, however, must be seen against the more equivocal statement in the Memorandum and Order that the bail report that had been prepared regarding Richardson indicated that his “primary language is Creole/English.”\footnote{\emph{United States v Kwame Richardson}, No. 09-CR-874(JFB), 2010 WL 5437206, at *2 (EDNY Dec. 23, 2010).} This suggests a conflation of Creole with English, notwithstanding that the bail report also indicated “that an interpreter was required.”\footnote{\emph{Ibid}.} The ideology that Caribbean English vernaculars are forms of English coupled with prevailing beliefs on the part of laypersons about communicative competencies of speakers of English as a second language are arguably enabling factors that bolster the view held by some legal professionals that they are capable, from a commonsensical perspective, of evaluating language proficiency and comprehension. This, in turn, would operate so as to encourage judicial invocation of the common knowledge exclusionary rule regarding certain kinds of expert linguistic evidence.

\subsection{Exclusion of evidence on the basis of relevance}

Documents\footnote{The Government’s Memorandum of Law in Opposition to the Defendant’s Motion to Suppress, 7--9, \emph{United States v Kwame Richardson}, No. 1:09-cr-00874(JFB), Document 51, Filed 09/30/10; Letter from US Attorney for EDNY, 6, \emph{US v Kwame Richardson}, No. 1:09-cr-00874(JFB), Document 66, Filed 02/14/11 (the government’s letter in support of its motion \emph{in limine} to exclude the testimony of the defendant’s proposed expert in which the prosecution outlines arguments against the admission of the expert’s testimony at trial).} filed in the \emph{Richardson} case show that the prosecution also intended to object to the use of some of the proposed expert testimony on the ground that it was not relevant. In law, all evidence, including expert opinion evidence, must be sufficiently relevant,\footnote{In US federal law, the FRE Rules 401 and 402 address relevance. In Anglophone Caribbean territories, the case of \emph{Jairam v The State} [2005] UKPC 21, on appeal from Trinidad and Tobago, is the applicable law.} i.e., it must have some bearing on the probability or not of a fact in issue so that it would assist the trier of fact in understanding and determining the issue. It may be that an expert’s opinion, while not running afoul of the common knowledge rule, is adjudged to be irrelevant.\footnote{See, for example, \emph{R v Turner} [1975] Q B 834, 841.}  

A dimension of the opinions contained in Devonish’s expert witness report in the \emph{Richardson} case was that Guyanese Creole is a language other than English and that the defendant was a speaker of a lower mesolectal variety of Guyanese Creole. It was suggested by the prosecution in certain court filings that this aspect of the opinion was irrelevant. The argument was that the fact that the defendant was a speaker of Guyanese Creole had no bearing on his English communication skills, particularly in view of the fact that the defendant had been living in the US for some five years at the time of the incident. It was also suggested by the prosecution in its filings that expert testimony regarding the “differences, if any, between ‘Guyanese Creole’ and English”\footnote{The Government’s Memorandum of Law in Opposition to the Defendant’s Motion to Suppress, 9, \emph{United States v Kwame Richardson}, No. 1:09-cr-00874(JFB), Document 51, Filed 09/30/10.} were not probative of the question of whether the defendant knowingly and voluntarily waived his \textit{Miranda} rights. The Government’s Memorandum of Law in Opposition to the Defendant’s Motion to Suppress stated that “[t]he relevant question is whether the defendant understands English, not whether ‘Guyanese Creole’ differs in some way from English.”\footnote{\emph{Ibid}.} 

It is useful for linguists offering their expertise in court cases to be aware of how legal rules relating to expert evidence may be deployed to craft challenges to such evidence which could threaten the use of their evidence. Such an awareness might inform the way in which their expert witness report and, where applicable, their \textit{viva voce} testimony are configured and expressed.  It is unfortunate that we do not have the benefit of judicial consideration and determination of the apparent intended arguments by the prosecution, and it is sometimes difficult to predict how a court may reason and decide on issues before it. I venture to suggest, though, that the intended arguments by the prosecution seem narrowly legalistic in the assertion that neither the distinctions between Guyanese and English nor the assessment of the defendant as a mesolectal Guyanese Creole speaker was relevant to the question of whether or not he understood the \emph{Miranda} warning told to him in English. Both these aspects of the expert’s report address and illuminate the confusion about the nature of the defendant’s speech as reflected in the Memorandum and Order on the motion to suppress, to which reference has been made in section 3.1. The report tends to infer that the resemblance between the languages is deceptive to the layperson, whose perception of the defendant’s speech as English, and thus judgment about the degree to which the defendant understands English is likely to be erroneous.

The prosecution’s intended argument that the fact that the defendant was a speaker of Guyanese did not rule out competence in English suggests the possibility of bilingual competence on the defendant’s part in both Guyanese and English. This seems to have overlooked the specific question posed in the expert report that was being answered in the findings of the report: “Is the language \textit{habitually} spoken by [the defendant], Guyanese Creole (Creolese), a language other than English?” (my emphasis). It appears, then, that it was not an issue that the defendant spoke Guyanese consistently, despite his time in the US. This, in turn, raised a critical question of whether his speech (Guyanese) was merely a variety of English. The findings contained in the expert’s report that the prosecution sought to challenge on the basis of relevance thus provide information that would assist the trier of fact to widen his appreciation of the nature of the defendant’s language proficiency. The trier of fact would be alerted to the fact that the defendant’s speech only superficially resembled English and that the structure of the defendant’s language distinguished it from English to the point where it has been regarded by language specialists as separate from English.  These findings in the report also provide the context for the second dimension of the expert’s finding -- that the defendant showed limited understanding of English, a language that differs from the language habitually spoken by the defendant. 

\largerpage
One English/Guyanese difference that seems to present comprehension problems concerns English lexical items which commence with the sound /a/ followed by a morpheme that can function autonomously. Examples of such English lexical items are \emph{appoint, assign, \textit{and} account}. In mesolectal varieties of Caribbean English vernaculars, including Guyanese, /a/ is a lexical item signifying the first person singular. English lexical items which are infrequently used such as the examples indicated are, in several instances, (mis)understood by Caribbean English vernacular speakers as first person singular (/a/) + verb (eg., /koʊnt/), English, \emph{I count}, rather than the meanings associated with the noun or verb, \emph{account}, in English (Brown-Blake and Chambers 2007: 279). This morphosyntactic-based confusion for habitual speakers of Caribbean English vernaculars is indeed relevant in the context of \emph{Miranda} warnings. As \citet[130]{Rogers2008} point out, many versions of the warning contain the word, \emph{appointed.}\footnote{\citet[130]{Rogers2008} also state that despite the frequency with which the word appears in versions of the warning, it is often not understood by persons who have not achieved secondary level education.} It is relevant to note that the test administered by Devonish to Richardson for the purposes of his expert report attempted to simulate the linguistic structure and lexical nature of the \emph{Miranda} warning. Devonish’s report, in which there is a review of the interview he conducted with the defendant, indicates Richardson’s apparent comprehension difficulty with the English word, \emph{assign.}\footnote{The report indicates that the following clause was put to Richardson as part of the test administered by Devonish: “One [a financial advisor] will be assigned to you...”.} This lexical item occurred in the test administered to Richardson by Devonish who reports that Richardson “incorrectly assumed” that it was related to the need to sign something.\footnote{Devonish’s report indicates that upon Devonish reading the relevant sentence in the test text in the course of the interview, Richardson “commented, ‘\emph{Asain…wa yu miin bai asain? A ga a sain it ar wa?}’” This was translated in the report as “Assign? What do you mean by assign? Do I have to sign it or what?”} This kind of misunderstanding is analogous to the kind of confusion that can potentially arise in relation to \emph{appointed}, a word frequently occurring in administered versions of the \emph{Miranda} warning. The linguistic source of the confusion is connected with differences between English and Guyanese.  

Another English/Guyanese linguistic difference pertinent to the nature of the language typically used in \emph{Miranda} warnings concerns the construction of the passive form in English versus Guyanese and indeed in all Caribbean English vernacular languages.\footnote{See, \citet[109--110]{DevonishThompson2010} and \citet[97]{Alleyne1980}.} The English passive construction occurs in parts of several versions of the Miranda warning.\footnote{For example, in the sentence, “[o]ne [an attorney] will be appointed to you.”} A Guyanese-dominant speaker’s unfamiliarity with this syntactic form in English arguably provides an example demonstrative of the English non-native speaker’s deficiency at the BLC level which could contribute to comprehension difficulties. It may be that unfamiliarity on Richardson’s part with the English passive form, a generally unexceptional structural form for native speakers of English, might also have compounded his misunderstanding surrounding the use of the word, \emph{assign}, in the test.

The linguistic differences highlighted thus indicate the relevance of such information offered in the expert report, since they bear on the likelihood of the defendant’s understanding or lack thereof, of the warning told to him in English. Thus, an appreciation on the part of a judicial officer of the nature of these linguistic differences, particularly within the context of the ramifications of the BICS/BLC and CALP/HLC distinction for L2 speakers, is capable of informing his or her decision-making on \emph{Miranda} warning understanding and waiver.

\largerpage
It should be noted too that the report provides sociolinguistic information\footnote{Such information includes the fact that the defendant lived in Guyana for the majority of his life and had had a rural upbringing there; that he had not completed primary education (the means by which native speakers of Caribbean vernaculars generally acquire English); that, during the time he lived in the US, he worked and socialised largely within a community of Caribbean English vernacular speakers. These factors combine to restrict the opportunities for the defendant to acquire a high level of proficiency in English.} that helps to explain the defendant’s low proficiency in English despite the number of years he had spent in the US. It is worth mentioning that comparable sociolinguistic information regarding a partial speaker of English, an Aboriginal accused, in the Australian case of \emph{Western Australia v Gibson}\footnote{[2014] WASC 240, paras. 68--72.} was provided by Diana Eades, linguist, in her expert testimony in the case. Her testimony was relied upon by the Australian court, on the issue, among others, of whether the accused’s language skills suggested that he understood the police caution. Arguably then, the expert’s opinions which the prosecution seemed intent on challenging in \emph{Richardson} on the basis of relevance were germane to the issue under consideration. They would have provided the trier of fact with information which bore on the crucial question of the defendant’s probable understanding, and ultimate waiver of the \emph{Miranda} warning administered to him in English only.

\subsection{Legal test for admissibility of expert evidence}

In the \emph{Richardson} case, the prosecution also intended to challenge, at trial, the admissibility of the expert linguistic testimony on the ground that it was unreliable.\footnote{See, Letter from US Attorney for EDNY, 8--9, \emph{US v Kwame Richardson}, No. 1:09-cr-00874(JFB), Document 66, Filed 02/14/11 (the government’s letter in support of its motion \emph{in limine} to exclude the testimony of the defendant’s proposed expert in which the prosecution outlines arguments against the admission of the expert’s testimony at trial).} This ground emanates from FRE, Rule 702 which states that an expert may give opinion evidence, if, among other things, “the testimony is the product of reliable principles and methods” and if those principles and methods have been reliably applied by the expert to the facts of the case.\footnote{The issue of reliability is also a key factor in the law of Anglophone Caribbean territories on expert evidence. See, \emph{Myers, Brangman and Cox v R} [2015] UKPC 40 esp paras 57--58, on appeal from Bermuda, in which the PC adopted the principles in \emph{Ahmed v R} [2011] EWCA Crim 184. The \emph{Ahmed} court accepted the proposition that the subject matter of an expert’s opinion must form “part of a body of knowledge or experience which is sufficiently organised or recognised to be accepted as … reliable”.}   

The important opinion of the US Supreme Court in \emph{Daubert v Merrell Dow Pharmaceuticals, Inc}\footnote{509 US 579 (1993).}, which itself triggered some of the current formulations of FRE, Rule 702, augments our understanding of this reliability principle. A trial judge must make “a preliminary assessment of whether the reasoning or methodology underlying the testimony is scientifically valid”\footnote{\emph{Ibid}, 592--593.} and whether that methodology can be applied appropriately to the facts in issue. The \emph{Daubert} court outlined four non-exhaustive factors that are useful for a judge to bear in mind in carrying out this assessment. They are, (a) whether the method or technique can be or has been tested; (b) whether it has been subjected to peer review and publication; (c) the rate of error associated with the method or technique; and, (d) its general acceptance within the relevant scientific community\footnote{\emph{Ibid}, 593--595.}.

In a US federal case, a judge, faced with proffered expert evidence, will likely use one or more of these factors or criteria\footnote{In the law applicable to territories in the Anglophone Caribbean, there is no enumeration of criteria similar to the ones itemised in the \emph{Daubert} case, although reliability, as already noted at note 43, \emph{supra}, is an important principle governing expert evidence in these territories.  Courts of these territories, however, may have regard to factors akin to those listed in \emph{Daubert}. See, for example, \emph{Bernal and Moore v R} (1997) 51 WIR 241, esp 252--253, in which the Privy Council accepted the correctness of the trial judge’s refusal to admit the evidence of the expert on polygraph testing. The trial judge was of the view that polygraph testing was not a recognised or sufficiently established area. This consideration appears comparable with the fourth criterion enumerated in \emph{Daubert}.} in evaluating whether the method underpinning the opinion evidence is scientifically adequate so that it generates reliable results. This evaluation is especially applicable to proffered evidence that is outside disciplines with a recognised history of scientific rigour \citep{Durston2005}. The judge’s evaluation will determine whether the proffered evidence is admitted or excluded. In performing this gatekeeping function, the judge must determine that the proffered evidence is appropriately grounded. In so doing, a judge must require proof, on a preponderance of evidence, “that the expert’s specific theory or technique works; that is, that the use of the theory or technique enables the expert to accurately make the inferential determination that the expert contemplates testifying to” \citep[759]{Imwinkelried2003}. As \citet[480]{Kaye2005} more bluntly puts it in his discussion of the meaning of the first \emph{Daubert} factor of testability, judges must determine whether a particular method “is worth betting on, and they would do well to place their bets on theories that are not only testable but that also are tested.” The idea then is that scientific adequacy or validity, and hence legal reliability, may be established by offering proof of suitable and tried testing methods; certainly, a lack of robustness in the scientific method undermines its validity and will, in all likelihood, rule out its admissibility.  
US case law indicates that a judge, in his or her evaluation of evidential reliability, has discretion as to which \emph{Daubert} factors, among others, may be applied\footnote{See \emph{Kumho Tire Co v Carmichael} 526 US 137, 141, 151 (1999).}. It has been suggested in \emph{Kumho Tires v Carmichael} that the particular criteria to be applied in a given case will depend on the facts of the case, the specific circumstances, the nature of the issue being determined, as well as nature of the expert’s specialisation and his or her testimony. Generally, it seems that expert evidence regarding the language proficiency of individuals, particularly in the context of \emph{Miranda} comprehension, has been accepted by American courts (\citealt[27--228]{TiersmaSolan2002}; \citealt[660]{AinsworthJanet2006}) and thus, implicitly, has met the \emph{Daubert} standard of evidential reliability\footnote{In the Anglophone Caribbean, there has, so far, been scant use of, or reliance on expert linguistic evidence in court cases \citep{Steele2009,Blake2019}.}.  This would tend to show that language proficiency testing is not unusual or novel and this general position would have favoured the admissibility of the expert evidence proffered in \emph{Richardson}.

\subsubsection{The purported challenges in Richardson}

The prosecution in \emph{Richardson}, however, intended to base their \emph{Daubert} challenge to the expert evidence partly on the nature of the specific data used to arrive at the opinion that the defendant had a limited understanding of English and that he would have been unable to understand the main aspects of the \emph{Miranda} warning told to him in English. They suggested an apparent paucity of the data used by Devonish and alluded to weaknesses in the quality of the data: 

\begin{quote}
    Dr Devonish’s opinions about the defendant’s language abilities are based entirely on a telephone interview with the defendant that lasted approximately 30 minutes, and in which the defendant’s wife was also participating. Dr Devonish had no opportunity to observe the defendant’s demeanour.\footnote{Letter from US Attorney for EDNY, 8, \emph{US v Kwame Richardson}, No. 1:09-cr-00874(JFB), Document 66, Filed 02/14/11 (the government’s letter in support of its motion in limine to exclude the testimony of the defendant’s proposed expert in which the prosecution outlines arguments against the admission of the expert’s testimony at trial).} 
\end{quote}
   
This, presumably, would have cast doubt on the reliability of the test employed to arrive at his expert opinion. The prosecution also intended to attack the expert evidence on the basis that the expert’s testing method did not seem to compensate for the possibility of the defendant faking his level of proficiency:

\begin{quote}
    The defendant was in complete control of what he said and how he conducted himself during the interview with Dr Devonish.  The defendant obviously had a strong incentive to speak in a manner that would lead Dr Devonish to conclude that the defendant does not understand English ...\footnote{\emph{Ibid}. at *8--9.}
\end{quote}

The testing method was also challenged by a collateral attack on the nature of Devonish’s expertise:

\begin{quote}
    Nothing in Dr Devonish’s qualifications establishes his expertise in evaluating the English skills of others ... There is no indication in his report that he has specialized knowledge in reliable methods of testing language skills, particularly in situations where the subject has a powerful motive to skew the results\footnote{\emph{Ibid}. at *9.}.
\end{quote}
 
These \emph{Daubert} challenges were never actually argued and consequently, their legal cogency in terms of their impact on the evidence reliability \textit{cum} admissibility question remains uncertain. They may, however, be instructive for linguists offering expert evidence in terms of plugging potential gaps that open the possibility of legal challenge.  

\subsubsection{Lessons for linguists acting as expert witnesses}

Drawing on \citet{Meintjes-vanderWalt2019}, it is conceivable that, under cross-examination, Devonish might have faced questions seeking to ascertain, for example, the extent to which the method he used in testing the defendant’s proficiency has been validated in other studies; whether the test employed had been developed independently of the pending trial; how, if at all, the test offsets the possibility of the test taker faking his proficiency level, which raises the larger question of error rates; and the extent to which the method has been used or accepted by other language proficiency testing professionals. Questions such as these would have, arguably, been justifiable to probe the validity and reliability of the test used, since it appeared that Devonish had designed a language proficiency test that specifically contemplated the case.  This could have suggested that the method employed was not adequately developed, tested, or established which, in turn, affects how courts assess reliability. 

Given the possibility of such attacks on the reliability of the test method, a useful research inquiry may be how the testing method employed by Devonish compares with other language proficiency/comprehension tests,\footnote{Some of the comparison points include the length of time over which such tests are administered, how they are administered (e.g. face-to-face, via telephone, electronically, written or orally), the material used (a version of the \emph{Miranda} warning or other text), the precise test method (examinee required to explain in his own words, close tests, specialised vocabulary tests, etc), any special considerations to be applied where examinees speak a language variety related to the language in which the warning or caution has been administered and strategies for detecting faking of proficiency levels.}  particularly those that have already been used in legal or judicial contexts concerning \emph{Miranda} comprehension. Such an academic inquiry might provide some indication as to the possibility of Devonish’s test passing judicial scrutiny against the reliability and validity factors, but it is beyond the scope of this paper. Tests emanating from the discipline of psychology, seem to have been used in US courts and to have had some degree of judicial acceptance even under \emph{Daubert} standards \citep{Ryba2007}. \citet{Brière1978} and \citet{Roy1990} report the use, in the context of \emph{Miranda} comprehension, of proficiency tests\footnote{Brière, in his evaluation of the English proficiency of a Thai native speaker, used the Michigan Test of English Language Proficiency, Form D, and parts of the Brown-Carlsen Listening Comprehension Test, Form Bm; while Roy, in assessing a Puerto Rican origin defendant, reports using the Language Assessment Battery (LAB) 1982 used in New York City schools to evaluate the English proficiency of non-native speakers. Roy also reports the use of a single feature focus test developed by him.} arising largely from within the field of language education. It should be noted that these latter accounts of the use of proficiency tests pre-date the \textit{Daubert} standard, and, in any event, Brière’s test was not subjected to the prevailing pre-\emph{Daubert} admissibility scrutiny since there was eventually no trial of the defendant in respect of whom the proficiency tests had been carried out.\footnote{\citet[243, note]{Briere1978}.} All these tests, though, provide a methodological blueprint that can inform appropriate design responses to issues of reliability and validity for comparable tests for use with speakers of Caribbean English vernacular-dominant speakers in assessing comprehension of \emph{Miranda} warnings or of police caution.

Linguists asked to provide expert evidence on the language proficiency of defendants should also be prepared for reliability-based challenges on the basis of the possibility that the defendant could be faking their proficiency and lack of understanding. The issue of faking low proficiency is central to validity – whether the results of the assessment are reflective of what it claims to test. Van Naerssen, a forensic linguist, states that a language expert testing L2 proficiency should assume the possibility of both deliberate faking as well as truthful performance, but she concedes that linguists “have not yet solidly demonstrated expertise in detecting deceit” \citep[1547--1548]{vanNaerssen2013}. While she has experimented with a test to detect deliberate faking \citep{vanNaerssen2011}, it remains difficult to assess deliberate underperformance. The approaches suggested by van Naerssen, require ample samples of text or communication produced by the L2 speaker\footnote{The idea, according to van Naerssen, is that it is improbable that a speaker will be able to maintain intentional underperformance “throughout lengthy samples of unplanned communication, especially at different times” \citep[1548]{vanNaerssen2013} without giving him/herself away.} \citep[1548--1549]{vanNaerssen2013}, which is likely to make resorting to them impracticable in many real-life situations.

Given that tests for detecting faking are in their nascent stages of development, it will be hard for language experts to vigorously counter suggestions put to them by a cross-examiner regarding the possibility that a defendant is deliberately underperforming their L2 proficiency. If faced with such suggestions, a language expert may perhaps be in a position to rely on his or her experience in administering L2 proficiency assessments which, over time, may have revealed types of discrepancies tending to indicate deliberate underperformance. The expert may, on the basis of such experience, be able to say that they noticed no discrepancies in the samples which would tend to indicate deliberate underperformance on the part of the examinee. Professionalism, objectivity, and independence would of course require that findings by an expert who may be adverse to the side who has consulted them be included in their report.\footnote{In Jamaica, the Civil Procedure Rules, 2002, r. 32.4 expressly provide that an expert must consider, and ought not to omit material facts which could detract from his or her concluded view. This is also the position in respect of expert reports for criminal matters (\emph{Myers, Brangman and Cox v R} [2015] UKPC 40, paras. 59--60).} It is then up to the instructing lawyers to make strategic decisions in response to the nature of the expert’s findings, including a decision not to rely on the expert’s opinion and thus not putting the report in evidence or not calling on the expert at all to give evidence at trial.

The other dimension of the intended challenge by the prosecution in \emph{Richardson} concerned the suitability of the expert’s credentials for the task he was requested to perform. This goes to whom a court will regard as an expert in the field in which the relevant expertise is required. Based on FRE Rule 702, an expert is someone sufficiently qualified by knowledge, skill, experience, training, or education in the particular field at issue, so that their opinion on the issue is likely to assist in determining a fact in issue.  This may be evidenced by qualifications and experience in the relevant field. It is clear that experts should only testify on issues within their field of expertise.  However, issues may arise in practice as to whether someone who possesses general qualifications in a field has the acceptable credentials to testify on a matter relating to a specialised sub-area within that field.  An automotive engineer, for example, could not testify whether it was probable that vehicle emissions would enter the passenger compartment while the vehicle was in operation because he had no expertise in aerodynamics.\footnote{See, \emph{Buzzerd v Flagship Carwash St Lucie}, 379 F App’x 797 (3rd Cir. 2010).} Consonant with this, in a case arising from Jamaica, the evidence of specialist engineers was, given the nature of the issue in question, held to be preferable to that given by engineers without the relevant specialist expertise.\footnote{\emph{West Indies Alliance Co Ltd v Jamaica Flour Mills Ltd} [1999] UKPC 35, paras. 92--107. }   

The caution then to both the expert and the lawyer is that an appropriate assessment should be made of whether someone with specialist expertise on the matter in issue would be more suitable than someone with general expertise in the overall field. The expert is, of course, important to such an assessment because, with superior knowledge of the field, they will be in a position to advise on whether the matter in issue is within their core competence or whether it may be more beneficial to engage an expert with more specialist knowledge. Such an assessment would be useful not only in deflecting a challenge regarding the suitability of the expertise offered but would also be useful in decisions about how an expert’s competencies are represented. The expert’s qualifications and experience are a critical peg in judicial decisions on the admissibility of his or her evidence. Furthermore, where an expert strays outside his sphere of competence in giving evidence, there is little or no value to the evidence offered,\footnote{\emph{Price Waterhouse v Caribbean Steel Company Ltd} [2011] JMCA Civ 29.} and the trier of fact should disregard it. It may also be the case that a cross-examiner may succeed in discrediting evidence from an expert on the basis of a lack of sufficient competence in the sub-field. In addition, where evidence of a specialist in a sub-field germane to the issue in question competes with evidence from someone with general expertise in the broad field, the former is likely to be preferred over the latter.


\section{Conclusion}

The discussion has shown that it is useful for language specialists offering expert evidence in court proceedings to be aware of the legal rules governing expert evidence. Such an awareness is likely to assist the expert in the preparation of his or her report. It should also help to alert him or her to the nature of the possible legal challenges to the evidence he or she plans to offer. Linguists will thus be in a better position to respond professionally to those challenges. As revealed by the \emph{Richardson} case, some legal challenges to expert evidence may emanate from incorrect assumptions by laypersons about language in general, and specifically about the nature of Caribbean English vernacular languages. Linguists, particularly those offering evidence in connection with these vernacular languages in English medium courtrooms, should be prepared, where necessary, to confront such assumptions and misconceptions. Language experts, like all other types of experts, should be aware, however, that it is the trier of fact who determines the facts in issue.\footnote{\emph{Robinson v The State} [2015] UKPC 34, para.16 (on appeal from Trinidad and Tobago).} Even persuasive evidence put forward by an expert may be rejected by the ultimate fact-finder. The expert’s role, in this context, is to place before the trier of fact relevant and reliable specialist knowledge which will provide perspectives not otherwise available, which can help the fact determiner to come to the most appropriate finding.   


% -- References:
{\sloppy\printbibliography[heading=subbibliography,notkeyword=this]}


\end{document}
