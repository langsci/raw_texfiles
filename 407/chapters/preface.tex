\addchap{Preface}

In the summer of 2020, the Society for Caribbean Linguistics (SCL) was scheduled to have its 23\textsuperscript{rd} Biennial Conference at The University of the West Indies, St. Augustine campus in Trinidad and Tobago. Plans were already well underway from the year before and scholars from across the world were getting ready to make their usual trek to the conference in a warm and accommodating Caribbean destination. Abstracts had started coming in close to the end of 2019 and by February 2020 a rudimentary conference schedule had started to take shape, all the while news of some new flu variant was making the rounds on the news circuits. Then, almost out of nowhere, the world came to a screeching halt in March when a global pandemic was declared. The SCL executive committee hoped that a conference would have been a possibility by August, but we had to pull the plug a few months before. The 23\textsuperscript{rd} Biennial SCL Conference was canceled. 

I sent out a proposal for a panel on language and the law during the call for papers and received two abstracts to be considered for the panel. Along with my paper, the panel had three presenters when the conference was canceled. It occurred to me shortly after the cancellation that this would be a good opportunity to convert the panel into a publication focusing on language rights in the Caribbean context. I extended the invitation to contribute a chapter to a few more individuals and received two more positive replies, bringing the number of contributions to five. It became apparent fairly quickly that this was a good decision since a published volume could accommodate many more contributions than a conference panel that is usually capped at four papers, and a full publication would encourage authors to develop their ideas beyond the scope of what was expected for a conference paper. By the end of 2020, all the contributions were ready for peer review and SCL’s Studies in Caribbean Languages book series hosted by Language Science Press was the obvious choice.

The summer of 2020 also saw a surge in the \emph{social} pandemic of anti-black racism, sparked by a tragedy in the USA which then reverberated across the globe in the form of protests and rallies, but also in the form of policy changes. The timing of this volume is apropos as it gives us the opportunity to reflect on the anti-black racist origin of attitudes towards Caribbean Creole languages, and helps us forge new directions in the ways that we study and interact with these languages. Linguists and other researchers interested in the connections between language and the law are likely to find the discussions in this volume useful, as well as individuals interested in the sociology of language, and advocacy in its broadest sense. The content is suited for a range of readers spanning senior undergraduates to established academics and could serve as targeted reading material for courses that have a language and law component or more generally, syllabi that highlight social issues surrounding Caribbean languages. 

Several of the authors in the present volume have looked at communicative issues coming out of the courtroom context where a Creole-dominant speaker has had to interact with officers of the court who use a standard(ised) European language. This is a common site for investigation since there is a “safe” bet that a Creole-speaking member of the public is likely to encounter challenges regarding effective communication in an ‘alien’ discourse space such as a courtroom. There are, however, other adjacent areas of research that are concerned with language and the law in the Caribbean context, but not necessarily with discourse patterns and interpretation issues inside the courtroom. This volume focuses primarily on one of those areas -- language rights. Each author interrogates the issue of language rights from a different perspective; whether through the lens of social justice, the right to an interpreter in court, or strategies for addressing linguistic discrimination, but we all converged on a singular observation -- significant work is still needed in this area.

The present volume is the second collection of essays dealing with language rights in the Caribbean context. The first such collection was Brown-Blake and Walicek (2013), “Language Rights and Language Policy in the Caribbean,” the second issue of the Sargasso Journal of Caribbean Language, Literature and Culture. They presented a snapshot of the status of language policies (or the lack thereof) in a select few Caribbean territories; most notably Jamaica, St. Lucia, Puerto Rico, and Haiti. The Brown-Blake and Walicek collection contained a mixture of original research and reflections from Caribbean linguists spanning three distinct generations. There was an interview with Mervyn Alleyne, one of the “older heads” in Caribbean linguistics, contributions from Hubert Devonish and Marta Dijkhoff, senior academics in the subfield, and chapters from Clive Forrester and R. Sandra Evans, newly minted PhDs at the time. The Brown-Blake and Walicek volume appeared only a year after the creation of the “Charter on Language Policy and Language Rights in the Creole-speaking Caribbean” was established at a meeting at UWI, Mona in 2011. The energy and expectations of the linguists and language advocates who met to draft the charter were high and the document represented the first attempt at articulating a fairly robust list of rights for speakers of Caribbean languages. 

Ten years later, the energy and expectations are still high, but as the present collection of essays demonstrates, many of the challenges still remain; the spectre of linguistic discrimination is ever present, communicative issues in the courtroom persist, and policymakers are still reticent about defending indigenous languages in the region. If anything, what the present volume demonstrates is that even though the goals of the charter on language rights are yet to be fully realized, there are still opportunities for language advocates of all stripes to devise impactful strategies in the ongoing struggle for the recognition of language rights. Caribbean language advocacy has come a long way, but it still has a long way to go. This current collection of essays is meant to bring us slightly closer to that point. 

\bigskip

{\large \noindent 
  Clive Forrester \\
  Waterloo, Canada. 
}


