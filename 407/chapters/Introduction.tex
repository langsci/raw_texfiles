\addchap{Introduction} 

It is not altogether clear how to categorize matters of language rights within their proper disciplinary foci. Is it a branch of linguistics, or a branch of law? Which set of researchers should ideally take charge of articulating the parameters and goals of this area? Linguists command expertise in language and communication but lack the depth of knowledge required to interrogate or critique human rights instruments. Lawyers are skilled at evaluating human rights legislation but lack proficiency in analyzing language issues in the micro- and macro-linguistic arenas. The linguist and the lawyer, proficient in their separate areas of research and cognizant of each other’s blindspots, can bring a collaborative perspective to the emerging area of linguistic rights in the Caribbean region. This volume aims to accomplish exactly that, i.e., to provide researchers from these two backgrounds with a platform to engage in rigorous interdisciplinary dialogue on language rights. 

Admittedly, the current volume is not a law book. It is decidedly oriented towards issues central to Caribbean linguistics, notwithstanding the fact that two of the contributors are trained lawyers (Brown-Blake and Murray). The goal in each contribution is not so much to point out the gaps in legislation in relation to language rights, but rather to highlight how current language-related issues in the Caribbean make it difficult for Creoles and other minoritised languages to be recognised as languages that should even be afforded rights. Indeed, the very idea of language rights is such a novel concept in the Caribbean that legislative gaps are to be expected even if we refuse to accept them in perpetuity. The contributors to this volume acknowledge what is absent from the legislative framework or the public consciousness as the case might be, and in turn, recommend strategies for ways in which full language rights recognition can become integrated both into the law as well into public life. 

The first chapter (\textbf{Clive Forrester}) explores attitudes toward discussions of language rights from the standpoint of speaker perception. Forrester analyzes samples of comments to an online newspaper responding to an article on language rights in an attempt to define a role for the Caribbean linguist in bringing analytical expertise to these issues. Forrester goes on to argue that the role of the linguist as it relates to language rights is not so much to commandeer the process but rather to serve the community interests from within the same community, and in the same discursive styles common in the community, as opposed to aloof, English-dominant, academic discursive styles as some online commenters suggest. This chapter is important in laying a foundation for the place of linguistic expertise in issues of language rights generally, but also specifically within the Jamaican context where language rights are seen as merely an academic preoccupation. 

\textbf{Kadian Walters} highlights the issue of linguistic discrimination in Chapter 2 when she turns attention to the uneven service given to speakers of Jamaican Creole when interacting with public servants in government offices. Walters juxtaposes the use of Jamaican Creole in public marketing campaigns to promote civic duties such as recycling and safe driving, with linguistic discrimination from public servants when face-to-face with speakers of Jamaican Creole. This contrast is made all the more striking against the background of a recent language attitude survey that echoes a similar sentiment from the first survey done in 2006 -- the public supports the use of the Jamaican language in public formal spaces. Walters argues that the government’s failure to act on mounting research that indicates that linguistic discrimination occurs and that Jamaicans are in favour of seeing their language in formal contexts is tantamount to ignoring the people’s “demand for justice”. 

The next two chapters handle different aspects of communication in the courtroom space. The first of these comes from \textbf{Celia Brown-Blake} in Chapter 3 with her examination of the role of the linguist in delivering expert witness testimony. She focuses on a case tried in the USA involving a speaker of Creolese (the English-related Creole in Guyana) and the use of an expert witness report -- from linguist Hubert Devonish -- in determining whether the accused individual was sufficiently competent in English to knowingly waive his Miranda Rights. Brown-Blake’s discussion centers on two of the primary challenges with giving expert witness testimony for Creole speakers: (i) the perception that differences between English-related Creoles and Standard English are insignificant, and (ii) lawyers believe they are experts at language anyway and therefore need no further expertise to make a submission on a linguistic matter. Brown-Blake, being one of the two lawyers in this volume, manages to bridge the two competing issues of legal obligations when making an arrest and communicative obligations during the said arrest. 

\textbf{Robertha Sandra Evans}’ discussion in Chapter 4 explores the access to lan\-guage-re\-lat\-ed rights afforded to speakers of St. Lucian French Creole -- Kwéyòl. The data used for the discussion in this chapter comes from interviews carried out with St. Lucian police officers who were questioned about the procedure used when interacting with a Kwéyòl speaker. Similar to the findings in Brown-Blake’s presentation in Chapter 3, Evans highlights that the police officers take communicative competence in Kwéyòl for granted simply by virtue of having grown up in St. Lucia. This situation is compounded by the fact that, from the officers’ own admission, efforts are made to ensure that speakers of Standard French get the services of an interpreter as stipulated in the St. Lucian Charter of Rights. Evans gives concrete examples from courtroom interactions on the kinds of misinterpretations that occur when Kwéyòl speakers are not extended the same kind of courtesy. 

The final chapter closes this volume with an overview of the legal and sociolinguistic hurdles in the way of ensuring the recognition of Creole speakers’ full linguistic rights in the courtroom context. \textbf{Murray and Anglin} focus their argument on Jamaica, but the similarities to the situation which exists in the St. Lucian legal framework are unmistakable. As such, many of the hurdles identified in this chapter, as well as the recommendations for overcoming them, are applicable to any Creole language situation. This final chapter is important since it is the sole chapter written entirely by non-experts in the field of linguistics -- Murray is trained in the law, and Anglin is trained in advocacy -- and affords the reader the opportunity to look at the problem from the perspective of law and social justice advocacy. 
