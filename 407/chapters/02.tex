\documentclass[output=paper,colorlinks,citecolor=brown]{langscibook}
\ChapterDOI{10.5281/zenodo.10103076}

\author{Kadian Walters\affiliation{The University of the West Indies, Mona}}
\title[We want justice]{We want justice: Linguistic discrimination in Jamaica’s public formal domains and the people’s cry for justice}
\abstract{The proverbial cry of “we want justice” is usually echoed during demonstrations when citizens bemoan some instance of injustice usually committed by law enforcement officers. The demonstrators are often seen with placards usually written in Jamaican Creole (JC). Similarly, Jamaicans have been crying out for language justice for a number of years. 

This injustice is manifested in the forms of direct and indirect linguistic discrimination in Jamaica’s public formal domains. The former occurs in interpersonal settings when someone is treated differently and unfairly because of his or her language use, while the latter exists when information is provided for the general public in a language (English) that the mass of the population does not understand.

This chapter explores research in the area of linguistic discrimination as a result of this English dominance in the courtroom, government agencies, and the mass media. Since the National Language Attitude Survey (2005) subsequent surveys over a ten-year period have shown that the majority of the informants indicated a desire for Jamaican Creole to be used in a more serious manner in several domains. These include JC becoming an official language alongside English, being used to read the news both on television and radio, and by public service agents in government agencies. The chapter highlights the public’s cry for the right to receive information and services in Jamaican Creole (JC) in a similar way it is given in English.}

\IfFileExists{../localcommands.tex}{
   \addbibresource{../localbibliography.bib}
   % add all extra packages you need to load to this file

\usepackage{tabularx,multicol}
\usepackage{url}
\urlstyle{same}

\usepackage{listings}
\lstset{basicstyle=\ttfamily,tabsize=2,breaklines=true}

\usepackage{langsci-basic}
\usepackage{langsci-optional}
\usepackage{langsci-lgr}
\usepackage{langsci-osl}
% \usepackage{./langsci/styles/langsci-lgr}
% \usepackage{./langsci/styles/langsci-osl}
% \usepackage{langsci-gb4e}

\usepackage{tikz}
\usetikzlibrary{patterns,calc}
\pgfdeclarepatternformonly{south east lines}{\pgfqpoint{-0pt}{-0pt}}{\pgfqpoint{3pt}{3pt}}{\pgfqpoint{3pt}{3pt}}{
    \pgfsetlinewidth{0.6pt}
    \pgfpathmoveto{\pgfqpoint{0pt}{3pt}}
    \pgfpathlineto{\pgfqpoint{3pt}{0pt}}
    \pgfpathmoveto{\pgfqpoint{.2pt}{-.2pt}}
    \pgfpathlineto{\pgfqpoint{-.2pt}{.2pt}}
    \pgfpathmoveto{\pgfqpoint{3.2pt}{2.8pt}}
    \pgfpathlineto{\pgfqpoint{2.8pt}{3.2pt}}
    \pgfusepath{stroke}}
    
\usepackage{stmaryrd}
\usepackage{wasysym}
\usepackage{multirow}
\usepackage{caption}
\usepackage{subcaption}
\usepackage{mathrsfs}
\usepackage{qtree}

\usepackage{linguex}


   %pminos do not split footnotes
% \interfootnotelinepenalty=10000 %Footnote in Laporte chapters has to be split SN


%\DeclareIndexNameFormat{default}{%
%\nameparts{#1}%
%\usebibmacro{index:name}%
%{\index[names]}%
%{\namepartfamily}%
%{\namepartgiveni}%
% {}% L1
% {}% L2
%{\namepartprefix}% generates spurious space L3
%{\namepartsuffix}% generates spurious space L4
%}

%  {\DeclareIndexNameFormat{default}{%
%     \usebibmacro{index:name}{\index[names]}{#1}{#3}{#5}{#7}}}

%\DeclareIndexNameFormat{default}{%
%  \usebibmacro{index:name}{\sindex[nom]}{#1}{#3}{#5}{#7}}

%\DeclareIndexNameFormat{default}{%
%  \usebibmacro{index:name}{\sindex[person]}{#1}{#3}{#5}{#7}}
%\DeclareIndexNameFormat{default}{%
%\nameparts{#1} \usebibmacro{index:name}{\sindex[person]]}{\namepartfamily}{‌​\namepartgiven}{\nam‌​epartprefix}{\namepa‌​rtsuffix}}

%\newcommand{\smiley}{:)}

%\renewbibmacro*{index:name}[5]{%
%\usebibmacro{index:entry}{#1}%
%{\iffieldundef{usera}{}{\thefield{usera}\actualoperator}\mkbibindexname{#2}{#3}{#4}{#5}}}

% \newcommand{\noop}[1]{}

%remove for final
%\overfullrule=1mm

\newcommand{\tobi}[2]}}
\renewcommand{\S}[1]{\tobi{#1}{\textsc{*}}}

% this volume references
% puts: [this volume]
% already defined: \citetv
%\newcommand{\citepv}[1]{(\citeauthor{#1} \citeyear*{#1} [this volume])}
\newcommand{\citealtv}[1]{\citeauthor{#1} \citeyear*{#1} [this volume]}

%parentheses around example number
\newcommand{\pref}[1]{(\ref{#1})}

% in-text examples

\newcommand{\lnex}[1]{\textit{#1}} %target lang word
\newcommand{\lnlit}[1]{(lit.: `#1')} %literal reading
\newcommand{\lnlat}[1]{(#1)} % latinization
\newcommand{\lntrans}[1]{`#1'} %translation
\newcommand{\lnexl}[2]%
{\lnex{#1}{} \lnlat{#2}} % ex with latinization
\newcommand{\lnexlat}[3]{\lnex{#1}{} \lnlat{#2}{} \lntrans{#3}} % ex with latinization and tranl.

%ch01
\newcommand{\co}[1]{\mbox{\textbf{#1}}}

%ch09

\newcommand{\cyrbulg}[1]{\begin{otherlanguage*}{bulgarian}#1\end{otherlanguage*}}


%ch10
\newcommand{\nlp}{{\small NLP}}
\newcommand{\mwe}{{\small MWE}}
\newcommand{\rae}{{\small RAE}}
\newcommand{\lvc}{{\small LVC}}
\newcommand{\pos}{{\small P}o{\small S}}
%\newcommand{\todo}[1]{ \textcolor{red}{#1} }

%\renewcommand{\labelenumi}{\theenumi}
%\ainamefmt{{vv}{ll}{, ff}{, jj}} % fullname

\newcommand{\biberror}[1]{{\color{red}#1}}

\newcommand{\osenovaitem}{--~}
   %% hyphenation points for line breaks
%% Normally, automatic hyphenation in LaTeX is very good
%% If a word is mis-hyphenated, add it to this file
%%
%% add information to TeX file before \begin{document} with:
%% %% hyphenation points for line breaks
%% Normally, automatic hyphenation in LaTeX is very good
%% If a word is mis-hyphenated, add it to this file
%%
%% add information to TeX file before \begin{document} with:
%% %% hyphenation points for line breaks
%% Normally, automatic hyphenation in LaTeX is very good
%% If a word is mis-hyphenated, add it to this file
%%
%% add information to TeX file before \begin{document} with:
%% \include{localhyphenation}
\hyphenation{
    Beck-man
    Ngu-yen
    back-chan-nel
    back-chan-nels
    mo-not-o-nous
    ste-reo-typ-i-cal
}

\hyphenation{
    Beck-man
    Ngu-yen
    back-chan-nel
    back-chan-nels
    mo-not-o-nous
    ste-reo-typ-i-cal
}

\hyphenation{
    Beck-man
    Ngu-yen
    back-chan-nel
    back-chan-nels
    mo-not-o-nous
    ste-reo-typ-i-cal
}

   \boolfalse{bookcompile}
   \togglepaper[23]%%chapternumber
}{}

\begin{document}
\maketitle

\section{Background}

\subsection{Jamaica’s public formal communication}

In Jamaica’s national communication system, the government and its institutions communicate with the public using one dominant language. Citizens have however demonstrated an increasing demand for basic language rights for speakers of Jamaican. This demand is indicated in the results of successive national language surveys discussed in this chapter. This signals a shift in the language ideology of the people, as previously held views that Jamaican is a broken and bastardized form of English, are gradually being eroded. Four unpublished language surveys reveal that Jamaicans desire their government to allocate more significant functions to their mother tongue in public formal domains: \citet{JamaicanLanguageUnit2015};  \citet{LanguageUse_inthe_CourtRoom20015}; \citet{LanguageUse_in_PublicAgencies2016}; \citet{LanguageUse_inthe_JamaicanMediaSurvey2017}. Such domains include legal, educational, media, and government contexts. 

\begin{sloppypar}
In his message to commemorate the International Year of Languages, the then Director General of UNESCO reiterated that: “thousands of languages, though mastered by those populations for whom it is the daily means of expression, are absent from education systems, the media, publishing and the public domain in general” \citep{Matsuura2007}. Some local political representatives have also agreed that Jamaican should take greater prominence in official contexts because it is the vernacular of the masses. Former Ministers of Education have both publicly stated that “Patois must be regarded as a first language for persons who have little exposure to the English language” \citep{Reid2017}, highlighting that “Jamaican is the first idiom of the majority” \citep{Thwaites2018}. Others have gone further to declare a push for official bilingualism: “We have our own language, and we need to `officialise it', while at the same time we extol the virtues of the English language” \citep{Smith2012}. Institutional bilingualism would see the government providing information and services in both English and Jamaican, in written and spoken form. Continued public education campaigns to teach the official writing system for Jamaican would be necessary before implementing a bilingual system.
\end{sloppypar}

\begin{sloppypar}
Previous research shows that speakers of Jamaican experience discrimination in various domains \citep{Linton-PhilpFrench2001,Brown-Blake2011,Walters2016}. Only a comprehensive national language policy and carefully planned implementation and enforcement can achieve fair, humane, and courteous treatment for Jamaican speakers. Such policy should consider Jamaican and English as the two dominant languages in Jamaica: one has numerical dominance of speakers and the other enjoys functional dominance in public formal domains. Regarding numerical dominance, most monolingual Jamaicans (36.5\%) speak Jamaican \citep{JamaicanLanguageUnit2006}, while most of the state communication takes place in English. Though Jamaican is the vernacular language of the masses, English dominates as the language widely used in formal contexts. Of course, this places the monolingual Jamaican speaker at a disadvantage, as they are unable to access information and services in a language in which they are competent. Each citizen supposedly has the “right to equal access to service and information” \citep[2]{Ministry_of_CommerceScience_and_Technology2003} but this is clearly not the case for those with limited knowledge of English.
\end{sloppypar}

\subsection{The Government of Jamaica’s communication policy}

Since the specific domain under investigation is that of public entities, it is necessary for us to look at the language situation in these contexts. Among public formal domains, services are provided for the public on behalf of the state within the judiciary, education, and the mass media. The public service includes all agencies of the state established by law to carry out the policies of the Government of Jamaica. It consists of those entities that are part of the civil service, public enterprises established by the Act of Parliament and companies incorporated under the Companies Act in which the State or one of its agencies has a majority or controlling interest – Canton \citeauthor{Davis2001} (Cabinet Secretary and Head of the public service) (\citeyear{Davis2001}). 

Before we examine the language practices of the Jamaican state, let us review the Government of Jamaica’s seven-year-old communication policy. This document gives the guidelines of conduct for various government entities (ministries, departments, and agencies) particularly during crises and emergency management through communication with citizens. It is the standard in communication policies for mention of specific languages to be used. In a complex sociolinguistic situation such as Jamaica’s, it is unacceptable that Jamaican was mentioned once. Throughout the policy there is a focus on “effective” and “clear” communication: 

The policy places emphasis on the consistent use of clear, understandable language and messaging with respect to communicating information on the policies, programmes, services, and initiatives of the Government while underscoring the need to facilitate uniform and wide appreciation of current issues, strategies, and opportunities. \citep{GovernmentJamaicaCommunicationPolicy2015} 

What is considered “clear understandable language”? Clear and understandable language for which subset of the population? Given Jamaica’s linguistic landscape, this could present a few scenarios:

\begin{enumerate}
    \item Clear and understandable language in English
    \item Clear and understandable language in Jamaican
    \item Clear and understandable language in both English and Jamaican
\end{enumerate}

The phrase “clear and understandable” is vague and its interpretation is left to the respective government agencies. With the complexities of the language situation, such ambiguity should not exist in a national policy outlining how the government should communicate with the public. There are still no concrete guidelines on which language(s) ought to be used. 

In its description of the context and situational analysis of Jamaica’s public communication system, in Section I, the document fails to address the true nature of the language situation. The government cannot continue to deny the usefulness of Jamaican in communicating with citizens, as political representative \citet{Smith2012} states, “We need to declare once and for all that the Patois is one of our languages”. 

National communication policies are usually detailed in specifying the languages the respective government should use and how they should use them. South Africa's \citet{GovernmentCommunicationPolic2018} outlines the official languages to be used. In Section 1.6, in addition to sign language and Braille, it lists the eleven official languages the government is expected to use in its internal and external communication. It stresses that consideration should be given to the linguistic preferences of the public:

\begin{quote}
    All departments must consider the usage, practicality, resources, regional circumstances and the balance of the needs and preferences of the public in deciding on the official language/s to use when communicating. \citep[9]{GovernmentCommunicationPolic2018}
\end{quote}

Further in Section 7.2.4 of the policy, language requirements are outlined with careful consideration of the language situation

\begin{enumerate}
    \item[i.] All communication by government institutions must comply with the Use of Official Languages Act (Act 12 of 2012).
    
    \item[ii.] Different audience segments have different communication needs. All marketing communication must consider the preferred official language of the segmented and target audience. \citep[45]{GovernmentCommunicationPolic2018}
\end{enumerate}

Despite the complex nature of the South African language situation, agencies are expected to observe the public’s communication needs when developing messages for citizens. Linguists researching Jamaica’s sociolinguistic situation have engaged successive administrations about the role of Jamaican in public formal domains. Despite this constant dialogue between government agencies and linguists, there is still no attention paid to how Jamaican should be used by government entities. Matsuura encouraged governments to develop “language policies that enable each linguistic community to use its first language, or mother tongue, as widely and as often as possible” \citep[par. 8]{Matsuura2007}.

While the policy mentions that the “diverse needs of the Jamaican people, whose communication skills and educational backgrounds differ, must also be recognized and accommodated in Government communication” \citep[11]{GovernmentJamaicaCommunicationPolicy2015}, it does not outline exactly what this diversity is and how state entities should accommodate the same. 

Further, Section III of the policy, which discusses issues and challenges, mentions that “cultural diversity needs to be consistently taken into consideration” \citep[13]{GovernmentJamaicaCommunicationPolicy2015}. This is, in fact, the only section in which direct reference (though limited) was made to Jamaican (\citeyear[33]{GovernmentJamaicaCommunicationPolicy2015}):


\begin{quote}
The communication needs of the diverse Jamaican public are not consistently considered with respect to: 
\begin{itemize}
    \item determining the content or presentation of government messaging;
    \item accommodating the use of the Jamaican language and cultural expressions in \emph{certain} official communication activities (e.g., oral and dramatic presentations); or 
    \item the selection of media platforms to which the public has access.
\end{itemize}
\end{quote}

In a national policy on communication, the aim of the government should be to utilize the languages used most by its citizens, not to simply accommodate them. In fact, the phrase “\emph{certain} official communication activities” is also problematic, since this suggests that there are still some contexts in which the use of Jamaican should be deemed inappropriate. Some are of the belief that Jamaican should be restricted to theatrical presentations such as “oral and dramatic” and other similar contexts. Again, the government fails to officially acknowledge the communication needs of Jamaican monolinguals. 

While this national communication policy was never intended to be a language rights document, it had the potential to afford Jamaican speakers the right to be served in a language they understand. The document took a “tolerance” approach to the Jamaican language; while the language is acknowledged, its speakers are not catered for by the state. 


\subsubsection{Language Policy in Jamaica}

In 2001, the Jamaican Language Unit led by Devonish was invited to make a proposal to the then Joint Select Committee that freedom from discrimination on the grounds of language and disability be included in the Charter of Rights \citep{Devonish2001}. The proposal’s introduction outlined the fact that the linguistic distance between Jamaican and English is like that between Spanish and Portuguese in the lexicon; Spanish and French in the phonology; English and German in the morphology and ranging from the distance between French syntax and Spanish to the distance between English and German \citep{Devonish2001}. The proposal then went on to state that both direct and indirect forms of linguistic discrimination exist within Jamaica’s sociolinguistic situation. Accordingly, the provision of services only in English was cited as an indirect form of discrimination and the form of discrimination that emerged in the study by \citet[3]{Linton-PhilpFrench2001} was cited as evidence of direct discrimination. The main thrust of the proposal states that:

\begin{quote}
Following on from Section 1, 24-(8) which currently reads, “In this section, the expression `discriminatory' means affording different treatment to different persons attributable wholly or mainly to their respective descriptions by gender, race, place of origin, social class, political opinions, colour or religion...” should have “language” ... added to the preceding list.   
\end{quote}
 
In its report on the proposal to include provisions against discrimination on the grounds of language in the Charter of Rights, the Joint Select Committee states that “the Committee does not have much difficulty with the matter of constitutional protection against the direct form of discrimination on the ground of language”. \citep[26]{JointSelectCommittee2002}. Committee members expressed concern that most Jamaicans do not view Jamaican as a language in its own right” \citep[5]{Brown-Blake2011} and they also encouraged research to be undertaken in this area. This discussion presents some of the surveys conducted since that recommendation. 


\subsection{Direct and indirect linguistic discrimination}

Linguistic discrimination is linked to people’s negative language attitudes. \citet[2]{Rickford_1983_reprint1985} in describing general attitudes towards Creole languages states that:

\begin{quote}
    The standard view of language attitudes in a Creole continuum is that the standard variety is good, and the non-standard varieties (including the mesolectal and basilectal varieties which are often referred to collectively as “Creole”) are bad. This view may be referred to as the standard one, not only because it is the orthodox one—the one usually reported in academic literature and the local press—but also because it assumes a positive orientation toward the standard variety alone.
\end{quote}

Of relevance to this discussion are the concepts of direct and indirect linguistic discrimination which occurs because of the continued use of English in public formal domains. Direct linguistic discrimination occurs when an individual treats another less favourably or differently based on their linguistic attributes. An example of this is when a service representative (SR) treats a monolingual speaker of Jamaican unfavourably because of their language use. \citet[10]{Walters2017} explains how SRs correcting speakers of Jamaican are similar to teachers correcting students in the classroom:

\begin{quote}
SRs corrected the callers because they did not use the “acceptable” code. One way in which the SRs performed correction was by the use of the phrase “pardon me”. The phrase “pardon me” and its variants “I beg your pardon” and “pardon” served as corrective markers that the SRs used when they attempted to correct the callers’ use of Jamaican.
\end{quote}
    
As \citet[329]{Ferguson1959} states, the importance of using the right variety in the right situation can hardly be overestimated. This practice of using a language that citizens do not understand to communicate with the masses is known as indirect linguistic discrimination. When state authorities offer information in a language other than that of its citizens, indirect linguistic discrimination is the result. This is particularly problematic in Caribbean territories where the standard language is the lexifier language for the Creole. Jamaican is an English-based Creole and for many, they are unable to distinguish between the two while some take it for granted that if you speak Jamaican, you must understand English. 

This common misconception that all Jamaicans speak and understand English, results in the neglect of the communication needs of monolingual speakers of Jamaican. In fact, one-third of the population encounters challenges when communicating in public agencies, where English is the dominant language spoken. The \citet{JamaicanLanguageUnit2006-NationalLangCompetence} revealed that 36.5\% of the population sampled are monolingual speakers of Jamaican, with only 17.1\% demonstrating monolingualism in English. According to the survey, less than half of the population is competent in both languages as 46.4\% demonstrated bilingualism. The implementation of institutional bilingualism will benefit all groups.

\citet{Smalling1983} conducted a comprehension study with adults enrolled in a remedial program known as JAMAL, now known as JFLL, Jamaica Foundation for Life Long Learning. She found that new participants demonstrated only 50\% comprehension of English material. Despite the date and nature of this study, it provides insight into the comprehension levels of monolingual speakers of Jamaican. The question of comprehension will be revisited in the subsequent section.

The state continues to violate the rights of Jamaican speakers, thereby denying them the opportunity to be functional citizens within Jamaican society. For monolinguals, the “Denial of the linguistic rights of the mass of the population means denying them the right to use the only language which they know in order to gain access to these important areas of their society” \citep[18]{Devonish1986}. According to \citet[18]{Phillipson1992}, “The use of one language [in a society] generally implies the exclusion of others, although this is by no means logically necessary”. Additionally, language plays a vital role in the process of production in any society. It is the medium by which production is organized and coordinated whenever more than one producer is involved \citep[16]{Devonish1986}.

\subsection{From traditional to transitional diglossia}

\subsubsection{Traditional diglossia \citep{Ferguson1959}}

Ferguson’s traditional diglossia holds that a superposed High (H) variety has greater prestige than dialects of H, considered the Low (L) varieties. The concept of diglossia expanded to include independent languages and multiple speech communities. An essential feature of diglossia is the functional allocation of the High and Low languages. This type of language situation is relatively stable with the H language usually reserved for more official and formal contexts, while the L variety is used in more informal settings. Traditionally, Jamaican was used in informal contexts and restricted to folklore, theatre and daily conversations. Today, we see the language being used in more formal domains. English has been the language of public formal communication and could be heard on the radio, television, and print media. 

\citet[330]{Ferguson1959} states that “many speakers of a language involved in diglossia characteristically prefer to hear a political speech or an expository lecture or a recitation of poetry in H even though it may be less intelligible to them than it would be in L”. That was the case in Jamaica since people held this view of English, holding it to a higher prestige since it is “the linguistic badge, which one wears when one wants to identify with a certain level of sophistication, of linguistic competence, and of having ‘arrived’ in a highly stratified society" \citep[9]{PollardVelma1994}. These perspectives are changing, as we will discuss throughout this chapter.  


\subsubsection{Transitional diglossia} 

Though diglossia has often been applied to Jamaica’s language situation (\citeauthor{Winford1985} \citeyear{Winford1985}; Devonish 1986) there are newly emerging patterns of language use signaling a \emph{transitional diglossia}. \citet{Walters2016} describes this as an inclusion of Jamaican in all domains of public formal communication, previously reserved for English only. This shift will eventually lead to official institutional bilingualism. A precursor to this transition is an ideological shift in how people viewed the Jamaican language. The catalyst for this could possibly be Jamaica’s performance during the 2008 Olympics, as the athletes excelled in the international competition which created a great sense of national pride. Increasingly, Jamaican monolinguals have been using their mother tongue in domains traditionally reserved for the use of English. These non-traditional domains in which Jamaican is now being used include traditional and social media, the classroom, parliament, and public agencies.

Not only do these practices signal transitional diglossia but a change in language attitudes as well. The results of the National Language Attitude Survey \citep{JamaicanLanguageUnit2005}, discussed in detail below, indicate that there was a shift in the attitudes of Jamaicans regarding the use of Jamaican in public formal domains. Most of the informants indicated that they would like to see Jamaican used in parliament and the classroom (textbooks and language of instruction), road signs, medicine bottles, and pesticides\footnote{The sample was asked if they would like to see Jamaican written in a standard form on the following items: a) road signs, b) school books, c) medicine bottles, d) government forms, e) weed spray.  57.3\% said they would like to see Jamaican written in standard form on school books. \citep{JamaicanLanguageUnit2005}}. This is a shift from the preference of Jamaican in only rum bar and roots play type contexts. 

These attitudes indicate the people’s desire for Jamaican to be elevated into other domains and influence the stakeholders (gatekeepers) to use the language in their respective domains. These gatekeepers include customer service representatives and teachers. 

Today, there is no domain in which Jamaican cannot be found. Gone are the days when only English could be heard on the radio or the television. Apart from daytime talk shows which have traditionally used Jamaican, the language can now be heard on current affairs and news programs such as Beyond the Headlines, CVM Live and the Jamaican News programme “Braadkyaas Jamiekan\footnote{Broadcast Jamaican is a news programme produced by the Jamaican Language Unit.}“ on News Talk 93 FM. From eyewitness reports, vox pop segments and news headlines, Jamaican can be heard. Additionally, service representatives have demonstrated the practice of code-switching to Jamaican when serving Customers who speak Jamaican \citep{Walters2015}. 
 
Jamaican can now be found in all the domains that were reserved for English only. Citizens are proudly using Jamaican in various “sacred” contexts such as valedictory speeches, principal’s addresses to students, and on social media. Public formal communication is far behind, and its practitioners need to incorporate the discourse practices of social media influencers and freelance journalists if they wish to communicate effectively with the masses.

Jamaican monolinguals are now using their language in such formal contexts since any attempt to use English in public contexts often leads to ridicule and mockery. This is the case when dominant Creole speakers try to use English, many tend to hypercorrect and are often subjected to taunting. What is necessary now is for Jamaican to be made an official language alongside English in order to provide equal treatment for monolingual Jamaican speakers. 


\section{The people’s demand for language justice}

\subsection{National language attitude survey}

Most Jamaicans express positive attitudes towards the language, though a subset of the population still rejects its expansion in other domains. In fact, “there still exists a minority within the speech community who still does not recognize JC as a valid code” \citep[225]{DevonishWalters2015}. On the one hand, Jamaicans view their language as a strong part of national and cultural identity, an indication of their rich African heritage and a unique source of pride. On the other hand, a few stigmatize Jamaican and ridicule others for their language use, particularly when they use it in formal domains. Language attitudinal research has shown that such attitudes can be “markedly polarized and tightly held – both institutionally and personally, openly and internally” \citep[58]{Beckford-Wassink1999}. 

The first comprehensive study of language attitudes in Jamaica is the Language Attitude Survey \citep{JamaicanLanguageUnit2005}. The Jamaican Language Unit (JLU) conducted the study within the Western, Central, and Eastern regions of Jamaica. Researchers surveyed a thousand (1,000) Jamaicans in the areas of Language Awareness, Language Use, Language Stereotypes, Education, and Writing in Standard Form. The survey was administered in either English or Jamaican, so questions were asked in one of the two languages depending on the informant's own language use. Regarding language awareness, 79.5\% of the respondents reported that they recognized Jamaican as a language and 68.5\% claimed that they supported Jamaican becoming an official language.

In reference to language use by public officials, 67.8\% indicated that if the Prime Minister and Minister of Finance gave speeches in Jamaican, they would communicate better with the public. In the most recent budget debate, Finance Minister, Dr. Nigel Clarke, incorporated the use of Jamaican in his presentation. He code-switched particularly while demonstrating how to use the new digital currency JAMDEX application. During his presentation\footnote{\url{https://www.youtube.com/watch?v=_0t_feUoz2M}} he declared “Wach mi nou...it get sen....an wach ya nou” “yu kyan piyee yuself!”.\footnote{“Watch me now...it has been sent...and watch here now” “you can pay yourself!”}

Regarding Jamaican in a standard form, 57.3\% indicated that they would like to see Jamaican in textbooks and 49\% agreed they would like to see the language written on road signs. Questions concerning language stereotypes revealed that 57\% and 67\% viewed English speakers as more intelligent and educated than speakers of Jamaican, respectively; despite some indication of change in the views about Jamaican discussed earlier. English maintains its position as the language of prestige, but Jamaican has gained many strides as a prestigious language. \citep{DevonishWalters2015}.

\subsection{Current research: Four language attitude surveys}

I have since collaborated with the Jamaican Language Unit in conducting subsequent quantitative surveys on language use in specific domains. These surveys have been incorporated into the Language Planning research course offered by the Department of Language, Linguistics and Philosophy at the University of the West Indies, Mona. The typical methodological approach has been to:

\begin{itemize}
    \item Select a public formal domain
    \item Develop a questionnaire based on current issues in the selected domains
    \item Conduct a pilot study
    \item Revise questionnaire
    \item Conduct actual survey
    \item Code the data sheets
    \item Enter data in SPSS
    \item Run SPSS analysis
    \item Prepare survey report
    \item Engage stakeholders at public symposium
\end{itemize}

These surveys solicit the views of a wide cross-section of individuals who have their profession, working in a market or an office. We gather information from these respondents in regular everyday settings. 

\subsubsection{The language attitude survey remix}

A follow-up \citet{JamaicanLanguageUnit2015} was conducted to investigate the attitudes towards Jamaican and if there were any major changes. This took place within two regions of Jamaica, Western and Eastern in both rural and urban centers with approximately 900 informants. A key component of the four surveys was a petition for the government to establish laws to end discrimination on the grounds of language. At the end of the questionnaire, the question was presented in this form:

\ea
    {\textit{Language Rights Petition}}\smallskip\\ 
    {\textit{English}}\smallskip\\
    I call on Parliament to pass a law which adds ‘the right to freedom from discrimination on the ground of language’ to the Charter of Rights to the Jamaican Constitution.\medskip\\
    {\textit{Patwa}}\smallskip\\
    Mi waahn Paaliment fi paas wan laa we se ‘piipl no fi get bad chriitment sieka di kain a langwij we dem taak’ an put i ina di Chaata a Rait we de ina di Jamieka kanstichuushan (di big laa we se ou di konchri fi ron).\medskip\\
    \begin{tabular}{@{}ll@{}}
       Signature:  & \rule{7.5cm}{0.5pt}\\ 
       Address:    & \rule{7.5cm}{0.5pt}\\ 
       Date:       & \rule{7.5cm}{0.5pt}\\ 
    \end{tabular}
\z

This was attached to a separate sheet of paper and later included on the questionnaire of the subsequent surveys. The aim here was to see if citizens would be willing to take action in support of their language attitudes. The comparative results of both surveys can be summed up in \tabref{tab:02:1}.

\begin{table}
\begin{tabularx}{\textwidth}{Qrr}
    \lsptoprule
    
                            & LAS 2005   & LAS 2015\\
                            & $n = 1000$ & $n=900$\\
                            & yes (\%) & yes (\%)\\\midrule
    Jamaican is a language in its own right & 79.5 & 77.8\\
    \tablevspace
    Jamaican should be made official alongside English & 68.5 & 69.5\\
    \tablevspace
    A bilingual school in which they teach children to read and write in Jamaican and English is best for the Jamaican child & 71.1 & 70.0\\
    \tablevspace
    Willingness to sign the petition. & -- & 71.3\\
    \lspbottomrule
\end{tabularx}
\caption{Comparison of language attitude surveys 2005 and 2015\label{tab:02:1}}
\end{table}

\begin{sloppypar}
\tabref{tab:02:1} shows that attitudes have been positive towards Jamaican, and they have remained the same over a ten-year period. The majority agrees that Jamaican is a language, parliament should make it official, and that bilingual schools are better for Jamaican children. The consistency in the results also speaks to the veracity of the surveys. The majority of the participants (71.3\%) signed the petition to safeguard against linguistic discrimination in Jamaica’s public agencies.
\end{sloppypar}

In the four language use surveys previously listed in the introduction, most of the respondents have indicated their desire to see Jamaican being used in public formal domains alongside English. Year after year, the results represent the constant demand of Jamaicans for the use of their language to be made official. These surveys capture the heart of Jamaican speakers as informants are from all walks of life across the island. An effort was made to capture the views of residents in both urban and rural areas across a wide range of occupational groups. They serve as a more credible and scientific source than letters written to the editors of major newspapers and those who participate in online discussions about the great 'Patwa Debate'. Many of these ‘debaters’ already have at their disposal competence in the English language. While their voices might seem louder, they do not represent the voice of the common man who stands to benefit the most from an officially bilingual Jamaica. Let us explore the results of these language attitudinal surveys in the next section.

\subsubsection{Language use in the media survey}

We may begin by looking at official government communication using radio, television and electronic media, which mirrors the nature of the distribution of languages in Jamaica’s other public formal domains. Both Jamaican and English share space in the Jamaican mass media. Public practices such as talk show hosts using Jamaican for a combination of “pragmatic purposes and/or acts of identity, and who thereby provide certain legitimacy for the use of Jamaican Creole in public/formal media" \citep[202]{ShieldsBrodber2022}, promote the use of Jamaican in domains usually reserved for English. Since the 1980s, Jamaican has been used in some public service messages and government broadcasts such as skits included in the Jamaica Information Service (JIS) Jamaica Magazine program to “ensure optimal intelligibility” \citep[9]{Akers1981}.

The Jamaican government also uses Jamaican as taglines for advertisements and public education campaigns. For example, \citet[35]{Westphal2010} cites a government advertisement promoting backyard gardening as using the basilectal variety of Jamaican to appeal to citizens who are of the “lower classes”. As is typically the case with mixed language government advertisements and public service announcements, any dialogue between the characters takes place in Jamaican and the official information that refutes or corrects the beliefs of the characters is shared in English. Though Jamaican is used in the public domain, it is often assigned a secondary or an inferior role. 

The anti-litter campaign of the Ministry of Tourism and Entertainment and the Jamaica Environment Trust (JET) launched in 2014 takes a similar approach. This is an example of how the government uses Jamaican to communicate with the masses. It is dubbed the “Nuh Dutty Up Jamaica” campaign and the logo in \figref{fig:Anti-LitterCampaignLogo} uses the unofficial or “chaka-chaka”\footnote{The term “chaka-chaka” is used to indicate that something is disorderly and untidy.} writing system as is often the case with such messages written in Jamaican.  

\begin{figure}
\begin{floatrow}
  \captionsetup{margin=.05\linewidth}%
  \ffigbox{\includegraphics[width=.75\linewidth]{figures/Anti-LitterCampaignLogo.jpg}}
          {\caption{Anti-Litter Campaign Logo \citep{JamaicaEnvironmentTrust2016}\label{fig:Anti-LitterCampaignLogo}}}%
  \ffigbox{\includegraphics[width=.75\linewidth]{figures/Big_up_wi_beach.jpg}}
          {\caption{Big up wi beach \citep{JamaicaEnvironmentTrust2016}\label{fig:Big_up_wi_beach}}}%
\end{floatrow} 
\end{figure}

All other subsequent posters in this campaign included a limited use of Jamaican as illustrated in Figures~\ref{fig:Big_up_wi_beach}--\ref{fig:Love_where_yu_live} and done to catch citizens' attention.


% Figure 2:
\begin{figure}
\begin{floatrow}
  \captionsetup{margin=.05\linewidth}%
  \ffigbox{\includegraphics[width=.75\linewidth]{figures/Nuh_dutty_up_di_road.jpg}}
          {\caption{Nuh dutty up di road \citep{JamaicaEnvironmentTrust2016}\label{fig:Nuh_dutty_up_di road}}}%
  \ffigbox{\includegraphics[width=.75\linewidth]{figures/Love_where_yu_live.jpg}}
          {\caption{Love where yu live \citep{JamaicaEnvironmentTrust2016}\label{fig:Love_where_yu_live}}}%
\end{floatrow}    
\end{figure}

The phrase “big up” is often used in Jamaican when positively acknowledging someone or something, paying homage or respect to a particular target. Note the use of Jamaican “wi” as the possessive plural pronoun. This is a phrase we might hear in everyday conversation “big up yuself” “big up wi konchri” and so on. This resonates with citizens who can easily relate to the language used.

This repeats the slogan “Nuh dutty up” but instead of “Jamaica”, it includes “di road”. Note the use of the Jamaican definite article “di” instead of English “the”. By code-mixing Jamaican markers in English sentences or using short catch phrases in Jamaican, the phrase grabs the attention of the citizens.

For this sentence to be fully Jamaican, “where” would have to become “we” or the more popularly used form “weh”. The limited use of marked words such as “di” and “yu” reflects a mere token usage of Jamaican.

\begin{figure}
    \centering
    \includegraphics[width=0.5\linewidth]{figures/fig5.jpg}
    \caption{Government road sign using elements of Jamaican \citep{Dray2010}\label{fig:fig5}}
\end{figure}

Another example of how the government uses Jamaican to transmit messages is in the form of road signage. \citet{Dray2010} shows how the government uses Creole terms to convey important messages to the public as exemplified in \figref{fig:fig5}. Many SJE speakers now accept some of these Jamaican terms and see them as a part of the mainstream language \citep{Christie2003,Irvine2005}. The phrase “walk good” is from the Jamaican phrase “waak gud” meaning travel safely. As \citet{Dray2010} points out, those responsible for the message on the sign considered the term to be English and not Creole.

Whilst we have seen the token usage of Jamaican, the government’s use of the language of the masses may be what \citet{Fairclough1994} classifies as “conversationalization”. Drawing from \citeauthor{Leech1966}’s (\citeyear{Leech1966}) notes on the “public-colloquial” style of advertising, \citet[242]{Fairclough1994} describes conversationalization as the “...modeling of public discourse upon the discursive practices of ordinary life, conversational practices in a broad sense”. \citet{Fairclough1994} argues that this modeling represents a shifting of the boundaries of public and private discourse conventions, a technique that authorities use to give power to the target group. We also see these conversationalization practices during political campaigns. Political parties use linguistic manipulation of both English and Jamaican in political strategizing \citep{Francis2010}. The Government’s use of Jamaican in public messages, though minimal, is always strategic, using language to give power to and to take it back from the people.

Jamaicans have used their language on several media platforms but not always in any formal way. Jamaican has always been on radio and television and print media. The traditional manner was for Jamaican to be used to discuss lighthearted topics, humorous segments and for dialogue. When the serious topics were to be discussed, English would be used. For instance, ‘Under the Law' is an educational program concerning laws surrounding everyday conflicts between citizens and the applicable laws of the land. The program would entail dialogue in Jamaican, for instance, two neighbours at odds over a broken fence or a fallen tree. After the scenario, the commentator would then point out the aspects of the law relevant to the situation, using English.

Though many of the national radio stations use Jamaican during talk shows, discussion forums and in advertisements, most of their newscasters use English to read the news. This of course is tantamount to linguistic discrimination in its indirect form. 

\begin{table}
\caption{Language use in the media survey results\label{tab:02:3}}
\begin{tabularx}{\textwidth}{Qrrr}
    \lsptoprule
    {Questions} & {Yes} & {No} & {Not sure}\\\midrule
    Should broadcasters use Jamaican on TV and radio? & 58.0\% & 33.4\% & 7.0\%\\
    \tablevspace
    Do speakers of Jamaican fully understand the news in English? & 54.7\% & 35.1\% & 10.2\%\\
    \tablevspace
    Should the government provide information about what is happening in the country in Jamaican on their JIS program? & 72.4\% & 26.7\% & 4\%\\
    \tablevspace
    Willingness to sign the petition & 53.8\% & 46.2\% & -- \\
    \lspbottomrule
\end{tabularx}
\end{table}

In 2017, a “Language Use in the Media” survey was done to ascertain how Jamaicans felt about the use of Jamaican in the Media. Approximately 900 informants were surveyed on the role of language in the media and the effectiveness of the Braadkyaas Jamiekan Nyuuz\footnote{Broadcast Jamaican News} program on News Talk 93 FM. The overall findings revealed that the majority of the respondents were in favour of Jamaican being used on various media platforms.

When asked if news broadcasters should use Jamaican to read the news on TV and Radio, most respondents indicated that they were in favour of this practice as 58\% indicated that they wanted broadcasters to use Jamaican to read the news on TV and radio. If radio stations were to follow suit, it is likely that most Jamaicans would begin to understand news content, thus becoming more informed citizens. Another 33.4\% indicated that they did not wish to hear newscasters read the news in Jamaican and 7\% indicated uncertainty. 

In our earlier discussion of Jamaican speakers fully understanding information shared by the state in English, the question was posed “Do you think citizens who speak Jamaican fully understand the news when it is read in English?” On the matter of comprehension, the majority of those polled indicated that they do not think Jamaican speakers fully understand the news in English. If this is indeed the case, the existing language barrier prevents many monolinguals from understanding vital information shared in the news. This therefore leads to “a differential ability to understand the information being disseminated, particularly in rural Jamaica \citep[42]{Justus1978}. A total of 54.4\% were of the view that speakers of Jamaican do not fully understand the news while 35.1\% felt that Jamaican speakers do understand the news in English. Those who indicated that they were not sure amounted to 10.2\%. 

\subsubsection{Government media broadcasts}

The Jamaica Information Service (JIS) is the agency responsible for government broadcasts and for informing the public of the various ministries’ new policies and projects. It is intended to keep the public aware of the government’s performance and to provide the information they need to access state services. The radio and the television departments within the JIS both transmit information in English. Their \textit{Jamaica Magazine} is aired every day on television and radio. 

The JIS is the official information arm of the Government of Jamaica and is mandated to disseminate information (in different formats and using all available media) that will enhance public awareness and increase knowledge of the government’s policies and programmes. The agency utilizes airtime allowed under the broadcast regulations, for government programming (\citeauthor{GovernmentJamaicaCommunicationPolicy2015} 91, \citeyear{GovernmentJamaicaCommunicationPolicy2015})
When asked if the government should provide information about what is happening in the country in Jamaican on their JIS program, the majority of the participants (72.4\%) responded yes, 26.7\% said no and 4\% said they were not sure. This is of great importance since the JIS is “the voice of Jamaica” and is tasked with sharing valuable information on behalf of the state. Monolingual Jamaican-speaking citizens would receive the opportunity to become more aware of government policies and programs. This would contribute to the development of an informed society. 

A change in language attitudes has resulted in Jamaican monolinguals desirous of being included in the communication of mainstream society. Monolingual Jamaican speakers have not been able to access information from the government in a language that they understand and as a result, they have been marginalized. There is great injustice in the national media system as Jamaicans who do not speak English do not have access to official government information. 

\subsubsection{Language use in public agencies}

It has been established that linguistic discrimination exists in Jamaica's public sector. \citet{Walters2015} found that direct discrimination also exists in Jamaica's public agencies. In a study of sixteen public agencies, she found that Jamaican speakers were twice as likely to receive negative treatment when they telephoned service representatives using Jamaican. The service representatives were more likely to be impolite and unprofessional and the information received was inadequate and, in some cases, inaccurate. Callers reported that when they called requesting information using Jamaican, they felt interrogated, belittled, dismissed, and ridiculed.

Public agencies provide official services on behalf of the government and citizens access their services on a regular basis. These agencies offer state services to citizens which cannot be accessed elsewhere. It is vital that one possesses basic competence in English to access these services. All official documents are published in English, and this poses a challenge to Jamaican monolinguals, as discussed earlier in Section 1. If a Jamaican monolingual wishes to get a Tax Registration Number (TRN) they must overcome the English language barrier in order to be successful. A typical experience of monolinguals is for them to be sent to the security guards for assistance. The security guard functions as interpreter who assists the individual with completing the necessary forms. I went to the National Insurance Scheme (NIS) office once and came upon a security guard assisting a young man complete a form. The security guard hurriedly asked me to continue assisting the young man as he had something else to tend to. In speaking with the young man, I realized that he was a monolingual Jamaican speaker. I had to act as an interpreter by translating from English to Jamaican. For instance, “place of residence” I asked, “we you liv” and so on. 

As \citet{Devonish2001} points out, one would only tend to find Jamaican in written form during role-playing when particular characters are being represented. In these cases, the chaka-chaka\footnote{The unofficial writing system developed overtime by those wishing to express themselves in Jamaican.} writing system is often used to do so. This practice does not greatly contribute to accessing basic services from public agencies unless in rare cases, informational pamphlets containing dialogue are produced for the public in Jamaican. In such cases, English remains the dominant language used for communicating with the public.

\citet{Irvine2004} denounces this inefficiency embedded in the continued functional dominance of English in public formal domains:

\begin{quote}
    It is also inefficient to continue presenting news, parliamentary proceedings, and all the information crucial to a functioning democracy in English only. If both were used, we would stand a better chance of having an informed public.
\end{quote}

English of course, plays an integral role in service delivery in the public sector, while Jamaican “is not the language of serious business” \citep[218]{Irvine2005}. \citet[4]{Brown-Blake2011} points out that such practices demonstrate that the state has adopted an “English monolingual policy”. Among the list of skills usually seen in job advertisements for employment with public agencies, particularly SRs, is having a good command of English. This is evidenced by the requirement of “acceptable passes in the subject in the secondary level qualifying examinations” \citep[32]{Brown-Blake2011}. In her description of language use in a state agency, \citet{Irvine2005} investigated the speech of what she calls “front line workers” at JAMPRO, as representative of what Jamaicans consider to be SJE. In her study, Irvine conducted 20–25-minute interviews with 104 members of staff, 82 of whom agreed to be recorded. Members of the front line staff believed that they were hired in these positions due to their English language competence, and light skin colour among other criteria \citep[271]{Irvine2005}. Though JAMPRO is responsible for promoting brand Jamaica to international investors, the entity represents the ideal language requirements for similar employees of other Jamaican state agencies. Based on Irvine’s findings, “the idealised member of this speech community is one who can manipulate both Creole and English” \citep[271]{Irvine2005}. Now that we have developed a picture of the language situation in the public entity domain, we will now move on to look at the phenomenon under investigation, linguistic discrimination.

The “Language Use in Public Agencies” survey was done in 2016 to ascertain attitudes towards Jamaican being used in government agencies. The majority of the 882 respondents indicated that they wanted customer service to be offered in Jamaican public agencies. The public service encounter is one in which a service representative interacts with customers by providing goods and services or the exchange of information. Whether one wants a driver’s license, a voter’s identification card, or a passport, one must participate in a public service encounter. It is the only means by which many citizens interact with the state and therefore forms a very integral part of the public formal domain. 

\begin{table}	
\caption{Language use in public agencies\label{tab:02:4}}
\begin{tabularx}{\textwidth}{Qccc}
    \lsptoprule
    Questions & Yes & No & Not sure\\\midrule
    Should the service representatives communicate with customers in a language that he/she understands? & 88.6\% & 11.4\% & -- \\
    \tablevspace
    Do Jamaican speakers have difficulty in fully understanding Service Representatives when they use English? & 55.3\% & 39.3\% & 5.4\%\\
    \tablevspace
    If there was a proper way to write Patwa, do you think that the government should provide agency documents in Patwa for Patwa speakers? & 55.7\% & 44.0\% & 0.3\%\\
    \tablevspace
    Should Jamaican be used in parliamentary budget debates? & 62.4\% & 32.7\% & 4.9\%\\
    \tablevspace
    Willingness to sign the petition & 65.5\% & 34.5\% & --\\
    \lspbottomrule
\end{tabularx}
\end{table}

On the matter of the language used by service representatives, when asked ‘Should the service representatives communicate with customers in a language that he/she understands?’, most of the respondents, 88.6\%, indicated that service representatives in public agencies should use the language of the people during service encounters. Only 11.4\% said that service representatives should not use a language the customer understands. 
                
On the matter of comprehension, when asked if they believe that Jamaican speakers have difficulty in fully understanding Service Representatives when they use English, 55.3\% indicated that they thought that Jamaican speakers did not fully understand the service representatives. This can of course mean that monolingual speakers only understand half of what takes place during such encounters. Only 39.3\% indicated that monolinguals have a full understanding of English service encounters and 5.4\% indicated uncertainty. 

Questions focused specifically on the language of the national budget debate as this is important for citizens to evaluate decisions on how the government intends to handle the public purse. Coupled with the use of English is the use of financial jargon which the average English speaker does not understand. Terms such as ‘consolidated fund', 'fiscal policy’, and ‘capital expenditure' have been listed in a glossary of terms on the website for the Jamaica Information Service \citep{JIS2020}. The majority of the respondents were in favour of Jamaican being used in parliamentary budget debates; 62.4\% indicated that the Minister of Finance should use Jamaican when explaining the details of the budget, 32.7\% indicated that the ministers should use English and 4.9\% indicated that they were not sure. 

Regarding written documents, when asked ‘If there was a proper way to write Patwa, do you think that the government should provide agency documents in Patwa for Patwa speakers?’, 55.7\% indicated that government documents should be printed in Jamaican, once there is a suitable writing system for the language. Another 44\% said there should be no written documents in Jamaican and 0.3\% indicated that they were not sure. In the language attitude survey results discussed in Section 2.2.1, the majority of respondents were in favour of Jamaican being written on government forms and road signs. The grassroots citizens have been demanding for their language to be used on official documents. 

Figure 6 shows an example of what a bilingual sign would look like when Jamaican and English are used on signage in public entities. The National Housing Trust offers mortgage loans to Jamaicans who make monthly contributions to the fund. It is one of the most accessible public entities in Jamaica and serves people from all walks of life. Should Jamaica be declared officially bilingual, we propose bilingual signs such as in \figref{fig:Bilingual_NHT}

% Figure 6:
\begin{figure}
    \centering
    \includegraphics[width=0.5\linewidth]{figures/Bilingual_NHT.jpg}
    \caption{Proposed Bilingual sign for the National Housing Trust}
    \label{fig:Bilingual_NHT}
\end{figure}

We thought it pertinent to ask about language use in the Jamaica Information Service (JIS) in this survey since this is the agency that supplies citizens with crucial information on government programmes. When asked ‘If given the opportunity, do you think Patwa speakers should receive government information in Patwa during broadcasts such as the JIS) Magazine program?’, again, the majority was in favour of Jamaican being used in such a domain. A total of 60.7\% believe Jamaican should be used on JIS broadcasts, 33.0\% said no and 6.2\% said not sure. 

Since the JIS communicates issues of national importance, why then isn’t the national language used to relay such information? Though there have been some recent attempts to increase the use of Jamaican, these maintain the same conversatonalization practices discussed in previously. Short skits and dialogues are often done in Jamaican but when they return to the studio, the presenter uses English. Based on the survey results, the respondents are calling for the use of Jamaican in a more meaningful way.

\subsubsection{Language use in the courtroom}

The West Kingston Commission of Enquiry was publicized on national television and on radio. In 2010, the United States issued an extradition request for community kingpin, Christopher 'Dudus' Coke who was on the run. This standoff led to war between the national security forces and citizens in West Kingston. The aftermath left approximately 69 residents murdered and many others injured \citep{Commission_of_Enquiry2016}. A Truth Commission was established in 2014-2016 to determine if police and soldiers were responsible for these deaths. 

\citet{Walters2017} discussed the results of focus groups and a survey that both solicited the attitudes of the average Jamaican but also eight witnesses who were dominant Jamaican speakers. 

\begin{sloppypar}
The results revealed that respondents preferred if lawyers used Jamaican with Jamaican monolinguals who serve as witnesses. This would mean that lawyers would have to formulate their questions in Jamaican and transcribers' notes would also be recorded in the same language. Additionally, respondents felt that interpreters should be provided if lawyers are not competent in Jamaican. While legalese is not easily understood by those outside of the profession, there is merit in translating it to Jamaican so that even a bilingual speaker may gain a greater understanding of the questions being posed.
\end{sloppypar}

\begin{table}
\caption{Language use in the courts\label{tab:02:5}}
\begin{tabularx}{\textwidth}{Qrrr}
    \lsptoprule
    {Questions} & {Yes} & {No} & {Both}\\\midrule
    If witnesses give their statements in Jamaican, should they be written in Jamaican? & 58\% & 37\% & 7\%\\
    \tablevspace
    Should there be interpreters for speakers of Jamaican in the courts? & 65\% & 35\% & --\\
    \tablevspace
    Did the witnesses have difficulty understanding the lawyers' questions? & 78\% & 22\% & -- \\
    \tablevspace
    If the lawyers used	Jamaican, would communicate better with the witnesses? & 77\% & 23\% & --\\
    \lspbottomrule
\end{tabularx}
\end{table}

\citet{Walters2017} highlighted some of what the focus group participants had to say when questioned on the issue of language and comprehension:

\begin{quote}      
    Nikila Brown: De aks wi wan kwestiyan aal sevn taim..an is WAN kwestiyan. Bot yu av tu lisn kierful an den se ‘maam’ ‘sir’ a duohn andastan..bring it dong tu mai levl. Kaaz dem de big, big, big word de mi no an- briek it op in silablz we mi kyan andastan den mi kyan se ‘yes maam’ ‘no maam ‘yes sor’ 
    
    [They ask you one question seven times...and its one question. But you have to listen careful and then say ‘maam’ ‘sir’ I don’t understand..bring it down to my level because those big, big, big words I don’t un- break it up in syllables that we can understand then I can say ‘yes maam’ ‘no maam’ ‘yes sir’] (my translation)
\end{quote}

The witness in giving her overview of the experience in court felt that the lawyers’ repetition of the same questions was unnecessary and that big words should be broken up into syllables to facilitate understanding. It is not so much a breaking up of the words that is necessary but an explanation of what the words actually mean in a language she can understand. The consensus among the residents was that lawyers expected them to answer questions they could not comprehend \citep[33]{Walters2016}.

Nikila’s statement above exemplifies the demand of a people forced to try and understand crucial information being thrown at them in a language in which they lack competence. This is an example of the difficulties speakers of Jamaican face because others assume that they are competent in English. Bilinguals can smoothly transition from one language to another without giving it a thought. This is not the same for Jamaican monolinguals and dominant Jamaican speakers. In fact, research has shown occasions of lack of understanding and overall miscommunication when Jamaican speakers engage with speakers of English and the possible consequences that may derive from such miscommunication \citep{Brown-Blake2007}.

It is not only the females who are demanding language justice, here’s what a male witness had to say about the lawyers’ language use:

\begin{quote}
    “Roshaine Grey: Wan a di biges prablem we mii fain wid di wie ou di laayaz in di inkwaiyeri..di-di dem puoz di kwestiyan tu di rezidens..ai woz wachin fram-fram-fram di staat an stof..an a get tu riyalaiz dat de aar yuuzin dier gud komaan af di Ingglish langwij agens porsnz uu duohn av a muor- duohn av a gud komaan a di Ingglish langwij az wel as dee duu..so de ten tu yuuz dat tu dier advantij..aahm- in di wie ou de puoz di kwestiyan..de puoz it in Standaad Ingglish an in a paatikyula wie tu mek di porzn an di stan luk STUUPID!

    [One of the biggest problems that I found with the way how the lawyers in the enquiry posed the questions to the residents..I was watching from-from-from the start and stuff..and I get to realize that they are using their good command of the English language against persons who don’t have a more – don’t have a good command of the English language as well as they do...so they tend to use that to their advantage..aahm- in the way how they pose the questions...they pose it in Standard English and in a particular way to make the person on the stand look STUPID!] (my translation) \citep[34]{Walters2017}
\end{quote}

As we have seen in the Jamaican case, the public has added its voice to the debate in favour of Jamaican being used in the courtroom. In those same proceedings, some lawyers code-switched to Jamaican when it suited them to do so.  Linguists and language rights advocates should seek to capitalize on this widespread interest in the use of Jamaican in the courts.

\subsection{Comparison of the findings from the four attitudinal surveys}

The findings from the previously discussed surveys are summarized in \tabref{tab:02:2}. It shows the consistency in the positive attitudes people demonstrated towards their language. Every year, since 2005, respondents have indicated that government should declare Jamaican an official language alongside English. This shows that Jamaicans wish to see their language in public formal domains.

\begin{table}
\caption{Twelve-year period of language attitudinal surveys. L: LAS, LR: LAS Remix, CR: Court Room, GM: Government Media, PA: Public Agencies\label{tab:02:2}}
\begin{tabularx}{\textwidth}{Q *5{c}}
    \lsptoprule
    Question & L  & LR  & CR  & GM  & PA\\
             & 2005 & 2015 & 2015 &  2016 &  2017\\
    \midrule
    Should JC be made an official language? & 68.5\% & 69.4\% & 63\% & 73\% & 68.8\%\\
    \tablevspace
    JC speakers do not fully understand in the domain & -- & -- & 78\% & 35.2\% & 53.3\%\\
    \tablevspace
    JC should be used by officials in specific domains &-- &-- & 58\% & 58.4\% & 82\%\\		
    \lspbottomrule
\end{tabularx}
\end{table}

\subsection{Discussion: A consistent message}

The results from each survey all show that informants want to see their language being used in the government offices, the media, and the courtroom. The surveys have given the common man, whether from rural or urban Jamaica, a voice in the “Patois debate”. Those who would not necessarily give their opinion on a social media post or write a letter to the editor were able to contribute to the ongoing discussion.

The message has been the same, regardless of the domain being examined and the year of the survey: the majority of the informants indicated a desire for Jamaican to be made a co-official language along with English. Lawyers and other members of the legal fraternity would have to use Jamaican when questioning speakers of Jamaican or employ the use of interpreters. Forms and recordings would have to be translated to Jamaican and SRs would be mandated to interact with Jamaican speakers using Jamaican. 

On the matter of Jamaican speakers fully understanding messages within the domains, most of the informants in both the legal (78\%) and public agencies (53.3\%) indicated that they did not think Jamaican speakers fully understood the information shared. Only 35\% of informants in the media survey indicated that Jamaican speakers did not fully understand. This could possibly be a result of Jamaican being frequently used on traditional and new media. There now exists a plethora of Jamaican bloggers on YouTube, giving the news in their own language and their own pace, hence the news is at everyone’s disposal.  


\newpage
On the matter of officials using Jamaican, 58\% endorsed this both in the courts and the media. We would therefore see lawyers and broadcasters using Jamaican when carrying out their tasks. For the public agencies, 82\% indicated that they would like to see SRs using Jamaican while interacting with customers. 

Such practices would reduce and eventually eliminate both direct and indirect discrimination, as there would now be institutionalized bilingualism. Both Jamaican and English speakers would be able to choose the language in which they wished to communicate. 


\section{A petition for justice: Addressing the discrepancy between the petition and survey results}

In 2019, the Jamaican Language Unit launched a petition for the Prime Minister to address the issue of Jamaican becoming an official language. Once any petition gains fifteen thousand (15,000) signatures, the Office of the Prime Minister would issue an official statement on the respective issues.  If the attitudes are so positive, why was this not reflected in the petition? One would have expected that the petition results would reflect the positive attitudes from the four surveys discussed. The online petition did not mirror the positive results on the willingness to sign the petition. 

This discrepancy could possibly be attributed to the fact that the petition was done online. Many Jamaicans do not have adequate access to the internet and would not have been able to participate. The online survey required a confirmation email, and most signees did not follow through by checking their emails and submitting their confirmation. Now that internet usage increased during the pandemic, the JLU should relaunch the online petition after observing the recommendations made in Section 5. 

The four face-to-face surveys discussed throughout the article, have managed to capture Jamaicans from three regions and nine different occupational groups, thus giving a more comprehensive representation of those desirous of seeing Jamaican in government institutions.  

\begin{sloppypar}
Though the language petition did not gather the requisite numbers, the Prime Minister addressed the issue when it was posed by youth parliamentarians. Amidst citing concerns on the learning of English, he said that the use of Jamaican in official business is inevitable and that the “institutionalization” of the language “will happen” \citep{NationwideNews2019}. With the public’s attitude already aligned with the notion of institutional bilingualism, this should happen sooner rather than later. As \citeauthor{Smith2012} stated, “We need to regularise and formalise it so that, as of this year, no Jamaican will feel inferior if he or she on the occasion that is necessary speaks Patois” (\citeyear{Smith2012}).
\end{sloppypar}

\section{Recommendations}

The overall positive results of the surveys offer a basis for linguists, language enthusiasts and advocates to continue engaging both the public and government representatives in a more strategic way. The push for Jamaican to be made official should take a bottom-up approach instead of the top-down approach used for several years. Apart from continued research, recommendations for the promoters of institutional bilingualism include:

\begin{enumerate}
    \item Meeting with all Jamaican language enthusiasts, including linguists, public figures, social media influencers, journalists, and educators. Linguists should lead these brainstorming sessions on the role of the key players mentioned in mobilizing the public to engage in a unified “Jamaican Language Movement”.  

    \item Mobilizing speakers of Jamaican by engaging citizens in all parishes, not just in Kingston and St. Andrew. The survey results show that informants in rural parishes tended to display more positive attitudes towards Jamaican becoming institutionalized. Advocates should capitalize on this by engaging monolingual Jamaican speakers through focus groups and town hall meetings. This is the group that stands to benefit the most from official bilingualism and they should therefore be adequately included.

    \item Launching a major fund-raising effort by appealing to non-governmental organizations, seeking grant funding through international organizations such as UNESCO. Members of the public, including the diaspora, should also be invited to become sponsors of the movement. The “Jamaican Language Movement” requires finances to fund the projects suggested below.  

    \item Launching an intense awareness campaign to educate the public on the official writing system of Jamaican, which will help to prepare the citizens for the successful implementation of institutional bilingualism. The JLU has conducted reading and writing Jamaican workshops on the Zoom platform during the pandemic, which were well received by the public. These training workshops should take on a face-to-face format to have a greater reach. 

    \item Targeting ministers of government and heads of public agencies. Language advocates should continue to strategically engage heads of key ministries, such as the Ministry of Education and Youth and the Ministry of Justice. 
\end{enumerate}

These recommendations can serve to make a difference in the call for Jamaican to be made official and the implementation of official bilingualism.

\section{Conclusion}

Most survey informants declared that Jamaican is a language, it should be made official and that it should be used in public formal domains. \citet[24]{ShahSanghavi2017} states that:

\begin{quote}
With the world becoming smaller and coming closer, and with international relations ever increasing, creoles will win the debate and soon become the official languages and medium of instruction in the countries where most of the population uses them as their first language.
\end{quote}

The language attitude surveys signal a demand for Jamaican to be used in public formal domains in a more serious and effective manner. Its speakers are no longer accepting the inconsistent and mediocre way the state uses its language. Monolingual Jamaican speakers deserve the opportunity to be served and to access information in their own language. This is their inalienable right, and the time has come for the Jamaican government to listen to the people and observe these language rights. 

A top-down approach has not worked in mobilizing Jamaicans to seriously advocate for their language to be made official. Language advocates and enthusiasts must engage the public and employ a bottom-up approach to involve the grassroots people in the “Jamaican Language Movement”.

{\sloppy\printbibliography[heading=subbibliography,notkeyword=this]}

\end{document}
