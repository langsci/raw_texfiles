\addchap{Acknowledgments}
 
This grammar is a somewhat revised version of my PhD thesis, which I defended in July 2016. The PhD project formally started in November 2012, but the journey leading to this grammar has been much longer. In 2004, my wife Antje and I went to live in \ili{French} Polynesia with our daughters Mattie and Nina, to assist language groups there with Bible translation work. After learning \ili{Tahitian}, in 2005 I started to study Rapa Nui as well and became involved in checking the Rapa Nui translation of the New Testament. In 2007 we moved to Easter Island and ended up living there for three years. Among other things, I was involved in Bible translation, the edition of educational materials and the elaboration of a lexical database. In the course of time I started to collect observations on the grammar of Rapa Nui. Coming from \ili{French} Polynesia, there was much of interest in a language so similar to \ili{Tahitian}, yet so different in many respects. 

This grammar would not have been possible without the help of many people. First of all I would like to thank Bob (Roberto) and Nancy Weber, who have devoted their lives to the Rapa Nui people and who have done a tremendous amount of work on vernacular education, Bible translation and linguistic research, as well as assisting the Rapa Nui community in anything having to do with the language. They were the ones who invited us to join them on Easter Island, made us feel welcome and helped us in many ways. Their observations, notes and suggestions helped me a great deal to learn to know the language. Over the years and decades, they have collected the texts which have served as corpus for this grammar.

I wish to thank the Rapa Nui translation team and various other people who welcomed us on the island and made their knowledge of the language available: María Eugenia Tuki Pakarati, long-time translator and linguistic worker; Alfredo Tuki Pakarati, who helped us through the visa application; Pai Hiti {\ꞌ}Uira Rano Moai; María Virginia Haoa Cardinali; Victoria Hereveri Tuki; Virginia Atan Tuki.

A big thank you to my supervisors, Lourens de Vries and Ross Clark, who guided me through the process of writing a book that had to meet the requirements of a descriptive grammar and a dissertation; encouraging, correcting, fine-tuning where needed. Thanks also to the members of the reading committee for their willingness to review this dissertation: Aone van Engelenhoven, Ben Hermans, Marian Klamer, Pieter Muysken and Ger Reesink. A special thank you to Marian Klamer and Ger Reesink for their many helpful comments on earlier versions of the manuscript.

I wish to thank Kevin and Mary Salisbury, for their hospitality during my stay in Auckland; Nico Daams, Albert Davletshin, Mary Walworth, Kevin and Mary Salisbury, for our discussions about Polynesian languages; Stephen Levinsohn, for reviewing in detail my analysis of Rapa Nui narrative discourse; Steven Roger Fischer, for clearing up many details of the transcription of Alfred Métraux’ notebooks; Bob and Nancy Weber, Ross Clark, Andrew Pawley and Albert Davletshin, for supplying valuable linguistic resources; René van den Berg, for encouraging me to start this PhD project.

I am grateful to Martin Haspelmath and Sebastian Nordhoff at Language Science Press, who accepted this grammar for publication and carefully coached me through the process of revision and electronic conversion. They never failed to respond promptly to any questions I raised. Thanks also to the anonymous reviewers for their many helpful comments, and the proofreaders who took the time to get this grammar into shape.

Nico and Pam Daams were the people who invited us to join them in Bible translation work in Polynesia. Their friendship, help and encouragement over the past fifteen years have been truly invaluable. 

Finally, I’d like to thank you, Antje, for your support and initiative all these years in so many visible and invisible ways. Without you this grammar would not have been written.

~

Paulus Kieviet

\textit{Alblasserdam, The Netherlands, January 2017}
