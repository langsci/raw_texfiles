\chapter[Closed word classes]{Closed word classes}\label{ch:4}
\section{Introduction}\label{sec:4.1}

As indicated in \sectref{sec:3.1}, there is a basic distinction in Rapa Nui between full words (notably nouns and verbs) and particles. The previous chapter dealt with word classes that are clearly full words: nouns and verbs and subtypes thereof. This section discusses word classes which have at least some characteristics of particles: they form closed classes and do not have a lexical meaning. All of these, except personal pronouns\is{Pronoun!personal}, occur in the periphery of the noun and/or verb phrase.

However, most of these words also share characteristics of full words. Numerals and (occasionally) demonstratives\is{Demonstrative} may also be a clause constituent. Pronouns and numerals, and to a lesser extent quantifiers\is{Quantifier} and adverbs\is{Adverb} as well, may form phrases containing pre- and or postnuclear particles\is{Particle!postnuclear}. 

\tabref{tab:17} lists these word classes in roughly descending order of full word status:\footnote{\label{fn:159}Other word classes are not discussed in this chapter, but in sections corresponding to their respective functions. This concerns negators (\sectref{sec:10.5}), the \isi{polar question} marker (\sectref{sec:10.3.1}), coordinating conjunctions (\sectref{sec:11.2}), preverbal subordinators (\sectref{sec:11.5}) and subordinating conjunctions (\sectref{sec:11.6}). Yet other words are particles occurring in fixed positions in the \isi{noun phrase} and the verb phrase; these are discussed in Chapters 5 and 7: determiners (\sectref{sec:5.3}), the proper article (\sectref{sec:5.9}), the collective marker (\sectref{sec:5.2}), plural markers (\sectref{sec:5.5}), the identity marker (\sectref{sec:5.13.2}), \isi{aspect} markers (\sectref{sec:7.2}), other preverbal particles (\sectref{sec:7.3}), evaluative markers (\sectref{sec:7.4}), directionals (\sectref{sec:7.5}) and the continuity marker (\sectref{sec:7.2.5.5}).}

\begin{table}
\begin{tabularx}{\textwidth}{XZ{9mm}Z{13mm}Z{18mm}Z{13mm}Z{16mm}}
\lsptoprule
& § & closed class & clause constituent & phrase head & NP/VP \newline periphery\\
\midrule
{personal pronouns\is{Pronoun!personal}} & \ref{sec:4.2.1}& x& x& x& \\
{possess. pronouns\is{Pronoun!possessive}} & \ref{sec:4.2.2}& x& x& x& x\\
numerals & \ref{sec:4.3}& x& x& x& x\\
{quantifiers\is{Quantifier}} & \ref{sec:4.4}& x&  & (x)& x\\
{adverbs\is{Adverb}} & \ref{sec:4.5}& x& (x)& (x)& x\\
{demonstratives\is{Demonstrative}} & \ref{sec:4.6}& x& (x)&  & x\\
{prepositions} & \ref{sec:4.7}& x&  &  & x\\
\lspbottomrule
\end{tabularx}
\caption{Closed word classes}
\label{tab:17}
\end{table}

\section{Pronouns}\label{sec:4.2}

Rapa Nui has a set of personal pronouns\is{Pronoun!personal}, two sets of possessive pronouns\is{Pronoun!possessive} and a set of benefactive pronouns\is{Pronoun!benefactive}. \sectref{sec:4.2.1} discusses personal pronouns\is{Pronoun!personal}; \sectref{sec:4.2.2} discusses possessive pronouns\is{Pronoun!possessive}; \sectref{sec:4.2.3} lists benefactive pronouns\is{Pronoun!benefactive}. Finally, \sectref{sec:4.2.4} discusses a few marked uses of pronouns.

NB Demonstrative particles are used as pronouns in limited contexts; this is discussed in \sectref{sec:4.6.6}.

\subsection{Personal pronouns}\label{sec:4.2.1}
\is{Pronoun!personal}\subsubsection[Forms]{Forms}\label{sec:4.2.1.1}

Personal pronouns\is{Pronoun!personal} are inflected for number (singular, dual, plural), person, and inclusiveness. The forms are given in the \tabref{tab:18}.\footnote{\label{fn:160}Apart from the loss of plural forms, the personal pronouns\is{Pronoun!personal} were inherited from \is{Eastern Polynesian}PEP without any changes (see the reconstructed forms by \citealt[98]{Wilson1985}); the singular forms are even unchanged from the \is{Proto-Polynesian}PPN forms as reconstructed by \citet[168]{Kikusawa2003}. Ultimately, the dual and plural forms go back to Proto-Oceanic, where the dual\is{Dual} forms had a suffix \textit{*-dua} ‘two’ and the plural forms a suffix \textit{*-tolu} ‘three’ \citep[37]{Pawley1972}.}

\begin{table}
\begin{tabularx}{.75\textwidth}{XZ{17mm}Z{12mm}Z{13mm}}
\lsptoprule
 & {singular}& {dual}\is{Dual}& {plural}\\
\midrule
1\textsuperscript{st} person inclusive & \textit{au}\is{au ‘1sg’}& \textit{tāua}\is{taua ‘1du.incl.’@tāua ‘1du.incl.’}& \textit{tātou}\is{tatou ‘1pl.incl.’@tātou ‘1pl.incl.’}\\
1\textsuperscript{st} person exclusive &  & \textit{māua}\is{maua ‘1du.excl.’@māua ‘1du.excl.’}& \textit{mātou}\is{matou ‘1pl.excl.’@mātou ‘1pl.excl.’}\\
2\textsuperscript{nd} person & \textit{koe}\is{koe ‘2sg’}& \multicolumn{2}{c}{\textit{kōrua}\is{korua ‘2du/pl’@kōrua ‘2du/pl’}

}\\
3\textsuperscript{rd} person & \textit{ia}\is{ia ‘3sg’}& \multicolumn{2}{c}{\textit{rāua}\is{raua ‘3du/pl’@rāua ‘3du/pl’}

}\\
\lspbottomrule
\end{tabularx}
\caption{Personal pronouns}
\label{tab:18}
\end{table}

The inclusive\is{Inclusive} forms indicate that the addressee is included in the group referred to by the pronouns: \textit{tāua} ‘you and me’, \textit{tātou} ‘we all, including you’. The exclusive\is{Exclusive} forms indicate that the addressee is not part of the group referred to: \textit{māua} ‘the two of us (but not you)’, \textit{mātou} ‘we (excluding you)’.

Most Polynesian languages have distinct dual\is{Dual} and plural pronouns in all persons. As \tabref{tab:18} shows, in Rapa Nui dual and plural are only distinguished in the first person. In the second and third person, the original dual forms \textit{kōrua} and \textit{rāua} extended their use to plural, while the \is{Eastern Polynesian}PEP plural forms \textit{*kōtou} and \textit{*rātou} were lost.\footnote{\label{fn:161}According to \citet[55]{Bergmann1963}, in some other Polynesian languages, dual pronouns have extended uses. Thus in \ili{Tongan}, the first person inclusive dual is often used with a plural sense (\citealt[124–125]{Churchward1953}). A similar process in Rapa Nui may have led to the extension in use of dual pronouns to include plurality, eventually superseding the original plural forms.}  

Personal pronouns\is{Pronoun!personal} tend to be used for animate\is{Animacy} referents only: humans and animals. Note however that possessive pronouns\is{Pronoun!possessive} can be used for inanimates as well. Here is an illustration from a description of a palm tree:

\ea\label{ex:4.1}
\gll Tumu nei e ai rō {\ꞌ}ā \textbf{tō{\ꞌ}ona} maŋa roaroa.\\
tree \textsc{prox} \textsc{ipfv} exist \textsc{emph} \textsc{cont} \textsc{poss.3sg.o} branch long:\textsc{red}\\

\glt
‘This tree has long branches (lit. there are its long branches).’ \textstyleExampleref{[R483.005]} 
\z

In the same text, personal pronouns\is{Pronoun!personal} are avoided to refer to the tree; full noun phrases are used instead:

\ea\label{ex:4.2}
\gll Tā{\ꞌ}aku aŋa he manava mate ki \textbf{te} \textbf{tumu} \textbf{nei}. E ai rō {\ꞌ}ana te maŋa pakapaka o \textbf{te} \textbf{niu} \textbf{nei}.\\
\textsc{poss.1sg.a} do \textsc{pred} stomach die to \textsc{art} tree \textsc{prox} \textsc{ipfv} exist \textsc{emph} \textsc{cont} \textsc{art} branch dry:\textsc{red} of \textsc{art} palm\_tree \textsc{prox}\\

\glt 
‘I always like this tree. This palm tree has dry branches.’ \textstyleExampleref{[R483.009–010]}
\z

Personal pronouns\is{Pronoun!personal} belong to the class of proper nouns\is{Noun!proper}. This means that in many syntactic contexts, they are preceded by the proper article\is{a (proper article)} \textit{a} (\sectref{sec:5.13.2.1}). 

\subsection{Possessive pronouns}\label{sec:4.2.2}

Rapa Nui has two sets of possessive pronouns\is{Pronoun!possessive}.\footnote{\label{fn:162}Following \citet[182]{Dryer2007Noun}, I use the term \textit{possessive pronoun} for any pronominal possessive form, whether used as a constituent on its own (\ili{English} ‘mine’, ‘yours’) or as a modifier within the \isi{noun phrase} (\ili{English} ‘my’, ‘your’). The latter are often called possessive adjectives, reserving the term possessive pronoun for independent forms which can function as nominal \isi{complement} or predicate. In Rapa Nui, the difference between the two sets of possessive forms does not correspond to the difference between so-called possessive adjectives and possessive pronouns\is{Pronoun!possessive}. Both can, for example, occur as modifier before the noun (\sectref{sec:6.2.1}). Moreover, the term \textit{possessive adjective} would not be entirely satisfactory for Rapa Nui, as possessors do not occur in the same position in the \isi{noun phrase} as adjectives (see the position chart in \sectref{sec:5.1}).} One set is based on the article \textit{te} and starts with \textit{t-}; I will call this series \textit{t-possessives}\is{Pronoun!possessive!t-class}. The other set, which does not start with \textit{t\nobreakdash-}, will be called \textit{zero possessives} (\textit{Ø-possessives}\is{Pronoun!possessive!Ø-class}). In addition, the singular pronouns in each set exhibit a distinction between \textit{o}{}- and \textit{a}{}-forms. This results in four forms, for example in the first person singular:

\ea
 \textit{tō{\ꞌ}oku  ~ ~   tā{\ꞌ}aku  ~ ~   ō{\ꞌ}oku   ~ ~ {\ꞌ}ā{\ꞌ}aku   }  ‘my, mine’
\z

The \textit{t}{}-possessives are discussed in \sectref{sec:4.2.2.1}, the Ø-possessives in \sectref{sec:4.2.2.2}.

In this chapter, only the forms of possessive pronouns\is{Pronoun!possessive} are given. Their use is discussed – together with possessive phrases in general – in Chapter 6 on \isi{possession}.

\subsubsection[T{}-possessives]{\textit{t}{}-possessives}\label{sec:4.2.2.1}
\is{Pronoun!possessive!t-class|(}\paragraph{Singular possessors}\label{sec:4.2.2.1.1}

\is{Possession!o/a distinction}In the singular, there are two classes of possessive pronouns\is{Pronoun!possessive}, characterised by the use of \textit{o} and \textit{a,} respectively. These classes indicate different types of relationships between \isi{possessor}\is{Possession} and \isi{possessee}; the issue of \textit{o-} and \textit{a}{}-\isi{possession} is discussed in \sectref{sec:6.3.3}.

The singular \textit{t}{}-possessives\is{Pronoun!possessive!t-class} are given in \tabref{tab:19}.

\begin{table}
\begin{tabularx}{.75\textwidth}{XL{30mm}L{30mm}} 
\lsptoprule
& {\textit{o}{}-class} & {\textit{a}{}-class}\\
\midrule
1 singular & \textit{tō{\ꞌ}oku}\is{tooku ‘my’@tō{\ꞌ}oku ‘my’} & \textit{tā{\ꞌ}aku}\is{taaku ‘my’@tā{\ꞌ}aku ‘my’}\\
2\textsuperscript{} singular & \textit{tō{\ꞌ}ou}\is{toou ‘your’@tō{\ꞌ}ou ‘your’}\textit{, tu{\ꞌ}u}\is{tuu ‘your’@tu{\ꞌ}u ‘your’}\textit{, to{\ꞌ}u}\is{tou ‘your’@to{\ꞌ}u ‘your’} & \textit{tā{\ꞌ}au}\is{taau ‘your’@tā{\ꞌ}au ‘your’}\textit{, ta{\ꞌ}a}\is{taa ‘your’@ta{\ꞌ}a ‘your’}\textit{, ta{\ꞌ}u}\is{tau ‘your’@ta{\ꞌ}u ‘your’}\\
3\textsuperscript{} singular & {\textit{tō{\ꞌ}ona}\is{toona ‘his/her’@tō{\ꞌ}ona ‘his/her’}} & {\textit{tā{\ꞌ}ana}\is{taana ‘his/her’@tā{\ꞌ}ana ‘his/her’}}\\
\lspbottomrule
\end{tabularx}
\caption{Singular t-possessive pronouns}
\label{tab:19}
\end{table}

\textit{Tu{\ꞌ}u} and \textit{to{\ꞌ}u} are shortened forms of \textit{tō{\ꞌ}ou}; \textit{ta{\ꞌ}a} and \textit{ta{\ꞌ}u} are shortened forms of \textit{tā{\ꞌ}au}.\footnote{\label{fn:163}According to \citet[13]{MulloyRapu1977}, \textit{ta{\ꞌ}a} and \textit{tu{\ꞌ}u} “demonstrate a relatively recent sound change” from the older forms \textit{tā{\ꞌ}au} and \textit{tō{\ꞌ}ou}. Note however that, while the shortened forms are indeed relatively rare in older texts, they do occur in MsE and Ley (though not in Mtx).}  

In older texts the short forms are rare; the long forms are used exclusively in all contexts:

\begin{itemize}
\item 
in the \isi{noun phrase}, before the noun (\textit{tō{\ꞌ}ou matu{\ꞌ}a} ‘your parent’, \textit{tā{\ꞌ}au poki} ‘your child’, \sectref{sec:6.2.1})

\item 
in verbless possessive clauses\is{Clause!possessive}, a construction now obsolete (\sectref{sec:9.3.3}). 

\end{itemize}

Nowadays, when \textit{t}{}-pronouns are used in the \isi{noun phrase}, only the shorter forms are used (\textit{tu{\ꞌ}u matu{\ꞌ}a} ‘your parent’, \textit{ta{\ꞌ}a poki} ‘your child’, \sectref{sec:6.2.1}). The long forms are only used nowadays in the partitive\is{Partitive} construction \textit{Poss o te N} (\sectref{sec:6.2.2}).

\paragraph[Plural possessors]{Plural possessors}\label{sec:4.2.2.1.2}

In the plural, \textit{a} and \textit{o} forms are not distinguished. Even so, there are two series of \textit{t-}possessive\is{Pronoun!possessive!t-class} pronouns\is{Pronoun!possessive}: one with \textit{to}\is{to (possessive prep.)}, one with \textit{te}\is{te (article)}. Their forms are given in \tabref{tab:20}.

\begin{table}
\begin{tabularx}{.75\textwidth}{L{35mm}L{25mm}L{25mm}} 
\lsptoprule
& \textit{to}-series & \textit{te}-series\\
\midrule
1 dual inclusive & \textit{to tāua} & \textit{te tāua}\\
1 dual exclusive & \textit{to māua} & \textit{te māua}\\
1 plural inclusive & \textit{to tātou} & \textit{te tātou}\\
1 plural exclusive & \textit{to mātou} & \textit{te mātou}\\
2\textsuperscript{} dual/plural & \textit{to kōrua} & \textit{te kōrua}\\
3 dual/plural & {\textit{to rāua}} & {\textit{te rāua}}\\
\lspbottomrule
\end{tabularx}
\caption{Plural t-possessive pronouns}
\label{tab:20}
\end{table}

There is no difference in meaning between the two series. The \textit{to}\is{to (possessive prep.)}{}-series is older; it is still used occasionally nowadays, but has an archaic ring to it. The \textit{te}{}-series is found occasionally in older texts (17x), but \textit{to} is predominant in these texts (176x).\footnote{\label{fn:164}12 of the 17 \textit{te}{}-forms in old texts are \textit{te kōrua} in Mtx; \textit{to kōrua} is only used once in Mtx. This may suggest that the change \textit{to}~{\textgreater}~\textit{te} started off as dissimilation before \textit{o} (\textit{kōrua} is the only plural pronoun with \textit{o} as first vowel); subsequently this was generalised to all pronouns.
In any case, the data show that the \textit{te}{}-possessives are a recent innovation, not a retention from \is{Eastern Polynesian}PEP as suggested by \citet[105–106]{Wilson1985}; \citet[298]{Wilson2012}.} In newer texts, \textit{te}\is{te (article)} is predominant: there are 127 \textit{to}{}-forms against 1314 \textit{te}{}-forms.
\is{Pronoun!possessive!t-class|)}
\subsubsection[Ø{}-possessives]{Ø-possessives}\label{sec:4.2.2.2}
\is{Pronoun!possessive!Ø-class|(}
\is{Possession!o/a distinction}The singular Ø-possessives\is{Pronoun!possessive!Ø-class} are listed in \tabref{tab:21}. They have the same form as the \textit{t}{}-possessives\is{Pronoun!possessive!t-class}, minus the initial \textit{t\nobreakdash-}. The \textit{a}{}-forms are spelled with an initial glottal\is{Glottal plosive}, just like the possessive \isi{preposition} \textit{{\ꞌ}a}\is{a (possessive prep.)@{\ꞌ}a (possessive prep.)} (\sectref{sec:2.2.5}).

\begin{table}
\begin{tabularx}{.75\textwidth}{L{25mm}L{30mm}L{30mm}} 
\lsptoprule	
& {\textit{o}{}-class} & {\textit{a}{}-class}\\
\midrule
1\textsuperscript{} singular & {\textit{ō{\ꞌ}oku}\is{ooku ‘my’@ō{\ꞌ}oku ‘my’}} & {\textit{{\ꞌ}ā{\ꞌ}aku}\is{aaku ‘my’@{\ꞌ}ā{\ꞌ}aku ‘my’}}\\
2\textsuperscript{} singular & {\textit{ō{\ꞌ}ou}\is{oou ‘your’@ō{\ꞌ}ou ‘your’}\textit{, u{\ꞌ}u}\is{uu ‘your’@u{\ꞌ}u ‘your’}\textit{, o{\ꞌ}u}\is{o{\ꞌ}u ‘your’}} & {\textit{{\ꞌ}ā{\ꞌ}au}\is{aau ‘your’@{\ꞌ}ā{\ꞌ}au ‘your’}\textit{, a{\ꞌ}a}\is{aa@a{\ꞌ}a ‘your’}\textit{, a{\ꞌ}u}\is{au@a{\ꞌ}u ‘your’}}\\
3\textsuperscript{} singular & {\textit{ō{\ꞌ}ona}\is{oona ‘his/her’@ō{\ꞌ}ona ‘his/her’}} & {\textit{{\ꞌ}ā{\ꞌ}ana}\is{aana ‘his/her’@{\ꞌ}ā{\ꞌ}ana ‘his/her’}}\\
\lspbottomrule
\end{tabularx}
\caption{Singular Ø-possessive pronouns}
\label{tab:21}
\end{table}

As with the \textit{t}{}-possessives\is{Pronoun!possessive!t-class}, there are shortened forms in the 2\textsuperscript{nd} person singular: \textit{u{\ꞌ}u} and \textit{o{\ꞌ}u} are shortened forms of \textit{ō{\ꞌ}ou}, \textit{a{\ꞌ}a} and \textit{a{\ꞌ}u} are shortened forms of \textit{{\ꞌ}ā{\ꞌ}au}. There is no difference in meaning between the longer and the shorter forms.

The plural forms are given in \tabref{tab:22}, t. In the plural, the Ø-possessives\is{Pronoun!possessive!Ø-class} are identical to the personal pronouns\is{Pronoun!personal} preceded by the genitive \isi{preposition} \textit{o}, as in a genitive \isi{noun phrase}. As with the \textit{t}{}-possessives\is{Pronoun!possessive!t-class}, the plural pronouns do not make a distinction between \textit{a} and \textit{o}{}-\isi{possession}.

\begin{table}
\begin{tabularx}{.5\textwidth}{XL{22mm}}
\lsptoprule
1 dual inclusive & {\textit{o tāua}}\\
1 dual exclusive & {\textit{o māua}}\\
1 plural inclusive & {\textit{o tātou}}\\
1 plural exclusive & {\textit{o mātou}}\\
2\textsuperscript{} dual/plural & {\textit{o kōrua}}\\
3 dual/plural & {\textit{o rāua}}\\
\lspbottomrule
\end{tabularx}
\caption{Plural Ø-possessive pronouns}
\label{tab:22}
\end{table}
\is{Pronoun!possessive!Ø-class|)}

\subsection{Benefactive pronouns}\label{sec:4.2.3}
\is{Pronoun!benefactive|(}
Benefactive pronouns\is{Pronoun!benefactive} express that something is destined/intended for the person in question. They are identical to the \textit{t}\nobreakdash-possessive\is{Pronoun!possessive!t-class} pronouns\is{Pronoun!possessive} (for the second person singular, the long form is used), but with an initial \textit{m-} instead of \textit{t-}. \is{Possession!o/a distinction}As with possessive pronouns\is{Pronoun!possessive}, there is an \textit{o}/\textit{a} distinction in the singular, but not in dual and plural. The forms are listed in \tabref{tab:23}.

\begin{table}
\begin{tabularx}{.66\textwidth}{XL{20mm}L{20mm}}
\lsptoprule
 & {\textit{o}{}-class} & {\textit{a}{}-class}\\
\midrule
1\textsuperscript{} singular & \textit{mō{\ꞌ}oku} & \textit{mā{\ꞌ}aku}\\
2\textsuperscript{} singular & \textit{mō{\ꞌ}ou} & \textit{mā{\ꞌ}au}\\
3\textsuperscript{} singular & {\textit{mō{\ꞌ}ona}} & {\textit{mā{\ꞌ}ana}}\\
1 dual inclusive & \textit{mo}\is{mo (benefactive prep.)} \textit{tāua} & { \textit{–}}\\
1 dual exclusive & \textit{mo māua} & { \textit{–}}\\
1 plural inclusive & \textit{mo tātou} & { \textit{–}}\\
1 plural exclusive & \textit{mo mātou} & { \textit{–}}\\
2\textsuperscript{} dual/plural & \textit{mo kōrua} & { \textit{–}}\\
3 dual/plural & {\textit{mo rāua}} & { \textit{–}}\\
\lspbottomrule
\end{tabularx}
\caption{Benefactive pronouns}
\label{tab:23}
\end{table}

Benefactive pronouns\is{Pronoun!benefactive} are the pronominal counterpart of the \isi{preposition} \textit{mo/mā} + NP, and have the same uses. The use of these prepositions is discussed in \sectref{sec:4.7.7}.
\is{Pronoun!benefactive|)}

\subsection{Uses of pronouns}\label{sec:4.2.4}

Personal pronouns are used in the same contexts as nouns: as subjects, objects, after prepositions et cetera. The uses of possessive pronouns will be discussed in more detail in Chapter 6. In this section, a few nonstandard uses of pronouns are discussed.

\subsubsection[Generic pronouns: ‘one’]{Generic pronouns: ‘one’}\label{sec:4.2.4.1}

As in many languages, the second person singular pronoun \textit{koe}\is{koe ‘2sg’} can be used in a generic way, referring to persons in general. 

\ea\label{ex:4.3}
\gll E ko takera e \textbf{koe} e noho {\ꞌ}ana, e riri {\ꞌ}ana {\ꞌ}o e tātake {\ꞌ}ana... \\
\textsc{ipfv} \textsc{neg.ipfv} see \textsc{ag} \textsc{2sg} \textsc{ipfv} sit \textsc{cont} \textsc{ipfv} angry \textsc{cont} or \textsc{ipfv} argue \textsc{cont} \\

\glt 
‘(describing someone’s character:) You would never see him angry or arguing...’ \textstyleExampleref{[R302.050]} 
\z

\ea\label{ex:4.4}
\gll E ri{\ꞌ}ari{\ꞌ}a nō \textbf{koe} {\ꞌ}i te kai ka hiko era. \\
\textsc{ipfv} afraid just \textsc{2sg} at \textsc{art} food \textsc{cntg} snatch \textsc{dist} \\

\glt
‘People (lit. you) were afraid because she would snatch the food away.’ \textstyleExampleref{[R368.104]} 
\z

\textit{Koe} as a generic pronoun can have a distributive\is{Distributive} sense: ‘each one, every one’. In the following example this is reinforced by the repeated \textit{te kope era} ‘that person’:

\ea\label{ex:4.5}
\gll He oho te taŋata, he to{\ꞌ}o mai \textbf{koe} i \textbf{tā{\ꞌ}au} vi{\ꞌ}e, te kope era, te kope era hoki ananake.\\
\textsc{ntr} go \textsc{art} man \textsc{ntr} take hither \textsc{2sg} \textsc{acc} \textsc{poss.2sg.a} woman \textsc{art} person \textsc{dist} \textsc{art} person \textsc{dist} also together\\

\glt
‘The men came, every one took a woman for himself, each and every young man.’ \textstyleExampleref{[Mtx-3-01.285]}
\z

This example also shows that possessive pronouns (here \textit{tā{\ꞌ}au} ‘your’) may have a generic sense as well. 

\subsubsection[Second person pronouns of personal involvement]{Second person pronouns of personal involvement}\label{sec:4.2.4.2}

There is yet another use of the second person singular personal and possessive pronouns, which could be labeled “personal involvement”. Even though no participant in the discourse is an addressee, someone – either a participant in the story or the hearer – is addressed directly, to communicate a degree of personal or emotional involvement from the part of the speaker.

Firstly: sometimes a participant in a \isi{narrative text} is referred to as \textit{koe}\is{koe ‘2sg’}, followed by a vocative\is{Vocative} phrase (\sectref{sec:8.11} on the vocative). The narrative is in the third person, i.e. no addressee is involved as a participant; yet the speaker is, as it were, addressing the participant:\footnote{\label{fn:165}\citet[400]{Fedorova1965} gives examples of this same construction in Mss. A and C (see Footnote \ref{fn:27} on p.~\pageref{fn:27}): \textit{koe e ... ē}, calling it “the article circumfix”.}

\ea\label{ex:4.6}
\gll He {\ꞌ}ara mai \textbf{koe} \textbf{e} \textbf{Tahoŋa} \textbf{ē} koia ko koa. \\
\textsc{ntr} wake\_up hither \textsc{2sg} \textsc{voc} Tahonga \textsc{voc} together \textsc{prom} happy \\

\glt 
‘Tahonga (lit. ‘you, O Tahonga’) woke up happy.’ \textstyleExampleref{[R301.351]} 
\z

\ea\label{ex:4.7}
\gll He tu{\ꞌ}u \textbf{koe} \textbf{e} \textbf{te} \textbf{korohu{\ꞌ}a} \textbf{nei} \textbf{ē} {\ꞌ}i ruŋa i tō{\ꞌ}ona hoi pakiroki. \\
\textsc{ntr} arrive \textsc{2sg} \textsc{voc} \textsc{art} old\_man \textsc{prox} \textsc{voc} at above at \textsc{poss.3sg.o} horse thin \\

\glt
‘The old man arrived on his skinny horse.’ \textstyleExampleref{[R363.017]} 
\z

As \REF{ex:4.7} shows, even when the participant is “addressed” in this way, for all other purposes it is still a third-person participant (\textit{tō{\ꞌ}ona hoi}, ‘his horse’).

Sometimes the pronoun could be paraphrased as ‘that dear one’, but in many cases its exact connotation is hard to convey in translation.

Secondly: the second person singular possessive pronouns\is{Pronoun!possessive} \textit{ta{\ꞌ}a}\is{taa ‘your’@ta{\ꞌ}a ‘your’} and \textit{ta{\ꞌ}u}\is{tau ‘your’@ta{\ꞌ}u ‘your’} (\sectref{sec:4.2.2.1.1}) can be used without a real possessive meaning.\footnote{\label{fn:166}This use is also noted by \citet[21]{Englert1978}, who distinguishes “\textit{taa} y \textit{taau} como artículos” (\textit{taa} and \textit{taau} as articles) from “el pronombre posesivo \textit{taau}” (the possessive pronoun \textit{taau}), and \citet[48]{Bergmann1963}.} This happens both in conversation and in third-person contexts. In conversation, they are used to imply that the noun is in some loose way connected to the hearer: ‘your thing’, i.e. the thing you were referring to, or the thing you asked about, or the thing that is of interest to you.

In \REF{ex:4.8}, two people are discussing a photograph. One of them points out a woman they both know:

\ea\label{ex:4.8}
\gll —{\ꞌ}Ai \textbf{ta{\ꞌ}u} vi{\ꞌ}e ko Eva. —{\ꞌ}Ai te rū{\ꞌ}au era ko Eva. \\
~~~~there \textsc{poss.2sg.a} woman \textsc{prom} Eva ~~~~there \textsc{art} old\_woman \textsc{dist} \textsc{prom} Eva \\

\glt
‘—Here is the (lit. ‘your’) woman Eva. —(Indeed), here is the old woman Eva.’ \textstyleExampleref{[R416.461–462]}
\z

The same use of possessive pronouns\is{Pronoun!possessive} is also found in narrative contexts where no second-person participant is involved. By using a second person pronoun the speaker is, as it were, addressing the listener, implying that the object or person under discussion is in some way relevant to him/her. One could say that the listener is made part of the story, a strategy which makes the story more vivid. One function of the pronoun in this construction is stressing familiarity: the person or object is already known to the listener, whether from the preceding text or from general knowledge. \textit{Ta{\ꞌ}a}/\textit{ta{\ꞌ}u} could thus be paraphrased as ‘the one you know’. 

\ea\label{ex:4.9}
\gll He to{\ꞌ}o mai \textbf{ta{\ꞌ}a} \textbf{ika} he totoi ki raro ki tou rua era. \\
\textsc{ntr} take hither \textsc{poss.2sg.a} fish \textsc{ntr} \textsc{red}:drag to below to \textsc{dem} hole \textsc{dist} \\

\glt 
‘They took that (lit. your) victim and dragged her down into the pit.’ \textstyleExampleref{[R368.099]} 
\z

\ea\label{ex:4.10}
\gll {\ꞌ}Ina mau ena \textbf{ta{\ꞌ}a} \textbf{hahau} tokerau o{\ꞌ}o atu a roto i te avaava  o te hare.\\
\textsc{neg} really \textsc{med} \textsc{poss.2sg.a} breeze wind enter away by inside at \textsc{art} crack:\textsc{red}  of \textsc{art} house\\

\glt
‘Really the (lit. your) breeze did not enter through the cracks of the house.’ \textstyleExampleref{[R347.055]} 
\z

In this loose sense, the possessive pronouns\is{Pronoun!possessive} \textit{ta{\ꞌ}u}\is{tau ‘your’@ta{\ꞌ}u ‘your’} and \textit{ta{\ꞌ}a}\is{taa ‘your’@ta{\ꞌ}a ‘your’} have lost their possessive force; rather, they have become a sort of demonstrative, similar to demonstrative determiners like \textit{tū} and \textit{tau}. However, the latter require a postnominal demonstrative\is{Demonstrative!postnominal} \textit{nei}, \textit{ena} or \textit{era}, while \textit{ta{\ꞌ}a} and \textit{ta{\ꞌ}u} don’t.

\section{Numerals}\label{sec:4.3}
\is{Numeral|(}
\is{Numeral, cardinal}Rapa Nui has a decimal counting system, as is usual in Eastern Polynesia (see \citealt{Lemaître1985}). As is equally usual, it has terms for several powers of ten.

Cardinal numerals are usually preceded by one of the particles \textit{e} (the default marker), \textit{ka} (the contiguity marker) and \textit{hoko} (when referring to a group of persons); these will be discussed in \sectref{sec:4.3.2}. Using these particles as a criterion, the \isi{interrogative} \textit{hia}\is{hia ‘how many’} ‘how many’ also classifies as a numeral (\sectref{sec:10.3.2.4}).

On the other hand, the archaic form \textit{{\ꞌ}aŋahuru}\is{anzahuru ‘ten’@{\ꞌ}aŋahuru ‘ten’} ‘ten’ does not qualify as a numeral in older texts, and neither do certain other forms which are obsolete nowadays (\sectref{sec:4.3.1.2}).

In this section, first the forms of the numerals are discussed (\sectref{sec:4.3.1}). \sectref{sec:4.3.2} discusses elements preceding and following the numerals in the numeral phrase, especially the numeral particles \textit{e}, \textit{ka} and \textit{hoko}. \sectref{sec:4.3.3} discusses ordinal numerals; \sectref{sec:4.3.4} discusses definite numerals, special forms with collective reference. Finally, \sectref{sec:4.3.5} discusses the expression of fractions.

In the \isi{noun phrase}, numerals occur either before or after the noun (\sectref{sec:5.1}); the use of numerals in the \isi{noun phrase} will be discussed in \sectref{sec:5.4}. Apart from that, numerals also occur as predicates of numerical clauses; these are discussed in \sectref{sec:9.5}.

\subsection{Forms of the numerals}\label{sec:4.3.1}
\subsubsection{Basic and alternative forms}\label{sec:4.3.1.0}

\paragraph{One to ten} The cardinal numerals from one to ten in modern Rapa Nui are given in \tabref{tab:24}. 

\begin{table}
\begin{tabularx}{.75\textwidth}{XL{35mm}L{30mm}}
\lsptoprule
 & {basic form} & {alternative form}\\
\midrule
one & {\textit{tahi}\is{tahi ‘one’}} & {\textit{ho{\ꞌ}e}\is{hoe@ho{\ꞌ}e ‘one’}}\\
two & {\textit{rua}\is{rua ‘two; other’}} & {\textit{piti}\is{piti ‘two’}}\\
three & {\textit{toru}\is{toru ‘three’}} & \\
four & {\textit{hā}\is{ha@hā ‘four’}} & {\textit{maha}\is{maha ‘four’}}\\
five & {\textit{rima}\is{rima ‘five’}} & {\textit{pae}\is{pae ‘five’}}\\
six & {\textit{ono}\is{ono ‘six’}} & \\
seven & {\textit{hitu}\is{hitu ‘seven’}} & \\
eight & {\textit{va{\ꞌ}u}\is{vau ‘eight’@va{\ꞌ}u ‘eight’}} & \\
nine & {\textit{iva}\is{iva ‘nine’}} & \\
ten & {\textit{ho{\ꞌ}e {\ꞌ}ahuru}\is{ahuru ‘ten’@{\ꞌ}ahuru ‘ten’}; \textit{{\ꞌ}aŋahuru}\is{anzahuru ‘ten’@{\ꞌ}aŋahuru ‘ten’}} & {\textit{ho{\ꞌ}e {\ꞌ}ahuru}}\\
\lspbottomrule
\end{tabularx}
\caption{Numerals 1–10}
\label{tab:24}
\end{table}

As this table shows, for certain numerals there are two forms: a basic form and an alternative form. The alternative numerals are used in compound\is{Compound} numerals, i.e. as part of numerals higher than ten. They are also used in a number of other cases, described in \sectref{sec:4.3.1.1}.

For ‘ten’, \textit{ho{\ꞌ}e {\ꞌ}ahuru} is the most common form nowadays. (\textit{{\ꞌ}Ahuru} is never used on its own, but always preceded by \textit{ho{\ꞌ}e} ‘one’ or a higher numeral.) \textit{\mbox{{\ꞌ}Aŋahuru}}\is{anzahuru ‘ten’@{\ꞌ}aŋahuru ‘ten’} is an older form which is still in use, but rare. It is especially used as a noun ‘a group of ten’, and as ordinal\is{Numeral!ordinal} number ‘tenth’ (\sectref{sec:4.3.3}). 

\begin{table}
\begin{tabularx}{.5\textwidth}{XL{34mm}}
\lsptoprule
eleven & {\textit{ho{\ꞌ}e {\ꞌ}ahuru}\is{ahuru ‘ten’@{\ꞌ}ahuru ‘ten’} \textit{mā ho{\ꞌ}e}}\\
twelve & {\textit{ho{\ꞌ}e {\ꞌ}ahuru mā piti}}\\
thirteen & {\textit{ho{\ꞌ}e {\ꞌ}ahuru mā toru}}\\
fourteen & {\textit{ho{\ꞌ}e {\ꞌ}ahuru mā maha}}\\
fifteen & {\textit{ho{\ꞌ}e {\ꞌ}ahuru mā pae}}\\
twenty & {\textit{piti {\ꞌ}ahuru}}\\
twenty-one & {\textit{piti {\ꞌ}ahuru mā ho{\ꞌ}e}}\\
twenty-two & {\textit{piti {\ꞌ}ahuru mā piti}}\\
thirty & {\textit{toru {\ꞌ}ahuru}}\\
fourty & {\textit{maha {\ꞌ}ahuru}}\\
fifty & {\textit{pae {\ꞌ}ahuru}}\\
sixty & {\textit{ono {\ꞌ}ahuru}}\\
seventy & {\textit{hitu {\ꞌ}ahuru}}\\
eighty & {\textit{va{\ꞌ}u {\ꞌ}ahuru}}\\
ninety & {\textit{iva {\ꞌ}ahuru}}\\
one hundred & {\textit{ho{\ꞌ}e hānere}}\\
\lspbottomrule
\end{tabularx}
\caption{Numerals 11–100}
\label{tab:25}
\end{table}

\paragraph{11 to 100} Numerals above 10 are illustrated in \tabref{tab:25}. As this table shows, the alternative numerals are used both for the tens (\textit{piti {\ꞌ}ahuru}, not \textit{*rua {\ꞌ}ahuru}) and the units (\textit{mā piti}, not *\textit{mā rua}). Tens and units are connected by the \isi{particle} \textit{mā}\is{ma ‘with’@mā ‘with’} ‘and, with’.\footnote{\label{fn:167}\textit{Mā} is common in Polynesian languages in the sense ‘and, with’ ({\textless} \is{Proto-Polynesian}PPN *\textit{mā}); in various languages it serves to connect to tens to units in numerals, like in Rapa Nui. In Rapa Nui, it is also used in circumstantial clauses (\sectref{sec:11.6.8}), a function shared with \ili{Tahitian} (\citealt[107, 196]{AcadémieTahitienne1986}); possibly Rapa Nui borrowed \textit{m}\textit{ā} from \ili{Tahitian}.} 

Like \textit{{\ꞌ}ahuru}, \textit{hānere}\is{hanere@hānere ‘hundred’} is always preceded by another numeral, whether \textit{ho{\ꞌ}e} ‘one’ or a higher numeral:

\ea\label{ex:4.11}
\gll E \textbf{ho{\ꞌ}e} \textbf{hānere} māmoe hāpa{\ꞌ}o {\ꞌ}ā{\ꞌ}ana... \\
\textsc{num} one hundred sheep care\_for \textsc{poss.3sg.a} \\

\glt
‘He had one hundred sheep he cared for...’ \textstyleExampleref{[R490.002]} 
\z

To indicate an unspecified number above ten, \textit{tūma{\ꞌ}a}\is{tuma{\ꞌ}a ‘more or less’@tūma{\ꞌ}a ‘more or less’} is used: ‘something, and a bit’.

\ea\label{ex:4.12}
\gll piti {\ꞌ}ahuru \textbf{tūma{\ꞌ}a} matahiti \\
two ten more\_or\_less year \\

\glt 
‘twenty-something years’
\z

\ea\label{ex:4.13}
\gll ...{\ꞌ}ātā ki tō{\ꞌ}ona hora mate era {\ꞌ}i te matahiti pae {\ꞌ}ahuru \textbf{tūma{\ꞌ}a} \\
~~~~until to \textsc{poss.3sg.o} time die \textsc{dist} at \textsc{art} year five ten more\_or\_less \\

\glt 
‘...until his death in the fifties (=1950s)’ \textstyleExampleref{[R539-1.493]}
\z

\paragraph{Above 100} \tabref{tab:26} shows numerals above 100. Just as in the numerals between 10 and 100, units as part of higher numerals are preceded by \textit{mā}. Between hundreds and tens, and between thousands and hundreds, the \isi{particle} \textit{e} can be used, but this is not obligatory. 

\begin{table}
\begin{tabularx}{.75\textwidth}{XL{76mm}}
\lsptoprule
101 & {\textit{ho{\ꞌ}e hānere}\is{hanere@hānere ‘hundred’} \textit{mā ho{\ꞌ}e}}\\
102 & {\textit{ho{\ꞌ}e hānere mā piti}}\\
110 & {\textit{ho{\ꞌ}e hānere (e) ho{\ꞌ}e {\ꞌ}ahuru}}\\
111 & {\textit{ho{\ꞌ}e hānere (e) ho{\ꞌ}e {\ꞌ}ahuru mā ho{\ꞌ}e}}\\
120 & {\textit{ho{\ꞌ}e hānere (e) piti {\ꞌ}ahuru}}\\
157 & {\textit{ho{\ꞌ}e hānere (e) pae {\ꞌ}ahuru mā hitu}}\\
200 & {\textit{piti hānere}}\\
678 & {\textit{ono hānere (e) hitu {\ꞌ}ahuru mā va{\ꞌ}u}}\\
1000 & {\textit{ho{\ꞌ}e ta{\ꞌ}utini}\is{tautini ‘thousand’@ta{\ꞌ}utini ‘thousand’}}\\
1001 & {\textit{ho{\ꞌ}e ta{\ꞌ}utini mā ho{\ꞌ}e}}\\
1100 & {\textit{ho{\ꞌ}e ta{\ꞌ}utini (e) ho{\ꞌ}e hānere}}\\
1722 & {\textit{ho{\ꞌ}e}\is{hoe@ho{\ꞌ}e ‘one’} \textit{ta{\ꞌ}utini (e) hitu hānere (e) piti {\ꞌ}ahuru mā}\is{ma ‘with’@mā ‘with’} \textit{piti}}\\
\lspbottomrule
\end{tabularx}
\caption{Numerals {\textgreater} 100}
\label{tab:26}
\end{table}

In spoken language, high numbers are often expressed with \ili{Spanish}\is{\ili{Spanish} influence} numerals. These are not preceded by a numeral \isi{particle}:

\ea\label{ex:4.14}
\gll He take{\ꞌ}a e māua i te \textbf{cien} peso. \\
\textsc{ntr} see \textsc{ag} \textsc{1du.excl} \textsc{acc} \textsc{art} hundred peso \\

\glt 
‘We found one hundred pesos.’ \textstyleExampleref{[R127.004]} 
\z

\ea\label{ex:4.15}
\gll \textbf{Tres} \textbf{mil} dorare i va{\ꞌ}ai ai a Kontiki. \\
three thousand dollar \textsc{pfv} give \textsc{pvp} \textsc{prop} Kontiki \\

\glt
‘Three thousand dollars Kontiki (=Thor Heyerdahl) gave.’ \textstyleExampleref{[R416.674]} 
\z

Common as this may be, speakers do not consider this to be proper Rapa Nui; \ili{Spanish} numerals are not accepted in formal spoken and written language.

\paragraph{Etymology of the numerals} The basic forms of the numerals are regular reflexes of the PPN forms, while the alternative numerals listed above (\textit{\mbox{ho{\ꞌ}e}}\is{hoe@ho{\ꞌ}e ‘one’}\textit{, piti}\is{piti ‘two’}\textit{, maha}\is{maha ‘four’}\textit{, pae}\is{pae ‘five’}\textit{, {\ꞌ}ahuru}\is{ahuru ‘ten’@{\ꞌ}ahuru ‘ten’}) are borrowed from \ili{Tahitian}\is{\ili{Tahitian} influence}.\footnote{\label{fn:168}The \ili{Tahitian} forms for 1, 2, 4 and 5 are language-internal developments, some of which may have occurred as late as the early 19\textsuperscript{th} century \citep[64]{White1968}.} The basic numerals are the original Rapa Nui forms, except \textit{va{\ꞌ}u} (the original form is \textit{varu}) and \textit{ho{\ꞌ}e \mbox{{\ꞌ}ahuru}}, which are also \ili{Tahitian}. The forms \textit{toru, ono, hitu} and \textit{iva} are common to both languages.\footnote{\label{fn:169}A \ili{Spanish} expedition in 1770 recorded a set of numerals totally different from the usual ones and unlike any numerals known from other Polynesian languages: \textit{coyana, corena, cogojui, quiroqui, majana, teuto, tehea, moroqui, vijoviri, queromata} (with some variation between different manuscripts; see \citealt{Ross1937}). These have sometimes been considered as evidence of a non-Polynesian substrate (\citealt{Ross1936}; \citealt{Schuhmacher1976,Schuhmacher1990}; \citealt{MangorSchuhmacher1998}). More likely, however, they represent Rapa Nui words which the \ili{Spanish} transcribers mistook for numerals (\citealt{Fischer1992}; \citealt{Fedorova1993}; \citealt{MellénBlanco1994}).

Four years later, in 1774, Johann Reinhold Forster recorded a set of regular Polynesian numerals \citep[184]{Fischer1992}.}

\textit{Hānere} is also a \ili{Tahitian} borrowing\is{\ili{Tahitian} influence}, derived from \ili{English} ‘hundred’. The origin of \textit{ta{\ꞌ}utini}\is{tautini ‘thousand’@ta{\ꞌ}utini ‘thousand’} is a little more complicated. It was probably borrowed from \ili{Tahitian}\is{\ili{Tahitian} influence} \textit{tauatini}, whereby the second \textit{a} disappeared and a glottal\is{Glottal plosive} was introduced between the first two vowels. \ili{Tahitian} \textit{tauatini} itself is a development from the older form \textit{tautani}, from Eng. ‘thousand’.\footnote{\label{fn:170}The development from \textit{tautani} to \textit{tauatini} probably happened under the influence of Tah. \textit{tini} ‘numerous’: the second half of \textit{tautani} was assimilated to \textit{tini}, which had a closely related meaning.}

We may conclude that in modern Rapa Nui, all numbers higher than seven are expressed by \ili{Tahitian} numerals. The remarkable extent of lexical replacement is evidence for the widespread influence of \ili{Tahitian} on the language (\sectref{sec:1.4.1}).\footnote{\label{fn:171}According to \citet[37]{LynchSpriggs1995}, in almost all Oceanic languages, “the basic monomorphemic numerals are well known and very frequently used”. Two notable exceptions are \ili{Chamorro} (Guam, see \citealt[166]{Topping1973}), in which the whole \isi{numeral system} was replaced by \ili{Spanish}, and \ili{Anejom} (Vanuatu, see \citealt{LynchSpriggs1995}), where all numerals above three were replaced by \ili{Bislama}/English forms. \citet{Clark2004}, however, suggests that numerals are quite susceptible to replacement by terms from a European language, because numerals are often used in domains of interaction with Europeans: in European culture, numbers play a much larger role than in traditional culture. For higher numerals, another reason for substitution is the length of terms: higher vernacular numerals tend to be much longer than \ili{English} equivalents. This is true for Rapa Nui as well. 

We also have to keep in mind that substitution by \ili{Tahitian} terms is different from borrowing from \ili{Spanish}. As \citet[397]{Fischer2007}; \citet[151]{Fischer2008Reversing} points out, \ili{Tahitian} forms are considered as indigenous; they do not stand out as \ili{Spanish} or \ili{English} borrowings\is{Borrowing} would (\sectref{sec:1.4.1}). Large-scale replacement of numerals also happened in other languages under \ili{Tahitian} influence. In \ili{Mangarevan} for example, all numbers higher than five are nowadays expressed with \ili{Tahitian} numerals (P. Auguste Uebe-Carlson, p.c.).}

\subsubsection[Other uses of the alternative numerals]{Other uses of the alternative numerals}\label{sec:4.3.1.1}

As described above, in numbers above ten, the alternative (\ili{Tahitian}) numerals are used. They are also used in dates and when telling the time, and sometimes in measures. These constructions are discussed here.

\paragraph[Days and dates]{Days and dates}\label{sec:4.3.1.1.1}

Most of the names of days of the week contain a (\ili{Tahitian}) numeral:

\ea\label{ex:4.16}
\gll mahana piti; mahana pae \\
day two day five \\

\glt
‘Tuesday; Friday’
\z

For numbering the days of the month, the \ili{Tahitian} numerals are used as well:

\ea\label{ex:4.17}
\gll {\ꞌ}i te \textbf{ho{\ꞌ}e} mahana o Mē \\
at \textsc{art} one day of May \\

\glt 
‘on the first of May’ \textstyleExampleref{[R231.045]}  
\z

\paragraph[Telling time]{Telling time}\label{sec:4.3.1.1.2}

‘X o’clock’ is expressed as \textit{hora X}, where X is a \ili{Tahitian} numeral:

\ea\label{ex:4.18}
\gll Hora \textbf{maha} nei, {\ꞌ}e hora \textbf{hitu} tātou ka tu{\ꞌ}u iho. \\
hour four \textsc{prox} and hour seven \textsc{1pl.incl} \textsc{cntg} arrive just\_then \\

\glt 
‘It is now four o’clock, and seven o’clock we will arrive.’ \textstyleExampleref{[R210.198]} 
\z

\paragraph{Measuring space and time}\label{sec:4.3.1.1.3}

With spatial measuring words like \textit{mētera} ‘meter’, \textit{māroa} ‘fathom’ and \textit{{\ꞌ}umi} ‘ten fathoms’, both Rapa Nui and \ili{Tahitian} numerals are used: \REF{ex:4.19} has the \ili{Tahitian} term \textit{pae}, while \REF{ex:4.20} has the Rapa Nui term \textit{rua}.

\ea\label{ex:4.19}
\gll Te tumu nei tumu nikoniko e ko oŋa te \textbf{pae} \textbf{mētera} o te roaroa. \\
\textsc{art} tree \textsc{prox} tree curl:\textsc{red} \textsc{ipfv} \textsc{neg.ipfv} appear \textsc{art} five meter of \textsc{art} long \\

\glt 
‘This tree is a twisted tree which doesn’t surpass five meters of length.’ \textstyleExampleref{[R478.055]} 
\z

\ea\label{ex:4.20}
\gll E \textbf{rua} \textbf{mētera} mā pae o te roaroa. \\
\textsc{num} two meter plus five of \textsc{art} long:\textsc{red} \\

\glt
‘He was 2.05 meters tall.’ \textstyleExampleref{[R250.177]} 
\z

With time words we also find an alternation between \ili{Tahitian} and Rapa Nui numerals:

\ea\label{ex:4.21}
\gll E u{\ꞌ}i nō rā, e \textbf{pae} \textbf{minuti} toe he tu{\ꞌ}u mai. \\
\textsc{exh} look just \textsc{intens} \textsc{num} five minute remain \textsc{ntr} arrive hither \\

\glt 
‘Just watch, in another five minutes he comes.’ \textstyleExampleref{[R437.037]} 
\z

\ea\label{ex:4.22}
\gll Ka \textbf{rima} \textbf{matahiti} ō{\ꞌ}oku...\\
\textsc{cntg} five year \textsc{poss.1sg.o}\\

\glt 
‘When I was five years old...’ \textstyleExampleref{[R242.001]} 
\z

\subsubsection[Old numerals]{Old numerals}\label{sec:4.3.1.2}

In older texts, only the original Rapa Nui numerals are used. The numbers one through seven and nine are identical to the basic forms still in use today, listed in \sectref{sec:4.3.1.0}. For ‘eight’, the old form is \textit{varu}\is{varu}.\footnote{\label{fn:172}It is interesting to note, that the \ili{Tahitian} form \textit{va{\ꞌ}u} appears already in MsE, the oldest text in the corpus, where it is used alongside \textit{varu}. In Ley and Mtx, however, \textit{varu} is consistently used. Englert’s grammar (\citealt[58]{Englert1978}), which otherwise does not mention \ili{Tahitian} numerals, states that, while \textit{varu} is the older form, nowadays only \textit{va{\ꞌ}u} is used. 

\textit{Va{\ꞌ}u} may have been replaced earlier than the other numerals because it is a relatively high number, and/or because the \ili{Tahitian} form is close to the Rapa Nui form. Moreover, the alternation between \textit{r} and glottal\is{Glottal plosive} is a process which occurs within Rapa Nui as well (\sectref{sec:2.5.2}).} 

For ‘ten’, the original numeral is \textit{{\ꞌ}aŋahuru}\is{anzahuru ‘ten’@{\ꞌ}aŋahuru ‘ten’}, which is still marginally in use today (\sectref{sec:4.3.1.0} above). In older texts, it is usually preceded by the article \textit{te} rather than the numeral particles \textit{e} or \textit{ka}. Between \textit{{\ꞌ}aŋahuru} and the noun there is a second article:

\ea\label{ex:4.23}
\gll He here e tahi te {\ꞌ}aŋahuru te taka.\\
\textsc{ntr} tie \textsc{num} one \textsc{art} ten \textsc{art} roll\\

\glt
‘They tie ten rolls (of \textit{mahute} fibers) together.’ \textstyleExampleref{[Ley-5-05.002]}
\z

Thus, \textit{{\ꞌ}aŋahuru} is more a noun than a numeral;\footnote{\label{fn:173}It is not unusual for higher numerals to have the status of nouns; see \citet[78]{Dixon2012}.} the counted item follows as a second \isi{noun phrase}. On the other hand, it is not quite a regular noun, as the construction \textit{te~N~te~N} is never used with other nouns.

When \textit{{\ꞌ}aŋahuru} is used nowadays, it behaves like any other numeral. For example, in the following example it is not preceded by an article:

\ea\label{ex:4.24}
\gll {\ꞌ}I ira i noho ai \textbf{e} \textbf{tahi} \textbf{{\ꞌ}aŋahuru} o te mahana. \\
at \textsc{ana} \textsc{pfv} stay \textsc{pvp} \textsc{num} one ten of \textsc{art} day \\

\glt
‘There they stayed ten days.’ \textstyleExampleref{[R420.047]} 
\z

Three other old – and obsolete – numerals are \textit{kauatu} ‘ten’, \textit{rau} ‘hundred’ and \textit{pīere} ‘thousand’. Like \textit{{\ꞌ}aŋahuru}, they are preceded by the article rather than by a numeral marker.

\subsection{The numeral phrase}\label{sec:4.3.2}

Cardinal numerals are always preceded by one of the particles \textit{e}, \textit{ka} and – less commonly – \textit{hoko}.\footnote{\label{fn:174}A \isi{prefix} found in other Polynesian languages but not in Rapa Nui is the distributive\is{Distributive} \isi{prefix} \textit{*taki-} (e.g. \textit{takitahi} ‘one each’), used e.g. in \ili{Tahitian} (\citealt[182]{LazardPeltzer2000}), \ili{Pa’umotu} \citep[492]{Stimson1964}, \ili{Māori} \citep[498]{Bauer1993}, \ili{Samoan} (\citealt[116]{MoselHovdhaugen1992}). In Rapa Nui, the distributive is expressed by repetition of the \isi{noun phrase}.} These are discussed in the following subsections. \sectref{sec:4.3.2.4} shows that numerals may be followed by a number of modifying particles.

\subsubsection[Neutral e]{Neutral \textit{e}}\label{sec:4.3.2.1}

\textit{e}\is{e (numeral \isi{particle})} is the neutral numeral \isi{particle}. In most contexts, cardinal numerals are preceded by \textit{e}. 

Numerals preceded by \textit{e} occur before or after the noun in the \isi{noun phrase} (\sectref{sec:5.4.1}–\ref{sec:5.4.2}); they also occur as the predicate of a numerical clause\is{Clause!numerical} (\sectref{sec:9.5}). Numerous examples of \textit{e} + numeral are provided in the referred sections, as well as in \sectref{sec:4.3.1} above.

\subsubsection[The contiguity marker ka]{The contiguity marker \textit{ka}}\label{sec:4.3.2.2}

\textit{Ka}\is{ka (numeral \isi{particle})} is an aspectual marker\is{Aspect marker} indicating contiguity between two events (\sectref{sec:7.2.6}). With numerals, \textit{ka} is used in counting, or when listing or summing up a series of things:

\ea\label{ex:4.25}
\gll ka tahi, ka rua, ka toru, ka hā \\
\textsc{cntg} one \textsc{cntg} two \textsc{cntg} three \textsc{cntg} four \\

\glt 
‘one, two, three, four’
\z

\ea\label{ex:4.26}
\gll He oho ki te hare hāpī, ka tahi mahana, ka rua, ka toru. \\
\textsc{ntr} go to \textsc{art} house learn \textsc{cntg} one day \textsc{cntg} two \textsc{cntg} three \\

\glt
‘He went to school, one day, two, three.’ \textstyleExampleref{[R399.013]}  
\z

When used within a \isi{noun phrase}, like \textit{ka tahi mahana} in \REF{ex:4.26}, \textit{ka} + numeral always occurs before the noun, never after the noun (different from \textit{e}). 

\textit{Ka} as a numeral marker is used when a number or quantity has been reached; it indicates an extent. It is especially common with time words, indicating that a certain time has elapsed. In the following example, \textit{ka} + numeral indicates that the age of ten years has been reached:

\ea\label{ex:4.27}
\gll E tahi poki te {\ꞌ}īŋoa ko Eva \textbf{ka} \textbf{ho{\ꞌ}e} \textbf{{\ꞌ}ahuru} matahiti. \\
\textsc{num} one child \textsc{art} name \textsc{prom} Eva \textsc{cntg} one ten year \\

\glt
‘There was a girl whose name was Eva, ten years old.’ \textstyleExampleref{[R210.001]} 
\z

In this sense ‘elapsed time’, \textit{ka} is used to indicate minutes after the hour (\sectref{sec:4.3.1.1.2}). 

As \textit{ka}\is{ka (numeral \isi{particle})} indicates a quantity which has been reached, it may emphasise the amount: ‘up to, as many as’. In \REF{ex:4.28} this emphasis is further enhanced by the use of \textit{rō atu}:

\ea\label{ex:4.28}
\gll Mo ai rō kona hore iho hai {\ꞌ}ārote e pu{\ꞌ}a era e ono  {\ꞌ}o \textbf{ka} \textbf{va}{\ꞌ}\textbf{u} \textbf{rō} \textbf{atu} {\ꞌ}uei.\\
if exist \textsc{emph} place cut just\_then \textsc{ins} plough \textsc{ipfv} cover \textsc{dist} \textsc{num} six  or \textsc{cntg} eight \textsc{emph} away ox\\

\glt 
‘When a field was ploughed for the first time, it was ploughed with six or even eight oxen.’ \textstyleExampleref{[R539-1.110–111]}
\z

\ea\label{ex:4.29}
\gll \textbf{Ka} \textbf{ono,} \textbf{ka} \textbf{ono} taŋata i mate {\ꞌ}i tau {\ꞌ}ura era ko tetu. \\
\textsc{cntg} six \textsc{cntg} six man \textsc{pfv} die at \textsc{dem} lobster \textsc{dist} \textsc{prom} huge \\

\glt 
‘As many as six men died by that enormous lobster.’ \textstyleExampleref{[Mtx-4-05.014]}
\z

\subsubsection[The person marker hoko]{The person marker \textit{hoko}}\label{sec:4.3.2.3}

The \isi{particle} \textit{hoko}\is{hoko (numeral \isi{particle})} is used when counting persons: \textit{hoko rua} ‘two people’, \textit{hoko toru} ‘three people’ etc.\footnote{\label{fn:175}A \isi{prefix} \textit{soko} or \textit{hoko} preceding numerals (\is{Proto-Polynesian}PPN \textit{*soko}) is found in a smattering of languages throughout Polynesia (Pollex, see \citealt{GreenhillClark2011}), though never exclusively referring to persons; it either means ‘just, exactly’ or ‘one, alone, a single’; the latter sense occurs in Rapa Nui in \textit{hokotahi} ‘solitary’. 
A numeral \isi{prefix} restricted to human reference is \is{Proto-Polynesian}PPN \textit{*toko}, which occurs in the majority of Polynesian languages. Possibly both \textit{*toko} and \textit{*hoko} existed originally in Rapa Nui; the two were conflated because of their formal and semantic similarity, resulting in the form \textit{hoko} with semantic features of both *\textit{hoko} and *\textit{toko}.} It is only used with numerals under ten. 

Numerals preceded by \textit{hoko} may occur after the noun as in \REF{ex:4.30}, before the noun as in \REF{ex:4.31} (though this is relatively rare), or on their own as in \REF{ex:4.32}:

\ea\label{ex:4.30}
\gll He e{\ꞌ}a ia tou ŋā kope era \textbf{hoko} \textbf{toru} {\ꞌ}i ruŋa o te vaka. \\
\textsc{ntr} go\_out then \textsc{dem} \textsc{pl} person \textsc{dist} \textsc{num.pers} three at above of \textsc{art} boat \\

\glt 
‘Those three people went out by boat.’ \textstyleExampleref{[R309.102]} 
\z

\ea\label{ex:4.31}
\gll I e{\ꞌ}a mai ai \textbf{hoko} \textbf{iva} taŋata o ruŋa i te {\ꞌ}avione mau {\ꞌ}ana... \\
\textsc{pfv} go\_out hither \textsc{pvp} \textsc{num.pers} nine man of above at \textsc{art} airplane really \textsc{ident} \\

\glt 
‘When nine men had come out of the airplane...’ \textstyleExampleref{[R539-2.215]}
\z

\ea\label{ex:4.32}
\gll He ha{\ꞌ}uru \textbf{hoko} \textbf{hā}, \textbf{hoko} \textbf{toru} ka {\ꞌ}ara ka vānaŋanaŋa nō. \\
\textsc{ntr} sleep \textsc{num.pers} four \textsc{num.pers} three \textsc{cntg} wake\_up \textsc{cntg} talk:\textsc{red} just \\

\glt
‘Four (men) slept, three were awake and were talking.’ \textstyleExampleref{[MsE-050.005]}
\z

\textit{Hoko}\is{hoko (numeral \isi{particle})} \textit{rua} and \textit{hoko tahi} have both developed certain lexicalised uses in which the sense is somewhat different from ‘a group of X persons’; in these cases, they are written as one word. \textit{Hokorua} is used as a noun ‘companion’ and as a verb ‘to accompany’; \textit{hokotahi} is used as an adjective ‘lonely, solitary’, or an \isi{adverb}\is{Adverb} ‘alone, on one’s own’:

\ea\label{ex:4.33}
\gll He \textbf{hokorua} a au i tō{\ꞌ}oku repahoa. \\
\textsc{ntr} accompany \textsc{prop} \textsc{1sg} \textsc{acc} \textsc{poss.1sg.o} friend \\

\glt 
‘I accompany my friend.’ \textstyleExampleref{[R208.138]} 
\z

\ea\label{ex:4.34}
\gll He u{\ꞌ}i mai a Ure {\ꞌ}a Reka \textbf{hokotahi} nō a Marama, {\ꞌ}ina he \textbf{hokorua}. \\
\textsc{ntr} look hither \textsc{prop} Ure a Reka solitary just \textsc{prop} Marama \textsc{neg} \textsc{pred} companion \\

\glt 
‘Ure a Reka saw that Marama was lonely, he had no companion.’ \textstyleExampleref{[Ley-7-48.013]}
\z

\subsubsection[Modifiers in the numeral phrase]{Modifiers in the numeral phrase}\label{sec:4.3.2.4}

Cardinal numerals may be followed by modifying elements like \textit{mau}\is{mau ‘really’} ‘really’, \textit{nō}\is{no ‘just’@nō ‘just’} ‘just’ or \textit{haka{\ꞌ}ou}\is{haka{\ꞌ}ou ‘again’} ‘again, more, other’, elements which also occur in the \isi{noun phrase} (\sectref{sec:5.8}).

\ea\label{ex:4.35}
\gll {\ꞌ}E ko tū me{\ꞌ}e {\ꞌ}ā i aŋa ai \textbf{e} \textbf{rua} \textbf{haka{\ꞌ}ou} mahana. \\
and \textsc{prom} \textsc{dem} thing \textsc{ident} \textsc{pfv} do \textsc{pvp} \textsc{num} two again day \\

\glt 
‘And he did that same thing two more days’ \textstyleExampleref{[R532-07.021]}
\z

\ea\label{ex:4.36}
\gll \textbf{E} \textbf{tahi} \textbf{mau} \textbf{nō} {\ꞌ}ā{\ꞌ}ana poki vahine.\\
\textsc{num} one really just \textsc{poss.3sg.a} child female\\

\glt 
‘He had just one daughter.’ \textstyleExampleref{[R372.004]} 
\z

Numerals preceded by \textit{ka}\is{ka (numeral \isi{particle})} may also be followed by the verb phrase particles \textit{rō}\is{o (asseverative)@{\ꞌ}ō (asseverative)} (\sectref{sec:7.4.2}) and \textit{{\ꞌ}ō} (\sectref{sec:4.5.4.5})\is{ro (emphatic marker)@rō (emphatic marker)}. \textit{Rō} (which may in turn be followed by \textit{atu}) emphasises the extent or limit of the number: ‘up to, as much as, even’:

\ea\label{ex:4.37}
\gll ¡\textbf{Ka} \textbf{rua} \textbf{{\ꞌ}ō} mahana {\ꞌ}ina kai tu{\ꞌ}u mai! \\
~\textsc{cntg} two really day \textsc{neg} \textsc{neg.pfv} arrive hither \\

\glt 
‘She hasn’t come for two days!’ \textstyleExampleref{[R229.132]} 
\z

\ea\label{ex:4.38}
\gll Mo ai rō kona hore iho hai {\ꞌ}ārote e pu{\ꞌ}a era e ono  {\ꞌ}o \textbf{ka} \textbf{va}{\ꞌ}\textbf{u} \textbf{rō} \textbf{atu} {\ꞌ}uei.\\
if exist \textsc{emph} place cut just\_then \textsc{ins} plow \textsc{ipfv} cover \textsc{dist} \textsc{num} six  or \textsc{cntg} eight \textsc{emph} away ox\\

\glt 
‘When a field was ploughed for the first time, it was covered with six or even eight oxen.’ \textstyleExampleref{[R539-1.110]}
\z

\subsection{Ordinal numerals}\label{sec:4.3.3}

Rapa Nui does not have separate forms for ordinal\is{Numeral!ordinal} numerals\is{Numeral!ordinal}, except \textit{ra{\ꞌ}e}\is{rae ‘first’@ra{\ꞌ}e ‘first’} ‘first’, which is an adjective, occurring after the noun. Other numerals are interpreted as ordinal\is{Numeral!ordinal} numerals by virtue of their position: they are ordinal\is{Numeral!ordinal} numerals\is{Numeral!ordinal} when they occur before the noun and are preceded by a determiner. 

\ea
\begin{tabbing}
 xxxxxxxxxxxxxxxxxx \= xxxxxxxxxx\kill
\textit{te taŋata ra{\ꞌ}e} \> ‘the first man’\\
\textit{te rua taŋata} \> ‘the second man’\\
\textit{te toru taŋata} \> ‘the third man’\\
\textit{te ho{\ꞌ}e {\ꞌ}ahuru taŋata} \> ‘the tenth man’
\end{tabbing}
\z 
The determiner can be the article \textit{te} as in the examples above, but also a possessive pronoun as in \REF{ex:4.39}, or the predicate marker \textit{he}\is{he (nominal predicate marker)} as in \REF{ex:4.40}:

\ea\label{ex:4.39}
\gll Ku aŋa {\ꞌ}ana i \textbf{tō{\ꞌ}ona} \textbf{rua} vaka era. \\
\textsc{prf} make \textsc{cont} \textsc{acc} \textsc{poss.3sg.o} two boat \textsc{dist} \\

\glt 
‘He built his second boat.’ \textstyleExampleref{[R539-1.168]}
\z

\ea\label{ex:4.40}
\gll Te hare pure \textbf{he} \textbf{rua} hare pure era. \\
\textsc{art} house pray \textsc{pred} two house pray \textsc{dist} \\

\glt
‘The church (shown in this picture) is the second church.’ \textstyleExampleref{[R412.203]} 
\z

\textit{Rua}\is{rua ‘two; other’} as an ordinal\is{Numeral!ordinal} numeral is also used in the sense ‘the other’ (out of two):

\ea\label{ex:4.41}
\gll E rua ŋāŋata hiva, e tahi taŋata he italiano, te \textbf{rua} taŋata  he harani.\\
\textsc{num} two men continent \textsc{num} one man \textsc{pred} Italian \textsc{art} two man  \textsc{pred} \ili{French}\\

\glt
‘There were two foreigners, one man was an Italian, the other man was a Frenchman.’ \textstyleExampleref{[Egt-02.185]}
\z

With a time word, \textit{rua} means ‘next’. 

\ea\label{ex:4.42}
\gll {\ꞌ}I te pō{\ꞌ}ā o te \textbf{rua} \textbf{mahana} he {\ꞌ}ara a Piu. \\
at \textsc{art} morning of \textsc{art} two day \textsc{ntr} wake\_up \textsc{prop} Piu \\

\glt 
‘In the early morning of the next day, Piu woke up.’ \textstyleExampleref{[R437.088]} 
\z

\subsection{Definite numerals}\label{sec:4.3.4}
\is{Numeral!definite}
To express ‘the two, the three’ et cetera, a special form of the numerals is used, in which the first \isi{mora}\is{Mora} is reduplicated (type 1 \isi{reduplication}, see \sectref{sec:2.6.1.1}). These forms, which can be labeled \textsc{definite numerals},\footnote{\label{fn:176}Other possible terms are “proper numerals” (because of the use of \textit{a}), or “collective numerals” (because they denote a collectivity).} are always preceded by the proper article\is{a (proper article)}~\textit{a} (\sectref{sec:5.13.2}; possibly \textit{a} is used because the definite numerals are functionally similar to pronouns). They are listed in \tabref{tab:27}.

\begin{table}
\begin{tabularx}{.75\textwidth}{XL{28mm}L{33mm}} 
\lsptoprule
& {modern form} & {archaic form}\\
\midrule
the two & \textit{ararua}\is{ararua ‘the two’} & \textit{a rurua}\footnotemark{}\\
the three & {\textit{a totoru}} & \\
the four & {\textit{a hahā}} & \\
the five & {\textit{a ririma}} & \\
the six & {\textit{a oono}} & \\
the seven & {\textit{a hihitu}} & \\
the eight & {\textit{a vava{\ꞌ}u}} & {\textit{a vavaru}}\\
the nine & {\textit{a iiva}} & \\
the ten & {\textit{a hoho{\ꞌ}e {\ꞌ}ahuru}} & {\textit{a~tatahi~te~{\ꞌ}aŋahuru;}} \\
& & {\textit{a~tatahi~te~kauatu}}\\
\lspbottomrule
\end{tabularx}
\caption{Definite numerals}
\label{tab:27}
\end{table}

\footnotetext{\textit{A rurua} is used only in Ley and MsE; Mtx has \textit{ararua}.}

Like all reduplications, these forms are written with a hyphen in standard \isi{orthography} (\textit{a to-toru} etc.). As the table shows, the original \textit{a rurua} has evolved into \textit{ararua}\is{ararua ‘the two’}. As this is a frozen form which is not recogniseable as a \isi{reduplication}\is{Reduplication}, it does not have a hyphen in the standard \isi{orthography}: \textit{ararua} versus \textit{a ru-rua}.

The definite numerals often occur on their own as in \REF{ex:4.43} below, but they are also used in the \isi{noun phrase}. In the latter case they may placed either before the noun as in \REF{ex:4.44}, or after the noun as in \REF{ex:4.45}. 

\ea\label{ex:4.43}
\gll He e{\ꞌ}a ia \textbf{a} \textbf{totoru} he oho he runu i te rāua tūava. \\
\textsc{ntr} go\_out then \textsc{prop} \textsc{red}:three \textsc{ntr} go \textsc{ntr} gather \textsc{acc} \textsc{art} \textsc{3pl} guava \\

\glt 
‘The three went out and gathered their guavas.’ \textstyleExampleref{[R496.029]} 
\z

\ea\label{ex:4.44}
\gll He oho mai \textbf{a} \textbf{oono} \textbf{ŋā} \textbf{io}, he tu{\ꞌ}u ki te hare o Kave Heke. \\
\textsc{ntr} go hither \textsc{prop} \textsc{red}:six \textsc{pl} young\_man \textsc{ntr} arrive to \textsc{art} house of Kave Heke \\

\glt 
‘The six young men came and arrived at the house of Kave Heke.’ \textstyleExampleref{[Ley-4-01.007]}
\z

\ea\label{ex:4.45}
\gll He oŋe te \textbf{{\ꞌ}aro} \textbf{a} \textbf{hahā} o nei. \\
\textsc{ntr} shortage \textsc{art} side \textsc{prop} \textsc{red}:four of \textsc{prox} \\

\glt
‘The four sides of the island (lit. of here) here suffered shortage.’ \textstyleExampleref{[Mtx-5-02.017]}
\z

Like the cardinal numerals, definite\is{Numeral!definite} numerals are never preceded by prepositions. They are usually found in subject position, where no \isi{preposition} is needed. However, they are also used occasionally in positions that would normally require a \isi{preposition}. In the following example, \textit{a vavaru} occurs in a locative phrase, where the \isi{preposition} \textit{{\ꞌ}i} ‘in’ is expected; the \isi{preposition} is left out.

\ea\label{ex:4.46}
\gll ...he tiŋa{\ꞌ}i e rima te {\ꞌ}aŋahuru \textbf{a} \textbf{vavaru} pū.\\
~~~~\textsc{ntr} kill \textsc{num} five \textsc{art} ten \textsc{prop} \textsc{red}:eight hole\\

\glt
‘...they killed fifty (people who were hiding) in the eight holes.’ \textstyleExampleref{[Mtx-3-01.237]}
\z

Notice that this restriction distinguishes the definite numerals from all other items preceded by the proper article\is{a (proper article)}: pronouns and common nouns\is{Noun!common} marked with the proper article\is{a (proper article)} can be preceded by prepositions without a problem (\sectref{sec:5.13.2.1}). 

Nowadays the definite numerals other than \textit{ararua}\is{ararua ‘the two’} are used less frequently than in the past.\footnote{\label{fn:178}In the corpus of old texts (122,600 words), there are 73 occurrences, roughly once in 1,700 words; in the much larger corpus of newer texts (367,500 words) there are only 39 occurrences, roughly once in 9,400 words. \textit{Ararua}, on the other hand, is common both in older and newer texts: in the former it occurs 136 times (once in 900 words), in the latter 865 times (once in 425 words).} Their role is partly taken over by \textit{ananake}\is{ananake ‘together’} (\sectref{sec:4.4.4}), which used to be the universal \isi{quantifier}\is{Quantifier} ‘all’, but which nowadays has the sense ‘together’. Both \textit{ananake} and the definite numerals are mainly used pronominally nowadays, i.e. without a head noun or pronoun. 

One could say that \textit{ararua} and \textit{ananake} form a mini-paradigm in modern Rapa Nui, with \textit{ararua} referring to a group of two entities and \textit{ananake} to more than two.\footnote{\label{fn:179}This is even clearer in comitative\is{Comitative} constructions: nowadays both \textit{ararua}\is{ararua ‘the two’} and \textit{ananake}\is{ananake ‘together’} are used as comitative\is{Comitative} markers, while other definite numerals\is{Numeral!definite} are not used as such (\sectref{sec:8.10.3}).} 

\subsection{Fractions}\label{sec:4.3.5}
\is{Fraction}
\textit{{\ꞌ}Afa}\is{afa ‘half’@{\ꞌ}afa ‘half’} means ‘half’. It is only used in \textit{{\ꞌ}e te {\ꞌ}afa} ‘and a half’, supplementing a whole number:

\ea\label{ex:4.47}
\gll e toru mētera \textbf{{\ꞌ}e} \textbf{te} \textbf{{\ꞌ}afa} \\
\textsc{num} three meter and \textsc{art} half \\

\glt
‘three and a half meters’ \textstyleExampleref{[Notes]}
\z

The expression as a whole was borrowed from \ili{Tahitian}\is{\ili{Tahitian} influence}, which in turn borrowed the word \textit{{\ꞌ}afa} from \ili{English} ‘half’.

There are no common terms to express other fractions. They can be circumscribed using \textit{{\ꞌ}apa}\is{apa ‘part’@{\ꞌ}apa ‘part’} ‘part’. In the Bible translation, where certain fractions occur, this may lead to constructions such as the following:

\ea\label{ex:4.48}
\gll Ko mate {\ꞌ}ana e tahi {\ꞌ}apa o te {\ꞌ}apa e toru o te taŋata. \\
\textsc{prf} die \textsc{cont} \textsc{num} one part of \textsc{art} part \textsc{num} three of \textsc{art} man \\

\glt
‘One third of the people (lit. one part of the three parts of the people) had died.’ \textstyleExampleref{[Rev. 9:20]}
\z

The word \textit{{\ꞌ}apa}\is{apa ‘part’@{\ꞌ}apa ‘part’} was probably borrowed from \ili{Tahitian}\is{\ili{Tahitian} influence}, where it means ‘half of a fish or animal, cut lengthwise’ or ‘piece of tissue, patch’ (\citealt[49]{AcadémieTahitienne1999}). It is not used in older texts.\is{Numeral}
\is{Numeral|)}

\section{Quantifiers}\label{sec:4.4}
\is{Quantifier|(}\subsection{Overview}\label{sec:4.4.1}

Quantifiers\is{Quantifier} are semantically similar to numerals in that they express a quantity; unlike numerals, quantifiers\is{Quantifier} do not indicate an exact amount. 

The quantifiers\is{Quantifier} of Rapa Nui are listed in \tabref{tab:28}. As the table shows, the \isi{quantifier}\is{Quantifier} system has undergone significant changes over the past century. A number of new quantifiers\is{Quantifier} have been introduced, while others have undergone semantic shifts. 

\begin{table}
\begin{tabularx}{.75\textwidth}{XL{32mm}L{31mm}}
\lsptoprule

{\isi{quantifier}\is{Quantifier}} & {modern Rapa Nui} & {older Rapa Nui}\\
\midrule
{\textit{ta{\ꞌ}ato{\ꞌ}a}\is{taatoa ‘all’@ta{\ꞌ}ato{\ꞌ}a ‘all’}} & {all} & —\\
{\textit{ananake}\is{ananake ‘together’}} & {together} & {all}\\
{\textit{paurō}\is{paurō ‘every’}} & {all, every} & —\\
{\textit{rauhuru}\is{rauhuru ‘diverse’}} & {diverse} & —\\
{\textit{tētahi}\is{tetahi ‘some’@tētahi ‘some’}} & {some, other, another} & {some, other, another}\\
\textit{me{\ꞌ}e rahi}{\is{mee rahi ‘many’@me{\ꞌ}e rahi ‘many’}} & {many} & — {\is{rahi ‘much/many’}}\\
{\textit{kā}\is{ka ‘each’@kā ‘each’}} & {each}  & —\\
{\textit{pura}\is{pura ‘mere’}} & {mere, purely, totally}  & —\\
\lspbottomrule
\end{tabularx}
\caption{List of quantifiers}
\label{tab:28}
\end{table}

\tabref{tab:29} lists a few words which are syntactically different from quantifiers\is{Quantifier} (i.e. they do not occur in the same positions in the \isi{noun phrase}) but are discussed in this section because they have a quantifying sense.

\begin{table}
\begin{tabularx}{\textwidth}{XL{27mm}L{65mm}}
\lsptoprule

{\isi{quantifier}\is{Quantifier}} & {sense} & {syntactic status}\\
\midrule
{\textit{rahi}\is{rahi ‘much/many’}} & {many, much} & {adjective (cf. \textit{me{\ꞌ}e rahi} above)}\\
{\textit{tahi}\is{tahi ‘all’}} & {all} & {verb phrase \isi{adverb}\is{Adverb}}\\
{\textit{kē}\is{ke ‘different’@kē ‘different’}} & {some, other} & {mainly adjective, occasionally \isi{quantifier}\is{Quantifier}}\\
{\textit{rua}} & {other} & {ordinal numeral (\sectref{sec:4.3.3})}\\
\lspbottomrule
\end{tabularx}
\caption{Quantifier-like words}
\label{tab:29}
\end{table}

In Rapa Nui, quantifiers\is{Quantifier} are syntactically like numerals in two respects:

\newpage 

\begin{itemize}
\item 
they occur as modifiers in the \isi{noun phrase}, before or after the noun;

\item 
they often exclude the use of the article.

\end{itemize}

There are important differences as well. Quantifiers\is{Quantifier} are not preceded by the numeral particles \textit{e}, \textit{ka} and \textit{hoko}. And even though they seem to occupy the same positions in the \isi{noun phrase}, on closer analysis they sometimes turn out to be in a different position. In fact, quantifiers\is{Quantifier} also differ from each other in the positions in which they can occur. They may be pre- or postnominal; if prenominal, they occur before or after the article or without article. \tabref{tab:30} lists the position(s) of each \isi{quantifier} in the \isi{noun phrase}.\footnote{\label{fn:180}Not included are \textit{ananake} (which rarely occurs within a \isi{noun phrase} in modern Rapa Nui) and the minor quantifiers\is{Quantifier} \textit{kā} and \textit{pūra}.}

\begin{table}
\begin{tabularx}{\textwidth}{L{19mm}L{18mm}Z{15mm}Z{15mm}Z{15mm}Z{18mm}} 
\lsptoprule
&  & \textit{QTF te N}& \textit{QTF N}& \textit{te QTF N}& \textit{(te) N QTF}\\
\midrule
{\textit{ta{\ꞌ}ato{\ꞌ}a}\is{taatoa ‘all’@ta{\ꞌ}ato{\ꞌ}a ‘all’}} & ‘all’ & marginal& yes& yes& yes\\
{\textit{paurō}\is{paurō ‘every’}} & ‘all’ & yes& marginal& no& marginal\\
{\textit{rauhuru}\is{rauhuru ‘diverse’}} & ‘diverse’ & yes& yes& yes& yes\\
{\textit{tētahi}\is{tetahi ‘some’@tētahi ‘some’}} & ‘some’ & no& yes& yes& no\\
{\textit{me{\ꞌ}e rahi}\is{mee rahi ‘many’@me{\ꞌ}e rahi ‘many’}} & ‘many’ & no\footnotemark{}& yes& no& no\\
\lspbottomrule
\end{tabularx}
\caption{Distribution of quantifiers in the noun phrase}
\label{tab:30}
\end{table}

\footnotetext{When \textit{me{\ꞌ}e rahi} occurs before the article, it is external to the \isi{noun phrase}.}

This table demonstrates that the position of quantifiers is lexically determined. All quantifiers occur before the noun, only some after the noun. \textit{Ta}\textit{{\ꞌ}ato{\ꞌ}a} ‘all’ and \textit{rauhuru} ‘diverse’ occur both pre- and postnominally; the other quantifiers only occur before the noun. The position with respect to the article \textit{te} is lexically determined as well: whereas \textit{paurō} ‘all’ is always followed by the article, the other quantifiers mostly occur without article or after the article. The question whether the position of the \isi{quantifier} has semantic repercussions, is discussed in the subsections on the individual quantifiers.

\subsection{\textit{Ta{\ꞌ}ato{\ꞌ}a} ‘all’}\label{sec:4.4.2}
\is{taatoa ‘all’@ta{\ꞌ}ato{\ꞌ}a ‘all’|(}\is{taatoa ‘all’@ta{\ꞌ}ato{\ꞌ}a ‘all’}
The universal \isi{quantifier}\is{Quantifier} \textit{ta{\ꞌ}ato{\ꞌ}a} ‘all’ is the most common \isi{quantifier}\is{Quantifier} in modern Rapa Nui. It is a relative newcomer, borrowed from \ili{Tahitian}\is{\ili{Tahitian} influence}.\footnote{\label{fn:182}It is found in Fel, Blx and newer texts.} It occurs in a variety of positions in the \isi{noun phrase}; a difference in position does not always imply a clear difference in meaning. 

\subparagraph{\textit{Te N ta{\ꞌ}ato{\ꞌ}a}} The most common position of \textit{ta{\ꞌ}ato{\ꞌ}a} is after the noun, before postnominal demonstratives\is{Demonstrative} (see the chart in \sectref{sec:5.1}). The noun is preceded by the article \textit{te} or another determiner:

\ea\label{ex:4.49}
\gll Te nūna{\ꞌ}a \textbf{ta{\ꞌ}ato{\ꞌ}a} nei i noho ai {\ꞌ}i {\ꞌ}Anakena.\\
\textsc{art} group all \textsc{prox} \textsc{pfv} stay \textsc{pvp} at Anakena\\

\glt 
‘All these people stayed at Anakena.’ \textstyleExampleref{[R376.036]} 
\z

\ea\label{ex:4.50}
\gll E haŋa koe ki te manu \textbf{ta{\ꞌ}ato{\ꞌ}a}, ki te {\ꞌ}animare \textbf{ta{\ꞌ}ato{\ꞌ}a}. \\
\textsc{exh} love \textsc{2sg} to \textsc{art} bird all to \textsc{art} animal all \\

\glt 
‘You must love all the birds, all the animals.’ \textstyleExampleref{[R213.053]} 
\z

\subparagraph{\textit{Te ta{\ꞌ}ato{\ꞌ}a N}} \textit{ta{\ꞌ}ato{\ꞌ}a}\is{taatoa ‘all’@ta{\ꞌ}ato{\ꞌ}a ‘all’} may also appear before the noun, after the determiner:

\ea\label{ex:4.51}
\gll {\ꞌ}I te mahana nei \textbf{te} \textbf{ta{\ꞌ}ato{\ꞌ}a} ŋā poki he porotē. \\
at \textsc{art} day \textsc{prox} \textsc{art} all \textsc{pl} child \textsc{ntr} parade \\

\glt
‘Today all the children participate in the parade.’ \textstyleExampleref{[R334.324]} 
\z

The difference between \textit{te N ta{\ꞌ}ato{\ꞌ}a} and \textit{te ta{\ꞌ}ato{\ꞌ}a N} is mainly a stylistic one: some speakers freely use \textit{ta{\ꞌ}ato{\ꞌ}a} prenominally, others feel that \textit{te~ta{\ꞌ}ato{\ꞌ}a~N} is less grammatical. On the whole, postnominal \textit{ta{\ꞌ}ato{\ꞌ}a} is much more common.

Yet there is also a slight difference in meaning: at least for some speakers, prenominal \textit{ta{\ꞌ}ato{\ꞌ}a}\is{taatoa ‘all’@ta{\ꞌ}ato{\ꞌ}a ‘all’} is somewhat emphatic. Compare \REF{ex:4.52} with \REF{ex:4.51} above: \REF{ex:4.51} indicates ‘all without exception’, while \REF{ex:4.52} is more neutral. 

\ea\label{ex:4.52}
\gll {\ꞌ}I te mahana nei te ŋā poki \textbf{ta{\ꞌ}ato{\ꞌ}a} he porotē. \\
at \textsc{art} day \textsc{prox} \textsc{art} \textsc{pl} child all \textsc{ntr} parade \\

\glt
‘Today all the children participate in the parade.’
\z

As \REF{ex:4.51} and \REF{ex:4.52} show, the \isi{noun phrase} containing \textit{ta{\ꞌ}ato{\ꞌ}a} may include a plural marker. In most cases, however, no plural marker is used; the use of \textit{ta{\ꞌ}ato{\ꞌ}a} itself is a sufficient indication of the plurality of the \isi{noun phrase}. 

\subparagraph{\textit{Ta{\ꞌ}ato{\ꞌ}a te N}} Very occasionally, \textit{ta{\ꞌ}ato{\ꞌ}a} occurs before the article:

\ea\label{ex:4.53}
\gll \textbf{Ta{\ꞌ}ato{\ꞌ}a} \textbf{te} taŋata o Rapa Nui \textbf{i} oho ai ki te pure pāpaku\\
all \textsc{art} person of Rapa Nui \textsc{pfv} go \textsc{pvp} to \textsc{art} prayer corpse\\

\glt
‘All the people of Rapa Nui went to the funeral mass.’ \textstyleExampleref{[R231.349]}  
\z

This is merely a stylistic variant of \textit{te ta{\ꞌ}ato{\ꞌ}a N}. This construction is never used when the \isi{noun phrase} is preceded by a \isi{preposition}.

\subparagraph{\textit{Ta{\ꞌ}ato{\ꞌ}a N}} \textit{Ta{\ꞌ}ato{\ꞌ}a}\is{taatoa ‘all’@ta{\ꞌ}ato{\ꞌ}a ‘all’} often occurs before the noun without a determiner. This is only possible when the \isi{noun phrase} is not preceded by a \isi{preposition}: prepositions require a deteminer to be present (\sectref{sec:5.3.2.1}). \textit{Ta{\ꞌ}ato{\ꞌ}a} without determiner mostly occurs in noun phrases at the beginning of the sentence or clause, as in \REF{ex:4.54}. However, it may occur further on in the sentence as well, as in \REF{ex:4.55}.

\ea\label{ex:4.54}
\gll \textbf{Ta{\ꞌ}ato{\ꞌ}a} me{\ꞌ}e rakerake e haka aŋa era ki a Puakiva. \\
all thing bad:\textsc{red} \textsc{ipfv} \textsc{caus} do \textsc{dist} to \textsc{prop} Puakiva \\

\glt 
‘He made Puakiva do all sorts of bad jobs.’ \textstyleExampleref{[R229.397]} 
\z

\ea\label{ex:4.55}
\gll E haŋa koe \textbf{ta{\ꞌ}ato{\ꞌ}a} nō manu, \textbf{ta{\ꞌ}ato{\ꞌ}a} nō {\ꞌ}animare. \\
\textsc{exh} love \textsc{2sg} all just bird all just animal \\

\glt
‘Love all the birds, all the animals.’ \textstyleExampleref{[R213.026]} 
\z

This use of \textit{ta{\ꞌ}ato{\ꞌ}a} may indicate a more generic, less exact quantification, without establishing a precisely defined group: ‘all sorts of, everything, whatever’.

\subparagraph{\textit{(Te) ta{\ꞌ}ato{\ꞌ}a}} \textit{Ta{\ꞌ}ato{\ꞌ}a} may occur without an accompanying noun, i.e. in a headless\is{Noun phrase!headless} \isi{noun phrase}\is{Noun phrase!headless} (\sectref{sec:5.6}). In this case, it can be translated as ‘all, everyone, the totality’. 

Headless \textit{ta{\ꞌ}ato{\ꞌ}a}\is{taatoa ‘all’@ta{\ꞌ}ato{\ꞌ}a ‘all’} may occur either with or without article:

\ea\label{ex:4.56}
\gll E aŋa tahi a ia i te ŋā me{\ꞌ}e nei mo te rivariva o \textbf{te} \textbf{ta{\ꞌ}ato{\ꞌ}a}. \\
\textsc{exh} do all \textsc{prop} \textsc{3sg} \textsc{acc} \textsc{art} \textsc{pl} thing \textsc{prox} for \textsc{art} good:\textsc{red} of \textsc{art} all \\

\glt 
‘He should do all these things for the good of all.’ \textstyleExampleref{[R647.043]} 
\z

\ea\label{ex:4.57}
\gll \textbf{Ta{\ꞌ}ato{\ꞌ}a} e tahuti era, e tari mai era i te kai. \\
all \textsc{ipfv} run \textsc{dist} \textsc{ipfv} carry hither \textsc{dist} \textsc{acc} \textsc{art} food \\

\glt
‘All (people) ran, carrying the food.’ \textstyleExampleref{[R210.155]} 
\z

The choice between \textit{ta{\ꞌ}ato{\ꞌ}a} and \textit{te ta{\ꞌ}ato{\ꞌ}a}\is{taatoa ‘all’@ta{\ꞌ}ato{\ꞌ}a ‘all’} in headless\is{Noun phrase!headless} noun phrases\is{Noun phrase!headless} is partly syntactically determined: when the \isi{noun phrase} is preceded by a \isi{preposition}, there needs to be a determiner. This is the case in \REF{ex:4.56}. When the context does not require a determiner, the determiner tends to be left out, as in \REF{ex:4.57}. However, this is no absolute rule. 

\subparagraph{With pronoun} Finally, \textit{ta{\ꞌ}ato{\ꞌ}a}\is{taatoa ‘all’@ta{\ꞌ}ato{\ꞌ}a ‘all’} may quantify a pronoun; usually it appears after the pronoun:

\ea\label{ex:4.58}
\gll E koro, {\ꞌ}ī a \textbf{mātou} \textbf{ta{\ꞌ}ato{\ꞌ}a} ia. \\
\textsc{voc} Dad \textsc{imm} \textsc{prop} \textsc{1pl.excl} all then \\

\glt 
‘Dad, here we all are!’ \textstyleExampleref{[R237.051]} 
\z
\is{taatoa ‘all’@ta{\ꞌ}ato{\ꞌ}a ‘all’|)}
\subsection{\textit{Paurō} ‘each’}\label{sec:4.4.3}
\is{paurō ‘every’|(}
Like \textit{ta{\ꞌ}ato{\ꞌ}a}, \textit{paurō} ‘each, every, all’ is a newcomer in Rapa Nui, borrowed from \ili{Tahitian} \textit{pauroa}\is{\ili{Tahitian} influence}. Interestingly, it already occurs in Mtx and Egt, but only once in each. It is much more common in recent texts.

\textit{Paurō} usually precedes the determiner and is mostly used with temporal nouns like \textit{mahana} ‘day’, \textit{vece} ‘time, turn’, \textit{matahiti} ‘year’. Some examples:

\ea\label{ex:4.59}
\gll \textbf{Paurō} \textbf{te} \textbf{mahana} he turu au ki te hāpī. \\
every \textsc{art} day \textsc{ntr} go\_down \textsc{1sg} to \textsc{art} learn \\

\glt 
‘Every day I go to school.’ \textstyleExampleref{[R151.059]} 
\z

\ea\label{ex:4.60}
\gll E ko puē au mo {\ꞌ}a{\ꞌ}amu atu ki a kōrua \textbf{paurō} \textbf{te} \textbf{vece}. \\
\textsc{ipfv} \textsc{neg.ipfv} can \textsc{1sg} for tell away to \textsc{prop} \textsc{2pl} every \textsc{art} time \\

\glt 
‘I can’t tell you every time.’ \textstyleExampleref{[R201.009]} 
\z

\ea\label{ex:4.61}
\gll E rua ŋā vi{\ꞌ}e \textbf{paurō} \textbf{te} \textbf{pō} māhina {\ꞌ}omotohi e vari era  ki te ika hī.\\
\textsc{num} two \textsc{pl} woman every \textsc{art} night moon full\_moon \textsc{ipfv} pass \textsc{dist}  to \textsc{art} fish to\_fish\\

\glt
‘There were two woman who went fishing every night of a full moon.’ \textstyleExampleref{[R532-12.001]}
\z

Occasionally \textit{paurō}\is{paurō ‘every’} is used with other nouns, mostly after the noun. In these cases it is equivalent to \textit{ta{\ꞌ}ato{\ꞌ}a}\is{taatoa ‘all’@ta{\ꞌ}ato{\ꞌ}a ‘all’}:

\ea\label{ex:4.62}
\gll Te aŋa he ri{\ꞌ}ari{\ꞌ}a nō \textbf{te} \textbf{taŋata} \textbf{paurō} {\ꞌ}i tū ŋā tahutahu era. \\
\textsc{art} do \textsc{pred} afraid just \textsc{art} man every at \textsc{dem} \textsc{pl} witch \textsc{dist} \\

\glt 
‘All the people were continually afraid of those witches.’ \textstyleExampleref{[R233.007]} 
\z
\is{paurō ‘every’|)}
\subsection{\textit{Ananake} ‘together’}\label{sec:4.4.4}
\is{ananake ‘together’|(}
In old texts, \textit{ananake} is the most common \isi{quantifier}\is{Quantifier}; in these texts it has a wide range of uses, much like \textit{ta{\ꞌ}ato{\ꞌ}a} nowadays. In modern Rapa Nui, the use of \textit{ananake} is semantically and syntactically restricted. In the following sections, these two stages are discussed separately.

\textit{Ananake} does not occur in other languages, but the simple form \textit{anake}\is{anake ‘all’}\footnote{\label{fn:183}\textit{Anake} also occurs in Rapa Nui, but only in older texts.} is widespread in Polynesian ({\textless} PNP \textit{*anake} = ‘completely, only’). \textit{Ananake} may have developed from \textit{anake} by analogy of the definite numerals (\sectref{sec:4.3.4})\is{Numeral!definite}: the development \textit{anake} {\textgreater} \textit{ananake} is very similar to \textit{(a)~toru} {\textgreater} \textit{a~totoru}. This would explain the otherwise unattested \isi{reduplication}\is{Reduplication} pattern, in which the penultimate \isi{syllable}\is{Syllable} of a three-\isi{syllable}\is{Syllable} word is reduplicated. As discussed in \sectref{sec:4.3.4} above, \textit{ananake} shows similarities in use to the definite numerals.

\subsubsection[Modern use]{Modern use}\label{sec:4.4.4.1}

The modern sense of \textit{ananake}\is{ananake ‘together’} is ‘together, all together’. It is mostly used pronominally: \textit{ananake} is not accompanied by a noun, nor preceded by an article. Its referent must be known from the preceding context. Some examples:

\ea\label{ex:4.63}
\gll He nonoho rō {\ꞌ}ai \textbf{ananake} {\ꞌ}i {\ꞌ}Ohovehi. \\
\textsc{ntr} \textsc{pl}:stay \textsc{emph} \textsc{subs} together at Ohovehi \\

\glt 
‘They lived together in Ohovehi.’ \textstyleExampleref{[R310.481]} 
\z

\ea\label{ex:4.64}
\gll {\ꞌ}I tū hora era \textbf{ananake} i u{\ꞌ}i ai rū{\ꞌ}au rima kore. \\
at \textsc{dem} time \textsc{dist} together \textsc{pfv} look \textsc{pvp} old\_woman hand lack \\

\glt 
‘At that moment they all (together) saw that the old woman had no hands.’ \textstyleExampleref{[R437.085]} 
\z

\ea\label{ex:4.65}
\gll He {\ꞌ}amo te {\ꞌ}ura \textbf{ananake}. \\
\textsc{ntr} carry \textsc{art} lobster together \\

\glt
‘Together they carried the lobsters.’ \textstyleExampleref{[R410.045]} 
\z

As these examples show, \textit{ananake} may occur after the verb in the subject position as in \REF{ex:4.63}, but also before the verb as in \REF{ex:4.64}, or at the end of the clause as in \REF{ex:4.65}.

Regardless of its position in the clause, \textit{ananake} always refers to the subject. For example, \REF{ex:4.65} does not mean ‘they carried all the lobsters’. Now this also has a semantic reason: \textit{ananake} normally has human reference; it is uncommon for \textit{ananake}\is{ananake ‘together’} to be used for animals or inanimate things. 

Another current use of \textit{ananake}\is{ananake ‘together’} is in the comitative\is{Comitative} construction (\sectref{sec:8.10.3}).

\subsubsection{\textit{Ananake} in older Rapa Nui}\label{sec:4.4.4.2}

The modern pronominal use, in which \textit{ananake} quantifies an implied subject, already occurs in older texts. More commonly, however, \textit{ananake}\is{ananake ‘together’} is used in these texts as a \isi{quantifier}\is{Quantifier} within the \isi{noun phrase}. This syntactic difference between the old and the modern language coincides with a semantic difference: while in modern Rapa Nui \textit{ananake} means ‘together’, in older texts it is a universal \isi{quantifier} ‘all’, a sense nowadays expressed by \textit{ta{\ꞌ}ato{\ꞌ}a}\is{taatoa ‘all’@ta{\ꞌ}ato{\ꞌ}a ‘all’} and \textit{paurō}\is{paurō ‘every’}.

Just like \textit{ta{\ꞌ}ato{\ꞌ}a} nowadays, \textit{ananake} in the older language may occur after a noun or pronoun:

\ea\label{ex:4.66}
\gll He hīhiŋa te mōai \textbf{ananake}. \\
\textsc{ntr} \textsc{pl}:fall \textsc{art} statue all \\

\glt 
‘All the statues fell.’ \textstyleExampleref{[Mtx-4-05.060]}
\z

\ea\label{ex:4.67}
\gll Ka oho mai kōrua \textbf{ananake}, he mate au. \\
\textsc{imp} go hither \textsc{2pl} all \textsc{ntr} die \textsc{1sg} \\

\glt
‘(The king said to his children:) Come, all of you, I am dying.’ \textstyleExampleref{[Ley-2-08.009]}
\z

It also occurs before the noun; in that case it precedes the article \textit{te}. 

\ea\label{ex:4.68}
\gll \textbf{Ananake} te mata ana haka uŋa e tahi taŋata. \\
all \textsc{art} tribe \textsc{irr} \textsc{caus} send \textsc{num} one man \\

\glt 
‘All the tribes sent one man.’ \textstyleExampleref{[Ley-5-36.001]}
\z

\ea\label{ex:4.69}
\gll He oho tau nuahine era \textbf{ananake} te motu. \\
\textsc{ntr} go \textsc{dem} old\_woman \textsc{dist} all \textsc{art} islet \\

\glt 
‘The old woman went to all the islets.’ \textstyleExampleref{[Mtx-3-06.045]}
\z

\ea\label{ex:4.70}
\gll E taū era \textbf{ananake} te ra{\ꞌ}ā. \\
\textsc{ipfv} fight \textsc{dist} all \textsc{art} day \\

\glt
‘They fought every day.’ \textstyleExampleref{[Mtx-3-05.006]}
\z

Prenominal \textit{ananake} is never preceded by a \isi{preposition}. Even so, the examples show that it may occur in noun phrases with a variety of functions, for example subject as in \REF{ex:4.68}, locational\is{Locational} adjunct as in \REF{ex:4.69}, or temporal adjunct as in \REF{ex:4.70}. But \textit{ananake}\is{ananake ‘together’} \textit{te N} is especially common with nouns denoting place or time, as in (\ref{ex:4.69}–\ref{ex:4.70}), a construction that has been taken over by \textit{paurō}\is{paurō ‘every’} \textit{te N} nowadays. 
\is{ananake ‘together’|)}

\subsection{\textit{Rauhuru} ‘diverse’}\label{sec:4.4.5}
\is{rauhuru ‘diverse’|(}
\textit{Rauhuru} means ‘diverse, manifold, many kinds’. It is a recent word, derived from \textit{rau}\is{rau ‘hundred’} ‘one hundred (archaic)’ + \textit{huru} ‘kind, sort’. Like \textit{ta{\ꞌ}ato{\ꞌ}a}, it occurs before and after the noun, with and without article, preceding and following the article.

\subparagraph{\textit{Rauhuru te N}}

\ea\label{ex:4.71}
\gll I noho era te oromatu{\ꞌ}a {\ꞌ}i nei, he take{\ꞌ}a \textbf{rauhuru} te me{\ꞌ}e mātāmu{\ꞌ}a. \\
\textsc{pfv} stay \textsc{dist} \textsc{art} priest at \textsc{prox} \textsc{ntr} see diverse \textsc{art} thing past \\

\glt 
‘When the priest lived here, he saw manifold things of the past.’ \textstyleExampleref{[R423.021]} 
\z

\subparagraph{\textit{Te rauhuru N}}

\ea\label{ex:4.72}
\gll {\ꞌ}I te hora nei he vānaŋa a tātou o ruŋa i te \textbf{rauhuru} \textbf{aŋa}  o te taŋata.\\
at \textsc{art} time \textsc{prox} \textsc{ntr} talk \textsc{prop} \textsc{1pl.incl} of above at \textsc{art} diverse work  of \textsc{art} man\\

\glt 
‘Now we are going to talk about the different kinds of work of people.’ \textstyleExampleref{[R334.203]} 
\z

\subparagraph{\textit{Rauhuru N}}

\ea\label{ex:4.73}
\gll ...mo aŋa \textbf{rauhuru} \textbf{me{\ꞌ}e} rivariva haŋa {\ꞌ}ā{\ꞌ}ana \\
~~~for do diverse thing good:\textsc{red} want \textsc{poss.3sg.a} \\

\glt
‘... to do all sorts of good things which he wants’ \textstyleExampleref{[2 Tim. 3:17]}
\z

\is{mee ‘thing’@me{\ꞌ}e ‘thing’}The use or non-use of the article is partly determined by syntax: after prepositions the article is obligatory. Partly it is a matter of style; the article is more common in this construction in certain texts than in others. 

\subparagraph{\textit{(Te) N rauhuru}} The postnominal use of \textit{rauhuru} is limited to some speakers. The article may or may not be used.

\ea\label{ex:4.74}
\gll Ko rahi {\ꞌ}ana te huru \textbf{rauhuru} o te kahu {\ꞌ}e tao{\ꞌ}a. \\
\textsc{prf} many \textsc{cont} \textsc{art} manner diverse of \textsc{art} cloth(es) and object \\

\glt 
‘There are many kinds (lit. many are the different kinds) of clothes and things.’ \textstyleExampleref{[R539-2.28]}
\z

\ea\label{ex:4.75}
\gll Te aŋa {\ꞌ}a Paio he {\ꞌ}oka ha{\ꞌ}a{\ꞌ}apu \textbf{rauhuru}. \\
\textsc{art} work of\textsc{.a} Paio \textsc{pred} plant crops diverse \\

\glt 
‘Paio’s work was planting all kinds of crops.’ \textstyleExampleref{[R439.005]} 
\z

\subparagraph{As a noun} Finally, \textit{rauhuru}\is{rauhuru ‘diverse’} itself can also be used as a noun, followed by a possessive phrase:

\ea\label{ex:4.76}
\gll Te aŋa {\ꞌ}ā{\ꞌ}ana he {\ꞌ}oka i te kai, i \textbf{te} \textbf{rauhuru} \textbf{o} \textbf{te} \textbf{me{\ꞌ}e}. \\
\textsc{art} work \textsc{poss.3sg.a} \textsc{pred} plant \textsc{acc} \textsc{art} food \textsc{acc} \textsc{art} diverse of \textsc{art} thing \\

\glt
‘His work was planting food, all kinds of things’ \textstyleExampleref{[R444.015–016]}
\z

Nominalised \textit{rauhuru} may or may not be preceded by the article. Again, this choice is partly prescribed by the syntax, partly free. 
\is{rauhuru ‘diverse’|)}
\subsection{\textit{Tētahi} ‘some, other’}\label{sec:4.4.6}
\is{tetahi ‘some’@tētahi ‘some’|(}
In this section first the syntax of \textit{tētahi} will be discussed (\sectref{sec:4.4.6.1}), then its meaning (\sectref{sec:4.4.6.2}).

\subsubsection[Syntax of tētahi: te + tahi?]{Syntax of \textit{tētahi}: \textit{te} + \textit{tahi}?}\label{sec:4.4.6.1}

\textit{Tētahi} ‘some, other’ is an ambiguous element. Its origin is clear: the article \textit{te}\is{te (article)} + the numeral \textit{tahi} ‘one’. \textit{Tētahi} still betrays this origin when it occurs after prepositions:

\ea\label{ex:4.77}
\gll \textbf{{\ꞌ}I} \textbf{tētahi} mahana ana ta{\ꞌ}o haka{\ꞌ}ou te tātou {\ꞌ}umu. \\
at some/other day \textsc{irr} cook\_in\_earth\_oven again \textsc{art} \textsc{1pl.incl} earth\_oven \\

\glt 
‘Another day we will cook in the (lit. our) earth oven again.’ \textstyleExampleref{[R333.546]} 
\z

\ea\label{ex:4.78}
\gll Ko māhani {\ꞌ}ana \textbf{ki} \textbf{tētahi} ŋā poki era. \\
\textsc{prf} accustom \textsc{cont} to some/other \textsc{pl} child \textsc{dist} \\

\glt 
‘She had gotten used to the other children.’ \textstyleExampleref{[R151.018]} 
\z

\ea\label{ex:4.79}
\gll He mate te manava ki te mā{\ꞌ}aŋa hāŋai \textbf{o} \textbf{tētahi} taŋata. \\
\textsc{ntr} die \textsc{art} stomach to \textsc{art} chick feed of some/other man \\

\glt
‘She fell in love (lit. the stomach died) with the adopted child (lit. the chick fed/raised) of another man.’ \textstyleExampleref{[Mtx-5-04.002]}
\z

These prepositions are obligatory followed by a determiner (\sectref{sec:5.3.2.1}). The fact that they can be followed by \textit{tētahi} shows that in these cases \textit{tētahi} contains a determiner, the most natural explanation being that \textit{tētahi} consists of the article \textit{te} followed by \textit{tahi}.

Yet in other cases \textit{tētahi}\is{tetahi ‘some’@tētahi ‘some’} does not incorporate a determiner. It can be preceded by determiners, such as the article \textit{te} \REF{ex:4.80} or a demonstrative \REF{ex:4.81}:

\ea\label{ex:4.80}
\gll {\ꞌ}Ina ko oho ki \textbf{te} \textbf{tētahi} \textbf{kona}. \\
\textsc{neg} \textsc{neg.ipfv} go to \textsc{art} some/other place \\

\glt 
‘Don’t go to another place.’ \textstyleExampleref{[R481.135]} 
\z

\ea\label{ex:4.81}
\gll He oho tahi ananake ko \textbf{tū} \textbf{tētahi} ŋā poki era. \\
\textsc{ntr} go all together \textsc{prom} \textsc{dem} some/other \textsc{pl} child \textsc{dist} \\

\glt
‘He went together with those other boys.’ \textstyleExampleref{[R161.027]} 
\z

Also, \textit{tētahi}\is{tetahi ‘some’@tētahi ‘some’} may follow the \isi{preposition} \textit{hai}\is{hai (instrumental prep.)}, a \isi{preposition} which is never followed by a determiner (\sectref{sec:4.7.9}):

\ea\label{ex:4.82}
\gll A Kontiki tako{\ꞌ}a i hā{\ꞌ}ū{\ꞌ}ū mai hai tara {\ꞌ}e \textbf{hai} \textbf{tētahi} atu me{\ꞌ}e. \\
\textsc{prop} Kontiki also \textsc{pfv} help hither \textsc{ins} money and \textsc{ins} some/other away thing \\

\glt
‘Kontiki (=Thor Heyerdahl) also helped with money and with other things.’ \textstyleExampleref{[R375.094]} 
\z

We may conclude that \textit{tētahi} has – at least in these cases – undergone a process of reanalysis and turned into a monomorphemic \isi{quantifier}\is{Quantifier} which no longer includes a determiner. 

\subsubsection[Use of tētahi]{Use of \textit{tētahi}}\label{sec:4.4.6.2}
\is{tetahi ‘some’@tētahi ‘some’}
\textit{Tētahi} can be used with singular nouns in the sense ‘another’:

\ea\label{ex:4.83}
\gll E hoki mai ho{\ꞌ}i koe {\ꞌ}i \textbf{tētahi} mahana. \\
\textsc{exh} return hither indeed \textsc{2sg} at some/other day \\

\glt
‘Come back another day.’ \textstyleExampleref{[R344.034]} 
\z

More commonly, the noun has a plural sense, and \textit{tētahi} means ‘some’ or ‘others’:

\ea\label{ex:4.84}
\gll \textbf{Tētahi} mahana, e ha{\ꞌ}uru era {\ꞌ}i ruŋa o te {\ꞌ}one. \\
some/other day \textsc{ipfv} sleep \textsc{dist} at above of \textsc{art} sand \\

\glt 
‘Some days he would sleep on the ground.’ \textstyleExampleref{[R309.060]} 
\z

\ea\label{ex:4.85}
\gll ¿Ko ai rā nei te \textbf{tētahi} nu{\ꞌ}u era? \\
~\textsc{prom} who \textsc{intens} \textsc{prox} \textsc{art} some/other people \textsc{dist} \\

\glt
‘Who are those other people?’ \textstyleExampleref{[R414.075]} 
\z

Multiple noun phrases can be conjoined in juxtaposed clauses using \textit{tētahi}\is{tetahi ‘some’@tētahi ‘some’} \textit{... tētahi}: ‘some ... others’:

\ea\label{ex:4.86}
\gll ...\textbf{tētahi} ŋā poki tane nunui he hāpī mo haka taŋi i te kītara.  \textbf{Tētahi} ŋā poki he hāpī i te {\ꞌ}ori rapa nui, \textbf{tētahi} haka{\ꞌ}ou mo {\ꞌ}ori i te cueca.\\
~~~some/other \textsc{pl} child male \textsc{pl}:big \textsc{ntr} learn for \textsc{caus} cry \textsc{acc} \textsc{art} guitar some/other \textsc{pl} child \textsc{ntr} learn \textsc{acc} \textsc{art} dance shine crouch some/other again for dance \textsc{acc} \textsc{art} \textit{cueca}\\

\glt
‘...some bigger boys learn to play the guitar. Other children learn Rapa Nui dancing, yet others dancing the \textit{cueca}.’ \textstyleExampleref{[R334.130–131]}
\z

As the last clause in \REF{ex:4.86} shows, \textit{tētahi}\is{tetahi ‘some’@tētahi ‘some’} can also be used without a following noun.
\is{tetahi ‘some’@tētahi ‘some’|)}

\subsection{\textit{Me{\ꞌ}e rahi} and \textit{rahi} ‘much, many’}\label{sec:4.4.7}
\is{mee rahi ‘many’@me{\ꞌ}e rahi ‘many’|(}\subsubsection{\textit{Me{\ꞌ}e rahi}: from \isi{noun phrase} to quantifier}\label{sec:4.4.7.1}
\is{Quantifier}
\textit{me{\ꞌ}e rahi}, lit. ‘many things’, is originally a \isi{noun phrase}, consisting of the noun \textit{me{\ꞌ}e} ‘thing’, modified by the adjective \textit{rahi} ‘much/many’. The few times when it is used in older texts (there are only four occurrences), it is used as such. In the following example, the \isi{noun phrase} \textit{me{\ꞌ}e rahi} is in initial position as the predicate of an attributive clause,\footnote{\label{fn:184}Attributive clauses commonly have the dummy noun \textit{me{\ꞌ}e} as anchor of the predicate adjective (\sectref{sec:9.2.7}).} followed by the subject \isi{noun phrase}.

\ea\label{ex:4.87}
\gll \textbf{Me{\ꞌ}e} \textbf{rahi} te manu o ruŋa. \\
thing many/much \textsc{art} bird of above \\

\glt
‘There were many birds (lit. many [were] the birds) on (the island).’ \textstyleExampleref{[Egt-02.083]}
\z

This example has the same structure as the attributive clause below (\sectref{sec:9.2.7}):

\ea\label{ex:4.88}
\gll Me{\ꞌ}e paŋaha{\ꞌ}a te kūmara.\\
thing heavy \textsc{art} sweet\_potato\\

\glt 
‘Sweet potatoes are heavy (food).’ \textstyleExampleref{[Ley-5-24.008]}
\z

Nowadays \textit{me{\ꞌ}e rahi}\is{mee rahi ‘many’@me{\ꞌ}e rahi ‘many’} is still used in the same way, i.e. as a predicate of an attributive clause. If this construction contains a verb, possibly with other arguments, this is constructed as a \isi{relative clause}\is{Clause!relative} following the subject.

\ea\label{ex:4.89}
\gll \textbf{Me{\ꞌ}e} \textbf{rahi} te taŋata [i mate {\ꞌ}i rā noho iŋa]. \\
thing many/much \textsc{art} man \textsc{~pfv} die at \textsc{dist} stay \textsc{nmlz} \\

\glt
‘Many people died (lit. many [were] the people who died) at that time.’ \textstyleExampleref{[R250.093]} 
\z

However, this is not the most common way in which \textit{me{\ꞌ}e rahi} is used nowadays. It has also developed into a frozen form which as a whole functions as a \isi{quantifier}\is{Quantifier}, occupying the \isi{quantifier}\is{Quantifier} position in the \isi{noun phrase}. \textit{Me{\ꞌ}e rahi} as a \isi{quantifier}\is{Quantifier} is distinguished by the following characteristics:

\begin{itemize}
\item 
Unlike the examples above, it is not followed by the article, but directly precedes the noun. Unlike most other quantifiers, \textit{me{\ꞌ}e rahi} cannot be preceded by the article either.

\item 
It does not need to occur clause-initially, but occurs in noun phrases in different positions in the clause; the \isi{noun phrase} may be subject \REF{ex:4.90}, direct object \REF{ex:4.91}, oblique \REF{ex:4.92}, time adjunct \REF{ex:4.93}. In all cases, the \isi{noun phrase} is not marked by a \isi{preposition} (\sectref{sec:5.3.2.1}).

\end{itemize}

\ea\label{ex:4.90}
\gll \textbf{Me{\ꞌ}e} \textbf{rahi} \textbf{nu{\ꞌ}u} i māmate. \\
thing many/much people \textsc{pfv} \textsc{pl}:die \\

\glt 
‘Many people died.’ \textstyleExampleref{[R532-05.002]}
\z

\ea\label{ex:4.91}
\gll Ko {\ꞌ}amo {\ꞌ}ana \textbf{me{\ꞌ}e} \textbf{rahi} \textbf{nō} \textbf{atu} {\ꞌ}ati. \\
\textsc{prf} carry \textsc{cont} thing many/much just away problem \\

\glt 
‘They have carried many kinds of sufferings.’ \textstyleExampleref{[1 Tim. 6:10]}
\z

\ea\label{ex:4.92}
\gll He tuha{\ꞌ}a te henua \textbf{me{\ꞌ}e} \textbf{rahi} \textbf{taŋata} mo {\ꞌ}oka i te rāua tarake. \\
\textsc{ntr} distribute \textsc{art} land thing many/much man for plant \textsc{acc} \textsc{art} \textsc{3pl} corn \\

\glt 
‘They distributed land to many people to plant corn.’ \textstyleExampleref{[R424.013]} 
\z

\ea\label{ex:4.93}
\gll {\ꞌ}I te kona nei i noho ai \textbf{me{\ꞌ}e} \textbf{rahi} \textbf{mahana}. \\
at \textsc{art} place \textsc{prox} \textsc{pfv} stay \textsc{pvp} thing many/much day \\

\glt
‘In this place this stayed many days.’ \textstyleExampleref{[R420.055]} 
\z

These examples show that reanalysis has taken place. As discussed above, in older Rapa Nui \textit{me{\ꞌ}e rahi}\is{mee rahi ‘many’@me{\ꞌ}e rahi ‘many’} was the predicate of a nominal clause\is{Clause!nominal}, optionally containing a \isi{relative clause}\is{Clause!relative}:
\ea \label{ex:4.i}
[ \textit{Me{\ꞌ}e rahi} ]\textsubscript{NP} [ \textit{te N} ([ \textit{i V} ]\textsubscript{Rel}) ]\textsubscript{NP} 
\z
This construction was reanalysed to a simple clause with initial subject, in which \textit{me{\ꞌ}e rahi} is a \isi{quantifier}\is{Quantifier} occurring before the article, by analogy of other quantifiers\is{Quantifier} which may occur in the same position (e.g. \textit{ta{\ꞌ}ato{\ꞌ}a te N}, \sectref{sec:4.4.2}):
\ea \label{ex:4.ii}
 [ \textit{Me{\ꞌ}e rahi te N} ]\textsubscript{NP} [ \textit{i V} ]\textsubscript{VP} 
\z
Once \textit{me{\ꞌ}e rahi} is part of the \isi{noun phrase}, the way is open for two developments:

%\setcounter{listWWviiiNumcviileveli}{0}
\begin{enumerate}
\item 
The determiner can be left out, as in \REF{ex:4.90} above:

\end{enumerate}
\ea \label{ex:4.iii}
 [ \textit{Me{\ꞌ}e rahi N} ]\textsubscript{NP} [ \textit{i V} ]\textsubscript{VP}
\z
\begin{enumerate}
\setcounter{enumi}{1}
\item 
\textit{me{\ꞌ}e rahi} may occur in non-initial noun phrases with different semantic roles, as in (\ref{ex:4.91}–\ref{ex:4.93}) above. 

\end{enumerate}

There is still one difference with quantifiers\is{Quantifier} like \textit{ta{\ꞌ}ato{\ꞌ}a}: \textit{me{\ꞌ}e rahi} is not preceded by the article. If the article is used, it follows \textit{me{\ꞌ}e rahi}.
\is{mee rahi ‘many’@me{\ꞌ}e rahi ‘many’|)}
\subsubsection{\textit{Rahi}\textit{} ‘many, much’}\label{sec:4.4.7.2}
\is{rahi ‘much/many’|(}
\textit{Rahi} is used in the expression \textit{me{\ꞌ}e rahi} (see above), but also has a number of other uses.

\textit{Rahi} occurs in older texts, but not nearly as frequently as in modern Rapa Nui.\footnote{\label{fn:185}In older texts, \textit{rahi} (including \textit{me{\ꞌ}e rahi}) occurs 20x (once per 6,100 words), in newer texts it occurs 896x (once per 410 words).} Though the word occurs throughout Polynesia, Rapa Nui may have borrowed it from \ili{Tahitian}, or extended its usage under the influence of \ili{Tahitian}.\footnote{\label{fn:186}The form of the word would be the same, whether inherited or borrowed.}  Apart from the marked increase in use, another indication for \ili{Tahitian} influence is the fact that \textit{rahi} can be followed by the \ili{Tahitian} nominaliser \textit{-ra{\ꞌ}a}. 

\subparagraph{Predicate} In older texts, \textit{rahi} is mainly used as a verbal/\isi{adjectival} predicate.

\ea\label{ex:4.94}
\gll \textbf{Ku} \textbf{rahi} \textbf{{\ꞌ}ā} te mamae o te vi{\ꞌ}e {\ꞌ}a Tau {\ꞌ}a Ure rāua ko  tā{\ꞌ}ana poki.\\
\textsc{prf} many/much \textsc{cont} \textsc{art} pain of \textsc{art} woman of\textsc{.a} Tau a Ure \textsc{3pl} \textsc{prom}  \textsc{poss.3sg.a} child\\

\glt
‘Tau a Ure’s wife and her child were in much pain (lit. Much was the pain of...)’ \textstyleExampleref{[Ley-9-63.019]}
\z

This usage is still common nowadays. \textit{Rahi}\is{rahi ‘much/many’}, preceded by an aspectual marker\is{Aspect marker}, can be the predicate of either a \isi{main clause} or a \isi{relative clause} after the noun:

\ea\label{ex:4.95}
\gll \textbf{Ko} \textbf{rahi} \textbf{{\ꞌ}ana} te mahana {\ꞌ}ina e tahi me{\ꞌ}e mo kai. \\
\textsc{prf} many/much \textsc{cont} \textsc{art} day \textsc{neg} \textsc{num} one thing for eat \\

\glt 
‘Many days there was nothing to eat.’ \textstyleExampleref{[R303.029]} 
\z

\ea\label{ex:4.96}
\gll He take{\ꞌ}a i te nu{\ꞌ}u \textbf{ko} \textbf{rahi} \textbf{{\ꞌ}ā} {\ꞌ}i roto i te hare. \\
\textsc{ntr} see \textsc{acc} \textsc{art} people \textsc{prf} many/much \textsc{cont} at inside at \textsc{art} house \\

\glt 
‘He saw that there were many people in the house.’ \textstyleExampleref{[R229.295]} 
\z

\subparagraph{Adverb} \textit{Rahi}\is{rahi ‘much/many’} often serves as \isi{adverb}\is{Adverb}, modifying a verb or adjective: ‘a lot, very (much)’. When modifying a verb, \textit{rahi} often implies quantification of the subject or object of the verb (in the same way as \textit{tahi} ‘all’, \sectref{sec:4.4.9}). E.g. in \REF{ex:4.98}, \textit{rahi} does not refer to many acts of seeing, but quantifies the object of seeing.

\ea\label{ex:4.97}
\gll E topa \textbf{rahi} era te {\ꞌ}ua he ai te mau o te mahiŋo. \\
\textsc{ipfv} descend many/much \textsc{dist} \textsc{art} rain \textsc{ntr} exist \textsc{art} abundance of \textsc{art} people \\

\glt 
‘When a lot of rain fell, the people had abundance.’ \textstyleExampleref{[Fel-19.006]}
\z

\ea\label{ex:4.98}
\gll {\ꞌ}Ina he take{\ꞌ}a \textbf{rahi} i te taŋata. \\
\textsc{neg} \textsc{ntr} see many/much \textsc{acc} \textsc{art} man \\

\glt 
‘He did not see many people.’ \textstyleExampleref{[R459.003]} 
\z

\subparagraph{Noun} When \textit{rahi}\is{rahi ‘much/many’} is used a noun, it means either ‘the many, the large number’ or ‘the majority, most’. The counted entity is expressed as a genitive phrase after \textit{rahi}.\footnote{\label{fn:187}Some speakers use \textit{rahira{\ꞌ}a} in the same senses, either ‘the many’ or ‘the majority’. (\textit{-ra{\ꞌ}a} is the \ili{Tahitian} nominaliser).}

\ea\label{ex:4.99}
\gll Ka u{\ꞌ}i rā koe i \textbf{te} \textbf{rahi} ena o te pua{\ꞌ}a ena mo tatau. \\
\textsc{imp} look \textsc{intens} \textsc{2sg} \textsc{acc} \textsc{art} many/much \textsc{med} of \textsc{art} cattle \textsc{med} for squeeze \\

\glt 
‘Look how many cows there are to milk.’ \textstyleExampleref{[R245.186]} 
\z

\ea\label{ex:4.100}
\gll {\ꞌ}I \textbf{te} \textbf{rahi} o te nehenehe i tupu ai, {\ꞌ}ina he take{\ꞌ}a mai te hakari o te tumu.\\
at \textsc{art} many/much of \textsc{art} fern \textsc{pfv} grow \textsc{pvp} \textsc{neg} \textsc{ntr} see hither \textsc{art} body of \textsc{art} tree\\

\glt 
‘Because of the many ferns, the body of the tree cannot be seen.’ \textstyleExampleref{[R497.005]} 
\z

\subparagraph{Adjective} Finally, \textit{rahi}\is{rahi ‘much/many’} is used adjectivally, i.e. as a noun modifier. As discussed in \sectref{sec:4.4.7}, in older texts the expression \textit{me{\ꞌ}e rahi} is found occasionally, in which \textit{rahi} is an adjective to the generic noun \textit{me{\ꞌ}e}. There is only one example in these texts of \textit{rahi} modifying a noun other than \textit{me{\ꞌ}e}:

\ea\label{ex:4.101}
\gll He to{\ꞌ}o mai i te moa, moa \textbf{rahi}. \\
\textsc{ntr} take hither \textsc{acc} \textsc{art} chicken chicken many/much \\

\glt
‘They took chickens, many chickens.’ \textstyleExampleref{[Ley-9-55.012]}
\z

Though one example does not carry too much weight, it is interesting to note that the adjective is not simply put after the noun \textit{moa}. Rather, \textit{moa} is repeated as an \isi{apposition}\is{Apposition}, yielding a sort of predicate \isi{noun phrase} to which \textit{rahi} is added. (Appositions in Rapa Nui are similar to predicate noun phrases.)

The use of \textit{rahi} as an adjective thus seems to be a recent development. \textit{Rahi} as an adjective is relatively common nowadays, though still not quite as common as the predicate and adverbial uses of \textit{rahi}. Speakers hesitate somewhat to use \textit{rahi} as an adjective; when they do so, it is often in situations where a construction with \textit{me{\ꞌ}e rahi} is difficult or impossible. \textit{Rahi}\is{rahi ‘much/many’} as an adjective is especially found in the following situations:

First: when the \isi{noun phrase} is preceded by a \isi{preposition} requiring a determiner.

\ea\label{ex:4.102}
\gll He ha{\ꞌ}ere mo haka ora {\ꞌ}i te rohirohi o tū \textbf{aŋa} \textbf{rahi} era.\\
\textsc{ntr} walk for \textsc{caus} live at \textsc{art} tired:\textsc{red} of \textsc{dem} work many/much \textsc{dist}\\

\glt
‘He went to rest from the fatigue of those many works.’ \textstyleExampleref{[R233.069]} 
\z

Second: when the \isi{quantifier} is negated by the constituent negator \textit{ta{\ꞌ}e}\is{tae (negator)@ta{\ꞌ}e (negator)}.

\ea\label{ex:4.103}
\gll Ika \textbf{ta{\ꞌ}e} \textbf{rahi} nō i rava{\ꞌ}a ai. \\
fish \textsc{conneg} many/much just \textsc{pfv} obtain \textsc{pvp} \\

\glt
‘They caught few fish.’ \textstyleExampleref{[R312.010]} 
\z

Third: when the noun modified by \textit{rahi} is itself a modifier:

\ea\label{ex:4.104}
\gll {\ꞌ}I te mahana \textbf{tokerau} \textbf{rahi}, e ko e{\ꞌ}a ki te ika hī. \\
at \textsc{art} day wind many/much \textsc{ipfv} \textsc{neg.ipfv} go\_out to \textsc{art} fish to\_fish \\

\glt
‘On days with much wind, (people) don’t go out fishing.’ \textstyleExampleref{[R334.254]} 
\z

Fourth: in predicate noun phrases, especially in attributive clauses\is{Clause!attributive}:

\ea\label{ex:4.105}
\gll \textbf{Nu{\ꞌ}u} \textbf{rahi} te nu{\ꞌ}u i mana{\ꞌ}u pē nei ē ko tētere {\ꞌ}ana ki Tahiti. \\
people many/much \textsc{art} people \textsc{pfv} think like \textsc{prox} thus \textsc{prf} \textsc{pl}:run \textsc{cont} to Tahiti \\

\glt
‘Many people (lit. many people were the people who) thought that they had fled to Tahiti.’ \textstyleExampleref{[R303.051]} 
\z

In fact, this is the same construction as \textit{me{\ꞌ}e rahi} when used as a \isi{noun phrase} (see (\ref{ex:4.87}–\ref{ex:4.89}) above).

Finally: with abstract nouns like \textit{riva} ‘goodness’, \textit{mamae} ‘pain’, \textit{haŋa} ‘love’, \textit{aŋa} ‘work’ and \textit{mana{\ꞌ}u} ‘thought’. \textit{Rahi} can be translated here as ‘much, great’:

\ea\label{ex:4.106}
\gll Te \textbf{pohe} \textbf{rahi} {\ꞌ}ā{\ꞌ}ana he haka piri he haka takataka  i te taŋata.\\
\textsc{art} desire many/much \textsc{poss.3sg.a} \textsc{pred} \textsc{caus} join \textsc{pred} \textsc{caus} gather:\textsc{red}  \textsc{acc} \textsc{art} man\\

\glt 
‘His great desire was to get people together.’ \textstyleExampleref{[R302.039]} 
\z

\ea\label{ex:4.107}
\gll Ko ai {\ꞌ}ā te \textbf{māuiui} \textbf{rahi} {\ꞌ}i nei {\ꞌ}i Rapa Nui. \\
\textsc{prf} exist \textsc{cont} \textsc{art} sick many/much at \textsc{prox} at Rapa Nui \\

\glt 
‘There is a severe disease here on Rapa Nui.’ \textstyleExampleref{[R398.002]} 
\z
\is{rahi ‘much/many’|)}

\subsection{Other quantifiers}\label{sec:4.4.8}
\is{Quantifier}\subsubsection{\textit{Kē} ‘some, others’}\label{sec:4.4.8.1}
\is{ke ‘different’@kē ‘different’}
\textit{Kē} is common as an adjective meaning ‘other, different’, but in modern Rapa Nui it also serves as a \isi{quantifier}\is{Quantifier} in the sense ‘some’ or ‘other(s)’. When used as a \isi{quantifier}\is{Quantifier}, it occurs before the noun; the \isi{noun phrase} has no determiner. 

\textit{Kē} is similar in meaning to \textit{tētahi}, but more than \textit{tētahi} it singles out a subgroup within a larger group. Often, two subgroups are juxtaposed: \textit{kē ... kē} ‘some ... others’. 

\ea\label{ex:4.108}
\gll \textbf{Kē} \textbf{ŋā} \textbf{poki} he oho he hohopu {\ꞌ}i raro o te rano. \\
different \textsc{pl} child \textsc{ntr} go \textsc{ntr} \textsc{pl}:bathe at below of \textsc{art} crater\_lake \\

\glt 
‘Some children went for a swim down in the crater lake.’ \textstyleExampleref{[R157.012]} 
\z

\ea\label{ex:4.109}
\gll \textbf{Kē} nu{\ꞌ}u he tu{\ꞌ}u, \textbf{kē} he māmate {\ꞌ}i vāeŋa {\ꞌ}ā o te ara. \\
different people \textsc{ntr} arrive different \textsc{ntr} \textsc{pl}:die at middle \textsc{ident} of \textsc{art} road \\

\glt
‘Some people arrived, others died during the voyage.’ \textstyleExampleref{[R303.002]} 
\z

This use of \textit{kē}\is{ke ‘different’@kē ‘different’} may be influenced by \ili{Spanish}, where quantifiers\is{Quantifier} like \textit{ciertos} and \textit{algunos} (both meaning ‘certain’) occur before the noun.

\subsubsection{\textit{Kā} ‘every’}\label{sec:4.4.8.2}
\is{ka ‘each’@kā ‘each’}
\textit{Kā} ‘every’ is an adaptation of \ili{Spanish}\is{\ili{Spanish} influence} \textit{cada}.\footnote{\label{fn:188}As intervocalic \textit{d} is pronounced very weakly in Chilean \ili{Spanish}, it tends to drop out completely in Rapa Nui borrowings\is{\ili{Spanish} influence} (\sectref{sec:2.5.3.1}).} It occurs before the noun and may be preceded by the article \textit{te}:

\ea\label{ex:4.110}
\gll {\ꞌ}I \textbf{te} \textbf{kā} \textbf{kona} e ai rō {\ꞌ}ā te {\ꞌ}āua va{\ꞌ}ehau. \\
at \textsc{art} each place \textsc{ipfv} exist \textsc{emph} \textsc{cont} \textsc{art} enclosure soldier \\

\glt 
‘In every place there was a garrison.’ \textstyleExampleref{[Notes]}
\z

\ea\label{ex:4.111}
\gll Ka tutututu tahi \textbf{kā} \textbf{hare} \textbf{ta{\ꞌ}ato{\ꞌ}a}. \\
\textsc{imp} set\_fire:\textsc{red} all each house all \\

\glt 
‘Burn every single house to the ground.’ \textstyleExampleref{[R368.059]} 
\z

Unlike \ili{Spanish} \textit{cada}, which precedes only singular nouns, \textit{kā}\is{ka ‘each’@kā ‘each’} is not limited to singulars: it may be followed by the plural marker \textit{ŋā} or the inherently plural noun \textit{nu{\ꞌ}u} ‘people’.

\ea\label{ex:4.112}
\gll He uru tahi \textbf{kā} \textbf{ŋā} \textbf{poki}. \\
\textsc{ntr} enter all each \textsc{pl} child \\

\glt 
‘All the children entered.’ \textstyleExampleref{[R151.016]} 
\z

\ea\label{ex:4.113}
\gll E noho era \textbf{kā} \textbf{nu{\ꞌ}u} {\ꞌ}i tō{\ꞌ}ona kona {\ꞌ}āua {\ꞌ}oka kai. \\
\textsc{ipfv} stay \textsc{dist} each people at \textsc{poss.3sg.o} place enclosure plant food \\

\glt 
‘Everyone lived at his plantation.’ \textstyleExampleref{[R107.038]} 
\z

\subsubsection{\textit{Pura} ‘mere, only’}\label{sec:4.4.8.3}
\is{pura ‘mere’}
\textit{Pura} is borrowed from \ili{Spanish} \textit{pura} (feminine of \textit{puro}) and means ‘mere, only, pure, sheer, simple’. It may or may not be preceded by the article or the predicate marker \textit{he}, depending on the syntactic requirements of the context.

\ea\label{ex:4.114}
\gll He \textbf{pura} ŋā vi{\ꞌ}e {\ꞌ}ō te me{\ꞌ}e o ruŋa i tū vaka era. \\
\textsc{pred} only \textsc{pl} woman really \textsc{art} thing of above at \textsc{dem} boat \textsc{dist} \\

\glt 
‘There are only women on that boat.’ \textstyleExampleref{[R416.513]} 
\z

\ea\label{ex:4.115}
\gll Te Tāpati Rapa Nui, he tāpati e tahi e hitu nō mahana o te \textbf{pura} {\ꞌ}ori.\\
\textsc{art} Tapati Rapa Nui \textsc{pred} week \textsc{num} one \textsc{num} seven just day of \textsc{art} only dance\\

\glt 
‘The Tapati Rapa Nui is a week, seven days of just dancing.’ \textstyleExampleref{[R240.003]}  
\z

\ea\label{ex:4.116}
\gll Kahu {\ꞌ}ō, \textbf{pura}\is{pura ‘mere’} kahu teatea e uru era {\ꞌ}i rā tiempo. \\
clothes really only clothes white:\textsc{red} \textsc{ipfv} dress \textsc{dist} at \textsc{dist} time \\

\glt 
‘As for the clothes, they wore only white clothes at that time.’ \textstyleExampleref{[R416.1272]}
\z

\subsection{\textit{Tahi} ‘all’}\label{sec:4.4.9}
\is{tahi ‘all’}
\textit{Tahi} is the numeral ‘one’ (\sectref{sec:4.3.1.0}), and as such it is always preceded by one of the numeral particles \textit{e}, \textit{ka} or \textit{hoko}. Apart from that, \textit{tahi} is also used as an \isi{adverb}\is{Adverb} in the verb phrase, in the sense ‘all’.\footnote{\label{fn:189}This use of \textit{tahi} does not occur in older texts and is probably borrowed from \ili{Tahitian}\is{\ili{Tahitian} influence}, where \textit{tahi} likewise occurs as a VP \isi{adverb} meaning ‘all’.} \textit{Tahi} has reference not to the action itself (in which case it would indicate that the action happens completely), but to one of the arguments of the verb. This argument is usually plural (whether explicitly indicated or not) and \textit{tahi} indicates that all of the entities referred to by the \isi{noun phrase} are concerned by the action.

\textit{Tahi} may have reference to an O argument as in \REF{ex:4.117}, an S argument as in \REF{ex:4.118}, or an A argument as in \REF{ex:4.119}:

\ea\label{ex:4.117}
\gll He haka hāŋai \textbf{tahi} i tū māmoe era. \\
\textsc{ntr} \textsc{caus} feed all \textsc{acc} \textsc{dem} sheep \textsc{dist} \\

\glt 
‘We fed all the sheep.’ \textstyleExampleref{[R131.008]} 
\z

\ea\label{ex:4.118}
\gll {\ꞌ}Arīnā he turu \textbf{tahi} mai tātou. \\
today.\textsc{fut} \textsc{ntr} go\_down all hither \textsc{1pl.incl} \\

\glt 
‘Today we all go down (to the school).’ \textstyleExampleref{[R315.384]} 
\z

\ea\label{ex:4.119}
\gll He tike{\ꞌ}a \textbf{tahi} te ŋā poki i te pahī tu{\ꞌ}u iho mai. \\
\textsc{ntr} see all \textsc{art} \textsc{pl} child \textsc{acc} \textsc{art} ship arrive just\_then hither \\

\glt
‘The children all saw the ship that had just arrived.’ \textstyleExampleref{[Notes]}
\z

When both arguments of a transitive verb are plural, the reference of \textit{tahi}\is{tahi ‘all’} may be ambiguous. In the following example, \textit{tahi} may quantify either the implied Agent, or the Patient ‘the sweet potatoes’.

\ea\label{ex:4.120}
\gll He keri \textbf{tahi} rāua i te kūmara. \\
\textsc{ntr} dig all \textsc{3pl} \textsc{acc} \textsc{art} sweet\_potato \\

\glt 
‘They dug up all the sweet potatoes’ or ‘They all dug up the sweet potatoes’ \textstyleExampleref{[Notes]}
\z

\subsection{The \isi{quantifier} phrase}\label{sec:4.4.10}
\is{Quantifier}
Unlike numerals, quantifiers\is{Quantifier} are not preceded by obligatory particles. However, like numerals they can be followed by certain particles; in other words, they are the nucleus of a \isi{quantifier}\is{Quantifier} phrase.

Universal quantifiers\is{Quantifier} are often followed by the limitative \isi{particle} \textit{nō}\is{no ‘just’@nō ‘just’} ‘simply, just’ (\sectref{sec:5.8.2}), which emphasises that the \isi{quantifier}\is{Quantifier} involves all people or things, without exception.

\ea\label{ex:4.121}
\gll He mau e tahi {\ꞌ}i te \textbf{ta{\ꞌ}ato{\ꞌ}a} \textbf{nō} kona {\ꞌ}i rā hora. \\
\textsc{pred} abundance \textsc{num} one at \textsc{art} all just place at \textsc{dist} time \\

\glt 
‘It (a kind of grass) was abundant just everywhere at that time.’ \textstyleExampleref{[R106.050]} 
\z

\ea\label{ex:4.122}
\gll Ka hāpa{\ꞌ}o nō i te \textbf{paurō} \textbf{nō} me{\ꞌ}e o te misione. \\
\textsc{cntg} care\_for just \textsc{acc} \textsc{art} every just thing of \textsc{art} mission \\

\glt
‘He took care of everything of the mission.’ \textstyleExampleref{[R539-1.067]}
\z

After \textit{tētahi}\is{tetahi ‘some’@tētahi ‘some’} ‘some/others’, and occasionally after \textit{me{\ꞌ}e rahi}\is{mee rahi ‘many’@me{\ꞌ}e rahi ‘many’} ‘many’ and \textit{rauhuru}\is{rauhuru ‘diverse’} ‘diverse’, the \isi{directional}\is{Directional} \isi{particle} \textit{atu}\is{atu ‘away’} is used. After verbs this \isi{particle} indicates a movement away from the speaker (\sectref{sec:7.5}), but it may also emphasise a quantity or extent (\sectref{sec:7.5.1.5}); the latter is relevant when it is used after a \isi{quantifier}\is{Quantifier}.

\ea\label{ex:4.123}
\gll He iri te poki ki {\ꞌ}uta tuatua i te kūmara,  ananake ko \textbf{tētahi} \textbf{atu} ŋā poki.\\
\textsc{ntr} ascend \textsc{art} child to inland dig:\textsc{red} \textsc{acc} \textsc{art} sweet\_potato  together \textsc{prom} other away \textsc{pl} child\\

\glt
‘The child went to the field to harvest sweet potatoes, together with other children.’ \textstyleExampleref{[Mtx-7-25.009]}
\z

The same quantifiers\is{Quantifier} may be followed by \textit{haka{\ꞌ}ou}\is{haka{\ꞌ}ou ‘again’} ‘again’ (\sectref{sec:4.5.3.4}), here in the sense ‘more, others’, which serves to single out a second or further subgroup:

\ea\label{ex:4.124}
\gll \textbf{Tētahi} ŋā poki he hāpī i te {\ꞌ}ori rapa nui, \textbf{tētahi} \textbf{haka{\ꞌ}ou} mo {\ꞌ}ori  i te cueca.\\
some \textsc{pl} child \textsc{ntr} learn \textsc{acc} \textsc{art} dance shine crouch other again for dance  \textsc{acc} \textsc{art} cueca\\

\glt 
‘Some children learn Rapa Nui dancing, others (learn) to dance the cueca.’ \textstyleExampleref{[R334.131]} 
\z

\subsection{Conclusions}\label{sec:4.4.11}

The sections above have shown that quantifiers\is{Quantifier} occur in different positions in the \isi{noun phrase}: after the noun, after the article, without article, sometimes before the article. The positional possibilities are different for each \isi{quantifier}, as shown in \tabref{tab:30} on p.~\pageref{tab:30}; however, there is a general tendency for prenominal placement, as well as a tendency to omit the article when the determiner is prenominal. In fact, the five most common quantifiers (\textit{ta{\ꞌ}ato{\ꞌ}a} and \textit{paurō} ‘all’, \textit{rauhuru} ‘diverse’, \textit{tētahi} ‘some’ and \textit{me{\ꞌ}e rahi} ‘many’) all occur in the construction \textit{QTF N}. For quantifiers occurring in multiple positions, there may be subtle semantic differences between different placements, but it does not seem to be possible to formulate general rules across the group. 

As \tabref{tab:28} on p.~\pageref{tab:28} shows, the \isi{quantifier}\is{Quantifier} system has undergone significant changes over the last century: 

\begin{itemize}
\item 
Three new quantifiers\is{Quantifier} have emerged, two of which (\textit{paurō}, \textit{ta{\ꞌ}ato{\ꞌ}a}) were borrowed from \ili{Tahitian}\is{\ili{Tahitian} influence}, while the third (\textit{rauhuru}) is a language-internal development. 

\item 
At the same time \textit{ananake}, which used to be the only universal \isi{quantifier}\is{Quantifier}, has specialised its meaning to ‘together’. 

\item 
Two less common quantifiers\is{Quantifier}, \textit{pura} ‘merely’ and \textit{kā} ‘each’, have been borrowed from \ili{Spanish}. 

\item 
The adjective \textit{kē} came to be used as a \isi{quantifier}\is{Quantifier} ‘some, certain’, probably also under \ili{Spanish} influence.

\end{itemize}

Interestingly, while \textit{ta{\ꞌ}ato{\ꞌ}a} ‘all’ and \textit{paurō} ‘all’ were borrowed from \ili{Tahitian}\is{\ili{Tahitian} influence}, their syntax differs from their \ili{Tahitian} equivalent. In \ili{Tahitian}, both quantifiers\is{Quantifier} only occur after the noun or pronoun they modify (\citealt[172]{LazardPeltzer2000}, \citealt[148–149]{AcadémieTahitienne1986}). They never occur before the noun, though \textit{ta{\ꞌ}ato{\ꞌ}a} does occur independently: \textit{te ta{\ꞌ}ato{\ꞌ}a} ‘the totality’ (\citealt[149]{AcadémieTahitienne1986}). Both elements also occur after verbs; in the examples given by \citet[147]{LazardPeltzer2000}, they quantify the subject of the verb, in the same way as \textit{tahi} in \ili{Tahitian} and Rapa Nui. By contrast, in Rapa Nui, \textit{ta{\ꞌ}ato{\ꞌ}a} occurs either before or after the noun or independently, but only rarely after verbs. When \textit{\mbox{ta{\ꞌ}ato{\ꞌ}a}} occurs independently in Rapa Nui, it may or may not be preceded by the article; in \ili{Tahitian}, the article is obligatory.

Likewise, Rapa Nui \textit{paurō} is quite different from its \ili{Tahitian} equivalent \textit{pauroa}: while the latter occurs after nouns and verbs, Rapa Nui \textit{paurō} usually precedes the article and mainly occurs with temporal nouns. The differences are summarised in \tabref{tab:31}.

\begin{table}
\fittable{
\begin{tabularx}{127mm}{L{21mm}Z{23mm}Z{21mm}Z{18mm}Z{8mm}Z{13mm}} 
\lsptoprule
& prenominal:& postnominal:& postverbal:& \multicolumn{2}{c}{independent:}\\
& \textit{(te)~QTF~(te)~N}& \textit{N~QTF}& \textit{V~QTF}& \textit{QTF}& \textit{te~QTF}\\
\midrule
{Tah. \textit{ta{\ꞌ}ato{\ꞌ}a}\is{taatoa ‘all’@ta{\ꞌ}ato{\ꞌ}a ‘all’}} & no& yes& yes& no& yes\\
{RN \textit{ta{\ꞌ}ato{\ꞌ}a}} & yes& yes& marginal& yes& yes\\
%\midrule
\tablevspace
{Tah. \textit{pauroa}\is{paurō ‘every’}} & no& yes& yes& no& no\\
{RN \textit{paurō}} & yes& marginal& no& no& no\\
\lspbottomrule
\end{tabularx}
\caption{Distribution of \ili{Tahitian} and Rapa Nui quantifiers}
\label{tab:31}
}
\end{table}

We may conclude that, even though the form and meaning of \textit{ta{\ꞌ}ato{\ꞌ}a}\is{taatoa ‘all’@ta{\ꞌ}ato{\ꞌ}a ‘all’} and \textit{paurō} were borrowed from \ili{Tahitian}\is{\ili{Tahitian} influence}, they acquired a distinctive Rapa Nui syntax, which they partly inherited from \textit{ananake}. For \textit{tahi}\is{tahi ‘all’} a different development took place: the word already existed in Rapa Nui as numeral ‘one’, but came also to be used as a \isi{quantifier}\is{Quantifier}{}-like \isi{adverb} in the VP. If this happened under the influence of \ili{Tahitian} – as seems likely – this means that an existing word acquired a new syntactic behaviour through borrowing. 

Another language-internal development in Rapa Nui is the change of \textit{tētahi}\is{tetahi ‘some’@tētahi ‘some’} ‘some, other’, originally a combination of article + numeral, into a monomorphemic \isi{quantifier}\is{Quantifier} which does not include a determiner.

Last of all, \textit{rahi} ‘much’ has undergone a significant syntactic shift. While it used to function predominantly as an \isi{adjectival} predicate, it came to be used as an \isi{adjectival} modifier of \textit{me{\ꞌ}e}\is{mee rahi ‘many’@me{\ꞌ}e rahi ‘many’} ‘thing’ (a construction already found in old texts, but only sporadically), a combination which subsequently developed into a \isi{quantifier}\is{Quantifier}.

\is{Quantifier}To summarise: the Rapa Nui \isi{quantifier}\is{Quantifier} system has radically changed over the past century, partly under \ili{Tahitian} and \ili{Spanish} influence, partly as a language-internal development. But even borrowed elements show a syntactic behaviour which is distinctly Rapa Nui.
\is{Quantifier|)}
\section{Adverbs}\label{sec:4.5}

There are two classes of adverbs\is{Adverb} in Rapa Nui: \textsc{verb phrase} \textsc{adverbs}\is{Adverb}, which are part of a verb phrase, and \textsc{sentential adverbs}\is{Adverb}, which form a separate constituent in the clause. These two classes are discussed in \sectref{sec:4.5.1} and \sectref{sec:4.5.2}, respectively. The two sets are largely distinct\is{Adverb}. 

In \sectref{sec:4.5.3}, a number of individual adverbs\is{Adverb} are discussed.

\subsection{Verb phrase adverbs}\label{sec:4.5.1}
\is{Adverb}
Adverbs in the verb phrase occur immediately after the verb (see the chart in \sectref{sec:7.1}). 

The following words function primarily as verb phrase adverbs\is{Adverb}: 

\begin{tabbing}
xxxx \= xxxxxxxxxxxxxxxxxx \= xxxxxxxxxxxxxxxx \kill
\> \textit{tahi}\is{tahi ‘all’} \> ‘all’ (\sectref{sec:4.4.9})\\
\> \textit{iho}\is{iho ‘just now’} \> ‘just now, just then, recently’ (\sectref{sec:4.5.3.1})\\
\> \textit{tako{\ꞌ}a}\is{tako{\ꞌ}a ‘also’} \> ‘also’ (\sectref{sec:4.5.3.2})\\
\> \textit{hoki} \> ‘also’ (obsolete) (\sectref{sec:4.5.3.3})\\
\> \textit{haka{\ꞌ}ou}\is{haka{\ꞌ}ou ‘again’} \> ‘again’ (\sectref{sec:4.5.3.4})\\
\> \textit{mau}\is{mau ‘really’} \> ‘really, completely’\\
\> \textit{tā{\ꞌ}ue}\is{taue ‘by chance’@tā{\ꞌ}ue ‘by chance’} \> ‘by chance, accidentally; suddenly’\\
\> \textit{tahaŋa}\is{tahanza ‘simply’@tahaŋa ‘simply’} \> ‘simply, spontaneously; without reason’\footnote{\label{fn:190}\textit{tahaŋa} {\textless} \is{Proto-Polynesian}PPN \textit{*tafaŋa}, which has reflexes in many languages in the sense ‘naked, bare’. The development to a postverbal \isi{adverb}\is{Adverb} in the sense ‘simply’ only took place in Rapa Nui and \ili{Rapa} (\citealt[180]{Walworth2015Thesis}; \citealt{Walworth2015Classifying}.}\\
\> \textit{koro{\ꞌ}iti}\is{koro{\ꞌ}iti ‘slowly, softly’} (var. \textit{kora{\ꞌ}iti}) \> ‘slowly; softly’
\end{tabbing}
\textit{Tako{\ꞌ}a} ‘also’ and \textit{koro{\ꞌ}iti} ‘slowly, softly’ are also used as sentential adverbs. \textit{\mbox{Tako{\ꞌ}a}, \mbox{haka{\ꞌ}ou}} ‘again’ and \textit{mau} ‘really’ also occur as adverbs in the \isi{noun phrase}.

Other words occur both as adjectives and as verb phrase adverbs\is{Adverb}; this includes words like \textit{rivariva} ‘good; well’, \textit{rahi} ‘much/many’, \textit{{\ꞌ}iti{\ꞌ}iti} ‘small; a bit’, \textit{ra{\ꞌ}e} ‘first’ (\sectref{sec:3.6.4.1}), \textit{\mbox{ri{\ꞌ}ari{\ꞌ}a}} ‘terrible; terribly, very’, \textit{kē}\is{ke ‘different’@kē ‘different’} ‘different(ly)’, \textit{pūai} ‘strong(ly)’. The first two occur very frequently as adverbs\is{Adverb}, the others somewhat less.

Still other words occur as adverbs\is{Adverb} very occasionally; they function primarily as adjectives or verbs. Examples are \textit{parauti{\ꞌ}a} ‘truth; true, truly’, \textit{hōrou} ‘quick(ly)’, \textit{nuinui} ‘big; in a big way, on a large scale’; \textit{ora} ‘to live; alive’, \textit{reoreo} ‘to lie; lying’, \textit{tano} ‘correct; somewhat (after an adj.)’, \textit{rikiriki} ‘small (pl.); a bit’. 

Though the verb phrase chart in \sectref{sec:7.1} shows a single \isi{adverb}\is{Adverb} slot, the verb may be followed by more than one \isi{adverb}\is{Adverb}, as the following examples show:

\ea\label{ex:4.125}
\gll Ki oti ana aŋa \textbf{iho} \textbf{haka{\ꞌ}ou} e tahi pērīkura.\\
when finish \textsc{irr} make just\_then again \textsc{num} one film\\

\glt 
‘Later, they may make yet another movie.’ \textstyleExampleref{[R647.253]} 
\z

\ea\label{ex:4.126}
\gll He vahivahi \textbf{rivariva} \textbf{tako{\ꞌ}a} a mātou i te henua...\\
\textsc{ntr} divide:\textsc{red} good:\textsc{red} also \textsc{prop} \textsc{1pl.excl} \textsc{acc} \textsc{art} land\\

\glt
‘We will also divide up the land well...’ \textstyleExampleref{[R648.224]} 
\z

All \isi{adverb}\is{Adverb} combinations in the corpus occur in a consistent order. For example, \textit{V rivariva tako{\ꞌ}a} is attested, but \textit{*V} \textit{tako{\ꞌ}a rivariva} is not. Therefore it is possible to establish a number of ordered \isi{adverb}\is{Adverb} slots, as shown in \tabref{tab:32}.\footnote{\label{fn:191}All adverbs\is{Adverb} in this table co-occur with at least one \isi{adverb}\is{Adverb} in the contiguous slot(s), i.e. all adverbs\is{Adverb} in slot 2 co-occur with an \isi{adverb}\is{Adverb} in slot 1 and with one in slot 3, and so on. Adverbs in the same slot do not co-occur in the corpus. Adverbs in the combined slot 1 + 2, such as \textit{{\ꞌ}iti}, do not co-occur with any \isi{adverb}\is{Adverb} in slot 1 or 2.} 

\begin{table}
%\begin{tabularx}{\textwidth}{Z{32mm}Z{28mm}p{25mm}p{23mm}}
\begin{tabularx}{\textwidth}{Z{32mm}Z{28mm}Z{25mm}Z{23mm}}
\lsptoprule
 {1}& {2}& {3}& {4}\\
\midrule
 \textit{rivariva} ‘well’;
 \textit{kē}~‘differently’;
 \textit{takataka}~’together’;
 \textit{{\ꞌ}iti{\ꞌ}iti}~‘a little’& \textit{iho}~’just~then, recently’;
 \textit{tahi} ‘all’;
 \textit{kora{\ꞌ}iti} ‘slowly’ \newline ~ & \textit{haka{\ꞌ}ou} ‘again’;
 \textit{tako{\ꞌ}a} ‘also’ \newline ~ \newline ~ & \textit{mau} ‘really’ \newline ~ \newline ~ \newline ~ \\ 
\tablevspace
\multicolumn{2}{>{\centering}p{6.5cm}}{\textit{{\ꞌ}iti} ‘a little’; \textit{tahaŋa} ‘just, without reason’; \textit{hōrou} ‘quickly’; \textit{rahi} ‘much’; \textit{{\ꞌ}ino} ‘badly’; \textit{pūai} ‘strongly’; \textit{parauti{\ꞌ}a} ‘truly’; \textit{tahaŋa} ‘just’...
} &  & \\
\lspbottomrule
\end{tabularx}
\caption{Order of postverbal adverbs}
%\todo[inline]{Couldn't get top alignment to work. See the alternative begin tabular line above: using p25mm doesn't lead to top alignment, whether or not preceded by backslash+centering. The only way that worked, was adding newline codes.}
\label{tab:32}
\end{table}

Another indication for the existence of multiple \isi{adverb}\is{Adverb} slots is found in nominalised phrases: \textit{tahi} and \textit{kora{\ꞌ}iti} (slot 2) occur before the nominalising\is{Nominalisation} suffix, while adverbs\is{Adverb} from slots 3 and 4 occur after the suffix (\sectref{sec:3.2.3.3}).

\subsection{Sentential adverbs}\label{sec:4.5.2}
\is{Adverb}
Sentential adverbs\is{Adverb} are a constituent on their own; they are not part of a \isi{noun phrase} or verb phrase. 

Sentential adverbs\is{Adverb} form a small class in Rapa Nui. They include words expressing temporal notions relating to the future\is{Adverb}, like \textit{{\ꞌ}anīrā} ‘later today’ and \textit{āpō} ‘tomorrow’ (\sectref{sec:3.6.4}).\footnote{\label{fn:192}Notions like ‘here’ and ‘there’ are not expressed by adverbs\is{Adverb}, but by a \isi{preposition} + locational\is{Locational} (\sectref{sec:4.6.5.1}). The same is true for temporal notions related to the past, like ‘yesterday’.} Apart from these, there are only a few common sentential adverbs\is{Adverb}: \textit{tako{\ꞌ}a} ‘also’, \textit{\mbox{kora{\ꞌ}iti}} or \textit{\mbox{koro{\ꞌ}iti}} ‘slowly’, \textit{koi{\ꞌ}ite} ‘perhaps, if perhaps’, \textit{korohaŋa} ‘even’ and \textit{pēaha} ‘perhaps, probably’.\footnote{\label{fn:193}A few other sentential level adverbs\is{Adverb} are used occasionally, such as \textit{pōrā/pōpōrā} ‘quickly’ and \textit{umarā} ‘hurriedly’. These will not be discussed separately.} Some of these are discussed individually in \sectref{sec:4.5.3} below. Two of them\is{Adverb}, \textit{\mbox{tako{\ꞌ}a}} and \textit{\mbox{koro{\ꞌ}iti}}, also occur in the verb phrase.

Sentential adverbs occur in different positions in the clause. For example, while \textit{tako{\ꞌ}a} ‘also’ as a sentential \isi{adverb} is usually clause-initial (see \REF{ex:4.134} below), \textit{pēaha} ‘perhaps’ occurs after the initial constituent as in \REF{ex:4.127}:

\ea\label{ex:4.127}
\gll Ku toke {\ꞌ}ā \textbf{pēaha} to tāua tāropa {\ꞌ}ura. \\
\textsc{prf} steal \textsc{cont} perhaps \textsc{art}:of \textsc{1du.incl} basket lobster \\

\glt
‘Our basket of lobsters seems to have been stolen.’ \textstyleExampleref{[Mtx-7-28.050]}
\z

Sentential adverbs\is{Adverb} can be modified by particles and form an \isi{adverb}\is{Adverb} phrase. For example, time adverbs\is{Adverb} may be followed by \textit{mau} ‘really’, \textit{nō} ‘just’, postnominal demonstratives\is{Demonstrative}, and the identity marker \textit{{\ꞌ}ā}. In the following example, \textit{āpō} is followed by no less than three particles:

\ea\label{ex:4.128}
\gll \textbf{Āpō} \textbf{mau} \textbf{ena} \textbf{{\ꞌ}ā} he hakaroŋo koe i a ia.\\
tomorrow really \textsc{med} \textsc{ident} \textsc{ntr} listen \textsc{2sg} \textsc{acc} \textsc{prop} \textsc{3sg}\\

\glt 
‘Tomorrow (‘Just tomorrow’ or ‘Tomorrow exactly’) you will hear him.’ \textstyleExampleref{[Act. 25:22]}
\z

\subsection{Individual adverbs}\label{sec:4.5.3}
\is{Adverb}
In this section, a number of adverbs\is{Adverb} is discussed in more detail. All of these are verb phrase adverbs, though \textit{tako{\ꞌ}a} ‘also’ is also used as a sentential \isi{adverb}.

\subsubsection{\textit{Iho} ‘just then’}\label{sec:4.5.3.1}

\textit{Iho}\is{iho ‘just now’} ({\textless} \is{Proto-Polynesian}PPN \textit{*hifo}) is originally a \isi{directional}\is{Directional} \isi{particle} ‘downwards’, which is widespread throughout the Polynesian languages. In all languages apart from Rapa Nui it is a \isi{directional}\is{Directional}, in the same class as \textit{mai} ‘hither’ and \textit{atu} ‘away’ (\sectref{sec:7.5}); additionally, in some languages it is used as a verb in the sense ‘to descend’. In many languages, \isi{directional}\is{Directional} particles have additional senses besides the \isi{directional}\is{Directional} one, such as deictic, aspectual and/or \isi{reflexive}\is{Reflexive}.\footnote{\label{fn:194}See e.g. \citet[427]{Cablitz2006} for \ili{Marquesan}, \citet[175, 217, 330]{AcadémieTahitienne1986} for \ili{Tahitian}, \citet[92–95]{ElbertPukui1979} for \ili{Hawaiian}.} However, only in Rapa Nui have the following two developments taken place: 

%\setcounter{listWWviiiNumlviileveli}{0}
\begin{enumerate}
\item 
\textit{Iho} has lost its spatial meaning altogether; instead, it indicates temporal proximity or immediacy: ‘recently; just then, just now’.\footnote{\label{fn:195}Notice that \textit{iho} can indicate recent past in \ili{Hawaiian} (\citealt[92]{ElbertPukui1979}) and \ili{Tahitian} (\citealt[175]{AcadémieTahitienne1986}).} 

\item 
\textit{Iho} has moved to the \isi{adverb}\is{Adverb} position, directly after the verb. As a result, \textit{iho} occurs before \textit{rō} and \textit{nō} (unlike directionals, see the chart in \sectref{sec:7.1}) and may co-occur with directionals\is{Directional} (see \REF{ex:4.129} below).

\end{enumerate}

\textit{Iho} indicates that an event takes place exactly at, or just prior to, the time of reference. This often implies that something will happen only at the time specified, not earlier. In a perfective clause, this means that the event has just happened: ‘recently, just’; in other aspects, \textit{iho} can be translated as ‘just at that moment, exactly then’. When \textit{iho} occurs in a \isi{main clause} with perfective sense, the aspectual tends to be left out, as \REF{ex:4.129} shows. 

\ea\label{ex:4.129}
\gll Tutu{\ꞌ}u \textbf{iho} nō mai te ŋā poki {\ꞌ}i {\ꞌ}aŋataiahi. \\
\textsc{pl}:arrive just\_then just hither \textsc{art} \textsc{pl} child at yesterday \\

\glt 
‘The children arrived just yesterday.’ \textstyleExampleref{[R245.225]} 
\z

\ea\label{ex:4.130}
\gll Hora maha nei, {\ꞌ}e hora hitu tātou ka tu{\ꞌ}u \textbf{iho}. \\
hour four \textsc{prox} and hour seven \textsc{1pl.incl} \textsc{cntg} arrive just\_then \\

\glt
‘It is now four o’clock, and we will (not) arrive (before) seven o’clock.’ \textstyleExampleref{[R210.198]} 
\z

Sometimes \textit{iho} means ‘for the first time’:

\ea\label{ex:4.131}
\gll He piri \textbf{iho} mai ki a au te roe ta{\ꞌ}e kai tihota. \\
\textsc{ntr} get\_together just\_then hither to \textsc{prop} \textsc{1sg} \textsc{art} ant \textsc{conneg} eat sugar \\

\glt 
‘This is the first time I meet an ant which doesn’t eat sugar.’ \textstyleExampleref{[R214.008]} 
\z

\subsubsection{\textit{Tako{\ꞌ}a} ‘also’}\label{sec:4.5.3.2}
\is{tako{\ꞌ}a ‘also’}\is{tako{\ꞌ}a ‘also’|(}
\textit{Tako{\ꞌ}a} (variants \textit{toko{\ꞌ}a}, \textit{takora}) is an additive connective: ‘also, as well’. It may have developed by \isi{metathesis} from \is{Proto-Polynesian}PPN \textit{*katoa} ‘all’ (with added glottal\is{Glottal plosive}): in several other \is{Eastern Polynesian}EP languages (\ili{Rarotongan}, \ili{Tahitian}, \ili{Pa’umotu}), reflexes of \textit{katoa} have the same sense ‘also’.

\textit{Tako{\ꞌ}a} is most commonly used to indicate a connection between two clauses. In this function it normally occurs as an \isi{adverb} in the verb phrase. The following is an example:

\ea\label{ex:4.132}
\gll I hīmene tahi era te ŋā poki i te hīmene o te reva,  he hīmene \textbf{tako{\ꞌ}a} a Tiare.\\
\textsc{pfv} sing all \textsc{dist} \textsc{art} \textsc{pl} child \textsc{acc} \textsc{art} song of \textsc{art} flag  \textsc{ntr} sing also \textsc{prop} Tiare\\

\glt 
‘When all the children sang the national anthem, Tiare also sang.’ \textstyleExampleref{[R334.340]} 
\z

\ea\label{ex:4.133}
\gll E hohopu nō {\ꞌ}ā, he u{\ꞌ}i atu ko te vave nuinui e tahi.  He take{\ꞌ}a \textbf{tako{\ꞌ}a} i te ika e tahi.\\
\textsc{ipfv} \textsc{pl}:bathe just \textsc{cont} \textsc{ntr} look away \textsc{prom} \textsc{art} wave big \textsc{num} one  \textsc{ntr} see also \textsc{acc} \textsc{art} fish \textsc{num} one\\

\glt
‘When they were swimming, they saw a big wave. They also saw a fish.’ \textstyleExampleref{[R338.003–004]}
\z

In these examples, \textit{tako{\ꞌ}a} indicates that the event or action applies not only to an entity mentioned previously, but to another entity as well, e.g. a different subject as in \REF{ex:4.132}, or a different object as in \REF{ex:4.133}.

In modern Rapa Nui, \textit{tako{\ꞌ}a}\is{tako{\ꞌ}a ‘also’} may also occur clause-initially, sometimes preceded by \textit{{\ꞌ}e} ‘and’. This construction may be influenced by \ili{Spanish}, where \textit{(y) además} ‘(and) moreover’ commonly occurs at the beginning of a sentence. This serves to create a link between what precedes and what follows, but unlike the examples above, there is not necessarily a constituent which is identical in both clauses. 

In the following example (from a text about marriage in the past), the two sentences linked by \textit{tako}\textit{{\ꞌ}}\textit{a} describe two aspects of the relationship between the families of the bride and the groom.

\ea\label{ex:4.134}
\gll Te hua{\ꞌ}ai o Iovani, ko {\ꞌ}ite {\ꞌ}ā ta{\ꞌ}e he hua{\ꞌ}ai o rāua te hua{\ꞌ}ai o Tiare. \textbf{Tako{\ꞌ}a}, {\ꞌ}ina a Iovani kai māhani hia ki a Tiare. \\
\textsc{ntr} family of Iovani \textsc{prf} know \textsc{cont} \textsc{conneg} \textsc{pred} family of \textsc{3pl} \textsc{art} family of Tiare also \textsc{neg} \textsc{prop} Iovani \textsc{neg.pfv} accustomed yet to \textsc{prop} Tiare \\

\glt 
‘Iovani’s family knew that Tiare’s family was not related to them. Also, Iovani did not know Tiare yet.’ \textstyleExampleref{[R238.004–005]}
\z

\textit{Tako{\ꞌ}a} also occurs in the \isi{noun phrase}. Just as in the verb phrase, it occurs in the \isi{adverb}\is{Adverb} position, before other particles (\sectref{sec:5.8.1}). Like \textit{tako{\ꞌ}a} in the verb phrase, it signals that an entity in the clause replaces an entity in the previous clause. It occurs in the \isi{noun phrase} in the following situations:

Firstly, in nominal clauses\is{Clause!nominal}, where there simply is no verb to attach to, as in \REF{ex:4.135}:

\ea\label{ex:4.135}
\gll He viri i te moeŋa {\ꞌ}i te kahu, {\ꞌ}i roto {\ꞌ}ana a Ure \textbf{tako{\ꞌ}a}. \\
\textsc{ntr} roll \textsc{acc} \textsc{art} mat at \textsc{art} cloth at inside \textsc{ident} \textsc{prop} Ure also \\

\glt
‘They rolled the mat (in which Ure was sleeping) in a cloth, Ure too was inside (the cloth).’ \textstyleExampleref{[Mtx-7-03.016]}
\z

Secondly, when the \isi{noun phrase} in question is preposed, as in \REF{ex:4.136}:

\ea\label{ex:4.136}
\gll A au \textbf{tako{\ꞌ}a} e hā{\ꞌ}ū{\ꞌ}ū rō {\ꞌ}ā ki tō{\ꞌ}oku matu{\ꞌ}a hāpa{\ꞌ}o  i te {\ꞌ}animare pē Mihaera.\\
\textsc{prop} \textsc{1sg} also \textsc{ipfv} help \textsc{emph} \textsc{cont} to \textsc{poss.1sg.o} parent care\_for  \textsc{acc} \textsc{art} animal like Mihaera\\

\glt
‘I also help my father to take care of the animals, like Mihaera.’ \textstyleExampleref{[R334.293]} 
\z

Thirdly, in elliptic clauses, where the predicate is omitted. In \REF{ex:4.137} below, only the contrastive constituent – the locative phrase – is expressed, and \textit{tako{\ꞌ}a}\is{tako{\ꞌ}a ‘also’} is added to this constituent. 

\ea\label{ex:4.137}
\gll Te ŋāŋata o te nohoŋa tuai era {\ꞌ}ā, {\ꞌ}i roto i te hare te moeŋa  haŋa, {\ꞌ}i roto i te {\ꞌ}ana \textbf{tako{\ꞌ}a}.\\
the men of \textsc{art} stay:\textsc{nmlz} old \textsc{dist} \textsc{ident} at inside at art hout \textsc{art} sleep:\textsc{nmlz}  \textsc{nmlz} at inside at \textsc{art} cave also\\

\glt 
‘The people of the old times, they slept in houses, and also in caves.’ \textstyleExampleref{[Ley-5-02.001]}\textstyleExampleref{} 
\z \is{tako{\ꞌ}a ‘also’|)}
\subsubsection{\textit{Hoki} ‘also’}\label{sec:4.5.3.3}

\textit{Hoki}\is{hoki ‘also’|(} ‘also’ is only used in older texts,\footnote{\label{fn:196}\textit{Hoki} ({\textless} PPN \textit{*foki} ‘also’) does not occur in MsE, but is common in Mtx and Ley. As \textit{tako{\ꞌ}a} also occurs in these corpora, the question is whether there is a clear difference between \textit{hoki} and \textit{tako{\ꞌ}a}. As far as there is any difference, it seems to be that \textit{hoki} indicates a stronger contrast. For example, while \textit{tako{\ꞌ}a} occurs with implicit subjects, \textit{hoki} never does (see \ref{ex:4.139} below). Also, while \textit{hoki} is used in preposed (i.e. focused) noun phrases, \textit{tako{\ꞌ}a} is not (see \ref{ex:4.140}), at least not in older texts.} apart from a few examples of what could be frozen usage in newer texts. It occurs at the end of a constituent; this constituent is typically a \isi{noun phrase} or verb phrase, but may also be a \isi{quantifier}\is{Quantifier} phrase (within an NP). \textit{Hoki} signals that the constituent it modifies is added to another constituent of the same kind and in some way parallel to it. 

\textit{hoki} can be used to connect NPs or to connect clauses. When it connects two noun phrases, it indicates that the NP is parallel to an earlier NP. This typically happens in lists, where a series of NPs all play the same role in a clause. In this case, \textit{hoki} is attached to the last element of the list. 

\ea\label{ex:4.138}
\gll {\ꞌ}I te tapa te matu{\ꞌ}a, a koro, a nua, te uka riva, te repa  riva \textbf{hoki}.\\
at \textsc{art} side \textsc{art} parents \textsc{prop} Dad \textsc{prop} Mum \textsc{prop} girls good \textsc{prop} young\_man  good also\\

\glt
‘To the side are the parents, the fathers, the mothers, the pretty girls, the handsome boys as well.’ \textstyleExampleref{[Ley-5-24.013]}
\z

When \textit{hoki} connects two clauses, it signals that the second clause (in which \textit{hoki} occurs) is parallel in some way to the first clause. Usually this means that both clauses are identical in one or two constituents,\footnote{\label{fn:197}An exception are clauses which are only identical in their subjects; these clauses are extremely common in narrative and don’t warrant the use of \textit{hoki}.} while they contrast in one or two other constituents.\footnote{\label{fn:198}\citet[92]{Levinsohn2007} distinguishes several ways in which clauses can be contrasted: “replacing focus” means that one constituent of the clause is replaced by another, while the rest of the clause is identical or synonymous; “prototypical\is{Prototype} contrast” means that clauses have one constituent in common and two points of contrast. Both of these can be indicated by \textit{hoki}.} 

When \textit{hoki} connects two clauses, it is usually added to the first constituent of the second clause. This is normally the verb phrase as in \REF{ex:4.139}, but it may be a preposed constituent as in \REF{ex:4.140}:

\ea\label{ex:4.139}
\gll He to{\ꞌ}o mai te nui, he {\ꞌ}akaveŋa. [He to{\ꞌ}o mai \textbf{hoki}] te {\ꞌ}iti,  he {\ꞌ}akaveŋa tako{\ꞌ}a.\\
\textsc{ntr} take hither \textsc{art} big \textsc{art} carry\_on\_back ~\textsc{ntr} take hither also \textsc{art} small  \textsc{ntr} carry\_on\_back also\\

\glt 
‘The oldest (girl) took (the food) and carried it on her back. The youngest also took (food) and also carried it on her back.’ \textstyleExampleref{[Mtx-7-24.041–042]}
\z

\ea\label{ex:4.140}
\gll He to{\ꞌ}o mai i te {\ꞌ}ō{\ꞌ}one... [Hai moa \textbf{hoki}] ana to{\ꞌ}o mai. \\
\textsc{ntr} take hither \textsc{acc} \textsc{art} soil ~\textsc{ins} chicken also \textsc{irr} take hither \\

\glt 
‘He took dirt... He also took a chicken.’ \textstyleExampleref{[Ley-5-28.002-004]}
\z \is{hoki ‘also’|)}
\subsubsection{\textit{Haka{\ꞌ}ou} ‘again’}\label{sec:4.5.3.4}
\is{haka{\ꞌ}ou ‘again’|(}
\textit{Haka{\ꞌ}ou}\footnote{\label{fn:199}\textit{Haka{\ꞌ}ou} has cognates in several \is{Eastern Polynesian}EP languages: \ili{Mangarevan} \textit{{\ꞌ}aka{\ꞌ}ou} ‘again’, \ili{Tahitian} \textit{fa{\ꞌ}ahou}, \ili{Pa’umotu} \textit{hakahou}. In these languages it consists of the \isi{causative}\is{Causative} \isi{prefix} (\is{Eastern Polynesian}PEP \textit{*faka}) plus a reflex of PPN *\textit{fo{\ꞌ}ou} ‘new’. The Rapa Nui reflex of \textit{*fōu} is \textit{hō{\ꞌ}ou}; the form \textit{haka{\ꞌ}ou} suggests that the word was borrowed from \ili{Mangarevan}, with the initial glottal\is{Glottal plosive} becoming \textit{h} by analogy of the RN \isi{causative}\is{Causative} \isi{prefix} \textit{haka}.}\is{haka{\ꞌ}ou ‘again’} (older variants \textit{hoko{\ꞌ}ou, hakahou}) ‘again’ marks various kinds of repetition. \textit{Haka{\ꞌ}ou} may indicate that an event which has happened before, is repeated:

\ea\label{ex:4.141}
\gll He hāŋai \textbf{haka{\ꞌ}ou} i te mā{\ꞌ}aŋa rikiriki. \\
\textsc{ntr} feed again \textsc{acc} \textsc{art} chick \textsc{pl}:small \\

\glt
‘He raised chicks again. (He had raised chickens before.)’ \textstyleExampleref{[Mtx-7-05.021]}
\z

More often \textit{haka{\ꞌ}ou} has a broader sense, indicating that the event has some element of repetition, without being repeated exactly. For example, the action expressed by the verb is performed again, even though the event as a whole is not the repetition of a previous event. In the following example, \textit{oho haka{\ꞌ}ou} signals that the people keep going, without implying that they had gone to Mount Pu’i before.

\ea\label{ex:4.142}
\gll {\ꞌ}Ai ka oho \textbf{haka{\ꞌ}ou} mai ira ki Pua Katiki. \\
there \textsc{cntg} go again from \textsc{ana} to Pua Katiki \\

\glt
‘Then they went (continued their way) from there to Pua Katiki.’ \textstyleExampleref{[R420.047]} 
\z

In a negated clause, \textsc{neg} + \textit{haka{\ꞌ}ou} means ‘not any more’ (cf. \ili{Spanish} \textit{ya no}):

\ea\label{ex:4.143}
\gll {\ꞌ}Ina koe ko taŋi \textbf{haka{\ꞌ}ou}.\\
\textsc{neg} \textsc{2sg} \textsc{neg.ipfv} cry again\\

\glt
‘(to someone who is crying): Don’t cry anymore.’ \textstyleExampleref{[R349.016]} 
\z

In the \isi{noun phrase}, \textit{haka{\ꞌ}ou}\is{haka{\ꞌ}ou ‘again’} means ‘other, another’.

\ea\label{ex:4.144}
\gll ...kī tū taŋata era ki tū poki era, ki tū taŋata \textbf{haka{\ꞌ}ou} era. \\
~~~say \textsc{dem} man \textsc{dist} to \textsc{dem} child \textsc{dist} to \textsc{dem} man again \textsc{dist} \\

\glt 
‘...said the man to the child and to the other man.’ \textstyleExampleref{[R102.020]} 
\z
\is{haka{\ꞌ}ou ‘again’|)}
\subsection{Sentential particles}\label{sec:4.5.4}

There is another small group of words which can be classified as sentential adverbs: they form a separate constituent in the clause and specify the clause as a whole. Unlike the adverbs discussed so far, these elements are not derived from lexical items: they are short, usually monosyllabic, and in this respect they are similar to particles occurring in the \isi{noun phrase} and the verb phrase. Also, their sense is more general and they are less straightforward to translate. In other words, they have a more grammatical, less lexical character than the adverbs described so far. Hence their characterisation as particles, even though – different from particles occurring in the NP and the VP – they form a constituent on their own.

These elements are described in the following subsections.

\subsubsection{Deictic particles}\label{sec:4.5.4.1}\is{Deictic \isi{particle}|(}
\paragraph[{\ꞌ}ī ‘here’]{\textit{{\ꞌ}Ī} ‘here’}\label{sec:4.5.4.1.1}
\is{i (deictic)@{\ꞌ}ī (deictic)|(}
\textit{{\ꞌ}Ī}\is{i (deictic)@{\ꞌ}ī (deictic)} is a deictic \isi{particle} expressing immediacy. It is used to point towards an object or event, expressing that it is close in space or time to the speech situation.\footnote{\label{fn:200}\textit{{\ꞌ}Ī} is similar in function to focus particles such as \textit{idou} in Koine \ili{Greek} and \textit{hinneh} in Biblical \ili{Hebrew} (see \citealt[58, 82]{Levinsohn2007}; \citealt{Bailey2009}).} By drawing attention to the object or event, the latter is put in focus\is{Focus}.

\textit{{\ꞌ}Ī} is used to draw attention to something which is nearby. 

\ea\label{ex:4.145}
\gll \textbf{{\ꞌ}Ī} au. \\
\textsc{imm} \textsc{1sg} \\

\glt 
‘Here I am.’ \textstyleExampleref{[R363.078]} 
\z

\ea\label{ex:4.146}
\gll Ka e{\ꞌ}a koe, \textbf{{\ꞌ}ī} tu{\ꞌ}u pāpā era {\ꞌ}i haho. \\
\textsc{imp} go\_out \textsc{2sg} \textsc{imm} \textsc{poss.2sg.o} father \textsc{dist} at outside \\

\glt
‘Go outside, here is your father outside.’ \textstyleExampleref{[R210.094]} 
\z

Clauses such as (\ref{ex:4.145}–\ref{ex:4.146}) could be labeled “presentational”: \textit{{\ꞌ}ī} followed by a nominal constituent serves to signal the presence of someone or something.\footnote{\label{fn:201}This does not mean that \textit{{\ꞌ}ī} is a general device to express presentational clauses, e.g. to introduce participants in narrative (on presentational clauses, see \citealt[4]{Bailey2009}). The use of \textit{{\ꞌ}ī} in presentational clauses is limited to deictic contexts, where the entity presented is visible to speaker and hearer.} 

\textit{{\ꞌ}Ī} may indicate that an event takes place immediately (‘right now’): 

\ea\label{ex:4.147}
\gll \textbf{{\ꞌ}Ī} au he oho rō {\ꞌ}ai mai ta{\ꞌ}e pō. \\
\textsc{imm} \textsc{1sg} \textsc{ntr} go \textsc{emph} \textsc{subs} from \textsc{conneg} night \\

\glt
‘I’m going now, before it gets dark.’ \textstyleExampleref{[R153.042]} 
\z

More generally, \textit{{\ꞌ}ī} expresses simultaneity\is{Simultaneity} with respect to a time of reference. In \REF{ex:4.147} above the time of reference is the present; in narrative discourse the time of reference is the time when events in the context take place. In combination with perfect \isi{aspect}\is{Aspect!perfect} \textit{ko V {\ꞌ}ā}\is{ko V {\ꞌ}ā (perfect \isi{aspect})}, \textit{{\ꞌ}ī} underlines that an event has just taken place.

\ea\label{ex:4.148}
\gll \textbf{{\ꞌ}Ī} ku e{\ꞌ}a haka{\ꞌ}ou mai {\ꞌ}ana a ruŋa mai te {\ꞌ}ara iŋa,   ka uru mai era {\ꞌ}i tū hora era, ka ŋau.\\
\textsc{imm} \textsc{prf} go\_out again hither \textsc{cont} by above from \textsc{art} look\_under\_water \textsc{nmlz}   \textsc{cntg} enter hither \textsc{dist} at \textsc{dem} time \textsc{dist} \textsc{cntg} bite\\

\glt
‘Just when he came up again from looking under water, the (shark) entered and bit.’ \textstyleExampleref{[R361.016]} 
\z

\textit{{\ꞌ}Ī} may convey immediacy and urgency to a statement or request: ‘I’m telling you, listen, look...’.

\ea\label{ex:4.149}
\gll E Pea ē, \textbf{{\ꞌ}ī} a Kava e taŋi mai nei ki a au,  mai te pō{\ꞌ}ā ki te hora nei.\\
\textsc{voc} Pea \textsc{voc} \textsc{imm} \textsc{prop} Kava \textsc{ipfv} cry hither \textsc{prox} to \textsc{prop} \textsc{1sg}  from \textsc{art} morning to \textsc{art} time \textsc{prox}\\

\glt 
‘Pea, (listen,) Kava is crying for me, from morning till now.’ \textstyleExampleref{[R229.017]} 
\z

\ea\label{ex:4.150}
\gll {\ꞌ}E \textbf{{\ꞌ}ī} a au ka hatu rō atu ki a koe. \\
and \textsc{imm} \textsc{prop} \textsc{1sg} \textsc{cntg} advise \textsc{emph} away to \textsc{prop} \textsc{2sg} \\

\glt 
‘Look, I’m advising/warning you.’ \textstyleExampleref{[R310.294]} 
\z

Often \textit{{\ꞌ}ī}\is{i (deictic)@{\ꞌ}ī (deictic)} occurs in combination with a perception verb. \textit{{\ꞌ}Ī} has the effect of putting the perceived object into focus\is{Focus}. What the participant sees or hears is something significant or even surprising. The act of perception may also be implied, as in \REF{ex:4.152}.

\ea\label{ex:4.151}
\gll \textbf{{\ꞌ}Ī} a Vai Ora ka u{\ꞌ}i atu ena, {\ꞌ}ina a Tahoŋa. \\
\textsc{imm} \textsc{prop} Vai Ora \textsc{cntg} look away \textsc{med} \textsc{neg} \textsc{prop} Tahonga \\

\glt 
‘Vai Ora looked: Tahonga wasn’t there!’ \textstyleExampleref{[R301.164]} 
\z

\ea\label{ex:4.152}
\gll \textbf{{\ꞌ}Ī} ka o{\ꞌ}o atu ena, e ha{\ꞌ}uru nō {\ꞌ}ā a Makita. \\
\textsc{imm} \textsc{cntg} enter away \textsc{med} \textsc{ipfv} sleep just \textsc{cont} \textsc{prop} Makita \\

\glt
‘He entered, and (look!) Makita was sleeping.’ \textstyleExampleref{[R243.183]} 
\z

As (\ref{ex:4.149}–\ref{ex:4.151}) show, the subject of the clause tends to be placed straight after \textit{{\ꞌ}ī}, before the verb (\sectref{sec:8.6.1.1}. This is not obligatory, though. 
\is{i (deictic)@{\ꞌ}ī (deictic)|)}

\paragraph{\textit{{\ꞌ}Ai} ‘there is’}\label{sec:4.5.4.1.2}
\is{ai (deictic)@{\ꞌ}ai (deictic)|(}
Just like \textit{{\ꞌ}ī}, \textit{{\ꞌ}ai} is a deictic \isi{particle}, calling attention to an object or event; it indicates greater distance.

\textit{{\ꞌ}Ai} is used to point at things at a certain distance:

\ea\label{ex:4.153}
\gll \textbf{{\ꞌ}Ai} te Padre Sebastian {\ꞌ}i muri i te mōai. \\
there \textsc{art} Father Sebastián at near at \textsc{art} statue \\

\glt 
‘There is Father Sebastián, next to the statue.’ \textstyleExampleref{[R412.180]} 
\z

\ea\label{ex:4.154}
\gll \textbf{{\ꞌ}Ai} a Toroa {\ꞌ}ai. \textbf{{\ꞌ}Ai} a Feripe {\ꞌ}ai.  \\
there \textsc{prop} Toroa there there \textsc{prop} Felipe there  \\

\glt 
‘There is \textit{Toroa} (=Father Seb. Englert). There is Felipe.’ \textstyleExampleref{[R411.134–135]}
\z

\ea\label{ex:4.155}
\gll \textbf{{\ꞌ}Ai} tu{\ꞌ}u tao{\ꞌ}a ko haka topa {\ꞌ}ā e te kape pahī {\ꞌ}i tū hora nei. \\
there \textsc{poss.2sg.o} object \textsc{prf} \textsc{caus} descend \textsc{cont} \textsc{ag} \textsc{art} captain ship at \textsc{dem} time \textsc{prox} \\

\glt
‘There are your belongings, which have just been disembarked by the captain of the ship.’ \textstyleExampleref{[R231.142]} 
\z

	(\ref{ex:4.154}–\ref{ex:4.155}) are presentational clauses, which indicate the presence of an entity in the distance, just like \textit{{\ꞌ}ī} presents entities nearby.\footnote{\label{fn:202}This use led \citet[319]{Fischer2001Hispan} to take \textit{{\ꞌ}ai} as derived from \ili{Spanish} existential marker \textit{hay}. However, the use of \textit{{\ꞌ}ai} to introduce presentational clauses already occurs in old texts.} As \REF{ex:4.154} shows, initial \textit{{\ꞌ}ai} may be followed by another \textit{{\ꞌ}ai} at the end of the clause, just like \textit{{\ꞌ}ī} may be followed by clause-final \textit{{\ꞌ}ī a{\ꞌ}a}.

Like \textit{{\ꞌ}ī}, \textit{{\ꞌ}ai} may have a temporal function; it marks a clause referring to a subsequent event:

\ea\label{ex:4.156}
\gll He haka ekeeke i te taŋata ki ruŋa ki te vaka, \textbf{{\ꞌ}ai} ka ma{\ꞌ}u  ki ruŋa i te pahī.\\
\textsc{ntr} \textsc{caus} go\_up:\textsc{red} \textsc{acc} \textsc{art} person to above to \textsc{art} boat there \textsc{cntg} carry  to above at \textsc{art} ship\\

\glt
‘They made the people embark the boat, then took them on board the ship.’ \textstyleExampleref{[R210.042]} 
\z

As this example shows, \textit{{\ꞌ}ai} is usually followed by the contiguity marker \textit{ka}.

Like \textit{{\ꞌ}ī}, \textit{{\ꞌ}ai} may lend emphasis to a clause: ‘I’m telling you...’:

\ea\label{ex:4.157}
\gll Ko mate era {\ꞌ}ana, \textbf{{\ꞌ}ai} koe ka mana{\ꞌ}u nō e ha{\ꞌ}uru {\ꞌ}ana.\\
\textsc{prf} die \textsc{dist} \textsc{cont} there \textsc{2sg} \textsc{cntg} think just \textsc{ipfv} sleep \textsc{cont}\\

\glt 
‘She has died, and there you are just thinking that she is asleep!’ \textstyleExampleref{[R229.303]} 
\z

\textit{{\ꞌ}Ai}\is{ai (\isi{preposition})@{\ꞌ}ai (\isi{preposition})} is marginally used as a deictic \isi{preposition} ‘there in/at’ (\sectref{sec:4.7.10}). Furthermore, \textit{{\ꞌ}ai} is obviously related to the postverbal \isi{particle} \textit{{\ꞌ}ai}, which occurs in the construction \textit{he V rō {\ꞌ}ai}\is{he (\isi{aspect} marker)!he V rō {\ꞌ}ai} (\sectref{sec:7.2.3.3}). It is similar in function: while deictic \textit{{\ꞌ}ai} frequently indicates sequential events, \textit{he V rō {\ꞌ}ai}\is{he (\isi{aspect} marker)!he V rō {\ꞌ}ai} marks final and culminating events in a series.
\is{ai (deictic)@{\ꞌ}ai (deictic)|)}
\paragraph{\textit{Nā} ‘there near you’}\label{sec:4.5.4.1.3}
\is{na (demonstrative)@nā (demonstrative)!deictic|(}
\is{Deictic particle}The demonstrative \textit{nā} (which indicates medial distance, see \sectref{sec:4.6.4.4}) is used as a deictic \isi{particle}. Like \textit{{\ꞌ}ī} and \textit{{\ꞌ}ai} it occurs clause-initially, and usually serves to point at something in the vicinity of the hearer. Different from \textit{{\ꞌ}ī} and \textit{{\ꞌ}ai}, \textit{nā} is used only in a spatial sense, not in a temporal sense.

\ea\label{ex:4.158}
\gll \textbf{Nā} ka u{\ꞌ}i rā kōrua, ka hia {\ꞌ}umu nei {\ꞌ}ā{\ꞌ}aku  e kā atu ena.\\
\textsc{med} \textsc{imp} look \textsc{intens} \textsc{2pl} \textsc{cntg} how\_many earth\_oven \textsc{prox} \textsc{poss.1sg.a}  \textsc{ipfv} light away \textsc{med}\\

\glt 
‘Now look, you guys, how many earth ovens I have been cooking!’ \textstyleExampleref{[R352.089]} 
\z

\ea\label{ex:4.159}
\gll ¿\textbf{Nā} {\ꞌ}ō koe, e māmārū{\ꞌ}au ē?\\
~\textsc{med} really \textsc{2sg} \textsc{voc} grandmother \textsc{voc}\\

\glt 
‘Is that you, grandmother?’ \textstyleExampleref{[R313.119]} 
\z

\ea\label{ex:4.160}
\gll {\ꞌ}Ē, ka iri mai koe, \textbf{nā} te vave \textbf{nā}. \\
hey \textsc{imp} ascend hither \textsc{2sg} \textsc{med} \textsc{art} wave \textsc{med} \\

\glt
‘Hey, come up, there is a wave!’ \textstyleExampleref{[R126.025]} 
\z

As \REF{ex:4.160} shows, \textit{nā} may be reinforced by another \textit{nā}\is{na (demonstrative)!deictic@nā (demonstrative)!deictic} at the end of the clause.
\is{na (demonstrative)!deictic@nā (demonstrative)!deictic|)}
\is{Deictic \isi{particle}|)}
\subsubsection{\textit{Ho{\ꞌ}i} and \textit{pa{\ꞌ}i}}\label{sec:4.5.4.2}

\textit{Ho{\ꞌ}i}\is{hoi@ho{\ꞌ}i ‘in fact’} and \textit{pa{\ꞌ}i}\is{pai ‘in fact’@pa{\ꞌ}i ‘in fact’} are discourse particles which are very common in spoken language; their function is not easy to pin down.\footnote{\label{fn:203}Both particles were borrowed from \ili{Tahitian}\is{\ili{Tahitian} influence}. They are very common in modern Rapa Nui discourse, but do not occur in older texts. Even in the texts collected by Felbermayer and by Blixen in the 1970s, they occur hardly or not (\textit{pa{\ꞌ}i} 0x, \textit{ho{\ꞌ}i} 2x). In \ili{Tahitian}, \textit{pa{\ꞌ}i} underlines a statement (‘indeed’); \textit{ho{\ꞌ}i} may have the same function, but may also connect a statement to the preceding context, for example providing a reason (‘for, as’), an addition (‘also’) or a contrast (‘however’). See \citet[381]{AcadémieTahitienne1986} and \citet[117]{LazardPeltzer2000}.} They usually occur after the first constituent of the clause; they lend emphasis to this constituent and/or provide a connection to the previous clause. 
\paragraph{\textit{Pa{\ꞌ}i} ‘in fact’}\label{sec:4.5.4.2.1} \textit{Pa{\ꞌ}i} \is{pai ‘in fact’@pa{\ꞌ}i ‘in fact’|(}is used to link clauses, indicating that the clause in some way builds upon, confirms or reinforces the preceding clause. In \REF{ex:4.161}, \textit{\mbox{pa{\ꞌ}i}} (2x) confirms what the other speaker has said. In \REF{ex:4.162}, a fragment from an oral text, \textit{pa{\ꞌ}i} appears to be sprinkled throughout the discourse without a very specific function. 

\ea\label{ex:4.161}
\gll —¿Ta{\ꞌ}e {\ꞌ}ō mai {\ꞌ}Anakena i haka eke ai? —{\ꞌ}Ēē, \textbf{pa{\ꞌ}i}. —{\ꞌ}I {\ꞌ}Anakena \textbf{pa{\ꞌ}i} tū hora ena i haka eke ai. \\
~~~~~\textsc{conneg} really from Anakena \textsc{pfv} \textsc{caus} go\_up \textsc{pvp} ~~~~yes~~~ in\_fact ~~~~at Anakena in\_fact \textsc{dem} time \textsc{dist} \textsc{pfv} \textsc{caus} go\_up \textsc{pvp} \\

\glt 
‘—Didn’t they take (the statue) up from Anakena? —Yes, indeed. —Indeed, when they took it up, it was in Anakena.’ \textstyleExampleref{[R412.159–160]}
\z

\ea\label{ex:4.162}
\gll He {\ꞌ}aroha atu \textbf{pa{\ꞌ}i} ki a kōrua, {\ꞌ}iorana \textbf{pa{\ꞌ}i} a kōrua ta{\ꞌ}ato{\ꞌ}a.  {\ꞌ}I te me{\ꞌ}e, ko haŋa {\ꞌ}ā \textbf{pa{\ꞌ}i} a au mo vānaŋa atu ki a koe... \\
\textsc{ntr} greet away in\_fact to \textsc{prop} \textsc{2pl} hello in\_fact \textsc{prop} \textsc{2pl} all  at \textsc{art} thing \textsc{prf} want \textsc{cont} in\_fact \textsc{prop} \textsc{1sg} for talk away to \textsc{prop} \textsc{2sg} \\

\glt
‘I’m greeting \textit{pa{\ꞌ}i} you; hello \textit{pa{\ꞌ}i} to you all. Because, I want \textit{pa{\ꞌ}i} to talk to you...’ \textstyleExampleref{[R403.001–003]}
\z

More commonly, \textit{pa{\ꞌ}i} is used in clauses providing the grounds for the previous clause: ‘for, as’ (\ili{Spanish} \textit{pues}):

\ea\label{ex:4.163}
\gll A Tiare {\ꞌ}ina kai {\ꞌ}ite, he turu iho, \textbf{pa{\ꞌ}i}, ki te hāpī. \\
\textsc{prop} Tiare \textsc{neg} \textsc{neg.pfv} know \textsc{ntr} go\_down just\_then in\_fact to \textsc{art} learn \\

\glt 
‘Tiare doesn’t know (the national anthem), as it’s the first time she goes to school.’ \textstyleExampleref{[R334.023]} 
\z

\ea\label{ex:4.164}
\gll ¿Pē hē a au ana hoŋi atu i a koe? {\ꞌ}Ina, \textbf{pa{\ꞌ}i}, koe o muri i a au. \\
~like \textsc{cq} \textsc{prop} \textsc{1sg} \textsc{irr} kiss away \textsc{acc} \textsc{prop} \textsc{2sg} \textsc{neg} in\_fact \textsc{2sg} of near at \textsc{prop} \textsc{1sg} \\

\glt
‘How could I kiss you? You are not with me.’ \textstyleExampleref{[R228.006–007]}
\z

\textit{Pa{\ꞌ}i}\is{pai ‘in fact’@pa{\ꞌ}i ‘in fact’} seems to have taken on the range of use of \ili{Spanish} \textit{pues}, which both specifies grounds or reasons (‘for, as’) and provides confirmation or emphasis (‘well, indeed’).
\is{pai ‘in fact’@pa{\ꞌ}i ‘in fact’|)}
\paragraph{\textit{Ho{\ꞌ}i} ‘indeed’}\label{sec:4.5.4.2.2} \textit{Ho{\ꞌ}i}\is{hoi@ho{\ꞌ}i ‘in fact’|(} gives (some) emphasis to the preceding constituent. It is used after a wide range of elements, such as deictic particles \REF{ex:4.165} and the \isi{negation} \textit{{\ꞌ}ina} \REF{ex:4.166}. Like \textit{pa{\ꞌ}i}, it may have a confirmatory function: ‘indeed’ \REF{ex:4.167}.

\ea\label{ex:4.165}
\gll {\ꞌ}Ai \textbf{ho{\ꞌ}i} te taŋata e ha{\ꞌ}amata era e tari era ki ruŋa i te pahī. \\
there indeed \textsc{art} person \textsc{ipfv} begin \textsc{dist} \textsc{ipfv} carry \textsc{dist} to above at \textsc{art} ship \\

\glt 
‘Then \textit{ho{\ꞌ}i} the people started to be transported on board the ship.’ \textstyleExampleref{[R210.040]} 
\z

\ea\label{ex:4.166}
\gll {\ꞌ}Ina \textbf{ho{\ꞌ}i} koe ko riri. He kori nō ho{\ꞌ}i nei me{\ꞌ}e. \\
\textsc{neg} indeed \textsc{2sg} \textsc{prom} angry \textsc{pred} play just indeed \textsc{prox} thing \\

\glt 
‘Don’t \textit{ho{\ꞌ}i} be angry. This is just a joke.’ \textstyleExampleref{[R315.040–041]}
\z

\ea\label{ex:4.167}
\gll —¿He {\ꞌ}ariki ho{\ꞌ}i rā? —{\ꞌ}Ēē, \textbf{ho{\ꞌ}i}. \\
~~~~~\textsc{pred} king indeed \textsc{intens} ~~~~yes~~~ indeed \\

\glt
‘—Is that a king/chief? —Yes, indeed.’ \textstyleExampleref{[R415.033]} 
\z

\textit{Ho{\ꞌ}i} may occur at the start of a new episode in discourse, marking a new topic or another initial constituent:

\ea\label{ex:4.168}
\gll A nua, \textbf{ho{\ꞌ}i}, e haka rito {\ꞌ}ā e tahi {\ꞌ}avahata kahu. \\
\textsc{prop} Mum indeed \textsc{ipfv} \textsc{caus} ready \textsc{cont} \textsc{num} one box clothes \\

\glt
‘(In the meantime,) Mum \textit{ho{\ꞌ}i} was preparing a box of clothes.’ \textstyleExampleref{[R210.027]} 
\z

The constituent marked with \textit{ho{\ꞌ}i} may be emphasised in opposition to another constituent. In this way, \textit{ho{\ꞌ}i} may come to express contrast:

\ea\label{ex:4.169}
\gll Kai {\ꞌ}ite mai... Ko koe \textbf{ho{\ꞌ}i} mo {\ꞌ}ite i ta{\ꞌ}a kai tunu nei  pa{\ꞌ}i e koe.\\
\textsc{neg.pfv} know hither \textsc{prom} \textsc{2sg} indeed for know \textsc{acc} \textsc{poss.2sg.a} food cook \textsc{prox}  in\_fact \textsc{ag} \textsc{2sg}\\

\glt
‘I don’t know... You \textit{ho{\ꞌ}i} are the one who knows what food you have cooked.’ \textstyleExampleref{[R236.029–030]}
\z

Altogether, \textit{ho{\ꞌ}i}\is{hoi@ho{\ꞌ}i ‘in fact’} can be characterised as a \textsc{spacer}: an element which marks the boundary between two constituents and indicates that the preceding constituent is special in some way (\citealt[37]{DooleyLevinsohn2001}). According to \citet[74]{Levinsohn2007}, it is not unusual for the same spacer in a given language to occur after a topic, a point of departure (such as a time phrase), or a constituent in focus.
\is{hoi@ho{\ꞌ}i ‘in fact’|)}
\subsubsection{\textit{Ia} ‘so, then’}\label{sec:4.5.4.3}

The \isi{particle} \textit{ia}\is{ia ‘then’|(} ‘so, then’ is a transition marker. It occurs in modern Rapa Nui only.\footnote{\label{fn:204}\textit{Ia} may be a borrowing from \ili{Tahitian}: \ili{Tahitian} \textit{ia} is “un anaphorique général qui renvoie d’une manière assez vague à ce qui précède, thème antéposé ou, plus généralement, contexte antérieur” (a general \isi{anaphoric} which refers in a rather vague way to what precedes, a preposed theme or, more generally, the preceding context, \citealt[118]{LazardPeltzer2000}).} When the clause starts with a verb phrase, \textit{ia} occurs after the verb phrase; \REF{ex:4.170} shows that it occurs after the VP{}-final \isi{particle} \textit{{\ꞌ}ana}:

\ea\label{ex:4.170}
\gll Ko koa atu {\ꞌ}ana \textbf{ia} a Tamy {\ꞌ}i tū hora era. \\
\textsc{prf} happy away \textsc{cont} then \textsc{prop} Tamy at \textsc{dem} time \textsc{dist} \\

\glt
‘Tamy was happy at that moment.’ \textstyleExampleref{[R315.300]} 
\z

When the verb phrase is not initial in the clause, \textit{ia} occurs either after the first constituent as in \REF{ex:4.171}, or after the verb phrase as in \REF{ex:4.172}. 

\ea\label{ex:4.171}
\gll {\ꞌ}Ai \textbf{ia} e raŋaraŋa mai era... \\
there then \textsc{ipfv} float:\textsc{red} hither \textsc{dist} \\

\glt 
‘Then he floated...’ \textstyleExampleref{[R108.117]} 
\z

\ea\label{ex:4.172}
\gll {\ꞌ}I tū hora era he ŋaro{\ꞌ}a \textbf{ia} e {\ꞌ}Uri{\ꞌ}uri i te ora. \\
at \textsc{dem} time \textsc{dist} \textsc{ntr} perceive then \textsc{ag} Uri’uri \textsc{acc} \textsc{art} life \\

\glt
‘At that moment, Uri’uri felt relieved.’ \textstyleExampleref{[R536.074]} 
\z

\textit{Ia} indicates that the event in the clause develops from events mentioned before. It may be the result of, or dependent on, other events (‘so, thus’), as in \REF{ex:4.173} below. In a weaker sense it marks events which are simply subsequent to other events (‘then’), or marks the apodosis of conditional clauses as in \REF{ex:4.174}.

\ea\label{ex:4.173}
\gll He ha{\ꞌ}amata \textbf{ia} te moto {\ꞌ}i tū ŋā poki era. \\
\textsc{ntr} begin then \textsc{art} fight at \textsc{dem} \textsc{pl} child \textsc{dist} \\

\glt 
‘(Some boys mocked Taparahi and he got angry.) So a fight started between the boys.’ \textstyleExampleref{[R250.013]} 
\z

\ea\label{ex:4.174}
\gll {\ꞌ}E mo ai ō{\ꞌ}ona he repahoa ō{\ꞌ}ou, e Okū ē,  he repahoa tako{\ꞌ}a \textbf{ia} ō{\ꞌ}oku!\\
and if exist \textsc{poss.3sg.o} \textsc{pred} friend \textsc{poss.2sg.o} \textsc{voc} Oku \textsc{voc}  \textsc{pred} friend also then \textsc{poss.1sg.o}\\

\glt
‘And if he is your friend, Oku, then he is also my friend!’ \textstyleExampleref{[R535.151]} 
\z

Often \textit{ia}\is{ia ‘then’} marks a new step in the discourse, for example, at the start of a new scene in a story, or a new topic in an exposition. The latter can be seen in the following example:

\ea\label{ex:4.175}
\gll He o{\ꞌ}o \textbf{ia} {\ꞌ}i te hora nei ki te aŋa iŋa o te hi{\ꞌ}o. \\
\textsc{ntr} enter then at \textsc{art} time \textsc{prox} to \textsc{art} make \textsc{nmlz} of \textsc{art} glass \\

\glt
‘(in an explanation of different aspects of diving:) Now let’s turn to the making of goggles.’ \textstyleExampleref{[R360.026]} 
\z

Some speakers use \textit{ia} in clauses which simply represent the next event in the discourse. Others use it sparingly, or not at all. The frequency of \textit{ia}\is{ia ‘then’} thus depends to a large degree on the preference of the speaker, just like ‘then’ in \ili{English} discourse.
\is{ia ‘then’|)}
\subsubsection[The intensifier rā]{The intensifier \textit{rā}}\label{sec:4.5.4.4}
\is{ra (intensifier)@rā (intensifier)|(}
The intensifying \isi{particle} \textit{rā}\footnote{\label{fn:205}The etymology of \textit{rā} is unknown, but it is probably related to \ili{Tahitian} \textit{rā}, which also occurs after the first clause constituent. \ili{Tahitian} \textit{rā} is a contrastive conjunction\is{Conjunction} ‘but’, but also serves as an intensifier in commands and conditional clauses\is{Clause!conditional} (\citealt[98]{LazardPeltzer2000}; \citealt[399]{AcadémieTahitienne1999}).} (not to be confused with demonstrative \textit{rā}) occurs in content questions\is{Question!content} and imperatives\is{Imperative}. It is placed after the first constituent of the clause; in questions this is the question phrase, in imperatives\is{Imperative} the verb phrase.\footnote{\label{fn:206}\textit{Rā} itself is not part of the verb phrase: in \REF{ex:4.176} it occurs after the direction \textit{mai}; in \REF{ex:4.177} it occurs after the VP{}-final \isi{particle} \textit{{\ꞌ}ana}.} \textit{Rā} occurs after the verb phrase-final \isi{particle} \textit{{\ꞌ}ana} as in \REF{ex:4.177}, but before other sentence-level particles like \textit{ia} ‘then’, as in \REF{ex:4.178}.

\ea\label{ex:4.176}
\gll Ka noho {\ꞌ}iti{\ꞌ}iti mai \textbf{rā} koe. \\
\textsc{imp} stay little:\textsc{red} hither \textsc{intens} \textsc{2sg} \\

\glt 
‘You wait a little.’ \textstyleExampleref{[R208.164]} 
\z

\ea\label{ex:4.177}
\gll ¿E aha {\ꞌ}ana \textbf{rā} koe? \\
~\textsc{ipfv} what \textsc{cont} \textsc{intens} \textsc{2sg} \\

\glt 
‘What are you doing?’ \textstyleExampleref{[R212.054]} 
\z

\ea\label{ex:4.178}
\gll ¿Ko ai \textbf{rā} ia koe? \\
~\textsc{prom} who \textsc{intens} then \textsc{2sg} \\

\glt 
‘Who then are you?’ \textstyleExampleref{[R314.099]} 
\z

\textit{Rā} adds an element of insistence to the question or command.\footnote{\label{fn:207}\citet{DuFeu1987,DuFeu1996} labels this \isi{particle} as [-REA] (as opposed to \textit{rō} [+REA]), in line with the fact that it does not occur in statements expressing a fact. \citet{WeberR2003} labels \textit{rā} as \textsc{dub}(itative).} It may be used in rhetorical questions, often adding a note of provocation or rebuke:

\ea\label{ex:4.179}
\gll ¿Mo aha \textbf{rā} koe i kī ai i ta{\ꞌ}a vānaŋa pē nā? \\
~~for what \textsc{intens} \textsc{2sg} \textsc{pfv} say \textsc{pvp} \textsc{acc} \textsc{poss.2sg.a} word like \textsc{med} \\

\glt 
‘Why did you say something like that?’ \textstyleExampleref{[R301.301]} 
\z

It is also used in non-rhetorical questions, to which the speaker expects a reply. \textit{Rā} conveys a certain vividness and inquisitiveness: the speaker is eager to get an answer. \REF{ex:4.180}, for example, is spoken by a curious child. \REF{ex:4.181} is spoken by one (teenage) friend to another.

\ea\label{ex:4.180}
\gll ¿A hē nei \textbf{rā} i ŋaro ai?~... ¿{\ꞌ}I hē \textbf{rā} e noho era  tō{\ꞌ}ona ŋā repahoa era?\\
~toward \textsc{cq} \textsc{prox} \textsc{intens} \textsc{pfv} disappear \textsc{pvp} ~at \textsc{cq} \textsc{intens} \textsc{ipfv} stay \textsc{dist}  \textsc{poss.3sg.o} \textsc{pl} friend \textsc{dist}\\

\glt 
‘Where did (the fish) disappear? Where do its friends live?’ \textstyleExampleref{[R301.179,182]}
\z

\ea\label{ex:4.181}
\gll ¿Pē hē \textbf{rā} koe, e Hiero ē? \\
~like \textsc{cq} \textsc{intens} \textsc{2sg} \textsc{voc} Hiero \textsc{voc} \\

\glt
‘How are you, Hiero?’ \textstyleExampleref{[R315.081]} 
\z

A question like \REF{ex:4.181}, with its somewhat insistent tone, is only appropriate when talking to friends or close acquaintances. When speaking to strangers, one would use the more neutral ¿\textit{Pē hē koe}? ‘How are you?’ (Nancy Weber, p.c.).

When \textit{rā}\is{ra (intensifier)@rā (intensifier)} is used in \isi{imperative}\is{Imperative} clauses, it marks insistence as well. The context may involve a certain emotion: enthusiasm as in \REF{ex:4.182}, defy as in \REF{ex:4.183}:

\ea\label{ex:4.182}
\gll Ka u{\ꞌ}i mai \textbf{rā} koe. Ko rava{\ꞌ}a {\ꞌ}ā e au e rima tara. \\
\textsc{imp} look hither \textsc{intens} \textsc{2sg} \textsc{prf} obtain \textsc{cont} \textsc{ag} \textsc{1sg} \textsc{num} five peso \\

\glt 
‘Look at me. I found five pesos!’ \textstyleExampleref{[R334.288–289]}
\z

\ea\label{ex:4.183}
\gll ¡Ka kī mai \textbf{rā} {\ꞌ}a {\ꞌ}ai a koe i pu{\ꞌ}a atu!\\
~\textsc{imp} say hither \textsc{intens} of\textsc{.a} who \textsc{prop} \textsc{2sg} \textsc{pfv} beat away\\

\glt 
‘(Soldiers are mocking Jesus:) Now tell us who hit you!’ \textstyleExampleref{[Mark 14:65]}
\z
\is{ra (intensifier)@rā (intensifier)|)}
\subsubsection[Asseverative {\ꞌ}ō]{Asseverative \textit{{\ꞌ}ō}}\label{sec:4.5.4.5}
\is{o (asseverative)@{\ꞌ}ō (asseverative)|(}
The \isi{particle} \textit{{\ꞌ}ō} (etymology unknown, possibly from the exclamation \textit{{\ꞌ}ō} ‘oh!’) is asseverative. It occurs after the first constituent of the clause and underlines the truth of the clause. Often, but not always, the clause expresses something unexpected.

\ea\label{ex:4.184}
\gll {\ꞌ}I te rua mahana... he u{\ꞌ}i ko mate {\ꞌ}ana \textbf{{\ꞌ}ō}. \\
at \textsc{art} two day \textsc{ntr} look \textsc{prf} die \textsc{cont} really \\

\glt 
‘The next day... they saw that (the sea monster) was dead (something they had not expected at all).’ \textstyleExampleref{[R402.015]} 
\z

\ea\label{ex:4.185}
\gll ¡Ko pō {\ꞌ}ana, {\ꞌ}ina \textbf{{\ꞌ}ō} kai tu{\ꞌ}u mai {\ꞌ}ana! \\
~\textsc{prf} night \textsc{cont} \textsc{neg} really \textsc{neg.pfv} arrive hither \textsc{cont} \\

\glt
‘Now it’s night, and he hasn’t arrived! (And you told me he would come today!)’ \textstyleExampleref{[R229.148]} 
\z

\textit{{\ꞌ}Ō} is often used in exclamative\is{Exclamative} constructions (\sectref{sec:10.4.2}), where it underlines that something is not according to normal expectations.

\ea\label{ex:4.186}
\gll ¡Ko te {\ꞌ}aroha \textbf{{\ꞌ}ō} i a koe! \\
~\textsc{prom} \textsc{art} pity really at \textsc{prop} \textsc{2sg} \\

\glt
‘Poor you! (How pitiable you are!)’ \textstyleExampleref{[R490.018]} 
\z

\textit{{\ꞌ}Ō}\is{o (asseverative)@{\ꞌ}ō (asseverative)} is used in rhetorical questions to which a negative answer is expected. As in other cases, \textit{{\ꞌ}ō} emphasises that the proposition expressed in the question is not in line with what one would expect.

\ea\label{ex:4.187}
\gll ¿Ko haŋa {\ꞌ}ana \textbf{{\ꞌ}ō} pēaha koe mo pako{\ꞌ}o tō{\ꞌ}oku rima? \\
~\textsc{prf} want \textsc{cont} really perhaps \textsc{2sg} for loose \textsc{poss.1sg.o} hand \\

\glt 
‘Do you want my hand to come loose?!’ \textstyleExampleref{[R215.020]} 
\z

\ea\label{ex:4.188}
\gll ¿Kai pāhono mai {\ꞌ}ana \textbf{{\ꞌ}ō} koe i tū vānaŋa era {\ꞌ}ā{\ꞌ}aku?\\
~\textsc{neg.pfv} answer hither \textsc{cont} really \textsc{2sg} \textsc{acc} \textsc{dem} word \textsc{dist}\textsc{} \textsc{poss.1sg.a}\\

\glt 
‘Don’t you answer to what I said?’ \textstyleExampleref{[R315.264]} 
\z
\is{o (asseverative)@{\ꞌ}ō (asseverative)|)}

\subsubsection[Dubitative hō]{Dubitative \textit{hō}}\label{sec:4.5.4.6}

\textit{Hō}\is{ho (dubitative)@hō (dubitative)|(} – a rather rare \isi{particle} – adds an element of un\isi{certainty} or doubt to questions (whether polar or content questions\is{Question!content}): ‘maybe...’. It occurs after the first constituent of the clause.

\ea\label{ex:4.189}
\gll ¿He ma{\ꞌ}u \textbf{hō} e au {\ꞌ}o {\ꞌ}ina? \\
~\textsc{ntr} carry \textsc{dub} \textsc{ag} \textsc{1sg} or \textsc{neg} \\

\glt 
‘Should I take it or not?’ \textstyleExampleref{[R460.002]} 
\z

\ea\label{ex:4.190}
\gll He aha \textbf{hō} te aura{\ꞌ}a o te vānaŋa era?\\
\textsc{pred} what \textsc{dub} \textsc{art} meaning of \textsc{art} word \textsc{dist}\\

\glt
‘What could be the meaning of those words?’ \textstyleExampleref{[Luke 1:29]}
\z

\textit{Hō} may be used in rhetorical questions to which the expected answer is ‘no’.\footnote{\label{fn:208}This use of \textit{hō} is only found in the Bible translation.} 

\ea\label{ex:4.191}
\gll ¿E ai rō {\ꞌ}ana \textbf{hō} te me{\ꞌ}e mo ta{\ꞌ}e rova{\ꞌ}a e te {\ꞌ}Atua mo aŋa? \\
~\textsc{ipfv} exist \textsc{emph} \textsc{cont} \textsc{dub} \textsc{art} thing for \textsc{conneg} obtain \textsc{ag} \textsc{art} God for do \\

\glt 
‘Would there be anything God is not able to do?’ \textstyleExampleref{[Gen. 18:14]}
\z
\is{ho (dubitative)@hō (dubitative)|)}
\section{Demonstratives}\label{sec:4.6}
\is{Demonstrative|(}\subsection{Forms}\label{sec:4.6.1}

Rapa Nui does not have a single class of demonstratives\is{Demonstrative}. Rather, it has four classes of particles with demonstrative functions. Each class consists of three particles indicating different degrees of distance: proximal (close to the speaker), medial (at some distance, often close to the hearer) and distal (removed from speaker and hearer).\footnote{\label{fn:209}Forms of all classes are glossed \textsc{prox}, \textsc{med} and \textsc{dist}, respectively.} The four classes are similar or even identical in form, but differ in syntactic status; besides, they exhibit certain differences in function. 

\begin{itemize}
\item 
\textsc{Demonstrative determiners} occur before the noun, in the same position as other determiners. In addition to the proximal, medial and distal forms, there are two forms which are neutral with respect to distance (glossed dem).

\item 
\textsc{Postnuclear} (= postnominal and postverbal) \textsc{demonstratives}\is{Demonstrative} occur after the noun or the verb.

\item 
\textsc{Deictic locationals}\is{Locational} are a subclass of the locationals\is{Locational} (\sectref{sec:3.6}). They point to a general location: ‘here, there’, and correspond to what \citet[228]{Dixon2010-2} labels “local adverbial demonstratives\is{Demonstrative}”. In addition to the proximal, medial and distal forms, there is also a neutral pro-form \textit{ira} which syntactically belongs to the same class.

\item 
\textsc{Demonstrative pronouns} are relatively rare and used in limited contexts. 

\end{itemize}

The first two are what \citet[225]{Dixon2010-2} calls “nominal demonstratives\is{Demonstrative}”, elements in the \isi{noun phrase} which specify nouns for \isi{definiteness}\is{Definiteness}, indicate distance with respect to the speaker or hearer, and enable participant tracking in discourse. The others have a more independent status.

The forms for each class are given in \tabref{tab:33}.\footnote{\label{fn:210}\citet{Clark1974} reconstructs two sets of demonstratives for \is{Proto-Polynesian}PPN: monomoraic unstressed forms \textit{*\nobreakdash-ni/*\nobreakdash-na/*-ra} and bimoraic long forms \textit{nei/naa/laa}. In Rapa Nui, as in some other languages, both sets are reflected, with the exception of \textit{-ni} (\textit{nī} patterns with the long forms and must have developed from \textit{nei} by monophthongisation). Rapa Nui is the only language to have \textit{e-} in the short form \textit{era}; \textit{ena} occurs in \ili{Tongan} as well. However, similar forms occur in \ili{Rarotongan} (\citealt[415–416]{Buse1963Nominal}) and \ili{Tahitian} (pers.obs.), though less overtly. In these languages, the enclitics \textit{na} and \textit{ra} cause lengthening of the preceding vowel, accompanied by stress shift:
\ea
\gll 
\textup{[te taˈɁata]} ~ ~ \textup{[te taɁaˈtaː ra]} ~ ~ ~ \textup{(\ili{Tahitian})} \\
  ‘the~man’ ~ ~   ‘that~man’\\
\z
Thus, \textit{na} and \textit{ra} in these languages actually consist of a CV \isi{syllable}\is{Syllable} preceded by an unspecified vowel (\textit{\textbf{V}}\textit{na}, \textit{\textbf{V}}\textit{ra}), which means that they are quite similar to Rapa Nui \textit{ena} and \textit{era}, respectively.}

\begin{table}
%\fittable{
\begin{tabular}{p{15mm}p{22mm}p{22mm}p{22mm}p{22mm}} 
\lsptoprule
& 
\parbox{2.3cm}{  {demonstrative} \newline {determiners}}& 
\parbox{2.4cm}{  {postnuclear} \newline {demonstratives}}& 
\parbox{2cm}{  {deictic} \newline {locationals}\is{Locational}}& 
\parbox{2.4cm}{  {demonstrative} \newline  {pronouns}}\\
\midrule
proximal & \textit{nei, nī}& \textit{nei}& \textit{nei}& \textit{nei}\\
medial & \textit{nā}& \textit{ena}& \textit{nā}& \textit{nā}\\
distal & \textit{rā}& \textit{era}& \textit{rā}& \textit{rā}\\
neutral & \textit{tau/tou/tū,~hū}&  & \textit{ira}& \\
\lspbottomrule
\end{tabular}
%}
\caption{Classes of demonstratives}
\label{tab:33}
\end{table}

The four classes will be discussed in the following subsections. First the neutral demonstrative determiners will be discussed (\sectref{sec:4.6.2}), followed by the postnominal demonstratives\is{Demonstrative} (\sectref{sec:4.6.3}), as these commonly occur together. The other demonstrative determiners are discussed in \sectref{sec:4.6.4}. \sectref{sec:4.6.5} deals with deictic locationals\is{Locational}, \sectref{sec:4.6.6} with demonstrative pronouns. 

Postverbal demonstratives are discussed in Chapter 7 (\sectref{sec:7.6}), as their use is closely tied to other verb phrase elements (especially \isi{aspect} markers).

\subsection{Neutral demonstrative determiners}\label{sec:4.6.2}
\is{Demonstrative!determiner}\subsubsection[The t{}-demonstrative: form and function]{The \textit{t}{}-demonstrative: form and function}\label{sec:4.6.2.1}

Rapa Nui has a set of demonstrative determiners of the form \textit{tVV}:

\ea\label{ex:3.192a}
\textit{tau}\is{tau (demonstrative)}{\rmfnm} ~~~ \textit{ tou}\is{tou (demonstrative)} ~~~ \textit{ tū}\is{tu (demonstrative determiner)@tū (demonstrative determiner)}
\z
\footnotetext{\label{fn:211}\textit{Tau} is probably related to \is{Eastern Polynesian}PEP *\textit{taua} (see \citealt[60]{Pawley1966}; \citealt[12]{Green1985}), which, however, only occurs in Tahitic languages. An indication for a relationship between the two is that \textit{taua}, like Rapa Nui \textit{tau}, is an \isi{anaphoric}\is{Anaphora} determiner which co-occurs with postnominal demonstratives\is{Demonstrative} – obligatorily so in \ili{Tahitian} (\citealt[64–65]{AcadémieTahitienne1986}), optionally in \ili{Māori} \citep[152]{Bauer1993}. 
\citet[462]{RigoVernaudon2004} consider \ili{Tahitian} \textit{taua} to consist of the article \textit{te} + a cognate of the demonstrative \textit{ua} which appears in \ili{Hawaiian} but has no cognates in any other language. They tentatively propose that this \textit{ua} is originally the same morpheme as the perfect aspectual \textit{ua} which occurs in both \ili{Tahitian} and \ili{Hawaiian}; however, the latter is a reflex of \is{Proto-Polynesian}PPN \textit{*kua}, while \textit{taua} also occurs in languages which have preserved \is{Proto-Polynesian}PPN *\textit{k}, like \ili{Māori}, \ili{Rarotongan} and \ili{Pa’umotu} (Pollex, see \citealt{GreenhillClark2011}).}

These forms are semantically and syntactically equivalent; they succeed each other in the history of Rapa Nui. In older texts, \textit{tau} is predominant; in some corpora it is the only form in use. \textit{Tou} occurs in both older and newer texts; nowadays, \textit{tū} is used. The sources thus show a gradual \isi{vowel assimilation} \textit{tau} {\textgreater} \textit{tou} {\textgreater} \textit{tū}.\footnote{\label{fn:212}A similar monophthongisation process may have taken place in \ili{Rapa} (=Rapa Iti): the definite marker \textit{tō} is probably derived from *\textit{taua}, through a development \textit{taua {\textgreater} tau {\textgreater} tou {\textgreater} tō} \citep[183]{Walworth2015Thesis}.} 

As the three forms are diachronic variants of the same \isi{particle}, they will be treated as a single “\textit{t-}demonstrative”. \is{tu (demonstrative determiner)@tū (demonstrative determiner)}The \textit{t}{}-demonstrative is a neutral form, which – unlike other demonstratives\is{Demonstrative} – is not differentiated for relative distance.\footnote{\label{fn:213}According to \citet[280]{AndersonKeenan1985}, one-term deictic systems, which do not indicate relative distance, are crosslinguistically very rare. \ili{French} \textit{ce} is another example, but like the Rapa Nui \textit{t-}demonstrative, it usually goes together with another demonstrative element which does express distance. Notice that the \textit{t-}demonstrative in combination with the identity marker \textit{{\ꞌ}ā}/\textit{{\ꞌ}ana} is a true one-term subsystem: in this construction no relative distance is expressed, despite the presence of a demonstrative. In such a case, as Anderson and Keenan suggest, the demonstrative is little different from a definite article.}  It is always accompanied by one of the following postnominal elements: either a postnominal demonstrative\is{Demonstrative!postnominal} (PND\is{Demonstrative!postnominal}) \textit{nei}, \textit{ena} or \textit{era} or the identity marker \textit{{\ꞌ}ā} or \textit{{\ꞌ}ana}, but never both. Of these two options, the PND\is{Demonstrative!postnominal} is by far the most common one.

In combination with a PND\is{Demonstrative!postnominal}, the \textit{t}{}-demonstrative has \textsc{anaphoric}\is{Anaphora} function: it signals that the entity referred to has been mentioned in the preceding context (and, by implication, is known to the hearer). In \REF{ex:4.192}, there are three referents: Ure a Ohovehi, the boat and the men. All have been mentioned before, and all are referred to with the same combination of a \textit{t}{}-demonstrative and a PND\is{Demonstrative!postnominal}. 

\ea\label{ex:4.192}
\gll He tike{\ꞌ}a e \textbf{tau} kope \textbf{era}, ko Ure {\ꞌ}a Ohovehi, \textbf{tau} vaka \textbf{era}  o \textbf{tau} ŋāŋata \textbf{era}.\\
\textsc{ntr} see \textsc{ag} \textsc{dem} person \textsc{dist} \textsc{prom} Ure a Ohovehi \textsc{dem} boat \textsc{dist}  of \textsc{dem} men \textsc{dist}\\

\glt
‘That man Ure a Ohovehi saw that boat of those people.’ \textstyleExampleref{[Blx-3.070]}
\z

The use of the \textit{t-}demonstrative with postnominal demonstratives\is{Demonstrative} is further discussed in \sectref{sec:4.6.3}.

In combination with the identity marker \textit{{\ꞌ}ā}\is{a (identity)@{\ꞌ}ā (identity)}\textit{/{\ꞌ}ana} the \textit{t}{}-demonstrative expresses \textsc{identity} with an entity previously mentioned; this is discussed in \sectref{sec:5.9}.

As demonstratives\is{Demonstrative} are the main \isi{anaphoric}\is{Anaphora} device to track participants in discourse, they are much more common than \ili{English} demonstratives\is{Demonstrative}. \is{tu (demonstrative determiner)@tū (demonstrative determiner)}Example \REF{ex:4.192} would sound unnatural in translation if all the demonstratives\is{Demonstrative} were translated by demonstratives\is{Demonstrative}.\footnote{\label{fn:214}See \citet[21]{Englert1978}: “El artícula \textit{tou-era} (a veces \textit{tau-era}) es pronombre demostrativo que se usa frequentemente como simple artículo definido.” (The article \textit{tou-era} (sometimes \textit{tau-era}) is a demonstratie pronoun which is often used simply as a definite article.)} 

\subsubsection{The demonstrative \textit{hū}}\label{sec:4.6.2.2}
\is{hu (demonstrative)@hū (demonstrative)}
The demonstrative \textit{hū}\footnote{\label{fn:215}Etymologically, \textit{hū} is more different from \textit{tū} than its shape may suggest. As \textit{hū} (unlike \textit{tū}) already occurs in older texts, it cannot be derived from \textit{tū} (e.g. by analogy of \textit{te} and \textit{he}). \textit{Hū} may be related to \ili{Marquesan} \textit{hua}, which likewise serves as an \isi{anaphoric}\is{Anaphora} article. (\citealt[62]{Cablitz2006}; \citealt[49]{Bergmann1963}.) Bergmann also suggests a tentative link to the \ili{Hawaiian} demonstrative \textit{ua}.} is always accompanied by a postnominal demonstrative\is{Demonstrative!postnominal} or an identity marker, just like the \textit{t-}demonstrative. It is much less common than the \textit{t-}forms and especially occurs in older texts, but is still in use. Like \textit{tū}, it indicates that the referent has been mentioned before; it may indicate a more pointed deixis: ‘just that, precisely that’.

\ea\label{ex:4.193}
\gll —Ta{\ꞌ}e ko Reŋa Roiti ta{\ꞌ}a me{\ꞌ}e ena. —¿He aha rā  \textbf{hū} me{\ꞌ}e era?\\
~~~~\textsc{conneg} \textsc{prom} Renga Roiti \textsc{poss.2sg.a} thing \textsc{med} ~~~~~\textsc{pred}~ what \textsc{intens}  \textsc{dem} thing \textsc{dist}\\

\glt 
‘—That one is not Renga Roiti. —Then what exactly is it?!’ \textstyleExampleref{[Ley-9-56.092–093]}
\z

\ea\label{ex:4.194}
\gll He kī ki te nu{\ꞌ}u mo oho a {\ꞌ}uta {\ꞌ}ana mo haka tau mo u{\ꞌ}i  {\ꞌ}atakea ko \textbf{hū} ŋā io era.\\
\textsc{ntr} say to \textsc{art} people for go toward inland \textsc{ident} for \textsc{caus} hang for look  if \textsc{prom} \textsc{dem} \textsc{pl} young\_man \textsc{dist}\\

\glt 
‘He told the people to go ashore and lie in waiting to see whether it would be those (same) boys.’ \textstyleExampleref{[R425.011]} 
\z

\subsection{Postnominal demonstratives} \label{sec:4.6.3}
\is{Demonstrative!postnominal|(}
The postnominal demonstratives\is{Demonstrative} \textit{nei}, \textit{ena} and \textit{era} (henceforth PND\is{Demonstrative!postnominal}) indicate different degrees of distance:
\begin{tabbing}
xxxx \= xxxxxx \= xxxxxxxxxxxxxx\kill
\> \textit{nei} \>  proximity, close to the speaker\\
\> \textit{ena}\>   medial distance, close to the hearer\\
\> \textit{era} \>  farther distance, removed from both speaker and hearer
\end{tabbing}
PNDs occur towards the right periphery of the \isi{noun phrase} (see the chart in \sectref{sec:5.1}).

As discussed in \sectref{sec:4.6.2}, PNDs\is{Demonstrative!postnominal} are obligatory when the noun is preceded by a \textit{t}{}-de\-mon\-stra\-tive (\textit{tau}/\textit{tou}/\textit{tū}\is{tu (demonstrative determiner)@tū (demonstrative determiner)}), unless the \isi{noun phrase} contains the identity marker \textit{{\ꞌ}ā}/\textit{{\ꞌ}ana}. PNDs also occur in combination with other determiners: articles\is{te (article)} as in \REF{ex:4.195}, possessive pronouns\is{Pronoun!possessive} as in \REF{ex:4.196}:

\ea\label{ex:4.195}
\gll te kona hare \textbf{era} \\
\textsc{art} place house \textsc{dist} \\

\glt 
‘home’ \textstyleExampleref{[R210.021]} 
\z

\ea\label{ex:4.196}
\gll tō{\ꞌ}ona koro \textbf{era} \\
\textsc{poss.3sg.o} Dad \textsc{dist} \\

\glt
‘his father’ \textstyleExampleref{[R380.010]} 
\z

PNDs may be used either deictically\is{Deixis} or anaphorically\is{Anaphora}. As deictic markers they serve to point at something which is visible in the nonlinguistic context. As \isi{anaphoric}\is{Anaphora} markers they refer to entities in the discourse context: entities which have been mentioned before or which are known by some other means. In practice, the \isi{anaphoric}\is{Anaphora} use is much more common in discourse.\footnote{\label{fn:216}\citet[363]{Hooper2010} notices the same in \ili{Tokelauan} discourse: situational (=deictic) use only plays a “very minor part” in texts.}

In the following sections, the PNDs\is{Demonstrative!postnominal} are discussed in turn, starting with the most common form \textit{era}.

\subsubsection[Distal/neutral era]{Distal/neutral \textit{era}}\label{sec:4.6.3.1}

\is{era (distal)!postnominal|(}When \textit{era} is used deictically\is{Deixis}, it serves to point at something at a distance from both speaker and hearer. 

\ea\label{ex:4.197}
\gll ¿Hē te haraoa o te poki \textbf{era}?\\
~\textsc{cq} \textsc{art} bread of \textsc{art} child \textsc{dist}\\

\glt 
‘Where is the bread of that child (over there)?’ \textstyleExampleref{[R245.041]} 
\z

\ea\label{ex:4.198}
\gll Ka noho, ki ma{\ꞌ}u mai tu{\ꞌ}u māmātia \textbf{era} i te kai mā{\ꞌ}au.\\
\textsc{imp} sit to carry hither \textsc{poss.2sg.o} aunt \textsc{dist} \textsc{acc} \textsc{art} food \textsc{ben.2sg.a}\\

\glt
‘Sit down, so your aunt (over there) can bring you food.’ \textstyleExampleref{[R245.065]} 
\z

Much more commonly, \textit{era} is \isi{anaphoric}\is{Anaphora}. \textit{Era} is by far the most common postnominal demonstrative\is{Demonstrative!postnominal} and the most neutral in sense. In its \isi{anaphoric}\is{Anaphora} use \textit{era} usually does not have a connotation of distance, but is simply a general-purpose demonstrative. 

\textit{Era} is especially common with the \textit{t}{}-demonstrative determiner (\sectref{sec:4.6.2}). The combination \textit{tū}\is{tu (demonstrative determiner)@tū (demonstrative determiner)}\textit{/tou/tau N era} is the most general device in narrative texts to refer to participants mentioned earlier in the context. This makes its use extremely common in discourse.\footnote{\label{fn:217}\citet[81]{Naess2004} notices that demonstratives\is{Demonstrative} in \ili{Pileni} (a Polynesian outlier) are “used to an extent which appears quite extraordinary for a language of this family, perhaps for any language”. The same is true for Rapa Nui\is{Demonstrative}: over the whole text-corpus, \textit{era} occurs almost 15,000 times and is the seventh most common word overall (after the determiners \textit{te} and \textit{he} and a number of prepositions). Given the fact that demonstratives not only serve to indicate spatial deixis but mark \isi{definiteness} and anaphora as well (functions carried out by definite articles in other languages), their high frequency is not as surprising as it may seem at first sight.} In the following example, the two main characters of the story, neither of whom is mentioned by name, are referred to as \textit{tau taŋata era} ‘that man’ and \textit{tau vi{\ꞌ}e era} ‘that woman’.

\ea\label{ex:4.199}
\gll He moe rō {\ꞌ}avai \textbf{tau} \textbf{taŋata} \textbf{era}. He koromaki ki \textbf{tau} \textbf{vi{\ꞌ}e} \textbf{era}  to{\ꞌ}o era e tō{\ꞌ}ona matu{\ꞌ}a. He moe \textbf{tau} \textbf{taŋata} \textbf{era}, kai kai. He {\ꞌ}ōtea, he pō haka{\ꞌ}ou, \textbf{tau} \textbf{taŋata} \textbf{era}, he mate \textbf{tau} \textbf{taŋata} \textbf{era}, he koromaki ki \textbf{tau} \textbf{vi{\ꞌ}e} \textbf{era}.\\
\textsc{ntr} lie\_down \textsc{emph} certainly \textsc{dem} man \textsc{dist} \textsc{ntr} miss to \textsc{dem} woman \textsc{dist}  take \textsc{dist} \textsc{ag} \textsc{poss.3sg.o} parent \textsc{ntr} lie\_down \textsc{dem} man \textsc{dist} \textsc{neg.pfv} eat \textsc{ntr} dawn \textsc{ntr} night again \textsc{dem} man \textsc{dist} \textsc{ntr} die \textsc{dem} man \textsc{dist} \textsc{ntr} miss to \textsc{dem} woman \textsc{dist}\\

\glt
‘The man slept. He longed for the woman that had been taken (back) by her father. The man slept, he did not eat. Day came, then night again; the man died, that man, out of longing for the woman.’ \textstyleExampleref{[Mtx-5-02.057-060]}
\z

In the following example, two participants (the father and the child) and one object (the child’s umbilical cord) are first introduced with the article \textit{te.} The next time they are mentioned, all are marked with \textit{tou/tū N era}.

\ea\label{ex:4.200}
\gll He poreko \textbf{te} \textbf{poki} o \textbf{te} \textbf{taŋata} \textbf{e} \textbf{tahi}. He uŋa mai te roŋo mo e{\ꞌ}a atu o te taŋata nei, mo oho, mo haha{\ꞌ}u i \textbf{te} \textbf{pito}. I e{\ꞌ}a era te taŋata nei, i oho era ki \textbf{tou} \textbf{pito} \textbf{era} o \textbf{tū} \textbf{poki} \textbf{era}  o \textbf{tū} \textbf{taŋata} \textbf{era} mo haha{\ꞌ}u...\\
\textsc{ntr} born \textsc{art} child of \textsc{art} man \textsc{num} one \textsc{ntr} send hither \textsc{art} message for go\_out away of \textsc{art} man \textsc{prox} for go for tie \textsc{acc} \textsc{art} navel \textsc{pfv} go\_out \textsc{dist} \textsc{art} man \textsc{prox} \textsc{pfv} go \textsc{dist} to \textsc{dem} navel \textsc{dist} of \textsc{dem} child \textsc{dist}  of \textsc{dem} man \textsc{dist} for tie\\

\glt 
‘A child was born to a certain man. A message was sent for this (other) man to come, to tie the navel (cord). When man had gone out to tie the navel (cord){\rmfnm} of the child of that man...’ \textstyleExampleref{[Blx-2-1.001-005]}
\z
\footnotetext{Lit. ‘gone out to the navel to tie’; for this construction, see \sectref{sec:11.6.3}.}

Another determiner-demonstrative combination is \textit{te}\is{te (article)} \textit{N era}, with the article \textit{te} instead of a demonstrative determiner. This combination is used to refer to something which is known to both speaker and hearer, whether or not it has been mentioned in the preceding context. This means that \textit{te N era} indicates \textsc{definiteness}\is{Definiteness}:\footnote{\label{fn:219}See the discussion of \isi{definiteness}\is{Definiteness} in section \sectref{sec:5.3.3}. The development of demonstratives to definite markers may have taken place in \ili{Tongan} as well: \citet{Clark1974} shows how the “definitive accent” (a stress shift to the final \isi{syllable} of the noun, marking \isi{definiteness}) may have derived from a postposed demonstrative \textit{*aa}.} it signals that speaker and hearer are both able to identify the referent of the \isi{noun phrase}. It is therefore the equivalent of the \ili{English} (or \ili{Spanish}) definite article, rather than a demonstrative. 

Like \textit{tau/tou/tū}\is{tu (demonstrative determiner)@tū (demonstrative determiner)} \textit{N era}, it may be used to refer to participants in a story who have been mentioned before. In \REF{ex:4.201}, \textit{tau poki era} and \textit{te poki era} refer to the same child:

\ea\label{ex:4.201}
\gll He oŋa mai \textbf{tau} \textbf{poki} \textbf{era} o tau taŋata era ko Kava te Rūruki. He tikera \textbf{te} \textbf{poki} \textbf{era}...\\
\textsc{ntr} appear hither \textsc{dem} child \textsc{dist} of \textsc{dem} man \textsc{dist} \textsc{prom} Kava te Ruruki \textsc{ntr} see \textsc{art} child \textsc{dist}\\

\glt 
‘The child of that man Kava te Ruruki observed him. The child saw it...’ \textstyleExampleref{[Ley-9-57.035]}
\z

\ea\label{ex:4.202}
\gll He tupu te poki o te vi{\ꞌ}e, he poreko... He hāŋai, he nuinui  \textbf{te} \textbf{poki} \textbf{era}.\\
\textsc{ntr} grow \textsc{art} child of \textsc{art} woman \textsc{ntr} born \textsc{ntr} feed \textsc{ntr} big:\textsc{red}  \textsc{art} child \textsc{dist}\\

\glt
‘A woman was with child, it was born. The child was raised and grew up.’ \textstyleExampleref{[Mtx-7-21.004–005]}
\z

\textit{Te}\is{te (article)} \textit{N era} may also refer to entities which are generally known, or which are presumed to be present in the context. In the following example, ‘the cliffs’ refers to the cliffs in general (which all hearers will presumably know to be part of the Rapa Nui coastline); no specific cliff is meant.

\ea\label{ex:4.203}
\gll I na{\ꞌ}a era a {\ꞌ}Oho Takatore i tū kūpeŋa era, he oho mai  ki \textbf{te} \textbf{kona} \textbf{{\ꞌ}ōpata} \textbf{era}.\\
\textsc{pfv} hide \textsc{dist} \textsc{prop} Oho Takatore \textsc{acc} \textsc{dem} net \textsc{dist} \textsc{ntr} go hither  to \textsc{art} place cliff \textsc{dist}\\

\glt
‘When Oho Takatore had hidden that net, he went to the cliffs (lit. the cliff place).’ \textstyleExampleref{[R304.110]} 
\z

\textit{Te}\is{te (article)} \textit{N era} may also refer to things which have not been previously mentioned, but which are definite because they are explained in the \isi{noun phrase} itself: a modifying phrase or \isi{relative clause}\is{Clause!relative} after the noun specifies what the noun refers to. In \REF{ex:4.204} below, the referent of \textit{te haŋa era} ‘the bay’ is specified by the genitive phrase \textit{o {\ꞌ}Akahaŋa}; in \REF{ex:4.205} \textit{te ha{\ꞌ}u era} ‘the hats’ is explained by the \isi{relative clause}\is{Clause!relative} \textit{e aŋa era hai rau toa} ‘made with sugarcane leaves’.

\ea\label{ex:4.204}
\gll {\ꞌ}I mu{\ꞌ}a i \textbf{te} \textbf{haŋa} \textbf{era} o {\ꞌ}Akahaŋa, te noho haŋa ō{\ꞌ}ona. \\
at front at \textsc{art} bay \textsc{dist} of Akahanga \textsc{art} stay \textsc{nmlz} \textsc{poss.3sg.o} \\

\glt 
‘His residence was in front of the bay of Akahanga.’ \textstyleExampleref{[Blx-2-3.002]}
\z

\ea\label{ex:4.205}
\gll O rā hora {\ꞌ}ā te ŋā vi{\ꞌ}e o nei pa{\ꞌ}ari era e hatu rō {\ꞌ}ana  i \textbf{te} \textbf{ha{\ꞌ}u} \textbf{era} e aŋa era hai rau toa...\\
of \textsc{dist} time \textsc{ident} \textsc{art} \textsc{pl} woman of \textsc{prox} adult \textsc{dist} \textsc{ipfv} weave \textsc{emph} \textsc{cont}  \textsc{acc} \textsc{art} hat \textsc{dist} \textsc{ipfv} make \textsc{dist} \textsc{ins} leaf sugarcane\\

\glt
‘At that time the older women here wove those hats which are made with sugarcane leaves....’ \textstyleExampleref{[R106.049]} 
\z

In these contexts, where the \isi{noun phrase} becomes definite by virtue of a modifier, \textit{tū N era} is not (or rarely) used. In other words, where \textit{Det N era} has a unique referent, \textit{tū} is used; where \textit{Det N era} as such does not have a unique referent but needs a modifier to pinpoint its reference, \textit{te} is used.

To summarise: 

\begin{itemize}
\item 
\textit{Te N era} is used when the \isi{noun phrase} is \textsc{definite}\is{Definiteness} for any reason (whether known from the context, by general knowledge, or defined by modifiers in the NP)

\item 
\textit{Tū N era} is \textsc{anaphoric}\is{Anaphora}, indicating that the referent of the \isi{noun phrase} is known from the preceding context.

\end{itemize}
\is{era (distal)!postnominal|)}
\subsubsection{Proximal \textit{nei}}\label{sec:4.6.3.2}
\is{nei (proximal)!postnominal|(}
\textit{Nei} indicates proximity. It is more commonly used with the \is{te (article)}article \textit{te} than with the demonstrative \textit{tū}\is{tu (demonstrative determiner)@tū (demonstrative determiner)}. When used deictically\is{Deixis}, \textit{nei} refers to something close to the speaker:

\ea\label{ex:4.206}
\gll \textbf{Te} \textbf{kona} \textbf{nei} {\ꞌ}i \textbf{te} \textbf{hare} \textbf{nei} mo te poki mā{\ꞌ}aŋa nei {\ꞌ}ā{\ꞌ}aku. \\
\textsc{art} place \textsc{prox} at \textsc{art} house \textsc{prox} for \textsc{art} child chick \textsc{prox} \textsc{poss.1sg.a} \\

\glt
‘This place (here) in this house is for my adopted child.’ \textstyleExampleref{[R229.271]} 
\z

The proximity indicated by \textit{nei} may also be temporal: the event takes place close to the time of speaking. This is especially clear when \textit{nei} is used with nouns denoting time.

\ea\label{ex:4.207}
\gll \textbf{{\ꞌ}I} \textbf{te} \textbf{hora} \textbf{nei} pa{\ꞌ}i ku ŋaro {\ꞌ}ana rā mauku.\\
at \textsc{art} time \textsc{prox} in\_fact \textsc{prf} disappear \textsc{cont} \textsc{intens} grass\\

\glt
‘Nowadays (lit. ‘in this time’) that grass has disappeared.’ \textstyleExampleref{[R106.050]} 
\z

However, temporal proximity is not necessarily related to the time of speaking. The reference time may also be the time of other events in the same text. In the following example, \textit{te noho iŋa nei} ‘this time/epoch’ refers to the time when the events in the story happened. 

\ea\label{ex:4.208}
\gll \textbf{{\ꞌ}I} \textbf{te} \textbf{noho} \textbf{iŋa} \textbf{nei}, ho{\ꞌ}i, {\ꞌ}ina he mōrī, {\ꞌ}ina he vai... \\
at \textsc{art} stay \textsc{nmlz} \textsc{prox} indeed \textsc{neg} \textsc{pred} light \textsc{neg} \textsc{pred} water \\

\glt
‘At this time there was no electricity, no water...’ \textstyleExampleref{[R539-1.092]}
\z

\textit{Nei} also has \isi{anaphoric}\is{Anaphora} uses. It may refer to something which has been mentioned just before; the referent is close in a textual sense.

\ea\label{ex:4.209}
\gll ‘¡Ka haka kore te kope ena {\ꞌ}e ka haka e{\ꞌ}a mai a \textbf{Varavā}!’  \textbf{Te} \textbf{taŋata} \textbf{nei} i puru ai {\ꞌ}o te haka tumu i te ture.\\
~~\textsc{imp} \textsc{caus} lack \textsc{art} person \textsc{med} and \textsc{imp} \textsc{caus} go\_out hither \textsc{prop} Barabbas  \textsc{art} man \textsc{prox} \textsc{pfv} close \textsc{pvp} because\_of \textsc{art} \textsc{caus} origin \textsc{acc} \textsc{art} quarrel\\

\glt
‘ “Away with that man, release Barabbas!” This man had been imprisoned for provoking a riot.’ \textstyleExampleref{[Luke 23:19]}
\z

Unlike other postnominal demonstratives\is{Demonstrative}, \textit{nei} is also used cataphorically\is{Cataphora}, pointing forward to what follows. One such cataphoric use is at the beginning of stories: here \textit{nei} is often used to introduce (main) participants.\footnote{\label{fn:220}This use is common in newer stories, but not found at all in older texts.} An example is:

\ea\label{ex:4.210}
\gll {\ꞌ}I {\ꞌ}Ohovehi te noho iŋa o \textbf{te} \textbf{ŋā} \textbf{roe} \textbf{nei} e rua. \\
at Ohovehi \textsc{art} stay \textsc{nmlz} of \textsc{art} \textsc{pl} ant \textsc{prox} \textsc{num} two \\

\glt
‘In Ohovehi was the place where these two ants lived.’ \textstyleExampleref{[R214.001]} 
\z

This sentence is the beginning of a story about two ants. The use of \textit{nei} signals to the reader that the two ants will be playing an important role in the story that follows.\footnote{\label{fn:221}\ili{English} has a similar – somewhat informal – use of \textit{this}, to introduce a participant at the start of a story: \textit{‘}Yesterday I met this guy...\textit{’}} This use of \textit{nei} can be considered as cataphoric: \textit{nei} directs the hearer to look forward to provide more information about the indicated participant. 

Another cataphoric use of \textit{nei} is after generic nouns like \textit{me{\ꞌ}e}\is{mee ‘thing’@me{\ꞌ}e ‘thing’} ‘thing’. Here \textit{nei} signals that more specific information follows:\footnote{\label{fn:222}Again, \ili{English} provides a parallel use of ‘this’: ‘Listen to this: ....’; ‘This is what you need...’} 

\ea\label{ex:4.211}
\gll Te me{\ꞌ}e nei he ruku e ai te ŋā \textbf{me{\ꞌ}e} \textbf{nei}: he pātia, he hi{\ꞌ}o... \\
\textsc{art} thing \textsc{prox} \textsc{pred} dive \textsc{exh} exist \textsc{art} \textsc{pl} thing \textsc{prox} \textsc{pred} harpoon \textsc{pred} glass \\

\glt
‘For diving you need the following things: a harpoon, glasses...’ \textstyleExampleref{[R360.001]} 
\z

The same use of \textit{nei} (though not in a \isi{noun phrase}) is found in the expression \textit{pē nei ē} ‘like this’, which introduces speech or thought (see (\ref{ex:4.231}–\ref{ex:4.232}) on p.~\pageref{ex:4.231}).\is{nei (proximal)!postnominal|)}

\subsubsection[Medial ena]{Medial \textit{ena}}\label{sec:4.6.3.3}

\is{ena (medial distance)!postnominal|(}\textit{Ena} indicates something removed from the speaker, but close to the hearer:

\ea\label{ex:4.212}
\gll {\ꞌ}Ina koe ko kai i \textbf{te} \textbf{me{\ꞌ}e} \textbf{ena} o roto o \textbf{te} \textbf{kete} \textbf{ena}. \\
\textsc{neg} \textsc{2sg} \textsc{neg.ipfv} eat \textsc{acc} \textsc{art} thing \textsc{med} of inside of \textsc{art} basket \textsc{med} \\

\glt
‘Don’t eat those things in that basket (you have there).’ \textstyleExampleref{[Blx-3.036]}
\z

However, the use of \textit{ena} is somewhat limited: while \textit{nei} is regularly used with first person pronouns, \textit{ena} is not used with second person pronouns.

\is{Demonstrative!postnominal}After temporal nouns like \textit{tāpati} ‘week’ or \textit{matahiti} ‘year’, \textit{ena} signifies ‘next’.

\ea\label{ex:4.213}
\gll \textbf{Matahiti} \textbf{ena} he hoki a au ki te hāpī. \\
year \textsc{med} \textsc{ntr} return \textsc{prop} \textsc{1sg} to \textsc{art} learn \\

\glt
‘Next year I return to school.’ \textstyleExampleref{[R210.003]} 
\z

Here, \textit{ena} signifies a referent which is in the future, one step removed from the time of speaking. To refer to a time one step removed in the past, \textit{ena} is used in combination with the verb \textit{oti} ‘finish’. The following example occurs in a newspaper published in May, i.e. it refers to the previous month:

\ea\label{ex:4.214}
\gll {\ꞌ}I tū \textbf{{\ꞌ}āva{\ꞌ}e} \textbf{oti} \textbf{ena} o Vai~Tu{\ꞌ}u~Nui i ha{\ꞌ}amata i keri ai  o koā Jo Anne...\\
at \textsc{dem} month finish \textsc{med} of April \textsc{pfv} begin \textsc{pfv} dig \textsc{pvp}  of \textsc{coll} Jo Anne\\

\glt 
‘In the past month of April, Jo Anne and the others started to dig...’ \textstyleExampleref{[R647.106]} 
\z\is
{ena (medial distance)!postnominal|)}
\is{Demonstrative!postnominal|)}
\subsection{Demonstrative determiners}\label{sec:4.6.4}
\is{Demonstrative!determiner|(}
\textit{Nei}, \textit{nī}, \textit{nā} and \textit{rā} are demonstrative determiners indicating relative distance. Like the \textit{t-}demonstrative they exclude the article, but unlike these, they are not accompanied by a postnominal demonstrative\is{Demonstrative!postnominal} (except \textit{nī}).\footnote{\label{fn:223}\citet[8]{Chapin1974} also mentions a demonstrative \textit{tenei}, supposedly used in Egt-02. However, in \citet{Englert1974}, which includes this text, the form in question is printed as \textit{to nei}. The forms \textit{teenei}, \textit{teenaa} and \textit{teeraa}, which are common in Nuclear Polynesian languages \citep[51]{Pawley1966}, do not occur in Rapa Nui (see also \citealt[21]{LangdonTryon1983}), though they may have existed at a prior stage: \textit{tenā} possibly appears in the old chant \textit{e timo te akoako} \citep[426]{Fischer1994}. The fact that the demonstrative determiners \textit{nei}, \textit{nā} and \textit{rā} hardly occur in older texts, suggests that they did not develop from the \is{Eastern Polynesian}PEP demonstrative determiners \textit{*teenei, *teenaa, *teeraa} through loss of \textit{tee-}, but are an independent recent development.} In fact, these demonstratives\is{Demonstrative} themselves are very similar in sense to postnominal demonstratives\is{Demonstrative}. They are a recent development: demonstrative determiners are rarely found in older texts. It is not unlikely that they developed under \ili{Spanish} influence: \textit{nei taŋata} ‘this man’ by analogy of Sp. \textit{este hombre}.\footnote{\label{fn:224}This is pointed out by \citet[389]{Fischer2007}.}

As \textit{rā} is the most common (and most neutral) form, it will be discussed first.

\subsubsection[Distal/neutral rā]{Distal/neutral \textit{rā}}\label{sec:4.6.4.1}

\textit{Rā}\is{ra (demonstrative)@rā (demonstrative)!demonstrative determiner|(} is similar in meaning to the postnominal \textit{era} (\sectref{sec:4.6.3.1}): just like \textit{era} is the neutral postnominal demonstrative\is{Demonstrative!postnominal}, \textit{rā} is the neutral, most common, demonstrative determiner. 

\textit{Rā} is used deictically\is{Deixis}, referring to something which has not been mentioned before, but which is present in the extralinguistic context and hence accessible\is{Accessibility} to both speaker and hearer. It is used in conversation, for example, when pointing out something at a certain distance, or when indicating something on a picture or map:

\ea\label{ex:4.215}
\gll {\ꞌ}I \textbf{rā} \textbf{hare} a mātou e noho ena. \\
at \textsc{dist} house \textsc{prop} \textsc{1pl.excl} \textsc{ipfv} stay \textsc{med} \\

\glt
‘(discussing a photograph:) In that house we lived’. \textstyleExampleref{[R416.961]} 
\z

Like \textit{tū N era}, \textit{rā} is also used anaphorically\is{Anaphora}. In the following examples, the noun in question has been introduced in the preceding context. 

\ea\label{ex:4.216}
\gll He mate rō {\ꞌ}ai {\ꞌ}i roto {\ꞌ}i \textbf{rā} \textbf{hare}. \\
\textsc{ntr} die \textsc{emph} \textsc{subs} at insids at \textsc{dist} house \\

\glt 
‘She died inside that house.’ \textstyleExampleref{[R532-14.034]}
\z

\ea\label{ex:4.217}
\gll Ko {\ꞌ}ite {\ꞌ}ā, pa{\ꞌ}i, a ia i \textbf{rā} \textbf{hīmene} {\ꞌ}i te hare hāpī. \\
\textsc{prf} know \textsc{cont} in\_fact \textsc{prop} \textsc{3sg} \textsc{acc} \textsc{dist} song at \textsc{art} house learn \\

\glt
‘For she had learned that song at school.’ \textstyleExampleref{[R334.341]} 
\z

This means that \textit{tū}\is{tu (demonstrative determiner)@tū (demonstrative determiner)} \textit{N era} and \textit{rā} are similar in function. Even so, there are differences between the two. 

First of all, \textit{rā} is somewhat more informal than \textit{tū N era}. It tends to be more common in conversation and \isi{direct speech}, while \textit{tū N era} occurs more commonly in narrative texts.

Secondly, there are also collocational differences: \textit{rā N} is especially common before words denoting a moment or period of time, like \textit{hora} ‘time, moment, hour’, \textit{mahana} ‘day’ and \textit{noho iŋa} ‘period, epoch’, while \textit{tū N era} is found more often with concrete nouns like \textit{hare} ‘house’ and \textit{taŋata} ‘man, person’. 

In the third place, the relation between \textit{rā} and \textit{tū N era} also has a diachronic \isi{aspect}. \textit{Rā} is extremely rare in older texts. The demonstrative \textit{rā} does occur in these texts, but almost always as a locational (\sectref{sec:4.6.5})\is{Locational}: \textit{{\ꞌ}i rā} ‘over there’. 

In newer texts (most of which date from the 1980s), \textit{rā} is common, but \textit{tū} still occurs about twice as often. However, in the Bible translation – the largest part of which was done, or at least thoroughly revised, after 2000 – \textit{rā} is about 50\% more frequent than \textit{tū}. In the Bible translation, \textit{rā} is commonly used to track participants in discourse. 

Only when the \isi{noun phrase} contains a modifier (an adjective, a \isi{possessor}\is{Possession} or a \isi{relative clause}\is{Clause!relative}), \textit{tū N era} continues to be the default choice, even in the Bible translation:

\ea\label{ex:4.218}
\gll tū taŋata matapō era \\
\textsc{dem} man blind \textsc{dist} \\

\glt
‘that blind man’ \textstyleExampleref{[John 9:6]}
\z

Taking these facts together, we arrive at the following explanation: \textit{rā} was originally a deictic locational (\sectref{sec:4.6.5}), used to point at things and locations: ‘there, over there’. \textit{Tau/tou} had a different role: tracking participants in discourse, i.e. referring to entities mentioned earlier in the context. 

When \textit{rā} started to be used as a prenominal demonstrative, it was initially with the same deictic role it already had, pointing to for example things and locations (‘that house there’, ‘that place over there’), and points in time (‘on that day’). Gradually it acquired a participant-tracking role as well, but until recently this role has been predominantly fulfilled by \textit{tau/tou/tū}. This use of \textit{rā} is becoming more and more frequent, to the point where it is now more common than \textit{tū}\is{tu (demonstrative determiner)@tū (demonstrative determiner)}\textit{/tou}. Only in complex noun phrases is \textit{tū} still preferred.
\is{ra (demonstrative)@rā (demonstrative)!demonstrative determiner|)}
\subsubsection[Proximal nei]{Proximal \textit{nei}}\label{sec:4.6.4.2}

Prenominal \textit{nei}\is{nei (proximal)!demonstrative determiner|(} is similar in meaning to postnominal \textit{nei} (\sectref{sec:4.6.3}): it indicates proximity in time, location or discourse. It may refer to something near the speaker as in \REF{ex:4.219}, something just mentioned as in \REF{ex:4.220}, or to a time close to the time of the preceding discourse as in \REF{ex:4.221}:

\ea\label{ex:4.219}
\gll Te me{\ꞌ}e aŋa mai nei e \textbf{nei} \textbf{vi{\ꞌ}e}... \\
\textsc{art} thing do hither \textsc{prox} \textsc{ag} \textsc{prox} woman \\

\glt 
‘What this woman (near the speaker) has done...’ \textstyleExampleref{[Mat. 26:12]}
\z

\ea\label{ex:4.220}
\gll Mai tētahi henua o te norte o Nueva Zelántia i oho mai ai  ki \textbf{nei} \textbf{henua}.\\
from other land of \textsc{art} north of New Zealand \textsc{pfv} go hither \textsc{pvp}  to \textsc{prox} land\\

\glt 
‘From other countries, to the north of New Zealand, they came to this island (= New Zealand).’ \textstyleExampleref{[R346.012]} 
\z

\ea\label{ex:4.221}
\gll {\ꞌ}E tako{\ꞌ}a pa{\ꞌ}i, \textbf{nei} \textbf{noho} \textbf{iŋa} kai rahi mai {\ꞌ}ā te me{\ꞌ}e  he {\ꞌ}aurī ki nei.\\
and also in\_fact \textsc{prox} stay \textsc{nmlz} \textsc{neg.pfv} much hither \textsc{cont} \textsc{art} thing  \textsc{pred} iron to \textsc{prox}\\

\glt
‘And also, at this time (the period just mentioned), there was not much iron here.’ \textstyleExampleref{[R353.006]} 
\z

Pre- and postnominal \textit{nei}\is{nei (proximal)!demonstrative determiner} are not completely identical in function: while postnominal \textit{nei} may be cataphoric, referring to something which has not been mentioned yet, prenominal \textit{nei} always refers something which has been mentioned before.
\is{nei (proximal)!demonstrative determiner|)}
\subsubsection[Proximal nī]{Proximal \textit{nī}}\label{sec:4.6.4.3}

\textit{Nī}\is{ni (demonstrative determiner)@nī (demonstrative determiner)|(} is a relatively rare demonstrative, which is not found in older texts. Its function is similar to \textit{nei;} it must have arisen from \textit{nei} by \isi{vowel assimilation}. That this only happened prenominally may be because the prenominal position is phonologically less prominent: unlike postnominal \textit{nei}, it never receives phrase stress\is{Stress}.

\textit{Nī} often refers to something which has been recently mentioned. In the following example, \textit{nī taŋata} refers back to \textit{e te taŋata e tahi} in the previous sentence.

\ea\label{ex:4.222}
\gll Pē ira i hīmene ai \textbf{e} \textbf{te} \textbf{taŋata} \textbf{e} \textbf{tahi}... i te hīmene e tahi.  Ko To{\ꞌ}o Raŋi te {\ꞌ}īŋoa o \textbf{nī} \textbf{taŋata}.\\
like \textsc{ana} \textsc{pfv} sing \textsc{pvp} \textsc{ag} \textsc{art} man \textsc{num} one \textsc{acc} \textsc{art} song \textsc{num} one  \textsc{prom} To’o Rangi \textsc{art} name of \textsc{prox} man\\

\glt
‘In that way one man... sang a song. To’o Rangi was the name of this man.’ \textstyleExampleref{[R539-1.127–128]}
\z

Unlike prenominal \textit{nei}, \textit{nī} can be accompanied by a postnominal demonstrative\is{Demonstrative!postnominal}. Interestingly, the latter is not necessarily \textit{nei}:

\ea\label{ex:4.223}
\gll Mai rā hora ŋa{\ꞌ}aha era o \textbf{nī} \textbf{iate} \textbf{nei} i tiaki ai {\ꞌ}i nei. \\
from \textsc{dist} time burst \textsc{dist} of \textsc{prox} yacht \textsc{prox} \textsc{pfv} wait \textsc{pvp} at \textsc{prox} \\

\glt 
‘From the moment this yacht had broken down, they waited here.’ \textstyleExampleref{[R539-1.686]}
\z

\ea\label{ex:4.224}
\gll Tītika ki \textbf{nī} \textbf{titi} \textbf{{\ꞌ}ōpata} \textbf{era} o {\ꞌ}Ōroŋo... \\
straight to \textsc{prox} border cliff \textsc{dist} of Orongo \\

\glt
‘Straight opposite these cliffs of Orongo...’ \textstyleExampleref{[R112.008]} 
\z

\textit{Nī}\is{ni (demonstrative determiner)@nī (demonstrative determiner)} tends to be used for referents which are not central participants in the discourse: minor participants, objects (\textit{iate} above), places \textit{({\ꞌ}ōpata} above), time words like \textit{mahana} ‘day’ and \textit{hora} ‘time’.
\is{ni (demonstrative determiner)@nī (demonstrative determiner)|)}
\subsubsection[Medial nā]{Medial \textit{nā}}\label{sec:4.6.4.4}

\textit{Nā}\is{na (demonstrative)@nā (demonstrative)!demonstrative determiner|(} is occasionally used as a prenominal demonstrative. It is similar in meaning to postnominal \textit{ena}, referring to something not close to the speaker, but close to the hearer. Therefore it typically appears in \isi{direct speech}, as in the following example:

\ea\label{ex:4.225}
\gll Ka to{\ꞌ}o mai \textbf{nā} \textbf{matā} ka vero ki rote haha. \\
\textsc{imp} take hither \textsc{med} obsidian \textsc{imp} throw to inside\_\textsc{art} mouth \\

\glt 
‘Take that obsidian spearhead and throw it into his mouth’. \textstyleExampleref{[R304.020]} 
\z
\is{Demonstrative!determiner|)}
\is{na (demonstrative)@nā (demonstrative)!demonstrative determiner|)}
\subsection{Deictic locationals}\label{sec:4.6.5}
\is{Locational|(}\is{Locational}\subsubsection{\textit{Nei}, \textit{nā} and \textit{rā} as deictic locationals}\label{sec:4.6.5.1}

Deictic locationals have the same form as demonstrative determiners (\sectref{sec:4.6.4}): \textit{nei}\is{nei (proximal)!deictic locational}, \textit{nā}\is{na (demonstrative)@nā (demonstrative)!deictic locational} and \textit{rā}\is{ra (demonstrative)@rā (demonstrative)!deictic locational}. As locationals, they are a nucleus in their own right, rather than modifiers of a head noun. Just like all locationals (\sectref{sec:3.6.1})\is{Locational}, they can be preceded by a \isi{preposition}, but not by a determiner. They usually have a deictic function. Some examples:

\ea\label{ex:4.226}
\gll \textbf{Mai} \textbf{nei} te pahī nei i oho ai ki Tahiti. \\
from \textsc{prox} \textsc{art} ship \textsc{prox} \textsc{pfv} go \textsc{pvp} to Tahiti \\

\glt 
‘From here (=Rapa Nui) the ship went to Tahiti.’ \textstyleExampleref{[R239.091]} 
\z

\ea\label{ex:4.227}
\gll ¡Ka to{\ꞌ}o te me{\ꞌ}e era ka hakarē \textbf{{\ꞌ}i} \textbf{rā}! \\
~\textsc{imp} take \textsc{art} thing \textsc{dist} \textsc{imp} leave at \textsc{dist} \\

\glt 
‘Take that and leave it over there!’ \textstyleExampleref{[R208.173]} 
\z

\ea\label{ex:4.228}
\gll E aha {\ꞌ}ā kōrua \textbf{{\ꞌ}i} \textbf{nā}?\\
\textsc{ipfv} what \textsc{cont} \textsc{2pl} at \textsc{med}\\

\glt 
‘What are you doing there?’ \textstyleExampleref{[R416.514]} 
\z

Preceded by \textit{pē}\is{pe ‘like’@pē ‘like’} ‘like’, the deictic locationals\is{Locational} tend to be used anaphorically\is{Anaphora} rather than deictically\is{Deixis}. \textit{Pē rā}\is{ra (demonstrative)@rā (demonstrative)!deictic locational} is similar in function to \textit{pē ira} (see the next section): it refers back to a state of affairs mentioned before, ‘like that, in the same manner’. An example:

\ea\label{ex:4.229}
\gll \textbf{Pē} \textbf{rā} {\ꞌ}ā e {\ꞌ}amo mai era {\ꞌ}i te mahana.\\
like \textsc{dist} \textsc{ident} \textsc{ipfv} carry hither \textsc{dist} at \textsc{art} day\\

\glt
‘In the same way (as described before) he carried (food) every day.’ \textstyleExampleref{[R372.139]} 
\z

\textit{Pē nei}\is{pe ‘like’@pē ‘like’!pē nei} is used anaphorically\is{Anaphora} as well. As \textit{nei} expresses proximity, \isi{anaphoric}\is{Anaphora} \textit{pē nei} indicates what can be called discourse proximity: it refers back to something which has just been mentioned.

\ea\label{ex:4.230}
\gll \textbf{Pē} \textbf{nei} a Nueva Zelántia i noho ai mo te paratane. \\
like \textsc{prox} \textsc{prop} New Zealand \textsc{pfv} stay \textsc{pvp} for the British \\

\glt
‘In this way (just described) New Zealand came to belong to the British.’ \textstyleExampleref{[R346.022]} 
\z

Especially common is \textit{pē}\is{pe ‘like’@pē ‘like’} \textit{nei ē}, which introduces the content of a speech or thought, or a piece of knowledge. 

\ea\label{ex:4.231}
\gll {\ꞌ}Ina koe ko taŋi mai \textbf{pē} \textbf{nei} \textbf{ē}: ko au te kai rivariva mai. \\
\textsc{neg} \textsc{2sg} \textsc{neg.ipfv} cry hither like \textsc{prox} thus \textsc{prom} \textsc{1sg} \textsc{art} eat good:\textsc{red} hither \\

\glt 
‘Don’t cry (thinking about the fact) that I am eating well’ \textstyleExampleref{[R304.033]} 
\z

\ea\label{ex:4.232}
\gll Nu{\ꞌ}u rahi te nu{\ꞌ}u i mana{\ꞌ}u \textbf{pē} \textbf{nei} \textbf{ē} ko tētere {\ꞌ}ana ki Tahiti. \\
people many \textsc{art} people \textsc{pfv} think like \textsc{prox} thus \textsc{prf} \textsc{pl}:run \textsc{cont} to Tahiti \\

\glt 
‘Many people thought that they had fled to Tahiti.’ \textstyleExampleref{[R303.051]} 
\z

\subsubsection{The \isi{anaphoric} locational \textit{ira}}\label{sec:4.6.5.2}
\is{Anaphora}\is{Locational}
\textit{Ira}\is{ira (anaphor)}\footnote{\label{fn:225}\textit{Ira} does not occur in any other Polynesian language. However, most other \is{Eastern Polynesian}EP languages have a locational\is{Locational} anaphor\is{Anaphora} \textit{reira/leila} (‘there’, referring to a place mentioned before); Rapa Nui \textit{ira} may be a truncated reflex of the same form. This would mean that \textit{*leila} is not a \is{Central-Eastern Polynesian}PCE innovation as suggested by \citet[12]{Green1985} and Pollex (\citealt{GreenhillClark2011}), but a \is{Eastern Polynesian}PEP innovation with subsequent shortening in Rapa Nui.} is a multipurpose anaphor\is{Anaphora} (hence its gloss \textsc{ana}). Whereas personal pronouns\is{Pronoun!personal} serve as anaphors referring back to animate\is{Animacy} referents mentioned in the preceding context, \textit{ira} refers back to locations or situations.\footnote{\label{fn:226}Together, personal pronouns\is{Pronoun!personal} and \textit{ira} cover a large part of the field of possible referents for anaphora\is{Anaphora}. For other referents, no anaphor\is{Anaphora} is available, however: 1) inanimates. To refer back to an inanimate, the general-purpose noun \textit{me{\ꞌ}e} ‘thing’ can be used: \textit{te me{\ꞌ}e era}, lit. ‘that thing’. 2) time. \textit{{\ꞌ}I ira} can only refer to place, not to time. To refer back to a moment in time, phrases like \textit{{\ꞌ}i te hora era} ‘at that time’ are used.}

When preceded by a locative \isi{preposition}, \textit{ira} refers to a \textsc{location} which has been mentioned before: ‘that place, there’. In \REF{ex:4.233} \textit{ira} refers back to ‘home’ in the previous clause, in \REF{ex:4.234} to ‘his country’:

\ea\label{ex:4.233}
\gll I tu{\ꞌ}u haka{\ꞌ}ou era a Mako{\ꞌ}i ki te kona hare era, \textbf{{\ꞌ}i} \textbf{ira} a Paepae.  \\
\textsc{pfv} arrive again \textsc{dist} \textsc{prop} Mako’i to \textsc{art} place house \textsc{dist} at \textsc{ana} \textsc{prop} Paepae  \\

\glt 
‘When Mako’i arrived home again, Paepae was there.’ \textstyleExampleref{[R214.071]} 
\z

\ea\label{ex:4.234}
\gll He tu{\ꞌ}u ki tō{\ꞌ}ona kāiŋa ko Ma{\ꞌ}uŋa Terevaka. I tu{\ꞌ}u era \textbf{ki} \textbf{ira}...\\
\textsc{ntr} arrive to \textsc{poss.3sg.o} homeland \textsc{prom} Mount Terevaka \textsc{pfv} arrive \textsc{dist} to \textsc{ana}\\

\glt
‘He went to his own place, mount Terevaka. When he arrived there...’ \textstyleExampleref{[R314.159–160]}
\z

One of the contexts in which locational\is{Locational} \textit{ira} can be used, is in a \isi{relative clause}\is{Clause!relative} with locative relativisation (see (\ref{ex:11.100}–\ref{ex:11.101}) on p.~\pageref{ex:11.100}).

Preceded by other than locative prepositions, \textit{ira} refers to a \textsc{situation}, a state of affairs which has been mentioned in an earlier clause. This happens with \textit{mo ira}\is{ira (anaphor)} ‘therefore, for that purpose’, \textit{{\ꞌ}o ira} ‘because of that’,\footnote{\label{fn:227}\textit{{\ꞌ}O ira} (with the reason \isi{preposition} \textit{{\ꞌ}o}, \sectref{sec:4.7.2.2}) should not be confused with \textit{o ira} ‘of there’ (with possessive \textit{o}), in which \textit{ira} has a locational\is{Locational} sense:
\ea
\gll 
He māta{\ꞌ}ita{\ꞌ}i ararua i te ŋā mōai era \textbf{o} \textbf{ira}.\\
  \textsc{ntr} observe the\_two \textsc{acc} \textsc{art} \textsc{pl} statue \textsc{dist} of \textsc{ana}\\
  \glt 
  ‘The two of them admired the statue there (lit. the statue of there)’ \textstyleExampleref{[R478.044]}\z  } and the very common \textit{pē ira} ‘like that, thus’:

\ea\label{ex:4.235}
\gll \textbf{Mo} \textbf{ira} te puka nei i aŋa ai. \\
for \textsc{ana} \textsc{art} book \textsc{prox} \textsc{pfv} make \textsc{pvp} \\

\glt 
‘Therefore I have made this book.’ \textstyleExampleref{[R531.014]} 
\z

\ea\label{ex:4.236}
\gll He me{\ꞌ}e kore mo kai, \textbf{{\ꞌ}o} \textbf{ira} au e taŋi nei. \\
\textsc{pred} thing lack for eat because\_of \textsc{ana} \textsc{1sg} \textsc{ipfv} cry \textsc{prox} \\

\glt 
‘There is nothing to eat, therefore I am crying.’ \textstyleExampleref{[R349.013]} 
\z

\ea\label{ex:4.237}
\gll Te mahana te mahana e raŋi era \textbf{pē} \textbf{ira}. \\
\textsc{art} day \textsc{art} day \textsc{ipfv} call \textsc{dist} like \textsc{ana} \\

\glt 
‘Day after day he cried like that.’ \textstyleExampleref{[R213.003]} 
\z
\is{Locational|)}
\subsection{Demonstrative pronouns}\label{sec:4.6.6}

Demonstrative pronouns are relatively rare. In order to refer to a situation in general (‘this’, ‘that’), the dummy noun \textit{me{\ꞌ}e}\is{mee ‘thing’@me{\ꞌ}e ‘thing’} is often used:

\ea\label{ex:4.238}
\gll Me{\ꞌ}e rivariva rahi \textbf{te} \textbf{me{\ꞌ}e} \textbf{nei} mo te orara{\ꞌ}a o te mahiŋo o Rapa Nui. \\
thing good:\textsc{red} much \textsc{art} thing \textsc{prox} for \textsc{art} life of \textsc{art} people of Rapa Nui \\

\glt 
‘This (the practices just described) was something very good for the life of the people of Rapa Nui.’ \textstyleExampleref{[R231.314]} 
\z

The demonstratives \textit{nei}\is{nei (proximal)!demonstrative pronoun}, \textit{nā}\is{na (demonstrative)@nā (demonstrative)!demonstrative pronoun} and \textit{rā}\is{ra (demonstrative)@rā (demonstrative)!demonstrative pronoun} are also used pronominally, but only as subject of a classifying or identifying clause (\sectref{sec:9.2.1}–\ref{sec:9.2.2}). In these constructions, the demonstrative is a constituent by itself; unlike personal pronouns\is{Pronoun!personal}, it is never preceded by the proper article\is{a (proper article)} or \textit{ko}, or followed by modifying particles. The \isi{constituent order} is always predicate—subject. Two examples:

\ea\label{ex:4.239}
\gll He {\ꞌ}ariko \textbf{nei}. \\
\textsc{pred} bean \textsc{prox} \\

\glt 
‘These are beans’ \textstyleExampleref{[Notes]}
\z

\ea\label{ex:4.240}
\gll Ko Rusinta \textbf{rā} {\ꞌ}i te tapa {\ꞌ}uta. \\
\textsc{prom} Rusinta \textsc{dist} at \textsc{art} side inland \\

\glt
‘That is Rusinta on the inland side.’ \textstyleExampleref{[R411.074]} 
\z

Just like demonstratives\is{Demonstrative} in the \isi{noun phrase}, the demonstrative pronoun can be used either deictically\is{Deixis} (pointing at something in the non-linguistic context) or anaphorically\is{Anaphora} or cataphorically (pointing back or forward to something mentioned in the text).

Certain postnominal elements belonging to the predicate \isi{noun phrase} occur after the subject: genitives as in \REF{ex:4.241}, relative clauses\is{Clause!relative} as in \REF{ex:4.242}. 

\ea\label{ex:4.241}
\gll He toru e{\ꞌ}a iŋa atu \textbf{nei} o Tāpura Re{\ꞌ}o. \\
\textsc{pred} three go\_out \textsc{nmlz} away \textsc{prox} of Tapura Re’o \\

\glt 
‘This is the third issue of (the newspaper) Tapura Re’o.’ \textstyleExampleref{[R649.001]} 
\z

\ea\label{ex:4.242}
\gll Famiria hope{\ꞌ}a \textbf{rā} oho mai mai kampō, mai {\ꞌ}Anakena. \\
family last \textsc{dist} go hither from countryside from Anakena \\

\glt
‘That was the last family who came from the countryside, from Anakena.’ \textstyleExampleref{[R413.889]} 
\z

Even though the demonstratives\is{Demonstrative} in these examples may seem to be postnominal particles which are part of the predicate \isi{noun phrase}, in reality they are pronominal, i.e. constituents in their own right. This is shown by the following evidence:

%\setcounter{listWWviiiNumcxxiileveli}{0}
\begin{enumerate}
\item 
Postnominal demonstratives\is{Demonstrative} have the forms \textit{nei}, \textit{ena}, \textit{era}; the forms under consideration here are different: \textit{nei}\is{nei (proximal)!demonstrative pronoun}, \textit{nā}\is{na (demonstrative)@nā (demonstrative)!demonstrative pronoun}, \textit{rā}\is{ra (demonstrative)@rā (demonstrative)!demonstrative pronoun}.

\item 
While a \isi{noun phrase} may contain only one postnominal demonstrative\is{Demonstrative!postnominal}, the forms considered here may co-occur with a postnominal demonstrative\is{Demonstrative!postnominal}, as the following example shows:

\end{enumerate}

\ea\label{ex:4.243}
\gll {\ob}{\ꞌ}Aka \textbf{era}\,{\cb} \textbf{rā} {\ob}o te parasa era o mu{\ꞌ}a o te hare hāpī\,{\cb}. \\
{\db}anchor \textsc{dist} \textsc{dist} {\db}of \textsc{art} courtyard \textsc{dist} of front of \textsc{art} house learn \\

\glt
‘That is the anchor (which is) in the courtyard in front of the school.’ \textstyleExampleref{[R413.675]} 
\z

This means that the nominal predicate in (\ref{ex:4.241}–\ref{ex:4.243}) is split in two, and interrupted by the subject. Split predicates also occur with other pronominal subjects (\sectref{sec:9.2.5}).\is{Demonstrative}
\is{Demonstrative|)}
\section{Prepositions}\label{sec:4.7}
\is{Preposition|(}\subsection{Introduction}\label{sec:4.7.1}

Prepositions express a semantic relationship between a \isi{noun phrase} and the rest of the clause. Rapa Nui has a variety of prepositions, some of which (like \textit{{\ꞌ}i} and \textit{ki}) have a wide range of uses, while others are more narrowly defined. They also serve to mark case, especially the prepositions \textit{i} (direct object) and \textit{e} (\isi{agentive} subject).

Syntactically, prepositions are characterised by the fact that they are followed by a \isi{noun phrase}. When the \isi{preposition} is followed by a common \isi{noun phrase}, this \isi{noun phrase} must contain a \textit{t}{}-determiner (\sectref{sec:5.3.2.1}). Two prepositions show different behaviour, however:

\begin{itemize}
\item 
\textit{hai}\is{hai (instrumental prep.)} ‘with (instrumental)’ is not followed by a determiner (with a few exceptions, see \sectref{sec:4.7.9})\is{Demonstrative};

\item 
\textit{pa/pē}\is{pe ‘like’@pē ‘like’} ‘like’ (not to be confused with \textit{pe} ‘toward’) is followed by the predicate marker \textit{he} (\sectref{sec:5.3.4}).

\end{itemize}

With a \isi{proper noun} or pronoun \isi{complement}, prepositions ending in \textit{i} (with the exception of \textit{hai}) are followed by the proper article\is{a (proper article)} \textit{a}, while others are directly followed by the (pro)noun (\sectref{sec:5.13.2.1}). 

Most prepositions can be followed by locationals\is{Locational} (\sectref{sec:3.6.2.1}); locationals\is{Locational} immediately follow the \isi{preposition}, without a determiner.

These patterns are summarised in \tabref{tab:34}.

\begin{table}
\begin{tabularx}{\textwidth}{XL{19mm}L{19mm}L{19mm}L{18mm}} 
\lsptoprule
&  {1} &  {2} &  {3} &  {4}\\
& \textit{ki} ‘to’ & \textit{mo} ‘for’ & \textit{hai} ‘with’ & \textit{pē} ‘like’\\
\midrule
\textit{te hare} ‘the house’ & \textit{ki te hare} & \textit{mo te hare} & \textit{hai hare} & \textit{pē he hare}\\
\textit{Māria} ‘Maria’ & \textit{ki a Māria} & \textit{mo Māria} & \textit{hai Māria}\footnotemark{} & \textit{pē Māria}\\
\textit{rātou} ‘3 \textsc{pl}’ & \textit{ki a rātou} & \textit{mo rātou} & \textit{hai rātou} & \textit{pē rātou}\\
{\textit{roto} ‘inside’} & {\textit{ki roto}} & {\textit{mo roto}} & \textit{—} & {\textit{—}}\\
\lspbottomrule
\end{tabularx}
\caption{Preposition classes}
\label{tab:34}
\end{table}


\footnotetext{In fact, animate\is{Animacy} complements of \textit{hai} are rare. They are found e.g. in the Bible translation: \textit{hai Ietū} ‘with/by Jesus’, \textit{hai ia} ‘with/by him’.}

Group 1 includes \textit{i} ‘object marker’, \textit{i/{\ꞌ}i} ‘locative’, \textit{ki} ‘to’ and \textit{mai} ‘from’. Group 2 includes \textit{a} ‘by, along’, \textit{pe} ‘toward’, \textit{e} ‘agent marker’, \textit{{\ꞌ}o} ‘because of’, \textit{o/{\ꞌ}a} ‘possessive’, \textit{to/ta} ‘possessive’, \textit{mo/ma} ‘benefactive’, \textit{ko} ‘prominence marker’ and \textit{{\ꞌ}ai} ‘there in/at’. Group 3 only includes \textit{hai}, group 4 only includes \textit{pa}/\textit{pē}. 

\is{o (possessive prep.)}In the following subsections, prepositions are discussed individually, except the Agent marker \textit{e}, the accusative marker\is{i (accusative marker)} \textit{i} (\sectref{sec:8.2}–\ref{sec:8.4}), and the possessive prepositions \textit{o} and \textit{{\ꞌ}a} (\sectref{sec:6.2}–\ref{sec:6.2.4}). \sectref{sec:4.7.2}–\ref{sec:4.7.6} deal with prepositions which are primarily locative, such as \textit{{\ꞌ}i} and \textit{ki}. The causal \isi{preposition} \textit{{\ꞌ}o} will be discussed together with \textit{{\ꞌ}i} in \sectref{sec:4.7.2.2}, as the two are similar in function.

\sectref{sec:4.7.7}–\ref{sec:4.7.9} discuss prepositions with other than locative functions, such as benefactive and instrumental. \sectref{sec:4.7.10} discusses the rare \isi{preposition} \textit{{\ꞌ}ai}. Finally, \sectref{sec:4.7.11} deals with the prominence marker \textit{ko}, which is different in function from other prepositions, but which is nevertheless a \isi{preposition} syntactically.

\subsection{The \isi{preposition} \textit{{\ꞌ}i}/\textit{i} ‘in, at, on’}\label{sec:4.7.2}

\is{i ‘in, at’@{\ꞌ}i ‘in, at’|(}
The \isi{preposition} \textit{{\ꞌ}i/i} ‘in, at, on’ is a reflex of \is{Proto-Polynesian}PPN \textit{*{\ꞌ}i} (Pollex, see \citealt{GreenhillClark2011}) or \textit{*i} \citep[41]{Clark1976} – due to the instable character of glottals\is{Glottal plosive} in particles \citep[22]{Clark1976}, it is impossible to tell with \isi{certainty} if the \isi{preposition} had a glottal\is{Glottal plosive} in \is{Proto-Polynesian}PPN. 

In most Polynesian languages this \isi{preposition} has a wide range of functions.\footnote{\label{fn:229}\citet[428]{Chapin1978} mentions “the extreme polyfunctionality of Polynesian prepositions, and of \textit{i} in particular”.} In the accepted Rapa Nui \isi{orthography} (\sectref{sec:1.4.4}), this \isi{preposition} is written either \textit{{\ꞌ}i} and \textit{i}, depending on its function: certain uses of this \isi{preposition} are written with glottal\is{Glottal plosive}, others without. The inclusion or non-inclusion of glottals\is{Glottal plosive} in particles is largely based on whether the \isi{particle} occurs mainly at the start or in the middle of prosodic units (\sectref{sec:2.2.5}). This means that functions of \textit{i/{\ꞌ}i} which tend to occur phrase-initially are written with glottal\is{Glottal plosive}, while functions mainly occurring in the middle of phrases, or at the start of phrases prosodically connected to the preceding context, are written without glottal\is{Glottal plosive}. As a result, the \isi{preposition} in a locative sense is written \textit{{\ꞌ}i}, while the \isi{preposition} occurring after locationals\is{Locational} is \textit{i}.

In the following sections, the uses of \textit{i}/\textit{{\ꞌ}i} are discussed.  Because \textit{{\ꞌ}i/i} is largely used in a locative\is{Locational} or temporal sense, it is glossed ‘at’.
The causal use of \textit{{\ꞌ}i} is discussed in \sectref{sec:4.7.2.2}, together with the \isi{preposition} \textit{{\ꞌ}o}, which is similar in function.

\subsubsection[Locative {\ꞌ}i]{Locative \textit{{\ꞌ}i}}\label{sec:4.7.2.1}

\textit{{\ꞌ}I} expresses stationary location: ‘in, on, at’. In this sense it is often followed by locationals\is{Locational} (\sectref{sec:3.6.2}). Here are examples where it is directly followed by a \isi{noun phrase}: in (\ref{ex:4.244}–\ref{ex:4.245}) in a spatial sense, in \REF{ex:4.246} in a temporal sense.

\ea\label{ex:4.244}
\gll He noho \textbf{{\ꞌ}i} te hare o te huŋavai. \\
\textsc{ntr} stay at \textsc{art} house of \textsc{art} parent\_in\_law \\

\glt 
‘She stayed in the house of her in-laws.’ \textstyleExampleref{[Mtx-5-03.002]}
\z

\ea\label{ex:4.245}
\gll I poreko ena a koe {\ꞌ}i Haŋa Roa. \\
\textsc{pfv} born \textsc{med} \textsc{prop} \textsc{2sg} at Hanga Roa \\

\glt 
‘You were born in Hanga Roa.’ \textstyleExampleref{[R380.156]} 
\z

\ea\label{ex:4.246}
\gll \textbf{{\ꞌ}I} tū hora era te taŋata ta{\ꞌ}ato{\ꞌ}a ko ri{\ꞌ}ari{\ꞌ}a tahi {\ꞌ}ana. \\
at \textsc{dem} time \textsc{dist} \textsc{art} person all \textsc{prf} afraid all \textsc{cont} \\

\glt
‘At that moment all the people were afraid.’ \textstyleExampleref{[R210.152]} 
\z

Temporal \textit{{\ꞌ}i} may be followed by a nominalised verb\is{Verb!nominalised}, making the \textit{{\ꞌ}i}{}-marked constituent similar to a \isi{temporal clause}\is{Clause!temporal}.

\ea\label{ex:4.247}
\gll \textbf{{\ꞌ}I} \textbf{te} \textbf{kī} nō o Puakiva ki a Pea i tā{\ꞌ}ana vānaŋa,  kai haka mou e Pea.\\
at \textsc{art} say just of Puakiva to \textsc{prop} Pea \textsc{acc} \textsc{poss.3sg.a} word  \textsc{neg.pfv} \textsc{caus} silent \textsc{ag} Pea\\

\glt
‘When Puakiva said her words to Pea, Pea didn’t silence her.’ \textstyleExampleref{[R229.489]} 
\z

In comparatives, \textit{{\ꞌ}i} marks the quality with respect to which the comparison is made (\sectref{sec:3.5.2.1}).

\subsubsection[General{}-purpose i]{General-purpose \textit{i}}\label{sec:4.7.2.3}
\is{i (\isi{preposition})|(}
The \isi{preposition} \textit{i} serves as a general-purpose oblique marker. It is used to mark noun phrases which are in some way related to the action. Generally speaking, the \textit{i}{}-marked constituent expresses a participant with respect to whom the event takes place; this constituent can be characterised as the locus of the event.

In some cases this \isi{noun phrase} has a possessive sense:

\ea\label{ex:4.256}
\gll \textbf{I} \textbf{a} \textbf{ia} i topa ai te {\ꞌ}āua era o Vaihū.\\
at \textsc{prop} \textsc{3sg} \textsc{pfv} happen \textsc{pvp} \textsc{art} enclosure \textsc{dist} of Vaihu\\

\glt 
‘To him was assigned the field of Vaihu.’ \textstyleExampleref{[R250.052]} 
\z

\ea\label{ex:4.257}
\gll Ku riro mau {\ꞌ}ana ho{\ꞌ}i \textbf{i} \textbf{tū} \textbf{taŋata} \textbf{era} te rē.\\
\textsc{prf} become really \textsc{cont} indeed at \textsc{dem} man \textsc{dist} \textsc{art} victory\\

\glt
‘That man became the winner (lit. To that man became the victory).’ \textstyleExampleref{[R372.154]} 
\z

Possessive \textit{i} is also found in proprietary clauses\is{Clause!proprietary} (see (\ref{ex:9.88}–\ref{ex:9.89}) on p.~\pageref{ex:9.88}).

With adjectives, \textit{i} expresses the \isi{possessor}\is{Possession} of a certain quality, i.e. the entity where the quality is located. Examples of this are the \textit{ko te X} exclamative\is{Exclamative} construction (see (\ref{ex:10.82}–\ref{ex:10.83}) on p.~\pageref{ex:10.82}) and cases such as the following:

\ea\label{ex:4.258}
\gll Me{\ꞌ}e {\ꞌ}aroha \textbf{i} \textbf{tū} \textbf{nanue\_para} \textbf{era} ana ai ko rava{\ꞌ}a {\ꞌ}ana e te nu{\ꞌ}u  hī ika.\\
thing pity at \textsc{dem} kind\_of\_fish \textsc{dist} \textsc{irr} exist \textsc{prf} obtain \textsc{cont} \textsc{ag} \textsc{art} people  to\_fish fish\\

\glt
‘Poor \textit{nanue para} fish (lit. A pity \textit{i} that \textit{nanue para}) if it is caught by fishermen’ \textstyleExampleref{[R301.320]} 
\z

In other cases the sense of \textit{i} is hard to define more precisely; however, it is clear that the \textit{i}{}-marked NP is involved in the action in some way; the event takes place with respect to the participant mentioned.

\ea\label{ex:4.259}
\gll I tatau era, {\ꞌ}ina he tehe te ū \textbf{i} \textbf{a} \textbf{Te} \textbf{Manu}.\\
\textsc{pfv} milk \textsc{dist} \textsc{neg} \textsc{ntr} flow \textsc{art} milk at \textsc{prop} Te Manu\\

\glt 
‘When they milked (the cows), the milk didn’t flow to Te Manu (he couldn’t get the milk to flow).’ \textstyleExampleref{[R245.192]} 
\z

\ea\label{ex:4.260}
\gll E ko pau \textbf{i} \textbf{a} \textbf{koe} te kona mo rao o tu{\ꞌ}u va{\ꞌ}e. \\
\textsc{ipfv} \textsc{neg.ipfv} run\_out at \textsc{prop} \textsc{2sg} \textsc{art} place for cross\_over of \textsc{poss.2sg.o} foot \\

\glt 
‘There are many places where you can go (lit. The places to put your foot do not finish to you).’ \textstyleExampleref{[R315.071]} 
\z

Two other uses of \textit{i} are discussed elsewhere:

\begin{itemize}
\item 
After locationals\is{Locational}, \textit{i} is the most common \isi{preposition} introducing locative complements (e.g. \textit{{\ꞌ}i roto i} ‘inside’, see \sectref{sec:3.6.2.2}).

\item 
\is{i (\isi{preposition})!agent marker}\textit{I} marks \isi{agentive} phrases which are not an argument of the verb (\sectref{sec:8.6.4.7}). 

\end{itemize}
\is{i ‘in, at’@{\ꞌ}i ‘in, at’|)}
\is{i (\isi{preposition})|)}

\subsection{Causes and reasons: \textit{{\ꞌ}i} and \textit{{\ꞌ}o}} \label{sec:4.7.2.2}
\is{o ‘because of’@{\ꞌ}o ‘because of’|(}
Causes and reasons may be expressed by a verbal clause (\sectref{sec:11.6.4}). More commonly, however, they are expressed by a \isi{noun phrase} marked with either \textit{{\ꞌ}i} or \textit{{\ꞌ}o}. This \isi{noun phrase} often contains a nominalised verb\is{Verb!nominalised} or an adjective.

\textit{{\ꞌ}I} is used to express causes. These causes can be events or states as in (\ref{ex:4.248}–\ref{ex:4.249}), but also non-human entities as in (\ref{ex:4.250}–\ref{ex:4.251}). Cf. the discussion on \isi{agentive} \textit{i} in \sectref{sec:8.6.4.7}.

\ea\label{ex:4.248}
\gll ¡He mate ta{\ꞌ}a māhaki \textbf{{\ꞌ}i} \textbf{te} \textbf{maruaki}! \\
~\textsc{ntr} die \textsc{poss.2sg.a} companion at \textsc{art} hunger \\

\glt 
‘Your friend dies from hunger!’ \textstyleExampleref{[R245.142]} 
\z

\ea\label{ex:4.249}
\gll He viriviri a Torometi \textbf{{\ꞌ}i} \textbf{te} \textbf{kata}. \\
\textsc{ntr}\textsc{} roll:\textsc{red} \textsc{prop} Torometi at \textsc{art} laugh \\

\glt 
‘Torometi fell down from laughing.’ \textstyleExampleref{[R245.105]} 
\z

\ea\label{ex:4.250}
\gll He hati te ŋao o {\ꞌ}Oto {\ꞌ}Uta \textbf{{\ꞌ}i} \textbf{te} \textbf{pureva}. \\
\textsc{ntr} break \textsc{art} neck of Oto Uta at \textsc{art} rock \\

\glt 
‘The neck of (the statue) Oto Uta broke by/from the rock.’ \textstyleExampleref{[MsE-089.002]}
\z

\ea\label{ex:4.251}
\gll Ku ŋarepe {\ꞌ}ā te kahu \textbf{{\ꞌ}i} \textbf{te} \textbf{{\ꞌ}ua}. \\
\textsc{prf} wet \textsc{cont} \textsc{art} clothes at \textsc{art} rain \\

\glt 
‘The clothes got soaked by the rain.’ \textstyleExampleref{[Egt. lexicon]}
\z

Causes are also expressed with the \isi{preposition} \textit{{\ꞌ}o}, but there is a difference. \textit{{\ꞌ}I} is used in situations where cause and effect are closely linked, i.e. for direct causes which automatically lead to a certain effect. In \REF{ex:4.249}, for example, laughter is not only the cause of falling down, but also accompanies the falling down: ‘He fell while laughing, he fell down with laughter’. Similarly, in \REF{ex:4.248}, hunger it not only the cause of death, but hunger and death go together. In some cases – such as \REF{ex:4.249} – cause and effect are so closely linked, that the \textit{{\ꞌ}i}\nobreakdash-marked constituent is similar to a circumstantial clause. \textit{{\ꞌ}O}, on the other hand, is used in situations where cause and effect are less closely linked. Here are a few examples:

\ea\label{ex:4.252}
\gll He taŋi \textbf{{\ꞌ}o} te mate o Huri {\ꞌ}a Vai. \\
\textsc{ntr} cry because\_of \textsc{art} die of Huri a Vai \\

\glt 
‘He cried because Huri a Vai had died.’ \textstyleExampleref{[R304.104]} 
\z

\ea\label{ex:4.253}
\gll {\ꞌ}I tū hora era te tokerau me{\ꞌ}e hūhū, \textbf{{\ꞌ}o} ira kai hini i oti tahi rō ai tū hare era te vera. \\
at \textsc{dem} time \textsc{dist} \textsc{art} wind thing roar:\textsc{red} because\_of \textsc{ana} \textsc{neg.pfv} delay \textsc{pfv} finish all \textsc{emph} \textsc{pvp} \textsc{dem} house \textsc{dist} \textsc{art} burn \\

\glt
‘The wind roared at that time, therefore it wasn’t long before the whole house burned down.’ \textstyleExampleref{[R250.120]} 
\z

\textit{{\ꞌ}O} is often used to express reasons, i.e. situations where cause and effect are mediated by a volitional decision:

\ea\label{ex:4.254}
\gll Hora kai, {\ꞌ}ina he haraoa mā{\ꞌ}au \textbf{{\ꞌ}o} tu{\ꞌ}u toke i te haraoa  o te poki era.\\
time eat \textsc{neg} \textsc{pred} bread \textsc{ben.2sg.a} because\_of \textsc{poss.2sg.o} steal \textsc{acc} \textsc{art} bread  of \textsc{art} child \textsc{dist}\\

\glt 
‘At dinnertime, there is no bread for you, because you stole the bread of that child.’ \textstyleExampleref{[R245.048]} 
\z

\ea\label{ex:4.255}
\gll {\ꞌ}Ina pa{\ꞌ}i o māua kona mo noho. \textbf{{\ꞌ}O} ira au i iri   mai nei ki a koe.\\
\textsc{neg} in\_fact of \textsc{1du.excl} place for stay because\_of \textsc{ana} \textsc{1sg} \textsc{pfv} ascend   hither \textsc{prox} to \textsc{prop} \textsc{2sg}\\

\glt 
‘We don’t have a place to stay. Therefore I have come to you.’ \textstyleExampleref{[R229.210]}\textstyleExampleref{} 
\z
\is{o ‘because of’@{\ꞌ}o ‘because of’|)}

\subsection{The \isi{preposition} \textit{ki} ‘to’}\label{sec:4.7.3}
\is{ki (\isi{preposition})|(}

\textit{Ki} ({\textless} \is{Proto-Polynesian}PPN \textit{*ki}) indicates movement in the direction of a goal. It is often found with verbs of movement:

\ea\label{ex:4.261}
\gll He hoki mai ararua \textbf{ki} te kona hare era. \\
\textsc{ntr} return hither the\_two to \textsc{art} place house \textsc{dist} \\

\glt 
‘The two returned home.’ \textstyleExampleref{[R166.007]} 
\z

\ea\label{ex:4.262}
\gll E tahi mahana he turu a Tiare \textbf{ki} te hare hāpī. \\
\textsc{num} one day \textsc{ntr} go\_down \textsc{prop} Tiare to \textsc{art} house learn \\

\glt
‘One day Tiare went down to school.’ \textstyleExampleref{[R170.001]} 
\z

\textit{Ki} is often followed by a locational\is{Locational} indicating the direction in which the movement takes place (see e.g. \REF{ex:3.133} on p.~\pageref{ex:3.133}). 

\textit{Ki} is used when the referent makes a movement touching the endpoint, for example with the verb \textit{tu{\ꞌ}u} ‘arrive’:

\ea\label{ex:4.263}
\gll He oho a Teke, he tu{\ꞌ}u \textbf{ki} te hare o Mā{\ꞌ}eha. \\
\textsc{ntr} go \textsc{prop} Teke \textsc{ntr} arrive to \textsc{art} house of Ma’eha \\

\glt 
‘Teke went and arrived at Ma’eha’s house.’ \textstyleExampleref{[MsE-059.005]}
\z

\ea\label{ex:4.264}
\gll ...ko te kutakuta o te vaikava e hāpaki era \textbf{ki} te {\ꞌ}ōpata.\\
~~~~\textsc{prf} \textsc{art} foam of \textsc{art} ocean \textsc{ipfv} slap \textsc{dist} to \textsc{art} cliff\\

\glt 
‘...the foaming water of the sea was slapping against the cliffs.’ \textstyleExampleref{[R408.105]} 
\z

\textit{Ki} is used when a referent is oriented towards an object, even when no movement towards this object is involved: in \REF{ex:4.265} the tree bark is exposed to the sun, but not moved towards the sun.

\ea\label{ex:4.265}
\gll Ki oti he to{\ꞌ}o mai, he tauaki \textbf{ki} te ra{\ꞌ}ā, he haka pakapaka. \\
when finish \textsc{ntr} take hither \textsc{ntr} dry\_in\_sun to \textsc{art} sun \textsc{ntr} \textsc{caus} dry:\textsc{red} \\

\glt 
‘After that they take (the tree bark) and put it in the sun to dry.’ \textstyleExampleref{[Ley-5-04.009]}
\z

In a temporal sense, \textit{ki} indicates the end of a stretch of time: ‘until’, as in \REF{ex:4.266}.\footnote{\label{fn:230}The preverbal marker \textit{ki} has the same function (see examples (\ref{ex:11.196}–\ref{ex:11.197}) on p.~\pageref{ex:11.196}).} In this sense, \textit{ki} may be reinforced by \textit{{\ꞌ}ātā}\is{ata ‘until’@{\ꞌ}ātā ‘until’} ({\textless} Sp. \textit{hasta} ‘until’) as in \REF{ex:4.267}.

\ea\label{ex:4.266}
\gll {\ꞌ}O ira e ko hakarē a au i a koe \textbf{ki} tō{\ꞌ}oku hope{\ꞌ}ara{\ꞌ}a.\\
because\_of \textsc{ana} \textsc{ipfv} \textsc{neg.ipfv} leave \textsc{prop} \textsc{1sg} \textsc{acc} \textsc{art} \textsc{2sg} to \textsc{poss.1sg.o} end\\

\glt 
‘Therefore I will not leave you till the end of my days.’ \textstyleExampleref{[R474.010]} 
\z

\ea\label{ex:4.267}
\gll Mai rā hora \textbf{{\ꞌ}ātā} \textbf{ki} te hora nei kai e{\ꞌ}a haka{\ꞌ}ou e tahi Rapa Nui.\\
from \textsc{dist} time until to \textsc{art} time \textsc{prox} \textsc{neg.pfv} go\_out again \textsc{num} one Rapa Nui\\

\glt 
‘From that time until now, not one Rapa Nui left (the island) anymore.’ \textstyleExampleref{[R303.211]} 
\z

\textit{Ki} has a wide range of metaphorical extensions; it is the default \isi{preposition} for semantic roles like Recipient, Beneficiary and Goal (\sectref{sec:8.8.2}) as in \REF{ex:4.268}. \textit{Ki} is used to indicate an opinion or point of view: ‘according to’, as in \REF{ex:4.269}. In addition, \textit{ki} marks the object of middle verbs\is{Verb!middle} (\sectref{sec:8.6.4.2}).

\ea\label{ex:4.268}
\gll He va{\ꞌ}ai a nua i te kai ki a koro. \\
\textsc{ntr} give \textsc{prop} Mum \textsc{acc} \textsc{art} food to \textsc{prop} Dad \\

\glt 
‘Mum gave the food to Dad.’ \textstyleExampleref{[R236.078]} 
\z

\ea\label{ex:4.269}
\gll \textbf{Ki} te kī o te nu{\ꞌ}u te repa nei ko Ure {\ꞌ}a Vai {\ꞌ}a Nuhe  he kope nehenehe.\\
to \textsc{art} say of \textsc{art} people \textsc{art} young\_man \textsc{prox} \textsc{prom} Ure a Vai a Nuhe  \textsc{pred} person beautiful\\

\glt
‘According to (lit. to the say of) the people, young Ure a Vai a Nuhe was a handsome man.’ \textstyleExampleref{[R532-07.006]}
\z

\is{ki (\isi{preposition})}Finally, in comparative\is{Comparative} constructions, \textit{ki} marks the standard of comparison (\sectref{sec:3.5.2.1}). 
\is{ki (\isi{preposition})|)}
\subsection{The \isi{preposition} \textit{mai} ‘from’}\label{sec:4.7.4}
\is{mai ‘from’|(}

\textit{Mai} indicates a spatial or temporal point of origin:

\ea\label{ex:4.270}
\gll \textbf{Mai} Haŋa Roa i iri ai ki {\ꞌ}Ōroŋo. \\
from Hanga Roa \textsc{pfv} ascend \textsc{pvp} to Orongo \\

\glt 
‘From Hanga Roa they went up to Orongo.’ \textstyleExampleref{[Ley-2-02.054]}
\z

\ea\label{ex:4.271}
\gll \textbf{Mai} te mahana nei {\ꞌ}ina a nua kai haka uŋa haka{\ꞌ}ou ki a Tiare. \\
from \textsc{art} day \textsc{prox} \textsc{neg} \textsc{prop} Mum \textsc{neg.pfv} \textsc{caus} send again to \textsc{prop} Tiare \\

\glt 
‘From this day on, Mum didn’t send Tiare anymore.’ \textstyleExampleref{[R179.046]} 
\z

When \textit{mai} is followed by a \isi{proper noun} or pronoun, the proper article\is{a (proper article)} \textit{a} is used (as with \textit{ki} and \textit{i/{\ꞌ}i}); however, the \isi{preposition} \textit{i} must be added between \textit{mai} and the proper article\is{a (proper article)}, as shown in the following example:

\ea\label{ex:4.272}
\gll Ararua nō pā{\ꞌ}eŋa e tu{\ꞌ}u mai era, mai Tahiti {\ꞌ}e \textbf{mai} \textbf{i} \textbf{a} Tire. \\
the\_two just side \textsc{ipfv} arrive hither \textsc{dist} from Tahiti and from at \textsc{prop} Chile \\

\glt 
‘The two sides came, from Tahiti and from Chile.’ \textstyleExampleref{[R539-2.221]}
\z

\is{mai ‘from’}The use of \textit{mai} as a preverbal marker in subordinate clauses is discussed in \sectref{sec:11.5.5}.
\is{mai ‘from’|)}
\subsection{The \isi{preposition} \textit{pe} ‘toward’}\label{sec:4.7.5}

\is{pe ‘towards’|(}
The \isi{preposition} \textit{pe} indicates a general direction or orientation.\footnote{\label{fn:231}This \isi{preposition} does not occur in other languages. It may be derived from \textit{pē}\is{pe ‘like’@pē ‘like’} ‘like’, but its syntax is different: while \textit{pē} ‘like’ is followed by the predicate marker \textit{he}, \textit{pe} ‘towards’ is followed by a \textit{t}{}-determiner, like most prepositions.} Its function is similar to \textit{ki}, but it is not goal-oriented: to go \textit{ki X} implies that one intends to arrive at X; \textit{pe X} does not have this implication.

\ea\label{ex:4.273}
\gll He iri te nuahine, he oho \textbf{pe} {\ꞌ}Ōroŋo. \\
\textsc{ntr} ascend \textsc{art} old\_woman \textsc{ntr} go toward Orongo \\

\glt 
‘The old woman went up, she went towards Orongo.’ \textstyleExampleref{[Ley-8-52.028]}
\z

\ea\label{ex:4.274}
\gll I {\ꞌ}ata oho atu era \textbf{pe} haho o te vaikava... \\
\textsc{pfv} more go away \textsc{dist} toward outside of \textsc{art} ocean \\

\glt 
‘When she went further outside toward the open sea...’ \textstyleExampleref{[R338.006]} 
\z

\ea\label{ex:4.275}
\gll E take{\ꞌ}a mai era e au mai te pena nō \textbf{pe} ruŋa. \\
\textsc{ipfv} see hither \textsc{dist} \textsc{ag} \textsc{1sg} from \textsc{art} belt just toward above \\

\glt
‘I saw him from the belt upwards.’ \textstyleExampleref{[R106.034]} 
\z

In a temporal sense, \textit{pe} is used with the locational\is{Locational} \textit{mu{\ꞌ}a}\is{mua ‘front’@mu{\ꞌ}a ‘front’} ‘front’ to refer to a period of time in the future, or posterior\is{Posteriority} to a time of reference (see (\ref{ex:3.153}–\ref{ex:3.154}) on p.~\pageref{ex:3.153}). 

\is{pe ‘towards’}\textit{Pe} may also indicate an approximate location as in \REF{ex:4.276}, or an approximate time as in \REF{ex:4.277}:

\ea\label{ex:4.276}
\gll Te me{\ꞌ}e hau mau o te rahi he mā{\ꞌ}ea, \textbf{pe} \textbf{ruŋa} \textbf{pe} \textbf{raro} o te ma{\ꞌ}uŋa.\\
\textsc{art} thing exceed really of \textsc{art} much \textsc{pred} stone toward above toward below of \textsc{art} mountain\\

\glt 
‘What’s really abundant (on the island) are stones, up and down the mountain.’ \textstyleExampleref{[R350.011]} 
\z

\ea\label{ex:4.277}
\gll I ahiahi era \textbf{pe} \textbf{te} hora toru, he turu mai tū Tamy era.\\
\textsc{pfv} afternoon \textsc{dist} toward \textsc{art} time three \textsc{ntr} go\_down hither \textsc{dem} Tamy \textsc{dist}\\

\glt 
‘Around three o’clock in the afternoon, Tamy went down.’ \textstyleExampleref{[R315.273]} 
\z
\is{pe ‘towards’|)}
\subsection{The \isi{preposition} \textit{a} ‘along; towards’}\label{sec:4.7.6}
\is{a (\isi{preposition})|(}

Unlike other locative prepositions, \textit{a} is used mainly in a spatial sense, rarely in temporal expressions.\footnote{\label{fn:232}The \isi{preposition} \textit{a} (different from possessive \textit{a} or \textit{{\ꞌ}a}) occurs in a few languages as a locative \isi{preposition} (Pollex, see \citealt{GreenhillClark2011}) and is reconstructed as \is{Eastern Polynesian}PEP *\textit{aa}.}  

The \isi{preposition} \textit{a} may indicate a spatial relation which is neither stationary \mbox{(\textit{{\ꞌ}i}),} nor involves a movement towards (\textit{ki}) or away from (\textit{mai}) an object. It is used when one object moves with respect to another object in some other way: by, along or via the other object.

\ea\label{ex:4.278}
\gll I hoki mai era ki te hare \textbf{a} te ara kē. \\
\textsc{pfv} return hither \textsc{dist} to \textsc{art} house by \textsc{art} road different \\

\glt 
‘He returned home by another road.’ \textstyleExampleref{[R408.038]} 
\z

\ea\label{ex:4.279}
\gll Ku oho {\ꞌ}ā Taŋaroa ki te kāiŋa \textbf{a} roto a te vaikava. \\
\textsc{prf} go \textsc{cont} Tangaroa to \textsc{art} homeland by inside by \textsc{art} ocean \\

\glt
‘Tangaroa has gone to the island by way of the sea.’ \textstyleExampleref{[Ley-1-06.007]}
\z

It is also used when a part of something is singled out as the location where an event takes place. 

\ea\label{ex:4.280}
\gll He ha{\ꞌ}i i a koro ararua ko nua \textbf{a} te ŋao, he hoŋihoŋi \textbf{a} te {\ꞌ}āriŋa.\\
\textsc{ntr} embrace \textsc{acc} \textsc{prop} Dad the\_two \textsc{prom} Mum by \textsc{art} neck \textsc{ntr} kiss:\textsc{red} by \textsc{art} face\\

\glt 
‘She embraced Dad and Mum by the neck and kissed them on the face.’ \textstyleExampleref{[R210.012]} 
\z

\ea\label{ex:4.281}
\gll E hā taura: \textbf{a} mu{\ꞌ}a, \textbf{a} tu{\ꞌ}a, \textbf{a} te mata{\ꞌ}u, \textbf{a} te maui.\\
\textsc{num} four rope by front by back by \textsc{art} right by \textsc{art} left\\

\glt 
‘There are four ropes (tied to the statue): at the front, at the back, to the right, to the left.’ \textstyleExampleref{[Ley-5-29.010–011]}
\z

\textit{A} often indicates a general direction; this direction is expressed by a locational\is{Locational} (such as \textit{mu{\ꞌ}a} in \REF{ex:4.282}), but there is no second referent involved with respect to which this movement takes place. 

\ea\label{ex:4.282}
\gll He oho, he ao \textbf{a} mu{\ꞌ}a, he pū \textbf{a} mu{\ꞌ}a. \\
\textsc{ntr} go \textsc{ntr} rush by front \textsc{ntr} come by front \\

\glt 
‘They went, they came forward, rushed forward.’ \textstyleExampleref{[Ley-4-05.012]}
\z

\ea\label{ex:4.283}
\gll He take{\ꞌ}a e Tahoŋa he rere \textbf{a} ruŋa {\ꞌ}i te koa. \\
\textsc{ntr} see \textsc{ag} Tahonga \textsc{ntr} jump by above at \textsc{art} happy \\

\glt
‘When Tahonga saw this, he jumped up from joy.’ \textstyleExampleref{[R301.210]} 
\z

\textit{A} indicating a direction is similar to \textit{pe}\is{pe ‘towards’} ‘toward’ (\sectref{sec:4.7.5} above.)\textstyleExampleref{} A difference between the two is, that \textit{a} is far more common with locationals\is{Locational} than \textit{pe}. Another difference is, that some expressions with \textit{pe} have a temporal rather than a spatial sense, whereas \textit{a} is usually spatial.

\is{a (\isi{preposition})}Certain combinations of \textit{a} followed by a locational\is{Locational} have a lexicalised meaning: \textit{a raro}, \textit{a vāeŋa} (\sectref{sec:3.6.2.3}) and \textit{a tu{\ꞌ}a} (\sectref{sec:3.6.2.4}).
\is{a (\isi{preposition})|)}
\subsection{The benefactive prepositions \textit{mo} and \textit{mā}}\label{sec:4.7.7}
\is{Benefactive \isi{preposition}|(}
\is{ma (benefactive prep.)@mā (benefactive prep.)|(}\is{mo (benefactive prep.)|(}
The benefactive prepositions \textit{mo} and \textit{mā} express benefactive relations in a broad sense; they are used in situations where an event or object is destined for or aimed at the participant. This pair of prepositions displays the \is{Possession!o/a distinction}\textit{o}/\textit{a} distinction between two classes of possessives (\sectref{sec:6.3.2}). This distinction is only made with proper nouns\is{Noun!proper} and singular pronouns: with proper nouns\is{Noun!proper} either \textit{mā} or \textit{mo} is used; singular benefactive pronouns\is{Pronoun!benefactive} start with \textit{mā{\ꞌ}a-} or \textit{mō{\ꞌ}o-} (\sectref{sec:4.2.3}). With common nouns\is{Noun!common} and with plural pronouns, \textit{mo} is used in all situations. 

Regarding the etymology of these prepositions, the \is{Proto-Polynesian}PPN forms are \textit{*mo{\ꞌ}o}, \mbox{\textit{*ma{\ꞌ}a}}.\footnote{\label{fn:233}The original form of both particles, with glottal\is{Glottal plosive}, appears in other languages that preserved the \is{Proto-Polynesian}PPN glottal\is{Glottal plosive}: \ili{East Uvean}, \ili{Rennell} and \ili{Tongan} (Pollex, see \citealt{GreenhillClark2011}).} In Rapa Nui the glottal\is{Glottal plosive} is still present in the singular pronouns \textit{mā{\ꞌ}aku} etc.; the lengthening\is{Vowel!lengthening} in these forms is the result of a general tendency to lengthen the first vowel of three-\isi{syllable}\is{Syllable} words. In the prepositions as separate words, the glottal\is{Glottal plosive} has disappeared.\footnote{\label{fn:234}See \citet{Wilson1985} on the loss of the glottal\is{Glottal plosive} in \textit{t}{}-possessives\is{Pronoun!possessive!t-class} and benefactives. For benefactives, he uses the term \textit{irrealis}\is{Irrealis}.}

A benefactive relation is in fact a possessive relationship in which a \isi{possessee} is destined/intended for the \isi{possessor}\is{Possession}.\footnote{\label{fn:235}In Polynesian linguistics, these forms are sometimes characterised as “irrealis\is{Irrealis} possessives” (see e.g. \citealt{Clark2000Possessive}; \citealt[48]{Wilson1982}): they indicate not-yet realised \isi{possession}, in contrast to the “realis\is{Realis} possessives” starting with \textit{n-} or Ø (see Footnote \ref{fn:290} on p.~\pageref{fn:290}). The likely origin of the \textit{m}{}-forms is an irrealis\is{Irrealis} marker \textit{m-} \citep[115]{Clark1976}.} Whether \textit{mo} or \textit{mā} is used, depends on the relationship between the prospective \isi{possessor}\is{Possession} and \isi{possessee}, as discussed in \sectref{sec:6.3.2}: \textit{mā} is used when the \isi{possessor}\is{Possession} has control, authority or responsibility over the \isi{possessee}, \textit{mo} in all other cases. Thus, \textit{mā} is used for possessions over which the \isi{possessor}\is{Possession} has control, as in \REF{ex:4.284}. \textit{Mo} is used for means of transport as in \REF{ex:4.285}.

\ea\label{ex:4.284}
\gll He aŋa e tō{\ꞌ}ona matu{\ꞌ}a vahine i te manu parau \textbf{mā{\ꞌ}ana}.\\
 \textsc{ntr} make \textsc{ag} \textsc{poss.3sg.o} parent female \textsc{acc} \textsc{art} bird paper \textsc{ben.3sg.a}\\

\glt 
‘His mother made a paper bird for him.’ \textstyleExampleref{[R476.002]}\is{Possession!o/a distinction} 
\z

\ea\label{ex:4.285}
\gll He pu{\ꞌ}a i te hoi e tahi \textbf{mō{\ꞌ}ona}, e tahi mo te matu{\ꞌ}a.\\
\textsc{ntr} cover \textsc{acc} \textsc{art} horse \textsc{num} one \textsc{ben.3sg.o} \textsc{num} one for \textsc{art} parent\\

\glt 
‘He saddled one horse for himself, one for the priest.’ \textstyleExampleref{[R167.001]} 
\z

With certain verbs, possessive \textit{mo/mā} may express a Goal or Recipient, indicating that the object of the verb is destined for this participant; this is discussed in \sectref{sec:8.8.2}. 

Apart from the uses discussed so far, both \textit{mo} and \textit{mā} have uses of their own. \textit{Mo} may indicate the person towards whom an action or attitude is directed. This happens for example with the verbs \textit{riri} ‘be angry’ and \textit{{\ꞌ}aroha} ‘be sorry’:

\ea\label{ex:4.286}
\gll {\ꞌ}Ina koe ko riri \textbf{mō{\ꞌ}oku}, e nua ē. \\
\textsc{neg} \textsc{2sg} \textsc{neg.ipfv} angry \textsc{ben.1sg.o} \textsc{voc} Mum \textsc{voc} \\

\glt 
‘Don’t be angry with me, Mum.’ \textstyleExampleref{[R229.497]} 
\z

\ea\label{ex:4.287}
\gll He {\ꞌ}aroha a Vai Ora \textbf{mo} \textbf{Tahoŋa}. \\
\textsc{ntr} compassion \textsc{prop} Vai Ora for Tahonga \\

\glt
‘Vai Ora felt sorry for Tahonga.’ \textstyleExampleref{[R301.249]} 
\z

\is{Benefactive preposition}\textit{Mo} may also indicate a participant from whose perspective the event expressed in the clause is true: ‘for X, as far as X is concerned’.\footnote{\label{fn:236}Cf. the use of \textit{ki} to express a point of view (\sectref{sec:4.7.3}).} This use may have been influenced by \ili{Spanish} \textit{para}. In \REF{ex:4.288}, the things described in the preceding context are news, not necessary for everyone, but for the person mentioned: as far as he is concerned, they are news. In \REF{ex:4.289}, the clause expresses a point of view which is true for the person expressed with \textit{mo}:

\ea\label{ex:4.288}
\gll {\ꞌ}E te ŋā me{\ꞌ}e ta{\ꞌ}ato{\ꞌ}a nei he parau {\ꞌ}āpī \textbf{mō{\ꞌ}ona}.\\
and \textsc{art} \textsc{pl} thing all \textsc{prox} \textsc{pred} word new \textsc{ben.3sg.o}\\

\glt 
‘And all of this was news for him.’ \textstyleExampleref{[R363.055]} 
\z

\ea\label{ex:4.289}
\gll \textbf{Mō{\ꞌ}oku} {\ꞌ}ina he {\ꞌ}ati te noho mai o rāua {\ꞌ}i te kona era.\\
\textsc{ben.1sg.o} \textsc{neg} \textsc{pred} problem \textsc{art} stay hither of \textsc{3pl} at \textsc{art} place \textsc{dist}\\

\glt
‘For me (as far as I am concerned), it is no problem if they live there.’ \textstyleExampleref{[R647.163]} 
\z

\textit{Mā}\is{ma (benefactive prep.)@mā (benefactive prep.)} marks the Agent in the \isi{imperfective} actor-emphatic\is{Actor-emphatic construction} construction (\sectref{sec:8.6.3}):

\ea\label{ex:4.290}
\gll Mā{\ꞌ}aku {\ꞌ}ā a koe e hāpa{\ꞌ}o atu. \\
\textsc{ben.1sg.a} \textsc{ident} \textsc{prop} \textsc{2sg} \textsc{ipfv} care\_for away \\

\glt 
‘I will take care of you myself.’ \textstyleExampleref{[R310.067]} 
\z
\is{Benefactive \isi{preposition}|)}
\is{ma (benefactive prep.)@mā (benefactive prep.)|)}\is{mo (benefactive prep.)|)}
\subsection{The \isi{preposition} \textit{pa/pē} ‘like’}\label{sec:4.7.8}
\is{pe ‘like’@pē ‘like’|(}
\is{pa ‘like’}
\textit{Pē} is an \isi{equative}\is{Equative} \isi{preposition}: it serves to compare two entities, expressing that one resembles the other.\footnote{\label{fn:237}\textit{Pē} ({\textless} \is{Proto-Polynesian}PPN *\textit{pee} ‘like’) occurs in \ili{Hawaiian} and \ili{Māori}, but only or mainly as a bound root, followed by a demonstrative (\textit{pēnei})\is{Demonstrative}. It is more common in non-EP languages.} Equative constructions are discussed in \sectref{sec:3.5.2.3}; in this section other syntactic and semantic particularities of \textit{pē} will be discussed.

First of all, \textit{pē} is usually followed by the predicate marker \textit{he}\is{he (nominal predicate marker)},\footnote{\label{fn:238}Interestingly, the same is true for the \isi{preposition} \textit{me} ‘like’ in \ili{Hawaiian}, \ili{Marquesan} and \ili{Māori} (\sectref{sec:5.3.3}).}  not only when the compared entity is generic as in \REF{ex:4.291}, but also when it is a single, identifiable\is{Identifiability} entity as in \REF{ex:4.292}:

\ea\label{ex:4.291}
\gll He u{\ꞌ}i atu a Eva ko te me{\ꞌ}e \textbf{pē} \textbf{he} tiare {\ꞌ}ā ka {\ꞌ}ī.\\
\textsc{ntr} look away \textsc{prop} Eva \textsc{prom} \textsc{art} thing like \textsc{pred} flower \textsc{ident} \textsc{cntg} full\\

\glt 
‘Eva saw something like flowers, in great numbers\textstyleExampleref{.’ [R210.193]} 
\z

\ea\label{ex:4.292}
\gll \textbf{Pē} \textbf{he} korohu{\ꞌ}a era ko Iovani {\ꞌ}Iti{\ꞌ}iti te {\ꞌ}āriŋa.\\
like \textsc{pred} old\_man \textsc{dist} \textsc{prom} Iovani Iti’iti \textsc{art} face\\

\glt
‘His face (looks) like the old man Iovani Iti’iti.’ \textstyleExampleref{[R416.1180]}
\z

Before \textit{he}, \textit{pē} is often dissimilated to \textit{pa}. The choice between \textit{pē} and \textit{pa} is free; certain speakers favour one over the other.

\ea\label{ex:4.293}
\gll {\ꞌ}Arero nei \textbf{pa} he {\ꞌ}arero rapa nui {\ꞌ}ā. \\
tongue \textsc{prox} like \textsc{pred} tongue Rapa Nui \textsc{ident} \\

\glt
‘This language is like the Rapa Nui language.’\textstyleExampleref{ [R231.272]} 
\z

Occasionally \textit{pē} is followed by a \textit{t-}determiner as in \REF{ex:4.294}, or a \isi{proper noun} or pronoun as in \REF{ex:4.295}:

\ea\label{ex:4.294}
\gll ¿\textbf{Pē} \textbf{tū} huru {\ꞌ}ā te kī iŋa o te ŋā vānaŋa nei?\\
~like \textsc{dem} manner \textsc{ident} \textsc{art} say \textsc{nmlz} of \textsc{art} \textsc{pl} word \textsc{prox}\\

\glt 
‘Are these words pronounced the same way (lit. is the saying like that [same] way)?’ \textstyleExampleref{[R615.231]} 
\z

\ea\label{ex:4.295}
\gll \textbf{Pē} \textbf{ia} {\ꞌ}ā te huru. \\
like \textsc{3sg} \textsc{ident} \textsc{art} manner  \\

\glt
‘He looks like him.’ \textstyleExampleref{[R415.886]} 
\z

As most of the examples above show, the comparison may be reinforced by the identity \isi{particle} \textit{{\ꞌ}ā} (\sectref{sec:5.9}).

In modern Rapa Nui, \textit{pē} also expresses the category to which someone belongs. In \REF{ex:4.296} below, \textit{pē he {\ꞌ}ōtare} does not mean that the speaker resembles an orphan, but that he is an orphan. This usage may be influenced by \ili{Spanish} \textit{como}.

\ea\label{ex:4.296}
\gll {\ꞌ}Ina ō{\ꞌ}oku matu{\ꞌ}a, {\ꞌ}o ira a au e noho nei \textbf{pē} he {\ꞌ}ōtare. \\
\textsc{neg} \textsc{poss.1sg.o} parent because\_of \textsc{ana} \textsc{prop} \textsc{1sg} \textsc{ipfv} stay \textsc{prox} like \textsc{pred} orphan \\

\glt 
‘I don’t have parents, therefore I live as an orphan.’ \textstyleExampleref{[R214.013]} 
\z
\is{pe ‘like’@pē ‘like’|)}
\subsection{The instrumental \isi{preposition} \textit{hai}}\label{sec:4.7.9}
\is{hai (instrumental prep.)|(}
\textit{Hai}\footnote{\label{fn:239}This \isi{preposition} is not found in any other language. It may have developed from \is{Proto-Polynesian}PPN \textit{*fai}, which occurs in several languages as a verb or \isi{prefix} meaning ‘have, possess’. Reflexes of \is{Proto-Polynesian}PPN \textit{*fai} occur in many non-EP languages; the only \is{Eastern Polynesian}EP language in which it occurs, is \ili{Māori} (Pollex, see \citealt{GreenhillClark2011}). The fact that Rapa Nui \textit{hai} is followed by a bare noun suggests that it originated from a \isi{prefix} \textit{*fai-}  (itself related to the root \textit{*fai} just mentioned) rather than a full word. As a \isi{prefix}, it occurs for example in \ili{Nukuoro}, where \textit{hai-} is – among other things – prefixed to nouns to form derived verbs: \textit{hai hegau} ‘do work’ = ‘to work’; \textit{hai bodu} ‘do spouse’ = ‘to marry’ (cf. \citealt[628]{CarrollSoulik1973}). It would be a relatively small step for such a \isi{prefix} to develop into a \isi{preposition} taking a bare noun \isi{complement}.} is an instrumental \isi{preposition}, indicating the means or tool with which something is done: ‘with, using, by means of’:

\ea\label{ex:4.297}
\gll He pu{\ꞌ}apu{\ꞌ}a \textbf{hai} pāoa; he mate.\\
\textsc{ntr} beat:\textsc{red} \textsc{ins} club \textsc{ntr} die\\

\glt
‘They beat her with a club and she died.’ \textstyleExampleref{[Egt-01.082]}
\z

As discussed in \sectref{sec:4.7.1}, \textit{hai} is not followed by a determiner, but by a bare noun. This correlates with the meaning of \textit{hai}, which tends to occur with non-specific\is{Specific reference!non-specific reference} referents as in \REF{ex:4.297} above. Occasionally, however, \textit{hai} is followed by pronouns or proper nouns\is{Noun!proper} as in \REF{ex:4.298}, or by definite nouns (preceded by a demonstrative) as in \REF{ex:4.299}:

\ea\label{ex:4.298}
\gll \textbf{Hai} Eugenio {\ꞌ}i te pū{\ꞌ}oko e aŋa era ananake.\\
\textsc{ins} Eugenio at \textsc{art} head \textsc{ipfv} work \textsc{dist} together\\

\glt 
‘With Eugenio at the head they worked together.’ \textstyleExampleref{[R231.307]} 
\z

\ea\label{ex:4.299}
\gll E puru rō {\ꞌ}ā te {\ꞌ}āriŋa ararua \textbf{hai} tū paratoa era o rāua. \\
\textsc{ipfv} close \textsc{emph} \textsc{cont} \textsc{art} face the\_two \textsc{ins} \textsc{dem} coat \textsc{dist} of \textsc{3pl} \\

\glt 
‘The two covered their faces with their coat.’ \textstyleExampleref{[R215.038]} 
\z

The semantic range of \textit{hai} is large. It may indicate the instrument or material with which an action is done, as in \REF{ex:4.297} above and the following examples:

\ea\label{ex:4.300}
\gll {\ꞌ}Ina he ruku \textbf{hai} raperape, ni \textbf{hai} haŋuhaŋu.\\
\textsc{neg} \textsc{ntr} dive \textsc{ins} swim\_fin nor \textsc{ins} breathe:\textsc{red}\\

\glt 
‘They didn’t dive with swimming fins or with snorkels.’ \textstyleExampleref{[R360.004]} 
\z

\ea\label{ex:4.301}
\gll E paru rō {\ꞌ}ā i te rāua hakari \textbf{hai} kī{\ꞌ}ea. \\
\textsc{ipfv} paint \textsc{emph} \textsc{cont} \textsc{acc} \textsc{art} \textsc{3pl} body \textsc{ins} red\_earth \\

\glt
‘They painted their bodies with red earth.’ \textstyleExampleref{[R231.095]} 
\z

\textit{Hai} may mark various kinds of noun phrases which are in some way instrumental to the action, such as the price paid as in \REF{ex:4.302}, or the language spoken as in \REF{ex:4.303}. 

\ea\label{ex:4.302}
\gll E ko ho{\ꞌ}o atu ki a koe \textbf{hai} moni tire, ni \textbf{hai} torare... \\
\textsc{ipfv} \textsc{neg.ipfv} trade away to \textsc{prop} \textsc{2sg} \textsc{ins} money Chile nor \textsc{ins} dollar \\

\glt 
‘They wouldn’t pay you with Chilean money, nor with dollars...’ \textstyleExampleref{[R239.077]} 
\z

\ea\label{ex:4.303}
\gll A au i haŋa ai mo vānaŋa atu \textbf{hai} {\ꞌ}arero o tātou {\ꞌ}ā. \\
\textsc{prop} \textsc{1sg} \textsc{pfv} want \textsc{pvp} for talk away \textsc{ins} tongue of \textsc{1pl.incl} \textsc{ident} \\

\glt
‘I wanted to speak in our own language.’ \textstyleExampleref{[R201.002]} 
\z

As \textit{hai} expresses the means by which something happens, it may indicate a resource. Used in a more abstract way, it indicates a reason or motive: ‘because of, on account of, thanks to’.

\ea\label{ex:4.304}
\gll \textbf{Hai} heruru o tu{\ꞌ}u vaikava a au e {\ꞌ}ara nei. \\
\textsc{ins} sound of \textsc{poss.2sg.o} ocean \textsc{prop} \textsc{1sg} \textsc{ipfv} wake\_up \textsc{prox} \\

\glt 
‘I wake up with/from the sound of your ocean.’ \textstyleExampleref{[R474.002]} 
\z

\ea\label{ex:4.305}
\gll \textbf{Hai} ha{\ꞌ}ere mai o Kontiki i ai ai te haŋu. \\
\textsc{ins} walk hither of Kontiki \textsc{pfv} exist \textsc{pvp} \textsc{art} breath \\

\glt
‘Thanks to Kontiki’s coming, there was relief (for the people).’ \textstyleExampleref{[R376.077]} 
\z

The NP marked with \textit{hai} may also be a resource which is needed but not found yet. This sense is found with verbs of asking or searching as in \REF{ex:4.306}, but also in other contexts as in \REF{ex:4.307}:

\ea\label{ex:4.306}
\gll He nono{\ꞌ}i e te korohu{\ꞌ}a nei \textbf{hai} haraoa. \\
\textsc{ntr} request \textsc{ag} \textsc{art} old\_man \textsc{prox} \textsc{ins} bread \\

\glt 
‘This old man asked for bread.’ \textstyleExampleref{[R335.019]} 
\z

\ea\label{ex:4.307}
\gll He e{\ꞌ}a tau vi{\ꞌ}e era mai tō{\ꞌ}ona hare \textbf{hai} ahi.  \\
\textsc{ntr} go\_out \textsc{dem} woman \textsc{dist} from \textsc{poss.3sg.o} house \textsc{ins} fire  \\

\glt 
‘The woman left her house (to look) for fire(wood).’ \textstyleExampleref{[Mtx-7-35.013]}
\z

Finally, \textit{hai} may mark Patient arguments (\sectref{sec:8.6.4.3}), especially when their role is similar to Instruments.
\is{hai (instrumental prep.)|)}
\subsection{The deictic \isi{preposition} \textit{{\ꞌ}ai}}\label{sec:4.7.10}
\is{ai (\isi{preposition})@{\ꞌ}ai (\isi{preposition})|(}
\textit{{\ꞌ}Ai} is a deictic \isi{particle} (\sectref{sec:4.5.4.1.2}). Occasionally it is used as a \isi{preposition} to point at something which is at a certain distance: ‘there at/in/on...’. Like other prepositions, it may be followed by locationals\is{Locational} as in (\ref{ex:4.308}–\ref{ex:4.309}) or nouns\is{Noun!common} as in \REF{ex:4.310}:

\ea\label{ex:4.308}
\gll E pāpā, ka u{\ꞌ}i koe \textbf{{\ꞌ}ai} \textbf{ruŋa} i te ma{\ꞌ}uŋa te moa e rua. \\
\textsc{voc} father \textsc{imp} look \textsc{2sg} there\_at above at \textsc{art} mountain \textsc{art} chicken \textsc{num} two \\

\glt 
‘Father, look, there on the mountain are two chickens.’ \textstyleExampleref{[R104.052]} 
\z

\ea\label{ex:4.309}
\gll Te pūtē \textbf{{\ꞌ}ai} \textbf{roto} i te hare. \\
\textsc{art} sack there\_at inside at \textsc{art} house \\

\glt 
‘The bag is there inside the house.’ \textstyleExampleref{[R333.349]} 
\z

\ea\label{ex:4.310}
\gll {\ꞌ}E \textbf{{\ꞌ}ai} \textbf{te} \textbf{pā{\ꞌ}eŋa} \textbf{era} a mātou. \\
and there\_at \textsc{art} side \textsc{dist} \textsc{prop} \textsc{1pl.excl} \\

\glt
‘And we were there on that side.’ \textstyleExampleref{[R623.047]}
\z

This \isi{preposition} may be a contraction of the deictic \isi{particle} \textit{{\ꞌ}ai}\is{ai (\isi{preposition})@{\ꞌ}ai (\isi{preposition})} + the \isi{preposition} \textit{{\ꞌ}i}. (The glottal\is{Glottal plosive} in \textit{{\ꞌ}i} is not pronounced when it is not preceded by a prosodic boundary, see \sectref{sec:2.2.5}.) 
\is{ai (\isi{preposition})@{\ꞌ}ai (\isi{preposition})|)}
\subsection{The prominence marker \textit{ko}}\label{sec:4.7.11}
\is{ko (prominence marker)|(}
The prominence marker \textit{ko} precedes common nouns\is{Noun!common}, proper nouns\is{Noun!proper} and pronouns.\footnote{\label{fn:240}Rapa Nui also has two other particles \textit{ko}, which should not be confused with the prominence marker: the \isi{negation} \textit{(e) ko} (\sectref{sec:10.5.4}) and the perfect marker \textit{ko/ku} (\sectref{sec:7.2.7}).}  Even though it does not mark grammatical or semantic relations in the same way as other prepositions do, it is a \isi{preposition} syntactically:

%\setcounter{listWWviiiNumxxxvleveli}{0}
\begin{enumerate}
\item 
It is never preceded or followed by another \isi{preposition}. 

\item 
When \textit{ko} is followed by a common noun, this noun always has a \textit{t}{}-determiner. Proper nouns\is{Noun!proper} and pronouns follow \textit{ko} without proper article\is{a (proper article)}. This places \textit{ko} in group 2 of the prepositions (\sectref{sec:4.7.1}).

\end{enumerate}

\textit{Ko} has many different uses, which can be summarised under the heading of \textsc{prominence}: \textit{ko} signals that the \isi{noun phrase} is in some way prominent within the context. \textit{Ko} has two main functions:

In the first place, it marks prominent topics in verbal clauses (\sectref{sec:8.6.2.1}):

\ea\label{ex:4.311}
\gll \textbf{Ko} ia i eke ki tu{\ꞌ}a o tū hoi era.\\
\textsc{prom} \textsc{3sg} \textsc{pfv} climb to back of \textsc{dem} horse \textsc{dist}\\

\glt
‘(He put the child on his horse, at the front.) He (himself) mounted on the back.’ \textstyleExampleref{[R399.046]} 
\z

Secondly, it marks predicates in identifying clauses\is{Clause!identifying} (\sectref{sec:9.2.2}): 

\ea\label{ex:4.312}
\gll Te kona hope{\ꞌ}a o te nehenehe \textbf{ko} {\ꞌ}Anakena. \\
\textsc{art} place last of \textsc{art} beautiful \textsc{prom} Anakena \\

\glt
‘The most beautiful place (of the island) is Anakena.’ \textstyleExampleref{[R350.013]} 
\z

As a nominal predicate marker, \textit{ko} also marks noun phrases in focus in cleft\is{Cleft} constructions (\sectref{sec:9.2.6}):

\ea\label{ex:4.313}
\gll \textbf{Ko} te nūna{\ꞌ}a era {\ꞌ}a {\ꞌ}Ōrare te nūna{\ꞌ}a i rē. \\
\textsc{prom} \textsc{art} group \textsc{dist} of\textsc{.a} Orare \textsc{art} group \textsc{pfv} win \\

\glt
‘Orare’s group was the group that won.’ \textstyleExampleref{[R539-3.313]}
\z

Bu \textit{ko} has a number of other uses as well:

It marks non-topicalised\is{Topicalisation} verbal arguments (\sectref{sec:8.6.4.5}):

\ea\label{ex:4.314}
\gll He poreko \textbf{ko} te heke {\ꞌ}Akaverio. \\
\textsc{ntr} born \textsc{prom} \textsc{art} octopus Akaverio \\

\glt
‘The octopus Akaverio was born.’ \textstyleExampleref{[Mtx-7-14.003]}
\z

It occurs in comitative\is{Comitative} constructions (\sectref{sec:8.10}):

\ea\label{ex:4.315}
\gll He noho Rano rāua \textbf{ko} tā{\ꞌ}ana poki, \textbf{ko} te vi{\ꞌ}e.\\
\textsc{ntr} stay Rano \textsc{3pl} \textsc{prom} \textsc{poss.3sg.a} child \textsc{prom} \textsc{art} woman\\

\glt
‘Rano lived with his child and his wife.’ \textstyleExampleref{[Mtx-7-18.001]}
\z

It marks noun phrases in \isi{apposition}\is{Apposition} (\sectref{sec:5.12}):

\ea\label{ex:4.316}
\gll He oho mai era te {\ꞌ}ariki \textbf{ko} Hotu Matu{\ꞌ}a. \\
\textsc{ntr} go hither \textsc{dist} \textsc{art} king \textsc{prom} Hotu Matu’a \\

\glt
‘King Hotu Matu’a came.’ \textstyleExampleref{[Mtx-2-02.043]}
\z

It occurs in the \isi{interrogative} pronoun \textit{ko ai} ‘who’ (\sectref{sec:10.3.2.1}):

\ea\label{ex:4.317}
\gll ¿\textbf{Ko} ai koe? \\
~\textsc{prom} who \textsc{2sg} \\

\glt
‘Who are you?’ \textstyleExampleref{[R304.097]} 
\z

It occurs in exclamative\is{Exclamative} clauses (\sectref{sec:10.4.2}):

\ea\label{ex:4.318}
\gll ¡\textbf{Ko} te manu hope{\ꞌ}a o te tau! \\
~\textsc{prom} \textsc{art} animal last of \textsc{art} pretty \\

\glt
‘What an extremely pretty animal!’ \textstyleExampleref{[R345.072]} 
\z

Finally, \textit{ko te} + verbal noun\is{Noun!verbal} expresses continuity of action (\sectref{sec:3.2.3.1.1}):

\ea\label{ex:4.319}
\gll \textbf{Ko} te kimi \textbf{ko} te ohu a nua. \\
\textsc{prom} \textsc{art} search \textsc{prom} \textsc{art} shout \textsc{prop} Mum \\

\glt
‘Mum kept searching and shouting.’ \textstyleExampleref{[R236.082]} 
\z

In the following subsections, only those uses of \textit{ko} are discussed which do not have a place elsewhere in this grammar. This is followed by a general discussion on the nature of \textit{ko}.

\subsubsection{\textit{Ko} in lists and in isolation}\label{sec:4.7.11.1}

\textit{Ko} is used to mark items in a list. These items may be proper nouns\is{Noun!proper} or common nouns\is{Noun!common} with definite reference. The list may be isolated from the syntactic context as in \REF{ex:4.320}, but it may also have a syntactic role in the clause: in \REF{ex:4.321} the noun phrases introduced by \textit{ko} are direct object, yet they are marked with \textit{ko} rather than the accusative marker\is{i (accusative marker)} \textit{i}.\footnote{\label{fn:241}Common nouns\is{Noun!common} in lists may also be marked with \textit{he} (\sectref{sec:5.3.4.1}).}

\ea\label{ex:4.320}
\gll ...i tētere ai {\ꞌ}i ruŋa i te vaka te nu{\ꞌ}u nei: \textbf{ko} Parano,  \textbf{ko} Hoi Hiva, \textbf{ko} Mā{\ꞌ}aŋa, \textbf{ko} Feri {\ꞌ}e \textbf{ko} Tira.\\
~~~\textsc{pfv} \textsc{pl}:run \textsc{pvp} at above at \textsc{art} boat \textsc{art} people \textsc{prox} \textsc{prom} Parano  \textsc{prom} Hoi Hiva \textsc{prom} Ma’anga \textsc{prom} Feri and \textsc{prom} Tira\\

\glt 
‘(On 2 March 1944) the following people fled by boat: Parano, Hoi Hiva, Ma’anga, Feri and Tira.’ \textstyleExampleref{[R539-1.592]}
\z

\ea\label{ex:4.321}
\gll He {\ꞌ}apa tahi \textbf{ko} te ŋā poki, \textbf{ko} te hare, \textbf{ko} te me{\ꞌ}e ta{\ꞌ}ato{\ꞌ}a. \\
\textsc{ntr} gather all \textsc{prom} \textsc{art} \textsc{pl} child \textsc{prom} \textsc{art} house \textsc{prom} \textsc{art} thing all \\

\glt
‘She gathered all the children, the house, everything.’ \textstyleExampleref{[R352.103]} 
\z

\textit{Ko} also marks noun phrases used in isolation, i.e. without a syntactic context. In a running text, examples of isolated noun phrases are hard to detect, as a \isi{noun phrase} which seems to be isolated, may actually be the predicate of a nominal clause\is{Clause!nominal} with implied subject (see (\ref{ex:9.15}–\ref{ex:9.16}) on p.~\pageref{ex:9.15}). Clearer examples of isolated noun phrases are found in titles of stories and other texts. The following examples show that isolated pronouns and proper nouns\is{Noun!proper} are marked \textit{ko}, while common nouns\is{Noun!common} in isolation are marked with either \textit{ko} or \textit{he} (\sectref{sec:5.3.4.1}).

\ea\label{ex:4.322}
\gll He tiare ko au he raŋi he hetu{\ꞌ}u\\
\textsc{pred} flower \textsc{prom} \textsc{1sg} \textsc{pred} sky \textsc{pred} star\\

\glt 
‘The flower, me, the sky and the stars’ \textstyleExampleref{[R222.000]} 
\z

\ea\label{ex:4.323}
\gll Ko Petero {\ꞌ}e ko tō{\ꞌ}ona repahoa\\
\textsc{prom} Peter and \textsc{prom} \textsc{poss.3sg.o} friend\\

\glt 
‘Peter and his friend’ \textstyleExampleref{[R428.000]} 
\z

\subsubsection{\textit{Ko} as a locative preposition}\label{sec:4.7.11.2}

Very occasionally, \textit{ko}\is{ko (locative prep.)} is used as a \isi{preposition} with a locative sense. This usage only occurs before locationals\is{Locational}. In modern Rapa Nui, it indicates immediacy: something is in a location without delay, in a flash.

\ea\label{ex:4.324}
\gll He tu{\ꞌ}u ki {\ꞌ}Apina, \textbf{ko} \textbf{raro} te rū{\ꞌ}au nei, \textbf{ko} \textbf{roto} \textbf{i} te hare, ki rote piha o Vai Ora. \\
\textsc{ntr} arrive to Apina \textsc{prom} below \textsc{art} old\_woman \textsc{prox} \textsc{prom} inside at \textsc{art} house to inside\_\textsc{art} room of Vai Ora \\

\glt
‘When she arrived at Apina, the old woman got off (her horse) straightaway, inside the house (she went), into Vai Ora’s room.’ \textstyleExampleref{[R301.111]} 
\z

In older texts, its use is somewhat different. The sense of immediacy is not obvious; \textit{ko} seems to be similar in sense to other locative prepositions like \textit{{\ꞌ}i}.

\ea\label{ex:4.325}
\gll He nunui ararua pā{\ꞌ}iŋa \textbf{ko} \textbf{tu{\ꞌ}a} \textbf{ko} te {\ꞌ}ana, \textbf{ko} \textbf{haho} \textbf{ko} te motu.\\
\textsc{pred} \textsc{pl}:big the\_two side \textsc{prom} back \textsc{prom} \textsc{art} cave \textsc{prom} outside  \textsc{prom} \textsc{art} islet\\

\glt 
‘Both groups of children grew up, those in the back of the cave and those outside on the islet.’ \textstyleExampleref{[Mtx-3-01.293]}
\z

\ea\label{ex:4.326}
\gll Ka varu mai te pū{\ꞌ}oko ki toe {\ꞌ}iti{\ꞌ}iti \textbf{ko} \textbf{vāeŋa} nō o te rau{\ꞌ}oho. \\
\textsc{imp} shave hither \textsc{art} head to remain little:\textsc{red} \textsc{prom} middle just of \textsc{art} hair \\

\glt 
‘Shave the head, so a little hair will remain only in the middle.’ \textstyleExampleref{[Ley-6-44.033]}
\z

\subsubsection[Lexicalised ko]{Lexicalised \textit{ko}}\label{sec:4.7.11.3}

In a number of cases, \textit{ko} has become lexicalised, i.e. become part of a word or expression. In these expressions, \textit{ko} is always used, even in syntactic contexts in which it would not occur otherwise. One example is the construction \textit{ko ŋā kope} ‘the people, the guys’ (\sectref{sec:5.5.2}). Another example is \textit{ta{\ꞌ}e ko {\ꞌ}iti}, which acts as a frozen expression meaning ‘not a few, a considerable number, many’:\footnote{\label{fn:242}All other adjectives are negated by \textit{ta{\ꞌ}e} without the use of \textit{ko} (see \REF{ex:10.147} on p.~\pageref{ex:10.147}).}

\ea\label{ex:4.327}
\gll He turu ia te taŋata \textbf{ta{\ꞌ}e} \textbf{ko} \textbf{{\ꞌ}iti} ki tū kona era o te pahī. \\
\textsc{ntr} go\_down then \textsc{art} person \textsc{conneg} \textsc{prom} small to \textsc{dem} place \textsc{dist} of \textsc{art} ship \\

\glt
‘Quite a few people went down to the place where the ship was.’ \textstyleExampleref{[R250.211]} 
\z

Thirdly, the word \textit{tetu} ‘huge, enormous’ is usually preceded by \textit{ko}. This combination \textit{ko tetu} is lexicalised, that is, its use cannot be predicted from \textit{ko} + \textit{tetu}.\footnote{\label{fn:243}\textit{Ko} in \textit{ko tetu} may have found its origin in the exclamatory \textit{ko}, discussed in sec. \sectref{sec:10.4.2}.} \textit{Ko tetu} is used very flexibly: as an adjective modifying a noun, but also freestanding as in \REF{ex:4.328}.

\ea\label{ex:4.328}
\gll Nā, te vave e tahi ko uru mai {\ꞌ}ā \textbf{ko} \textbf{tetu}. \\
\textsc{med} \textsc{art} wave \textsc{num} one \textsc{prf} enter hither \textsc{cont} \textsc{prom} huge \\

\glt 
‘Look, there comes a huge wave.’ \textstyleExampleref{[R243.028]} 
\z

\subsubsection[What is ko?]{What is \textit{ko}?}\label{sec:4.7.11.4}

\sectref{sec:4.7.11} started out with the observation that \textit{ko} is a \isi{preposition}. The question remains, how the function of \textit{ko} should be characterised in general – if this is possible at all.

The multitude of uses of \textit{ko} discussed in various parts of this grammar make clear that \textit{ko} is a marker with an extremely wide range of use. The most common (and probably syntactically most significant) uses are those where \textit{ko} marks a core constituent: a topicalised\is{Topicalisation} subject of a verbal clause, the predicate of an identifying clause, or a \isi{noun phrase} in focus\is{Focus} in a cleft\is{Cleft} construction.

So on the one hand, \textit{ko} marks NPs in focus\is{Focus}, a function associated with high information load: focus highlights new and significant information. On the other hand, \textit{ko} marks topical\is{Topic, topicality} NPs, a function associated with a relatively low information load – topicalised\is{Topicalisation} NPs represents information already established in the context (cf. \citealt[51–52]{Levinsohn2007}). Several authors have pointed out this dual nature of Polynesian \textit{ko} (e.g. \citealt{Clark1976} on \is{Proto-Polynesian}PPN, \citealt{Bauer1991} and \citealt{Pearce1999} on \ili{Māori}; \citealt{MassamLee2006} on \ili{Niuean}\footnote{\label{fn:244}In \ili{Niuean}, \textit{ko} has an even wider range of uses than in Rapa Nui, as it also occurs before verbs. Incidentally, \citet[15]{MassamLee2006} mistakenly assume that the same is possible in Rapa Nui, based on confusion of the prominence marker \textit{ko} and perfect \textit{ko}.}).\footnote{\label{fn:245}According to \citet[46]{Clark1976}, the functions of \textit{ko} can possibly be reduced to “nominal predicate” and “topic”, and the two should not be confused.} Pragmatically, these two functions can be combined under the label \textit{prominence}\is{Prominence}: in both functions, the \isi{noun phrase} is in some way prominent or highlighted. For this reason, \textit{ko} is uniformly glossed as \textsc{prom}.

However, the list in \sectref{sec:4.7.11} above shows that the range of functions of \textit{ko} is much wider than topic and focus. Some uses can be reduced to the categories above; for example, \textit{ko ai} in questions is a constituent in focus; the same may be true for \textit{ko} in exclamative\is{Exclamative} clauses, while \textit{ko} in isolated NPs such as titles may be topical\is{Topic, topicality}. Not all uses are easy to categorise, however: it is less clear how \textit{ko} in appositions\is{Apposition}, lists, comitative\is{Comitative} constructions (‘X with \textit{ko} Y’) and with verbal nouns\is{Noun!verbal} should be analysed as either topic or focus\is{Focus}. The only feature connecting these functions, is that they involve a function not marked by any other \isi{preposition}. The conclusion seems justified that \textit{ko} is a \textsc{default preposition} for noun phrases which have no thematic\is{Thematicity} role in the clause (i.e. no role marked by any other \isi{preposition}), an analysis proposed by \citet[45]{Clark1976} for \is{Proto-Polynesian}Proto-Polynesian, and adopted by \citet{MassamLee2006} for \ili{Niuean}. This analysis is plausible for Rapa Nui as well. Most uses of \textit{ko} involve a \isi{noun phrase} which either does not have a semantic role, or which has been moved out of its normal argument position. (The only exceptions are non-topicalised\is{Topicalisation} subjects marked with \textit{ko}, see \sectref{sec:8.6.4.5}.)

In many functions, \textit{ko} is in complementary distribution with the nominal predicate marker \textit{he}\is{he (nominal predicate marker)}. (This does not imply that both are structurally identical: while \textit{ko} is a \isi{preposition}, \textit{he} is a determiner.) \tabref{tab:35} shows how both are used in similar contexts.

\begin{table} 
% \fittable{
\begin{tabularx}{\textwidth}{Xp{1cm}p{15mm}p{12mm}l} 
\lsptoprule
& \multicolumn{2}{c}{proper nouns}   & \multicolumn{2}{c}{common nouns}\\
& \multicolumn{2}{c}{incl.~pronouns}\\
\midrule
\raggedright
topicalisation in verbal clauses & \textit{ko}  & \sectref{sec:8.6.2.1} & \textit{ko} or \textit{he}  & \sectref{sec:8.6.2.1}; \sectref{sec:8.6.2.2}\\
\tablevspace
\raggedright
\isi{complement} of naming verbs\is{Verb!naming} & \textit{ko}  & \sectref{sec:8.6.4.5} & \textit{he}  & \sectref{sec:8.6.4.5}\\
\tablevspace
{NP predicates} & \textit{ko}  & \sectref{sec:9.2.2} & \textit{he} or \textit{ko}  & \sectref{sec:9.2.1}; \sectref{sec:9.2.2}\\
\tablevspace
{appositions\is{Apposition}} & \textit{ko}  & \sectref{sec:5.12.2} & \textit{he} or \textit{ko}  & \sectref{sec:5.12.1}\\
\tablevspace
{content questions\is{Question!content}} & \textit{ko}  & \sectref{sec:10.3.2.1} & \textit{he}  & \sectref{sec:10.3.2.2}\\
& \multicolumn{2}{l}{(\textit{ko ai} ‘who’)} & \multicolumn{2}{l}{(\textit{he aha} ‘what’)} \\
\tablevspace
in isolation & \textit{ko}  & \sectref{sec:4.7.11.1} & \textit{he}  & \sectref{sec:5.3.4.1}\\
\tablevspace
in lists & \textit{ko}  & \sectref{sec:4.7.11.1} & \textit{ko or he}  & \sectref{sec:5.3.4.1}\\
\lspbottomrule
\end{tabularx} 
% }
\caption{Comparison of \textit{ko} and \textit{he}}
% \todo[inline]{top-align}
\label{tab:35}
\end{table}

As discussed in \sectref{sec:5.3.4.1}, \textit{he} marks non-referential noun phrases, while other determiners indicate referentiality\is{Referentiality}. We may conclude that noun phrases in non-thematic\is{Thematicity} positions are either non-referential, in which case they are marked with the predicate marker \textit{he}\is{he (nominal predicate marker)}, or referential, in which case they get the default \isi{preposition} \textit{ko}. For common nouns\is{Noun!common}, both strategies are possible. Pronouns and proper nouns\is{Noun!proper}, on the other hand, are necessarily referential, so they are always marked with \textit{ko}. \is{ko (prominence marker)}
\is{Preposition|)}
\is{ko (prominence marker)|)}

\section{Conclusions}\label{sec:4.8}

Closed word classes in Rapa Nui can be placed on a continuum ranging from full words (= open classes of words occurring in the nucleus of a phrase which is a constituent of the clause) to particles (= closed classes occurring in the periphery of a phrase). 

Pronouns are close to the full word end of the continuum: they are a closed class, but serve as clause constituents and may take some of the same \isi{noun phrase} modifiers as proper nouns. They are differentiated for singular, dual, and plural, though the dual/plural distinction was lost in the second and third person.

Both numerals and quantifiers show a massive shift between older and modern Rapa Nui under \ili{Tahitian} influence. All numerals above seven (or even above five) were replaced by \ili{Tahitian} equivalents, and in certain contexts the \ili{Tahitian} terms are used even for lower numerals. On the other hand, a set of reduplicated numerals unique to Rapa Nui (the definite numerals) was preserved, though their use is on the wane (except \textit{ararua} ‘the two’, which was lexicalised).

Three quantifiers were introduced from \ili{Tahitian}, while existing quantifiers underwent semantic shifts. Interestingly, the introduced quantifiers were incorporated into Rapa Nui in ways not predictable from their \ili{Tahitian} origin; their syntax shows features not found in \ili{Tahitian}.

Demonstratives are very common in Rapa Nui. One set of demonstratives is differentiated for distance (proximal, medial, distal); it actually consists of four subsets with similar forms, which occur in different syntactic contexts: as determiners, pronouns, locationals and postnuclear particles. The other set consists of a single member \textit{tū}, not differentiated for distance. Demonstratives are extremely common in discourse; in combination with articles, they serve to indicate \isi{definiteness}, deixis and anaphora.

Rapa Nui has about a dozen prepositions. Prepositions impose restrictions on the following \isi{noun phrase}: after most prepositions the \isi{noun phrase} must be introduced by a determiner. The instrumental \isi{preposition} \textit{hai}, however, precludes the use of a determiner (perhaps reflecting its origin as a \isi{prefix}), while \textit{pē} ‘like’ is usually followed by the predicate marker \textit{he}, just like its counterparts in other Polynesian languages (even when these are not etymologically related to \textit{pē}).

The most versatile \isi{preposition} is \textit{ko}; it marks noun phrases with a wide range of functions: prominent topics, constructions in focus, nominal predicates, et cetera. It can be characterised as a default \isi{preposition}, marking all noun phrases not marked otherwise.
