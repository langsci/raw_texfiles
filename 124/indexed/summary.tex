\addchap{Summary}

% \todo[inline]{Summary is in smaller font and paragraphs are not indented. I want blank lines between the chapters, so I added subparagraphs. Any other solution would be fine as well.}

This dissertation describes the grammar of Rapa Nui, the language of Easter Island. It is mainly based on the analysis of an extensive and varied corpus of texts, dating from c. 1920 till the present.

\medskip Chapter 1 provides a short overview of the origins and history of the Rapa Nui people. The provenance and date of settlement of the island bear on the question of Rapa Nui’s position in the Polynesian language family and the status of its protolanguage, Proto-Eastern Polynesian (PEP). Re-examination of the evidence for Eastern Polynesian and Central-Eastern Polynesian shows that the evidence for the latter is much weaker than hitherto assumed; this suggests that Rapa Nui split off at a time when PEP was already diverging.

34 innovations are identified which set Rapa Nui apart from its closest relatives, as well as 11 innovations which took place in the last century. Some of the latter are due to \ili{Spanish} and \ili{Tahitian} influence; both languages have exerted a massive influence on Rapa Nui. Even so, in its grammar Rapa Nui has maintained its integrity vis-à-vis \ili{Spanish} and \ili{Tahitian}.

\medskip Chapter 2 provides a concise treatment of the phonology. The phoneme inventory is small, with 10 consonants, 5 short and 5 long vowels. Unlike most other Polynesian languages, Rapa Nui retained the Proto-Polynesian glottal plosive, which is contrastive both word-initially and -medially. Utterance-initially the glottal plosive is not contrastive; this means, for example, that there is no phonetic contrast between \textit{{\ꞌ}e} “and” and \textit{e} “\textsc{ipfv}”, despite the difference in spelling.

All words in Rapa Nui consist of bimoraic trochaic feet; only the first foot may be degenerate. This means that a heavy \isi{syllable} is never followed by an odd number of light syllables. All VV sequences are bisyllabic; diphthongs do not occur. The final foot of the word is stressed; in connected speech, the final foot of the phrase is stressed. As a consequence, postnuclear particles often receive the main stress, with secondary stress falling on the nucleus of the phrase.

Utterance-final vowel \isi{devoicing} after voiceless consonants is common and affects all vowels.

Lexicalised sound changes are pervasive: \isi{metathesis}, vowel changes, insertion and deletion of glottals, monophthongisation et cetera. Borrowings are usually adjusted to the Rapa Nui phoneme inventory and metrical structure.

Two types of \isi{reduplication} can be distinguished, monomoraic (type 1) and bimoraic (type 2). There is no principal distinction between full and partial \isi{reduplication}; full \isi{reduplication} is merely the result of type 2 \isi{reduplication} of bimoraic words. Vowel lengthening and shortening in the base or the reduplicant can be derived from a number of metrical constraints.

\medskip Chapter 3 discusses nouns and verbs and their subcategories. For Polynesian languages, the existence of a noun-verb distinction in the lexicon has often been denied, but there are good reasons to maintain this distinction. The semantic relation between nominal and verbal uses of a word is often unpredictable, hence lexically specified. Moreover, words that are the nucleus of a \isi{noun phrase} (hence “nouns” according to a syntactic approach often propagated) may either be true nouns with a nominal sense and syntax, or have a verbal sense, function and syntax. This can be accounted for by a prototypical approach to noun- and verbhood, which allows for non-prototypical forms and constructions without abolishing the noun/verb distinction. Moreover, a distinction must be made between lexical and syntactic nominalisation.

Nouns can be divided into common nouns (which take determiners), proper nouns (which take the proper article \textit{a}) and locationals (which take neither).

There is evidence for an adjective category as a subclass of the verb, though this can only be based on a range of “soft” criteria.

\medskip Chapter 4 deals with other word classes: pronouns, numerals, quantifiers, adverbs, demonstratives and prepositions. The inventory of numerals and quantifiers has been heavily influenced by \ili{Tahitian}. Even so, the syntax of \ili{Tahitian} quantifiers was not adopted; the borrowed quantifiers have syntactic characteristics (such as their position in the \isi{noun phrase}) not found in \ili{Tahitian}. Another new \isi{quantifier} is \textit{me{\ꞌ}e rahi} “much/many”; originally a noun+adjective combination functioning as nominal predicate, it developed into a prenominal quantifer.

Rapa Nui is the only Polynesian language to have a set of definite numerals, formed by \isi{reduplication} of the cardinal numerals. 

Rapa Nui has two similar sets of demonstrative forms. The first set functions as demonstrative determiners and deictic locationals (“here, there”); the second set functions as postnominal and postverbal demonstratives, and marginally as demonstrative pronouns. Both sets exhibit a three-way distance distinction (proximal, medial and distal), though the distal forms are the default choice in many contexts. An additional demonstrative determiner \textit{tū} is not specified for distance. Together, determiners and postnominal demonstratives mark noun phrases for \isi{definiteness} and anaphora.

The \isi{preposition} \textit{ko} has a variety of uses: it marks highlighted topics, constituents in focus, appositions, certain nominal predicates et cetera. In view of this diversity, \textit{ko} is best analysed as the default \isi{preposition} for noun phrases that do not have an argument role, nor are marked by other prepositions.

\medskip Chapter 5 discusses the elements of the \isi{noun phrase}. The common \isi{noun phrase} contains 17 different slots. Quantifiers and numerals occur in several different positions; for example, certain quantifiers occur before, others after the determiner. Numerals after the noun tend to have a more specific sense than numerals before the noun.

The article \textit{te} marks referentiality, not \isi{definiteness} or specificity. \textit{te} (or another \textit{t}{}-determiner) is obligatory in most syntactic contexts and excluded in others. The determiner \textit{he} marks noun phrases as predicates.

The \isi{noun phrase} may also contain a plural marker, a selection of adverbs, a deictic \isi{particle} and a postnominal demonstrative. 

Proper nouns can be modified by the same elements as common nouns, except quantifying elements. They take the proper article \textit{a} in certain contexts; different from what the label “article” may suggest, \textit{a} is not in determiner position. 

\medskip Chapter 6 discusses \isi{possession}. Possessive forms are used in a wide range of constructions; for example, they may express the S/A argument in certain subordinate clauses and – occasionally – in main clauses. 

Like most Polynesian languages, Rapa Nui exhibits a contrast between \textit{a}{}- and \textit{o}{}-possessive forms, but only in singular pronouns and with proper nouns. Which form is used, depends on the semantic relation between \isi{possessor} and \isi{possessee}. \textit{a}{}-forms are used when the \isi{possessor} has an active and/or dominant role with respect to the \isi{possessee}. The use of \textit{o-}forms, on the other hand, encompasses such a wide range of relationships, that \textit{o} must be regarded as the default possessive form. This is also suggested by the fact that in plural pronouns and with common nouns, where the \textit{a/o} distinction was neutralised, only the \textit{o}{}-forms have been maintained.

\medskip Chapter 7 deals with the elements of the verb phrase. In most contexts, the verb is obligatorily preceded by a preverbal marker, which may express \isi{aspect}, mood, subordination or \isi{negation}. This means that aspectual distinctions are neutralised in clauses containing a preverbal subordinator or negator. 

Rapa Nui has five aspectual markers. Four of these are common in Polynesian languages; the fifth, neutral \textit{he}, developed from the nominal predicate marker. \textit{he} is by far the most common \isi{aspect} marker, used for example to mark theme line events in discourse.

Of the Proto-Polynesian set of \isi{directional} markers, only two were retained in Rapa Nui, \textit{mai} “toward deictic centre” and \textit{atu} “away from deictic centre”. Apart from their deictic use, where they indicate orientation with respect to speaker and/or hearer, directionals serve to single out participants or locations in discourse as deictic centre. Examples from different narrative texts show that this deictic centre may be either stable or shifting. With motion, speech and perception verbs, directionals mark orientation; with certain (groups of) verbs there is a statistical preference for one \isi{directional} over the other.

Rapa Nui is the only Polynesian language to have a serial verb construction in which the preverbal marker is repeated. In this construction, two or more verbs together form a single verb phrase; this predicate has a single argument structure and expresses a single event or macro-event.

\medskip The verbal clause is discussed in Chapter 8. The default \isi{constituent order} is VS/VAO; other orders occur, with frequencies depending on the degree of variation from the default order.

Rapa Nui is an accusative language: S/A is unmarked or has the agent marker \textit{e}; O has the accusative marker \textit{i.} The accusative character of the language is somewhat obscured by the high frequency of the agent marker \textit{e}. Unlike its cognates in other Polynesian languages, \textit{e} is used in \isi{intransitive} as well as transitive clauses. Its use depends on a combination of semantic, syntactic and pragmatic factors. For example, it is very common with verbs of uncontrolled perception; it is obligatory in VOA clauses; it is common with subjects high in agentivity. Another factor obscuring the accusative character of the language is the frequent omission of the accusative marker \textit{i}. This too is motivated by syntactic and pragmatic factors; for example: \textit{i} tends to be omitted with the verb \textit{rova{\ꞌ}a} “obtain” and with non-salient non-human objects, and is excluded with preverbal objects.

Despite the absence of a \isi{passive} suffix, Rapa Nui has a \isi{passive} construction, which is characterised by VOA order, absence of the accusative marker and presence of the agent marker. 

Rapa Nui has various non-canonical constructions, among which are topicalised arguments, as well as the actor-emphatic construction, in which the S/A argument is expressed as a \isi{possessor}.

Rapa Nui has a variety of comitative constructions (“with”); two or more concomitant elements may be connected by a dual or plural pronoun or by a collective \isi{quantifier} (“together”). For a looser connection, the connector \textit{koia ko} “with” is used. 

The \isi{causative} construction is extremely common. It can be applied to any verb or adjective and may express various types of causation. In some cases a \isi{causative} form does not change the argument structure of the verb, but adds an element of intentionality or intensity.

\medskip Chapter 9 discusses clauses without a lexical verb. One major type concerns clauses with a nominal predicate. There is a distinction between classifying clauses, in which the predicate (marked with \textit{he}) expresses a category to which the subject belongs, and identifying clauses, in which the predicate (marked with \textit{ko} + determiner) identifies the subject with a certain referent. The latter construction is only used when the predicate meets strict requirements of identifiability.

Adjectives are used as verbal predicates to express non-inherent (and potentially transient) properties of the subject (Chapter 3). They cannot be used by themselves as nominal predicates; for an adjective to express an inherent property of the subject, it must be embedded in a \isi{noun phrase}: “this horse is a black horse”. This means that attributive clauses are similar in form to classifying clauses; however, the nominal predicate marker \textit{he} is usually omitted.

Existential clauses show a shift over time: in older texts they are predominantly verbless, in modern Rapa Nui the existential verb \textit{ai} is more common. 

In modern Rapa Nui, a \isi{copula} verb construction is emerging: the existential verb \textit{ai} is occasionally used as a \isi{copula} “to be”; \textit{riro} “to become” was borrowed from \ili{Tahitian} and became a \isi{copular} verb.

\medskip Chapter 10 deals with clause types other than positive declarative clauses: imperatives, interrogatives and exclamatives. Negation is discussed as well. 

Polar questions are often marked by \isi{intonation} only, though there is an optional marker \textit{hoki}. For content questions, there is a set of four question words, all of which belong to different word classes: \textit{ai} “who” is a \isi{proper noun}, \textit{aha} “what” is a common noun, \textit{hia} “how much/many” is a numeral, \textit{hē} “which, when, where” is an adjective. Question constituents are always fronted and in focus; “who” and “what” questions are often constructed as clefts.

Rapa Nui has a neutral negator \textit{{\ꞌ}ina} and preverbal negators \textit{kai} (perfective) and \textit{e~ko} (\isi{imperfective}). The latter can both be reinforced by \textit{{\ꞌ}ina}. \textit{{\ꞌ}ina} is a phrase nucleus which has some predicate-like properties; in this respect \textit{{\ꞌ}ina} is similar to other clause-initial elements. On the other hand, there is no reason to analyse \textit{{\ꞌ}ina} as a verb, unlike nuclear negations in other Polynesian languages.

Negations in other contexts than main clauses are expressed by \textit{ta{\ꞌ}e}.

\medskip Chapter 11 discusses the combination of clauses into sentences. There are various strategies to combine clauses: syndetic and asyndetic coordination, juxtaposition of independent clauses, subordinating conjunctions and preverbal subordinators. 

Relative clauses follow the head noun without a special marker. Any constituent can be relativised; most \isi{relative clause} constructions involve gapping, while a few non-core constituents involve a resumptive pronoun. These two strategies do not entirely conform to the \isi{noun phrase} accessibility hierarchy, as formulated by \citet{KeenanComrie1977}. 

A particularity of relative clauses is, that the preverbal marker may be omitted, something which is uncommon otherwise in Rapa Nui. In this case, the verb of the \isi{relative clause} tends to occur immediately after the head noun, before other \isi{noun phrase} elements.

Rapa Nui has a set of preverbal modal markers, the most common of which are \textit{mo} “if; in order to” and \textit{ana} “irrealis”. The latter has a wide range of functions, including intention, potentiality, obligation, general practice and condition.

