\chapter{Phonology}\label{ch:2}
\section{Introduction}\label{sec:2.1}

As this grammar is primarily based on corpus research, it does not include a complete phonology; rather, what follows is a relatively brief phonological sketch. The following topics are discussed: 

\begin{itemize}
\item 
the phoneme inventory (\sectref{sec:2.2});

\item 
phonotactics: syllable\is{Syllable} structure (\sectref{sec:2.3.1}), word structure (\sectref{sec:2.3.2}) and cooccurrence restrictions (\sectref{sec:2.3.3});

\item 
suprasegmentals: word and phrase stress\is{Stress} (\sectref{sec:2.4.1}) and intonation (\sectref{sec:2.4.2});

\item 
phonological processes: regular phonological processes (\sectref{sec:2.5.1}), lexicalised sound changes and alternations (\sectref{sec:2.5.2}) and the phonological treatment of borrowings\is{Borrowing} (\sectref{sec:2.5.3}).

\end{itemize}

Rapa Nui is one of the few Polynesian languages in which the glottal plosive is a contrastive phoneme; it is discussed in detail in \sectref{sec:2.2.4}–\ref{sec:2.2.5}. The discussion will show that while the glottal plosive is clearly contrastive in lexical words, in prenuclear particles the situation is different.

Phonological processes such as metathesis and vowel shifts have profoundly affected the lexicon of Rapa Nui, perhaps more so than in other Polynesian languages. These processes are described and illustrated in \sectref{sec:2.5.2}.

Finally, \sectref{sec:2.6} deals with reduplication. Rapa Nui has two types of reduplication; first the form, then the function of each type is discussed\is{Reduplication}.

The research for this grammar does not include formal acoustic analysis (though for certain topics a speech corpus was used). This means that the pronunciation of phonemes is only indicated in general terms (\sectref{sec:2.2.1}–\ref{sec:2.2.2}). Likewise, the treatment of intonation is limited to general statements. A full analysis of the phonetics of Rapa Nui has never been carried out so far.

\section{Phonemes}\label{sec:2.2}

The phoneme inventory\is{Phoneme inventory} of Rapa Nui consists of 10 consonants and 10 vowels (5 short and 5 long).

\subsection{Consonants}\label{sec:2.2.1}

\subparagraph{Inventory} The consonant inventory\is{Consonant inventory} of Rapa Nui is given in \tabref{tab:1}.


\begin{table}
\begin{tabularx}{\textwidth}{L{30mm}Z{15mm}Z{18mm}Z{15mm}Z{11mm}Z{12mm}}
\lsptoprule
 & bilabial& labiodental& alveolar& velar& glottal\is{Glottal plosive}\\
\midrule
voiceless~plosive\is{Plosive} & \textstyleIPA{p}&  & \textstyleIPA{t}& \textstyleIPA{k}& \textstyleIPA{Ɂ}\\
nasal\is{Nasal} & \textstyleIPA{m}&  & \textstyleIPA{n}& \textstyleIPA{ŋ}& \\
voiceless~fricative\is{Fricative} &  & \textstyleIPA{(f)}& \textstyleIPA{(s)}&  & \textstyleIPA{h}\\
voiced~fricative\is{Fricative} &  & \textstyleIPA{v}&  &  & \\
flap &  &  & \textstyleIPA{r}&  & \\
\lspbottomrule
\end{tabularx}
\caption{Consonant inventory}
\label{tab:1}
\end{table}

\textstyleIPA{/p/} is a voiceless unaspirated bilabial plosive.

\textstyleIPA{/t/} is a voiceless unaspirated alveolar plosive.

\textstyleIPA{/k/} is a voiceless unaspirated velar plosive. Before front vowels /e/ and /i/ it is somewhat fronted towards the palatal position.

\is{Plosive}\textstyleIPA{/Ɂ/} is an unaspirated glottal\is{Glottal plosive} plosive\is{Glottal plosive}. It is sometimes realised as creaky voice on the surrounding vowels. It is not unusual for the glottal\is{Glottal plosive} plosive\is{Glottal plosive} to be elided\is{Elision}; this happens especially in rapid speech and/or between identical vowels (e.g. \textit{to{\ꞌ}o} ‘take’, \textit{nu{\ꞌ}u} ‘people’). Elision\is{Elision} of the glottal\is{Glottal plosive} plosive\is{Glottal plosive} is more common with certain speakers than with others. 

\textstyleIPA{/m/} is a voiced bilabial nasal\is{Nasal}.

\textstyleIPA{/n/} is a voiced alveolar nasal\is{Nasal}.

\textstyleIPA{/ŋ/} is a voiced velar nasal\is{Nasal}. 

\textstyleIPA{/h/} is a voiceless glottal\is{Glottal plosive} fricative\is{Fricative}. Between vowels, it may become voiced in rapid speech.

\textstyleIPA{/v/} is a voiced labiodental fricative\is{Fricative}. In rapid speech it may become a labiodental approximant \textstyleIPA{[ʋ]}.\footnote{\label{fn:31}\citet[14]{GuerraEissmann1993} notice that a few of their consultants tend to pronounce it as a bilabial fricative\is{Fricative}. However, \citet[317–318]{Fischer2001Hispan}, quoting – among others – \citet{WeberWeber1982}, confirms that despite pervasive \ili{Spanish} influence on the language, /v/ is still a labiodental.}

\textstyleIPA{/r/} is a voiced alveolar flap \textstyleIPA{[ɾ]}, not a trill \textstyleIPA{[r]}.

The remaining two consonants only occur in loanwords\is{Borrowing}: 

\textstyleIPA{/f/} is a voiceless labiodental fricative\is{Fricative}. 

\textstyleIPA{/s/} is a voiceless alveolar fricative\is{Fricative}. 

Even in loanwords\is{Borrowing}, \textstyleIPA{/f/} and \textstyleIPA{/s/} are often changed to native phonemes (\sectref{sec:2.5.1} below).

In this grammar, all phonemes are written in accordance with standard Rapa Nui orthography\is{Orthography} (\sectref{sec:1.4.4}), i.e. \textstyleIPA{/Ɂ/} is written as \textit{{\ꞌ}}, \textstyleIPA{/ŋ/} as \textit{ŋ}.

\subparagraph{Contrastive sets} All consonants are contrastive both word-initially and between vowels. The following minimal sets show contrastivity for groups of similar consonants.

\newpage 
Bilabials and labiodental: \textit{p}, \textit{m}, \textit{v}

\ea
\begin{tabbing}
 xxxxxxxxxxxxxxxxxx \= xxxxxxxxxxxxxxxxxxxxx \= xxxxxxxxxxxxxxx \kill
 \textit{pā} ‘to encircle’  \> \textit{mā} ‘plus’  \> \textit{vā} ‘to resonate’\\
 \textit{{\ꞌ}apa} ‘part’  \> \textit{{\ꞌ}ama} ‘to burn’ \>  \textit{{\ꞌ}ava} ‘liquor’
\end{tabbing}
\z 
Alveolars: \textit{t}, \textit{n}, \textit{r}

\ea
\begin{tabbing}
 xxxxxxxxxxxxxxxxxx \= xxxxxxxxxxxxxxxxxxxxx \= xxxxxxxxxxxxxxx \kill
\textit{tō} ‘to rise (sun)’  \> \textit{nō} ‘just’  \> \textit{rō} ‘\textsc{emph}’\\
\textit{pota} ‘leaf vegetable’  \> \textit{pona} ‘to tie a net’  \> \textit{pora} ‘reed floater’
\end{tabbing}
\z
Velars and glottals\is{Glottal plosive}: \textit{k}, \textit{ŋ}, \textit{{\ꞌ}}, \textit{h}

\ea
\begin{tabbing}
 xxxxxxxxxxxxxxxx \= xxxxxxxxxxxxxxxx \= xxxxxxxxxxxxxxxxx \= xxxxxxxxxxxxxxx \kill
 \textit{kau} ‘to swim’  \> \textit{ŋau} ‘to bite’  \> \textit{{\ꞌ}au} ‘smoke’  \> \textit{hau} ‘cord’\\
 \textit{haka} ‘\textsc{caus}’  \> \textit{haŋa} ‘to want’   \> \textit{ha{\ꞌ}a} ‘cooked leaves’  \> \textit{haha} ‘mouth’
\end{tabbing}
\z 
Glottal versus Ø

\ea
\begin{tabbing}
 xxxxxxxxxxxxxxxxxx \= xxxxxxxxxxxxxxxxxxxxx \= xxxxxxxxxxxxxxx \kill
 \textit{ono} ‘six’  \> \textit{{\ꞌ}ono} ‘rich’\\
 \textit{uru} ‘enter’  \> \textit{{\ꞌ}uru} ‘breadfruit’\\
 \textit{moa} ‘chicken’  \> \textit{mo{\ꞌ}a} ‘to respect’\\
 \textit{hau} ‘cord’  \> \textit{ha{\ꞌ}u} ‘hat’\\
 \textit{ui} ‘generation’  \> \textit{{\ꞌ}ui} ‘to ask’  \> \textit{u{\ꞌ}i} ‘to watch’\\
 \textit{ao} ‘to serve food’  \> \textit{{\ꞌ}ao} ‘dance paddle’  \> \textit{a{\ꞌ}o} ‘speech’
\end{tabbing}
\z 
These examples show that the glottal\is{Glottal plosive} plosive\is{Glottal plosive} is contrastive word-initially; however, this does not mean that it is contrastive phrase-initially (\sectref{sec:2.2.5})\is{Prosodic phrase}.
\\

Plosives\is{Plosive}: \textit{p}, \textit{t}, \textit{k}, \textit{{\ꞌ}}

\ea
\begin{tabbing}
 xxxxxxxxxxxxxxxx \= xxxxxxxxxxxxxxxx \= xxxxxxxxxxxxxxxxx \= xxxxxxxxxxxxxxx \kill
\textit{pā} ‘to encircle’\>   \textit{tā} ‘to tattoo’  \> \textit{kā} ‘to kindle’ \>  \textit{{\ꞌ}ā} ‘\textsc{cont}’\\
\textit{tapa} ‘side’  \> \textit{tata} ‘to wash’  \> \textit{taka} ‘round’  \> \textit{ta{\ꞌ}a} ‘your’
\end{tabbing}
\z
Nasals\is{Nasal}: \textit{m}, \textit{n}, \textit{ŋ}

\ea
\begin{tabbing}
 xxxxxxxxxxxxxxxxxx \= xxxxxxxxxxxxxxxxxxxxx \= xxxxxxxxxxxxxxx \kill
   \textit{mao} ‘fine, OK’ \>  \textit{nao} ‘temple’  \> \textit{ŋao} ‘neck’\\
  \textit{tumu} ‘tree’  \> \textit{tunu} ‘to cook’  \> \textit{tuŋu} ‘to cough’\\
 \textit{mama} ‘limpet’  \> \textit{mana} ‘power’ \>  \textit{maŋa} ‘branch’
\end{tabbing}
\z 
Fricatives\is{Fricative}: \textit{v}, \textit{h}

\ea
\begin{tabbing}
 xxxxxxxxxxxxxxxxxx \= xxxxxxxxxxxxxxxxxxxxx \= xxxxxxxxxxxxxxx \kill
  \textit{vī} ‘stubborn’  \> \textit{hī} ‘to fish’\\
  \textit{ava} ‘ditch’  \> \textit{aha} ‘what’\\
 \textit{heve} ‘perchance’ \>  \textit{hehe} ‘cooked sweet potato’
\end{tabbing}
\z 

\newpage 
\textit{v} versus \textit{u} (notice that the segmental difference in these pairs also implies a difference in syllable structure: \textit{{\ꞌ}a.va.hi} versus \textit{{\ꞌ}a.u.a.hi}, see \sectref{sec:2.3.1})

\ea
\begin{tabbing}
 xxxxxxxxxxxxxxxxxx \= xxxxxxxxxxxxxxxxxxxxx \= xxxxxxxxxxxxxxx \kill
 \textit{{\ꞌ}avahi} ‘to split’  \> \textit{{\ꞌ}auahi} ‘chimney’\\
 \textit{rava} ‘sufficient’ \>  \textit{rāua} ‘they’\\
 \textit{vaka} ‘boat’ \>  \textit{{\ꞌ}uaka} ‘rod’
\end{tabbing}
\z 
\is{Perfective aspect}\textit{h} versus Ø

\ea
\begin{tabbing}
 xxxxxxxxxxxxxxxxxx \= xxxxxxxxxxxxxxxxxxxxx \= xxxxxxxxxxxxxxx \kill
\textit{ai} ‘exist’  \>\textit{hai} ‘\textsc{ins}’ \> \textit{ahi} ‘fire’\\
 \textit{vai} ‘water’  \>\textit{vahi}  ‘to separate’\\
 \textit{tui} ‘string’  \>\textit{tuhi}  ‘to point out’
\end{tabbing}
\z 
\textit{ŋ} is relatively rare word-initially. Only about 1/6 of its token occurrences in the text corpus are word-initial, and 2/3 of these concern the plural marker \textit{ŋā}. (Likewise, of all occurrences of \textit{ŋ} in the lexicon, less than 1/6 is word-initial.)

\subparagraph{Derivation} The consonant correspondences between Rapa Nui and its ancestors (\is{Proto-Polynesian}Proto-Polynesian, Proto-Nuclear Polynesian and \is{Eastern Polynesian}Proto-Eastern Polynesian)\footnote{See the Polynesian language tree in \figref{fig:1} on p.~\pageref{fig:1}.} are given in \tabref{tab:2} (data from \citealt[23–24]{Marck2000}). The consonants of \is{Central-Eastern Polynesian}Proto-Central-Eastern Polynesian and \ili{Tahitian} are also included, not only because \is{Central-Eastern Polynesian}Central-Eastern languages are Rapa Nui’s closest relatives (\sectref{sec:1.2.1}), but also because Rapa Nui borrowed\is{Tahitian influence} extensively from \ili{Tahitian} (\sectref{sec:1.4.1}; \sectref{sec:2.5.3.2}).\is{Borrowing}

\begin{table}
\resizebox{\textwidth}{!}{
\begin{tabularx}{127mm}{lrrrrrrrrrrrrr}
\lsptoprule

\is{Proto-Polynesian}PPN &  \textstyleIPA{*p}&  \textstyleIPA{*t}&  \textstyleIPA{*k}&  \textstyleIPA{*m}&  \textstyleIPA{*n}&  \textstyleIPA{*ŋ}&  \textstyleIPA{*ʔ}&  \textstyleIPA{*f}&  \textstyleIPA{*s}&  \textstyleIPA{*h}&  \textstyleIPA{*w}&  \textstyleIPA{*l}&  \textstyleIPA{*r}\\
\tablevspace
PNP &  \textstyleIPA{*p}&  \textstyleIPA{*t}&  \textstyleIPA{*k}&  \textstyleIPA{*m}&  \textstyleIPA{*n}&  \textstyleIPA{*ŋ}&  \textstyleIPA{*ʔ}&  \textstyleIPA{*f}&  \textstyleIPA{*s}&  \textstyleIPA{Ø/*h}&  \textstyleIPA{*w}&  \textstyleIPA{*l}&  \textstyleIPA{*l}\\
\tablevspace
\is{Eastern Polynesian}PEP &  \textstyleIPA{*p}&  \textstyleIPA{*t}&  \textstyleIPA{*k}&  \textstyleIPA{*m}&  \textstyleIPA{*n}&  \textstyleIPA{*ŋ}&  \textstyleIPA{*ʔ}&  \textstyleIPA{*f}&  \textstyleIPA{*s}\footnotemark{}&  \textstyleIPA{Ø/*h}&  \textstyleIPA{*w}&  \textstyleIPA{*r}&  \textstyleIPA{*r}\\
\tablevspace
Rapa Nui &  \textstyleIPA{p}&  \textstyleIPA{t}&  \textstyleIPA{k}&  \textstyleIPA{m}&  \textstyleIPA{n}&  \textstyleIPA{ŋ}&  \textstyleIPA{ʔ}&  \textstyleIPA{h}&  \textstyleIPA{h}&  \textstyleIPA{Ø}&  \textstyleIPA{v}&  \textstyleIPA{r}&  \textstyleIPA{r}\\
\tablevspace
\is{Central-Eastern Polynesian}PCE &  \textstyleIPA{*p}&  \textstyleIPA{*t}&  \textstyleIPA{*k}&  \textstyleIPA{*m}&  \textstyleIPA{*n}&  \textstyleIPA{*ŋ}&  \textstyleIPA{Ø/(ʔ)}&  \textstyleIPA{*f}&  \textstyleIPA{*s}&  \textstyleIPA{Ø/*h}&  \textstyleIPA{*w}&  \textstyleIPA{*r}&  \textstyleIPA{*r}\\
\tablevspace
\ili{Tahitian} &  \textstyleIPA{p}&  \textstyleIPA{t}&  \textstyleIPA{ʔ}&  \textstyleIPA{m}&  \textstyleIPA{n}&  \textstyleIPA{ʔ}&  \textstyleIPA{Ø}&  \textstyleIPA{f/h}&  \textstyleIPA{h}&  \textstyleIPA{Ø}&  \textstyleIPA{v}&  \textstyleIPA{r}&  \textstyleIPA{r}\\
\lspbottomrule
\end{tabularx}
}
\caption{Derivation of consonant phonemes}
\label{tab:2}
\end{table}

\footnotetext{ \is{Proto-Polynesian}PPN \textit{*s} was still present in \is{Eastern Polynesian}PEP and \is{Central-Eastern Polynesian}PCE, but \ili{Penrhyn} is the only EP language to retain it; in all others, \textit{*s} became \textit{h}, as in Rapa Nui.}

\is{a (postverbal)@{\ꞌ}ā (postverbal)}As \tabref{tab:2} shows, the \is{Proto-Polynesian}PPN glottal\is{Glottal plosive} plosive was retained in Rapa Nui but lost in \is{Central-Eastern Polynesian}PCE (though it is sporadically retained in some words in CE languages, see \citealt[335]{Wilson2012}). This means that Rapa Nui is the only \is{Eastern Polynesian}EP language where it was preserved. The phonemic status of the glottal is discussed in more detail in \sectref{sec:2.2.4} below. 

\is{Eastern Polynesian}PEP \textit{*f} and \textit{*s} became \textit{h} in all environments in Rapa Nui. In fact, \textit{*f} merged with \textit{*s} in all \is{Eastern Polynesian}EP languages, either in some or in all environments.\footnote{\label{fn:33}In \ili{Hawaiian} and Rapa Nui, \textit{*}\textit{f} > \textit{h} in all environments; in \ili{Mangarevan}, \ili{Rapa} and \ili{Rarotongan}, \textit{*f} > \textit{{\ꞌ}} in all environments; in \ili{Tahitian} and \ili{Māori}, \textit{*f} > \textit{h} medially and before round vowels, though not without exceptions (see \citealt{Harlow1998}).} One change which occurs in all \is{Central-Eastern Polynesian}CE languages but not in Rapa Nui, is \textit{*faf-} > \textit{*vah-}: \is{Proto-Polynesian}PPN/PEP \textit{*fafa} ‘mouth’ > PCE \textit{*vafa}, but Rapa Nui \textit{haha}; PNP/\is{Eastern Polynesian}PEP \textit{*fafie} ‘firewood’ > \is{Central-Eastern Polynesian}PCE \textit{*vafie}, but Rapa Nui \textit{hahie} (\sectref{sec:1.2.2} no. 13).

\is{Proto-Polynesian}PPN \textit{*h} is lost in most languages. In some NP languages (including some \is{Eastern Polynesian}EP languages), \is{Proto-Polynesian}PPN \textit{*h} is reflected as \textit{s} or \textit{h} in a few words (\citealt{Marck2000}; \citealt{Rutter2001} argues that some of these actually reflect \is{Proto-Polynesian}PPN \textit{*s} rather than \textit{*h}). In Rapa Nui, it is reflected as \textit{\textup{a glottal plosive}} in a few words (for examples see \sectref{sec:2.5.2}, cf. \citealt{Davletshin2015}). 

\subsection{Vowels}\label{sec:2.2.2}

The vowel inventory\is{Vowel inventory} of Rapa Nui is given in \tabref{tab:3}.

\begin{table}
\begin{tabularx}{110mm}{L{10mm}Z{12mm}Z{12mm}Z{12mm}Z{12mm}Z{12mm}Z{12mm}} 
\lsptoprule
& \multicolumn{2}{c}{front unrounded} & \multicolumn{2}{c}{central unrounded} & \multicolumn{2}{c}{back rounded}\\
& short& long& short& long& short& long\\
\midrule
high & \textstyleIPA{i}& \textstyleIPA{iː}&  &  & \textstyleIPA{u}& \textstyleIPA{uː}\\
mid & \textstyleIPA{e}& \textstyleIPA{eː}&  &  & \textstyleIPA{o}& \textstyleIPA{oː}\\
low &  &  & \textstyleIPA{a}& \textstyleIPA{aː}&  & \\
\lspbottomrule
\end{tabularx}
\caption{Vowel inventory}
\label{tab:3}
\end{table}

In this grammar, vowel length is represented by a macron over the vowel, in accordance with standard Rapa Nui orthography\is{Orthography}.

The vowel system was inherited without change from \is{Proto-Polynesian}PPN. All vowels are contrastive in word-initial, -medial and -final position. \tabref{tab:4} gives minimal sets of two or more contrastive short vowels.

\begin{table}
\resizebox{\textwidth}{!}{
\begin{tabularx}{\textwidth}{XXXXX}
\lsptoprule
\textbf{\textit{a}} & \textbf{\textit{e}} & \textbf{\textit{i}} & \textbf{\textit{o}} & \textbf{\textit{u}}\\
\midrule
\textit{a}\newline ‘towards’ & \textit{e}\newline ‘\textsc{ipfv}’ & \textit{i}\newline ‘at’ & \textit{o}\newline ‘\textsc{poss}’   \\[1em]
\textit{haka}\newline ‘\textsc{caus}’ & \textit{heka}\newline ‘soft’ & \textit{hika}\newline ‘make fire by friction’ &  & \textit{huka}\newline ‘stubborn’\\[1em]
&  &  & \textit{hono}\newline ‘patch’ & \textit{honu}\newline ‘turtle’\\[1em]
\textit{hara}\newline ‘sin’ & \textit{hare}\newline ‘house’ & \textit{hahari}\newline ‘to comb’ & \textit{haro}\newline ‘to pull’ & \textit{haru}\newline ‘to grab’\\[1em]
&  & \textit{heruri}\newline ‘distressed’ &  & \textit{heruru}\newline ‘noise’\\[1em]
& \textit{ho{\ꞌ}e}\newline ‘one’ & \textit{ho{\ꞌ}i}\newline ‘in fact’ & \textit{ho{\ꞌ}o}\newline ‘buy/sell’  \\[1em]
\textit{karaŋa}\newline ‘shouting’ & \textit{kareŋa}\newline ‘property’ &  & \textit{karoŋa}\newline ‘eyelids’  \\[1em]
\textit{haha{\ꞌ}u}\newline ‘to tie’ &  &  &  & \textit{hahu{\ꞌ}u}\newline ‘fish sp.’\\[1em]
&  & \textit{māhina}\newline ‘moon’ &  & \textit{māhuna}\newline ‘pimple’\\[1em]
\lspbottomrule
\end{tabularx}
}
\caption{Vowel contrasts}
\label{tab:4}
\end{table}
Vowel length\is{Vowel length} is contrastive. Some examples of monosyllabic minimal pairs:

\ea
\begin{tabbing}
 xxxxxxxxxxxxxxxxxxxxxxx \= xxxxxxxxxxxxxxxxxxxxx \kill
  \textit{ki} \textstyleIPA{/ki/} ‘to’ \> \textit{kī} \textstyleIPA{/kiː/} ‘to say’\\
  \textit{ka} \textstyleIPA{/ka/} ‘\textsc{imp}’ \> \textit{kā} \textstyleIPA{/kaː/} ‘to kindle’\\
\textit{{\ꞌ}o} \textstyleIPA{/ʔo/} ‘lest’ \> \textit{{\ꞌ}ō} \textstyleIPA{/ʔoː/} ‘really’\\
 \textit{{\ꞌ}i} \textstyleIPA{/ʔi/} ‘at’ \> \textit{{\ꞌ}ī} \textstyleIPA{/ʔiː/} ‘full’
\end{tabbing}
\z
Notice that in all these pairs the short-vowel word is a prenuclear particle\is{Particle!prenuclear}, while the long-vowel word is a lexical word\is{Lexical word} or postnuclear particle\is{Particle!postnuclear}. As a result, the two words will never occur in an identical context.

For bisyllabic words, most minimal pairs concern final vowels; in these cases, the length distinction\is{Vowel length} also implies a difference in stress\is{Stress} (\sectref{sec:2.4.1}): \textit{ha{\ꞌ}i} \textstyleIPA{/ˈhaɁi/} ‘to embrace’ versus \textit{ha{\ꞌ}ī} \textstyleIPA{/haˈɁiː/} ‘to wrap up’. Other examples include:

\ea
\begin{tabbing}
 xxxxxxxxxxxxxxxxxxxxxxx \= xxxxxxxxxxxxxxxxxxxxx \kill
  \textit{mata} \textstyleIPA{/ˈmata/} ‘eye’   \> \textit{matā} \textstyleIPA{/maˈtaː/} ‘obsidian’\\
  \textit{pua} \textstyleIPA{/ˈpua/} ‘flower’   \> \textit{puā} \textstyleIPA{/puˈaː/} ‘to touch’\\
  \textit{ruru} \textstyleIPA{/ˈruru/} ‘bundle’   \> \textit{rurū} \textstyleIPA{/ruˈruː/} ‘to tremble’\\
  \textit{huhu} \textstyleIPA{/ˈhuhu/} ‘to strip’  \>  \textit{huhū} \textstyleIPA{/huˈhuː/} ‘to move, sway in the wind’
\end{tabbing}
\z
There are a few pairs of words which only differ in vowel length in the antepenultimate (hence unstressed\is{Stress}) syllable\is{Syllable}; these words are distinguished by vowel length only.

\ea
\begin{tabbing}
 xxxxxxxxxxxxxxxxxxxxxxx \= xxxxxxxxxxxxxxxxxxxxx \kill
  \textit{momore} \textstyleIPA{/moˈmore/} ‘harvest’   \>  \textit{mōmore} \textstyleIPA{/moːˈmore/} ‘cut’\\
   \textit{rurū} \textstyleIPA{/ruˈruː/} ‘tremble’  \>  \textit{rūrū} \textstyleIPA{/ruːˈruː/} ‘to shake something’\\
  \textit{vavā} \textstyleIPA{/vaˈvaː/} ‘to resonate’  \>  \textit{vāvā} \textstyleIPA{/vaːˈvaː/} ‘to insult’\\
    ~~~~(redup. of \textit{vā})     \>     ~~~~({\textless} Tah. \textit{vāvā})
\end{tabbing}
\z 
\subsection{Phoneme frequencies}\label{sec:2.2.3}

In the text corpus, totalling over 1.6 million segments, the token frequency of each segment is as given in \tabref{tab:5}.\footnote{\label{fn:34}Counts are based on written data, so elision\is{Elision} (e.g. of the glottal\is{Glottal plosive} plosive\is{Glottal plosive}) is not taken into account. The corpus also contains 10,600 non-Rapa Nui characters, such as \textit{s} and \textit{l} (both around 2000 times), which occur in borrowings\is{Borrowing} and proper names. These do not affect the overall percentages.

Alternatively, phoneme frequencies could be based on a list of lexemes. However, as the PLRN lexical database collates data from all lexical sources, contains a relatively high proportion of words occurring in one or two older sources (especially \citealt{Roussel1908}) which were never part of the language.}

\begin{table}
\begin{tabularx}{6.7cm}{Z{1cm}R{1cm}Z{0.5cm}Z{1cm}R{1cm}}
\lsptoprule
\multicolumn{2}{c}{{consonants}

} & ~~~~~& \multicolumn{2}{c}{{vowels}

}\\
\hhline{--~--}
 \textit{h}&  6.0\%&  &  \textit{a}&  17.9\%\\
 \textit{k}&  5.6\%&  &  \textit{ā}&  2.1\%\\
 \textit{m}&  3.5\%&  &  \textit{e}&  11.3\%\\
 \textit{n}&  3.2\%&  &  \textit{ē}&  0.5\%\\
 \textit{ŋ}&  1.6\%&  &  \textit{i}&  9.4\%\\
 \textit{p}&  2.0\%&  &  \textit{ī}&  0.5\%\\
 \textit{r}&  5.7\%&  &  \textit{o}&  8.2\%\\
 \textit{t}&  8.2\%&  &  \textit{ō}&  1.0\%\\
 \textit{v}&  1.0\%&  &  \textit{u}&  5.6\%\\
 \textit{{\ꞌ}}&  6.2\%&  &  \textit{ū}&  0.5\%\\
\lspbottomrule
\end{tabularx}
\caption{Phoneme frequencies}
\label{tab:5}
\end{table}

56.9\% of all segments are vowels (52.3\% short, 4.6\% long), 43.1\% are consonants.

The most common phonemes, in descending order of frequency, are \textit{a e i t o {\ꞌ} h}. If corresponding short and long vowels\is{Vowel length} are considered as instances of the same vowel (i.e. figures for \textit{a} and \textit{ā} are added up), the order is \textit{a e i o t {\ꞌ}} \textit{u h}. The least common phonemes, in ascending order of frequency, are \textit{ū ē ī ō v ŋ}.

\subsection{The glottal plosive}\label{sec:2.2.4}
\is{Glottal plosive|(}\is{Glottal plosive}
As indicated above, Rapa Nui is the only \is{Eastern Polynesian}Eastern Polynesian language which preserved the \is{Proto-Polynesian}PPN glottal\is{Glottal plosive} plosive\is{Glottal plosive}. The glottal\is{Glottal plosive} was lost various times in the history of Polynesian; apart from Rapa Nui, it was preserved only in \ili{Tongan}, \ili{Rennell}-Bellona, \ili{East Uvean} and \ili{East Futunan}. Within \is{Eastern Polynesian}Eastern Polynesian, the presence of the \is{Proto-Polynesian}PPN glottal\is{Glottal plosive} is one of the features distinguishing Rapa Nui from the \is{Central-Eastern Polynesian}Central-Eastern languages.

The glottal\is{Glottal plosive} plosive\is{Glottal plosive} was not recorded in early lexical sources: neither in word lists by \citet{Philippi1873}, \citet{Geiseler1883}, \citet{Thomson1889}, \citet{Cooke1899}, nor in the lexica by \citet{Roussel1908,Roussel1917}, \citet{Churchill1912} and \citet{Martínez1913}. Métraux, who stayed on the island in 1934–1935, explicitly mentions that “so far as I can trust my ear there is no trace of the glottal\is{Glottal plosive} stop on Easter Island” (\citealt[32]{Métraux1971}).

Englert, who lived on the island from 1935 until 1968, did notice the significance of the glottal\is{Glottal plosive} plosive\is{Glottal plosive}: he lists minimal pairs, where the presence or absence of the “hiato” changes the meaning of the word (\citealt[16]{Englert1978}). All of the glottals\is{Glottal plosive} he noticed occur word-medially between non-identical vowels (e.g. \textit{va{\ꞌ}e} ‘foot’ versus \textit{vae} ‘choose’).

The first linguist to fully recognise the glottal\is{Glottal plosive} plosive\is{Glottal plosive} as a phoneme in Rapa Nui was \citet{Ward1961,Ward1964}. Ward compared occurrences of the glottal\is{Glottal plosive} plosive\is{Glottal plosive} with cognates in other Polynesian languages that retain the \is{Proto-Polynesian}PPN glottal\is{Glottal plosive}, and concluded that the glottal\is{Glottal plosive} in Rapa Nui corresponds to the original \is{Proto-Polynesian}PPN glottal\is{Glottal plosive} (apart from \ili{Tahitian} borrowings\is{Tahitian influence}, see below). An example is Rapa Nui \textit{hō{\ꞌ}ou} ‘new’; \ili{Tongan}, \ili{East Futunan}, \ili{East Uvean} \textit{\mbox{fo{\ꞌ}ou}}, \ili{Rennell} \textit{ho{\ꞌ}ou}, but \ili{Hawaiian} and \ili{Tahitian} \textit{hou}. Around the same time, \citet[4]{Bergmann1963} included the glottal\is{Glottal plosive} in his phoneme inventory of Rapa Nui, though he suggested that it has disappeared in the modern language.

The adoption of the glottal\is{Glottal plosive} as a full-fledged consonant phoneme was confirmed in later phonological analyses: \citet{Blixen1972}, \citet{Salas1973}, \citet{WeberWeber1982} and \citet{GuerraEissmann1993}.\footnote{\label{fn:35}According to \citet[24, 69]{Marck2000}, the Rapa Nui glottal\is{Glottal plosive} was lost in the environment \textit{a\_\_a}; however, this is based partly on sources with defective orthography\is{Orthography}, such as \citet{Fuentes1960} (e.g. RN \textit{*haaki} ‘to inform’ {\textless} PNP \textit{*fa{\ꞌ}aki}; the actual Rapa Nui form is \textit{hā{\ꞌ}aki}), partly on \ili{Tahitian} loans (RN \textit{tane} ‘male’ {\textless} \ili{Tahitian}\is{Tahitian influence} \textit{tāne}, cf. PNP \textit{*ta{\ꞌ}ane}). See also \citet{Davletshin2015}.} Despite Métraux’ and Bergman’s assertions to the contrary, in current Rapa Nui the glottal\is{Glottal plosive} stop is consistently present. Only a minority of speakers (especially those for whom Rapa Nui is not their first language) tend to elide\is{Elision} it frequently. 

While most instance of the glottal\is{Glottal plosive} plosive\is{Glottal plosive} in Rapa Nui correspond to the PPN glottal\is{Glottal plosive}, a second source for the glottal\is{Glottal plosive} plosive\is{Glottal plosive} is \ili{Tahitian}. Rapa Nui borrowed extensively from \ili{Tahitian} (\sectref{sec:1.4.1}); this includes words containing glottals, like \textit{ho{\ꞌ}o} ‘buy/sell’, \textit{{\ꞌ}a{\ꞌ}amu} ‘story’, \textit{{\ꞌ}āno{\ꞌ}i} ‘to mix’ and \textit{ha{\ꞌ}ari} ‘coconut’. The fact that the glottal\is{Glottal plosive} was already part of the phoneme system doubtlessly facilitated the adoption of these words without elision\is{Elision} of the glottal\is{Glottal plosive} (\sectref{sec:2.5.3.2}).

\subsection{The glottal plosive in particles}\label{sec:2.2.5}
\largerpage
\is{Glottal plosive}\is{Glottal plosive}
As shown in \sectref{sec:2.2.1} above, the glottal\is{Glottal plosive} plosive\is{Glottal plosive} is contrastive both word-initially and after vowels. Now all examples given there concern full words\is{Full word} (\sectref{sec:3.1}), i.e. content words; full words in natural speech are usually preceded by a particle, e.g. an aspectual or a determiner. They hardly ever occur at the start of a prosodic phrase\is{Prosodic phrase}. 

Words which do occur at the start of prosodic phrases are prenuclear particles\is{Particle!prenuclear}, such as aspect markers\is{Aspect marker} and prepositions. In the standard Rapa Nui orthography\is{Orthography}, some of these are written with a glottal\is{Glottal plosive}: \textit{{\ꞌ}e} ‘and’, \textit{{\ꞌ}a} ‘of’; others do not have a glottal\is{Glottal plosive}: \textit{e} ‘\textsc{ag}; \textsc{ipfv}’, \textit{o} ‘of’. 

The question is, whether there is a real phonetic distinction between the presence and the absence of a glottal\is{Glottal plosive} in these particles. To answer this question, I analysed the pronunciation of eight particles – four with orthographic\is{Orthography} glottal\is{Glottal plosive}, four without – in an oral text corpus, spoken by a number of speakers of different genders and age groups.\footnote{\label{fn:36}The corpus consists of Bible passages used for the dubbing of a Biblical movie. These texts were rehearsed recitation, partly read from paper, which may favour a pronunciation in line with the orthography\is{Orthography}; however, the passages were practiced until pronounced smoothly and naturally, which should have mitigated the “orthography\is{Orthography} effect”.}  For each occurrence, I determined:

\begin{itemize}
\item 
whether or not the particle is pronounced with a glottal\is{Glottal plosive};

\item 
whether or not the particle occurs at the start of a prosodic phrase, indicated by a pause or an intonational\is{Intonation} break.

\end{itemize}

\begin{table}
%\small
\begin{tabularx}{\textwidth}{lrrrrrrr}
\lsptoprule 
& 
\parbox{1.5cm}{\raggedleft\textit{a}\is{a (preposition)} \\ ‘toward’}& 
\parbox{1.5cm}{\raggedleft\textit{e}\is{e (imperfective)} \\ ‘\textsc{ag}; \textsc{ipfv}’}& 
\parbox{1cm}{\raggedleft\textit{i} \\ ‘at’}& 
\parbox{1.8cm}{\raggedleft\textit{o}\is{o (possessive prep.)} \\ ‘of’}& 
\parbox{1cm}{\raggedleft total \\ ~} & \\
\midrule
with \textstyleIPA{[ʔ]}:\\
initial &  43&  189&  138&  70&  ~~~~~440&  30.1\%\\% \textit{(=}\\
  non-initial &  3&  21&  5&  13&  42&  2.9\% \\%\textit{33.0\%)}\\
\midrule
              &   &    &   &    &    & 33.0\% \\
\tablevspace
without \textstyleIPA{[ʔ]}:  \\
initial &  1&  17&  15&  0&  33&  2.3\% \\%\textit{(=}\\
  non-initial &  245&  188&  321&  191&  945&  64.7\%\\%& \textit{67.0\%)}\\
\midrule
              &   &    &   &    &    & 67.0\%\\
% \lsptoprule

\\
&
\parbox{1.5cm}{\raggedleft\is{a (possessive prep.)@{\ꞌ}a (possessive prep.)}\textit{{\ꞌ}a} \\‘of’}& 
\parbox{1.5cm}{\raggedleft\textit{{\ꞌ}e}\is{e ‘and’@{\ꞌ}e ‘and’} \\ ‘and’}& 
\parbox{1cm}{\raggedleft\textit{{\ꞌ}i}\is{i ‘in, at’@{\ꞌ}i ‘in, at’} \\ ‘at’}& 
\parbox{1.8cm}{\raggedleft\textit{{\ꞌ}o}\is{o ‘because of’@{\ꞌ}o ‘because of’} \\ ‘because~of’}& 
\parbox{1cm}{\raggedleft total \\ ~} & \\
\midrule
with \textstyleIPA{[ʔ]}: \\
initial &  4&  153&  133&  25&  315&  75.0\% \\%\textit{(=}\\
  non-initial &  3&  7&  13&  3&  26&  6.2\%\\%& \textit{81.2\%)}\\
\midrule
              &   &    &   &    &    &81.2\%\\
\tablevspace
without \textstyleIPA{[ʔ]}:  \\
initial &  0&  3&  1&  0&  4&  1.0\%& \\%\textit{(=}\\
  non-initial &  21&  1&  50&  3&  75&  17.9\%\\% & \textit{18.8\%)}\\
\midrule
              &   &    &   &    &    &18.8\%\\
\lspbottomrule
\end{tabularx}
\caption{Pronunciation of glottals in particles}
\label{tab:6}
\end{table}

This yields the statistics given in \tabref{tab:6}. This table shows that particles written with a glottal\is{Glottal plosive} are indeed overwhelmingly pronounced with a glottal\is{Glottal plosive} (81.2\%), while particles written without glottal\is{Glottal plosive} are predominantly pronounced without glottal\is{Glottal plosive} (67.0\%). However, this effect is largely due to the distribution of these particles. At the start of a prosodic phrase\is{Prosodic phrase}, most Rapa Nui speakers almost automatically pronounce initial vowels with a sharp onset, i.e. a non-phonemic glottal\is{Glottal plosive} plosive\is{Glottal plosive}. As \tabref{tab:6} shows, phrase-initial particles are overwhelmingly pronounced with a glottal: 440+315 against 33+4 without glottal, i.e. 755 out of 792 (95.3\%). On the other hand, \is{Glottal plosive}non-phrase-initial particles tend to be pronounced without a glottal: 945+75 against 42+26 with glottal, i.e. 1020 out of 1088 (93.8\%)\is{Glottal plosive}. The fact that this strongly correlates with the presence or absence of the \textit{written} glottal\is{Glottal plosive}, is because certain particles happen to occur much more frequently after boundaries than others. For example, the conjunction\is{Conjunction} \textit{{\ꞌ}e} ‘and’ is almost always preceded by a pause, while the proper article\is{a (proper article)} \textit{a} is very often preceded by a preposition, hence non-initial. In other words, the glottal\is{Glottal plosive} is not phonemic in these particles; it just tends to be pronounced phrase-initially and omitted otherwise.

The following example illustrates this. The first line represents the orthography\is{Orthography}, the second line is a broad phonetic transcription. {\textbar} indicates a prosodic phrase break; \_  represents a vowel onset without glottal\is{Glottal plosive}.

\ea\label{ex:2.1}
\glll  ~~Te nu{\ꞌ}u ~ \textbf{e} rerehu rō \textbf{{\ꞌ}i} te ri{\ꞌ}ari{\ꞌ}a. ~ \textbf{{\ꞌ}I} rā hora he take{\ꞌ}a \textbf{i} te Poki ~o te Taŋata ~ ka topa mai ~  \textbf{{\ꞌ}i} ruŋa \textbf{i} te raŋi ~ \textbf{{\ꞌ}i} ruŋa \textbf{i} \textbf{a} ia  ~ te pūai \textbf{~{\ꞌ}e} te {\ꞌ}ana{\ꞌ}ana \textbf{o} te {\ꞌ}Atua.\\
[~te nuʔu {\textbar} \textbf{ʔe} rerehu roː \textbf{\_i} te riʔariʔa {\textbar} \textbf{ʔi} raː hora he takea \textbf{\_i} te poki \textbf{\_o} te taŋata {\textbar} ka topa mai {\textbar} \textbf{ʔi} ruŋa \textbf{\_i} te raŋi {\textbar} \textbf{ʔi} ruŋa \textbf{\_i} \textbf{\_a} ia {\textbar} te puːai \textbf{ʔe} te ʔanaʔana \textbf{\_o} te ʔatua~]\\
\textsc{art} people ~ \textsc{ipfv} faint \textsc{emph} at \textsc{art} afraid  ~  at \textsc{dist} time \textsc{ntr} see \textsc{acc} \textsc{art} child of \textsc{art} man  ~  \textsc{cntg} descend hither  ~  at above at \textsc{art} heaven  ~  at above at \textsc{prop} \textsc{3sg}  ~  \textsc{art} power and \textsc{art} glory of \textsc{art} God\\

\glt
‘The people will faint from fear. At that time they will see the Son of Man descending in heaven, on him the power and glory of God.’ \textstyleExampleref{[R630-13.010]}
\z

As this fragment shows, the preposition \textit{{\ꞌ}i} ‘in’ is pronounced with glottal\is{Glottal plosive} after a pause (3x), but without a glottal\is{Glottal plosive} within a prosodic phrase\is{Prosodic phrase} (line 1). The conjunction\is{Conjunction} \textit{{\ꞌ}e} ‘and’ is pronounced with glottal\is{Glottal plosive} after a pause (line 3), but so is the imperfective marker \textit{e} (line 1). The preposition \textit{i} is never pronounced with glottal\is{Glottal plosive} in this fragment, but then, it does not occur phrase-initially. The same is true for the proper article\is{a (proper article)} \textit{a} and the possessive preposition \textit{o}.

This example also shows that the orthography\is{Orthography} is accurate as far as glottals\is{Glottal plosive} in content words are concerned: glottals\is{Glottal plosive} are usually pronounced where they are written, both word-medially (\textit{nu{\ꞌ}u, ri{\ꞌ}ari{\ꞌ}a}) and word-initially (\textit{{\ꞌ}Atua}). The same is true for longer particles (which do not occur in this example), such as the postverbal markers \textit{{\ꞌ}ā} ‘\textsc{cont}’ and \textit{{\ꞌ}ai} ‘\textsc{subs}’; these are consistently pronounced with glottal\is{Glottal plosive}.

We may conclude that the glottal\is{Glottal plosive} is not contrastive in prenuclear particles\is{Particle!prenuclear}. The glottal\is{Glottal plosive} is a phonetic reality only to the extent that particles occur post-pausally.\footnote{\label{fn:37}Cf. \citet[222]{Clark1976}: particles fail to follow the normal correspondences, which “is probably a result of their typically phrase-initial position”. He points out that there is a universal tendency to insert a glottal\is{Glottal plosive} after a pause, so “glottal\is{Glottal plosive} stop in such position is of dubious value” (23). He gives several examples of initial particles which have an initial glottal\is{Glottal plosive} stop in \ili{Tongan} (a language which has preserved the \is{Proto-Polynesian}PPN glottal\is{Glottal plosive}), no glottal\is{Glottal plosive} stop in \ili{Rennell} (idem), but a glottal\is{Glottal plosive} stop in \ili{Tahitian} (which does not retain the PPN glottal\is{Glottal plosive}).

For this reason, it is difficult to reconstruct the protoform of phrase-initial particles; there is some discussion whether the \is{Proto-Polynesian}PPN possessive markers were \textit{a} or \textit{{\ꞌ}a}, and \textit{o} or \textit{{\ꞌ}o} (see \citealt{Fischer2000Rapanui}, Lichtenberk 200; \citealt{Wilson1982}; \citealt{Lynch1997}).} This does not mean that the use of the glottal\is{Glottal plosive} symbol in these particles is without justification: it helps the reader to distinguish possessive \textit{{\ꞌ}a} from the proper article\is{a (proper article)} \textit{a}, and the conjunction\is{Conjunction} \textit{{\ꞌ}e} from the many particles \textit{e}. Yet one should keep in mind that the distinction is in a sense superficial. This is especially important in the case of the prepositions \textit{{\ꞌ}i} and \textit{i}, which are etymologically a single preposition (\sectref{sec:4.7.2}). \is{Glottal plosive}
\is{Glottal plosive|)}
\section{Phonotactics}\label{sec:2.3}
\is{Syllable|(}\subsection{Syllable structure}\label{sec:2.3.1}
\is{Syllable}
The syllable\is{Syllable} structure of Rapa Nui is (C)V(ː). The syllable\is{Syllable} contains a single short or long vowel, optionally preceded by one consonant. A syllable\is{Syllable} cannot contain two or more vowels. This means that all sequences of non-identical vowels are disyllabic, even those often analysed as diphthongs\is{Diphthong} in other Polynesian languages: words like \textit{kai} ‘to eat’ and \textit{mau} ‘really’ do not contain a diphthong, but two syllables. 

In older descriptions of Rapa Nui, such as \citet{Englert1978}, \citet{Fuentes1960} and \citet{Salas1973}, certain VV sequences are analysed as diphthongs\is{Diphthong}; for example, \citet[16]{Englert1978} mentions \textit{ai}, \textit{au} and \textit{oi}.\footnote{\label{fn:38}In other Polynesian languages the diphthong inventory may be different. For example, in \ili{Tahitian}, all VV sequences in which the first vowel is more open, are considered diphthongs\is{Diphthong} (\citealt[5]{AcadémieTahitienne1986}). The same is true in \ili{Māori} (\citealt[69]{Harlow2007Maori}).} This is understandable, as phonetically it is often impossible to distinguish two separate syllable\is{Syllable} nuclei in sequences like \textit{ai} and \textit{au}. Even so, there are several reasons to consider all VV sequences as disyllabic.\footnote{\label{fn:39}The first three of these are also mentioned by \citet{WeberWeber1982}.} 

%\setcounter{listWWviiiNumciileveli}{0}
\begin{enumerate}
\item 
Reduplication\is{Reduplication} data. If \textit{kai} ‘to eat’ would constitute a single syllable\is{Syllable}, it would be impossible to produce the reduplication\is{Reduplication} \textit{kakai} ‘to eat (Pl)’ (\sectref{sec:2.6.1.1}). In prosodic terms: if \textit{kai} is a single syllable\is{Syllable}, the reduplication\is{Reduplication} base \textit{ka-} does not constitute a prosodic unit; rather, it consists of an onset and a partial nucleus. On the other hand, if \textit{kai} is disyllabic, \textit{kakai} can be analysed as copying (the segmental content of) the first syllable\is{Syllable} of the root.

\item 
Stress\is{Stress} patterns. As discussed in \sectref{sec:2.4.1} below, when the final syllable\is{Syllable} of a word is short, the penultimate syllable\is{Syllable} is stressed. This happens regardless of the occurrence of consecutive vowels: even when the antepenultimate + penultimate vowel would be likely candidates for diphthongisation\is{Diphthong}, the penultimate vowel receives stress\is{Stress}.\footnote{\label{fn:40}In this area, the difference with \ili{Tahitian} is especially obvious. While cognates of the first four items occur in \ili{Tahitian} as well (\textit{r}\textit{āua} and \textit{maika} are shared cognates, \textit{māuiui} and \textit{haraoa} were borrowed from \ili{Tahitian}\is{Tahitian influence}), the pronunciation in \ili{Tahitian} is markedly different because of diphthongisation\is{Diphthong} and stress\is{Stress} shift: \textstyleIPA{[ˈraːᵘa   ˈmaⁱɁa   ˈmaːᵘiᵘi   faˈraːᵒa]}.} Some examples:
\end{enumerate}
\ea
\begin{tabbing}
 xxxxxxxxxx \= xxxxxxxxxxx \= xxxxxxxxx \kill
 \textit{rāua} \>  \textstyleIPA{/raːˈua/} \>  ‘\textsc{3pl}’\\
 \textit{maika} \>  \textstyleIPA{/maˈika/} \>  ‘banana’\\
 \textit{māuiui} \>  \textstyleIPA{/ˌmaːuiˈui/} \>  ‘sick’\\
 \textit{haraoa} \>  \textstyleIPA{/ˌharaˈoa/} \>  ‘bread’\\
 \textit{i a ia} \>  \textstyleIPA{/ˌiaˈia/} \>  ‘3\textsc{sg}.\textsc{acc}’
\end{tabbing}
\z
\begin{itemize} 
\item[]
If, for example, \textit{ai} in \textit{maika} would be a diphthong, it would be impossible for \textit{i} to receive stress\is{Stress}; rather, the stress\is{Stress} pattern would be \textstyleIPA{[ˈmaika]}.
\end{itemize} 

% \todo[inline]{indent this paragraph as sequel of numbered item 2}
% \todo[inline]{Indentation not taken care of yet...}

\begin{enumerate}

\setcounter{enumi}{2}
\item 
If some VV sequences are considered as bisyllabic, for reasons of symmetry it is satisfactory to treat all such sequences in the same way.\footnote{\label{fn:41}\citet[24]{Berg1989} makes the same observation for \ili{Muna}. Similarly, \citet[127]{Rehg2007} points out that the wide range of diphthongs\is{Diphthong} in \ili{Hawaiian} (as opposed to e.g. \ili{English}) suggests that they are VV sequences rather than occupying a single V slot (though in his analysis, they are not bisyllabic).}

\item 
Finally, metrical\is{Metrical structure} constraints on word structure suggest that VV sequences are bisyllabic, unlike long vowels\is{Vowel length}. This is discussed in the next section.
\is{Syllable|)}
\end{enumerate}
\subsection{Word structure}\label{sec:2.3.2}
\subsubsection{Constraints on word structure}\label{sec:2.3.2.1}

The phonological structure of words in Rapa Nui can be described using concepts from metrical\is{Metrical structure|(} theory (\citealt{Kager1995}; \citealt{Hayes1995}). Words and phrases are organised in metrical\is{Metrical structure} units; in ascending order: morae\is{Mora} (µ), syllables\is{Syllable} ($\sigma $), feet\is{Foot} (F) and prosodic words\is{Prosodic word}. The following constraints apply:

%\setcounter{listWWviiiNumcvleveli}{0}
\begin{enumerate}
\item 
Short vowels form a monomoraic (light) syllable\is{Syllable}; long vowels\is{Vowel length} form a bimoraic (heavy) syllable\is{Syllable}.

\item 
Onset consonants are non-moraic, i.e. the presence or absence of a consonant does not affect the weight of the syllable\is{Syllable}.\footnote{\label{fn:42}Coda consonants, which occur in some borrowings\is{Borrowing}, are non-moraic as well. Coda consonants can occur in any non-final syllable\is{Syllable}, including the penultimate: \textit{toro}\textbf{m}\textit{po} ‘spinning top’, \textit{ase}\textbf{r}\textit{ka} ‘chard’, \textit{ramie}\textbf{n}\textit{ta} ‘tool’. If these consonants were moraic, the penultimate syllable\is{Syllable} would be heavy, while the final syllable\is{Syllable} is light. They would thus violate the *...HL constraint formulated below.}

\item 
All morae\is{Mora} are parsed into trochaic feet (i.e. a strong mora\is{Mora} followed by a weak mora\is{Mora}).

\item 
A foot\is{Foot} cannot span a syllable\is{Syllable} boundary.\footnote{\label{fn:43}This condition was formulated as a universal constraint by \citet[50]{Hayes1995}.} 

\item 
All non-initial feet are bimoraic. Only the initial foot of the word can be degenerate\is{Foot!degenerate}, i.e. monomoraic.\footnote{\label{fn:44}An alternative would be to state that the initial mora\is{Mora} can be left unparsed. However, the fact that initial syllables\is{Syllable} sometimes receive secondary stress\is{Stress}, suggests that they are in fact parsed into a degenerate\is{Foot!degenerate} foot. Cf. Kager’s principle of exhaustivity (\citealt[370]{Kager1995}), which requires all syllables\is{Syllable} of a word to be parsed.} 

\end{enumerate}

These constraints are inviolable and apply to all words, including reduplications, compounds and borrowings\is{Reduplication}. This means that all words in the language conform to a single rule: a heavy syllable\is{Syllable} is never followed by an odd number of morae\is{Mora}.\footnote{\label{fn:45}In other Polynesian languages a similar tendency operates, though usually in a weaker form. For example, in \ili{Samoan}, the penultimate vowel in trisyllabic words cannot be long \citep[102]{Hovdhaugen1990}.} In other words, when heavy syllables\is{Syllable} are followed by light syllables\is{Syllable}, the latter always occur in pairs; patterns such as the following do not occur: 

\ea
  * H L ~~~~ * L H L H ~~~~ * H L L L ~~~~ * H L L L H
\z

\tabref{tab:7} lists all occurring word patterns. Certain patterns\is{Metrical structure} are common, while others are rare or nonexistent. Foot boundaries are indicated by dots. Column 2 gives the number of morae\is{Mora}, column 3 lists the frequency of each pattern.\footnote{\label{fn:46}Counts are based on the PLRN lexical database (\sectref{sec:1.6.1}). All words in the database are included, including obsolete words (the length of which cannot be ascertained, though Englert’s lexicon often records length accurately), as well as words which may never have been genuine Rapa Nui words, but which occur in less reliable sources such as \citet{Roussel1908}. Homonyms are counted separately.}

\begin{table}[b]
\begin{tabularx}{125mm}{>{\raggedleft}p{18mm}p{8mm}>{\raggedleft}p{7mm}p{75mm}}
%NB Leftmost column is designed in such a way that the foot boundaries (indicated by a full stop) lined up. This is done by using small spaces and a monospaced font.
%\tablehead
\lsptoprule
{ pattern} & { µ} &  { {\#}}& { examples}\\
\midrule
{ L.}& 1 &  45& \textit{te} ‘article’, \textit{i} ‘\textsc{pfv}’, \textit{ki} ‘to’ \\[.1em]
 \tablevspace
{ LL.}& 2 &  1237& \textit{ai} ‘exist’, \textit{{\ꞌ}ana} ‘cave’, \textit{e{\ꞌ}a} ‘go out’, \textit{haha} ‘mouth’, \newline \textit{kope} ‘person’, \textit{nui} ‘big’, \textit{paru} ‘paint’, \textit{tau} ‘pretty’\\
\tablevspace
 { H{\tiny {\db}}.}& 2 &  87& {\textit{ū} ‘milk’, \textit{pō} ‘night’, \textit{hū} ‘burn, roar’, \textit{rō} ‘\textsc{emph}’, \textit{hai} ‘\textsc{ins}’}\\
\tablevspace
 { L.LL.}& 3 &  1010& {\textit{{\ꞌ}a{\ꞌ}amu} ‘story’, \textit{makenu} ‘move’, \textit{mauku} ‘grass’, \newline \textit{hohopu} ‘bathe (Pl)’; borrowings\is{Borrowing}: \textit{kamita} ‘shirt’, \textit{rivuho} ‘drawing’}\\
\tablevspace
 { L.{\tiny {\db}}H{\tiny {\db}}.} & 3 &  54& {\textit{kurī} ‘cat’, \textit{maŋō} ‘shark’, \textit{pahī} ‘ship’, \textit{ra{\ꞌ}ā} ‘sun’; borrowings\is{Borrowing}: \textit{kampō} ‘inland’, \textit{panā} ‘empanada’}\\
\tablevspace
 { LL.LL.}& 4 &  1290& {\textit{haŋupotu} ‘youngest child’, \textit{huŋavai} ‘parent-in-law’, \textit{manavai} ‘rock garden’, \textit{takaure} ‘fly’, \textit{teatea} ‘white’}\\
\tablevspace
 { LL.{\tiny {\db}}H{\tiny {\db}}.} & 4 &  70& {\textit{hatatū} ‘gizzard’\textit{, keretū} ‘pumice’\textit{,} \textit{raupā} ‘leaf’; borrowings\is{Borrowing}: \textit{korop}\textit{ā} ‘crowbar’, \textit{Kiritō} ‘Christ’}\\
\tablevspace
 { H{\tiny {\db}}.LL.}& 4 &  453& {\textit{{\ꞌ}āriŋa} ‘face’, \textit{kūmara} ‘sweet potato’, \textit{hīhiŋa} ‘to fall (Pl)’; borrowings\is{Borrowing}: \textit{hāmara} ‘hammer’, \textit{kānato} ‘basket’}\\
\tablevspace
 { H{\tiny {\db}}.{\tiny {\db}}H{\tiny {\db}}.}& 4 &  69& {\textit{{\ꞌ}āpī} ‘new’, \textit{kōpū} ‘belly’, \textit{hāhā} ‘to touch’; borrowings\is{Borrowing}: \textit{kāpē} ‘coffee’, \textit{{\ꞌ}ātā} ‘until’}\\
\lspbottomrule
\end{tabularx}
\caption{Metrical word structures}
% \todo[inline]{All columns should be top-aligned instead of center.}
\label{tab:7}
\end{table}

\begin{table}[p]
\begin{tabularx}{125mm}{>{\raggedleft}p{20mm}p{8mm}>{\raggedleft}p{5mm}p{75mm}}
\lsptoprule
{ pattern} & { µ} &  { {\#}}& { examples}\\
\midrule
 
 { L.LL.LL.}& 5 &  101& {\textit{hakanonoŋa} ‘fishing zone’, \textit{oromatu{\ꞌ}a} ‘priest’, \textit{pipihoreko} ‘cairn, rock pile’; borrowings\is{Borrowing}: \textit{{\ꞌ}etaretia} ‘church’, \textit{sanaoria} ‘carrot’}\\
\tablevspace
 { L.LL.{\tiny {\db}}H{\tiny {\db}}.} & 5 &  10& {\textit{pipirimā} ‘Gemini’, \textit{mairepā} ‘northwest wind’; borrowing: \textit{{\ꞌ}epikipō} ‘bishop’}\\
\tablevspace
 { L.{\tiny {\db}}H{\tiny {\db}}.{\tiny {\db}}H{\tiny {\db}}.} & 5 &  3& {\textit{{\ꞌ}anīrā} ‘later today’}\\
\tablevspace
 { LL.LL.LL.}& 6 &  35& {\textit{{\ꞌ}aŋataiahi} ‘yesterday’, \textit{pakapakakina} ‘to explode.\textsc{red}’, \textit{taure{\ꞌ}are{\ꞌ}a} ‘young person’}\\
\tablevspace
{ LL.LL.{\tiny {\db}}H{\tiny {\db}}.} & 6 &  5& {\textit{{\ꞌ}au{\ꞌ}auē} ‘to cry.\textsc{red}’\textit{, ma{\ꞌ}u{\ꞌ}aurī} ‘prison’, \textit{kere\-kere\-tū} ‘pumice’}\\
\tablevspace
 { LL.{\tiny {\db}}H{\tiny {\db}}.LL.}& 6 &  5& {\textit{manupātia} ‘wasp’\textit{, ha{\ꞌ}amā{\ꞌ}ore} ‘shameless’}\\
\tablevspace
 { LL.{\tiny {\db}}H{\tiny {\db}}.{\tiny {\db}}H{\tiny {\db}}.}& 6 &  4& {\textit{{\ꞌ}aŋarīnā/{\ꞌ}aŋanīrā} ‘earlier today’}\\
\tablevspace
 { H{\tiny {\db}}.LL.LL.}& 6 &  130& {\textit{hānautama} \textit{\textup{‘}}\textit{pregnant’, pō{\ꞌ}auahi} ‘hell’, \textit{vānaŋanaŋa} ‘to talk.\textsc{red}’, \textit{vānavanaŋa} ‘to talk.\textsc{red}’}\\
\tablevspace
 { H{\tiny {\db}}.LL.{\tiny {\db}}H{\tiny {\db}}.} & 6 &  1& {\textit{tātaurō} ‘cross’}\\
\tablevspace
 { H{\tiny {\db}}.{\tiny {\db}}H{\tiny {\db}}.LL.}& 6 &  16& {\textit{{\ꞌ}āpārima} ‘dance’\textit{, mātāmu{\ꞌ}a} ‘ancestors’\textit{, tōtōamo} ‘trumpetfish’; borrowings\is{Borrowing}: \textit{{\ꞌ}ōpītara} ‘hospital’, \textit{pērīkura} ‘movie’}\\
\tablevspace
 { H{\tiny {\db}}.{\tiny {\db}}H{\tiny {\db}}.{\tiny {\db}}H{\tiny {\db}}.} & 6 &  12& {\textit{pātōtō} ‘to knock’, \textit{tōtōā} ‘to hurt, harm’, \textit{hā{\ꞌ}ū{\ꞌ}ū} ‘to help’\footnotemark{}}\\
\tablevspace
 { L.LL.{\tiny {\db}}H{\tiny {\db}}.LL.}& 7 &  1& {\textit{matamatāika} ‘hail’}\\
\tablevspace
 { LL.{\tiny {\db}}H{\tiny {\db}}.LL.LL.}& 8 &  1& {\textit{ha{\ꞌ}amāuruuru} ‘to thank’}\\
\tablevspace
 { H{\tiny {\db}}.LL.{\tiny {\db}}H{\tiny {\db}}.LL.}& 8 &  1& {\textit{tōuamāmari} ‘yellow’}\\
\tablevspace
 { H{\tiny {\db}}.{\tiny {\db}}H{\tiny {\db}}.{\tiny {\db}}H{\tiny {\db}}.LL.}& 8 &  4& {\textit{māmārū{\ꞌ}au} ‘grandmother’, \textit{mōrī{\ꞌ}ārahu} ‘kerosene’} \\
\lspbottomrule
\end{tabularx}

\caption{Metrical word structures (cont.)}
% \todo[inline]{All columns should be top-aligned instead of center.}
\label{tab:7b}
\end{table}

\footnotetext{Most of these have identical final and penultimate syllables\is{Syllable}, but for none of them there is clear evidence that they are reduplications\is{Reduplication}.}

The table shows that words containing up to six morae\is{Mora} are common. Longer words are rare; in fact, all 7–8 mora words are either reduplications\is{Reduplication} or compounds (e.g. \textit{tōuamāmari} {\textless} \textit{tōua} ‘yolk’ + \textit{māmari} ‘egg’).

\clearpage 
Below are examples of the metrical\is{Metrical structure} structure of \textit{hānautama} ‘pregnant’, \textit{keretū} ‘pumice’ and \textit{mauku} ‘grass’. (Feet are indicated by round brackets; the strong mora\is{Mora} within the foot is marked as x, the weak mora\is{Mora} as a dot.)

\resizebox{.95\textwidth}{!}{
\begin{forest}
[,phantom, for tree={calign=first,align=center}
  [,phantom
    [,phantom[~\\feet:,tier=metric]]
  ]
  [\syl
    [\mor[ha\\(x,name=ha,tier=metric]]
    [\mor,name=mor2[\\.), no edge,tier=metric]]
  ]
  [\syl
    [\mor[na\\(x,tier=metric]]
  ]
  [\syl
    [\mor[u\\.),tier=metric]]
  ]
  [\syl
    [\mor[ta\\(x,tier=metric]]
  ]
  [\syl
    [\mor[ma\\.),tier=metric]]
  ]
  [,phantom[,phantom]]
  [,phantom[,phantom]]
  [\syl
    [\mor[ke\\(x,tier=metric]]
  ]
  [\syl
    [\mor[re\\.),tier=metric]]
  ]
  [\syl
    [\mor[tu\\(x,name=tu,tier=metric]]
    [\mor,name=mor4[\\.),no edge,tier=metric]]
  ]
  [,phantom[,phantom]]
  [,phantom[,phantom]]
  [\syl
    [\mor[ma\\(x),tier=metric]]
  ]
  [\syl
    [\mor[u\\(x,tier=metric]]
  ]
  [\syl
    [\mor[ku\\.),tier=metric]]
  ]
]
\draw (mor2.south) -- (ha.north);
\draw (mor4.south) -- (tu.north);
\end{forest}
}
% \todo[inline]{The trees above were reduced to text width. The alternative would be to tell forest to use less horizontal space between the syllables - I just don't know how.}

The absence of certain structures follows straightforwardly from these constraints. For example, the \ili{Tahitian} word \textit{tāne} ‘man, male’ was borrowed into Rapa Nui, but with shortening of the first vowel: \textit{tane}. The form \textit{*tāne} (with the non-attested pattern *HL) would involve either a degenerate\is{Foot!degenerate} foot\is{Foot} at the end of the word (violating constraint 4), a foot spanning a syllable\is{Syllable} boundary (violating constraint 3), or an unparsed syllable\is{Syllable} (violating constraint 2). These alternatives are illustrated below.\footnote{\label{fn:48}\citet[171]{Finney1999} notes that in most Polynesian languages, words can end in V\textsubscript{1}V\textsubscript{1}CV. In that case, speakers “tend to treat the antepenult and the penult as a foot, a single long syllable\is{Syllable}, even though that violates the normal [process of right-to-left foot formation]”. We may conclude that Rapa Nui differs from other Polynesian languages in that the constraints on foot formation impose absolute constraints on word formation.} 

\begin{forest}
[,phantom, for tree={calign=first,align=center}
  [,phantom
    [,phantom[~\\feet:,tier=metric]]
  ]
  [{*\syl}
    [\mor[ta\\(x,name=ta1,tier=metric]]
    [\mor,name=mor2[\\.),no edge]]
  ]
  [\syl
    [\mor[ne\\(x)]]
  ]
  [,phantom[,phantom]]
  [,phantom[,phantom]]
  [{*\syl}
    [\mor[ta\\(x),name=ta2,tier=metric]]
    [\mor,name=mor4[\\(x,no edge]]
  ]
  [\syl
    [\mor[ne\\.)]]
  ]
  [,phantom[,phantom]]
  [,phantom[,phantom]]
  [{*\syl}
    [\mor[ta\\(x,name=ta3,tier=metric]]
    [\mor,name=mor6[\\.),no edge]]
  ]
  [\syl
    [\mor[ne\\]]
  ]
\draw (mor2.south) -- (ta1.north);
\draw (mor4.south) -- (ta2.north);
\draw (mor6.south) -- (ta3.north);
]
\end{forest}

We may conclude that the prosodic shape of words is determined by a set of non-violable metrical\is{Metrical structure} constraints. Once these constraints are established, a number of other issues can be addressed: minimal words, vowel sequences, and the frequency of metrical patterns. These are discussed in the following sections.

\subsubsection{Minimal words}\label{sec:2.3.2.2}
\largerpage
Content words minimally consist of one bimoraic foot\is{Foot}: \textit{pō} ‘night’, \textit{kai} ‘to eat’, \textit{hare} ‘house’, \textit{oho} ‘to go’. Postnuclear particles are minimally bimoraic as well (in fact, most of these are bimoraic\is{Mora}): \textit{nō} ‘just’, \textit{era} ‘distal’, \textit{mai} ‘hither’. The same is true for particles occurring in isolation, such as \textit{{\ꞌ}ina} ‘\textsc{neg}’ and \textit{{\ꞌ}ī} ‘\textsc{imm}’. Only prenuclear particles\is{Particle!prenuclear} may be monomoraic: \textit{te} ‘\textsc{art}’, \textit{e} ‘\textsc{ipfv}’, \textit{ki} ‘to’.\footnote{\label{fn:49}This means that prenuclear particles are not independent phonological words, which can be taken as evidence to analyse them as clitics. Another reason to consider them as clitics is the fact that they occupy a fixed (i.e. initial) position in the phrase, without being attached to a single category: a prenuclear particle precedes whatever comes next in the phrase. In the end, whether or not they are to be considered clitics may be a matter of terminological preference. (Cf. \citealt[23]{Payne1997}: it is uncommon to use the term clitic for elements such as adpositions and tense/aspect markers.)}

\subsubsection{Vowel sequences}\label{sec:2.3.2.3}

In the previous section, several reasons were mentioned to analyse sequences of two non-identical vowels as disyllabic sequences rather than diphthongs\is{Diphthong}. The conditions on metrical\is{Metrical structure} structure provide another argument for a disyllabic analysis. As pointed out above, (C)Vː(C)V words such as \textit{*tāne} do not occur in Rapa Nui, a fact which can be explained by metrical\is{Metrical structure} constraints ruling out *HL patterns. On the other hand however, (C)V\textsubscript{1}V\textsubscript{2}(C)V words are common: \textit{mauku} ‘grass’, \textit{hau{\ꞌ}a} ‘smell’, \textit{maika} ‘banana’, \textit{koia} ‘with’, \textit{paihi} ‘torn’, \textit{taote} ‘doctor’, et cetera. Now if \textit{au, ai, oi} and \textit{ao} would be monosyllabic (i.e. diphthongs)\is{Diphthong}, these words would have an HL pattern, and it would be unclear why these words are possible while \textit{tāne} is not. On the other hand, if these sequences are disyllabic, these words have a LLL pattern just like \textit{makenu} ‘to move’ and \textit{poreko} ‘to be born’, a pattern which is metrically well-formed and which is in fact very common.\footnote{\label{fn:50}Following a similar reasoning, \citet{AndersonOtsuka2006} conclude that long vowels\is{Vowel length} in \ili{Tongan} must be disyllabic, as they may span a foot boundary.} 

\subsubsection{Common and uncommon patterns}\label{sec:2.3.2.4}

For words consisting of 1, 2, 3, 4 or 6 morae\is{Mora}, all possible patterns are attested. (Longer words are very rare overall.) Even so, some patterns are more common than others.

In general, light syllables\is{Syllable} are more common than heavy syllables\is{Syllable}. Patterns with an LL foot in a given position in the word are much more common than patterns with a H foot in the same position, e.g. L.LL (1010) versus L.H (54); LL.LL (1290) versus LL.H (70) or H.LL (453). The only exception is H.LL.LL (130), which is more common than LL.LL.LL (35).

H syllables\is{Syllable} are more common word-initially than word-finally. Not counting monosyllabic H words, there are 686 words with initial H, against 228 words with final H. Medial H is also relatively uncommon; it mainly occurs when the preceding or following syllable\is{Syllable} is also H. Of all 329 three- and four-foot words (the only ones in which medial H is possible), 164 have initial H, 47 have one or two medial H, while 35 have final H.

The patterns listed above, as well as etymological data, suggest that there is a tendency to avoid degenerate\is{Foot!degenerate} feet.\footnote{\label{fn:51}\citet[17]{Englert1978} already notices the tendency to lengthen\is{Vowel!lengthening} antepenultimate vowels. Cf. \citet[399]{Kager1995}: languages employ various strategies to avoid degenerate\is{Foot!degenerate} feet, such as lengthening\is{Vowel!lengthening} and reparsing.} Lengthening an initial vowel turns a degenerate\is{Foot!degenerate} foot\is{Foot} into a complete bimoraic one. If the reconstructed forms in Pollex (\citealt{GreenhillClark2011}) are correct, the initial syllable\is{Syllable} was lengthened in words such as \textit{\mbox{hō{\ꞌ}ou}} ‘new’ (\is{Proto-Polynesian}PPN \textit{\mbox{*fo{\ꞌ}ou}}), \textit{{\ꞌ}ūnahi} ‘fish scale’ (\is{Proto-Polynesian}PPN \textit{*{\ꞌ}unafi}), \textit{hōhonu} ‘deep’ (\is{Eastern Polynesian}PEP \textit{*fofonu}) and \textit{pū{\ꞌ}oko} ‘head’ (by metathesis from PEP \textit{*upoko}). In longer words this tendency is even stronger: there are more H.LL.LL words in the lexicon (130) than L.LL.LL (101).\footnote{\label{fn:52}In actual language use the difference is even more marked: many of the L.LL.LL words in the lexicon are borrowings\is{Borrowing}, some of which only occur in older sources such as \citet{Roussel1908} and which are no longer (or never were) in use.} Certain reduplication\is{Reduplication} patterns show a tendency to lengthen L.LL.LL to H.LL.LL (\sectref{sec:2.6.1.2}). On the other hand, the pressure toward whole feet is not sufficiently strong to prevent the occurrence of many hundreds of LLL words; in fact, this is the third most common pattern overall.

Another issue related to metrical\is{Metrical structure} structure is stress\is{Stress} assignment. This will be discussed in \sectref{sec:2.4.1}.\is{Metrical structure|)}

\largerpage
\subsection{Cooccurrence restrictions}\label{sec:2.3.3}
\is{Cooccurrence restrictions}\subsubsection{Between vowels}\label{sec:2.3.3.1}
\is{Cooccurrence restrictions|(}
As discussed above, all sequences of non-identical vowels are possible. This is illustrated in \tabref{tab:8}.

\begin{table}
\begin{tabularx}{\textwidth}{p{1.4cm}p{1.8cm}p{1.5cm}p{1.8cm}p{1.8cm}p{2cm}}
\lsptoprule

 \backslashbox{V1}{V2} &  { a}& { e}& { i}& { o}& { u}\\
% \hhline{~------}
{ a}   & (\textit{hā} \newline ‘breathe’) & \textit{hae} \newline ‘savage’ & \textit{hai} \newline ‘\textsc{ins}’ & \textit{hao} \newline ‘to plant’ & \textit{hau} \newline ‘more’\\
\tablevspace
{ e}   & \textit{mea} \newline ‘gill’ & (\textit{hē} \newline ‘\textsc{cq}’) & \textit{hei} \newline ‘headdress’ & \textit{heo} \newline ‘collarbone’ & \textit{heu} \newline ‘half-breed’\\
\tablevspace
{ i}   & \textit{hia} \newline‘how  many’ & \textit{hiero} \newline ‘radiance’ & (\textit{hī} \newline ‘to fish’) & \textit{hio} \newline ‘strong’ & \textit{hiu} \newline ‘moth larva’\\
\tablevspace
{ o}   & \textit{hoa} \newline ‘friend’ & \textit{hoe} \newline ‘paddle’ & \textit{hoi} \newline ‘horse’ & (\textit{hō} \newline ‘\textsc{dub}’) & \textit{hou} \newline ‘to drill’\\
\tablevspace
{ u}   & {\textit{hua} \newline ‘bear fruit’} & {\textit{hue} \newline ‘vine’} & {\textit{hu}\textit{ī} \newline ‘lineage’} & {\textit{uo} ‘k.o. \newline  tattoo’} & {(\textit{hū} \newline ‘to burn’)}\\
\lspbottomrule
\end{tabularx}
\caption{Examples of VV sequences}
% \todo[inline]{Lefthand column (a e i o u) should be top-aligned}
\label{tab:8}
\end{table}

Not all VV sequences are equally common, however. The text corpus contains 90,700 disyllabic VV sequences; their relative frequencies are given in \tabref{tab:9}.

\begin{table} 
\begin{tabularx}{\textwidth}{XR{1.8cm}R{1.5cm}R{1.8cm}R{1.8cm}R{2cm}}
\lsptoprule

 \backslashbox{V1}{V2}& \parbox{1cm}{\centering a}& \parbox{1cm}{\centering e}& \parbox{1cm}{\centering i}& \parbox{1cm}{\centering o}& \parbox{1cm}{\centering u}\\
% \hhline{~------}
{ a}  &  1.1\%&  31.2\%&  1.8\%&  12.0\%\\
{ e}  &  1.8\%&  &  6.8\%&  0.5\%&  0.4\%\\
{ i}  &  6.4\%&  0.8\%&  &  2.2\%&  0.7\%\\
{ o}  &  4.8\%&  4.4\%&  1.7\%&  &  6.0\%\\
{ u}  &  12.5\%&  0.9\%&  4.0\%&  0.04\%& \\
\lspbottomrule
\end{tabularx}
\caption{Frequencies of VV sequences}
\label{tab:9}
\end{table}

As \tabref{tab:9} shows, sequences of a high and a low vowel in either order (\textit{ai}, \textit{au}, \textit{ia}, \textit{ua}) are much more common than those containing a mid vowel or consisting of two high vowels. The former four sequences together account for 62.1\% of the total, while the other fourteen sequences account for 37.9\%. \textit{uo} hardly occurs at all. \textit{eo} and \textit{eu} are rare, as well as \textit{iu}, \textit{ie}\footnote{\label{fn:53}Most of the occurrences of \textit{ie} are due to \ili{Spanish} influence. While \textit{ie} is quite rare in Rapa Nui words (apart from some proper names), it is very common in \ili{Spanish} and often occurs in loanwords\is{Borrowing}: \textit{fiesta, noviembre, tiene}...} and \textit{ue}. 

VVV sequences are common as well: \textit{māua} ‘we (dual excl.)’, \textit{pūai} ‘strong’, \textit{tāea} ‘to throw a lasso’, \textit{tōua} ‘egg yolk’, \textit{tūai} ‘ancient’.

\subsubsection{Between vowels and consonants}\label{sec:2.3.3.2}

Any consonant can be followed by any vowel, with one exception: the syllable\is{Syllable} \textit{vu} is extremely rare. Apart from the loanword \textit{vuto} ‘sweet’ ({\textless} Sp. \textit{dulce}), it only occurs in \textit{vuhi} (and its reduplication\is{Reduplication} \textit{vuhivuhi}) ‘to whistle’, a word not occurring in the text corpus.\footnote{\label{fn:54}\citet{DeLacy1997} reports similar restrictions in \ili{Māori}, where \textit{*wu, *wo, *whu} and \textit{*who} do not occur. In \ili{Tuvaluan}, \textit{*vu-} is unattested, while \textit{*vo-} is rare \citep[612]{Besnier2000}.} 

\textit{vo} is not very common either; it occurs in eight lexical entries, such as \textit{vovo} ‘dear girl’ and \textit{vo{\ꞌ}u} ‘to shout’.

\subsubsection{Between consonants}\label{sec:2.3.3.3}
\largerpage[2]
As discussed in \sectref{sec:2.3.1} above, consonants are always separated by a vowel; contiguous consonants do not occur. (The only exceptions occur in loanwords\is{Borrowing}, see \sectref{sec:2.5.3.1} below.) Even so, there are a few co-occurrence restrictions between consonants in adjacent syllables\is{Syllable}.\footnote{\label{fn:55}Similar restrictions operate in other Polynesian languages (see \citealt[24]{MoselHovdhaugen1992} on cooccurrence restrictions between labial and labiodental consonants; \citealt[554]{Bauer1993} on a tendency towards consonant dissimilation in adjacent syllables).}

Firstly, the co-occurrence of a homorganic nasal\is{Nasal} + stop\is{Plosive} (in that order) within a root is very rare, though not completely excluded. The co-occurrence of homorganic stop + nasal\is{Nasal} is somewhat less rare, though by no means common. The data are given in \tabref{tab:9a}.


\begin{table}[h]
\begin{tabularx}{\textwidth}{Xp{105mm}}
\lsptoprule
\textit{mVp-} & The only example is \textit{māpē} ‘kidney’.\\
% \tablevspace
\textit{pVm-} & Only in borrowings\is{Borrowing} such as \textit{pamu} ‘pump; to fumigate’ and in a few rare words such as \textit{pōmiti} ‘thunderstruck’.\\
% \tablevspace
\textit{nVt-} & Does not occur word-initially. Non-initially, the only example is \textit{{\ꞌ}onotau} ‘epoch’, as well as borrowings\is{Borrowing} such as \textit{kānato} ‘basket’ ({\textless} Sp. \textit{canasto})\\
% \tablevspace
\textit{tVn-} & Is common: \textit{tano} ‘correct’, \textit{tono} ‘push’, \textit{tunu} ‘cook’ etc.\\
% \tablevspace
\textit{ŋVk-} & Does not occur.\\
% \tablevspace
\textit{kVŋ-} & Word-initially, the only examples are \textit{koŋokoŋo} ‘to grunt’ and \textit{kuŋukuŋu} ‘hoarse’, both of which occur in one lexigraphical source only. Non\nobreakdash-initially, the only example is \textit{kokoŋo} ‘mucus’.{\rmfnm}\\
\lspbottomrule
\end{tabularx}
\caption{Nasal + plosive cooccurrence restrictions}
% \todo[inline]{Lefthand column should be top-aligned}
\label{tab:9a}
\end{table}

\footnotetext{\label{fn:56}Non-initial \textit{kVŋ-} is not uncommon in forms containing the nominaliser \textit{-ŋa}, such as \textit{pikoŋa} ‘hiding place’, but here the two consonants are separated by a morpheme break.}


\clearpage 
Secondly, \citet{Davletshin2015} mentions a constraint\is{Cooccurrence restrictions} against the occurrence of glottal\is{Glottal plosive} stops in adjacent syllables\is{Syllable} of bimoraic words.\footnote{\label{fn:57}This constraint does not operate in words having more than two morae\is{Mora}: glottals\is{Glottal plosive} in adjacent syllables\is{Syllable} occur in bisyllabic words with long vowels\is{Vowel length} (\textit{{\ꞌ}ī{\ꞌ}ī} ‘slightly spoiled (food)’, \textit{{\ꞌ}u{\ꞌ}ū} ‘to groan’), and in trisyllabic words (\textit{{\ꞌ}a{\ꞌ}aru} ‘to grab’, \textit{ha{\ꞌ}i{\ꞌ}a} ‘Malay apple’).} This is confirmed by my data: in the lexicon and the text corpus there are no words of the shape \textit{{\ꞌ}V{\ꞌ}V}.\is{Cooccurrence restrictions}
\is{Cooccurrence restrictions|)}

\section{Suprasegmentals}\label{sec:2.4}
\subsection{Stress}\label{sec:2.4.1}
\is{Stress|(}
Metrical structure as described in \sectref{sec:2.3.2} above allows a simple description of word stress\is{Stress}: the final foot\is{Foot} of the word is prominent. This results in the following pattern:

\begin{itemize}
\item 
When the final syllable\is{Syllable} of the word is long, it is stressed. Being a heavy syllable\is{Syllable}, it contains a whole foot.

\item 
When the final syllable\is{Syllable} of the word is short, the penultimate syllable\is{Syllable} is stressed. The penultimate and the final syllable constitute the final foot\is{Syllable}; as the foot is trochaic (i.e. the first mora\is{Mora} is strong), the penultimate receives stress\is{Stress}. 

\end{itemize}

The strong morae\is{Mora} of the other feet receive secondary stress\is{Stress}. This results in a rhythm of alternating strong and weak morae\is{Mora}. Some examples:

\ea
\begin{tabbing}
 xxxxxxxxxxxx \= xxxxxxxxxxxxxx \= xxxxxxxxx \kill
  \textit{pō} \> \textstyleIPA{/ˈpoː/} \> ‘night’\\
  \textit{noho} \> \textstyleIPA{/ˈnoho/} \> ‘to sit, stay’\\
 \textit{maŋō} \> \textstyleIPA{/maˈŋoː/} \> ‘shark’\\
 \textit{mauku} \> \textstyleIPA{/maˈuku/} \> ‘grass’\\
 \textit{pāpa{\ꞌ}i} \> \textstyleIPA{/ˌpaːˈpaʔi/} \> ‘to write’\\
 \textit{haŋupotu} \> \textstyleIPA{/ˌhaŋuˈpotu/} \> ‘youngest child’\\
 \textit{keretū} \> \textstyleIPA{/ˌkereˈtuː/} \> ‘pumice’\\
 \textit{hānautama} \> \textstyleIPA{/ˌhaːˌnauˈtama/} \> ‘pregnant’
\end{tabbing}
\z 
Not all non-final feet have the same level of stress\is{Stress}. For example, when the initial foot is degenerate\is{Foot!degenerate}, either the initial or the second syllable\is{Syllable} may be slightly more prominent:

\ea
\begin{tabbing}
 xxxxxxxxxxxx \= xxxxxxxxxxxxxx \= xxxxxxxxx \kill
 \textit{oromatu{\ꞌ}a} \> \textstyleIPA{/ˌoromaˈtuʔa/} \> ‘priest’\\
 \textit{vanavanaŋa}  \> \textstyleIPA{/ˌvanavaˈnaŋa/} \> ‘to chat’\\
 \textit{vanavanaŋa} \> \textstyleIPA{/vaˌnavaˈnaŋa/} \> ‘to chat’
\end{tabbing}
\z
More study is needed to determine which factors determine levels of lower-order stress\is{Stress}.

In connected speech, \textsc{phrase stress}\is{Stress} is more conspicuous than word stress\is{Stress}. Stress\is{Stress} is assigned at the level of the prosodic phrase\is{Prosodic phrase}, according to the same rule as word stress\is{Stress}: the final foot of the phrase is prominent. In other words, stress\is{Stress} falls on the phrase-final syllable\is{Syllable} if it is long, and on the penultimate syllable\is{Syllable} otherwise. Prosodic phrase breaks usually coincide with breaks between syntactic constituents, but not all syntactic phrases constitute a separate prosodic phrase\is{Prosodic phrase}.

In the examples below, prosodic phrase\is{Prosodic phrase} breaks are represented by {\textbar}. 

\ea\label{ex:2.2}
\gll E ˌai rō ˈ{\ꞌ}ā {\textup{\textbar}} e ˌtahi ˌoromaˈtu{\ꞌ}a {\textup{\textbar}} te ˌ{\ꞌ}īˌŋoa  ko ˌTahaˈria.\\
\textsc{ipfv} exist \textsc{emph} \textsc{cont} ~ \textsc{num} one priest ~ \textsc{art} name \textsc{prom} Zechariah\\
[ʔeˌairoːˈʔaː  ʔeˌtahiˌoromaˈtuʔa  teˌʔiːˌŋoakoˌtahaˈria]\\

\glt 
‘There was a priest named Zechariah.’ \textstyleExampleref{[R630-01.002]}
\z

\ea\label{ex:2.3}
\gll E ˌha{\ꞌ}aˌtura  rō ˈ{\ꞌ}ana {\textup{\textbar}} ki tāˌ{\ꞌ}ana ˈroŋo {\textup{\textbar}} ˌ{\ꞌ}e ki tāˌ{\ꞌ}ana ˌhaka ˌtere ˈiŋa.\\
\textsc{ipfv} respect \textsc{emph} \textsc{cont} ~  to \textsc{poss.3sg.a} message ~ and to \textsc{poss.3sg.a} \textsc{caus} run \textsc{nmlz}\\

[ʔeˌhaʔaˌturaroːˈʔana  kitaːˌʔanaˈroŋo  ˌʔekitaːˌʔanaˌhakaˌtereˈiŋa]\\

\glt 
‘They obeyed his word and his law.’ \textstyleExampleref{[R630-01.002]}
\z
% \todo[inline]{the stress marks are very close to the glottal. -> keep this in mind, we will address this during final typesetting. http://tex.stackexchange.com/questions/74353/what-commands-are-there-for-horizontal-spacing}

\ea\label{ex:2.4}
\gll ˈMatu {\textup{\textbar}} ki ˌoho tāˌtou ki Vēˈrene {\textup{\textbar}} ki ˌu{\ꞌ}i i te ˌme{\ꞌ}e ˌhaka ˌ{\ꞌ}ite mai ˈena.\\
come\_on ~ \textsc{hort} go \textsc{1pl.incl} to Bethlehem ~ to look \textsc{acc} \textsc{art} thing \textsc{caus} know hither \textsc{med}\\

[ˈmatu  kiˌohotaːˌtoukiveˈrene kiˌuʔiteˌmeʔeˌhakaˌʔitemaiˈena]\\

\glt 
‘Come, let’s go to Bethlehem, to see the thing announced (to us).’ \textstyleExampleref{[R630-02.008]}
\z

\ea\label{ex:2.5}
\gll ˌKi a ˌkōˈrua, {\textup{\textbar}} ki te ˌnu{\ꞌ}u ˌhakaˌroŋo ˈmai, {\textup{\textbar}} ˌ{\ꞌ}ī a ˌau he ˌkī ˈatu...\\
to \textsc{prop} \textsc{2pl} ~ to \textsc{art} people listen hither ~ \textsc{imm} \textsc{prop} \textsc{1sg} \textsc{ntr} say away\\

[ˌkiaˌkoːˈrua  kiteˌnuʔuˌhakaˌroŋoˈmai  ˌʔiːaˌauheˌkiːˈatu]\\

\glt
‘To you, to the people listening, I tell you...’ \textstyleExampleref{[R630-04.063]}
\z

As these examples show, primary stress\is{Stress} always falls on the final foot of the prosodic phrase\is{Prosodic phrase}, whether this is a lexeme (\textit{oromatu}\textit{{\ꞌ}}\textit{a} in \REF{ex:2.2}), a continuous marker (\textit{{\ꞌ}ā} in \REF{ex:2.2}), a nominaliser (\textit{iŋa} in \REF{ex:2.3}), a postnuclear demonstrative (\textit{ena} in \REF{ex:2.4}), or a directional\is{Directional} (\textit{mai} in \REF{ex:2.5}). All other feet potentially receive secondary stress\is{Stress}. However, secondary stress\is{Stress} is not always conspicuous, especially on or near long vowels\is{Vowel length}, when two contiguous syllables\is{Syllable} both contain a strong mora\is{Mora}.
Not all secondary stresses are equally strong, though this has not been indicated in the examples above. A more refined analysis is needed to determine how different levels of non-primary stress\is{Stress} are assigned. Two factors that seem to play a role are:

%\setcounter{listWWviiiNumcleveli}{0}
\begin{itemize}
\item 
semantic or pragmatic prominence. The nucleus of the phrase (often the only lexical word) tends to get relatively heavy secondary stress\is{Stress}, especially the syllable\is{Syllable} that would be stressed according to the word stress\is{Stress} rules; e.g. in \textit{ha{\ꞌ}atura} in \REF{ex:2.3} and \textit{haka {\ꞌ}ite} in \REF{ex:2.4}, the second foot receives more stress\is{Stress} than the first. The deictic particle \is{Deixis}\textit{{\ꞌ}ī} in \REF{ex:2.5} is relatively prominent as well.

\item 
linear distance. Feet immediately preceding the main phrase stress\is{Stress} are not heavily stressed. This means that the stressed syllable\is{Syllable} of content words may not receive a high degree of stress\is{Stress} if it is immediately followed by the phrase stress: in \textit{hakaroŋo} in \REF{ex:2.5}, the initial syllable\is{Syllable} receives more stress\is{Stress} than the penultimate one, despite the word stress\is{Stress} on the latter.\is{Stress}
\is{Stress|)}

\end{itemize}
\subsection{Intonation}\label{sec:2.4.2}
\is{Intonation|(}
This section describes a number of intonation patterns in declarative and interrogative clauses. Examples are given from basic sentences, i.e. monoclausal sentences with standard constituent order. A full treatment of intonation would require precise acoustic analysis and is outside the scope of this grammar.\footnote{\label{fn:58}Intonation\is{Intonation} in imperative\is{Imperative} clauses is not illustrated. Imperative clauses tend to show a high rise, followed by a gradual decline. This means that the intonation pattern is superficially identical to the intonation of declarative clauses. A more precise analysis could reveal subtle differences between declarative and imperative intonation, e.g. in the shape or timing of the rise.} 

\subsubsection{In declarative clauses}\label{sec:2.4.2.1}

Intonation\is{Intonation} in declarative clauses is characterised by a peak on the stressed syllable\is{Syllable} of the predicate. Subsequently, the pitch may gradually drop:\footnote{\label{fn:59}Intonation graphs were created using Speech Analyzer 3.1 (SIL International, 2012). In the examples, syllables bearing phrase stress are underlined.}

  
%%please move the includegraphics inside the {figure} environment
%%\includegraphics{figures/AgrammarofRapaNuiCURRENTDOC-img2.pdf}

% \todo[inline]{JPGs are all I have for the intonation graphs. The only alternative would be the intonation notation above the line (as described in the LSP guidelines) but that would be less clear.}
\ea\label{ex:2.6}
\includegraphics[scale=1]{figures/AgrammarofRapaNuiCURRENTDOC-img2.pdf}\\
\gll ~ ~ ~ ~ ~~~~~~~~~~~~ He ~~~~ aŋi ~~~~~~ \textbf{{\ꞌ}ā} ~~~~~~~ tā\nobreakdash-\nobreakdash-\nobreakdash---\nobreakdash-\nobreakdash-\nobreakdash-\nobreakdash-\nobreakdash-\nobreakdash-{\ꞌ}au.\\
 ~ ~ ~ ~ ~ \textsc{ntr} ~ certain ~ \textsc{cont} ~ \textsc{poss.2sg.a}\\

\glt 
‘You are right.’ \textstyleExampleref{[R630-05.036]}
\z

\newpage 
\ea\label{ex:2.7}
\includegraphics[width=.9\textwidth]{figures/AgrammarofRapaNuiCURRENTDOC-img3.pdf}\\
\gll ~ ~ ~ ~ ~~~~~~~~~~~ He ~~ māere ~~~ te ~ \textbf{nu}{\ꞌ}u ~~~~~~~~~~~~~~~~~~~ {\ꞌ}i ~~ tū me{\ꞌ}e ~ \textbf{e}ra.\\
 ~ ~ ~ ~ ~ \textsc{ntr} ~ surprised ~ \textsc{art} ~ people ~ at ~ \textsc{dem} thing ~ \textsc{dist}\\

\glt
‘The people were amazed about that.’ \textstyleExampleref{[R630-07.038]}
\z
% \todo[inline]{Rescaled this graphic. For some reason, this causes the ex.no. to move from the bottom to the top of the graphic. For sake of consistency I added [scale=1] to the other graphics.}
The final constituent may show a second peak, as on \textit{poki era} in the next example:

\protectedex{ 
\ea\label{ex:2.8}
\includegraphics[width=.9\textwidth]{figures/AgrammarofRapaNuiCURRENTDOC-img5.pdf}\\
\gll ~ ~ ~ ~ ~~~~~~~~~~~~~~~~~ He ~~~ \textbf{o}ra ~~~~~~ ia ~~ tū ~~~~~ po---------ki  ~~~~~ \textbf{e}ra.\\
 ~ ~ ~ ~ ~ \textsc{ntr} ~ live ~ then ~ \textsc{dem} ~ child ~ \textsc{dist}\\

\glt
‘Then the child was alive.’ \textstyleExampleref{[R630-06.016]}
\z
}

Alternatively, the sentence may end in a high plateau. In the next example, there is a high rise on the second (stressed) syllable\is{Syllable} of \textit{pāhono}; the pitch remains on this level throughout the final syllable\is{Syllable}.\footnote{\label{fn:60}The example is from a younger male speaker. Data from a range of speakers could show if this is pattern is limited to certain age groups.} 

 
\ea\label{ex:2.9}
\includegraphics[scale=1]{figures/AgrammarofRapaNuiCURRENTDOC-img6.pdf}\\
\gll ~ ~ ~ ~ ~~~~~~~~~~~~~~ Ko ~~ tano ~ \textbf{{\ꞌ}ā} ~ ta{\ꞌ}a  pā\nobreakdash-\nobreakdash-\nobreakdash-\nobreakdash---\nobreakdash-\nobreakdash-hono.\\
 ~ ~ ~ ~ ~ \textsc{prf} ~ correct ~ \textsc{cont} ~ \textsc{poss.2sg.a}  answer\\

\glt 
‘Your answer is correct.’ \textstyleExampleref{[R630-07.015]}
\z

\subsubsection{In questions}\label{sec:2.4.2.2}

\is{Interrogative}In polar questions\is{Question!polar} (\sectref{sec:10.3.1}), there is usually a high rise on the stressed syllable of the first constituent; after that the pitch is low or falling, but on or just before the final stressed syllable\is{Syllable} the pitch quickly goes up. After a quick rise it tends to drop somewhat in post-stress\is{Stress} syllables\is{Syllable}, but not all the way back to the previous low level. 

\is{Intonation}Below are two examples. In both cases there is a rise on the first constituent. The last stressed syllable\is{Syllable} of the sentence also exhibits a sharp rise; in \REF{ex:2.10} this rise is higher than the first one, while in \REF{ex:2.11} it is somewhat less high. 

 
\ea\label{ex:2.10}
% \includegraphics[scale=1]{figures/AgrammarofRapaNuiCURRENTDOC-img7.pdf}\\
\includegraphics[width=.9\textwidth]{figures/AgrammarofRapaNuiCURRENTDOC-img7.pdf}\\
\gll ~ ~ ~ ~ ~~~~~~~~~~ ¿Ko koe ~~~~ mau ~~ \textbf{{\ꞌ}ā} ~ te  me{\ꞌ}e ~~~~ era ~~~ mo ~ tu{\ꞌ}u ~ mai? \\
 ~ ~ ~ ~ ~ \textsc{prom} \textsc{2sg} ~ really ~ \textsc{ident} ~ \textsc{art}  thing ~ \textsc{dist} ~ for ~ arrive ~ hither \\

\glt 
‘Are you really the one who was to come?’ \textstyleExampleref{[R630-05.019]}
\z

% \todo{please check the vectorized graphics as compared to the jpg}

\ea\label{ex:2.11}
\includegraphics[scale=1]{figures/AgrammarofRapaNuiCURRENTDOC-img8.pdf}\\
\gll ~ ~ ~ ~ ~~~~~~~~~ ¿{\ꞌ}Ina ~~~~~~~ \textbf{{\ꞌ}ō} ~~~~~~ kōrua ~~~~~~ me{\ꞌ}e ~~~~~~~~ mo ~~~~~~~~~~~~ \textbf{kai}? \\
 ~ ~ ~ ~ ~ ~~\textsc{neg} ~ really ~ \textsc{2pl} ~ thing ~ for ~ eat \\

\glt 
‘Don’t you have something to eat?’ \textstyleExampleref{[R630-15.041]}
\z

Content questions (\sectref{sec:10.3.2}) are characterised by a high rise on the stressed syllable\is{Syllable} of the question constituent, followed by a sharp drop. There may be a moderate rise on the final stressed syllable\is{Syllable}, but the question may also end in a low pitch. Here are two examples. Both exhibit a high rise on the stressed syllable\is{Syllable} of the interrogative constituent; \REF{ex:2.12} has a falling pitch at the end of the question, while \REF{ex:2.13} has a rise to mid-range pitch.

 
\ea\label{ex:2.12}
\includegraphics[scale=1]{figures/AgrammarofRapaNuiCURRENTDOC-img10.pdf}\\
\gll ~ ~ ~ ~ ~~~~~~~ ¿Ko ~~~ ai ~ \textbf{ma}u ~~~ te kope nei? \\
 ~ ~ ~ ~ ~ \textsc{prom} ~ who ~ really ~ \textsc{art} person \textsc{prox} \\

\glt 
‘Who is this person?’ \textstyleExampleref{[R630-05.063]}
\z

\ea\label{ex:2.13}
\includegraphics[scale=1]{figures/AgrammarofRapaNuiCURRENTDOC-img11.pdf}\\
\gll ~ ~ ~ ~ ~~~~~~~~~~~ ¿He ~~~ \textbf{a}ha ~~~~~~ ia ~~ te ~~ me{\ꞌ}e ~ mo ~~ aŋa? \\
 ~ ~ ~ ~ ~ ~\textsc{pred} ~ what ~ then ~ \textsc{art} ~ thing ~ for ~ do \\

\glt 
‘What should (we) do then?’ \textstyleExampleref{[R630-03.007]}\is{Intonation|)}\textstyleExampleref{} 
\z



\section{Phonological processes}\label{sec:2.5}

Rapa Nui is poor in morphology; as a consequence, morpho-phonological processes are uncommon. The only exception is found in the area of reduplication\is{Reduplication} (\sectref{sec:2.6.1}). This section describes phonological processes which are not morphologically conditioned. \sectref{sec:2.5.1} discusses three regular phonological processes: word-final devoicing, pre-stress lengthening and elision. 

Other phonological processes are lexically determined and result in lexical items having a different form than expected on the basis of cognates, or having two or more alternate forms; these are discussed in \sectref{sec:2.5.2}. Finally, \sectref{sec:2.5.3} deals with the (more or less regular) phonological adaptation of borrowings\is{Borrowing}.

\subsection{Regular processes}\label{sec:2.5.1}

This section discusses three regular phonological processes, i.e. processes which are not limited to certain lexical items. All three are optional. Two processes, word-final vowel devoicing\is{Devoicing} and pre-stress\is{Stress} lengthening\is{Vowel!lengthening}, take place in certain well-defined phonological contexts. For a third process, elision\is{Elision}, no specific conditions can be formulated without extensive further analysis.
 
 \largerpage
\subsubsection[Word{}-final vowel devoicing]{Word-final vowel devoicing}\label{sec:2.5.1.1}
\is{Devoicing}
Word-final short (hence unstressed) vowels are optionally devoiced after voiceless consonants. This happens especially at the end of an utterance, or at least the end of a prosodic phrase\is{Prosodic phrase}.\footnote{\label{fn:61}Vowel devoicing\is{Devoicing} occurs in other Polynesian languages as well: \ili{Māori} (\citealt[76]{Harlow2007Maori}; \citealt[556]{Bauer1993}); \ili{Niuafo’ou} (\citealt[23-25]{Tsukamoto1988}; \citealt{DeLacy2001}), \ili{Tongan} \citep[137]{Feldman1978}. The conditions under which devoicing\is{Devoicing} occurs in these languages, are different from those in Rapa Nui. In general, high vowels are affected more than low vowels.}

In a stretch of careful speech by different speakers (about 7,400 words), I counted 80 instances of word-final devoicing\is{Devoicing}. 75 of these occur at the end of a prosodic phrase\is{Prosodic phrase}, 72 of which occur at the end of an utterance. All vowels undergo the process: \textit{a} (19x), \textit{e}, (15x), \textit{i} (9x), \textit{o} (12x) and \textit{u} (25x). Devoicing occurs after all voiceless consonants: \textit{p} (2x), \textit{t} (24x), \textit{k} (17x), \textit{{\ꞌ}} (27x), \textit{f} (1x, a foreign name), \textit{h} (9x). It never occurs after voiced consonants or in non-final syllables\is{Syllable}. 

{\sloppy
Some examples: 
\textit{tahataha} [tahaˈtah%
% \hspace{-3pt}
ḁ] ‘edge’, 
%
\textit{taŋata} [taˈŋat%
% \hspace*{-3pt}
ḁ] ‘person,’ 
%
\textit{vi{\ꞌ}e} \mbox{[ˈviɁ%
% \hspace{-3pt}
e̥]} ‘wom\-an’, 
%
\textit{{\ꞌ}ariki} [Ɂaˈɾik%
% \hspace{-3pt}
i̥] ‘king’, 
%
\textit{mō{\ꞌ}oku} [moːˈɁok%
% \hspace{-3pt}
u̥] ‘for me’, 
%
\textit{oho} \mbox{[ˈoh%
̥o]} ‘go’.
}

\subsubsection[Pre{}-stress lengthening]{Pre-stress lengthening}\label{sec:2.5.1.2}
\is{Stress}\is{Vowel!lengthening}
\is{Lengthening}Occasionally, a short vowel immediately preceding the main phrase stress\is{Stress} is lengthened.\footnote{\label{fn:62}In \ili{Nukeria}, monosyllabic prenuclear particles are lengthened before a bimoraic root \citep{Davletshin2016}.} I have noticed this phenomenon 
in particular with the contiguity/imperative\is{Imperative} marker \textit{ka}: \textit{ka tanu} \textstyleIPA{[kaːˈtanu]} ‘bury’; \textit{ka pure} \textstyleIPA{[kaːˈpure]} ‘pray’, \textit{ka tu{\ꞌ}u} \textstyleIPA{[kaːˈtuɁu]} ‘arrive’. The phenomenon occurs with other particles as well: the proper article\is{a (proper article)} \textit{a} in \textit{ki~a~ia} \textstyleIPA{\mbox{[kiaːˈia]}} ‘to him’; the negator \textit{e ko} in \textit{e ko pau} \textstyleIPA{[ekoːˈpau]} ‘does not run out’, the exhortative\is{Exhortative} marker \textit{e} in \textit{e {\ꞌ}ite} \textstyleIPA{[eːˈɁite]} ‘(you must) know’. 

A possible explanation for this lengthening\is{Vowel!lengthening} is the preference for whole feet; this preference is noticeable on the word level (\sectref{sec:2.3.2}) and could be operative on the phrase level as well. This would explain why \textit{\textup{(}}\textit{ka}\textit{\textup{)}}\textit{\textsubscript{F}}\textit{~}\textit{\textup{(}}\textit{pure}\textit{\textup{)}}\textit{\textsubscript{F}} – with a degenerate\is{Foot!degenerate} initial foot – is lengthened to \textit{\textup{(}}\textit{ka:}\textit{\textup{)}}\textit{\textsubscript{F}}\textit{~}\textit{\textup{(}}\textit{pure}\textit{\textup{)}}\textit{\textsubscript{F}}. However, this does not explain the lengthening\is{Vowel!lengthening} of \textit{a} in \textit{\textup{(}}\textit{ki a}\textit{\textup{)}}\textit{\textsubscript{F}}\textit{~}\textit{\textup{(}}\textit{ia}\textit{\textup{)}}\textit{\textsubscript{F}} and \textit{ko} in \textit{\textup{(}}\textit{e ko}\textit{\textup{)}}\textit{\textsubscript{F}}\textit{~}\textit{\textup{(}}\textit{pau}\textit{\textup{)}}\textit{\textsubscript{F}}, which already have two complete feet.

\subsubsection[Elision]{Elision}\label{sec:2.5.1.3}
\is{Elision}
It is not uncommon for phonemes or whole syllables\is{Syllable} to be elided\is{Elision}. \citet[45–47]{GuerraEissmann1993} give examples of elision\is{Elision} of almost all consonants and vowels in a spoken speech corpus, such as \textit{o Rapa Nui} \textstyleIPA{[oˈrapaːi]} ‘of Rapa Nui’; \textit{{\ꞌ}ina e tahi} \textstyleIPA{[inaˈtai]} ‘not one’; \textit{me{\ꞌ}e rivariva} \textstyleIPA{[meːriːˈriːa]} ‘a good thing’. They do not indicate if any conditions on elision\is{Elision} can be formulated; answering this question would require careful analysis of a corpus of spoken texts by different speakers, including different speech styles. Such an analysis lies outside the scope of the present investigation.

\subsection{Lexicalised sound changes}\label{sec:2.5.2}

Even though there are regular sound correspondences between Rapa Nui and its protolanguages (\sectref{sec:2.2.1}–\ref{sec:2.2.2}), there is a considerable number of words for which Rapa Nui has an irregular reflex of the protoform, i.e. where sound changes have been at work. This includes numerous lexical items for which Rapa Nui has two or more alternate forms. These processes are productive: the same patterns can be observed in recent borrowings\is{Borrowing}. \citet{Davletshin2015}, who illustrates these processes in detail, points out that they should be labeled “incomplete” rather than “sporadic” or “irregular”: they are not completely unpredictable, but follow certain patterns. 

Below is an overview of these sound changes. The etymology is given where known.

\is{Metathesis|(}\subparagraph{Metathesis} Methathesis is very common in Rapa Nui (cf. \citealt[166]{DuFeuFischer1993}), mostly between onset consonants of adjacent syllables\is{Syllable}, occasionally between vowels of adjacent syllables\is{Syllable}, and very occasionally between whole syllables\is{Syllable}. It is especially common between the antepenultimate and penultimate syllable\is{Syllable} of trisyllabic words, but may occur in any pair of adjacent syllables\is{Syllable}. The consonants affected are often similar, e.g. two plosives (\textit{t}/\textit{k}), or two glottal consonants (\textit{{\ꞌ}}/\textit{h}).

\ea
Consonants (a–f = irregular reflexes, g–k = alternates within Rapa Nui): 
\ea
\textit{ha{\ꞌ}i} ‘to embrace’ {\textless} \is{Proto-Polynesian}PPN \textit{*{\ꞌ}afi} ‘to hold or carry under the arm’
\ex
\textit{ha{\ꞌ}i{\ꞌ}a} ‘Malay apple’ {\textless} Tah. \textit{{\ꞌ}ahi{\ꞌ}a}
\ex
\textit{kōtini} ‘sock, stocking’ {\textless} Eng. ‘stocking’; Thomson recorded \textit{tokin} in 1889 (\citealt[157]{Thomson1980})
\ex
\textit{ŋaro{\ꞌ}a}\is{nzaroa ‘to perceive’@ŋaro{\ꞌ}a ‘to perceive’} ‘perceive’ {\textless} \is{Proto-Polynesian}PPN \textit{*roŋo} + \textit{{}-na}
\ex
\textit{tako{\ꞌ}a}\is{tako{\ꞌ}a ‘also’} ‘also’ {\textless} \is{Proto-Polynesian}PPN \textit{*katoa}
\ex
\textit{tike{\ꞌ}a}\is{tike{\ꞌ}a ‘to see’} ‘to see’ {\textless} \is{Proto-Polynesian}PPN \textit{*kite} + \textit{\nobreakdash-a}. 
\ex
\textit{{\ꞌ}arīnā}\is{arina ‘later today’@{\ꞌ}arīnā ‘later today’} / \textit{{\ꞌ}anīrā}\is{anira ‘later today’@{\ꞌ}anīrā ‘later today’} ‘later today’
\ex
\textit{{\ꞌ}avahata} / \textit{ahavata} / \textit{ha{\ꞌ}avata} ‘box’
\ex
\textit{{\ꞌ}avai} / \textit{va{\ꞌ}ai} ‘to give’
\ex
\textit{rava}\is{rava ‘usually’} / \textit{vara} ‘usually’
\ex
\textit{\mbox{rava{\ꞌ}a}}\is{rava{\ꞌ}a ‘to obtain’} / \textit{vara{\ꞌ}a}\is{vara{\ꞌ}a ‘to obtain’} ‘to obtain’ ({\textless} Tah. \textit{roa{\ꞌ}a})
\z
\z
\ea
Vowels: 
\ea
\textit{hariu} / \textit{harui} ‘to turn’
\ex
\textit{nokinoki} / \textit{nikoniko} ‘to meander’
\z
\z
\ea
Whole syllables\is{Syllable}: 
\ea
\textit{kia}{}-\textit{kia} ‘seagull sp.’ {\textless} \is{Proto-Polynesian}PPN \textit{*}\textit{aki}{}-\textit{aki}
\z
\z
\ea
\is{Metathesis}Sometimes the pattern is more intricate: 
\ea
\textit{ta{\ꞌ}oraha} {\textless} PNP \textit{*}\textit{tafola{\ꞌ}a} shows metathesis between \textit{{\ꞌ}} and *\textit{f} in non-adjacent syllables\is{Syllable}
\ex
\textit{hōŋa{\ꞌ}a} ‘nest’ {\textless} \is{Proto-Polynesian}PPN \textit{*ofaŋa} (Ø C\textsubscript{1} C\textsubscript{2} > C\textsubscript{1} C\textsubscript{2} \textit{{\ꞌ}}); cf. \is{Central-Eastern Polynesian}PCE *\textit{kōfaŋa}
\z
\z
\is{Metathesis|)}
\subparagraph{Vowel changes} Vowel changes are common. Most of these occur either in \ili{Tahitian} borrowings\is{Tahitian influence} or as variants alongside the original form. Most of these involve a single degree of height (\textit{a/e}, \textit{a/o}, \textit{e/i}, \textit{o/u}), but other alternations occur as well.

\newpage 
\ea
\textit{a/e}: 
\ea
\textit{hatuke / hetuke} ‘sea-urchin’ ({\textless} \is{Eastern Polynesian}PEP \textit{*fatuke})
\ex
\textit{māria} ‘calm (sea)’ {\textless} \is{Proto-Polynesian}PPN \textit{*mālie}; Thomson recorded \textit{marie} in 1889 (\citealt[155]{Thomson1980})
\ex
\textit{taupe{\ꞌ}a} ‘porch’ {\textless} Tah. \textit{taupe{\ꞌ}e}
\z
\z
\ea
\textit{a/o}: 
\ea
\textit{{\ꞌ}auhau} / \textit{{\ꞌ}ouhou} ‘to pay’ ({\textless} Tah. \textit{{\ꞌ}auhau})
\ex
\textit{kora{\ꞌ}iti}\is{koro{\ꞌ}iti ‘slowly, softly’} / \textit{koro{\ꞌ}iti} ‘slowly; softly’
\ex
\textit{\mbox{rava{\ꞌ}a}}\is{rava{\ꞌ}a ‘to obtain’} / \textit{rova{\ꞌ}a}\is{rova{\ꞌ}a ‘to obtain’} ‘to obtain’ ({\textless} Tah. \textit{roa{\ꞌ}a})
\z
\z
\ea
\textit{e/i}: 
\ea
\textit{eŋo}{}-\textit{eŋo} / \textit{iŋo}{}-\textit{iŋo} ‘dirty’
\ex
\textit{pā{\ꞌ}eŋa / pā{\ꞌ}iŋa} ‘side’
\ex
\textit{pē{\ꞌ}iku} / \textit{pī{\ꞌ}iku} ‘sugarcane fibers’
\z
\z
\ea
\textit{o/u}: 
\ea
\textit{kāhui} / \textit{kāhoi} ‘bunch’ ({\textless} \is{Eastern Polynesian}PEP \textit{*kāfui})
\ex
\textit{ku} / \textit{ko}\is{ko V {\ꞌ}ā (perfect aspect)} ‘\textsc{prf}’ ({\textless} \is{Proto-Polynesian}PPN \textit{*kua})
\ex
\textit{tautoru} ‘to help’ {\textless} Tah. \textit{tauturu}
\z
\z
\ea
\textit{i/u}: 
\ea
\textit{miritoni} / \textit{miritonu} ‘seaweed sp.’
\ex
\textit{rīpoi} \textit{/} \textit{rīpou} ‘well made’ 
\ex
\textit{pō{\ꞌ}iri} / \textit{pō{\ꞌ}uri} ‘to get dark’ (see 4 below)
\z
\z
\ea
\textit{a/i}: 
\ea
\textit{take{\ꞌ}a} \textit{/} \textit{tike{\ꞌ}a} ‘to see’ ({\textless} PPN \textit{*kite} + \textit{{}-{\ꞌ}a})
\z
\z
 
\subparagraph{The liquid \textit{r}} This consonant alternates with either a glottal\is{Glottal plosive} or zero in a number of words.\footnote{\label{fn:63}In \ili{Marquesan}, \textit{r} > \textit{{\ꞌ}} is a regular – though not exceptionless – change \citep{Clark2000R}.} 
\ea
glottal\is{Glottal plosive}/\textit{r}: especially in final syllables.
\ea
\textit{kio{\ꞌ}e} ‘rat’ ({\textless} PNP \textit{*kiole})
\ex
\textit{tike{\ꞌ}a}\is{tike{\ꞌ}a ‘to see’} / \textit{tikera} ‘to see’ ({\textless} \is{Proto-Polynesian}PPN \textit{*kite} + \textit{{}-{\ꞌ}a})
\ex
\textit{ŋoriŋori} / \textit{ŋo{\ꞌ}iŋo{\ꞌ}i} ‘tiny’
\ex
\textit{hatu{\ꞌ}a} / \textit{hatura} ‘cinch, belt’ ({\textless} \is{Eastern Polynesian}PEP \textit{*fātu{\ꞌ}a})
\z
\z

\newpage 
\ea
Ø/\textit{r}: 
\ea
\textit{emu} ‘to drown’ {\textless} \is{Proto-Polynesian}PPN \textit{*lemo}
\ex
\textit{{\ꞌ}ōhiohio} / \textit{{\ꞌ}ōhirohiro} ‘whirlwind’ (cf. \is{Proto-Polynesian}PPN \textit{*siosio})
\z
\z
As these examples show, in those cases where the etymology is known, the \textit{r} is usually – but not always – secondary.

\subparagraph{Glottals} The glottal plosive is\is{Glottal plosive} sometimes added, occasionally deleted.

\ea
Added glottals\is{Glottal plosive}: 
\ea
\textit{ka{\ꞌ}ika{\ꞌ}i} ‘sharp’ {\textless} PNP \textit{*kai}\footnote{\label{fn:64}This word does not have a glottal\is{Glottal plosive} in other glottal\is{Glottal plosive}{}-preserving languages. The same is true for \is{Proto-Polynesian}PPN \textit{*osi} and \textit{*pao} below.} 
\ex
\textit{{\ꞌ}ohi} ‘stem’ {\textless} \is{Proto-Polynesian}PPN \textit{*osi}
\ex
\textit{pa{\ꞌ}o} ‘to chop’ {\textless} PPN \textit{*pao}
\ex
\textit{ha{\ꞌ}ata{\ꞌ}ahinu} / \textit{ha{\ꞌ}atāhinu} ‘to administer the last rites’ ({\textless} Tah. \textit{fa{\ꞌ}atāhinu})
\ex
\textit{pō{\ꞌ}iri} ‘darkness’ is probably a borrowing from Tah. \textit{pōiri}, with an inserted glottal\is{Glottal plosive} by analogy of the synonym \textit{pō{\ꞌ}uri} ({\textless} \is{Proto-Polynesian}PPN \textit{*pō{\ꞌ}uli}).
\ex
\textit{ta{\ꞌ}utini}\is{tautini ‘thousand’@ta{\ꞌ}utini ‘thousand’} {\textless} Tah. \textit{tauatini}, with glottal\is{Glottal plosive} inserted possibly by analogy of \textit{\mbox{ta{\ꞌ}u}} ‘year’
\z
\z
\ea
In some words, a glottal\is{Glottal plosive} plosive\is{Glottal plosive} reflects PPN \textit{*h}: 
\ea
\textit{maŋeo / maŋe{\ꞌ}o} ‘sour, bitter’ {\textless} \is{Proto-Polynesian}PPN \textit{*maŋeho}\footnote{\label{fn:65}\is{Proto-Polynesian}PPN \textit{*h} was lost in most languages. In the case of \textit{*maŋeho} it was not preserved in any other \is{Eastern Polynesian}EP language, so the PEP form may have been \textit{*maŋeo}. Interestingly, the \ili{Hawaiian} reflex is \textit{mane{\ꞌ}o}, with a glottal\is{Glottal plosive} as in Rapa Nui.}
\ex
\textit{{\ꞌ}īŋoa} ‘name’ {\textless} PPN \textit{*hiŋoa}
\ex
\textit{{\ꞌ}ai}/\textit{ai} ‘who’ {\textless} \is{Proto-Polynesian}PPN \textit{*hai} (see Footnote \ref{fn:486} on p.~\pageref{fn:486})
\ex
\textit{{\ꞌ}aŋahuru}\is{anzahuru ‘ten’@{\ꞌ}aŋahuru ‘ten’} ‘ten’ {\textless} PPN \textit{*haŋafulu}
\z
\z
\ea
Deleted glottals\is{Glottal plosive}: 
\ea
benefactive prepositions \textit{mo}\is{mo (benefactive prep.)} ({\textless} \is{Proto-Polynesian}PPN \textit{*mo{\ꞌ}o}) and \textit{mā}\is{ma (benefactive prep.)@mā (benefactive prep.)} ({\textless} \is{Proto-Polynesian}PPN \textit{*ma{\ꞌ}a}); the glottal\is{Glottal plosive} was retained in the pronominal forms \textit{mō{\ꞌ}oku, mā{\ꞌ}ana} etc (\sectref{sec:4.2.3}).
\z
\z
\ea
Glottal elision\is{Elision} is especially common in borrowings\is{Borrowing} from \ili{Tahitian}: 
\ea
\textit{pē} ‘gone’ {\textless} Tah. \textit{pe{\ꞌ}e}
\ex
\textit{hāpī} ‘to learn’ {\textless} Tah. \textit{ha{\ꞌ}api{\ꞌ}i}
\ex
\textit{ha{\ꞌ}amaitai} ‘to bless’ {\textless} Tah. \textit{ha{\ꞌ}amaita{\ꞌ}i} 
\z
\z

\subparagraph{The consonant \textit{h}} In a few cases, \textit{h} alternates with zero: 
\ea
\ea
\textit{aŋa} ‘to make, do, work’ {\textless} \is{Proto-Polynesian}PPN \textit{*saŋa} (the regular reflex would be \textit{*haŋa})
\ex
\textit{ia}\is{ia ‘not yet’} \textit{/ hia}\is{hia ‘not yet’} ‘yet’
\z
\z

\newpage 
\subparagraph{Nasal consonants} Some words exhibit shifts between different nasal\is{Nasal} consonants, mostly between \textit{n} and \textit{ŋ}:\footnote{\label{fn:66}\citet[10]{Blixen1972} notices a few cases of \textit{n} > \textit{ŋ} after \textit{i}, though none of them are certain, e.g. \textit{mahiŋo} ‘people with common bond’ {\textasciitilde} \ili{Tongan} \textit{mahino} ‘distinguished’.}
\ea
\ea
\textit{{\ꞌ}aŋa-} ‘recent past’ {\textless} \is{Proto-Polynesian}PPN \textit{*{\ꞌ}ana-} (\sectref{sec:3.6.4})
\ex
\textit{tiŋa{\ꞌ}i} ‘to kill’ {\textless} PNP \textit{*tina{\ꞌ}i}
\ex
\textit{tumu} / \textit{tuŋu} ‘cough’ ({\textless} \is{Eastern Polynesian}PEP *\textit{tuŋu})
\ex
\textit{norinori} / \textit{ŋoriŋori} ‘tiny’
\ex
\textit{nako} ‘fat, marrow’ {\textless} PPN \textit{*ŋako}
\ex
\textit{kona} ‘place’ {\textless} \is{Proto-Polynesian}PPN \textit{*koŋa} ‘fragment, part, place’\footnote{\label{fn:67}Notice that \textit{kona}, with \textit{n} rather than \textit{ŋ}, is also found in \ili{Mangarevan} (‘bed; dwelling’); cf. also PNP \textit{*kona} ‘nook, corner’.}
\z
\z

In the last two examples, \textit{ŋ} dissimilated to \textit{n} in the vicinity of \textit{k}.

\subparagraph{Monophthongisation} A number of particles exhibit monophthongisation of a VV cluster, resulting in a single short or long vowel:

\ea
\ea
\textit{ku}\is{ku} ‘\textsc{prf}’ {\textless} PPN \textit{*kua}
\ex
\textit{nō}\is{no ‘just’@nō ‘just’} ‘just’ {\textless} \is{Proto-Polynesian}PPN \textit{*noa}
\ex
\textit{rō}\is{ro (emphatic marker)@rō (emphatic marker)} ‘\textsc{emph}’ {\textless} PPN \textit{*roa}
\ex
\textit{hē}\is{he (content question marker)@hē (content question marker)} ‘\textsc{cq}’ {\textless} \is{Proto-Polynesian}PPN \textit{*hea} (see Footnote \ref{fn:490} on p.~\pageref{fn:490})
\ex
\textit{tū}\is{tu (demonstrative determiner)@tū (demonstrative determiner)} ‘demonstrative’ {\textless} older Rapa Nui \textit{tou} {\textless} \textit{tau} (\sectref{sec:4.6.2.1})
\ex
\textit{ki}\is{ki (preverbal)} ‘purpose marker’ {\textless} \is{Proto-Polynesian}PPN \textit{*kia} (\sectref{sec:11.5.3})
\ex
Another possible example is \textit{{\ꞌ}o}\is{o ‘lest’@{\ꞌ}o ‘lest’} ‘lest’ {\textless}? \is{Proto-Polynesian}PPN \textit{*{\ꞌ}aua} ‘neg. imperative\is{Imperative}‘ (see Footnote \ref{fn:527} on p.~\pageref{fn:527}).
\z
\z

The opposite process occurs in \textit{toa} ‘sugarcane’ {\textless} PPN \textit{*tō}, and \textit{roe} ‘ant’ {\textless} \is{Proto-Polynesian}PPN \textit{*rō} (though cf. \ili{Pa’umotu} \textit{rōe}).

\subparagraph{Elision} Some words with identical vowels in the penultimate and final syllables\is{Syllable} have a reduced variant in which the final consonant is elided\is{Elision}: 
\ea
\ea
\textit{kūmara / kūmā} ‘sweet potato’
\ex
\textit{rova{\ꞌ}a} / \textit{rovā} ‘to obtain’
\ex
\textit{pūtītī / putī} ‘blistered’
\ex
\textit{{\ꞌ}ana}\is{a (postverbal)@{\ꞌ}ā (postverbal)} \textit{/ {\ꞌ}ā} ‘continuity marker’
\z
\z

\subsection{The phonology of borrowings}\label{sec:2.5.3}
\is{Borrowing}
As discussed in \sectref{sec:1.4}, Rapa Nui has incorporated numerous borrowings\is{Borrowing}, especially from \ili{Tahitian} and \ili{Spanish}. It is well known that borrowings\is{Borrowing} are often adapted to the phonological structure of the recipient language, both in phoneme inventory and in phonotactics (\citealt{TentGeraghty2004}; \citealt{MatrasSakel2007}). The degree of adaptation may vary within a language,\footnote{\label{fn:68}See \citet[17]{Sakel2007}; \citet{Mosel2004} on \ili{Samoan}; \citet{Fischer2007} and \citet{Makihara2001Adaptation} on Rapa Nui.} depending for example on:

\begin{itemize}
\item 
the speech style and situation (formal or informal, oral or written);

\item 
the speaker (younger or older, more or less educated);

\item 
the status of the borrowing (spontaneous versus codified/integrated loanwords\is{Borrowing});

\item 
language attitudes (purism).\footnote{\label{fn:69}Puristic attitudes are widespread in Polynesian languages, especially where languages are perceived as endangered. This may lead to the rejection of borrowings\is{Borrowing} (see \citealt[154]{Harlow2004} on \ili{Māori}), or increased adaptation of borrowings\is{Borrowing} to the recipient language phonology. In \ili{Tahitian}, there is a tendency to remove formerly accepted non-\ili{Tahitian} consonants from European borrowings\is{Borrowing}; in \ili{Rapa}, \ili{Tahitian} borrowings are consciously adjusted to the Rapa phonological system (\citealt{KievietKieviet2006}; \citealt{Walworth2015Thesis}).}

\end{itemize}

This also happens in Rapa Nui, as illustrated below. Borrowings from \ili{Spanish}\is{Spanish influence} and from \ili{Tahitian} will be discussed separately. Rapa Nui has also incorporated some words from other European languages (\ili{English}, \ili{French}); these follow the same general principles as borrowings\is{Borrowing} from \ili{Spanish}.

\subsubsection[Borrowings from \ili{Spanish}]{Borrowings from Spanish}\label{sec:2.5.3.1}
\is{Spanish influence}
This section deals with codified borrowings\is{Borrowing}, loanwords\is{Borrowing} which are commonly used and have become part of the language. Codified borrowings\is{Borrowing} should be distinguished from spontaneous borrowings\is{Borrowing}, such as the following:

\ea\label{ex:2.14}
\gll \textbf{Cincuenta} matahiti o te hāipoipo, pa{\ꞌ}i. \\
cincuenta year of \textsc{art} marry in\_fact \\

\glt
‘The wedding was fifty years ago, in fact.’ \textstyleExampleref{[R415.498]} 
\z

Spontaneous borrowings\is{Borrowing} are instances of code switching\is{Code switching}, even though they involve just a single word (cf. \citealt{Fischer2007}). They are inserted without phonological adjustments. Codified borrowings\is{Borrowing}, on the other hand, tend to be adapted to Rapa Nui phonology to a greater or lesser degree. This adaptation does not follow hard and fast rules; the same word may be adjusted in various degrees and various ways. For example, \textit{olvida} (‘forgets’) may be pronounced as \textit{orvida}, \textit{orvira}, \textit{orovida} or \textit{orovira} \citep[195]{Makihara2001Adaptation}. This means that the adjustments described below may or may not apply in individual cases, depending on the factors mentioned above.

\paragraph{Phoneme level} On the phoneme level, no adjustments are needed in vowel quality, as both Rapa Nui and \ili{Spanish} have a five-vowel system. 

In the area of consonants, on the other hand, the two languages are considerably different. Many \ili{Spanish} consonants do not occur in Rapa Nui; these tend to be adjusted to Rapa Nui phonology. 

\subparagraph{Voiceless plosives\is{Plosive} and nasals} These consonants\is{Nasal} do not need adjustment.

\subparagraph{Voiced plosives} The treatment of voiced plosives\is{Plosive} can be explained from their pronunciation in (Chilean) \ili{Spanish}. Word-initially and after consonants, these are pronounced as plosives\is{Plosive} in Spanish. After vowels, they are pronounced as voiced fricatives; in Chilean \ili{Spanish}, these tend to be very weak: they often become approximants or almost disappear. In connected speech, word-initial voiced plosives\is{Plosive} after a vowel are pronounced as fricatives as well.

In Rapa Nui, \ili{Spanish} \textit{g} is consistently adjusted to \textit{k}; \textit{d} is usually either adjusted to \textit{r} or elided\is{Elision} (the latter only after vowels); \textit{b} is either adjusted to \textit{v} (word-initially) or elided\is{Elision} (word-initially before \textit{u}; after vowels). 
\ea
\begin{tabbing}
 xxxxxxxx \= xxxxxxxxxxxxxxxxxxxxxxxxxxxxxxxxx\kill
 \textit{g} > \textit{k} \>  \textit{karapone} ‘barn’ {\textless} Sp. \textit{galpón}; \textit{rēkaro} ‘present, gift’ {\textless} Sp. \textit{regalo} \\
 \textit{d} > \textit{r} \>  \textit{rivuho} ‘drawing’ {\textless} Sp. \textit{dibujo}; \textit{{\ꞌ}īrea} ‘idea’ {\textless} Sp. \textit{idea} \\
 \textit{d} > Ø \>  \textit{kā} ‘each’ {\textless} Sp. \textit{cada}; \textit{revaura} ‘yeast’ {\textless} Sp. \textit{levadura}; \\
 \> \textit{noverā} ‘news’ {\textless} Sp. \textit{novedad} \\
 \textit{b} > \textit{v} \>  \textit{vata} ‘dress’ {\textless} Sp. \textit{bata}; \textit{veteraka} ‘beetroot’ {\textless} Sp. \textit{betarraga} \\
 \textit{b} > Ø \>  \textit{ueno} ‘OK’ {\textless} Sp. \textit{bueno}; \textit{suerekao} ‘sub-delegate’ {\textless} Sp. \textit{subdelegado}
\end{tabbing}
\z 
\subparagraph{Fricatives} The fricative\is{Fricative} \textit{s} (also spelled \textit{c} before \textit{i/e} and \textit{z} before \textit{a/o/u}) is either maintained or becomes \textit{t}; \textit{j} (= velar fricative\is{Fricative} [x]) becomes \textit{k} or \textit{h}. \textit{f} is maintained or changed to \textit{p}.

\ea
\begin{tabbing}
 xxxxxxxx \= xxxxxxxxxxxxxxxxxxxxxxxxxxxxxxxxx\kill
 \textit{s} > \textit{s} \> \textit{resera} ‘foolishness’ {\textless} Sp. \textit{lesera}; \textit{siera} ‘sawfish’ {\textless} Sp. \textit{sierra} \\
\textit{s} > \textit{t} \>  \textit{tapatia} ‘sandal’ {\textless} Sp. \textit{zapatilla}; \textit{kamita} ‘shirt’ {\textless} Sp. \textit{camisa} \\
 \textit{j} > \textit{h} \>  \textit{rivuho} ‘drawing’ {\textless} Sp. \textit{dibujo}; \textit{hākima} ‘muzzle’ {\textless} Sp. \textit{jaquima} \\
\textit{j} > \textit{k} \>  \textit{Kāpone} ‘Japan’ {\textless} Sp. \textit{Japón}; \textit{karo} ‘jug’ {\textless} Sp. \textit{jarro} \\
 \textit{f} > \textit{f} \> \textit{asufre} ‘sulphur’ {\textless} Sp. \textit{azufre} \\
 \textit{f} > \textit{p} \>  \textit{kāpē} ‘coffee’ {\textless} Sp. \textit{café} 
\end{tabbing}
\z 
\subparagraph{Affricates} The affricate \textit{ch} ([tʃ]) becomes a plosive \textit{t} or a fricative\is{Fricative} \textit{s}:

\ea
\begin{tabbing}
 xxxxxxxx \= xxxxxxxxxxxxxxxxxxxxxxxxxxxxxxxxx\kill
 \textit{ch} > \textit{t} \> \textit{tarakī} ‘beef jerky’ {\textless} Sp. \textit{charqui}; \textit{Tire} ‘Chile’ {\textless} Sp. \textit{Chile} \\
 \textit{ch} > \textit{s}  \> \textit{supeta} ‘pacifier’ {\textless} Sp. \textit{chupeta} 
\end{tabbing}
\z 
\subparagraph{Liquids} \ili{Spanish} \textit{rr} (= trill [r]) and \textit{r} (= flap [ɾ]) both become \textit{r}, which is a flap in Rapa Nui. \textit{l} is likewise adjusted to \textit{r}:

\ea
\begin{tabbing}
 xxxxxxxx \= xxxxxxxxxxxxxxxxxxxxxxxxxxxxxxxxx\kill
 \textit{rr} > \textit{r} \> \textit{karetia} ‘wheelbarrow’ {\textless} Sp. \textit{carretilla}; \textit{karo} ‘jug’ {\textless} Sp. \textit{jarro} \\
 \textit{l} > \textit{r} \>  \textit{rēkaro} ‘present, gift’ {\textless} Sp. \textit{regalo}; \textit{Tire} ‘Chile’ {\textless} Sp. \textit{Chile} 
\end{tabbing}
\z 
\subparagraph{Other} \ili{Spanish} \textit{ll}, which is a voiced palatal approximant [j] or fricative\is{Fricative} [ʝ] in Chilean \ili{Spanish}, becomes \textit{i}: \textit{kaio} {\textless} \textit{callo} ‘callus’, \textit{kameio} {\textless} Sp. \textit{camello} ‘camel’. After \textit{i} it is elided\is{Elision}: \textit{tapatia} {\textless} Sp. \textit{zapatilla} ‘slipper’.

\paragraph{Phonotactics} Borrowings are also adjusted to the phonotactics of Rapa Nui; this affects the syllable\is{Syllable} structure and stress\is{Stress} pattern. 

\subparagraph{Final consonants} Final consonants are not allowed. This is resolved by adding a final vowel\is{Epenthesis}, which is either \textit{e} or identical to the previous vowel: \textit{{\ꞌ}avion\textbf{e}} ‘airplane’ {\textless} Sp. \textit{avión}; \textit{kōror\textbf{e}} ‘colour’ {\textless} Sp. \textit{color}; \textit{tampur\textbf{u}} ‘drum’ {\textless} Sp. \textit{tambor}. Alternatively, the final consonant is elided\is{Elision}; this happens especially with consonants such as \textit{d} and \textit{s/z}, which have a weak pronunciation postvocalically in Chilean \ili{Spanish}: \textit{noverā} ‘news’ {\textless} Sp. \textit{noveda\textbf{d}}; \textit{kapatā} ‘foreman’ {\textless} Sp. \textit{capata\textbf{z}}.

\subparagraph{Consonant clusters}\is{Consonant cluster} Consonant clusters\is{Consonant cluster} are disfavoured. Word-initial consonant clusters\is{Consonant cluster} are not allowed, with the exception of \textit{pr-}. Some clusters are allowed word-medially, especially homorganic nasal\is{Nasal} + plosive: \textit{ka\textbf{mp}ō} ‘countryside’ {\textless} Sp. \textit{campo}; \textit{a\textbf{tr}asao} ‘delayed’ {\textless} Sp. \textit{atrasado}; \textit{re\textbf{nt}ara} ‘apron’ {\textless} Sp. \textit{delantal}.

Clusters can be resolved by vowel epenthesis\is{Epenthesis}: \textit{{\ꞌ}ar\textbf{a}mā} ‘army’ {\textless} Sp. \textit{armada}; \textit{kar\textbf{e}sone} ‘underwear’ {\textless} Sp. \textit{calzón}; \textit{k\textbf{u}rua} ‘crane’ {\textless} Sp. \textit{grúa}, \textit{p\textbf{a}rata} ‘silver’ {\textless} Sp. \textit{plata}. Another strategy is consonant elision; this is especially common with nasals\is{Nasal} or continuants preceding another consonant: \textit{{\ꞌ}ātā} ‘until’ {\textless} Sp. \textit{ha\textbf{s}ta}; \textit{rito} ‘ready’ {\textless} Sp. \textit{li\textbf{s}to}; \textit{matakia} ‘butter’ {\textless} Sp. \textit{ma\textbf{n}tequilla}; \textit{tēnero} ‘calf’ {\textless} Sp. \textit{te\textbf{r}nero}.

\subparagraph{Word shortening} Long words are somewhat disfavoured; some words are shortened by elision\is{Elision} of an unstressed syllable\is{Syllable}: \textit{apenti} ‘appendix’ {\textless} Sp. \textit{apéndi\textbf{ce}}; \textit{tafate} ‘dish’ {\textless} Sp. \textit{\textbf{a}zafate}; \textit{rentara} ‘apron’ {\textless} Sp. \textit{de\textbf{la}ntal}; \textit{pīnere} ‘longline fishing’ {\textless} Sp. \textit{\textbf{es}pinel}.

\subparagraph{Vowel lengthening} Sometimes, vowels are lengthened\is{Vowel!lengthening}. This may serve to keep the stress\is{Stress} in the same position: \textit{kā}ˈ\textit{pē} (not \textit{*}\textit{ˈ}\textit{kape}) ‘coffee’ {\textless} Sp. \textit{ca}ˈ\textit{fé}; \textit{nove}ˈ\textit{rā} ‘news’ {\textless} Sp. \textit{nove}ˈ\textit{dad}; \textit{pā}ˈ\textit{rē} ‘wall’ {\textless} Sp. \textit{pa}ˈ\textit{red}. However, there are also cases where no adjustments are made to prevent stress\is{Stress} shift: \textit{pērī}ˈ\textit{kura} ‘movie’ {\textless} Sp. \textit{pe}ˈ\textit{lícula}.

In other cases, lengthening\is{Vowel!lengthening} may serve to avoid degenerate\is{Foot!degenerate} feet\is{Foot}, conforming the word to a preferential metrical\is{Metrical structure} pattern (\sectref{sec:2.3.2}). For example, the antepenultimate vowel is lengthened in \textit{mūseo} ‘museum’ {\textless} Sp. \textit{museo}. 

In yet other words, the reasons for lengthening\is{Vowel!lengthening} are unclear. In four-syllable\is{Syllable} words, there is a tendency to lengthen the first two vowels, creating a HHLL pattern: \textit{{\ꞌ}ōpītara} ‘hospital’ {\textless} Sp. \textit{hospital}; \textit{{\ꞌ}āpōtoro} ‘apostle’ {\textless} Sp. \textit{apóstol}. This happens even though LLLL is a common pattern in the language (\sectref{sec:2.3.2}). Lengthening may even shift the stress\is{Stress} with respect to the \ili{Spanish} original: \textit{kara}ˈ\textit{pā} ‘tent’ {\textless} Sp. ˈ\textit{carpa}; \textit{Kiri}ˈ\textit{tō} ‘Christ’ {\textless} Sp. ˈ\textit{Christo}. 

\subsubsection[Borrowings from \ili{Tahitian}]{Borrowings from Tahitian}\label{sec:2.5.3.2}
\is{Tahitian influence}
Most borrowings\is{Borrowing} from \ili{Tahitian} do not need any phonological adjustment: all \ili{Tahitian} phonemes are also part of the Rapa Nui phoneme inventory, with the exception of \textit{f} (\is{Eastern Polynesian}see \sectref{sec:2.2.1}). In fact, borrowings\is{Borrowing} from \ili{Tahitian} are often not perceived as borrowings\is{Borrowing} at all. 

In some words, \textit{f} is retained (\textit{fata} ‘altar’{\textless} Tah. \textit{fata}), but more commonly, it becomes \textit{h}: \textit{haraoa} ‘bread’ {\textless} Tah. \textit{faraoa}; \textit{hauha}\textit{{\ꞌ}a} ‘value’{\textless} Tah. \textit{faufa{\ꞌ}a}.

Some long vowels\is{Vowel length} are shortened: \textit{hoho{\ꞌ}a} {\textless} Tah. \textit{hōho{\ꞌ}a} ‘image’. Shortening may serve to avoid an illicit metrical\is{Metrical structure} pattern: \textit{tane} {\textless} Tah. \textit{tāne} ‘male’ (\sectref{sec:2.3.2}).

Glottals\is{Glottal plosive} are usually preserved\is{Elision}, but in a number of words, they are elided\is{Elision}: \textit{hāpī} ‘to learn’ {\textless} Tah. \textit{ha{\ꞌ}api{\ꞌ}i}; \textit{ha{\ꞌ}amaitai} {\textless} Tah. \textit{ha{\ꞌ}amaita{\ꞌ}i} ‘to bless’. 

Occasionally, vowels are modified: \textit{ha{\ꞌ}amuri} ‘to worship’ {\textless} Tah. \textit{ha{\ꞌ}amori}; \textit{mana{\ꞌ}u} ‘to think’ {\textless} Tah. \textit{mana{\ꞌ}o}; \textit{mareti} ‘plate’ \textit{{\textless}} Tah. \textit{merēti} (\sectref{sec:2.5.2}). 

Even when the phonemic content of \ili{Tahitian} borrowings\is{Tahitian influence} is exactly retained, borrowing may involve phonotactic shifts, especially because Rapa Nui differs from \ili{Tahitian} in diphthongisation\is{Diphthong} and stress\is{Stress} placement (see Footnote \ref{fn:40} on p.~\pageref{fn:40}): \textit{māuiui} \textstyleIPA{[ˌmaːuiˈui]} ‘sick’ {\textless} Tah. \textit{māuiui} \textstyleIPA{[ˈmaː}ᵘ\textstyleIPA{i}ᵘ\textstyleIPA{i]}; \textit{pāpa{\ꞌ}i} \textstyleIPA{[ˌpaːˈpaɁi]} ‘to write’ {\textless} Tah. \textit{pāpa{\ꞌ}i} \textstyleIPA{[ˈpaːpɁa}ⁱ\textstyleIPA{]}; \textit{haraoa} \textstyleIPA{[ˌharaˈoa]} ‘bread’ {\textless} Tah. \textit{faraoa} \textstyleIPA{[faˈra(ː)}ᵒ\textstyleIPA{a]} ({\textless} Eng. ‘flour’).

\section{Reduplication}\label{sec:2.6}
\is{Reduplication|(}
Reduplication\is{Reduplication} is a process whereby all or part of the root is copied and prefixed or suffixed to the root. The copied part of the root is called the base; the copy is called the reduplicant. In the following example, the root is placed between brackets; the base is underlined, while the reduplicant is in bold:

\ea
\gll  \textup{taŋi}      \textup{>} ~ \textup{\textbf{ta}[\underline{ta}ŋi]}\\
  {\textit{taŋi} ‘to cry’} ~ ~  {\textit{tataŋi} ‘to cry (plural)’}\\
\z
In Rapa Nui orthography\is{Orthography} the reduplicant and the root are separated by a hyphen, a practice not adopted in this grammar (\sectref{sec:1.4.4}).

Reduplication\is{Reduplication} is very common in Rapa Nui, just as it is in Polynesian languages in general. It occurs with many verbs (including adjectives) and is productive, to the extent that it is even applied to borrowings\is{Borrowing}.\footnote{\label{fn:70}Bob Weber (p.c.) \ia{Weber, Robert}once heard someone commenting at the telephone exchange that the line was engaged all the time: \textit{ko okuokupao {\ꞌ}ā} ({\textless} Sp. \textit{ocupado} ‘occupied, engaged’). \citet[198]{Makihara2001Adaptation} gives an example of \textit{kamikamiare} (\textit{kamiare} ‘to change’ {\textless} Sp. \textit{cambiar}).} Nouns are generally not reduplicated, though a nominal root may be reduplicated to form a denominal verb, while a few verbal roots are reduplicated to form a deverbal noun (\sectref{sec:2.6.2.2} below). 

The patterns of reduplication\is{Reduplication} will be discussed in \sectref{sec:2.6.1}, while \sectref{sec:2.6.2} deals with the functions of reduplication\is{Reduplication}. \sectref{sec:2.6.3} briefly discusses reduplications\is{Reduplication} for which the base form does not exist independently.

\subsection{Patterns of reduplication}\label{sec:2.6.1}
\is{Reduplication}
Although there is a wide variety of reduplication\is{Reduplication} patterns, all of these can be reduced to two types:

%\setcounter{listWWviiiNumcivleveli}{0}
\begin{enumerate}
\item 
Monomoraic reduplication\is{Reduplication}: the initial mora\is{Mora} is prefixed to the root.

\item 
Bimoraic reduplication\is{Reduplication}: the initial two morae\is{Mora} are prefixed to the root, or the final two morae\is{Mora} are suffixed to the root. With bimoraic roots, this results in complete reduplication\is{Reduplication}; with longer roots, it results in partial reduplication\is{Reduplication}.\footnote{\label{fn:71}This means that there is no principled distinction between full and partial reduplication\is{Reduplication}; cf. \citet[39]{Blust2001}. (\citealt{Davletshin2015} does take a full/partial distinction as primary.)} 

\end{enumerate}

These two patterns will be referred to as type 1 and 2, respectively. They will be discussed in turn in the following subsections. They are analysed using concepts from prosodic morphology (\citealt{McCarthyPrince1995,McCarthyPrince1996}; \citealt{InkelasZec1995}), which allows segmental content and prosodic structure\is{Metrical structure} to be subject to distinct processes and/or constraints. This allows reduplication\is{Reduplication} to be described in terms of prosodic structure\is{Metrical structure} (i.e. feet\is{Foot}, syllables\is{Syllable} and morae\is{Mora}), even though the segmental content affected does not necessarily coincide with prosodic constituents, and may vary in size and shape.

\subsubsection[The morphology of type 1 reduplication]{The morphology of type 1 reduplication}\label{sec:2.6.1.1}
\is{Reduplication}
Type 1 reduplication\is{Reduplication} occurs with a good number of bisyllabic verbs, as well as a few trisyllabic verbs and – to my knowledge – one quadrusyllabic verb.\footnote{\label{fn:72}Possibly type 1 reduplication\is{Reduplication} also occurs with a few monosyllabic verbs: \textit{kīkī} ‘to say repeatedly’ can be analysed as reduplication\is{Reduplication} of the initial mora\is{Mora} + secondary lengthening\is{Vowel!lengthening}. However, the function of this form (iterative\is{Iterative}, not plural) suggests that this is a case of type 2 reduplication\is{Reduplication}. The same is true for other reduplicated monosyllabic roots.} 

\tabref{tab:10} illustrates the different patterns in terms of light (L) and heavy (H) syllables\is{Syllable}. Each pattern will be discussed below.

\begin{table}
\begin{tabularx}{\textwidth}{p{5mm}p{30mm}p{35mm}X}
\lsptoprule
 (a)  & L L  > L L L & {\textit{{\ꞌ}ara} ‘to wake up’}& \textit{{\ꞌ}a{\ꞌ}ara} ‘to wake up (\textsc{pl})’\\
&& \textit{eke} ‘to mount’&\textit{eeke} ‘to mount (\textsc{pl})’\footnotemark{}\\
&& \textit{ha{\ꞌ}i} ‘to embrace’& \textit{haha{\ꞌ}i} ‘to embrace (\textsc{pl})’\\
&& \textit{rahi} ‘much’& \textit{rarahi} ‘important’\\
&& \textit{rehu} ‘to forget’& \textit{rerehu} ‘to faint’\\
&& \textit{turu} ‘to go down’ 
&\textit{tuturu} ‘to go down (\textsc{pl})’\\
\tablevspace
(b)  & L L  > H L L & {\textit{mate} ‘to die’}& {\textit{māmate} ‘to die (\textsc{pl})’}\\
&& \textit{piko} ‘to hide’ & \textit{pīpiko} ‘to hide (\textsc{pl})’\\
&& \textit{tere} ‘to run’ 
& \textit{tētere} ‘to run (\textsc{pl})’\\
\tablevspace
 (c)  & L L L  > H L L & {\textit{ha{\ꞌ}uru} ‘to sleep’}& \textit{hā{\ꞌ}uru} ‘to sleep (\textsc{pl})’ \\
 &&\textit{ha{\ꞌ}ere} ‘to walk’ & \textit{hā{\ꞌ}ere} ‘to walk (\textsc{pl})’\\
&& \textit{tahuti} ‘to run’ &\textit{tāhuti} ‘to run (\textsc{pl})’\\
\tablevspace
 (d)  & L L L L  > L L L L L & {\textit{paŋaha{\ꞌ}a} ‘heavy’} & \textit{papaŋaha{\ꞌ}a} ‘heavy (\textsc{pl})’\\
\lspbottomrule
\end{tabularx}
\caption{Patterns of type 1 reduplication}
% \todo[inline]{all columns should be top-aligned}
\label{tab:10}
\end{table}

\footnotetext{ When the root is vowel-initial, the reduplication\is{Reduplication} contains a bisyllabic VV-sequence, which is not merged to a single long vowel.}

As \tabref{tab:10} shows, for most bisyllabic words the reduplicant is a short syllable\is{Syllable}, i.e. it is an exact copy of the first syllable\is{Syllable} of the root.\footnote{\label{fn:74}As discussed in sec. \sectref{sec:2.3.2}, the first syllable\is{Syllable} of bisyllabic words is always short.} For some verbs, however, the vowel of the reduplicant is lengthened\is{Vowel!lengthening}. The choice between the two patterns is lexically determined: there is no difference in function between both patterns, nor is there a phonological motivation for the choice.

Both patterns can be accounted for by stating that type 1 reduplication\is{Reduplication} adds one mora\is{Mora} to the root. This mora\is{Mora} must be integrated to the prosodic structure\is{Metrical structure}, which means that an additional foot\is{Foot} is added to the word. This is illustrated in the following structure:\footnote{\label{fn:75}For sake of conciseness, the PrWd level is not included in the structure trees in this section.}

\begin{forest} 
[,phantom, for tree={calign=first}, 
[F
  [\syl
    [\mor[ra]]
  ]
  [\syl
    [\mor[hi]]
  ]
]
[,phantom
  [,phantom
    [,phantom[,phantom]]
  ]
]
[,phantom
  [,phantom
    [,phantom[,phantom]]
  ]
]
[F
  [\syl
    [\mor[\textbf{ra}]]
  ]
]
[F
  [\syl
    [\mor[{[\underline{ra}}]]
  ]
  [\syl
    [\mor[{hi]}]]
  ]
]
]
\end{forest}

The initial foot\is{Foot} only has one mora\is{Mora}, i.e. it is degenerate\is{Foot!degenerate}. Word-initial degenerate\is{Foot!degenerate} feet are allowed in Rapa Nui, but there is pressure towards a pattern of whole feet (\sectref{sec:2.3.2}). For some words, this leads to the addition of a second mora\is{Mora} to the initial foot, which is filled by spreading the first vowel:\\

\begin{forest} 
[,phantom, for tree={calign=first}
[F
  [\syl
    [\mor[hi]]
  ]
  [\syl
    [\mor[ŋa]]
  ]
]
[,phantom
  [,phantom
    [,phantom[,phantom]]
  ]
]
[,phantom
  [,phantom
    [,phantom[,phantom]]
  ]
]
[F
  [\syl
    [\mor[\textbf{hi},name=hi1]]
    [\mor,name=mor2]
  ]
]
[F
  [\syl
    [\mor[{[\underline{hi}}]]
  ]
  [\syl
    [\mor[{ŋa]}]]
  ]
]
]
\draw (mor2.south) -- (hi1.north);
\end{forest}

\newpage 
For trisyllabic roots, a mora\is{Mora} is added to the existing degenerate\is{Foot!degenerate} foot; no additional foot\is{Foot} is needed. Moreover, no segmental content needs to be added, as the additional mora\is{Mora} can be filled by spreading the initial vowel of the root:\footnote{\label{fn:76}With bisyllabic roots, spreading of the vowel (\textit{rahi} > *\textit{rāhi}) is not possible, as the resulting long vowel would cross a foot boundary, creating an impossible prosodic pattern (\sectref{sec:2.3.2}).}

\begin{forest} 
[,phantom, for tree={calign=first}
[F
  [\syl
    [\mor[ha]]
  ]
]
[F
  [\syl
    [\mor[{\ꞌ}e]]
  ]
  [\syl
    [\mor[re]]
  ]
]
[,phantom
  [,phantom
    [,phantom[,phantom]]
  ]
]
[,phantom
  [,phantom
    [,phantom[,phantom]]
  ]
]
[,phantom
  [,phantom
    [,phantom[,phantom]]
  ]
]
[F
  [\syl
    [\mor[ha,name=ha]]
    [\mor,name=mor2]
  ]
]
[F
  [\syl
    [\mor[{\ꞌ}e]]
  ]
  [\syl
    [\mor[re]]
  ]
]
]
\draw (mor2.south) -- (ha.north);
\end{forest}

We may conclude that, even though the surface result of reduplication\is{Reduplication} is quite different for trisyllabic roots than for bisyllabic roots, both can be analysed as involving the same process: addition of a single mora\is{Mora} to the root. Another indication that both groups of words involve the same type of reduplication\is{Reduplication}, is that in both cases the most common function of reduplication\is{Reduplication} is plurality: \textit{hā{\ꞌ}ere} is the plural of \textit{ha{\ꞌ}ere}, just like \textit{tuturu} is the plural of \textit{turu}. This will be discussed in more detail in \sectref{sec:2.6.2.1} below.

The only example of a quadrusyllabic word shows the same mechanism at work: a mora\is{Mora} is added, resulting in an additional degenerate\is{Foot!degenerate} foot, which is filled with a copy of the initial syllable\is{Syllable} of the root: \textit{paŋaha{\ꞌ}a} > \textit{papaŋaha{\ꞌ}a}.

\subsubsection[The morphology of type 2 reduplication]{The morphology of type 2 reduplication}\label{sec:2.6.1.2}
\is{Reduplication}
Type 2 reduplication\is{Reduplication} has the following features:

%\setcounter{listWWviiiNumlxxxvileveli}{0}
\begin{enumerate}
\item 
Two morae\is{Mora} at the edge of the root are copied: either the initial two morae\is{Mora} are reduplicated as a prefix, or the final two morae\is{Mora} are reduplicated as a suffix. 

\item 
With trisyllabic LLL- and HLL-roots, suffixing is far more common; with quadrusyllabic roots and trisyllabic LLH roots, only prefixing occurs.

\item 
If the first vowel of the root is short, it is lengthened when the reduplicant is suffixed, as in (d) below.

\item 
If the first vowel of the root is long, it is shortened when the reduplicant is prefixed, as in (e) below. In this case, the reduplication\is{Reduplication} base consists of the first two short syllables\is{Syllable}, rather than the initial long syllable\is{Syllable}.

\end{enumerate}

\begin{table}
\begin{tabularx}{\textwidth}{p{5mm}p{26mm}p{26mm}p{56mm}}
\lsptoprule
{(a)} & H > H H & {\textit{pā} ‘to fold’} \newline 
{\textit{kī} ‘to say’} & {\textit{pāpā} ‘to fold repeatedly’} \newline 
{\textit{kīkī} ‘to say repeatedly’}\\
%\tablevspace
{(b)} & L L > L L L L & {\textit{hoa} ‘to throw’} \newline 
{\textit{riva} ‘good’} & {\textit{hoahoa} ‘to throw various things’}\newline
{\textit{rivariva} ‘good’}\\
%\tablevspace
{(c)} & L L L, prefixing \newline
> L L L L L & {\textit{ha{\ꞌ}ere} ‘to walk’} \newline 
{\textit{mana{\ꞌ}u} ‘think’} & {\textit{ha{\ꞌ}eha{\ꞌ}ere} ‘to stroll’}\newline
{\textit{manamana{\ꞌ}u} ‘to think repeatedly’}\\
%\tablevspace
{(d)} & L L L, suffixing \newline 
> H L L L L & {\textit{ha{\ꞌ}ere} ‘to walk’} \newline 
{\textit{tiŋa{\ꞌ}i} ‘to kill’} & {\textit{hā{\ꞌ}ere{\ꞌ}ere} ‘to stroll’}\newline
{\textit{tīŋa{\ꞌ}iŋa{\ꞌ}i} ‘to kill several people’}\\
%\tablevspace
{(e)} & H L L, prefixing \newline
> L L L L L & {\textit{mā{\ꞌ}ea} ‘stone’} \newline 
{\textit{vānaŋa} ‘to talk’} & {\textit{ma{\ꞌ}ema{\ꞌ}ea} ‘stony, rocky’} \newline 
{\textit{vanavanaŋa} ‘to chat’}\\
%\tablevspace
{(f)} & H L L, suffixing \newline
> H L L L L & {\textit{vānaŋa} ‘to talk’} \newline 
{\textit{pāhono} ‘answer’} & {\textit{vānaŋanaŋa} ‘to chat’} \newline
{\textit{pāhonohono} ‘argumentative’}\\
%\tablevspace
{(g)} & L L H, prefixing \newline
> L L L L H & {\textit{{\ꞌ}auē} ‘to cry out’} & {\textit{{\ꞌ}au{\ꞌ}auē} ‘to cry repeatedly’}\\
%\tablevspace
{(h)} & L L L L, prefixing \newline 
> L L L L L L & {\textit{tokerau} ‘wind’} & {\textit{toketokerau} ‘windy’}\\
\lspbottomrule
\end{tabularx}
\caption{Patterns of type 2 reduplication}
\label{tab:11}
\end{table}

The patterns of type 2 reduplication\is{Reduplication} are listed in \tabref{tab:11}. These patterns are united by a single feature: the addition of a foot\is{Foot} to the word, which is filled in some way by copying two morae\is{Mora} from the root. The different patterns are discussed in turn below.

\subparagraph{Bimoraic words (patterns a–b)} For bimoraic words (whether mono- or bisyllabic), prefixing and suffixing yield the same result. In both cases the whole root is copied, resulting in a two-foot word. Below are examples of reduplications\is{Reduplication} of H and LL words (here prefixing is assumed, cf. Footnote \ref{fn:79} on p.~\pageref{fn:79}):\\
 

\begin{forest}
[,phantom, for tree={calign=first}
[F
  [\syl
    [\mor[\textbf{pa},name=pa1]]
    [\mor,name=mor2]
  ]
]
[F
  [\syl
    [\mor[{[\underline{pa}]},name=pa2]]
    [\mor,name=mor4]
  ]
]
\draw (mor2.south) -- (pa1.north);
\draw (mor4.south) -- (pa2.north);
[,phantom
  [,phantom
    [,phantom[,phantom]]
  ]
]
[,phantom
  [,phantom
    [,phantom[,phantom]]
  ]
]
[F
  [\syl
    [\mor[\textbf{ho}]]
  ]
  [\syl
    [\mor[\textbf{a}]]
  ]
]
[F
[\syl
    [\mor[{[\underline{ho}}]]
  ]
  [\syl
    [\mor[{\underline{a}]}]]
  ]
]
]
\end{forest}
 

\begin{table}
\begin{tabularx}{90mm}{L{16mm}R{5mm}L{24mm}R{5mm}L{24mm}}
\lsptoprule
{root} & & {prefixing} & & {suffixing}\\
\midrule
L L L & (c) & L L L L L & (d) & H L L L L\\
\textit{ha{\ꞌ}ere} & & \textit{ha{\ꞌ}eha{\ꞌ}ere} & & \textit{hā{\ꞌ}ere{\ꞌ}ere}\\
\tablevspace
H L L & (e) & L L L L L & (f) & H L L L L\\
\textit{vānaŋa} & & \textit{vanavanaŋa}  & & \textit{vānaŋanaŋa}\\
\lspbottomrule
\end{tabularx}
\caption{Type 2 reduplication of trisyllabic roots}
\label{tab:12}
\end{table}

\subparagraph{Trisyllabic words (patterns c–f)} For trisyllabic LLL and HLL words, the pattern is more intricate. The relevant data are repeated in \tabref{tab:12}.\footnote{\label{fn:77}The same patterns of lengthening\is{Vowel!lengthening} and shortening also occur in \ili{Māori}; \citet[148]{MeyerhoffReynolds1996} give examples of patterns d–f.} As this table shows, regardless the length of the root vowels, in prefixing forms all vowels are short, while in suffixing forms the first vowel is long. These data can be accounted for by the following constraints:

\newpage 
\textit{Non-violable:}

%\setcounter{listWWviiiNumxxxiiileveli}{0}
\begin{enumerate}
\item 
The reduplicated word contains three feet, i.e. one foot more than the base.

\item 
The reduplicant consists of either the first two syllables\is{Syllable} of the root, which are prefixed, or the final two syllables\is{Syllable} of the root, which are suffixed.

\item 
Only the first vowel of the root may be long, and only if it is word-initial.

\end{enumerate}

\textit{Violable:}

\begin{enumerate}
\setcounter{enumi}{3}
\item 
All feet are complete.

\end{enumerate}

Constraint 4 is a general soft constraint in Rapa Nui (\sectref{sec:2.3.2}) which can be fulfilled – if possible – by vowel lengthening\is{Vowel!lengthening}. But in type-two reduplications\is{Reduplication}, the data show that the possibilities of lengthening\is{Vowel!lengthening} are limited (constraint 3): only the first vowel of the root may be lengthened (\textit{h}\textit{ā{\ꞌ}ere{\ꞌ}ere}), not the first vowel of the reduplicant (\textit{*hā{\ꞌ}eha{\ꞌ}ere}). Moreover, the first root vowel is lengthened only word-initially, not when it is preceded by the reduplicant (\textit{*ha{\ꞌ}ehā{\ꞌ}ere}).\footnote{\label{fn:78}An alternative option to account for \textit{ha{\ꞌ}eha{\ꞌ}ere} would be, to assume that the boundary of the root coincides with a foot boundary, so that the initial foot is complete, while the second foot is degenerate\is{Foot!degenerate}: 

\ea
  \textup{(ha{\ꞌ}e)\textsubscript{F}   [ (ha)\textsubscript{F}   ({\ꞌ}ere)\textsubscript{F} ]}
\z
But this would mean assuming an otherwise unattested pattern containing a non-initial degenerate\is{Foot!degenerate} foot. Moreover, it would raise the question why the root-initial vowel of \textit{vanavanaŋa} (based on the foot \textit{vānaŋa}) is short, rather than long; one would expect:
\ea
  \textup{*(vana)\textsubscript{F}   [ (vā)\textsubscript{F}   (naŋa)\textsubscript{F} ]}
\z
Another reason not to adopt this analysis, is that some speakers put secondary stress\is{Stress} on the second vowel: [haˌʔehaˈʔere]. This suggests a foot structure where the second syllable\is{Syllable} is prominent, i.e. foot-initial:
\ea
\textup{(ha)\textsubscript{F}~  ({\ꞌ}eha)\textsubscript{F}~  ({\ꞌ}ere)\textsubscript{F}}
\z
Pattern (i) is proposed for derivations like \textit{haapai} > \textit{hapahapai} in \ili{Māori} by \citet[161]{MeyerhoffReynolds1996}; in their analysis, \textit{*hapahaapai} would violate a correspondence constraint which requires that every element in the reduplicant has a correspondent in the base. Notice that \ili{Māori} is metrically different from Rapa Nui: degenerate\is{Foot!degenerate} feet are disallowed, and main stress\is{Stress} falls on the leftmost foot.}\textstyleFootnoteSymbol{} When vowel lengthening\is{Vowel!lengthening} is not possible, the initial foot is degenerate\is{Foot!degenerate}, in accordance with the following non-violable constraint in the language (\sectref{sec:2.3.2}):

\begin{enumerate}
\setcounter{enumi}{4}

\item \glt
All non-initial feet\is{Foot} are complete; the initial foot may be degenerate.

\end{enumerate}

Though constraint 3 may seem to be somewhat arbitrary, it corresponds to a general tendency in Rapa Nui: the statistics in \sectref{sec:2.3.2} show that long vowels\is{Vowel length} are much more common word-initially than in other positions; moreover, they are very rare when surrounded by short vowels. (\textit{*ha}\textit{{\ꞌ}ehā{\ꞌ}ere} would correspond in prosodic structure\is{Metrical structure} to \textit{manupātia} ‘wasp’, one of the few LLHLL words.)\footnote{\label{fn:79}These constraints may explain why suffixing is much more common with these words than prefixing, even though in other cases where prefixing and suffixing can be distinguished (type 1 reduplication\is{Reduplication}; type 2 for quadrumoraic words) only prefixing occurs: suffixing allows the initial vowel to be lengthened (constraint 3), so the word consists of three whole feet (satisfying constraint 4); on the other hand, prefixing results in a degenerate\is{Foot!degenerate} foot. In general, prefixing reduplication\is{Reduplication} is much more common in Polynesian, and in Austronesian in general \citep{Finney1999}.}

The constraints under discussion result in the following structures for LLL words:

\begin{tabularx}{\textwidth}{ll}
%{
%  \raggedleft
\begin{forest}
[,phantom, for tree={calign=first}
[F
  [\syl
    [\mor[ha]]
  ]
]
[F
  [\syl
    [\mor[{\ꞌ}e]]
  ]
  [\syl
    [\mor[re]]
  ]
]
[,phantom
]
[,phantom
]
[,phantom
]
]
\end{forest}
&
\begin{forest}
[,phantom, for tree={calign=first}
[{suffixing:}
]
[F
  [\syl
    [\mor[{[ha},name=ha1]]
    [\mor,name=mor2]
  ]
]
[F
  [\syl
    [\mor[\underline{{\ꞌ}e}]]
  ]
  [\syl
    [\mor[{\underline{re}]}]]
  ]
]
[F
  [\syl
    [\mor[\textbf{{\ꞌ}e}]]
  ]
  [\syl
    [\mor[\textbf{re}]]
  ]
]
]
]
\draw (mor2.south) -- (ha1.north);
\end{forest}
\\
&
\begin{forest}
[,phantom, for tree={calign=first}
[{prefixing:}
]
[F
  [\syl
    [\mor[\textbf{ha}]]
  ]
]
[F
  [\syl
    [\mor[\textbf{{\ꞌ}e}]]
  ]
  [\syl
    [\mor[{[\underline{ha}}]]
  ]
]
[F
  [\syl
    [\mor[\underline{{\ꞌ}e}]]
  ]
  [\syl
    [\mor[{re]}]]
  ]
]
]
\end{forest}

%}
\end{tabularx}
% \z
%\todo[inline]{I’d like to align “suffixing” and “prefixing” in the trees above. Likewise in the trees below.}
%\todo[inline]{You could always use a 3x2 table, if you want to align the left margin of the right column}
For HLL words the situation is identical, except that the root consists of two complete feet\is{Foot}. Interestingly, the length of the initial vowel is not carried over into the reduplication\is{Reduplication}. This is somewhat surprising, as in other cases long vowels\is{Vowel length} remain long under type 2 reduplication\is{Reduplication} (see (a) above and (g) below).

\begin{tabularx}{\textwidth}{ll}
\begin{forest}
[,phantom, for tree={calign=first}
[F
  [\syl
    [\mor[\textbf{va},name=va1]]
    [\mor,name=mor2]
  ]
]
[F
  [\syl
    [\mor[na]]
  ]
  [\syl
    [\mor[ŋa]]
  ]
]
[,phantom
]
[,phantom
]
[,phantom
]
]
\draw (mor2.south) -- (va1.north);
\end{forest}
&
\begin{forest}
[,phantom, for tree={calign=first}
[{suffixing:}
]
[F
  [\syl
    [\mor[{[va},name=va2]]
    [\mor,name=mor4]
  ]
]
[F
  [\syl
    [\mor[\underline{na}]]
  ]
  [\syl
    [\mor[{\underline{ŋa}]}]]
  ]
]
[F
  [\syl
    [\mor[\textbf{na}]]
  ]
  [\syl
    [\mor[\textbf{ŋa}]]
  ]
]
]
\draw (mor4.south) -- (va2.north);
\end{forest}\\

&

\begin{forest}
[,phantom, for tree={calign=first}
[{prefixing:}
]
[F
  [\syl
    [\mor[\textbf{va}]]
  ]
]
[F
  [\syl
    [\mor[\textbf{na}]]
  ]
  [\syl
    [\mor[{[\underline{va}}]]
  ]
]
[F
  [\syl
    [\mor[\underline{na}]]
  ]
  [\syl
    [\mor[{ŋa]}]]
  ]
]
]
\end{forest}
\end{tabularx}

\subparagraph{Quadrumoraic words (patterns g–h)} Finally, the reduplication\is{Reduplication} of quadrumoraic words (LLH words like \textit{{\ꞌ}auē} ‘cry’, LLLL words like \textit{tokerau} ‘wind’) is illustrated below. In both cases, a complete foot\is{Foot} is added, which is filled segmentally by copying the first two syllables\is{Syllable} of the root:

\begin{forest}
[,phantom, for tree={calign=first}
[F
  [\syl
    [\mor[\textbf{{\ꞌ}a}]]
  ]
  [\syl
    [\mor[\textbf{u}]]
  ]
]
[F
  [\syl
    [\mor[{[\underline{{\ꞌ}a}}]]
  ]
  [\syl
    [\mor[\underline{u}]]
  ]
]
[F
  [\syl
    [\mor[{ē]},name=e]]
    [\mor,name=mor2]
  ]
]
[,phantom
  [,phantom
    [,phantom[,phantom]]
  ]
]
[,phantom
  [,phantom
    [,phantom[,phantom]]
  ]
]
[F
  [\syl
    [\mor[\textbf{to}]]
  ]
  [\syl
    [\mor[\textbf{ke}]]
  ]
]
[F
  [\syl
    [\mor[{[\underline{to}}]]
  ]
  [\syl
    [\mor[\underline{ke}]]
  ]
]
[F
  [\syl
    [\mor[ra]]
  ]
  [\syl
    [\mor[{u]}]]
  ]
]
]
\draw (mor2.south) -- (e.north);
\end{forest}

Occasionally, type 1 and type 2 reduplication\is{Reduplication} are applied in sequence: the result of type 1 reduplication\is{Reduplication} serves as the base of type 2 reduplication\is{Reduplication}. This is only attested with a few LL roots; the process can be described as follows:

\ea
  µ\textsubscript{1} µ\textsubscript{2} > \textbf{µ}\textbf{\textsubscript{1}} [µ\textsubscript{1} µ\textsubscript{2}] > \textbf{µ}\textbf{\textsubscript{1}} \textbf{µ}\textbf{\textsubscript{1}} [µ\textsubscript{ 1} µ\textsubscript{1} µ\textsubscript{2}]
\z

The result is a form in which the initial syllable\is{Syllable} of the root is repeated four times. A few examples:

\ea
\begin{tabbing}
 xxxxxxxxxxxxxxxxxxxxxxxxxxxxxxxxxxx \kill
\textit{{\ꞌ}uri} ‘black’ > \textit{{\ꞌ}u{\ꞌ}u{\ꞌ}u{\ꞌ}uri} ‘black (many things)’\\
  \textit{tea} ‘white’ > \textit{tetetetea} ‘white (many things)’\\
  \textit{kikiu}\footnote{\label{fn:80}The root \textit{kiu} does not occur in isolation. However, \textit{kiukiu} does occur, hence \textit{kikiu} can be analysed as a reduplication\is{Reduplication}.} ‘to shriek, squeak’ > \textit{kikikikiu} ‘to shriek again and again’
\end{tabbing}
\z 
\subsection{Functions of reduplication}\label{sec:2.6.2}
\is{Reduplication}
The basic function of type 1 reduplication\is{Reduplication} of a verb is expressing plurality of its S/A argument; the basic sense of type 2 reduplication\is{Reduplication} is repetition or intensity.\footnote{\label{fn:81}The same is true in Polynesian languages in general, see \citet{Finney1999}.} However, in both cases exceptions and lexicalised meanings are not uncommon. Both types are discussed in turn below.

\subsubsection[Type 1: plurality]{Type 1: plurality}\label{sec:2.6.2.1}

The sense of type 1 reduplication\is{Reduplication} is lexically determined. For most verbs, it indicates a plural\is{Plural!verb} (i.e. more than one) S/A argument. Some examples:

\ea\label{ex:2.15}
\gll He \textbf{totopa} o mātou ki raro. \\
\textsc{ntr} \textsc{pl}:descend of \textsc{1pl.excl} to below \\

\glt 
‘We went down.’ \textstyleExampleref{[R157.040]} 
\z

\ea\label{ex:2.16}
\gll Te aŋa o koā Eugenio he \textbf{pīpiko} nō {\ꞌ}i roto i te rāua hare. \\
\textsc{art} do of \textsc{coll} Eugenio \textsc{pred} \textsc{pl}:hide just at inside at \textsc{art} \textsc{3pl} house \\

\glt 
‘Eugenio and his friend used to hide inside their house.’ \textstyleExampleref{[R231.279]} 
\z

\ea\label{ex:2.17}
\gll Ka \textbf{nonoho} kōrua ka \textbf{uunu} {\ꞌ}i ra{\ꞌ}e i te kōrua ū. \\
\textsc{imp} \textsc{pl}:sit \textsc{2pl} \textsc{imp} \textsc{pl}:drink at first \textsc{acc} \textsc{art} \textsc{2pl} milk \\

\glt
‘Sit down (pl.) and first drink your milk.’ \textstyleExampleref{[R334.117]} 
\z

Most verbs do not have a plural\is{Plural!verb} form at all.\footnote{\label{fn:82}The lexical database includes 56 plural forms with type 1 reduplication\is{Reduplication}, on a total of over 3500 verbs and adjectives.} For those verbs that do have a plural form, its use is not obligatory – in other words, the base form is not limited to singular argument. In \REF{ex:2.18} the basic form \textit{tu{\ꞌ}u} is used, even though a plural form \textit{tutu{\ꞌ}u} exists.

\ea\label{ex:2.18}
\gll He \textbf{tu{\ꞌ}u} mai tou ŋā uka era. \\
\textsc{ntr} arrive hither \textsc{dem} \textsc{pl} girl \textsc{dist} \\

\glt 
‘Those girls arrived.’ \textstyleExampleref{[Blx-3.053]}
\z

Some type 1 reduplications have a different sense; this is lexically determined, hence unpredictable. 
\ea
\begin{tabbing}
xxxxxxxxxxxxxxxxxxxxxx \= xxxxxxxxxxxxxxxxxx\kill
\textit{hati} ‘to break (intr.)’ \> \textit{hahati} ‘to break (tr.)’\\
\textit{more} ‘to be cut, wounded’ \> \textit{momore} ‘to harvest, pick; to break’\\
\textit{puhi} ‘to blow’ \> \textit{pupuhi} ‘to shoot (with a weapon)’\\
\textit{rehu} ‘to be forgotten’ \> \textit{rerehu} ‘to faint’\\
\textit{rahi} ‘much’ \> \textit{rarahi} ‘important’
\end{tabbing}
\z 
As this list shows, for a few of these verbs the base form is intransitive\is{Verb!intransitive}, while the reduplicated form is transitive\is{Verb!transitive}. Here is a pair of examples:

\ea\label{ex:2.19}
\gll \textit{He} \textbf{more} {\ꞌ}ino Rau Nui {\ꞌ}i te mā{\ꞌ}ea.\\
\textsc{ntr} cut bad Rau Nui at \textsc{art} stone\\

\glt 
‘Rau Nui was badly wounded by the stone.’ \textstyleExampleref{[Fel-64.081]}
\z

\ea\label{ex:2.20}
\gll Ki oti he oho he \textbf{momore} i te tarake. \\
when finish \textsc{ntr} go \textsc{ntr} \textsc{red}:cut \textsc{acc} \textsc{art} corn \\

\glt 
‘After that, he goes and picks corn.’ \textstyleExampleref{[R156.013]} 
\z

\subsubsection[Type 2: iterativity and intensity]{Type 2: iterativity and intensity}\label{sec:2.6.2.2}
\is{Iterative}
Like type 1 reduplication, type 2 mainly affects verbs, though unlike type 1, there are some examples where a noun is involved; the latter mostly concerns cases of denominal verbs (see below). However, only for verbs is reduplication\is{Reduplication} productive. Its function depends largely on the nature of the verb.\footnote{\label{fn:83}\citet{Johnston1978}, after a detailed lexical study, concludes that reduplication\is{Reduplication} in Rapa Nui indicates 1) repetition; 2) quantification (of the subject); 3) duration; 4) the degree of vigour in which the action is carried out. I have not found any case where (3) is the sole factor involved; whenever reduplication\is{Reduplication} may be taken as indicating duration, this is usually by virtue of iterativity\is{Iterative}. “quantification” may involve either the subject (usually with type 1, but occasionally with type 2) or the object; see below in this section.}

\paragraph{Repetition}\label{sec:2.6.2.2.1} Type 2 reduplication\is{Reduplication} often adds an element of repetition to the event expressed by the verb:

%\todo{here and elsewhere, I would prefer to have the offset items numbered}

\ea
\begin{tabbing}
 xxxxxxxxxxxxxxxxxxxxxxx \= xxxxxxxxxxxxxxxxxx\kill
\textit{rei} ‘to step’ \> \textit{reirei} ‘to step repeatedly, to trample; \\
\> ~~~to sew with a sewing machine’\\
  \textit{tumu} ‘to cough’ \>   \textit{tumutumu} ‘to cough repeatedly’\\
  \textit{rapu} ‘to gesture’  \>  \textit{rapurapu} ‘to make repeated gestures’\\
  \textit{ŋae{\ꞌ}i} ‘to move’  \>  \textit{ŋāe{\ꞌ}ie{\ꞌ}i} ‘to move back and forth’\\
  \textit{e{\ꞌ}a} ‘to go out, make a trip’  \>  \textit{e{\ꞌ}ae{\ꞌ}a} ‘to make various trips’  
\end{tabbing}
\z 
For some verbs, reduplication\is{Reduplication} indicates repetition of the parts or stages making up the event, rather than the event as a whole:\footnote{\label{fn:84}\citet{Haji-AbdolhosseiniMassam2002}, describing reduplication\is{Reduplication} in \ili{Niuean}, use the term \textit{phase repetition}.}

\ea
\begin{tabbing}
 xxxxxxxxxxxxxxxxxxxxxxx \= xxxxxxxxxxxxxxxxxx\kill
  \textit{hore} ‘to cut’  \>  \textit{horehore} ‘to cut with various movements’\\
  \textit{kokoti} ‘to cut’   \> \textit{kotikoti} ‘to cut repeatedly; cut with scissors’\\
  \textit{pa{\ꞌ}o} ‘to chop’  \>  \textit{pa{\ꞌ}o\nobreakdash-pa{\ꞌ}o} ‘to make various chopping \\
 \> ~~~movements, to chip away (e.g. at a tree)’
\end{tabbing}
\z 
\paragraph{Distributive}\label{sec:2.6.2.2.2} Repetition of the event may imply a distributive\is{Distributive} reading, involving different participants: the event happens repeatedly, each time affecting a different Patient or Recipient, or performed by a different Agent.

\ea
\begin{tabbing}
 xxxxxxxxxxxxxxxxxxxxxxx \= xxxxxxxxxxxxxxxxxx\kill
  \textit{hono} ‘to patch’  \>  \textit{honohono} ‘to put various patches’\\
  \textit{na{\ꞌ}a} ‘to hide’  \>  \textit{na{\ꞌ}ana{\ꞌ}a} ‘to hide several things’\\
  \textit{ohu} ‘to shout’  \>  \textit{ohuohu} ‘to shout (various people)’\\
  \textit{poreko} ‘to be born’  \>  \textit{pōrekoreko} ‘to be born (different children, \\
 \> ~~~at different times)’\\
  \textit{vahi} ‘to divide’  \>  \textit{vahivahi} ‘to divide in various parts; \\
 \> ~~~to divide among various people’
\end{tabbing}
\z 
With plural\is{Plural!verb} Agents, the sense of the verb may seem to be similar to a type 1 reduplication. However, the type 2 reduplication\is{Reduplication} still refers to a series of separate events: each Agent performs the action individually (possibly at different times), not as a group. In the following example, \textit{tu{\ꞌ}utu{\ꞌ}u} expresses multiple events of arriving, i.e. different ships arriving at different occasions. The plural \textit{tutu{\ꞌ}u} (type 1) would imply that different ships arrived in a single event.

\ea\label{ex:2.21}
\gll Mai te taŋata nei i ha{\ꞌ}amata ai i \textbf{tu{\ꞌ}utu{\ꞌ}u} mai ai te pahī papa{\ꞌ}ā. \\
from \textsc{art} man \textsc{prox} \textsc{pfv} begin \textsc{pvp} \textsc{pfv} arrive:\textsc{red} hither \textsc{pvp} \textsc{art} ship foreign \\

\glt 
‘Starting with this man (=the explorer Jacob Roggeveen), foreign ships started to arrive (on Rapa Nui).’ \textstyleExampleref{[R111.014]} 
\z

The choice between mere repetition and a distributive\is{Distributive} reading results to some degree from the semantics of the verb. Transitive\is{Verb!transitive} verbs are more likely to have a distributive\is{Distributive} sense: repetition of a transitive\is{Verb!transitive} event will often affect different objects. However, the precise meaning of the reduplication\is{Reduplication} is not lexically specified, but may vary depending on the context. The two examples below show different uses of type-2 reduplication\is{Reduplication} of \textit{u{\ꞌ}i} ‘to look’. While in \REF{ex:2.22} \textit{u{\ꞌ}iu{\ꞌ}i} has an iterative\is{Iterative} sense, in \REF{ex:2.23} it is distributive\is{Distributive} (and effectively reciprocal\is{Reciprocal}).

\ea\label{ex:2.22}
\gll Pē rā {\ꞌ}ā e \textbf{u{\ꞌ}iu{\ꞌ}i} era a tu{\ꞌ}a koi{\ꞌ}ite e tute rō mai  e tū {\ꞌ}amahiŋo era ko Mako{\ꞌ}i.\\
like \textsc{dist} \textsc{ident} \textsc{ipfv} look:\textsc{red} \textsc{dist} by back perhaps \textsc{ipfv} chase \textsc{emph} hither  \textsc{ag} \textsc{dem} evil\_person \textsc{dist} \textsc{prom} Mako’i\\

\glt 
‘Like that he kept looking behind him, to see if he was followed by that wicked Mako’i.’ \textstyleExampleref{[R214.038]} 
\z

\ea\label{ex:2.23}
\gll He \textbf{u{\ꞌ}iu{\ꞌ}i} ia te {\ꞌ}āriŋa a totoru. \\
\textsc{ntr} look:\textsc{red} then \textsc{art} face \textsc{prop} \textsc{red}:three \\

\glt 
‘The three looked at each other.’ \textstyleExampleref{[R313.005]} 
\z

\newpage 
\paragraph{Intensity}\label{sec:2.6.2.2.3} With many adjectives, reduplication\is{Reduplication} signifies increased intensity:
\ea
\begin{tabbing}
xxxxxxxxxxxxxxxxxxxxxxx \= xxxxxxxxxxxxxxxxxx \= xxxxxxxxxxxxxx \kill
  \textit{piro} ‘rotten’ \> \textit{piropiro} ‘completely rotten’\\
  \textit{{\ꞌ}ehu} ‘blurry’ \> \textit{{\ꞌ}ehu{\ꞌ}ehu} ‘very blurry, barely visible’\\
  \textit{tea} ‘light in colour’ (\textit{moana tea} = light blue) \>\> \textit{teatea} ‘white’\\
  \textit{{\ꞌ}uri} ‘dark in colour’ (\textit{meamea {\ꞌ}uri} = dark red) \>\> \textit{{\ꞌ}uri{\ꞌ}uri} ‘black’
\end{tabbing}
\z 
However, reduplication\is{Reduplication} of an adjective does not always imply intensity: \textit{rivariva} ‘good’, not ‘very good’. See \sectref{sec:2.6.2.2.5} below.

\paragraph{Lexicalised meanings}\label{sec:2.6.2.2.4} For some verbs, the sense of the reduplicated form is lexicalised and unpredictable, even though it is obviously related to the meaning of the root.

\ea
\begin{tabbing}
xxxxxxxxxxxxxxxxxxxxxxx \= xxxxxxxxxxxxxxxxxx\kill
  \textit{{\ꞌ}omo} ‘to smoke’ \> \textit{{\ꞌ}omo{\ꞌ}omo} ‘to suck’\\
  \textit{mana{\ꞌ}u} ‘to think’ \> \textit{māna{\ꞌ}una{\ꞌ}u} ‘to be worried’\\
  \textit{taka} ‘to roll up’ \> \textit{takataka} ‘round’\\
  \textit{poto} ‘short (in size)’ \>  \textit{potopoto} ‘short (in distance)’\\
  \textit{roa} ‘distant’ \> \textit{roaroa} ‘tall’\\
  \textit{haŋu} ‘to breathe’ \> \textit{haŋuhaŋu} ‘to pant’
\end{tabbing}
\z 
In some cases the meaning of the reduplication\is{Reduplication}, even though lexicalised, is clearly derived from an iterative\is{Iterative} sense. In the case of \textit{māna{\ꞌ}una{\ꞌ}u} the specialised sense ‘to be worried’ developed from the iterative\is{Iterative} sense ‘to think much’. (In fact, \textit{mana{\ꞌ}u rahi} ‘think much’ is used with a similar sense.)

\paragraph{Reduplication as basic form}\label{sec:2.6.2.2.5} For certain words, the reduplicated form is more common than the root. In these cases, the simple form is often limited in use. This is especially common with adjectives (\sectref{sec:3.5.1.2}): \textit{nuinui} ‘big’ is much more common than \textit{nui} ‘big’, which is used in limited contexts. For other words, the simple form is not in use at all; these are discussed in \sectref{sec:2.6.3} below. 

\paragraph{Conversion}\label{sec:2.6.2.2.6} A number of reduplications\is{Reduplication} are denominal verbs\is{Verb!denominal} or adjectives:

\ea 
\begin{tabbing}
 xxxxxxxxxxxxxxxxxxxxxxx \= xxxxxxxxxxxxxxxxxx \= xxxxxxxxxxxxxx \kill
  \textit{māmari} ‘egg; biceps’ \> \textit{māmamamari} ‘to build muscles’\\
  \textit{ŋutu} ‘mouth’ \> \textit{ŋutuŋutu} ‘to talk excessively, be talkative’\\
  \textit{pia} ‘starch’  \>\textit{piapia} ‘starchy’\\
  \textit{tore} ‘stripe’ \> \textit{toretore} ‘striped’\\
  \textit{vai} ‘water’ \> \textit{vaivai} ‘moist, wet’
\end{tabbing}
\z 
A few reduplications\is{Reduplication} are deverbal nouns\is{Noun!deverbal}:
\ea
\begin{tabbing}
 xxxxxxxxxxxxxxxxxxxxxxx \= xxxxxxxxxxxxxxxxxx \= xxxxxxxxxxxxxx \kill
  \textit{poko} ‘hollow’  \>\textit{pokopoko} ‘hollow place, basin’\\
  \textit{toke} ‘to steal’ \> \textit{toketoke} ‘to steal frequently; thief’
\end{tabbing}
\z 
\paragraph{Attenuative}\label{sec:2.6.2.2.7} Finally, reduplication\is{Reduplication} may have an attenuative\is{Attenuative} function, implying a certain weakening. Iterativity may mean that the event – or a phase of the event – takes place repeatedly, but each time to a small extent:
\ea 
\begin{tabbing}
 xxxxxxxxxxxxxxxxxxxxxxxxx \= xxxxxxxxxxxxxxxxxx \= xxxxxxxxxxxxxx \kill
  \textit{mate} ‘to die, be extinguished’ \> \textit{matemate} ‘to flicker’\\
  \textit{{\ꞌ}ua} ‘to rain’ \> \textit{{\ꞌ}ua{\ꞌ}ua} ‘to drizzle’\\
  \textit{taŋi} ‘to cry’ \> \textit{taŋitaŋi} ‘to cry intermittently’\\
  \textit{tere} ‘to run, travel’ \> \textit{teretere} ‘to tack (in sailing)’\\
  \textit{hiŋa} ‘to fall’  \>\textit{hiŋahiŋa} ‘to totter, stagger (to fall \\
 \> ~~~a little again and again)’\\
  \textit{vānaŋa} ‘to talk’ \> \textit{vānaŋanaŋa} / \textit{vanavanaŋa} ‘to chatter, \\
 \> ~~~do small talk’
\end{tabbing}
\z 
With adjectives, the reduplication\is{Reduplication} may indicate a weaker, ‘more or less’ sense. I have found this sense only with one adjective; it is probably not accidental that in this case, an intensified sense (‘very cooked’) does not fit in well with the semantics of the word.

\ea 
\begin{tabbing}
xxxxxxxxxxxxxxxxxxxxxxx \= xxxxxxxxxxxxxxxxxx \= xxxxxxxxxxxxxx \kill
  \textit{mata} ‘ripe; cooked’ \> \textit{matamata} ‘half-ripe; half-cooked’
\end{tabbing}\z 
\subsection{Reduplications without base form}\label{sec:2.6.3}

There are a number of type 2 reduplications\is{Reduplication} for which the base does not occur on its own. Most of these are either nouns or adjectives with a bisyllabic base: \textit{hiohio} ‘strong’, \textit{kutakuta} ‘foam’, \textit{rairai} ‘thin, flat’, \textit{naonao} ‘mosquito’, \textit{rohirohi} ‘tired’, \textit{tokotoko} ‘walking stick’. Examples with a trisyllabic base are \textit{māuruuru} ‘to thank; thank you’ and \textit{māuiui} ‘sick’. Sometimes there is evidence that the simple form did exist in Rapa Nui: \textit{paka} ‘dry’ is found in older texts, but in modern Rapa Nui only \textit{pakapaka} is used. Other forms (e.g. \textit{naonao} and \textit{māuruuru}) were borrowed as a whole from \ili{Tahitian}.

Some of these reduplication\is{Reduplication}{}-only forms have a plural of type 1, based on the root: \textit{kaokao} ‘narrow’, \textit{kakao} ‘narrow (Pl)’; \textit{ka{\ꞌ}ika{\ꞌ}i} ‘sharp’, \textit{kaka{\ꞌ}i} ‘sharp (Pl)’.\footnote{\label{fn:85}Interestingly, this leads to a situation where the plural is shorter than the corresponding singular (cf. \citealt[40]{Blust2001}).}

There are also verbs which have the shape of a type 1 reduplication\is{Reduplication} ($\sigma $\textsubscript{1} $\sigma $\textsubscript{1} $\sigma $\textsubscript{2}), but for which the non-reduplicated form does not occur: \textit{{\ꞌ}a{\ꞌ}aru} ‘to grab’, \textit{totoi} ‘to drag’, \textit{nēne{\ꞌ}i} ‘to defecate’, \textit{nono{\ꞌ}i} ‘to ask, beg’. For some of these, it is clear that the base form was known in the past: \textit{ne{\ꞌ}i} ‘defecate’ occurs in Englert’s dictionary, \textit{toi} is found once in an older text, but neither is used nowadays. For other verbs such as \textit{{\ꞌ}a{\ꞌ}aru}, the base form is not attested at all. Even so, they are treated as reduplications\is{Reduplication} in the accepted orthography (i.e. they are written with a hyphen), because a type 2 reduplication\is{Reduplication} of the same base does exist with a typical type 2 sense such as iterativity\is{Iterative}. For example, while there is no simple form *\textit{{\ꞌ}aru}, there is a type 2 reduplication\is{Reduplication} \textit{{\ꞌ}aru{\ꞌ}aru} ‘to grab several things’; hence, \textit{{\ꞌ}a{\ꞌ}aru} is considered a type 1 reduplication\is{Reduplication} and written with a hyphen (\textit{{\ꞌ}a\nobreakdash-{\ꞌ}aru}). 

In fact, most words with identical first and second syllables\is{Syllable} can be considered reduplications\is{Reduplication} for one of the reasons above. Exceptions are e.g. \textit{{\ꞌ}a{\ꞌ}amu} ‘story’ (neither \textit{*{\ꞌ}amu} nor \textit{*{\ꞌ}amu{\ꞌ}amu} is attested), \textit{rarama} ‘inspect’ (there are no related lexemes \textit{*rama} or \textit{*rama\-rama}), and \textit{tātara} ‘to make a speech’ (there are no related lexemes \textit{*tara} or \textit{*taratara} in Rapa Nui, though \is{Proto-Polynesian}PPN \textit{*tala} ‘to talk; story’ has reflexes in many other languages).
\is{Reduplication|)}
\section{Conclusions}\label{sec:2.7}

The preceding sections have given an overview of Rapa Nui phonology. The phoneme inventory of Rapa Nui is small (10 consonants, 5 short and 5 long vowels) and closely reflects the phoneme inventory of Rapa Nui’s protolanguages. The glottal plosive is contrastive in lexical words, both word-initially and word-medially, but acoustic analysis shows that it is not contrastive phrase-initially. This means that it is not contrastive in certain prenuclear particles; the latter tend to have a glottal only when they occur at the start of a prosodic phrase.

The syllable structure of Rapa Nui is (C)V(ː). There are no (C)V\textsubscript{1}V\textsubscript{2} syllables: sequences of non-identical vowels are analysed as disyllabic. One argument for this is stress assignment: the second vowel of a VV sequence may be stressed, which shows that it does not form a syllable with the preceding vowel. Another argument is word structure. Rapa Nui has a strict constraint on the metrical structure of words: long (i.e. heavy) syllables cannot be followed by an odd number of morae; in other words, the penultimate syllable cannot be long when the final syllable is short. This means that a word like \textit{mauku} ‘grass’ must be trisyllabic, as a long penultimate syllable \textit{mau-} would be metrically impossible.

Stress – both on word and phrase level – falls on the penultimate mora; in connected speech, stress is assigned on the phrase level. Interestingly, all postnuclear elements are (minimally) bimoraic, which avoids a possible conflict between word and phrase stress.

Two phonological processes which are regular but optional, are word-final vowel devoicing and pre-stress lengthening. The former is especially common.

A wide range of sporadic sound changes can be detected in the lexicon, resulting either in variants within Rapa Nui, or irregular reflexes of protoforms. Metathesis is rampant; other sound changes especially affect vowels, glottals and the liquid /r/.

Borrowings – especially from \ili{Spanish} – tend to be adjusted to the phonology of Rapa Nui, but in various ways and to varying degrees. Some non-native phonological features are more liable to be accepted (hence not adjusted) than others, especially certain word-medial consonant clusters and the fricative /s/.

Finally, this chapter deals with reduplication. Two basic types can be distinguished: monomoraic (expressing plurality) and bimoraic (expressing repetition or intensity). Re\-du\-plication may be full or partial, but there is no principled distinction between the two: whether all or part of the root is reduplicated, simply depends on the size of the root.

Depending on the prosodic shape of the root, various processes of lengthening and shortening take place; these can be explained by metrical constraints, most of which correspond to general phonological tendencies in the language.

\newpage 
Further research could throw more light on the following areas:

\begin{itemize}
\item 
the pronunciation of vowels (formant frequencies);

\item 
the phonetic correlates of stress (loudness, pitch);

\item 
levels of stress (especially on phrase level);

\item 
intonation patterns.

\end{itemize}
