\chapter[Nouns and verbs]{Nouns and verbs}\label{ch:3}
\section{Introduction: word classes in Rapa Nui}\label{sec:3.1}
This chapter and the next deal with the description of word classes. In this area, the most basic distinction in Rapa Nui – as in other Polynesian languages – is that between full words and particles.\footnote{\label{fn:86}\citet{Buse1965} uses these same terms for \ili{Rarotongan}. \citet{Biggs1961} uses the terms “bases” and “minor morphemes” for \ili{Māori}; in \citet{Biggs1973} the latter term has been replaced by “particles”. \citet[71]{MoselHovdhaugen1992} distinguish full words, particles, proforms and interjections in \ili{Samoan}.} \textsc{Full words}\is{Full word} occur in the nucleus of a phrase and mostly form large, open classes (though certain types of full words, such as locationals\is{Locational}, are closed classes). \textsc{Particles} are a closed class: they can be exhaustively listed. They occur in fixed positions before or after the nucleus, and most of them are highly frequent.

In Rapa Nui, full words and postnuclear particles\is{Particle!postnuclear} have a minimal length of two morae\is{Mora}; prenuclear particles\is{Particle!prenuclear} may be one mora\is{Mora}.

\is{Pro-form}\textsc{Pro-forms}\is{Pro-form} have an intermediate status between full words and particles. Like full words, they occur in the nucleus of a phrase and can be preceded and followed by particles. Unlike full words, they do not have a lexical meaning, and like particles, they form a closed class. Pro-forms include personal, possessive and benefactive pronouns\is{Pronoun!benefactive}, as well as interrogative words. 

Two other intermediate categories are the negator \textit{{\ꞌ}ina}\is{ina (negator)@{\ꞌ}ina (negator)} and the numerals. Both of these form a closed class, yet they function as phrase nuclei, as they can be followed by postnuclear particles\is{Particle!postnuclear}, while numerals are also preceded by a particle. 

Full words\is{Full word} can be divided into word classes (parts of speech) on the basis of grammatical and semantic criteria. Some word classes can be defined by a single unambiguous criterion. These include the following:

\begin{itemize}
\item 
\textsc{Locationals}\is{Locational} (\sectref{sec:3.6}), a subclass of nouns, are immediately preceded by prepositions and do not take articles.

\item 
\textsc{Proper nouns}\is{Noun!proper} (\sectref{sec:3.3.2}) are preceded by the proper article\is{a (proper article)} \textit{a}.

\item 
\textsc{Cardinal numerals} (\sectref{sec:4.3.1}) are preceded by one of the numeral particles \textit{e}, \textit{ka} and \textit{hoko}.\footnote{\label{fn:87}By contrast, quantifiers\is{Quantifier} (\sectref{sec:4.4}) cannot be grouped together as a word class on the basis of distributional criteria, as different quantifiers\is{Quantifier} show a different distribution.} 

\end{itemize}

For verbs and common nouns the situation is much less clear\is{Noun!common}. In \sectref{sec:3.2}, the distinction between nouns and verbs is discussed, and various aspects of their interaction are explored. 

The remainder of this chapter discusses other issues concerning nouns (\sectref{sec:3.3}) and verbs (\sectref{sec:3.4}).\footnote{\label{fn:88}See also Chapters 5 and 7 on noun and verb phrases, respectively.} \sectref{sec:3.5} discusses adjectives, a subclass of verbs, while \sectref{sec:3.6} discusses locationals\is{Locational}, a subclass of nouns. Other – minor – word classes will be discussed in Chapter 4.

\section{Nouns and verbs}\label{sec:3.2}

There are three types of nouns in Rapa Nui: common nouns\is{Noun!common}, proper nouns\is{Noun!proper} and locationals\is{Locational}. Proper nouns\is{Noun!proper} and locationals\is{Locational} are easily distinguished from other types of nouns and from other word classes, as indicated above. For common nouns\is{Noun!common}, the distinction with other parts of speech – especially verbs – is less obvious. This section deals with the noun/verb distinction in Rapa Nui; in this discussion, \textit{noun} should be read as a shorthand for \textit{common noun}. \sectref{sec:3.2.1} deals with the question whether there is a distinction between nouns and verbs in the lexicon of Rapa Nui; the existence of this distinction has been denied for some Polynesian languages. As I will show in \sectref{sec:3.2.1.1}, there are various types of evidence suggesting that Rapa Nui does have a distinction between lexical nouns and verbs. A given word may seem to be both noun and verb, but in most cases the two are either lexically distinguished (often with unpredictable relationships between nominal and verbal meanings), or the word is primarily a verb which may enter in nominalised constructions. In \sectref{sec:3.2.1.2} I propose an analysis in terms of prototypes; this analysis captures the fact that there is a tendency to congruence of form and function: verbal meaning tends to go together with verbal syntax, nominal meaning with nominal syntax. At the same time, various non-prototypical types also occur: words and constructions having features of both nouns and verbs. The latter are discussed in the\textsubscript{} next subsections (\sectref{sec:3.2.2}–\ref{sec:3.2.4}). Finally, \sectref{sec:3.2.5} brings together evidence for a general tendency in Rapa Nui to maximise the nominal domain.

\subsection{The noun/verb distinction}\label{sec:3.2.1}

Polynesian languages are known to be very flexible in use of nouns and verbs: many words seem to be used both as nouns and verbs. This is also true for Rapa Nui. In \REF{ex:3.1} below, \textit{poki} ‘child’ occurs in a noun phrase (preceded by the article \textit{te}) which is subject of the clause; in \REF{ex:3.2}, it occurs in a verb phrase (preceded by the imperfective marker\is{Aspect marker} \textit{e}) which is the clause predicate:

\ea\label{ex:3.1}
\gll He pōrekoreko \textbf{te} \textbf{ŋā} \textbf{poki} {\ꞌ}i Tāhai.\\
\textsc{ntr} born:\textsc{red} \textsc{art} \textsc{pl} child at Tahai\\
\glt 
‘Children were born in Tahai.’ \textstyleExampleref{[Ley-4-08.10]}
\z

\ea\label{ex:3.2}
\gll Mai te hora era ō{\ꞌ}oku \textbf{e} \textbf{poki} nō {\ꞌ}ana... \\
from \textsc{art} time \textsc{dist} \textsc{poss.3sg.o} \textsc{ipfv} child just \textsc{cont} \\
\glt
‘From the time when I was a child...’ \textstyleExampleref{[R539-1.614]}
\z

Likewise, in the following examples, \textit{\mbox{{\ꞌ}a{\ꞌ}amu}} is first used as a noun ‘story’ (in a noun phrase functioning as clause subject), then as a verb ‘to tell’ (in a verb phrase functioning as clause predicate):

\ea\label{ex:3.3}
\gll ¿He parauti{\ꞌ}a \textbf{te} \textbf{{\ꞌ}a{\ꞌ}amu} \textit{nei?}\\
~\textsc{pred} truth \textsc{art} story \textsc{prox}\\

\glt 
‘Is this story true?’ \textstyleExampleref{ [R616.608]} 
\z

\ea\label{ex:3.4}
\gll \textbf{He} \textbf{{\ꞌ}a{\ꞌ}amu} ia e mātou i te {\ꞌ}ati ta{\ꞌ}ato{\ꞌ}a nei o tātou o Rapa Nui.\\
\textsc{ntr} tell then \textsc{ag} \textsc{1pl.excl} \textsc{acc} \textsc{art} problem all \textsc{prox} of \textsc{1pl.incl} of Rapa Nui\\

\glt
‘We told about all the problems we have on Rapa Nui.’ \textstyleExampleref{[R649.238]} 
\z

Like all Polynesian languages, Rapa Nui has hundreds of words which, like \textit{\mbox{{\ꞌ}a{\ꞌ}amu}}, are defined both as a noun and a verb (These will be discussed in more detail in \sectref{sec:3.2.2}). Moreover, there is no inflectional morphology in the language which would facilitate distinguishing nouns from verbs. It is therefore not surprising that the existence of a lexical noun/verb distinction in Polynesian languages has been denied.\footnote{\label{fn:89}See e.g. \citet{MoselHovdhaugen1992} for \ili{Samoan}, \citet{LazardPeltzer1991,LazardPeltzer2000} for \ili{Tahitian}, and \citet{ElbertPukui1979} for \ili{Hawaiian}. A similar approach recognises a large class of “generals” \citep{Biggs1961} or “universals” \citep{Biggs1973}, besides smaller classes of (pure) nouns and verbs. See e.g. \citet{Buse1963Verbal,Buse1965} for \ili{Rarotongan}, \citet{Tchekhoff1979} for \ili{Tongan} and \citet{Biggs1961,Biggs1973} for \ili{Māori}. See \citet{Vonen2000} for an overview of the different approaches.}  In such an approach, the terms \textit{noun} and \textit{verb} are used in a purely syntactic sense: whatever occurs in the nucleus of a noun phrase is a noun, whatever occurs in the nucleus of a verb phrase is a verb. Such a distinction is workable as there is a strict distinction between nominal and verbal phrases,\footnote{\label{fn:90}In some languages the distinction is not as strict. \citet[168]{Moyse-Faurie2005} points out that in \ili{East Futunan}, aspect markers\is{Aspect marker} and articles may co-occur.} a distinction which also applies in Rapa Nui.

Nevertheless, I will argue that there are good reasons to maintain a lexical distinction between noun and verbs. That is, words are defined as noun or verb in the lexicon. This does not mean that all occurrences of these words are completely and unambiguously nominal and/or verbal. Lexical verbs very commonly enter into constructions which have certain nominal features; less frequently, lexical nouns are used in constructions with certain verbal features (as in \REF{ex:3.2} above). Moreover, many words are specified as both noun and verb in the lexicon, as \textit{\mbox{{\ꞌ}a{\ꞌ}amu}} in (\ref{ex:3.3}–\ref{ex:3.4}) above.

\sectref{sec:3.2.1.1} lists reasons to maintain a lexical distinction between nouns and verbs. In addition, several reasons are given why a syntactic approach to the noun/verb distinction is unsatisfactory. \sectref{sec:3.2.1.2} proposes a definition of nouns and verbs in terms of prototypes. This approach maintains a lexical distinction between noun and verbs, while at the same time recognising that the two cannot always be unambiguously distinguished.

\subsubsection[Reasons to maintain a lexical noun/verb distinction]{Reasons to maintain a lexical noun/verb distinction}\label{sec:3.2.1.1}

Firstly: despite the flexibility in the use of nouns and verbs, the large majority of noun phrases have a nucleus denoting an entity, while an overwhelming majority of verb phrases have a nucleus denoting an event. While all action words can be used in nominal phrases, many entity words are never used in verb phrases, or only in very specific, uncommon contexts. For example, the word \textit{oho} ‘go’ is very often preceded by the imperfective marker \textit{e}, but the word \textit{taŋata} ‘man’ is never preceded by this particle in the text corpus. Other words, like \textit{\mbox{{\ꞌ}a{\ꞌ}amu}} in (\ref{ex:3.3}–\ref{ex:3.4}) above, are commonly used both as noun and as verb, but with a different sense. Somehow generalisations like these should be accounted for in the grammar. To assume one large class of words, which can be indiscriminately slotted into noun or verb phrases, does not do justice to actual usage.

A second reason not to abandon the notion of nouns and verbs in the lexicon, is that the semantic relationship between nouns and verbs is not always predictable. In other words, it is not always possible to derive the nominal and verbal meanings of a word from an underlying acategorial sense. This will be illustrated in \sectref{sec:3.2.2}. This could be accounted for by analysing nouns and verbs of the same form as homophones (i.e. separate lexical items), but in that case the relationship between corresponding nouns and verbs is lost: under a homophone analysis, a lexical item used in a noun phrase is unrelated to an identical-sounding item with a related meaning in a verb phrase. This is unsatisfactory from a semantic point of view, for even though the relation between nominal and verbal senses may be unpredictable, the senses are always clearly related. 

A third argument that the apparent freedom of use does not imply the absence of lexical categories, comes from a rare phenomenon: very occasionally, words from other (minor) word classes are used as a noun or a verb. For example, a pronoun may occur in the nucleus of a verb phrase as in \REF{ex:3.5}; demonstrative particles may be the nucleus of a verb phrase, as in \REF{ex:3.6}:\footnote{\label{fn:91}Postverbal \textit{era}\is{era (distal)!postverbal} \textit{{\ꞌ}ā} indicates a finished action (sometimes equivalent to a perfect or pluperfect).} 

\ea\label{ex:3.5}
\gll ¿Ko {\ꞌ}ite {\ꞌ}ā kōrua he aha i \textbf{mātou} ai?\\
~\textsc{prf} know \textsc{cont} \textsc{2pl} \textsc{pred} what \textsc{pfv} \textsc{1pl.excl} \textsc{pvp}\\

\glt 
‘Do you know what we did?’ \textstyleExampleref{[Notes N. Weber]}
\z

\ea\label{ex:3.6}
\gll —¿Ku oti {\ꞌ}ā? —¡\textbf{Ko} \textbf{era} \textbf{{\ꞌ}ā} ta{\ꞌ}a me{\ꞌ}e. \\
~~~~~\textsc{prf} finish \textsc{cont}  ~~~~\textsc{prf} \textsc{dist} \textsc{cont} \textsc{poss.2sg.a} thing \\

\glt
‘—Is it finished? —I’m done (lit. something like ‘there is your thing’).’ \textstyleExampleref{[R230.105]} 
\z

These words belong to well-defined categories (pronouns, demonstratives\is{Demonstrative}), so it is clear that they are not acategorial; yet they occur in a noun phrase or verb phrase. This suggests that the absence of a strict boundary between word classes can be explained by freedom of cross-categorial use rather than the absence of lexical categories: the possibility of cross-categorial use is present in the grammar anyway.

We may conclude that the distinction between lexical nouns and verb should be maintained. 

In addition, there are a number of reasons why the syntactic approach to nouns and verbs common in Polynesian linguistics is unsatisactory. In this approach, nounhood and verbhood depends wholly on syntactic criteria: a word is a noun when it is the nucleus of a noun phrase, and a verb when it is the nucleus of a verb phrase. This can be further reduced to a single criterion: a word is a noun when preceded by a determiner, and a verb when preceded by an aspect marker\is{Aspect marker}.\footnote{\label{fn:92}For examples of this approach, see \citet[19]{Biggs1973}; \citet[76]{MoselHovdhaugen1992}; \citet[21]{LazardPeltzer2000}.} There are syntactic, semantic and pragmatic problems with this assumption.

Syntactic: the presence of a determiner does not necessarily mean that the phrase is entirely nominal. The nucleus of such a phrase may control verbal arguments:\footnote{\label{fn:93}\citet{Waite1994}, working in a generative framework, captures this insight by proposing that D (=determiner) in \ili{Māori} can take not only NP complements, but VP and AdjP as well. This means that a verb may occur in a nominal context (DP) while retaining its verbal character.} in \REF{ex:3.7} below, the subject of \textit{vānaŋa} has the agent marker \textit{e}; in \REF{ex:3.8}, \textit{runurunu} is followed by a direct object marked with the accusative marker\is{i (accusative marker)} \textit{i}. 

\ea\label{ex:3.7}
\gll I oti era \textbf{te} \textbf{vānaŋa} e te vi{\ꞌ}e...\\
\textsc{pfv} finish \textsc{dist} \textsc{art} speak \textsc{ag} \textsc{art} woman\\

\glt 
‘When the woman had finished speaking...’ \textstyleExampleref{[Egt-01.095]}
\z

\ea\label{ex:3.8}
\gll He turu mai ia ki te hare hāpī koia ko \textbf{te} \textbf{runurunu} mai i te rāua tūava.\\
\textsc{ntr} go\_down hither then to \textsc{art} house learn \textsc{com} \textsc{prom} \textsc{art} gather:\textsc{red} hither \textsc{acc} \textsc{art} \textsc{3pl} guava\\

\glt
‘They went down to school, while picking (lit. with the picking) guavas.’ \textstyleExampleref{[R211.012]} 
\z

The phrase may also contain other VP elements like directionals\is{Directional}, such as \textit{mai} in \REF{ex:3.8}. These elements do not appear in ordinary noun phrases, i.e. phrases headed by entity words like \textit{taŋata} ‘man’ or \textit{hare} ‘house’. In conclusion, a phrase introduced by a determiner may still have certain VP characteristics.

Semantic: despite the presence of a determiner, the nucleus may have a verbal sense, referring to an event rather than an entity. Even though it is preceded by a determiner, the verb may therefore have a different meaning from a real noun with the same form.\footnote{\label{fn:94}The same point is made by \citet[511]{Besnier2000} for \ili{Tuvaluan}.} This can be illustrated with the word \textit{vānaŋa}, which may denote an action ‘to talk’, or an entity ‘word, spoken utterance’. In \REF{ex:3.9} \textit{vānaŋa} denotes an event and occurs in a verb phrase (preceded by the aspect marker \textit{e}); in \REF{ex:3.10} it denotes an entity and occurs in a noun phrase (preceded by the article \textit{te}):

\ea\label{ex:3.9}
\gll \textbf{E} \textbf{vānaŋa} \textbf{rō} \textbf{mai} \textbf{{\ꞌ}ā} paurō te mahana ki a au.\\
\textsc{ipfv} speak \textsc{emph} hither \textsc{cont} every \textsc{art} day to \textsc{prop} \textsc{1sg}\\

\glt 
‘Every day he speaks to me.’ \textstyleExampleref{[R655.018]} 
\z

\ea\label{ex:3.10}
\gll Ka tai{\ꞌ}o pūai \textbf{te} \textbf{ŋā} \textbf{vānaŋa} \textbf{nei}: raŋi, rano, rapu.\\
\textsc{imp} read strong \textsc{art} \textsc{pl} word \textsc{prox} \textit{raŋi} \textit{rano} \textit{rapu}\\

\glt
‘Read the following words aloud: \textit{raŋi, rano, rapu}.’ \textstyleExampleref{[R616.147]} 
\z

Clearly, in \REF{ex:3.9} \textit{vānaŋa} is a verb, while in \REF{ex:3.10} it is a noun. So far, so good. In \REF{ex:3.7} above however, \textit{vānaŋa} denotes an event, even though it is preceded by a determiner. It serves as the complement of \textit{oti}, a verb which commonly takes a nominalised event word as complement. Thus, \textit{te vānaŋa} in \REF{ex:3.7} is not nominal in the same way as \textit{te ŋā vānaŋa nei} in \REF{ex:3.10}. Notice that this semantic difference correlates with certain syntactic differences: in \REF{ex:3.10}, \textit{vānaŋa} is preceded by the plural marker \textit{ŋā}, a noun phrase element; in \REF{ex:3.7} it is followed by a subject marked with the agentive \textit{e}, something to be expected of a verb. 

Pragmatic: in some constructions, a nominal phrase is syntactically not a clause predicate, yet it expresses an event and functions as a predicate pragmatically. This happens in the nominalised actor-emphatic\is{Actor-emphatic construction} construction, in which the actor is expressed as a possessive, while the event is expressed in a noun phrase (\sectref{sec:8.6.3}). Syntactically these constructions are nominal clauses\is{Clause!nominal} with the nominalised verb\is{Verb!nominalised} as subject; pragmatically, however, they express an event with the verb as nucleus. One example:

\ea\label{ex:3.11}
\gll {\ꞌ}Ā{\ꞌ}ana \textbf{te} \textbf{haka} \textbf{tere} i te henua.\\
\textsc{poss.3sg.a} \textsc{art} \textsc{caus} run \textsc{acc} \textsc{art} land\\

\glt
‘He (was the one who) governed the country.’ \textstyleExampleref{[R370.005]} 
\z

Constructions such as \REF{ex:3.11} are only found with event words, not with entity words. If the underlined phrases were regarded as noun phrases because of the presence of the article, they would be undistinguishable from “normal” noun phrases, which never enter into this construction.

We may conclude that it would be unsatisfactory to consider a word as noun whenever it is preceded by a determiner. Event words preceded by determiners may have either a nominal sense and nominal function, or a verbal sense and verbal function (possibly with verbal syntactic trappings). This suggests that we should make a distinction between \textsc{lexical nominalisation}\is{Nominalisation!lexical}, which turns a verb into a true noun, and \textsc{syntactic nominalisation}\is{Nominalisation!syntactic}, in which a verb is used as nucleus of a noun phrase, while retaining its verbal meaning and other verbal characteristics, such as the possibility to take verbal arguments. The examples above show that both occur in Rapa Nui: \REF{ex:3.10} is an example of lexical nominalisation\is{Nominalisation!lexical}, while \REF{ex:3.7} is an example of syntactic nominalisation\is{Nominalisation!syntactic}. These processes will be discussed in sections \sectref{sec:3.2.2} and \sectref{sec:3.2.3}, respectively.

\largerpage
\subsubsection[Prototypicality]{Prototypicality}\label{sec:3.2.1.2}
\is{Prototype}
As discussed in the previous section, it would be unsatisfactory to deny a basic distinction between nouns and verbs: most instances are either nominal or verbal both syntactically and semantically; the semantic relation between nominal and verbal uses of a word is often predictable, hence lexical; and the fact that words from minor (and well-defined) word classess can be used cross-categorially suggests that the freedom of use of nouns and verbs can also be accounted for as crosscategorial use, rather than by an absence of categories. Moreover, an alternative approach, which defines nounhood in terms of the occurrence of a noun phrase (minimially defined by a determiner) does not do justice to the often very verbal nature of noun phrases: a phrase which seems to be a noun phrase because of the presence of a determiner, may yet have a strongly verbal character. It may contain certain VP elements, while certain NP elements are excluded; it may function as a predicate; it may denote an event rather than an entity. 

\largerpage
The nominal and a verbal domain are not divided by a sharp boundary in Rapa Nui. Rather, “verbness” and “nounness” can be conceived of as a continuum, defined in terms of prototypes: at one end there are constructions which are entirely nominal (prototypical\is{Prototype} nouns), at the other end there are constructions which are entirely verbal (prototypical\is{Prototype} verbs). In between is a range of constructions which share characteristics of both.\footnote{\label{fn:95}See \citet[34–38]{Payne1997} for discussion of prototypicality in word classes. \citet{Croft2000} defines word classes as unmarked combinations of a pragmatic function and a lexical semantic class: 
\begin{tabbing}
xxxx \= xxxxxxxxxxxx \= xxxxxxxxxxxxxxxx \= xxxxxxxx \= xxxxxxxxx \kill
\> \textit{word class} \> \textit{pragmatic function} \> \>  \textit{semantic class}\\
\> noun \> reference \> to an \> object\\
\> adjective \> modification \> by a\>  property\\
\> verb \> predication \> of an\>  action
\end{tabbing}
Other combinations are possible: an object word may be used in predication (predicate nominals), action words may be used as modifier (e.g. participles), et cetera. Croft reserves the terms \textit{noun} and \textit{verb} for the unmarked combinations, i.e. prototypical\is{Prototype} nouns and verbs.} 

As the discussion above has made clear, prototypical\is{Prototype} nouns and verbs cannot be defined solely on the basis of lexical meaning, nor solely on the basis of syntactic properties. Rather, a prototypical\is{Prototype} form combines syntactic, semantic and pragmatic characteristics. I suggest the following definitions:

A \textsc{prototypical}\is{Prototype}\textsc{ verb} is a word which

\begin{itemize}
\item 
denotes an event or action;

\item 
functions as the predicate of the clause;

\item 
occurs as head of a verb phrase. A prototypical\is{Prototype} verb phrase has an aspect or mood marker and may contain various other elements, expressing for example aspectual nuances, degree and direction;

\item 
governs canonical arguments such as subject and/or direct object.

\end{itemize}

A \textsc{prototypical}\is{Prototype}\textsc{ noun} is a word which

\begin{itemize}
\item 
denotes a concrete entity;

\item 
is used as a referring expression;

\item 
occurs in a noun phrase. A prototypical\is{Prototype} noun phrase contains a determiner and may contain various other elements with quantifying, deictic and anaphoric functions;

\item 
may take a possessor\is{Possession} to express various relations with a dependent noun.

\end{itemize}

This approach enables us to account for flexibility in use, while at the same time maintaining the basic noun-verb distinction: \textit{taŋata} ‘man’ can be defined as a noun, even though it occasionally occurs in a verb phrase; the latter is simply a case of non-prototypical\is{Prototype} use.\footnote{\label{fn:96}\citet[257]{Besnier2000} takes a similar approach for \ili{Tuvaluan}: each word in the language has one basic word-class membership; use of the same word in other word classes is marked (e.g. nominals may be used as a verb, but this use is less frequent than their use as a noun and may be subject to structural restrictions). \citet[184]{Moyse-Faurie2005} likewise argues for an approach starting from the prototypical sense and function of a lexeme (as opposed to an approach based on a syntactic dichotomy between noun phrase and verb phrase).}

Between prototypical\is{Prototype} nouns and prototypical\is{Prototype} verbs lies a whole range of non-prototyp\-i\-cal\is{Prototype} constructions, as illustrated above. Any attempt to divide this area up by drawing a line separating the “noun area” from the “verb area” is arbitrary. However, for practical reasons I will use the term \textit{verb} for any word which is lexically (i.e. semantically) a verb, and \textit{noun} for any word which is lexically a noun. Thus, in the examples above, the underlined lexical item in (\ref{ex:3.1}–\ref{ex:3.3}) and \REF{ex:3.10} is called a noun, while the underlined word in \REF{ex:3.9} and \REF{ex:3.11} is called a verb. \textit{Vānaŋa} is a verb when it denotes the action ‘to talk’, whether it occurs in a prototypical\is{Prototype} VP or in a phrase that also has nominal properties. When \textit{vānaŋa} denotes an entity ‘word, utterance’, it is a noun. As these two senses are obviously related, the relation between the two can be defined as polysemy (one lexical item having two related but not identical senses) rather than homophony (two unrelated lexical items which happen to share the same phonological form). 

For many words, the semantic criterion is sufficient to classify them as either nouns or verbs: they primarily designate either an entity or an event. However, with non-concrete words this criterion does not work as well; it is not always obvious whether a notion should be classified as an event or an entity. Take for example natural phenomena: without a syntactic context, should ‘rain’\is{Verb!weather} be classified as an event (‘it rains’) or an entity (‘the rain falls’)? Is ‘flood’ a thing or an event? The same is true for abstract nouns (\textit{haŋa} ‘to love; love’; \textit{mana{\ꞌ}u} ‘to think; thought’). Such words are hard or impossible to assign to a word class apart from syntax; they are by nature not prototypical nouns or verbs. For these words, therefore, syntactic criteria are needed to assign them to a word class. One possible criterion is the presence of a determiner, but as discussed above, this is not a very strong clue; the determiner is a very weak criterion for nounhood. There are other syntactic criteria, however:

\begin{itemize}
\item 
Verbs may be modified by VP elements (see Chapter 7 and \sectref{sec:3.2.3.3}): apart from aspect/mood markers, there may be degree modifiers, the constituent negator \textit{ta{\ꞌ}e}, directionals\is{Directional}.

\item 
Verbs may take arguments which are marked as subject, object or oblique.

\item 
Nouns may be modified by NP elements (see Chapter 5): quantifiers\is{Quantifier}, numerals and a plural marker.

\item 
Nouns may take a possessor\is{Possession}.

\end{itemize}

These criteria tend to converge into the same direction; for concrete words, this is usually the same direction as suggested by the semantics of the word: an entity word is usually modified by NP elements and may take a possessor\is{Possession}; an event word is usually modified by VP elements and may take canonical arguments. In other words, nouns and verbs tend to show prototypical\is{Prototype} behaviour.\footnote{\label{fn:97}\citet[96]{Croft2000} points out that the meaning of words tends to shift towards the unmarked sense associated with their syntactic use: action words used in referring expressions tend to denote an object typically associated with the activity (e.g. ‘learn+NOM’ {\textgreater} ‘school’); object words used as a predicate tend to denote an action typically associated with the object (e.g. ‘baggage+V’ {\textgreater} ‘to pack’).} The same syntactic criteria can now be used to assign abstract words (for which the semantics do not provide a strong clue) to a word class. For an example of how syntax can help to categorise an abstract word, see the discussion on \REF{ex:3.14} in \sectref{sec:3.2.2.1} below.

In two cases, there are morphological clues for noun- or verbhood.

%\setcounter{listWWviiiNumxxxviileveli}{0}
\begin{enumerate}
\item 
The causative\is{Causative} prefix \textit{haka}\is{haka (causative)} (\sectref{sec:8.12}) turns a root into a verb. There are a few cases where \textit{haka} + root is lexicalised as something else than a verb (e.g. \textit{haka{\ꞌ}ou} ‘again’, \textit{hakanonoŋa} ‘fishing zone’), but the vast majority of \textit{haka} forms are verbs. However, like all verbs, they may take on certain nominal roles and function as a noun phrase head: see e.g. \REF{ex:3.29} below.

\item 
The nominalising\is{Nominalisation} suffix, usually \textit{haŋa}\is{hanza (nominaliser)@haŋa (nominaliser)} or \textit{iŋa}\is{inza (nominaliser)@iŋa (nominaliser)} (\sectref{sec:3.2.3}), turns a root into a noun. As discussed in \sectref{sec:3.2.3.1}, the resulting forms have a more nominal sense than non-suffixed verb forms, and are used in nominal contexts.

\end{enumerate}

In the following sections, the area between prototypical\is{Prototype} nouns and prototypical\is{Prototype} verbs is further explored. \sectref{sec:3.2.2} discusses lexical noun/verb pairs; \sectref{sec:3.2.3} discusses syntactic nominalisation\is{Nominalisation!syntactic}; \sectref{sec:3.2.4} briefly discusses the use of nouns in verbal contexts.

\subsection{Lexical noun/verb correspondences}\label{sec:3.2.2}
\is{Nominalisation|(}
\is{Verb!nominalised|(}
Many words in Rapa Nui are used both as nouns and as verbs, without any difference in form but with a difference in meaning. As discussed in the previous section, these are best considered as cases of polysemy, a single lexical item having both a nominal and a verbal sense. 

First a note on terminology. Common terms like \textit{nominalisation}\is{Nominalisation!lexical} and \textit{deverbal noun} indicate that a noun is derived from a verb. While this is often the case, for other words the verb is derived from the noun, or the direction of derivation is undetermined. As the verb and the noun are identical in form, there are no morphological clues for the direction of derivation. For this reason the neutral term \textit{noun/verb correspondences} is used here. 

\newpage 
\sectref{sec:3.2.2.1} explores the semantic relationships between these homophonous noun/verb pairs.\footnote{\label{fn:98}To obtain the data for this section, I listed all words in my lexical database that have both a nominal and a verbal definition. As this database incorporates data from all previous dictionaries and word lists of Rapa Nui, it includes many doubtful definitions, translational equivalents for which it is not clear that the word is actually used in that particular sense. Besides, the lexical resources include many words not attested in the text corpus, either because they are obsolete or because the corpus is limited in size. This leaves just over 200 words that are attested in the corpus in both verbal and nominal senses; it is from these words that the data in this section are taken.} \sectref{sec:3.2.2.2} discusses the – much rarer – derivations involving a nominalising\is{Nominalisation} suffix.\footnote{\label{fn:99}Apart from the nominalising\is{Nominalisation} suffix and the causative\is{Causative} prefix, there are no productive derivative affixes in Rapa Nui. The lexicon does show traces of derivative suffixes, but in all cases the word was certainly or probably borrowed or inherited as a whole. For example, \textit{tāmiti} ‘to salt, cure’ is obviously related to \textit{miti} ‘salt’, but \textit{tāmiti} was probably borrowed from \ili{Tahitian}\is{Tahitian influence}, where \textit{tā-} frequently occurs as a (non-productive) factive prefix.} 

\subsubsection[Homophonous noun/verb pairs]{Homophonous noun/verb pairs}\label{sec:3.2.2.1}

\paragraph{Concrete entities}\label{sec:3.2.2.1.1} In many cases, the noun denotes a concrete entity (an object or person), while the verb denotes an activity in which this entity plays a certain role. Different semantic relationships can be discerned:

\subparagraph{Instrument} The noun indicates a physical object, while the verb denotes an action performed with that object as instrument: ‘to use N, to do something with N’. Examples: \textit{hiahia} ‘saw;\footnote{\label{fn:100}\ili{English} definitions not preceded by ‘to’ are nouns.} to saw’; \textit{hoe} ‘paddle; to paddle’; \textit{harihari} ‘comb; to comb hair’. 

Sometimes the verb is more specific in sense than the noun: \textit{rama}\textit{\textsubscript{N}} ‘torch’, \textit{rama}\textit{\textsubscript{V}} ‘to fish with a torch’ (a fishing technique done at night). In other cases the noun is more specific: \textit{raŋo}\textit{\textsubscript{V}} ‘to support’, \textit{raŋo}\textit{\textsubscript{N}} ‘stretcher, handbarrow’; \textit{haŋuhaŋu}\textit{\textsubscript{V}} ‘to pant, breathe heavily’, \textit{haŋuhaŋu}\textit{\textsubscript{N}} ‘bellows; forge’. 

\subparagraph{Product} The noun denotes the product or result of the action, often a concrete object. Examples: \textit{hoho{\ꞌ}a}\textit{\textsubscript{}} ‘to take a picture; a picture’; \textit{taka} ‘to roll up; a roll, spool’; \textit{tūtia} ‘to sacrifice; offering’; \textit{tarakī} ‘to dry meat; dried meat’. One of the senses may be more specialised: \textit{pū}\textit{\textsubscript{N}} ‘hole’; \textit{pū}\textit{\textsubscript{V}} has the underlying basic sense ‘to make a hole, pierce, perforate’ but is only used in several specific senses: ‘to hit with a bullet, to hook a fish, to dig out tubers’. \textit{Para}\textit{\textsubscript{V}} has a wide range of senses: ‘to decay, ripen, rot, rust’, while \textit{para}\textit{\textsubscript{N}} only means ‘rust’.

\subparagraph{Utterance} Similar to the preceding cases are verbs of speaking\is{Verb!speech}, where the corresponding noun expresses the utterance produced by the action of speaking: \textit{vānaŋa} ‘to speak; word, utterance’; \textit{\mbox{{\ꞌ}a{\ꞌ}amu}} ‘to tell; story’; \textit{reoreo} ‘to lie; a lie’. This category also includes \textit{mana{\ꞌ}u} ‘to think; thought’. It seems that all words in this semantic domain can be both verb and noun; however, the semantic relationship may be idiosyncratic: \textit{pure}\textit{\textsubscript{V}} ‘to pray’, \textit{pure}\textit{\textsubscript{N}} ‘prayer’ but also ‘mass’. Notice that the nominal sense of these words is not just ‘the act of performing X’: one can expose a lie or print a story, without being involved in the act of lying or storytelling itself. 

\subparagraph{Patient} Other cases in which the noun is the Patient of the corresponding verb are those in which the noun is an entity undergoing the action or affected by the action: \textit{kai} ‘to eat; food’; \textit{{\ꞌ}akaveŋa} ‘to carry on the back; basket carried on the back’.

\subparagraph{Agent} For a number of words, the noun denotes the Agent of the corresponding action. In some cases the noun denotes a profession: \textit{ha{\ꞌ}avā} ‘to judge; a judge’; \textit{tāvini} ‘to serve; servant’. For other words the Agent may be anyone who performs the action, whether incidentally or regularly: \textit{mata{\ꞌ}ite} ‘to testify; eyewitness’; \textit{reoreo} ‘to lie; liar’. 

\subparagraph{Location} Words indicating the place where the action happens, are rare. One example is \textit{haka iri} ‘to ascend; slope’. \textit{Hāpī} ‘to learn’ may be used in the sense ‘school’ (\textit{turu ki te hāpī} ‘go down to school’), but more commonly this sense is expressed by \textit{hare} \textit{hāpī} ‘house + learn’. 

\paragraph{Abstract words}\label{sec:3.2.2.1.2} For abstract words, it is harder to distinguish distinct nominal and verbal senses. Noun and verb often refer to the same ‘thing’, but with an aspectual difference: while the verb expresses an event taking place in time, the noun denotes the same event as a bounded whole. This suggests that the distinction is syntactic rather than lexical. 

\subparagraph{Natural phenomena} Many natural phenomena\is{Verb!weather} (e.g. meteorological conditions) can be expressed as either noun or verb. The following pair of examples illustrate this for \textit{a{\ꞌ}a} ‘flood’: in \REF{ex:3.12} it is a verb with the flooded object as subject, in \REF{ex:3.13} it is a noun in idiomatic collocation with the verb \textit{rere} ‘fly’.

\ea\label{ex:3.12}
\gll Ku \textbf{a{\ꞌ}a} {\ꞌ}ā te hare {\ꞌ}i te vai.\\
\textsc{prf} flood \textsc{cont} \textsc{art} house at \textsc{art} water\\

\glt 
‘The house was flooded with water.’ \textstyleExampleref{[Egt]}
\z

\ea\label{ex:3.13}
\gll He rere te \textbf{a{\ꞌ}a}. \\
\textsc{ntr} fly \textsc{art} flood \\

\glt
‘The flood came up.’ \textstyleExampleref{[Mtx-7-17.012]}
\z

Other words in this category only occur as nouns: \is{Verb!weather}\textit{{\ꞌ}ua} ‘rain’\footnote{\label{fn:101}Only rarely is \textit{{\ꞌ}ua} used as a verb, without a subject: \textit{e {\ꞌ}ua rō {\ꞌ}ā} ‘it was raining’ (R475.003).} (with \textit{hoa} ‘throw’: \textit{He hoa te {\ꞌ}ua}, ‘It rained’, lit. ‘The rain threw’); \textit{tokerau} ‘wind’ (often with \textit{puhi} ‘blow’ or \textit{hū} ‘roar’).

{\sloppy
\subparagraph{Human experiences} There is a large category of words for the expression of \mbox{human} experiences: feelings and propensities (\textit{mataku} ‘to be afraid; fear’; \textit{nounou} ‘to be greedy; greed’); physical experiences (\textit{mamae} ‘pain; to suffer pain’; \textit{mare} ‘asthma; to have asth\-ma’).
}

\subparagraph{Other abstract words} There are many other abstract words. Some of these express telic events, events with a natural endpoint; in that case the noun expresses a bounded entity, the event conceived as an object: \textit{hāipoipo} ‘to marry; wedding ceremony, wedding party’; \textit{{\ꞌ}ā{\ꞌ}ati} ‘to compete; competition’; \textit{tau{\ꞌ}a} ‘to fight; battle’. For other words the semantic distinction between the nominal and the verbal sense is less clear: \textit{hāpī} ‘to learn, to teach; schooling, lesson, education’; \textit{ha{\ꞌ}amata} ‘to begin; beginning’; \textit{ha{\ꞌ}uru} ‘to sleep; sleep’; \textit{mate} ‘to die; death’. 

It is questionable whether abstract nouns are lexically distinct from the corresponding verbs. In a few cases, the noun has developed more specific senses: \textit{makenu}\textit{\textsubscript{V}} ‘to move about’, \textit{makenu}\textit{\textsubscript{N}} ‘action, movement; development; party/feast’; \textit{rē}\textit{\textsubscript{V}} ‘to win’; \textit{rē}\textit{\textsubscript{N}} ‘victory; goal (in soccer)’. Further lexical research could show if other abstract words show subtle meaning differences between noun and verb. 

As suggested in \sectref{sec:3.2.1.2}, syntactic criteria could also help to determine the existence of lexical nouns and verbs. The consistent absence of verb phrase particles could indicate that the nucleus is a lexical noun, not a nominalised verb\is{Verb!nominalised}. Another syntactic criterion is the syntactic context in which the noun phrase appears. As discussed in \sectref{sec:3.2.3.1} below, in certain constructions nominalised verbs\is{Verb!nominalised} occur in noun phrases without a suffix, while in other nominal positions they tend to have a nominalising\is{Nominalisation} suffix. If a word occurs in one of the latter contexts \textit{without} a nominalising\is{Nominalisation} suffix, this suggests that it is a lexical noun. In the following example, \textit{mana{\ꞌ}u} ‘think’ and \textit{ŋaro{\ꞌ}a} ‘perceive’ both occur in the direct object position, a position in which verbs usually take a nominalising\is{Nominalisation} suffix. \textit{\mbox{Ŋaro{\ꞌ}a}} does indeed have the suffix \textit{iŋa}; \textit{mana{\ꞌ}u} however does not, which suggests that it is a lexical noun.

\ea\label{ex:3.14}
\gll A au e haka {\ꞌ}ite atu ena i tō{\ꞌ}oku \textbf{mana{\ꞌ}u}, i tō{\ꞌ}oku  \textbf{ŋaro{\ꞌ}a} \textbf{iŋa}.\\
\textsc{prop} \textsc{1sg} \textsc{ipfv} \textsc{caus} know away \textsc{med} \textsc{acc} \textsc{poss.1sg.o} think \textsc{acc} \textsc{poss.1sg.o}  perceive \textsc{nmlz}\\

\glt
‘I will make known what I think, what I feel.’ \textstyleExampleref{[R443.013–015]}
\z

In many other situations it is hard to classify the abstract word as a noun or a verb, and for these words the existence of a lexical noun/verb distinction could be called into question. For many concrete words, on the other hand, there is a clear lexical noun/verb distinction. As indicated above, the noun often denotes a participant in the event rather than the event itself. Moreover, either the verb or the noun may have idiosyncratic senses.

\largerpage
Another indication that nominal and verbal senses are lexically determined is the fact that many noun/verb pairs which could be expected to exist, do not occur.\footnote{\label{fn:102}\citet{Clark1983Maori} presents similar observations for \ili{Māori}.} A few examples:

\begin{itemize}
\item 
Some words express both an action and the agent of that action (1e above). Others, however, can only express the action itself: \textit{hāpī} ‘teach’, not ‘teacher’; \textit{aŋa} ‘to do, make’, not ‘builder’. \textit{Kori} means both ‘to steal’ and ‘thief’, but \textit{toke} means ‘to steal’, not ‘thief’.

\item 
Some words express both an action and an object brought about or affected by the action (1b–1d); others do not. \textit{Kai} ‘to eat; food’ is both a noun and a verb, but \textit{unu} ‘to drink’ is only a verb: one may \textit{kai i te kai} ‘to eat food’, but one cannot \textit{*unu i te unu} ‘to drink a drink’. \textit{Tarakī} ‘to dry meat; dried meat’ is both a noun and a verb, but other verbs of food preparation (like \textit{tunu} ‘to cook’, \textit{tunuahi} ‘to roast’) cannot be used in a nominal sense to refer to the cooked food.

\item 
Many objects have an action typically associated with them, which can be expressed by the corresponding verb: \textit{rama} ‘torch; to fish with a torch’; \textit{hoho{\ꞌ}a} ‘picture; take a picture’. Other objects also have an action typically associated with them, yet do not express that action with the same word: \textit{kahu} ‘clothes’, not ‘to be/get dressed’\textit{; hoi} ‘horse’, not ‘to ride a horse’; \textit{vaka} ‘canoe’, not ‘to travel by canoe’; \textit{mata} ‘eye’, not ‘to look’.

\end{itemize}

This confirms that noun/verb correspondences are – at least for certain words – defined in the lexicon. 

\subsubsection[Lexical nominalisation involving a suffix]{Lexical nominalisation involving a suffix}\label{sec:3.2.2.2}
\is{Nominalisation!lexical}
While hundreds of words in the Rapa Nui lexicon show zero derivation, cases of lexical nominalisation\is{Nominalisation!lexical} involving a nominalising\is{Nominalisation} suffix are much less numerous. As discussed in \sectref{sec:3.2.3.2} below, there are various nominalising\is{Nominalisation} suffixes, without a sharp distinction in meaning and use: \textit{-ŋa}, \textit{haŋa}\is{hanza (nominaliser)@haŋa (nominaliser)}, \textit{iŋa}\is{inza (nominaliser)@iŋa (nominaliser)}, \textit{aŋa}, \textit{eŋa}, \textit{oŋa}. In the standard Rapa Nui orthography (\sectref{sec:1.4.4}), all of these are written as separate words, with the exception of \textit{-ŋa}. All these forms can be used in lexical nominalisation\is{Nominalisation!lexical} as well as syntactic nominalisation\is{Nominalisation!syntactic}, often with the same verb. An extreme example is the verb \textit{noho} ‘to sit, stay’, which occurs with all suffixes: \textit{nohoŋa}, \textit{noho haŋa}, \textit{noho iŋa}, \textit{noho aŋa}, \textit{noho eŋa}, \textit{noho oŋa}; all of these may have the lexicalised sense ‘epoch, period’.

As discussed in the previous section, lexical noun/verb pairs without suffix may have various meaning correspondences. In the same way, suffixed nominalisations may be related to the root verb in various ways. Some indicate an object associated with the event or action: \textit{moe} ‘to lie’, \textit{moeŋa} ‘mat’; \textit{hatu} ‘to weave leaves’, \textit{hatuŋa} ‘woven roofing’; \textit{toe} ‘to remain’, \textit{toeŋa} ‘leftovers’.

Others refer to a place where the action is performed: \textit{puhi} ‘to fish for lobsters and eels at night’, \textit{puhiŋa} ‘a place where lobsters and eels are caught at night’; \textit{piko} ‘to hide’, \textit{pikoŋa} or \textit{piko haŋa} (both obsolete) ‘hiding place’. 

Other derivations yet have a more abstract sense. \textit{Noho} + \textsc{nmlz} is mentioned above. Another example is \textit{haka tere iŋa} ‘system, culture, religion’, from \textit{haka tere} ‘to lead, rule, govern’. 

All these examples concern lexical nominalisation. The use of the nominalising\is{Nominalisation} suffix in syntactic nominalisation\is{Nominalisation!syntactic} will be discussed in \sectref{sec:3.2.3} below.

\subsubsection[Cross{}-categorial use of borrowings]{Cross-categorial use of borrowings}\label{sec:3.2.2.3}
\is{Borrowing}
The Rapa Nui lexicon has incorporated a large number of \ili{Spanish} borrowings\is{Borrowing} (\sectref{sec:1.4.2}). These are used cross-categorially with great freedom. For many \ili{Spanish}\is{Spanish influence} noun/verb pairs, Rapa Nui has borrowed one form, usually either the noun or the verb in the 3\textsuperscript{rd} sg. present, and this form is used as both noun and verb. Below are two examples from the text corpus.\footnote{\label{fn:103}See also \citet{Makihara2001Adaptation}, who gives many examples from a corpus of spoken texts.} In \REF{ex:3.15}, \textit{rivuho}, originally a noun (Sp. \textit{dibujo} ‘drawing’), is used as a verb; in \REF{ex:3.16}, the verb \textit{agradece} (Sp. \textit{agradece} ‘gives thanks’) is used as a noun.

\ea\label{ex:3.15}
\gll Ku \textbf{rivuho} atu {\ꞌ}ā i tū {\ꞌ}avione era.\\
\textsc{prf} drawing away \textsc{cont} \textsc{acc} \textsc{dem} airplane \textsc{dist}\\

\glt 
‘They drew that airplane.’ \textstyleExampleref{[R379.057]} 
\z

\ea\label{ex:3.16}
\gll Me{\ꞌ}e huru kē tō{\ꞌ}ona \textbf{agradece} ki te hau nei he rapa nui. \\
thing manner different \textsc{poss.3sg.o} thank to \textsc{art} race \textsc{dist} \textsc{pred} Rapa Nui \\

\glt
‘Her gratitude for this race, the Rapa Nui, is exceptional.’ \textstyleExampleref{\citep[204]{Makihara2001Adaptation}} 
\z

\sectref{sec:3.2.2.1} showed that there are lexical restrictions and idiosyncrasies in the cross-cat\-e\-go\-rial use of Rapa Nui words. Further research could show whether similar restrictions apply in the use of borrowings\is{Borrowing}. 
\is{Verb!nominalised|)}

\subsection{Syntactic nominalisation}\label{sec:3.2.3}
\is{Nominalisation!syntactic|(}
Syntactic nominalisation\is{Nominalisation!syntactic} refers to constructions in which a lexical verb enters into a construction which has “some of the formal trappings of a noun phrase” \citep[65]{Clark1981}. As discussed in \sectref{sec:3.2.1.1}, no change in meaning is involved; the verb retains its verbal sense, while the phrase may retain VP characteristics. In Rapa Nui, the minimal criterion for nominalisation is that the verb is preceded by a determiner (see the inventory of determiners in \sectref{sec:5.3.1}). This is usually the article \textit{te}\is{te (article)}, occasionally a demonstrative determiner, but it may also be the nominal predicate marker \textit{he}\is{he (nominal predicate marker)}: see the discussion of \REF{ex:3.23} below.\footnote{\label{fn:104}\textit{Pace} \citet[136]{Moyse-Faurie2011}: “In Polynesian languages, only the specific article may nominalise a verb phrase”.}

Just like lexical nominalisation\is{Nominalisation!lexical}, syntactic nominalisation\is{Nominalisation!syntactic} occurs both with and without a nominalising suffix. In the first case, the suffix is usually \textit{iŋa} or \textit{haŋa}, occasionally \textit{eŋa}, \textit{aŋa} or \textit{oŋa}; the form of the suffix is discussed in \sectref{sec:3.2.3.2}. In the second case (zero nominalisation), the nominalised form is identical to the verb itself\is{Noun!verbal}.

\is{Nominalisation}In \sectref{sec:3.2.3.1} the use of both types of nominalisations is discussed. In \sectref{sec:3.2.3.3}, the nominalised phrase is examined in further detail, showing that this phrase retains certain verbal characteristics.

\subsubsection[Uses of zero and suffixed nominalisation]{Uses of zero and suffixed nominalisation}\label{sec:3.2.3.1}

The choice between zero and suffixed nominalisation depends to a large extent on the syntactic context. Generally speaking, zero nominalisations are used in more verbal contexts, while suffixed forms are used in more nominal contexts. However, there is no clear watershed between the two sets of contexts: in certain environments either one can be used. The difference between the two can be formulated as follows: zero nominalisation presents the event as an \textsc{event}, i.e. as something which has a temporal duration, and which may be progressive\is{Progressive} or habitual\is{Aspect!habitual}. Suffixed nominalisation noun\is{Noun!verbal} presents the event as an \textsc{object}, i.e. as a bounded entity, seen as a whole.\footnote{\label{fn:105}\citet[79]{Clark1981} makes a similar distinction, when he suggests “the hypothesis that unsuffixed nominalizations denote activities or processes [...] whereas suffixed nominalizations denote events, which can be enumerated and located in time”.} Often it refers to one particular occasion when the event took place, or to a set of such occasions. By contrast, zero forms may refer to potential occurrences. Broadly speaking, suffixed nominalisation are realis\is{Realis}, while zero nominalisations may be irrealis\is{Irrealis}.\footnote{\label{fn:106}Realis, as defined by \citet[244]{Payne1997}, asserts that an event has actually happened; the irrealis\is{Irrealis} mode does not assert that the event actually happened, nor that it did not happen (\sectref{sec:11.5.2}).}

The event/object distinction goes a long way towards explaining the distribution of both items. The different syntactic contexts will be listed and illustrated below, but here are some general observations. Zero nominalisations are commonly used as main clause predicate, a typical verbal context. Aspectual verbs\is{Verb!aspectual verb} like \textit{ha{\ꞌ}amata} ‘begin’ refer to the temporal structure of an event, so it is not surprising that they take a zero nominalisation as complement. By contrast, in typical nominal positions (subject, possessor...) suffixed forms are more common. 

When the event is negated (an irrealis\is{Irrealis} context), zero nominalisations are common, while suffixed forms are extremely rare. 

\ea\label{ex:3.17}
\gll {\ꞌ}I \textbf{te} \textbf{ta{\ꞌ}e} \textbf{hakaroŋo}, he ŋaro rō atu {\ꞌ}ai. \\
at \textsc{art} not listen \textsc{ntr} lost \textsc{emph} away \textsc{subs} \\

\glt
‘Because (the sheep) did not listen, it got lost.’ \textstyleExampleref{[R490.005]} 
\z

When the event is modified by a numeral (i.e. is countable), a verbal noun\is{Noun!verbal} is used:

\ea\label{ex:3.18}
\gll He take{\ꞌ}a mai ka teka \textbf{e} \textbf{rua} \textbf{haka} \textbf{teka} \textbf{iŋa} {\ꞌ}i muri o te motu. \\
\textsc{ntr} see hither \textsc{cntg} revolve \textsc{num} two \textsc{caus} revolve \textsc{nmlz} at near of \textsc{art} islet \\

\glt
‘I saw (the bird) making two rounds near the islet.’ \textstyleExampleref{[R338.014]} 
\z

Not all distributional facts are easily explained, though. Certain constructions take a suffixed nominalisation, even though they denote an event with temporal duration (e.g. the \textit{ko te V} construction in (\ref{ex:3.24}–\ref{ex:3.25}) below). On the other hand, a reason clause may refer to one particular instance, yet contain a zero nominalisation (see \REF{ex:3.29}). 

We may conclude that the choice between the two forms is partly based on semantics, partly conventionalised (certain constructions always or usually take one form), and partly free.

Regardless the syntactic position of the phrase, suffixed forms are used when the word refers to the place, time\footnote{\label{fn:107}The temporal sense is found with “stage words” (see \citealt[148]{Broschart1997}): certain adjectives like \textit{{\ꞌ}iti{\ꞌ}iti} ‘small’, and the noun \textit{poki} ‘child’: \textit{poki iŋa} ‘childhood’. In the corpus, \textit{poki} is the only noun taking the nominaliser.}  or manner of the event, as the following examples show:

\ea\label{ex:3.19}
\gll Tō{\ꞌ}ona \textbf{noho} \textbf{haŋa} {\ꞌ}i Ahu te Peu.\\
\textsc{poss.3sg.o} stay \textsc{nmlz} at Ahu te Peu\\

\glt 
‘He lived (lit. his living) in Ahu te Peu.’ \textstyleExampleref{[R233.002]} 
\z

\ea\label{ex:3.20}
\gll Kai ŋaro i a au mai tō{\ꞌ}oku \textbf{{\ꞌ}iti{\ꞌ}iti} \textbf{iŋa} {\ꞌ}ā ki te hora nei.\\
\textsc{neg.pfv} forgotten at \textsc{prop} \textsc{1sg} from \textsc{poss.1sg.o} small:\textsc{red} \textsc{nmlz} \textsc{ident} to \textsc{art} time \textsc{prox}\\

\glt 
‘I have not forgotten it from my childhood (lit. smallness) until now.’ \textstyleExampleref{[R416.936]} 
\z

\ea\label{ex:3.21}
\gll Pē nei te \textbf{aŋa} \textbf{haŋa} o te rā{\ꞌ}au nei.\\
like \textsc{prox} \textsc{art} make \textsc{nmlz} of \textsc{art} medicine \textsc{prox}\\

\glt 
‘This is how you make (lit. the making of) this medicine.’ \textstyleExampleref{[R313.159]} 
\z

In the remainder of this section, the different contexts in which the two nominalisations are used, are listed and illustrated.

\paragraph{Nominalised main clauses}\label{sec:3.2.3.1.1} A number of constructions involve a main clause which is nominalised, even though they express an event. In most of these, zero nominalisation is used.

\subparagraph{The actor-emphatic} The actor-emphatic is a very common construction\is{Actor-emphatic construction}, in which an Agent is preposed as a possessive pronoun or phrase (\sectref{sec:8.6.3}). In one actor-emphatic\is{Actor-emphatic construction} construction (there are three in Rapa Nui), the verb is nominalised (i.e. preceded by the article). 

\ea\label{ex:3.22}
\gll {\ꞌ}Ā{\ꞌ}ana te \textbf{kai} i te me{\ꞌ}e ririva ta{\ꞌ}ato{\ꞌ}a. \\
\textsc{poss.3sg.a} \textsc{art} eat \textsc{acc} \textsc{art} thing good:\textsc{red} all \\

\glt
‘He (was the one who) ate all the best things.’ \textstyleExampleref{[R532-01.011]}
\z

\subparagraph{\textit{Ko S te V}} Much less common is the \textit{ko S te V} construction: a topicalised subject marked by \textit{ko}, followed by a zero nominalisation (see (\ref{ex:8.89}–\ref{ex:8.90}) on p.~\pageref{ex:8.89}).

\subparagraph{\textit{He V te aŋa}} Another common construction is \textit{he V te aŋa} (lit. ‘the doing was V-ing’), which indicates an habitual\is{Aspect!habitual} action, event or attitude. This construction involves two nominalised verbs\is{Verb!nominalised}: \textit{aŋa} ‘to do’ is nominalised and serves as the subject of the clause; the other verb serves as nominal predicate. An example:

\ea\label{ex:3.23}
\gll {\ꞌ}I rā noho iŋa \textbf{he} \textbf{tu{\ꞌ}u} nō mai \textbf{te} \textbf{aŋa} o te nu{\ꞌ}u pa{\ꞌ}ari ki tō{\ꞌ}oku koro u{\ꞌ}i.\\
at \textsc{dist} stay \textsc{nmlz} \textsc{pred} arrive just hither \textsc{art} do of \textsc{art} people adult to \textsc{poss.1sg.o} Dad look\\

\glt
‘At that time the old people always came to see my father (lit. just arriving was the doing of the old people).’ \textstyleExampleref{[R649.101]} 
\z

As \textit{he} can precede both verbs (the aspect marker \textit{he}) and nouns (the predicate marker \textit{he})\is{Aspect marker}, it may not be immediately obvious that \textit{he tu{\ꞌ}u} is nominalised. However, the further contents of the clause show that this is the case: the subject of the clause is \textit{te aŋa}, which is not an argument of \textit{tu{\ꞌ}u}. Conversely, the Agent of \textit{tu{\ꞌ}u} is not expressed as subject of the clause, but as a genitive phrase after \textit{aŋa}. (Another indication that the verb in this construction is nominalised is, that its object may be incorporated; see \REF{ex:5.128} on p.~\pageref{ex:5.128}.)

\subparagraph{\textit{Ko te V}} The construction \textit{ko}\is{ko (prominence marker)} \textit{te} + verb signifies that an action or situation is ongoing or persisting. In most cases a suffixed nominalisation is used, followed by the identity particle \textit{{\ꞌ}ana}\is{ana (identity)@{\ꞌ}ana (identity)}\textit{/{\ꞌ}ā}\is{a (identity)@{\ꞌ}ā (identity)}, as in \REF{ex:3.24}. Sometimes the identity particle is left out, in which case a zero nominalisation may be used, as in \REF{ex:3.25}. 

\ea\label{ex:3.24}
\gll \textbf{Ko} \textbf{te} \textbf{ai} \textbf{iŋa} \textbf{{\ꞌ}ana} te kona mai ira e punua ena te naonao. \\
\textsc{prom} \textsc{art} exist \textsc{nmlz} \textsc{cont} \textsc{art} place from \textsc{ana} \textsc{ipfv} hatch \textsc{med} \textsc{art} mosquito \\

\glt 
‘There are still places from where the mosquito breeds.’ \textstyleExampleref{[R535.054]} 
\z

\ea\label{ex:3.25}
\gll \textbf{Ko} \textbf{te} \textbf{kimi} \textbf{ko} \textbf{te} \textbf{ohu} a nua. \\
\textsc{prom} \textsc{art} search \textsc{prom} \textsc{art} shout \textsc{prop} Mum \\

\glt
‘Mum kept searching and shouting.’ \textstyleExampleref{[R236.082]} 
\z

\subparagraph{Other main clauses} Occasionally zero-nominalised verbs occur in main clauses in other cases. This happens especially with verbs like \textit{haŋa} ‘want’ and \textit{kī} ‘say’ (cf. \sectref{sec:9.2.6}).\footnote{\label{fn:108}Interestingly, in \ili{Māori} there is also a tendency to express ‘wish’-type predicates nominally, followed by a purpose clause \citep[459]{Bauer1993}.} Notice that their S/A argument\footnote{\label{fn:109}Following \citet{Comrie1978}, the following terms are used in this grammar to refer to verb arguments without specifying a semantic role: S = the single argument of an intransitive verb; A = the most agentive argument of a transitive verb (typically an Agent or Experiencer); O = the least agentive argument of a transitive verb (typically a Patient or Theme).} is expressed as a possessive.

\ea\label{ex:3.26}
\gll Te \textbf{haŋa} era o Malo mo ai ko Hepu mo rē. \\
\textsc{art} want \textsc{dist} of Malo for exist \textsc{prom} Hepu for win \\

\glt 
‘Malo wants (lit. ‘Malo’s wish’) Hepu to win.’ \textstyleExampleref{[R408.064]} 
\z

\ea\label{ex:3.27}
\gll Tā{\ꞌ}ana \textbf{kī}: ta{\ꞌ}e tātou hokotahi nō. \\
\textsc{poss.3sg.a} say \textsc{conneg} \textsc{1pl.incl} alone just \\

\glt
‘What she said, was: we are not alone.’ \textstyleExampleref{[R649.191]} 
\z

\subparagraph{Reason clause} Finally, a construction with nominalised verb is sometimes used to express reasons (see (\ref{ex:11.258}–\ref{ex:11.259}) on p.~\pageref{ex:11.258}).

\paragraph{Subordinate clauses}\label{sec:3.2.3.1.2} In subordinate clauses, either suffixed or zero nominalisations are used, depending on the type of clause:

\subparagraph{\textit{{\ꞌ}O} + nominalised verb} In \is{Clause!causal}causal clauses\is{Clause!reason}, after the preposition \textit{{\ꞌ}o}\is{o ‘because of’@{\ꞌ}o ‘because of’}, nominalised verbs\is{Verb!nominalised} are common (\sectref{sec:4.7.2.2}):

\ea\label{ex:3.28}
\gll Ko koa rivariva {\ꞌ}ana te ŋā poki {\ꞌ}o te \textbf{turu} haka{\ꞌ}ou  o rāua ki te hāp{\=\i.}\\
\textsc{prf} happy good:\textsc{red} \textsc{cont} \textsc{art} \textsc{pl} child because\_of \textsc{art} go\_down again  of \textsc{3pl} to \textsc{art} learn\\

\glt
‘The children are really happy because they go back to school.’ \textstyleExampleref{[R334.128]} 
\z

\subparagraph{\textit{{\ꞌ}I} + nominalised verb} The preposition \textit{{\ꞌ}i}\is{i ‘in, at’@{\ꞌ}i ‘in, at’} followed by a verb has various usages. It may indicate a reason, in which case it is followed by either a zero or a suffixed nominalisation\is{Noun!verbal}; the latter is more common.

\ea\label{ex:3.29}
\gll Ku mate atu {\ꞌ}ā a au {\ꞌ}i te kata {\ꞌ}i tū \textbf{haka} \textbf{paka} era  i a ia.\\
\textsc{prf} die away \textsc{cont} \textsc{prop} \textsc{1sg} at \textsc{art} laugh at \textsc{dem} \textsc{caus} conspicuous \textsc{dist}  \textsc{acc} \textsc{prop} \textsc{3sg}\\

\glt 
‘I laughed my head off (lit. died with laughing) because of his boasting.’ \textstyleExampleref{[R230.172]} 
\z

\ea\label{ex:3.30}
\gll Ko ha{\ꞌ}umani {\ꞌ}ana {\ꞌ}i te \textbf{kai} \textbf{iŋa} nō i te moa.\\
\textsc{prf} fed\_up \textsc{cont} at \textsc{art} eat \textsc{nmlz} just \textsc{acc} \textsc{art} chicken\\

\glt
‘He was fed up with eating only chicken.’ \textstyleExampleref{[R617.202]} 
\z

\textit{{\ꞌ}I} is also used in a temporal sense; in that case the clause usually has a suffixed nominalisation\is{Noun!verbal}:

\ea\label{ex:3.31}
\gll {\ꞌ}I te \textbf{tu{\ꞌ}u} \textbf{iŋa} haka{\ꞌ}ou mai era mai Hiva...\\
at \textsc{art} arrive \textsc{nmlz} again hither \textsc{dist} from mainland\\

\glt
‘When he returned (lit. in the returning) again from the mainland....’ \textstyleExampleref{[R487.021]} 
\z

\subparagraph{Temporal clauses} In temporal clauses\is{Clause!temporal} introduced by \textit{ki} or \textit{{\ꞌ}ātā}\is{ata ‘until’@{\ꞌ}ātā ‘until’} \textit{ki} ‘until’, suffixed forms\is{Noun!verbal} are used:

\ea\label{ex:3.32}
\gll E tiaki rō atu ki tu{\ꞌ}u \textbf{topa} \textbf{haŋa} atu.\\
\textsc{ipfv} wait \textsc{emph} away to \textsc{poss.2sg.o} descend \textsc{nmlz} away\\

\glt
‘I will wait until you come down.’ \textstyleExampleref{[R230.047]} 
\z

However, \textit{ante}\is{ante ‘before’} \textit{ki} ‘before’ is followed by a zero nominalisation\is{Verb!nominalised} – possibly because its sense is more irrealis\is{Irrealis} than \textit{{\ꞌ}ātā ki}.

\ea\label{ex:3.33}
\gll ante ki te \textbf{uru} ki roto\\
before to \textsc{art} enter to inside\\

\glt
‘before she went inside’ \textstyleExampleref{[R181.005]} 
\z

\subparagraph{Circumstantial clauses} Occasionally in circumstantial clauses, after \textit{koia ko}\is{koia ko ‘with’}, a zero nominalisation is used; more commonly, however, \textit{koia ko} is followed by a verb (\sectref{sec:8.10.4.2}).

\subparagraph{Purpose clauses} \is{Complement}Purpose clauses\is{Clause!purpose}, introduced by \textit{mo}\is{mo (preverbal)} ‘in order to’, usually have a (non-nominalised) verb (\sectref{sec:11.5.1}). Interestingly, occasionally they have a suffixed nominalisation directly following \textit{mo}. This is the only construction in which a suffixed form is not preceded by a determiner:

\ea\label{ex:3.34}
\gll {\ꞌ}Ina he mā{\ꞌ}eha mo \textbf{u{\ꞌ}i} \textbf{iŋa} i te kai.\\
\textsc{neg} \textsc{pred} light for see \textsc{nmlz} \textsc{acc} \textsc{art} food\\

\glt 
‘There was no light to see the food.’ \textstyleExampleref{[R352.070]} 
\z

\paragraph{Nominal roles}\label{sec:3.2.3.1.3} In nominal positions in general, suffixed forms are much more common than zero nominalisations\is{Verb!nominalised}.

\subparagraph{Subject} Suffixed nominalisations may be the subject of verbal or nominal clauses\is{Clause!nominal}:

\ea\label{ex:3.35}
\gll He riva nō te \textbf{hī} \textbf{iŋa} ki te nu{\ꞌ}u o mu{\ꞌ}a {\ꞌ}ana i te siera.\\
\textsc{pred} good just \textsc{art} to\_fish \textsc{nmlz} for \textsc{art} people of before \textsc{ident} \textsc{acc} \textsc{art} sawfish\\

\glt 
‘For the people of the past, fishing for sawfish was something nice.’ \textstyleExampleref{[R364.019]} 
\z

\ea\label{ex:3.36}
\gll I ha{\ꞌ}amata ai te \textbf{noho} \textbf{iŋa} {\ꞌ}i ira mai te matahiti toru {\ꞌ}ahuru mā pae. \\
\textsc{pfv} begin \textsc{pvp} \textsc{art} stay \textsc{nmlz} at \textsc{ana} from \textsc{art} year three ten plus five \\

\glt
‘His living there started in the year ‘35.’ \textstyleExampleref{[R539-1.492]}
\z

For more examples, see \REF{ex:3.19} and \REF{ex:3.21} above.

\newpage 
However, the subject may also be a zero nominalisation\is{Verb!nominalised}. This tends to happen when the verb refers to a potential or general situation, rather than an event which happens at a specific time:

\ea\label{ex:3.37}
\gll {\ꞌ}O ira te \textbf{oho} tai e oho hai mahana rivariva.\\
because\_of \textsc{ana} \textsc{art} go sea \textsc{ipfv} go \textsc{ins} day good:\textsc{red}\\

\glt
‘Therefore, going to sea is done on beautiful days.’ \textstyleExampleref{[R354.016]} 
\z

Notice however, that \REF{ex:3.35} does not refer to a specific instance either, yet it involves a suffixed form\is{Noun!verbal}. 

These examples suggest that there is a certain freedom in the use of both forms.

\subparagraph{Direct object} In direct object\is{Object!nominalised verb} position, zero nominalisations are common with two classes of verbs: aspectual verbs and certain cognitive verbs.

\textsc{Aspectual verbs}\is{Verb!aspectual verb} include \textit{oti} ‘to finish’, \textit{ha{\ꞌ}amata} ‘to begin’ and \textit{hōrou} ‘to hurry, do in a haste’, as well as a few less common verbs like \textit{haka mao} ‘to terminate’. They may take a nominalised verb as complement, which may or may not be introduced by the object marker \textit{i} (\sectref{sec:11.3.2}). 

\ea\label{ex:3.38}
\gll I oti era i te \textbf{hakaroŋo} e Kāiŋa...\\
\textsc{pfv} finish \textsc{dist} \textsc{acc} \textsc{art} listen \textsc{ag} Kainga\\

\glt 
‘When Kainga had finished listening...’ \textstyleExampleref{[R304.011]} 
\z

\ea\label{ex:3.39}
\gll He ha{\ꞌ}amata rō {\ꞌ}ai te me{\ꞌ}e ta{\ꞌ}ato{\ꞌ}a te \textbf{aŋa}.\\
\textsc{ntr} begin \textsc{emph} \textsc{subs} \textsc{art} thing all \textsc{art} do\\

\glt
‘All the things began to be done.’ \textstyleExampleref{[R378.022]} 
\z

Aspectual verbs\is{Verb!aspectual verb} are not always constructed with a nominalised verb\is{Verb!nominalised}, however. For more details, see \sectref{sec:11.3.2}.

\textsc{Cognitive verbs}\is{Verb!cognitive} include, among others, \textit{{\ꞌ}ite} ‘to know’, \textit{hāpī} ‘to learn’ and \textit{māhani} ‘to be or get used to’. These often take a zero nominalisation when the content of knowledge is a skill, a ‘how to’: 

\ea\label{ex:3.40}
\gll ¿Kai {\ꞌ}ite {\ꞌ}ō koe i te \textbf{tatau} i te ū?\\
~\textsc{neg.pfv} know really \textsc{2sg} \textsc{acc} \textsc{art} squeeze \textsc{acc} \textsc{art} milk\\

\glt 
‘Don’t you know how to milk cows?’ \textstyleExampleref{[R245.184]} 
\z

\ea\label{ex:3.41}
\gll Ki oti he hāpī mai i te \textbf{pāpa{\ꞌ}i} {\ꞌ}i te mākini.\\
when finish \textsc{ntr} learn hither \textsc{acc} \textsc{art} write at \textsc{art} machine\\

\glt
‘After that, we learned typing.’ \textstyleExampleref{[R206.008]} 
\z

However, a suffixed form may also be used, possibly indicating the manner of performing an activity:

\ea\label{ex:3.42}
\gll Mo hāpī rivariva ō{\ꞌ}ou i te \textbf{pāpa{\ꞌ}i} \textbf{haŋa} o te ŋā me{\ꞌ}e nei...\\
for learn good:\textsc{red} \textsc{poss.2sg.o} \textsc{acc} \textsc{art} write \textsc{nmlz} of \textsc{art} \textsc{pl} thing \textsc{prox}\\

\glt
‘In order for you to learn well the (way of) writing these things...’ \textstyleExampleref{[R617.003]} 
\z

With complements of other verbs, for example verbs of perception\is{Verb!perception} and speech, suffixed forms are much more common\is{Verb!speech}:

\ea\label{ex:3.43}
\gll E ŋaro{\ꞌ}a nō {\ꞌ}ā e au te \textbf{hetu} \textbf{iŋa} o tu{\ꞌ}u māhatu.\\
\textsc{ipfv} perceive just \textsc{cont} \textsc{ag} \textsc{1sg} \textsc{art} strike \textsc{nmlz} of \textsc{poss.2sg.o} heart\\

\glt 
‘I hear the beating of your heart.’ \textstyleExampleref{[R505.015]} 
\z

\ea\label{ex:3.44}
\gll He vānaŋa tahi i te \textbf{mate} \textbf{eŋa} era o tū poki era.\\
\textsc{ntr} speak all \textsc{acc} \textsc{art} die \textsc{nmlz} \textsc{dist} of \textsc{dem} child \textsc{dist}\\

\glt 
‘He told all about the death of that child.’ \textstyleExampleref{[R102.105]} 
\z

\subparagraph{Possessives} When a verb is used as a possessive, suffixed nominalisations\is{Noun!verbal} are often used:

\ea\label{ex:3.45}
\gll E ai rō {\ꞌ}ana e rua huru o te \textbf{u{\ꞌ}i} \textbf{iŋa} o te taŋata. \\
\textsc{ipfv} exist \textsc{emph} \textsc{cont} \textsc{num} two manner of \textsc{art} look \textsc{nmlz} of \textsc{art} man \\

\glt
‘There are two ways in which people see it (lit. two ways of seeing).’ \textstyleExampleref{[R648.218]} 
\z

Zero nominalisations\is{Verb!nominalised} also occur in this position, especially after temporal nouns:

\ea\label{ex:3.46}
\gll Ka rua matahiti o te \textbf{poreko} o Puakiva...\\
\textsc{cntg} two year of \textsc{art} born of Puakiva\\

\glt
‘Two years after Puakiva’s birth...’ \textstyleExampleref{[R229.007]} 
\z

One might expect a suffixed form here, as the birth is a one-time event which has happened; yet zero forms are more common when modifying a temporal noun.

\subparagraph{After prepositions} Suffixed nominalisations are found after most prepositions: \textit{mai} ‘from’, \textit{hai} ‘with’, \textit{pē} ‘like’, \textit{ki} ‘to’ (often temporal ‘until’, see above), and after locationals\is{Locational}. Two examples:

\ea\label{ex:3.47}
\gll Mai tai nei, mai te \textbf{hopu} \textbf{iŋa} mātou ko kuā Tonere. \\
from sea \textsc{prox} from \textsc{art} bathe \textsc{nmlz} \textsc{1pl.excl} \textsc{prom} \textsc{coll} Tonere \\

\glt 
‘We are coming from the shore, from swimming with Tonere.’ \textstyleExampleref{[R245.084]} 
\z

\ea\label{ex:3.48}
\gll He hati te va{\ꞌ}e pa he \textbf{hati} \textbf{iŋa} era {\ꞌ}ā o tō{\ꞌ}oku va{\ꞌ}e.\\
\textsc{ntr} break \textsc{art} foot like \textsc{pred} break \textsc{nmlz} \textsc{dist} \textsc{ident} of \textsc{poss.1sg.o} foot\\

\glt 
‘He broke his leg, like I broke my leg (lit. like the breaking of my leg).’ \textstyleExampleref{[R492.021]} 
\z

\subparagraph{Nominal predicates} With the nominal predicate marker \textit{he}\is{he (nominal predicate marker)}, suffixed nominalisations are used (except in the construction \textit{he V te aŋa}, see \sectref{sec:3.2.3.1.1} above, ex. \REF{ex:3.23}). This happens for example in titles as in \REF{ex:3.49}, in existential clauses\is{Clause!existential}, and in classifying clauses\is{Clause!nominal} as in \REF{ex:3.50}. 

\ea\label{ex:3.49}
\gll He \textbf{tu{\ꞌ}u} \textbf{iŋa} mai o Hotu Matu{\ꞌ}a\\
\textsc{pred} arrive \textsc{nmlz} hither of Hotu Matu’a\\

\glt 
‘The arrival of Hotu Matu’a’ (title of a story) \textstyleExampleref{[R369.000]} 
\z

\ea\label{ex:3.50}
\gll Te me{\ꞌ}e nehenehe o te aŋa nei... he \textbf{aŋa} \textbf{iŋa} o te hi{\ꞌ}o.\\
\textsc{art} thing beautiful of \textsc{art} work \textsc{prox} \textsc{pred} make \textsc{nmlz} of \textsc{art} glass\\

\glt 
‘The beautiful thing of this work was the making of the glass.’ \textstyleExampleref{[R360.038]} 
\z

\subsubsection[The form of the nominalising suffix]{The form of the nominalising suffix}\label{sec:3.2.3.2}
\is{Nominalisation}
As indicated above, there are various forms of the nominalising\is{Nominalisation} suffix: \textit{haŋa}\is{hanza (nominaliser)@haŋa (nominaliser)}, \textit{iŋa}\is{inza (nominaliser)@iŋa (nominaliser)}, \textit{eŋa}, \textit{oŋa}, \textit{-ŋa}.\footnote{\label{fn:110}The forms \textit{-(C)aŋa} and \textit{-ŋa} occur throughout the Polynesian languages. Originally, the initial consonant in \textit{-Caŋa} was lexically determined; this is still the case in languages like \ili{Māori} \citep[512]{Bauer1993} and \ili{Samoan} (\citealt[194]{MoselHovdhaugen1992}). In other languages, the paradigm was simplified, as in \ili{Tahitian}, where only \textit{-ra{\ꞌ}a} ({\textless} \textit{*raŋa}) was retained. In Rapa Nui the situation is more complicated, as this section shows.} In older texts, both \textit{haŋa} (86x) and \textit{-ŋa} (132x) are common, while \textit{iŋa} (9x) and \textit{aŋa} (3x) occur sporadically. In newer texts, \textit{haŋa} still occurs (255x), but \textit{iŋa} is now the predominant form (914x). \textit{-ŋa} has become very rare (12x), but other forms have developed: besides \textit{aŋa} (9x), \textit{eŋa} (25x) and \textit{oŋa} (14x) are found. The latter two are the result of vowel assimilation: \textit{oŋa} occurs after \textit{noho} ‘to stay’ and \textit{oho} ‘to go’, while \textit{eŋa} occurs after various verbs ending in \textit{-e} and \textit{-o}; in the first case \textit{eŋa} is the result of total assimilation, in the second case it results from vowel height assimilation: \textit{noho iŋa {\textgreater} noho eŋa}. 

The predominant forms, then, are \textit{haŋa} and \textit{-ŋa}\footnote{\label{fn:111}One could ask whether forms like \textit{nohoŋa} in older texts actually contain a long vowel (\textit{nohōŋa}) or even a disyllabic double vowel (\textit{noho oŋa}). The former is theoretically possible: long vowels\is{Vowel length} are poorly represented in older texts, and in other languages (e.g. \ili{Samoan}), the vowel before \textit{-ŋa} may be lengthened as well. Notice, however, that Rapa Nui has an absolute constraint against long vowels\is{Vowel length} in penultimate syllables\is{Syllable} (\sectref{sec:2.3.2}). Concerning the possibility of \textit{noho oŋa} underlying \textit{nohoŋa}, there is no positive evidence for this; on the contrary, the occurrence of *\textit{-ŋa} in many other Polynesian languages and the rarity of \textit{Vŋa} in older Rapa Nui texts suggest that \textit{-ŋa} is an original form while \textit{-Vŋa} is a recent development, even though the occurrence of \textit{-iŋa/-aŋa} in some languages could be taken as evidence to the contrary. A possible scenario is, that a form like \textit{u{\ꞌ}iŋa} ‘look \textsc{nmlz}’ developed into \textit{u{\ꞌ}i iŋa}; the form \textit{iŋa} was then generalised to verbs not ending in \textit{i}, supplanting \textit{-ŋa}.} in older Rapa Nui, and \textit{haŋa} and \textit{iŋa} in modern Rapa Nui. The question is, if there is a distinction between the two forms in each stage.

In older Rapa Nui, \textit{haŋa} has a wide range of uses, corresponding to the uses of suffixed nominalisations\is{Noun!verbal} described in the previous section. \textit{-ŋa} often has a more nominal and sometimes lexicalised sense: \textit{ohoŋa} ‘go \textsc{nmlz} = trip’; \textit{nohoŋa} ‘stay \textsc{nmlz} = epoch’. The \textit{-ŋa} form may refer to an object related to the event: \textit{toeŋa} ‘remain \textsc{nmlz} = leftovers’; \textit{hatuŋa} ‘weave \textsc{nmlz} = roofing’; \textit{moeŋa} ‘lie \textsc{nmlz} = mat’. 

However, the distinction between \textit{haŋa} and \textit{-ŋa} is by no means clear-cut. On the one hand, \textit{haŋa} forms are used with nominal senses, especially in the sense of place, manner and time (see (\ref{ex:3.19}–\ref{ex:3.21}) above): \textit{noho haŋa} means ‘epoch’, just like \textit{nohoŋa}; \textit{{\ꞌ}iti{\ꞌ}iti haŋa}\is{hanza (nominaliser)@haŋa (nominaliser)} ‘small \textsc{nmlz} = infanthood’; \textit{piko haŋa} ‘hide \textsc{nmlz} = hiding place’. On the other hand, \textit{-ŋa} forms may be used with a verbal sense, just like \textit{haŋa} forms:

\ea\label{ex:3.51}
\gll Ki roaroa te \textbf{mimiroŋa}, he vīviri te henua.\\
when long:\textsc{red} \textsc{art} spin:\textsc{nmlz} \textsc{ntr} roll \textsc{art} land\\

\glt 
‘When he has turned around for a while (lit. ‘when the spinning is long’), he will get dizzy (lit. the land rolls).’ \textstyleExampleref{[Ley-8-52.013]}
\z

In modern Rapa Nui, the distinction between different nominalisers is even harder to pinpoint. \textit{Haŋa} (255x) is less common than \textit{iŋa}\is{inza (nominaliser)@iŋa (nominaliser)} (914x), but occurs in a wide variety of texts, in a wide variety of uses, and with no less than 82 different verbs. To give two examples: 

\begin{itemize}
\item 
Both \textit{topa} \textit{iŋa o te ra{\ꞌ}ā} and \textit{topa haŋa o te ra{\ꞌ}ā} (‘descend nmlz of the sun’) are used in the sense ‘sunset’ or ‘the place where the sun sets, the west’. 

\item 
Both \textit{noho iŋa} and \textit{noho haŋa} occur in the sense ‘epoch, period’. 

\end{itemize}

More generally, both suffixes occur in nominalisations used as subject, object, genitive, after prepositions, and in time clauses introduced by \textit{{\ꞌ}i}. The only construction in which \textit{haŋa} never occurs, is the predicate construction \textit{ko te V} (see (\ref{ex:3.24}–\ref{ex:3.25}) above). Speaker preference may play a role: it is telling that the Bible translation consistently uses \textit{iŋa}, almost never \textit{haŋa}\is{hanza (nominaliser)@haŋa (nominaliser)}. Apart from this, I have not been able to find a distinction between the two.

\subsubsection[The nominalised phrase]{The nominalised phrase}\label{sec:3.2.3.3}

In \sectref{sec:3.2.1.1} it was pointed out that verbs preceded by a determiner may still be accompanied by certain verb phrase elements, as well as certain noun phrase elements. The noun phrase is discussed in detail in Chapter 5, the verb phrase in Chapter 7. This section is limited to a brief listing of elements occurring with nominalised verbs\is{Verb!nominalised}, which shows the hybrid character of nominalised verb phrases.

\subparagraph{Verb phrase elements} Some verb phrase elements never occur with nominalised verbs\is{Verb!nominalised}: aspectual and modal markers, the intensifier \textit{rō}, and the VP{}-final particle \textit{ai} or \textit{{\ꞌ}ai}. However, other elements do occur:

Nominalised verbs\is{Verb!nominalised} may be followed by an \textsc{adverb}\is{Adverb} (\sectref{sec:4.5.1}) like \textit{haka{\ꞌ}ou} ‘again’ or \textit{\mbox{tako{\ꞌ}a}} ‘also’. Notice that \textit{haka{\ꞌ}ou} and \textit{tako{\ꞌ}a} may also occur in the noun phrase (\sectref{sec:5.8.1}). With suffixed nominalisations\is{Noun!verbal}, \textit{haka{\ꞌ}ou} and \textit{tako{\ꞌ}a} occur after the nominalising suffix. 

The adverbs\is{Adverb} \textit{tahi} ‘all’ and \textit{kora{\ꞌ}iti} ‘slowly’ – which do not occur in the noun phrase – both occur once in the corpus with a nominaliser; interestingly, they precede the suffix:

\ea\label{ex:3.52}
\gll Ko te \textbf{turu} \textbf{kora{\ꞌ}iti} \textbf{iŋa} {\ꞌ}ā te vai.\\
\textsc{prom} \textsc{art} go\_down slowly \textsc{nmlz} \textsc{ident} \textsc{art} water\\

\glt
‘The water went slowly down.’ \textstyleExampleref{[Gen. 8:5]}
\z

\is{mai ‘hither’}Both suffixed and zero nominalisations may be followed by a \textsc{directional}\is{Directional} \textit{mai} or \textit{atu} (\sectref{sec:7.5}):

\ea\label{ex:3.53}
\gll Ko rivariva {\ꞌ}ana tō{\ꞌ}ona rere iŋa \textbf{mai}.\\
\textsc{prf} good:\textsc{red} \textsc{cont} \textsc{poss.3sg.o} jump \textsc{nmlz} hither\\

\glt
‘His jump(ing) was good.’ \textstyleExampleref{[R408.022]} 
\z

Zero nominalisations may take the \textsc{constituent negator} \textit{ta{\ꞌ}e}\is{tae (negator)@ta{\ꞌ}e (negator)} (\sectref{sec:10.5.6}):\footnote{\label{fn:112}\textit{ta{\ꞌ}e} does not occur within the noun phrase; when it modifies a noun, it occurs before the predicate marker \textit{he}.}

\ea\label{ex:3.54}
\gll {\ꞌ}I te \textbf{ta{\ꞌ}e} hakaroŋo, he ŋaro rō atu {\ꞌ}ai. \\
at \textsc{art} \textsc{conneg} listen \textsc{ntr} lost \textsc{emph} away \textsc{subs} \\

\glt
‘Because (the sheep) did not listen, it got lost.’ \textstyleExampleref{[R490.005]} 
\z

\textit{Ta{\ꞌ}e} hardly ever goes together with a suffixed nominalisation\is{Noun!verbal}, possibly because the realis\is{Realis} character of the verbal noun\is{Noun!verbal} precludes its use with a negation.

The \textsc{limitative particle} \textit{nō}\is{no ‘just’@nō ‘just’} ‘just, still’ (\sectref{sec:7.4.1}) occurs with zero and suffixed nominalised verbs\is{Verb!nominalised} (see \REF{ex:3.30} above).

\subparagraph{Noun \& verb phrase elements} Certain particles occur in both noun phrase and verb phrase:

\is{Demonstrative!postnominal}\is{Demonstrative!postverbal}The \textsc{demonstrative} particles \textit{nei}, \textit{ena} and \textit{era}. In the verb phrase, they co-occur with certain aspectual markers (\sectref{sec:7.6}); in the noun phrase, they co-occur with any determiner (\sectref{sec:4.6.3}). They do not occur with zero  nominalisations\is{Verb!nominalised}, but they do occur with suffixed nominalisations\is{Noun!verbal}, for example in \REF{ex:3.48}, here repeated:

\ea\label{ex:3.48a}
\gll He hati te va{\ꞌ}e pa he hati iŋa \textbf{era} {\ꞌ}ā o tō{\ꞌ}oku va{\ꞌ}e.\\
\textsc{ntr} break \textsc{art} foot like \textsc{pred} break \textsc{nmlz} \textsc{dist} \textsc{ident} of \textsc{poss.1sg.o} foot\\

\glt
‘He broke his leg, like I broke my leg (lit. like the breaking of my leg).’ \textstyleExampleref{[R492.021]}
\z

The marker \textit{{\ꞌ}ā}\is{a (identity)@{\ꞌ}ā (identity)}\textit{/{\ꞌ}ana} occurs in the verb phrase as a continuous marker, co-occurring with certain aspectual markers (\sectref{sec:7.2.5.5}); in the noun phrase it serves as an identity marker (\sectref{sec:5.9}). It occurs with verbal nouns\is{Noun!verbal}, as illustrated in \REF{ex:3.48} above. In this context, where a comparison is involved, \textit{{\ꞌ}ā} is clearly an identity marker. 

\subparagraph{Noun phrase elements} Nominalised verbs\is{Verb!nominalised} may also be accompanied by noun phrase particles. They may be preceded by any kind of determiner: the article \textit{te}\is{te (article)}, demonstratives\is{Demonstrative} like \textit{tū} as in \REF{ex:3.29} above, possessive pronouns\is{Pronoun!possessive} as in \REF{ex:3.27}, and the predicate marker \textit{he}\is{he (nominal predicate marker)} as in \REF{ex:3.48a}. Suffixed nominalisations tend to denote single instances of an event, so they may be countable: they can be modified by a numeral (see \REF{ex:3.18}), or by quantifiers\is{Quantifier} like \textit{\mbox{ta{\ꞌ}ato{\ꞌ}a}} ‘all’. The corpus does not contain examples of the plural marker \textit{ŋā}\is{nza (plural marker)@ŋā (plural marker)} with verbal nouns\is{Noun!verbal}, but this may be accidental.

In conclusion, both zero and suffixed nominalisations retain a partly verbal character in their phrase. The latter are more nominal than the former\is{Verb!nominalised}, as they allow quantifying elements but do not allow negation.\is{Nominalisation}
\is{Nominalisation!syntactic|)}

\subsection{Nouns used as VP nucleus}\label{sec:3.2.4}
\is{Verb!denominal|(}
Any noun (i.e. entity word) can be used as the nucleus of a verb phrase. Usually, the noun is used in a predicative sense: a verb phrase headed by noun N signifies that the subject is or becomes N; it possesses or acquires property N. These constructions are somewhat similar to nominal predicates marked with \textit{he} (\sectref{sec:9.2.1}), yet they are different: the noun may be preceded by any preverbal marker, e.g. an aspectual as in \REF{ex:3.55} or a negator as in \REF{ex:3.56}, and it may be followed by verb phrase particles such as \textit{rō {\ꞌ}ā} in \REF{ex:3.55}. Also, the clause may express a process (‘become’), while nominal predicates only express a state (‘be’).

\ea\label{ex:3.55}
\gll {\ꞌ}Ai te nunui o te pa{\ꞌ}ahia \textbf{e} \textbf{toto} \textbf{rō} \textbf{{\ꞌ}ā} e viri era. \\
there \textsc{art} \textsc{pl}:big of \textsc{art} sweat \textsc{ipfv} blood \textsc{emph} \textsc{cont} \textsc{ipfv} roll \textsc{dist} \\

\glt 
‘Big drops of sweat became blood and fell down.’ \textstyleExampleref{[Luke 22:44]}
\z

\ea\label{ex:3.56}
\gll \textbf{Kai} \textbf{oromatu{\ꞌ}a} \textbf{hia} i oho rō mai era ki nei. \\
\textsc{neg.pfv} priest yet \textsc{pfv} go \textsc{emph} hither \textsc{dist} to \textsc{prox} \\

\glt
‘When he had not yet become a priest, he came here.’ \textstyleExampleref{[R423.004]} 
\z

Very occasionally, the noun does not indicate ‘be/become N’, but a typical action associated with N:

\ea\label{ex:3.57}
\gll ...i e{\ꞌ}a mai ai e tahi rū{\ꞌ}au \textbf{e} \textbf{tokotoko} \textbf{rō} \textbf{{\ꞌ}ana}.\\
~~~\textsc{pfv} go\_out hither \textsc{pvp} \textsc{num} one old\_woman \textsc{ipfv} cane \textsc{emph} \textsc{cont}\\

\glt
‘...an old woman appeared leaning on a cane.’ \textstyleExampleref{[R437.079]} 
\z

Nouns in a verb phrase are in fact rare in texts, with the exception of temporal nouns. The latter are commonly used as verbs, usually expressing that a period of time passes. 

\ea\label{ex:3.58}
\gll \textbf{Ko} \textbf{tāpati} \textbf{{\ꞌ}ā} i tu{\ꞌ}u iho atu ai. \\
\textsc{prf} week \textsc{cont} \textsc{pfv} arrive just\_then away \textsc{pvp} \\

\glt 
‘When a week had passed, he arrived.’ \textstyleExampleref{[R416.515]}\textstyleExampleref{} 
\z
\is{Verb!denominal|)}
\is{Nominalisation|)}
\subsection{Nominal drift}\label{sec:3.2.5}

In \sectref{sec:3.2.3.1.1} above (examples (\ref{ex:3.22}–\ref{ex:3.27})), a number of constructions are listed in which a verb is nominalised, even though they are main clauses expressing an event: the actor-emphatic construction, \textit{ko te} + verb, et cetera. \sectref{sec:3.2.3.1.1} (\ref{ex:3.28}–\ref{ex:3.34}) lists a number of nominalised subordinate constructions, e.g. \textit{{\ꞌ}o te} + verb to indicate cause or reason. (As shown in \sectref{sec:11.6.4}, various other nominal constructions are used as well to express reason.)

These examples illustrate a tendency in Rapa Nui to maximise the nominal domain. This tendency reveals itself in a number of other areas as well:

\subparagraph{Complements} Motion verbs may be followed by a nominal Goal complement as in \REF{ex:3.59}, even though the goal is semantically an event (\sectref{sec:11.6.3}). As the example shows, the event may be expressed by a verb following the nominal complement. Likewise, perception verbs may be followed by a nominal object + a verbal clause, as in \REF{ex:3.60} (\sectref{sec:11.3.1.2}).

\ea\label{ex:3.59}
\gll He iri ararua \textbf{ki} \textbf{te} \textbf{rāua} \textbf{hoi} {\ꞌ}a{\ꞌ}aru mai.\\
\textsc{ntr} ascend the\_two to \textsc{art} \textsc{3pl} horse grab hither\\

\glt 
‘Both of them went to grab their horse.’ \textstyleExampleref{[R170.002]} 
\z

\ea\label{ex:3.60}
\gll He take{\ꞌ}a \textbf{i} \textbf{a} \textbf{Hoto} \textbf{Vari} ka pū mai. \\
\textsc{ntr} see \textsc{acc} \textsc{prop} Hoto Vari \textsc{cntg} approach hither \\

\glt
‘He saw Hoto Vari approaching.’ \textstyleExampleref{[R304.004]} 
\z

\subparagraph{Compounding} In a peculiar case of compounding, an event is expressed by a verb attached as modifier to one of its arguments; the argument is syntactically the head of the construction (\sectref{sec:5.7.2.3}):

\ea\label{ex:3.61}
\gll {\ꞌ}I tō{\ꞌ}ona mahana he ai mai te aŋa he \textbf{{\ꞌ}āua} \textbf{titi},  {\ꞌ}o he \textbf{rau} \textbf{kato}.\\
at \textsc{poss.3sg.o} day \textsc{ntr} exist hither \textsc{art} work \textsc{pred} enclosure build  or \textsc{pred} leaf pick\\

\glt
‘On certain days there were jobs like making fences or picking leaves.’ \textstyleExampleref{[R380.084]} 
\z

\subparagraph{Arguments as possessives} In a number of constructions, verbal arguments – especially S and A – may be expressed as possessives, even when the verb is not nominalised. This is the default way to express the S/A argument in clauses introduced by \textit{m}\textit{o} as in \REF{ex:3.62} (\sectref{sec:11.5.1.2}); it commonly happens in relative clauses as in \REF{ex:3.63} (\sectref{sec:11.4.4}); and under certain conditions it happens in main clauses as in \REF{ex:3.64} (\sectref{sec:8.6.4.1}).

\ea\label{ex:3.62}
\gll Mo haŋa \textbf{ō{\ꞌ}ou} mo {\ꞌ}ite a hē a au e ŋaro nei...\\
if want \textsc{poss.2sg.o} for know by \textsc{cq} \textsc{prop} \textsc{1sg} \textsc{ipfv} disappear \textsc{prox}\\

\glt 
‘If you want to know where I disappear (then come with me).’ \textstyleExampleref{[R212.010]} 
\z

\ea\label{ex:3.63}
\gll ¿He aha \textbf{te} \textbf{kōrua} me{\ꞌ}e [i aŋa {\ꞌ}i {\ꞌ}Apina]? \\
~\textsc{ntr} what \textsc{art} \textsc{2pl} thing \textsc{~pfv} do at Apina \\

\glt 
‘What did you do (lit. what [is] your thing did) in Apina?’ \textstyleExampleref{[R301.197]} 
\z

\ea\label{ex:3.64}
\gll He kī \textbf{o} \textbf{tū} \textbf{rū{\ꞌ}au} \textbf{era}...\\
\textsc{ntr} say of \textsc{dem} old\_woman \textsc{dist}\\

\glt 
‘The old woman said...’ \textstyleExampleref{[R313.171]} 
\z

\section{Nouns}\label{sec:3.3}
\subsection{Classification of nouns}\label{sec:3.3.1}
\is{Noun|(}
Apart from locationals (\sectref{sec:3.6}), there are two main types of nouns in Rapa Nui: \textsc{common nouns}\is{Noun!common} and \textsc{proper nouns}\is{Noun!proper}. Common nouns\is{Noun!common}, such as \textit{hare} ‘house’ or \textit{poki} ‘child’, designate a class of entities characterised by certain properties; they can be used as nominal predicates, and it is only within a referential noun phrase that they acquire reference to one or more entities. Proper nouns\is{Noun!proper}, such as \textit{Tiare} ‘Tiare’ and \textit{koro} ‘Dad’, are inherently referential; they are not used as predicates and have unique reference in a given context.

These classes impose different constraints on the noun phrase of which they are the head. The most important differences are:

%\setcounter{listWWviiiNumxiiileveli}{0}
\begin{enumerate}
\item 
Common nouns\is{Noun!common} are in most contexts obligatorily preceded by a determiner, proper nouns\is{Noun!proper} are not. 

\item 
Common nouns\is{Noun!common} may be modified by various elements which are incompatible with proper nouns\is{Noun!proper}: quantifiers\is{Quantifier}, adjectives, plural markers and relative clauses\is{Clause!relative}. (See the structure of common NPs in \sectref{sec:5.1} and the structure of proper NPs in \sectref{sec:5.13.1}.)

\item 
Proper nouns\is{Noun!proper} are in many contexts preceded by the proper article\is{a (proper article)} \textit{a}\is{a (proper article)} (\sectref{sec:5.13.2.1}).

\item 
Though both common and proper noun phrases may be preceded by the particle \textit{ko}, proper nouns\is{Noun!proper} have \textit{ko} in a wider range of contexts (\sectref{sec:4.7.11}).

\end{enumerate}

Prototypical common nouns\is{Noun!common} denote classes of concrete, bounded entities, for example persons (\textit{taŋata}) and objects (\textit{hare} ‘house’, \textit{toki} ‘adze’). Prototypical proper nouns\is{Noun!proper} are names of persons. The precise extent of each category can be deduced from the syntactic behaviour of nouns, with \REF{ex:3.1} and \REF{ex:3.3} above as main criteria: nouns preceded by the proper article\is{a (proper article)} \textit{a} are proper nouns\is{Noun!proper}; nouns preceded by determiners like the article \textit{te} are common nouns\is{Noun!common}. This will be explored in the next section.

Both common and proper nouns\is{Noun!proper} function as head of a noun phrase. The structure of the common noun phrase is discussed in \sectref{sec:5.1}; the structure of the proper noun phrase is briefly discussed in \sectref{sec:5.13}.

Within the class of common nouns\is{Noun!common}, we may distinguish countable\is{Noun!countable} and non-countable nouns. Non-countable nouns\is{Noun!non-countable} include mass nouns like \textit{toto} ‘blood’ and \textit{{\ꞌ}ō{\ꞌ}one} ‘earth, soil’, and abstract nouns like \textit{haŋa} ‘love’ and \textit{mamae} ‘pain’. There is no morphological or syntactic difference between countable and non-countable nouns in Rapa Nui, except that the latter cannot be combined with noun phrase elements related to quantification: plural markers, numerals and universal quantifiers\is{Quantifier}.

A third group of nouns is the class of \textsc{locationals}\is{Locational}, which are preceded neither by determiners nor by the proper article\is{a (proper article)}. This class contains a small group of locational\is{Locational} terms like \textit{mu{\ꞌ}a} ‘front’, as well as deictic terms like \textit{nei} ‘here, nearby’. Locationals\is{Locational} are discussed in \sectref{sec:3.6}.

Geographical names\is{Geographical names} mostly pattern with locationals\is{Locational}, but in some situations they behave like proper nouns\is{Noun!proper} (\sectref{sec:3.3.2}). 

The properties of the different types of nouns are summarised in \tabref{tab:13}.

\begin{table} 
% \begin{tabularx}{\textwidth}{L{25mm}Z{8mm}Z{12mm}Z{12mm}Z{12mm}Z{15mm}Z{12mm}} 
\begin{tabularx}{\textwidth}{p{25mm}p{1.2cm}p{1.1cm}p{1.1cm}p{1.1cm}p{1.1cm}p{1.1cm}} 
\lsptoprule
& \rotatehead{open~class}&
  \rotatehead{determiners}&
  \rotatehead{\mbox{proper~article}\is{a (proper article)}}&
  \rotatehead{adjectives}&
  \rotatehead[2cm]{\mbox{quantif.~elements}}&
  \rotatehead{other~modif.}\\
\midrule
common nouns\is{Noun!common}: &  &  &  &  &  & \\
~~~~~count nouns& ×& ×&  & ×& ×& ×\\
~~~~~mass nouns& ×& ×&  & ×&  & ×\\
proper nouns\is{Noun!proper} & ×&  & ×&  &  & ×\\
{locationals\is{Locational}} &  &  &  &  &  & ×\\
\lspbottomrule
\end{tabularx} 
\caption{Types of nouns}
\label{tab:13}
\end{table}

\is{Noun!proper|(}
\subsection{Proper nouns}\label{sec:3.3.2}
\is{Noun!proper}\is{a (proper article)|(}

\is{a (proper article)}The class of proper nouns\is{Noun!proper} contains those items which are – in the appropriate contexts – preceded by the proper article\is{a (proper article)} \textit{a}\is{a (proper article)}. This includes the following categories:

\subparagraph{\ref{sec:3.3.2}.1~ Proper names of persons} Some examples:

\ea\label{ex:3.65}
\gll He oho \textbf{a} \textbf{Hotu} ki te hare. \\
\textsc{ntr} go \textsc{prop} Hotu to \textsc{art} house \\

\glt 
‘Hotu went home.’ \textstyleExampleref{[R273.003]} 
\z

\ea\label{ex:3.66}
\gll He u{\ꞌ}i i \textbf{a} \textbf{Vaha}. \\
\textsc{ntr} look \textsc{acc} \textsc{prop} Vaha \\

\glt
‘He saw Vaha.’ \textstyleExampleref{[Mtx-3-01.144]}
\z

Geographical names\is{Geographical names} do not take the proper article\is{a (proper article)}, whether they designate countries or islands, towns, mountains or any other geographical entity. Nor do they take the common noun article \textit{te}:

\ea\label{ex:3.67}
\gll He hoki rāua \textbf{ki} \textbf{Rapa} \textbf{Nui}. \\
\textsc{ntr} return \textsc{3pl} to Rapa Nui \\

\glt 
‘They returned to Rapa Nui.’ \textstyleExampleref{[Notes]}
\z

\ea\label{ex:3.68}
\gll Te kona noho {\ꞌ}i tu{\ꞌ}a, \textbf{{\ꞌ}i} \textbf{Pōike} {\ꞌ}i roto i te {\ꞌ}ana. \\
\textsc{art} place stay at back at Poike at inside at \textsc{art} cave \\

\glt
‘They lived back in Poike in a cave.’ \textstyleExampleref{[Ley-5-26b.003]}
\z

This characteristic distinguishes geographical names from both common and proper nouns\is{Noun!proper}, and includes them with locationals\is{Locational} (\sectref{sec:3.6}). There are some exceptions though.

Firstly, \textit{Tire} ‘Chile’ is the only geographical name which always takes the personal name in the appropriate contexts.

\ea\label{ex:3.69}
\gll Ararua nō pā{\ꞌ}eŋa e tu{\ꞌ}u mai era, mai Tahiti {\ꞌ}e \textbf{mai} \textbf{i} \textbf{a} \textbf{Tire}. \\
the\_two just side \textsc{ipfv} arrive hither \textsc{dist} from Tahiti and from at \textsc{prop} Chile \\

\glt
‘Both sides arrived, from Tahiti and from Chile.’ \textstyleExampleref{[R539-2.221]}
\z

Secondly, other geographical names may take the proper article\is{a (proper article)} when topicalised\is{Topicalisation} (personal names and pronouns would also \textit{a} this context):

\ea\label{ex:3.70}
\gll \textbf{A} \textbf{Rapa} \textbf{Nui} he henua {\ꞌ}iti{\ꞌ}iti e tahi... \\
\textsc{prop} Rapa Nui \textsc{pred} land small:\textsc{red} \textsc{num} one \\

\glt
‘Rapa Nui is a small island...’ \textstyleExampleref{[R351.001]}  
\z

Finally, the proper article\is{a (proper article)} is used before geographical names used metonymically for their inhabitants. In this case, the presence of \textit{a} shows that the geographical name has been transformed into a personal name:

\ea\label{ex:3.71}
\gll He aŋa \textbf{a} \textbf{Rapa} \textbf{Nui} i to rāua riu tuai. \\
\textsc{ntr} make \textsc{prop} Rapa Nui \textsc{acc} \textsc{art}:of \textsc{3pl} song ancient \\

\glt 
‘(The people of) Rapa Nui made their old songs.’ \textstyleExampleref{[R620.013]} 
\z

\subparagraph{\ref{sec:3.3.2}.2~ Personal pronouns}\is{Pronoun!personal}

\ea\label{ex:3.72}
\gll He turu \textbf{a} \textbf{ia} ki tai. \\
\textsc{ntr} go\_down \textsc{prop} \textsc{3sg} to sea \\

\glt 
‘He went down to the sea.’ \textstyleExampleref{[Notes]}
\z

\ea\label{ex:3.73}
\gll {\ꞌ}I rā hare \textbf{a} \textbf{mātou} e noho ena. \\
at \textsc{dist} house \textsc{prop} \textsc{1pl.excl} \textsc{ipfv} stay \textsc{med} \\

\glt 
‘In that house we lived.’ \textstyleExampleref{[R416.961]} 
\z

\subparagraph{\ref{sec:3.3.2}.3~ Kinship terms}\is{Kinship term}\is{Proper article}The proper article\is{a (proper article)} is common with certain kinship terms, especially \textit{koro} ‘father, older man’ and \textit{nua} ‘mother, older woman’. These words are used in the same way as ‘Dad’ and ‘Mum’ in \ili{English}: like personal names, they have a unique referent in the context, and therefore do not need a determiner.

\ea\label{ex:3.74}
\gll He kī \textbf{a} \textbf{koro} ki \textbf{a} \textbf{nua}... \\
\textsc{ntr} say \textsc{prop} Dad to \textsc{prop} Mum \\

\glt
‘Dad said to Mum...’ \textstyleExampleref{[R333.303]} 
\z

By contrast, \textit{matu{\ꞌ}a} ‘parent’ is a common noun. It does not have a unique referent; in order to refer to a particular parent, its reference must be defined, e.g. by a possessive pronoun:

\ea\label{ex:3.75}
\gll He kī ia a Tiare ki \textbf{tō{\ꞌ}ona} matu{\ꞌ}a vahine era... \\
\textsc{ntr} say then \textsc{prop} Tiare to \textsc{poss.3sg.o} parent female \textsc{dist} \\

\glt 
‘Then Tiare said to her mother...’ \textstyleExampleref{[R481.137]} 
\z

\subparagraph{\ref{sec:3.3.2}.4~ General terms referring to people} The word \textit{māhaki} ‘friend’ (which has a certain compassionate connotation: ‘poor one’) usually takes the proper article\is{a (proper article)}:

\ea\label{ex:3.76}
\gll Ka turu kōrua, ka u{\ꞌ}i i \textbf{a} \textbf{māhaki}. \\
\textsc{imp} go\_down \textsc{2pl} \textsc{imp} look \textsc{acc} \textsc{prop} companion \\

\glt
‘Go down to have a look at (our) friend.’ \textstyleExampleref{[Ley-2-05.011]}
\z

The same applies to a few similar, but less common words: \textit{vērā} ‘that poor one’, \textit{\mbox{{\ꞌ}e{\ꞌ}ete}} ‘so-and-so’\textit{, taureka} ‘that guy’\textit{.}

\subparagraph{\ref{sec:3.3.2}.5~ The collective marker} The collective marker \textit{kuā}\is{kua (collective)@kuā (collective)}/\textit{koā} (\sectref{sec:5.2}) is usually followed by a proper name or another word from the categories mentioned above, but even when followed by a common noun, it may be preceded by the proper article\is{a (proper article)}. In the following example, \textit{korohu{\ꞌ}a} is preceded by the plural marker \textit{ŋā}\is{nza (plural marker)@ŋā (plural marker)}, something which only happens with common nouns\is{Noun!common}. Even so, \textit{kuā} is preceded by the proper article\is{a (proper article)}.

\ea\label{ex:3.77}
\gll {\ꞌ}O ira \textbf{a} \textbf{koā} ŋā korohu{\ꞌ}a e ma{\ꞌ}u hio-hio era  i te haha{\ꞌ}u iŋa o te pātia.\\
because\_of \textsc{ana} \textsc{prop} \textsc{coll} \textsc{pl} old\_man \textsc{ipfv} carry strong:\textsc{red} \textsc{dist}  \textsc{acc} \textsc{art} tie \textsc{nmlz} of \textsc{art} harpoon\\

\glt 
‘Therefore the old people tied the cable of their harpoons well.’ \textstyleExampleref{[R360.020]} 
\z

\subparagraph{\ref{sec:3.3.2}.6~ Names of months}\is{Months, names of} Names of months always take the proper article\is{a (proper article)}, regardless which names are used: the old Rapa Nui names as in \REF{ex:3.78}, the modern \ili{English}-based names as in \REF{ex:3.79}, or \ili{Spanish} names as in \REF{ex:3.80}:

\ea\label{ex:3.78}
\gll E tiaki {\ꞌ}ātā ki \textbf{a} \textbf{Hora~Nui}.  \\
\textsc{exh} wait until to \textsc{prop} September  \\

\glt 
‘You must wait until September.’ \textstyleExampleref{[R647.238]} 
\z

\ea\label{ex:3.79}
\gll {\ꞌ}I \textbf{a} \textbf{Noema} o nei matahiti {\ꞌ}ā i hoki haka{\ꞌ}ou ai ki nei henua. \\
at \textsc{prop} November of \textsc{prox} year \textsc{ident} \textsc{pfv} return again \textsc{pvp} to \textsc{prox} land \\

\glt 
‘In November of this same year he returned again to this island.’ \textstyleExampleref{[R343.016]} 
\z

\ea\label{ex:3.80}
\gll Ki oti te Tāpati nei he piri tātou {\ꞌ}i \textbf{a} \textbf{marzo}. \\
when finish \textsc{art} Tapati \textsc{prox} \textsc{ntr} join \textsc{1pl.incl} at \textsc{prop} March \\

\glt 
‘When the Tapati (= festival week) is finished we are close to March.’ \textstyleExampleref{[R625.131]} 
\z

\subparagraph{\ref{sec:3.3.2}.7~ \textit{Hora}\is{hora ‘summer’} ‘summer’}

\ea\label{ex:3.81}
\gll ¿Pē hē a kōrua i noho ai {\ꞌ}i \textbf{a} \textbf{hora}? \\
~~like \textsc{cq} \textsc{prop} \textsc{2pl} \textsc{pfv} stay \textsc{pvp} at \textsc{prop} summer \\

\glt
‘How were you during summer?’ \textstyleExampleref{[R334.051]} 
\z

NB \textit{Hora}\is{hora ‘time’} ‘time’, a different lexeme, is a common noun. \textit{Toŋa} ‘winter’ is also a common noun.

\subparagraph{\ref{sec:3.3.2}.8~ Definite numerals} Definite numerals like \textit{a totoru} ‘the three’\is{Numeral!definite} are always preceded by the proper article\is{a (proper article)} (\sectref{sec:4.3.4}). Unlike all other elements that take the proper article\is{a (proper article)}, they cannot occur after prepositions.

The use of the proper article\is{a (proper article)} \textit{a} is limited to certain syntactic contexts. This is discussed in \sectref{sec:5.13.2.1}.\textstyleExampleref{\textbf{}} 
\is{Noun|)}
\is{Noun!proper|)}
\is{a (proper article)|)}
\section{Verbs}\label{sec:3.4}
\subsection{Classification of verbs}\label{sec:3.4.1}
\is{Verb|(}
As discussed in \sectref{sec:3.2.1.2}, a prototypical\is{Prototype} verb is a word which denotes an event, functions as clause predicate and is the head of a verb phrase. Verb phrases will be discussed in Chapter 7, verbal clauses in Chapter 8. This section will be limited to a brief discussion of verb types.

Verbs may have zero, one, two or three arguments. Zero-argument verbs are, for example, words indicating a moment in time or the passage of time.\footnote{\label{fn:113}On zero-argument verbs, see \citet[267]{Dryer2007Clause}. Crosslinguistically, zero-argument verbs typically involve weather conditions (‘It rains’). In Rapa Nui however, weather terms are not zero-argument verbs: as discussed in \sectref{sec:3.2.2.1.2}, weather conditions are expressed by subject–predicate collocations, i.e. one-argument predicates.} \textit{{\ꞌ}Ōtea} ‘to dawn’ in \REF{ex:3.82} and \textit{ahiahi} ‘to be evening’ in \REF{ex:3.83} do not have a subject or any other argument, whether overt or implied. The bracketed clause consists of a predicate only.

\ea\label{ex:3.82}
\gll {\ob}I \textbf{{\ꞌ}ōtea} era\,{\cb} he turu he oho a Kava...\\
{\db}\textsc{pfv} dawn \textsc{dist} \textsc{ntr} go\_down \textsc{ntr} go \textsc{prop} Kava\\

\glt 
‘When it dawned, Kava went down...’ \textstyleExampleref{[R229.198]} 
\z

\ea\label{ex:3.83}
\gll {\ob}He ahiahi\,{\cb}, he ma{\ꞌ}oa te {\ꞌ}umu. \\
{\db}\textsc{ntr} afternoon \textsc{ntr} open \textsc{art} earth\_oven \\

\glt 
‘(When) it was late afternoon, they opened the earth oven.’ \textstyleExampleref{[Mtx-7-15.030]}
\z

One-argument verbs include:

%\setcounter{listWWviiiNumxlvileveli}{0}
\begin{enumerate}
\item 
active intransitive\is{Verb!intransitive} verbs\is{Verb!intransitive}, i.e. verbs involving an Agent, such as \textit{oho} ‘go’, \textit{hopu} ‘to bathe, swim’, \textit{piko} ‘to hide oneself’;

\item 
patientive\is{Verb!patientive} verbs\is{Verb!patientive}, i.e. verbs involving a Patient undergoing a process, such as \textit{mate} ‘to die’, \textit{hiŋa} ‘to fall’, \textit{rehu} ‘to be forgotten’;

\item 
adjectives, i.e. words expressing a property, such as \textit{nuinui} ‘(be) big’, \textit{teatea} ‘(be) white’.

\end{enumerate}

Two-argument verbs in Polynesian languages are often divided into two groups: ca\-nonical transitives and middle verbs\is{Verb!middle}.\footnote{\label{fn:114}See e.g. \citet{Chung1978}, \citet{Hooper1984Neuter}, \citet{Harlow2007Maori}. \citet{Bauer1983} uses the term \textit{experience verbs}. In Chung’s description, the difference concerns the affectedness of the patient; \citet{Pawley1973} and \citet{ElbertPukui1979} focus on the difference between deliberate and spontaneous actions. Both classifications yield the same sets of verbs. Syntactic differences between canonical transitives and middle verbs\is{Verb!middle} are language-specific: (a) in ergative\is{Ergativity} languages, they take different transitive\is{Verb!transitive} constructions (\sectref{sec:8.2.1}); (b) when nominalised, they may take different possessive markers (e.g. in \ili{Hawaiian}, \citealt[48]{ElbertPukui1979}); (c) middle verbs\is{Verb!middle} may take the \textsc{acc} marker \textit{ki} rather than \textit{i}. The latter is true in Rapa Nui and \ili{Māori} (\citealt{Bauer1983}; \citealt[267]{Bauer1997}). In \ili{Hawaiian} and \ili{Tahitian}, the development \textit{k {\textgreater}} glottal\is{Glottal plosive} neutralises the difference between \textit{ki} and \textit{i}, as initial glottals\is{Glottal plosive} in particles are usually not contrastive.} The former involve an Agent which acts voluntarily and deliberately, and a Patient affected by the action. Examples are \textit{kai} ‘to eat’ and \textit{\mbox{tiŋa{\ꞌ}i}} ‘to kill, hit’. With middle verbs\is{Verb!middle}, the object is not affected by the action, and the action may be spontaneous rather than voluntary. This category includes verbs of cognition\is{Verb!cognitive}, affection and perception\is{Verb!perception}: ‘to know’, ‘to love’, ‘to see’. As discussed in \sectref{sec:8.6.4.2}, in Rapa Nui the difference has consequences for the marking of the object.

Many verbs may be either transitive\is{Verb!intransitive} or intransitive\is{Verb!intransitive}, depending on whether an object is expressed or implied.\footnote{\label{fn:115}In this grammar, any clause in which a Patient/Theme argument is either expressed or implied, is considered transitive. See also Footnote \ref{fn:379} on p.~\pageref{fn:379} on transitivity.} For example, the verb \textit{kai} ‘to eat’ is transitive\is{Verb!intransitive} when a certain (type of) food is mentioned or implied in the context: in \REF{ex:3.84} below it is transitive\is{Verb!transitive}; in \REF{ex:3.85} it is transitive\is{Verb!transitive} as well, even though the object is implicit (it has been mentioned just before); in \REF{ex:3.86} it is intransitive\is{Verb!intransitive}.\footnote{\label{fn:116}Whether a verb is transitive\is{Verb!transitive} or intransitive\is{Verb!intransitive} may have syntactic repercussions, even when no object is expressed. See the discussion on causativisation of transitive\is{Verb!transitive} verbs in \sectref{sec:8.12.3}, esp. examples \REF{ex:8.235} and \REF{ex:8.236}.} 

\ea\label{ex:3.84}
\gll Kai haŋa a Puakiva mo kai i tū kai era. \\
\textsc{neg.pfv} want \textsc{prop} Puakiva for eat \textsc{acc} \textsc{dem} food \textsc{dist} \\

\glt 
‘Puakiva did not want to eat that food.’ \textstyleExampleref{[R229.145]} 
\z

\ea\label{ex:3.85}
\gll Mo kai ō{\ꞌ}ou he mate koe. \\
if eat \textsc{poss.2sg.o} \textsc{ntr} die \textsc{2sg} \\

\glt 
‘If you eat (the poison), you will die.’ \textstyleExampleref{[R310.063]} 
\z

\ea\label{ex:3.86}
\gll ¿Ko kai {\ꞌ}ā koe? \\
~\textsc{prf} eat \textsc{cont} \textsc{2sg} \\

\glt 
‘Have you eaten?’ \textstyleExampleref{[R245.058]} 
\z

Three-argument verbs involve an Agent, a Patient, and a participant to which the action is directed in some way; depending on the verb, this may be a Goal, Addressee, Recipient or Beneficiary. Examples are \textit{va{\ꞌ}ai} ‘to give’, \textit{tuha{\ꞌ}a} ‘to distribute’, \textit{hāpī} ‘to teach’, \textit{\mbox{{\ꞌ}a{\ꞌ}amu}} ‘to tell’. Usually the Patient is expressed as direct object, while the other argument is marked with either \textit{ki} or \textit{mo}. This is discussed in \sectref{sec:8.8.2}; one example:

\ea\label{ex:3.87}
\gll He va{\ꞌ}ai a nua i te kai ki a koro. \\
\textsc{ntr} give \textsc{prop} Mum \textsc{acc} \textsc{art} food to \textsc{prop} Dad \\

\glt
‘Mum gave the food to Dad.’ \textstyleExampleref{[R236.078]} 
\z

There is one exception to this pattern: the verb \textit{hāpī} ‘teach’ may take two direct objects; the first of these expresses the person taught, the second the content of teaching:

\ea\label{ex:3.88}
\gll He hāpī i te taŋata i te pure.\\
\textsc{ntr} teach \textsc{acc} \textsc{art} person \textsc{acc} \textsc{art} pray\\

\glt
‘He taught people to pray.’ \textstyleExampleref{[R231.304]} 
\z

Three-argument verbs also include causativisations\is{Causative} of transitive\is{Verb!transitive} verbs, such as \textit{haka take{\ꞌ}a} ‘\textsc{caus} see = to show’, \textit{haka aŋa} ‘cause to make’, \textit{haka {\ꞌ}amo} ‘make (someone) carry’; these are discussed in \sectref{sec:8.12.3}. One example:

\ea\label{ex:3.89}
\gll He haka tike{\ꞌ}a e Te Pitu ki a Uka Oho Heru i te {\ꞌ}ō{\ꞌ}one meamea.\\
\textsc{ntr} \textsc{caus} see \textsc{ag} Te Pitu to \textsc{prop} Uka Oho Heru \textsc{acc} \textsc{art} soil red:\textsc{red}\\

\glt 
‘Te Pitu showed (=made see) Uka Oho Heru the red soil.’ \textstyleExampleref{[Fel-1978.070]}
\z

\subsection{Active, stative, intransitive}\label{sec:3.4.2}
\is{Verb!intransitive}
Transitive\is{Verb!transitive} and active intransitive\is{Verb!intransitive} verbs together form the class of \textsc{active verbs}\is{Verb!active}\textbf{.} These are characterised by

%\setcounter{listWWviiiNumliileveli}{0}
\begin{enumerate}
\item 
the possibility for the subject to have the agent marker \textit{e}\is{e (agent marker)} (\sectref{sec:8.3.1.2});

\item 
the possibility to occur in the actor-emphatic\is{Actor-emphatic construction} construction (\sectref{sec:8.6.3}).

\end{enumerate}

The remaining verbs form the class of \textsc{stative verbs}\is{Verb!stative}. This class is well-established in Polynesian linguistics.\footnote{\label{fn:117}The term was introduced by \citet{Buse1965} and adopted e.g. by \citet{Hohepa1969Not}, \citet{Biggs1973}, \citet{ElbertPukui1979}, \citet{Chung1978}, \citet{Seiter1980}, \citet{MoselHovdhaugen1992}.} Criteria for this class vary per language. In Rapa Nui, statives are characterised only by the two criteria above: they do not occur in the actor-emphatic\is{Actor-emphatic construction} and their subject cannot be marked with \textit{e}. In other languages, criteria may include the impossibility of passivisation\is{Passive} and the impossibility to be used in the imperative\is{Imperative}.\footnote{\label{fn:118}See \citet{Biggs1973,Biggs1974} on statives in \ili{Māori}. Within this class, Biggs distinguishes between stative adjectives and stative verbs\is{Verb!stative} (discussed as “neuter verbs” in \citealt{Hooper1984Neuter}); the latter are a small class of verbs with inherently passive meaning, distinguished by the impossibility to enter into a nominal construction. In Rapa Nui, no such distinction can be made.}  

Regarding the latter criterion, the incompatibility of stative verbs\is{Verb!stative} with the imperative\is{Imperative} is probably semantically/pragmatically motivated: there are simply few contexts in which it is appropriate to use a property word in a command. In Rapa Nui, the word \textit{koa} ‘happy’ – which is otherwise a typical adjective (\sectref{sec:3.5.1.4}) – does occur in the imperative\is{Imperative}:

\ea\label{ex:3.90}
\gll \textbf{Ka} \textbf{koa} mai {\ꞌ}āpī {\ꞌ}e mai nehenehe a koe. \\
\textsc{imp} happy while new and while beautiful \textsc{prop} \textsc{2sg} \\

\glt
‘Be happy as long as you are young and beautiful.’ \textstyleExampleref{[R453.018]} 
\z

Stative verbs in Rapa Nui are also characterised by the use of the perfect aspect\is{Aspect!perfect} \textit{ko V {\ꞌ}ā}\is{ko V {\ꞌ}ā (perfect aspect)} to express a present situation; however, this use also occurs with certain categories of active verbs\is{Verb!active} (\sectref{sec:7.2.7.2}).

\textsc{Intransitive verbs}\is{Verb!intransitive} are united by two features:

\begin{enumerate}
\item 
they have a single argument;

\item 
apart from this argument, an (extra) Agent may be expressed, marked with \textit{i}\is{i (preposition)!agent marker}:

\end{enumerate}

\ea\label{ex:3.91}
\gll He mate koe \textbf{i} \textbf{a} \textbf{au}. \\
\textsc{ntr} die \textsc{2sg} at \textsc{prop} \textsc{1sg} \\

\glt
‘You will die by me = I will kill you.’ \textstyleExampleref{[Mtx-3-01.147]}
\z

As discussed in \sectref{sec:8.6.4.7}, this mainly happens with non-agentive verbs (categories 2 and 3 in the previous section), but given the right context, it may also occur with agentive intransitives (category 1).

\textsc{Adjectives} can be considered as a subclass of stative verbs\is{Verb!stative} and will be discussed in \sectref{sec:3.5}. Even though there are no clear-cut criteria to distinguish adjectives from other verbs (especially from patientives\is{Verb!patientive}), in \sectref{sec:3.5.1} it will be shown that there are sufficient grounds to recognise adjectives as a separate subcategory.

\tabref{tab:14} lists the different types of verbs with their features.

\begin{table}
\small{
\begin{tabularx}{\textwidth}{p{16mm}p{24mm}XXZ{8mm}Z{8mm}Z{8mm}Z{9mm}Z{8mm}}
\lsptoprule
 class &   examples  &  val.  & S/A  & \textit{i}-mkd \newline \textsc{acc} & other\newline arg.& actor-emph. &\textit{e}-mkd.\newline Agent& \textit{i}-mkd. \newline Agent\\
\midrule
zero-arg. & {\textit{ahiahi} ‘evening’,  \newline \textit{{\ꞌ}ōtea}~’dawn’} & 0&  &  &  &  &  & \\
\tablevspace
patientives\is{Verb!patientive} & {\textit{mate} ‘die’,\newline \textit{rehu}    ‘be~forgotten’} & 1& ×&  &  &  &  & ×\\
\tablevspace
adjectives & {\textit{nuinui} ‘big’,  \newline \textit{teatea}~’white’} & 1& ×&  &  &  &  & ×\\
\tablevspace
active \newline intransitives & {\textit{oho} ‘go’,  \newline \textit{hopu}~’bathe’} & 1& ×&  &  & ×& ×& ×\\
\tablevspace
canonical  \newline transitives & {\textit{kai} ‘eat’,  \newline \textit{tiŋa{\ꞌ}i}~’kill’} & 2& ×& ×&  & ×& ×& \\
\tablevspace
middle \newline  verbs\is{Verb!middle} & {\textit{haŋa} ‘love’, \newline  \textit{tiaki}~’wait’} & 2& ×&  & ×& ×& ×& \\
\tablevspace
three-arg. \newline  verbs & {\textit{va{\ꞌ}ai} ‘give’,  \newline \textit{\mbox{{\ꞌ}a{\ꞌ}amu}}~’tell’} & 3& ×& ×& ×& ×& ×& \\
\lspbottomrule
\end{tabularx}
}
\caption{Types of verbs}
\label{tab:14}
\end{table}
\is{Verb|)}

 
\section{Adjectives}\label{sec:3.5}
\is{Adjective|(}
Adjectives are words denoting properties. As \citet{Bhat1994} points out, adjectives differ from nouns in that they refer to a single property, while nouns refer to a cluster of properties. Adjectives differ from verbs in that they denote a time-stable property, while verbs denote a transient event. 

  
\sectref{sec:3.5.1} discusses the question whether adjectives form a separate part of speech in Rapa Nui and how they can be distinguished from other words, especially verbs.\footnote{\label{fn:119}According to \citet{Croft2000}, adjectives are intermediate between verb and noun. A prototypical\is{Prototype} adjective describes a property and acts as a modifier; properties are intermediate between objects and actions (one could think of a scale of time-stability here), while modification is intermediate between reference and predication. Therefore, in a language like Rapa Nui, where there is so much interaction between noun and verb, it is only to be expected that adjectives are even harder to distinguish.

According to Dixon, it is probable that every language has a class of adjectives (\citealt{Dixon2004}; \citealt[53]{Dixon2010-1}; \citealt[62, 104]{Dixon2010-2}; different from \citealt{Dixon1982}), though the criteria to distinguish adjectives from either nouns or verbs may be subtle and not obvious at first sight. \citet{Dixon2004}, \citet[70–73]{Dixon2010-2} suggests criteria to distinguish adjectives from verbs and nouns. Note however, that out of thirteen language descriptions in \citet{AikhenvaldDixon2004}, five authors consider adjectives as members of the verb class, even though there are differences between adjectives and (other) verbs (e.g. \citealt{Hajek2004}; \citealt{Hyslop2004}).}  

\sectref{sec:3.5.2} discusses degrees of comparison, a grammatical category largely confined to adjectives.

\subsection{Does Rapa Nui have adjectives?}\label{sec:3.5.1}
\subsubsection[Adjectives as a prototypical category]{Adjectives as a prototypical category}\label{sec:3.5.1.1}
\is{Prototype}
In Polynesian languages – and in Oceanic languages in general – property words such as ‘big’ and ‘good’ tend to behave like verbs; for example, they are often preceded by an aspect marker\is{Aspect marker} and function as predicate of the clause. Many grammars therefore deny that adjectives are a separate word class; rather, they are considered as verbs. On the other hand, property words are sufficiently different from action words to be classified as a separate subclass of verbs. As discussed in \sectref{sec:3.4.2} above, in Rapa Nui – as in other Polynesian languages – a class of stative verbs\is{Verb!stative} can be distinguished; this class includes typical adjectives such as size and colour terms, but also non-active verbs\is{Verb!active} like ‘die’ and ‘be forgotten’. 

The question is, whether it is possible in Rapa Nui to distinguish a subcategory of adjectives within the stative verbs\is{Verb!stative}. \citet[28]{Englert1978} remarks: “Es dudoso si en el idioma rapanui existe el adjetivo propiamente así llamado. Tal vez hay solamente adjetivos verbales o participios.” (It is dubious if the adjective properly so called exists in the Rapanui language. Perhaps there are only verbal adjectives or participles.) Property words in Rapa Nui behave like verbs in most respects. On the other hand, there are also significant differences, as will be shown in the following sections. These differences are sufficiently far-reaching to recognise adjectives as a separate subclass within the category of verbs. At the same time, it is impossible to draw a sharp line between adjectives and other verbs; I have not found a single criterion which sharply and clearly defines a category of adjectives. The boundary between adjectives and verbs is fluid in two ways. First, it is not possible to give an exact list of adjectives; some words are more adjectival than others.\footnote{\label{fn:120}\citet[8]{Hohepa1969Not} lists adjectives in \ili{Māori} (as distinguished from stative verbs\is{Verb!stative}) on the basis of a number of syntactic and morphological criteria. However, as \citet[106]{Harlow2007Maori} points out, other attempts to list \ili{Māori} adjectives exhausively have resulted in somewhat different lists.} Second, some contexts are more adjectival than others, so that a given word may show more adjectival or more verbal behaviour, depending on the context. The adjectival category can therefore best be defined in terms of a \textsc{prototype} (cf. the same approach for nouns and verbs in \sectref{sec:3.2.1.2}), which unites certain semantic, pragmatic and syntactic properties. A prototypical\is{Prototype} adjective

\begin{itemize}
\item 
denotes a property such as dimension, colour or value;

\item 
modifies a referent, by specifying a property of that referent;

\item 
occurs in a noun phrase, directly following the head noun, without a preceding aspect marker\is{Aspect marker}.

\end{itemize}

This raises the question whether less prototypical\is{Prototype} cases are also labelled as adjectives, and if so, how far the use of this label is extended. For practical reasons, in this grammar the term \textit{adjective} is used for property words modifying a noun, and in a looser sense also for property words in other syntactic positions.

In the following sections, I will discuss adjectival characteristics and show to what extent these may serve to distinguish adjectives from other words.

\subsubsection[Morphology of adjectives]{Morphology of adjectives}\label{sec:3.5.1.2}

Two things can be said about the morphology of adjectives.

Firstly, some adjectives are full reduplications\is{Reduplication}. This is true for 

%\setcounter{listWWviiiNumxxvileveli}{0}
\begin{enumerate}
\item 
a number of very common “basic” adjectives: \textit{nuinui} ‘big’, \textit{{\ꞌ}iti{\ꞌ}iti} ‘small’, \textit{rivariva} ‘good’ and \textit{rakerake} ‘bad’; 

\item 
a number of colour terms: \textit{teatea} ‘white’, \textit{meamea} ‘red’, \textit{ritorito} ‘clear, transparent, white’, \textit{{\ꞌ}uri{\ꞌ}uri} ‘black, dark’.\footnote{\label{fn:121}Reduplications as basic colour terms are common in Oceanic languages, even though (a) the use of reduplications\is{Reduplication} as basic lexemes is unusual in Austronesian; (b) it is typologically unusual to have morphologically complex words as basic colour terms (\citealt{Blust2001}; \citealt[304]{Blust2013}. \citet[42]{Blust2001} suggests that reduplications\is{Reduplication} originally had an intensive sense, which lost its intensity over time through frequent use.}  

\end{enumerate}

For most of these, the simple form also exists, but with a marked sense and limited use. For example, even though both \textit{{\ꞌ}iti} and \textit{{\ꞌ}iti{\ꞌ}iti} are used adnominally and adverbially, \textit{{\ꞌ}iti} is more common as an adverb\is{Adverb}, while \textit{{\ꞌ}iti{\ꞌ}iti} is predominantly adnominal. While \textit{rivariva} means ‘good’, adnominal \textit{riva} means either ‘good’ or ‘pretty’. The reduplicated forms may have had an intensifying sense originally, but nowadays they are the default forms in most contexts. In some case the sources exhibit a shift over time: while \textit{rake} ‘bad’ occurs in old texts, in modern Rapa Nui only \textit{rakerake} is found.

Secondly: Just like some verbs, a number of adjectives have a separate plural\is{Plural!adjective} form, which is partially reduplicated. For example: \textit{roaroa} ‘long’, \textit{roroa} ‘long (Pl)’; \textit{rivariva} ‘good’, \textit{ririva} ‘good (Pl)’. The plural forms may be used when the denoted entity is plural, but their use is optional. 

\textit{{\ꞌ}Iti{\ꞌ}iti} ‘small’ has a suppletive plural \textit{rikiriki}; the use of this form is obligatory when the adjective modifies a plural noun or is a predicate with a plural subject.

\subsubsection[Syntactic function: adnominal and other uses]{Syntactic function: adnominal and other uses}\label{sec:3.5.1.3}

The prototypical\is{Prototype} syntactic function of adjectives, which distinguishes it from nouns and verbs, is adnominal: adjectives typically modify a head noun \citep{Croft2000}. Now this fact alone is not sufficient to distinguish adjectives from nouns and verbs, as the latter are used adnominally as well (\sectref{sec:5.7.1}). Moreover, no adjective is used \textit{exclusively} as a noun modifier: the same words also serve as predicates, NP heads and/or adverbs\is{Adverb}, and many also serve as a base for causativisation. The following examples of \textit{rivariva} ‘good’ illustrate this:

\ea\label{ex:3.92}
\gll He hāŋai hai kai \textbf{rivariva}. ~~~\textup{(adnominal)}\\
\textsc{ntr} feed \textsc{ins} food good:\textsc{red} \\

\glt 
‘She fed (him) with good food.’ \textstyleExampleref{[Mtx-7-26.030]}
\z

\ea\label{ex:3.93}
\gll Ko \textbf{rivariva} {\ꞌ}ā {\ꞌ}i te hora nei. ~~~\textup{(predicate)}\\
\textsc{prf} good:\textsc{red} \textsc{cont} at \textsc{art} time \textsc{prox} \\

\glt 
‘She is well now.’ \textstyleExampleref{[R103.234]} 
\z

\ea\label{ex:3.94}
\gll Ku tike{\ꞌ}a {\ꞌ}ana te \textbf{rivariva} o tū rere era. ~~~\textup{(noun)}\\
\textsc{prf} see \textsc{cont} \textsc{art} good:\textsc{red} of \textsc{dem} jump \textsc{dist} \\

\glt 
‘He saw how well he had jumped (lit. the good of the jump).’ \textstyleExampleref{[R408.025]} 
\z

\ea\label{ex:3.95}
\gll Ko {\ꞌ}ite \textbf{rivariva} {\ꞌ}ā koe {\ꞌ}ina ō{\ꞌ}oku matu{\ꞌ}a. ~~~\textup{(adverb\is{Adverb})}\\
\textsc{prf} know good:\textsc{red} \textsc{cont} \textsc{2sg} \textsc{neg} \textsc{poss.1sg.o} parent \\

\glt 
‘You know well that I don’t have parents.’ \textstyleExampleref{[R214.013]} 
\z

\ea\label{ex:3.96}
\gll He haka \textbf{rivariva} i tā{\ꞌ}ana me{\ꞌ}e h{\=\i.} ~~~\textup{(causative\is{Causative})}\\
\textsc{ntr} \textsc{caus} good:\textsc{red} \textsc{acc} \textsc{poss.3sg.a} thing to\_fish \\

\glt 
‘He prepared his fishing gear.’ \textstyleExampleref{[R237.111]} 
\z

Even though adnominal use as such cannot serve as an absolute criterion, the \textit{frequency} of adnominal use may be used as a diagnostic. Words denoting events and objects (i.e. verbs and nouns) are used adnominally only occasionally, while for property words adnominal use is quite common. 

The frequency of adnominal use differs considerably between different adjectives: some are mainly used adnominally, others are mainly used in other functions.\footnote{\label{fn:122}In the frequency counts in this paragraph, adjectives that are part of a name are excluded. Also excluded are syntactically isolated adjectives, e.g. in lists and appositions.} For example, \textit{nuinui} ‘big’ is adnominal in 58.3\% of all occurrences in the text corpus,\footnote{\label{fn:123}403 occurrences in total; 25.8\% are predicate, 9.7\% are NP heads and 3.7\% are adverbs\is{Adverb}.} while \textit{rivariva} ‘good, well’ is adnominal in only 24.6\% of all occurrences.\footnote{\label{fn:124}837 occurrences in total; 19.6\% are predicate, 4.3\% are NP heads and 37.8\% are adverbs\is{Adverb}.} Even so, for both of these, adnominal use is considerably more common than for the noun \textit{taŋata} ‘man’, which is adnominal in 2.3\% of all occurrences (72 out of 3120), or the verb \textit{oho} ‘to go’, which is adnominal in 1.0\% of all occurrences (51 out of 5011).

When adjectives are grouped in semantic categories, such as suggested by \citet[73]{Dixon2010-2}, some patterns emerge, as shown in \tabref{tab:15}.\footnote{\label{fn:125}For this and the following section, I analysed a number of common adjectives from different semantic categories. See the following footnotes for a listing. In the table, values over 20\% are in bold.} This table gives the total number of occurrences for the following categories:

\begin{itemize}
\item 
\textsc{colour}: \textit{meamea} ‘red’; \textit{moana} ‘blue’; \textit{ritomata} ‘green’; \textit{ritorito} ‘clear, transparent, white’; \textit{teatea} ‘white’; \textit{tetea} ‘white (Pl)’; \textit{tōuamāmari} ‘yellow’; \textit{{\ꞌ}uri} ‘dark, black’; \textit{{\ꞌ}uri{\ꞌ}uri} ‘dark, black’

\item 
\textsc{age}: \textit{{\ꞌ}āpī} ‘new’; \textit{hō{\ꞌ}ou} ‘new’; \textit{mātāmu{\ꞌ}a} ‘past’; \textit{pa{\ꞌ}ari} ‘adult’; \textit{tahito} ‘old’; \textit{tuai} ‘old’

\item 
\textsc{dimension}: \textit{{\ꞌ}iti} ‘small, a bit’; \textit{{\ꞌ}iti{\ꞌ}iti} ‘small’; \textit{nui} ‘big’; \textit{nuinui} ‘big’; \textit{parera} ‘deep’; \textit{popoto} ‘short (Pl)’; \textit{potopoto} ‘short’; \textit{raro nui} ‘deep’; \textit{rikiriki} ‘small (Pl)’; \textit{roaroa} ‘long’; \textit{roroa} ‘long (Pl)’; \textit{ruŋa nui} ‘high’

\item 
\textsc{value}: \textit{hauha{\ꞌ}a} ‘important; value’; \textit{hōnui} ‘respected’; \textit{{\ꞌ}ino} ‘bad’; \textit{kino} ‘bad (arch.)’; \textit{ma{\ꞌ}itaki} ‘clean, pretty’; \textit{nehenehe} ‘beautiful’; \textit{rakerake} ‘bad’; \textit{ririva} ‘good (Pl)’; \textit{riva} ‘good’; \textit{rivariva} ‘good, well’; \textit{ta{\ꞌ}e} \textit{au} ‘unpleasant’

\item 
\textsc{physical property}: \textit{hiohio} ‘strong’; \textit{māuiui} ‘sick’; \textit{paŋaha{\ꞌ}a} ‘heavy’; \textit{pūai} ‘strong’; \textit{tītika} ‘straight’

\item 
\textsc{position}: \textit{hāhine} ‘near’; \textit{poto} ‘nearby; short of breath’; \textit{roa} ‘far’

\item 
\textsc{other}: \textit{huru} \textit{kē} ‘different, strange’; \textit{koa} ‘happy’; \textit{hōrou} ‘quick(ly)’; \textit{aŋarahi} ‘difficult’; \textit{parauti{\ꞌ}a} ‘true, truth’; \textit{tano} ‘correct’

\end{itemize}

\begin{table}[b]
\fittable{
\begin{tabular}{lrrrrrr}
\lsptoprule
&  total & adnominal & predicate & noun & adverb & causative\\
\midrule
colour &  337&  {\bfseries 82.5\%}&  13.6\%&  3.3\%&  0.3\%&  0.3\%\\
age &  619&  {\bfseries 77.2\%}&  8.1\%&  13.6\%\parbox{0mm}{
% \footnotemark{}
\footnote{Most nominal uses are cases of \textit{mātāmu{\ꞌ}a} ‘past’, which is often used as a noun ‘the past, the old days’, and \textit{hō{\ꞌ}ou} ‘new’, which is used idiomatically as a term of endearment. Without these two, figures for this category would be as follows:
% \begin{tabbing}
% xxxx \= xxxxxxx \= xxxxxxxxxxx \= xxxxxxxxx \= xxxxxx \= xxxxxxx \= xxxxxx \kill
%  \> total  \>  adnominal   \> predicate   \> noun   \> adverb\is{Adverb}   \> causative\is{Causative} \\
%    \> 495   \> 86.5\%   \> 9.7\%   \> 2.4\%   \> 0.8\%   \> 0.6\%
% \end{tabbing}

\begin{tabular}{lllllll}
 & total  &  adnominal   & predicate   & noun   & adverb   & causative  \\
   & 495   & 86.5\%   & 9.7\%   & 2.4\%   & 0.8\%   & 0.6\%
  \end{tabular}
}
}&  0.6\%&  0.5\%\\
dimension &  1315&  {\bfseries 61.7\%}&  13.5\%&  11.9\%&  8.7\%&  4.1\%\\
value &  1842&  {\bfseries 36.8\%}&  {\bfseries 22.6\%}&  18.1\%&  13.2\%&  9.3\%\\
physical property &  805&  15.3\%&  {\bfseries 26.6\%}&  {\bfseries 33.5\%}&  5.3\%&  19.3\%\\
position &  542&  14.9\%&  {\bfseries 41.0\%}&  {\bfseries 30.6\%}&  0.4\%&  13.1\%\\
other &  1426&  16.3\%&  {\bfseries 57.4\%}&  4.1\%&  15.1\%&  7.1\%\\
\lspbottomrule
\end{tabular}
}
\caption{Uses of adjectives}
% \todo[inline]{vertical alignment top column incorrect – perhaps the last column is a tiny bit too narrow so “causative” takes an extra line? Also: top-align all columns.}
\label{tab:15}
\end{table}



\tabref{tab:15} shows that words denoting colour, age and dimension are mostly used adnominally. For value terms, the adnominal function is the most common one as well, though it accounts for only 36.8\% of all occurrences. For all other categories, less than 20\% of the occurrences are adnominal; these words are more commonly used as predicate or as noun. We may conclude that dimension, age and colour terms are the most prototypical\is{Prototype} adjectives, as far as their syntactic function is concerned; value adjectives are close to prototypical\is{Prototype}. This coincides with Dixon’s generalisation (\citealt[73]{Dixon2012}) that if a language has any adjectives at all, it will have at least some adjectives from (some of) these four categories.

\subsubsection[Adnominal adjectives versus adnominal nouns and verbs]{Adnominal adjectives versus adnominal nouns and verbs}\label{sec:3.5.1.4}
 
The previous paragraph showed, that adjectives show a high frequency of adnominal use compared to nouns and verbs. Apart from this, adnominal adjectives are also different in function and syntax from adnominal nouns and verbs. In the first place, modifying \textsc{nouns}\is{Noun!as modifier} are usually part of a compound\is{Compound}, expressing a single concept together with the head noun, while modifying adjectives specify an additional property of the concept expressed by the head noun (\sectref{sec:5.7.1}). Modifying nouns are incorporated into the head noun; different from adjectives, they cannot be followed by modifying particles, while adjectives may be accompanied by e.g. degree markers and adverbs\is{Adverb} (\sectref{sec:5.7.3.2}).

Modifying \textsc{verbs}\is{Verb!as modifier} occur in two constructions. First, they may form a compound\is{Compound} together with the head noun (\sectref{sec:5.7.2.3}); in this case, they express a single concept together with the head noun, and the same constraints apply as with modifying nouns. Alternatively, modifying verbs may be the head of a relative clause\is{Clause!relative} (\sectref{sec:11.4}), which consists of a verb phrase optionally followed by one or more arguments or adjuncts. The verb in a relative clause\is{Clause!relative} is often preceded by an aspect marker\is{Aspect marker}. By contrast, prototypical\is{Prototype} adjectives – such as terms of dimension, age and colour – are never preceded by an aspectual marker\is{Aspect marker} when used adnominally. 

  
Less prototypical\is{Prototype} adjectives (such as those of position and physical property) do occur in aspect-marked relative clauses\is{Clause!relative}, though only occasionally. In the following example, \textit{hāhine} ‘near’+  is used in a relative clause\is{Clause!relative}: 

\ea\label{ex:3.97}
\gll {\ꞌ}Ina tako{\ꞌ}a o Oceanía te ta{\ꞌ}ato{\ꞌ}a henua era \textbf{e} \textbf{hāhine} \textbf{era} ki Asia.\\
\textsc{neg} also of Oceania \textsc{art} all land \textsc{dist} \textsc{ipfv} near \textsc{dist} to Asia\\

\glt
‘Not all the islands that are close to Asia belong to Oceania either.’ \textstyleExampleref{[R342.008]} 
\z

\textit{Hāhine} is mostly used as predicate; its adnominal use is relatively rare, which suggests that it is not a prototypical\is{Prototype} adjective.

Now Rapa Nui also has “bare\is{Clause!relative!bare} relative clauses\is{Clause!relative}”, relative clauses\is{Clause!relative} in which the verb is not preceded by an aspect marker\is{Aspect marker} (\sectref{sec:11.4.5}). One could ask whether an adnominal adjective is structurally identical to the verb in a bare relative clause\is{Clause!relative}. After all, there are certain similarities between both, besides the absence of the aspect marker. For one thing, adnominal adjectives may be preceded by degree markers and followed by adverbs\is{Adverb} (\sectref{sec:5.7.3.2}), elements which also occur in verb phrases (\sectref{sec:7.3.2}; \sectref{sec:4.5.1}). Adjectives may enter into the comparative\is{Comparative} construction, but verbs occasionally enter into this construction as well (see \REF{ex:3.94} in \sectref{sec:7.3.2}).

However, there are also structural differences between adnominal adjectives and bare relative clauses\is{Clause!relative}. Adnominal adjectives do not take the full range of postverbal particles: they are never followed by the evaluative markers \textit{rō}\is{ro (emphatic marker)@rō (emphatic marker)} and \textit{nō}\is{no ‘just’@nō ‘just’}, or by directionals\is{Directional} \textit{mai} and \textit{atu}. This is true for all adjectives included in \tabref{tab:15} in the preceding section, not just the prototypical\is{Prototype} categories. Verbs in relative clauses\is{Clause!relative}, on the other hand, do take the full range of postverbal particles.\footnote{\label{fn:134}See also sec. \sectref{sec:5.7.2.3} on the difference between modifying verbs as compounds and bare relative clauses\is{Clause!relative}.}

When adjectives are used predicatively, these restrictions do not hold: not only are predicate adjectives preceded by an aspectual marker\is{Aspect marker}, they can be followed by evaluative markers, or by a directional\is{Directional} marker as in the following example:\footnote{\label{fn:135}Examples such as \REF{ex:3.98} are not very common, as time-stable properties are not naturally associated with directionality. In the example above, \textit{atu} is used in the sense of extent (\sectref{sec:7.5.1.5}).}

\ea\label{ex:3.98}
\gll Ku rikiriki \textbf{atu} {\ꞌ}ā te ika nei pē he tapatea {\ꞌ}ana.\\
\textsc{prf} small:\textsc{pl}:\textsc{red} away \textsc{cont} \textsc{art} fish \textsc{prox} like \textsc{pred} kind\_of\_eel \textsc{ident}\\

\glt
‘These fish are quite small, just like \textit{tapatea}.’ \textstyleExampleref{[R364.015]} 
\z

Another difference between verbs and adnominal adjectives is, that the latter are only followed by a limited set of adverbs\is{Adverb}, all of which express a degree: \textit{rahi} ‘much’, \textit{ri{\ꞌ}ari{\ꞌ}a} ‘very, terribly’, \textit{taparahi-ta{\ꞌ}ata} ‘terribly’, or \textit{tano} ‘in a moderate degree’ (\sectref{sec:5.7.3.2}). With the exception of \textit{rahi}, these adverbs\is{Adverb} do not occur in the verb phrase, while on the other hand most verb phrase adverbs\is{Adverb} do not occur in the adjective phrase (\sectref{sec:4.5.1}).

We may conclude that there are subtle but clear semantic and structural differences between adnominal adjectives and verbs. Together with the higher frequency of adnominal use of adjectives, this suggests that the prototypical\is{Prototype} adjective is different from a verb.

\subsubsection[Predicate adjectives]{Predicate adjectives}\label{sec:3.5.1.5}

Adjectives are used as verbal predicates (i.e. predicates marked with verbal particles) to express non-permanent properties, properties which characterise their argument during a moment or a period of time. Permanent properties are expressed in nominal clauses\is{Clause!nominal}, in which the adjective modifies a nominal predicate (\sectref{sec:9.2.7}).

Adjectives and verbal predicates may take the full range of aspect markers discussed in \sectref{sec:7.2}\is{Aspect marker}: neutral \textit{he}, perfective \textit{i}, imperfective \textit{e}, contiguity \textit{ka} and perfect \textit{ko}. Below are some remarks on specifically adjectival uses (or non-uses) of aspect markers\is{Aspect marker}.

\subparagraph{The contiguity marker \textit{ka}\is{ka (aspect marker)}} \textit{ka} is used with adjectives in the same way as with any verb. However, there is one use of \textit{ka} which only occurs with certain adjectives, the exclamative\is{Exclamative} construction discussed in \sectref{sec:10.4.1}.

\subparagraph{Imperfective \textit{e}\is{e (imperfective)}} As discussed in \sectref{sec:7.2.5.4}, \textit{e} with adjectives commonly occurs in the construction \textit{e V (nō/rō) {\ꞌ}ā}, but rarely in the construction \textit{e V} \textit{PVD}\is{Demonstrative!postverbal}. \textit{E V nō {\ꞌ}ā}\is{e (imperfective)!e V nō {\ꞌ}ana} indicates that a state still exists, implying that it could end at some point, but has not ended yet.

\ea\label{ex:3.99}
\gll Te poki nei \textbf{e} \textbf{{\ꞌ}iti{\ꞌ}iti} \textbf{nō} \textbf{{\ꞌ}ā}.\\
\textsc{art} child \textsc{prox} \textsc{ipfv} small:\textsc{red} just \textsc{cont}\\

\glt 
‘This child is still small.’ \textstyleExampleref{[R532-14.007]}
\z

\subparagraph{Perfect \textit{ko}\is{ko V {\ꞌ}ā (perfect aspect)} \textit{V {\ꞌ}ā}} \textit{ko V {\ꞌ}ā} indicates that a state has been reached as the result of an otherwise unstated process:

\ea\label{ex:3.100}
\gll \textbf{Ko} \textbf{koa} \textbf{{\ꞌ}ā} a au {\ꞌ}i te hora nei. \\
\textsc{prf} happy \textsc{cont} \textsc{prop} \textsc{1sg} at \textsc{art} time \textsc{prox} \\

\glt 
‘I am happy now.’ \textstyleExampleref{[R214.053]} 
\z

\ea\label{ex:3.101}
\gll \textbf{Ko} \textbf{rivariva} \textbf{{\ꞌ}ā} {\ꞌ}i te hora nei, {\ꞌ}ina he māuiui haka{\ꞌ}ou.\\
\textsc{prf} good:\textsc{red} \textsc{cont} at \textsc{art} time \textsc{prox} \textsc{neg} \textsc{ntr} sick again\\

\glt
‘He is well now, he is not sick any more.’ \textstyleExampleref{[R103.234]} 
\z

Now this use of \textit{ko V {\ꞌ}ā}\is{ko V {\ꞌ}ā (perfect aspect)} is not restricted to adjectives, but occurs with a much wider range of verbs, including certain types of active verbs\is{Verb!active} (\sectref{sec:7.2.7.2}).

\subparagraph{Neutral \textit{he}\is{he (aspect marker)}} \textit{He} with adjectives expresses a state as such. 

\ea\label{ex:3.102}
\gll \textbf{He} \textbf{rivariva} tā{\ꞌ}ana aŋa era ka aŋa era.\\
\textsc{ntr} good:\textsc{red} \textsc{poss.3sg.a} work \textsc{dist} \textsc{cntg} do \textsc{dist}\\

\glt
‘The work he was doing, was good.’ \textstyleExampleref{[R313.116]} 
\z

\textit{He} + adjective may be used in situations where a state starts to exist, as in the following examples:

\ea\label{ex:3.103}
\gll I oho era, \textbf{he} \textbf{māuiui} haka{\ꞌ}ou tū mata era.\\
\textsc{pfv} go \textsc{dist} \textsc{ntr} sick again \textsc{dem} eye \textsc{dist}\\

\glt 
‘Later, his eyes got sick again.’ \textstyleExampleref{[R237.084]} 
\z

\ea\label{ex:3.104}
\gll I hini era \textbf{he} \textbf{paŋaha{\ꞌ}a} rō atu {\ꞌ}ai {\ꞌ}i te ha{\ꞌ}uru. \\
\textsc{pfv} delay \textsc{dist} \textsc{ntr} heavy \textsc{emph} away \textsc{subs} at \textsc{art} sleep \\

\glt
‘Later they got heavy with sleep.’ \textstyleExampleref{[R536.027]} 
\z

In such cases, the clause can be labeled inchoative; however, this is not expressed by \textit{he} as such, but simply a feature which can be inferred from the context.

\subparagraph{Other preverbal markers} Just like verbs, adjectives can also be used with the modal markers \textit{ana}, \textit{mo} and \textit{ki}, and be preceded by the verbal negators\is{Negation} \textit{{\ꞌ}ina}, \textit{kai} and \textit{e ko}. Two examples:

\ea\label{ex:3.105}
\gll \textbf{Ki} \textbf{nuinui} he ma{\ꞌ}u he haka hāipoipo ki te taŋata hauha{\ꞌ}a.\\
when big:\textsc{red} \textsc{ntr} carry \textsc{ntr} \textsc{caus} marry to \textsc{art} man value\\

\glt 
‘When (the child) was big, they would take it and marry it off to a rich man.’ \textstyleExampleref{[R399.004]} 
\z

\ea\label{ex:3.106}
\gll He noho Makemake hokotahi nō, \textbf{{\ꞌ}ina} \textbf{kai} \textbf{riva}.\\
\textsc{ntr} stay Makemake solitary just \textsc{neg} \textsc{neg.pfv} good\\

\glt 
‘Makemake lived on his own, it was not good.’ \textstyleExampleref{[Ley-1-01.001]}
\z

\subsubsection[Nominal use of adjectives]{Nominal use of adjectives}\label{sec:3.5.1.6}

\is{Adjective!used nominally}As pointed out in \sectref{sec:3.5.1.3} above, adjectives can be used nominally, i.e. as heads of noun phrases. Nominal adjectives refer to a property as such, not to an object possessing the property: \textit{rivariva} ‘goodness’, not ‘a good one’ (\sectref{sec:5.6}):

\ea\label{ex:3.107}
\gll he me{\ꞌ}e mo \textbf{te} \textbf{rivariva} o Rapa Nui pe mu{\ꞌ}a ka oho ena\\
\textsc{pred} thing for \textsc{art} good:\textsc{red} of Rapa Nui toward front \textsc{cntg} go \textsc{med}\\

\glt 
‘something for the good of Rapa Nui in the future’ \textstyleExampleref{[R470.011]} 
\z

\ea\label{ex:3.108}
\gll mata nunui pa he matā {\ꞌ}ā \textbf{te} \textbf{{\ꞌ}uri{\ꞌ}uri}\\
eye \textsc{pl}:big like \textsc{pred} obsidian \textsc{ident} \textsc{art} black:\textsc{red}\\

\glt
‘big eyes, black as obsidian (lit. like obsidian itself the black)’ \textstyleExampleref{[R310.021]} 
\z

Verbs are also used nominally in a variety of constructions (\sectref{sec:3.2.3.1}). However, two nominal constructions occur only with adjectives, not with verbs.\footnote{\label{fn:136}See \citet[29]{Bhat1994}: adjectives are typically able to be the basis of exclamations.} Both have an exclamative\is{Exclamative} sense.

%\setcounter{listWWviiiNumxxviiileveli}{0}
\begin{enumerate}
\item 
Exclamative \textit{{\ꞌ}ai te X} is only found with adjectives of size, such as \textit{nuinui} ‘big’ and \textit{kumi} ‘long’ (\sectref{sec:10.4.3}).

\item 
Exclamative \textit{ko te X} is used with both nouns and a wide range of adjectives (value, physical property, size etc.) (\sectref{sec:10.4.2}).

\end{enumerate}

Nominally used adjectives usually do not have a nominalising\is{Nominalisation} suffix; in this respect they differ from verbs. For example, in \REF{ex:3.94} in \sectref{sec:3.5.1.3} above, \textit{rivariva} is used as object of a verb of perception; in this context, verbs normally get a nominalising\is{Nominalisation} suffix (\sectref{sec:3.2.3.1}), but \textit{rivariva} does not.

There are two contexts in which adjectives do have a nominalising\is{Nominalisation} suffix:

%\setcounter{listWWviiiNumxcileveli}{0}
\begin{enumerate}
\item 
When referring to a time, stage or occasion when a certain property applies. This happens especially with stage adjectives like \textit{{\ꞌ}āpī} ‘young’ and \textit{{\ꞌ}iti{\ꞌ}iti} ‘small’, but occasionally with other adjectives as well.

\end{enumerate}

\ea\label{ex:3.109}
\gll mai te \textbf{rikiriki} \textbf{haŋa} {\ꞌ}ātā ki te \textbf{nunui} \textbf{haŋa}\\
from \textsc{art} small:\textsc{pl}:\textsc{red} \textsc{nmlz} until to \textsc{art} \textsc{pl}:big \textsc{nmlz}\\

\glt 
‘from the time they were small until the time they grew up’ \textstyleExampleref{[R236.097]} 
\z

\ea\label{ex:3.110}
\gll He \textbf{rakerake} \textbf{iŋa} o te vaikava\\
\textsc{pred} bad:\textsc{red} \textsc{nmlz} of \textsc{art} sea\\

\glt
‘The sea gets rough (lit. the bad of the sea) (title of a story)’ \textstyleExampleref{[Acts 27:12]}
\z

\begin{enumerate}
\setcounter{enumi}{1}
\item 
In the construction \textit{ko te V iŋa} \textit{\textsf{{\ꞌ}}}\textit{ā} (\sectref{sec:3.2.3.1.1}):

\end{enumerate}

\ea\label{ex:3.111}
\gll Ko te \textbf{hiohio} \textbf{iŋa} {\ꞌ}ana te taura {\ꞌ}aka era.\\
\textsc{prom} \textsc{art} strong:\textsc{red} \textsc{nmlz} \textsc{ident} \textsc{art} rope anchor \textsc{dist}\\

\glt 
‘The anchor rope kept being taut.’ \textstyleExampleref{[R361.061]} 
\z

\subsubsection{Conclusions}\label{sec:3.5.1.7}

The previous sections have shown that property words differ in their syntactic behaviour from event words in a number of respects:

\begin{itemize}
\item 
They are often used adnominally.

\item 
When used adnominally, they form adjective phrases, which differ from verb phrases: aspect markers\is{Aspect marker} and certain postverbal particles do not occur in the adjective phrase, while the set of adverbs\is{Adverb} in the adjective phrase is different from verb phrase adverbs\is{Adverb}. This means that adnominal adjectives are distinct from relative clauses.

\item 
When property words are used predicatively, they enter into the same range of constructions as verbs, but there are some minor differences.

\item 
Property words functioning as head of a noun phrase show two differences from verbs in the noun phrase: they enter into certain exclamative\is{Exclamative} constructions, and they rarely take the nominalising\is{Nominalisation} suffix.

\end{itemize}

This allows the conclusion that Rapa Nui has an adjective category. However, given the close correspondence with verbs, it is best to consider adjectives as a subclass of verbs, more specifically, of stative verbs\is{Verb!stative}. 

The discussion has also shown that the adjectival category is not a monolithic one. Some adjectives – especially those denoting colour, age and dimension – are more prototypical\is{Prototype} than others. 

\subsection{Degrees of comparison}\label{sec:3.5.2}
\subsubsection[The comparative]{The comparative}\label{sec:3.5.2.1}
\is{Comparative}
Rapa Nui has a number of different comparative\is{Comparative} constructions.\footnote{\label{fn:137}For the different elements in comparative\is{Comparative} constructions, I use the following terms:
\begin{tabbing}
xxxx \= xxxxxxxxxx \= xxxx \= xxxxxxxx \= xxxxxxxxxx \= xxxxxxxxxx  \kill
\> \textit{Susan} \> \textit{(is)} \> \textit{more} \> \textit{intelligent} \> \textit{than} \textit{Mary}\\
  \> comparee  \>  \> index  \> parameter  \> standard
\end{tabbing}
  } In one of these, the particle \textit{{\ꞌ}ata}\is{ata ‘more’@{\ꞌ}ata ‘more’} serves as index of comparison (‘more, -er’); it precedes the adjective expressing the parameter of comparison. This construction can be used whether the adjective is adnominal as in \REF{ex:3.112} or predicative as in \REF{ex:3.113}. The standard of comparison is expressed by \textit{ki}\is{ki (preposition)} + noun phrase.

\ea\label{ex:3.112}
\glll E ai rō {\ꞌ}ā... {\ob}te~~~~~poki\,{\cb} {\ob}{\ꞌ}ata\,{\cb} {\ob}nuinui\,{\cb} ... {\ob}ki~~a~~~~~~Taparahi\,{\cb}. \\
\textsc{ipfv} exist \textsc{emph} \textsc{cont} {\db}\textsc{art}~~child {\db}more {\db}big ~ {\db}to~~\textsc{prop}~~Taparahi \\
~ ~ ~ ~    \textsc{~comparee} \textsc{~index} \textsc{~parameter} ~ \textsc{~standard} \\

\glt 
‘There were children bigger than Taparahi.’ \textstyleExampleref{[R250.011]} 
\z

\ea\label{ex:3.113}
\glll ¿{\ob}{\ꞌ}Ata\,{\cb} {\ob}maneŋe\,{\cb} {\ob}koe\,{\cb} {\ob}ki~~te~~~~~poki~~~era~~ai\,{\cb}? \\
~~~more {\db}medium\_size {\db}\textsc{2sg} {\db}to~~\textsc{art}~~child~~\textsc{dist}~~there \\
~~\textsc{~index} \textsc{~~parameter} \textsc{~~comparee} \textsc{~~standard}\\

\glt
‘Are you smaller than that boy there?’ \textstyleExampleref{[}\textstyleExampleref{R415.176]}
\z
\textit{{\ꞌ}Ata} also functions as a degree marker in front of event verbs (\sectref{sec:7.3.2}, where its etymology is also discussed). With verbs, it may also form a complete comparative\is{Comparative} construction, including a standard of comparison (see \REF{ex:7.94} on p.~\pageref{ex:7.94}).

A second construction uses the verb \textit{hau}\is{hau ‘to exceed’} ‘to exceed, surpass, be superior’, with the comparee as subject. The parameter of comparison is marked with the locative preposition \textit{{\ꞌ}i}. The standard of comparison is expressed by \textit{ki} + noun phrase, as in the \textit{{\ꞌ}ata}{}-construction above. 

\ea\label{ex:3.114}
\glll {\ob}E~~~~~\textbf{hau}~~~~~rō~~~~~~~atu\,{\cb} {\ob}a~~~~~~ia\,{\cb} {\ob}{\ꞌ}i~~te~~~~roroa\,{\cb} {\ob}ki~~a~~~~~~au\,{\cb}. \\
{\db}\textsc{ipfv}~exceed~\textsc{emph}~away {\db}\textsc{prop}~\textsc{3sg} {\db}at~\textsc{art}~\textsc{red}:long {\db}to~~\textsc{prop}~\textsc{1sg} \\
\textsc{~index} \textsc{~comparee} \textsc{~parameter} \textsc{~standard} \\

\glt
‘He is taller than me (lit. he is more/surpassing in length to me).’ \textstyleExampleref{[Notes]}
\z

\textit{Hau} can in turn be reinforced by \textit{{\ꞌ}ata}\is{ata ‘more’@{\ꞌ}ata ‘more’}, in which case the aspectual marker\is{Aspect marker} before \textit{hau} tends to be left out.

\ea\label{ex:3.115}
\glll {\ob}\textbf{{\ꞌ}Ata}~~~\textbf{hau}\,{\cb} ho{\ꞌ}i {\ob}a~~~~~~Veriamo\,{\cb} {\ob}{\ꞌ}i~~~te~~~~~reherehe\,{\cb} {\ob}ki~~a~~~~~~~~me{\ꞌ}e~~~ki~~~a~~~~~~~Eva\,{\cb}. \\
{\db}more~~exceed indeed {\db}\textsc{prop}~~Veriamo {\db}at~~\textsc{art}~~weak:\textsc{red} {\db}to~~\textsc{prop}~~thing~~to~~\textsc{prop}~~Eva \\
\textsc{~index} ~ \textsc{~comparee} \textsc{~parameter} \textsc{~standard} \\

\glt
‘Veriamo was weaker than what’s-her-name, than Eva.’ \textstyleExampleref{[R416.171]} 
\z
In the older language, comparisons are sometimes made without any marking; only \textit{ki}\is{ki (preposition)} indicates that a comparison is made:

\ea\label{ex:3.116}
\gll Te poki nei poki ma{\ꞌ}ori ki tētahi poki. \\
\textsc{art} child \textsc{prox} child expert to other child \\

\glt
‘This child is more intelligent than the other.’ \textstyleExampleref{[\citealt[30]{Englert1978}]}
\z

Although this sentence still sounds acceptable nowadays, speakers of modern Rapa Nui would tend to add \textit{{\ꞌ}ata} in front of \textit{ma{\ꞌ}ori}.

\subsubsection[The superlative]{The superlative}\label{sec:3.5.2.2}
\is{Superlative}
The superlative\is{Superlative} can be expressed by \textit{hope{\ꞌ}a}\is{hope{\ꞌ}a ‘last’} ‘last’ (a \ili{Tahitian} loan not found in old texts), followed by a genitive phrase which contains a nominalised adjective:

\ea\label{ex:3.117}
\gll Te kona \textbf{hope{\ꞌ}a} o te nehenehe ko {\ꞌ}Anakena. \\
\textsc{art} place last of \textsc{art} beautiful \textsc{prom} Anakena \\

\glt 
‘The most beautiful place (lit. the place last of the beauty) is Anakena.’ \textstyleExampleref{[R350.013]} 
\z

\ea\label{ex:3.118}
\gll He autoridad \textbf{hope{\ꞌ}a} o te nuinui o te Quinta Región. \\
\textsc{pred} authority last of \textsc{art} big:\textsc{red} of the Fifth Region \\

\glt
‘He is the highest authority of the Fifth Region.’ \textstyleExampleref{[R203.018]} 
\z

\textit{Hope{\ꞌ}a} can also be used in a superlative\is{Superlative} sense without a qualifying adjective, to express that something is ‘ultimate, extreme’, whether in a positive or negative sense:

\ea\label{ex:3.119}
\gll \textbf{Te} \textbf{tai} \textbf{hope{\ꞌ}a} mo te hāhaki he tai pāpaku. \\
\textsc{art} sea last for \textsc{art} gather\_shellfish \textsc{pred} sea thin \\

\glt
‘The best tide for gathering shellfish is low tide.’ \textstyleExampleref{[R353.018]} 
\z

\textit{Hope{\ꞌ}a}\is{hope{\ꞌ}a ‘last’} + genitive is also used for the absolute superlative\is{Superlative}: ‘very’.

\ea\label{ex:3.120}
\gll ¡Ko te manu \textbf{hope{\ꞌ}a} \textbf{o} \textbf{te} \textbf{tau}! \\
~\textsc{prom} \textsc{art} animal last of \textsc{art} pretty \\

\glt 
‘What a very pretty animal!’ \textstyleExampleref{[R345.072]} 
\z

\ea\label{ex:3.121}
\gll E tahi {\ꞌ}ōpītara nuinui, \textbf{hope{\ꞌ}a} \textbf{o} \textbf{te} \textbf{rivariva}. \\
\textsc{num} one hospital big:\textsc{red} last of \textsc{art} good:\textsc{red} \\

\glt 
‘There was a big hospital, very good.’ \textstyleExampleref{[R239.055]} 
\z

In the older language, the superlative\is{Superlative} can be expressed by the adjective as such, without any special marking; such unmarked superlatives are obsolete nowadays.

\ea\label{ex:3.122}
\gll Te ma{\ꞌ}uŋa Terevaka te ma{\ꞌ}uŋa \textbf{nuinui} o te kāiŋa. \\
\textsc{art} mountain Terevaka \textsc{art} mountain big of \textsc{art} homeland \\

\glt 
‘Mount Terevaka is the biggest hill of the island.’ \textstyleExampleref{[\citealt[30]{Englert1978}]}
\z

\subsubsection[The equative]{The equative}\label{sec:3.5.2.3}

The equative, ‘X is as [Adj] as Y’, is expressed using the preposition \textit{pē}\is{pe ‘like’@pē ‘like’} ‘like’ (\sectref{sec:4.7.8}). The quality with respect to which the two entities are compared, may be expressed as a noun modifier, such as \textit{rikiriki} in the following example:

\ea\label{ex:3.123}
\gll He hakarē i a Tiare {\ꞌ}i muri i te tētahi ŋā poki \textbf{rikiriki} \textbf{pē} \textbf{ia} \textbf{{\ꞌ}ā}. \\
\textsc{ntr} leave \textsc{acc} \textsc{prop} Tiare at near at \textsc{art} other \textsc{pl} child small:\textsc{pl}:\textsc{red} like \textsc{3sg} \textsc{ident} \\

\glt
‘He left Tiare with the other children that were as small as her.’ \textstyleExampleref{[R481.034]} 
\z

But more commonly, it is expressed as a noun phrase:

\ea\label{ex:3.124}
\gll Te ma{\ꞌ}uŋa e take{\ꞌ}a mai era mai tū roa era o Ao Tea Roa  \textbf{pē} \textbf{he} \textbf{ŋa{\ꞌ}oho} \textbf{{\ꞌ}ana} \textbf{te} \textbf{rikiriki}.\\
\textsc{art} mountain \textsc{ipfv} see hither \textsc{dist} from \textsc{dem} far \textsc{dist} of Ao Tea Roa  like \textsc{pred} pebble \textsc{ident} \textsc{art} small:\textsc{pl}\\

\glt 
‘The mountains of Ao Tea Roa in the distance were small like pebbles.’ \textstyleExampleref{[R347.078]} 
\z

\ea\label{ex:3.125}
\gll {\ꞌ}Ī a au e kimi {\ꞌ}ā i te tiare tu{\ꞌ}u \textbf{pē} \textbf{koe} \textbf{te} \textbf{nehenehe}. \\
\textsc{imm} \textsc{prop} \textsc{1sg} \textsc{ipfv} search \textsc{cont} \textsc{acc} \textsc{art} flower seem like \textsc{2sg} \textsc{art} beautiful \\

\glt 
‘I’m looking for a flower that looks as beautiful as you.’ \textstyleExampleref{[R433.003–004]}\textstyleExampleref{} 
\z

\is{Adjective|)}
\section{Locationals}\label{sec:3.6}
\is{Locational|(}\subsection{Introduction}\label{sec:3.6.1}

Rapa Nui has a set of words serving to locate entities in space. These words behave somewhat like nouns, yet are a class of their own, and are called \textsc{locationals}\is{Locational} in this grammar.\footnote{\label{fn:138}All Polynesian languages have such a word class. They have been called local nouns (\citealt{Churchward1953}; \citealt{Bauer1997}; \citealt{MoselHovdhaugen1992}; \citealt{Besnier2000}), locationals (\citealt[59]{Clark1976}, \citealt{Clark1986}, \citealt{DuFeu1996}), L-nouns \citep[55]{Clark1976}, locative nouns (\citealt{ElbertPukui1979}) and locatives (\citealt{Biggs1973}; \citealt{Bowden1992}). For the relative locationals\is{Locational}, \citet[145]{Harlow2007Maori} uses the term \textit{relator nouns}.} 

Different groups of locationals\is{Locational} can be distinguished. 
\begin{enumerate}
\item 
\textsc{Relative locationals.} One group indicates basic spatial relationships such as ‘before, behind, under, above’. Often they indicate the relative position of a referent with respect to another specific referent in the context:

\ea\label{ex:3.126}
\gll A nua {\ꞌ}i \textbf{roto} i te hare. \\
\textsc{prop} Mum at inside at \textsc{art} house \\

\glt
‘Mother is in the house.’ \textstyleExampleref{[R333.284]} 
\z

\textit{Roto} locates mother with respect to the house. The preceding preposition \textit{{\ꞌ}i} indicates that this locative relationship is stable: there is no movement involved towards a position inside the house, or from the inside to the outside.

In this grammar, this first group is called relative locationals\is{Locational}.\footnote{\label{fn:139}Note, however, that there is not always a second referent involved. These same locationals\is{Locational} can also indicate a general direction:
\ea
\gll
He rere a \textbf{ruŋa}.\\
  \textsc{ntr} jump by above\\
 \glt
 ‘He jumped up.’
\z
}
\item
\textsc{Absolute locationals}\is{Locational}. Absolute locationals locate the referent with respect to certain generally known geographical points of reference:

\ea\label{ex:3.127}
\gll He turu a koro ki \textbf{tai}. \\
\textsc{ntr} go\_down \textsc{prop} Dad to sea \\

\glt
‘Dad went down to the seashore.’ \textstyleExampleref{[R333.388]} 
\z

\item
\textsc{Deictic locationals}\is{Locational}\textbf{.} Deictic locationals which indicate distance with respect to the speaker or the discourse situation:

\ea\label{ex:3.128}
\gll E va{\ꞌ}u mahana i noho ai {\ꞌ}i \textbf{nei}. \\
\textsc{ipfv} eight day \textsc{pfv} stay \textsc{pvp} at \textsc{prox} \\

\glt
‘He stayed here (=on Rapa Nui) for eight days.’ \textstyleExampleref{[R374.005]} 
\z

\item
\textsc{Temporal locationals.} There is a small group of time words belonging to the locational\is{Locational} class, such as \textit{{\ꞌ}aŋataiahi} ‘yesterday’.
\end{enumerate}
What all locationals\is{Locational} have in common is that they can be preceded by prepositions, like common nouns\is{Noun!common}. Unlike common nouns\is{Noun!common}, they do not take determiners: the preposition immediately precedes the locational\is{Locational}. Nor do they take the proper article\is{a (proper article)}, as proper nouns\is{Noun!proper} do.

Another class of lexical items commonly used in Rapa Nui discourse for spatial orientation, is the class of geographical names, such as \textit{Tahiti} ‘Tahiti’\is{Geographical names}. As discussed in \sectref{sec:3.3.2}, these can be immediately preceded by prepositions; unlike personal names, they do not take the proper article\is{a (proper article)}. Therefore they do not belong to the class of proper nouns\is{Noun!proper}, but to the locationals\is{Locational}.\footnote{\label{fn:140}\citet[54]{Clark1976} likewise classifies proper names of places among the locationals.}  Geographical names will not be discussed in further detail. 

The following sections discuss relative (\sectref{sec:3.6.2}) and absolute (\sectref{sec:3.6.3}) locationals\is{Locational}. Deictic locationals\is{Locational} are very similar in form and function to demonstratives\is{Demonstrative} and are discussed in the section on demonstratives\is{Demonstrative} (\sectref{sec:4.6.5}). \sectref{sec:3.6.4} discusses temporal words belonging to the class of locationals\is{Locational}. \sectref{sec:3.6.5} shows which modifying elements may occur in the locative phrase.

Finally, the interrogative \textit{hē} partly behaves like a locational\is{Locational} as well; it is discussed in \sectref{sec:10.3.2.3}.

\subsection{Relative locationals}\label{sec:3.6.2}
\is{Locational}
Relative locationals\is{Locational}, in Polynesian linguistics often simply called ‘locationals\is{Locational}’, indicate basic spatial relationships. They are listed in \tabref{tab:15a}. \is{Locational}

\begin{table}
\begin{tabularx}{.50\textwidth}{L{16mm}L{40mm}}
\lsptoprule
\textit{mu{\ꞌ}a}\is{mua ‘front’@mu{\ꞌ}a ‘front’} & front\\
\textit{tu{\ꞌ}a}\is{tua ‘back’@tu{\ꞌ}a ‘back’} & back, behind\\
\textit{ruŋa}\is{ruŋa ‘above’} & above, higher place\\
\textit{raro}\is{raro ‘below’} & under, lower place\\
\textit{roto}\is{roto ‘inside’} & inside\\
\textit{haho}\is{haho ‘outside’} & outside\\
\textit{muri}\is{muri ‘near’} & older RN: back, behind\\
& modern RN: proximity, nearby place\\
\textit{tupu{\ꞌ}aki}\is{tupu{\ꞌ}aki ‘near’} & proximity, nearby place\\
\textit{vāeŋa}\is{vaeŋa ‘middle’@vāeŋa ‘middle’} & middle\\
\lspbottomrule
\end{tabularx}
\caption{Relative locationals}
\label{tab:15a}
\end{table}

\clearpage 

Most of these have the same basic sense throughout the Polynesian languages, though the Rapa Nui locationals\is{Locational} underwent some idiosyncratic developments.\footnote{\label{fn:141}This becomes clear when we compare the Rapa Nui forms and meanings with their \is{Proto-Polynesian}PPN equivalents, as given in Pollex (\citealt{GreenhillClark2011}): 
\is{Proto-Polynesian}PPN \textit{*muri} meant ‘behind, after, to follow, be last’. Rapa Nui is the only language in which its meaning shifted to ‘proximity’; the original meaning is still present in older texts. 
\is{Proto-Polynesian}PPN \textit{*tupuaki} meant ‘the top of the head’. In no other language did it develop into a locational\is{Locational}.
\is{Proto-Polynesian}PPN \textit{*waheŋa} is glossed as ‘division, portion, share, piece of land; middle’. In many languages it is a common noun, and \citet{Clark1976} does not list it as a locational\is{Locational} in \is{Proto-Polynesian}PPN; however, in most \is{Eastern Polynesian}EP languages it does occur as a locational\is{Locational}: \ili{Marquesan} \textit{vaveka} (\citealt[331–332]{Cablitz2006}, \textit{k} {\textless} \is{Proto-Polynesian}PPN \textit{*ŋ}), \ili{Māori} \textit{vaenganui} \citep[41]{Biggs1973}, \ili{Hawaiian} \textit{waena} (\citealt[121]{ElbertPukui1979}), \ili{Pa’umotu} \textit{vaeŋa} \citep[594]{Stimson1964}, \ili{Mangarevan} \textit{vaega} \citep[118]{Tregear2009}. It does not occur in \ili{Tahitian}.} 

In the following subsections, these locationals\is{Locational} are discussed in detail. Sections \sectref{sec:3.6.2.1} and \sectref{sec:3.6.2.2} discuss the syntax of locational\is{Locational} constructions. \sectref{sec:3.6.2.3} discusses the semantics of certain locationals\is{Locational} and locational\is{Locational} expressions. This is continued in \sectref{sec:3.6.2.4}, which discusses the temporal use of certain locationals\is{Locational}.




\subsubsection[Adverbial expressions]{Adverbial expressions}\label{sec:3.6.2.1}

Locationals\is{Locational} are usually preceded by one of the locative prepositions discussed in \sectref{sec:4.7}. Together with these prepositions, the locationals\is{Locational} form adverbial expressions of location. Here are a few examples:

\ea\label{ex:3.129}
\gll He uru koe he noho \textbf{{\ꞌ}i} \textbf{roto}. \\
\textsc{ntr} enter \textsc{2sg} \textsc{ntr} stay at inside \\

\glt 
‘You go in and stay inside.’ \textstyleExampleref{[R310.295]} 
\z

\ea\label{ex:3.130}
\gll He marere te hare \textbf{ki} \textbf{raro}. \\
\textsc{ntr} scatter \textsc{art} house to below \\

\glt 
‘The house fell down.’ \textstyleExampleref{[Ley-2-12.006]}
\z

\ea\label{ex:3.131}
\gll \textbf{Mai} \textbf{ruŋa} he raŋi atu a Eva... \\
from above \textsc{ntr} call away \textsc{prop} Eva \\

\glt 
‘From above, Eva cried...’ \textstyleExampleref{[R210.111]} 
\z

\ea\label{ex:3.132}
\gll Me{\ꞌ}e rahi te manu \textbf{o} \textbf{ruŋa}. \\
thing many \textsc{art} bird of above \\

\glt
‘There were many birds on (the island).’ \textstyleExampleref{[Egt-02.083]}
\z

Adverbial expressions like these often have an absolute sense. For example, in \REF{ex:3.130} \textit{ki raro} indicates ‘down, in a lower direction’. In other cases, the locational\is{Locational} is interpreted relative to a second referent, which is implied. In \REF{ex:3.132}, the context makes clear that \textit{o ruŋa} is to be interpreted with respect to an island.

\subsubsection[Locationals with complement: prepositional expressions]{Locationals with complement: prepositional expressions}\label{sec:3.6.2.2}
\is{Locational}
The adverbial expressions discussed in the previous section can be followed by a preposition + noun phrase to indicate a spatial relationship with respect to a second referent. The combination of preposition + locational\is{Locational} + preposition acts as a sort of complex preposition, in which the locational\is{Locational} indicates the spatial relationship between two referents, and the initial preposition the way in which this relationship holds. In the following example, \textit{roto} expresses that the spatial relationship is such that referent A (the cat) is inside referent B (the house). The preposition \textit{ki} expresses that referent A moves towards that location.

\ea\label{ex:3.133}
\gll He uru te kurī ki roto i te hare.\\
textsc{ntr} enter \textsc{art} cat to inside at \textsc{art} house \\

\glt
‘The cat entered into (lit. to inside) the house.’ \textstyleExampleref{[Notes]}
\z

The second preposition does not have any semantic contribution; it serves just to provide a syntactic link between the locational\is{Locational} and its complement. The following examples show different ways in which this preposition can be realised:

\ea\label{ex:3.134}
\gll {\ꞌ}I te rua mahana i tu{\ꞌ}u mai ai \textbf{ki} \textbf{mu{\ꞌ}a} \textbf{o} Haŋa Kaokao. \\
at \textsc{art} two day \textsc{pfv} arrive hither \textsc{pvp} to front of Hanga Kaokao \\

\glt 
‘On the second day, they arrived in front of Hanga Kaokao.’ \textstyleExampleref{[R539-1.570]}
\z

\ea\label{ex:3.135}
\gll He e{\ꞌ}a \textbf{mai} \textbf{roto} \textbf{mai} te koro.\\
\textsc{ntr} go\_out from inside from \textsc{art} feast\_house\\

\glt 
‘They went out of the feast house.’ \textstyleExampleref{[Mtx-6-03.090]}
\z

\ea\label{ex:3.136}
\gll He eke māua \textbf{ki} \textbf{ruŋa} \textbf{ki} te hoi.\\
\textsc{ntr} go\_up \textsc{1du.excl} to above to \textsc{art} horse\\

\glt 
‘We mounted (on top of) the horses.’ \textstyleExampleref{[R126.045]} 
\z

\ea\label{ex:3.137}
\gll He {\ꞌ}oka te pua \textbf{{\ꞌ}i} \textbf{raro} \textbf{i} te rano {\ꞌ}i Rano {\ꞌ}Aroi.  \\
\textsc{ntr} plant \textsc{art} kind\_of\_plant at below at \textsc{art} crater\_lake at Rano Aroi  \\

\glt 
‘He planted \textit{pua} down in the crater of Rano Aroi.’ \textstyleExampleref{[Mtx-6-05.006]}
\z

\ea\label{ex:3.138}
\gll Ko Meta te me{\ꞌ}e \textbf{o} \textbf{tu{\ꞌ}a} \textbf{o} Juan Hotu.\\
\textsc{prom} Meta \textsc{art} thing of back of Juan Hotu\\

\glt
‘Meta is the one behind Juan Hotu.’ \textstyleExampleref{[R412.214]} 
\z

As these examples show, the second preposition may be either \textit{i}\is{i (preposition)} as in \REF{ex:3.133}, \textit{o}\is{o (possessive prep.)} as in \REF{ex:3.134}, or a copy of the first preposition as in (\ref{ex:3.135}–\ref{ex:3.136}). When the first preposition is \textit{{\ꞌ}i} or \textit{o}, the analysis of the second preposition is ambiguous: in \textit{{\ꞌ}i raro i} in \REF{ex:3.137}, the second preposition may be either a default preposition \textit{i}, or a copy of the first preposition (\textit{{\ꞌ}i} and \textit{i} are variants of the same preposition, see \sectref{sec:4.7.2}). The same is true for \textit{o tu{\ꞌ}a o} in \REF{ex:3.138}.

As \citet[54–55]{Clark1976} points out, all other Polynesian languages use either \textit{i} or \textit{o} as second preposition;\footnote{\label{fn:142}Vaitupu (a dialect of \ili{Tuvaluan}) is the only variety apart from Rapa Nui where both \textit{o} and \textit{i} are used, without apparent difference in meaning.} Rapa Nui is the only language in which the second preposition may be a copy of the first.\footnote{\label{fn:143}As \citet[56–57]{Clark1976} indicates, the copying construction could have arisen from cases like \REF{ex:3.137} or \REF{ex:3.138}: the second preposition, which originally was an invariable \textit{i} or \textit{o}, was reanalysed as a repetition of the first one. This reanalysis could have been facilitated by constructions like the following (quoted by Clark):
\ea
\gll
He topa mai te timo \textbf{ki} \textbf{roto} \textbf{ki} \textbf{te} \textbf{{\ꞌ}ana} o {\ꞌ}Ana te Ava Nui.\\
\textsc{ntr} descend hither \textsc{art} warrior to inside to \textsc{art} cave of Ana te Ava Nui\\
\glt 
‘The warriors were dragged into the cave of Ana te Ava Nui.’ (Mtx-3-03.231)
\z While such constructions could originally have consisted of two parallel phrases: ‘inside, to the cave’ they could easily be reanalysed as a single phrase ‘into the cave’, in which the second \textit{ki} is a copy of the first.

According to \citet[27–28]{FinneyAlexander1998}, \textit{ki ... ki} also occurs in Vaitupu and, in some constructions, in \ili{Māori}; however, this does not amount to a generalised copying strategy as in Rapa Nui.} 

In older texts the copying strategy is used in an overwhelming majority of the cases. Not counting the ambiguous \textit{{\ꞌ}i} LOC \textit{i}\is{i (preposition)} and \textit{o}\is{o (possessive prep.)} LOC \textit{o} constructions, the second preposition is a copy of the first in 93\% of all PREP + LOC + PREP constructions in this corpus (768 out of 826). Thus, constructions like (\ref{ex:3.135}–\ref{ex:3.136}) are common in older texts, while constructions such as (\ref{ex:3.133}–\ref{ex:3.134}) are rare. 

In modern Rapa Nui the copying strategy is still in use, as illustrated by \REF{ex:3.136} above, but it has become relatively rare, occurring in only 10\% of all nonambiguous cases (175 out of 1761).\footnote{\label{fn:144}This tendency is even stronger in the Bible translation, which is more recent than most of the newer texts: in the new Testament, the preposition is \textit{i} or \textit{o} in over 99\% of the prepositions, with \textit{i} in the overwhelming majority (88\%).} And some of these are, on a closer look, not copies at all, but prepositions introducing a new constituent. The following example illustrates this:

\ea\label{ex:3.139}
\gll I oti era he turu \textbf{ki} \textbf{raro} \textbf{ki} \textbf{te} \textbf{teata} māta{\ꞌ}ita{\ꞌ}i.\\
\textsc{pfv} finish \textsc{dist} \textsc{ntr} go\_down to below to \textsc{art} cinema observe\\

\glt
‘After that they went down to the theatre to watch.’ \textstyleExampleref{[R210.145]} 
\z

This is not a case of a complex preposition ‘to below N’: \textit{ki raro} is not interpreted relative to the second referent \textit{te teata} (in that case, people would go to a location below the cinema); rather, \textit{ki raro} and \textit{ki te teata} are two separate, parallel constituents.

Instead of a copy of the first preposition, the second preposition is usually \textit{i} or \textit{o} nowadays; both are used without a clear difference in meaning.\footnote{\label{fn:145}Just like the copying construction may be the result of reanalysis (see Footnote \ref{fn:143} above), the choice for \textit{i} or \textit{o} could also have been brought about by reanalysis: in expressions like \textit{{\ꞌ}i ruŋa i} and \textit{o roto o}, the second preposition (which was a copy of the first) was reanalysed as default \textit{i} or \textit{o}, and their use was subsequently generalised. \textit{I} lends itself to a generalised use as it is the most general locative preposition; \textit{o} lends itself to a generalised use as it is common as genitive marker. Notice that it is not uncommon for spatial relationships to be expressed by the genitive (see \citealt[285]{Dixon2010-2}). \citet[324]{Fischer2001Hispan} considers the generalisation of \textit{i} as second preposition as a development under \ili{Spanish} influence.}

In general, \textit{i} is more common in modern Rapa Nui than \textit{o}: over the whole corpus of modern texts, \textit{i} outnumbers \textit{o} in a proportion of 2:1.\footnote{\label{fn:146}\textit{Pace} \citet[28]{FinneyAlexander1998}, who claim that “\textit{o} has largely displaced earlier \textit{i} as right-side preposition”.} The choice between the two is free to a certain degree, but certain tendencies can be observed:

%\setcounter{listWWviiiNumxlivleveli}{0}
\begin{enumerate}
\item 
When the second referent is pronominal, \textit{i} tends to be used, followed by the proper article\is{a (proper article)}.
\end{enumerate}

\ea\label{ex:3.140}
\gll Poki ra{\ꞌ}e {\ꞌ}ā{\ꞌ}aku ka e{\ꞌ}a nei mai muri \textbf{i} \textbf{a} \textbf{au}.\\
child first \textsc{poss.1sg.a} \textsc{cntg} go\_out \textsc{prox} from near at \textsc{prop} \textsc{1sg}\\

\glt
‘You are my first child to leave my side (lit. to go out from near me).’ \textstyleExampleref{[R210.049]} 
\z

\begin{enumerate}
\setcounter{enumi}{1}
\item 
The choice between \textit{i}\is{i (preposition)} and \textit{o}\is{o (possessive prep.)} also correlates with the choice of locational\is{Locational}: \textit{o} is more common with \textit{raro}, \textit{mu{\ꞌ}a} and \textit{tu{\ꞌ}a}, while \textit{i} is more common with \textit{roto}, \textit{ruŋa} and \textit{muri}. The preposition preceding the locational\is{Locational} does not play a role.

\end{enumerate}

The locational\is{Locational} \textit{roto} and the following article \textit{te} are often contracted: \textit{roto (i/o) te {\textgreater}} \textit{rote}. This contraction is a recent development; it does not occur in older texts. 

\ea\label{ex:3.141}
\gll E koro, ¿e aha {\ꞌ}ā koe \textbf{{\ꞌ}i} \textbf{rote} {\ꞌ}ua? \\
\textsc{voc} Dad ~\textsc{ipfv} what \textsc{cont} \textsc{2sg} at inside\_\textsc{art} rain \\

\glt 
‘Dad, what are you doing in the rain?’ \textstyleExampleref{[R210.097]} 
\z

\subsubsection[The semantics of some locationals]{The semantics of some locationals}\label{sec:3.6.2.3}
\is{Locational}
This section discusses the meaning of some individual locationals\is{Locational}, and of some locational\is{Locational} expressions.

\paragraph{}\label{sec:3.6.2.3.1} \textit{Muri}\is{muri ‘near’} in older texts means ‘after’: either in spatial sense (‘behind’), or in a temporal sense (‘afterward’). 

\ea\label{ex:3.142}
\gll He oho te ŋāŋata {\ꞌ}i \textbf{muri} i tau ŋā io era. \\
\textsc{ntr} go \textsc{art} men at after at \textsc{dem} \textsc{pl} young\_man \textsc{dist} \\

\glt 
‘The men went after those youngsters.’ \textstyleExampleref{[Mtx-7-37.018]}
\z

\ea\label{ex:3.143}
\gll Ka tiŋa{\ꞌ}i kōrua te vi{\ꞌ}e ena, mo \textbf{muri} au ana tiŋa{\ꞌ}i. \\
\textsc{imp} kill \textsc{2pl} \textsc{art} woman \textsc{med} for after \textsc{1sg} \textsc{irr} kill \\

\glt
‘Kill that woman, after that kill me.’ \textstyleExampleref{[Mtx-7-21.037]}
\z

In modern Rapa Nui, \textit{muri}\is{muri ‘near’} indicates spatial proximity, ‘close to, next to’:

\ea\label{ex:3.144}
\gll He tu{\ꞌ}u ki \textbf{muri} ki te pahī, he ekeeke ki ruŋa. \\
\textsc{ntr} arrive to near to \textsc{art} ship \textsc{ntr} go\_up:\textsc{red} to above \\

\glt 
‘They came alongside the ship and went on board.’ \textstyleExampleref{[R210.081]} 
\z

\paragraph{}\label{sec:3.6.2.3.2} \textit{Tu{\ꞌ}a}\is{tua ‘back’@tu{\ꞌ}a ‘back’} refers to the back of something. \textit{{\ꞌ}I tu{\ꞌ}a} normally refers to a location behind, on the outside of something: \textit{{\ꞌ}i tu{\ꞌ}a o te hare} ‘behind the house’. But in some situations it may refer to a location within, at the back side. This may occasionally lead to ambiguities:

\ea\label{ex:3.145}
\gll Ka hakarē te bombona \textbf{{\ꞌ}i} \textbf{tu{\ꞌ}a} \textbf{o} Te kamioneta. \\
\textsc{imp} leave \textsc{art} gas\_bottle at back of \textsc{art} van \\

\glt
‘Put the gas bottle behind the van’, or: ‘...in the back of the van.’
\z

\textit{{\ꞌ}I tu{\ꞌ}a} in this example refers to a location either inside or outside the car. 

\paragraph{}\label{sec:3.6.2.3.3} \textit{Ruŋa}\is{ruŋa ‘above’} may be either ‘on, on top of (touching)’ or ‘above (not touching)’:

\ea\label{ex:3.146}
\gll Te puka \textbf{{\ꞌ}i} \textbf{ruŋa} i te {\ꞌ}amurama{\ꞌ}a. \\
\textsc{art} book at above at \textsc{art} table \\

\glt 
‘The book is on the table.’ \textstyleExampleref{[Notes]}
\z

\ea\label{ex:3.147}
\gll E revareva rō {\ꞌ}ā te mōrī \textbf{a} \textbf{ruŋa} i te {\ꞌ}amurama{\ꞌ}a. \\
\textsc{ipfv} stand\_out:\textsc{red} \textsc{emph} \textsc{cont} \textsc{art} oil by above at \textsc{art} table \\

\glt 
‘The lamp is hanging above the table.’ \textstyleExampleref{[Notes]}
\z

\paragraph{}\label{sec:3.6.2.3.4} \textit{Vāeŋa}\is{vāeŋa ‘middle’} refers to the middle, the centre of something: 

\ea\label{ex:3.148}
\gll {\ꞌ}I \textbf{vāeŋa} o te vaikava he topa te {\ꞌ}ati nuinui.\\
at middle of \textsc{art} ocean \textsc{ntr} happen \textsc{art} problem big:\textsc{red}\\

\glt
‘In the middle of the ocean a big accident happened.’ \textstyleExampleref{[Fel-40-026]}
\z

In relation to a set of two referents it indicates a location in between the two:

\ea\label{ex:3.149}
\gll \textbf{{\ꞌ}I} \textbf{vāeŋa} o te hare nei {\ꞌ}e o te hare era te karapone. \\
at middle of \textsc{art} house \textsc{prox} and of \textsc{art} house \textsc{dist} \textsc{art} shed \\

\glt
‘The shed is between this house and that house.’ \textstyleExampleref{[Notes]}
\z

\paragraph{}\label{sec:3.6.2.3.5} Some combinations of preposition + locational\is{Locational} have specialised meanings:

\smallskip 

\textit{a raro}\is{raro ‘below’} ‘on foot’:

\ea\label{ex:3.150}
\gll Ko koro \textbf{a} \textbf{raro} {\ꞌ}ā i iri ai. Ko nua a ruŋa te hoi. \\
\textsc{prom} Dad by below \textsc{ident} \textsc{pfv} ascend \textsc{pvp} \textsc{prom} Mum by above \textsc{art} horse \\

\glt
‘Dad goes up (to the field) on foot. Mum goes on horse.’ \textstyleExampleref{[R184.052–053]}
\z

\newpage 
\textit{a vāeŋa}\is{vāeŋa ‘middle’} ‘in half’:


\ea\label{ex:3.151}
\gll Ana haŋa he {\ꞌ}avahi \textbf{a} \textbf{vāeŋa}, hoa hai miti...\\
\textsc{irr} want \textsc{ntr} divide by middle throw \textsc{ins} salt\\

\glt
‘If you want, you cut (the fish) in half, put salt on...’ \textstyleExampleref{[R185.007]}
\z

\textit{o ruŋa}\is{ruŋa ‘above’} \textit{i} ‘about’, in the sense of a topic of knowledge or discourse:

\ea\label{ex:3.152}
\gll He {\ꞌ}ui{\ꞌ}ui nō te aŋa \textbf{o} \textbf{ruŋa} \textbf{i} te {\ꞌ}a{\ꞌ}amu tuai. \\
\textsc{pred} ask:\textsc{red} just \textsc{art} do of above at \textsc{art} story ancient \\

\glt 
‘He was always asking about the old stories.’ \textstyleExampleref{[R302.018]} 
\z

\subsubsection[Temporal use of locationals]{Temporal use of locationals}\label{sec:3.6.2.4}

While \textit{mu{\ꞌ}a}\is{mua ‘front’@mu{\ꞌ}a ‘front’} ‘front’ and \textit{tu{\ꞌ}a}\is{tua ‘back’@tu{\ꞌ}a ‘back’} ‘back’ are primarily spatial terms, they are also used temporally, referring to past and future. However, the temporal dichotomy between past and future does not coincide with the spatial dichotomy between front and back – in other words, it is not the case that \textit{mu{\ꞌ}a} refers to the future and \textit{tu{\ꞌ}a} to the past, or the other way around. Rather, \textit{mu{\ꞌ}a and tu{\ꞌ}a} acquire specific temporal senses in combination with certain prepositions.\footnote{\label{fn:147}See \citet{Tetahiotupa2005} for an equally complex situation in \ili{Tahitian}. Temporal reference leads itself easily to ambiguity, as there are two fundamentally different ways to conceptualise the passage of time: either the world is seen as fixed and time moves from the future to the past, or time is fixed and we travel through it from the past to the future (see \citealt[296]{AndersonKeenan1985}). In the second case, the future is clearly ahead, while the past is behind. On the other hand, as the past is known and therefore visible while the future is unknown and invisible, the past can be conceived as being before our eyes, while the future is behind our backs.}

 
\textit{Pe}\is{pe ‘towards’} \textit{mu{\ꞌ}a}\is{mua ‘front’@mu{\ꞌ}a ‘front’} (often in the expression \textit{pe mu{\ꞌ}a ka oho ena}) means ‘later, in the future’:

\ea\label{ex:3.153}
\gll Mai te hora nei \textbf{pe} \textbf{mu{\ꞌ}a}, e ko take{\ꞌ}a haka{\ꞌ}ou au e koe. \\
from \textsc{art} time \textsc{prox} toward front \textsc{ipfv} \textsc{neg.ipfv} see again \textsc{1sg} \textsc{ag} \textsc{2sg} \\

\glt 
‘From now on, you won’t see me anymore.’ \textstyleExampleref{[R309.070]} 
\z

\ea\label{ex:3.154}
\gll \textbf{Pe} \textbf{mu{\ꞌ}a} ka oho ena, he haka aŋa rō au i te hare.      \\
toward front \textsc{cntg} go \textsc{med} \textsc{ntr} \textsc{caus} make \textsc{emph} \textsc{1sg} \textsc{acc} \textsc{art} house      \\

\glt 
‘Later, I will have a house built.’ \textstyleExampleref{[R229.029]} 
\z

\textit{A}\is{a (preposition)} \textit{tu{\ꞌ}a}\is{tua ‘back’@tu{\ꞌ}a ‘back’} either means ‘before, ago’ or ‘later, afterwards’:

\ea\label{ex:3.155}
\gll E ai rō {\ꞌ}ā te rivuho ... me{\ꞌ}e rahi matahiti \textbf{a} \textbf{tu{\ꞌ}a} i aŋa ai. \\
\textsc{ipfv} exist \textsc{emph} \textsc{cont} \textsc{art} drawing ~ thing many year by back \textsc{pfv} make \textsc{pvp} \\

\glt 
‘There is a drawing... made many years ago.’ \textstyleExampleref{[R296.010–013]}
\z

\ea\label{ex:3.156}
\gll Ka rima ta{\ꞌ}u \textbf{a} \textbf{tu{\ꞌ}a} ... he mana{\ꞌ}u haka{\ꞌ}ou a ia ki a Roke{\ꞌ}aua ararua ko Makita.\\
\textsc{cntg} five year by back ~ \textsc{ntr} think again \textsc{prop} \textsc{3sg} to \textsc{prop} Roke’aua the\_two \textsc{prom} Makita\\

\glt 
‘Five years later he thought again of Roke’aua and Makita.’ \textstyleExampleref{[R243.205]} 
\z

\textit{{\ꞌ}I/o mu{\ꞌ}a {\ꞌ}ā} means ‘first, in the past’:

\ea\label{ex:3.157}
\gll Te mana \textbf{{\ꞌ}i} \textbf{mu{\ꞌ}a} \textbf{{\ꞌ}ā} me{\ꞌ}e pūai. \\
\textsc{art} power at front \textsc{ident} thing strong \\

\glt 
‘\textit{Mana} (supernatural power) was something strong in the past.’ \textstyleExampleref{[R634.001]} 
\z

\subsection{Absolute locationals}\label{sec:3.6.3}
\is{Locational}
Polynesian languages have a small set of locationals\is{Locational} which locate a person or object with respect to a certain generally known geographical area.\footnote{\label{fn:148}Cf. \citet[21]{LevinsonWilkins2006}: “The absolute frame of reference in ordinary language use requires fixed bearings that are instantly available to all members of the community.”
See \citet{Cablitz2005} for a discussion of absolute or geocentric localisation in another Polynesian language, \ili{Marquesan}.} These can be labelled ‘absolute locationals\is{Locational}‘.

The absolute locationals are listed in \tabref{tab:15b}\is{Locational}:
\begin{table}
\begin{tabularx}{.66\textwidth}{p{20mm}X}
\lsptoprule
\textit{tai}\is{tai ‘seaside’}  & seashore (as opposed to land)\\
\textit{{\ꞌ}uta}\is{uta ‘inland’@{\ꞌ}uta ‘inland’} & land, inland (as opposed to sea)\\
\textit{tahatai}\is{tahatai ‘seashore’} & seashore\\
\textit{kampō}\is{kampō ‘countryside’} & countryside\\
\textit{kōnui}\is{konui ‘far’@kōnui ‘far’} & far{\rmfnm}\\
\lspbottomrule
\end{tabularx}
\caption{Absolute locationals}
\label{tab:15b}
\end{table}

\footnotetext{\label{fn:149}Based on its meaning \textit{kōnui} would seem to belong to the category of deictic locationals\is{Locational} (\sectref{sec:4.6.5} below). However, syntactically it behaves like the absolute locationals\is{Locational}, in that it can be followed by the postnominal demonstrative\is{Demonstrative!postnominal} \textit{era}; see section \sectref{sec:3.6.5} about elements modifying locationals\is{Locational}.}

\textit{Kampō} is borrowed from \ili{Spanish} \textit{campo} ‘field, countryside’.\footnote{\label{fn:150}Rapa Nui is not the only language in which the class of locationals\is{Locational} has been extended with borrowings\is{Borrowing}. For example, in \ili{Tongan}, \textit{uafa} ‘wharf’, \textit{piliisone} ‘prison’ and \textit{sitima} ‘steamer’ are locationals\is{Locational}. See \citet[55]{Clark1976}.}  The other words are common in the Polynesian languages.\footnote{\label{fn:151}Most Polynesian languages have a locational\is{Locational} \textit{kō} ‘there’, often modified by deictics \textit{nei}, \textit{ena} or \textit{era} to indicate the degree of distance. \textit{Kōnui}, in which \textit{kō} is modified by \textit{nui} ‘big’, is its only Rapa Nui reflex. (Similarly, in Rapa Nui \textit{raro nui} became lexicalised, meaning ‘deep’, and \textit{ruŋa nui}, meaning ‘high’.)} 

Like the relative locationals\is{Locational}, these words are immediately preceded by prepositions. Unlike the relative locationals\is{Locational}, they cannot be followed by a prepositional phrase indicating a second referent with respect to which the spatial relation holds.

The following sections discuss each of these locationals\is{Locational} in turn. First, however, a general note on spatial reference. As the list above shows, the main reference points for spatial orientation in Rapa Nui are related to the sea. Spatial reference in Rapa Nui reflects the geography of the environment in which the language is spoken: a single island, a closed world of limited dimensions. In this world, the coast is always close; it is either visible, or one knows at least in which direction it is. It is not surprising that orientation happens predominantly with respect to the sea.\footnote{\label{fn:152}A correlation between the geographical environment and grammaticalisation of spatial reference systems is crosslinguistically common; \citet{Palmer2015} captures this generalisation as the \textit{\textup{Topographic Correspondence Hypothesis}}: “absolute coordinate systems are not merely anchored in, but are motivated by the environment” (210).} 

As the speech community is small and the area is limited, common orientation points (most of them on the island, a few outside, like Tahiti and the mainland) are generally known by name. Therefore, spatial reference in stories often happens by place names\is{Geographical names}. The following is a typical example:

\ea\label{ex:3.158}
\gll He e{\ꞌ}a ki ruŋa, he tere he oho mai ki Ma{\ꞌ}uŋa Teatea, ki Mahatua ... He oho, he tu{\ꞌ}u ki Vaipū... \\
\textsc{ntr} go\_out to above \textsc{ntr} run \textsc{ntr} go hither to Ma’unga Teatea to Mahatua ~  \textsc{ntr} go \textsc{ntr} arrive to Vaipu \\

\glt
‘They got up and travelled to Ma’unga Teatea, to Mahatua ... They went and arrived at Vaipu...’ \textstyleExampleref{[Mtx-3-01.214–216]}
\z

  
Another feature of Rapa Nui geography is, that the Rapa Nui population is concentrated in one town. The rest of the island is largely uninhabited (though easily accessible) and can be designated as a whole by a couple of generic locationals\is{Locational}: either \textit{{\ꞌ}uta}\is{uta ‘inland’@{\ꞌ}uta ‘inland’} ‘inland’ where agriculture takes place, or \textit{kampō}\is{kampō ‘countryside’}, ‘the countryside’ where one goes for an outing. This will be discussed in more detail below.

The cardinal points (north, east, south, west) are not used for spatial orientation. \is{Proto-Polynesian}Proto-Polynesian does have words for two of these: \textit{*toŋa} ‘south, southern wind’ and \textit{*tokelau} ‘north, north wind’; these are reflected in many daughter languages, but in Rapa Nui they have a different sense: \textit{toŋa} = ‘winter’, \textit{tokerau} = ‘wind (in general)’.

\subsubsection{\textit{Tai} ‘seashore’; \textit{{\ꞌ}uta} ‘inland’}\label{sec:3.6.3.1}

\textit{Tai}\is{tai ‘seaside’} indicates orientation with respect to the seashore:\footnote{\label{fn:153}There is a difference in meaning between the locational\is{Locational} \textit{tai}, which refers to the seashore, and the noun \textit{tai}, which refers to the surface or condition of the sea:
\ea
\gll   
Ko māria {\ꞌ}ā te tai. \\
  \textsc{prf} calm \textsc{cont} \textsc{art} sea\\
  \glt 
  ‘The sea is calm.’\zlast}
% \todo[inline]{Is it possible to get rid of the blank line at the bottom of the footnote? It is probably caused by the interlinear example, but rather ugly.}
\ea\label{ex:3.159}
\gll Ko takataka tahi {\ꞌ}ā te ŋā poki \textbf{{\ꞌ}i} \textbf{tai}. \\
\textsc{prf} gather:\textsc{red} all \textsc{cont} \textsc{art} \textsc{pl} child at sea \\

\glt 
‘All the children gathered near the shore.’ \textstyleExampleref{[R161.013]} 
\z

\ea\label{ex:3.160}
\gll He turu \textbf{ki} \textbf{tai} hāhaki rāua ko te poki. \\
\textsc{ntr} go\_down to sea gather\_shellfish \textsc{2pl} \textsc{prom} \textsc{art} child \\

\glt 
‘She went down to the seashore to gather shellfish with her child.’ \textstyleExampleref{[Mtx-7-14.034]}
\z

\ea\label{ex:3.161}
\gll He oho atu te hānau momoko \textbf{a} \textbf{tai} \textbf{{\ꞌ}ā}. \\
\textsc{ntr} go away \textsc{art} race slender by sea \textsc{ident} \\

\glt 
‘The ‘slender race’{\rmfnm} went along the seashore.’ \textstyleExampleref{[Ley-3-06.029]}
\z
\footnotetext{There is some uncertainty about the meaning of the terms \textit{hānau {\ꞌ}e{\ꞌ}epe} and \textit{hānau momoko}. The traditional interpretation is ‘long ears’ and ‘short ears’, but \citet{Englert1978} translates ‘raza corpulenta’ and ‘raza delgada’, respectively (see \citealt{Mulloy1993}). More recently, \citet{Langdon1994} has defended the traditional interpretation.}

As \REF{ex:3.160} shows, the verb used for a movement in the direction of the sea is \textit{turu}\is{turu ‘to go down’} ‘go down’. This verb is always used for seaward movement, even when no vertical movement is involved. Note, however, that in the hilly landscape of Rapa Nui a movement towards the sea will often involve some downward movement.

The locational\is{Locational} \textit{tai}\is{tai ‘seaside’} is only used for movement and location on land. A movement at sea toward land is indicated with \textit{{\ꞌ}uta}\is{uta ‘inland’@{\ꞌ}uta ‘inland’} ‘inland’ (see the next section).

\textit{{\ꞌ}Uta} indicates orientation towards the inland, away from the coast. It may indicate a location on land (as opposed to the sea), or a place well inland (as opposed to the coastal region).

For example, \textit{ki {\ꞌ}uta}\is{uta ‘inland’@{\ꞌ}uta ‘inland’} either indicates a movement from sea to land as in \REF{ex:3.162}, or a movement from a place on land to a place further inland as in \REF{ex:3.163}. In the first case the verb \textit{tomo}\is{tomo ‘to go ashore’} ‘go ashore’ is used, in the second case \textit{iri}\is{iri ‘to go up’} ‘go up’.

\ea\label{ex:3.162}
\gll He tomo te taŋata \textbf{ki} \textbf{{\ꞌ}uta}. \\
\textsc{ntr} go\_ashore \textsc{art} man to inland \\

\glt 
‘The people went ashore.’ \textstyleExampleref{[Ley-2-03.036]}
\z

\ea\label{ex:3.163}
\gll He iri tau kope era \textbf{ki} \textbf{{\ꞌ}uta} ki te tau{\ꞌ}a. \\
\textsc{ntr} ascend \textsc{dem} person \textsc{dist} to inland to \textsc{art} battle \\

\glt 
‘That man went (further) inland to the battle.’ \textstyleExampleref{[Mtx-7-35.012]}
\z

\textit{Tai}\is{tai ‘seaside’} and \textit{{\ꞌ}uta}\is{uta ‘inland’@{\ꞌ}uta ‘inland’} are not only used for large-scale movement, but also for movement and localisation on a small scale. They may serve, for example, to localise people in a group, or objects on a table:

\ea\label{ex:3.164}
\gll Te me{\ꞌ}e ena o te \textbf{pā{\ꞌ}eŋa} \textbf{{\ꞌ}uta} ko tō{\ꞌ}oku māmā era. \\
\textsc{art} thing \textsc{med} of \textsc{art} side inland \textsc{prom} \textsc{poss.1sg.o} mother \textsc{dist} \\

\glt 
‘(looking at people in a photo:) The one on the inland side is my mother.’ \textstyleExampleref{[R411.057]} 
\z

\ea\label{ex:3.165}
\gll Ka va{\ꞌ}ai mai te ensalada o te \textbf{pā{\ꞌ}iŋa} \textbf{{\ꞌ}uta}. \\
\textsc{imp} give hither \textsc{art} salad of \textsc{art} side inland \\

\glt 
‘Pass me the salad on the inland side.’ \textstyleExampleref{[Notes]}
\z

\subsubsection{\textit{Tahatai} ‘seashore’}\label{sec:3.6.3.2}
\is{tahatai ‘seashore’}
\textit{Tahatai} indicates the seashore. Its meaning is similar to \textit{tai}\is{tai ‘seaside’} (\sectref{sec:3.6.3.1}), but seems to focus more narrowly on the line separating land and sea. Like the other locationals\is{Locational}, it may be preceded by different prepositions:

\ea\label{ex:3.166}
\gll He ŋā poki e kokori {\ꞌ}ā \textbf{{\ꞌ}i} \textbf{tahatai}. \\
\textsc{pred} \textsc{pl} child \textsc{ipfv} \textsc{pl}:play \textsc{cont} at seashore \\

\glt 
‘There are children playing on the seashore.’ \textstyleExampleref{[R415.950]} 
\z

\ea\label{ex:3.167}
\gll He turu te taŋata \textbf{ki} \textbf{tahatai} he ruku i te ika. \\
\textsc{ntr} go\_down \textsc{art} man to seashore \textsc{ntr} dive \textsc{acc} \textsc{art} fish \\

\glt 
‘The men went down to the seashore and fished underwater.’ \textstyleExampleref{[R372.016]} 
\z

\ea\label{ex:3.168}
\gll He ha{\ꞌ}ere a au he oho \textbf{a} \textbf{tahatai}. \\
\textsc{ntr} walk \textsc{prop} \textsc{1sg} \textsc{ntr} go by seashore \\

\glt
‘I walked along the seashore.’ \textstyleExampleref{[R475.010]} 
\z


Like \textit{tai}, \textit{tahatai}\is{tahatai ‘seashore’} is only used for movement on land. Movement from the sea to the shore is indicated by \textit{{\ꞌ}uta}.

\subsubsection{\textit{Kampō} ‘countryside’}\label{sec:3.6.3.3}
\is{kampō ‘countryside’}
\textit{Kampō}, from \ili{Spanish} \textit{campo}, indicates the area outside town. 

\ea\label{ex:3.169}
\gll He eke ararua ki ruŋa i te hoi he oho \textbf{ki} \textbf{kampō}. \\
\textsc{ntr} go\_up the\_two to above at \textsc{art} horse \textsc{ntr} go to countryside \\

\glt 
‘The two mounted their horse and went to the countryside.’ \textstyleExampleref{[R178.013]} 
\z

\ea\label{ex:3.170}
\gll {\ꞌ}I te mahana era {\ꞌ}i {\ꞌ}Ōvahe {\ꞌ}o {\ꞌ}i tētahi kona \textbf{o} \textbf{kampō}, i tomo era  te ika, he ha{\ꞌ}a{\ꞌ}ī ki ruŋa i te pere{\ꞌ}oa.\\
at \textsc{art} day \textsc{dist} at Ovahe or at other place of countryside \textsc{pfv} go\_ashore \textsc{dist}  \textsc{art} fish \textsc{ntr} fill to above at \textsc{art} car\\

\glt
‘On days when in Ovahe or another place in the country the fish would come ashore, they would load it on a wagon...’ \textstyleExampleref{[R539-1.482]}
\z

As \textit{kampō}\is{kampō ‘countryside’} is principally used with reference to outings, and as outings typically take place near the shore, \textit{kampō} usually refers to a place near the coast. In this respect it is different from \textit{{\ꞌ}uta}\is{uta ‘inland’@{\ꞌ}uta ‘inland’} ‘inland’, which often refers to areas inland where people grow their crops.

\subsubsection{\textit{Kōnui} ‘far’}\label{sec:3.6.3.4}
\is{konui ‘far’@kōnui ‘far’}
\textit{Kōnui} ‘far, distant’ does not indicate an absolute point of reference, but any point far away from the reference point. The reference point may be the starting point of a movement as in \REF{ex:3.171}, or the place where the action takes place as in \REF{ex:3.172}. 

\ea\label{ex:3.171}
\gll He tere he piko a Manu \textbf{ki} \textbf{kōnui} era. \\
\textsc{ntr} run \textsc{ntr} hide \textsc{prop} bird to far \textsc{dist} \\

\glt 
‘Manu fled and hid far away.’ \textstyleExampleref{[R459.007]} 
\z

\ea\label{ex:3.172}
\gll He u{\ꞌ}i atu \textbf{mai} \textbf{kōnui} nei {\ꞌ}ā ko te pua{\ꞌ}a ka teka, ka teka. \\
\textsc{ntr} look away from far \textsc{prox} \textsc{ident} \textsc{prom} \textsc{art} cow \textsc{cntg} revolve \textsc{cntg} revolve \\

% \largerpage
\glt 
‘From afar he saw a cow that was turning round and round.’ \textstyleExampleref{[R250.137]} 
\zlast


  
\subsection{Temporal locationals}\label{sec:3.6.4}
\is{Locational}
There are a number of words referring to time which, like locationals\is{Locational}, are preceded by prepositions. (They could be labelled ‘temporals’.) These words all share the non-productive prefix \textit{{\ꞌ}aŋa-}\is{anza- ‘recent past’@{\ꞌ}aŋa- ‘recent past’}, which indicates recent past.\footnote{\label{fn:155}This prefix occurs with a similar meaning in many other Polynesian languages, but always as a reflex of \is{Proto-Polynesian}PPN \textit{*{\ꞌ}ana}; Rapa Nui is the only language in which \textit{*n} became \textit{ŋ}. 

\citet[12]{Green1985} mentions \textit{*ina(a)fea} ‘when (past)’ as a \is{Central-Eastern Polynesian}PCE innovation; in fact, this reflects a more general shift from \is{Eastern Polynesian}PEP \textit{*{\ꞌ}ana-} to \is{Central-Eastern Polynesian}PCE \textit{*ina-}. This shift is not only reflected in *\textit{inafea}, but also in \ili{Māori} \textit{inapoo}, \ili{Tahitian} \textit{inapō} ‘last night’; \ili{Māori} \textit{inakuanei}, \ili{Tahitian} \textit{inā{\ꞌ}uanei}, \ili{Pa’umotu} \textit{inākuanei} ‘just now’; \ili{Tahitian} \& \ili{Māori} \textit{inanahi} ‘yesterday’ (Pollex, \citealt{GreenhillClark2011}; \citealt{Bauer1993}; \citealt{AcadémieTahitienne1999}). The Rapa Nui forms \textit{{\ꞌ}aŋa-} show that the shift \textit{a} {\textgreater} \textit{i} took place after Rapa Nui split from PEP. (Notice also that all reflexes of \textit{*ina} are from Tahitic languages, except \ili{Marquesan} \textit{inehea} ‘when (past)’.)} They are listed in \tabref{tab:15c}.

\begin{table}
\begin{tabularx}{\textwidth}{p{35mm}X}
\lsptoprule

\textit{{\ꞌ}aŋataiahi}\is{anzataiahi ‘yesterday’@{\ꞌ}aŋataiahi ‘yesterday’} &‘yesterday’\\

\textit{{\ꞌ}aŋapō}\is{anzapō ‘last night’@{\ꞌ}aŋapō ‘last night’}  & ‘last night’\\

\textit{{\ꞌ}aŋanīrā}\is{anzanīrā ‘earlier today’@{\ꞌ}aŋanīrā ‘earlier today’}\textit{/{\ꞌ}aŋarīnā}\is{anzarīnā ‘earlier today’@{\ꞌ}aŋarīnā ‘earlier today’}  & ‘earlier today’; also more general ‘today, nowadays’\\

\textit{{\ꞌ}aŋahē}\is{anzahe ‘when (past)’@{\ꞌ}aŋahē ‘when (past)’}  &‘when (past)’ (\sectref{sec:10.3.2.3}){\rmfnm}\\
\lspbottomrule
\end{tabularx}
\caption{Temporal locationals}
\label{tab:15c}
\end{table}


\footnotetext{\label{fn:156}About the origin of these terms: \textit{{\ꞌ}aŋapō} and \textit{{\ꞌ}aŋahē} are transparent: \textit{pō} ‘night’, \textit{hē} ‘content interrogative particle’ ({\textless} \is{Proto-Polynesian}PPN \textit{*fea}, see Footnote \ref{fn:490} on p.~\pageref{fn:490}). For \textit{-nīrā} and \textit{-rīnā} Pollex does not give any cognates (only \ili{Samoan} \textit{*analeilaa} ‘earlier today’ is a possible candidate). It is not clear which form is original in Rapa Nui, as both appear in older texts. For \textit{{\ꞌ}aŋataiahi}, the only known cognate in Pollex is \ili{Māori} (Eastern dialect) \textit{tainahi} ‘yesterday’. However, the second part \textit{-ahi} is common as part of a word meaning ‘yesterday’: most Polynesian languages have a reflex of \is{Proto-Polynesian}PPN *\textit{nanafi} ‘yesterday’, sometimes preceded by \textit{i-} or \textit{a-}.}

Like other locationals\is{Locational}, these words are preceded by prepositions, such as locative \textit{{\ꞌ}i} \REF{ex:3.173} or genitive \textit{o} \REF{ex:3.174}:

\ea\label{ex:3.173}
\gll Kai ha{\ꞌ}uru mātou \textbf{{\ꞌ}i} \textbf{{\ꞌ}aŋapō}. \\
\textsc{neg.pfv} sleep \textsc{1pl.excl} at last\_night \\

\glt 
‘We did not sleep last night.’ \textstyleExampleref{[R250.126]} 
\z

\ea\label{ex:3.174}
\gll Te nu{\ꞌ}u ruku \textbf{o} \textbf{{\ꞌ}aŋanīrā} ko ai {\ꞌ}ana te raperape, te hi{\ꞌ}o...\\
\textsc{art} people dive of today.\textsc{past} \textsc{prom} exist \textsc{cont} \textsc{art} swimming\_fin \textsc{art} glass\\

\glt 
‘Today’s divers have swimming fins, goggles...’ \textstyleExampleref{[R539-1.348]}
\z
 
There is also a set of three time words referring to the future; these are listed in \tabref{tab:15d}. 

\begin{table}
\begin{tabularx}{.5\textwidth}{p{28mm}X}
\lsptoprule
\textit{{\ꞌ}anīrā}\is{anira ‘later today’@{\ꞌ}anīrā ‘later today’}\textit{/{\ꞌ}arīnā}\is{arina ‘later today’@{\ꞌ}arīnā ‘later today’}  & ‘later today’\\
\textit{āpō}\is{apo ‘tomorrow’@āpō ‘tomorrow’}  & ‘tomorrow’\\
\textit{a hē} & ‘when (future)’\\
\lspbottomrule
\end{tabularx}
\caption{Temporal terms with future reference}
\label{tab:15d}
\end{table}
  
 
These are not locationals\is{Locational} but adverbs\is{Adverb}: they are not preceded by prepositions but form a clause adjunct on their own. The initial \is{a- ‘future’}\textit{a} in all three words reflects \is{Proto-Polynesian}PPN \textit{*{\ꞌ}ā-}, a prefix indicating near future (Pollex, \citealt{GreenhillClark2011}), despite the variety in spelling in its Rapa Nui reflexes (\textit{{\ꞌ}a}, \textit{ā}, \textit{a}).\footnote{\label{fn:157}This prefix occurs in different words in several languages, e.g. \ili{Samoan} \textit{aa taeao} ‘tomorrow’; \ili{Tongan} \textit{{\ꞌ}apogipogi} ‘tomorrow’; \ili{Tahitian} \textit{{\ꞌ}ā{\ꞌ}uanei} ‘shortly, in a while’, \textit{afea} ‘when (future)’ (cf. \textit{ina{\ꞌ}uanei} ‘just now, a while ago’, \textit{inahea} ‘when (past)’). \ili{Māori}, like Rapa Nui, has a whole set of expressions sharing this morpheme: \textit{aapoopoo} ‘tomorrow’, \textit{aa hea} ‘when (future)’, \textit{aaianei} ‘now’, \textit{aakuanei} ‘presently’, \textit{aa teeraa tau} ‘next year’ (see \citealt[79]{Biggs1973}).}

  
Some examples:

\ea\label{ex:3.175}
\gll E vovo, \textbf{{\ꞌ}anīrā} he hoki māua ki {\ꞌ}uta. \\
\textsc{voc} dear\_girl today.\textsc{fut} \textsc{ntr} return \textsc{1du.excl} to inland \\

\glt 
‘My girl, today we will return to the field.’ \textstyleExampleref{[R235.038]} 
\z

\ea\label{ex:3.176}
\gll \textbf{Āpō} he e{\ꞌ}a tātou ki ruŋa ki te vaka. \\
tomorrow \textsc{ntr} go\_out \textsc{1pl.incl} to above to \textsc{art} boat \\

\glt 
‘Tomorrow we will go out by boat.’ \textstyleExampleref{[R368.045]} 
\z

\ea\label{ex:3.177}
\gll ¿\textbf{A} \textbf{hē} tātou ka iri haka{\ꞌ}ou mai mo piroto? \\
~\textsc{fut} \textsc{cq} \textsc{1pl.incl} \textsc{cntg} ascend again hither for soccer \\

\glt
‘When are we going to play soccer again?’ \textstyleExampleref{[R155.007]} 
\z

Interestingly, Rapa Nui has no generic temporal words ‘now’ and/or ‘then’. To express these, the noun \textit{hora}\is{hora ‘time’} ‘time’ is used, in combination with a postnominal demonstrative (\sectref{sec:4.6.3}): \textit{hora nei} indicates temporal proximity ‘now’, \textit{hora era} expresses temporal distance ‘then’.

\subsubsection{\textit{Ra{\ꞌ}e} ‘first’}\label{sec:3.6.4.1}
\is{rae ‘first’@ra{\ꞌ}e ‘first’}
One more element needs to be mentioned here. \textit{Ra{\ꞌ}e} ‘first’ is used in a variety of constructions: it can be an adjective modifying a noun (\sectref{sec:4.3.3}), a verb, or an adverb\is{Adverb} modifying a verb. It is also used as a locational\is{Locational}, always preceded by the preposition \textit{{\ꞌ}i}. \textit{{\ꞌ}I ra{\ꞌ}e}\is{rae ‘first’@ra{\ꞌ}e ‘first’} functions as an adverbial phrase\is{Adverb} ‘first, before anything else’ (\sectref{sec:11.6.2.4}).

\ea\label{ex:3.178}
\gll He kai ia \textbf{{\ꞌ}i} \textbf{ra{\ꞌ}e} e tahi {\ꞌ}apa haraoa.\\
\textsc{ntr} eat then at first \textsc{num} one part bread\\

\glt 
‘First I will eat a piece of bread.’ \textstyleExampleref{[R476.031]}  
\z

\subsection{The locational phrase}\label{sec:3.6.5}
\is{Locational}
Like other nouns, locationals\is{Locational} can be modified by certain noun phrase elements. The full range of possibilities is represented in \tabref{tab:16}.

\begin{table} 
\fittable{
\begin{tabular}{>{\centering}p{8mm}>{\centering}p{16mm}c>{\centering}p{12mm}>{\centering}p{12mm}>{\centering}p{15mm}>{\centering}p{12mm}c}
\lsptoprule
 0& 1& 2& 3& 4& 5& 6& 7\\
prep.  & nucleus & adverb\newline & emph. marker & limit. marker & postnom. dem. &ident. marker & complement\newline\\
\midrule
 \textit{{\ꞌ}i},\newline \textit{mai, pe...}& 
 locational\is{Locational}& 
 \textit{tako{\ꞌ}a}& 
 \textit{mau}\is{mau ‘really’}&
 \textit{nō}\is{no ‘just’@nō ‘just’}&  
 \parbox[t]{1cm}{\centering \textit{nei}; \newline \textit{ena}; \newline \textit{era}\newline} &
 \textit{{\ꞌ}ā}\is{a (identity)@{\ꞌ}ā (identity)};  \newline \textit{{\ꞌ}ana}& 
 prep. + NP\\
\lspbottomrule
\end{tabular} 
}
\caption{Structure of the locational phrase}
% \todo[inline]{Small detail: in col. 5, nei, ena and era are not centered. Can't see why this happens.}
\label{tab:16}
\end{table}

Position 5 is only available for relative and absolute locationals\is{Locational}, not for deictic locationals\is{Locational}. This is not surprising, as postnominal demonstratives\is{Demonstrative} have (almost) the same form and function as the deictic \is{ena (medial distance)}locationals\is{Locational} themselves. Position 7, which connects the locational\is{Locational} to a second referent, is only available for relative locationals\is{Locational}, not for absolute and deictic locationals\is{Locational}.

\is{Deixis}Here are a few examples:

\ea\label{ex:3.179}
\gll Ki \textbf{roto} \textbf{mau} \textbf{{\ꞌ}ana} a Kekoa e haŋa era mo rere mai. \\
to inside really \textsc{ident} \textsc{prop} Kekoa \textsc{ipfv} want \textsc{dist} for fly hither \\

\glt 
‘Into (that pool) Kekoa wanted to jump.’ \textstyleExampleref{[R408.012]} 
\z

\ea\label{ex:3.180}
\gll \textbf{Pē} \textbf{rā} \textbf{nō} e kai e oho era.\\
like \textsc{dist} just \textsc{ipfv} eat \textsc{ipfv} go \textsc{dist}\\

\glt 
‘Just like that he kept eating.’ \textstyleExampleref{[R310.225]} 
\z

\ea\label{ex:3.181}
\gll Mai \textbf{{\ꞌ}uta} \textbf{era} au, mai roto mai te koro.\\
from inland \textsc{dist} \textsc{1sg} from inside hither \textsc{art} feast\_house\\

\glt
‘I’m coming from inland, from the feast house.’ \textstyleExampleref{[Mtx-7-20.034]}
\z

Compared to the common noun phrase (see the chart in \sectref{sec:5.1}), adjectives are absent from the locational phrase,\footnote{\label{fn:158}There is one exception: \textit{ruŋa} ‘above’ and \textit{raro} ‘below’ may be followed by \textit{nui} ‘big’, in both cases with idiomatic sense: \textit{ruŋa nui} ‘high’, \textit{raro nui} ‘deep’. Notice that the same element \textit{nui} has also been added to the original \is{Eastern Polynesian}PEP locative \textit{*kō} ‘there’, resulting in \textit{kōnui} ‘far’ (\sectref{sec:3.6.3}).} as well as anything related to quantification: determiners, quantifiers\is{Quantifier}, numerals, plural markers and the collective marker \textit{kuā}. The locational\is{Locational} phrase is very similar to the proper noun phrase (\sectref{sec:5.13.1}), which also excludes quantifying elements; the main difference is, that the latter includes the proper article\is{a (proper article)} \textit{a}.
\is{Locational|)}
\section{Conclusions}\label{sec:3.7}

Like other Polynesian languages, Rapa Nui has no inflectional (and little derivational) morphology; moreover, many lexical items are freely used in both the noun phrase and the verb phrase. The existence of a distinction between nouns and verbs in the lexicon has been questioned for Polynesian languages. However, in this chapter I argue that there are good grounds to maintain this distinction. Approaches which conflate the two classes (or which define the bulk of the lexicon as “universals”) do not do justice to the fact that the semantic relationship between the nominal and verbal uses of a lexeme is often unpredictable, and the fact that many words are either predominantly nominal or predominantly verbal in meaning and use. Rather, the occurrence of words with a typically verbal sense in the noun phrase can be regarded as cross-categorial use.

The boundary between nouns and verbs is not clear-cut; hence, the two can be defined in terms of a prototype, an intersection of certain syntactic, semantic and pragmatic features. In actual use, these features are not randomly distributed but tend to converge: a word referring to an entity tends to occur in a noun phrase, modified by noun phrase particles, and function as a referring expression.

The common cross-categorial use of nouns and verbs can be described in terms of two processes: lexical nominalisation (which turns a verb into a true noun, with a nominal sense) and syntactic nominalisation (where a verb is used in a construction which has certain nominal features). In both cases, the resulting nominal form may or may not have a suffix. While in lexical nominalisation the suffix is relatively uncommon, in syntactic nominalisation the use of the suffix depends on the construction; generally speaking, suffixed nominalisations are used when the event is presented as an object, a bounded entity, rather than as an event happening over time.

Syntactic nominalisation is in fact very common in Rapa Nui. In several constructions, a main clause predicate is constructed nominally; in addition, nominalised verbs are used in various subordinate constructions, such as causal clauses and certain complement clauses. The variety and frequency of nominal constructions are evidence of a “nominal drift”, a tendency to maximise the use of nominal constructions.

Nouns can be subdivided into common nouns (which are preceded by determiners), proper nouns (which take the proper article \textit{a}) and locationals (which take neither). 

Verbs can be subdivided into several classes, based on criteria such as the number of arguments, the use of the agent marker \textit{e} and the possibility to enter into the actor-emphatic construction. Adjectives are a subclass of verbs; they are characterised by frequency of adnominal use, as well by the presence of certain modifiers and the absence of modifiers occurring with other verbs.
