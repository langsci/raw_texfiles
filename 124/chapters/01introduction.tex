\chapter[Introduction]{Introduction}\label{ch:1}
\section{Rapa Nui: the island and the language}\label{sec:1.1}
\subsection{The island and its name}\label{sec:1.1.1}

%At the start of the text are works that should be included in the bibliography:%
% \todo[inline]{Include (c) info for maps in the front matter. If maps are in front matter, make sure they have no page header. Otherwise, place maps after the first page of ch. 1.}
\nocite{Shopen2007-1}\nocite{Shopen2007-2}\nocite{Shopen2007-3}\nocite{Shopen1985-1}\nocite{Shopen1985-3}\nocite{Buse1963Verbal}\nocite{Buse1963Nominal}Rapa Nui, also known as Easter Island, is located at 27°05’S 109°20’W. The island is known for its giant statues (\textit{mōai}), as well as for its extreme isolation: while the nearest islands (Sala y Gómez) are at a 400 km distance, the nearest inhabited island is tiny Pitcairn, 2100 km away. The closest population centres are Tahiti in French Polynesia (over 4200 km to the west) and Valparaíso on the Chilean coast (3700 km to the east).

The island forms a triangle, composed of three extinct volcanoes, with a surface of about 165 km\textsuperscript{2}. The highest point is Mt. Terevaka (507m).

At the last census (2012), the island’s population numbered 5,761.\footnote{\label{fn:1}http://www.ine.cl/canales/chile\_estadistico/familias/demograficas\_vitales.php (accessed 27 October 2015); the projected population for 2016 is 6,600.} Almost all inhabitants live in the town of Hanga Roa. Roughly half of the island’s population is of Rapa Nui origin; other inhabitants include continental Chileans, as well as small numbers from other nationalities. Conversely, numerous ethnic Rapa Nui live in continental Chile, while there is also a Rapa Nui community of a few hundred people on Tahiti.\footnote{\label{fn:2}It is extremely hard to estimate the total number of ethnic Rapa Nui. Estimates over the past years range from 2,600 to 7,748 (Bob Weber\ia{Weber, Robert}, p.c.). According to the 2012 census, 63.81\% of the inhabitants of Rapa Nui (i.e. 3,676 out of 5,761) belong to an indigenous people group. According to the same census, 2,697 people on Rapa Nui, and 4,934 people in Chile as a whole, are able to conduct a conversation in Rapa Nui. Though these figures seem to be impossibly high, they may give an indication of the number of people on the island and on the Chilean mainland adhering to the Rapa Nui identity.} 

The number of ethnic Rapa Nui does not coincide with the number of speakers of the Rapa Nui language. \citet{Wurm2007} estimates the number of speakers at 2,400–2,500, but the actual number is probably lower. \citet[192]{Makihara2001Adaptation} gives an estimate of 1,100 speakers, out of 1,800 ethnic Rapa Nui on the island; linguists \ia{Weber, Robert}\ia{Weber, Nancy}Bob \& Nancy Weber (p.c.) give a rough estimate of 1,000 speakers.

The name Rapa Nui, literally ‘great Rapa’, is used for the island, the people and the language.\footnote{\label{fn:3}The name is often spelled as a single word: Rapanui. In this grammar, the spelling Rapa Nui is used, in accordance with the accepted orthography (\sectref{sec:1.4.4}). The spelling sparked some debate in the \textit{Rapa Nui Journal}: \citet{Fischer1991,Fischer1993Or,Fischer1993Hoki}; \citet{WeberWeber1991}.} It may have been coined in 1862, when Rapa Nui people came in contact  
with people from Rapa, the southernmost island of what is nowadays French Polynesia (\citealt[64]{Fischer1993Naming}; \citealt[91]{Fischer2005}); the latter is also called Rapa Iti, ‘little Rapa’.\footnote{\label{fn:4}The meaning of the name Rapa itself is unknown, despite Caillot’s assertion that there cannot be any doubt that it means “en dehors, à l’extérieur [...] de l’autre côté” (outside, at the exterior, on the other side; \citealt[69]{Caillot1932}); the lexical sources for \ili{Rapa} (\citealt{Stokes1955}; \citealt{Fischer1996Green} [= James L. Green, 1864]; \citealt{KievietKieviet2006}) do not list a lexeme \textit{rapa}. In Rapa Nui there are two lexemes \textit{rapa}: 1) ‘to shine, be lustrous’; 2) ‘ceremonial paddle’.} 

The island has been known by many other names \citep{Fischer1993Naming}, all of them of post-contact origin. The name Easter Island and its corollaries in other European languages (Isla de Pascua, Osterinsel, Paaseiland, et cetera) dates back to 1722; it was given by the \ili{Dutch} explorers who discovered the Island on Easter Sunday, April 5. No pre-contact name for the island or the people has been transmitted, and none may ever have existed.

\subsection{Origins}\label{sec:1.1.2}
\largerpage
Linguistic, biological and archaeological data unambiguously indicate that the Rapa Nui people are Polynesians (\citealt{Green2000}; \citealt{Kirch2000}; \citealt{StefanCollinsCuny2002} and refs. there). In a certain sense, the early history of the island is uncomplicated. The island has a single language and there is no evidence that it was settled more than once. The date of settlement of the island is usually assumed to coincide with the date at which Rapa Nui split off from its protolanguage. 

Even so, the prehistory of the Rapa Nui people is still surrounded by uncertainty, despite extensive archaeological, biological and linguistic research. The only virtually uncontested fact is, that the first settlers of the island came from somewhere in east Polynesia.\footnote{\label{fn:5}Thor Heyerdahl’s theory that the Rapa Nui came from South America, is commonly rejected (see \citealt{BahnFlenley1992} for an extensive critique), though \citet{Schuhmacher1990} continues to explore the possibility of (secondary) influence of South-American languages on Rapa Nui. On the discussion about possible non-Polynesian elements in the Rapa Nui language, see Footnote \ref{fn:169} on p.~\pageref{fn:169}.} They probably arrived by a voyage of purposeful exploration rather than by chance (\citealt[72–79]{BahnFlenley1992}; \citealt[199]{KirchKahn2007}). Some scholars suggest an origin from the Marquesas (cf. \citealt[66]{BahnFlenley1992}), but the current consensus is that an origin from southeast Polynesia is much more likely, given the distance and prevailing winds. This means that the people who first discovered Rapa Nui probably arrived from Mangareva, Pitcairn and/or Henderson (\citealt{Green1998}; \citealt{StefanCollinsCuny2002}). Henderson Island, the closest habitable island to Rapa Nui (c. 1900 km), is deserted nowadays but was populated in the past, possibly as early as 700–900 AD (\citealt{Weisler1998}; \citealt{GreenWeisler2002}).

A more southern origin, from or through the Austral islands, has also been proposed (\citealt{LangdonTryon1983}), but is generally rejected (\citealt{Green1985,Green1998}).

\is{Proto Eastern Polynesian}The date of initial settlement of the island is much debated. In the past, attempts were made to date the split-off of Rapa Nui from its protolanguage by means of glottochronology (using the amount of lexical change and an assumed rate of change), but these did not give satisfactory results: \citet{Emory1963} obtained glottochronological dates as far apart as 1025 BC and 500 AD, and settled on an estimate of 500 AD on the basis of a single radiocarbon date provided by \citet[395]{HeyerdahlFerdon1961}.\footnote{\label{fn:6}In general, \is{Eastern Polynesian}Eastern Polynesian languages have changed vocabulary at a much higher rate than other Polynesian languages. \citet{Pawley2009} calculates replacement rates of 0.67–2.0\% per century for a number of non-EP languages, against 2.0–3.4\% per century for \is{Eastern Polynesian}EP languages (2.5\% for Rapa Nui), based on retention of basic \is{Proto-Polynesian}PPN vocabulary. This is explained by the “founder effect”, i.e. rapid change in a small isolated speech community (\citealt[138]{Marck2000}, \citealt[1818]{WilmshurstHunt2011}.} \citet{Green1967}, \citet[21]{Green1985}, \citet{Emory1972} and \citet{Kirch1986} also give an estimate between 400 and 600 AD based on radiocarbon dates. \citet{DuFeuFischer1993} and \citet{Fischer1992} suggest a possible split between Rapa Nui and its relatives as early as the first century AD. Others give later dates: between 600 and 800 AD (\citealt{Fischer2005}, based on a radiocarbon date of 690±130 given by \citealt{Ayres1971}; \citealt{GreenWeisler2002}), or between 800 and 1000 AD (\citealt[74]{Green2000}; \citealt{SpriggsAnderson1993}; \citealt{Martinsson-WallinCrockford2001}).

More recently, even later dates have been proposed. Re-examination of radiocarbon dates from Rapa Nui and other islands in east Polynesia, eliminating those samples not deemed reliable indicators for initial settlement, has led some scholars to date the onset of colonisation after 1200 AD (\citealt{HuntLipo2006,HuntLipo2007}; \citealt{Hunt2007}; \citealt{WilmshurstHunt2011}).

Others continue to propose dates late in the first millennium AD (\citealt{KirchKahn2007}; \citealt{MiethBork2005,MiethBork2010}).

The date of settlement of Rapa Nui is closely linked to the question of the colonisation of \is{Eastern Polynesian}east Polynesia as a whole, an issue which is in turn linked to the relative chronology of the different archipelagos in east Polynesia. Here as well, a wide range of dates has been proposed. Settlement of east Polynesia started either in the Society Islands, with Tahiti at the centre (\citealt{Emory1963}; \citealt{Kirch2000}; \citealt{WilmshurstHunt2011}), in the Marquesas (\citealt[290]{Wilson2012}, \citealt{Green1966}), or in the Societies/Marquesas area as a whole (\citealt[9]{Kirch1986}; \citealt[138]{Marck2000}). According to \citet{SpriggsAnderson1993}, there is solid archaeological evidence for human presence in the Marquesas from about 300–600 AD and in the Society Islands from 600–800 AD. \citet[9]{Kirch1986} suggests that the Marquesas may have been peopled as early as 200 BC. On the other hand, \citet{WilmshurstHunt2011} date the initial settlement of the Societies as late as 1025–1120 AD, while all the other archipelagos in east Polynesia (including Rapa Nui) followed after 1190.

The relation between the Rapa Nui language and \is{Eastern Polynesian}Eastern Polynesian is discussed in \sectref{sec:1.2.2} below.

\subsection{Snippets of history}\label{sec:1.1.3}

After its initial settlement, Rapa Nui may have maintained contact with other islands in east Polynesia, despite its geographical isolation (\citealt[424]{Clark1983Review}; \citealt{Green1998,Green2000}; \citealt{KirchKahn2007}).\footnote{\label{fn:7}\citet{Walworth2015Classifying} gives four words uniquely shared between Rapa Nui and \ili{Rapa}. However, three of these (Rapa \textit{matu} ‘to advance’, \textit{kakona} ‘sweet-smelling’, \textit{reka} ‘happy’) are also shared with other \is{Eastern Polynesian}EP languages, and the fourth (\textit{honi} ‘peel’) is a shared semantic innovation rather than a uniquely shared lexeme. Moreover, unique shared lexemes are not a strong indication of direct contact: Rapa Nui uniquely shares two words (\textit{ua} ‘war club’, \textit{ma{\ꞌ}a} ‘to know’) with \ili{Rennell} in the Solomon Islands, even though direct contact between the two islands is very unlikely.} At some point, there must even have been contact between at least one Polynesian island and South America, given the fact that the sweet potato and the bottle gourd spread from South America throughout Polynesia prior to European contact; \citet[98]{Green1998} suggests that Rapa Nui people may have travelled to South America, returning either to Rapa Nui or to another island.

However, contact between Rapa Nui and other islands was probably very intermittent; Rapa Nui language and culture developed in relative isolation, an isolation which at some point became complete. This explains the high amount of lexical innovation noticed by \citet{Emory1963}, \citet[45]{LangdonTryon1983} and \citet[36]{Bergmann1963}.

The history of Rapa Nui is described in \citet{BahnFlenley1992}, \citet{McCall1994}, \citet{FlenleyBahn2002} and \citet{Fischer2005}. Rapa Nui’s prehistory is the tale of a society constructing hundreds of increasingly large stone statues (\textit{mōai}) and transporting them to almost all corners of the island; a number of often feuding tribes whose names survive in legends; the gradual deforestation of an island once covered with giant palm trees;\footnote{\label{fn:8}The causes of the deforestation of Rapa Nui (human or by rats?) and the question whether it led to a socio-cultural collapse (traditionally dated around 1680) have been the subject of much debate, see \citet{FlenleyBahn2002}; \citet{Diamond2005}; \citet{MulrooneyLadefoged2007,MulrooneyLadefoged2009}; \citet{Hunt2007}; \citet{MiethBork2010}; \citet{Boersema2011}.} and the ‘birdman’ cult, which involved an annual contest between young warriors for the season’s first tern egg on one of the islets off the coast.

Rapa Nui entered written history on Easter Sunday, April 5, 1722, when it was sighted by a \ili{Dutch} fleet of three ships, commanded by Jacob Roggeveen\ia{Roggeveen, Jacob}. Later in the 18th century, the island was visited by a \ili{Spanish} expedition led by Don Felipe González\ia{González, Don Felipe} in 1770, followed by James Cook\ia{Cook, James} in 1774 and Count La Pérouse \ia{La Pérouse, Count} in 1786. From the early 19th century on, many explorers, traders and whalers called at the island.

The repeated arrival of foreigners caused epidemic diseases, which in turn led to depopulation and a major socio-cultural upheaval. A greater trauma was yet to follow: in 1862–1863, ships raided the Pacific in search of cheap labour for mines, plantations and households in Peru. Several of these visited Rapa Nui and at least 1400 people were abducted or lured away and taken to Peru.\footnote{\label{fn:9}In the course of these events, the name Rapa Nui may have emerged, see \sectref{sec:1.1.2} above \citep[91]{Fischer2005}.} Most of them died of smallpox; when a few survivors were repatriated late 1863, they brought the disease with them. As a result, the population of Rapa Nui dropped even further. The events of 1863 were fatal for Rapa Nui culture, leading to the collapse of the structure of society and ultimately to the loss of old customs and traditions \citep[391]{Knorozov1965}.

In 1870, the \ili{French} trader/adventurer Dutrou-Bornier\ia{Dutrou-Bornier, Jean-Baptiste}, acting for a Scottish company, managed to acquire title to most of the island and started to convert it into a giant sheep ranch. As the traditional power structure had collapsed, Dutrou-Bornier had free rein. When the situation for the remaining Rapa Nui seemed hopeless, bishop Tepano Jaussen \ia{Jaussen, Mgr. Tepano} of Tahiti formed the plan to evacuate all remaining inhabitants of the island; only the limited capacity of the vessel come to fetch them forced 230 people to stay on the island, while 275 left to settle in Mangareva and Tahiti. (\is{Borrowing}In the 1880s, some of them returned, bringing with them \ili{Tahitian}\is{Tahitian influence} elements which were subsequently incorporated into the Rapa Nui language.) The number of people on Rapa Nui further decreased to 111 in 1877, after which it started to rise slowly again, doubling by 1897 and again by 1934. 

\largerpage
In 1888, Rapa Nui was annexed by Chile. Even so, the island remained a sheep ranch under commercial control until 1953, when it passed under naval authority. During much of that time, islanders were not permitted to leave the island (presumably because of leprosy, an illness imported in the 1880s from Tahiti), so contact with the outside world was largely limited to the few foreign residents and visitors to the island.

In 1966, Rapa Nui became a civil territory, a department (since 1974 a province) within the 5\textsuperscript{th} region of Chile, consisting of a single municipality (\textit{comuna}). The Rapa Nui people received Chilean citizenship.  From 1960 on, Rapa Nui came out of its isolation. More and more Rapa Nui started to travel to the Chilean mainland for education and jobs; many of them settled there or emigrated to other countries. On the other hand, tourists and other visitors started to arrive in great numbers after the construction of the airport in 1967.\footnote{\label{fn:10}From 1970 on, Rapa Nui has been serviced by long-range jet airliners. As of October 2015, there were eight weekly flights to/from Santiago and one flight to/from Tahiti.} More jobs came available in the public sector (administration, education, health...), while the quickly expanding tourist industry also started to provide a host of job opportunities in hotels and guest houses, the building industry, retail and traditional crafts. As a result, over the past decades the island has experienced rapid economic development, but also a large influx of non-Rapa Nui residents (mainly from Chile). Tourism has continued to grow; currently the island attracts more than 40,000 people annually.

\section{Genetic affiliation}\label{sec:1.2}
\subsection{Rapa Nui in the Polynesian language family}\label{sec:1.2.1}

\is{Polynesian languages|(}Rapa Nui is a member of the Austronesian language family; its complete classification according to the \textit{Ethnologue} \citep{LewisSimons2015} is as follows: Austronesian, Ma\-la\-yo-Polynesian, Central-Eastern Malayo-Polynesian, Eastern Malayo-Polynesian, Oce\-anic, Central-Eastern Oceanic, Remote Oceanic, Central Pacific, East Fijian-Polynesian, Polynesian, Nuclear, East. The language has no dialects.

\begin{figure}
\includegraphics[width=\textwidth]{figures/LanguageTree.pdf}
\caption{Genetic classification of the Polynesian languages}
\label{fig:1}
\end{figure}

Rapa Nui’s immediate relatives are the other Polynesian languages,\footnote{\label{fn:11}See \citet{Krupa1982} for a typological overview of Polynesian languages and \citet{Krupa1973} for a history of research. More recent overviews are available for larger groupings: \citet{LynchRoss2002} for Oceanic, \citet{Blust2013} for Austronesian.} which number around 35. These languages are spoken within a triangle delineated by New Zealand in the south-west, Hawaii in the north and Rapa Nui in the east; a number of Polynesian groups (known as Outliers) are located outside this area.

The basic subgrouping of the Polynesian languages was established in the 1960s. While earlier approaches used lexicostatistics and glottochronology to measure relative distance between languages (see e.g. \citealt{Elbert1953}; \citealt{Emory1963}), in the mid-1960s research started to focus on shared innovations: languages are likely to form a subgroup when they have a significant number of phonological, lexical and/or grammatical innovations in common. This resulted in a hypothesis which became the standard theory for Polynesian subgrouping (see \citealt{Pawley1966}; \citealt{Green1966}; \citealt{Marck2000}), and which is represented in \figref{fig:1} (based on \citealt{Pawley1966}; \citealt{Clark1983Review}; \citealt{Marck2000}). In this subgrouping, all but two languages belong to the Nuclear Polynesian (NP) branch. NP is divided in two branches: Samoic-Outlier (SO) and Eastern Polynesian (\is{Eastern Polynesian}EP). Within EP, Rapa Nui forms a branch on its own, coordinate with Central-Eastern (\is{Central-Eastern Polynesian}CE) languages. CE in turn branches into Tahitic (TA) and Marquesic (MQ).\footnote{\label{fn:12}The evidence for \is{Eastern Polynesian}EP and \is{Central-Eastern Polynesian}CE will be reviewed in \sectref{sec:1.2.2} below.}  

Though there is a wide consensus on the basic tenets of this subgrouping, various refinements and modifications have been proposed.\footnote{\label{fn:13}One proposal generally rejected is that by \citet{LangdonTryon1983}, who propose a Futunic subgroup including \ili{East Futunan}, \ili{East Uvean}, \ili{Rennell} and Rapa Nui. The evidence for this subgroup is scant (see \citealt{Clark1983Review}; \citealt{Green1985}).} I will mention a few which directly or indirectly affect the position of Rapa Nui.

Firstly: within SO, there is evidence for a subgroup consisting of the Northern Outliers (NO), spoken in the northern Solomons (including the North Solomons province of Papua New Guinea) such as \ili{Takuu} and \ili{Luangiua}. A slightly larger group has also been suggested, consisting of the NO languages plus \ili{Kapingamarangi}, \ili{Nukuoro} and \ili{Tuvaluan} (‘Ellicean’, see \citealt{Howard1981}; \citealt{Pawley2009}). \citet{Wilson1985,Wilson2012} discusses a number of innovations shared between the Northern Outliers and \is{Eastern Polynesian}EP: a thorough restructuring and reduction of the pronominal system, as well as various other grammatical and lexical innovations. This leads him to suggest a NO-EP subgroup; in this hypothesis, the East Polynesians originated from the Northern Outliers, possibly migrating through the Ellice and Line Islands.

Secondly, \citet{Marck1996Subgrouping,Marck2000} proposes a few refinements within \is{Central-Eastern Polynesian}CE languages: nuclear Tahitic includes all Tahitic languages except New Zealand \ili{Māori}; nuclear Marquesic includes \ili{Marquesan} and \ili{Mangarevan}, but not \ili{Hawaiian}.\footnote{\label{fn:14}A different grouping is presented by \citet{Fischer2001Doublets}, who proposes a subgroup on the basis of doublets in \ili{Mangarevan}, such as \textit{{\ꞌ}a{\ꞌ}ine} ‘woman’ {\textasciitilde} \textit{ve{\ꞌ}ine} ‘wife’. The first member of these doublets has not participated in the sound change \textit{*faf} {\textgreater} \textit{*vah}, which is common to all \is{Central-Eastern Polynesian}CE languages but does not occur in Rapa Nui (\sectref{sec:1.2.2} below). According to Fischer, this constitutes evidence for a Proto-Southeastern Polynesian substratum, a subgroup which predates the differentiation of \is{Central-Eastern Polynesian}PCE into PTa and PMq, and which includes Rapa Nui. However, this analysis has been questioned: the doublets can also be explained as an incomplete phonological change \citep{Rutter2002}, and even if they suggest a substratum in \ili{Mangarevan} predating \is{Central-Eastern Polynesian}PCE, there is no evidence that this branch includes Rapa Nui \citep{Marck2002}.}

Finally: more recently, the validity of Tahitic and Marquesic as clear-cut subgroups has been questioned. \citet{Walworth2012} points out that the evidence for both subgroups is not very strong, something which has been recognised before (see e.g. \citealt{Green1966}; \citealt{Marck1996Subgrouping}). Neither subgroup is characterised by regular sound changes or morphological innovations; the only evidence consists of lexical and semantic innovations \citep{Green1966} and sporadic sound changes \citep{Marck2000}. Walworth suggests that there never was a Proto-Marquesic or a Proto-Tahitic language; rather, both branches may have developed through diffusion of features over certain geographical areas. She maintains the status of Proto-CE, which will be discussed in the next section.
\is{Polynesian languages|)}
\subsection{Evidence for Eastern Polynesian and Central-Eastern Polynesian}\label{sec:1.2.2}

\is{Eastern Polynesian|(}\is{Central-Eastern Polynesian|(}As Rapa Nui is the only language distinguishing EP from CE, it is worthwile to examine the evidence for both groupings. This evidence was collected by \citet{Pawley1966} and \citet{Green1985} and reviewed by \citet{Marck1996Subgrouping}; while Pawley did not differentiate between EP and CE for lack of data on Rapa Nui, Green did take Rapa Nui into account, though on the basis of limited data. \citet{Marck2000} provided further evidence on the basis of incidental sound changes. Here I will review the evidence adduced for both subgroups in the light of data and analysis for Rapa Nui presented in this grammar. In the list below, each proposed innovation is evaluated as valid (OK), invalid for the subgroup under consideration (\textbf{X}), or questionable (\textbf{??}).

For Eastern Polynesian, the following innovations have been suggested:

\subparagraph{Morphology}

%\setcounter{listWWviiiNumxxxileveli}{0}
\settowidth\jamwidth{OK}
\begin{enumerate}
\item 
The past tense marker \is{i (perfective)}\textit{i} (non-EP languages have \textit{na}, \textit{ne} or \textit{ni}): occurs in Rapa Nui.  \jambox{\textbf{OK}}

\item 
The negation \textit{k}\textit{āore/}\is{kore (negator)}\textit{kore}: occurs in Rapa Nui, though with limited use.   \jambox{\textbf{OK}}

\item 
\textit{pafa} ‘perhaps’: probably reflected in Rapa Nui \textit{pēaha}\is{peaha ‘perhaps’@pēaha ‘perhaps’}.   \jambox{\textbf{OK}}

\item 
Progressive \is{e (imperfective)}\is{ana (postverbal)@{\ꞌ}ana (postverbal)}\textit{e V {\ꞌ}ana}: occurs in Rapa Nui.   \jambox{\textbf{OK}}

\item 
\is{aha ‘what’}\textit{afa} ‘what’, against SO \textit{aa} and TO \textit{haa}: incorrect. \textit{*Afa} goes back to \is{Proto-Polynesian}PPN \textit{*hafa}, the form \textit{afa/aha} occurs in several SO languages.   \jambox{\textbf{X}}

\item 
\textit{e aha ... ai} ‘why’: occurs in Rapa Nui; however, the same construction occurs at least in \ili{Nukuoro} as well.   \jambox{\textbf{X}}

\item 
\textit{hei} ‘future location’: only occurs in Tahitic languages; the supposed Rapa Nui cognate \textit{he} is a different lexical item.   \jambox{\textbf{X}}

\end{enumerate}

\subparagraph{Lexicon}

\begin{enumerate}
\setcounter{enumi}{7}
\item 
157 entries in Pollex (2009)\footnote{\label{fn:16}Lexical data for individual languages have mostly been taken from the lexical database Pollex (2009 version; \citealt{GreenhillClark2011}).} are reconstructed for PEP.

\end{enumerate}

\subparagraph{Sporadic sound changes} \citep[131]{Marck2000}

\begin{enumerate}
\setcounter{enumi}{8}
\item 
PNP \textit{*maŋawai} {\textgreater} PEP \textit{*manavai} ‘tributary water course’; Rapa Nui \textit{manavai} ‘rock garden’.   \jambox{\textbf{OK}}

\item 
PNP \textit{*salu} {\textgreater} PEP \textit{*seru} ‘to scrape’; Rapa Nui \textit{heru}.   \jambox{\textbf{OK}}

\end{enumerate}

The following innovations are considered characteristic for Central-Eastern Polynesian:

\subparagraph{Phonology}

\begin{enumerate}
\setcounter{enumi}{10}
\item 
Loss of the \is{Proto-Polynesian}PPN glottal plosive\is{Glottal plosive}: basically correct, though the glottal left traces in some CE languages (\citealt[70–71]{Marck2000}). In any case, loss of the glottal happened several times independently in Polynesian languages and is no strong evidence for subgrouping.   \jambox{\textbf{??}}

\item 
\textit{*f} merges with \textit{*s} medially and before round vowels: this is in fact an EP innovation. The same happened in Rapa Nui, where \textit{*f} and \textit{*s} both became \textit{*h} in all environments (\sectref{sec:2.2.1}).   \jambox{\textbf{X}}

\item 
\textit{*f} {\textgreater} \textit{v} before \textit{*-af}: Rapa Nui \textit{haha} ‘mouth’ {\textasciitilde} PCE \textit{*vaha}; Rapa Nui \textit{haho} ‘outside’ {\textasciitilde} PCE \textit{*vaho}. However, both \ili{Marquesan} (\textit{haha/fafa} ‘mouth’) and \ili{Mangarevan} (\textit{{\ꞌ}a{\ꞌ}a} ‘mouth’, \textit{{\ꞌ}a{\ꞌ}ine} ‘woman’) have forms in which the change did not take place (\citealt[509]{Elbert1982}, \citealt[351–352]{Wilson2012}, \citealt{Fischer2001Doublets}).   \jambox{\textbf{??}}

\end{enumerate}

\subparagraph{Morphology}

\begin{enumerate}
\setcounter{enumi}{13}
\item 
\textit{tei} ‘present position’: only occurs in Tahitic languages → PTa rather than PCE.   \jambox{\textbf{X}}

\item 
\textit{inafea} ‘when’ (past): this is part of a larger change \textit{*ana} {\textgreater} *\textit{ina}, which only occurs in Tahitic languages (see Footnote \ref{fn:155} on p.~\pageref{fn:155}), except \ili{Marquesan} \textit{inehea} ‘when’.   \jambox{\textbf{??}}

\item 
The pronominal anaphor \textit{leila}: reflected in Rapa Nui \textit{ira}\is{ira (anaphor)} (\sectref{sec:4.6.5.2}). Moreover, it also occurs in \ili{Samoan} \citep[45]{Pawley1966}.   \jambox{\textbf{X}}

\item 
Possessives starting in \textit{nō/nā}: as I argue in Footnote \ref{fn:290} on p.~\pageref{fn:290}, these probably date back to PEP; in Rapa Nui, they merged with \is{Pronoun!possessive!Ø-class}Ø-possessives.   \jambox{\textbf{X}}\is{Possession}

\item 
\textit{me} ‘and, plus’ ({\textless} PNP \textit{*ma}): \textit{me} indeed occurs in a range of CE languages but not in Rapa Nui; however, the original \textit{ma/mā} continues in CE as well. \textit{Mā} occurs in Rapa Nui, but probably as a \ili{Tahitian} loan (see Footnote \ref{fn:167} on p.~\pageref{fn:167}); this means that the shift \textit{ma {\textgreater} me} is indeterminate between EP and CE.   \jambox{\textbf{??}}

\item 
\textit{taua} ‘demonstrative’: reflected in Rapa Nui \textit{tau} (see Footnote \ref{fn:211} on p.~\pageref{fn:211}).  \jambox{\textbf{X}}

\item 
\textit{ānei} ‘interrogative’: occurs in \ili{Tahitian} and \ili{Pa’umotu}, but I have not found the supposed reflexes in \ili{Mangarevan} and \ili{Hawaiian} → PTa rather than PCE.\footnote{\label{fn:15}Alternatively, \textit{ānei} may reflect an earlier stage than PEP, as suggested by \textit{anii} ‘question marker’ in \ili{Takuu} (see \citealt[23]{Moyle2011}).}   \jambox{\textbf{X}}

\item 
\textit{vai} ‘who’ ({\textless} \is{Proto-Polynesian}PPN \textit{*ai}). According to \citet[300]{Wilson2012}, \textit{vai} only occurred by PTa; \ili{Hawaiian} \textit{vai} could be under \ili{Tahitian} influence.   \jambox{\textbf{X}}

\item 
\textit{vau} ‘1sg’ as variant of \textit{au.} Only in \ili{Tahitian} and \ili{Pa’umotu}, and as a rare variant in \ili{Hawaiian}.   \jambox{\textbf{??}}

\end{enumerate}

\subparagraph{Syntax}

\begin{enumerate}
\setcounter{enumi}{22}
\item 
Loss of \is{Ergativity}ergative traces. However, Rapa Nui is fully accusative (\sectref{sec:8.4.2}), so ergative traces may have been lost by PEP.   \jambox{\textbf{X}}

\end{enumerate}

\subparagraph{Lexicon \& semantics}

\begin{enumerate}
\setcounter{enumi}{23}
\item 
553 entries in Pollex are reconstructed for PCE. Notice, however, that given the fact that PCE is distinguished from PEP by a single language, a lexeme reconstructed for PCE is not necessarily a PCE innovation: it could also be a PEP lexeme that was lost in Rapa Nui, or for which there are no data for Rapa Nui (cf. \citealt[445]{Geraghty2009}).\footnote{\label{fn:17}In fact, for any language X in family A, there will be a number of proto-A reconstructions for which there is no reflex in language X. This means that a subfamily B can be set up consisting of all languages of family A except language X, however implausible such a subgrouping may be on other grounds. For example, out of 710 \is{Eastern Polynesian}EP+\is{Central-Eastern Polynesian}CE reconstructions, only 67 are represented in \ili{Rapa}. On the basis of lexical data alone, one could thus propose a subgroup – let’s call it North-Eastern Polynesian – consisting of EP minus \ili{Rapa}, with no less than 643 reconstructions, while EP itself would be represented by only 67 reconstructions. ‘NEP’ would thus seem to be even more strongly motivated than \is{Central-Eastern Polynesian}CE. Even so, no one has ever proposed such a grouping. The small number of \ili{Rapa} reflexes can be explained by a small vocabulary (i.e. widespread loss) and lack of data.

For both \ili{Rapa} and Rapa Nui – and in fact for all \is{Eastern Polynesian}EP languages – the total number of reflexes in \is{Eastern Polynesian}EP and \is{Central-Eastern Polynesian}CE reconstructions is roughly in proportion to the total number of reflexes in Pollex as a whole.

A lexeme occurring in a branch of languages is likely to be an innovation of that branch if it can be shown to replace a lexeme with the same meaning occurring in a higher-order branch.} In fact, the Rapa Nui lexicon is known to show a high degree of innovation (\citealt[45]{LangdonTryon1983}, \citealt[36]{Bergmann1963}).   \jambox{\textbf{??}}

\item 
\is{Proto-Polynesian}PPN \textit{*tafito} ‘base’ {\textgreater} PCE ‘ancient’ (cf. Rapa Nui \textit{tahito} ‘base’; cf. PPN \textit{*tuai} ‘ancient’, also reflected in Rapa Nui).   \jambox{\textbf{OK}}

\item 
\is{Proto-Polynesian}PPN \textit{*kite} ‘to see’ {\textgreater} PCE ‘to know’ (cf. Rapa Nui \textit{tike{\ꞌ}a} ‘to see’; cf. old Rapa Nui \textit{ma{\ꞌ}a} ‘to know’, modern Rapa Nui \textit{{\ꞌ}ite} ‘to know’ {\textless} Tah.)   \jambox{\textbf{OK}}

\end{enumerate}

\subparagraph{Sporadic sound changes} (\citealt[41, 96–97]{Marck2000})

\begin{enumerate}
\setcounter{enumi}{26}
\item 
For PEP \textit{*hugovai} {\textgreater} PCE \textit{*hugavai} ‘parent-in-law’, reflexes of the PEP form occur not only in Rapa Nui, but in \ili{Māori} and \ili{Pa’umotu} as well.   \jambox{\textbf{X}}

\item 
PEP \textit{*kai {\textgreater}} PCE \textit{*koi} ‘sharp’, cf. Rapa Nui \textit{ka{\ꞌ}i}.   \jambox{\textbf{OK}}

\item 
PEP \textit{*tafora{\ꞌ}a} {\textgreater} PCE \textit{tofora{\ꞌ}a}, cf. Rapa Nui \textit{ta{\ꞌ}oraha} (an irregular reflex, but displaying the PEP vowel pattern).   \jambox{\textbf{OK}}

\item 
\citet{Marck2000} gives four more PCE sporadic sound changes; as none of the words in question occurs in Rapa Nui, these sound changes are indeterminate between EP and CE.   \jambox{\textbf{X}}

\end{enumerate}

\newpage 
To summarise: 

\begin{itemize}
\item 
EP is supported by four morphological changes (1, 2, 3, 4), two sporadic sound changes (9, 10), and a number of lexical innovations (8). In addition, one phonological and two morphological changes attributed to CE are actually EP innovations (12, 17, 19); the same may be true for one or two other morphological changes (16, 18), one syntactic change (23) and four sporadic sound changes (30).

\item 
CE is supported by two semantic innovations (25, 26), two sporadic sound changes (28, 29) and possibly a third (30), and a number of lexical innovations (24). In addition, it may be characterised by one or two phonological changes (11, 13) and possibly up to three morphological changes (15, 18, 22). 

\end{itemize}

We may conclude that both subgroups are reasonably well established, though on re-examination the evidence for CE is considerably weaker than has been suggested so far. This provides at least a partial solution to the challenge posed by newer theories of settlement, according to which eastern Polynesia was colonised late and rapidly (\sectref{sec:1.1.2} above). In these scenarios, there is not much time for EP and CE to develop in isolation, so a small number of innovations for both groups is expected.

The evidence still suggests that there is a CE subgroup within EP. However, the small number of innovations and a possibly shorter chronology call into question the identity of PCE: was there ever a community speaking PCE? In other words, did all the CE innovations occur in a unified language, before subgroups (TA and MQ) and individual languages started to diverge? Or did these innovations spread over the PCE area through contact, possibly after the protolanguage had started to diverge into different dialects? \citet{Walworth2012} proposes that innovations in Tahitic and Marquesic were not part of a unified protolanguage but spread by diffusion through different speech communities. The data above suggest that the same is true for PCE. 

This also means that the first colonisers of Rapa Nui did not necessarily leave an EP homeland where PEP was spoken as a unified language. If Rapa Nui was settled from southeast Polynesia, as is the growing consensus (\sectref{sec:1.1.2}), it is conceivable that the language spoken in that area, at the time Rapa Nui split off, was already starting to differentiate from PEP towards a proto-Marquesic speech variety. This possibility is suggested by the fact that Rapa Nui shares considerably more lexemes with Marquesic than with Tahitic (\citealt[94]{Emory1963}; \citealt[42–44]{LangdonTryon1983}; \citealt[424]{Clark1983Review}). This scenario is not in contradiction with the standard theory (according to which Marquesic and Tahitic languages together form the CE branch): it is altogether likely that speech communities within Eastern Polynesia, especially those relatively close together such as the Societies, the Tuamotus, Marquesas and Mangareva, remained in close contact, which facilitated the diffusion of subsequent “CE” innovations. In other words, CE innovations did not necessarily predate the onset of differentiation between Tahitic and Marquesic.
\is{Eastern Polynesian|)}\is{Central-Eastern Polynesian|)}
\section{The Rapa Nui language: typology and innovations}\label{sec:1.3}
\subsection{General typology}\label{sec:1.3.1}

Rapa Nui is characterised by the following typological features, most of which are shared by the Polynesian languages in general:

\begin{itemize}
\item 
The phoneme inventory is small: ten consonants, five short vowels and five long vowels.

\item 
Syllable structure is restricted to CV(ː). Moreover, there are strict metrical constraints on phonological words.

\item 
The basic constituent order is Verb – Subject – Object. Determiners and adpositions precede the noun; adjectives, possessives (except pronominal possessives) and relative clauses follow the noun.

\item 
In the area of word classes, there is a basic distinction between full words and (pre- and postnuclear) particles. There is a great freedom of cross-categorial use of nouns and verbs, to the extent that the existence of lexical nouns and verbs has been denied in some analyses of Polynesian languages.

\item 
Verbs are preceded by a preverbal marker. These markers form a multi-category paradigm, indicating either aspect, mood, subordination or negation.

\item 
Rapa Nui is an isolating language, even to a greater degree than other Polynesian languages, because of the loss of the passive suffix. There is no agreement marking on verbs, nor number marking on nouns.

\item 
In first person pronouns, there is a distinction between dual and plural, and between inclusive and exclusive. Unlike other Polynesian languages, Rapa Nui does not have a dual/plural distinction in second and third person pronouns.

\item 
There are two semantic categories of possession. These are not structurally different, but marked by a distinction between \textit{o} and \textit{a} in the possessive preposition or pronoun.

\item 
There is a general preference for nominal(ised) constructions (\sectref{sec:3.2.5}). 

\end{itemize}
\subsection{Innovations and losses in Rapa Nui}\label{sec:1.3.2}

In the course of history, a number of developments took place in Rapa Nui which did not take place in \is{Central-Eastern Polynesian}PCE (though they may have taken place independently in daughter languages). In this section, only phonological and grammatical changes are listed; lexical changes are not included.

%\setcounter{listWWviiiNumcxiileveli}{0}
\begin{enumerate}
\item 
Merger of \textit{*f} and \textit{*s} in all environments (\sectref{sec:2.2.1}). (This development also took place independently in a number of \is{Central-Eastern Polynesian}CE languages: \ili{Mangarevan}, \ili{Hawaiian}, \ili{Rapa}, \ili{Rarotongan}.)

\item 
Enforcement of strict \is{Metrical structure}metrical constraints, so that all word forms conform to a metrical scheme (\sectref{sec:2.3.2}).

\item 
A large number of sporadic phonological changes, such as \is{Metathesis}metathesis and vowel shifts (\sectref{sec:2.5.2}).

\item 
Monophthongisation (sometimes with shortening) of a number of CVV particles (\sectref{sec:2.5.2}).

\item 
A copying strategy for prepositions around locationals (\sectref{sec:3.6.2.2}).

\item 
Extension of the second and third person dual \is{Pronoun}pronouns to plurality (\sectref{sec:4.2.1.1}).

\item 
Loss of the neutral set of \is{Pronoun!possessive}possessive pronouns \textit{taku, tō, tana} (\citealt{Wilson2012}, cf. \sectref{sec:4.2.2.1.1}).

\item 
Loss of possessive pronouns starting with \textit{na-} and \textit{no-}; their function was taken over by Ø-forms (see Footnote \ref{fn:290} on p.~\pageref{fn:290}).

\item 
Development of definite \is{Numeral!definite}numerals, formed by reduplication (\sectref{sec:4.3.4}).

\item 
Loss of the \is{Proto-Polynesian}PPN numeral distributive prefix \textit{*taki-} (see Footnote \ref{fn:174} on p.~\pageref{fn:174}).

\item 
Possibly: merger of the numeral prefixes \textit{*hoko-} and \textit{*toko-} (see Footnote \ref{fn:175} on p.~\pageref{fn:175}).

\item 
Development of certain sentential particles: deictic \textit{{\ꞌ}ī} and \textit{{\ꞌ}ai}, intensifier \textit{rā}, dubitative \textit{hō}, asseverative \textit{{\ꞌ}ō} (\sectref{sec:4.5.4}).

\item 
Loss of the \is{Demonstrative!determiner}demonstrative determiners \textit{*teenei, *teenaa, *teeraa} (see Footnote \ref{fn:223} on p.~\pageref{fn:223}; cf. \citealt[44]{Pawley1966}).

\item 
Development of the prepositions \textit{pē} ‘like’ (\sectref{sec:4.7.8}) and \textit{pe} ‘towards’ (\sectref{sec:4.7.5}).

\item 
Development of the instrumental preposition \is{hai (instrumental prep.)}\textit{hai}, probably from the prefix \textit{hai-} (\sectref{sec:4.7.9}).

\item 
Emergence of the collective marker \textit{kuā}\is{kua (collective)@kuā (collective)}\textit{/koā} (\sectref{sec:5.2}).

\item 
Restriction of prenominal possessives to pronouns; full noun phrases as possessives only occur after the noun (see Footnote \ref{fn:284} on p.~\pageref{fn:284}).

\item 
Loss of the distinction between \is{Possession!o/a distinction}\textit{o-} and \textit{a-}possession in common nouns and plural pronouns (\sectref{sec:6.3.2}).

\item 
Possibly a shift in marking of the Agent in nominalised constructions: possessive agents are \textit{o-}marked, against \textit{a}{}-marking in other \is{Eastern Polynesian}Polynesian languages (see Footnote \ref{fn:301} on p.~\pageref{fn:301}).

\item 
Development of the plural marker \textit{ŋā}\is{nza (plural marker)@ŋā (plural marker)} from a determiner into a particle co-occur\-ring with determiners (though there are traces of this development in other \is{Eastern Polynesian}EP languages as well) (\sectref{sec:5.5.1.1}).

\item 
Loss of certain \is{Noun phrase!headless}headless noun phrase constructions (\sectref{sec:5.6}). For example, headless relative clauses (including clefts) are excluded (\sectref{sec:11.4.1}, \sectref{sec:9.2.6}); attributive clauses need a predicate noun (\sectref{sec:9.2.7}). 

\item 
Extension of the use of the postverbal continuity marker \is{ana (postverbal)@{\ꞌ}ana (postverbal)}\textit{{\ꞌ}ana} to the noun phrase (\sectref{sec:5.9}).

\item 
Development of the nominal predicate \is{he (nominal predicate marker)}marker \is{he (aspect marker)}\textit{he} into an aspect marker (\sectref{sec:7.2.3}).

\item 
Obligatory occurrence of the continuity marker \textit{{\ꞌ}ana/{\ꞌ}ā} after the perfect aspect marker \textit{ku/ko} (\sectref{sec:7.2.7}).

\item 
Restriction of the postverbal anaphoric particle \is{ai (postverbal)}\textit{ai} to the perfective aspectual \textit{i}, with extension in use from an anaphoric marker to a general postverbal demonstrative (\sectref{sec:7.6.5}).

\item 
Development of the preverbal modifier \is{rava ‘usually’}\textit{rava} ‘usually, given to’ (\sectref{sec:7.3.1}).

\item 
Reduction of the set of \is{Directional}directionals to \textit{mai} ‘hither’ and \textit{atu} ‘away’ (\sectref{sec:7.5}). A third directional, \textit{iho}, was reanalysed as an adverb (\sectref{sec:4.5.3.1}); others were lost.

\item 
Emergence of a \is{Serial verb}serial verb construction with repetition of the preverbal marker (\sectref{sec:7.7}).

\item 
On the premise that \is{Eastern Polynesian}PEP had accusative case marking: extension of the \is{e (agent marker)}agentive marker \textit{e} from passive to active clauses (\sectref{sec:8.2}–\ref{sec:8.4}).

\item 
Emergence of a nominal \is{Actor-emphatic construction}actor-emphatic construction, besides a perfective and an imperfective actor-emphatic (\sectref{sec:8.6.3}).

\item 
Restructuring of the \is{Negation}negation system, with the development of \textit{{\ꞌ}ina} as neutral, \textit{e ko} as imperfective and \textit{kai} as perfective negator, while \textit{ta{\ꞌ}e} is relegated to constituent negation and \textit{kore} to noun negation (\sectref{sec:10.5}).

\largerpage
\item 
Possibly: development of bare \is{Clause!relative!bare}relative clauses, i.e. without preverbal marker\\ (\sectref{sec:11.4.5}).

\item 
Development of the benefactive preposition \is{mo (preverbal)}\textit{mo} into a preverbal purpose marker (\sectref{sec:11.5.1}; cf. \citealt[27]{FinneyAlexander1998}).

\item 
Possibly: emergence of the preverbal irrealis marker \is{ana ‘irrealis’}\textit{ana} (\sectref{sec:11.5.2}; NB preverbal \textit{ana} is used in certain contexts in \ili{Māori} as well).

\end{enumerate}

\newpage 
In recent times, the following developments took place:

\begin{enumerate}
\setcounter{enumi}{34}
\item 
Disappearance of the preposition copying strategy around locationals (\sectref{sec:3.6.2.2}).

\item 
Replacement of \is{Numeral}numerals by \is{Tahitian influence}\ili{Tahitian} equivalents: numerals 1–7 in some contexts, those above 7 in all contexts (\sectref{sec:4.3.1}).

\item 
Restructuring of the quantifier system through borrowing and reanalysis of \ili{Tahitian} (and, to a lesser extent, \is{Spanish influence}\is{Quantifier}\ili{Spanish}) quantifiers (\sectref{sec:4.4.1}, \sectref{sec:4.4.11}).

\item 
Development of \is{Demonstrative!determiner}demonstrative determiners \textit{nei}, \textit{nā} and \textit{rā} (\sectref{sec:4.6.4}).

\item 
Extension in use of the collective marker \textit{ku}\textit{ā}\is{kua (collective)@kuā (collective)}\textit{/koā} (\sectref{sec:5.2}).

\item 
Increased use of the existential verb \is{ai ‘to exist’}\textit{ai} in existential and possessive clauses (\sectref{sec:9.3.1}, \sectref{sec:9.3.3}).

\item 
Extension of the use of agentive marker \is{e (agent marker)}\textit{e} (\sectref{sec:8.3.1.5}).

\item 
Incipient development of \is{Verb!copula}copula verbs (\sectref{sec:9.6}).

\item 
Emergence of \is{Exclamative}exclamative constructions introduced with the prominence marker \textit{ko} (\sectref{sec:10.4.2}).

\item 
Introduction of \is{Conjunction}conjunctions \textit{{\ꞌ}e} ‘and’ (\sectref{sec:11.2.1}) and \textit{{\ꞌ}o} ‘or’ (\sectref{sec:11.2.2}), as well as \textit{{\ꞌ}ātā} ‘until’ (\sectref{sec:11.6.2.5}), \textit{ante} ‘before’ (\sectref{sec:11.6.2.4}), \textit{pero} ‘but’ (\sectref{sec:11.2.1}).

\item 
Introduction of \is{Verb!modal}modal verbs from \ili{Spanish}`: \textit{pu}\textit{ē} ‘can’, \textit{tiene que} ‘must’ (\sectref{sec:11.3.6}).

\end{enumerate}
\section{Sociolinguistic situation}\label{sec:1.4}

Rapa Nui has undergone profound influence from two major sources: \ili{Tahitian} and \ili{Spanish}.

\subsection{Influence from Tahitian}\label{sec:1.4.1}

\is{Tahitian influence|(}\ili{Tahitian} started to exert its influence in the 1880s, when Rapa Nui speakers who had migrated to Tahiti in the 1870s started to remigrate (\sectref{sec:1.1.3} above). After 1889, contacts between Rapa Nui and Tahiti were scarce \citep[141]{Fischer2005}; they slowly resumed in the mid-20th century. To this day, a few hundred Rapa Nui live on Tahiti, and a weekly flight enables regular contact between the two islands.

The influence of \ili{Tahitian} on modern Rapa Nui is striking. In my lexical database, which contains 5,833 lexical entries, 543 items are marked as (probably) of \ili{Tahitian} origin, and another 89 as possibly \ili{Tahitian}. Many of these can be distinguished phonologically, as the \ili{Tahitian} and Rapa Nui consonant inventories are different, especially in the distribution of the glottal plosive (\sectref{sec:2.2.1}). Others can be recognised because of their semantics and/or recent introduction (see e.g. the discussion about \textit{riro} ‘to become’ in \sectref{sec:9.6.2}). \ili{Tahitian} vocabulary includes a number of very common words, such as \textit{{\ꞌ}ite} ‘to know’, \textit{ha{\ꞌ}amata} ‘to begin’, \textit{{\ꞌ}ī} ‘full’, \textit{hāpī} ‘to learn’, \textit{māuruuru} ‘thank you’ and the everyday greeting \textit{{\ꞌ}iorana}.

One reason why \ili{Tahitian} elements are easily adopted into the language, is their ‘vernacular feel’. \ili{Tahitian} words match the Rapa Nui phoneme inventory and word-forming constraints, with a few exceptions (\sectref{sec:2.5.3.2}). As a result, \ili{Tahitian} borrowings are not perceived as intrusions; unlike \ili{Spanish} borrowings, they are not avoided in written language and formal styles.

On historical grounds it seems plausible to date the intrusion of \ili{Tahitian} elements to the 1880s (cf. \citealt[315]{Fischer2001Hispan}), when Rapa Nui remigrated from Tahiti. This remigration happened at the time when the population was at an all-time low, a situation conducive to rapid language change. Moreover, in the same period \ili{Tahitian} catechists came to Rapa Nui, as well as foremen and labourers for the sheep ranch (\citealt[101]{DiCastri1999}). According to \citet[32]{Métraux1971}, by 1935 many \ili{Tahitian} words had entered the language; already in 1912, \citet[65]{Knoche1912} noticed that \ili{Tahitian} had exercised “einen grossen Einfluss auf Sitten und Sprache der Insulaner” (a large influence on the customs and speech of the islanders).

However, when we look at Rapa Nui texts from the 1910s–1930s (\sectref{sec:1.6.2}), the scarcity of \ili{Tahitian} influence is striking, in comparison to modern Rapa Nui. The \ili{Tahitian} numerals (\sectref{sec:4.3.1}) are not used, except the occasional \textit{va{\ꞌ}u} ‘eight’ (though the original \textit{varu} is much more common). The \ili{Tahitian} quantifiers \textit{ta{\ꞌ}ato{\ꞌ}a} and \textit{paurō} ‘all’ (\sectref{sec:4.4.2}, \sectref{sec:4.4.3}) do not occur either. Certain \ili{Tahitian} words are commonly used in older texts (\textit{\mbox{rava{\ꞌ}a}} ‘to obtain’, \textit{{\ꞌ}ī} ‘full’, \textit{mana{\ꞌ}u} ‘think’, \textit{\mbox{{\ꞌ}a{\ꞌ}amu}} ‘story’), but many words common nowadays occur rarely or not at all in older texts: \textit{{\ꞌ}ite} ‘to know’, \textit{riro} ‘to become’, \textit{{\ꞌ}onotau} ‘epoch’, \textit{ha{\ꞌ}amata} ‘to begin’, \textit{māere} ‘to be surprised’, \textit{māhatu} ‘heart’, \textit{māuruuru} ‘to thank; thanks’, \textit{nehenehe} ‘beautiful’, \textit{{\ꞌ}e} ‘and’, \textit{\mbox{nu{\ꞌ}u}} ‘people’, and so on.\footnote{\label{fn:18}Of these words, only \textit{{\ꞌ}ite} is found in Englert’s dictionary (first published in 1948). Notice, however, that Englert does not include words known to be of recent origin.} 

\largerpage
This suggests that many \ili{Tahitian} words common nowadays only came into use after the 1930s. The \ili{Tahitian} influence noticed by Knoche and Métraux must have been less pervasive than it is today. An alternative explanation would be, that the language of the older texts is archaic and reflects a variety which was current before 1880, possibly through verbatim transmission of old legends; after all, many of these texts represent old traditions. This is not very likely, however: it would leave unexplained why certain \ili{Tahitian} words are very common, while many others – equally common nowadays – do not occur at all. Neither would it explain why roughly the same picture emerges from all corpora of older texts (Egt, Ley, Mtx and MsE), including a long text which tells of post-1880 events (Ley-9-63, memories of catechist Nicolás Pakarati, recounted by his widow).\footnote{\label{fn:19}Only for a few words do the corpora differ mutually: \textit{{\ꞌ}ati} ‘problem’ occurs in Mtx and Ley, but not in MsE.} It is hard to conceive that scores of words borrowed 50 or 60 years previously would have been completely avoided in traditional stories, while others were freely used. Rather, the picture that emerges is one of two waves of \ili{Tahitian} intrusions: one in the 1870s and 1880s, followed by a much bigger one after 1960, when intensive contacts between Rapa Nui and the outside world (including Tahiti) were established.
\is{Tahitian influence|)}
\subsection{Influence from Spanish}\label{sec:1.4.2}

\is{Spanish influence|(}The \ili{Spanish} influence on modern Rapa Nui is likewise massive. This influence is not noticeable in the older texts, even though Rapa Nui had been a Chilean territory for almost 50 years by the time these texts were collected. \ili{Spanish} influence only started to make itself felt from the 1960s on, when Rapa Nui speakers acquired Chilean citizenship, began to participate actively in government and politics, acquired jobs for which proficiency in \ili{Spanish} was a prerequisite, and increasingly took part in secondary and tertiary education. \ili{Spanish} is also the language of the media, the predominant language of the Roman Catholic church, and the language of the many Chileans from the mainland who moved to the island (ultimately resulting in a high proportion of intermarriage). All of this led to a gradual incursion of \ili{Spanish} elements into the language. 

My lexical database contains 201 lexemes of \ili{Spanish} origin, but this only represents words well entrenched in the language, often with adaptation to Rapa Nui phonology (\sectref{sec:2.5.3.1}). In everyday speech, the number of \ili{Spanish} words is much higher. Most of these are not considered as part of the Rapa Nui lexicon but as foreign intrusions, i.e. as instances of code mixing. 

Code mixing is extremely common in modern Rapa Nui speech, involving single words, phrases, sentences or longer stretches of speech; see \citet{Makihara2001Changing,Makihara2001Adaptation,Makihara2004,Makihara2007,Makihara2009} for examples and discussion. In most modern Rapa Nui texts in my corpus, the amount of code mixing is considerably lower than in Makihara’s examples. This can be explained by the fact that a large part of my corpus consists of text types for which the use of \ili{Spanish} is considered less acceptable: (a) traditional stories; (b) written texts; (c) edited spoken texts.\footnote{\label{fn:20}Another reason for the discrepancy may be that much of the corpus is slightly older (1977–1990) than Makihara’s data (after 1990). However, relatively high amounts of code mixing are found in some of the oldest (informal) texts in the corpus.} Moreover, traditional stories make less reference to modern institutions and artifacts, so there is less need for the use of \ili{Spanish} elements.

\citet{Makihara1998,Makihara2009} signals a growing trend of purism, in which people attempt to speak Rapa Nui free of \ili{Spanish} influence. This happens especially in political discourse, but is spreading to other domains.

The extent of \ili{Tahitian} influence has led \citet[47]{Fischer1996Review} to characterise modern Rapa Nui as a “Rapanui-\ili{Tahitian} hybrid”, a product of “language intertwining” \citep[151]{Fischer2008Reversing}. However, while the lexicon of modern Rapa Nui is heavily influenced by \ili{Tahitian}, the grammar has not been affected to the same degree, as the following chapters will make clear (cf. \citealt[194]{Makihara2001Adaptation}). Even in areas where massive replacement by \ili{Tahitian} terms has taken place, e.g. quantifiers (\sectref{sec:4.4.11}) and numerals (\sectref{sec:4.3.1}), these terms have been reinterpreted into a “native” Rapa Nui syntax.

The same is true for \ili{Spanish}. \ili{Spanish} has certainly influenced the grammar of Rapa Nui, but \ili{Spanish} borrowings have been integrated into Rapa Nui grammar without transfer of their syntactic features. For example, the \ili{Spanish} noun \textit{kampō} ‘countryside’ ({\textless} \textit{campo}) became a locational (\sectref{sec:3.6.3.3}); \textit{kā} ‘each’ ({\textless} \textit{cada}) became a quantifier compatible with plurality (\sectref{sec:4.4.8.2}). The modal verbs \textit{puē} ‘can’ and \textit{tiene que} ‘must’ were borrowed (\sectref{sec:11.3.6}), but the third person singular of these verbs is used with all persons and numbers, and they are used with Rapa Nui syntactic features like \textit{mo}{}-complements. On the other hand, certain \ili{Spanish} semantic and syntactic features have become common without borrowing of the lexical items: \textit{kē} ‘several’ (\sectref{sec:4.4.8.1}), copular verbs (\sectref{sec:9.6}), the coordinating conjunction \textit{{\ꞌ}e} ‘and’ (\sectref{sec:11.2.1}), the construction \textit{oho mo} ‘to be about to’ (\sectref{sec:11.3.2.4}), et cetera. These elements have affected Rapa Nui grammar to a certain degree, but the same cannot be said of the numerous \ili{Spanish} words and phrases interspersed in everyday speech. The fact that \ili{Spanish} intrusions are avoided in certain types of discourse, confirms that these are instances of code switching and belong to the domain of language use (\textit{parole}), without having profound effect on the linguistic system (\textit{langue}) of modern Rapa Nui (cf. \citealt[193]{Makihara2001Adaptation}).
\is{Spanish influence|)}

\subsection{Language use and vitality}\label{sec:1.4.3}
\largerpage
As indicated in \sectref{sec:1.2.1} above, Rapa Nui does not have dialects. On the other hand, there is considerable idiolectal variation between the speech varieties of individual speakers and of different families, e.g. in the use of certain lexical items and the degree of Tahitianisation (cf. \citealt[154]{Fischer2008Reversing}).

While Rapa Nui grammar has retained its distinctive character and has not become a Rapa Nui-\ili{Tahitian} and/or Rapa Nui-\ili{Spanish} mix, the language is certainly endangered. The factors mentioned above which led to \ili{Spanish} influence on the language (participation in Chilean civil life, education, jobs, immigration of mainland Chileans, intermarriage) also led to a gradual increase in the use of \ili{Spanish} by Rapa Nui people. From the 1960s on, Rapa Nui people started to aspire to “being Chilean” \citep[315]{Fischer2001Hispan}, something for which proficiency in \ili{Spanish} was essential. As a result, it became common for Rapa Nui people to use \ili{Spanish}, initially in interaction with mainland Chileans, but then also between each other, both in public and at home. From the 1980s on, this meant that many children – even those from two Rapa Nui parents – learned \ili{Spanish} as their first language. \citet{WeberWeber1990Sobrevivir} found that the number of primary school children who were fully proficient in Rapa Nui (either as first language or by being bilingual) had decreased from 77\% in 1977 to 25\% in 1989. This can only partly be explained from an increased proportion of children from continental or mixed households. In 1997, a production/comprehension test among primary school children living on the island showed that only 49 out of 558 children (9\%) were fully bilingual; an additional 80 (14\%) had a reasonable level of comprehension and production in Rapa Nui (a score of 4 or higher on a scale of 0–7); 329 (59\%) had virtually no proficiency at all \citep{WeberWeber1998}. 

This trend did not go unnoticed. Various measures were taken to enhance the chances of survival of the language, many of these initiated or assisted by the \textit{Programa Lengua Rapa Nui}. One of these was the institution of an immersion program in the local primary school, extending from kinder until year 4. This program has achieved a varying degree of success \citep{Makihara2009}. Other initiatives include the publication of two series of textbooks (\citealt{WeberWeber1990Mai,WeberWeber1990Mo}) and other educational materials, the foundation of a language academy (\textit{Academia de la lengua}) and an annual Language Day (\textit{Día de la lengua}). At the same time, the use of Rapa Nui in public domains increased, e.g. in politics \citep[204]{Makihara2001Adaptation}. 

In 2011, a new survey was conducted using the same criteria for comprehension and production as in 1997 \citep{CalderónHaoaMakihara2011}. In this survey, the same persons included in the 1997 survey were interviewed again (as far as they could be traced), as well as young people in the age 5–19. The results were as follows: out of 1338 interviewees, 138 (10.3\%) were fully bilingual; another 235 (17.6\%) had a score of 4–7 in comprehension and production; 721 (53.9\%) had virtually no proficiency. This means that proficiency in Rapa Nui had somewhat increased since 1997, despite the fact that the proportion of children from a non-Rapa Nui background was higher than in 1997.

Ultimately, the survival of Rapa Nui will depend on whether speakers succeed in passing the language on to the next generation.

\subsection{Orthography}\label{sec:1.4.4}

Even \is{Orthography|(}though Rapa Nui has a small phoneme inventory (\sectref{sec:2.2}), in three areas an orthographical choice needs to be made between various alternatives: the velar nasal /ŋ/, the glottal plosive \textstyleIPA{/ʔ/} and vowel length. 

In old word lists and lexicons, such as \citet{Roussel1908}, neither the glottal plosive nor vowel length is marked. In later sources, if the glottal plosive\is{Glottal plosive} is marked, it is usually written as an apostrophe, either straight ({\ꞌ}) or curled (‘ or ’); a few sources (\citealt{Fuentes1960}; \citealt{Salas1973}) use the IPA glottal or a similar symbol (\textit{ʔ} \textit{?} \textit{ˀ} ). 

Vowel length is represented in various ways: \textit{aa} (\citealt{Fuentes1960}; \citealt{Salas1973}), \textit{â} (\citealt{Englert1978}, \citealt{ConteOliveros1996}), \textit{á} (\citealt{DuFeu1996}), or \textit{ā} (\citealt{Blixen1972}; \citealt{Chapin1978}). 

The velar nasal has been represented as \textit{ng} (\citealt{Métraux1971} [1940]; \citealt{Blixen1972}; \citealt{ConteOliveros1996}) or \textit{g} (\citealt{Roussel1908}; \citealt{Chapin1978}). Engert was the first to use the \textit{ŋ} symbol, a practice adopted by \citet{Fuentes1960}, \citet{Salas1973} and \citet{DuFeu1996}.

In the \textit{Programa Lengua Rapa Nui} (PLRN, see \sectref{sec:1.6.2} below), the following choices were made:

\begin{itemize}
\item 
In the typewriter era, \textstyleIPA{/ŋ/} was written as \textit{\"{g}} (\citealt{WeberWeber1985}); later this was replaced by \textit{ŋ} (\citealt{WeberWeber2005}). 

\item 
The glottal plosive is represented by a straight apostrophe {\ꞌ}.\footnote{\label{fn:21}To prevent word processors from turning {\ꞌ} into curly brackets (‘ or ’), which take up more space and disrupt the visual unity of the word, a special font was used in the past containing a symbol {\ꞌ}. More recently, the development of Unicode has obviated the need for a special font; the code point UA78C (‘Latin small letter saltillo’) is now available for a symbol \textit{\textup{{\ꞌ}}} which is not confused with an apostrophe by word processors.} Though it is considered an alphabetic character (named \textit{e{\ꞌ}e}), it is not alphabetised separately but according do the following vowel (\textit{a {\ꞌ}a a{\ꞌ}a...}). Also, the glottal does not have a lowercase/capital distinction; if a glottal-initial word is capitalised, it is the vowel after the glottal which gets the capital: \textit{\mbox{{\ꞌ}A{\ꞌ}amu}} ‘story’.

\item 
Vowel length is represented by a macron over the vowel. 

\end{itemize}

These choices are presented and discussed by \citet{WeberWeber1985}; \citet{WeberWeber2005}. 

Another issue concerns word boundaries: should the causative marker \textit{haka} be connected to the root (\textit{hakaoho} ‘to cause to go’) or be treated as a separate word (\textit{haka~oho})? The same question applies to nominalisers like \textit{iŋa}: \textit{vānaŋaiŋa} or \textit{vānaŋa iŋa} ‘speaking’? In most Polynesian languages, these elements are connected to the root, but in the PLRN orthography of Rapa Nui, they are written as separate words.\footnote{\label{fn:22}This not only serves to avoid long words like \textit{hakamāramarama} ‘to cause to be intelligent’ but also prevents potential spacing conflicts: both \textit{haka} and \textit{iŋa} may be separated from the root by certain particles or adverbs (see \REF{ex:3.52} on p.~\pageref{ex:3.52}; \REF{ex:7.99} on p.~\pageref{ex:7.99}).} 

Other grammatical elements are written as separate words as well: determiners, the proper article \textit{a}, prepositions, aspect markers, et cetera. The same is true for phrasal proper nouns, hence \textit{Rapa Nui}, not \textit{Rapanui}; \textit{Haŋa Roa} (town); \textit{Te Moko {\ꞌ}a Raŋi Roa} (protagonist of a legend). On the other hand, certain lexical compounds are written as a single word (\sectref{sec:5.7.2}; \citealt[27]{WeberWeber1985}). 

One more choice which differs from the current practice in most Polynesian languages concerns the orthography of reduplications. In most languages, these are connected to the root; in Rapa Nui, they are separated from the root by a hyphen: \textit{\mbox{riva-riva}} ‘good’, \textit{\mbox{tē-tere}} ‘to run (Pl)’, \textit{\mbox{vānaŋa-naŋa}} ‘to talk repeatedly’. This applies even to lexical reduplications, for which the base does not occur independently in Rapa Nui: \textit{\mbox{nao-nao}} ‘mosquito’, \textit{\mbox{{\ꞌ}ā-{\ꞌ}anu}} ‘to spit’ (\sectref{sec:2.6.3}). 

Over the years, the PLRN orthography has gained acceptance among the Rapa Nui community, including teachers and members of the Rapa Nui Language Academy. It is increasingly seen in publications (e.g. \citealt{GleisnerMontt2014}). In this grammar the same orthography is used, with two exceptions:

\begin{itemize}
\item 
Reduplications are written as single words: instead of the PLRN orthography \textit{\mbox{riva-riva}} ‘good’, this grammar has \textit{rivariva}. Use of the hyphen would create confusion in interlinear glossing and violation of the Leipzig glossing rules,\footnote{\label{fn:23}See http://grammar.ucsd.edu/courses/lign120/leipziggloss.pdf. \citet{Lehmann2004} notices that there is no satisfactory solution for hyphens that do not correspond to morpheme breaks, as in \textit{vis-à-vis}.} as the reduplicant does not have a ‘glossable’ sense separate from the root.

\item 
A few words separated in the PLRN orthography are a single word in this grammar, as they have a non-composite sense. These words start with the causative marker \textit{haka}, followed by a root which does not occur in Rapa Nui or which has a totally unrelated meaning. This affects the following words: \textit{haka{\ꞌ}ou} ‘again’, \textit{hakaroŋo} ‘to listen’, \textit{hakarē} and \textit{hakarere} ‘to leave’, \textit{hakame{\ꞌ}eme{\ꞌ}e} ‘to mock’ and \textit{hakatiu} ‘to watch, wait’. 

\end{itemize}
\is{Orthography|)}

\section{Previous work on the language}\label{sec:1.5}
\subsection{Lexicon}\label{sec:1.5.1}

A good number of early visitors to the island gathered a short word list of the language. The first of these was compiled by Don Francisco Antonio de Agüera during the \ili{Spanish} expedition in 1770 (\citealt{Ross1937}; \citealt{Corney1908}), followed by the German botanist Johann Forster, part of Cook’s expedition in 1774 \citep{Schuhmacher1977}. Father Hippolyte Roussel, who stayed on the island in the late 1860s, compiled a dictionary which was published posthumously.\footnote{\label{fn:24}The \ili{French} original was published in \citet{Roussel1908}, a \ili{Spanish} translation in \citet{Roussel1917}; the latter was republished in \citet{Foerster2013}, with a critical introduction by Bob Weber\ia{Weber, Robert}.} It contains almost 6,000 \ili{Spanish} lemmas with a total of about 1,800 unique Rapa Nui words; unfortunately it is heavily contaminated by \ili{Mangarevan} and \ili{Tahitian} vocabulary \citep{Fischer1992} and therefore far from reliable. Other early vocabularies include \citet{Philippi1873}, \citet{Geiseler1883} (see also \citealt{AyresAyres1995}), \citet{Thomson1889}, \ili{Spanish} translation \citet{Thomson1980}, \citet{Cooke1899} and the short dictionary by \citet{Martínez1913}. The extensive vocabulary in \citet{Churchill1912} is based on Roussel’s dictionary and some of the other vocabularies.

Father Sebastian Englert, who served on the island as parish priest from 1935 until his death in 1969, was the first person to study the language in depth. His dictionary (published in \citealt{Englert1948} and revised in \citealt{Englert1978}) is an invaluable resource for the language as it was spoken in the first half of the 20th century. Another extensive dictionary is \citet{Fuentes1960}. Recent dictionaries include \citet{Fedorova1988}, \citet{ConteOliveros2000}, \citet{HernándezSallés2001} and \citet{HotusChavez2008}. Publications on specific lexical domains include \citet{Gunckel1968} and \citet{RauchIbañez1996} on flora, \citet{PinochetCarte1980} on mollusks, \citet{RandallEgaña1984} on fish, and \citet{BierbachCain1996} on religion.

Over the past years, a number of phrase books for the wider public have been published: \citet{HaoaRapahangoLiller1996}, \citet{HotusTuki2001} and \citet{PaulyHukeAtán2008}.

\subsection{Grammar and sociolinguistics}\label{sec:1.5.2}

The first grammar of Rapa Nui was written by Father Sebastian Englert (included in \citealt{Englert1948}, revised version 1978). It is relatively short but remarkably accurate. Other grammar sketches and concise grammars include \citet{Fuentes1960}, \citet{Chapin1978}, \citet{Munro1978}, \citet{Fedorova1988} (Russian), \citet{ConteOliveros1996} and \citet{Rubino1998}. The latter is a reordering of material from \citet{DuFeu1996}.

The most extensive grammar is \citet{DuFeu1996}, published in the Descriptive Grammars series.\footnote{\label{fn:25}This grammar suffers from some serious flaws, as pointed out in reviews by \citet{Mosel1997} and \citet{WeberWeber1999}. It follows the Descriptive Grammars questionnaire closely rather than presenting material in categories relevant to the language. Moreover, the analysis presented is often unclear, incomplete or incorrect. Some of the examples adduced are unnatural or even incorrect, while the glosses are often inadequate.} 

Several theses, articles and unpublished papers have been written on specific aspects of the language. 

The phonology of Rapa Nui is described in \citet{DuFeu1985}, \citet{GuerraEissmann1993}, \citet{Salas1973} and \citet{WeberWeber1982}. An important landmark in Rapa Nui linguistics was the discovery that Rapa Nui preserves the \is{Proto-Polynesian}PPN glottal plosive, a phoneme which has disappeared in all other \is{Eastern Polynesian}EP languages. The glottal plosive was largely ignored in early descriptions, though Englert’s dictionary registers it in many words. Its phonemic status was brought to light by \citet{Ward1961,Ward1964} and \citet{Blixen1972}. 

The noun phrase is described in \citet{DuFeu1987} in broad outline. Another paper on the noun phrase is \citet{Gordon1977}.

The verb phrase is discussed by \citet{WeberR1988} (\ili{Spanish} version \citealt{WeberR2003}), who offers a thorough analysis of aspect marking. Papers by \citet{Fuller1980} and \citet{Wittenstein1978} deal with the directional markers \textit{mai} and \textit{atu}. \citet{Chapin1974} analyses the use of the postverbal particle \textit{ai}, which is difficult to define in Rapa Nui. 

Grammatical relations in Rapa Nui have been the subject of several studies, especially Agent marking. The supposedly ergative traits of the case system have drawn the attention of several linguists\footnote{\label{fn:26}According to \citet[182]{Mosel1997}, “The most striking feature of Rapanui is that it shows traces of ergativity and hence similarities with West Polynesian languages.”} (\citealt{Alexander1981OL,Alexander1981Minnesota,Alexander1982,FinneyAlexander1998}; \citealt{Finney2000,Finney2001}). \citet{WeberR1988} (\ili{Spanish} version \citealt{WeberR2003}) argues against an ergative analysis.

Other grammatical topics include the following: modality \citep{DuFeu1994}; interrogatives \citep{DuFeu1995}; possession \citep{MulloyRapu1977}; reduplication \citep{Johnston1978}, nominalisation \citep{McAdams1980}, relative clauses \citep{Silva-Corvalán1978}, sentence structure \citep{Smith1980}, and negation \citep{Stenson1981}.

Sociolinguistic aspects (language use and vitality) are discussed by \citet{WeberWeber1984,WeberWeber1990Sobrevivir,WeberWeber1998}, \citet{GómezMacker1977,GómezMacker1979} and \citet{HaoaCardinali2012}. \citet{Makihara1998,Makihara1999,Makihara2001Changing,Makihara2001Adaptation,Makihara2004,Makihara2007,Makihara2009} has studied the use of Rapa Nui and \ili{Spanish} in spoken language. Other studies on the influence of \ili{Spanish} on modern Rapa Nui are included in \citet{StolzBakker2008}. 

\section{About this grammar}\label{sec:1.6}
\subsection{A corpus-based study}\label{sec:1.6.1}

This grammar is based on the analysis of a large corpus of Rapa Nui texts, in addition to observations and discussion/elicitation sessions during the time when I lived on Easter Island (November 2007 – December 2010). In addition to grammatical research, I developed a comprehensive lexical database (hitherto unpublished) based on all available lexical sources and text materials (2008–2010), and carried out an exegetical check of the Rapa Nui translation of the New Testament (2005–2013). The lexical database has served as an additional resource for this grammar, providing data for example in the area of the relation between nouns and verbs (Chapter 3).

A corpus-based approach has several advantages (cf. \citealt[12]{McEneryWilson1996}): it is based on actual, natural data, which are not biased by the linguist’s interest; a large corpus includes data from a wide range of speakers; it enables discourse analysis; the data are verifiable; and finally, a large corpus allows statistical analysis. Moreover, the corpus used for this grammar allows diachronic analysis (see below). Two possible disadvantage of corpus-based research are, that less common phenomena are harder to analyse, as they are rare in texts \citep{Chapin1978}, and that the corpus only shows what is possible, not what is impossible \citep[412]{Biggs1974}. These problems were overcome to a certain degree (a) by using a large corpus (over 500,000 words), and (b) by supplementing corpus analysis with personal observations and elicitation/discussion sessions with a speaker of the language. The corpus is described in \sectref{sec:1.6.2} below.

All texts in the corpus were digitised and converted to the accepted Rapa Nui orthography (\sectref{sec:1.4.4}), with consistent marking of glottal plosives and vowel length. The corpus has been formatted as a Toolbox database, which is linked to the lexical database mentioned above.

The analysis in the following chapters is based on the corpus as a whole. For certain topics (especially aspect marking and clause structure \& case marking), a subcorpus of 29 texts was analysed in more detail (c. 58,000 words; see Footnote \ref{fn:310} on p.~\pageref{fn:310}, Footnote \ref{fn:380} on p.~\pageref{fn:380}).

This grammar also has a comparative component: for many grammatical elements and constructions, the historical derivation and occurrence in related languages is discussed, mostly in footnotes. Comparative data are mainly taken from languages for which a thorough description is available. Data from \is{Eastern Polynesian}Eastern Polynesian languages (\ili{Tahitian}, \ili{Māori}, \ili{Hawaiian}, etc.) are of primary importance; sometimes, reference is made to non-EP \is{Eastern Polynesian}languages (\ili{Samoan}, \ili{Tongan}, \ili{Tuvaluan}, etc.). 

Finally, this grammar has a diachronic dimension. The corpus includes texts from the past 90 or 100 years, a period during which the language has changed considerably; this offers a certain historical perspective which has been taken into account in the analysis.

This grammar is written within the tradition of “basic linguistic theory”, the approach which has become common in descriptive linguistics and which eclectically employs concepts from both traditional linguistics and various theoretical frameworks (\citealt{Dryer2001,Dryer2006}; \citealt{Dixon2010-1,Dixon2010-2,Dixon2012}). 

\subsection{The corpus}\label{sec:1.6.2}

The corpus used as data for this grammar contains two subcorpora: older texts (c. 1910–1940, 124,500 words) and newer texts (c. 1977–2010, 399,000 words). In addition, there is a small collection of texts from the early 1970s (14,500 words). This section gives a description of the different parts of the corpus. The texts in the corpus are referenced with three-letter abbreviations in this grammar; a full listing is given in Appendix B. In this grammar, the term \textit{older Rapa Nui} is used for features only found in pre-1940 texts; features only occurring after 1970 are labelled \textit{modern Rapa Nui}. These labels are used for convenience, without implying that the pre-1940 texts reflect the pre-contact language sometimes referred to “Old Rapa Nui”.

The corpus contains a wide variety of texts. Narrative texts – both spoken and written – are the largest category. Other genres include speeches, conversations, radio interviews, poetry, newspaper articles, procedural texts (e.g. descriptions of traditional customs and techniques) and expository texts (e.g. episodes of the history of the island). The sources are as follows:

\newpage
%\setcounter{listWWviiiNumxcixleveli}{0}
\begin{enumerate}
\item 
In the first decades of the 20th century, a number of Rapa Nui men wrote down a cycle of traditions in what came to be known as Manuscript E (MsE).\footnote{\label{fn:27}MsE is one of six manuscripts (labeled A–F) discovered during the Norwegian archeological expedition in 1955; see \citet{Barthel1965}, \citet{HorleyLópezLabbé2014}. MsE is by far the most extensive of the six; the others mainly contain lists and fragmentary material. \citet[298]{Barthel1978} considers MsE as a copy of an original written before 1914. Recently, a set of photographs of a hitherto unpublished manuscript were discovered; the ms. was written in the same hand as MsE and is now labelled Manuscript H (\citealt{HorleyLópezLabbé2014,HorleyLópezLabbé2015}).} The manuscript was published and translated by Thomas Barthel \citep{Barthel1978} and recently republished in Rapa Nui with a \ili{Spanish} translation \citep{Frontier2008}. 

\item 
In the 1930s, a large number of legends and other stories was collected by Father Sebastian Englert. Many of these were included in \citet{Englert1939Huru1,Englert1939Huru2,Englert1939Tradiciones}; the full compilation was published posthumously with \ili{Spanish} translation in \citet{Englert1980} and with \ili{English} translation in \citet{Englert2001}.\footnote{\label{fn:28}Despite the late date of publishing, most – possibly all – of these texts were collected in the 1930s. Many were published (sometimes with minor variations) in \citet{Englert1939Huru1,Englert1939Huru2,Englert1939Tradiciones}; all of these were written in 1936. Of the stories not included in these publications, the majority were transmitted by the same narrators mentioned in \citet{Englert1939Huru1,Englert1939Huru2,Englert1939Tradiciones}: Mateo Veriveri, Juan Tepano and Arturo Teao. Other stories were told by the wife and sons of the catechist Nicolás Ure Potahi (1851-1927). 

Many of the texts in \citet{Englert1939Huru2} are not included in \citet{Englert1980}; these are not included in the corpus, as I only discovered this publication in November 2015.} A few other texts were included in \citet{Englert1948}.

\item 
The Swiss ethnologist Alfred Métraux, who visited the island in 1934–1935, included a large number of stories in his ethnography (\citealt{Métraux1971}): some in Rapa Nui with translation, others in translation only. For the latter, the original text was preserved in his notebooks (\citealt{Métraux1935}), which I transcribed and added to the corpus.\footnote{\label{fn:29}According to \citet{Fischer2008Correspondence,Fischer2009}, the original notebooks were lost, though a photocopy was preserved; Davletshin\ia{Davletshin, Albert} (p.c. 2016) pointed out to me that the originals still exist, and are in the Thomas Barthel Nachlass in Tübingen.} 

\item 
In the 1970s, Rapa Nui texts were published by Fritz Felbermayer (\citealt{Felbermayer1971,Felbermayer1973,Felbermayer1978}) and Olaf Blixen (\citealt{Blixen1973,Blixen1974}). 

\item 
In 1977, SIL linguists Robert and Nancy Weber \ia{Weber, Robert}\ia{Weber, Nancy}started the \textit{Programa Lengua Rapa Nui} (PLRN), a collaboration between the Pontifica Universidad Católica de Valparaíso and SIL International, which aimed at language preservation, education and documentation. They started collecting texts, recording and transcribing stories by notable storytellers, commissioning written texts, transcribing radio emissions, et cetera. Many new texts were written and published during two writers’ workshops in 1984 and 1985; the texts from the first workshop were republished in \citet{PatéTukiTukiTepano1986}. Other texts were added to the corpus during the preparation of a series of school books (\citealt{WeberWeber1990Mai,WeberWeber1990Mo}). Over time, many more texts were added, for example texts by Rapa Nui authors for which the Webers acted as linguistic consultants, such as \citet{CuadrosHucke2008} and \citet{PakaratiTuki2010}. Details about the texts are listed in Appendix B.

\item 
Finally, the largest single text in Rapa Nui is the translation of the New Testament, as well as portions of the Old Testament. This translation (as yet unpublished) was made by a number of Rapa Nui speakers, with exegetical and linguistic advice from Robert \& Nancy Weber\ia{Weber, Robert}\ia{Weber, Nancy}. In 2006–2012, the New Testament was meticulously checked for naturalness by a team of Rapa Nui speakers. In this grammar, the Bible translation is used as a secondary resource, especially to illustrate phenomena for which few or no clear examples are available otherwise.

\end{enumerate}

Not included in the corpus are a number of other Rapa Nui texts:

\begin{itemize}
\item 
The oldest surviving Rapa Nui text is the catechism translated by Father Hippolyte Roussel in 1868 \citep{Roussel1995}. Roussel, who had worked in the Tuamotus and on Mangareva, used a language heavily influenced by the language varieties spoken in those islands.

\item 
Songs, chants and recitations have been handed down from the past (see e.g. \citealt{Campbell1970}; \citealt{Barthel1960}); these are often syntactically fragmentary and difficult to interpret. See \citet{Fischer1994} for an interpretation of an old chant.

\item 
A distinctive corpus is formed by the \textit{kōhau rongorongo}, a number of wooden tablets inscribed by a script unique to Rapa Nui. Several attempts at interpretation have been made (\citealt{Barthel1958}; \citealt{Fischer1997}), but the script has not been definitively deciphered so far \citep{Davletshin2012}.\footnote{\label{fn:30}Several scholars have suggested that \textit{rongorongo} was developed after the Rapa Nui witnessed writing in 1770, when \ili{Spanish} explorers drew up a deed of cession in which the island was handed over to the \ili{Spanish} crown (\citealt{Emory1972}; \citealt{Fischer1996Reply,Fischer1997}).}

\item 
\citet{GleisnerMontt2014} include a number of stories and descriptive texts (c.~36,000 words); this corpus came to my attention when this grammar was nearly finished. Another recent collection is \citet{TepanoKaituoe2015}, a bilingual edition of 75 notebooks of Rapa Nui text by Uka Tepano Kaituoe (1929–2014).

\end{itemize}
\subsection{Organisation of this grammar}\label{sec:1.6.3}

This grammar is organised as follows.

Chapter 2 deals with the phonology of Rapa Nui. The following topics are discussed in turn: phonemes (with special attention to the glottal plosive), syllable and word structure, stress, intonation, phonological processes, and reduplication. 

Chapter 3 deals with nouns and verbs. In many analyses of Polynesian language, the existence of lexical nouns and verbs is denied; rather, the two categories are defined syntactically (“a noun is any word preceded by a determiner”). Arguments are given to show that this approach obscures various differences between nouns and verbs, and that the distinction between both should be maintained. A classification of nouns is proposed, as well as a classification of verbs. Adjectives (a subclass of verbs) and locatives (a subclass of nouns) are discussed.

Other word classes are discussed in Chapter 4: pronouns, numerals, quantifiers, adverbs, demonstratives and prepositions. Not treated in this chapter are words exclusively occurring as particles in the noun and/or verb phrase, such as determiners and aspect markers.

The noun phrase potentially contains a large number of elements; these are discussed in Chapter 5. Two determiners are discussed extensively: the article \textit{te} (which marks referentiality\is{Referentiality}, not definiteness or specificity) and the predicate marker \textit{he}. Possessive relationships are also discussed in this chapter; possessives can be marked with \textit{o} or \textit{a}, depending on the relationship between possessor and possessee. Another area discussed here is compounding, and the difference between compounding and modification.

Chapter 6 deals with possessive constructions. Possessors occur as modifiers in the noun phrase, as predicates of nominal clauses, and in various other constructions. A common feature in Polynesian is the distinction between \textit{o}{}- and \textit{a-}marked possessors; this is discussed in detail.

Chapter 7 discusses the verb phrase. A major topic is the use of aspect markers, a set of five preverbal particles. Other common verb phrase particles include directionals and postverbal demonstratives. Finally, a section is devoted to serial verbs, a construction not found in other Polynesian languages.

Some Polynesian languages are accusative, others are (partly) ergative; at first sight, Rapa Nui does not seem to fit either pattern. In Chapter 8 on the verbal clause, I show that Rapa Nui is accusative, and that case marking of Agent and Patient is governed by an interplay of syntactic, semantic and pragmatic factors. A passive construction is shown to exist, even though it is less obvious than in related languages. Other topics in this chapter include non-canonical case marking, constituent order, comitative constructions and causatives.

Nonverbal clauses are common in Rapa Nui; these are discussed in Chapter 9. Two major types are classifying and identifying clauses, respectively. Existential clauses can be verbal or non-verbal. The chapter closes with an unusual feature in Rapa Nui (compared to other Polynesian languages): the emergence of copula verbs in classifying clauses.

Chapter 10 deals with mood (imperatives, interrogatives, exclamatives) and negation. 

Constructions involving multiple clauses are discussed in Chapter 11: coordination, relative clauses, clausal complements and adverbial clauses.

Appendix A provides illustrative interlinear texts. Appendix B lists the texts in the corpus used as data for this grammar.

This grammar does not contain a separate section on discourse issues. 
Discourse-based analysis has been applied to a number of phenomena in different sections of the grammar instead: pre- and postnominal demonstratives (\sectref{sec:4.6}), aspect marking (\sectref{sec:7.2}), directional particles (\sectref{sec:7.5}), subject and object marking (\sectref{sec:8.3}–\ref{sec:8.4}), non-canonical subject marking and non-standard constituent orders (\sectref{sec:8.6}).
