\addchap{List of abbreviations}


(Abbreviations for text sources are listed in Appendix B.)

\section*{Grammatical categories}

\begin{tabbing}
xxxxxxxxxx \= xxxxxxxxxxxxxxxxxxxxxxxxxxxxxxxxxxxx \kill
%\begin{tabularx}{\textwidth}{L{15mm}X}
%\lsptoprule
* \> ungrammatical; reconstructed protoform\\
(*XX) \> ungrammatical if XX is included\\
*(XX) \> ungrammatical if XX is omitted, i.e. XX is obligatory\\
{\textgreater} \> becomes\\
Ø \> zero\\
/.../ \> phonemic transcription\\
{[...]} \> in Chapter 2: phonetic transcription; elsewhere: constituent\\
1, 2, 3 \> first, second, third person\\
\textsc{a} \> (in possessives:) \textit{a}{}-class possession\\
A \> (verb argument:) the most agentive argument of a transitive verb\\
A/M \> aspect/mood marker\\
{\scshape acc} \> accusative (\textit{i})\\
{\scshape ag} \> agentive (\textit{e})\\
{\scshape ana} \> general anaphor (\textit{ira})\\
{\scshape art} \> article (\textit{te})\\
{\scshape ben} \> benefactive\\
{\scshape C} \> consonant\\
{\scshape caus} \> {causative (\textit{haka})}\\
{\scshape cntg} \> contiguous (\textit{ka})\\
{\scshape coll} \> collective (\textit{kuā}/\textit{koā})\\
{\scshape com} \> {comitative (\textit{koia})}\\
{\scshape cont} \> continuous (\textit{{\ꞌ}ā/{\ꞌ}ana})\\
{\scshape cq} \> content question (\textit{hē})\\
{\scshape deic} \> deictic\\
{\scshape dem} \> demonstrative\\
{\scshape dist} \> distal (far from speaker)\\
{\scshape DO} \> direct object\\
{\scshape du} \> dual\\
{\scshape dub} \> dubitative (\textit{hō})\\
{\scshape emph} \> emphatic (\textit{rō})\\
%\lspbottomrule
%\end{tabularx}

%\begin{tabularx}{\textwidth}{L{15mm}X}
%\lsptoprule
{\scshape excl} \> exclusive\\
{\scshape exh} \> {exhortative (\textit{e})}\\
{\scshape fut} \> future\\
{\scshape hort} \> hortative (\textit{ki})\\
{\scshape ident} \> identity (\textit{{\ꞌ}ā/{\ꞌ}ana})\\
{\scshape imm} \> immediate (\textit{{\ꞌ}ī})\\
{\scshape imp} \> {imperative (\textit{ka})}\\
{\scshape incl} \> inclusive\\
{\scshape ins} \> instrumental (\textit{hai})\\
{\scshape intens} \> intensifier (\textit{rā})\\
{\scshape ipfv} \> imperfective (\textit{e})\\
{\scshape irr} \> irrealis (\textit{ana})\\
LOC \> locative; locational\\
{\scshape med} \> medial distance\\
N \> noun\\
{\scshape neg} \> negation (\textit{{\ꞌ}ina})\\
{\scshape conneg} \> constituent negation (\textit{ta{\ꞌ}e})\\
{\scshape nmlz} \> nominaliser\\
{\scshape ntr} \> neutral aspect (\textit{he})\\
{\scshape num} \> numeral marker\\
{\scshape num.pers} \> personal numeral marker (\textit{hoko})\\
\textsc{o} \> (in possessives:) \textit{o}{}-class possession\\
O \> (verb argument:) the least agentive argument of a transitive verb\\
{\scshape past} \> past\\
{\scshape pfv} \> perfective (\textit{i})\\
{\scshape pl} \> plural\\
PND \> {postnominal demonstrative}\\
{\scshape poss} \> possessive\\
{\scshape pq} \> polar question (\textit{hoki})\\
{\scshape pred} \> predicate marker (\textit{he})\\
{\scshape prf} \> perfect (\textit{ko}/\textit{ku})\\
{\scshape prom} \> prominence marker (\textit{ko})\\
{\scshape prop} \> {proper article (\textit{a})}\\
{\scshape prox} \> proximal (near speaker)\\
PVD  \> {postverbal demonstrative}\\
{\scshape pvp} \> postverbal particle (\textit{ai})\\
QTF \> quantifier\\
{\scshape red} \> reduplication\\
S \> the single argument of an intransitive verb\\
{\scshape sg} \> singular\\
%\lspbottomrule
%\end{tabularx}
%\begin{tabularx}{\textwidth}{L{15mm}X}
%\lsptoprule
{\scshape subs} \> subsequent (\textit{{\ꞌ}ai})\\
SVC \> serial verb construction\\
{\scshape V} \> verb; vowel\\
{\scshape V\textsubscript{Nom}} \> {nominalised verb}\\
{\scshape voc} \> {vocative (\textit{e})}\\
%\lspbottomrule
%\end{tabularx}
\end{tabbing} 
\section*{Language groups and protolanguages}

%\begin{tabularx}{\textwidth}{L{15mm}X}
%\lsptoprule
\begin{tabbing}
xxxxxxxxxx \= xxxxxxxxxxxxxxxxxxxxx \kill
PAN \> Proto-Austronesian\\
POc \> Proto-Oceanic\\
PEO \> Proto-Eastern Oceanic\\
PPN \> Proto-Polynesian\\
(P)NP \> (Proto) Nuclear Polynesian\\
(P)SO \> (Proto) Samoic-Outlier\\
(P)EP \> (Proto) Eastern Polynesian\\
(P)CE \> (Proto) Central-Eastern Polynesian\\
(P)Ta \> (Proto) Tahitic\\
(P)Mq \> (Proto) Marquesic\\
%\lspbottomrule
%\end{tabularx}
\end{tabbing}
