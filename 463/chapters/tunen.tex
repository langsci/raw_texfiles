\documentclass[output=paper,colorlinks,citecolor=brown
% showindex
]{langscibook}
\ChapterDOI{10.5281/zenodo.14833606}
\author{Elisabeth J.\ Kerr\orcid{0000-0002-0329-269X}\affiliation{Ghent University}}
\title{The expression of information structure in Tunen}
\abstract{This chapter provides a descriptive overview of the expression of information structure in Tunen (A44, Cameroon), based primarily on original field data. The overall conclusion is that, in contrast to other Bantu languages, grammatical roles have a greater influence on syntax in Tunen than discourse roles do. A key finding is that the canonical word order S-Aux-O-V-X is compatible with a variety of information-structural contexts, rather than being an information structure-conditioned variant of an unmarked VO pattern. Non-subject information focus may be left unmarked via the canonical word order, while subject focus must be expressed via a basic cleft, showing a subject/non-subject asymmetry. The marker \textit{á}, considered in previous work to be a monoclausal contrast marker, is argued on the basis of new data to be the specificational/identificational copula found within biclausal cleft constructions, with a subject/non-subject distinction visible between the use of a basic cleft and a reverse pseudocleft structure. In terms of topic expression, it is shown that both topical and non-topical material may occupy the subject position within the canonical word order, with no obligatory marking of topicality and no requirement to mark contrastive topics. The chapter ends with discussion of functional equivalents of passive constructions and consideration of how referent expression varies over discourse.}



\IfFileExists{../localcommands.tex}{
   \addbibresource{../localbibliography.bib}
   \usepackage{langsci-optional}
\usepackage{langsci-gb4e}
\usepackage{langsci-lgr}

\usepackage{listings}
\lstset{basicstyle=\ttfamily,tabsize=2,breaklines=true}

%added by author
% \usepackage{tipa}
\usepackage{multirow}
\graphicspath{{figures/}}
\usepackage{langsci-branding}

   
\newcommand{\sent}{\enumsentence}
\newcommand{\sents}{\eenumsentence}
\let\citeasnoun\citet

\renewcommand{\lsCoverTitleFont}[1]{\sffamily\addfontfeatures{Scale=MatchUppercase}\fontsize{44pt}{16mm}\selectfont #1}
  
   %% hyphenation points for line breaks
%% Normally, automatic hyphenation in LaTeX is very good
%% If a word is mis-hyphenated, add it to this file
%%
%% add information to TeX file before \begin{document} with:
%% %% hyphenation points for line breaks
%% Normally, automatic hyphenation in LaTeX is very good
%% If a word is mis-hyphenated, add it to this file
%%
%% add information to TeX file before \begin{document} with:
%% %% hyphenation points for line breaks
%% Normally, automatic hyphenation in LaTeX is very good
%% If a word is mis-hyphenated, add it to this file
%%
%% add information to TeX file before \begin{document} with:
%% \include{localhyphenation}
\hyphenation{
affri-ca-te
affri-ca-tes
an-no-tated
com-ple-ments
com-po-si-tio-na-li-ty
non-com-po-si-tio-na-li-ty
Gon-zá-lez
out-side
Ri-chárd
se-man-tics
STREU-SLE
Tie-de-mann
}
\hyphenation{
affri-ca-te
affri-ca-tes
an-no-tated
com-ple-ments
com-po-si-tio-na-li-ty
non-com-po-si-tio-na-li-ty
Gon-zá-lez
out-side
Ri-chárd
se-man-tics
STREU-SLE
Tie-de-mann
}
\hyphenation{
affri-ca-te
affri-ca-tes
an-no-tated
com-ple-ments
com-po-si-tio-na-li-ty
non-com-po-si-tio-na-li-ty
Gon-zá-lez
out-side
Ri-chárd
se-man-tics
STREU-SLE
Tie-de-mann
}
   \boolfalse{bookcompile}
   \togglepaper[2]%%chapternumber
}{}

\begin{document}
\maketitle
\label{ch:2}


\section{Introduction}
Tunen (or ``Nen'', ISO 639-3 code [tvu], Guthrie code A44; \citealt{Maho2003, Maho2009}) is a Narrow Bantu language spoken by somewhere over 35,000 people (likely around 70,000–100,000; \citealp[33--34]{KerrFut}), predominantly in central Cameroon \citep{Mous2003, Gordon2005}. The language is spoken in the \textit{Centre} region of Cameroon around the town Ndikiniméki and in an area stretching southwestwards into the \textit{Littoral} region, reaching Douala \citep{Dugast1955, Dugast1971, Mous2003}. It is therefore spoken at the far North-Western region of Narrow Bantu, bordering non-Bantu Bantoid languages of the Grassfields subgroup, and as such has various unusual properties compared to (other) Narrow Bantu languages. As well as phonological differences including the loss of final vowels and tone lowering in utterance-final position and an ATR vowel harmony system \citep{Dugast1971, vanLeynseele1977,vanderHulstEtAl1986, Bancel1991,Mous1986, Mous2003, Boyd2015}, Tunen's morphosyntax is more analytical than the canonical Bantu language, with the subject marker and tense marker (and the negative marker or directionality affixes, if present) separate from the verb \citep{Dugast1971, Mous1997, Mous2003, KongneWelaze2010}. Most relevantly for understanding the information structure of the language, Tunen is known to be highly unusual for a Bantu language in having SOV as its canonical word order \citep{Dugast1971, Bearth2003, Mous1997, Mous2003, Mous2005, Mous2014, Kerr2024, KerrFut}. This chapter starts by investigating the information-structural contexts in  which this SOV word order is found, and then turns to consider how else information-structural notions are expressed in the language.  
 
The data presented in this chapter, unless indicated otherwise, are drawn from fieldwork conducted by the author under the Bantu Syntax and Information Structure (BaSIS) research project at Leiden University. The fieldwork was conducted over a total period of approximately 7 months, split across Mar–Jun 2019 (3.5 months) and Oct 2021–Feb 2022 (3.25 months), in Ndikiniméki and Yaoundé, Cameroon, with some follow-up fieldwork conducted remotely via WhatsApp. There are four dialects of Tunen: Tɔbɔ́áŋɛ, Hiliŋ, Fombo, and Ndogtuna. Tɔbɔ́áŋɛ is the standard dialect of Ndikiniméki town and the main dialect on which previous work on Tunen has been based. Most consultants for this study spoke the Tɔbɔ́áŋɛ dialect, but one consultant, EO, speaks Hiliŋ, and another consultant, DM speaks Fombo (though has lived in the Tɔbɔ́áŋɛ-speaking area for a long time). The dialects are mutually intelligible \citep[8]{Dugast1971}, with minor differences in pronunciation and lexicon. In this chapter, consultant initials and the unique ID of the form in the corresponding Dative database are provided in square brackets  alongside each example. For example, ``[PM 316]'' is a form provided by consultant PM with UID 316 in the database. The data consist of a mixture of elicitation based on a controlled discourse context and natural speech of different subgenres (dialogues, storytelling, instructional monologues). Further information on the data collection and information on how to access the archival corpus \citep{KerrDatabase} is provided in \citet[Chapter 3]{KerrFut}.
 
 This chapter is structured as follows. \sectref{secwordorder} presents evidence for S-Aux-O-V-X as the canonical word order and shows the different information-structural contexts in which this word order can appear. \sectref{secnoncanonwordorder} discusses departures from this canonical word order, including the VO word order patterns discussed in \citet{Mous1997, Mous2003}. \sectref{secclefts} looks at how clefts and the marker \textit{á} are used for focus expression, \sectref{sectopics} covers left-peripheral topic expression and contrastive topics, \sectref{secpassive} considers how expression of the agent can be avoided using the \textit{-átɔ} participle and \textit{bá-} functional passive constructions, \sectref{secreference} considers how referent expression varies over discourse. \sectref{seccomparison} reflects briefly on how Tunen compares to other Bantu languages in its expression of information structure, and \sectref{secconc} concludes. The terminology used for information structure is explained in the general introduction to this volume \parencite{chapters/intro}, which also provides an introduction to the conceptual background of information structure and the methodology used in the BaSIS project.

\section{Canonical word order S-Aux-O-V-X} \label{secwordorder}
\subsection{S-Aux-O-V-X as the canonical word order}
Tunen is widely known to have SOV as its canonical word order \citep{Dugast1971, Bearth2003, Mous1997, Mous2003, Mous2005, Mous2014, Kerr2024, KerrFut}, specifically the subtype S-Aux-O-V-X, where Aux refers to an auxiliary element (in Tunen, the TAM complex) and X to other elements, such as locative adjuncts. This word order is highly unusual for a Bantu language, where SVO is the expected word order \citep{Bearth2003}. Moreover, S-Aux-O-V-X is unusual in Niger-Congo and is also rare cross-linguistically, found in some languages of West and Central Africa (see e.g. \citealt{DryerGensler2005, GenslerGüldemann2003, Güldemann2008}). S-Aux-O-V(-X) surface word order (at least in some TAM contexts) is reported for example for Mande languages \citep{Claudi1993, Creissels2005, Nikitina2011, SandeEtAl2019}, Kru languages \citep{Gensler1994, SandeEtAl2019}, and the Senufo branch of Gur \citep{Gensler1994}, with only Tunen and its close neighbour Nyokon (Guthrie no. A45) and nearby Ewondo/Eton (A72) as Benue-Congo languages reported to show (a degree of) OV word order \citep{Mous2005,Mous2014,Kerr2024,KerrFut}. 

One question is the extent to which a language's basic word order varies dependent on information structure. In this vein, it has been claimed that the word order of some Bantu languages is better captured by making reference to discourse roles than using grammatical role-oriented labels like ``SVO'' (see e.g. \citealt{Morimoto2000, Morimoto2006, Good2010, Yoneda2011, KerrEtAl2023}). This means that a more appropriate characterisation of a language's word order may be in terms of discourse roles like topic and focus. In this section I show that Tunen's word order is largely influenced by grammatical role (i.e. subjecthood vs. objecthood; see \citealt{KerrEtAl2023} for more discussion on the nature of grammatical role-oriented versus discourse role-oriented word order in Bantu). I show that S-Aux-O-V-X is a pragmatically neutral word order in Tunen that is compatible with various information-structural interpretations. Tunen therefore contrasts with the other Bantu languages in this volume, which show SVO canonical word order that is conditioned more strongly by information-structural considerations than by grammatical role. 

\subsubsection{Thetics}
One typical diagnostic for a language's canonical word order is the word order found in thetic sentences. Thetics are defined as being all-new and thus differ from categorical sentences which show a topic\hyp comment distinction \parencite{Sasse1987, Sasse1996, Lambrecht1994, chapters/intro}. In Tunen, S-Aux-O-V-X word order is used for thetics, as illustrated in (\ref{ex:elephantthetic}--\ref{criminalattack}) below.\footnote{Elicitation sessions were conducted in French and so the French translations are provided as they are what was agreed with the consultant. English translations are my own additions. The first transcription line gives a broad phonetic transcription, while the second line shows the underlying form with morpheme segmentation. The phonetic transcription is similar to the official orthography \citep{SatreEtAl2008} but differs in transcribing surface tone, noting tonal downstep, and in not transcribing non-pronounced final vowels (which can be identified from the underlying representation line); see \citet[Chapter 4]{KerrFut} for further discussion. A list of glossing abbreviations can be found at the end of this chapter.}
\largerpage[2]

%%EAX
\ea
%%JUDGEMENT
%%LABEL
\label{ex:elephantthetic}
%%CONTEXT
(Context: You are at the riverside outside the village and see an elephant, which very rarely occurs, so run to tell the others.) \\
%%LINE1
\glll
{\db}mɛ nɔ́	misəku	siəkin! \\
%%LINE2
/mɛ nɔ́	misəku	siəkinə/ \\
%%LINE3
{\db}\SM{}.1\SG{} \PST{}1{}	9.elephant	see.\DUR{} \\
%%TRANS1
\glt
`Je viens de voir un éléphant !'\\  `I just saw an elephant!' \jambox*{[PM 316] }
%%TRANS2
%%EXEND

 \z

%%EAX
\ea
%%JUDGEMENT
%%LABEL
\label{churchthetic}
%%CONTEXT
(Context: Your friend asks what happened at church.) \\
%%LINE1
\glll
{\db}mɔtát a ná imbə́nu yɛ fəkin nɛ́ Yə́səs ɔ Yɛrúsalɛm nɔŋɔnak. \\
%%LINE2
/mɔ-táta a ná ɛ-mbə́nu yɛ fəkinə nɛ́ Yə́səsu ɔ Yɛrúsalɛmɛ nɔŋɔnɔ-aka/ \\
%%LINE3
{\db}1-pastor 1\SM{} \PST{}2{} 9-news 9.\ASS{} 5.entrance 5.\ASS{} Jesus \PREP{} Jerusalem tell-\DUR{} \\
%%TRANS1
\glt
`Le pasteur a raconté des nouvelles de l'entrée de Jésus à Jerusalem'. \\  `The pastor told the news of Jesus' entrance into Jerusalem.' \jambox*{[DM 166] }
%%TRANS2
%%EXEND
%\a
%
%\ Context: Your interlocutor wants to know what happened at church so you describe it. \\ bɛndɔ báná matala sɔ́mbák. \\
% /ba-ndɔ bá-ná matala sɔ́mb-aka/ \\
% 2-person 2\SM{}-\PST{}2{} 6.palm.fronds cut-\DUR{}  \\
%\glt Des gens ont coupé des palmes.' \\ `People cut palm fronds.' \jambox*{[DM 165] }
%
%\a
\z

%%EAX
\ea
%%JUDGEMENT
%%LABEL
\label{criminalattack}
%%CONTEXT
(Context: Imagine someone came in the room right now during our field session, and you are explaining to someone else later what happened.) \\ 
%%LINE1
\glll
{\db}tɔ́ ndɔ bá u miímə́ yi isukúlú, mutʃə́ŋə́ a nɔ́ nda bəəsú kasɔn! \\
%%LINE2
/tɔ \textsuperscript{H}ndɔ bá ɔ miímə́ yɛ ɛ-sukúlú mɔ-tʃə́ŋə́ a nɔ́ nda bəəsú kasɔna/ \\
%%LINE3
{\db}\SM{}.1\PL{} \PRS{} be \PREP{} 9.room \ASS{}.{}9 7-school 1-criminal 1\SM{} \PST{}1{} \VEN{} \PRO{}.1\PL{} attack \\
%%TRANS1
\glt
`Nous étions dans la salle de classe, un bandit est venu nous agresser !' \\ `We were in the classroom, a criminal came in to attack us!' \jambox*{[EE+EB 1847] }
%%TRANS2
%%EXEND

\z

The consistent S-Aux-O-V(-X) word order across different types of thetics shows that this word order is compatible with an all-new context where no element is topical or in focus. One point of complexity here, however, is that conversational participants can be argued to be always retrievable topics via world knowledge (see e.g. \citealt{Givón1983, Erteschik-Shir2007} for relevant discussion). However, examples like \xref{churchthetic} with third-person subjects provide evidence that non-topical subjects can also be in the initial position, and example \xref{criminalattack} further shows that OV order is found even when the object is the first person. In any case, regardless of the position of the subject, the word order for Tunen thetics is markedly different from the other Bantu languages in this volume in that the object precedes the verb in Tunen (S\textit{OV}). Versions with VO order were judged as ungrammatical \xref{criminaltheticVO}.

%%EAX
\ea
%%JUDGEMENT
[*]{
%%LABEL
\label{criminaltheticVO}
%%CONTEXT
%%LINE1
\glll
{\db}tɔ́ ndɔ bá u miímə́ yi isukúlú, mutʃə́ŋə́ a nɔ́ nda \textbf{kasɔn} \textbf{bəəsu}! \\
%%LINE2
/tɔ \textsuperscript{H}ndɔ bá ɔ miímə́ yɛ ɛ-sukúlú mɔ-tʃə́ŋə́ a nɔ́ nda kasɔna bəəsú/ \\
%%LINE3
{\db}\SM{}.1\PL{} \PRS{} be \PREP{} 9.room \ASS{}.{}9 7-school 1-criminal 1\SM{} \PST{}1 \VEN{} attack \PRO{}.1\PL{} \\
%%TRANS1
\glt
int. `Nous étions dans la salle de classe, un bandit est venu nous agresser !' \\ int. `We were in the classroom, a criminal came in to attack us!' \jambox*{[EE+EB 2239] }
%%TRANS2
}
%%EXEND

\z

Note also that while subject inversion constructions have been reported as a means of detopicalising subjects in a thetic context in other Bantu languages \citep{MartenvanderWal2014}, and indeed are found to express thetics in all other languages in this volume except Teke-Kukuya, such inversion constructions are completely ungrammatical in Tunen (regardless of the position of temporal and locative adjuncts), as shown in \xref{chiefinversion} and \xref{cowinversion}.\footnote{In this chapter, the following standard conventions are used for presentation of judgements: *(X) indicates that the sentence would be ungrammatical if X were omitted (in other words, X is obligatory); (*X) indicates that the sentence would be ungrammatical if X were included; $\{$X$\}$ ... $\{$X$\}$ means that the judgement holds with X in either position in the sentence (not that it occurs in both at the same time).}\footnote{The non-inverted equivalents (with the nominal subject directly before the subject marker) are grammatical ([JO 2623, 2627]; [JO 2600-1]).}

%%EAX
\ea
%%JUDGEMENT
[*]{
%%LABEL
\label{chiefinversion}
%%CONTEXT
%%LINE1
\glll
{\db}$\{$naánɛkɔla$\}$ a ka nyɔkɔ $\{$naánɛkɔla$\}$ \textbf{kíŋgə} \\
%%LINE2
/$\{$naánɛkɔla$\}$ a ka nyɔ-aka $\{$naánɛkɔla$\}$ kíŋgə/ \\
%%LINE3
{\db}yesterday 1\SM{} \PST{}3{} work-\DUR{} yesterday 1.chief \\
%%TRANS1
\glt
int. `Le chef a travaillé hier.' \\ int. `The chief worked yesterday.' \jambox*{[JO 2629--30] }
%%TRANS2
}
%%EXEND

\z

%%EAX
\ea
%%JUDGEMENT
[*]{
%%LABEL
\label{cowinversion}
%%CONTEXT
%%LINE1
\glll
{\db}bɛ́ ká fámáka \textbf{bɛfɔŋɔ} naánɛkɔla ɔ ɛtɔbɔtɔ́bɔ́. \\
%%LINE2
/bɛ́ ka fámá-aka bɛ-fɔŋɔ naánɛkɔla ɔ ɛ-tɔbɔtɔ́bɔ́/ \\
%%LINE3
{\db}8\SM{} \PST{}3{} arrive-\DUR{} 8-cow yesterday \PREP{} 7-field \\
%%TRANS1
\glt
int. `Les vaches sont apparues dans le champ hier.' \\ int. `The cows appeared in the field yesterday.' \jambox*{[JO 2608] }
%%TRANS2
}
%%EXEND

\z

This unavailability of subject inversion matches other languages in the area such as Basaá (Guthrie no. A43, Cameroon), which is shown by \citet{HamlaouiMakasso2015} to not allow inversion constructions (see also \citealt{Hamlaoui2022}). The unavailability of inversion constructions in Tunen means that the canonical word order S-Aux-O-V-X is the only means of expressing theticity in Tunen.

\subsubsection{VP focus}
Word order when the entire verb phrase is in focus (VP focus) is a second criterion that can be invoked to determine a language's canonical word order. In Tunen, VP focus questions can felicitously be answered with S-Aux-O-V-X word order patterns \xref{MariaVPvarnish}, providing further evidence for S-Aux-O-V-X as the canonical word order.\footnote{Here and throughout, the scope of focus is indicated in the translation lines by square brackets with subscript \FOC{} for ``focus".}\largerpage[2]

%%EAX
\ea
%%JUDGEMENT
%%LABEL
\label{MariaVPvarnish}
%%CONTEXT
(What did Maria do?) \\
%%LINE1
\glll
{\db}Malíá a ná bilə́liə fɔfɔ́kíə́ ɔmbambala na makat. \\
%%LINE2
/Malíá a ná bɛ-lə́liə fɔfɔ́kíə́ ɔ-mbambala na ma-kátá/ \\
%%LINE3
{\db}1.Maria 1\SM{} \PST{}2{} 8-varnish anoint.\DUR{} 3-wall with 6-hand \\
%%TRANS1
\glt
`Maria a [oint le vernis sur le mur avec la main]\textsubscript{\FOC{}}.' \\ `Maria [applied the varnish to the wall by hand]\textsubscript{\FOC{}}.' \jambox*{[JO 2516] }
%%TRANS2
%%EXEND

\z

Now we have seen evidence from thetics and VP focus for S-Aux-O-V-X as the canonical word order, we can investigate the other information-structural contexts in which it occurs.

\subsection{S-Aux-O-V-X for object focus}

S-Aux-O-V-X word order is compatible with term focus on the theme object, as seen in the  answers to the object question in \xref{carrotobj} below, illustrated for two different consultants.

\ea \label{carrotobj} (What is the man holding? (+ hand-drawn picture stimulus)) \\
%%EAX
\ea
%%JUDGEMENT
%%LABEL
%%CONTEXT
%%LINE1
\glll
{\db}mɔndɔ a ná \textbf{kalɔ́tɔ} ití. \\
%%LINE2
/mɔ-ndɔ a ná kalɔ́tɔ itíə́/ \\
%%LINE3
{\db}1-person 1\SM{} \PST{}2{} 9.carrot hold \\
%%TRANS1
\glt
`L'homme tient [une carotte]\textsubscript{\FOC{}}.' \\ `The man is holding [a carrot]\textsubscript{\FOC{}}.' \jambox*{[JO 1107] }
%%TRANS2
%%EXEND

%%EAX
\ex
%%JUDGEMENT
%%LABEL
%%CONTEXT
%%LINE1
\glll
{\db}mɔndɔ a ná \textbf{kalɔ́tɔ} itíə́ ɔ mɔkat.  \\
%%LINE2
/mɔ-ndɔ a ná kalɔ́tɔ itíə́ ɔ mɔ-kátá/ \\
%%LINE3
{\db}1-person 1\SM{} \PST{}2{} 9.carrot hold \PREP{} 3-hand \\
%%TRANS1
\glt
`L'homme tient [une carotte]\textsubscript{\FOC{}} à la main.' \\ `The man is holding [a carrot]\textsubscript{\FOC{}} in his hand.' \jambox*{[PM 1264] }
%%TRANS2
%%EXEND

\z
\z

Both SVOX and SVXO orders were judged as ungrammatical \xxref{carrotobjsvox}{carrotobjsvxo}.

\ea  (What is the man holding? (+ hand-drawn picture stimulus)) \\ 
%%EAX
\ea
%%JUDGEMENT
[*]{
%%LABEL
\label{carrotobjsvox}
%%CONTEXT
%%LINE1
\glll
{\db}a ná itíə́ \textbf{kalɔ́tɔ} ɔ mɔkata. \\
%%LINE2
/a ná itíə́ kalɔ́tɔ ɔ mɔ-kátá/ \\
%%LINE3
{\db}1\SM{} \PST{}2{} hold 9.carrot \PREP{} 3-hand \\
%%TRANS1
\glt
int. `Il tient [une carrotte]\textsubscript{\FOC{}} à la main.' \\ int. `He is holding [a carrot]\textsubscript{\FOC{}} in his hand.' \jambox*{[JO 1626] }
%%TRANS2
}
%%EXEND

%%EAX
\ex
%%JUDGEMENT
[*]{
%%LABEL
\label{carrotobjsvxo}
%%CONTEXT
%%LINE1
\glll
{\db}a ná itíə́ ɔ mɔkata \textbf{kalɔ́tɔ}. \\
%%LINE2
/a ná itíə́ ɔ mɔ-kátá kalɔ́tɔ/ \\
%%LINE3
{\db}1\SM{} \PST{}2{} hold \PREP{} 3-hand 9.carrot \\
%%TRANS1
\glt
int. `Il tient [une carrotte]\textsubscript{\FOC{}} à la main.' \\ int. He is holding [a carrot]\textsubscript{\FOC{}} in his hand.' \jambox*{[JO 1625] }
%%TRANS2
}
%%EXEND

\z
\z

These examples provide evidence for two things. Firstly, when taken together with what we have seen for other discourse contexts, we see that S-Aux-O-V-X is a pragmatically neutral word order in Tunen: it is possible for an all-new thetic context, VP focus, and narrow information focus on the theme object. Secondly, the examples show that information focus does not need to be morphosyntactically marked in Tunen, at least for theme objects, as the object in \xref{carrotobj} is  left in-situ without any special marking.\largerpage[2]

The Q-A pairs seen above show that S-Aux-O-V-X word order is found for information focus on the theme object. This word order is not possible in a different focus context where the object is marked by the focus-sensitive particles \textit{ata} `even' \xref{eveneatpotatoinsitu} or \textit{ɔ́maná} `only' \xref{onlyset1.a}. In such exclusive and exhaustive focus cases, the object must be ex-situ, typically fronted \xref{eveneatpotatoexsitu} or clefted \xref{onlyset1.b}, and marginally postposed \xref{eveneatpotatoright}.\footnote{Note that the preference for fronted over postposed differs from the previous description in \citet{Mous1997}, who only reports the postposed strategy.}

\ea
\label{evenpotatoset}
%%EAX
\ea
%%JUDGEMENT
[*]{
%%LABEL
\label{eveneatpotatoinsitu} 
%%CONTEXT
%%LINE1
\glll
{\db}a ná nɛ́áká mɔŋɛ́ŋ, a ná \textbf{ata} \textbf{bɛŋgwɛtɛ} nɛak. \\
%%LINE2
/a ná nɛ́á-aka mɔŋɛ́ŋa a ná ata bɛ-ŋgwɛtɛ nɛ́á-aka/ \\
%%LINE3
{\db}1\SM{} \PST{}2{} eat-\DUR{} much 1\SM{} \PST{}2{} even 8-potato eat-\DUR{} \\
%%TRANS1
\glt
int. `Il a beaucoup mangé, il a même mangé [des patates]\textsubscript{\FOC{}}.' \\ int. `He ate a lot, he even ate [potatoes]\textsubscript{\FOC{}}.' \jambox*{[PM (+DM) 2265] }
%%TRANS2
}
%%EXEND

%%EAX
\ex
%%JUDGEMENT
[]{
%%LABEL
\label{eveneatpotatoexsitu}
%%CONTEXT
%%LINE1
\glll
{\db}a ná nɛ́áká mɔŋɛ́ŋ, \textbf{ata} \textbf{bɛŋgwɛtɛ} a ná nɛak. \\
%%LINE2
/a ná nɛ́á-aka mɔŋɛ́ŋa ata bɛ-ŋgwɛtɛ a ná nɛ́á-aka/ \\
%%LINE3
{\db}1\SM{} \PST{}2{} eat-\DUR{} much  even 8-potato  1\SM{} \PST{}2{} eat-\DUR{} \\
%%TRANS1
\glt
`Il a beaucoup mangé, il a même mangé [des patates]\textsubscript{\FOC{}}.' \\ `He ate a lot, he even ate [potatoes]\textsubscript{\FOC{}}.' \jambox*{[PM (+DM) 2264] }
%%TRANS2
}
%%EXEND

%%EAX
\ex
%%JUDGEMENT
[\textsuperscript{?}]{
%%LABEL
\label{eveneatpotatoright}
%%CONTEXT
%%LINE1
\glll
{\db}a ná nɛ́áká mɔŋɛ́ŋ, a ná nɛ́áká \textbf{ata} \textbf{bɛŋgwɛtɛ}. \\
%%LINE2
/a ná nɛ́á-aka mɔŋɛ́ŋa a ná nɛ́á-aka ata bɛ-ŋgwɛtɛ/ \\
%%LINE3
{\db}1\SM{} \PST{}2{} eat-\DUR{} much   1\SM{} \PST{}2{} eat-\DUR{} even 8-potato \\
%%TRANS1
\glt
int. `Il a beaucoup mangé, il a même mangé [des patates]\textsubscript{\FOC{}}.' \\ int. `He ate a lot, he even ate [potatoes]\textsubscript{\FOC{}}.' \jambox*{[PM (+DM) 2263] }
%%TRANS2
}
%%EXEND

\z
\z

\ea \label{onlyset1}
%%EAX
\ea
%%JUDGEMENT
[*]{
%%LABEL
\label{onlyset1.a}
%%CONTEXT
%%LINE1
\glll
{\db}a ná mɔná \textbf{ɔ́maná} \textbf{imítə́} túmbi. \\
%%LINE2
/a ná mɔ-ná ɔ́maná ɛ-mítə́ túmbiə/ \\
%%LINE3
{\db}1\SM{} \PST{}2{} 1-child only 9-calabash return \\
%%TRANS1
\glt
int. `C'est seulement [la calebasse]\textsubscript{\FOC{}} qu'elle a donné à l'enfant.' \\ int. `She only gave [the calabash]\textsubscript{\FOC{}} to the child.' \jambox*{[JO 1593] }
%%TRANS2
}
%%EXEND

%%EAX
\ex
%%JUDGEMENT
[]{
%%LABEL
\label{onlyset1.b}
%%CONTEXT
%%LINE1
\glll
{\db}\textbf{ɔ́maná} \textbf{imítə́} á a ná mɔná túmbi. \\
%%LINE2
ɔ́maná ɛ-mítə́ á a ná mɔ-ná túmbiə/ \\
%%LINE3
{\db}only 9-calabash \COP{} 1\SM{} \PST{}2{} 1-child return \\
%%TRANS1
\glt
`C'est seulement [la calebasse]\textsubscript{\FOC{}} qu'elle a donné à l'enfant.' \\ `She only gave [the calabash]\textsubscript{\FOC{}} to the child.' \jambox*{[JO 1592] }
%%TRANS2
}
%%EXEND

\z
\z


Note also that there is a distinction questions and declaratives, with questions formed ex-situ rather than in the S-Aux-O-V-X order. This is illustrated in \xref{whatseeQA} below, where \textit{yatɛ́} `what' is fronted in the question, while the answer is provided with the canonical S-Aux-O-V-X order.\footnote{The numeral modifier of the object appears in a discontinuous position after the verb, despite also falling within the scope of focus. I will come back to the interpretation of such discontinuous modifiers in \sectref{secdiscon}; see \citet[Chapter 7]{KerrFut} for more detailed discussion.}

\ea \label{whatseeQA}
%%EAX
\begin{xlist}
\exi{Q:}
%%JUDGEMENT
%%LABEL
%%CONTEXT
%%LINE1
\glll
{\db}\textbf{yatɛ́} ɔ́ ndɔ́ sin? \\
%%LINE2
/yatɛ́ ɔ \textsuperscript{H}ndɔ	sinə/ \\
%%LINE3
{\db}what	\SM{}.2\SG{} \PRS{}	see \\
%%TRANS1
\glt
`Qu'est-ce que tu vois?' \\ `What do you see?' \jambox*{[EO 396] }
%%TRANS2
\end{xlist}
%%EXEND

%%EAX
\begin{xlist}
\exi{A:}
%%JUDGEMENT
%%LABEL
%%CONTEXT
%%LINE1
\glll
{\db}mɛ́ ndɔ	\textbf{tunoní}	sinə tɔ́lál. \\
%%LINE2
/mɛ \textsuperscript{H}ndɔ tɔ-noní sinə tɔ́-lálɔ́/ \\
%%LINE3
{\db}\SM{}.1\SG{} \PRS{}	13-bird	see	13-three \\
%%TRANS1
\glt
`Je vois [trois oiseaux]\textsubscript{\FOC{}}.' \\ `I see [three birds]\textsubscript{\FOC{}}.' \jambox*{[EO 397]}
%%TRANS2
\end{xlist}
%%EXEND
\z

Example \xref{whatcookQQa} shows that an in-situ object question is not possible.\footnote{An exception to the ban on in-situ object questions is an echo question context, in which case the object question word can be left in-situ, as in the elicited example \xref{echoQ} below. Note however that a clefted ex-situ question was provided as the first answer here, and short movement of the question word to the left periphery of the embedded clause is also possible.
%%EAX
\ea
%%JUDGEMENT
%%LABEL
\label{echoQ}
%%CONTEXT
%%LINE1
\glll
{\db}Elísabɛ́tɛ	á ndɔ	laa	ásɛ	a ná \textbf{yatɛ́}	ɔ́ndɔ́kɔ	eé? \\
%%LINE2
/Elísabɛtɛ	a \textsuperscript{H}ndɔ	láá	a-sɛ́á	a ná yatɛ́	ɔ́ndɔ́-aka	eé/ \\
%%LINE3
{\db}1.Elisabeth 1\SM{} \PRS{} say 1\SM{}-say 1\SM{} \PST{}2{} what buy-\DUR{} \Q{} \\
%%TRANS1
\glt
`Elisabeth dit qu'elle a acheté quoi ?' \\ `Elisabeth said that she bought what?' \jambox*{[EB+JO 2782]; \citep[132]{KerrFut}}
%%TRANS2
%%EXEND
\z
}

\ea \label{whatcookQQ}
%%EAX
\ea
%%JUDGEMENT
[*]{
%%LABEL
\label{whatcookQQa}
%%CONTEXT
%%LINE1
\glll
{\db}ɔ́ ndɔ \textbf{yatɛ́} talɛ́áka	nɛɔfɛ́nɛ eé? \\
%%LINE2
/ɔ \textsuperscript{H}ndɔ yatɛ́ talɛ́á-aka nɛɔfɛ́nɛ eé/ \\
%%LINE3
{\db}\SM{}.2\SG{} \PRS{} what cook-\DUR{} today \Q{} \\
%%TRANS1
\glt
int. `Qu'est-ce que tu vas cuisiner aujourd'hui ?' \\ int. `What will you cook today?' \jambox*{[JO 1600] }
%%TRANS2
}
%%EXEND


%%EAX
\ex
%%JUDGEMENT
[]{
%%LABEL
%%CONTEXT
%%LINE1
\glll
{\db}\textbf{yatɛ́} ɔ́ ndɔ talɛ́áka nɛɔfɛ́nɛ eé? \\
%%LINE2
/yatɛ́	ɔ \textsuperscript{H}ndɔ talɛ́á-aka nɛɔfɛ́nɛ eé/ \\
%%LINE3
{\db}what \SM{}.2\SG{} \PRS{}  cook-\DUR{} today \Q{} \\
%%TRANS1
\glt
`Qu'est-ce que tu vas cuisiner aujourd'hui ?' \\ `What will you cook today?' \jambox*{[JO 1601] }
%%TRANS2
}
%%EXEND

\z
\z

We therefore see that S-Aux-O-V-X is compatible with information focus on the object in declarative sentences, while exclusive and exhaustive focus on the object -- as evidenced by association with the focus-sensitive particles \textit{ata} `even' and \textit{ɔ́maná} `only' -- require an ex-situ word order, and foci in questions are ex-situ. \sectref{secclefts} will discuss the use of reverse pseudoclefts as a means of expressing more contrastive term focus on the object. I turn now to testing for other discourse contexts in which the canonical word order S-Aux-O-V-X can be used.

\subsection{No S-Aux-O-V-X for subject focus}
In contrast to focused declarative objects, focused declarative subjects cannot be left in-situ and must be focused via a cleft \xref{subjfoccleftnocleft}; the \textit{á} \COP{} cannot be omitted.\footnote{The infelicity judgement of example \xref{subjfoccleftnocleft} without the \textit{á} \COP{} also holds when the subject marker is expressed in the non-relative form (with a low tone).}

%%EAX
\ea
%%JUDGEMENT
%%LABEL
\label{subjfoccleftnocleft}
%%CONTEXT
(Which politician died?)  \\
%%LINE1
\glll
\textbf{$\#$(á)} Píɛlə á ná wə. \\
%%LINE2
/á Piɛlə á ná wə́/ \\
%%LINE3
{\db}\COP{} 1.Pierre 1\SM{}.\REL{} \PST{}2{} die \\
%%TRANS1
\glt
`C'est [Pierre]\textsubscript{\FOC{}} qui est mort.' \\  `[Pierre]\textsubscript{\FOC{}} died.' \jambox*{[EO 271] }
%%TRANS2
%%EXEND

\z

In other words, the canonical S-Aux-O-V-X word order cannot be used to express subject focus, even in a non-contrastive information focus context. This matches subject/non-subject focus asymmetries reported in other language families, where subject focus is obligatorily marked while non-subject focus may be expressed using the canonical word order (see e.g. \citealt{FiedlerEtAl2010} for an overview).
 
For subject foci associated with the focus-sensitive particles \textit{ata} `even' \xref{evenpeterate} and \textit{ɔ́maná} `only' \xref{onlyemanuel}, evidence for the subject being ex-situ is given by the obligatory presence of the copula \textit{á}, which will be discussed further in \sectref{secclefts}.

\ea \label{evenpeterate}
%%EAX
\ea
%%JUDGEMENT
%%LABEL
%%CONTEXT
%%LINE1
\glll
{\db}\textbf{ata} \textbf{*(á)} \textbf{Bitə} a ná bɛŋgɛtɛ nɛak. \\
%%LINE2
/ata á Bitə a ná bɛ-ŋgɛtɛ nɛ́á-aka/ \\
%%LINE3
{\db}even \COP{} 1.Peter 1\SM{} \PST{}2{} 8-potato eat-\DUR{} \\
%%TRANS1
\glt
`Même [Peter]\textsubscript{\FOC{}} a mangé des patates.' \\ `Even [Peter]\textsubscript{\FOC{}} ate potatoes.' \jambox*{[PM 2260] }
%%TRANS2
%%EXEND

\pagebreak
%%EAX
\ex
%%JUDGEMENT
%%LABEL
\label{onlyemanuel}
%%CONTEXT
(Context: You are a school teacher marking the exams for a class, and are shocked by how badly the students did.) \\ 
%%LINE1
\glll
{\db}\textbf{ɔ́maná} \textbf{Ɛmánúwɛlɛ} \textbf{na} \textbf{Natanayɛ́lɛ} \textbf{á} bá ná tɔ́mbá ɔ nɛkɔsɔna! \\
%%LINE2
/ɔ́maná Ɛmánúɛlɛ na Natanayɛ́lɛ á bá ná tɔ́mba ɔ nɛ-kɔsɔna/ \\
%%LINE3
{\db}only 1.Emmanuel and 1.Nathaniel \COP{} 2\SM{} \PST{}2{} pass \PREP{} 5-exam \\
%%TRANS1
\glt
`Seulement [Emmanuel et Nathaniel]\textsubscript{\FOC{}} ont réussi à l'examen !' \\ `Only [Emmanuel and Nathaniel]\textsubscript{\FOC{}} passed the exam!' \jambox*{[JO 527] }
%%TRANS2
%%EXEND

\z
\z

For subject questions, it is harder to tell whether the canonical word order is used, as fronting of the subject question word results in the same linear order S-Aux-O-V-X and is therefore string-vacuous. As will be discussed further in \sectref{secclefts}, there is some evidence that subject questions are formed as clefts, as suggested for example by the relative form (visible from the high tone) of the subject marker in example \xref{whospoon} below.

%%EAX
\ea
%%JUDGEMENT
%%LABEL
\label{whospoon}
%%CONTEXT
%%LINE1
\glll
{\db}\textbf{ɛ́yánɛ́}	\textbf{á} lɛa	na hioso? \\
%%LINE2
/ɛ́yánɛ́	á lɛ́á na	hɛ-ɔ́sɔ/ \\
%%LINE3
{\db}who	1\SM{}.\REL{} be	with	19-spoon \\
%%TRANS1
\glt
`Qui a une cuillère ?' \\`Who has a spoon?' \jambox*{[EO 1433] }
%%TRANS2
%%EXEND


\z

We will come back to the expression of subject focus via clefting in \sectref{secclefts}. Note that the inability for subjects to be focused in a non-clefted construction illustrates that Tunen does not have a dedicated focus position in which different grammatical roles can be focused, unlike the languages in this volume which have an immediate before verb (IBV; \cite{chapters/teke}), immediate after verb (IAV; e.g. Makhuwa; \cite{chapters/makhuwa}, Rukiga; \cite{chapters/rukiga}) and sentence\hyp final focus position (Kirundi; \cite{chapters/kirundi}) in which subjects and non-subjects alike may be focused. This difference in availability of focus positions is discussed further in \citet{KerrEtAl2023}.

\subsection{S-Aux-O-V-X for non-argument focus}
Non-arguments in Tunen pattern with objects in being able to be focused in-situ in the canonical S-Aux-O-V-X order \REF[A]{wherebuilthouseA}. They may also be focused via a cleft, and time adverbials are generally more flexible in their position than other adverbials/adjuncts \citep[114--115]{KerrFut}. Like with what we saw above for objects, non-argument questions are generally formed by fronting/clefting \REF[Q1]{wherebuilthouseQ} rather than having the question word in the canonical position, although the in-situ options are accepted more so than for objects \REF[Q2]{wherebuilthouseQ2}.

\ea \label{wherebuilthouse}
%%EAX
\begin{xlist}[Q2:]
\exi{Q1:}
%%JUDGEMENT
%%LABEL
\label{wherebuilthouseQ}
%%CONTEXT
%%LINE1
\glll
{\db}\textbf{hə́níə́}	á \ds{}ná	yayɛ́á	miímə	lúmə́kə́ eé? \\
%%LINE2
/hə́níə́	á \textsuperscript{L}ná	yayɛ́á	miímə	lúmə́-aka eé/ \\
%%LINE3
{\db}where	1\SM{}.\REL{} \PST{}3.\REL{}	3.\POSS{}.\PRO{}.1{}	3.house	build-\DUR{}  \Q{} \\
%%TRANS1
\glt
`Où est-qu'il a construit sa maison ?' \\ `Where did he build his house?' \jambox*{[JO 1115] }
%%TRANS2
%%EXEND

%\todo{Check spacing /align}
%%EAX
\exi{Q2:}
%%JUDGEMENT
%%LABEL
\label{wherebuilthouseQ2}
%%CONTEXT
%%LINE1
\glll
{\db}a ka yáyɛ́á miímə lúmə́kə́ \textbf{hə́ní(ə́)} eé? \\
%%LINE2
/a ka yáyɛ́á miímə lúmə́-aka hə́níə́ eé/ \\
%%LINE3
{\db}1\SM{} \PST{}3	3.\POSS{}.\PRO{}.1{}	3.house	build-\DUR{} where \Q{} \\
%%TRANS1
\glt
`Où est-qu'il a construit sa maison ?' \\ `Where did he build his house?' \jambox*{[JO 1118] }
%%TRANS2
%%EXEND


%%EAX
\exi{A:}
%%JUDGEMENT
%%LABEL
\label{wherebuilthouseA}
%%CONTEXT
%%LINE1
\glll
{\db}a ka yayɛ́á miímə lúmə́kə́ \textbf{ɔ} \textbf{iNdíki}. \\
%%LINE2
/a ka yayɛ́á miímə lúmə́-aka ɔ iNdíki/ \\
%%LINE3
{\db}1\SM{} \PST{}3{} 3.\POSS{}.\PRO{}.1{} 3.house build-\DUR{} \PREP{} Ndiki \\
%%TRANS1
\glt
`Il a construit sa maison [à Ndiki]\textsubscript{\FOC{}}.' \\ He built his house [in Ndiki]\textsubscript{\FOC{}}.' \jambox*{[JO, 1121] }
%%TRANS2
\end{xlist}
%%EXEND
\z

Note that example \REF[A]{wherebuilthouseA} also shows that a given object (here, an object already mentioned in the question) may be preverbal, providing further evidence for S-Aux-O-V-X as a pragmatically neutral word order, as the object position can be filled by focused or given objects alike.

Like we saw above for objects, non-arguments modified by the focus-sensitive particle \textit{ɔ́maná} `only' are commonly fronted \xref{onlyyaoundeleft}. However, it is also possible to have what appears to be the S-Aux-O-V-X word order with the X element modified by \textit{ɔ́maná} `only', although note that this is linearly equivalent to an alternative analysis in which the focused phrase is postposed \xref{onlyyaounderight}.\largerpage

\ea (Context: Someone incorrectly says you have been to both Yaoundé and Kribi.) 
%%EAX
\ea
%%JUDGEMENT
%%LABEL
\label{onlyyaoundeleft}
%%CONTEXT
%%LINE1
\glll
{\db}bɔ́ɔ,	\textbf{ɔ́maná}	\textbf{ɔ}	\textbf{Yəhənd}	á	mɛ́ ná ká hul. \\
%%LINE2
/bɔ́ɔ	ɔ́maná	ɔ	Yəhəndə	á	mɛ́ ná ka  húlə́/ \\
%%LINE3
{\db}no	only	\PREP{}	Yaounde	\COP{}	\SM{}.1\SG{}.\REL{} \PST{}2{} \AND{} return \\
%%TRANS1
\glt
`Non, ce n'est que à [Yaoundé]\textsubscript{\FOC{}} que je suis parti(e).' \\ `No, I only went to [Yaoundé]\textsubscript{\FOC{}}.' \jambox*{[JO 1607] }
%%TRANS2
%%EXEND

%%EAX
\ex
%%JUDGEMENT
%%LABEL
\label{onlyyaounderight}
%%CONTEXT
%%LINE1
\glll
{\db}bɔ́ɔ,	mɛ ná \ds{}ká	hulə	\textbf{ɔ́maná}	\textbf{ɔ} \textbf{Yəhənd}. \\
%%LINE2
/bɔ́ɔ	mɛ ná ka	húlə́	ɔ́maná	ɔ Yəhəndə/ \\
%%LINE3
{\db}no \SM{}.1\SG{} \PST{}2{} \AND{}	return	only	\PREP{} Yaounde \\
%%TRANS1
\glt
`Non, ce n'est que à [Yaoundé]\textsubscript{\FOC{}} que je suis parti(e).' \\ `No, I only went to [Yaoundé]\textsubscript{\FOC{}}.' \jambox*{[JO 1608] } 
%%TRANS2
%%EXEND

\z
\z

In \sectref{subsec:revpseudo} below, we will see that non-arguments also pattern with objects with respect to cleft formation and fragment answers.

\subsection{S-Aux-O-V-X for predicate-centred focus}
Finally, the canonical word order S-Aux-O-V-X can be used for predicate-centred focus (PCF), as seen already for VP focus in \xref{MariaVPvarnish} above and as further illustrated below for polarity focus \xref{seesheepQ} and state-of-affairs (verb) focus \xref{contrastiveobject1}, \xref{eveneatmanioc}.

%%EAX
\ea
%%JUDGEMENT
%%LABEL
\label{seesheepQ}
%%CONTEXT
Context: `Do you see the sheep?' (polarity focus) \\
%%LINE1
\glll
{\db}mɛ́ nd(ɔ)	ɛndɔ́mbá	sin. \\
%%LINE2
/mɛ \textsuperscript{H}ndɔ	ɛ-ndɔ́mbá	sinə/ \\
%%LINE3
{\db}\SM{}.1\SG{} \PRS{}	4-sheep	see \\
%%TRANS1
\glt
`Je vois les moutons.' \\ `I see the sheep.' \jambox*{[EO 695] }
%%TRANS2
%%EXEND


\z

%%EAX
\ea
%%JUDGEMENT
%%LABEL
\label{contrastiveobject1}
%%CONTEXT
(What did he do with the beans and the plantains? (SoA focus)) \\ (What happened? (thetic)) \\
%%LINE1
\glll
{\db}a ka makɔnd͡ʒɛ nɛáka. a ná bilikó lu.  \\
%%LINE2
/a ka ma-kɔnd͡ʒɛ nɛ́á-aka a ná bɛ-likó luə/ \\
%%LINE3
{\db}1\SM{} \PST{}3{} 6-plantain eat-\DUR{} 1\SM{} \PST{}2{} 8-bean sell \\
%%TRANS1
\glt
`Il a [mangé]\textsubscript{\FOC{}}  les plantains. Il a [vendu]\textsubscript{\FOC{}}  les haricots.' \\ `He [ate]\textsubscript{\FOC{}} the plantains. He [sold]\textsubscript{\FOC{}}  the beans.' \jambox*{[JO 908] }
%%TRANS2
%%EXEND


\z
%%EAX
\ea
%%JUDGEMENT
%%LABEL
\label{eveneatmanioc}
%%CONTEXT
(Context: Johannes dislikes the taste of cassava so normally cooks it but doesn't eat it. This time, he buys it, he cooks it, and he even eats it (which is surprising). (SoA focus)) \\
%%LINE1
\glll
{\db}Yɔhánɛsɛ	a ná	ɛsasɔma	nɛ́áká	sɛ́ɛb. \\
%%LINE2
/Yɔhánɛsɛ	a ná	ɛ-sasɔma	nɛ́á-aka	sɛ́ɛbɛ/ \\
%%LINE3
{\db}1.Johannes 1\SM{} \PST{}2{} 7-cassava eat-\DUR{} even \\
%%TRANS1
\glt
`Johannes a même [mangé]\textsubscript{\FOC{}} du manioc.' \\`Johannes even [ate]\textsubscript{\FOC{}} the cassava.' \jambox*{[PM 2282] }
%%TRANS2
%%EXEND


\z

We see in \xref{eveneatmanioc} that the canonical word order can be used for PCF for exclusive focus as well as information focus, as the exclusive focus-sensitive particle \textit{sɛ́ɛbɛ} `only' may modify the predicate. Note here that the exclusive focus-sensitive particle \textit{ata} `even' seen previously is only found for term focus and cannot be used in a PCF construction \xref{noatapcf},\footnote{Although a possible example of \textit{ata} `even' with PCF was found in the following example from a story, where \textit{ata} modifies a clause after a left-dislocated topic:

%%EAX
\ea
%%JUDGEMENT
%%LABEL
%%CONTEXT
%%LINE1
\glll
{\db}«mɔkand͡ʒakand͡ʒ \textbf{ata} á ndɔ hɔ́ taka a báka bá lɛ́ wɛ́ɛya ɔkɛ́n.» \\
%%LINE2
/mɔ-kand͡ʒakand͡ʒa ata a \textsuperscript{H}ndɔ hɔ́ taka a bá-aka bá lɛ́ wɛ́ɛya ɔ-kɛ́na/ \\
%%LINE3
{\db}1-liar even 1\SM{} \PRS{} talk truth 1\SM{} be-\DUR{} 2\SM{} \NEG{} \PRO{}.\EMPH{}.1{} \INF{}-believe \\
%%TRANS1
\glt
`« Un menteur, même quand il dit la vérité, on ne lui croit pas. »' \\ `"Even if a liar is telling the truth, nobody believes them."' \jambox*{[JO 2039] }
%%TRANS2
%%EXEND

\z} with \textit{sɛ́ɛbɛ} (literally translatable as `self') used instead. Unlike with \textit{ata}, \textit{sɛ́ɛbɛ} follows the focused constituent and does not require it to be ex-situ.

\ea\label{noatapcf}
%%EAX
\ea
%%JUDGEMENT
[*]{
%%LABEL
%%CONTEXT
%%LINE1
\glll
{\db}Yɔhánɛsɛ	a ná	ɛsasɔma	\textbf{ata}	\textbf{nɛ́áká}. \\
%%LINE2
/Yɔhánɛsɛ	a ná	ɛ-sasɔma	ata	nɛ́á-aka/ \\
%%LINE3
{\db}1.Johannes	1\SM{} \PST{}2{}	7-cassava	even	eat-\DUR{} \\
%%TRANS1
\glt
int. `Johannes a même [mangé]\textsubscript{\FOC{}} du manioc.' \\ int. `Johannes even [ate]\textsubscript{\FOC{}} the cassava.' \jambox*{[PM 2284] }
%%TRANS2
}
%%EXEND

%%EAX
\ex
%%JUDGEMENT
[*]{
%%LABEL
%%CONTEXT
%%LINE1
\glll
{\db}\textbf{ata}	\textbf{nɛ́áká}	Yɔhánɛsɛ	a ná	ɛsasɔma. \\
%%LINE2
/ata	nɛ́á-aka	Yɔhánɛsɛ	a ná	ɛ-sasɔma/ \\
%%LINE3
{\db}even	eat-\DUR{}	1.Johannes	1\SM{} \PST{}2{}	7-cassava \\
%%TRANS1
\glt
int. `Johannes a même [mangé]\textsubscript{\FOC{}} du manioc.' \\ int. `Johannes even [ate]\textsubscript{\FOC{}} the cassava.' \jambox*{[PM 2285] }
%%TRANS2
}
%%EXEND

\z
\z

Note also that, unlike the other Bantu languages in this volume, Tunen does not have a predicate doubling construction for the expression of PCF (i.e. a construction in which an infinitival/nominal form of the verb appears together with a finite verb form; see e.g. \citealt{GüldemannEtAl2015} and references therein).\largerpage

So far then, we have seen that S-Aux-O-V-X is the pragmatically neutral word order in Tunen, and therefore can be taken as the canonical word order. This canonical word order is found for thetics, VP focus, information focus on the theme object, non-argument focus, and predicate-centred focus (PCF). It is not possible to use S-Aux-O-V-X word order for focus on the subject, which must instead be expressed by a cleft. More contrastive term focus is generally expressed ex-situ, as shown by association with the focus-sensitive particles \textit{ata} `even' and \textit{ɔ́maná} `only' (while more contrastive PCF can be left in-situ).


\subsection{S-Aux-O-V-X in double object constructions}
Having seen that the canonical word order for Tunen transitives is S-Aux-O-V-X, let us now consider ditransitives. Here, the O slot of S-Aux-O-V-X can be filled by multiple objects, specifically in S-O\textsubscript{\textsc{goal}}-O\textsubscript{\textsc{theme}}-V order, as previously noted by \citet{Mous1997, Mous2003}. I show in this section that this S-O\textsubscript{\textsc{goal}}-O\textsubscript{\textsc{theme}}-V order is consistent across different information-structural contexts, meaning that the order of objects is not conditioned by information structure. In other words, when both objects are preverbal, the goal (i.e. recipient/beneficiary) object always precedes the theme object \xxref{ditrans2}{ditransgourdtheme}. The reverse order S-O\textsubscript{\textsc{theme}}-O\textsubscript{\textsc{goal}}-V is not grammatical \xref{ditransbad}.

\ea  (`Who is the woman giving a gourd to?' (+ photo from BaSIS stimuli)) \\
%%EAX
\ea
%%JUDGEMENT
[]{
%%LABEL
\label{ditrans2}
%%CONTEXT
%%LINE1
\glll
{\db}a nɔ́ \textbf{ɔsɔ́kɔ́} \textbf{hɛtɛ́tɛ́} indi.  \\
%%LINE2
/a nɔ́ ɔsɔ́kɔ́ hɛ-tɛ́tɛ́ índíə́/ \\
%%LINE3
{\db}1\SM{} \PST{}1{} 1.other 19-gourd give \\
%%TRANS1
\glt
`Elle donne une gourde à [l'autre]\textsubscript{\FOC{}}.' \\ `She gives a gourd to [the other (woman)]\textsubscript{\FOC{}}.' \jambox*{[PM 1541] }
%%TRANS2
}
%%EXEND


%%EAX
\ex
%%JUDGEMENT
[*]{
%%LABEL
\label{ditransbad}
%%CONTEXT
%%LINE1
\glll
{\db}a nɔ́ \textbf{hɛtɛ́tɛ́} \textbf{ɔsɔ́kɔ́} indi.  \\
%%LINE2
/a nɔ́ hɛ-tɛ́tɛ́ ɔsɔ́kɔ́ índíə́/ \\
%%LINE3
{\db}1\SM{} \PST{}1{} 19-gourd 1.other give \\
%%TRANS1
\glt
int. `Elle donne une gourde à [l'autre]\textsubscript{\FOC{}}.' \\ int. `She gives a gourd to [the other (woman)]\textsubscript{\FOC{}}.' \jambox*{[PM 1542] }
%%TRANS2
}
%%EXEND

\z
\z

\ea
%%EAX
\begin{xlist}
\exi{Q:}
%%JUDGEMENT
%%LABEL
\label{ditransgourdq}
%%CONTEXT
%%LINE1
\glll
{\db}yatɛ́ (á) muəndú á ndɔ mɔná túmbi? \\
%%LINE2
/yatɛ́ á mɔ-əndú a \textsuperscript{H}ndɔ mɔ-ná túmbiə/ \\
%%LINE3
{\db}what \COP{} 1-woman 1\SM{} \PRS{} 1-child return \\
%%TRANS1
\glt
`Qu'est-ce que la femme remet à l'enfant ?' \\ `What is the woman returning to the child?' \jambox*{[JO 1588] }
%%TRANS2
\end{xlist}
%%EXEND


%%EAX
\begin{xlist}
\exi{A:}
%%JUDGEMENT
%%LABEL
\label{ditransgourdtheme}
%%CONTEXT
%%LINE1
\glll
{\db}muəndú á ndɔ \textbf{mɔná} \textbf{imítə́} túmbi. \\
%%LINE2
/mɔ-əndú a \textsuperscript{H}ndɔ mɔ-ná ɛ-mítə́ túmbiə/ \\
%%LINE3
{\db}1-woman 1\SM{} \PRS{} 1-child 9-calabash return \\
%%TRANS1
\glt
`La femme remet [le calebasse]\textsubscript{\FOC{}} à l'enfant.' \\ `The woman returns [the calabash]\textsubscript{\FOC{}} to the child.' \jambox*{[JO 1587] }
%%TRANS2
\end{xlist}
%%EXEND
\z

The same O\textsubscript{\textsc{goal}}-O\textsubscript{\textsc{theme}} order is also found in questions \xref{whofoodcatq} and imperatives \xref{givecalchild}.

%%EAX
\ea
%%JUDGEMENT
%%LABEL
\label{whofoodcatq}
%%CONTEXT
%%LINE1
\glll
{\db}ɛ́yánɛ́ á ná \textbf{himuísimuísí} \textbf{híɔfɔ́} indi? \\
%%LINE2
/ɛ́yánɛ́ á ná hɛ-muísimuísi hɛ-ɔfɔ́ indíə́/\\
%%LINE3
{\db}who 1\SM{}.\REL{} \PST{}2{} 19-cat 19-fish give \\
%%TRANS1
\glt
`Qui a donné du poisson au chat ?' \\ `Who gave fish to the cat?' \jambox*{[EO 278] }
%%TRANS2
%%EXEND

\z

%%EAX
\ea
%%JUDGEMENT
%%LABEL
\label{givecalchild}
%%CONTEXT
%%LINE1
\glll
{\db}índíə́ \textbf{mɔná} \textbf{imit}! \\
%%LINE2
/índíə́ mɔ-ná ɛ-mítə́/ \\
%%LINE3
{\db}give 1-child 9-calabash \\
%%TRANS1
\glt
`Donne la calebasse à l'enfant !' \\ `Give the calabash to the child!' \jambox*{[JO 1594] }
%%TRANS2
%%EXEND

\z

These data provide evidence for the  S-O\textsubscript{\textsc{goal}}-O\textsubscript{\textsc{theme}}-V-X as the canonical word order in Tunen, meaning that the `O' of S-Aux-O-V-X covers both the theme and the goal (i.e. recipient/beneficiary) object. Note that this is in contrast to certain West African languages described as having S-Aux-O-V-X basic word order that only permit a single preverbal object (\citealt{GenslerGüldemann2003, Creissels2005}; see \citealt[Chapter 6]{KerrFut} for further discussion).

Note also that this canonical order is not the only word order found for double object constructions. Firstly, \REF[Q]{ditransgourdq} above and \xref{ditransalt1q} below show that an ex-situ strategy is standard for questioning an object. Additionally, recipient objects marked by the focus-sensitive particle \textit{ɔ́maná} `only' must be moved out of their canonical position, typically to the left \xref{onlyset.d} but sometimes also to the right \xref{onlyset.e}, with the canonical word order not possible \xref{onlyset.c}.\footnote{I gloss \textit{á} as \COP/\PREP{} here due to uncertainty as to how many \textit{á}s are present underlyingly and the nature of \textit{á} as a preposition; see the end of this section for further discussion.}

\ea \label{onlyset2}
%%EAX
\ea
%%JUDGEMENT
[]{
%%LABEL
\label{onlyset.d}
%%CONTEXT
%%LINE1
\glll
{\db}\textbf{ɔ́maná} \textbf{á} \textbf{mɔná} á ndɔ imítə́ túmbi. \\
%%LINE2
/ɔ́maná á mɔ-ná a \textsuperscript{H}ndɔ ɛ-mítə́ túmbiə/ \\
%%LINE3
{\db}only \COP/\PREP{} 1-child 1\SM{} \PRS{} 9-calabash return \\
%%TRANS1
\glt
‘C'est seulement [à l'enfant]\textsubscript{\FOC{}} qu'elle a donné la calebasse.' \\ `She only gave a calabash [to the child]\textsubscript{\FOC{}}.' \jambox*{[JO 1590] }
%%TRANS2
}
%%EXEND


%%EAX
\ex
%%JUDGEMENT
[]{
%%LABEL
\label{onlyset.e}
%%CONTEXT
%%LINE1
\glll
{\db}a nɔ́ hɛtɛ́tɛ́ indi \textbf{ɔ́maná} \textbf{á} \textbf{Ilísabɛt}. \\
%%LINE2
/a nɔ́ hɛ-tɛ́tɛ́ índíə́ ɔ́maná á Ilisabɛtɛ/ \\
%%LINE3
{\db}1\SM{} \PST{}1{} 19-gourd give only \COP/\PREP{} 1.Elisabeth \\
%%TRANS1
\glt
`C'est uniquement [à Elisabeth]\textsubscript{\FOC{}} qu'elle a donné la gourde.' \\ `She only gave a gourd [to Elisabeth]\textsubscript{\FOC{}}.' (and nobody else) \jambox*{[PM 1559] }
%%TRANS2
}
%%EXEND


%%EAX
\ex
%%JUDGEMENT
[*]{
%%LABEL
\label{onlyset.c}
%%CONTEXT
%%LINE1
\glll
{\db}muəndú á ndɔ \textbf{ɔ́maná} \textbf{á} \textbf{mɔná} imítə́ túmbi. \\
%%LINE2
/mɔ-əndú a \textsuperscript{H}ndɔ ɔ́maná á mɔ-ná ɛ-mítə́ túmbiə/ \\
%%LINE3
{\db}1-woman 1\SM{} \PRS{} only \COP/\PREP{} 1-child 9-calabash return \\
%%TRANS1
\glt
int. ‘C'est seulement [à l'enfant]\textsubscript{\FOC{}} que la femme a donné la calebasse.' \\ int. `The woman only gave a calabash [to the child]\textsubscript{\FOC{}}.' \jambox*{[JO 1591] }
%%TRANS2
}
%%EXEND

\z
\z

This pattern matches what we saw already for theme objects in \xref{onlyset1} above. These data taken together support a general observation that exhaustively\hyp focused elements in Tunen must be ex-situ, while information focus (for non-subjects) is typically unmarked and left in-situ in the S-Aux-O-V-X word order.

As is common cross-linguistically (see e.g. \citealt{MalchukovEtAl2010}), in addition to the double object construction, an alternative ditransitive construction is available in which the recipient object is introduced by a preposition. In this case, the word order is S-O\textsubscript{\textsc{theme}}-V-Prep-O\textsubscript{\textsc{goal}}, with the goal object an oblique in the postverbal position (S-Aux-O-V-\textit{X}). The examples below illustrate this construction in a new information focus context \xref{ditransalt1a} and in an imperative \xref{ditransalt2}.

\ea \label{ditransalt1}
\ea
%%EAX
\begin{xlist}
\exi{Q:}
%%JUDGEMENT
%%LABEL
\label{ditransalt1q}
%%CONTEXT
(Context: BaSIS photo stimulus of woman giving another woman a gourd.) \\ 
%%LINE1
\glll
{\db}ɔwanɛ́ á múə́ndú á ndɔ imítə́ túmbi? \\
%%LINE2
/ɔ-anɛ́ á mɔ-əndú á \textsuperscript{H}ndɔ ɛ-mítə́ túmbiə/ \\
%%LINE3
{\db}\PREP{}-who \COP{} 1-woman 1\SM{}.\REL{} \PRS{} 9-calabash return \\
%%TRANS1
\glt
`À qui la femme remet la calebasse ?' \\ `Who is the woman returning the calabash to?' \jambox*{[JO 1583] } 
%%TRANS2
\end{xlist}
%%EXEND

\ex
%%EAX
\begin{xlist}
\exi{A:}
%%JUDGEMENT
%%LABEL
\label{ditransalt1a}
%%CONTEXT
%%LINE1
\glll
{\db}muəndú á ndɔ imítə́ túmbiə \textbf{ɔ} \textbf{mɔn}. \\
%%LINE2
/mɔ-əndú a \textsuperscript{H}ndɔ ɛ-mítə́ túmbiə ɔ mɔ-ná/ \\
%%LINE3
{\db}1-woman 1\SM{} \PRS{} 9-calabash return \PREP{} 1-child \\
%%TRANS1
\glt
`La femme remet la calebasse [à l'enfant]\textsubscript{\FOC{}}.' \\ `The woman returns the calabash [to the child]\textsubscript{\FOC{}}.' \jambox*{[JO 1586] }
%%TRANS2
\end{xlist}
%%EXEND

%%EAX
\ex
%%JUDGEMENT
%%LABEL
\label{ditransalt2}
%%CONTEXT
%%LINE1
\glll
{\db}índíə́ imítə́ \textbf{á} \textbf{mɔná}! \\
%%LINE2
/indíə́ ɛ-mítə́ á mɔ-ná/ \\
%%LINE3
{\db}give 9-calabash \PREP{} 1-child \\
%%TRANS1
\glt
`Donne la calebasse à l'enfant !' \\ `Give the calabash to the child!' \jambox*{[JO 1595] }
%%TRANS2
%%EXEND

\z
\z

Initial analysis suggests that S-O\textsubscript{\textsc{theme}}-V-Prep-O\textsubscript{\textsc{goal}} is a lower-frequency pattern than S-O\textsubscript{\textsc{theme}}-O\textsubscript{\textsc{goal}}-V, although both strategies are found across speakers. The preposition used in the prepositional variant is generally the general preposition \textit{ɔ}, but sometimes surfaces as \textit{á} or \textit{a}. Whether \textit{á} is underlying low or high-toned and whether it is an allomorph, separate preposition (cf.\ the \textit{á} preposition found in nearby Bantoid languages), or a borrowing from French \textit{à} `to, at' is a topic for further research. In any case, its status as a preposition means that these constructions can be understood as instances of the canonical S-Aux-O-V-X word order, where X here is a prepositional phrase. The generalisation is therefore that prepositional objects are postverbal, while non-prepositional objects are preverbal (with fixed O\textsubscript{\textsc{goal}}-O\textsubscript{\textsc{theme}} order).

\subsection{Section summary}
This section presented evidence that S-Aux-O-V-X is a pragmatically neutral word order compatible with different information-structural contexts, and therefore should be taken as the canonical word order in Tunen. The order of objects in double object constructions is determined by grammatical role rather than information structure, with S-O\textsubscript{\textsc{goal}}-O\textsubscript{\textsc{theme}}-V word order or a prepositional construction S-O\textsubscript{\textsc{theme}}-V-Prep-O\textsubscript{\textsc{goal}}. While the S-Aux-O-V-X order is compatible with thetics, information focus on an object, non-argument focus, and predicate-centred focus, subjects cannot be focused in-situ, and content questions are formed ex-situ. It was noted that subject inversion of the type found in Eastern and Southern Bantu languages is also ungrammatical in Tunen and predicate doubling constructions are likewise not found. This unavailability of inversion constructions matches what was found for the Cameroonian Bantu language Basaá (Guthrie no. A43) by \citet{HamlaouiMakasso2015}, which they argue to be a feature of Northwestern Bantu more generally. The pragmatically neutral preverbal position of the object in Tunen is a further peculiarity compared to most other Bantu languages \citep{Bearth2003, Mous1997, Mous2003}. The neighbouring Cameroonian Bantu language Nyokon (A45) also has OV patterns, although only in a subset of TAM contexts; Ewondo (A72, Cameroon) and Tikar (Bantoid, Cameroon) also have partial OV patterns, but Tunen is the only known language with robust S-Aux-O-V-X basic word order, as discussed further in  \citet{Mous1997, Mous2005, Mous2014} and (\citealt{Kerr2024}; \citeyear[Chapter 6]{KerrFut}). Finally, we saw that foci modified by the focus-sensitive particles \textit{ata} `even' or \textit{ɔ́maná} `only' cannot be expressed in-situ, showing that the preverbal object position is specific to information focus, and we saw that questions are generally formed ex-situ rather than using the S-Aux-O-V-X word order (with in-situ question formation most accepted for non-argument focus).

\section{Non-canonical word order}\label{secnoncanonwordorder}
Having seen that S-Aux-O-V-X is the canonical word order in Tunen, we can look at the use of alternative word orders in different information-structural contexts. This section considers VO orders found, both with and without the \textit{á} marker, as well as argument fronting and discontinuous noun phrases.

\subsection{VO order}
Some Niger-Congo languages in Central and West Africa are known to have alternations between OV and VO word order, including Tunen's close neighbour, Nyokon (A45, Cameroon; \citealt{Mous2005}). This raises the question as to whether VO is found alongside OV in Tunen. In other work I show that Tunen OV is consistent across tense/aspect contexts (\citealt{Kerr2024}, \citeyear[Chapter 6]{KerrFut}), corroborating earlier work by \citet{Mous1997, Mous2005}. This means that there is no tense/aspect-conditioned OV/VO alternation in Tunen as reported for Nyokon and for other Niger-Congo languages, such as those of the Kwa branch \citep{Heine1976, Creissels2005, Creissels2018, SandeEtAl2019}.

While OV is thus consistent across tenses in Tunen, \citet{Mous1997, Mous2003} has proposed that Tunen does have a VO strategy, which is used for contrastive focus on the theme object and formed with a marker \textit{á} preceding the object (termed a ``contrast'' marker by \citealt{Mous2003}, although intended specifically for contrastive focus; Mous p.c.). \citet{Mous1997} also notes that VO is found for objects modified by the focus-sensitive particle \textit{ha} `only'. This relates to the data reported above of rightwardly-postposed contrastive objects, which I argue to be ex-situ cases rather than in-situ (and with fronting a more common strategy). Such a description of OV vs. VO variation dependent on information structure has been picked up in summaries of Tunen word order, such as \citet[273--274]{DowningMarten2019}.

The alternation in position of the object dependent on contrastiveness has been discussed by \citet{Güldemann2007} as an example of a more general pattern of preverbal objects in Benue-Congo being extrafocal while postverbal objects are more contrastive. Under such an account, the prediction is therefore that OV order in Tunen is found with extrafocal objects, existing in alternation with a VO pattern used for contrastive foci.

In this section I discuss the V\textit{á}O construction presented by \citet{Mous1997}. While the V\textit{á}O construction is found in my field data, it is infrequent and was only seen in elicitation contexts. I show that this construction shows evidence of becoming monoclausal, but argue against \textit{á} as a general contrast or focus marker, instead treating it as the identificational/specificational copula as part of a bicalusal cleft construction. Next, I argue that objects modified by `only' must be ex-situ, which can result in VO order on the surface but should not be taken as a basic VO order of the kind found in some West African languages. Finally, I show that some VO constructions are possible without \textit{á}, although these are rare in the data and the extent to which such patterns depend on information structure rather than independent factors such as prosodic weight or predicate type needs further testing.

\subsubsection{V\textit{á}O} \label{subsec:Vao}
In \citeauthor{Mous1997}'s \citeyearpar{Mous1997, Mous2003} analysis of Tunen syntax, he identifies an SVO construction in Tunen where the postverbal object is preceded by what he calls a ``contrast marker'', \textit{á}, marking contrastive focus, as illustrated in \xref{yamseat} below from Mous' own elicitation data.\footnote{Glosses for the subject marker, tense marker, and noun have been adapted for consistency with the rest of the examples in this chapter; the \CONTR{} gloss and data line are left unchanged.} 

%%EAX
\ea
%%JUDGEMENT
%%LABEL
\label{yamseat}
%%CONTEXT
%%LINE1
%%LINE2
\gll mɛ́-ndò ní \textbf{á} bónìàk.\\
%%LINE3
{\db}\SM{}.1\SG{}-\PRS{} eat \CONTR{} 14.yam \\
%%TRANS1
\glt
`What I eat is yams.' \jambox*{\citep[304]{Mous2003} }
%%TRANS2
%%EXEND

\z

A key question is whether this \textit{á} marker is best treated as a contrast marker (\CONTR{}), focus marker (\FOC{}), or copula (\COP{}). If \textit{á} were a general focus marker, we may expect it to be able to appear on focused objects in other positions. However, it is not possible to have \textit{á} precede the object in the canonical preverbal position \xref{SáOVgourd}, despite S-Aux-O-V-X being compatible with information focus on the object (as we saw in \xref{carrotobj} above).

%%EAX
\ea[]{
\label{SáOVgourd}
(`What did the woman give to the other woman?' (+ BaSIS photo stimulus)) \\
}
%\sn
%%JUDGEMENT
%[*]{
%%LABEL
%%CONTEXT
%%LINE1
\glll
{\db}a nɔ́ ɔsɔ́kɔ́ \textbf{(*á)} \textbf{hɛtɛ́tɛ́} indiə. \\
%%LINE2
/a nɔ ɔsɔkɔ á hɛ-tɛ́tɛ́ índíə́/ \\
%%LINE3
{\db}1\SM{} \PST{}1{} 1.other \COP{} 19-gourd give \\
%%TRANS1
\glt
int. `Elle a donné à l'autre [une gourde]\textsubscript{\FOC{}}.' \\ int. `She gave the woman [a gourd]\textsubscript{\FOC{}}.' \jambox*{[PM 1541, 1549] }
%%TRANS2
% }
%%EXEND

\z\largerpage

We should therefore not take \textit{á} to be a general focus marker. This observation matches the data from exhaustive focus marked by the focus-sensitive particle \textit{ɔ́maná} `only', which requires that the object is moved from its base position \xref{onlyset1}, \xref{onlyset2}. These data therefore are compatible with Mous' \citeyearpar{Mous1997, Mous2003} analysis of \textit{á} as a contrastive focus marker, but not as a general focus marker.


One key difference in my data compared to \citeauthor{Mous2003}'s \citeyearpar{Mous2003} presentation is that SV\textit{á}O was a low frequency strategy in my corpus. In \citegen{Isaac2007} study of 6 of the longer Tunen texts transcribed in \citet{Dugast1975}, he also reports that there were no clear examples of this construction \citep[61]{Isaac2007}. Furthermore, the construction was even judged ungrammatical in the following elicitation session, regardless of the position of the object with respect to the postverbal adjunct \xref{SVáOcarrotset}.



%%EAX
\ea
%%JUDGEMENT
[*]{
%%LABEL
\label{SVáOcarrotset}
%%CONTEXT
%%LINE1
\glll
{\db}a ná itíə́ $\{$ɔ mɔkata$\}$ \textbf{á} \textbf{kalɔ́tɔ} $\{$ɔ mɔkata$\}$. \\
%%LINE2
/a ná itíə́ $\{$ɔ mɔ-kátá$\}$ á kalɔ́tɔ $\{$ɔ mɔ-kátá$\}$/ \\
%%LINE3
{\db}1\SM{} \PST{}2{} hold \PREP{} 3-hand \COP{} 9.carrot \PREP{} 3-hand \\
%%TRANS1
\glt
int. `Ce qu'il tient à la main, c'est [une carotte]\textsubscript{\FOC{}}.' \\ int. `He is holding [a carrot]\textsubscript{\FOC{}} in his hand.' \jambox*{[JO 1627--8] }
%%TRANS2
}
%%EXEND

\z


Instead, the focused object can be left in-situ unmarked by \textit{á} (S-Aux-O-V-X), the typical expression of information focus, or be focused via a reverse pseudocleft cleft with \textit{á}, as used for contrastive and exhaustive focus (see \sectref{secclefts}). The SV\textit{á}O word order is therefore a less frequent pattern than it may seem from Mous' description. We will see in \sectref{secclefts} that a reverse pseudocleft of the form O\textit{á}SVX is a more common strategy for contrastive focus than the pseudocleft type here; recall as well from \xref{evenpotatoset} above that fronting was preferred to postposing when an object is modified by a focus-sensitive particle. In my natural speech recordings, the V\textit{á}O construction did not appear.

In earlier oral presentation of this work, I argued that the SV\textit{á}O construction is a pseudocleft, on the basis of evidence of constructions with \textit{á} showing properties of relative clauses and therefore a biclausal cleft structure (to be seen for other cleft types in \sectref{secclefts}). However, remotely elicited data explicitly testing this hypothesis for this low-frequency SV\textit{á}O construction found that this construction does not allow relative tense morphology or the relative form of the subject marker, thus providing evidence for monoclausality rather than biclausality. The dataset in \xref{killspiderset} illustrate this point.\footnote{For these remotely-elicited data, the form ID in square brackets refers to the session number followed by the example number in this session.}

\ea
\label{killspiderset}
%%EAX
\ea
%%JUDGEMENT
[]{
%%LABEL
\label{killspidercleft}
%%CONTEXT
%%LINE1
\glll
{\db}\textbf{nɛlala} \textbf{á} babá \textbf{á} \textbf{\ds{}ná} húánána ɔwɔ́n. \\ 
%%LINE2
/nɛ-lala á babá á {\textsuperscript{L}ná} húánána ɔ-ɔ́nɔ/ \\
%%LINE3
{\db}5-spider \COP{} 1.father 1\SM{}.\REL{} \PST{}3.\REL{} must \INF{}-kill \\
%%TRANS1
\glt
`C'est [l'araignée]\textsubscript{\FOC{}} que papa devait tuer.' \\ `It's [the spider]\textsubscript{\FOC{}} that dad had to kill.' \jambox*{[PM 70.61] }
%%TRANS2
}
%%EXEND


%%EAX
\ex
%%JUDGEMENT
[]{
%%LABEL
\label{killspidervaomono}
%%CONTEXT
%%LINE1
\glll
{\db}babá \textbf{a} \textbf{ná} húánána ɔwɔ́nɔ \textbf{á} \textbf{nɛ́lal}. \\
%%LINE2
/babá a ná húánána ɔ-ɔ́nɔ á nɛ-lala/ \\
%%LINE3
{\db}1.father 1\SM{} \PST{}3{} must \INF{}-kill \CONTR{} 5-spider \\
%%TRANS1
\glt
`Ce que papa devait tuer n'est que [l'araignée]\textsubscript{\FOC{}}.' \\ `Dad had to kill (only) the [spider]\textsubscript{\FOC{}}.' \jambox*{[PM 70.57] }
%%TRANS2
}
%%EXEND


%%EAX
\ex
%%JUDGEMENT
[*]{
%%LABEL
\label{killspidervaobi}
%%CONTEXT
%%LINE1
\glll
{\db}babá \textbf{á} \textbf{\ds{}na} húánána ɔwɔ́nɔ \textbf{á} \textbf{nɛ́lal}. \\
%%LINE2
/babá a \textsuperscript{L}na húánána ɔ-ɔ́nɔ á nɛ-lala/ \\
%%LINE3
{\db}1.father 1\SM{}.\REL{} \PST{}3.\REL{} must \INF{}-kill \CONTR{} 5-spider \\
%%TRANS1
\glt
int. `Ce que papa devait tuer n'est que [l'araignée]\textsubscript{\FOC{}}.' \\ int. `Dad had to kill (only) [the spider]\textsubscript{\FOC{}}.' \jambox*{[PM 70.62] }
%%TRANS2
}
%%EXEND

\z
\z

Here in \xref{killspidercleft} we see a reverse pseudocleft construction, in which the focused object \textit{nɛlala} `spider' is followed by the \textit{á} copula and then a reduced relative. As we will see in \sectref{secclefts}, the reduced relative environment is evidenced by the H tone on the class 1 subject marker \textit{á} -- which is \textit{a} in non-dependent clause contexts \citep[§4.3.3.2]{KerrFut} -- in addition to the dependent clause form of the third-degree past tense marker \textit{\ds{}ná} (which is \textit{ka} in main clause contexts, \citealt{Dugast1971, Mous2003}, \citealt[§4.4.5]{KerrFut}). We therefore expect to see these indicators in the relative subject marker and TAM contexts for the SV\textit{á}O construction. However, when the SV\textit{á}O construction is used, the subject marker and tense marker are in the main clause form \xref{killspidervaomono} and cannot be in the dependent clause form \xref{killspidervaobi}. These data therefore suggest that the SV\textit{á}O construction is not a biclausal pseudocleft and instead is grammaticalising into a monoclausal construction marking contrastive focus. Note also that the translations indicate that the SV\textit{á}O construction gives a sense of exhaustivity. A similar idea of contrast was provided by another speaker, and so I gloss the \textit{á} in the SV\textit{á}O construction as \CONTR{} for contrast, while \textit{á} is elsewhere glossed as \COP{} for copula.\footnote{In follow-up discussion of these examples, JO confirmed that both (\ref{killspidercleft}) and (\ref{killspidervaomono}) can be felicitously continued with \textit{, tátá á ɔnd͡ʒɛlɛ́} `, not the lizard', supporting this idea of exhaustivity.}

In summary then, we see that the V\textit{á}O strategy discussed by \citet{Mous1997, Mous2003} exists, with elicitation data providing evidence that it is at least in the process of becoming monoclausal and not simply a pseudocleft, as evidenced by the lack of relative SM and TAM forms. However, it is a low-frequency pattern that was not always judged as grammatical by consultants \xref{SVáOcarrotset} and did not appear in the natural speech data at all, with a reverse pseudocleft a much more common construction for expression of contrastive focus on the object. 

\subsubsection{VO without á}
While OV order is by far the most common word order pattern in Tunen, a few instances in my own and \citeauthor{Dugast1971}'s \citeyearpar{Dugast1971, Dugast1975} data show VO order, with no \textit{á} marker and no indication of biclausality. These instances are rare; VO was generally judged ungrammatical in elicitation, with only a few cases in which it was accepted. When checking a set of 10 natural speech texts containing approximately 400 utterances, only 4 potential VO constructions were found, but all of these can be excluded from being actual instances of VO syntax. 1 can be excluded due to being a case of secondary predication rather than a true DP object (\ref{ex:7vosecpred}), 1 can be excluded due to being a case of hesitation (\ref{ex:7vohesitation}), and 1 shows switching from Tunen to French (\ref{ex:7vofrench}).

%%EAX
\ea
%%JUDGEMENT
%%LABEL
\label{ex:7vosecpred}
%%CONTEXT
(Context: EO and PM are discussing the funeral of a local figure called Papa Daniel.) \\
%%LINE1
\glll
{\db}bá sɛ .. mukót ... mba a ka \textbf{híána} \textbf{munɛn}. \\
%%LINE2
/bá sɛ́á .. mɔ-kóto ... mba a ka híána mɔ-nəni/ \\
%%LINE3
\SM{}.2 say .. 1-Bamileke ... but \SM.1 \PST{}3 become 1-Nen \\
%%TRANS1
\glt
`On dit que c'est un Bamileke, mais il est devenu un Munen.' \\ `They say he's a Bamileke, but he became a Munen.' \jambox*{[EO 1037] }
%%TRANS2
%%EXEND
\z

%%EAX
\ea
%%JUDGEMENT
%%LABEL
\label{ex:7vohesitation}
%%CONTEXT
(Context: PM is giving EO instructions for the QUIS map task.) \\
%%LINE1
\glll
{\db}mɛ́ ndɔ \textbf{sinə} ... \textbf{mɛnyama} \textbf{ɛ-káhɔ}. \\
%%LINE2
/mɛ \textsuperscript{H}ndɔ sinə ... mɛ-nyama ɛ-kahɔ/ \\
%%LINE3
{\db}\SM.1\SG{} \PRS{} see ... 9-animal 9-beef \\
%%TRANS1
\glt
`Je vois ... une vache.' \\ `I see ... a cow.' \jambox*{[EO 664]}
%%TRANS2
%%EXEND

 \z


%%EAX
\ea
%%JUDGEMENT
%%LABEL
\label{ex:7vofrench}
%%CONTEXT
(PM: `I myself saw the first vehicle he bought, it was Inyas who drove (it)-' \\ EO: `I (also) saw (it)'.) \\
%%LINE1
\glll
{\db}-a ka	\textbf{tiləkə}	« \textbf{Dieu}	\textbf{haït}	\textbf{les} {\textbf{méchants} »} \\
%%LINE2
/a ka tilə-aka	« Dieu	haït	les {méchants »/} \\
%%LINE3
{\db}\SM{}.1 \PST{}3	write-\DUR{} { } God.FR	hates.FR	the.FR wicked.people.FR \\
%%TRANS1
\glt
`- Il avait écrit « Dieu haït les méchants ».' \\ `- He wrote ``God hates the wicked".' \jambox*{[PM 1047] }
%%TRANS2
%%EXEND

 \z


The final example, given in (\ref{ex:7voclause}), illustrates the occasional ambiguity in classifying a construction as VO or OV. Here, the objet \textit{tɔ́ánd͡ʒɛ} `leaves' can be either taken to be the complement of the verb in the first clause or a fronted topic in the second clause (with zero expression of the object in the first clause due to givenness, for which see \sectref{secreference}). Although originally transcribed as VO, when asking JO remotely in follow-up work, she interpreted the object as a fronted topic, meaning that this utterance would also be OV. 



%%EAX
\ea
%%JUDGEMENT
%%LABEL
\label{ex:7voclause}
%%CONTEXT
(Context: JO is explaining how to make the dish \textit{kok} [\textit{hɛkɔkɛ} leaves boiled with smoked fish and ground peanuts].) \\
%%LINE1
\glll
{\db}Mɛ ka ákán(a) (ɔ) ɛmbɔ́m, mɛ ná hɛkɔkɛ kɛ́ták, mɛ ná nda híáná ɔ ɔmbɛ́l, mɛ ná \textbf{tábɔ́náka} \textbf{tɔ́ánd͡ʒɛ} \textbf{tɔbíá} mɛ \textbf{ombokok}. [...] \\
%%LINE2
/mɛ ka ákáná {\db}ɔ ɛ-mbɔ́ma mɛ ná hɛ-kɔkɛ kɛ́táka mɛ ná nda híáná ɔ ɔ-mbɛ́la mɛ ná tábɔ́ná-aka tɔ-ánd͡ʒɛ tɔ-bíá mɛ ombokoko/ \\
%%LINE3
{\db}\SM.1\SG{} \PST{}3 leave \PREP{} 7-bush \SM.1\SG{} \PST{}2 19-kok gather-\DUR{} \SM.1\SG{} \PST{}2 \VEN{} enter \PREP{} 3-house \SM.1\SG{} \PST{}2 arrange-\DUR{} 13-leaf 13-bad \SM.1\SG{} throw.\REP{} \\
%%TRANS1
\glt
`Je suis partie en brousse, j'ai cueilli le \textit{kok}, je suis revenue à la maison, j'ai arrangé les mauvaises feuilles, je les ai jété, [...]' \\ `I went to the bush, I gathered \textit{kok}, I returned home, I arranged the bad leaves, I threw them out, [...]' \jambox*{[JO 1339] }
%%TRANS2
%%EXEND

 \z


However, VO constructions do sometimes show up, and some examples are found in the Dugast texts. \citet{Mous2003} notes that objects in such VO constructions are prosodically phrased with the verb, as evidenced by H tone spread. This is illustrated in \xref{palmwinetap}, where I have added an underlying representation line and adapted the glosses to show that the H tone of \textit{kemá} `tap' spreads rightwards onto the underlyingly L-toned class 6 prefix \textit{ma-}.

%%EAX
\ea
%%JUDGEMENT
%%LABEL
\label{palmwinetap}
%%CONTEXT
%%LINE1
\glll
{\db}à-ná kèmáká mʷə́lùk. \\
%%LINE2
/a-ná kemá-aka ma-lukə/ \\
%%LINE3
{\db}1\SM{}-\PST{}2{} tap-\DUR{} 6-palm.wine \\
%%TRANS1
\glt
`He tapped palm wine!' \jambox*{(\citealt[58]{Dugast1971}, \citealt[304]{Mous2003})}
%%TRANS2
%%EXEND

\z

While such examples appear in Dugast's work, JO considered this sentence ungrammatical when asked in follow-up work. Furthermore, the validity of H tone spread as a diagnostic of syntactic phrasing is uncertain, as discussed in \citet[304--305]{KerrFut}.

At this stage, it is therefore not clear whether there is generalisation accounting for when these VO examples can appear and the extent to which this depends on information structure. Multiple other factors could play a role, including prosodic weight, predicates requiring extraposition, and postverbal modifier placement. Prosodic weight alone would not account for example \xref{palmwinetap} above, and we saw already in \xref{churchthetic} above that prosodically heavy preverbal objects are possible. However, prosodic weight is a factor that could explain the discontinuous relative clauses modiyfing objects that we will see in \sectref{secdiscon} below.

A second context in which VO order is found without \textit{á} is with objects modified by certain modifiers, most commonly numerals, as reported in \citep[Chapter 7]{KerrFut}. The default order (i.e. the most common order, found across different information-structural contexts) for such objects is S-O-V-Mod, i.e. a discontinuous noun phrase. However, the object can also appear adjacent to the modifier, leading to the VO order S-V-O-Mod. The order S-O-Mod-V is dispreferred.  For example, in the sub-DP focus context in \xref{subDPvoset} below in which the focus falls on the numeral modifier of the theme object, V-O-Num was allowed \xref{subDPvo} as well as the discontinuous order O-V-Num \xref{subDPdisc}, while the preverbal order SONumV was considered marginal (despite SOV being generally allowed and numerals always following the noun they modify; \citealp{Dugast1971, Mous2003, Kerr2020, KerrFut}) \xref{subDPov}. The availabilty of S-V-O-Num for narrow focus on the numeral is illustrated for another consultant in \xref{animalkill}.

\ea \label{subDPvoset} (`How many people do you see?' (+ picture))
%%EAX
\ea
%%JUDGEMENT
[]{
%%LABEL
\label{subDPvo}
%%CONTEXT
%%LINE1
\glll
{\db}mɛ́ ndɔ sinə \textbf{bɛndɔ} \textbf{báfandɛ}. \\
%%LINE2
/mɛ \textsuperscript{H}ndɔ sinə bɛndɔ bá-fandɛ́/ \\
%%LINE3
{\db}\SM{}.1\SG{} \PRS{} see 2.person 2-two \\
%%TRANS1
\glt
`Je vois [deux]\textsubscript{\FOC{}} personnes.' \\ `I see [two]\textsubscript{\FOC{}} people.' \jambox*{[JO 541] }
%%TRANS2
}
%%EXEND

\ex
[]{
\label{subDPdisc}
% \gl
mɛ́ ndɔ \textbf{bɛndɔ} sinə \textbf{báfandɛ}. \jambox*{[JO 542] }
% mɛ-\textsuperscript{H}ndɔ \textbf{bɛ-ndɔ} sinə \textbf{ba-\textsuperscript{H}fandɛ́} \\
% \SM{}.1\SG{}-\PRS{} 2-person see 2-two \\
%\glt `Je vois [deux]\textsubscript{\FOC{}} personnes.' \\ `I see [two]\textsubscript{\FOC{}} people.'  \jambox*{[JO 542] }
}

\ex
[?]{
\label{subDPov}
% \gl
mɛ́ ndɔ \textbf{bɛndɔ} \textbf{báfandɛ́} sinə. \jambox*{[JO 543] }
% /mɛ-\textsuperscript{H}ndɔ \textbf{bɛ-ndɔ} \textbf{ba-\textsuperscript{H}fandɛ́} sinə/ \\
% \SM{}.1\SG{}-\PRS{} 2-person 2-two see \\
%\glt `Je vois deux personnes' \\ `I see two people.' \jambox*{[JO 543] }
}
\z
\z

%%EAX
\ea
%%JUDGEMENT
%%LABEL
\label{animalkill}
%%CONTEXT
(`How many animals did he kill?') \\ 
%%LINE1
\glll
{\db}a ná ɔnɔkɔ \textbf{mɛnyama} \textbf{ímoti}. \\
%%LINE2
/a ná ɔ́nɔ-aka mɛ-nyama ɛ́-mɔtɛ́/ \\
%%LINE3
{\db}1\SM{} \PST{}2{} kill-\DUR{} 9-animal 9-one \\
%%TRANS1
\glt
`Il en a tué [un]\textsubscript{\FOC{}} (seul).' \\ `He killed [one]\textsubscript{\FOC{}} animal.' \jambox*{[EO 1416] }
%%TRANS2
%%EXEND

\z


The S-V-O-Num pattern was also provided for contexts other than narrow focus on the modifier, for example with term focus on the entire object \xref{whatseeV-O-Num} and as an answer to a polar question (polarity focus) \xref{twobirdsSV-O-Num}. While this VO order was judged grammatical, it is worthwhile noting that the discontinuous S-O-V-Num order was the first response \xref{twobirdsS-O-V-Num}.

%%EAX
\ea
%%JUDGEMENT
%%LABEL
\label{whatseeV-O-Num}
%%CONTEXT
(Context: MPI scope image 1/77 + ex-situ object question \textit{yatɛ́ ɔ́ ndɔ sin?} (`What do you see?')) \\ 
%%LINE1
\glll
{\db}mɛ́ ndɔ	sinə \textbf{bɔlɛ́á}	\textbf{bɔmɔ́tɛ}. bɔ́ báka na tunoní tuəŋ.  \\
%%LINE2
/mɛ \textsuperscript{H}ndɔ sinə bɔ-lɛ́á	bɔ́-mɔtɛ́	bɔ́ bá-aka na tɔ-noní tɔ-əŋí/ \\
%%LINE3
{\db}\SM{}.1\SG{} \PRS{} see 14-tree	14-one	14\SM{} be-\DUR{}	with 13-bird	13-many \\
%%TRANS1
\glt
`Je vois [un arbre]\textsubscript{\FOC{}}. Il a beaucoup d'oiseaux.' \\`I see [a tree]\textsubscript{\FOC{}}. It has many birds.' \jambox*{[JO 1151] }
%%TRANS2
%%EXEND

\z


\ea \label{twobirdsS-O-V-Num} (`Do you see two birds?' (+ hand-drawn picture stimulus) (polarity focus)) \\
%%EAX
\ea
%%JUDGEMENT
%%LABEL
%%CONTEXT
%%LINE1
\glll
{\db}ɛ́ɛ, mɛ́ ndɔ \textbf{tunoní} sinə \textbf{tɔ́fandɛ}. \\
%%LINE2
/ɛ́ɛ mɛ \textsuperscript{H}ndɔ tɔ-noní sinə tɔ́-fandɛ́/ \\
%%LINE3
{\db}yes \SM{}.1\SG{} \PRS{} 13-bird see 13-two \\
%%TRANS1
\glt
`Oui, je vois deux oiseaux.' \\ `Yes, I see two birds.' \jambox*{[EO 1408] }
%%TRANS2
%%EXEND

\ex\label{twobirdsSV-O-Num}
% \gl
ɛ́ɛ, mɛ́ndɔ sinə \textbf{tunoní} \textbf{tɔ́fandɛ}. \jambox*{[EO 1409] }
% /ɛ́ɛ mɛ-\textsuperscript{H}ndɔ sinə \textbf{tɔ-noní} \textbf{tɔ-fandɛ́} / \\
% yes \SM{}.1\SG{}-\PRS{} see 13-bird 13-two \\
%\glt `Oui, je vois deux oiseaux.' \\ `Yes, I see two birds.' \jambox*{[EO 1409] }
\z
\z

In general, V-O-Num examples are less common than a discontinuous order in which the object is preverbal while its modifier is postverbal, which I will discuss more in \sectref{secdiscon} below.  I suggest that the S-V-O-Num order is related to the postverbal position being the preferred position for certain quantifiers like numerals, together with the desire to preserve the contiguity of the noun phrase constituent, rather than being related to the information-structural status of the noun and/or modifier. The following example shows that the universal quantifier \textit{-kimə} `all' can also appear in this slot when in focus.\largerpage
 
 
 %%EAX
\ea
%%JUDGEMENT
%%LABEL
%%CONTEXT
(Someone mistakenly says that not all the children did their homework (knowing that some are more studious than others).) \\ 
%%LINE1
\glll
{\db}bɔ́ɔ, bá ná masɔ́ma kiak \textbf{bə́kim}. \\
%%LINE2
/bɔ́ɔ bá ná ma-sɔ́ma kɛa-aka bá-kimə/ \\
%%LINE3
{\db}no 2\SM{} \PST{}2{} 6-homework do-\DUR{} 2-all \\
%%TRANS1
\glt
`Non, c'est [tous]\textsubscript{\FOC{}} qui ont fait les devoirs.' \\ `No, [all]\textsubscript{\FOC{}} of them did the homework.' \jambox*{[EE+EB 1824] }
%%TRANS2
%%EXEND

 \z
 
We therefore see that this VO order may be explained by the appearance of a quantifier element, rather than being conditioned by information-structural considerations. \sectref{secdiscon} (see also \citealp[Chapter 7]{KerrFut}) will cover such discontinuous nominals in further detail, showing that the discontinuous modifier placement is, somewhat surprisingly, the pragmatically neutral word order.

A final context where VO is found without \textit{á} is with focus-sensitive particles, as in the example below from \citet{Mous1997} and as already seen in \sectref{secwordorder} for objects modified by \textit{ata} `even' and \textit{ɔ́maná} `only.\footnote{Glosses have been standardised; the transcription line is unaltered.}\footnote{This example was checked in follow-up work with JO; she accepts it, but rejects it if the focus-sensitive particle is omitted.}

%%EAX
\ea
%%JUDGEMENT
%%LABEL
\label{mous1997onlyVO}
%%CONTEXT
%%LINE1
%%LINE2
\gll
mɛna nya ha mʷənif.\\
%%LINE3
\SM{}.1\SG{}.\PST{}2{} drink only 6.water \\
%%TRANS1
\glt 
`I drank only water.' \jambox*{\citet[125]{Mous1997} }
%%TRANS2
%%EXEND

\z

This VO order arises due to exhaustively-focused objects needing to move from the canonical position. Recall from \sectref{secwordorder} that while such movement to the right is possible, movement to the left is a more common strategy. The availability of VO order here  is expressed by Mous as needing to ``mak[e] a statement about the relation of a particular object against other possible objects'' \citep[127]{Mous1997}; he claims that the relation is not one of focus or new information but of contrast. I follow the other authors in this volume in calling such objects focused objects of a more contrastive type than information focus (see e.g. \citealt{BianchiEtAl2015, Cruschina2021}). This requirement to be ex-situ extends across all contrastively focused terms and is not specific to objects.\largerpage

Finally, it is worth noting that in elicitation contexts, most speakers reject VO examples; VO examples are therefore quite low-frequency in my corpus. These constructions could therefore be better investigated through a larger-scale corpus study with more natural speech examples and controlling for independent factors such as prosodic weight. Given the prevalence of VO orders in Benue-Congo and the variation between OV and VO in other languages with OV word order, the lack of VO in Tunen is particularly interesting from a comparative and historical perspective (see \citealp[Chapter 6]{KerrFut} for further reflections).


\subsection{Discontinuity} \label{secdiscon}
As noted by \citet{Mous1997}, modifiers of theme objects -- including numerals, quantifiers, and relative clauses -- may appear in Tunen in a discontinuous position, separated from the object by the verb, resulting in the discontinuous S-O-V-Mod order. Crosslinguistically, discontinuity is a low frequency word order strategy that relates directly to information structure: discontinuous noun phrases are used for focus on the modifier (see e.g. \citealt{LouagieVerstraete2016}), with a common pattern involving scrambling to a left-peripheral topic or focus phrase \citep{FanselowĆavar2002}. In Bantu, discontinuous noun phrases are very rare \citep[909]{VandeVelde2022}. In Tunen, however, discontinuous modifiers are found frequently, the modifier and object do not move to the left periphery, and they do not require narrow focus on the modifier. Instead, this word order appears to be pragmatically neutral: as I show in this section (see also \citealp[Chapter 7]{KerrFut}), it is possible with narrow focus on the postverbal modifier, with focus on the whole object, or even to introduce new discourse referents (as already observed in \citealt{Isaac2007}). Note that this analysis of discontinuity as a pragmatically neutral order runs against the analysis of \citet[133]{Mous1997}, who argues that discontinuous modifiers have ``contrastive force'' in Tunen, likening them to postverbal objects preceded by \textit{á} or a focus-sensitive particle `only'.\footnote{To be precise, Mous identifies the discontinuous position as the standard position for Tunen numerals and argues that numerals are inherently contrastive in that they are ``selective'' \citep{Mous1997}.}

Examples of discontinuous modifiers are given below, first for the universal quantifier  \textit{-kimə} `all' and secondly for a numeral, both in contexts where the quantity is surprising.\largerpage

%%EAX
\ea
%%JUDGEMENT
%%LABEL
%%CONTEXT
(Context: You are a farmer who has lost all of your animals, but by a stroke of luck, you find them all again.) \\
%%LINE1
\glll
{\db}mɛ ná Húɛ́lɛ́ hóli ɛ́\ds{}sɛ́á mɛ́ ná \textbf{biá\ds{}míá} \textbf{bɛhɔ́sɛ} bɔŋɔ \textbf{bikim}. \\
%%LINE2
/mɛ ná Húɛ́lɛ́ hólíə́ ɛ-sɛ́á mɛ́ ná biámíá bɛ-hɔ́sɛ bɔ́ŋɔ́ bɛ́-kimə/ \\
%%LINE3
{\db}\SM{}.1\SG{} \PST{}2{} God thank 7-say \SM{}.1\SG{}.\SBJV{} \PST{}2{} \POSS{}.\PRO{}.1\SG{}.8{} 8-horse find 8-all \\
%%TRANS1
\glt
`Je remercie Dieu comme j'ai retrouvé tous mes chevaux.' \\ `I thank the Lord that I've found all my horses.' \jambox*{[EE+EB 1827] }
%%TRANS2
%%EXEND

%%EAX
\ex
%%JUDGEMENT
%%LABEL
%%CONTEXT
(Context: You wake up after a party and see a surprising number of empty wine bottles in the room.) \\ 
%%LINE1
\glll
{\db}naánɛkɔla ɛkɔlakɔ́lá tɔ ka \textbf{mindíŋgə} nɛ́ákɛna \textbf{ɛ́námanɛ} \textbf{yɛ́} \textbf{mə́luk}! \\
%%LINE2
/naánɛkɔla ɛ-kɔlakɔ́lá tɔ ka mɛ-ndíŋgə nɛ́ákɛna ɛ́-námanɛ yɛ́ mə́lukə/ \\
%%LINE3
{\db}yesterday 7-evening \SM{}.1\PL{} \PST{}3{} 4-bottle drink 4-eight \ASS{}.4 6.wine \\
%%TRANS1
\glt
`Hier soir nous avons bu huit bouteilles du vin !' \\  `We drank eight bottles of wine last night!' \jambox*{[JO 1941] }
%%TRANS2
%%EXEND

\z

Example \xref{twobirdsS-O-V-Num} above and example \xref{didgivecalabash} below show that the modifier does not have to be new information in order to be discontinuous in Tunen. In \xref{didgivecalabash}, we see a discontinuous modifier used in a polarity focus context (a subtype of predicate-centred focus), in which both the object and modifier were given through prior expression in the discourse and are therefore non-focal.

%%EAX
\ea
%%JUDGEMENT
%%LABEL
\label{didgivecalabash}
%%CONTEXT
%%LINE1
\glll
{\db}(ɛ́ɛ,) mɛ ná \textbf{imitə} \textbf{yə} \textbf{mwənífí} indi \textbf{mɛŋɛ́ŋ} ɔ hɛlɔ́bátɔ. \\
%%LINE2
/(ɛ́ɛ) mɛ ná ɛ-mítə́ yɛ ma-nífə́ índíə́ mɛŋɛ́ŋa ɔ hɛ-lɔ́bátɔ/ \\
%%LINE3
{\db}yes \SM{}.1\SG{} \PST{}2{} 9-calabash \ASS{}.{}9 6-water give 9.big \PREP{} 19-child \\
%%TRANS1
\glt
`(Oui,) j'ai donné la grand calebasse (de l'eau) à l'enfant.' \\`(Yes,) I gave the large calabash of water to the child.' \jambox*{[EE+EB 1830; cf. \citealt[305]{Mous2003}] }
%%TRANS2
%%EXEND

\z

Discontinuous orders are not just found for modifiers of objects; discontinuity is also possible for quantifiers modifying subjects, as in the natural speech examples in \xref{quistreetask} below. In \xref{quisthreementreestart} we see the use of a discontinuous structure for the introduction of a discourse referent; in \xref{quisthreementreeend} we see the use of the discontinuous order when concluding the explanation. In both cases, the discontinuous numeral is neither focal nor contrastive.

\ea \label{quistreetask} (Context: QUIS dialogue task: EO has a picture from the end of a storyboard and must find out from PM (who has the rest of the storyboard) what happened.) 
%%EAX
\ea
%%JUDGEMENT
%%LABEL
\label{quisthreementreestart}
%%CONTEXT
%%LINE1
\glll
{\db}mba \textbf{bɛndɔ} bá báka háha \textbf{balal}, yatɛ́ bá ndɔ kɛ? \\
%%LINE2
/mba bɛndɔ bá bá-aka háaha bá-lálɔ́ yatɛ́ bá \textsuperscript{H}ndɔ kɛa/ \\
%%LINE3
{\db}but 2.person 2\SM{} be-\DUR{} \DEM{}.\PROX{}.\LOC{} 2-three what 2\SM{} \PRS{} do \\
%%TRANS1
\glt
`Mais il y a trois personnes ici, que font-ils?' \\ `But there are three people here, what are they doing?' \jambox*{{[EO 581]}}
%%TRANS2
%%EXEND


%%EAX
\ex
%%JUDGEMENT
%%LABEL
\label{quisthreementreeend}
%%CONTEXT
Context: After concluding the explanation. \\ 
%%LINE1
\glll
{\db}mhm. ɔ́\ds{}há \textbf{bɛndɔ} bándɔ wɛɛ́ya sinə \textbf{balal}. \\
%%LINE2
/mhm ɔhá bɛndɔ ba \textsuperscript{H}ndɔ wɛ́ɛya sinə bá-lálɔ́/ \\
%%LINE3
{\db}mhm for.that 2.person 2\SM{} \PRS{} \PRO{}.1{} see 2-three \\
%%TRANS1
\glt
`Mhm. C'est pour ça que trois gens le regardent.' \\ `Mhm. That's why three people are looking at him.' \jambox*{[PM 597] }
%%TRANS2
%%EXEND

\z
\z

The postverbal modifier position therefore seems to be a neutral position rather than related to a particular information-structural configuration. I discuss the details of the discontinuous noun phrase construction in \citet[Chapter~7]{KerrFut}; for current purposes we can conclude that the construction is not restricted to a particular discourse context of contrastive focus (contra \citealt{Mous1997}).
 
Finally, note that in addition to numeral and quantifier modifiers, relative clauses modifying objects are frequently discontinuously-positioned (O-V-Rel; \xref{relativemotorbike}, \xref{relativevehicle}), though they can also be continuous before the verb (O-Rel-V; \xref{relativeMartinfood}) or continuous after the verb (V-O-Rel; \xref{relativeguyfell}).

%%EAX
\ea
%%JUDGEMENT
%%LABEL
\label{relativemotorbike}
%%CONTEXT
%%LINE1
\glll
{\db}mɛ ná \textbf{wááyɛ́} \textbf{múə́ndu} siəkin \textbf{ɔwánákáná} {} \textbf{ɔ} \textbf{bulí} \textbf{na} \textbf{móto}, tátá wə́n. \\
%%LINE2
/mɛ ná wááyɛ́ mɔ-əndú siəkinə ɔwá-á-ánákáná ɔ bɔ-lí na móto tátá wə́ni \\
%%LINE3
{\db}\SM{}.1\SG{} \PST{}2 \DEM{}.\DISC{}.1 1-woman see.\DUR{} \REL{}.1{}-1\SM{}.\REL{}.\PST{}2{}-leave \PREP{} 14-work with 6.motorcycle \COP{}.\NEG{} \DEM.\DIST.1 \\
%%TRANS1
\glt
`J'ai vu la femme qui est allée au travail avec le moto, pas l'autre.' \\ `I saw the woman who went to work by motorbike, not the other one.' \jambox*{[PB 2019] }
%%TRANS2
%%EXEND

 
 \z
 
 %%EAX
\ea
%%JUDGEMENT
%%LABEL
\label{relativevehicle}
%%CONTEXT
%%LINE1
\glll
{\db}mɛ ka ámɛ \textbf{yáyɛ́á} \textbf{ibəŋuluəkə} \textbf{yí} \textbf{búsíə́} siəkinə \textbf{ɔyɛ́á} \textbf{á} \textbf{\ds{}ná} \textbf{ɔnd}, [...] \\
%%LINE2
/mɛ ka ámɛ yayɛ́á ɛ-bəŋuluəkə yɛ́ búsíə́ siəkinə ɔyɛ́á á \textsuperscript{L}ná ɔ́ndɔ́/ \\
%%LINE3
{\db}\SM{}.1\SG{} \PST{}3{} \PRO{}.1\SG{} 7{}.\PRO{}.\POSS{}.1{} 7-car \ASS{}.7{} front see.\DUR{} \REL{}.7{} 1\SM{} \PST{}3{} buy \\
%%TRANS1
\glt
`Moi j'avais vu la première véhicule qu'il a acheté, [..]' \\ `I myself saw the first vehicle he bought, [...] \jambox*{[PM 1084] }
%%TRANS2
%%EXEND

 \z 
 
 %%EAX
\ea
%%JUDGEMENT
%%LABEL
\label{relativeMartinfood}
%%CONTEXT
%%LINE1
\glll
{\db}(nɛɔfɛ́nɛ) Mə́tinə a ná \textbf{bɛlábɔ́nɛ́á} \textbf{bikimə} $\{$?ɔkɔlɔkɛn$\}$ \textbf{ɔbɛ́á} \textbf{yamíá} \textbf{inyə́} \textbf{a} \textbf{\ds{}ná} \textbf{tálɛ́áká} naánɛkɔla $\{$ɔkɔlɔkɛn(a)$\}$ (nɛɔfɛ́n). \\
%%LINE2
/(nɛɔfɛ́nɛ) Mə́tinə a ná bɛ-lábɔ́nɛ́á bɛ́-kimə $\{$ɔkɔlɔkɛna$\}$ ɔbɛ́á yamíá inyə́ a \textsuperscript{L}ná tálɛ́á-aka naánɛkɔla $\{$ɔkɔlɔkɛna$\}$ (nɛɔfɛ́nɛ)/ \\
%%LINE3
{\db}(today) 1.Martin 1\SM{} \PST{}2{} 8-food 8-all $\{$taste$\}$ \REL{}.8{} 9{}.\POSS{}.\PRO{}.1\SG{} 9.mother 1\SM{} \PST{}3.\REL{} cook-\DUR{} yesterday $\{$taste$\}$ (today) \\
%%TRANS1
\glt
`Martin a goûté (aujourd'hui) toute la nourriture que ma mère a cuisiné hier.' \\ `(Today), Martin has tasted all the food that my mother cooked yesterday.' \jambox*{[PM 498] } 
%%TRANS2
%%EXEND

 \z

%%EAX
\ea
%%JUDGEMENT
%%LABEL
\label{relativeguyfell}
%%CONTEXT
%%LINE1
\glll
{\db}ba l(ɛ) utíbíniə \textbf{ɛbɔ́ka} \textbf{ɔyɛ́á} \textbf{mwití} \textbf{a} \textbf{ná} \textbf{fálɛ́}. \\
%%LINE2
/ba lɛa ɔ-tíbíniə ɛ-bɔ́ka ɔyɛ́á mwití a ná fálɛ́á/ \\
%%LINE3
{\db}2\SM{} be \INF{}-observe 7-place \REL{}.7{} \PRO.\OBJ.1 1\SM{} \PST{}2{} tumble \\
%%TRANS1
\glt
`Ils sont en train d'observer l'endroit du la personne a degringolé.' \\ `They're looking at the place the guy fell.'\jambox*{[PM 582] }
%%TRANS2
%%EXEND

\z
 
I suggest that the variability in attachment of the relative clause modifying the object is related to independent factors such as prosodic weight and processing ease. As I am not aware of any influence of information structure on this variation, I leave the topic aside here.


\subsection{Fronting}
As noted in \citet{Mous1997}, another means in which word order may vary for information-structural reasons is by fronting a constituent, i.e. placing it at the beginning of the sentence. This will be discussed as a type of topic expression strategy in \sectref{sectopics}. In focus contexts, most apparently fronted constituents are in fact clefted, although some examples are found without a copula or relative marking, as in \xref{howmanyfronted} below. Recall as well that we saw in \sectref{secwordorder} above that questions are formed by fronting or clefting.

%%EAX
\ea
%%JUDGEMENT
%%LABEL
\label{howmanyfronted}
%%CONTEXT
(`How many children do you see?') \\ 
%%LINE1
\glll
{\db}\textbf{mɔná ɔmɔtɛ} mɛ́ ndɔ sin. \\
%%LINE2
/mɔ-ná ɔ́-mɔtɛ́ mɛ \textsuperscript{H}ndɔ sinə/ \\
%%LINE3
{\db}1-child 1-one \SM{}.1\SG{} \PRS{} see \\
%%TRANS1
\glt
`Je vois [un]\textsubscript{\FOC{}} seul enfant.' \\ `I see [one]\textsubscript{\FOC{}} child.' \jambox*{[DM 147] }
%%TRANS2
%%EXEND

\z

In some cases, a focused object may appear to be simply fronted, but further analysis shows evidence of an underlying cleft structure -- specifically a reverse pseudocleft -- which may be obscured by vowel elision or be ambiguous due to the noun class and tense marker. An example is in the object focus example in \xref{carrotcleft} below, where the H tone on the subject marker shows a dependent clause environment, with the copula \textit{á} analysable as elided due to vowel elision (see \sectref{secclefts} on clefts for more detail).\largerpage

%%EAX
\ea
%%JUDGEMENT
%%LABEL
\label{carrotcleft}
%%CONTEXT
(`What is the man holding in his hand?') \\ 
%%LINE1
\glll
{\db}kalɔ́tɔ \textbf{á} ná itíə́ ɔ mɔkat. \\
%%LINE2
/kalɔ́tɔ á-á ná itíə́ ɔ mɔ-kátá/ \\
%%LINE3
{\db}9.carrot \COP{}-1\SM{}.\REL{} \PST{}2{} hold \PREP{} 3-hand \\
%%TRANS1
\glt
`C'est [une carotte]\textsubscript{\FOC{}} qu'il tient à la main.' \\ `He is holding [a carrot]\textsubscript{\FOC{}} in his hand.' \jambox*{[JO 1630] }
%%TRANS2
%%EXEND

\z

Such ex-situ focus constructions will be covered in more detail in \sectref{secclefts} below. Fronted topic phrases are covered in more detail in \sectref{sectopics} on topic expression.

While contrast at the sub-DP level (i.e. on a modifier of the noun) does not require any special marking and can be left in-situ, it can also be expressed by fronting. Example \xref{nomarkcontrast} shows unmarked contrast between adjectival modifiers, while \xref{bigcalafrontset} shows that contrastive focus at the sub-DP level can alternatively be expressed by fronting the modifier together with the non-contrasted noun. Such fronting is optional, as the noun phrase can be left in the canonical position, an instance of the canonical S-Aux-O-V-X word order \xref{bigcalainsitu}, or be discontinuous \xref{bigcaladiscon}.

%%EAX
\ea
%%JUDGEMENT
%%LABEL
\label{nomarkcontrast}
%%CONTEXT
%%LINE1
\glll
{\db}ɔ	iNdíkiə	\textbf{nioní}	\textbf{nɛtɛ́\ds{}tɛ́}	nɛ-bɔkɔyiilə tɛ́á,	\textbf{nioní}	\textbf{nɛŋɛ́ŋa}	ɔ	ninúmbə́	(tɛ́). \\
%%LINE2
/ɔ	iNdíkiə	nɛ-oní	nɛ-tɛ́\textsuperscript{L}tɛ́á	nɛ-bɔkɔyiilə tɛ́	nɛ-oní	nɛ-ŋɛ́ŋa	ɔ	nɛ-númbə́	(tɛ́á)/ \\
%%LINE3
{\db}\PREP{}	Ndiki	5-market	5-small	5-Wednesday every,	5-market	5-big	\PREP{}	5-Saturday	(every) \\
%%TRANS1
\glt
`À Ndiki il y a un petit marché chaque mercredi et un grand marché (chaque) samedi.' \\ `In Ndiki, there is a small market every Wednesday and a large market every Saturday.' \jambox*{[PM 193] }
%%TRANS2
%%EXEND

\z

\ea \label{bigcalafrontset}
%%EAX
\ea
%%JUDGEMENT
%%LABEL
\label{bigcalafront}
%%CONTEXT
%%LINE1
\glll
{\db}bɔ́ɔ,	\textbf{imítə́}	\textbf{mɛŋɛ́ŋa}	mɛ ná	índíə	ɔ hɛlɔ́bat,	tátá	ɔ	mɛ́tɛ́\ds{}tɛ́. \\
%%LINE2
/bɔ́ɔ	ɛ-mítə́	 mɛŋɛ́ŋa	mɛ ná	índíə́	ɔ hɛ-lɔ́bátɔ	tátá	ɔ	mɛ́tɛ́\textsuperscript{L}tɛ́a/ \\
%%LINE3
{\db}no	9-calabash	9.big	\SM{}.1\SG{} \PST{}2{}	give	\PREP{} 19-child	\COP{}.\NEG{}	\PREP{}	 9.small \\
%%TRANS1
\glt
`Non, c'est la grande calebasse que j'ai donné à l'enfant, pas la petite.' \\ `No, I gave the big calabash to the child, not the small one.' \jambox*{[EE+EB 1832] }
%%TRANS2
%%EXEND

\ex \label{bigcalainsitu}
% \gl
bɔ́ɔ,	mɛ ná	\textbf{imítə́}	\textbf{mɛŋɛ́ŋ}	indiə	ɔ hɛlɔ́bátɔ,	(tátá	ɔ	mɛ́\ds{}tɛ́tɛ́). \jambox*{[EE+EB 1834] }
% /bɔ́ɔ	mɛ-ná	\textbf{ɛ-mítə́}	\textbf{mɛŋɛ́ŋa}	indiə	ɔ hɛ-lɔ́bátɔ	tátá	ɔ	mɛ̂tɛ́tɛ́a/ \\
% no	\SM{}.1\SG{}-\PST{}2{}	7-calabash	big	give	\PREP{} 19-child	\COP{}.\NEG{}	\PREP{}	small \\
%\glt `Non, j'ai donné la grande calebasse à l'enfant, (pas la petite).' \\ `No, I gave the big calabash to the child, (not the small one).' \jambox*{[EE+EB, 1834] }

\ex \label{bigcaladiscon}

% \gl
bɔ́ɔ,	mɛ ná	\textbf{imítə́}	indiə	\textbf{mɛŋɛ́ŋ}	ɔ hɛlɔ́bátɔ,	(tátá	ɔ	mɛ́\ds{}tɛ́tɛ́). \jambox*{[EE+EB 1833] }
% /bɔ́ɔ	mɛ-ná	\textbf{ɛ-mítə́}	indiə	\textbf{mɛŋɛ́ŋa}	ɔ hɛ-lɔ́bátɔ	tátá	ɔ	mɛ̂tɛ́tɛ́a/ \\
% no	\SM{}.1\SG{}-\PST{}2{}	7-calabash	give	big	\PREP{} 19-child	\COP{}.\NEG{}	\PREP{}	small \\
%\glt `Non, j'ai donné la grande calebasse à l'enfant, (pas la petite).' \\ `No, I gave the big calabash to the child, (not the small one).' \jambox*{[EE+EB, 1833] }
\z
\z

Note here that the adjective cannot be fronted without the noun; the noun must be pied-piped. This means that term focus on the sub-DP level is marked in the same way as term focus scoping over the entire DP.


\subsubsection{The right periphery}
As is common crosslinguistically, the right periphery is used for afterthoughts or repairs, as in the natural speech example in \xref{squirrelmaybemole} below, where an alternative noun is added as a suggestion for the subject.

%%EAX
\ea
%%JUDGEMENT
%%LABEL
\label{squirrelmaybemole}
%%CONTEXT
Context: PM and EO perform the QUIS map task \citep[155--157]{SkopeteasEtAl2006}, where PM must give EO directions using a map with various objects drawn on it. \\
%%LINE1
\glll
{\db}hɛkɔlɛ hɛ́ ka báká hə́ní u busí \textbf{káasɛ} \textbf{himondokóloŋ}. \\
%%LINE2
/hɛ-kɔlɛ hɛ́ ka bá-aka hə́níə́ ɔ busíə́ káasɛ hɛ-mondokóloŋo/  \\
%%LINE3
{\db}19-squirrel 19\SM{} \AND{} be-\DUR{} \DEM{}.\DIST{}.\LOC{} \PREP{} 14.front maybe 19-mole \\
%%TRANS1
\glt
`Là-bas il y a un écureuil, ou peut-être une taupe.' \\ `There's a squirrel there, or maybe a mole.' \jambox*{[PM 707] }
%%TRANS2
%%EXEND

\z

Further investigation of fronting and the right periphery could be done on the basis of a larger text corpus; in my field data, neither strategy was very commonly found. Instead of fronting, focus is typically expressed by the canonical word order (for non-subjects; \sectref{secwordorder}, \sectref{secnoncanonwordorder}) or else by clefting (\sectref{secclefts} below).

\subsection{Section summary}
Although S-Aux-O-V-X is the canonical word order, we saw in this section that other word order patterns are found in Tunen.  Objects may appear postverbally in certain contexts, often with a modifier or relative (although discontinuous structures are more common). In contrast to the presentation in \citet{Mous1997, Mous2003}, I argued that postverbal objects preceded by the marker \textit{á} are uncommon, although they show evidence for monoclausality. Aside from clefts, fronting is another possible strategy for focus expression, although this strategy is less commonly used for foci. Fronting for topics will be covered further in \sectref{sectopics}. Some modifiers are frequently discontinuous in Tunen; this is a pragmatically neutral word order pattern rather than a particular strategy for focussing the modifier, unlike what is found for other languages with (apparent) discontinuity in the nominal domain, and unusually for a Bantu language. Finally, the right periphery can be used for afterthoughts, as is common crosslinguistically.


\section{Clefts and the marker \textit{á}}\label{secclefts}
We have seen already that Tunen can use clefts to express focus, which is a common strategy for question formation and found also with declaratives. This section discusses these cleft constructions and their interpretation. As clefts are composed of a copula, focused NP, and a relative clause component \citep{HarrisCampbell1995a}, I begin by describing the form of copular clauses in Tunen, before looking into clefts specifically. 

\subsection{Copular clauses in Tunen}
A common typology of copular clauses is to split them into four types: identificational, predicational, specificational, and equative copular clauses (see e.g. \citealt{Higgins1979, Mikkelsen2011, Heycock2012}). In this section I show that Tunen does not differentiate between identificational and specificational copular clauses, and shows no evidence for equative copular clauses as a distinct class, and so the typology can be simplified as predicational vs. identificational/specificational copular clauses.\footnote{I refer to the second type as “identificational\slash specificational” in order to remain agnostic as to whether the identificational or the specificational copular clause is the most basic/general type.}

Firstly, consider predicational copular clauses, where a property is assigned to a referent. Predicational copular clauses in Tunen are formed with the copula \textit{lɛa} `be' or copula verb \textit{bá(ka)} `be' -- which are generally interchangeable (\citealp[347--350]{Dugast1971}, \citealp[124--125]{KerrFut}) -- as illustrated in \xref{copwaterclean} and \xref{copatmarket} below for a non-locative \xref{copwaterclean} and locative use respectively.
 
%%EAX
\ea
%%JUDGEMENT
%%LABEL
\label{copwaterclean}
%%CONTEXT
(Is the water clean for drinking?) \\
%%LINE1
\glll
{\db}bɔ́ɔ,	má lɛ́ \textbf{bá}	mas. \\
%%LINE2
/bɔ́ɔ	má lɛ bá	ma-ɛsɛ/  \\
%%LINE3
{\db}no	6{}\SM{} \NEG{} be	6-good \\
%%TRANS1
\glt
`Non, ce n'est pas pure.' `No, it isn't potable.', `No, it isn't clean.' \jambox*{[JO 612]}
%%TRANS2
%%EXEND


%%EAX
\ex
%%JUDGEMENT
%%LABEL
\label{copatmarket}
%%CONTEXT
(Where are you?) \\
%%LINE1
\glll 
{\db}mɛ \textbf{lɛ}	ɔ	nioní. \\
%%LINE2
/mɛ lɛa ɔ nɛ-oní/ \\
%%LINE3
{\db}\SM{}.1\SG{} be \PREP{} 5-market \\
%%TRANS1
\glt
`Je suis au marché.' \quad `I am at the market.' \jambox*{[PM 102]}
%%TRANS2
%%EXEND

  
 \z\largerpage[2]

Identificational copular clauses, on the other hand, are marked by \textit{á} \COP{} in Tunen. Example \xref{identificationalcop1} below shows the use of \textit{á} as the copula in a clause which identifies a referent.

 %%EAX
\ea
%%JUDGEMENT
%%LABEL
\label{identificationalcop1}
%%CONTEXT
%%LINE1
\glll
{\db}wɛ́ɛyɛ mɔndɔ wɛ́ɛyɛ ɔwá tɔ́ \ds{}ná siəkinə, \textbf{á} mutíkə wa bɔnɔŋɔ bɔ́ iNdikinímɛ́ki(ə). \\
%%LINE2
/wɛ́ɛyɛ mɔ-ndɔ wɛ́ɛyɛ ɔwá tɔ́ \textsuperscript{L}ná  siəkinə á mɔ-tíkə wa bɔ-nɔŋɔ bɔ́ iNdikinímɛkiə/ \\
%%LINE3
{\db}\DEM{}.\DISC{}.1{} 1-person \DEM{}.\DISC{}.1{} \REL{}.1{} \SM{}.1\PL{} \PST{}3.\REL{} see.\DUR{} \COP{} 1-mayor \ASS{}.1{} 14-country \ASS{}.14{} Ndikiniméki \\
%%TRANS1
\glt
‘Cet homme-là que nous avons vu (hier), c’est le maire du Ndikiniméki.' \\ `That man there that we saw (yesterday) is the mayor of Ndikiniméki.' \jambox*{[PM 780] }
%%TRANS2
%%EXEND

  
 \z

Specificational copular clauses are defined as having the structure \textit{A is B}, where A is typically non-referential and B is referential, and A is definite \citep{Heycock2012}. These are also marked by \textit{á} \COP{} in Tunen \xref{specificationalcop1}.

%%EAX
\ea
%%JUDGEMENT
%%LABEL
\label{specificationalcop1}
%%CONTEXT
%%LINE1
\glll
{\db}mɔná	ɔwá	á lɛ́á	na ɛmanya tɔ́mbálánátɔ	\textbf{á} Patiáns.\\
%%LINE2
/mɔ-ná	ɔwá	á lɛ́á	na	ɛ-manya tɔ́mbálánátɔ	á	Patiánsɛ/\\
%%LINE3
{\db}1-child	\REL{}.1{}	1\SM{}.\REL{} be	with	7-knowledge surpass.\PTCP{}	\COP{}	1.Patience\\
%%TRANS1
\glt
`L’enfant qui est le plus intelligent, c’est Patience.’ \\ `The smartest child is Patience.' \jambox*{[JO 854] }
%%TRANS2
%%EXEND
  
 \z

The final type of copular clause proposed in the literature on copular clauses is equatives, where A is said to be identical to B (e.g. ``The morning star is the evening star'' in English). Whether or not equatives are truly a distinct class is subject to some debate (see e.g. \citealt{Heycock2012}). When eliciting such examples in Tunen, consultants either rephrased the construction by using lexical verb (e.g. ``A gives B'' ) or used a specificational copula with \textit{á} \COP{}. The only possible example of a true equative is in the story below, which can either be analysed as a fragment or an instance of \textit{á} \COP{} (if the form \textit{miaŋɔ́á} `me' is taken to include \textit{á}; cf. \citealt{Dugast1971}). There is therefore no convincing evidence to identify a separate  equatives subclass of copulars in Tunen.\largerpage[3]

%\todo{check gloss align}
%%EAX
\ea
%%JUDGEMENT
%%LABEL
%%CONTEXT
Context: A shepherd lied/cried wolf that there was a panther. His concerned neighbours ran over... \\
%%LINE1
\glll
{\db}bá \ds{}ná ka fam, a ná sanɛ́á ɔ tuɔn, {asɛ́á :} {« \textbf{miaŋɔ́á}} {\textbf{mɛkɔ} ! »}. \\
%%LINE2
/bá \textsuperscript{L}ná ka fámá a ná sanɛ́á ɔ tɔ-ɔnɔ a-sɛ́á  {\db}miaŋɔ́á mɛ-kɔ/ \\
%%LINE3
{\db}2\SM{} \PST{}3.\DEP{} \AND{} arrive 1\SM{} \PST{}2{} burst.out \PREP{} 13-laughter \SM{}.1-say { }\PRO{}.\EMPH{}.1\SG{} 9-panther \\
%%TRANS1
\glt
`Quand ils sont arrivés, il a éclaté de rire, il a dit, «\,c'est moi la panthère\,!\,»' \quad `When they arrived, he burst out laughing and said `I'm the panther!"' [JO 2033]
%%TRANS2
%%EXEND


\z

In summary, Tunen forms identificational/specificational copular clauses differently from predicational clauses, as shown in \tabref{tbl1} below: predicational copular clauses use the verbs \textit{lɛa} and \textit{bá(ka)} `to be', while identificational and specificational clauses use \textit{á}.

\begin{table}[ht]
% \centering
\begin{tabularx}{0.75\textwidth}{Xl}
\lsptoprule
Copular clause type & Copula element \\
\midrule
Predicational & -\textit{lɛa} / -\textit{bá(ka)} `to be'\\
Identificational/specificational & \textit{á} \\
\lspbottomrule
\end{tabularx}
\caption{Copular clauses in Tunen}
\label{tbl1}
\end{table}

The predicational copula \textit{lɛa} and \textit{bá(ka)} take a subject marker and are negated by a negative marker (as seen in \xref{copwaterclean}). In contrast to these copula forms, the identificational/specifica\-tional copula \textit{á} is invariant\footnote{There is some indication of a human/non-human distinction with a \textit{ɔ́} variant used for non-personified non-human animates and inanimates; see \citet[120]{KerrFut} for further detail.\label{footnote18}} and has a negative form \textit{tátá}, glossed as \COP{}.\NEG{} \xref{twoplusthree}.

%%EAX
\ea
%%JUDGEMENT
%%LABEL
\label{twoplusthree}
%%CONTEXT
%%LINE1
\glll
{\db}(bɔ́ɔ,) bɛ́fandɛ́ kɔndá bɛ́lálɔ́ \textbf{tátá} bɛ́lɛndálɔ. \\
%%LINE2
/(bɔ́ɔ) bɛ́-fandɛ́ kɔndá bɛ́-lálɔ́ tátá bɛ́-lɛ́ndálɔ/ \\
%%LINE3
{\db}(no) 8-two add 8-three \COP{}.\NEG{} 8-six \\
%%TRANS1
\glt
`Non, deux plus trois ne font pas six.' \\ `No, two plus three doesn't equal six.' \jambox*{[PM 784] }
%%TRANS2
%%EXEND

\z

Now we have seen that \textit{á} \COP{} is used for identificational/specificational copulars in Tunen, we can consider clefts, which I show contain \textit{á} as a copular component, matching the common crosslinguistic pattern of identificational/specifica\-tional copular elements in clefts.

\subsection{Relativisation}
The next component of a cleft is a relative clause. Relative clauses are identified in Tunen by (i) a relativiser of form \textit{ɔXá}, where the shape of X depends on the noun class of the head noun, (ii) H-tone on normally L-toned subject markers, and (iii) dependent-clause tense marking, as visible in the third-degree past tense (\PST{}3{}) and in negative clauses. For example, the object relative example in \xref{relativeinyas} below shows the main clause third-degree past tense marker \textit{ka} followed by the dependent third-degree past tense marker \textit{\ds{}ná} in the relative clause, as well as high tone on the class 1 subject marker \textit{á} in the relative clause, contrasting with the low-toned main clause first person singular subject marker \textit{mɛ}.

%%EAX
\ea
%%JUDGEMENT
%%LABEL
\label{relativeinyas}
%%CONTEXT
%%LINE1
\glll
{\db}mɛ \textbf{ka} ámɛ yáyɛ́á ibəŋuluəkə yɛ́ búsíə́ siəkinə \textbf{ɔyɛ́á} \textbf{á} \textbf{\ds{}ná} ɔnd, [...] \\
%%LINE2
/mɛ ka ámɛ yáyɛ́á ɛ-bəŋuluəkə yɛ́ búsíə́ siəkinə ɔyɛ́á á \textsuperscript{L}ná ɔ́ndɔ/ \\
%%LINE3
{\db}\SM{}.1\SG{} \PST{}3{} \PRO{}.1\SG{} 7{}.\PRO{}.\POSS{}.1{} 7-car \ASS{}.7{} front see.\DUR{} \REL{}.7{} 1\SM{}.\REL{} \PST{}3.\REL{} buy \\
%%TRANS1
\glt
`Moi j'avais vu le premier véhicule qu'il a acheté, [...]' \\ `I myself saw the first vehicle he bought,' [...] \jambox*{[PM 1045] }
%%TRANS2
%%EXEND

\z

In Tunen clefts, relatives are reduced in the sense of lacking the \textit{ɔXá} relativiser. While there is no overt relativiser and while non-human noun classes and many TAM contexts have identical marking to main clauses, marking of a relative clause can still be seen by H-tone on underlyingly L-toned subject markers and the use of dependent TAM markers in third-degree past tense and negative contexts. For example, the following example provides evidence for there being a reduced relative in a Tunen cleft, as the third-degree past tense marker must be the dependent clause form \textit{\ds{}ná} instead of the main clause affirmative form \textit{ka} \xref{ex:whichmomentbarkq}. This provides evidence for a relative clause environment despite the lack of an overt relativiser.

%%EAX
\ea
%%JUDGEMENT
%%LABEL
\label{ex:whichmomentbarkq}
%%CONTEXT
%%LINE1
\glll
{\db}ɔ́	yə́níə́	ikúílí	á	ɛmɔ́á	yɛ́ \textbf{$\{$\ds{}ná|*ka}$\}$ bɔmɔkɔ	mɔŋɛŋa	eé? \\
%%LINE2
/ɔ	yə́níə́	ɛ-kúílí	á	ɛ-mɔ́á	yɛ́ $\{$\textsuperscript{L}ná|*ka$\}$ bɔmɔ-aka	mɔŋɛ́ŋa	eé/ \\
%%LINE3
{\db}\PREP{}	which	7-time	\COP{}	7-dog	7\SM{} $\{$\PST{}3.\REL{}|*\PST{}3{}$\}$ bark-\DUR{}	much	\Q{} \\
%%TRANS1
\glt
`A quel moment le chien a-t-il beaucoup aboyé ?'\\ `When did the dog bark a lot?' \jambox*{[PM, 1255--6] }
%%TRANS2
%%EXEND


\z

As non-human noun classes have H-toned subject markers in both dependent and main clauses, and as there is only a visible difference in tense marking in affirmatives in the third-degree past tense, many examples of clefts with \textit{á} are in fact ambiguous between the biclausal or monoclausal analysis. There is likely a change in progress between the biclausal and monoclausal structures, as discussed for different languages in \citet{HarrisCampbell1995a}. As the Tunen \textit{á} marker is likely in the process of grammaticalising to being a focus marker in a monoclausal construction, the most accurate gloss is debatable. While the presence of \textit{á} may be obscured due to vowel elision, my consultants indicated that there was a \textit{á} underlying even if it was elided on the surface, and therefore I maintain the copular analysis of \textit{á} in the glossing in this chapter and gloss it as \COP{}. I turn now to the different cleft constructions found in Tunen.

\subsection{Clefts}

Clefts are obligatory in Tunen for subject focus (which cannot be focused in-situ) and are used across all grammatical roles for exhaustive focus. There are two main forms of cleft in Tunen. In this section we will see that there is a basic cleft used for subject focus and a reverse pseudocleft used for non-subject focus. I argued in \sectref{subsec:Vao} above that the V\textit{á}O construction that resembles a pseudocleft instead shows monoclausal properties, and therefore do not include it here.

\subsubsection{Basic cleft}
Human animate subjects must be focused with a basic cleft construction, as schematised in \xref{subjcleft} below.\footnote{At this point, it is unclear whether the primary conditioning factor for the use of a basic cleft is subjecthood or humanness, given that most examples in the data of subjects are either human or personified animals. The discussion in this chapter should therefore be taken to apply to the prototypical human subject, with the potential role of animacy on cleft structure a question for further research.\label{footnote19}}

\ea \label{subjcleft}
Basic cleft: \\ \textit{á} + NP\textsubscript{\FOC{}} + reduced relative
\z

Example \xref{subjfocuswhoshutdoor} below shows the use of the basic cleft to express subject focus. The identificational/specificational copula \textit{á} is used, followed by the focused noun phrase \textit{Píɛ́l} `Pierre' and then a reduced relative. The reduced relative clause environment is recognisable due to the high tone on the class 1 subject marker \textit{á}, which is low-toned in main clause environments. Clefting the subject is obligatory in this context; leaving the subject in-situ is not felicitous (though would be grammatical in a thetic context)\footnote{The original fieldnotes for this form kept the H tone on the class 1 subject marker; based on other examples, I report the judgement with the subject marker appearing as in a thetic context.} \xref{subjfocuswhoshutdoorbad}.

\largerpage[-1]
\pagebreak
\ea \label{subjfocuswhoshutdoor} (Who shut the door?'') \\ 
%%EAX
\ea
%%JUDGEMENT
[]{
%%LABEL
%%CONTEXT
%%LINE1
\glll
{\db}\textbf{á} Píɛ́l \textbf{á} ná nikí kwiyí. \\
%%LINE2
/á Píɛ́lɛ á ná nɛ-kí kwiyíə/ \\
%%LINE3
{\db}\COP{} 1.Pierre 1\SM{}.\REL{} \PST{}2{} 5-door shut \\
%%TRANS1
\glt
`C'est [Pierre]\textsubscript{\FOC{}} qui a fermé la porte.' \\ `[Pierre]\textsubscript{\FOC{}} shut the door.' \jambox*{[EO 273]}
%%TRANS2
}
%%EXEND

%%EAX
\ex
%%JUDGEMENT
[\textsuperscript{$\#$}]{
%%LABEL
\label{subjfocuswhoshutdoorbad}
%%CONTEXT
%%LINE1
\glll
{\db}Píɛ́l a ná nikí kwiyí. \\
%%LINE2
/Píɛ́lɛ a ná nɛ-kí kwiyíə/ \\
%%LINE3
{\db}1.Pierre 1\SM{} \PST{}2{} 5-door shut \\
%%TRANS1
\glt
int. `[Pierre]\textsubscript{\FOC{}} a fermé la porte.' \\ int. `[Pierre]\textsubscript{\FOC{}} shut the door.' \jambox*{[EO 277] }
%%TRANS2
}
%%EXEND

\z
\z

Note that fragment answers also require the \textit{á} for subject focus \xref{ex:whichpoldiefrag}, suggesting that they are elided from an underlying cleft structure.

%%EAX
\ea
%%JUDGEMENT
%%LABEL
\label{ex:whichpoldiefrag}
%%CONTEXT
(Which politician died?) \\ 
%%LINE1
\glll
{\db}\textbf{*(á)} Píɛlə (á ná wə). \\
%%LINE2
/á Piɛlə á ná wə́./ \\
%%LINE3
{\db}\COP{} Pierre 1\SM{}.\REL{} \PST{}2{} die \\
%%TRANS1
\glt
`C'était [Pierre]\textsubscript{\FOC{}} (qui est mort).' \\`It was [Pierre]\textsubscript{\FOC{}} (who died).' \jambox*{[EO 270--1] }
%%TRANS2
%%EXEND

\z

As we will see for other arguments below, focused XPs in clefts are typically said to have an exhaustive interpretation. This is illustrated for subject focus below with the continuation `not another' in \xref{exhshutdoor} and the confirmation from the speakers of \xref{eatrice} that nobody else could have eaten the rice.\footnote{Note that the subject marker in \xref{eatrice} and in a few other examples in this chapter is low-toned, while we would expect a high tone in a relative clause environment. Such low tones could either be indication of the development from a biclausal to a monoclausal structure, or be related to a methodological issue of repeating transcriptions word-for-word, in which case consultants may have simply used the low-toned citation form of the subject marker when repeating the utterance, despite pronouncing it as a high in this context in fluent speech.}

%%EAX
\ea
%%JUDGEMENT
%%LABEL
\label{exhshutdoor}
%%CONTEXT
%%LINE1
\glll
{\db}miaŋɔ́á á mɛ ná nikí kwiyí, tátá mɔnə́munə́. \\
%%LINE2
/miaŋɔ́á á mɛ́ ná nɛ-ki kwiyíə, tátá mɔ-nə́munə́/ \\
%%LINE3
{\db}\PRO{}.\EMPH{}.1\SG{} \COP{} \SM{}.1\SG{}.\REL{} \PST{}2{} 5-door shut not 1-another \\
%%TRANS1
\glt
`C'est [moi]\textsubscript{\FOC{}} qui a fermé la porte, pas un autre.' \\ `It was [me]\textsubscript{\FOC{}} who shut the door, not someone else.' \jambox*{[EO 274] }
%%TRANS2
%%EXEND

\z


%%EAX
\ea
%%JUDGEMENT
%%LABEL
\label{eatrice}
%%CONTEXT
%%LINE1
\glll
{\db}á	Samuɛ́lɛ	a ná	ɔlɛ́sa	nɛak. \\
%%LINE2
/á	Samuɛ́lɛ	á ná	ɔ-lɛ́sa	nɛ́á-aka/ \\
%%LINE3
{\db}\COP{}	1.Samuel	1\SM{}.\REL{} \PST{}2{}	3-rice	eat-\DUR{} \\
%%TRANS1
\glt
`C'est [Samuel]\textsubscript{\FOC{}} qui a mangé du riz.' (pas quelqu'un d'autre) \\ `[Samuel]\textsubscript{\FOC{}} ate rice.' (it wasn't somebody else) \jambox*{[EE + EB 1661] }
%%TRANS2
%%EXEND

\z 

That being said, the question arises as to how non-exhaustive focus is expressed for subjects in Tunen. One piece of data suggesting that clefted subjects are not neccesarily exhaustive is \xref{exhevenNancy} below, where the marker \textit{á} appears after the exclusive particle \textit{ata} `even' modifying Nancy, in a context where Mary also has a bottle of water.

%%EAX
\ea
%%JUDGEMENT
%%LABEL
\label{exhevenNancy}
%%CONTEXT
(Does Maria have a bottle of water? (+ BaSIS photo stimulus)) \\
%%LINE1
\glll
{\db}ɛ́ɛ Maliá a báka na məndíŋgə wɔ́ mə́nif, \textbf{ata} \textbf{*(á)} Nansí $\{$\textbf{tɔ́na}$\}$ a báka $\{$\textbf{tɔ́na}$\}$ na məndíŋgə wɔ́ mə́nif. \\
%%LINE2
/ɛ́ɛ Maliá a bá-aka na mɔ-ndíŋgə wɔ́ ma-nífə́ ata á Nansí tɔ́na a bá-aka tɔ́na na mɔ-ndíŋgə wɔ́ ma-nífə́/ \\
%%LINE3
{\db}yes 1.Maria 1\SM{} be-\DUR{} with 3-bottle \ASS{}.3 6-water even \COP{} 1.Nancy also 1\SM{} be-\DUR{} also with 3-bottle \ASS{}.3 6-water \\
%%TRANS1
\glt
`Oui, Maria a une bouteille de l'eau, Nancy a une bouteille de l'eau aussi.' \\ `Yes, Maria has a bottle of water, Nancy also has a bottle of water.' \jambox*{[JO 2347] }
%%TRANS2
%%EXEND

\z

It therefore seems that exhaustivity is compatible with a cleft structure and is often understood pragmatically, but strictly speaking the basic cleft is not exhaustive, as it can be used in non-exhaustive contexts \xref{exhevenNancy}.


\subsubsection{Reverse pseudo-clefts} \label{subsec:revpseudo}
Non-subjects can be focused with a reverse pseudo-cleft construction, which takes the form schematised in \xref{reversepseudocleftform}.

\ea \label{reversepseudocleftform}
Reverse pseudo-cleft \\
NP\textsubscript{\FOC{}} + \textit{á} + reduced relative
\z

This is illustrated below for information focus \xref{carrotholdclefft}, \xref{cookppcleft} and corrective focus \xref{correctiveplantainsonly} on the theme object.%
\largerpage[-1]\pagebreak

%%EAX
\ea
%%JUDGEMENT
%%LABEL
\label{carrotholdclefft}
%%CONTEXT
(What is the man holding in his hand?) \\ 
%%LINE1
\glll
{\db}\textbf{kalɔtɔ} \textbf{á} mɔndɔ a ná itíə́ ɔ mɔkata \\
%%LINE2
/kalɔtɔ á mɔ-ndɔ a ná itíə́ ɔ mɔ-kátá/ \\
%%LINE3
{\db}9.carrot \COP{} 1-person  1\SM{} \PST{}2{} hold \PREP{} 3-hand \\
%%TRANS1
\glt
`C'est [une carotte]\textsubscript{\FOC{}} que l'homme tient dans sa main.' \\ `The man is holding [a carrot]\textsubscript{\FOC{}} in his hand.', `[A carrot]\textsubscript{\FOC{}} is what the man is holding.'  \jambox*{[JO 1624] }
%%TRANS2
%%EXEND
\z
%%EAX
\ea
%%JUDGEMENT
%%LABEL
\label{cookppcleft}
%%CONTEXT
(What will you cook today?) \\ 
%%LINE1
\glll
{\db}\textbf{mɔkɔnd͡ʒɛ} \textbf{na} \textbf{mɛkɔnɛ́fɛ́} \textbf{á} mɛ́ ndɔ talɛak. \\
%%LINE2
/mɔkɔnd͡ʒɛ na mɛkɔnɛ́fɛ́ á mɛ́ \textsuperscript{H}ndɔ talɛ́á-aka/ \\
%%LINE3
{\db}6.plantain with 6.pork \COP{} \SM{}.1\SG{}.\REL{} \PRS{} cook-\DUR{} \\
%%TRANS1
\glt
`C'est [les plantains et le porc]\textsubscript{\FOC{}} que je vais cuisiner (aujourd'hui).' \\ `I will cook [plantains and pork]\textsubscript{\FOC{}} today.', `Plantains and pork are what I will cook (today).' \jambox*{[PM 1512 (+ JO 1602)] }
%%TRANS2
%%EXEND

\z

%%EAX
\ea
%%JUDGEMENT
%%LABEL
\label{correctiveplantainsonly}
%%CONTEXT
%%LINE1
\glll
{\db}bɔ́ɔ, \textbf{mɔkɔnd͡ʒɛ} \textbf{na} \textbf{mɛkɔnɛ́fɛ} \textbf{á} mɛ́ ndɔ talɛaka nɛɔfɛ́n. \\
%%LINE2
/bɔɔ mɔkɔnd͡ʒɛ na mɛkɔnɛ́fɛ́ á mɛ́ \textsuperscript{H}ndɔ talɛ́á-aka nɛɔfɛ́nɛ/ \\
%%LINE3
{\db}no 6.plantain with 6.pork \COP{} \SM{}.1\SG{}.\REL{} \PRS{} cook-\DUR{} today \\
%%TRANS1
\glt
`Non, c'est [les plantains et le porc]\textsubscript{\FOC{}} que je vais préparer aujourd'hui.' (pas d'autres choses) \\ `No, I will cook [plantains and pork]\textsubscript{\FOC{}} today.', `No, plantains and pork are what I will cook today'. (and nothing else) \jambox*{[PM 1516] }
%%TRANS2
%%EXEND

\z

As the translation of \xref{correctiveplantainsonly} indicates, using a cleft construction suggests an exhaustive interpretation of the nominal, and is therefore more contrastive than the in-situ focus strategy. At this point, we may ask whether the same pattern is found as for subject clefts above, where compatibility with \textit{ata} `even' indicates that the cleft is not inherently exhaustive. We find that human objects modified by \textit{ata} `even'  can appear in a cleft, although of a basic cleft structure \xref{evenobjNat}, while non-human objects modified by \textit{ata} do not take \textit{á} \xref{evenobjshoes}, possibly due to difference in animacy.

\ea \label{evenobjset}
%%EAX
\ea
%%JUDGEMENT
%%LABEL
\label{evenobjNat}
%%CONTEXT
%%LINE1
\glll
{\db}\textbf{ata} *(\textbf{á}) Natanyɛ́lɛ mɛ ná siəkin. \\
%%LINE2
/ata á Natanyɛ́lɛ mɛ ná siəkinə/ \\
%%LINE3
{\db}even \COP{} 1.Nathaniel \SM{}.1\SG{} \PST{}2{} see.\DUR{} \\
%%TRANS1
\glt
J'ai vu même [Nathaniel]\textsubscript{\FOC{}}.' \\ `I even saw [Nathaniel]\textsubscript{\FOC{}}.' \jambox*{[PM 2276]}
%%TRANS2
%%EXEND

%%EAX
\ex
%%JUDGEMENT
%%LABEL
\label{evenobjshoes}
%%CONTEXT
%%LINE1
\glll
{\db}\textbf{ata}	\textbf{(*á)}	bɛtafɛna	Lídia	a ná	sɔák. \\
%%LINE2
/ata	(*á)	bɛ-tafɛna	Lídia	a ná	sɔ́á-aka/ \\
%%LINE3
{\db}even	\COP{}	8-shoe	1.Lydia	1\SM{} \PST{}2{}	wash-\DUR{} \\
%%TRANS1
\glt
`Lydia a lavé même [des chaussures]\textsubscript{\FOC{}}.' \\ `Lydia even washed [the shoes]\textsubscript{\FOC{}}.' \jambox*{[PM 2268] }
%%TRANS2
%%EXEND

\z
\z
 
Again then, it seems that the cleft constructions are not inherently exhaustive, although consultants generally interpret them as exhaustive and they are compatible with the exhaustive focus-sensitive particle \textit{ɔ́maná} `only', as seen already in \xref{onlyset1}, \xref{onlyset2}.

Reverse pseudo-clefts are not possible for subject focus, showing a subject\slash non-subject asymmetry (though recall the point in Footnotes~\ref{footnote18}--\ref{footnote19} above about animacy\slash humanness as a potential alternative factor):

%%EAX
\ea (Who shut the door?) \\
%%JUDGEMENT
% [*]{
%%LABEL
%%CONTEXT
%(Who shut the door?) \\
%%LINE1
\glll
*{\db}Píɛ́l á á ná nikí kwiyí. \\
%%LINE2
/Piɛlə á á ná nɛ-kí kwiyí/ \\
%%LINE3
{\db}Pierre \COP{} 1\SM{}.\REL{} \PST{}2{} 5-door shut \\
%%TRANS1
\glt
int. `C'est [Pierre]\textsubscript{\FOC{}} qui a fermé la porte.' \\ int. `It was Pierre who shut the door.' \jambox*{[EO 276] }
%%TRANS2
% }
%%EXEND

\z

However, a reverse pseudocleft rather was found for sub-DP focus on the modifier of a subject \xref{cleftdance}.

\ea \label{cleftdance}
%%EAX
\begin{xlist}
\exi{Q:}
%%JUDGEMENT
%%LABEL
%%CONTEXT
%%LINE1
\glll
{\db}bɛndɔ bá\ds{}nɛ́á á \ds{}bá ná binək? \\
%%LINE2
/bɛndɔ bá-nɛ́á á bá ná binə-aka/ \\
%%LINE3
{\db}2.person 2-how.many \COP{} 2\SM{} \PST{} dance-\DUR{} \\
%%TRANS1
\glt
`Combien de personnes ont dansé ?' \\ `How many people danced?' \jambox*{[PM 1211] }
%%TRANS2
\end{xlist}
%%EXEND

%%EAX
\begin{xlist}
\exi{A:}
%%JUDGEMENT
%%LABEL
%%CONTEXT
%%LINE1
\glll
{\db}bɛndɔ bálálɔ́ á bá ná binək. \\
%%LINE2
/bɛndɔ bá-lálɔ́ á bá ná binə-aka/ \\
%%LINE3
{\db}2.person 2-three \COP{} 2\SM{} \PST{}2{} dance-\DUR{} \\
%%TRANS1
\glt
`[Trois]\textsubscript{\FOC{}} personnes ont dansé.' \\ `[Three]\textsubscript{\FOC{}} people danced.' \jambox*{[PM 1214] }
%%TRANS2
\end{xlist}
%%EXEND
\z

Non-arguments pattern with objects in being found without \textit{á} preceding the focused XP in fragments, with the full version in the form of a reverse pseudocleft \xref{todayrevpseudocleft}, \xref{beansfragment}.

%%EAX
\ea
%%JUDGEMENT
%%LABEL
\label{todayrevpseudocleft}
%%CONTEXT
(Context: `Because he went to die in his home village, they went there to get the body;') \\
%    \ mɛ́ tɔ̀mbà, ah bon, ɛ́sɛ̀(à) ìkúílí ɔ̀yɛ́á mɛ́\ds{}ná\ds{}bá mɛ́ndɔ̀ ámɛ̀ mànyà mɛ̀sɛ́á nɛ̀ɔ̀fɛ́nɛ̀ á wə́yíə́ ùmììmə̀ wɔ́\ds{}ndɔ́ ndà fàmàk bɔ̀kàsɛ́á àkà kə̀ wínə̀ ɔ̀ ɔ̀wàyɛ́ bɔ̀nɔ̀ŋ(ɔ̀), báná ká ùmììmə̀ ɛ̀tà. nɛ̀ɔ̀fɛ́nɛ̀ á báná ndà fàmàn(à) \\
%%LINE1
   \glll 
   \textbf{nɛɔfɛ́nɛ} \textbf{á} bá ná nda faman. \\
%%LINE2
/nɛɔfɛ́nɛ á bá ná nda fámána/ \\
%%LINE3
{\db}today \COP{} 2\SM{} \PST{}2{} \VEN{} arrive.\APPL{} \\
%%TRANS1
    \glt
	`C'est [aujourd'hui]\textsubscript{\FOC{}} qu'on est arrivé avec.'  \\ `It's [today]\textsubscript{\FOC{}} that they arrived with it.'  \jambox*{[PM 1012] }
%%TRANS2
%%EXEND
\z

Non-subject fragments are found without \textit{á} \xref{carrotfragments}, \xref{beansfragment}, thus differing from subject fragments. The focused noun phrase cannot be preceded by \textit{á} \xref{carrotfragaXP}, contrasting with what we saw for subject fragments. The marker \textit{á} also cannot follow the focused XP \xref{carrotfragXPa}. This suggests that the \textit{á} is the copula part of what in non-elided form is a cleft (rather than acting as a grammaticalised focus marker).

\ea \label{carrotfragments} (What is the man holding?)
%%EAX
\ea
%%JUDGEMENT
[]{
%%LABEL
%%CONTEXT
%%LINE1
\glll
{\db}kalɔ́t. \\
%%LINE2
/kalɔ́tɔ/ \\
%%LINE3
{\db}9.carrot \\
%%TRANS1
\glt
`[Une carotte]\textsubscript{\FOC{}}.' \\ `[A carrot]\textsubscript{\FOC{}}.' \jambox*{[PM 1266] }
%%TRANS2
}
%%EXEND

\ex
[*]{
\label{carrotfragaXP}
        á kalɔ́t \\
     /á kalɔ́tɔ/ \\
    \COP{} 9.carrot \\
   int. `[Une carotte]\textsubscript{\FOC{}}.' \\ int. `[A carrot]\textsubscript{\FOC{}}.' \jambox*{[PM 1267] }
}

%%EAX
\ex
%%JUDGEMENT
[*]{
%%LABEL
\label{carrotfragXPa}
%%CONTEXT
%%LINE1
%\gl
kalɔ́t á   \\
%%LINE2
     /kalɔ́tɔ á/ \\
%%LINE3
9.carrot \COP{} \\
%%TRANS1
\glt
int. `[Une carotte]\textsubscript{\FOC{}}.' \\ int. `[A carrot]\textsubscript{\FOC{}}.' \jambox*{[PM 1267--8]}
%%TRANS2
}
%%EXEND

\z
\z

%%EAX
\ea
%%JUDGEMENT
%%LABEL
\label{beansfragment}
%%CONTEXT
(Where are the beans?) \\
%%LINE1
\glll
{\db}\textbf{ɔ}	\textbf{hisíní}	\textbf{núúmə} (\textbf{á}	bilikó	bɛ́ lɛ́á) \\
%%LINE2
/ɔ	hɛ-sini	nuumə	(a	bɛ-liko	bɛ lɛ́á)/ \\
%%LINE3
{\db}\PREP{}	19-casserole	inside	\COP{}	8-bean	 8\SM{} be \\
%%TRANS1
\glt
`C'est [dans la cassérole]\textsubscript{\FOC{}} qu'il y a des haricots.' \\ `The beans are [in the pot]\textsubscript{\FOC{}} (and nowhere else).' \jambox*{[PM 477] }
%%TRANS2
%%EXEND

\z 

Again, the translation of \xref{beansfragment} suggests that reverse pseudoclefted non\hyp arguments are typically interpreted as exhaustive.


Finally, in corrective focus contexts, which are argued to be more contrastive types of foci (see e.g. \citealt{Cruschina2021}), ex-situ clefting can be used, as seen already in \xref{exhshutdoor} and as further illustrated in \xref{speakfrench}.

%%EAX
\ea
%%JUDGEMENT
%%LABEL
\label{speakfrench}
%%CONTEXT
(Context: Someone says incorrectly that you speak Tunen.) \\ 
%%LINE1
\glll
{\db}bɔ́ɔ, fɛlɛ́nd͡ʒ \textbf{á} mɛ́ nd(ɔ) ɔ́k. \\
%%LINE2
/bɔ́ɔ, fɛlɛ́nd͡ʒɛ á mɛ́ \textsuperscript{H}ndɔ ɔ́kɔ/ \\
%%LINE3
{\db}no French \COP{} \SM{}.1\SG{}.\REL{} \PRS{} understand \\
%%TRANS1
\glt
`Non, c'est [le français]\textsubscript{\FOC{}} que je comprends.' \\  `No, it's [French]\textsubscript{\FOC{}} that I understand.' \jambox*{[PM 93--4] }
%%TRANS2
%%EXEND

\z

We therefore see that clefts can be used for more contrastive focus contexts than information focus, but the canonical word order is still possible for the expression of corrective/contrastive non-subject focus.

Alternatives can be marked explicitly by means of the particle \textit{ɔbanɔ} `rather' and/or by directly naming the incorrect argument.

%%EAX
\ea
%%JUDGEMENT
%%LABEL
%%CONTEXT
%%LINE1
\glll
{\db}bɔ́ɔ, tátá mɔndɔ ɔwá ɛŋɔŋɔ á ná wə, á Acteur $\{$\textbf{ɔban}$\}$ a ná wə $\{$\textbf{ɔban}$\}$. \\
%%LINE2
/bɔ́ɔ tátá mɔ-ndɔ ɔwá ɛ-ŋɔŋɔ á ná wə́ á Acteur $\{$ɔbanɔ$\}$ a ná wə́ $\{$ɔbanɔ$\}$/ \\
%%LINE3
{\db}no \COP{}.\NEG{} 1-person \REL{}.1{} 7-politics 1\SM{}.\REL{} \PST{}2{} die \COP{} 1.Acteur rather 1\SM{}.\REL{} \PST{}2{} die rather \\
%%TRANS1
\glt
`Non, ce n'est pas le politicien qui est mort, c'est plutôt Acteur.'  \\  `No, it wasn't a politician who died, it was actually Acteur.'  \jambox*{[EE+GE+PB 2716] }
%%TRANS2
%%EXEND

\z 

%%EAX
\ea
%%JUDGEMENT
%%LABEL
%%CONTEXT
%%LINE1
\glll
{\db}mɛ́ ndɔ	fɛlɛ́nd͡ʒ(ɛ)	ɔ́k,	\textbf{mba}	\textbf{tátá}  \textbf{*(á)}	túnən. \\
%%LINE2
/mɛ \textsuperscript{H}ndɔ	fɛlɛ́nd͡ʒɛ	ɔ́kɔ	mba	tátá *(á)	tɔ-nəni/ \\
%%LINE3
{\db}\SM{}.1\SG{} \PRS{}	French	understand	but	\COP{}.\NEG{} *(\COP{})	13-Nen \\
%%TRANS1
\glt
`Je comprends le français, mais pas le tunen.' \\`I understand French, but not Tunen.' \jambox*{[PM 92] }
%%TRANS2
%%EXEND

\z


%%EAX
\ea
%%JUDGEMENT
%%LABEL
\label{animalspass}
%%CONTEXT
(Context: `Lots of animals passed on the bridge.') \\
%%LINE1
\glll
{\db}bɔ́ɔ,	\textbf{tátá}	mɛnyama,	á	yɛ́ ná	tɔmbak, (\textbf{mba})	bibəŋuluəkə. \\
%%LINE2
/bɔ́ɔ	tátá	mɛnyama	á	yɛ́ ná	tɔmba-aka mba	bɛ-bəŋuluəkə/ \\
%%LINE3
{\db}no	\COP{}.\NEG{}	10.animal	\COP{}	10\SM{} \PST{}2{}	pass-\DUR{} but	8-vehicle \\
%%TRANS1
\glt
`Non, ce n’est pas des animaux qui sont passés, ce sont des véhicules.’ \\ `No, it wasn't animals that passed, it was vehicles.' \jambox*{[PM 1579] }
%%TRANS2
%%EXEND

\z

\subsection{Section summary}
In summary, there are two types of cleft construction available for focus expression in Tunen, the basic cleft (\textit{á} NP\textsubscript{\FOC{}} Rel) and the reverse pseudocleft (NP\textsubscript{\FOC{}} \textit{á} Rel). Both constructions show indications of biclausality through the presence of a focused phrase, a copula, and a reduced relative clause, the latter being identifiable through relative clause subject marker and TAM forms. These cleft constructions are used for more contrastive foci types and are generally interpreted as exhaustive, but appear to not be inherently exhaustive, as they are compatible with the exclusive focus marker \textit{ata} `even'. Interestingly, no pseudocleft strategy was found, with the V\textit{á}O construction shown in \sectref{subsec:Vao} above to have monoclausal properties, likely due to grammaticalisation from an earlier biclausal cleft construction. Complexities related to identification of clefts are the ambiguity of many subject marker and TAM contexts with respect to main clause versus relative clause marking and the regular vowel elision rule in Tunen, which may lead to elision of the \textit{á} copula.
%XX add discussion of structure and interpretation of pseudoclefts

\section{Left-peripheral topics (zero, \textit{ɔ},  \textit{aba/áká})}\label{sectopics}
%\todo{Can't use $\varnothing$ in section heading} % changed ø to 'zero'
As is common crosslinguistically (see e.g. \citealt{Gundel1988}), Tunen topical constituents can appear in a left-peripheral position. In these cases, there are three strategies for topic expression: (i) zero-marking (i.e. fronting the topic without morphological marking), (ii) marking by the preposition \textit{ɔ}, and (iii) marking by \textit{aba}/\textit{áká}, which elsewhere function as the conditional marker `if'. This section will go through each strategy in turn. Note that while left-peripheral topics is one common strategy for expressing topics, topics may also be left in-situ and do not need to be fronted. These in-situ topics do not appear with any topic marking.

\subsection{Zero-marking}
A topical constituent can be fronted without any marking, as shown  in \xref{chieftop} and \xref{Noahtop} below for an aboutness topic.


%%EAX
\ea
%%JUDGEMENT
%%LABEL
\label{chieftop}
%%CONTEXT
%%LINE1
\glll
{\db}\textbf{kíŋgə,}	a ka	nyɔkɔ	naánɛkɔl. \\
%%LINE2
/kíŋgə	a ka	nyɔ-aka	naánɛkɔla/ \\
%%LINE3
{\db}1.chief	1\SM{} \PST{}3{} work-\DUR{}	yesterday \\
%%TRANS1
\glt
`Le chef, il a travaillé hier.' \\ `The chief, he worked yesterday.' \jambox*{[JO 2625] }
%%TRANS2
%%EXEND

\z

%%EAX
\ea
%%JUDGEMENT
%%LABEL
\label{Noahtop}
%%CONTEXT
%%LINE1
\glll
{\db}\textbf{Nɔ́a,} yɛ́ ndɔ kɛa ɔwá á ndɔ náá. \\
%%LINE2
/Nɔa yɛ́ \textsuperscript{H}ndɔ kɛ́á ɔwá a \textsuperscript{H}ndɔ náá/ \\
%%LINE3
{\db}Noah 7\SM{} \PRS{} do \REL{}.1{} 1\SM{} \PRS{} be.sick \\
%%TRANS1
\glt
`Quant à Noah, il semble qu'il est malade.' \\ `As for Noah, it seems that he is sick.' \jambox*{[JO 1306]}
%%TRANS2
%%EXEND

\z

When objects are topicalised and prosodically separated from the main clause by a pause, resumption in the main clause is not required, as shown by the lack of object indexation in \xref{housetop}. The ability for zero indexation of objects will be covered in more detail in \sectref{secreference} below.

%%EAX
\ea
%%JUDGEMENT
%%LABEL
\label{housetop}
%%CONTEXT
%%LINE1
\glll
{\db}\textbf{miímə,}	mɔndɔ	á \ds{}ná	katák. \\
%%LINE2
/miímə	mɔ-ndɔ	á \textsuperscript{L}ná	katá-aka/ \\
%%LINE3
{\db}3.house	1-person	1\SM{}.\REL{} \PST{}3{} destroy-\DUR{} \\
%%TRANS1
\glt
`La maison, c'est quelqu'un qui l'a détruite.' \\ `The house, it's somebody who destroyed it.' \jambox*{[EB+JO 2692] }
%%TRANS2
%%EXEND

\z


It is often unclear as to whether a topical subject is fronted or left in-situ, as the canonical position of subjects is sentence-initial (\textit{S}-Aux-O-V-X), which is linearly equivalent to the position they appear in if fronted to a left-peripheral position. For example, in \xref{peanuttop} below, the referent of peanuts is a topic in that it is visibly present and has been previously mentioned in the discourse, and serves as the topic to which the comment of having cooled applies, but the word order is the same as what we saw for thetics in \xref{churchthetic} above.

%%EAX
\ea
%%JUDGEMENT
%%LABEL
\label{peanuttop}
%%CONTEXT
Context: JO has shown how to dry peanuts in order to prepare the \textit{kok} dish [\textit{hɛkɔkɛ} leaves boiled
with smoked fish and ground peanuts]. \\ 
%%LINE1
\glll
{\db}\textbf{tɔmbaŋa} tú nú huhək. \\
%%LINE2
/tɔ-mbaŋa tɔ́ ná huhə-aka/ \\
%%LINE3
{\db}13-peanut 13\SM{} \PST{}2{} cool-\DUR{} \\
%%TRANS1
\glt
`Les arachides se sont refroidies.' \\ `The peanuts have cooled.' \jambox*{[JO 1358] }
%%TRANS2
%%EXEND

\z

When studying the \citet{Dugast1975} texts, it can be seen that topical subjects are frequently transcribed as ending with a glottal stop ʔ, which reflects a prosodic break and therefore can be taken as evidence for a left-dislocated topic (as pointed out in \citealt[59]{Isaac2007}). An example is given in \xref{dugasttopic} below (I have adapted glosses for consistency). The Dugast data differ systematically from my own field data in not having subject indexation of non-dislocated topics, i.e. in lacking a subject marker when the topic is not dislocated.\footnote{As no recordings are available for Dugast's data, dislocation is only evidenced by her ʔ notation, and commas, when they are used. \citet{Isaac2007} considers the lack of subject marker to be evidence for a non-dislocated topic, but this argument is used somewhat circularly in the absence of any indication of prosody in Dugast's transcriptions. As all my consultants consistently use subject markers regardless of whether the topic is dislocated, there appears to have been a syntactic change in the time since Dugast with regards to the relation of clause-external topics and subject indexation. I discuss this further in \citet[363--365]{KerrFut}.}

%%EAX
\ea
%%JUDGEMENT
%%LABEL
\label{dugasttopic}
%%CONTEXT
%%LINE1
%%LINE2
\gll \textbf{wəbúə} \textbf{mon} \textbf{òwá} \textbf{ba} \textbf{na-ba} \textbf{ba} \textbf{ndò-hikiəʔ}, à nə̄-wə ton.\\
%%LINE3
{\db}1.\PRO{}.\POSS{}.1{} 1.child \REL{}.1{} 2\SM{} \PST{}2-be 2\SM{} \PRS{}-like 1\SM{} \PST{}2-{}die also \\
%%TRANS1
\glt
`Their child that they liked, he died also.' \jambox*{(\citealt[395]{Dugast1975}, cited in \citealt[165]{Isaac2007}) }
%%TRANS2
%%EXEND

\z

Turning now to obliques, while the neutral word order in Tunen is S-Aux-O-V-X, where X stands for other elements, including time adverbials and prepositional phrases, such items can also be fronted when they function as scene-setting topics, where the topical constituent is thus not an argument of the verb \citep{Lambrecht1994}. This is often found for time adverbials in natural speech. Compare the elicited example with S-Aux-O-V-X order in \xref{seePierrenow} below and the natural speech example with a fronted time adverbial in \xref{cookkoknow}.

%%EAX
\ea
%%JUDGEMENT
%%LABEL
\label{seePierrenow}
%%CONTEXT
%%LINE1
\glll
{\db}mɛ́ ndɔ Biə́lə sin \textbf{isiŋak}. \\
%%LINE2
/mɛ \textsuperscript{H}ndɔ Biɛ́lɛ sinə ɛsɛ́áŋáka/ \\
%%LINE3
{\db}\SM{}.1\SG{} \PRS{} 1.Pierre see now \\
%%TRANS1
\glt
`Je vois Pierre maintenant.' \\ `I see Pierre now.' \jambox*{[EO 1412] }
%%TRANS2
%%EXEND

\z

%%EAX
\ea
%%JUDGEMENT
%%LABEL
\label{cookkoknow}
%%CONTEXT
(Context: Instructional video where JO is demonstrating how to cook the \textit{kok} dish [\textit{hɛkɔkɛ} leaves boiled with smoked fish and ground peanuts].) \\ 
%%LINE1
\glll
{\db}\textbf{ɛsɛ́áŋáka} mɛ́ hɛkɔkɛ sɔ́áka. \\
%%LINE2
/ɛsɛ́áŋáka mɛ=\textsuperscript{H} hɛ-kɔkɛ sɔ́á-aka/ \\
%%LINE3
{\db}now \SM{}.1\SG{}=\PROC{} 19-kok wash-\DUR{} \\
%%TRANS1
\glt
`Maintenant, je lave le \textit{kok}.' \\ `Now, I wash the \textit{kok}.' \jambox*{[JO 1343] }
%%TRANS2
%%EXEND

\z

The following examples show fronted time adverbials or prepositional phrases indicating a switch between events \xref{angerhawk} and to set the scene at the beginning of a story \xref{chickenpartridgebeginning}.

%%EAX
\ea
%%JUDGEMENT
%%LABEL
\label{angerhawk}
%%CONTEXT
(Context: `The hawk waited and waited and waited, but he didn't see the cockroach, and his child died.') \\ 
%%LINE1
\glll
{\db}hilóbi hɛ́ ná wɛ́ɛya iti, \textbf{isíŋáka} ɔnd͡ʒɛlɛ́ a n(á) ákan asɛ : [...] \\
%%LINE2
/hɛ-lóbi hɛ́ ná wɛ́ɛya itíə́ ɛsɛ́áŋáka ɔ-nd͡ʒɛlɛ́ a ná akáná a-sɛ́á [...]/ \\
%%LINE3
{\db}19-anger 19\SM{} \PST{}2{} \PRO{}.\EMPH{}.1{} hold now 3-lizard 1\SM{} \PST{}2 leave 1\SM{}-say [...] \\
%%TRANS1
\glt
`Il s'est mis en colère, maintenant le lézard est parti, il dit : [...]' \\ `He became enraged, and now the lizard came by, and said: [...]' \jambox*{[JO 2063] }
%%TRANS2
%%EXEND
\z
%%EAX
\ea
%%JUDGEMENT
%%LABEL
\label{chickenpartridgebeginning}
%%CONTEXT
(Context: Start of the story \textit{The Chicken and the Partridge}.) \\ 
%%LINE1
\glll
{\db}\textbf{ɔ} \textbf{hí\ds{}tɛ́yí} \textbf{hiɔŋɔ,} məhuə má sa bá mas. \\
%%LINE2
/ɔ hítɛ́\textsuperscript{L}yí hɛ-ɔŋɔ ma-huə ma sa bá ma-ɛsɛ/ \\
%%LINE3
{\db}\PREP{} \DEM{}.\DISC{}.\EMPH{}.19{} 19-year 6-harvest 6\SM{} \NEG{} be 6-good \\
%%TRANS1
\glt
`Cette année, la récolte n'était pas bonne.' \\ `This year, the harvest wasn't good.' \jambox*{[JO 1744] }
%%TRANS2
%%EXEND

\z

Example \xref{chickenpartridgebeginning} is zero-marking in the sense of having no additional morphological\slash phonological marking compared to the form in the canonical word order, although it shares the property of being introduced by the preposition \textit{ɔ} as the examples to be discussed in the next subsection.

\subsection{\textit{ɔ}}

Fronted topics are often marked by the general preposition \textit{ɔ} \PREP{}. The following example comes from a dialogue task based on asking each other questions about their preferences. EO first fronts the time adverbial \textit{isiŋíáka} `now' to shift the topic from the previous question and then introduces the topic of food by using the preposition \textit{ɔ} \xref{alternativeqplantainrice}.

%%EAX
\ea
%%JUDGEMENT
%%LABEL
\label{alternativeqplantainrice}
%%CONTEXT
%%LINE1
\glll
{\db}isiŋíáka, \textbf{ɔ} \textbf{bɛlábɔ́nɛ́á} \textbf{ɔnɛ́}, áká mɛsɛa mɛ́ aŋɔ́á ɛ́lɛ́ákɛ́n, yatɛ́ ɛbáka ɔ aŋɔ́á hikəki, makɔnd͡ʒɛ alɛ́(á) kón? \\
%%LINE2
/ɛsɛ́áŋáka ɔ bɛ-lábɔ́nɛ́á ɔ-nɛ́á áká mɛ-sɛ́á mɛ́ aŋɔ́á ɛ́lɛ́ákɛ́na yatɛ́ ɛ-bá-aka ɔ aŋɔ́á hikəkiə ma-kɔnd͡ʒɛ alɛ́á kóni/ \\
%%LINE3
{\db}now \PREP{} 8-food \INF{}-eat if \SM{}.1\SG{}-say \SM{}.1\SG{}.\SBJV{} \PRO{}.2\SG{} invite what 7\SM{}-be-\DUR{} \SM{}.2\SG{} \PRO{}.2\SG{} like 6-plantain or.rather 9.rice \\
%%TRANS1
\glt
`Maintenant, à propos de la nourriture, si je veux t'inviter, qu'est-ce que tu aimerais, les plantains ou bien le riz ?' \\ `Now, with regards to food, if I were to invite you round, what would you like, plantains or rice?' \jambox*{[EO 966] }
%%TRANS2
%%EXEND

\z

For the next example \xref{ex:whycookfood}, the speaker said the \textit{ɔ} preposition was good in the discourse context in which the food has already been mentioned and that omitting it would mean that it has not been mentioned, suggesting that the preposition marks an aboutness topic.

\ea
\label{ex:whycookfood}
(Why did they cook this food here?) \\ 
%%EAX
\sn
%%JUDGEMENT
[\textsuperscript{$\#$}]{
%%LABEL
%%CONTEXT
%%LINE1
\glll
{\db}\textbf{(ɔ)} \textbf{bɛ́ɛ(bɛ)} \textbf{bɛlábɔ́nɛ́á} \textbf{bɛ́ɛbɛ,} bá ná talɛ́áká ɛlɔ́áyɛ́ ɛŋganda yɛ Básɛka. \\
%%LINE2
/\textsuperscript{$\#$}(ɔ) bɛ́ɛ(bɛ) bɛ-lábɔ́nɛ́á bɛ́ɛbɛ bá ná talɛ́á-aka ɛlɔ́áyɛ́ ɛ-ŋganda yɛ básɛka/ \\
%%LINE3
{\db}\PREP{} \DEM{}.\PROX{}.8{} 8-food \DEM{}.\PROX{}.8{} 2\SM{} \PST{}2{} cook-\DUR{} for 9-holiday \ASS{}.9{} Easter \\
%%TRANS1
\glt
`(Quant à cette nourriture-ci,) ils l'ont préparée pour la fête de Pâcques.' \\ (As for this food here,) they cooked it for Easter.' \jambox*{[PM 508] }
%%TRANS2
}
%%EXEND

\z

\subsection{\textit{aba}/\textit{áká}}
A limited number of examples had \textit{ábá} or \textit{áka} as a marker preceding a left-peripheral aboutness/shift topic, which function elsewhere as the conditional marker `if' and the related temporal marker `when' \citep[211--212]{Dugast1971}.\footnote{Note also \textit{ɛ́bɛ} `si, dans le cas où' [`if, in the case where'], which \citet[213]{Dugast1971} lists as an alternative for \textit{ába} (although she does not discuss whether it can be used in the same topic-marking function).} In the elicited example below, JO first gave the answer with \textit{ábá}, and then rejected the same sentence with \textit{ɔ} in place of \textit{ábá}.

%%EAX
\ea
%%JUDGEMENT
%%LABEL
\label{asforfuneralaba}
%%CONTEXT
%%LINE1
\glll
{\db}\textbf{$\{$ábá|*ɔ$\}$} ɛŋganda yɛ buwə́, yɛ́ sá bá yɛs. \\
%%LINE2
/$\{$ábá|*ɔ$\}$ ɛ-ŋganda yɛ bu-wə́ yɛ sa bá yɛ-ɛsɛ/ \\
%%LINE3
{\db}$\{$if|*\PREP{}$\}$ 9-celebration \ASS{}.9{} 14-death 9\SM{} \NEG{} be 7-good \\
%%TRANS1
\glt
`Quant à la fête du deuil, il n'était pas bon.' \\ `As for the funeral, it was not good.' \jambox*{[JO 1648--9] }
%%TRANS2
%%EXEND
\z

Directly after the above elicitation in the same session, JO first accepted the following sentence with \textit{ɔ} alone and then suggested it with \textit{ábá} preceding \textit{ɔ}, thus combining the two strategies \xref{asforfoodaba}. At this point, the data are insufficient to be able to account for why both variants were accepted for \xref{asforfoodaba} but not for \xref{asforfuneralaba}.\largerpage

%%EAX
\ea
%%JUDGEMENT
%%LABEL
\label{asforfoodaba}
%%CONTEXT
%%LINE1
\glll
{\db}\textbf{$\{$ɔ|ábá$\}$} bɛ́ɛbɛ bɛlabɛ́nɛ́ bá ná talɛ́áká ɛlɔ́áyɛ́ ɛŋganda yɛ básɛk. \\
%%LINE2
/$\{$ɔ|ábá$\}$ bɛ́ɛbɛ bɛ-labɛnɛa bá ná talɛ́á-aka ɛlɔ́áyɛ́ ɛ-ŋganda yɛ Básɛka/ \\
%%LINE3
{\db}\PREP{} \DEM{}.\PROX{}.8{} 8-food 2\SM{} \PST{}2{} cook-\DUR{} for 9-celebration \ASS{}.9 Easter \\
%%TRANS1
\glt
`Quant à cette nourriture, on l'a preparé pour la fête du Pacques.' / `Si c'est pour cette nourriture, on l'a preparé pour la fête du Pacques.' \\ `As for this food, they cooked it for Easter.' \jambox*{[JO 1650--1] }
%%TRANS2
%%EXEND

\z

The use of a conditional marker for topics was also found in natural dialogue by other speakers. In the first example below, the marker \textit{áká}, another form for `if' \citep[212]{Dugast1971},\footnote{\citet[212, 318]{Dugast1971} transcribes what appears to be the same marker as \textit{ɛ́kɛ}, translated as `si, quand' [`if, when'].} is first used before the speaker restarts using the prepositional strategy. In the second example, the speaker uses \textit{ábá} `if' with a prosodic break before the nominal (resulting in lowering of the final H tone via the utterance-final tone reduction rule; see \citealp[72--74, 82]{KerrFut}).

%%EAX
\ea
%%JUDGEMENT
%%LABEL
%%CONTEXT
(Context: EE describes how the harvests differed between crops cultivated by women and crops cultivated by men.)  \\
%%LINE1
\glll
{\db}\textbf{ák(á)} \textbf{ɛlɔ́áyɛ́} \textbf{ɔ} ... \textbf{ɔ} ... \textbf{ɔtɔ́mbákɛna} \textbf{ɔ} \textbf{bɛ} ... bɛlɔŋɔtɛ́ bɛ́ balɛ́mɛndɔ́, bɛ́ sá áyɛ́ wúu(wu) ɔyáá háá; \\
%%LINE2
/áká ɛlɔ́áyɛ́ ɔ ... ɔ ... ɔ-tɔ́mbá-aka-ɛna ɔ bɛ ... bɛ-lɔŋɔtɛ́ bɛ́ ba-lɛ́mɛndɔ́ bɛ́ sá áyɛ́ wúuwu ɔyáá háaha/ \\
%%LINE3
{\db}if for \PREP{} ... \PREP{} ... \INF{}-pass-\DUR{}-\REP{} \PREP{} 8 ... 8-production \ASS{}.8 2-man 8\SM{} \NEG{} \PRO{}.1{} \DEM.\PROX.? ? \DEM.\PROX.\LOC{} \\
%%TRANS1
\glt
`Si on parle des… des… des cultures des hommes, il n’y a pas eu de la production cette fois-ci.' \\ `If it's for... for... as for the... the mens' crops, there wasn't the production this time round.' \jambox*{[EE 1700] }
%%TRANS2
%%EXEND

\z
\largerpage[-1]\pagebreak
%%EAX
\ea
%%JUDGEMENT
%%LABEL
%%CONTEXT
(Context: PB and PM are discussing how they were impacted by the heavy rains that morning.) \\ 
%%LINE1
\glll
{\db}\textbf{ába}, ɔwámɛ yɛ́ ná ká sɔ́álátákɛn. \\
%%LINE2
/ábá ɔ-ámɛ yɛ́ ná ka sɔ́álátákɛna/ \\
%%LINE3
{\db}if \PREP{}-\PRO{}.1{} 7\SM{} \PST{}2 \AND{} whip.\DUR{}.\REP{} \\
%%TRANS1
\glt
`C'est moi que ça a fouetté.' \\ `It's me who got whipped by it.' \jambox*{[PM 1784] }
%%TRANS2
%%EXEND

\z

Compare these topical examples to the conditional example in \xref{condgetsick} below, where \textit{ábá} marks the conditional protasis.

%%EAX
\ea
%%JUDGEMENT
%%LABEL
\label{condgetsick}
%%CONTEXT
%%LINE1
\glll
{\db}\textbf{ábá}	Yɔhánasɛ	a ná ɛsasɔma nɛ́ák, á ndɔ	náák. \\
%%LINE2
/ábá	Yɔhánɛsɛ	a ná	ɛ-sasɔma	nɛ́á-aka a \textsuperscript{H}ndɔ	náá-aka/ \\
%%LINE3
{\db}if	1.Johannes	1\SM{} \PST{}2{} 7-cassava eat-\DUR{} 1\SM{} \PRS{} be.sick-\DUR{} \\
%%TRANS1
\glt
`Si Johannes va manger du manioc, il va tomber malade.' \\ `If Johannes eats cassava, he will get sick.' \jambox*{[PM 2288] }
%%TRANS2
%%EXEND

\z

This use of the same strategy to mark conditionals and topics has been found in many signed and spoken languages (\citealt{Haiman1978, Traugott1985, Janzen1999}, i.a.) with \citet{Haiman1978} arguing that there is an inherent link between conditionality and topicality. \citet[292]{Traugott1985} notes that markers of givenness are one of five sources of conditional markers crosslinguistically, with examples including Sanskrit \textit{yád} `topic, conditional' and Indonesian \textit{kalua} `if, as for'. This analysis would suggest that the topic marker use in Tunen predates the use as a conditional marker. I leave this for further research and retain the gloss `if' in this chapter without committing to `if' as the basic or original meaning.

\subsection{Multiple topics}
Multiple topic expressions can appear in the left periphery. We already saw in \xref{alternativeqplantainrice} that a frame-setting time adverbial can co-occur with a nominal topic. Another type of multiple topic expression is illustrated in the natural speech example from \xref{whichmangoes} below.%
\largerpage[-1]\pagebreak

%%EAX
\ea
%%JUDGEMENT
%%LABEL
\label{whichmangoes}
%%CONTEXT
%%LINE1
\glll
{\db}\textbf{á} \textbf{móŋgolo} \textbf{matɛ́\ds{}tɛ́}, \textbf{ɔ} \textbf{máama} \textbf{mə́súə́} \textbf{moŋgolo} \textbf{má} \textbf{háaha} \textbf{ɔ} \textbf{bɔnɔŋɔ}, \textbf{ɔmá} \textbf{abáka} \textbf{tɔ} \textbf{siəkinə} \textbf{háaha} \textbf{(ɔ)} \textbf{uwəsú} \textbf{bɔnɔŋɔ} \textbf{bɔ́} \textbf{Kəməlún}, ɛbáka ɔ maáta hikəkiə?  \\
%%LINE2
/á ma-óŋgolo ma-tɛ́\textsuperscript{L}tɛ́á ɔ máama mə́súə́ ma-óŋgolo má háaha ɔ bɔ-noŋɔ, ɔmá a-bá-aka tɔ siəkinə háaha ɔ uwəsú bɔ-nɔŋɔ bɔ́ kəməlúnə ɛ-bá-aka ɔ maáta hikəkiə/ \\
%%LINE3
{\db}\PREP{} 6-mango 6-small \PREP{} \DEM{}.\PROX{}.6{} \PRO{}.\POSS{}.1\PL{}.6{} 6-mango \ASS{}.6{} \DEM.\PROX.\LOC{} \PREP{} 14-country \REL{}.6{} 1\SM{}-{}be-\DUR{} \SM{}.1\PL{} see.\DUR{} \DEM.\PROX.\LOC{} \PREP{} \PRO{}.\POSS{}.1\PL{}.14{} 14.country \ASS{}.14{} Cameroon 7\SM{}-{}be-\DUR{} \SM{}.2\SG{} \PRO{}.\OBJ{}.6{} like \\
%%TRANS1
\glt
`Les petites mangues, nos petites mangues-ci du pays ici au Cameroun, tu les aimes ?' \\ `As for small mangoes, the small mangoes we get here in Cameroon, do you like them?' \jambox*{[PM 950] }
%%TRANS2
%%EXEND
\z

Here, we see multiple nominal topic expressions stacked in the left periphery. Matching the pattern found in other languages (\citealt[see e.g.][]{PaulWhitman2017} and the other chapters in this volume), the first topic phrase (small mangoes) is a superset of the second phrase (the small mangoes we get here in Cameroon), showing a progressive narrowing down of the topic to which the comment relates.

\subsection{Contrastive topics}
Unlike other languages like Rukiga \parencite{chapters/rukiga}, contrastive topics generally have no special marking in Tunen, neither for subjects nor objects. While the second topic may be optionally fronted, as indicated by the comma notation in \xref{topdontknowCecile}, we see in \xref{topemdavid} and \xref{contrastivetwocontexts} that no marking is required, with the same S-Aux-O-V(-X) canonical word order used as in an all-new thetic context.
\largerpage[2]

%%EAX
\ea
%%JUDGEMENT
%%LABEL
\label{topdontknowCecile}
%%CONTEXT
%%LINE1
\glll
{\db}mɛ́ ndɔ manya ɔwá \textbf{Matɛ́ŋɛ} a ka hiəfulə fanak, mba(,) \textbf{Sɛsília}(,) mɛ lɛ́ ndɔ many. \\
%%LINE2
/mɛ \textsuperscript{H}ndɔ manya ɔwá Matɛ́ŋɛ a ka hɛ-əfulə fana-aka mba Sɛsília mɛ lɛ \textsuperscript{H}ndɔ manya/ \\
%%LINE3
{\db}\SM{}.1\SG{} \PRS{} know \REL{}.1{} \textit{Martin} 1\SM{} \PST{}3{} 19-book read-\DUR{} but Cecile \SM{}.1\SG{} \NEG{} \PRS{} know \\
%%TRANS1
\glt
`Je sais que Martin a lu le livre, mais quant à Cecile, je ne sais pas.' \\ `I know that Martin has read the book, but I don't know about Cecile.' \jambox*{[JO 907] } 
%%TRANS2
%%EXEND

\z

%%EAX
\ea
%%JUDGEMENT
%%LABEL
\label{topemdavid}
%%CONTEXT
(Context: You are a teacher explaining to the parents of the students Emanuel and David how each child did in their exams.) \\ 
%%LINE1
\glll
{\db}\textbf{Ɛmánúɛ́lɛ} a ná tɔ́mbá, \textbf{Tə́witi} a ná kɔ. \\
%%LINE2
/Ɛmánúɛ́lɛ a ná tɔ́mbá Tə́witi a ná kɔa/ \\
%%LINE3
{\db}1{}.Emmanuel 1\SM{} \PST{}2{} pass 1{}.David 1\SM{} \PST{}2{} fail \\
%%TRANS1
\glt
`Emmanuel a réussi, David a échoué.' \\ `Emmanuel passed, David failed.' \jambox*{[JO 533] }
%%TRANS2
%%EXEND

\z

%%EAX
\ea
%%JUDGEMENT
%%LABEL
\label{contrastivetwocontexts}
%%CONTEXT
(What did the woman hold? What did the man hold? (+ QUIS picture stimulus)) \\ (What happened? (+ QUIS picture stimulus)) \\ 
%%LINE1
\glll
{\db}\textbf{mɔndɔ́} a ná hiɔ́sɔ itíə́. \textbf{muəndú} a ná ɔmbána itíə́.\\
%%LINE2
/mɔ-ndɔ a ná hɛ-ɔ́sɔ itíə́ mɔ-əndú a ná ɔ-mbána itíə́/ \\
%%LINE3
{\db}1-person 1\SM{} \PST{}2{} 19-spoon hold 1-woman 1\SM{} \PST{}2{} 3-knife hold \\
%%TRANS1
\glt
`L'homme tient une cuillière. La femme tient un couteau.' \\ `The man held a spoon. The woman held a knife.'\jambox*{[JO 629] }
%%TRANS2
%%EXEND

\z

These data can be taken to evidence the lack of morphosyntactic marking sensitive to a feature of contrast that covers both topic and focus.


\section{Functional passives (verbal participle \textit{-átɔ}; \textit{bá-} impersonals)} \label{secpassive}
Passives are a common cross-linguistic strategy used to restructure the information in a way that demotes the agent. While many Bantu languages have a passive morpheme cognate with Proto-Bantu *\textit{-ʊ/-ɪbʊ} (\citealp{Stappers1967}, \citealp[78--79]{Schadeberg2003}, \citealp{GuéroisFut}), Tunen does not have any verbal marker of the passive, a property it shares with other North-Western Bantu languages (see also \cite{chapters/teke} for the lack of a passive morpheme in Teke-Kukuya [B77]). Sentences that may be passivised in other languages are often given as active sentences in Tunen. In active sentences, the agent must be expressed as the syntactic subject. However, if the speaker does not want to express the agent or the agent is unknown, there are two options which perform as functional equivalents of a passive construction: (i) the use of the verbal participle \textit{-átɔ} in combination with the copular verb, and (ii) an impersonal construction with the class 2 subject marker \textit{bá-}.

\subsection{Verbal participles \textit{-átɔ}}

A copular construction with a verbal participle marked by the ending \textit{-átɔ} can be used in order to avoid expressing the agent. The \textit{-átɔ} form is analysed by \citet[362]{Dugast1971} as a verbal adjective and by \citet{Mous2003} as an adverb used as the complement of a copula to describe a resultant state, quality, or capacity. In my data, I gloss \textit{-átɔ} as \PTCP{} for ``participle", as discussed in \citet[109]{KerrFut}. The participle follows either the \textit{lɛ} or \textit{bá} copula \citep[362]{Dugast1971}, as illustrated in \xref{atoset} below.

\ea \label{atoset}
%%EAX
\ea
%%JUDGEMENT
%%LABEL
\label{cutwoodato}
%%CONTEXT
%%LINE1
\glll
{\db}hinyí hɛ́ lɛ sɔ́mb\textbf{átɔ}. \\
%%LINE2
/hɛ-nyí hɛ́ lɛa sɔ́mba-átɔ/ \\
%%LINE3
{\db}19-firewood 19\SM{} be cut-\PTCP{} \\
%%TRANS1
\glt
`Le bois de chauffage est coupé.' \\ `The firewood is cut.' \jambox*{[EE+EB 1671] }
%%TRANS2
%%EXEND

%%EAX
\ex
%%JUDGEMENT
%%LABEL
%%CONTEXT
%%LINE1
\glll
{\db}botɛ báka titə́k\textbf{áto}. \\
%%LINE2
/bɔ-tɛ bá-aka titə́-aka-átɔ/ \\
%%LINE3
{\db}14-savannah be-\DUR{} burn-\DUR{}-\PTCP{} \\
%%TRANS1
\glt
`La savanne est brûlée.' \\ `The savannah is burned.' \jambox*{(\citealt[362]{Dugast1971}, adapted) }
%%TRANS2
%%EXEND

\z
\z

An agent cannot be expressed using the comitative marker \textit{na} `with' in this verbal participle construction, in contrast to other Bantu languages that allow agent expression with \textit{na} in passives, such as Rukiga \parencite{chapters/rukiga}, Makhuwa \parencite{chapters/makhuwa} and Swahili and Shona \citep{Fleisch2005}. When a \textit{na}-phrase was added to \xref{cutwoodato}, it was interpreted as a discontinuous continuation of the theme object rather than as the agent \xref{dontcutsusanne}.\largerpage[2]


%%EAX
\ea
%%JUDGEMENT
[\textsuperscript{$\#$}]{
%%LABEL
\label{dontcutsusanne}
%%CONTEXT
%%LINE1
\glll
{\db}hinyí hɛ́ lɛ sɔ́mbátɔ \textbf{na} \textbf{Susan}. \\
%%LINE2
/hɛ-nyi hɛ lɛa sɔ́mba-átɔ na Susána/ \\
%%LINE3
{\db}19-firewood 19\SM{} be cut-\PTCP{} with 1.Susan \\
%%TRANS1
\glt
int. `Le bois du chauffage était coupé par Susanne.' \\  `Le bois du chauffage et Susanne étaient coupés.' \\ int. `The firewood was cut by Susanne.' \\ `The firewood and Susanne were cut.' \jambox*{[EE+EB 1673] }
%%TRANS2
}
%%EXEND
 \z

\ea
(What wounded the hunter?) \\
%%EAX
\sn
%%JUDGEMENT
[*]{
%%LABEL
%%CONTEXT
%%LINE1
\glll
{\db}a lɛ tanákátɔ \textbf{na} \textbf{mɔndɔ}. \\
%%LINE2
/a lɛa tanáká-átɔ na mɔ-ndɔ/ \\
%%LINE3
{\db}1\SM{} be wound.\DUR{}-\PTCP{} with 1-person \\
%%TRANS1
\glt
int. `Il était blessé par l'homme.' \\ int. He was wounded by a man.' \jambox*{[JO 1615] }
%%TRANS2
}
%%EXEND

\z

The participle \textit{-átɔ} construction is therefore used when the agent is not expressed, and so constitutes a functional equivalent to the passive.

\subsection{Impersonal \textit{bá-}}
If the agent is not known, an impersonal construction can also be used \citep{Mous2008}. This construction is formed with the class 2 subject marker \textit{bá-}, which does not agree with any referent in the discourse. Such \textit{bá-} impersonals are found in many other Bantu languages and are commonly referred to in the Bantu literature as \textit{ba-}passives (see e.g. \citealt{Fleisch2005}, \cite{chapters/rukiga}, \cite{chapters/teke}, and \citealt{Taylor1999}, the latter of whom shows \textit{bá}-passives in the neighbouring Bantu
language Nomaandé (A46, Cameroon)). 

%%EAX
\ea
%%JUDGEMENT
%%LABEL
%%CONTEXT
(Context: QUIS picture of a child falling; you cannot see who pushed them. EK asks in Tunen `What happened?') \\ 
%%LINE1
\glll
{\db}\textbf{bá} ná mɔná lúmə́ na mɔk əlim. \\
%%LINE2
/bá ná mɔ-ná lúmə́ na mɔkɔ əlimə/ \\
%%LINE3
{\db}2\SM{} \PST{}2{} 1-child throw with 3.stone behind \\
%%TRANS1
\glt
`On a lancé une pierre à l'enfant par derrière.' \\ `Someone threw a stone at the child from behind.' \jambox*{[EO 439] }
%%TRANS2
%%EXEND

\z

%%EAX
\ea
%%JUDGEMENT
%%LABEL
%%CONTEXT
(Context: EO is reporting a conversation he had with PM on the phone, telling PM that he is standing in front of the shop where car oil is sold so that PM can find him. `I'm here, where they sell crude oil,') \\
%%LINE1
\glll 
... ɔhá \textbf{bá} ndɔ moló má mátɔ́á sɛm. \\
%%LINE2
/[...] ɔhá bá \textsuperscript{H}ndɔ moló má ma-tɔ́á sɛma/   \\
%%LINE3
{\db}... \REL{}.\LOC{} 2\SM{} \PRS{} 6.oil \ASS{}.6{} 6-car sell \\
%%TRANS1
\glt
`...où on vend les carburants pour les voitures.' \\ `...where they sell oil for cars.' \jambox*{[EO 1029] }
%%TRANS2
%%EXEND

\z

%%EAX
\ea
%%JUDGEMENT
%%LABEL
%%CONTEXT
%%LINE1
\glll
{\db}\textbf{ba}́ ná hinyí sɔmb. \\
%%LINE2
/bá ná hɛ-nyi sɔ́mba/ \\
%%LINE3
{\db}2\SM{} \PST{}2{} 19-firewood cut \\
%%TRANS1
\glt
`On a coupé le bois du chauffage.' \\ `The firewood has been cut.' \jambox*{[EE+EB 1672] }
%%TRANS2
%%EXEND

\z

As with the participle construction, expression of the agent with \textit{na} `with' is not allowed in the \textit{bá-} construction in Tunen (again a point of crosslinguistic variation within the Bantu family; \citealt{Fleisch2005}).

%%EAX
\ea
%%JUDGEMENT
[*]{
%%LABEL
%%CONTEXT
%%LINE1
\glll
{\db}\textbf{ba}́ ná hinyí sɔmb na Susan. \\
%%LINE2
/bá ná hɛ-nyi sɔ́mba na Susána/ \\
%%LINE3
{\db}2\SM{} \PST{}2{} 19-firewood cut with 1.Susan \\
%%TRANS1
\glt
int. `Le bois de chauffage était coupé par Susanne.' \\ int. `The firewood was cut by Susanne.' \jambox*{[EE+EB 1674] }
%%TRANS2
}
%%EXEND

\z


We therefore see that Tunen has two constructions that can be used for the demotion of the agent: the \textit{-átɔ} participle form and the \textit{bá-} impersonal construction. Bantu languages are known to vary as to whether and how the agent is expressed in a passive construction \citep{Fleisch2005}. In Tunen, agent expression is not possible.

\subsection{The middle prefix \textit{bɛ́-}}

Finally, note that Tunen has a prefix \textit{bɛ́-} which \citet{Dugast1971} treats as a passive and reflexive marker. \citet{Mous2008} gives a detailed discussion of this marker, arguing that it is in fact a middle prefix, suggesting an etymology of a first person plural pronoun and noting a cognate form in other A40/A60 languages of Cameroon. In \citet{KerrFut} I analyse it formally as a Voice head within the verbal spine. This middle prefix shows some functional overlap with the \textit{bá-} impersonal construction, as seen in \xref{passivemid} below.\footnote{Note that the \textit{bɛ́-} prefix is transcribed as \textit{bé-} in Mous' \citeyear{Mous2003} orthography (see \citealt[Chapter 4]{KerrFut} on orthographical differences between sources).}


\ea \label{passivemid} 
%%EAX
\ea
%%JUDGEMENT
%%LABEL
%%CONTEXT
%%LINE1
%%LINE2
\gll
a-ná \textbf{bé}-tóŋona mɛkɔ.\\
%%LINE3
{\db}1\SM{}-\PST{}2{} \MID{}-transform 9:leopard \\
%%TRANS1
\glt
`He transformed into a leopard.' \\
%%TRANS2
%%EXEND

%%EAX
\ex
%%JUDGEMENT
%%LABEL
%%CONTEXT
%%LINE1
%%LINE2
\gll
\textbf{bá}-ná mondo tóŋóná mɛkɔ.\\
%%LINE3
{\db}2\SM{}-\PST{}2{} 1:man transform 9:leopard \\
%%TRANS1
\glt
‘They transformed the man into a leopard. \jambox*{(\citealt[310]{Mous2008}, adapted)  }
%%TRANS2
%%EXEND

\z
\z

The overlap between these construction is to be expected considering the cross-linguistic overlap in middle/neutro-passives and passives (\citeauthor{GuéroisFut} to appear). The interested reader can find more detail about the specific contexts of use of the \textit{bɛ́-} prefix in \citet{Mous2008}.

\section{Referent expression in discourse}\label{secreference}
This section will show how the form of nominals in Tunen varies dependent on its information-structural status in the discourse. Referent expression across languages varies dependent on givenness/activation status, i.e. the cognitive notion of how accessible the referent is at a particular point of discourse, as affected by factors such as recency of mention and number of intervening referent expressions \citep{GundelEtAl1993, Ariel2001}. The general pattern is that more accessible referents are referred to with less linguistic encoding. As noted in previous work by \citet{Isaac2007}, Tunen follows this general pattern, with full noun phrases typically used to introduce new discourse referents, after which less material is used. The full scale of options for referent expression in discourse in Tunen is shown in \xref{refhierarchy} below, ordered from least to most linguistic encoding.

\ea \label{refhierarchy} Tunen referent expression hierarchy \\
Zero/null > verbal marker > modifier only > non-emphatic pronoun > emphatic pronoun > demonstrative > full DP > compound DP > modified DP
\z

The following extract from a story shows how after a subject is referred to with a compound DP (\textit{muití ɛ́mbɔ́ma} `owner of the field'), it can then be referred to using the verbal subject marker only, that is the same noun class (class 1 \SM{} \textit{a-}). 

\ea
%%EAX
\ea
%%JUDGEMENT
%%LABEL
%%CONTEXT
%%LINE1
\glll
{\db}\textbf{muit(í)} \textbf{ɛ́mbɔ́ma} \textbf{a} ná wɛ́ɛya halɛ́n.  \\
%%LINE2
/mɔ-ití \textsuperscript{H}=ɛ-mbɔ́ma a ná wɛ́ɛya halɛ́na/ \\
%%LINE3
{\db}1-owner \ASS{}=7-field 1\SM{} \PST{}2{} \PRO{}.1{} catch \\
%%TRANS1
\glt
`Le propriétaire du champ l'a arreté.' \\ `The owner of the field caught her.'  \jambox*{[JO 1765] }
%%TRANS2
%%EXEND

%%EAX
\ex
%%JUDGEMENT
%%LABEL
%%CONTEXT
%%LINE1
\glll
{\db}\textbf{a} ná wɛ́ɛya ákanána ɔ wáayɛ́ ɔmbɛl. \\
%%LINE2
/a ná wɛ́ɛya ákánána ɔ wáayɛ́ ɔ-mbɛ́la/ \\
%%LINE3
{\db}1\SM{} \PST{}2{} \PRO{}.1{} leave.\APPL{} \PREP{} 3{}.\POSS{}.1 3-house \\
%%TRANS1
\glt
`Il l'a amené dans sa maison.' \\ `He took her into his house.' \jambox*{[JO 1766] }
%%TRANS2
%%EXEND

\z
\z

While subjects are always expressed by a verbal subject marker (\SM{}) in Tunen, objects can be zero-expressed, i.e. dropped. Unlike most Bantu languages, Tunen does not have any object marker slot on the verb \citep[107]{KerrFut}, so there is no available object marker strategy. Again, this is a property common to North-Western Bantu languages that sets Tunen apart from the Eastern and Southern Bantu languages in this volume (\citealt{Polak1986}, \citealt[69--70]{vanderWal2022}) and shows overlap with Grassfields Bantu (Bantoid) languages.

Object expression in Tunen varies dependent on givenness. When an object is first mentioned, a full/compound/modified DP is used. When the object is given (i.e. retrievable from the discourse context), it is often null, as in the example below where the object \textit{bɛɔnɔ́} `eggs' is first introduced with a DP and then dropped in the next clause, as indicated by `$\emptyset$'.\footnote{The ability for objects to be unexpressed in Tunen raises questions about the transitivity of verbs like \textit{nɛ́á} `eat' in Tunen. An alternative analysis would be to say that these verbs have homophonous intransitive forms, in which case the object would not properly be considered to be ``dropped'' as it is not required in the verb's lexical entry. In this chapter, I use ``zero expression'' and ``dropped'' to mean that there is no object expression with a predicate that in non-given contexts takes an object.}

%%EAX
\ea
%%JUDGEMENT
%%LABEL
%%CONTEXT
%%LINE1
\glll
{\db}a \textbf{bɛɔnɔ́} nɛakak, bɛ́ndɔ bá nɛak. \\
%%LINE2
/a \textbf{bɛ-ɔnɔ́} nɛaka-aka bɛndɔ bá \textbf{$\emptyset$} nɛ́á-aka/ \\
%%LINE3
{\db}1\SM{} 8-egg make-\DUR{} 2.person 2\SM{} 8.\OBJ{} eat-\DUR{} \\
%%TRANS1
\glt
`Elle pond des œufs; les hommes les mangent.' \\ `She lays eggs; people eat them.' \jambox*{[JO 1769] }
%%TRANS2
%%EXEND

\z

Dropping given objects like this is very common. An example is provided in the dialogue below, where speaker PM introduces the referent \textit{ibuŋuluəkə} `car' and speaker EO uses zero-expression \xref{carreference}. The example set in \xref{kokreference} from a monologue instructional video shows the same zero-expression of an object when it is given, with the full DP being used at the end again \xref{kokend}, where the need for re-activation can be considered in terms of both linguistic and temporal distance from the last explicit mention (the latter indicated by the timestamp next to each example). Note that the highly-accessible first-person singular subject is consistently referred to with a subject marker, which is the minimal means to express Tunen subjects.\largerpage

\ea \label{carreference} (Context: PM and EO perform the QUIS map task (PM gives instructions to EO).) \\
%%EAX
\ea
%%JUDGEMENT
%%LABEL
%%CONTEXT
%%LINE1
\glll
{\db}mɔkátá wɔ́ bɛ́nɔ́mɛ wú búsíə́ \textbf{ibuŋuluəkə} yɛ́ nda báká háha ɔ matá. \\
%%LINE2
/mɔ-kátá wɔ́ bɛ́nɔ́mɛ wɔ́ búsíə́ ɛ-buŋuluəkə yɛ́ nda bá-aka háaha ɔ matá/ \\
%%LINE3
{\db}3-hand \ASS{}.3{} right \ASS{}.3{} front 7-car 7\SM{} \VEN{} be-\DUR{} \DEM.\PROX.\LOC{} \PREP{} bottom \\
%%TRANS1
\glt
`'Il y a un véhicule en bas au premier embranchement à droite.' \\ `There's a car at the bottom of the first road on the right.' \jambox*{[PM 671] }
%%TRANS2
%%EXEND

%%EAX
\ex
%%JUDGEMENT
%%LABEL
%%CONTEXT
%%LINE1
\glll
{\db}ɛ́ɛ, mɛ́ ndɔ sin. \\
%%LINE2
/ɛ́ɛ mɛ \textsuperscript{H}ndɔ $\emptyset$ sinə/ \\
%%LINE3
{\db}yes \SM{}.1\SG{} \PRS{} \OBJ{}.7 see \\
%%TRANS1
\glt
`Oui, je le vois.' \\ `Yes, I see it.' \jambox*{[EO 672] }
%%TRANS2
%%EXEND

\z
\z

\ea \label{kokreference} (Context: JO demonstrates how to prepare the dish \textit{kɔk}.) \\
%%EAX
\ea
%%JUDGEMENT
%%LABEL
%%CONTEXT
%%LINE1
\glll
{\db}ɛsɛ́áŋáka	mɛ́ \textbf{hɛkɔkɛ}	sɔ́áka \\
%%LINE2
/ɛsɛ́áŋáka	mɛ=\textsuperscript{H}	hɛ-kɔkɛ	sɔ́á-aka/ \\
%%LINE3
{\db}now	\SM{}.1\SG{}=\PROC{}	19-kok	wash-\DUR{} \\
%%TRANS1
\glt
`Maintenant, je lave le kok.' \\ `Now, I wash the kok.' \jambox*{[JO 1343; 00:00:38] }
%%TRANS2
%%EXEND

%%EAX
\ex
%%JUDGEMENT
%%LABEL
%%CONTEXT
%%LINE1
\glll
{\db}mɛ́	\textbf{hɛkɔkɛ}	sɔ́áka \\
%%LINE2
/mɛ=\textsuperscript{H}	hɛ-kɔkɛ	sɔ́á-aka/ \\
%%LINE3
{\db}\SM{}.1\SG{}=\PROC{}	19-kok	wash-\DUR{} \\
%%TRANS1
\glt
`Je lave le kok.' \\ `I wash the kok.' \jambox*{[JO 1344; 00:00:58] }
%%TRANS2
%%EXEND

%%EAX
\ex
%%JUDGEMENT
%%LABEL
%%CONTEXT
%%LINE1
\glll
{\db}mɛ ná hɔ́á ɔ ɔsɔa \\
%%LINE2
/mɛ ná	hɔ́á	ɔ	ɔ-sɔ́á/ \\
%%LINE3
{\db}\SM{}.1\SG{} \PST{}2{}	finish	\PREP{}	 \INF{}-wash\\
%%TRANS1
\glt
`J'ai fini de laver.' \\ `I've finished washing (it).' \jambox*{[JO 1345; 00:01:34] }
%%TRANS2
%%EXEND

%%EAX
\ex
%%JUDGEMENT
%%LABEL
%%CONTEXT
%%LINE1
\glll
{\db}mɛ́	əmbə́kínə	ɔ	mol \\ %\hspace{7em} \\
%%LINE2
/mɛ=\textsuperscript{H}	əmbə́kínə	ɔ	moló/ \\
%%LINE3
{\db}\SM{}.1\SG{}=\PROC{}	throw.\REP{} \PREP{} 6.oil \\
%%TRANS1
\glt
`Je (le) lance dans l'huile.' \\ `I'm throwing (it) into the oil.' \jambox*{[JO 1346; 00:01:38 ] }
%%TRANS2
%%EXEND

%%EAX
\ex
%%JUDGEMENT
%%LABEL
\label{kokend}
%%CONTEXT
%%LINE1
\glll
{\db}mɛ ná	\textbf{hɛkɔkɛ}	əmbínə	ɔ	moló \\ %\hspace{7em}\\
%%LINE2
/mɛ ná	hɛ-kɔkɛ	əmbínə	ɔ	moló/ \\
%%LINE3
{\db}\SM{}.1\SG{} \PST{}2{}	19-kok	throw	\PREP{}	6.oil \\
%%TRANS1
\glt
`J'ai lancé le kok dans l'huile.'\\ `I've thrown the kok into the oil.' \jambox*{[JO 1347; 00:03:19] }
%%TRANS2
%%EXEND

\z 
\z 

While this strategy of zero-expression of given objects is common, it is not possible when the verb has an applicative extension, in which case overt expression of the object (by DP or pronoun) is syntactically required. Example \xref{whatMariasay} below shows that it is not grammatical to have an unexpressed recipient object with an applicativised verb form.

%%EAX
\ea
%%JUDGEMENT
%%LABEL
\label{whatMariasay}
%%CONTEXT
%%LINE1
\glll
{\db}yatɛ́	Malíá	á ná	$\{$\textbf{láá}|\textbf{*lɛná}$\}$	eé? \\
%%LINE2
/yatɛ́	Malíá	á ná	$\{$lá|*lɛ́ná$\}$	eé/ \\
%%LINE3
{\db}what	1.Maria	1\SM{} \PST{}2{} $\{$say|say.\APPL{}$\}$ \Q{} \\
%%TRANS1
\glt
int. `Qu'est-ce que Maria a dit ?' \\ int. `What did Maria say?' \jambox*{[JO 2448--9] }
%%TRANS2
%%EXEND
\z

The consequence of the applicative's valency requirement means that pronominal expression is fairly frequent for recipient objects in the corpus, as the standard way of reporting speech in a story uses an applicative form of the verb `say' (followed by the complementiser formed from \mbox{-\textit{sɛ́á}} `say'), which requires either a pronoun or lexical DP subject.

%%EAX
\ea
%%JUDGEMENT
%%LABEL
\label{lizardsaid}
%%CONTEXT
%%LINE1
\glll
{\db}ɔnd͡ʒɛlɛ́ a ná \textbf{wɛ́ɛya} \textbf{lɛ́ná} asɛ : ... \\
%%LINE2
/ɔ-nd͡ʒɛlɛ́ a ná wɛ́ɛya lɛ́ná a-sɛ́á [...]/ \\
%%LINE3
{\db}3-lizard 1\SM{} \PST{}2{} \PRO{}.1{} say.\APPL{} 1\SM{}-{}say [...] \\
%%TRANS1
\glt
`Le lézard lui a dit : [...]' \\ `The lizard told him: [...]' \jambox*{[JO 2068] }
%%TRANS2
%%EXEND
\z

%%EAX
\ea
%%JUDGEMENT
%%LABEL
%%CONTEXT
%%LINE1
\glll
{\db}Yə́susu a ná \textbf{bəə́bu} \textbf{lɛ́na} a sɛ́á : [...] \\
%%LINE2
/Yə́susu a ná bə́əbu lɛ́ná a-sɛ́á [...] / \\
%%LINE3
{\db}1.Jesus 1\SM{} \PST{}2{} \PRO{}.2{} say.\APPL{} 1\SM{}-say [...] \\
%%TRANS1
\glt
`Jésus leur a dit : [...]' \\ `Jesus said to them: [...]' \jambox*{[Luke 9.52: \citealp[159]{CABTAL2019}] }
%%TRANS2
%%EXEND
\z

This finding is significant as it challenges previous classification of Tunen's pronominal system, in which different pronominal forms are analysed as varying in degree of a loosely-defined notion of ``emphasis" (\citealt[128--130]{Dugast1971}, \citealt[49--51]{Isaac2007}; \citealt[97--100]{KerrFut}). The possible confound of the applicative verb form on pronoun use is a topic worth more detailed investigation, in order to better understand the extent to which pronoun form and frequency reflects referent accessibility rather than confounding factors such as valency requirements of the verb.

In summary then, Tunen referent expression follows crosslinguistic tendencies to use less material to refer to given/accessible referents \citep{GundelEtAl1993, Ariel2001}, with full noun phrases used to introduce discourse referents \citep{Isaac2007}. Compared to other Bantu languages, Tunen is typical in its use of verbal subject markers without a lexical DP for given subjects, but unusual in lacking object markers and therefore having frequent zero-reference for objects. Pronouns can be used and are often found to meet valency requirements when the verb has an applicative extension, suggesting a confound that could be investigated further. For our current purposes, we see that Tunen referent expression follows crosslinguistic tendencies to use more linguistic material to encode less accessible discourse referents.

\section{Comparison to other Bantu languages}\label{seccomparison}

Before concluding, I will reflect briefly on how Tunen compares to other Bantu languages in its expression of information structure. We have seen in this chapter that Tunen is unusual for a Bantu language in the following respects: (i) grammatical roles are more important for word order than discourse roles; (ii) S-Aux-O-V-X (and not SVO) is the canonical word order; (iii) there is no morphological passive, (iv) there is no dedicated focus position, and (v) no object marking is permissible to refer to given objects. Furthermore, (vi) no inversion constructions are found, and (vii) there is no predicate doubling (unlike other Bantu languages in this volume; see also \citealt{GüldemannFiedler2022}). These properties have been suggested before as areal features related to Tunen's position in the Northwest of the Bantu-speaking area. For example, \citet{HamlaouiMakasso2015} report the same lack of inversion constructions and object marking for Basaá, another Cameroonian Bantu language of the A40 group, and \citet{Güldemann2008} has proposed O-V-X as an areal syntactic property of the Macro-Sudan Belt (a proposed linguistic area which in which Tunen is spoken), as I discuss in \citet[§6.7--6.8]{KerrFut}. Finally, in our own work on the BaSIS project we have shown that Tunen has no dedicated focus position and have argued that grammatical roles are less important than information-structural roles for determining Tunen's word order, which we have suggested is linked to its position in the Northwest \citep{KerrEtAl2023}. Note that this reliance on grammatical role differs from the Cameroonian/Nigerian Bantoid language Naki studied by \citet{Good2010}, which was argued to show evidence for information structure as the principal determiner of word order, and is also distinct from Teke-Kukuya, which \textcite{chapters/teke, LiFut} shows has innovated a dedicated focus position (see also \citealt{DeKind2014} on Kisikongo (H16a), \citealt{BostoenMundeke2012} on Mbuun (B87), and \citealt{KoniMuluwaBostoen2014} on Nsong (B85d)). This highlights the fact that there is variation in Northwestern Bantu and Bantoid languages in the expression of information structure, meaning that detailed studies of individual languages are required.

This is not to say however that Tunen has no similarities with other Bantu languages in its expression of information structure. Like other Bantu languages in this volume, and as matches crosslinguistic patterns for focus marking, different cleft strategies are available to express focus (see e.g.\ \citealp{FiedlerEtAl2010, FeryIshihara2017}), in which case there is typically an exhaustivity reading. Information focus can be left unmarked (for non-subjects). Also like the other languages and the crosslinguistically common pattern \citep{Gundel1988}, topics can be left-peripheral, in which case they may be marked or unmarked. We also see overlap between Tunen and the zone B77 Bantu language Teke-Kukuya in \textcite{chapters/teke}, which similarly lacks a morphological passive and has no inversion constructions. Finally, subject markers are the minimal means of subject expression, as in the other languages.

\section{Summary}\label{secconc}
This chapter has shown that Tunen's canonical word order is S-Aux-O-V-X, which is compatible with various different information-structural contexts. Alternatives to the canonical S-Aux-O-V-X word order are possible for the expression of information-structural notions, with clefting a common strategy for expressing focus, and fronting a means of marking topics, which may additionally be marked by the preposition \textit{ɔ} or \textit{ábá/aka} `if'. Finally, a short comparison between Tunen and other Bantu languages in terms of the expression of information structure was provided.

Further areas for research on Tunen information structure would be a more detailed corpus-based approach to frequencies of different word-order patterns, taking into account other potential factors such as prosodic weight; a prosodic analysis of potential correlates of information structure; a more detailed investigation of the use of conditional marking for introducing topics; and a more detailed study of fine-grained distinctions in referent expression, such as the use between basic and emphatic pronouns. A more detailed comparative study of languages of the Northwest as compared to Eastern and Southern Bantu languages would also be valuable, as well as a comparison of Northwestern Bantu and the Southern Bantoid languages of the Grassfields Bantu group.

\section*{Acknowledgements}
This research was conducted at Leiden University under the Bantu Syntax and Information Structure (BaSIS) project (NWO VIDI grant 276-78-001). I thank all of the Tunen consultants from Ndikiniméki, namely Patient (JP) Batal Batangken, Edmond Biloungloung, Emmanuel Enganayat, Alain Georges Essomo, Daniel Mbel, Angel Molel, Pierre Molel, Étienne Ondjem, and Jeanne Ongmolaleba. Next I thank BaSIS colleagues Allen Asiimwe, Patrick Kanampiu, Zhen Li, Amani Lusekelo, Nelsa Nhantumbo, Ernest Nshemezimana, and Jenneke van der Wal for group discussions on information structure. I also thank Maarten Mous and the two anonymous reviewers for additional comments, as well as audience members of the BaSIS brainstorm workshop in 2019, Bantu9, CALL50, CALL52, and WOCAL10 for helpful feedback. All mistakes are my own.


\section*{Abbreviations}
\begin{multicols}{2}
\begin{tabbing}
MMMM \= Bantu\kill
%%% All Leipzig abbreviations are commented out, following the LangSci guidelines of only listing non-Leipzig abbreviations.
1, 2, 3... \> Bantu noun class  \\
1\SG{} \> 1st person singular \\
2\SG{} \> 2nd person singular \\
\AND{} \> anditive/thither \\
\APPL{} \> applicative extension \\
\ASS{} \> associative marker \\ \> (connective) \\
\CONTR{} \> contrast (gloss from \\ \> \citealt{Mous2003}) \\
% \COP{} \> copula \\
\COP{}2 \> non-human/inanimate copula \\
% \DEM{} \> demonstrative \\
% \DIR{} \> directional \\
\DISC{} \> discourse \\
% \DIST{} \> distal \\
% \DUR{} \> durative verbal extension \\
\EMPH{} \> emphatic (greater contrast) \\
\EXCL{} \> exclamation \\
FR \> French \\
% \DUR{} \> durative suffix \\
% \FOC{} \> focus marker \\
% \INF{} \> infinitive \\
\MID{} \> middle \\
% \NEG{} \> negation \\
\OBJ{} \> object \\
\PST{}1 \> first-degree past  \\ \> tense  (a few moments ago) \\
\PST{}2 \> second-degree past \\ \> tense (hodiernal) \\
\PST{}3 \> third-degree past  \\ \> tense (yesterday and back) \\
\PST{}4 \> fourth-degree past \\ \> tense (ancient past) \\
% \POSS{} \> possessive \\
\PREP{} \> preposition \\
% \PRS{} \> present tense marker \\
\PRO{} \> pronoun \\
\PROC{} \> procedural tense \\
% \PROX{} \> proximal \\
% \PTCP{} \> past participle \\
% \Q{} \> question particle \\
% \REL{} \> relative marker \\
\REP{} \> repetitive suffix (action \\ \> repeated) \\ 
\SBJV{} \> subjective mood \\
\SM{} \> subject marker \\
\VEN{} \> venitive/hither
\end{tabbing}
\end{multicols}

\printbibliography[heading=subbibliography,notkeyword=this]
\end{document}
