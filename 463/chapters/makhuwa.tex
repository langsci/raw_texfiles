\documentclass[output=paper]{langscibook}
\ChapterDOI{10.5281/zenodo.14833618}
\author{Jenneke van der Wal\orcid{}\affiliation{Leiden University}}
\title{The expression of information structure in Makhuwa-Enahara}
\abstract{This chapter describes which morphosyntactic strategies Makhuwa-Enahara uses to structure the information in a sentence. Makhuwa-Enahara is a language in which word order plays an important role in expressing information structure: the preverbal domain is reserved for topics, non-topical subjects occur in subject inversion, and there is an immediate after verb (IAV) focus position. The conjoint verb form expresses exclusive focus on the IAV constituent, whereas the disjoint verb form is an elsewhere form. A topic doubling construction involving a fronted infinitive may be used to express verum, and three types of cleft sentences are used to express focus. Pronouns and (emphatic) demonstratives are used to indicate referent activation and contrast.}
\IfFileExists{../localcommands.tex}{
  \addbibresource{../localbibliography.bib}
  \usepackage{langsci-optional}
\usepackage{langsci-gb4e}
\usepackage{langsci-lgr}

\usepackage{listings}
\lstset{basicstyle=\ttfamily,tabsize=2,breaklines=true}

%added by author
% \usepackage{tipa}
\usepackage{multirow}
\graphicspath{{figures/}}
\usepackage{langsci-branding}

  
\newcommand{\sent}{\enumsentence}
\newcommand{\sents}{\eenumsentence}
\let\citeasnoun\citet

\renewcommand{\lsCoverTitleFont}[1]{\sffamily\addfontfeatures{Scale=MatchUppercase}\fontsize{44pt}{16mm}\selectfont #1}
   
  %% hyphenation points for line breaks
%% Normally, automatic hyphenation in LaTeX is very good
%% If a word is mis-hyphenated, add it to this file
%%
%% add information to TeX file before \begin{document} with:
%% %% hyphenation points for line breaks
%% Normally, automatic hyphenation in LaTeX is very good
%% If a word is mis-hyphenated, add it to this file
%%
%% add information to TeX file before \begin{document} with:
%% %% hyphenation points for line breaks
%% Normally, automatic hyphenation in LaTeX is very good
%% If a word is mis-hyphenated, add it to this file
%%
%% add information to TeX file before \begin{document} with:
%% \include{localhyphenation}
\hyphenation{
affri-ca-te
affri-ca-tes
an-no-tated
com-ple-ments
com-po-si-tio-na-li-ty
non-com-po-si-tio-na-li-ty
Gon-zá-lez
out-side
Ri-chárd
se-man-tics
STREU-SLE
Tie-de-mann
}
\hyphenation{
affri-ca-te
affri-ca-tes
an-no-tated
com-ple-ments
com-po-si-tio-na-li-ty
non-com-po-si-tio-na-li-ty
Gon-zá-lez
out-side
Ri-chárd
se-man-tics
STREU-SLE
Tie-de-mann
}
\hyphenation{
affri-ca-te
affri-ca-tes
an-no-tated
com-ple-ments
com-po-si-tio-na-li-ty
non-com-po-si-tio-na-li-ty
Gon-zá-lez
out-side
Ri-chárd
se-man-tics
STREU-SLE
Tie-de-mann
} 
  \togglepaper[8]%%chapternumber
}{}

\begin{document}
\maketitle
\label{ch:8}



\section{Introduction}

Makhuwa is spoken in the north of Mozambique and the south of Tanzania. It is coded P31 in Guthrie’s classification, ISO code [vmw], but there are many variants of the language: Emeetto, Emihavani, Emoniga, Enlayi, Empamela, Emarevone, Esaakha, Imithupi, Erati, Exirima, Emwaja, and the variant studied for this chapter: Enahara (P31E).

An estimated 26.5\% of the Mozambican population has (a variant of) Makhuwa as their first language or language spoken most at home, which means around 5,890,000 speakers \citep{INE2017}. This makes it the largest language in Mozambique, even before the official language Portuguese (16.8\%).

Previous literature on the language includes various dictionaries, e.g. \citet{PiresPrata1960,PiresPrata1973,PiresPrata1990}, as well as work on tone (\citealt{CassimjeeKisseberth1999}, Ikorovere: \citealt{ChengKisseberth1979,ChengKisseberth1980,ChengKisseberth1981,ChengKisseberth1982}) and morphosyntax (Esaaka: \citealt{Katupha1983,Katupha1991}; Central: \citealt{Centis2001}). Specifically on information structure, \citet{Stucky1979,Stucky1979a,Stucky1985} analyses the interaction of tone and focus, and the influence of discourse on word order for Imithupi; \citet{Kröger2010} discusses the discourse function of inverted passives; \citet{Poeta2016} studies referent tracking in Swahili and Emeetto; and my own work on Enahara investigates focus in the conjoint/disjoint alternation \citep{vanderWal2009a,vanderWal2011,vanderWal2014} and the influence of information structure on word order \citep{vanderWal2008,vanderWal2009a,vanderWal2012a}. For information on general grammatical properties of Makhuwa-Enahara, I refer to \citet[chapter 2]{vanderWal2009a}.

The data were collected during various visits to Ilha de Moçambique between 2005 and 2022; if no source is mentioned, the example comes from the two Enahara databases (one in FileMaker Pro and one Online Language Database via Dative – see \url{https://www.dative.ca} and the introduction to this volume), but I also draw on earlier publications and indicate the source for examples from these publications. For the data in Dative, I used the BaSIS project methodology, available through the Leiden Repository.\footnote{\url{https://scholarlypublications.universiteitleiden.nl/handle/1887/3608096}} The data are presented in three lines, or at times in four to make the morphology transparent (transcription with surface tone, morpheme break, morpheme gloss, free translation).

The goal of the current chapter is to provide a general overview of the linguistic strategies used in Makhuwa\hyp Enahara to structure the information in a sentence. As each sentence has a verb, I first introduce the conjoint\slash disjoint alternation in \sectref{bkm:Ref95378676}, so that its influence can be understood in the other sections. Most important regarding this alternation in verb forms is that the constituent directly following the conjoint verb form is interpreted as exclusive focus. Word order is a second important strategy to express information structure, with a non-focal preverbal domain, and a dedicated immediate after verb focus position, as explained along with subject inversion in \sectref{bkm:Ref117578671}. \sectref{bkm:Ref117578690} presents a phenomenon that has not been discussed for Makhuwa previously: predicate doubling. Makhuwa uses topic doubling to express verum and contrast. In addition to the exclusive focus in the position following the conjoint verb form, Makhuwa also uses three types of copular constructions to express focus, as presented in \sectref{bkm:Ref117578839}: the basic cleft, pseudocleft, and what looks like a reverse pseudocleft. Finally, \sectref{bkm:Ref117578879} discusses independent pronouns and (emphatic) demonstratives in their use as contrastive topic markers and referent trackers marking the relative activation of referents.

\section{Conjoint/disjoint alternation}
\label{bkm:Ref95378676}
In four verbal conjugations,\footnote{The alternation is not present in other conjugations; see \citet{vanderWal2017} for a discussion of which conjugational categories typically show the alternation.} Makhuwa shows two verb forms: the conjoint and the disjoint form, illustrated per conjugation in (\ref{cjdjprs}--\ref{cjdjppfv}). The two forms differ in the verbal morphology, but also in the tone pattern on the following constituent: after the disjoint form, the noun appears as in citation form, e.g. \textit{epaphélo} ‘letter’, whereas after the conjoint form, the tonally lowered form occurs: \textit{epapheló} ‘letter’ (see also \sectref{bkm:Ref95489959} on the immediate after verb position).\footnote{\label{fn:makhuwa:predlow}Note that this tonal pattern is the same that occurs with a nominal predicate and is therefore called ``predicative lowering" (\citealt{SchadebergMucanheia2000} for Ekoti, \citealt{vanderWal2006} for Makhuwa-Enahara, \citealt{Guérois2015} for Cuwabo), also known under the name ``focus lowering" (\citealt{Devos2017} for Shangaji).} Note also that each conjugational category has its own tonal melody – see \citet[89--91]{vanderWal2009a} for a full overview of verbal conjugations.\footnote{\label{fn:makhuwa:pfv-imbricated}The perfective alternatively appears in an imbricated form, where a nasal precedes the last consonant of the verbal base, e.g. \textit{ki-le<m>p-é} ‘I have written’.}

\ea
\label{cjdjprs}
\begin{xlist}
%%EAX
\exi{\CJ}
%%JUDGEMENT
[]{
%%LABEL
%%CONTEXT
%%LINE2
\gll
Ki-\textbf{n}-lép-\textbf{á} e-papheló.\\
%%LINE3
1\SG{}.\SM{}-\PRS{}.\CJ{}-write-\FV{}  9-letter\\
%%TRANS1
\glt
‘I’m writing a letter.’\\
%%TRANS2
}
%%EXEND

%%EAX
\exi{\DJ}
%%JUDGEMENT
[]{
%%LABEL
%%CONTEXT
%%LINE2
\gll
Ki-\textbf{náá}-lép-\textbf{a}.\\
%%LINE3
1\SG{}.\SM{}-\PRS{}.\DJ{}-write-\FV{}\\
%%TRANS1
\glt
‘I’m writing (it).’\\
%%TRANS2
}
%%EXEND

\end{xlist}
\z

\ea
\label{cjdjpfv}
\begin{xlist}
%%EAX
\exi{\CJ}
%%JUDGEMENT
[]{
%%LABEL
%%CONTEXT
%%LINE2
\gll
Ki-lep-\textbf{alé} e-papheló.\\
%%LINE3
1\SG{}.\SM{}-write-\PFV{}.\CJ{}  9-letter\\
%%TRANS1
\glt
‘I’ve written a letter.’\\
%%TRANS2
}
%%EXEND

%%EAX
\exi{\DJ}
%%JUDGEMENT
[]{
%%LABEL
%%CONTEXT
%%LINE2
\gll
K-\textbf{oo}-lép-\textbf{a}.\\
%%LINE3
1\SG{}.\SM{}-\PFV{}.\DJ{}-write-\FV{}\\
%%TRANS1
\glt
‘I’ve written (it).’\\
%%TRANS2
}
%%EXEND
\end{xlist}
\z

\ea
\label{cjdjipfv}
\begin{xlist}
%%EAX
\exi{\CJ}
%%JUDGEMENT
[]{
%%LABEL
%%CONTEXT
%%LINE2
\gll
K-\textbf{aa}-lép-\textbf{á} e-papheló.\\
%%LINE3
1\SG{}.\SM{}-\IPFV{}-write-\FV{}.\CJ{}  9-letter\\
%%TRANS1
\glt
‘I wrote a letter.’\\
%%TRANS2
}
%%EXEND

%%EAX
\exi{\DJ}
%%JUDGEMENT
[]{
%%LABEL
%%CONTEXT
%%LINE2
\gll
K-\textbf{aánáa}-lép-\textbf{a}.\\
%%LINE3
1\SG{}.\SM{}-\IPFV{}.\DJ{}-write-\FV{}\\
%%TRANS1
\glt
‘I wrote (it).’\\
%%TRANS2
}
%%EXEND

\end{xlist}
\z


\ea
\label{cjdjppfv}
\begin{xlist}
%%EAX
\exi{\CJ}
%%JUDGEMENT
[]{
%%LABEL
%%CONTEXT
%%LINE2
\gll
K-\textbf{aa}-lep-\textbf{álé} e-papheló.\\
%%LINE3
1\SG{}.\SM{}-\PST{}-write-\PFV{}.\CJ{}  9-letter\\
%%TRANS1
\glt
‘I had written a letter.’\\
%%TRANS2
}
%%EXEND

%%EAX
\exi{\DJ}
%%JUDGEMENT
[]{
%%LABEL
%%CONTEXT
%%LINE2
\gll
K-\textbf{aahí}-lép-\textbf{a}.\\
%%LINE3
1\SG{}.\SM{}-\PST{}.\PFV{}.\DJ{}-write-\FV{}\\
%%TRANS1
\glt
‘I had written (it).’\\
%%TRANS2
}
%%EXEND

\end{xlist}
\z


The two forms do not differ in their tense-aspect semantics, as they can be used for example in a question-answer pair, as in \xref{bkm:Ref95489793}, where tense and aspect are typically kept constant.

\ea
\label{bkm:Ref95489793}
\begin{xlist}[Q (\CJ)]
%%EAX
\exi{Q (\CJ)}
%%JUDGEMENT
%%LABEL
%%CONTEXT
%%LINE1
Aniíríhani\footnote{Class 2 (plural) marking is also used for singular referents to express respect, both on nouns and in agreement.} ehópá iye?\\
%%LINE2
\gll
a-n-iir-ih-a=ni  e-hopa  iye\\
%%LINE3
2\SM{}-\PRS.\CJ{}-do-\CAUS{}-\FV{}=what  10-fish  10.\DEM{}.\DIST{}\\
%%TRANS1
\glt
‘What is s/he doing with/to the fish?’\\
%%TRANS2
%%EXEND

%%EAX
\exi{A (\DJ)}
%%JUDGEMENT
%%LABEL
%%CONTEXT
%%LINE1
Anámwááríka.\\
%%LINE2
\gll
a-na-arik-a\\
%%LINE3
2\SM{}-\PRS{}.\DJ{}-fry-\FV{}\\
%%TRANS1
\glt ‘S/he is frying them.’
%%TRANS2
%%EXEND
\end{xlist}
\z

Instead, the forms differ in the relation between the verb and following constituent. The conjoint form cannot occur clause-finally, whereas the disjoint form can, as illustrated in \xref{bkm:Ref129072678}.\footnote{Note that this concerns main clauses, as the formally identical form may be used finally in a relative clause – but the relative form does not show the alternation. See \sectref{bkm:Ref117578839} for discussion on relative forms in clefts.} 

\ea
\label{bkm:Ref129072678}
\begin{xlist}
%%EAX
\exi{\CJ}
%%JUDGEMENT
[]{
%%LABEL
%%CONTEXT
%%LINE2
\gll
Ki-n-tthár-á  e-seetá.\\
%%LINE3
1\SG{}.\SM{}-\PRS{}.\CJ{}-follow-\FV{}  9-sign\\
%%TRANS1
\glt
‘I’m following the sign.’\\
%%TRANS2
}
%%EXEND

%%EAX
\exi{\CJ}
%%JUDGEMENT
[*]{
%%LABEL
%%CONTEXT
%%LINE2
Ki-n-tthár-a. \\
%%LINE3
%%TRANS1
%%TRANS2
}
%%EXEND

%%EAX
\exi{\DJ}
%%JUDGEMENT
[]{
%%LABEL
%%CONTEXT
%%LINE2
\gll
Ki-náá-tthár-a.\\
%%LINE3
1\SG{}.\SM{}-\PRS{}.\DJ{}-follow-\FV{}\\
%%TRANS1
\glt
‘I’m following (it).’\\
%%TRANS2
}
%%EXEND

\end{xlist}
\z

The constituent following the conjoint form is in focus. In earlier work \citep{vanderWal2009a,vanderWal2011}, I argue that the conjoint form encodes exclusive focus on the constituent directly following it (in the immediate after verb position, see \sectref{bkm:Ref95489959}). The focus interpretation can be seen, for example, in the requirement for inherently focused interrogative constituents to follow a conjoint (not a disjoint) verb form, as in \xref{bkm:Ref95490193}.

\ea
\label{bkm:Ref95490193}
%%EAX
\begin{xlist}
\exi{\CJ}
%%JUDGEMENT
[]{
%%LABEL
%%CONTEXT
%%LINE1
Mwaapeyalé tsáyi nhútsí ulá?\\
%%LINE2
\gll
mu-apey-ale  tsayi  n-hutsi  ola\\
%%LINE3
2\PL{}.\SM{}-cook-\PFV{}.\CJ{}  how  3-sauce  3.\DEM{}.\PROX{}\\
%%TRANS1
\glt
‘How did you cook this sauce?’\\
%%TRANS2
}
%%EXEND

%%EAX
\exi{\DJ}
%%JUDGEMENT
[*]{
%%LABEL
%%CONTEXT
%%LINE1
Ohaápéya tsáyi nhútsí ulá?\\
%%LINE2
\gll
o-o-apey-a  tsayi  n-hutsi  ola\\
%%LINE3
2\SG{}.\SM{}-\PFV{}.\DJ{}-cook-\FV{}  how  3-sauce  3.\DEM{}.\PROX{}\\
%%TRANS1
\glt
int. ‘How did you cook this sauce?’
%%TRANS2
}
%%EXEND

\end{xlist}
\z


Among a range of tests that show the exclusive aspect of focus is the incompatibility with the inclusive focus-sensitive particle \textit{hata} ‘even’. If the predicate is true for \textit{even} the least likely referent, then no referents are excluded. This predicts the incompatibility with the exclusive meaning of the conjoint verb form, as borne out in \xref{bkm:Ref95490991}.

\ea
\label{bkm:Ref95490991}
(I ate many animals…)
\begin{xlist}
%%EAX
\exi{\CJ}
%%JUDGEMENT
[*]{
%%LABEL
%%CONTEXT
%%LINE1
…hatá kinkhuuralé mileká.\\
%%LINE2
\gll
hata  ki-n-khuur-ale  mileka\\
%%LINE3
until  1\SG{}.\SM{}-1\OM{}-eat-\PFV{}.\CJ{}  1.giraffe\\
%%TRANS1
\glt
int. ‘…I even ate giraffe.’\\
%%TRANS2
}
%%EXEND

%%EAX
\exi{\DJ}
%%JUDGEMENT
[]{
%%LABEL
%%CONTEXT
%%LINE1
…hatá końkhúúrá miléka.\\
%%LINE2
\gll
hata  ki-o-n-khuur-a  mileka\\
%%LINE3
until  1\SG{}.\SM{}-\PFV{}.\DJ{}-1\OM{}-eat-\FV{}  1.giraffe\\
%%TRANS1
\glt
‘…I even ate giraffe.’
%%TRANS2
}
%%EXEND

\end{xlist}
\z


The exhaustive particle \textit{paahi} ‘only’ on the other hand is compatible with the conjoint form and not the disjoint, as shown in \xref{bkm:Ref105317933}.

\ea
\label{bkm:Ref105317933}
\begin{xlist}
%%EAX
\exi{\CJ{}}
%%JUDGEMENT
[]{
%%LABEL
%%CONTEXT
%%LINE2
\gll
Ki-n-thúm’  é-tomati  paáhi.\\
%%LINE3
1\SG{}.\SM{}-\PRS{}.\CJ{}-{}buy  10-tomato  only\\
%%TRANS1
\glt
‘I buy only tomatoes.’\\
%%TRANS2
}
%%EXEND

%%EAX
\exi{\DJ}
%%JUDGEMENT
[*]{
%%LABEL
%%CONTEXT
%%LINE1
%%LINE2
\gll
Ki{}-náá{}-thúm’  é-tomati  paáhi.\\
%%LINE3
1\SG.\SM-\PRS.\DJ{}-buy  10-tomato  only\\
%%TRANS1
\glt
int. ‘I buy only tomatoes.’ \citep[1739]{vanderWal2011}
%%TRANS2
}
%%EXEND

\end{xlist}

\z

Exclusivity is also seen in the interpretation of the object \textit{ntthu} ‘person’: following the disjoint verb form, it is interpreted as a non-specific indefinite ‘someone’ as in \xref{bkm:Ref95808158:a}, but the exclusive focus following the conjoint form in \xref{bkm:Ref95808158:b} requires a set of alternatives to be generated and (partially or fully) excluded. The only way to exclude alternatives is to force a generic interpretation as ‘human being’.

\ea
\label{bkm:Ref95808158}
\ea
\label{bkm:Ref95808158:a}
\begin{xlist}
%%EAX
\exi{\DJ{}}
%%JUDGEMENT
%%LABEL
%%CONTEXT
%%LINE1
Koḿwéha ńtthu.\\
%%LINE2
\gll
ki-o-n-weh-a  n-tthu\\
%%LINE3
1\SG{}.\SM{}.\PFV{}.\DJ{}-1\OM{}-look-\FV{}  1-person\\
%%TRANS1
\glt
‘I saw someone.’\\
%%TRANS2
%%EXEND

\end{xlist}

\ex
\label{bkm:Ref95808158:b}
\begin{xlist}
%%EAX
\exi{\CJ{}}
%%JUDGEMENT
%%LABEL
%%CONTEXT
%%LINE2
\gll
Ki-m-weh-alé  n-tthú,  nki-weh-álé  e-náma.\\
%%LINE3
1\SG{}.\SM{}-1\OM{}-look-\PFV{}.\CJ{}  1-person  \NEG{}.1\SG{}-look-\PFV{}  9-animal\\
%%TRANS1
\glt
‘I saw a person/human being, not an animal.’ \citep[1740]{vanderWal2011}\\
%%TRANS2
%%EXEND
\end{xlist}

\z
\z

Similarly, a conjoint form cannot felicitously be used in a ``mention some" situation. When there are clearly multiple referents for which the predicate is true, and therefore exclusion is not warranted by the context, you have to use the disjoint form, as illustrated in \xref{bkm:Ref109479522}.

\ea
\label{bkm:Ref129265860}\label{bkm:Ref109479522}(Context: You went to the market and met many people; upon coming home you are asked ‘Who did you meet?’)

\begin{xlist}
%%EAX
\exi{\DJ{}}
%%JUDGEMENT
[]{
%%LABEL
%%CONTEXT
%%LINE1
Ko\'{m}phwánya Fernáántu.\\
%%LINE2
\gll
ki-o-n-phwany-a  Fernaantu\\
%%LINE3
1\SG{}.\SM{}-\PFV{}.\DJ{}-1\OM{}-find-\FV{}  1.Fernando\\
%%TRANS1
\glt
‘I met Fernando.’\\
%%TRANS2
}
%%EXEND

%%EAX
\exi{\CJ{}}
%%JUDGEMENT
[\textsuperscript{\#}]{
%%LABEL
%%CONTEXT
%%LINE2
\gll
Ki{}-m{}-phwany{}-alé  Fernaantú.\\
%%LINE3
1\SG{}.\SM{}-1\OM{}-find-\PFV{}.\CJ{}  1.Fernando\\
%%TRANS1
\glt
‘I met Fernando.’\\
%%TRANS2
}
%%EXEND
\end{xlist}

\z

As mentioned in \citet{vanderWal2011}, a universal quantifier cannot follow the conjoint verb form unless a contrast is made with another referent. Therefore, if there is only shima and no other dishes (context 2 in \xref{bkm:Ref109482986}), only the disjoint form may be used and not the conjoint. As shown in the different context 1 for the \CJ{} and \DJ{} forms in \xref{bkm:Ref109482986}, the conjoint form is felicitous when shima can be selected from among alternatives. The disjoint form, on the other hand, is felicitously used in a polarity-focus context.

\ea
%%EAX
\begin{xlist}
\exi{\CJ}
%%JUDGEMENT
%%LABEL
\label{bkm:Ref109482986}
%%CONTEXT
(Context 1: Various dishes on the table, which s/he hasn’t touched.\\
*Context 2: There was just a pan of shima on the table, no other dishes.)\\
%%LINE1
Ocaal’ éshima yootéene.\\
%%LINE2
\gll
o-c-ale  e-shima  e-oteene\\
%%LINE3
1\SM{}-eat-\PFV{}.\CJ{}  9-shima  9-all\\
%%TRANS1
\glt
‘S/he ate all the shima.’\\
%%TRANS2
%%EXEND

%%EAX
\exi{\DJ}
%%JUDGEMENT
%%LABEL
%%CONTEXT
(Context 1: Has the sick person eaten?\\
Context 2: There was just a pan of shima on the table, no other dishes.)\\
%%LINE1
Ohoóc’ éshima yootéene (oomálíha).\\
%%LINE2
\gll
o-o-c-a  e-shima  e-oteene  o-o-mal-ih-a\\
%%LINE3
1\SM{}-\PFV{}.\DJ{}-eat-\FV{}  9-shima  9-all  1\SM{}-\PFV{}.\DJ{}-finish-\CAUS{}-\FV{}\\
%%TRANS1
\glt 
‘S/he ate all the shima (and finished it).’
%%TRANS2
%%EXEND


\end{xlist}
\z

However, we need to be a bit more precise: the conjoint form is used when the following constituent is the focus \textit{or part of the focus}. The conjoint form is also used when focus is percolated from the constituent in the immediate after verb (IAV) focus position to the verb phrase, as seen in the question-answer pair in \xref{bkm:Ref95382615}, expressing VP focus.

%%EAX
\ea
%%JUDGEMENT
%%LABEL
\label{bkm:Ref95382615}
%%CONTEXT
(Context: Watching a video of a woman on the market. ‘What is the woman doing?’)\\
%%LINE1
%%LINE2
\gll
O-m{}-phím{}-á  maakhá.\\
%%LINE3
1\SM{}-\PRS{}.\CJ{}-measure-\FV{}  6.salt\\
%%TRANS1
\glt
‘She is measuring salt.’\\
%%TRANS2
%%EXEND


\z

The constituent following the conjoint verb form can be an object (as in \ref{bkm:Ref95382615} and other examples above), an adverb as in \xref{bkm:Ref95810342}, or even a dependent clause, such as the negative counterexpectational situative \textit{ohinátthi} \textit{ophiya} ‘not yet having arrived’ in \xref{bkm:Ref95810699} forming the answer to a content question, or the adverbial relative \textit{mahútté} \textit{wariípáyáaya} ‘when the clouds became dark’ in \xref{bkm:Ref95810696}. The latter was indicated to have an exhaustive reading on the whole adverbial phrase: only when there were dark clouds did it rain.

\ea
\label{bkm:Ref95810342}
\begin{xlist}
%%EAX
\exi{\CJ{}}
%%JUDGEMENT
%%LABEL
%%CONTEXT
%%LINE1
Killímá n’ iihipá.\\
%%LINE2
\gll
ki-n-lim-a  ni  e-hipa\\
%%LINE3
1\SG{}.\SM{}-\PRS{}.\CJ{}-cultivate  with  9-hoe\\
%%TRANS1
\glt
‘I am cultivating with a hoe.’\\
%%TRANS2
%%EXEND

%%EAX
\exi{\CJ{}}
%%JUDGEMENT
%%LABEL
%%CONTEXT
%%LINE2
\gll
E-shímá  e-ruw-iy-é  tsiítsáale / nańnáanová.\\
%%LINE3
9-shima  9\SM{}-stir-\PASS{}-\PFV{}.\CJ{}  like.that / right.now\\
%%TRANS1
\glt
‘(The) Shima was cooked like that/right now.’\\
%%TRANS2
%%EXEND

%%EAX
\exi{\CJ{}}
%%JUDGEMENT
%%LABEL
%%CONTEXT
%%LINE2
\gll
Ni-n-rúp-á  wa-khaámá-ni.\\
%%LINE3
1\PL{}.\SM{}-\PRS{}.\CJ{}-sleep-\FV{}  16-bed-\LOC{}\\
%%TRANS1
\glt
‘We sleep in a bed.’ (\citealt[49]{vanderWal2014}, cf. \citealt[221]{vanderWal2009a})\\
%%TRANS2
%%EXEND
\end{xlist}

\z

%%EAX
\ea
%%JUDGEMENT
%%LABEL
\label{bkm:Ref95810699}
%%CONTEXT
(When was the shima prepared?)\\
%%LINE1
%%LINE2
\gll
E-ruw-iy-é  Coáó  o-hi-ná-tthí  o-phíya.\\
%%LINE3
9\SM{}-stirr-\PASS{}-\PFV{}.\CJ{}  1.João  1\SM{}-\NEG{}-\CE{}-\AUX{}  15-arrive\\
%%TRANS1
\glt
‘It was prepared when João hadn’t arrived yet.’\\
%%TRANS2
%%EXEND

 \citep[57]{vanderWal2014}
\z

%%EAX
\ea
%%JUDGEMENT
%%LABEL
\label{bkm:Ref95810696}
%%CONTEXT
%%LINE1
Epúlá yaarumpé mahútté wariípályáaya.\\
%%LINE2
\gll
e-pula  e-aa-rup-ale  ma-hutte  wa{\footnotemark}-riip-ale-aaya\\
%%LINE3
9-rain  9\SM{}-\PST{}-rain-\PFV{}.\CJ{}  6-cloud  16-become.dark-\PFV{}.\REL{}-\POSS{}.2\\
\footnotetext{Class 16 can head an adverbial relative clause with a locative (‘where…’), temporal (‘when…’), or conditional meaning (‘if…’). See \citet{vanderWal2012a} for details.}
%%TRANS1
\glt
‘It rained after the clouds had become dark.’\\
%%TRANS2
{}[It rained, but not all day, only when the clouds were there.] \citep[59]{vanderWal2014}
%%EXEND

\z

Further discussion on focused clauses can be found in \citet{vanderWal2014}, where I also indicate that the focused interpretation of complement clauses is still unclear: either verb form can be used preceding a complement clause \xref{bkm:Ref95811763}, and the possible subtle differences in interpretation and/or use between the two is as yet unclear.

\ea
\label{bkm:Ref95811763}
\begin{xlist}
%%EAX
\exi{\CJ{}}
%%JUDGEMENT
%%LABEL
%%CONTEXT
%%LINE2
\gll
Ki-n-tsúwél-a  wiírá  e-tthépó  tsi-hááná  mpwína.\\
%%LINE3
1\SG{}.\SM{}-\PRS{}.\CJ{}-know-\FV{}  \COMP{}  10-elephant  10\SM{}-have  4.trunk\\
%%TRANS1
\glt
‘I know that elephants have trunks.’\\
%%TRANS2
%%EXEND

%%EAX
\exi{\DJ{}}
%%JUDGEMENT
%%LABEL
%%CONTEXT
%%LINE2
\gll
K-oo-tsúwél-a  wiírá  e-tthépó  tsi-hááná  mpwína.\\
%%LINE3
1\SG{}.\SM{}-\PFV{}.\DJ{}-know-\FV{}  \COMP{}  10-elephant  10\SM{}-have  4.trunk\\
%%TRANS1
\glt
‘I knew that elephants have trunks.’ \citep[60]{vanderWal2014}\\
%%TRANS2
%%EXEND
\end{xlist}

\z

Where the conjoint form has been analysed as expressing exclusive focus on the following constituent, the function of the disjoint verb form is less clear, and disjoint can hence be said to be an ``elsewhere form" (\citealt{vanderWal2009a}, see also \citegen[56]{Stucky1985} characterisation for Makhuwa-Imithupi as ``simply used to indicate that the action took place"). That is, the disjoint form is used sentence-finally, and when the following constituent is not in exclusive focus. This may be when the directly postverbal constituent cannot be focused, for example in the case of idiomatic objects \xref{bkm:Ref110502241} (see also \citealt{vanderWal2021}), or when there is simply no exclusive focus on a constituent. This is the case, for example, in a subject inversion construction (see \sectref{bkm:Ref95915158}), a polar question \xref{bkm:Ref109490423}, or when the object is already given and is not the point of the assertion, as further explained and illustrated in \sectref{bkm:Ref109481548}.

\ea
\label{bkm:Ref110502241}
%%EAX
\ea
%%JUDGEMENT
%%LABEL
%%CONTEXT
%%LINE1
%%LINE2
\gll
M-o-kí-ítth-á  m-ma-khúvá=ni.\\
%%LINE3
2\PL{}.\SM{}-\PFV{}.\DJ{}-1\SG{}.\OM{}-pour-\FV{}  18-6-bone-\LOC{}\\
%%TRANS1
\glt
‘You have made me demoralised.’\\
%%TRANS2%
\textsuperscript{?}`You have poured me in the bones.’
%%EXEND

%%EAX
\ex
%%JUDGEMENT
%%LABEL
%%CONTEXT
%%LINE1
%%LINE2
\gll
N-ki{}-itth{}-alé  m{}-ma{}-khúvá=ni.\\
%%LINE3
2\PL{}.\SM{}-1\SG{}.\OM{}-pour-\PFV{}.\CJ{}  18-6-bone-\LOC{}\\
%%TRANS1
\glt
*`You have made me demoralised.’\\
%%TRANS2
\textsuperscript{?}`You have poured me in the bones.’
%%EXEND

\z
\z

%%EAX
\ea
%%JUDGEMENT
%%LABEL
\label{bkm:Ref109490423}
%%CONTEXT
%%LINE2
\gll
W-oo-khúúr’  e-hópa?\\
%%LINE3
2\SG{}.\SM{}-\PFV{}.\DJ{}-eat-\FV{}  9-fish\\
%%TRANS1
\glt
‘Have you eaten fish?’\\
%%TRANS2
%%EXEND

\z

For state-of-affairs focus (focus on the lexical value of the verb), speakers prefer that the verb be the only element left in the comment, being located at the end of the sentence. This automatically makes the verb form disjoint. A possible object therefore typically precedes the verb, but may also follow. This is exemplified in \xref{bkm:Ref109490787}, where the object can appear in the periphery on either side – although the most natural answer would consists of just the inflected verb (and see Sections~\ref{bkm:Ref114751328} and~\ref{bkm:Ref95489959} on the preverbal topical status and postverbal background status of peripheral objects). Another example of state-of-affairs focus is given in \xref{bkm:Ref129854789}.

%%EAX
\ea
%%JUDGEMENT
%%LABEL
\label{bkm:Ref109490787}
%%CONTEXT
(What are you doing with the fish? Are you frying or grilling it?)\\
%%LINE1
(Ehóp’ éela) Kinámwáaneéla (ehópa).\\
%%LINE2
\gll
e-hopa  ela  ki-na-aaneel-a  e-hopa\\
%%LINE3
9-fish  9.\DEM{}.\PROX{}  1\SG{}.\SM{}-\PRS{}.\DJ{}-grill-\FV{}  9-fish\\
%%TRANS1
\glt
‘\{This fish\} I’m grilling it \{, the fish\}.’\\
%%TRANS2
%%EXEND


\z
\pagebreak
%%EAX
\ea
%%JUDGEMENT
%%LABEL
\label{bkm:Ref129854789}
%%CONTEXT
(Context: The government financially supports people during the Covid-19 crisis. The officials want a bribe to put your name on the list. So you pay the bribe but you didn’t get on the list and didn’t receive the money.)\\
%%LINE1
Só ninúúlíva paáhi (masi khaninaakéńle).\\
%%LINE2
\gll
so  ni-nuu-liv-a  paahi  masi  kha-ni-aakel-ale\\
%%LINE3
just  1\PL{}.\SM{}-\PFV{}.\PERS{}-pay-\FV{}  only  but  \NEG{}-1\PL{}.\SM{}-receive-\PFV{}\\
%%TRANS1
\glt
‘We only paid (but we didn’t receive).’\\
%%TRANS2
%%EXEND


\z

The conjoint/disjoint alternation goes hand in hand with the immediate after verb (IAV) focus position, as word order in Makhuwa is a major marker of information structure. We turn to the influence of information structure on word order next.

\section{Constituent order}
\label{bkm:Ref117578671}
Constituent order in Makhuwa is better characterised in terms of information structure than grammatical roles such as subject and object, and it can in that sense be called ``discourse configurational" \citep[see][]{KerrEtAl2023,É.Kiss1995,vanderWal2009a}. \citet[56]{Stucky1985} writes for Makhuwa-Imithupi that:
\begin{quote}
    If there is an NP at the beginning of the sentence, then that NP is what is being talked about (i.e. the topic). […] The rest of the sentence then constitutes a comment. The organization within the comment is, so far as I can tell, based on putting novel information after the verb and expected information before the verb. \citep[56]{Stucky1985}
\end{quote}
In this section I first discuss the preverbal domain with its ban on focus and preference for topics (\sectref{bkm:Ref114751328}), then the postverbal domain with focus immediately after the conjoint verb (\sectref{bkm:Ref95489959}) and non\hyp focal information in the right periphery and\slash or following the disjoint verb form (\sectref{bkm:Ref109481548}), and finally the subject inversion construction for thetics (\sectref{bkm:Ref95915158}).

\subsection{Preverbal domain}
\label{bkm:Ref114751328}
Makhuwa has a clear restriction against preverbal focus. Neither inherently focal interrogative phrases \xref{bkm:Ref96972813}, nor arguments modified by the exhaustive focus-sensitive particle \textit{paahi} ‘only’ are accepted preverbally \xref{bkm:Ref96972827}. Instead, subjects should appear in a cleft, and objects either in a cleft (see \sectref{bkm:Ref117578839}) or directly following the conjoint form (see \sectref{bkm:Ref95489959}).

\ea
\label{bkm:Ref96972813}
%%EAX
\ea
%%JUDGEMENT
[*]{
%%LABEL
%%CONTEXT
%%LINE2
\gll
Pani  o-naa-w-a?\\
%%LINE3
1.who  1\SM{}-\PRS{}.\DJ{}-come-\FV{}\\
%%TRANS1
\glt
int. ‘Who comes?’\\
%%TRANS2
}
%%EXEND

%%EAX
\ex
%%JUDGEMENT
[*]{
%%LABEL
%%CONTEXT
%%LINE2
\gll
Eshééní  o-náá-wéh-a?\\
%%LINE3
9.what  2\SG{}.\SM{}-\PRS{}.\DJ{}-look-\FV{}\\
%%TRANS1
\glt
int. ‘What do you see?’ \citep[170]{vanderWal2009a}\\
%%TRANS2
}
%%EXEND

\z
\z

\ea
\label{bkm:Ref96972827}
%%EAX
\ea
%%JUDGEMENT
[*]{
%%LABEL
%%CONTEXT
%%LINE2
\gll
E-kanétá  y-oóríipa  paáhi  y-oo-mór-él-a  va-thí.\\
%%LINE3
9-pen  9-black  only  9\SM{}-\PFV{}.\DJ{}-fall-\APPL{}-\FV{}  16-down\\
%%TRANS1
\glt
int. ‘Only the black pen fell down.’\\
%%TRANS2
}
%%EXEND


%%EAX
\ex
%%JUDGEMENT
[*]{
%%LABEL
%%CONTEXT
%%LINE2
\gll
Coakí paáhí k-aahí-ḿ-weh-a.\\
%%LINE3
1.Joaquim  only  1\SG.\SM-\PST.\PFV.\DJ-1\OM{}-look-\FV{}\\
%%TRANS1
\glt
int. ‘I saw only Joaquim.’ \citep[171]{vanderWal2009a}\\
%%TRANS2
}
%%EXEND

\z
\z

Furthermore, indefinite non-specific referents, which cannot form the topic, are not allowed in preverbal position. For the non-specific reading, the presentational construction in \xref{bkm:Ref129154547} is most appropriate.

%%EAX
\ea
%%JUDGEMENT
[\textsuperscript{\#}]{
%%LABEL
%%CONTEXT
%%LINE1
\'{N}tthú oomóra masi nkiníńtsúwéla ti paní.\\
%%LINE2
\gll
n-tthu  o-o-mor-a  masi  nki-n-n-tsuwel-a  ti  pani\\
%%LINE3
1-person  1\SM{}-\PFV{}.\DJ{}-fall-\FV{}  but  \NEG{}.1\SG{}.\SM{}-\PRS{}-1\OM{}-know-\FV{}  \COP{}  1.who\\
%%TRANS1
\glt
‘Someone fell but I don’t know who it is.’\\
%%TRANS2
}
%%EXEND

\z

%%EAX
\ea
%%JUDGEMENT
[]{
%%LABEL
\label{bkm:Ref129154547}
%%CONTEXT
%%LINE1
Oháávo omonré masi nkiníńtsúwéla (ti paní).\\
%%LINE2
\gll
o-haavo  o-mor-ale  masi  nki-n-n-tsuwel-a  ti  pani\\
%%LINE3
1\SM{}-be.there  1-fall-\PFV{}.\REL{}  but  \NEG{}.1\SG{}.\SM{}-\PRS{}-1\OM{}-know-\FV{}  \COP{}  1.who\\
%%TRANS1
\glt
‘There is someone who fell but I don’t know who it is.’\\
%%TRANS2
}
%%EXEND


\z

Instead, the preverbal domain is reserved for topics. The prototypical topic is the subject, which may be in its canonical position, or further fronted in the left periphery. A left-peripheral position of the subject is clear in \xref{bkm:Ref95912732}, as an adverbial phrase ‘a long time ago’ intervenes between the subject and the verb, and there is a prosodic break; in \xref{bkm:Ref95912734} the subject is marked by demonstratives (see further \sectref{bkm:Ref95380848}) and separated from the rest of the sentence by the adverbial clause \textit{wahalalyááwé} ‘when he stayed behind’.

%%EAX
\ea
%%JUDGEMENT
%%LABEL
\label{bkm:Ref95912732}
%%CONTEXT
%%LINE2
\gll
Namárókoló,  ekhálái  ekhalaí,  aa-rí  m-patthaní  a  nsátóro.\\
%%LINE3
1.hare  long.ago  \RED{}  1\SM{}.\PST{}-be  1-friend.\PRL{}  1.\CONN{}  1.administrator\\
%%TRANS1
\glt
‘(the) Hare, a long time ago, (he) was the friend of the administrator.’ \citep[183]{vanderWal2009a}\\
%%TRANS2
%%EXEND

\z

%%EAX
\ea
%%JUDGEMENT
%%LABEL
\label{bkm:Ref95912734}
%%CONTEXT
%%LINE1
%%LINE2
\gll
Ólé  n-lópwán’  oolé  wa-hal-aly-ááwé,   o-h-i\'{v}v’  é-púri. \\
%%LINE3
1.\DEM{}.\DIST{}  1-man  1.\DEM{}.\DIST{}  16-stay-\PFV{}.\REL{}-\POSS{}.1   1\SM{}-\PFV{}.\DJ{}-kill  9-goat \\
%%TRANS1
\glt
‘That man, when he stayed behind, (he) killed a goat.’\\
%%TRANS2
%%EXEND

\z

Objects often occur in a left-peripheral position as well, clearly functioning as a topic. It seems to be the case that objects can be preposed in order to ``evacuate" the postverbal focus position, for example when another constituent is questioned, as in \xref{bkm:Ref95381718} and \xref{bkm:Ref129080305}. A resumptive pronoun (the object marker in \xref{bkm:Ref129080194}) is present if one exists (only speech act participants and classes 1 and 2 have object markers, other classes do not).

%%EAX
\ea
%%JUDGEMENT
%%LABEL
\label{bkm:Ref95381718}
%%CONTEXT
%%LINE2
\gll
Ekólé  elá  ki{}-pwesh{}-ék{}-é  ni  sheéni?\\
%%LINE3
9.coconut  9.\DEM{}.\PROX{}  1\SG{}.\SM{}-break-\DUR{}-\SBJV{}  with  9.what\\
%%TRANS1
\glt
‘This coconut, what should/shall I break it with?’\\
%%TRANS2
%%EXEND

\z

%%EAX
\ea
%%JUDGEMENT
%%LABEL
\label{bkm:Ref129080305}
%%CONTEXT
(Context: You see that your friend has money.)\\
%%LINE1
Ntsúrúkhu uyo ophwannyé váyî?\\
%%LINE2
\gll
n-tsurukhu  oyo  o-phwany-ale  vayi\\
%%LINE3
3-money  3.\DEM{}.\MED{}  2\SG{}.\SM{}-find-\PFV{}.\CJ{}  where\\
%%TRANS1
\glt
‘Where did you get that money?’\\
%%TRANS2
%%EXEND


\z

%%EAX
\ea
%%JUDGEMENT
%%LABEL
\label{bkm:Ref129080194}
%%CONTEXT
(Context: a radio programme on child nutrition.)\\
%%LINE1
\textbf{Mwaáná} apiha mweérí sitá ohááná antséráka o\textbf{ḿ}váha maháatsa.\\
%%LINE2
\gll
mw-aana  a-phiy-ih-a  mweeri  sita  o-haan-a  o-ants-er-ak-a  o-n{}-vaha  ma-haatsa \\
%%LINE3
1-child  1\SM{}.\SIT{}-arrive-\CAUS{}-\FV{}  4.months  six   2\SG{}.\SM{}-have-\FV{}  2\SG{}.\SM{}.\SIT{}-begin-\APPL{}-\DUR{}-\FV{}  15-1\OM{}{}-give  6-porridge\\
%%TRANS1
\glt
‘The child, when s/he reaches six months, you have to start to give him/her porridge.’
%%TRANS2
%%EXEND

\z


Contrastive topics also appear in the left periphery, as illustrated in \xref{bkm:Ref96973064} where \textit{enanahi} ‘pineapples’ is fronted and contrasted with bananas, both being topics, and as is typical for contrastive topics, the sentence also contains a contrastive focus; in this case the quantifiers are constrasted. In \xref{bkm:Ref110322922}, a particular bunch of grass is selected for the protagonist and contrasted with other bunches of grass, and again the second contrast is the addressee (hence expressed as an independent pronoun \textit{wé}) vs. the friends.

%%EAX
\ea
%%JUDGEMENT
%%LABEL
\label{bkm:Ref96973064}
%%CONTEXT
%%LINE1
Kithumalé enika yińcéene, \textbf{enanáhí} kithumalé vakhaáni.\\
%%LINE2
\gll
ki-thum-ale  e-nika  e-inceene  e-nanahi  ki-thum-ale  va-khaani\\
%%LINE3
1\SG{}.\SM{}-buy-\PFV{}.\CJ{}  9-banana  9-much  9-pineapple  1\SG{}.buy-\PFV{}.\CJ{}  16-small\\
%%TRANS1
\glt
‘I bought many bananas; pineapples I bought few.’\\
%%TRANS2
%%EXEND


\z

%%EAX
\ea
%%JUDGEMENT
%%LABEL
\label{bkm:Ref110322922}
%%CONTEXT
(Context: A mother is trying to trick her son by having the Hyena hide in the middle bunch of the cut grass. The next day she said ‘My child, fetch the grass that is in the middle.’)\\
%%LINE1
Ari veeríyári wákúshe wé, makín’ áawó poótí okúshátsa ashíkwáawo.\\
%%LINE2
\gll
a-ri  va-eriyari  w-a-kush-e  we  ma-kin-aawo  pooti  o-kush-ats-a  a-shi-khw-aawo  \\
%%LINE3
6-be.\REL{}  16-middle  2\SG{}.\SM{}-\SUBS{}-carry-\SBJV{}  2\SG{}.\PRO{}  6-other-\POSS{}.2\SG{}  can  15-carry-\PLUR{}-\FV{}  2-\DIM{}-fellow-\POSS{}.2\SG{} \\
%%TRANS1
\glt
‘The one that is in the middle you should go and carry, the other ones your friends can carry.’\\
%%TRANS2
%%EXEND

\z

Topic shifts are typically marked by demonstratives (see also \sectref{bkm:Ref96973948}) and a position in the left periphery, as in \xref{bkm:Ref96973972} and \xref{bkm:Ref129081312}. As can also be seen in these examples, the pragmatic marker \textit{vano} ‘now’ is often used to mark an episode boundary – a prototypical place in narratives and discourse to switch topics.

\ea
(A: Muhammad had seven wives. B: Seven wives! I have only one and it is heavy… Now with 7 wives and not be jealous? A: But jealousy hurts. B: Jealousy hurts. A: Jealousy hurts.)\\
\begin{xlist}
%%EAX
\exi{A:}
%%JUDGEMENT
%%LABEL
\label{bkm:Ref96973972}
%%CONTEXT
%%LINE1
Vánó \textbf{ntthú úle} wuútsívelé... íi!\\
%%LINE2
\gll
vano  n-tthu  ole  o-u-tsivel-e  ii\\
%%LINE3
now  1-person  1.\DEM{}.\DIST{}  1\SM{}-2\SG{}.\OM{}-please-\SBJV{}  \EXCLAM{}\\
%%TRANS1
\glt
‘Now about that person that you like... Eh!’\\
%%TRANS2
%%EXEND

\end{xlist}

\z

%%EAX
\ea
%%JUDGEMENT
%%LABEL
\label{bkm:Ref129081312}
%%CONTEXT
(He\textsubscript{i} jumped into the tree. His\textsubscript{i} friend\textsubscript{k}, when he\textsubscript{k} saw the lion\textsubscript{m}, he\textsubscript{k} fainted. The lion\textsubscript{m} came, he\textsubscript{m} smelled him\textsubscript{k}, threw sand on him\textsubscript{k}, peed on him\textsubscript{k}. He\textsubscript{m} left.)\\
%%LINE1
%%LINE2
\gll
Váno  \textbf{ólé\textsubscript{i}}  \textbf{n{}-tsulú}  \textbf{ḿmwe},  masi  a{}-wehá$\sim$wéh{}-ak{}-a  e-tthw’    íye  ts{}-ootéene.\\
%%LINE3
now  1.\DEM{}.\DIST{}  18-up  18.\DEM{}.\DIST{}  but  1\SM{}.\SIT{}$\sim$\RED{}-look-\DUR{}-\FV{}  10-thing  10.\DEM{}.\DIST{}  10-all\\
%%TRANS1
\glt
‘Now the one up there seeing everything,…’
%%TRANS2
%%EXEND

\z

Multiple topics are equally allowed, and occur with some frequency in narratives and spontaneous discourse, as illustrated in \xxref{bkm:Ref96973203}{bkm:Ref96973205}. This can be a mix of arguments and adverbial phrases, as was also seen above in \xref{bkm:Ref95912732} and \xref{bkm:Ref95912734}. Note also that a boundary high tone may be present to indicate continuation, which in \xref{bkm:Ref109900589} results in a rising tone on the last vowel of the topic ‘the/a child who starts to eat’.

%%EAX
\ea
%%JUDGEMENT
%%LABEL
\label{bkm:Ref96973203}
%%CONTEXT
(There once were a man and his wife. They got a daughter. Their daughter grew up and reached puberty. The man said:)\\
%%LINE1
%%LINE2
\gll
Numwáár’  uulá,  ń-tthú  o-ni-ń-thél-a,  a-kush-ék-é  e-ttánká  nlokó  iya-íya.\\
%%LINE3
1.virgin  1.\DEM{}.\PROX{}  1-person  1{}-\PRS{}-1\OM{}-marry-\FV{}.\REL{}  1\SM{}-carry-\DUR{}-\SBJV{}  10-basket  10.ten  10.\DEM{}.\PROX{}-\RED{}\\
%%TRANS1
\glt
‘This girl, the one who wants to marry her should take these ten baskets.’ \citep[182]{vanderWal2009a}\\
%%TRANS2
%%EXEND

%%EAX
\ex
%%JUDGEMENT
%%LABEL
\label{bkm:Ref109900589}
%%CONTEXT
(We wanted to know, this child who is starting to eat, how many times each day should s/he be fed?)\\
%%LINE1
Mwaámáne ompácéra óc\textbf{ǎ}, kula nihíkú, nihááná nincíháká, ekóhá tthaarú.\\
\gll
%%LINE2
mwaamane  o-n-pacer-a  o-ca  kula  ni-hiku  ni-haan-a  ni-n-c-ih-ak-a  ekoha  tthaaru.\\
%%LINE3
1.child  1-\PRS{}-start-\FV{}.\REL{}  15-eat  each  5-day  1\PL{}.\SM{}-have-\FV{}  1\PL{}.\SM{}-1\OM{}-eat-\CAUS{}-\DUR{}-\FV{}  10.times  three\\
%%TRANS1
\glt
‘The/a child who starts to eat, every day we have to feed him/her three times.’\\
%%TRANS2
%%EXEND
\pagebreak
%%EAX
\ex
%%JUDGEMENT
%%LABEL
\label{bkm:Ref96973205}
%%CONTEXT
%%LINE1
Átthw’ úúlúpale íyó etthw’ íyu anámwáne ntokó híyáánó khiívó entsúwélḛḛhu.\\
%%LINE2
\gll
a-tthu  a-ulupale  eyo  e-tthu  eyo  anamwane  ntoko  hiyaano   kha-e-haavo  e-n-tsuwel-ḛḛhu \\
%%LINE3
2-person  2-big  9.\DEM{}.\MED{}  9-thing  9.\DEM{}.\MED{}  2.child  like  1\PL{}.\PRO{}  \NEG{}-9\SM{}-be.there  9{}-\PRS{}-know.\REL{}-\POSS{}.1\PL{} \\
%%TRANS1
\glt
‘Old people; those things; children like us don’t know anything about.’\\
%%TRANS2
lit. ‘Old people; those things; children like us; there isn’t what we know.’\\
%%EXEND

\z

Finally, tail-head linking also structures the information: after an event is presented in the comment, it is then repeated at the start of the next sentence so as to establish it as the new anchoring point. In Makhuwa, this is done by using the resumptive infinitive, as in \xref{bkm:Ref117580122}.

%%EAX
\ea
%%JUDGEMENT
%%LABEL
\label{bkm:Ref117580122}
%%CONTEXT
(A long time ago, when the world was virgin, there was a man and his wife.)\\
%%LINE1
%%LINE2
\gll
Hú-nyar-ák-á  mwan’  áyá  n-thíyána.\\
%%LINE3
\NARR{}-bear-\DUR{}-\FV{}  1.child  1.\POSS{}.2  1-woman\\
%%TRANS1
\glt
‘They gave birth to a daughter.’\\
%%TRANS2
%%EXEND

%%EAX
\sn
%%JUDGEMENT
%%LABEL
%%CONTEXT
%%LINE1
\textbf{Nuunyara mwan’ áya}, mwaáná ohuńnúwá.\\
%%LINE2
\gll
nuu-nyar-a  mw-aana  aya  mw-aana  o-o-unnuw-a\\
%%LINE3
\RSM{}-bear-\FV{}  1-child  1.\POSS{}.2  1-child  1\SM{}-\PFV{}.\DJ{}-grow-\FV{}\\
%%TRANS1
\glt
‘Having given birth to their child, the child grew.’
%%TRANS2
%%EXEND
\z

In summary, the preverbal domain preferentially contains topics, establishing the anchoring point for the new information in the comment. We now turn to the postverbal domain, where a dedicated focus position can be distinguished from the non-focal part of the comment, as presented in Sections~\ref{bkm:Ref95489959} and~\ref{bkm:Ref109481548}, respectively.

\subsection{Postverbal: IAV focus position}
\label{bkm:Ref95489959}
The position immediately after the conjoint verb form is reserved for exclusive focus, as argued in \citet{vanderWal2011}. Constituent order and verbal conjugation thus act together: both the position and the conjoint form of the verb (see \sectref{bkm:Ref95378676}) conspire to mark the focus in Makhuwa. We can see the effect of the immediate after verb (IAV) position for inherently focused interrogatives, as illustrated in \xref{bkm:Ref95378898} for the adverb ‘how’ and in \xref{bkm:Ref95378900} for the questioned Theme argument in a ditransitive: when postverbal (i.e. not in a cleft), these must occur in IAV position.

\ea
\label{bkm:Ref95378898}
%%EAX
\ea
%%JUDGEMENT
[]{
%%LABEL
%%CONTEXT
%%LINE1
Mwaapeyalé tsáyi nhútsí ulá?\\
%%LINE2
\gll
mu-apey-ale  tsayi  n-hutsi  ola\\
%%LINE3
2\PL{}.\SM{}-cook-\PFV{}.\CJ{}  how  3-sauce  3.\DEM{}.\PROX{}\\
%%TRANS1
\glt
‘How did you cook this sauce?’\\
%%TRANS2
}
%%EXEND

%%EAX
\ex
%%JUDGEMENT
[*]{
%%LABEL
%%CONTEXT
%%LINE1
Mwaapeyalé nhútsí ula tsayí?\footnote{One speaker said this might be possible as a sceptical question when the sauce is not as you think it should be. This can be understood as the question word forming its own phrase: ‘You cooked the sauce. How?!’}\\
%%LINE2
\gll
mu-apey-ale  n-hutsi  ola  tsayi\\
%%LINE3
2\PL{}.\SM{}-cook-\PFV{}.\CJ{}  3-sauce  3.\DEM{}.\PROX{}  how\\
%%TRANS1
\glt
int. ‘How did you cook this sauce?’\\
%%TRANS2
}
%%EXEND


\z

\ex
\label{bkm:Ref95378900}
%%EAX
\ea
%%JUDGEMENT
[]{
%%LABEL
%%CONTEXT
%%LINE1
Apórosóóri yaakawenlé eliivuru shéeni anámwánê?\\
%%LINE2
\gll
a-porosoori  a-aa-kaw-el-ale  [e-liivuru  sheeni] 
 [anamwane] \\
%%LINE3
2-teacher  2\SM{}-2\OM{}-distribute-\APPL{}-\PFV{}.\CJ{}  {\db}10-book  what  {\db}2.child\\
%%TRANS1
\glt
‘Which books did the teacher distribute to the children?’\\
%%TRANS2
}
%%EXEND

%%EAX
\ex
%%JUDGEMENT
[*]{
%%LABEL
%%CONTEXT
%%LINE1
Yaakawenle anamwane eliivuru sheeni?\\
%%LINE2
\gll
a-aa-kaw-el-ale  [anamwane] [e-liivuru  sheeni]\\
%%LINE3
2\SM{}-2\OM{}-distribute-\APPL{}-\PFV{}.\CJ{}  {\db}2.children  {\db}10-book  what\\
%%TRANS1
\glt
int. ‘Which books did s/he distribute to the children?’\\
%%TRANS2
}
%%EXEND


\z
\z

As mentioned in \sectref{bkm:Ref95378676}, the conjoint form of the verb combines with yet another marker: predicative lowering (\PRL{}). Regardless of the interpretation, the lowering only applies to the IAV position, as seen in \xref{bkm:Ref110326752}. The citation tones (cit) of the two objects are \textit{ánáńtéko} ‘workers’ and \textit{ekamítsa} ‘shirts’; the lowered forms are \textit{anańtéko} and \textit{ekamitsa}.\largerpage[2]

\ea
\label{bkm:Ref110326752}
%%EAX
\ea
%%JUDGEMENT
[]{
%%LABEL
%%CONTEXT
%%LINE1
   Aakawenlé anańtéko ekamítsa.    \jambox[1in]{\PRL{}-cit}
%%LINE2
\gll
o-a-kawel-ale  a-nanteko  e-kamitsa\\
%%LINE3
1\SM{}-2\OM{}-distribute-\PFV{}.\CJ{}  2-worker  10-shirt\\
%%TRANS1
\glt
‘S/he distributed shirts to the workers.’\\
%%TRANS2
}
%%EXEND

\ex[*]{
Aakawenlé ánáńtéko ekamitsá.    \jambox[1in]{*cit-\PRL{}}
}

%%EAX
\ex
%%JUDGEMENT
[]{
%%LABEL
%%CONTEXT
%%LINE1
  Aakawenlé ekamitsa áná\'{n}téko.    \jambox[1in]{\PRL{}-cit}
%%LINE2
\gll
o-a-kawel-ale  e-kamitsa  a-nanteko\\
%%LINE3
1\SM{}-2\OM{}-distribute-\PFV{}.\CJ{}  10-shirt  2-worker\\
%%TRANS1
\glt
‘S/he distributed shirts to the workers.’\\
%%TRANS2
}
%%EXEND

\ex[*]{
Aakawenlé ekamítsá anańtéko.    \jambox[1in]{*cit-\PRL{}}
}
\ex[*]{
Aakawenlé ekamitsa anantéko.    \jambox[1in]{* \PRL{}-\PRL{}}
}
\z
\z

It is also expected that arguments modified by the exhaustive focus particle ‘only’ are restricted to appear in IAV position. However, earlier data already showed that this was degraded but not completely ungrammatical \citep[226]{vanderWal2009a}, and the judgements in 2019 and 2022 allowed ‘only’ in a second (non-IAV) position, as in \xref{bkm:Ref95379118}. This difference between focused interrogative phrases on the one hand and exhaustive focus with ‘only’ on the other requires further investigation.\largerpage[2]

\ea
\label{bkm:Ref95379118}
%%EAX
\ea
%%JUDGEMENT
%%LABEL
%%CONTEXT
%%LINE1
Eléló Aptúlí, omvanhé [kwaatu paáhi] [eyoóca], khamvánhé mwaánúni.\\
%%LINE2
\gll
elelo  Aptuli  o-n-vah-ale  kwaatu  paahi  eyooca   kha-o-n-vah-ale  mwaanuni\\
%%LINE3
today  1.Abdul  1\SM{}-1\OM{}-give-\PFV{}.\CJ{}  1.cat  only  9.food  \NEG{}-1\SM{}-1\OM{}-give-\PFV{}  1.bird\\
%%TRANS1
\glt
‘Today Abdul gave food only to the cat, he didn’t give to the bird.’
%%TRANS2
%%EXEND

%%EAX
\ex
%%JUDGEMENT
%%LABEL
%%CONTEXT
%%LINE1
Eléló Aptúlí, omvanhé [eyoocá] [kwaátú paáhi], khamvánhé mwaánúni.\\
%%LINE2
\gll
elelo  Aptuli  o-n-vah-ale  eyooca  kwaatu  paahi     kha-o-n-vah-ale  mw-aanuni\\
%%LINE3
today  1.Abdul  1\SM{}-1\OM{}-give-\PFV{}.\CJ{}  9.food  1.cat  only  \NEG{}-1\SM{}-1\OM{}-give-\PFV{}  1-bird\\
%%TRANS1
\glt ‘Today Abdul gave food only to the cat, he didn’t give to the bird.’
%%TRANS2
%%EXEND


\z
\z

In fact, the scope of ‘only’ at the end of the phrase seems to be flexible between associating with the constituent it directly follows, the constituent in IAV, or the whole VP, as indicated in the possible follow-up phrases in \xref{bkm:Ref117580342}.

%%EAX
\ea
%%JUDGEMENT
%%LABEL
\label{bkm:Ref117580342}
%%CONTEXT
%%LINE2
\gll
O-ni-ḿ-váh-a  kwaatu  eyoócá  paáhi…\\
%%LINE3
1\SM{}-\PRS{}.\CJ{}-1\OM{}-give-\FV{}  1.cat  9.food  only\\
%%TRANS1
\glt
‘S/he only gives the cat food,…’\\
%%TRANS2
%%EXEND

%%EAX
\sn
%%JUDGEMENT
%%LABEL
%%CONTEXT
%%LINE1
%%LINE2
{\gll
  …kha-ni-ḿ-váh-a  é-tthú  e-kínáku.   \\
%%LINE3
\NEG{}-1\SM{}-1\OM{}-give-\FV{}  9-thing  9-other \\
}\jambox*{[Theme]}
%%TRANS1
\glt ‘…s/he doesn’t give anything else.’
%%TRANS2
%%EXEND


%%EAX
\sn
%%JUDGEMENT
%%LABEL
%%CONTEXT
%%LINE1
%%LINE2
{\gll
  …mw-alápwá  kha-ni-ḿ-váha.   \\
%%LINE3
1-dog  \NEG{}.1\SM{}-\PRS{}-1\OM{}-give-\FV{} \\
}\jambox*{[Recipient]}
%%TRANS1
\glt ‘…the dog s/he doesn’t give.’
%%TRANS2
%%EXEND


%%EAX
\sn
%%JUDGEMENT
%%LABEL
%%CONTEXT
%%LINE1
  …areéke.  \jambox*{[VP]}
%%LINE2
\gll
a-row-ak-e\\
%%LINE3
1\SM{}-go-\DUR{}-\SBJV{}\\
%%TRANS1
\glt ‘…and goes away.’
%%TRANS2
%%EXEND

\z

For example \xref{bkm:Ref115119526}, the two interpretations given were association of ‘only’ with the IAV object and with the VP.

%%EAX
\ea
%%JUDGEMENT
%%LABEL
\label{bkm:Ref115119526}
%%CONTEXT
%%LINE1
%%LINE2
\gll
Amáátí  o-hel-el-alé  maakha  n-karáfá=ni  paáhí.\\
%%LINE3
1.Amade  1\SM{}-put-\APPL{}-\PFV{}.\CJ{}  6.salt  18-bottle-\LOC{}  only\\
%%TRANS1
\glt
‘Amade put only salt in the bottle.’ (He put nothing else)\\
%%TRANS2
‘Amade put salt in the bottle and that’s it.’ (He did nothing else)
%%EXEND

\z


Note also that the focus may not just be a \textit{larger} constituent than the object, but can also be \textit{smaller}, contrasting a part of the object, in sub-NP focus. As indicated in \xref{bkm:Ref114753582}, the contrasted adjective can appear pronominalised by itself or as part of a full overt NP (and repetition of the verb is always optional). We can understand the focus on the adjective as still triggering alternatives on the level of the NP rather than the adjective, the alternative set consisting of \{small cups, medium cups, big cups\}.

%%EAX
\ea
%%JUDGEMENT
%%LABEL
\label{bkm:Ref114753582}
%%CONTEXT
(Did you buy big cups?)\\
%%LINE1
Naáta, (nithummé) (ekoopo) tsikháani.\\
%%LINE2
\gll
naata  ni-thum-ale  e-koopo  tsi-khaani\\
%%LINE3
no  1\PL{}.\SM{}-buy-\PFV{}.\CJ{}  10-cup  10-small\\
%%TRANS1
\glt
‘No, (we bought) small (cups).’\\
%%TRANS2
%%EXEND


\z

Regardless of the status of the IAV focus position, Makhuwa does not allow multiple interrogatives (i.e. multiple foci), neither in postverbal position \xref{bkm:Ref114753872:a}, nor in a cleft \xref{bkm:Ref114753882}. Multiple postverbal interrogatives were only accepted as two independent questions, with a prosodic break, as in \xref{bkm:Ref114753872:b}.

\ea
\label{bkm:Ref114753872}
%%EAX
\ea
%%JUDGEMENT
[*]{
%%LABEL
\label{bkm:Ref114753872:a}
%%CONTEXT
%%LINE1
Omvahalé páni eshéeni?\\
%%LINE2
\gll
o-n-vah-ale  pani  esheeni\\
%%LINE3
2\SG{}.\SM{}-1\OM{}-give-\PFV{}.\CJ{}  1.who  9.what\\
%%TRANS1
\glt
‘Who did you give what?’ (database)\\
%%TRANS2
}
%%EXEND

%%EAX
\ex
%%JUDGEMENT
[]{
%%LABEL
\label{bkm:Ref114753872:b}
%%CONTEXT
%%LINE2
\gll
O-m-vah-alé  pánî,  eshéénî?\\
%%LINE3
2\SG{}.\SM{}-1\OM{}-give-\PFV{}.\CJ{}  1.who  9.what\\
%%TRANS1
\glt
‘To whom did you give it? what?’ \citep[250]{vanderWal2009a}\\
%%TRANS2
}
%%EXEND

\z
\z

%%EAX
\ea
%%JUDGEMENT
[*]{
%%LABEL
\label{bkm:Ref114753882}
%%CONTEXT
%%LINE1
Ti paní oraalé vayí?\\
%%LINE2
\gll
ti  pani  o-row-ale  vayi\\
%%LINE3
\COP{}  1.who  1{}-go-\PFV{}.\REL{}  where\\
%%TRANS1
\glt
‘Who went where?’\\
%%TRANS2
}
%%EXEND


\z

For the focus interpretation of the element in IAV as exclusive, see \sectref{bkm:Ref95378676} on the conjoint verb form and \citet{vanderWal2011}.

\subsection{Postverbal domain: Non-focus}
\label{bkm:Ref109481548}
Apart from the IAV focus position, the domain following the verb may also contain information that is neither topical nor focal, as well as afterthoughts in the right periphery.

Afterthoughts follow the main information of the sentence, specifying referents that were assumed by the speaker to be identifiable, but added for explicit clarity or emphasis. There may be a pause preceding the afterthought, and there is typically a phonological phrase boundary. In \xref{bkm:Ref95913286}, we see two referents in the right periphery: \textit{hiyaano} ‘we’ refers back to the contrastive topic at the beginning of the sentence and is repeated to highlight the contrast with ‘they’; and \textit{Muhámat’ ḛḛhu} ‘our Muhammad’ follows after a pause (indicated by the comma). 

%%EAX
\ea
%%JUDGEMENT
%%LABEL
\label{bkm:Ref95913286}
%%CONTEXT
(They believe strongly in God, when they marry in church “till death do us part”.)\\
%%LINE1
Vánó híyáanǒ, híyáánó ninińtthára Mohúmét’ ḛḛhu [híyááno], [Muhámat’ ḛḛhu].\\
%%LINE2
\gll
vano  hiyaano  hiyaano  ni-n-n-tthar-a  Muhamat  ḛḛhu     hiyaano  Muhamat  ḛḛhu \\
%%LINE3
now  1\PL{}.\PRO{}  1\PL{}.\PRO{}  1\PL{}.\SM{}-\PRS{}.\CJ{}-1\OM{}-follow-\FV{}  1.Muhammad  1.\POSS{}.1\PL{}  1\PL{}.\PRO{}  1.Muhammad  1.\POSS{}.1\PL{} \\
%%TRANS1
\glt
‘We on the other hand, we follow our Muhammad, we, our Muhammad.’\\
%%TRANS2
%%EXEND

\z

This example also illustrates the marking of phonological phrase boundaries. In Makhuwa, high tones are normally doubled onto the next mora, unless that mora is final in the relevant constituent. The high tone on the first occurrence of \textit{ḛḛhu} ‘our’ is not doubled onto the final syllable of the possessive (it is not \textit{ḛḛhú}), which indicates a prosodic boundary preceding the first afterthought \textit{hiyaano} ‘we’.

The same phonological boundary and pause can be seen in \xref{bkm:Ref95913831}, as the verb surfaces as \textit{kóhoótta} (low toned final vowel), not \textit{kohoóttá} (high toned final vowel). The afterthought here refers back to earlier in the conversation where the speaker explained that back in the day, it was the young girls who used to apply \textit{nsiro} (a mixture of ground treebark and water applied to the skin as a mask).

%%EAX
\ea
%%JUDGEMENT
%%LABEL
\label{bkm:Ref95913831}
%%CONTEXT
%%LINE1
\label{bkm:Ref110501136}Mí kóhoótta, enúmwáari yáaka.\\
%%LINE2
\gll
mi  ki-o-oott-a  e-numwaari  e-aka\\
%%LINE3
1\SG{}.\PRO{}  1\SG{}.\SM{}-\PFV{}.\DJ{}-smear-\FV{}  9-adolescence  9-\POSS{}.1\SG{}\\
%%TRANS1
\glt
‘I (too) used (it), in my adolescence.’\\
%%TRANS2
%%EXEND


\z

Demonstratives are also used to mark that the information in the afterthought is information that should be familiar to the listener but may require reactivation. In \xref{bkm:Ref95914183}, the radio DJ reminds the listeners where to find the station with the emphatic demonstrative \textit{yoola} (see also Sections~\ref{bkm:Ref96973948} and~\ref{bkm:Ref109492056} on demonstratives).

%%EAX
\ea
%%JUDGEMENT
%%LABEL
\label{bkm:Ref95914183}
%%CONTEXT
%%LINE1
...wiírá mwiiréké mpántta wa prógram’ ḛḛhu \textbf{yoolá} wa séntu trés ponto nóove.\\
%%LINE2
\gll
wiira  mu-iir-ak-e  mpantta  o-a  programa  ḛḛhu yoola  o-a  sentu  tres  ponto  nove \\
%%LINE3
\COMP{}  2\PL{}.\SM{}-do-\DUR{}-\SBJV{}  3.part  3-\CONN{}  1.programme  1.\POSS{}.1\PL{}  1.\E.\DEM{}.\PROX{}  1-\CONN{}  hundred  three  point  nine \\
%%TRANS1
\glt
‘...so that you can be part of our programme, this one on 103.9.’\\
%%TRANS2
%%EXEND

\z

It is not always easy to distinguish afterthoughts from other constituents that are postverbal but not dislocated. A case in point are postverbal subjects: these follow the verb but are part of the new information. Subject inversion is used in thetic sentences (as further detailed in \sectref{bkm:Ref95915158}), where the information is presented as neither topical nor focal, but as one piece of information with the verb. In \xref{bkm:Ref95915123}, the postverbal subject \textit{ekalawa} ‘boat’ is presented as part of the presented information in the ``out of the blue" sentence.

%%EAX
\ea
%%JUDGEMENT
%%LABEL
\label{bkm:Ref95915123}
%%CONTEXT
(Context: In a remote village on the river where boats hardly ever arrive, a child comes running from the river screaming.)\\
%%LINE1
Yoophíyá ekaláwa!\\
%%LINE2
\gll
e-o-phiy-a  e-kalawa\\
%%LINE3
9\SM{}-\PFV{}.\DJ{}-arrive-\FV{}  9-boat\\
%%TRANS1
\glt
‘A boat arrived!’\\
%%TRANS2
%%EXEND


\z

Other constituents following a disjoint verb form can similarly be analysed as neither topical nor focal, as opposed to the constituents following the conjoint form, which must be part of the focus (see \sectref{bkm:Ref95378676}). In \xref{bkm:Ref95916636}, the children are not the topic about which the information is, and neither do they form the exclusive focus, as they are active referents in the conversation and are not contrasted – they follow the \DJ{} verb form \textit{ninámwáápwátíha} ‘we make them lose out’, together forming the comment.
\pagebreak

%%EAX
\ea
%%JUDGEMENT
%%LABEL
\label{bkm:Ref95916636}
%%CONTEXT
(Context: A discussion about what we should feed babies, whether fish is harmful or not, after which it is said that fish is not the problem, but we caretakers are.)\\
%%LINE1
Ninámwáápwátíh’ \textbf{aan’ ḛḛhú} nláttú woóháavahatsa eyoólyá íyé tsimpwánéla.\\
%%LINE2
\gll
ni-naa-aa-pwat-ih-a  aana  ḛḛhu  n-lattu    o-a-o-hi-aa-vah-ats-a  eyooca  iye  tsi-n-pwanel-a \\
%%LINE3
1\PL{}.\SM{}-\PRS{}.\DJ{}-2\OM{}-lose-\CAUS{}-\FV{}  2.children  2.\POSS{}.1\PL{}  3-problem  3-\CONN{}-15-\NEG{}-2\OM{}-give-\PLUR{}-\FV{}  10.food  10.\DEM{}.\DIST{}  10-\PRS{}-improve-\FV{}.\REL{} \\
%%TRANS1
\glt
‘We are making our children lose out because of not giving them the food that makes them better.’\\
%%TRANS2
%%EXEND

\z

Similarly in \xref{bkm:Ref95916824}, the adverbial phrase \textit{yavoliyá} ‘when they are hungry’ is not the topic, but it is given together with the verb as the new information: about dogs, we assert that they bark when hungry, possibly in a verum context. Had the verb been in the conjoint form this would have meant ‘dogs bark only when hungry but not otherwise’; with the disjoint form there is no such aspect of meaning and no alternatives are implied.

\ea
\label{bkm:Ref95916824}
(You may not believe me because they are quiet now, but…)\\
\begin{xlist}
%%EAX
\exi{\DJ{}}
%%JUDGEMENT
%%LABEL
%%CONTEXT
%%LINE1
Alápwá anáákhúwá yavoliyá.\\
%%LINE2
\gll
alapwa  a-naa-khuw-a  a-a-vol-iy-a\\
%%LINE3
2.dogs  2\SM{}-\PRS{}.\DJ{}-bark-\FV{}  2\SM{}-\SIT{}-torment-\PASS{}-\FV{}\\
%%TRANS1
\glt
‘(the) Dogs bark when they are hungry.’\\
%%TRANS2
%%EXEND

\end{xlist}


\z

The postverbal domain hence contains the IAV focus position after the conjoint verb, and may host non-focal non-topical information as part of the comment, as well as afterthoughts in the right periphery. There is a bit more to say about postverbal subjects, which we turn to now.

\subsection{Subject inversion}
\label{bkm:Ref95915158}
In parallel with the preference for topics in the preverbal domain, non-topical subjects in Makuwa-Enahara appear postverbally. Of the various types of subject inversion constructions (see the overview in \citealt{MartenvanderWal2014}), Makhuwa only allows Agreeing Inversion, whereby the subject always determines the subject marker. Subject inversion in Makhuwa expresses a thetic statement, announcing new information (see \citealt{Sasse1996,Sasse2006} for overviews of theticity), as in \xref{bkm:Ref95377065} and \xref{bkm:Ref95379220}. See also \citet{Kröger2010} for the function of inversion in the larger discourse for Emarevone.

%%EAX
\ea
%%JUDGEMENT
%%LABEL
\label{bkm:Ref95377065}
%%CONTEXT
(Context: Sharing the news you just heard.)\\
%%LINE1
Yootsámá ekaláwá oMasákása.\\
%%LINE2
\gll
e-o-tsam-a  e-kalawa  o-Masakasa\\
%%LINE3
9\SM{}-\PFV{}.\DJ{}-sink-\FV{}  9-boat  17-Masakasa\\
%%TRANS1
\glt
‘A boat sunk at Masakasa.’\\
%%TRANS2
%%EXEND


\z

%%EAX
\ea
%%JUDGEMENT
%%LABEL
\label{bkm:Ref95379220}
%%CONTEXT
%%LINE2
\gll
O-náá-ki-weréy-á  n-thána.\\
%%LINE3
3\SM{}-\PRS{}.\DJ{}-1\SG{}-hurt-\FV{}  3-back\\
%%TRANS1
\glt
‘My back hurts.’ \citep[198]{vanderWal2009a}\\
%%TRANS2
%%EXEND

\z


In subject inversion, the verb may also be inflected as the narrative (marked by prefix \textit{khu}{}-), which is used in sequences of events \citep{vanderWal2009a}. This is shown in the following lines from a tufo\footnote{Tufo is a traditional dance performed in northern Mozambique, accompanied by drums and singing.} song about the history of Ilha de Moçambique.

\ea
%%EAX
\ea
%%JUDGEMENT
%%LABEL
%%CONTEXT
%%LINE1
Khíyátekiya manúpá ni misíkítthi (x2).\\
%%LINE2
\gll
khu-ya-tek-iy-a  ma-nupa  ni  mi-sikitthi\\
%%LINE3
\NARR{}-\IPFV{}-build-\PASS{}-\FV{}  6-house  and  4-mosque\\
%%TRANS1
\glt
‘Houses and mosques were built,’\\
%%TRANS2
%%EXEND

%%EAX
\ex
%%JUDGEMENT
%%LABEL
%%CONTEXT
%%LINE1
Wasańtímíńkú khíyátekiya epońti (x2).\\
%%LINE2
\gll
wa-santuminku  khu-ya-tek-iy-a  e-ponti\\
%%LINE3
16-Santo.Domingo  \NARR{}-\IPFV{}-build-\PASS{}-\FV{}  9-bridge\\
%%TRANS1
\glt
‘at Santo Domingo a bridge was built,’\\
%%TRANS2
%%EXEND

%%EAX
\ex
%%JUDGEMENT
%%LABEL
%%CONTEXT
%%LINE1
Khíyáw’ apaphóro khíyákelá ápákéti.\\
%%LINE2
\gll
khu-ya-w-a  a-paphoro  khu-ya-kel-a  a-paketi\\
%%LINE3
\NARR{}-\IPFV{}-come-\FV{}  2-gasoline  \NARR{}-\IPFV{}-enter-\FV{}  2-ship\\
%%TRANS1
\glt
‘and there came gasoline and there entered ships.’\\
%%TRANS2
%%EXEND


\z
\z

For the introduction of referents (entity-central thetic) and for presentational and existential thetic utterances, Makhuwa uses Agreeing Inversion with \mbox{-\textit{haavo}}. This is lexicalised from \nobreakdash-\textit{khala} ‘stay, be’ and the medial demonstrative in class 16 \textit{vo} ‘there’. The noun phrase referring to the introduced referent is typically followed by a relative clause. Examples \xref{bkm:Ref95375476} and \xref{bkm:Ref95375477} illustrate the affirmative and negative form.

%%EAX
\ea
%%JUDGEMENT
%%LABEL
\label{bkm:Ref95375476}
%%CONTEXT
%%LINE1
Aháávó átthú amphééláká wiíshtárakári elaáré y’ ańkwáaya.\\
%%LINE2
\gll
a-haavo  a-tthu  a-n-pheel-ak-a  o-istarakari    elaare  e-a  a-nkw-aya \\
%%LINE3
2\SM{}-be.there  2-person  2\SM{}.\SIT{}-\PRS{}-want-\DUR{}-\FV{}  15-ruin    9.home  9-\CONN{}  2-fellow{}-\POSS{}.2 \\
%%TRANS1
\glt
‘There are people who want to ruin the home of their mates.’\\
%%TRANS2
%%EXEND

\z

%%EAX
\ea
%%JUDGEMENT
%%LABEL
\label{bkm:Ref95375477}
%%CONTEXT
%%LINE1
…khaávó ńtthú onlyá woóhíwuryá.\footnote{This speaker speaks a slightly different variant, using \textit{olya} for ‘to eat’ instead of \textit{oca}.}\\
%%LINE2
\gll
kha-haavo  n-tthu  o-n-c-a  o-a-o-hi-wurya\\
%%LINE3
\NEG{}.1\SM{}-be.there  1-person  1-\PRS{}-eat-\FV{}.\REL{}  1-\CONN{}-15-\NEG{}-drink\\
%%TRANS1
\glt
‘(I just want to tell you that) there is noone who eats without drinking.’\\
%%TRANS2
%%EXEND


\z

As shown in \citet{vanderWal2012a}, inversion constructions in Makhuwa-Enahara only take the disjoint form in conjugations that have the alternation (the conjoint form results in a pseudocleft; see \sectref{bkm:Ref95375695}), although conjugations without the alternation may also be used (the habitual, for example). The use and interpretation of subject inversion shows that it is only used in thetic utterances (as shown above). The postverbal subject cannot be in narrow focus, as shown for the question words in \xref{bkm:Ref95375798} and \xref{bkm:Ref95375801}. This is where subject inversion in Makhuwa crucially differs from the Nguni languages, for example, which are ambiguous between theticity and subject focus in (default or locative) inversion \citep[e.g.][]{Buell2005,Zerbian2006}; compare also the subject inversion constructions in other chapters in this volume.

%%EAX
\ea
%%JUDGEMENT
[*]{
%%LABEL
\label{bkm:Ref95375798}
%%CONTEXT
%%LINE2
\gll
Aahí-phíy-a  páni?\\
%%LINE3
1\SM{}.\PST{}.\PFV{}.\DJ{}-arrive-\FV{}  1.who\\
%%TRANS1
\glt
int. ‘Who arrived?’ \citep[196]{vanderWal2009a}\\
%%TRANS2
}
%%EXEND

\z

%%EAX
\ea
%%JUDGEMENT
[*]{
%%LABEL
\label{bkm:Ref95375801}
%%CONTEXT
%%LINE2
\gll
O-náá-wóóv-a  á-ráǹttáatsi  páni? \\
%%LINE3
1\SM{}-\PRS{}.\DJ{}-fear-\FV{}  2-spider  1.who\\
%%TRANS1
\glt
int: ‘Who is afraid of spiders?’ \citep[221]{vanderWal2012a}\\
%%TRANS2
}
%%EXEND

\z


Modification by the focus particle ‘only’ as in \xref{bkm:Ref110092014} is equally infelicitous in affirmative subject inversion – it is allowed in negative clauses though, negating the exhaustivity, as shown in \xref{bkm:Ref110329486}. To express exhaustivity on the subject, a cleft should be used.
\largerpage[-1]
\pagebreak

%%EAX
\ea
%%JUDGEMENT
%%LABEL
\label{bkm:Ref110092014}
%%CONTEXT
(Context: Many people are gathered in front of a house. What happened?)\\
%%LINE1
%%LINE2
\gll
A-núú-khw-á  á-páápa  (\textsuperscript{\#}paáhí).\\
%%LINE3
2\SM{}-\PFV{}.\PERS{}-die-\FV{}  2-father   {\db}only\\
%%TRANS1
\glt
‘(*only) Father died.’\\
%%TRANS2
%%EXEND

%%EAX
\ex
%%JUDGEMENT
%%LABEL
\label{bkm:Ref110329486}
%%CONTEXT
%%LINE1
Khawaálé akúnyá paáhi, n’ aálúp’ óóríipá ahoówa.\\
%%LINE2
\gll
kha-w-ale  a-kunya  paahi ni  a-lupa  a-a-o-riipa  a-o-w-a\\
%%LINE3
\NEG.2\SM{}-come-\PFV{}  2-white.person  only  and  2-person  2-\CONN{}-15-be.dark  2\SM{}-\PFV{}.\DJ{}-{}come-\FV{}\\
%%TRANS1
\glt
‘Not only white people came, black people also came.’\\
%%TRANS2
%%EXEND

\z

Makhuwa allows subject inversion with unaccusative \xref{bkm:Ref95376756}, unergative \xref{bkm:Ref95376764}, and transitive predicates \xref{bkm:Ref95376775}, as well as passives \xref{bkm:Ref95376783}. In contrast with Agreeing Inversion in Matengo, inversion with a transitive predicate takes VOS order in Makhuwa (not VSO, see \citealt{vanderWal2012a}).

%%EAX
\ea
%%JUDGEMENT
%%LABEL
\label{bkm:Ref95376756}
%%CONTEXT
(He sat down, leaned against the tree, and slept.)\\
%%LINE1
%%LINE2
\gll
Válé  ni-hoó-wá  n-láikha.\\
%%LINE3
16.\DEM{}.\DIST{}  5\SM{}-\PFV{}.\DJ{}-come  5-angel\\
%%TRANS1
\glt
‘Now there came an angel.’ \citep[189]{vanderWal2009a}\\
%%TRANS2
%%EXEND

%%EAX
\ex
%%JUDGEMENT
%%LABEL
\label{bkm:Ref95376764}
%%CONTEXT
%%LINE2
\gll
Ni-húkú  ni-motsa  ohíyú  waa-nú-mwááryá  mw-eéri.\\
%%LINE3
5-day  5-one  14.night  3\SM{}.\PST{}-\PERS{}-shine  3-moon\\
%%TRANS1
\glt
‘One night the moon was shining.’ \citep[189]{vanderWal2009a}\\
%%TRANS2
%%EXEND

%%EAX
\ex
%%JUDGEMENT
%%LABEL
\label{bkm:Ref95376775}
%%CONTEXT
(Context: You see trousers on the clothes line.)\\
%%LINE1
%%LINE2
\gll
O-núú-kátth-á  e-kaálásá  Shavyére.\\
%%LINE3
1\SM{}-\PFV{}.\PERS{}-wash-\FV{}  10-trousers  1.Xavier\\
%%TRANS1
\glt
‘Xavier washed (the) trousers!’\\
%%TRANS2
%%EXEND

\ex
\label{bkm:Ref95376783}
%%EAX
\ea
%%JUDGEMENT
%%LABEL
%%CONTEXT
%%LINE2
\gll
Noo-vár-íy-á  n-uḿmé  ni-motsá.\\
%%LINE3
5\SM{}.\PFV{}.\DJ{}-grab-\PASS{}-\FV{}  5-toad  5-one\\
%%TRANS1
\glt
‘One toad was caught.’ \citep[189]{vanderWal2009a}\\
%%TRANS2
%%EXEND



%%EAX
\ex
%%JUDGEMENT
%%LABEL
%%CONTEXT
%%LINE2
\gll
Aa-váh-íy-a  e-kanétá  anámwáne.\\
%%LINE3
2\SM{}.\PFV{}.\DJ{}-give-\PASS-\FV{}  10-pen  2.child \\
%%TRANS1
\glt
‘The children were given pens.’ \citep[198]{vanderWal2009a}\\
%%TRANS2
%%EXEND

\z
\z

Given that negative indefinites such as ‘nobody’ and ‘nothing’ cannot form topics, these too have to be postverbal. Makhuwa forms such subjects in Agreeing Inversion with a negative verb, as in \xref{bkm:Ref109898867:a}, where the negation scopes over the postverbal subject. An alternative is the presentational construction in \xref{bkm:Ref109898867:b}.

\ea
\label{bkm:Ref109898867}(Who came?)

%%EAX
\ea
%%JUDGEMENT
%%LABEL
\label{bkm:Ref109898867:a}
%%CONTEXT
%%LINE2
\gll
Kha-w-aálé  ń-tthu.\\
%%LINE3
\NEG{}.1\SM{}-come-\PFV{}  1-person\\
%%TRANS1
\glt
‘Nobody came.’\\
%%TRANS2
%%EXEND


%%EAX
\ex
%%JUDGEMENT
%%LABEL
\label{bkm:Ref109898867:b}
%%CONTEXT
%%LINE1
Khaávó owaalé.\\
%%LINE2
\gll
kha-haavo  o-w-ale\\
%%LINE3
\NEG{}.1\SM{}-be.there  1-come-\PFV{}.\REL{}\\
%%TRANS1
\glt
‘Nobody came.’, lit. ‘There isn’t who came.’\\
%%TRANS2
%%EXEND


\z
\z

To summarise, Makhuwa does not allow preverbal focus and strongly prefers a word order topic-verb-focus-rest. The IAV focus position is dependent on the use of the conjoint verb form, and what follows the disjoint verb form, if anything, is typically neither topic nor focus.

\section{Predicate doubling}
\label{bkm:Ref117578690}
The preverbal topic can also be an infinitive, and when it is followed by an inflected form of the same predicate, as in \xref{bkm:Ref95391381}, this is called predicate doubling. The predicate doubling construction (or more correctly, topic doubling) has not been described for Makhuwa. 

%%EAX
\ea
%%JUDGEMENT
%%LABEL
\label{bkm:Ref95391381}
%%CONTEXT
\label{bkm:Ref117581626}(Did you get water?)\\
%%LINE1
Oríká (maátsi) kooríkǎ.\\
%%LINE2
\gll
o-rika  maatsi  ki-o-rik-a\\
%%LINE3
15-draw  6.water  1\SG{}.\SM{}-\PFV{}.\DJ{}-draw-\FV{}\\
%%TRANS1
\glt
‘I did (already) draw water.’, lit. ‘To draw (water), I did draw (it).’\\
%%TRANS2
%%EXEND


\z

The infinitive can consist of just the verb, or the verb plus object. Note the rising tone on the final vowel in \xref{bkm:Ref117581626} to add emphasis on the truth value: I really did it. Even a postverbal subject is possible for the infinitive, as in \xref{bkm:Ref110331747}.

%%EAX
\ea
%%JUDGEMENT
%%LABEL
\label{bkm:Ref110331747}
%%CONTEXT
(Context: Someone discovered that Ali told a lie, and the fact that Ali lies is then confirmed in a conversation with someone who already knew that he lies.)\\
%%LINE1
W-oóth’ Álí, o-h-oóth-a.\\
%%LINE2
\gll
o-otha  Ali  o-o-oth-a\\
%%LINE3
15-lie  1.Ali  1\SM{}-\PFV{}.\DJ{}-lie-\FV{}\\
%%TRANS1
\glt
‘As for Ali lying, he lies indeed.’
%%TRANS2
%%EXEND

\z

Note also that the object marker (if one is present) must appear on the inflected verb, and optionally appears on the infinitive too. All four logical combinations, three of which are acceptable, are shown in \xref{bkm:Ref110331585}. It is unknown at this point what determines whether the object marker is present on the infinitive to distinguish \xref{bkm:Ref110331585:a} and \xref{bkm:Ref110331585:b} – but note the different interpretation of \xref{bkm:Ref110331585:d}.

\ea
\label{bkm:Ref110331585}
%%EAX
\ea
%%JUDGEMENT
[]{
%%LABEL
\label{bkm:Ref110331585:a}
%%CONTEXT
%%LINE2
{\gll
O-thélá,  o-ná-ń-thél-a.  \\
%%LINE3
15-marry  1\SM{}-\PRS{}.\DJ{}-1\OM{}-marry-\FV{}\\
}\jambox[1in]{[--OM,+OM]}
%%TRANS1
\glt
‘He will marry her.’\\
%%TRANS2
}
%%EXEND

%%EAX
\ex
%%JUDGEMENT
[]{
%%LABEL
\label{bkm:Ref110331585:b}
%%CONTEXT
%%LINE1
%%LINE2
{\gll
O-ń-théla  o-ná-ń-thél-a.        \\
%%LINE3
15-1\OM{}-marry  1\SM{}-\PRS{}.\DJ{}-1\OM{}-marry-\FV{}\\
}\jambox[1in]{[+OM,+OM]}
%%TRANS1
\glt
‘He will marry her.’\\
%%TRANS2
}
%%EXEND


%%EAX
\ex
%%JUDGEMENT
[*]{
%%LABEL
\label{bkm:Ref110331585:c}
%%CONTEXT
%%LINE1
%%LINE2
{\gll
O-ń{}-théla  o-náá-thél-a.  \\
%%LINE3
15-1\OM{}-marry  1\SM{}-\PRS{}.\DJ{}-marry-\FV{} \\
}\jambox[1in]{[+OM,--OM]}
%%TRANS1
\glt
int. ‘He will marry her.’\\
%%TRANS2
}
%%EXEND


%%EAX
\ex
%%JUDGEMENT
[]{
%%LABEL
\label{bkm:Ref110331585:d}
%%CONTEXT
%%LINE2
{\gll
O-thél-á  o-náá-thél-a.  \\
%%LINE3
15-marry  1\SM{}-\PRS{}.\DJ{}-marry-\FV{}\\
}\jambox[1in]{[--OM,--OM]}
%%TRANS1
\glt
‘He will marry.’ (He must; no specific person)\\
%%TRANS2
}
%%EXEND

\z
\z

Although the infinitive functions as a topic here, the interpretation of the whole construction is typically that of verum, as can be deduced from the contexts for use in \xref{bkm:Ref95391381} and \xref{bkm:Ref95391914}. 

%%EAX
\ea
%%JUDGEMENT
%%LABEL
\label{bkm:Ref95391914}
%%CONTEXT
(Don’t you know how to swim?)\\
%%LINE1
\gll
O-ráḿpeléla, ki{}-náá{}-rampeléla.\\
%%LINE2
%%LINE3
15-swim  1\SG{}.\SM{}-\PRS{}.\DJ{}-swim-\FV{}\\
%%TRANS1
\glt
‘I do know how to swim.’
%%TRANS2
%%EXEND

\z

The infinitive can also be interpreted as a contrastive topic, contrasting it with a different action, as in \xref{bkm:Ref95391807}.

%%EAX
\ea
%%JUDGEMENT
%%LABEL
\label{bkm:Ref95391807}
%%CONTEXT
(Hey, are you even listening to me?)\\
%%LINE1
Wi\'{w}wá, kińní\`{w}wá, só kinááhítthúna\footnote{At present I do not understand this verb form.} owáákhúla.\\
%%LINE2
\gll
o-iwwa  ki-nni-iww-a  so  ki-naa-hi-tthun-a  o-aakhula\\
%%LINE3
15-hear  1\SG{}.\SM{}-\HAB{}-hear-\FV{}  just  1\SG{}.\SM{}-\PRS{}.\DJ{}?-\NEG{}-want-\FV{}  15-reply\\
%%TRANS1
\glt
‘I am hearing you, it’s just that I don't want to reply.’\\
%%TRANS2
%%EXEND


\z

Note that polarity focus and verum can also be expressed simply by the verb itself, as in \xref{bkm:Ref96768886}. That is, predicate doubling is possible but not necessary in a context of contrast on the truth.

%%EAX
\ea
%%JUDGEMENT
%%LABEL
\label{bkm:Ref96768886}
%%CONTEXT
(Daniel didn’t arrive.)\\
%%LINE1
Oophíya (hḭ noomóona).\\
%%LINE2
\gll
o-o-phiy-a  hi  ni-o-m-oon-a\\
%%LINE3
1\SM{}-\PFV{}.\DJ{}-arrive-\FV{}  1\PL{}.\PRO{}  1\PL{}.\SM{}-\PFV{}.\DJ{}-1\OM{}-see-\FV{}\\
%%TRANS1
\glt
‘He did arrive (we saw him).’\\
%%TRANS2
%%EXEND


\z

As expected with a polarity focus interpretation, the disjoint verb form is used in predicate doubling for those conjugations that have the alternation. This is the only option when the verb is final, of course, but even when it is not, the conjoint form is not acceptable, as shown in \xref{bkm:Ref95391935}. The interpretation of the object does not seem to be an afterthought; rather, the verb and object together are asserted.

\ea
\label{bkm:Ref95391935}(Context: You’re ill and they want to know whether you have already eaten something.)
%%EAX
\ea
%%JUDGEMENT
[]{
%%LABEL
%%CONTEXT
%%LINE1
Ócá, kihoócá eshíma.\\
%%LINE2
\gll
o-ca  ki-o-c-a  e-shima\\
%%LINE3
15-eat-\FV{}  1\SG{}.\SM{}-\PFV{}.\DJ{}-eat-\FV{}  9-shima\\
%%TRANS1
\glt
‘I ate shima.’\\
%%TRANS2
}
%%EXEND

%%EAX
\ex
%%JUDGEMENT
[\textsuperscript{\#}]{
%%LABEL
%%CONTEXT
%%LINE1
Ócá, kicaalé eshimá.\\
%%LINE2
\gll
o-c-a  ki-c-ale  e-shima\\
%%LINE3
15-eat-\FV{}  1\SG{}.\SM{}-eat-\PFV{}.\CJ{}  9-shima\\
%%TRANS1
\glt
int. ‘I ate/did eat shima.’\\
%%TRANS2
}
%%EXEND

\z
\z

Predicate doubling with a conjoint verb is not preferred, but judged acceptable only in a context that clearly indicates focus on the postverbal object, as in \xref{bkm:Ref110329810}.

%%EAX
\ea
%%JUDGEMENT
%%LABEL
\label{bkm:Ref110329810}
%%CONTEXT
(What did you eat?)\\
%%LINE1
Ócá kicaal’ éshimá; owúryá kiwunryé maatsí.\\
%%LINE2
\gll
o-ca  ki-c-ale  e-shima  o-wurya  ki-wury-ale  maatsi\\
%%LINE3
15-eat  1\SG{}.\SM{}-eat-\PFV{}.\CJ{}  9-shima  15-drink  1\SG{}.\SM{}-drink-\PFV{}.\CJ{}  6.water\\
%%TRANS1
\glt
‘As for eating I ate shima, as for drinking I drank water.’\\
%%TRANS2
%%EXEND

\z

Predicate doubling is most natural when a discussion has been going on and a point needs to be made, as illustrated by the contexts for \xref{bkm:Ref109548576}. Context 1 is presumably out because the predicate forms the new information in the answer, but appears as a topic in the left periphery.

%%EAX
\ea
%%JUDGEMENT
%%LABEL
\label{bkm:Ref109548576}
%%CONTEXT
(\textsuperscript{\#}Context 1: ‘Where did you get that money?’\\
Context 2: After a discussion about being paid or not.)\\
%%LINE1
Olívíya kinúúlívíya / koolívíya.\\
%%LINE2
\gll
o-liv-iy-a  ki-nuu-liv-iy-a   / ki-o-liv-iy-a\\
%%LINE3
15-pay-\PASS{}-\FV{}  1\SG{}.\SM{}-\PFV{}.\PERS{}-pay-\PASS{}-\FV{}  / 1\SG{}.\SM{}-\PFV{}.\DJ{}-pay-\PASS{}-\FV{}\\
%%TRANS1
\glt
‘I was (indeed) paid.’ \\
%%TRANS2
%%EXEND

\z

The depreciative interpretation that was found for other languages (see Kîîtharaka, Rukiga, Kinyakyusa, Kirundi in this volume) is also found in Makhuwa-Enahara, as illustrated in \xref{bkm:Ref110329886} and \xref{bkm:Ref110331993}, but interestingly the intensive reading does not seem possible.

%%EAX
\ea
%%JUDGEMENT
%%LABEL
\label{bkm:Ref110329886}
%%CONTEXT
%%LINE1
%%LINE2
\gll
O-ttíkh-á  ni-náá-ttíkh-a,  masi  kha-ni-ń-tsúwel-á feto  ni-náá-kanyári. \\
%%LINE3
15-throw  1\PL{}.\SM{}-\PRS{}.\DJ{}-throw-\FV{}  but  \NEG{}-1\PL{}.\SM{}-\PRS{}-know-\FV{}   if  1\PL{}.\SM{}-\PRS{}.\DJ{}-win-\FV{} \\
%%TRANS1
\glt
‘We played, but we don’t know if we’ll win.’\\
%%TRANS2
%%EXEND

%%EAX
\ex
%%JUDGEMENT
%%LABEL
\label{bkm:Ref110331993}
%%CONTEXT
(If you want to lose weight, maybe you should run.)\\
%%LINE1
%%LINE2
\gll
O-tthyáwá  ki-náá-tthyáw-a  masi  nki-ḿ-vúkuw-a.\\
%%LINE3
15-run  1\SG{}.\SM{}-\PRS{}.\DJ{}-run-\FV{}  but  \NEG{}.1\SG{}.\SM{}-\PRS{}-diminish-\FV{}\\
%%TRANS1
\glt
‘I do run, but I’m not getting slimmer.’\\
%%TRANS2
%%EXEND


\z

Instead, an intensive reading is found with a relative clause, or what looks like a reverse pseudocleft, modifying the infinitive, as illustrated in \xxref{bkm:Ref129163666}{bkm:Ref129163670}. It does not form an independent clause, however, and it remains uncertain at this point what the structure of the clause is.\footnote{The same construction can also be used to indicate immediacy, as in \xref{ex:slappedhim}.\largerpage[2.25]

%%EAX
\ea
%%JUDGEMENT
%%LABEL
\label{ex:slappedhim}
%%CONTEXT
%%LINE1
Okhúmá okhumalyááká kimwiitthenlé epuraatá.\\
%%LINE2
\gll
o-khuma  o-khum-ale-aaka  ki-n-iith-el-ale  epuraata\\
%%LINE3
15-exit  15-exit-\PFV{}.\REL{}-\POSS{}.1\SG{}  1\SG{}.\SM{}-1\OM{}-pour-\APPL{}-\PFV{}.\CJ{}  9.slap\\
%%TRANS1
\glt
‘As soon as I came out, I slapped him.’\\
%%TRANS2
%%EXEND
\zlast
}

%%EAX
\ea
%%JUDGEMENT
%%LABEL
\label{bkm:Ref129163666}
%%CONTEXT
%%LINE1
Othéyá othennyáaká końkówa wiirutselátsa.\\
%%LINE2
\gll
o-theya  o-they-ale-aaka  ki-o-onkow-a  o-ii-ruts-el-ats-a\\
%%LINE3
15-laugh  15-laugh-\PFV{}.\REL{}-\POSS{}.1\SG{}  1\SG{}.\SM{}-\PFV{}.\DJ{}-lack-\FV{}  15-\REFL{}-pee-\APPL{}-\PLUR{}-\FV{}\\
%%TRANS1
\glt
‘The way I laughed I nearly peed myself.’\\
%%TRANS2
%%EXEND


\z

%%EAX
\ea
%%JUDGEMENT
%%LABEL
%%CONTEXT
%%LINE1
Wamoralyááwé, othéyá túthé\`{n}nyááká...\\
%%LINE2
\gll
wa-mor-ale-awe  o-theya  ti-o-they-ale-aaka\\
%%LINE3
16-fall-\PFV{}.\REL{}-\POSS{}.1  15-laugh  \COP{}-?-laugh-\PFV{}.\REL{}-\POSS{}.1\SG{}\\
%%TRANS1
\glt
‘When s/he fell down, the way/how much that I laughed...’\\
%%TRANS2
%%EXEND

%%EAX
\ex
%%JUDGEMENT
%%LABEL
\label{bkm:Ref129163670}
%%CONTEXT
%%LINE1
Othípá túthípalyáaká, matátá otééne khíyákiweréya.\\
%%LINE2
\gll
o-thipa  ti-o-thip-ale-aaka  ma-tata  a-otene  khiya-ki-werey-a\\
%%LINE3
15-dig  \COP{}-?-dig-\PFV{}.\REL{}-\POSS{}.1\SG{}  6-hand  6-all  \NARR{}.\IPFV{}-1\SG{}.\OM{}-hurt-\FV{}\\
%%TRANS1
\glt
‘The way in which I dug, both hands are hurting me!’\\
%%TRANS2
%%EXEND


\z

An alternative to express the intensive reading is to use a negated form of the auxiliary \nobreakdash-\textit{reere} ‘to lack’ and an infinitive, as in \xref{bkm:Ref129161298}.

%%EAX
\ea
%%JUDGEMENT
%%LABEL
\label{bkm:Ref129161298}
%%CONTEXT
%%LINE1
Anámwáne alé khareéré weétta!\\
%%LINE2
\gll
anamwane  ale  kha{}-reer-e  o-eetta\\
%%LINE3
2.children  2.\DEM{}.\DIST{}  \NEG{}.2\SM{}-lack-\FV{}  15-walk\\
%%TRANS1
\glt
‘Those children walked a lot!’\\
%%TRANS2
%%EXEND


\z

Unlike other languages (cf. Kîîtharaka, Kirundi, Rukiga), Makhuwa does not have in-situ doubling \xref{bkm:Ref109548598}, where an infinitive follows the verb and is in focus, neither does it allow cleft doubling \xref{bkm:Ref109899052}.

%%EAX
\ea
%%JUDGEMENT
[*]{
%%LABEL
\label{bkm:Ref109548598}
%%CONTEXT
%%LINE2
\gll
O-katth-alé  o-katthá.\\
%%LINE3
1\SM{}-wash-\PFV{}.\CJ{}  15-wash\\
%%TRANS1
\glt
int. ‘S/he is really washing.’\\
%%TRANS2
}
%%EXEND

%%EAX
\ex
%%JUDGEMENT
[*]{
%%LABEL
\label{bkm:Ref109899052}
%%CONTEXT
%%LINE1
Otheyá/*Ti othéyá othennyáaka.\\
%%LINE2
\gll
otheya/  ti  o-theya  o-they-ale-aaka\\
%%LINE3
15-laugh.\PRL{}  \COP{}  15-laugh  15-laugh-\PFV{}.\REL{}-\POSS{}.1\SG{}\\
%%TRANS1
\glt
int. ‘It’s laughing that I did.’\\
%%TRANS2
}
%%EXEND


\z

It remains for future research to establish the precise interpretational possibilities of predicate doubling, as well as the possible positions of an NP subject and object in this construction. 

\section{Clefts}
\label{bkm:Ref117578839}
As cleft constructions involve predicative nouns, I first briefly present nominal predication in Makhuwa. There are three ways to form a non-verbal predicate: by predicate lowering, the copula \textit{ti/pi}, and existential/locative verbs (\nobreakdash-\textit{ri} or \nobreakdash-\textit{khala}) (\citealt{KujathvanderWal2023}). 

In predicate lowering, as expected the noun occurs in its lowered form, that is, without the underlying first high tone (and its surface doubling). As shown in \xref{bkm:Ref95813480}, the noun \textit{maráti} takes a LHL tone pattern in citation form, but functions as the predicate with the ``lowered" LLH pattern. See \citet[119--125]{vanderWal2009a} and \citet{KujathvanderWal2023} on nominal predication in Makhuwa, and \citet{vanderWal2006} for an analysis of predicate lowering.

\ea
\label{bkm:Ref95813480}
\ea   maráti  \jambox*{LHL (citation form)}
‘disease’

%%EAX
\ex
%%JUDGEMENT
{
%%LABEL
%%CONTEXT
%%LINE2
\gll
O-síkhíni  ma{}-ratí.\\
%%LINE3
14-poverty  6-disease.\PRL{}\\}\jambox*{LLH (lowered form)}
%%TRANS1
\glt
‘Poverty is a disease.’\\
%%TRANS2
%%EXEND

\z
\z

Predicate lowering is not possible on all (pro)nouns, and the alternative is to use the copula \textit{ti} for the present tense, as in \xref{bkm:Ref95813996}. This example also shows that for the past tense, the inflected verb \nobreakdash-\textit{ri} ‘to be’ is used. 

%%EAX
\ea
%%JUDGEMENT
%%LABEL
\label{bkm:Ref95813996}
%%CONTEXT
%%LINE1
Shtóríya yaalankáaká ti / yaarí ya khálái.\\
%%LINE2
\gll
shtoriya  e-alak-ale-aka  ti  /  e-a-ri  e-a  khalai\\
%%LINE3
9.story  9-tell-\PFV{}.\REL{}-\POSS{}.1\SG{}  \COP {} /  9-\PST{}-be  9-\CONN{}  old.times\\
%%TRANS1
\glt
‘The story that I told is/was old.’\\
%%TRANS2
%%EXEND


\z

The verb \nobreakdash-\textit{ri} is also used to predicate presence in a location, e.g. ‘they are in Nacala’. The future tense uses the verb \nobreakdash-\textit{khala} ‘stay, be’. In Makhuwa non-verbal predication, tense, location, and morphophonology of the noun are the primary factors determining the predication strategy.\footnote{There are cases when either \PRL{} or \textit{ti} is possible, and here the choice seems to depend on the interpretation as predication (‘Liz is a researcher’) vs. identification/specification (‘the researcher is Liz’), as studied in \citet{Kujath2022}, and see \citet{KujathvanderWal2023}.} We will see each of these predication strategies in the different types of cleft constructions below.

  All clefts also involve a relative verb. The relative form of the verb shows a prefix agreeing with the relativised noun: in \xref{bkm:Ref129266812} it is in class 5 agreeing with \textit{ntsipo} ‘song’. The subject in a non-subject relative may be expressed as a possessive suffix on the noun, as is the case for the 1\textsuperscript{st} person plural \textit{\nobreakdash-ḛ́ḛ́hú} in \xref{bkm:Ref129266812}. See \citet{vanderWal2010a} for further analysis of Makhuwa relative clauses. Since neither the form of the verb nor the prefix show specific relative morphology, I indicate relativisation in the gloss as \REL{} at the end of the verb gloss.

%%EAX
\ea
%%JUDGEMENT
%%LABEL
\label{bkm:Ref129266812}
%%CONTEXT
%%LINE1
Ntsípó niipalḛ́ḛ́hú ninaátsímíyá “ophéńtána”.\\
%%LINE2
\gll
n-tsipo  ni{}-iip-ale-ḛḛhu  ni-naa-atsim-iy-a  o-phent-an-a\\
%%LINE3
5-song  5{}-sing-\PFV{}.\REL{}-\POSS{}.1\PL{}  5\SM{}-\PRS{}.\DJ{}-{}call-\PASS{}-\FV{}  15-love-\RECP{}-\FV{}\\
%%TRANS1
\glt
‘The song that we sang is called \textit{ophentana}.’\\
%%TRANS2
%%EXEND


\z
\subsection{Basic clefts}

The basic cleft consists of a predicative noun followed by a (free) relative clause. Subject focus obligatorily uses a (pseudo)cleft, which is why many subject questions (and answers) take the form of a cleft, as in \xref{bkm:Ref96955103}, but objects \xref{bkm:Ref95918424} and adverbs \xref{bkm:Ref95918470} can be clefted as well. The copula \textit{ti} is present in \xref{bkm:Ref96955103:a}, and \xref{bkm:Ref96955103:b} uses predicative lowering (as indicated in the gloss); example \xref{bkm:Ref96769425} shows an alternative form of the copula, \textit{pi}.

\ea
\label{bkm:Ref96955103}
%%EAX
\ea
%%JUDGEMENT
%%LABEL
\label{bkm:Ref96955103:a}
%%CONTEXT
%%LINE2
\gll
Ti  paní  o-tthik-ale  e-rrańca?\\
%%LINE3
\COP{}  1.who  1-throw-\PFV{}.\REL{}  10-orange\\
%%TRANS1
\glt
‘Who has thrown oranges?’\\
%%TRANS2
%%EXEND

%%EAX
\ex
%%JUDGEMENT
%%LABEL
\label{bkm:Ref96955103:b}
%%CONTEXT
%%LINE2
\gll
Namarokoló  o-tthik-alé.\\
%%LINE3
1.hare.\PRL{}  1-throw-\PFV{}.\REL{}\\
%%TRANS1
\glt
‘It was Hare who threw (them).’ \citep[172]{vanderWal2009a}\\
%%TRANS2
%%EXEND

\z
\z

%%EAX
\ea
%%JUDGEMENT
%%LABEL
%%CONTEXT
(Is it Maria who is holding water?)\\
%%LINE1
%%LINE2
\gll
Áyo,  ti  María  o-kush-ale  maátsi.\\
%%LINE3
yes  \COP{}  1.Maria  1-carry-\PFV{}.\REL{}  6.water\\
%%TRANS1
\glt
‘Yes, it is Maria who holds water’\\
%%TRANS2
%%EXEND


\z

%%EAX
\ea
%%JUDGEMENT
%%LABEL
\label{bkm:Ref95918424}
%%CONTEXT
%%LINE1
Etomatí tsintúmíhááwé nthíyána ola?\\
%%LINE2
\gll
e-tomati  tsi-n-thum-ih-aawe  n-thiyana  ola\\
%%LINE3
10-tomato.\PRL{}  10-\PRS{}-buy-\CAUS{}.\REL{}-\POSS{}.1  1-woman  1.\DEM{}.\PROX{}\\
%%TRANS1
\glt
‘Is it tomatoes that the woman is selling?’\\
%%TRANS2
%%EXEND


\z

%%EAX
\ea
%%JUDGEMENT
%%LABEL
\label{bkm:Ref95918470}
%%CONTEXT
%%LINE1
Ti vayí / nipuro shééni ninrówááká othúmá moóce?\\
%%LINE2
\gll
ti  vayi /  ni-puro  sheeni  ni-n-row-aaka  o-thuma  m-ooce\\
%%LINE3
\COP{}  where /  5-place.\PRL{}  what  5-\PRS{}-go.\REL{}-\POSS{}.1\SG{}  15-buy  6-egg\\
%%TRANS1
\glt
‘Where is it that I go to buy eggs?’\\
%%TRANS2
%%EXEND


\z

%%EAX
\ea
%%JUDGEMENT
%%LABEL
\label{bkm:Ref96769425}
%%CONTEXT
%%LINE2
\gll
A-tthw’  aá-lé  aa-rów-á:  “Pi-ítthú  e-nní-rúwan-el-áu?”\\
%%LINE3
2-person  2.\DEM{}.\DIST{}  2\SM{}.\IPFV{}-go-\FV{}  \COP{}-9.thing  9-\HAB{}-insult-\APPL{}.\REL{}-\POSS{}.2\SG{}\\
%%TRANS1
\glt
‘Those people were like “Is this why you are insulting us?”‘\\
%%TRANS2
%%EXEND

\z

Although examples with full NPs are accepted, the large majority of cleft examples concerns interrogatives, as well as demonstrative and personal pronouns. Full NPs seem to prefer the reverse pseudocleft construction discussed in \sectref{bkm:Ref110335230}.

The question-answer pairs above already indicate that the basic cleft is used to express focus. This is confirmed by the fact that part of an idiom loses its idiomatic meaning in a cleft \xref{bkm:Ref110335301}.

%%EAX
\ea
%%JUDGEMENT
%%LABEL
\label{bkm:Ref110335301}
%%CONTEXT
%%LINE1
%%LINE2
\gll
Maatsi  a-wury-aly-áawe.\\
%%LINE3
6.water.\PRL{}  6-drink-\PFV{}.\REL{}-\POSS{}.1\\
%%TRANS1
\glt ‘It’s water that s/he drank.’\\
%%TRANS2
*`It’s that she gave birth.’
%%EXEND


\z

More specifically, the Makhuwa cleft seems to have an exclusive focus interpretation, as indicated by how a constituent modified by exhaustive ‘only’ can appear in a cleft, as in \xref{bkm:Ref96954945} but not by scalar inclusive ‘even’, as in \xref{bkm:Ref109583490}. 

%%EAX
\ea
%%JUDGEMENT
%%LABEL
\label{bkm:Ref96954945}
%%CONTEXT
%%LINE2
\gll
E-shima  paáhí  e-c-aaly-áaka  (e-tthw’  íí-kíná,  naáta).\\
%%LINE3
9-shima.\PRL{}  only  9-eat-\PFV{}.\REL{}-\POSS{}.1\SG{}  {\db}9-thing  9-other  no\\
%%TRANS1
\glt
‘It’s only shima that I ate (other things not).’\\
%%TRANS2
%%EXEND

\z

%%EAX
\ea
%%JUDGEMENT
%%LABEL
\label{bkm:Ref109583490}
%%CONTEXT
%%LINE2
\gll
(*Hata)  E-kokhola  tsi-m-vél-íy-a.\\
%%LINE3
until  10-rubbish.\PRL{}  10-\PRS{}-sweep-\PASS{}-\FV{}.\REL{}\\
%%TRANS1
\glt
‘It’s (*even) rubbish that is being swept.’\\
%%TRANS2
%%EXEND

\z

Furthermore, a continuation excluding other referents is spontaneously given \xref{bkm:Ref96771985}, the universal quantifiers ‘every’ \xref{bkm:Ref96772046} or ‘all’ \xref{bkm:Ref109582797} are not accepted,\footnote{The same surface structure with ‘all’ can be accepted but with a different underlying structure, where the relative clause modifies the noun rather than being a free relative. That is, the whole NP including the relative clause forms the predication.
%%EAX
\ea
%%JUDGEMENT
%%LABEL
%%CONTEXT
%%LINE1
%%LINE2
\gll
  E-liivuru  ts{}-ootééne  tsi{}-som{}-aly{}-áaka.\\
%%LINE3
  10-book.PL  10-all  10-read-\PFV{}.\REL{}-\POSS{}.1\SG{}\\
%%TRANS1
  \glt ‘They’re all the books I’ve read.’ (you see a stack of books that you recognise)\\
%%TRANS2
  *‘It’s all the books that I’ve read.’
%%EXEND
  \zlast
}
and in a mention-some situation, it is infelicitous to reply with a cleft, as in \xref{bkm:Ref109587813} (compare to \xref{bkm:Ref129265860}). All these diagnostics suggest that the clefted constituent is singled out to the exclusion of some alternatives.

%%EAX
\ea
%%JUDGEMENT
[]{
%%LABEL
\label{bkm:Ref96771985}
%%CONTEXT
%%LINE2
\gll
E-nupá  e{}-tek{}-ale  Hasáani,  kha{}-tek{}-álé  e{}-ki\'{n}táali.\\
%%LINE3
9-house.\PRL{}  9-build-\PFV{}.\REL{}  1.Hasan  \NEG{}.1-build-\PFV{}  9-compound.\\
%%TRANS1
\glt
‘It is a house that Hamisi has built, he didn’t build a compound.’\\
%%TRANS2
}
%%EXEND

\z

%%EAX
\ea
%%JUDGEMENT
[*]{
%%LABEL
\label{bkm:Ref96772046}
%%CONTEXT
%%LINE2
\gll
Kata  n-tthú  o-m-múmúl-a.\\
%%LINE3
every  1-person.\PRL{}  1-\PRS{}-breathe.\REL{}\\
%%TRANS1
\glt
int. ‘It is every person who breathes.’\\
%%TRANS2
}
%%EXEND

\z

%%EAX
\ea
%%JUDGEMENT
[]{
%%LABEL
\label{bkm:Ref109582797}
%%CONTEXT
%%LINE1
Ti nyúwááno (*otééné) uuphwanyantsáákáni.\\
%%LINE2
\gll
ti  nyuwaano  a-oteene  uu-phwany-ats-ale-aaka=ni\\
%%LINE3
\COP{}  2\PL{}.\PRO{}  2{}-all  2\PL{}-find-\PLUR{}-\PFV{}.\REL{}-\POSS{}.1\SG{}=\PLA{}\\
%%TRANS1
\glt
‘It’s (*all of) you that I met.’\\
%%TRANS2
}
%%EXEND


\z

%%EAX
\ea
%%JUDGEMENT
[\textsuperscript{\#}]{
%%LABEL
\label{bkm:Ref109587813}
%%CONTEXT
(You went to the market and met many people. When you get home someone asks ‘Who did you meet at the market?’)\\
%%LINE1
%%LINE2
\gll
Fernaantú  o-phwany-aly-áaka.\\
%%LINE3
1.Fernando.\PRL{}  1-find-\PFV{}.\REL{}-\POSS{}.1\SG{}\\
%%TRANS1
\glt
‘It’s Fernando I met.’\\
%%TRANS2
}
%%EXEND


\z

As the IAV position after the conjoint form is also analysed as encoding exclusive focus, one of the questions is whether (and if so, how) the interpretation of the basic cleft differs from that of the post-conjoint form, i.e. what distinguishes \xref{bkm:Ref129853814:a} and \xref{bkm:Ref129853814:b}?

\ea
\label{bkm:Ref129853814}
%%EAX
\ea
%%JUDGEMENT
%%LABEL
\label{bkm:Ref129853814:a}
%%CONTEXT
%%LINE1
Mwinttharalé páni?\\
%%LINE2
\gll
n-n-tthar-ale  pani\\
%%LINE3
2\PL{}.\SM{}-1\OM{}-follow-\PFV{}.\CJ{}  1.who\\
%%TRANS1
\glt
‘Who did you follow?’\\
%%TRANS2
%%EXEND

%%EAX
\ex
%%JUDGEMENT
%%LABEL
\label{bkm:Ref129853814:b}
%%CONTEXT
%%LINE1
Ti paní onttharalényu vá?\\
%%LINE2
\gll
ti  pani  o-n-tthar-ale-inyu  va\\
%%LINE3
\COP{}  1.who  1-1\OM{}-follow-\PFV{}.\REL{}-\POSS{}.2\PL{}  \PP{}\\
%%TRANS1
\glt
‘Who is it that you followed?’\\
%%TRANS2
%%EXEND


\z
\z

While more targeted investigation is required to answer that question satisfactorily, there are two clear differences: first, subjects can only be focused in clefts and not in IAV position; and second, the post-conjoint focus can project to the verb phrase, whereas in a cleft that part of the clause which is not the clefted and focused constituent is explicitly backgrounded in a relative clause.

\subsection{Pseudoclefts}
\label{bkm:Ref95375695}
A pseudocleft is basically a copular structure in which the first part consists of a free relative and the second, predicative, part identifies the referent, e.g. ‘Who I met was Theresa’. The predicative postcopular noun is in focus here. Subjects \xref{bkm:Ref110408215} as well as objects \xref{bkm:Ref95923673} can be focused in a pseudocleft.

%%EAX
\ea
%%JUDGEMENT
%%LABEL
\label{bkm:Ref110408215}
%%CONTEXT
%%LINE1
Iitthukenlé nlénsó noóshéerya ti páni?\\
%%LINE2
\gll
o-ii-tthuk-el-ale  n-lenso  ni-a-o-sheerya  ti  pani\\
%%LINE3
1-\REFL{}-tie-\APPL{}-\PFV{}.\REL{}  5-cloth  5-\CONN{}-15-be.red  \COP{}  1.who\\
%%TRANS1
\glt
‘Who is the one wearing a red tie?’\\
%%TRANS2
%%EXEND


\z

%%EAX
\ea
%%JUDGEMENT
%%LABEL
\label{bkm:Ref95923673}
%%CONTEXT
%%LINE1
Álé yaarú\'{m}myááká, yaarí ashinamwáne.\\
%%LINE2
\gll
ale  a-a-rum-ale-aaka  a-a-ri  a-shi-namwane\\
%%LINE3
2.\DEM{}.\DIST{}  2-\PST{}-send-\PFV{}.\REL{}-\POSS{}.1\SG{}  2\SM{}-\PST{}-be  2-\DIM{}-child\\
%%TRANS1
\glt
‘The ones that I sent were my children.’\\
%%TRANS2
%%EXEND


\z

The predicative element is in focus in this construction, illustrated for interrogatives \xref{bkm:Ref95921031}, answers \xref{bkm:Ref95921039}, and constituents modified by ‘only’ \xref{bkm:Ref95921046}. 

%%EAX
\ea
%%JUDGEMENT
%%LABEL
\label{bkm:Ref95921031}
%%CONTEXT
%%LINE1
Onuúpúwela wiira ivvale ti paní?\\
%%LINE2
\gll
o-n-upuwel-a  wiira  o-ivv-ale  ti  pani\\
%%LINE3
2\SG{}.\SM{}-\PRS{}.\CJ{}-think-\FV{}  \COMP{}  1-kill-\PFV{}.\REL{}  \COP{}  1.who\\
%%TRANS1
\glt
‘Who do you think is the murderer?’\\
%%TRANS2
‘lit. You think that the one who killed is who?’

%%EXEND

\z

%%EAX
\ea
%%JUDGEMENT
%%LABEL
\label{bkm:Ref95921039}
%%CONTEXT
(I can’t see what is in the pan and ask what there is.)\\
%%LINE1
Orivó nramá.\\
%%LINE2
\gll
o-ri=vo  n-rama\\
%%LINE3
3-be=16.\DEM{}.\MED{}  3-rice.\PRL{}\\
%%TRANS1
\glt
‘What is there is rice.’\\
%%TRANS2
%%EXEND


\z

%%EAX
\ea
%%JUDGEMENT
%%LABEL
\label{bkm:Ref95921046}
%%CONTEXT
(The teacher gave pencils to the adults.)\\
%%LINE1
Naáta, amváhiya só anamwáne.\\
%%LINE2
\gll
naata  a-m-vah-iy-a  so  anamwáne\\
%%LINE3
no  2-\PRS{}-give-\PASS{}-\FV{}.\REL{}  only  2.children.\PRL{}\\
%%TRANS1
\glt
‘No, (the ones) who were given were only the children.’\\
%%TRANS2
%%EXEND


\z

As expected, idiomatic objects lose their interpretation in a pseudocleft, as in \xref{bkm:Ref110335791}.

%%EAX
\ea
%%JUDGEMENT
%%LABEL
\label{bkm:Ref110335791}
%%CONTEXT
%%LINE1
Wookolalyááwe mwettó.\\
%%LINE2
\gll
u-ookol-ale-awe  mwetto\\
%%LINE3
3-stretch-\PFV{}.\REL{}-\POSS{}.1  3.leg.\PRL{}\\
%%TRANS1
\glt
‘What s/he stretched is the leg.’\\
%%TRANS2
*`What s/he did is die.’

%%EXEND

\z

The function of the pseudocleft is to identify the referent for which the predicate is true, as is particularly clear in \xref{bkm:Ref110409158} but can be seen throughout this subsection.


%%EAX
\ea
%%JUDGEMENT
%%LABEL
\label{bkm:Ref110409158}
%%CONTEXT
(A woman went to the well and couldn’t carry her container of water. So she was stuck. She saw a rustle in the grass (and said) ‘Hmm, who is it?’)\\
%%LINE1
%%LINE2
\gll
“Ti  mí”,  o{}-n{}-iír{}-á  khutsupá.\\
%%LINE3
\COP{}  1\SG{}.\PRO{}  1-\PRS{}-say-\FV{}.\REL{}  1.hyena.\PRL{}\\
%%TRANS1
\glt
‘It’s me – who spoke is Hyena.’\\
%%TRANS2
%%EXEND


\z

Identification requires singling out one referent, which is why \textit{ntthu} ‘person’, which can in other contexts be used for ‘someone/anyone’, cannot form the focus unless modified by ‘one’ in \xref{bkm:Ref95921657} or interpreted generically as in \xref{bkm:Ref110409464}. 


%%EAX
\ea
%%JUDGEMENT
%%LABEL
\label{bkm:Ref95921657}
%%CONTEXT
%%LINE2
\gll
O-wa-alé  n-tthu  *(m-mótsá).\\
%%LINE3
1-come-\PFV{}.\REL{}  1-person.\PRL{}  1-one\\
%%TRANS1
\glt
‘(Only) one person came.’\\
%%TRANS2
%%EXEND

%%EAX
\ex
%%JUDGEMENT
%%LABEL
\label{bkm:Ref110409464}
%%CONTEXT
%%LINE2
\gll
E-shímá  elá,  o-c-aalé  ń{}-tthu;  khacaálé  e-náma.\\
%%LINE3
9-shima  9.\DEM{}.\PROX{}  1-eat-\PFV{}.\REL{}  1-person.\PRL{}  \NEG{}.1\SM{}-eat-\PFV{}  10-animal\\
%%TRANS1
\glt
‘This shima, who ate it was a human being, animals didn’t eat it.’\\
%%TRANS2
%%EXEND

\z

Pseudoclefts thus identify one referent to the exclusion of other referents. This can be seen in correction, as in the nice combination in \xref{bkm:Ref95922324} of the basic cleft followed by a pseudocleft, as well as the ungrammaticality of inclusive ‘even’ in \xref{bkm:Ref95921014}.


%%EAX
\ea
%%JUDGEMENT
[]{
%%LABEL
\label{bkm:Ref95922324}
%%CONTEXT
(Context: QUIS picture of three monkeys – Did that monkey steal the bread?)\\
%%LINE1
Naáta, kahi yéná iiyalé, iiyalé t’ uúla.\\
%%LINE2
\gll
naata  kahi  yena  o-iiy-ale  o-iiy-ale  ti  ola\\
%%LINE3
no  \NEG{}.\COP{}  1.\PRO{}  1-steal-\PFV{}.\REL{}  1-steal-\PFV{}.\REL{}  \COP{}  1.\DEM{}.\PROX{}\\
%%TRANS1
\glt
‘No, he wasn’t the one who stole (it), the one who stole (it) is this one.’\\
%%TRANS2
}
%%EXEND


\z

%%EAX
\ea
%%JUDGEMENT
[*]{
%%LABEL
\label{bkm:Ref95921014}
%%CONTEXT
%%LINE2
\gll
A-furahiy{}-ale  hata  alapwa.\\
%%LINE3
2-be.happy-\PFV{}.\REL{}  until  2.dog\\
%%TRANS1
\glt
int. ‘Even the dogs are happy.’ (lit. ‘The ones who are happy are even the dogs.’)\\
%%TRANS2
}
%%EXEND

\z

The focus here may be exclusive, but it does not seem to be exhaustive, as the pseudocleft can be used in the mention-some context in \xref{bkm:Ref110409947}. Notice the different contexts in \xref{bkm:Ref129268576} and \xref{bkm:Ref110410194}, where alternatives are required to make the selection. 


%%EAX
\ea
%%JUDGEMENT
%%LABEL
\label{bkm:Ref110409947}
%%CONTEXT
(What is sold in this shop?)\\
%%LINE1
Entúmíhíyá makhurá, ntsóró, meéle, sapáu...\\
%%LINE2
\gll
e-n-thum-ih-iy-a  ma-khura  n-tsoro  meele  sapau\\
%%LINE3
9-\PRS{}-buy-\CAUS{}-\PASS{}-\FV{}.\REL{}  6-oil.\PRL{}  3-rice  6.fine.maize  9.soap\\
%%TRANS1
\glt
‘What is sold is oil, rice, maize, soap, ...’\\
%%TRANS2
%%EXEND


\z

%%EAX
\ea
%%JUDGEMENT
%%LABEL
\label{bkm:Ref129268576}
%%CONTEXT
(\textsuperscript{\#}Context 1: We’re doing research into hair length. I need someone with short hair.\\
Context 2: I’m pointing out Nando in a group of people who don’t have short hair.)\\
%%LINE1
Orin’ ekharárí tsoókhúvéya Naantú.\\
%%LINE2
\gll
o-ri-na  e-kharari  tsi-a-o-khuveya  Naantu\\
%%LINE3
1-be-with.\REL{}  10-hair  10-\CONN{}-15-be.short  1.Nando.\PRL{}\\
%%TRANS1
\glt
‘The one who has short hair is Nando.’
%%TRANS2
%%EXEND

\z

%%EAX
\ea
%%JUDGEMENT
%%LABEL
\label{bkm:Ref110410194}
%%CONTEXT
(\textsuperscript{\#}Context 1: There is only one pot on the table, containing porridge.\\
Context 2: There are various dishes on the table but they chose only this one.)\\
%%LINE1
Aciyé mahaatsa otéene.\\
%%LINE2
\gll
a-c-iy-ale  ma{}-haatsa  a-oteene\\
%%LINE3
6-eat-\PASS{}-\PFV{}.\REL{}  6{}-porridge.\PRL{}  6-all\\
%%TRANS1
\glt
‘What was eaten is all the porridge.’
%%TRANS2
%%EXEND

\z

In this respect the pseudocleft is similar to the reverse pseudocleft, which we turn to now.

\subsection{\COP{}-V (reverse pseudo) clefts}
\label{bkm:Ref110335230}
Makhuwa also uses a reverse pseudocleft, in which the focused element is followed by the copula \textit{ti}/\textit{pi}+relative clause, as in \xref{bkm:Ref110413319} and \xref{bkm:Ref110495426}. The prefix on the relative verb (determined by the focused referent) is often invisible\slash unclear\slash merged with the copula. Therefore, the gloss contains a question mark in some examples, and in the first line of the examples the copula is written together with the relative verb (which is how it is pronounced).

%%EAX
\ea
%%JUDGEMENT
%%LABEL
\label{bkm:Ref110413319}
%%CONTEXT
(Context: Picture of 3 different animals. Between these animals, which one flies?)\\
%%LINE1
Mwaánúni tí\'{m}váva.\\
%%LINE2
\gll
mw-aanuni  ti  o-n-vav-a\\
%%LINE3
1-bird  \COP{}  1{}-\PRS{}-fly-\FV{}.\REL{}\\
%%TRANS1
\glt
‘The bird is the one that flies.’\\
%%TRANS2
%%EXEND


\z

%%EAX
\ea
%%JUDGEMENT
%%LABEL
\label{bkm:Ref110495426}
%%CONTEXT
(I bought a bucket but there was something else, I forgot…)\\
%%LINE1
Eparáthó títsíliyanlyáwo.\\
%%LINE2
\gll
e-paratho  ti  tsi-liyal-ale-au\\
%%LINE3
10-plate  \COP{}  10-forget-\PFV{}.\REL{}-\POSS{}.2\SG{}\\
%%TRANS1
\glt
‘Plates is what you forgot.’\\
%%TRANS2
%%EXEND


\z

Not just subjects, but objects \xref{bkm:Ref96779105} and adverbials \xref{bkm:Ref96779116} can also occur in the initial position.


%%EAX
\ea
%%JUDGEMENT
%%LABEL
\label{bkm:Ref96779105}
%%CONTEXT
(Do you prefer my shirt or his shirt?)\\
%%LINE1
Ekamis' aáwé pí/tímpheeláaka.\\
%%LINE2
\gll
e-kamisa  e-awe  pi/ti  e-m-pheel-aka\\
%%LINE3
9-shirt  9-\POSS{}.1  \COP{}  9-\PRS{}-want.\REL{}-\POSS{}.1\SG{}\\
%%TRANS1
\glt
‘His shirt is what I want.’\\
%%TRANS2
%%EXEND

  
\z

%%EAX
\ea
%%JUDGEMENT
%%LABEL
\label{bkm:Ref96779116}
%%CONTEXT
%%LINE1
Ntsúwá noóthékuwá tímwáaka.\\
%%LINE2
\gll
n-tsuwa  n-a-o-thekuwa  ti  ?-m-w-aka\\
%%LINE3
3-sun  3-\CONN{}-15-set  \COP{}  ?-\PRS{}-come.\REL{}-\POSS{}.1\SG{}\\
%%TRANS1
\glt
‘In the afternoon is when I come.’\\
%%TRANS2
%%EXEND


\z

The construction can also be used without an initial NP, and in such use it typically refers to earlier statements (discourse deixis) as in \xref{bkm:Ref129270545}, or to a manner, or it may be used as a sort of conclusion, as in \xref{bkm:Ref97026059}.

%%EAX
\ea
%%JUDGEMENT
%%LABEL
\label{bkm:Ref129270545}
%%CONTEXT
%%LINE1
Tíkíwaanale vá.\\
%%LINE2
\gll
ti  e?-ki-waan-ale  va\\
%%LINE3
\COP{}  9?{}-1\SG{}.\OM{}-bring-\PFV{}.\REL{}  16.\DEM{}.\PROX{}\\
%%TRANS1
\glt
‘That’s why I came here.’ / ‘That’s what brought me here.’\\
%%TRANS2
%%EXEND


\z

%%EAX
\ea
%%JUDGEMENT
%%LABEL
\label{bkm:Ref97026059}
%%CONTEXT
%%LINE1
Tíńkhalááya wiírá akhílí ahááná ekúrú vińcééné khaḿpá owáli.\\
%%LINE2
\gll
ti  ?-n-khal-aya  wiira  akhili  a-haana  e-kuru  vinceene   khampa  o-wali\\
%%LINE3
\COP{}  ?-\PRS{}-stay.\REL{}-\POSS{}.2  \COMP{}  intelligence  \SM{}-have  9-strength  much    than  14-strength\\
%%TRANS1
\glt
‘Therefore intelligence is stronger than force.’
%%TRANS2
%%EXEND

\z


Presumably to highlight the exhaustive identification (on which see below), the reverse pseudocleft is also used with the verb ‘to be’ in the free relative, as in \xref{bkm:Ref110415187} and \xref{bkm:Ref110415189}.

%%EAX
\ea
%%JUDGEMENT
%%LABEL
\label{bkm:Ref110415187}
%%CONTEXT
%%LINE1
Patáréro kanalisatóré kalápíǹtééró, kaĺpínteéró tárí / tírí / tákhánle mpátthány’ ááká muúlúpale.\\
%%LINE2
\gll
patarero  kanalisatore  kalapinteero  kalapinteero ti  a-ri/  ti  o{}-ri/  ti  a-khal-ale  m-patthani  aka  m-ulupale\\
%%LINE3
1.builder  1.plumber  1.carpenter  1.carpenter  \COP{}  2{}-be.\REL{}/  \COP{}  1-be.\REL{}/  \COP{}  2-stay-\PFV{}.\REL{}  1-friend  1.\POSS{}.1\SG{}  1-big\\
%%TRANS1
\glt ‘Of the builder, plumber and carpenter (who are my friends), the carpenter is my best/biggest friend.’
%%TRANS2
%%EXEND

\z

%%EAX
\ea
%%JUDGEMENT
%%LABEL
\label{bkm:Ref110415189}
%%CONTEXT
(You follow Muhammad. Why?)\\
%%LINE1
Tári muúlúpale etiíní áaka.\\
%%LINE2
\gll
ti  a-ri  m-ulupale  etiini   aka\\
%%LINE3
\COP{}  1-be.\REL{}  1-big  9.religion  9.\POSS{}.1\SG{}\\
%%TRANS1
\glt
‘That’s who is the leader of my religion.’\\
%%TRANS2
%%EXEND


\z

The initial NP in the reverse pseudocleft must be identifiable, and hence it cannot be an interrogative, as shown in \xref{bkm:Ref110412196} and \xref{bkm:Ref110412198}, and neither can it be an indefinite non-specific: \textit{ntthu} ‘person’ in \xref{bkm:Ref110412255} is instead interpreted generically.


%%EAX
\ea
%%JUDGEMENT
[*]{
%%LABEL
\label{bkm:Ref110412196}
%%CONTEXT
%%LINE1
Eshééni tícáalyáwo?\\
%%LINE2
\gll
esheeni  ti  e-c-ale-au\\
%%LINE3
9.what  \COP{}  9-eat-\PFV{}.\REL{}-\POSS{}.2\SG{}\\
%%TRANS1
\glt
int. ‘What is it that you eat?’\\
%%TRANS2
}
%%EXEND


\z

%%EAX
\ea
%%JUDGEMENT
[*]{
%%LABEL
\label{bkm:Ref110412198}
%%CONTEXT
%%LINE1
Pani tilepale va?\\
%%LINE2
\gll
pani  ti  o-lep-ale  va\\
%%LINE3
1.who  \COP{}  1-write-\PFV{}.\REL{}  16.\DEM{}.\PROX{}\\
%%TRANS1
\glt
int. ‘Who (is the one who) has written here?’\\
%%TRANS2
}
%%EXEND


\z

%%EAX
\ea
%%JUDGEMENT
%%LABEL
\label{bkm:Ref110412255}
%%CONTEXT
%%LINE1
\'{N}tthú táshókonlé (enámá khińshókhola).\\
%%LINE2
\gll
n-tthu  ti  a-shokhol-ale  e-nama  kha-e-n-shokhol-a\\
%%LINE3
1-person  \COP{}  2-gather.shellfish-\PFV{}.\REL{}  9-animal  \NEG{}-9-\PRS{}-gather.shellfish-\FV{}\\
%%TRANS1
\glt
‘People are the ones who gather shellfish (animals don’t).’\\
%%TRANS2
%%EXEND


\z

Like the pseudocleft, the reverse pseudocleft is also typically used in contexts where a referent is identified to the exclusion of others, as in \xref{bkm:Ref110413319} above. Similarly, in \xref{bkm:Ref96776223}, the speaker conveys that out of the possible people to be wearing \textit{nsiro}, it is the young girls. 

%%EAX
\ea
%%JUDGEMENT
%%LABEL
\label{bkm:Ref96776223}
%%CONTEXT
%%LINE1
Masi ílé, hat’ éwórá yaaryááká eteḿpú el’ eélé, ashínúmwaarí, \textbf{álé tíyóotta} máís nsíro.\\
%%LINE2
\gll
masi  ile  hata  ewora  e-yar-iy-aka   e-tempu  ele  ele   a-shi-numwaari  ale  ti  a-ootta  mais  n-siro\\
%%LINE3
but  9.\DEM{}.\DIST{}  until  9.hour  9-bear-\PASS{}.\REL{}-\POSS{}.1\SG{}   9-time  9.\DEM{}.\DIST{}  9.\DEM{}.\DIST{} 2-\DIM{}-maiden  2.\DEM{}.\DIST{}  \COP{}  2-smear.\REL{}  more  3-nsiro\\
%%TRANS1
\glt
‘But that, until the time I was born, that very time, girls, they are (the ones) who put nsiro more.’\\
%%TRANS2
%%EXEND

\z



Furthermore, the initial NP seems to require exhaustive focus. We use the same diagnostics again, showing that (some or all) alternatives must be excluded. First, the universal quantifier ‘every’ cannot appear in this construction \xref{bkm:Ref110413508}.


%%EAX
\ea
%%JUDGEMENT
[*]{
%%LABEL
\label{bkm:Ref110413508}
%%CONTEXT
%%LINE1
Kuta mmakhuwa tímphéélá oca saána.\\
%%LINE2
\gll
kuta  m-makhuwa  ti  o-m-pheela  o-ca  saana\\
%%LINE3
every  1-Makhuwa  \COP{}  1-\PRS{}-want.\REL{}  15-eat  well\\
%%TRANS1
\glt
‘It is every Makhuwa that wants to eat well.’\\
%%TRANS2
}
%%EXEND


\z

Second, the felicitous use with ‘only’ \xref{bkm:Ref96775954} and association of ‘only’ with the initial NP in \xref{bkm:Ref110415432} rather than the final NP, as well as the ungrammatical ‘even’ \xref{bkm:Ref96775962} also show the exclusive interpretation of the reverse pseudocleft.


%%EAX
\ea
%%JUDGEMENT
%%LABEL
\label{bkm:Ref96775954}
%%CONTEXT
%%LINE1
Manínyá paáhí tá-wá-alé.\\
%%LINE2
\gll
Maninya  paahi  ti  a-w-ale\\
%%LINE3
1.Maninha  only  \COP{}  2-come-\PFV{}.\REL{}\\
%%TRANS1
\glt
‘Only Maninha came.’ / ‘Maninha was the only one who came.’\\
%%TRANS2
%%EXEND
 \citep[260]{vanderWal2009a}
\z

%%EAX
\ea
%%JUDGEMENT
%%LABEL
\label{bkm:Ref110415432}
%%CONTEXT
%%LINE1
Nánsi tíkúshale maátsi paáhi?\\
%%LINE2
\gll
Nansi  ti  o-kush-ale  maatsi  paahi\\
%%LINE3
1.Nancy  \COP{}  1-carry-\PFV{}.\REL{}  6.water  only\\
%%TRANS1
\glt
‘Is Nancy the only one holding water?’\\
%%TRANS2
*`Is Nancy holding only water?’

%%EXEND

\z

%%EAX
\ea
%%JUDGEMENT
%%LABEL
\label{bkm:Ref96775962}
%%CONTEXT
%%LINE1
(*Hatá) Ashílópwáná táníina.\\
%%LINE2
\gll
hata  a-shi-lopwana  ti  a-n-iina\\
%%LINE3
until  2-\DIM{}-man  \COP{}  2-\PRS{}-dance.\REL{}\\
%%TRANS1
\glt
‘(Even) The men (are the ones who) dance.’\\
%%TRANS2
%%EXEND


\z

Third, a numeral receives an exact reading in this construction, as in \xref{bkm:Ref110414003}, where a minimum amount reading is not possible (as indicated by the follow-up phrase). Here, it differs interestingly with a numeral following the conjoint verb form, as in \xref{bkm:Ref110413993}. Even though this is only one example, and judgements on minimum vs. exact amount are tricky, this may be one of the few environments in which we can see a difference between exclusive focus (some alternatives are excluded, but maybe not all) and exhaustive focus (all alternatives are false).


%%EAX
\ea
%%JUDGEMENT
%%LABEL
\label{bkm:Ref110414003}
%%CONTEXT
%%LINE1
Sínku míl tí/píḿpheelḛ́ḛ́hṵ́ par’ otthúnél’ enúpa (\textsuperscript{\#}poótí ophéélakátho).\\
%%LINE2
\gll
sinku  mil  ti-?-n-pheel-ḛḛhu  para  o-tthunela  e-nupa  pooti  o-pheel-ak-a=tho \\
%%LINE3
five  thousand  \COP{}-\PRS{}-want.\REL{}-\POSS{}.1\PL{}  for  15-cover  9-house can  15-want-\DUR{}-\FV{}=\REP{} \\
%%TRANS1
\glt
‘Five thousand is what we need to cover the house (\textsuperscript{\#}maybe more).’\\
%%TRANS2
%%EXEND

%%EAX
\ex
%%JUDGEMENT
%%LABEL
\label{bkm:Ref110413993}
%%CONTEXT
%%LINE1
Nimphééla sínku míl par’ otthúnél’ enúpa (poótí ophéélakátho).\\
%%LINE2
\gll
ni-n-pheel-a  sinku  mil  para  o-tthunela  e-nupa pooti  o-pheel-ak-a=tho \\
%%LINE3
1\PL{}.\SM{}-\PRS{}.\CJ{}-{}want-\FV{}  five  thousand  for  15-cover  9-house can  15-want-\DUR{}-\FV{}=\REP{} \\
%%TRANS1
\glt
‘We need five thousand to cover the house (maybe more).’\\
%%TRANS2
%%EXEND

\z



Otherwise, it has proven difficult to find a context in which the reverse pseudocleft can be distinguished from the pseudocleft, and in many situations either construction can be used, as illustrated in \xref{bkm:Ref96776086} and \xref{bkm:Ref117584290}.\largerpage


\ea
\label{bkm:Ref96776086}
\begin{xlist} 
\exi{X:} ‘Who is sleeping inside? Abdul?’
\exi{Y:} ‘No, it’s not Abdul, …’
\end{xlist}

%%EAX
\ea
%%JUDGEMENT
%%LABEL
%%CONTEXT
%%LINE1
%%LINE2
\gll
o-n-rúp-á  ti  Coána.\\
%%LINE3
    1\SM{}-\PRS{}-sleep-\FV.\REL{}  \COP{}  1.Joana\\
%%TRANS1
\glt
    lit. ‘… the one who sleeps is Joana’\\
%%TRANS2
%%EXEND

%%EAX
\ex
%%JUDGEMENT
%%LABEL
%%CONTEXT
%%LINE2
\gll
Coáná  t’  í-ń-rupa.\\
%%LINE3
1.Joana  \COP{}  1-\PRS{}-sleep-\FV.\REL{}\\
%%TRANS1
\glt
    lit. ‘… Joana is the one who sleeps’ \citep[261]{vanderWal2009a}\\
%%TRANS2
%%EXEND

\z
\z

\ea
\label{bkm:Ref117584290}(In this picture, who is drinking?)
%%EAX
\ea
%%JUDGEMENT
%%LABEL
%%CONTEXT
%%LINE1
Nthíyán’ ola vá tímwúrya. (pointing)\\
%%LINE2
\gll
n-thiyana  ola  va  ti  o-n-wury-a\\
%%LINE3
1-woman  1.\DEM{}.\PROX{}  16.\DEM{}.\PROX{}  \COP{}  1-\PRS{}-drink-\FV{}.\REL{}\\
%%TRANS1
\glt
‘This woman here is the one who is drinking.’\\
%%TRANS2
%%EXEND

%%EAX
\ex
%%JUDGEMENT
%%LABEL
%%CONTEXT
%%LINE1
Amwúryá maátsí athiyána.\\
%%LINE2
\gll
a-n-wury-a  maatsi  a-thiyana\\
%%LINE3
2-\PRS{}-drink-\FV{}.\REL{}  6.water  2-woman.\PRL{}\\
%%TRANS1
\glt
‘Who is drinking water is the woman.’\\
%%TRANS2
%%EXEND


\z
\z

In \xref{bkm:Ref117584314}, the speaker first uses the pseudocleft ‘who will know are grown-up people’ (with predicative lowering on \textit{atthu} ‘people’, and then rephrases it as an inverse pseudocleft ‘they are the ones who will know’. The NP \textit{átthú a khálái} ‘old people’ could be an afterthought after \textit{atthu uúlúpalé}, or a topic of a (repetitive) next sentence, or the initial NP in a reverse pseudocleft. Given the break following this NP, though, the last analysis seems less likely.\largerpage[-1]

\ea
\label{bkm:Ref117584314}
(B: How did the capulana begin? B: I don’t know. That’s what I said, that the capulana I don’t know how it emerged, because I when I was born, I found it being worn and I also wore it.)\\
%%EAX
\begin{xlist}
\exi{B:}
%%JUDGEMENT
%%LABEL
%%CONTEXT
%%LINE1
Éyó [antthúná otsúwéla atthu uúlúpalé], átthú a khálái, [táńtthún’ ótsúwéla], aá.\\
%%LINE2
\gll
eyo  a-n-tthun-a  o-tsuwela  a-tthu  a-ulupale    a-tthu  a  khalai  ti  a-n-tthun-a  o-tsuwela  aa  \\
%%LINE3
9.\DEM{}.\MED{}  2-\PRS{}-want.\REL{}  15-know  2-person.\PRL{}  2-big  2-person  2.\CONN{}  old.times  \COP{}  2-\PRS{}-want.\REL{}  15-know  yes \\
%%TRANS1
\glt
‘This, [who will know are older people], very old people, [that’s the ones who will know], yes.’\\
%%TRANS2
\end{xlist}
%%EXEND

\z


It is uncertain whether this construction can be analysed as a left-peripheral NP plus a basic cleft, as suggested for other languages in this volume (i.e. a literal translation as “Joanna, it is HER who is sleeping”) – one vital difference is the lack of an independent pronoun in the Makhuwa construction. That construction is only found for personal pronouns, as in \xref{bkm:Ref115120262}.\largerpage[-1]\pagebreak

%%EAX
\ea
%%JUDGEMENT
%%LABEL
\label{bkm:Ref115120262}
%%CONTEXT
(present tense: In a group of people preparing there is no cook so someone is appointed to do the cooking.)\\
%%LINE1
Wé ti w’ óri/waarí shawuríya.\\
%%LINE2
\gll
we  ti  we  o-ri  /o-a-ri  shawuriya\\
%%LINE3
2\SG{}.\PRO{}  \COP{}  2\SG{}.\PRO{}  2\SG{}.\SM{}-be.\REL{}  /2\SG{}.\SM{}-\PST{}-be.\REL{}  1.cook\\
%%TRANS1
\glt
‘You are (the one who is/was) the cook.’\\
%%TRANS2
‘You, it’s you who is/was the cook.’

%%EXEND

\z

In summary, Makhuwa-Enahara shows three types of clefts, all of which express focus. Since there is no other way of focussing a subject, focused subjects must always choose a cleft construction. The precise factors determining the choice of cleft require more detailed investigation, but two hypotheses are that 1) the pseudocleft expresses identification while the basic cleft and the reverse pseudocleft express exhaustive focus, and 2) the basic cleft is used for pronouns whereas full noun phrases are focused in the reverse pseudocleft.

In the description and analysis so far, we have seen how in Makhuwa-Enahara, verb form, constituent order, predicate doubling, and clefts are used to express the information-structural functions (topic, focus) and we have seen their precise interpretations. There is, however, another vital aspect of information structure, which is the status of referents and their degree of mental accessibility. How this influences the language is taken up in the next (and final) section.

\section{Referent tracking}
\label{bkm:Ref117578879}
The choice of expression to refer to a referent largely depends on the status of the referent in terms of mental accessibility – see the introduction to this volume. Inspired by \citet{Ariel1990,Ariel2001} (and see further references on activation in \cite{chapters/intro}), I proposed an accessibility hierarchy of referential expressions specifically for Makhuwa \citep[195]{vanderWal2010}, modified as in \xref{bkm:Ref110501885}, where more accessible referents tend to be coded by expressions on the right side of the hierarchy, and less accessible referents prefer expressions on the left:

\ea
\label{bkm:Ref110501885}
N+modifier > N > \DEM{} N \DEM{} > N+\DEM{} > \DEM{}/pronoun > prefix > zero
\z

Brand-new referents are presented by full NPs in postverbal position (see \sectref{bkm:Ref95915158} on subject inversion and the presentational construction with \nobreakdash-\textit{haavo}), and on the other end of the continuum of accessibility, highly accessible referents are encoded as subject prefixes for subjects and object prefixes or zero for objects. 

Illustrating the right end of the hierarchy, Theme objects can easily be omitted altogether if they are highly accessibly and in a class other than 1/2, as illustrated in \xref{bkm:Ref110500117} and \xref{bkm:Ref110500118}. There are no examples of Recipient drop, as Recipients are typically in class 1/2 or 1\textsuperscript{st}/2\textsuperscript{nd} person and therefore need to be expressed by an object marker on the verb.

%%EAX
\ea
%%JUDGEMENT
%%LABEL
\label{bkm:Ref110500117}
%%CONTEXT
(Did you buy fish and wash the dishes?)\\
%%LINE1
%%LINE2
\gll
Áyo,  k{}-oo{}-thúm{}-á  ni  k{}-oo{}-ráp{}-íh{}-a.\\
%%LINE3
yes  1\SG{}.\SM{}-\PFV{}.\DJ{}-buy-\FV{}  and  1\SG{}.\SM{}-\PFV{}.\DJ{}-wash-\CAUS{}-\FV{}\\
%%TRANS1
\glt
‘Yes, I bought (it) and I washed (them).’\\
%%TRANS2
%%EXEND


\z

%%EAX
\ea
%%JUDGEMENT
%%LABEL
\label{bkm:Ref110500118}
%%CONTEXT
(But he knew what she wanted to do; he took the cloth,)\\
%%LINE1
%%LINE2
\gll
o{}-o{}-pér{}-ák{}-ats{}-ááá... \\
%%LINE3
1\SM{}-\PFV{}.\DJ{}-tear-\DUR{}-\PLUR{}-\FV{}\\
%%TRANS1
\glt
‘He tore (it)...’\\
%%TRANS2
%%EXEND

%%EAX
\sn
%%JUDGEMENT
%%LABEL
%%CONTEXT
%%LINE1
ohaáváha ashíkhwáawe yaarí athá\'{n}n’ ámótsa.\\
%%LINE2
\gll
o-h-a-vah-a  a-shikhw-aawe  a-a-ri  a-thanu  ni  a-motsa\\
%%LINE3
1\SG.\SM-\PFV.\DJ-2\OM{}-give-\FV{}  2-fellow-\POSS{}.1  2-\PST{}-be.\REL{}  2-five  and  2-one\\
%%TRANS1
\glt
‘he gave (it/them) to his six friends.’
%%TRANS2
%%EXEND
\z


\citet{Poeta2016} investigated the hypothesis that Makhuwa would show more object drop than Swahili, given the fact that Swahili has the option of expressing the object by an object marker for all relevant noun classes, whereas Makhuwa only has object markers for 1\textsuperscript{st} and 2\textsuperscript{nd} person and class 1/2. Her study did not confirm this for Makhuwa-Emeetto: she found a similar amount of object drop in Swahili as in Makhuwa texts. Interestingly, Makhuwa does use full NPs more often for anaphoric reference – see \citet{Poeta2016} for details.

Where the encoding at the far ends of the accessibility hierarchy is relatively (!) clear, the situation in between shows interesting uses of pronominal expressions, discussed in Sections~\ref{bkm:Ref110501311}--\ref{bkm:Ref109492056}. It should be kept in mind that referent tracking is influenced by, and plays a role in structuring, the higher level of discourse and narration as well – something I will not go into in this chapter.

\subsection{Personal pronouns}
\label{bkm:Ref110501311}
Makhuwa has independent pronouns for 1\textsuperscript{st} and 2\textsuperscript{nd} person, and classes 1 and 2~-- see \tabref{pronouns}. For 1\textsuperscript{st} and 2\textsuperscript{nd} person, a short and a long form exist, where the long form seems to be ``more emphatic" (whatever this turns out to be in future research).

\begin{table}
\begin{tabularx}{\textwidth}{llQ lll}
\lsptoprule
\SG{}   &  1 &  mi / miyaano             & \PL{}   & 1 & h\~\i / hiyaano\\
        & 2 & we / weyaano               &         & 2 & nyutse / nyuwaanotse\\
        & 2\RESP{} &  nyu / nyuwaano     &         & cl.2 & ayena(tse)\\
        & cl.1 & yena\\
\lspbottomrule
\end{tabularx}
\caption{\label{pronouns}Personal pronouns \citep[64]{vanderWal2009a}}
\end{table}

These independent pronouns are used after prepositions \xref{bkm:Ref110500800}, in non-verbal predication, as in \xref{bkm:Ref110409158} above and \xref{bkm:Ref110500835}, but more relevant for information structure, they are also used in contrastive topics and foci, illustrated in \xref{bkm:Ref110322922} and \xref{bkm:Ref110501136} above, and below in \xref{bkm:Ref129273753} for topic and \xref{bkm:Ref110501192} for focus.\largerpage

%%EAX
\ea
%%JUDGEMENT
%%LABEL
\label{bkm:Ref110500800}
%%CONTEXT
%%LINE1
Nihááná nimváháka eyoólyá ekhwááwé para \textbf{yéná} okhálá saáná.\\
%%LINE2
\gll
ni-haan-a  ni-m-vah-ak-a  eyooca  e-khw-aawe para  yena  o-khala  saana \\
%%LINE3
1\PL{}.\SM{}-have-\FV{}  1\PL{}.\SM{}-1\OM{}-give-\DUR{}-\FV{}  9.food  9-fellow-\POSS{}.1  for  1.\PRO{}  15-stay  well  \\
%%TRANS1
\glt
‘We have to give other food so that s/he stays well.’\\
%%TRANS2
%%EXEND


%%EAX
\ex
%%JUDGEMENT
%%LABEL
\label{bkm:Ref110500835}
%%CONTEXT
%%LINE1
Vá óle orí ntsulú ḿmwě kahi \textbf{yéná} mpáttháni áwo.\\
%%LINE2
\gll
va  ole  o-ri  n-tsulu  mmwe kahi  yena  m-patthani  au \\
%%LINE3
16.\DEM{}.\PROX{}  1.\DEM{}.\DIST{}  1\SM{}-be.\REL{}  18-up  18.\DEM{}.\DIST{}  \NEG{}.\COP{}  1.\PRO{}  1-friend  1.\POSS{}.2\SG{} \\
%%TRANS1
\glt
‘Now the one who is up there is not your friend.’\\
%%TRANS2
%%EXEND

%%EAX
\ex
%%JUDGEMENT
%%LABEL
\label{bkm:Ref129273753}
%%CONTEXT
%%LINE1
\textbf{Hḭ}, nimpéél’ otthyawá, \textbf{yéná} omphééla weettá.\\
%%LINE2
\gll
hi  ni-n-pheel-a  o-tthyawa  yena  o-n-pheel-a  o-eetta\\
%%LINE3
1\PL{}.\PRO{}  1\PL{}.\SM{}-\PRS{}.\CJ{}-want   15-run  1.\PRO{}  1\SM{}-\PRS{}-want-\FV{}  15-walk\\
%%TRANS1
\glt
‘We want to run, s/he wants to walk.’\\
%%TRANS2
%%EXEND

%%EAX
\ex
%%JUDGEMENT
%%LABEL
\label{bkm:Ref110501192}
%%CONTEXT
%%LINE1
\textbf{Áyéná} tahiwaale.\\
%%LINE2
\gll
a-yena  ti  a-hi-w-ale\\
%%LINE3
2-\PRO{}  \COP{}  1-\NEG{}-come-\PFV{}.\REL{}\\
%%TRANS1
\glt
‘They (are the ones who) didn’t come.’\\
%%TRANS2
%%EXEND
\z

Furthermore, in a list of topics all with the same predicate, the last topic can optionally be marked by \textit{ni yena} ‘and 1.\PRO{}’, roughly translatable as ‘him/her/it too’, which may be called marking of an ``additive topic". This is possible for both subjects and objects, but only for class 1 referents that are animate, as seen in \xxref{bkm:Ref109481398}{bkm:Ref109900273}.\footnote{I have not seen a plural counterpart.} As mentioned, there are no independent pronouns for other classes, and demonstratives are not used in this way, as can be seen in the comparison between \xref{bkm:Ref109481398:a} and \xref{bkm:Ref109481398:b}.

\ea
\label{bkm:Ref109481398}
%%EAX
\ea
%%JUDGEMENT
%%LABEL
\label{bkm:Ref109481398:a}
%%CONTEXT
%%LINE1
%%LINE2
\gll
E-nyómpé  e-n-khúúr-a  ma-lashí,      n’  ii-púrí  e-n-khúúr-á  ma-lashí,      namárókolo  \textbf{ni}  \textbf{yéná}  (també)  o-n-khúúr-á  ma-lashí.\\
%%LINE3
9-cow  9\SM{}-\PRS{}.\CJ{}-eat-\FV{}  6-grass and  9-goat  9\SM{}-\PRS{}.\CJ{}-eat-\FV{}  6-grass 1.hare  and  1.\PRO{}  {\db}also  1\SM{}-\PRS{}.\CJ{}-eat-\FV{}  6-grass\\
%%TRANS1
\glt
‘The cow eats grass, the goat eats grass, the hare too eats grass.’\\
%%TRANS2
%%EXEND

%%EAX
\ex
%%JUDGEMENT
%%LABEL
\label{bkm:Ref109481398:b}
%%CONTEXT
%%LINE1
%%LINE2
\gll
Namárókolo  o-n-khúúr-a  ma-lashí,      e-nyómpé  e-n-khúúr-a  ma-lashí,      n’  ii-púrí  (*\textbf{ni}  \textbf{yéná}/  *\textbf{n’}  \textbf{iílé})  e-n-khúúr-a  malashí  (també).\\
%%LINE3
1.hare  1\SM{}-\PRS{}.\CJ{}-eat-\FV{}  6-grass   9-cow  9\SM{}-\PRS{}.\CJ{}-eat-\FV{}  6-grass  {\db}and  9-goat  and  1.\PRO{}  and  9.\DEM{}.\DIST{}  9\SM{}-\PRS{}.\CJ{}-eat-\FV{}  6-grass  {\db}also\\
%%TRANS1
\glt ‘The hare eats grass, the cow eats grass, and the goat eats grass (too).’
%%TRANS2
%%EXEND

\z

\ex
%%EAX
%%JUDGEMENT
%%LABEL
%%CONTEXT
%%LINE1
%%LINE2
\gll
Zyórzyé  ki-nú-ń-kanyárí,\\
%%LINE3
1.Jorge  1\SG{}.\SM{}-\PFV{}.\PERS{}-1\OM{}-win\\
%%TRANS1
\glt
%%TRANS2
%%EXEND

%%EAX
%%JUDGEMENT
%%LABEL
%%CONTEXT
%%LINE1
%%LINE2
\gll
  Amáátí  ki-nú-ń-kanyárí,\\
%%LINE3
1.Amade  1\SG{}.\SM{}-\PFV{}.\PERS{}-1\OM{}-win\\
%%TRANS1
\glt
%%TRANS2
%%EXEND


%%EAX
%%JUDGEMENT
%%LABEL
%%CONTEXT
%%LINE1
%%LINE2
\gll
  Ambási  també  / \textbf{ni}  \textbf{yéná}  / *n’  uúlé  ki-nú-\'{n}-kanyárí.\\
%%LINE3
1.Ambasse  also /  and  1.\PRO{} /  and  1.\DEM{}.\DIST{}  1\SG{}.\SM{}-\PFV{}.\PERS{}-1\OM{}-win\\
%%TRANS1
\glt
‘I won from Jorge, I won from Amade, and Ambasse him too I beat.’
%%TRANS2
%%EXEND

%%EAX
\ex
%%JUDGEMENT
%%LABEL
\label{bkm:Ref109900273}
%%CONTEXT
%%LINE1
Sinkíyá końkhúura, kohoócá mathaápá, ni kharáká (*\textbf{ni yéná}/*n’ uúle) końkhúura.\\
%%LINE2
\gll
sinkiya  ki-o-n-khuur-a  ki-o-c-a  ma-thaapa  ni  kharaka  ni  yena /  ni  ole  ki-o-n-khuur-a \\
%%LINE3
1.pumpkin  1\SG{}.\SM{}-\PFV{}.\DJ{}-1\OM{}-eat-\FV{}  1\SG{}.\SM{}-\PFV{}.\DJ{}-eat-\FV{}  6-mathapa and  1.cassava  and  1.\PRO{} /  and  1.\DEM{}.\DIST{}  1\SG{}.\SM{}-\PFV{}.\DJ{}-1\OM{}-eat-\FV{} \\
%%TRANS1
\glt
‘Pumpkin I ate, I ate mathapa, and cassava I ate (too).’\\
%%TRANS2
%%EXEND
\z


For 3\textsuperscript{rd} persons, Makhuwa typically uses demonstratives for pronominal reference, in contexts further specified in the following subsections.


\subsection{(Circum)demonstratives for less accessible referents}
\label{bkm:Ref95380848}\label{bkm:Ref96973948}
Makhuwa has three series of demonstratives distinguished according to distance from the speaker and addressee, plus an emphatic form of each, as in \tabref{bkm:Ref252212434}.\footnote{Vowel liaison may result in a pronunciation as \textit{ula}/\textit{uyo}/\textit{ule}, \textit{ila}/\textit{iyo}/\textit{ile}. and in lengthening.} Demonstratives are a conspicuous tool for referent tracking in Makhuwa\hyp Enahara (although see \citealt{Poeta2016} for a different conclusion for Makhuwa-Emeetto). 

\begin{table}
\small
\begin{tabular}{r @{~} 
                >{\itshape}l @{~} llll 
                >{\raggedright\arraybackslash}p{\widthof{móomu}} 
                >{\raggedright\arraybackslash}p{\widthof{wéwwo/}} 
                >{\raggedright\arraybackslash}p{\widthof{mwémwe/}} 
                }
\lsptoprule
   &        &             & \multicolumn{3}{c}{Standard} & \multicolumn{3}{c}{Emphatic}\\\cmidrule(lr){4-6}\cmidrule(lr){7-9}
\multicolumn{3}{l}{Class} & \PROX{} & \MED{} & \DIST{} & \PROX{} & \MED{} & \DIST{}\\\midrule
1  & mwaáná & ‘child’     & óla & óyo & óle & yóola & yóoyo & yóole\\
2  & aáná   & ‘children’  & ála & áyo & ále & yáala & yáayo & yáale\\
3  & nvéló  & ‘broom’     & óla & óyo & óle & yóola & yóoyo & yóole\\
4  & mivéló & ‘brooms’    & íya & íyo & íye & yéyya\slash tséyya & yéyyo\slash tséyyo & yéyye\slash tséyye\\
5  & ntátá  & ‘hand’      & ńna & ńno & ńne & nénna & nénno & nénne\\
6  & matátá & ‘hands’     & ála & áyo & ále & yáala & yáayo & yáale\\
9  & emáttá & ‘field’     & éla & éyo & éle & yéela & yéeyo & yeéle\\
10 & emáttá & ‘fields’    & íya & íyo & íye & yéyya\slash tséyya & yéyyo\slash tséyyo & yéyye\slash tséyye\\
14 & orávó  & ‘honey’     & óla & óyo & óle & yóola & yóoyo & yóole\\
16 &        &             & vá & vó & vále & vááva & váavo & váávale\\
17 &        &             & ńno & \'{\~u}wo & \'{\~u}we & wénno & wéwwo\slash wówwo & wéwwe\\
18 &        &             & mú & ḿmo & ḿmwe & móomu & mómmo & mwémwe\slash wómwe\\
\lspbottomrule
\end{tabular}
\caption{\label{bkm:Ref252212434}Overview demonstratives (based on \citealt[47]{vanderWal2009a})}

\end{table}

My study of Makhuwa narratives \citep{vanderWal2010} revealed that demonstratives are used to refer to accessible (but not active) referents, that is, demonstratives help the reactivation of referents. The pronominal demonstrative is typically found in topic shifts, and after an episode boundary. I illustrated this with the sequence in \xref{bkm:Ref96938313} \citep[197]{vanderWal2010}. In \xref{bkm:Ref96938313:a}, the topic and subject is \textit{the Portuguese} (class 2 ‘they’). The topic shifts to the just-introduced fisherman in \xref{bkm:Ref96938313:b}, where the demonstrative \textit{ole} is used. Continuing the same topic, \xref{bkm:Ref96938313:c} refers to the fisherman only by a subject prefix, but in \xref{bkm:Ref96938313:d} the topic shifts to the Portuguese again, and the demonstrative \textit{ale} occurs.

\ea
\label{bkm:Ref96938313}(The Portuguese went to India. Passing on open sea, when they looked around, they saw this very island and they thought “Let’s take our boat, let’s go and take a rest on that island”. When they arrived there, …)
%%EAX
\ea
%%JUDGEMENT
%%LABEL
\label{bkm:Ref96938313:a}
%%CONTEXT
%%LINE2
\gll
A-ḿ-phwányá  n-lópwáná  m-motsá.\\
%%LINE3
2\SM{}.\PFV{}.\DJ{}-1\OM{}-meet  1-man  1-one\\
%%TRANS1
\glt
‘They met a man.’\\
%%TRANS2
%%EXEND



%%EAX
\ex
%%JUDGEMENT
%%LABEL
\label{bkm:Ref96938313:b}
%%CONTEXT
%%LINE2
\gll
\textbf{Ólé}  aa-rí  nákhavokó,\\
%%LINE3
1.\DEM{}.\DIST{}  1\SM{}.\PST{}-be  1.fisherman.\PRL{}\\
%%TRANS1
\glt
    ‘He was a fisherman,’\\
%%TRANS2
%%EXEND


%%EAX
\ex
%%JUDGEMENT
%%LABEL
\label{bkm:Ref96938313:c}
%%CONTEXT
%%LINE2
\gll
aa-ríná  e-kalawa  ts-áwé  ts-a  khavóko.\\
%%LINE3
1\SM{}.\PST{}-have  10-boat  10-\POSS{}.1  10-\CONN{}  fishing\\
%%TRANS1
\glt
    ‘he had his fishing boats.’\\
%%TRANS2
%%EXEND


%%EAX
\ex
%%JUDGEMENT
%%LABEL
\label{bkm:Ref96938313:d}
%%CONTEXT
%%LINE2
\gll
\textbf{Álé}  a-ḿ-wéh-áts-a.\\
%%LINE3
2.\DEM{}.\DIST{}  2\SM{}.\PFV{}.\DJ{}-1\OM{}-look-\PLUR{}-\FV{}\\
%%TRANS1
\glt
    ‘They looked at him.’  \citep[197]{vanderWal2010}\\
%%TRANS2
%%EXEND

\z
\z

The distal adnominal demonstrative is also used anaphorically, and it is similarly used for reactivation of earlier-mentioned referents. In example \xref{bkm:Ref96939989}, the owl (\textit{etsíítsí}) is referred to in sentence 6 of the story, again four sentences later and again two sentences later, when it is referred to with a noun modified by a demonstrative \citep[93]{vanderWal2010}.

%%EAX
\ea
%%JUDGEMENT
%%LABEL
\label{bkm:Ref96939989}
%%CONTEXT
%%LINE2
\gll
Hw-íir-a  “Mí  e-tsíítsí,  ki-náá-vár-á,  ki-náá-khúur-a”.\\
%%LINE3
\NARR{}-say-\FV{}  1\SG{}.\PRO{}  9-owl  1\SG{}.\SM{}-\PRS{}.\DJ{}-grab-\FV{}  1\SG{}.\SM{}-\PRS{}.\DJ{}-eat-\FV{}\\
%%TRANS1
\glt
‘And he (the fox) said: the owl, I will catch him and eat him.’\\
%%TRANS2
%%EXEND

  […]

%%EAX
%%JUDGEMENT
%%LABEL
%%CONTEXT
%%LINE2
\gll
Hatá  ni-húkú  ni-motsá,  ólé  khweelí  o-ḿ-phwány’  e-tsíítsí...{\footnotemark}  \\
%%LINE3
until  5-day  5-one  1.\DEM{}.\DIST{}  certainly  1\SM{}.\PFV{}.\DJ{}-1\OM{}-meet  9-owl\\
\footnotetext{Note also that the verb form \textit{oḿphwánya} here is disjoint, as the point of the sentence if not to focus the owl (as opposed to other animals), but rather the fact that the owl was caught (which may be seen as polarity focus).}
%%TRANS1
\glt
‘Until one day he found the owl…’\\
%%TRANS2
%%EXEND

  […]


%%EAX
%%JUDGEMENT
%%LABEL
%%CONTEXT
%%LINE2
\gll
  O-o-phíy-á  válé,  o-o-túph-á, o-o-vár-á  \textbf{e-tsííts’}  \textbf{íile}. \\
%%LINE3
1\SM{}-\PFV{}.\DJ{}-arrive-\FV{}  16.\DEM{}.\DIST{}  1\SM{}-\PFV{}.\DJ{}-jump-\FV{}  1\SM{}-\PFV{}.\DJ{}-grab-\FV{}  9-owl  9.\DEM{}.\DIST{}\\
%%TRANS1
\glt
‘He arrived there, he jumped, (and) he caught the/that owl.’\\
%%TRANS2
%%EXEND

\z

Both the adnominal and pronominal use are illustrated in \xref{bkm:Ref96850547}, where the pronominal demonstrative is used (\textit{ólé ookhúmá}) rather than only the subject marker, in order to switch from one referent (Tortoise) to another (Leopard). But then the narrator realises that it may at that point not be entirely clear who is who, adding the full name plus demonstrative afterwards (\textit{havárá ole}).

%%EAX
\ea
%%JUDGEMENT
%%LABEL
\label{bkm:Ref96850547}
%%CONTEXT
(Leopard\textsubscript{i} saves Tortoise\textsubscript{m} from the fire. Tortoise says “Tomorrow you must come to my\textsubscript{m} house, when the fire has stopped.” He\textsubscript{i} said “I have heard you.” So he\textsubscript{m} went away, crawling crawling crawling to his\textsubscript{m} house. He\textsubscript{m} arrived but the fire didn’t reach (the house).)\\
%%LINE1
%%LINE2
\gll
Masi  seertú  nróttó  áyá,  nuu-thowa-thówá  moóró  \textbf{olé},    \textbf{ólé}  oo-khúm-á,  oo-rów-a, \\
%%LINE3
but  truly  day.after.tomorrow  \POSS{}.2  \RSM{}-extinguish-\RED{}  3.fire  3.\DEM{}.\DIST{} 1.\DEM{}.\DIST{}  1\SM{}.\PFV{}.\DJ{}-{}exit-\FV{}  1\SM{}.\PFV{}.\DJ{}-go-\FV{} \\
%%TRANS1
\glt
‘But as promised, the day after, when the fire had extinguished, he\textsubscript{i} went out, he\textsubscript{i} went…’\\
%%TRANS2
%%EXEND

%%EAX
\sn
%%JUDGEMENT
%%LABEL
%%CONTEXT
%%LINE1
%%LINE2
\gll
havárá  \textbf{ole}  oo-rów-á  wa-khápá  \textbf{óle}.{\footnotemark}\\
%%LINE3
1.Leopard  1.\DEM{}.\DIST{}  1\SM{}.\PFV{}.\DJ{}-go-\FV{}  16-1.Tortoise  1.\DEM{}.\DIST{}\\
\footnotetext{The tonal patterns on the demonstratives here derive from a combination of the following factors: adnominal vs. pronominal use, final lowering, and boundary continuation tones. See \citet[chapter~2]{vanderWal2009a} for further information on tone rules.}
%%TRANS1
\glt
‘… Leopard\textsubscript{i} went to Tortoise’s\textsubscript{m}.’
%%TRANS2
%%EXEND
\z

There is yet another way in Makhuwa to (re)activate referents: circumdemonstratives. Demonstratives may occur both before and after the noun in three (re)activating contexts: first, when reactivating a referent that has not been mentioned for some time; second, in tail-head linking; and third, at episode boundaries. I repeat the illustration and explanation I gave in my earlier paper \citep[200--201]{vanderWal2010}: 

\begin{quote}
The last mention of the subject of the first sentence in [\xref{bkm:Ref117585368}], was in sentence 70. After an episode speaking solely about the man whose wife was taken from him, the story comes back to the one who took the wife, which is sentence 98 of the story. The narrator starts by referring to him with a pronominal demonstrative \textit{ólé}, but realises that the referent is not sufficiently accessible to be identified directly. In a pseudocleft construction (who answered was the/that man who…), he adds information so as to make sure that the listeners pick the right referent, and uses the doubled demonstrative to refer to the less accessible referent.
\end{quote}


%%EAX
\ea
%%JUDGEMENT
%%LABEL
\label{bkm:Ref117585368}
%%CONTEXT
%%LINE1
ólé  khwíira\\
%%LINE2
\gll
ole  khu-ira\\
%%LINE3
1.\DEM{}.\DIST{}  \NARR{}-say\\
%%TRANS1
\glt
‘and he said,’\\
%%TRANS2
%%EXEND


%%EAX
%%JUDGEMENT
%%LABEL
%%CONTEXT
%%LINE1
aakhulle  t’  \textbf{uúlé}  nlópwán’  \textbf{oolé}  aamwaákhálé  mwaár’  áw’  oole \\
%%LINE2
\gll
  a-a-akhul-ale  ti  ole  n-lopwana  ole  a-a-m-akh-ale  mwaara  awe  ole \\
%%LINE3
1-\PST{}-answer-\PFV{}.\REL{}  \COP{}  1.\DEM{}.\DIST{}  1-man  1.\DEM{}.\DIST{}  1{}-\PST{}-1\OM{}-pull-\PFV{}.\REL{}  1.woman  1.\POSS{}.1  1.\DEM{}.III\\
%%TRANS1
\glt
‘(the one) who answered was that man who had snatched his wife away,’\\
%%TRANS2
%%EXEND


%%EAX
%%JUDGEMENT
%%LABEL
%%CONTEXT
%%LINE1
hwíra  mpattháni…\\
%%LINE2
\gll
khu-íra  m-pattháni…\\
%%LINE3
\NARR{}-say  1-friend\\
%%TRANS1
\glt
‘he said: my friend…’  \citep[201]{vanderWal2010}\\
%%TRANS2
%%EXEND

\z

 Newly introduced referents that are presented in the comment of a sentence are often repeated at the start of the following sentence. In this repetition, two demonstratives may be used to enhance activation of the referent and make them more prominent for future reference as well. Examples \xref{bkm:Ref117580101} and \xref{bkm:Ref96953104} are from different stories, and both show the introduction of a woman (one in subject inversion, the other as object), who in the next sentence is marked by two demonstratives in order to function as the topic of that sentence.

%%EAX
\ea
%%JUDGEMENT
%%LABEL
\label{bkm:Ref117580101}
%%CONTEXT
(A long long time ago, when the world was bald, water had a tail, and the trees jumped,)\\
%%LINE1
Aanúúkhálá nthíyána mmotsá, aarí n’ iirukulu yuulupále.\\
%%LINE2
\gll
a-a-nuu-khal-a  n-thiyana  m-motsa  a-a-ri  ni  e-rukulu  e-ulupale\\
%%LINE3
2\SM{}-\PST{}-\PERS{}-stay-\FV{}  1-woman  1-one  2\SM{}-\PST{}-be  with  9-belly  9-big\\
%%TRANS1
\glt
‘There was a woman, she had a big belly.’\\
%%TRANS2
%%EXEND

%%EAX
\sn
%%JUDGEMENT
%%LABEL
%%CONTEXT
%%LINE1
\textbf{Ólé nthíyán’ ole} aakúshá ncómá áwé, oorééla maátsi opuúsu.\\
%%LINE2
\gll
ole  n-thiyana  ole  a-o-kush-a  n-coma  awe  o-o-row-el-a  maatsi  o-puusu\\
%%LINE3
1.\DEM{}.\DIST{}  1-woman  1.\DEM{}.\DIST{}  2\SM{}-\PFV{}.\DJ{}-carry-\FV{}  3-container  3.\POSS{}.1  1\SM{}-\PFV{}.\DJ{}-go-\APPL{}-\FV{}  6.water  17-well\\
%%TRANS1
\glt ‘This woman carried her container and went to fetch water at the well.’
%%TRANS2
%%EXEND

\z

%%EAX
\ea
%%JUDGEMENT
%%LABEL
\label{bkm:Ref96953104}
%%CONTEXT
%%LINE2
\gll
O-ḿ-phwány-a  n-thíyáná  m-motsá.\\
%%LINE3
1\SM{}.\PFV{}.\DJ{}-1\OM{}-meet-\FV{}  1-woman  1-one\\
%%TRANS1
\glt
‘He met a woman.’\\
%%TRANS2
%%EXEND


%%EAX
%%JUDGEMENT
%%LABEL
%%CONTEXT
%%LINE1
\textbf{Ólé}  nthíyán’  \textbf{uule}  khoóthá  aapáh’  ólumweńku.\\
%%LINE2
\gll
ole  n-thiyana  ole  kha-a-oth-a  a-aa-pah-a  o-lumwenku.\\
%%LINE3
1.\DEM{}.\DIST{}  1-woman  1.\DEM{}.\DIST{}  \NEG{}-1\SM{}.\IPFV{}-lie-\FV{}  1\SM{}-\IPFV{}-light-\FV{}  14-world\\
%%TRANS1
\glt
‘This woman didn’t just lie, she set the world on fire!’ \citep[180]{vanderWal2009a}\\
%%TRANS2
%%EXEND

\z

According to \citet{Ariel2001}, referents are also less accessible when they occur after an episode boundary. They need to be reestablished in the current episode, and this too happens by demonstratives in Makhuwa – pronominal or circumdemonstrative. The episode in \xref{bkm:Ref96953650} ends with the mother’s death, and the new episode starts with the protagonist as the (shifted) topic, marked by two demonstratives. 

\ea
\label{bkm:Ref96953650}
(Now his mother became ill. She called her son and said: “[…]. Do you remember?” He said “I remember”. “That’s what I am telling you”)\\
%%EAX
\ea
%%JUDGEMENT
%%LABEL
%%CONTEXT
%%LINE2a
\gll
Ólé  khú-khw-a.\\
%%LINE2b
%%LINE3
1.\DEM{}.\DIST{}  \NARR{}-die-\FV{}\\
%%TRANS1
\glt
‘and then she died.’\\
%%TRANS2
%%EXEND


%%EAX
\ex
%%JUDGEMENT
%%LABEL
%%CONTEXT
%%LINE2
\gll
\textbf{Ólé}  rapásy’  \textbf{úúlé}  oo-khál-á  oo-khálá  oo-khálá.\\
%%LINE3
1.\DEM{}.\DIST{}  1.boy  1.\DEM{}.\DIST{}  1\SM{}.\PFV{}.\DJ{}-stay-\FV{}  \RED{}  \RED{}\\
%%TRANS1
\glt
    ‘The boy stayed and stayed and stayed.’\\
%%TRANS2
%%EXEND


%%EAX
\ex
%%JUDGEMENT
%%LABEL
%%CONTEXT
%%LINE2
\gll
Oo-phíy-á  okáthí  w’  oóthéla.\\
%%LINE3
14\SM{}.\PFV{}.\DJ{}-arrive-\FV{}  14.time  14.\CONN{}  15.marry\\
%%TRANS1
\glt
    ‘(Until) it was time to get married.’  \citep[203--204]{vanderWal2010}\\
%%TRANS2
%%EXEND

\z
\z

For further discussion and examples of demonstratives and reference tracking, I refer to \citet{vanderWal2010}.

\subsection{Emphatic demonstrative}
\label{bkm:Ref109492056}
The emphatic form of the demonstrative consists of the demonstrative with an agreeing prefix, glossed as E as in \xxref{bkm:Ref96847545}{bkm:Ref96847549}. Its function seems to be an emphatic (re\nobreakdash-)identification of a previously established referent, or in \citegen{Floor1998} words a “confirmative demonstrative”, translated as ‘the very (same)’. It is typically used pronominally.

%%EAX
\ea
%%JUDGEMENT
%%LABEL
\label{bkm:Ref96847545}
%%CONTEXT
%%LINE1
Yaanúḿvól’ etálá hatá khúc’ éshíma (\textbf{yééle} ehiníńtsivelaka).\\
%%LINE2
\gll
e-a-nuu-n-vol-a  e-tala  hata  khu-c-a   e-shima   yeele  e-hi-n-n-tsivel-ak-a \\
%%LINE3
9\SM{}-\PST{}-\PERS{}-1\OM{}-torment-\FV{}  9-hunger  until  \NARR{}-eat-\FV{}  9-shima    9.E\DEM{}.\DIST{}  9-\NEG{}-\PRS{}-1\OM{}-please-\DUR{}-\FV{}.\REL{} \\
%%TRANS1
\glt
‘He was hungry to such an extent that he ate shima (the same that he doesn’t like).’\\
%%TRANS2
%%EXEND

\z


%%EAX
\ea
%%JUDGEMENT
%%LABEL
%%CONTEXT
%%LINE2
\gll
Válé  o-khúmá  ni-húkú  \textbf{né\'{n}né}…\\
%%LINE3
16.\DEM{}.\DIST{}  15-exit  5-day  5.E\DEM{}.\DIST{}\\
%%TRANS1
\glt
‘as of that day/ from that day on…’ \citep[48]{vanderWal2009a}\\
%%TRANS2
%%EXEND

\z

%%EAX
\ea
%%JUDGEMENT
%%LABEL
%%CONTEXT
(He searched and searched for a woman. He found one woman – this woman didn’t just lie, she set the world on fire (with her lies). So when he heard about this lying woman, …)\\
%%LINE1
%%LINE2
\gll
…hw-íira:  “Paáhi,  ki-n-iń-thél-a  \textbf{yóoyo}.”\\
%%LINE3
\NARR{}-say  only  1\SG{}.\SM{}-\PRS{}-1\OM{}-marry-\FV{}  1.E\DEM{}.\MED{}\\
%%TRANS1
\glt
‘… he said: “That’s it, I will marry that very one”.’\\
%%TRANS2
%%EXEND


\z

%%EAX
\ea
%%JUDGEMENT
%%LABEL
\label{bkm:Ref96847549}
%%CONTEXT
(She went straight to the police. She said ‘my husband at home has killed!’. ‘He has killed?’ She said ‘Yes. Go there and blood will be found.’)\\
%%LINE1
%%LINE2
\gll
\textbf{Yoólé}  mpákhá  wa-ámútsy’  aáwe\\
%%LINE3
1.E\DEM{}.\DIST{}  until  16-2.family  2.\POSS{}.1\\
%%TRANS1
\glt
‘She/the same went to his family’s place.’ \citep[48]{vanderWal2009a}\\
%%TRANS2
%%EXEND

\z

The emphatic demonstrative can be formally distinguished from the reduplicated demonstrative, illustrated in \xref{bkm:Ref96847566}, which consists simply of a repetition of the simplex demonstrative (compare again the columns in \tabref{bkm:Ref252212434}).

%%EAX
\ea
%%JUDGEMENT
%%LABEL
\label{bkm:Ref96847566}
%%CONTEXT
%%LINE2
\gll
ni  mwalápw’  \textbf{ool’}  \textbf{oólé}  oo-lúm-ák-ats-íy{}-á…\\
%%LINE3
and  1.dog  1.\DEM{}.\DIST{}  \RED{}  1\SM{}.\PFV{}.\DJ{}-bite-\DUR{}-\PLUR-\PASS-\FV{}\\
%%TRANS1
\glt
‘and that dog was bitten…’ \citep[48]{vanderWal2009a}\\
%%TRANS2
%%EXEND

\z

It remains to be studied when exactly the reduplicated demonstrative is used, compared to a single demonstrative, and compared to the emphatic demonstrative.

\section{Summary and conclusion}

In summary, information structure has a rather fundamental impact on the grammar of Makhuwa-Enahara. Primarily, it influences the constituent order in placing topics preverbally and having a dedicated immediate-after-verb focus position, and it is expressed in four main clause conjugations (present, present perfective, imperfective, and past perfective) where a choice is forced between the conjoint form (exclusive focus on the following constituent) and the disjoint verb form (elsewhere). The general template for a Makhuwa phrase not involving a copular construction (including clefts), is as in \figref{fig:makhuwawo}, and illustrated from different spontaneous discourse and narratives in \xxref{bkm:Ref129852497}{bkm:Ref129852498}.


\begin{figure}
%% FIGTAB
%% Not treating this as a table -- doesn't make sense to use formatting of a regular table here (e.g. no vertical lines, double horizontal lines)
%% If needed, this can be replaced with a figure without cell boundaries
\begin{tabular}{c c c c |c|}
\hline
\multicolumn{1}{|c}{discourse marker} & \multicolumn{1}{|c}{topic} & \multicolumn{2}{|c|}{comment} & \multirow{2}{*}{afterthought}\\
\cline{1-4}
&  & \multicolumn{1}{|c|}{focus} & background & \\
\cline{3-5}
\end{tabular}
\caption{Template of Makhuwa constituent order}
\label{fig:makhuwawo}
\end{figure}

%%EAX
\ea
%%JUDGEMENT
%%LABEL
\label{bkm:Ref129852497}
%%CONTEXT
%%LINE1
{}[Vánó…]\textsubscript{DM} [élé ekuw’ éele valé]\textsubscript{TOP} [ekhumalé [tsáyi?]\textsubscript{FOC} ]\textsubscript{COMM}\\
%%LINE2
\gll
vano  ele  ekuwo  ele  vale  e-khum-ale  tsayi\\
%%LINE3
now  9.\DEM{}.\DIST{}  9.cloth  9.\DEM{}.\DIST{}  16.\DEM{}.\DIST{}  9\SM{}-exit-\PFV{}.\CJ{}  how\\
%%TRANS1
\glt
‘Now, the capulana, where did it come from/how did it start?’\\
%%TRANS2
%%EXEND

\z

%%EAX
\ea
%%JUDGEMENT
%%LABEL
%%CONTEXT
%%LINE1
{}[Mí]\textsubscript{TOP} [kóhoótta]\textsubscript{COMM}, [enúmwáari yáaka.]\textsubscript{AFT}\\
%%LINE2
\gll
mi  ki-o-oott-a  enumwaari  e-aka\\
%%LINE3
1\SG{}.\PRO{}  1\SG{}.\SM{}-\PFV{}.\DJ{}-smear-\FV{}  9.adolescence  9-\POSS{}.1\SG{}\\
%%TRANS1
\glt
‘I too used (it), in my adolescence.’\\
%%TRANS2
%%EXEND

\z

%%EAX
\ea
%%JUDGEMENT
%%LABEL
%%CONTEXT
%%LINE1
{}[Vánó]\textsubscript{DM}  [hw-íyá-w-aak-á  khutsúpa.]\textsubscript{COMM}\\
%%LINE2
\gll
vano  khu-iya-w-ak-a  khutsupa.\\
%%LINE3
now  \NARR{}-\IPFV{}-come-\DUR{}-\FV{}  1.hyena\\
%%TRANS1
\glt
‘Then came Hyena.’\\
%%TRANS2
%%EXEND


\z

%%EAX
\ea
%%JUDGEMENT
%%LABEL
%%CONTEXT
%%LINE1
{}[O-r-aalé  [w-eetheyá  ettuúra.]\textsubscript{FOC} ]\textsubscript{COMM}\\
%%LINE2
\gll
o-row-ale  o-eetheya  ettuura\\
%%LINE3
1\SM{}-go-\PFV{}.\CJ{}  15-wet  9.ashes\\
%%TRANS1
\glt
‘He went to make the ashes wet.’\\
%%TRANS2
%%EXEND


\z

%%EAX
\ea
%%JUDGEMENT
%%LABEL
\label{bkm:Ref129852498}
%%CONTEXT
%%LINE1
[Mahíkw’ éény’ áala vá]\textsubscript{TOP} [ashínúmwaary’ otééné]\textsubscript{TOP} [amphéélá [weettakátsá pwitipwíiti.]\textsubscript{FOC}]\textsubscript{COMM}\\
%%LINE2
\gll
ma-hiku  ene  ala  va  a-shi-numwaari  a-oteene  a-m-pheel-a   o-eett-ak-ats-a  pwitipwiti \\
%%LINE3
6-day  \INT{}  5.\DEM{}.\PROX{}  \PP{}  2-\DIM{}-maiden  2-all  2\SM{}-\PRS{}.\CJ{}-want-\FV{}  15-walk-\DUR{}-\PLUR{}-\FV{}  naked \\
%%TRANS1
\glt
‘These days, all the girls want to walk naked.’\\
%%TRANS2
%%EXEND

\z

To express polarity focus and verum in Makhuwa, a simple verb form suffices, but an infinitive may be placed in the left periphery of the sentence functioning as a (potentially contrastive) topic, with the same verb repeated as an inflected form – the construction known as topic doubling. Topic doubling had not been previously described for Makhuwa, and the possibilities for lexical subjects and objects in this construction, together with their interpretation and possible contexts for use, remain to be studied. This will likely also provide further insight into the expression of predicate-centred focus in general.

As in all languages, cleft sentences can also be formed to express focus in Makhuwa-Enahara, whether as a basic cleft, pseudocleft, or reverse pseudocleft. All three constructions show signs of an exclusive interpretation, and further investigation should reveal whether the distinction should be sought in a categorial difference (pronoun vs. full NP), a grammatical difference (subject vs. object), or an interpretational difference (explicit backgrounding, implied vs. encoded exclusivity).

Finally, pronouns and demonstratives are used for reactivation and contrast of mental referents. The interaction between the lack of object markers and the anaphoric use of other pronominal elements as well as NPs is currently being investigated by Teresa Poeta.

While there are numerous areas for further investigation (as indicated throughout the chapter), it can certainly be concluded that it would be impossible to analyse Makhuwa morphosyntax and constituent order without reference to information structure.

\section*{Acknowledgements}

This research was supported by NWO Vidi grant 276-78-001 as part of the BaSIS “Bantu Syntax and Information Structure” project at Leiden University. I thank Ali Buanale, Joaquim Nazário, Zanaira N’gamo, and other Makhuwa speakers for their patience in explaining their language to me; Davety Mpiuka at UEM for his help in accessing the demographic information; Gail Woods, Eva Sandberg, Kyra la Snuzzi, Escondidinho, and BCI on Ilha de Moçambique for their help in the 2019 and 2022 visits;  the reviewers for their comments on an earlier version, and the BaSIS colleagues for their support. Any remaining errors are mine alone.

\section*{Abbreviations and symbols}

Numbers refer to noun classes except when followed by \SG{}/\PL{}, in which case they refer to persons. Commas indicate a pause. The orthographic 〈tt〉 reflects a retroflex voiceless stop and nasalisation is indicated by a tilde under the vowel 〈V̰〉.

\begin{multicols}{2}
\begin{tabbing}
MMMM \= ungrammatical\kill
%%% All Leipzig abbreviations are commented out, following the LangSci guidelines of only listing non-Leipzig abbreviations.
* \> ungrammatical\\
\textsuperscript{\#} \> infelicitous in the given \\ \> context\\
*(X) \> the presence of X is obligatory \\ \>  and cannot grammatically \\ \>  be omitted\\
(*X) \> the presence of X would make\\ \>  the sentence ungrammatical\\
(X) \> the presence of X is optional\\
% \APPL{} \> applicative\\
\CE{} \> counterexpectational\\
\CJ{} \> conjoint\\
% \CAUS{} \> causative\\
cit. \> citation form\\
% \COMP{} \> complementiser\\
\CONN{} \> connective\\
% \COP{} \> copula\\
% \DEM{} \> demonstrative\\
\DIM{} \> diminutive\\
% \DIST{} \> distal\\
\DJ{} \> disjoint\\
\DM{} \> discourse marker\\
% \DUR{} \> durative\\
\E.\DEM{} \> emphatic demonstrative\\
\EXCLAM{} \> exclamative\\
\FV{} \> final vowel\\
\INT{} \> intensifier\\
% \IPFV{} \> imperfective\\
% \LOC{} \> locative\\
\MED{} \> medial\\
\NARR{} \> narrative\\
% \NEG{} \> negation\\
\OM{} \> object marker\\
\OPT{} \> optative\\
% \PASS{} \> passive\\
\PERS{} \> persistive\\
% \PFV{} \> perfective\\
\PLA{} \> plural addressee\\
\PLUR{} \> plurative\\
% \POSS{} \> possessive\\
\PP{} \> pragmatic particle\\
\PRL{} \> predicative lowering\\
\PRO{} \> pronoun\\
% \PROX{} \> proximal\\
% \PRS{} \> present\\
\RED{} \> reduplication\\
% \REFL{} \> reflexive\\
% \REL{} \> relative\\
\RSM{} \> resumptive infinitive\\
\REP{} \> repetitive\\
\RESP{} \> respect\\
% \SBJV{} \> subjunctive\\
\SIT{} \> situative\\
\SM{} \> subject marker\\
\SUBS{} \> subsecutive
\end{tabbing}
\end{multicols}

\printbibliography[heading=subbibliography,notkeyword=this]
\end{document}
