\documentclass[output=paper]{langscibook}
\ChapterDOI{10.5281/zenodo.14833604}
\author{Jenneke van der Wal\affiliation{Leiden University} and Allen Asiimwe\affiliation{Makerere University} and Patrick N.\ Kanampiu\affiliation{University of Edinburgh} and Elisabeth J.\ Kerr\affiliation{Ghent University} and Zhen Li\affiliation{Peking University} and Amani Lusekelo\affiliation{Dar es Salaam College of Education} and Nelsa Nhantumbo\affiliation{Universidade Eduardo Mondlane} and Ernest Nshemezimana\affiliation{Université du Burundi}}
\title{On the expression of information structure in Bantu}
\abstract{This chapter provides an introduction to the volume. It sketches the background to the BaSIS project and methodology, explains the basics of information structure and terms in the field, exemplifies common constructions for information structure in Bantu languages (specifically cleft constructions, word order, and predicate doubling), and concisely explains the main diagnostics from the BaSIS methodology that are used in the following chapters.
}
\IfFileExists{../localcommands.tex}{
  \addbibresource{../localbibliography.bib}
  % add all extra packages you need to load to this file

\usepackage{tabularx,multicol}
\usepackage{url}
\urlstyle{same}

\usepackage{listings}
\lstset{basicstyle=\ttfamily,tabsize=2,breaklines=true}

\usepackage{langsci-basic}
\usepackage{langsci-optional}
\usepackage{langsci-lgr}
\usepackage{langsci-osl}
% \usepackage{./langsci/styles/langsci-lgr}
% \usepackage{./langsci/styles/langsci-osl}
% \usepackage{langsci-gb4e}

\usepackage{tikz}
\usetikzlibrary{patterns,calc}
\pgfdeclarepatternformonly{south east lines}{\pgfqpoint{-0pt}{-0pt}}{\pgfqpoint{3pt}{3pt}}{\pgfqpoint{3pt}{3pt}}{
    \pgfsetlinewidth{0.6pt}
    \pgfpathmoveto{\pgfqpoint{0pt}{3pt}}
    \pgfpathlineto{\pgfqpoint{3pt}{0pt}}
    \pgfpathmoveto{\pgfqpoint{.2pt}{-.2pt}}
    \pgfpathlineto{\pgfqpoint{-.2pt}{.2pt}}
    \pgfpathmoveto{\pgfqpoint{3.2pt}{2.8pt}}
    \pgfpathlineto{\pgfqpoint{2.8pt}{3.2pt}}
    \pgfusepath{stroke}}
    
\usepackage{stmaryrd}
\usepackage{wasysym}
\usepackage{multirow}
\usepackage{caption}
\usepackage{subcaption}
\usepackage{mathrsfs}
\usepackage{qtree}

\usepackage{linguex}


  %pminos do not split footnotes
% \interfootnotelinepenalty=10000 %Footnote in Laporte chapters has to be split SN


%\DeclareIndexNameFormat{default}{%
%\nameparts{#1}%
%\usebibmacro{index:name}%
%{\index[names]}%
%{\namepartfamily}%
%{\namepartgiveni}%
% {}% L1
% {}% L2
%{\namepartprefix}% generates spurious space L3
%{\namepartsuffix}% generates spurious space L4
%}

%  {\DeclareIndexNameFormat{default}{%
%     \usebibmacro{index:name}{\index[names]}{#1}{#3}{#5}{#7}}}

%\DeclareIndexNameFormat{default}{%
%  \usebibmacro{index:name}{\sindex[nom]}{#1}{#3}{#5}{#7}}

%\DeclareIndexNameFormat{default}{%
%  \usebibmacro{index:name}{\sindex[person]}{#1}{#3}{#5}{#7}}
%\DeclareIndexNameFormat{default}{%
%\nameparts{#1} \usebibmacro{index:name}{\sindex[person]]}{\namepartfamily}{‌​\namepartgiven}{\nam‌​epartprefix}{\namepa‌​rtsuffix}}

%\newcommand{\smiley}{:)}

%\renewbibmacro*{index:name}[5]{%
%\usebibmacro{index:entry}{#1}%
%{\iffieldundef{usera}{}{\thefield{usera}\actualoperator}\mkbibindexname{#2}{#3}{#4}{#5}}}

% \newcommand{\noop}[1]{}

%remove for final
%\overfullrule=1mm

\newcommand{\tobi}[2]}}
\renewcommand{\S}[1]{\tobi{#1}{\textsc{*}}}

% this volume references
% puts: [this volume]
% already defined: \citetv
%\newcommand{\citepv}[1]{(\citeauthor{#1} \citeyear*{#1} [this volume])}
\newcommand{\citealtv}[1]{\citeauthor{#1} \citeyear*{#1} [this volume]}

%parentheses around example number
\newcommand{\pref}[1]{(\ref{#1})}

% in-text examples

\newcommand{\lnex}[1]{\textit{#1}} %target lang word
\newcommand{\lnlit}[1]{(lit.: `#1')} %literal reading
\newcommand{\lnlat}[1]{(#1)} % latinization
\newcommand{\lntrans}[1]{`#1'} %translation
\newcommand{\lnexl}[2]%
{\lnex{#1}{} \lnlat{#2}} % ex with latinization
\newcommand{\lnexlat}[3]{\lnex{#1}{} \lnlat{#2}{} \lntrans{#3}} % ex with latinization and tranl.

%ch01
\newcommand{\co}[1]{\mbox{\textbf{#1}}}

%ch09

\newcommand{\cyrbulg}[1]{\begin{otherlanguage*}{bulgarian}#1\end{otherlanguage*}}


%ch10
\newcommand{\nlp}{{\small NLP}}
\newcommand{\mwe}{{\small MWE}}
\newcommand{\rae}{{\small RAE}}
\newcommand{\lvc}{{\small LVC}}
\newcommand{\pos}{{\small P}o{\small S}}
%\newcommand{\todo}[1]{ \textcolor{red}{#1} }

%\renewcommand{\labelenumi}{\theenumi}
%\ainamefmt{{vv}{ll}{, ff}{, jj}} % fullname

\newcommand{\biberror}[1]{{\color{red}#1}}

\newcommand{\osenovaitem}{--~} 
  %% hyphenation points for line breaks
%% Normally, automatic hyphenation in LaTeX is very good
%% If a word is mis-hyphenated, add it to this file
%%
%% add information to TeX file before \begin{document} with:
%% %% hyphenation points for line breaks
%% Normally, automatic hyphenation in LaTeX is very good
%% If a word is mis-hyphenated, add it to this file
%%
%% add information to TeX file before \begin{document} with:
%% %% hyphenation points for line breaks
%% Normally, automatic hyphenation in LaTeX is very good
%% If a word is mis-hyphenated, add it to this file
%%
%% add information to TeX file before \begin{document} with:
%% \include{localhyphenation}
\hyphenation{
    Beck-man
    Ngu-yen
    back-chan-nel
    back-chan-nels
    mo-not-o-nous
    ste-reo-typ-i-cal
}

\hyphenation{
    Beck-man
    Ngu-yen
    back-chan-nel
    back-chan-nels
    mo-not-o-nous
    ste-reo-typ-i-cal
}

\hyphenation{
    Beck-man
    Ngu-yen
    back-chan-nel
    back-chan-nels
    mo-not-o-nous
    ste-reo-typ-i-cal
}
 
  \togglepaper[1]%%chapternumber
}{}

\begin{document}
\renewcommand{\lsChapterFooterSize}{\footnotesize}
\lehead{van der Wal et al.}
\maketitle 
\label{ch:1}
%\shorttitlerunninghead{}%%use this for an abridged title in the page headers


\section{Introduction}

All users of all languages structure the information they give to their addressees so that the addressees can more easily incorporate the information into their current knowledge. This process involves indicating what is already known and what is new, and also which information contrasts with what the addressee might have in mind. Languages differ in terms of which linguistic strategies they have available for the language user to employ to express information structure. While Germanic languages for example use pitch accent to express what information is new or contrastive (“She saw the MAN” vs “She SAW the man”), in the Bantu (Niger-Congo) languages, morphosyntactic strategies are key for the expression of information structure \citep{DowningHyman2016,vanderWal2015,DowningMarten2019}, with many different strategies available. For example, we can see in example \xref{ex:kirundi:1} how word order reflects information structure in Kirundi. Here, the constituent in clause-final position is in focus: this final element forms the new or contrastive information (a more precise definition will follow in \sectref{sec:3:is}).

\ea
\label{ex:kirundi:1}
Kirundi (\chapref{ch:5}, \cite{chapters/kirundi})\footnote{See \sectref{sec:6} and the abbreviation list at the end of this chapter for information on the representation of examples and glosses.}
%%EAX
\ea
%%JUDGEMENT
%%LABEL
%%CONTEXT
(Where did the mother (that we were talking about) give the milk to the children?)\\
%%LINE1
Nya muvyéeyi yaheereye amatá abáana \textbf{mu} \textbf{nzu}. \\
%%LINE2
\gll
nya  mu-vyéeyi  a-a-$\varnothing$-há-ir-ye  a-ma-tá  a-ba-áana \textbf{mu}  \textbf{n-zu}  \\
%%LINE3
  1.\DEM{}7  1-mother  1\SM-\N.\PST{}-\CJ{}-give-\APPL-\PFV{}  \AUG-{}6-milk  \AUG-{}2-child \textbf{18}  \textbf{9-house} \\
%%TRANS1
\glt ‘It is [in the house]\textsubscript{\FOC} that the mother gave milk to the children.’\\
%%TRANS2
%%EXEND

%%EAX
\ex
%%JUDGEMENT
%%LABEL
%%CONTEXT
(What did the mother (that we were talking about) give to the children in the house?)\\
%%LINE1
Nya muvyéeyi yaheereye abáana mu nzu \textbf{amatá}.\\
%%LINE2
\gll
nya  mu-vyéeyi  a-a-$\varnothing$-há-ir-ye  a-ba-áana  mu  n-zu  \textbf{a-ma-tá}\\
%%LINE3
1.\DEM{}7  1-mother  1\SM-\N.\PST-\CJ{}-give-\APPL-\PFV{}  \AUG-2-child  18  9-house  \textbf{\AUG-6-milk}\\
%%TRANS1
\glt
‘It is [the milk]\textsubscript{\FOC} that the mother gave to the children in the house.’\\
%%TRANS2
%%EXEND


%%EAX
\ex
%%JUDGEMENT
%%LABEL
%%CONTEXT
(To whom did the mother (that we were talking about) give the milk in the house?)\\
%%LINE1
Nya muvyéeyi yaheereye amatá mu nzu \textbf{abáana}.\\
%%LINE2
\gll
nya  mu-vyéeyi  a-a-$\varnothing$-há-ir-ye  a-ma-tá  mu  n-zu  \textbf{a-ba-áana}\\
%%LINE3
1.\DEM{}7  1-mother  1\SM-\N.\PST-\CJ{}-give-\APPL-\PFV{}  \AUG-6{}-milk  18  9-house  \textbf{\AUG-2{}}\textbf{{}-child}\\
%%TRANS1
\glt
‘It is [to the children]\textsubscript{\FOC} that the mother gave milk in the house.’\\
%%TRANS2
%%EXEND


\z
\z


Another example of morphosyntactic expression of information structure in Bantu is how focus determines the form of the verb in Makhuwa-Enahara. In \xref{ex:makhuwa:1}, we see that the so-called conjoint verb form (\CJ{}) is only felicitous in a context that indicates focus on the constituent following the verb, whereas the disjoint verb form (\DJ{}) is not acceptable in such an environment (and instead is used when the focus is on the verb itself).

\ea
\label{ex:makhuwa:1}
Makhuwa-Enahara \citep[463]{KerrvanderWal2023}\\
\begin{xlist}
%%EAX
\exi{\CJ}
%%JUDGEMENT
%%LABEL
%%CONTEXT
\textsuperscript{\#}Context 1: Are you frying or grilling the fish? (focus on verb)\\
Context 2: What are you grilling? (focus on object)\\
%%LINE1
Ki\textbf{n}aánéélá ehopá.\\
%%LINE2
\gll
ki-\textbf{n}{}-aaneel-a  ehopa\\
%%LINE3
1\SG.\SM-\PRS.\CJ{}-grill-\FV{}  9.fish\\
%%TRANS1
\glt
\glt ‘I’m grilling a fish.’
%%TRANS2
%%EXEND

%%EAX
\exi{\DJ}
%%JUDGEMENT
%%LABEL
%%CONTEXT
Context 1: Are you frying or grilling the fish? (focus on verb)\\
\textsuperscript{\#}Context 2: What are you grilling? (focus on object)\\
%%LINE1
Ki\textbf{ná}mwáaneéla (ehópa).\\
%%LINE2
\gll
ki-\textbf{na}{}-aaneel-a  ehopa\\
%%LINE3
1\SG{}.\SM{}-\PRS{}.\DJ{}-grill-\FV{}  9.fish\\
%%TRANS1
\glt
\glt ‘I’m grilling it (the fish).’
%%TRANS2
%%EXEND
\end{xlist}
\z


Given these examples, an interesting question arises: how exactly does information structure influence the syntax of Bantu languages? This is the question underlying the project “Bantu Syntax and Information Structure” (BaSIS), funded by an NWO Vidi grant and hosted at Leiden University from December 2017 to July 2023. Insights into this question are important in order to develop linguistic models that accurately reflect how information structure interacts with other components of the grammar. In order to answer this more theoretical question, however, it is first necessary to better understand how information structure works cross-linguistically. Hence, the first step on the path to a better model of information structure in the grammar is to find systematic data on the expression of information structure in Bantu languages. That descriptive endeavour is the aim of this book. 

This book presents detailed descriptions of the expression of information structure in the eight Bantu languages that formed part of the BaSIS project: Tunen (Cameroon), Teke-Kukuya (Congo), Kîîtharaka (Kenya), Kirundi (Burundi), Rukiga (Uganda), Kinyakyusa (Tanzania), Makhuwa-Enahara (Mozambique), and Cicopi (Mozambique). While a more in-depth analysis is given for some topics in some languages, the consistent core aim is to describe the morphosyntactic strategies that are used in each language to structure information. The book is therefore aimed at Bantuists, typologists, and linguists interested in information structure.

The current chapter is intended as an introduction to the investigation of the expression of information structure in Bantu, serving as common background to the chapters on individual languages. We introduce the languages of the BaSIS project in \sectref{bkm:Ref117846249},  discuss the key conceptual notions in information structure as well as the diagnostics for information structure used in the project in \sectref{sec:3:is}, and mention three common morphosyntactic strategies for the expression of information structure in Bantu in \sectref{common}: cleft constructions, word order, and predicate doubling. We then describe the general methodology used in the project in Sections~\ref{bkm:Ref117846286} and~\ref{bkm:Ref117846329} and finally provide a reading guide of what (not) to expect in the book in \sectref{bkm:Ref117846450}.

\section{The languages in the book}
\label{bkm:Ref117846249}
As mentioned above, the Bantu languages are interesting to study for their information structure, as it seems to affect the core of the grammar, with visible effect on morphology and syntax. Another reason that such research is interesting for comparative purposes is the fact that the Bantu language family is large, comprising of some 555 closely-related languages \citep{Hammarström2019} spoken across a large geographical range of sub-Saharan Africa. Despite the number of languages and geographic spread, these languages show a lot of similarities: all languages have a noun class system where the class of a noun is visible in the morphology, on the noun itself and/or in the concord in the phrase and agreement in the clause; all languages apart from those in the north-west show extensive agglutinative verbal morphology, with the verb taking derivational and inflectional suffixes and prefixes; and a large majority of categories are head-initial in a large majority of languages (see for overviews of Bantu clausal and nominal syntax \citealt{DowningMarten2019} and \citealt{VandeVelde2019}, respectively). This relative grammatical similarity makes the Bantu languages important for the discovery of general linguistic principles, as it creates the possibility to study microvariation in the expression of information structure, giving us potential insight into the parameters of language variation as well as language change.

One of the strengths of the BaSIS project is its collaboration between native speaker linguists and external linguists. In this way, the investigation benefits from the inside as well as the outside perspective, with each researcher asking questions and making associations that the other would not as easily get to. The selection of languages for the project was therefore dependent on the collaborations, plus the two languages investigated by the PhD candidates, leading to what one might call a convenience sample (only that ten languages is perhaps too small to even properly be called a sample). Although there is a fair geographical spread, the languages are very few and not at all a representative sample of the Bantu family, with a greater representation of Eastern Bantu languages. We therefore hope that this research may inspire others to explore the same questions for other (Bantu) languages in order to arrive at a more complete understanding of information structure in the family. 

The languages at the start of the project were the following, each specified with its ISO code, Guthrie number,\footnote{As is customary in Bantu linguistics, an alphanumeric classification is provided based on the classification of \citet{Guthrie1948} and updated by \citet{Maho2003,Maho2009} and \citet{Hammarström2019}. This Guthrie classification is based on the location of the language spoken, ranging from zone A (languages in the north-west) to zone S (languages in southern Africa).} country where it is primarily spoken, and the respective primary researcher indicated (see also the map in \figref{fig:langmap}, page~\pageref{fig:langmap}):

\begin{itemize}
\item Tunen (ISO 639-3 [tvu], Guthrie A44, Cameroon; Elisabeth J. Kerr)
\item  Teke-Kukuya ([kkw], B77a, Congo; Zhen Li)
\item  Kîîtharaka ([thk], E54, Kenya; Patrick N. Kanampiu)
\item  Kirundi ([run], JD62, Burundi; Ernest Nshemezimana)
\item  Rukiga ([nyn], JE14, Uganda; Allen Asiimwe)
\item  Kinyakyusa ([nyy], M31, Tanzania; Amani Lusekelo)
\item  Makhuwa-Enahara ([vmw], P31E, Mozambique; Jenneke van der Wal)
\item  Otjiherero ([her], R30, Namibia; Jekura U. Kavari)
\item  Changana ([tso], S53, South Africa, Mozambique; Aurélio Z. Simango)
\item  Cicopi ([cce], S61, Mozambique; Nelsa Nhantumbo)
\end{itemize}
The map below shows the approximate locations of these languages, based on coordinates from Glottolog \citep{HammarströmEtAl2022}. Further information about each of the languages, such as geography and demography, is given in the respective chapters.

  
\begin{figure}
\includegraphics[width=\textwidth]{figures/map.png}
\caption{Map showing the geographical distribution of the languages in the BaSIS project after the removal of Herero.}
\label{fig:langmap}
\end{figure}

Two of these languages unfortunately do not form part of this book, as you will have noticed. The COVID-19 pandemic caused significant delay in the fieldwork, which meant that the collaboration with Jekura Kavari planned for 2020/21 was no longer feasible. Otjiherero was therefore removed from the project. For Changana, the collaboration started with a fieldwork visit in 2019 which resulted in a database, but, unfortunately, the collaborator for the language, Aurélio Simango, passed away in January 2022, hence the work was not completed.

For the eight remaining languages, each chapter in this book describes the morphosyntactic strategies that express (some aspect of) information structure. We do not give the background on general properties of the grammar of each language, instead diving right into the expression of information structure. For some languages, the description is more in depth than for others, as for some we could draw on already published sources (e.g. Nshemezimana’s \citeyear{Nshemezimana2016} thesis on Kirundi information structure) or separate articles published during the project (e.g. \citealt{Li2020} on the Kukuya passive, and \citealt{AsiimwevanderWal2021} on the Rukiga contrastive topic marker). While some linguistic strategies are covered in every chapter, such as word order and clefts, each language also has its own specific topics, depending on the special features of the language, and also on the interest of the researcher(s) involved. For the data collection in all languages we used the same methodology, which we explain in detail in Sections~\ref{common} and~\ref{bkm:Ref117846329}. First, however, we explain the concepts related to information structure that form the research focus of the chapters in this book.

\section{Information structure}
\label{sec:3:is}
There are two main aspects that play a role in information structure. The first is the \textit{activation status} of referents, and the second is the \textit{functions} that referents may take within the sentence (specifically, topic and focus). We discuss each of these in turn.

\subsection{Referent activation status}

As introduced above, information structure looks at how we construct sentences dependent on how we want to package information for the addressee. So, what counts as information? One key component is (the mental representation of) referents, and the expressions used to refer to them. For example, we may start a story by introducing the main characters (e.g. “There was a chicken and a cockroach”), and then continue to refer to these characters over the story (“the cockroach”, “he”, “that chicken” etc.). Here, there is an important distinction between on the one hand the \textit{referents} themselves – which can either be conceptualised as objects existing in the real world, or as mental representations existing in the minds of the speakers – and on the other hand the linguistic \textit{expressions} used to refer to those referents. In this way, a single referent (e.g. the chicken from the story above) may be referred to using different linguistic strategies (e.g. a noun phrase “the chicken”, pronoun “it”, or modified noun phrase “that young chicken”). \textit{Referent tracking} looks at how such referencing works in discourse.

Referents have a certain \textit{activation status}, relating to their cognitive accessibility. If they have not been mentioned and are not in the direct context, they are \textit{inactive}, whereas if they are currently under discussion, they are highly \textit{active} \citep[see e.g.][]{Chafe1987}. A referent’s activation status has an influence on which expressions we use to refer to it, as proposed in various guises by \citet{Givón1976,Givón1983}, \citet{Chafe1976,Chafe1987}, \citet{Ariel1990,Ariel2001} and \citet{GundelEtAl1993}. \citeauthor{Ariel1990} formalises the intuition that referring expressions mark varying degrees of mental accessibility of referents, as an inverse relation between linguistic encoding and accessibility. A new referent is inaccessible (i.e. inactive), and thus requires the most linguistic encoding (e.g. full NP + modifier), whereas a previously mentioned referent is considered accessible (active), and hence needs less linguistic encoding (with zero anaphora being the lowest in the scale). The degree of accessibility forms a scale referred to as the ``accessibility hierarchy", as represented in \xref{bkm:Ref122686119} (see also the Givenness Hierarchy of \citealt{GundelEtAl1993}).\footnote{Apart from prototypical referents (things, people, places), the lexical content of verbs (events, states) may also be accessible to a higher or lower degree. Givenness of events may be expressed through ellipsis (e.g. ‘Did you eat the last biscuit?’\slash ‘I did \sout{eat the last biscuit}’), and may influence the expression of the (often nominalised, \citealt{Givón2001}) predicate as a topic. This is discussed further in literature on ellipsis (see e.g. \citealt{Merchant2001} on givenness of elided material), and predicate topicalisation and predicate doubling (see \citealt{GüldemannFiedler2022} for an overview of such constructions in Bantu). See also \citet{HegartyEtAl2001} and \citet{GundelEtAl2003} on discourse deixis and accessibility of entities introduced by a clause (e.g. ‘We believe her, the court does not, and \textit{that/it} resolves the matter’, \citealt[282]{GundelEtAl2003}).}

\ea
\label{bkm:Ref122686119}
Accessibility Hierarchy \citep[31]{Ariel2001}:\\
full name + modifier > full name > long definite description > short definite description > last name > first name > distal demonstrative + modifier > proximate demonstrative + modifier > distal demonstrative + NP > proximate demonstrative + NP
> distal demonstrative > proximate demonstrative > stressed pronoun + gesture > stressed pronoun > unstressed pronoun > cliticized pronoun > verbal person inflections > zero 
\z

The accessibility/activation status not only influences the expression used for a referent, but also which information-structural function the referent is suited to play in the sentence. This is because the referent in the function of topic (which will be covered in the next section) needs to be able to provide an anchoring point for the addressee to attach the new information to that is provided in a sentence, and therefore it cannot be completely new. This is captured in \citegen{Lambrecht1994} Topic Acceptability Scale:

\ea
Topic Acceptability Scale \citep[165]{Lambrecht1994}\\
\newlength{\introscalebwidth}
\settowidth{\introscalebwidth}{ unanchored}
\mbox active > accessible > unused > brand-new anchored > \parbox[t][2.5\baselineskip]{\introscalebwidth}{brand-new unanchored}\\
% \begin{tabularx}{\linewidth}{llllllXlX}
% active & > & accessible & > & unused & > & brand-new anchored & > & brand-new unanchored \\
% \end{tabularx}
\textit{more acceptable} \hfill  \textit{less acceptable}
\z

Two main information-structural functions have been proposed and widely accepted within the field of information structure (although see \citealt{MatićWedgwood2013} and \citealt{Ozerov2018,Ozerov2021} for critical views): \textit{topic} and \textit{focus}. While the function of a referent and its activation status interact, they are crucially different aspects. We discuss each function in more detail in the following sections.

\subsection{Topic}

In terms of information structure, most sentences can be split into a \textit{topic} and a \textit{comment}.\footnote{This concerns the sentence topic, and not the discourse topic. The latter refers to the broader topic of the paragraph or conversation; see \citet{Lambrecht1994}.} The comment, in turn, may contain a focused constituent, with the rest as background, as represented in \figref{fig:clausefunx} (see also \citealt{NeelemanEtAl2009}, for example, for a separation in this way).


\begin{figure}
%% FIGTAB
%% Not treating this as a table -- doesn't make sense to use formatting of a regular table here (e.g. no vertical lines, double horizontal lines)
%% If needed, this can be replaced with a figure without cell boundaries
\begin{tabular}{c c |c|}
\hline
\multicolumn{1}{|c}{topic} & \multicolumn{2}{|c|}{comment}\\
\hline
& \multicolumn{1}{|c|}{focus} & background\\
\cline{2-3}
\end{tabular}
\caption{Functions in the clause}
\label{fig:clausefunx}
\end{figure}

The topic can be defined as “what the sentence is about” \citep{Reinhart1981}, or the “spatial, temporal, or individual framework within which the main predication holds” \citep[50]{Chafe1976}. This can be said to hold for all subtypes of topic. The topic-comment structure helps the addressee incorporate the new information into their knowledge, where the topic serves as an anchor to attach the comment to previously known information. Consider example \xref{ex:teke:tcs}: the question in \xref{ex:teke:tcs:q} targets the action of the ‘woman’, who represents the given information and functions as a topic. In the answer  to this question in \xref{ex:teke:tcs:a}, the topic referent ‘woman’ is expressed by a pronoun \textit{ndé}, and the VP is the part that answers the content question, therefore forming the comment (and the focus in this case), expressed in SVO word order corresponding to Topic-Comment order.

\ea
Teke-Kukuya (\chapref{ch:3}, \cite{chapters/teke})\\
\label{ex:teke:tcs}Topic-comment structure (VP focus)
%%EAX
\ea
%%JUDGEMENT
%%LABEL
\label{ex:teke:tcs:q}
%%CONTEXT
%%LINE1
%%LINE2
\gll
Mu-kái  kí-má  ká-sî?\\
%%LINE3
1-woman  7-what  1\SM.\PST{}-do.\PST{}\\
%%TRANS1
\glt
‘What did the woman do?’\\
%%TRANS2
%%EXEND

%%EAX
\ex
%%JUDGEMENT
%%LABEL
\label{ex:teke:tcs:a}
%%CONTEXT
%%LINE1
%%LINE2
\gll
{}[Ndé]\textsubscript{TOPIC} [á-búnum-i  baa-ntaba]\textsubscript{COMMENT}.\\
%%LINE3
{\db}1.\PRO{}   {\db}1\SM.\PST{}-feed-\PST{}  2-goat  \\
%%TRANS1
\glt
‘She fed the goats.’\\
%%TRANS2
%%EXEND


\z
\z


Different subtypes of topic have been proposed. These subtypes are seen as conceptually distinct and this conceptual difference may correspond to differences in expression. At least in part, these different subtypes can be understood by the interaction of a referent’s activation status and its topic function. When the referent is highly active and has already functioned as a topic before the current sentence, this is called a \textit{familiarity topic} or \textit{continuing} \textit{topic} \citep[e.g.][]{FrascarelliHinterhölzl2007,Givón1983,Reinhart1981}. In all languages in our sample, such topics are preferably expressed through a pronoun (i.e. a reduced form). This is illustrated for Tunen in \xref{bkm:Ref131685493}, where the class 1 subject marker \textit{a} is used in the second and third sentences to refer to the personified referent ‘chicken’, which functions as a topic in each sentence and was previously introduced by a full noun phrase \textit{miɔkɔ́} ‘chicken’.

\ea
Tunen (database Kerr)
\label{bkm:Ref145315632}\label{bkm:Ref131685493}
%%EAX
\sn
%%JUDGEMENT
%%LABEL
%%CONTEXT
%%LINE1
u bótólókiə ɔ hitɛ́\textsuperscript{!}yí hinó, miɔkɔ́ a ná hɛ́áná,\\
%%LINE2
\gll
ɔ	bótólókiə	ɔ	hitɛ́\textsuperscript{L}yí	hɛ-nó	\textbf{miɔkɔ́}	a	ná	hɛ́áná\\
%%LINE3
\PREP{}  begin  \PREP{}  \DEM.\DISC{}.\EMPH{}.19  19-day  \textbf{9.chicken}  1\SM{}  \PST{}2  become\\
%%TRANS1
\glt
‘\`{A} partir de ce jour, la poule est devenue -’\\
%%TRANS2
‘From this day on, the chicken has become -’

%%EXEND



%%EAX
\sn
%%JUDGEMENT
%%LABEL
%%CONTEXT
%%LINE1
a níŋə́kə na bɛ́ndɔ ɔmbɛ́l,\\
%%LINE2
\gll
\textbf{a}  níŋə́-aka	na	bɛndɔ	ɔmbɛ́la\\
%%LINE3
\textbf{1\SM{}}  live-\DUR{}  with  2.person  3.house\\
%%TRANS1
\glt
‘Elle vive avec les hommes à la maison.’\\
%%TRANS2
‘She lives with humans in their homes.’

%%EXEND

\pagebreak

%%EAX
\sn
%%JUDGEMENT
%%LABEL
%%CONTEXT
%%LINE1
a bɛɔnɔ́ nɛakak,  \\
%%LINE2
\gll
\textbf{a}  bɛ-ɔnɔ́	nɛaka-aka\\
%%LINE3
\textbf{1\SM{}}  8-egg  make-\DUR{}\\
%%TRANS1
\glt
‘Elle pond des œufs ;’\\
%%TRANS2
‘She lays eggs;’

%%EXEND



%%EAX
\sn
%%JUDGEMENT
%%LABEL
%%CONTEXT
%%LINE1
bɛ́ndɔ bá nɛak.\\
%%LINE2
\gll
bɛndɔ  bá  nɛa-aka\\
%%LINE3
2.person  2\SM{}  eat-\DUR{}\\
%%TRANS1
\glt
‘les hommes (les) mangent.’\\
%%TRANS2
‘people eat (them).’

%%EXEND

\z

Naturally, the topic must sometimes change. New topics like this are called \textit{shift} \textit{topics} \citep{Erteschik-Shir2007}. A shift topic is illustrated above in \xref{bkm:Ref145315632} where in the last phrase the topic (and subject) shift from the chicken to \textit{bɛndɔ} ‘the people’ and is expressed with a full noun phrase, i.e. using more linguistic material. We can see the same for Kîîtharaka in \xref{bkm:Ref122686153}, where the topic (and subject) shifts from the referent ‘the wife’ to the referent ‘the stick’. The stick therefore forms the shift topic, and because of its relatively less accessible status, \textit{mûti} ‘stick’ is marked with a demonstrative, rather than using a more reduced encoding.


%%EAX
\ea
%%JUDGEMENT
(Previous story: Very long ago there was a man who had a wife and a daughter. They lived well until one day...)
%%LABEL
\label{bkm:Ref122686153}
%%CONTEXT
Kîîtharaka (database Kanampiu \& Van der Wal)\\
%%LINE1
Mwekûrû akiuna nkû aakûrûtwa i mûtî.\\
%%LINE2
\gll
\textbf{mû-ekûrû}  a-kî-un-a  n-kû  a-a-kûrût-w-a  ni  mû-tî\\
%%LINE3
\textbf{1-wife}  1\SM-\DEP{}-break-\FV{}  9-firewood  1\SM-\PST{}-scratch-\PASS-\FV{}  \FOC{}  3-stick\\
%%TRANS1
\glt
‘The wife was injured by a stick while collecting firewood.’\\
%%TRANS2
%%EXEND

%%EAX
\sn
%%JUDGEMENT
%%LABEL
%%CONTEXT
%%LINE1
\textbf{Mûti ûyû} n'wari na cûmû.\footnote{Note that the prefix \textit{ni}\nobreakdash- scopes over the verb and is independent of the initial phrase being the topic -- see for further details \chapref{ch:4} on Kîîtharaka, \cite{chapters/kiitharaka}}\\
%%LINE2
\gll
\textbf{mû-tî}  \textbf{ûyû}  ni-w-a-rî  na  cûmû\\
%%LINE3
\textbf{3-stick}  \textbf{3.\DEM.\PROX{}}  \FOC{}-3\SM{}-\PST{}-be  with  poison\\
%%TRANS1
\glt
‘This stick was poisonous.’\\
%%TRANS2
%%EXEND


\z

A third type of topic is a \textit{contrastive topic} \citep{Büring2016,Repp2010,Vermeulen2012}, which occurs when two topics are compared – even if one of them may be implicit. \citet{AsiimwevanderWal2021} show for Rukiga that contrastive topics may be marked by a particle, as illustrated in example \xref{bkm:Ref122686235} with the class 11 particle \textit{rwo}. This particle indicates that there is a contrast with other topical referents: here, when talking about various areas to be weeded (the topic referents), the particle indicates that the predicate holds for the banana plantation, but perhaps not for the maize garden or the sorghum field.

%%EAX
\ea
%%JUDGEMENT
%%LABEL
\label{bkm:Ref122686235}
%%CONTEXT
Rukiga \citep[11]{AsiimwevanderWal2021}\\
(Context: They have not weeded the maize garden, or there is another task they have not done)\\
%%LINE1
 Orutookye \textbf{rwó} báárubágara.\\
%%LINE2
\gll
  o-ru-tookye  \textbf{ru-o}  ba-aa-ru-bagar-a\\
%%LINE3
\AUG{}-11-banana.plantation  \textbf{11-\CM{}}  2\SM-\N.\PST{}-11\OM{}-weed{}-\FV{}\\
%%TRANS1
\glt ‘As for the banana plantation, they have weeded it.’\\
%%TRANS2
%%EXEND


\z

Contrastive topics seem to be connected to mirativity, that is, the expression of surprise (for the speaker or addressee), and also to intensive and depreciative readings. In the BaSIS languages, we see these interpretations for topic doubling (see \sectref{bkm:Ref147299258}), and for the use of the contrastive topic marker \nobreakdash-\textit{o}, as in \xref{bkm:Ref147299352}. Recent research also shows a mirative interpretation of object-marker doubling, see \citet{LangadaCâmaraEtAlFut, LippardEtAlFut, SikukuDiercks2021, KerrvanderWal2023}.

%%EAX
\ea
%%JUDGEMENT
%%LABEL
\label{bkm:Ref147299352}
%%CONTEXT
Kîîtharaka (\chapref{ch:4}, \cite{chapters/kiitharaka})\\
(Has it rained that much?)\\
%%LINE1
 Mbúra \textbf{yoó} îkuúra.\\
%%LINE2
\gll
m-bura  \textbf{î-o}  î-kû-ur-a\\
%%LINE3
9-rain  \textbf{9-\CM{}}  9\SM-\PRS{}-rain-\FV{}\\
%%TRANS1
\glt ‘It has really rained.’ (intensive interpretation)\\
%%TRANS2
%%EXEND


\z

In addition to these three types of topics, a sentence may be preceded by a \textit{frame-setting} or \textit{scene-setting} topic. Such topics are not always arguments of the sentence. For example, a temporal phrase sets the scene in \xref{bkm:Ref145317716}, and example \xref{bkm:Ref145317717} shows a location, neither of which are arguments. Such topics may well be universal: \citet[231]{Gundel1988} states that “Every language has syntactic topic constructions in which an expression which refers to the topic of the sentence is adjoined to the left of a full sentence comment.” However, note that such ``types" of topics only tell us something about the strategy to express them, i.e. the form, as their function as topics is still the same, namely restricting what the comment applies to.

%%EAX
\ea
%%JUDGEMENT
%%LABEL
%%CONTEXT
Makhuwa (database Van der Wal)\\
%%LINE1
\label{bkm:Ref145317716}\textbf{Mahíkw’ éen’ áalá} tsinááthówa mivélo.\\
%%LINE2
\gll
\textbf{ma-hiku}  \textbf{ene}  \textbf{ala}  tsi-naa-thow-a  mi-velo\\
%%LINE3
\textbf{6-day}  \textbf{\INT{}} \textbf{6.\DEM.\PROX{}}  4\SM-\PRS.\DJ{}-lack-\FV{}  4-broom\\
%%TRANS1
\glt
‘These days there are not many brooms.’\\
%%TRANS2
%%EXEND


%%EX
\ex
%%JUDGEMENT
%%LABEL
\label{bkm:Ref145317717}
%%CONTEXT
Kirundi (database Nshemezimana \& Van der Wal)\\
%%LINE1
\textbf{Mu Rumonge} amashuúre ntaruúgurura\\
%%LINE2
\gll
\textbf{mu}  \textbf{Rumonge}  a-ma-shuure  nti-a-raa-ugurur-a\\
%%LINE3
\textbf{18}  \textbf{Rumonge}  \AUG{}-6-school  \NEG-6\SM-\INCP{}-open-\FV{}\\
%%TRANS1
\glt
‘In Rumonge schools are not open yet.’\\
%%TRANS2
%%EXEND


\z

Topics can also combine and jointly restrict what the sentence is about. This is illustrated in \xref{bkm:Ref145318359}, where the time frame is specified further with each topic phrase (marked by square brackets).\largerpage

%%EAX
\ea
%%JUDGEMENT
%%LABEL
\label{bkm:Ref145318359}
%%CONTEXT
Kinyakyusa (\chapref{ch:7}, \cite{chapters/kinyakyusa})\\
(Previous lines: Jackson has a dog which he loves very much. Also, he has a frog which he put in a bottle. It stayed and slept there.)\\
%%LINE1
{[Akabalilo kamo] [pakilo] [Jakisoni n’mbwa jake] [bo bikulonda ukubuuka nkulambalala] bakikeetile ikyula.}\\
%%LINE2
\gll
a-ka-balilo  ka-mo  pa-kilo  Jakisoni  na  mbwa  ji-ake  bo   ba-ku-lond-a  u-ku-buuk-a  mu-ku-lambalal-a  ba-ki-keet-ile  i-ki-ula \\
%%LINE3
\AUG{}-12-time  12-one  16-night  1.Jackson  and  9.dog  9-\POSS.1{}  when 2\SM-\PRS{}-want-\FV{}  \AUG{}-15-go-\FV{}  18-15-lie.down-\FV{}  2\SM-7\OM{}-look-\PFV{}  \AUG{}-7-frog \\
%%TRANS1
\glt
‘One day, at night, when Jackson and his dog wanted to go to sleep, they looked at the frog.’\\
%%TRANS2
%%EXEND

\z

While speakers greatly prefer to partition sentences to first indicate the topic and then add a comment on that topic (a so-called \textit{categorical} sentence), some sentences have no topic expression. These are \textit{thetic} sentences, which present the information as one piece. Thetics have been said to be about the ``here and now" – the topic referent is then the ``stage topic" \citep{Erteschik-Shir1997,Erteschik-Shir2007,Gundel1974}. Thetics tend to occur when presenting a new entity, for example as in \xref{ex:kinyakyusa:pstvbsubj}, or when announcing something out of the blue. In these cases, none of the referents are suitable to take on the role of topic. Since there is a well-known close association between subject and topic \citep{LiThompson1976}, the (logical) subject in a thetic sentence is typically explicitly marked as \textit{not} being the topic, for example by placing it in a postverbal position, as in \xref{ex:kinyakyusa:pstvbsubj}, leading to departures from the canonical word order. Theticity is discussed in detail in \citet{Sasse1987,Sasse1996,Sasse2006}.

%%EAX
\ea
%%JUDGEMENT
%%LABEL
\label{ex:kinyakyusa:pstvbsubj}
%%CONTEXT
Kinyakyusa (database Lusekelo, Msovela, and Van der Wal)\\
(Previous lines: An old woman walked through the forest and was pierced by a thorn. She sat down and felt sorry for herself.)\\
%%LINE1
Nakalinga akiindile \textbf{undumyana jumo}; ingamu jaake jo Lwitiko.\\
%%LINE2
\gll
nakalinga  a-kiind-ile  u-n-lumyana  ju-mo  i-ngamu  ji-ake  jo  Lwitiko\\
%%LINE3
soon  1\SM{}-pass-\PFV{}  \AUG{}-1-young  1-one  \AUG{}-9.name  9-\POSS.\SG{}  9.\IDCOP{}  1.Lwitiko\\
%%TRANS1
\glt
‘Soon there passed a certain boy; his name is Lwitiko.’\\
%%TRANS2
%%EXEND


\z

We thus see that the function of topic may be taken by referents that are at least identifiable, and that both topic and non-topic may be marked in languages, with the latter type visible in subjects of thetic sentences, which do not contain a topic-comment split.

\subsection{Focus}

In a categorical sentence (divided into topic and comment), the comment can be subdivided into the focus and the background, as seen above in Figure \ref{fig:clausefunx}. We have informally referred to focus as the ``new or contrastive" information, but a more formal definition is that focus “indicates the presence of alternatives that are relevant for the interpretation of linguistic expressions” (\citealt[7]{KrifkaMusan2012}; see \citealt{Rooth1992,Rooth1996,Rooth1985} for the Alternative Semantics approach). The constituent in focus thus triggers a set of relevant alternatives. For instance, in example \xref{ex:kiitharaka:ashabread}, the referent ‘bread’ is in focus. This means that it triggers alternatives, i.e. other things that Asha could have baked, such as cake, chapati, doughnuts, and so on. The answer in \xref{ex:kiitharaka:ashabread} expresses that Asha baked bread as opposed to those alternatives – in other words, it selects bread as the relevant alternative out of the set of alternatives generated by the question word ‘what?’.\largerpage

%%EAX
\ea
%%JUDGEMENT
%%LABEL
\label{ex:kiitharaka:ashabread}
%%CONTEXT
Kîîtharaka (database Kanampiu \& Van der Wal)\\
(What has Asha baked?)\\
%%LINE1
Áshá akáándíré [mûgááté]\textsubscript{FOC}.\\
%%LINE2
\gll
Asha   a-kaand-ire   mû-gaate\\
%%LINE3
1.Asha   1\SM{}-bake-\PFV{}   3-bread\\
%%TRANS1
\glt
‘Asha baked bread.’\\
%%TRANS2
%%EXEND

\gll
Alternatives:   {\{cake, bread, chapati, doughnuts, …\}}\\
Answer:     {\{cake, \textbf{bread}, chapati, doughnuts, …\}}\\

\z

There are two axes of variation in focus: the \textit{scope} of focus and the \textit{type} of focus. In terms of scope, there is a distinction between term focus (meaning focus on an argument or adjunct) and predicate-centred focus (PCF; following \citealt{Güldemann2003,Güldemann2009}).\footnote{We do not take \citegen{Lambrecht1994} \textit{sentence focus} to belong to this overview, as it concerns thetic sentences, discussed above.} Within term focus, there can be focus on the whole noun phrase (whole NP focus) or focus on only a sub-element, such as a nominal modifier (sub-NP focus). Within PCF, there can be focus on the lexical value of the verb itself (the \textit{state of affairs}), the truth value (also called \textit{polarity} or \textit{verum}), tense/aspect/mood (also referred to in terms of \textit{operator focus}), or the whole verb phrase (\textit{VP focus}). This is schematically represented in \figref{fig:focuselements} and illustrated in the examples below.

\begin{figure}
\includegraphics[width=\textwidth]{figures/fig3-focuselements.png}

% \begin{forest} for tree={draw,grow'=east}
% [{focus\\triggers alternatives}
%   [nothing else
%     [simple focus]
%   ]
%   [{operation on\\ alternatives}
%     [exclusivity]
%     [scalarity]
%   ]
% ]
% \end{forest}
\caption{Different parts of the sentence which may be in focus (\citealt{vanderWal2021}, building on \citealt{Dik1997,Güldemann2009,Zimmermann2016}, \citealt[540]{GüldemannFiedler2022}).}
\label{fig:focuselements}
\end{figure}



\ea
Term focus argument: Rukiga (\chapref{ch:6}, \cite{chapters/rukiga})
%%EAX
\ea
%%JUDGEMENT
%%LABEL
%%CONTEXT
%%LINE1
Hélen atwire \textbf{ki}?\\
%%LINE2
\gll
Helen  a-twar-ire  ki\\
%%LINE3
1.Hellen  1\SM-{}take-\PFV{}  what\\
%%TRANS1
\glt
‘What did Hellen take?’\\
%%TRANS2
%%EXEND


%%EAX
\ex
%%JUDGEMENT
%%LABEL
%%CONTEXT
%%LINE1
Hélen atwir’ \textbf{ékikópo}.\\
%%LINE2
\gll
Helen  a-twar-ire  [e-ki-kopo]\textsubscript{FOC}\\
%%LINE3
1.Hellen  1\SM{}-take-\PFV{}  {\db}\AUG-{}7-cup \\
%%TRANS1
\glt
‘Hellen took [a cup]\textsubscript{FOC}.’\\
%%TRANS2
%%EXEND


\z
\z

\ea
Term focus adjunct: Teke-Kukuya (\chapref{ch:3}, \cite{chapters/teke})

%%EAX
\ea
%%JUDGEMENT
%%LABEL
%%CONTEXT
%%LINE1
%%LINE2
\gll
Mwáana  \textbf{munkí}  ká-dzí  ntsúi?\\
%%LINE3
1.child  when  1\SM.\PST{}-eat.\PST{}  1.fish\\
%%TRANS1
\glt
‘When did the child eat the fish?’\\
%%TRANS2
%%EXEND

%%EAX
\ex
%%JUDGEMENT
%%LABEL
%%CONTEXT
%%LINE1
%%LINE2
\gll
Ndé  ntsúi  [\textbf{mu}  \textbf{ngwaalí}]\textsubscript{FOC}  ká-dzí.\\
%%LINE3
1.\PRO{}  1.fish  {\db}18.\LOC{}  9.morning  1\SM.\PST{}-eat.\PST{}  \\
%%TRANS1
\glt
‘S/He ate the fish [in the morning]\textsubscript{FOC}.’\\
%%TRANS2
%%EXEND


\z
\z



\ea
Sub-NP focus: Rukiga \citep[1297]{AsiimweEtAl2023}\\
(Which car should we take?)

%%EAX
\ea
%%JUDGEMENT
[]{
%%LABEL
%%CONTEXT
%%LINE1
%%LINE2
\gll
Tu-twar-e  é-mótoka  \textbf{é-y-ângye}.\\
%%LINE3
1\PL.\SM{}-take-\SBJV{}  \AUG{}-9.car  \AUG{}-9-\POSS.1\SG{}\\
%%TRANS1
\glt
‘We take \textit{my} car.’\\
%%TRANS2
}
%%EXEND

%%EAX
\ex
%%JUDGEMENT
[\textsuperscript{\#}]{
%%LABEL
%%CONTEXT
%%LINE1
%%LINE2
\gll
Tu-twar-e   é-mótoka  \textbf{y-angye}.\\
%%LINE3
1\PL.\SM{}-take-\SBJV{}  \AUG{}-9.car  9-\POSS.1\SG{}\\
%%TRANS1
\glt
‘We take my car.’\\
%%TRANS2
}
%%EXEND


\z
\z

%%EAX
\ea
%%JUDGEMENT
%%LABEL
%%CONTEXT
State-of-Affairs focus: Kîîtharaka (database Kanampiu \& Van der Wal)\\
%%LINE1
I kûrííngá t\'{û} rííngiré ng’óombé – tûtíracíthaika.\\
%%LINE2
\gll
ni  kû-riinga  tû-riing-ire  ng’-oombe  tû-ti-ra-ci-thaik-a\\
%%LINE3
\FOC{}  15-hit  1\PL.\SM{}-hit-\PFV{}  10-cows  1\PL.\SM-\NEG-\YPST{}-10\OM{}-tie-\FV{}\\
%%TRANS1
\glt ‘We {\itshape hit} the cows, we didn’t tie them.’
%%TRANS2
%%EXEND

\z

%%EAX
\ea
%%JUDGEMENT
%%LABEL
%%CONTEXT
VP focus: Changana (database Simango \& Van der Wal)\\
(Context: Picture of a man with a pumpkin in a wheelbarrow. Is this man playing football?)\\
%%LINE1
Ángábélí bó:lú, óxíxa kwe:mbe.\\
%%LINE2
\gll
a-nga-bel-i  bolu  a-o-xix-a  kwembe\\
%%LINE3
1\SM-\NEG{}-play-\NEG{}  5.ball  1\SM-\EXCL{}-unload-\FV{}  5.pumpkin\\
%%TRANS1
\glt
‘No, he is not playing football, he is unloading a/the pumpkin.’\\
%%TRANS2
%%EXEND


\z
%%EAX
\ea
%%JUDGEMENT
%%LABEL
%%CONTEXT
Verum (Cicopi; \citealt[485]{KerrvanderWal2023})\\
(You are not eating the cake that I bought. It’ll go bad.)\\
%%LINE1
Kudya hâ:dyá.\\
%%LINE2
\gll
ku-dya  hi-a-dy-a\\
%%LINE3
15-eat  1\PL.\SM{}-\DJ{}-eat-\FV{}\\
%%TRANS1
\glt
‘We {\itshape are} eating (it).’\\
%%TRANS2
%%EXEND


\z

Turning now to the type of focus, we need to distinguish between the semantic and pragmatic aspects of interpretation, following \citet{Krifka2008,ZimmermannOnea2011}. The semantic aspects affect what \citet[249]{Krifka2008} calls the ``common ground content" and therefore the truth-conditions of the utterance, whereas the pragmatic aspects concern the ``common ground management", which does not affect the truth-conditions. It must be noted that it is not always clear whether an aspect of interpretation is rooted in the semantics or pragmatics. The types of focus are shown in the overview in \figref{fig:focustypes}.

\begin{figure}
\includegraphics[width=\textwidth]{figures/fig4-focustypes.png}
\caption{Types of semantic and pragmatic focus}
\label{fig:focustypes}
\end{figure}

The semantic aspect of focus concerns the set of alternatives. Under the Alternative Semantics approach to focus introduced at the start of this section, all focus types by definition trigger the generation of a set of alternatives. If nothing else happens, this is called \textit{simple focus}, also known as ``(new) information focus" \citep{É.Kiss1998} or ``assertive focus" \citep{HymanWatters1984}. For simple focus, the proposition is true for the focused referent, and there are alternatives for which we do not know if the proposition is true. This was illustrated using a question-answer context in \xref{ex:kiitharaka:ashabread} above, and is illustrated again for a simple focus context ‘Who did you meet?’ -- ‘I met Alex’ in \xref{bkm:Ref134522227} below.

\ea
\label{bkm:Ref134522227}
\gll
Alternatives:   {\{Alex, Robin, Sam, Sacha, …\}}\\
{Simple focus}:  {\{\textbf{Alex}, Robin, Sam, Sacha, …\}}\\
\z

On the other hand, an additional operation can be performed on the set of alternatives. This can be a scalar ordering of the referents, for example for degree of expectedness. Another option is that there are alternatives for which the proposition is false – in other words, not only is an alternative selected, but other alternatives are excluded. This type of focus is therefore called exclusive focus. This means that ``there is at least one individual other than [the intended referent] for whom the proposition does not hold" \citep[147]{Kenesei2006}. For example, in the sentence ‘it is Alex that I met’, the focus does not stop at identifying the referent to which the proposition holds (Alex); it also excludes (at least some of) the possible alternatives (Robin, Sam, Sacha), implying that the proposition is false for them, as indicated by the strikethroughs in \xref{bkm:Ref1}. 

\ea
\label{bkm:Ref1}
\gll
{Alternatives:}   {\{Alex, Robin, Sam, Sacha, …\}}\\
{Exclusive focus:}  {\{\textbf{Alex}, \sout{Robin}, \sout{Sam}, Sacha, …\}}\\
\z

If the interpretation of the structure is that it excludes \textit{all} the possible alternatives, it becomes \textit{exhaustive} focus (which can be seen as a subtype of exclusive focus); see \citet{Szabolcsi1981} for early discussion of exhaustivity; \citet{É.Kiss1998} and many after her for discussion of the exhaustivity of Hungarian preverbal focus \citep[e.g.][]{OneaBeaver2009}; and see \citet{Horn1981,DrenhausEtAl2011}, among many others (e.g. the work of Destruel, as in \citealt{DestruelEtAl2015,DestruelDeVeaugh-Geiss2018}) for exhaustivity in clefts. 

\ea
\gll
{Alternatives:}   {\{Alex, Robin, Sam, Sacha, …\}}\\
{Exhaustive focus:}  {\{\textbf{Alex}, \sout{Robin}, \sout{Sam}, \sout{Sacha}, \sout{…}\}}\\
\z

Another aspect of focus is that a clause may contain a presupposition that there exists an entity for which the proposition is true, and this entity is then identified by the focus constituent. For instance, in the sentence ‘the person I met is Alex’, the phrase \textit{the person I met} presupposes that there is a person whom the speaker met. This person is then identified as Alex through the focused NP ‘Alex’. The role of focus in this structure can thus be said to be identificational. Note here that this is a different use of ``identificational" than used by \citet{É.Kiss1998}, who uses this term for exhaustive focus.

On a pragmatic level, many more types of focus can be described, all dependent on the discourse context. We mention three types here (see \citealt[]{Cruschina2021,Dik1997,Krifka2008} among others for further types, e.g. confirmative, mirative, expanding). A first proposed type is \textit{replacing focus}, also known as \textit{corrective} \textit{focus}. When the speaker suspects or knows that the addressee has a different referent in mind than the one intended by the speaker, they may try to correct this and replace the wrong referent with the intended one, thereby correcting the information.

A second ``flavour" of focus is \textit{contrastive focus}. This is possibly the most confusing term, since it has been used in two ways in the literature: firstly to express a contrast with a referent that is present in the context, and secondly to express contrast with referents in the set of alternatives. To some extent these overlap, of course, since referents in the context are also alternatives to the focused referent. However, the two definitions of contrastive focus are not identical. As the contrast with the set of alternatives can be captured as the exclusion of those alternatives, we refer to that as exclusive focus (see also \citealt{ByramWashburn2013}), and take contrastive focus to involve a pragmatic contrast with a contextual referent. What others have named ``contrastive focus" can therefore be seen as \textit{exclusive focus} in a \textit{contrastive context}. For example in ‘The children ate only \textit{ugali} and the parents ate \textit{rice}’, the object ugali excludes alternative foods, and it contrasts in this context with rice. Note that, similarly, corrective/replacing focus is \textit{exclusive} \textit{focus} in a \textit{corrective} \textit{context}, as the referent that was replaced/corrected is excluded, e.g. the rice in ‘The children ate ugali, not rice’. See \citet{Repp2010,Molnár2002} among others for further discussion on the notion of contrast, and \citet{Cruschina2021} for a more detailed discussion on degrees of contrast. In our book, we use the term contrastive only in a pragmatic sense.

A third pragmatic type of focus is \textit{selective focus}. This is found in a context where a set of alternatives is mentioned and one (or more) referents are selected from that set. Alternative questions such as ‘Do you want coffee or tea?’ establish a set of two referents (namely, coffee and tea) from which the addressee is to choose one. This choice, however, does not necessarily exclude the other referent and is therefore the same semantic type of focus as simple focus – the perceived exclusion of the non-selected alternative is thus in the pragmatics. 

To sum up, focus relates to the generation of alternatives and, together with the background, forms the comment of a categorical sentence. Focus can be categorised in terms of the scope of the focus, with a basic split between term focus and predicate-centred focus, and in terms of the semantic and pragmatic type of focus. Semantically, all focus types trigger a set of alternatives. These alternatives may simply be there with no further operation upon them (\textit{simple focus}), or they may be ordered (\textit{scalar}) or (partly) excluded (\textit{exclusive/exhaustive focus}). The pragmatic context may induce further variation in the interpretation, such as corrective focus, contrastive focus, and selective focus. Languages may express these different types of focus by different means.

\section{Common constructions for information structure expression in Bantu}
\label{common}

While the general aim of this chapter is to introduce the conceptual basics of information structure, it is also useful to introduce some constructions that occur in the Bantu languages covered in this book. We therefore turn now to zoom in on common constructions used for information structure expression in the Bantu languages. In particular, we discuss cleft constructions, word order variation, and predicate doubling here. Clefts in particular are present in each of the BaSIS languages and always play a role in the expression of focus (see \citealt{Gundel1988} and \citealt{Creisels2020} for clefts as a universally available construction). There are three types of cleft constructions: \textit{basic clefts}, \textit{pseudoclefts}, and \textit{reverse pseudoclefts}.

\subsection{Cleft constructions}

Basic cleft constructions\footnote{We use this as the more language-neutral term for what has been called for English the ``it-cleft" \citep{LafkiouiEtAl2016}; note that this is not to say that there is anything simpler about this cleft in comparison to other cleft types.} are bi-clausal structures that comprise a matrix clause consisting of a predicative nominal (typically headed by a copula) and a relative clause that is co-indexed with the predicative argument in the main clause (see \citealt[1949]{Jespersen1937}; \citealt[467]{Lambrecht2001}). In Tunen for example, a basic cleft is formed for subjects by the copula \textit{á} preceding the focused subject, followed by a reduced relative clause. This is illustrated in \xref{bkm:Ref151383611}, where the high tone in the subject marker indicates relativisation (contrasting with the low-toned non-relative class 1 subject marker \textit{a}).

%%EAX
\ea
Tunen (\chapref{ch:2}, \cite{chapters/tunen})\\
%%JUDGEMENT
%%LABEL
\label{bkm:Ref151383611}
%%CONTEXT
(Which politician died?)\\
%%LINE1
Á Píɛ́l á ná wə.\\
%%LINE2
\gll
á	Píɛ́lɛ	á	ná	wə́\\
%%LINE3
\COP{}  1.Pierre  1\SM.\REL{}  \PST{}2  die\\
%%TRANS1
\glt
‘It’s Pierre who died.’\\
%%TRANS2
%%EXEND

\glt COP   [NP]\textsubscript{FOC}   [reduced relative]
\z

The pseudocleft and the reverse pseudocleft \citep[see][]{Collins1991,Higgins2015} share with the basic cleft that they contain a relative clause, a copula, and a highlighted element \citep[482]{Collins1991} or clefted constituent (as \citealt{HedbergFadden2007} call it). The difference is in the order of these elements.

In pseudoclefts, the free relative clause presents the topic of the sentence while the post-copular nominal presents the focus \citep{HedbergFadden2007,É.Kiss1998}. The structure in \xref{bkm:Ref122688149} below illustrates these two parts. Note that the free relative is headed in English by a question word, but in other languages a free relative is simply marked in whichever way a relative can be marked – in the case of Kîîtharaka this has developed from the distal demonstrative, and in other languages in the BaSIS sample it also typically is or derives from a demonstrative.

%%EAX
\ea
%%JUDGEMENT
%%LABEL
\label{bkm:Ref122688149}
%%CONTEXT
Kîîtharaka (\chapref{ch:4}, \cite{chapters/kiitharaka})\\
%%LINE1
(What did Karîmi kick?)\\
%%LINE2
\gllll {}[ Kî-rá  Karîmi  á-rííng-iré ] í  [ mû-bíírá ]. \\
%%LINE3
{} 7-\RM{}  Karimi  1\SM{}-kick-\PFV{} {}  \COP{} {}  3-ball {}\\
%%EXTRA
{}[ {} free relative  ]  \COP{}  [ NP ]\\
{}[ {} cleft clause  ]  \COP{}  [ {clefted constituent} ]\\
% This is aligned after the translation in the original document, so I tried to align it here.
%%TRANS1
\glt
‘What Karîmi kicked is a ball.’\\
%%TRANS2
% {}[ free relative ]  \COP{}  [ NP ]\\
% {}[ cleft clause  ]  \COP{}  [ clefted constituent ]\\
%%EXEND

\z

In a pseudocleft, the free relative appears at the initial position. In a reverse pseudocleft, the order is inverted so that the NP (the clefted constituent) appears at the initial position and the relative follows the copula. The relative clause in the (reverse) pseudocleft is in fact a free relative, which functions as an independent phrase \citep[see e.g.][]{Šimík2018}. Therefore, (reverse) pseudoclefts can simply be seen as copular clauses of which one constituent is a free relative clause \citep{DenDikken2006}. An example of a reverse pseudocleft is given in \xref{bkm:Ref134538120} below.

%%EAX
\ea
%%JUDGEMENT
%%LABEL
\label{bkm:Ref134538120}
%%CONTEXT
Kinyakyusa (\chapref{ch:7}, \cite{chapters/kinyakyusa})\\
(I did not find him/her in the river.)\\
%%LINE1
{}[Ubibi jumo] jo [uju ambele].\\
%%LINE2
\gll
u-bibi  ju-mo  jo  uju  a-m-pa-ile\\
%%LINE3
\AUG{}-1.grandmother  1-one  1.\IDCOP{}  1.\DEM.\PROX{}  1\SM-1\SG.\OM{}-give-\PFV{}\\
%%TRANS1
\glt
‘An old woman is the one who gave (him/her) to me.’\\
%%TRANS2
%%EXEND

\gll {[NP] {} {}}               \COP{}   {[free relative]}\\
{[clefted constituent] {}}   \COP{}    {[cleft clause]}\\
\z

In a grammaticalisation process that has been shown to happen in languages all across the world (\citealt{HarrisCampbell1995,HeineTraugott1991}), cleft constructions – particularly the basic cleft – develop from a biclausal construction into a monoclausal focus construction, as schematically represented in \xref{bkm:Ref122688512} (from \citealt[9]{vanderWalManiacky2015}).

\ea
\label{bkm:Ref122688512}
%% JAMBOX OPTION
\settowidth{\jamwidth}{ > MAUD made pancakes}
{[copula \nobreakdash- NP] -- [relative clause]}  \jambox{> [NP\textsubscript{FOC} Verb]}
  {it is Maud -- (the one) who made pancakes}  \jambox{> {\itshape Maud} made pancakes}
%% GLL OPTION

%% TABULARX OPTION
% \begin{tabularx}{\linewidth}{Xl}
% [copula \nobreakdash- NP] -- [relative clause] & > [NP\textsubscript{FOC} Verb]\\
% it is Maud -- (the person/one) who made pancakes & > MAUD made pancakes\\
% \end{tabularx}
\z

\citet{HarrisCampbell1995} define three stages in the development from biclausal to monoclausal structures, as shown in \xref{bkm:Ref122688490}. The most obvious characteristics relevant for the languages in this book are the presence and form of the copula and the marking of relativisation, which are present in stage 1, but may disappear towards stage 3. 

\ea
\label{bkm:Ref122688490}Development cleft > focus construction \citep[166]{HarrisCampbell1995}

\begin{xlist}[Stage 3:]
\exi{Stage 1:} The structure has all of the superficial characteristics of a biclausal structure and none of the characteristics of a monoclausal one.

\exi{Stage 2:} The structure gradually acquires some characteristics of a monoclausal structure and retains some characteristics of a biclausal one.

\exi{Stage 3:} The structure has all of the characteristics of a monoclausal structure and no characteristics of a biclausal one.
\end{xlist}
\z

In various chapters, we will see properties indicating an in-between stage between biclausal clefts and monoclausal focus constructions, where the status of the erstwhile copula in a biclausal construction may later be analysed as a focus marker in a monoclausal construction (see discussion for Kîîtharaka and Kirundi, Chapters~\ref{ch:4} and \ref{ch:5}), or where the position has grammaticalised (as seen in Teke-Kukuya, \chapref{ch:3}).

On the interpretational side, the biclausal cleft typically expresses exhaustive focus, as a result of the combination of the relative clause and the predicative noun. The relative clause comes with a maximality presupposition: it refers to the complete set of referents for which the proposition holds. This maximal set of referents is then equated to the referent of the noun phrase. For example, in \xref{bkm:Ref125387736}, we know that there is a unique individual who threw the oranges, and this individual is identified as Hare.

%%EAX
\ea
%%JUDGEMENT
%%LABEL
\label{bkm:Ref125387736}
%%CONTEXT
Makhuwa-Enahara \citep[172]{vanderWal2009}\\
(Who has thrown oranges?)\\
%%LINE1
%%LINE2
\gll
Namarokoló  o-tthik-alé.\\
%%LINE3
1.hare.\PRL{}  1-throw-\PFV.\REL{}\\
%%TRANS1
\glt
‘It was Hare who threw (them).’\\
%%TRANS2
%%EXEND


\z

In a monoclausal focus construction, the exhaustive focus interpretation that used to be the result of the whole biclausal construction is now associated with the focus marker (the erstwhile copula, for example) and/or a particular position. To illustrate, for Kîîtharaka, where the basic cleft has an exhaustive interpretation, \citet{AbelsMuriungi2008} argue that the morpheme \textit{ni} should no longer be analysed as a copula, but as a focus marker, shown in \xref{bkm:Ref134538593}.\largerpage[-1]\pagebreak

%%EAX
\ea
%%JUDGEMENT
%%LABEL
%%CONTEXT
Kîîtharaka (\chapref{ch:4}, \cite{chapters/kiitharaka})\\
%%LINE1
\label{bkm:Ref134538593}\'{I} mûgaáté Áshá agûkáánda.\\
%%LINE2
\gll
ni  mû-gaate  Asha  a-kû-kaand-a\\
%%LINE3
\FOC{}  3-bread  1.Asha  1\SM-\PRS-{}bake-\FV{}\\
%%TRANS1
\glt
‘It’s bread that Asha is baking.’\\
%%TRANS2
%%EXEND


\z

While the aim of the chapters is not to analyse the underlying syntax in detail, but rather to provide a descriptive overview of the form-function mappings in each language, the diachronic reanalysis process is nevertheless useful to keep in mind. It is an interesting point of comparison between the Bantu languages, which vary in the extent of grammaticalisation into monoclausal focus constructions.

\subsection{Word order}

A second syntactic strategy used to express information structure is the manipulation of word order. Note here that we use the term ``word order" in the chapters in this book, although in many instances we more specifically mean ``constituent order": we are typically dealing with the order in which constituent \textit{phrases} rather than words per se are expressed.

In almost all of the BaSIS languages (see \figref{fig:langmap} above), word order plays a role in the expression of information structure, as we have shown in more detail in \citet{KerrEtAl2023}. Many eastern and southern Bantu languages show (some variant of) a topic-V-focus order \citep[see e.g.][]{vanderWal2009,Yoneda2011}, resulting in a restriction on preverbal focus, using so-called ``subject inversion constructions" to express theticity and/or subject focus, as illustrated in \xref{bkm:Ref131666802}. The logical subject here appears in the ``inverted" position after the verb because it is not the topic; the topic instead is the locative ‘on the bridge’ (as indicated in the context question), which therefore appears preverbally and determines subject agreement on the verb.

%%EAX
\ea
%%JUDGEMENT
%%LABEL
\label{bkm:Ref131666802}
%%CONTEXT
Rukiga \citep[2]{KerrEtAl2023}\\
%%LINE1
(What has happened on the bridge?)\\
%%LINE2
\gll
A-ha  ru-tindo        ha-a-raba=h\'{o}                e-motoka   ny-îngi.\\
%%LINE3
\AUG{}-16  11-bridge     16\SM-\N.\PST-{}pass=16  \AUG{}-10.car   10-many\\
%%TRANS1
\glt
‘On the bridge have passed many cars.’\\
%%TRANS2
%%EXEND
\z\pagebreak

If we think in terms of grammatical roles, these Bantu languages can be said to display SVO order with a (high) degree of flexibility (\citealp{Bearth2003,vanderWal2015,DowningMarten2019}, a.o.); for example, \xref{bkm:Ref131666802} shows a change to VS word order (i.e. locative inversion). However, if we describe word order in information-structural terms, as \citet{Good2010} does, word order is not that flexible at all for some of the Bantu languages. The word order can in some languages be more insightfully described in terms of the following restrictions (i) non-topical/focal referents must be expressed after the verb, and (ii) topics are expressed preverbally, regardless of their syntactic or semantic role. In \citet{KerrEtAl2023} we also show that three of the BaSIS languages have a dedicated focus position, defined as a position in which an element is always interpreted as focal: immediately before the verb (IBV) in Teke-Kukuya, immediately after the verb (IAV) in Makhuwa-Enahara, and clause-final in Kirundi (see example \xref{ex:kirundi:1} above).

Nevertheless, there is a lot of variation in the use of word order to express information structure in Bantu, as can be seen in comparing the chapters of this volume (and see the discussion from a comparative perspective in \citet{KerrEtAl2023}). One particular outlier is Tunen in the far northwest of the Bantu area, where subject inversion is impossible, and the same S-Aux-O-V word order is used to express a variety of information-structural interpretations. Tunen subjects, however, cannot be focused in their canonical position, and must be clefted, as we saw in \xref{bkm:Ref151383611} above.

\subsection{Predicate doubling}
\label{bkm:Ref147299258}
A third type of construction that we found in the majority of languages is predicate doubling. These are constructions in which the predicate appears once in an inflected form and once in a non-finite form. There are three types of predicate doubling (see also \citealt{GüldemannFiedler2022}, who use slightly different terminology): \textit{cleft doubling}, \textit{topic doubling}, and \textit{in-situ doubling}. In cleft doubling, the non-finite form (typically an infinitive in class 15) forms the clefted constituent, as in \xref{bkm:Ref151383677}, which is followed by the inflected (relative) form of the same predicate. This type of predicate doubling is typically used to express state-of-affairs focus.

%%EAX
\ea
%%JUDGEMENT
%%LABEL
\label{bkm:Ref151383677}
%%CONTEXT
Kîîtharaka (database Kanampiu \& Van der Wal)\\
(Context: Someone thinks that you sang or cooked.)\\
%%LINE1
I kûr\'{î} má tûrîmiré.\\
%%LINE2
\gll
ni  kû-rîma  tû-rîm-ire\\
%%LINE3
\FOC{}  15-dig  1\PL.\SM{}-dig-\PFV{}\\
%%TRANS1
\glt ‘We (only) dug.’
%%TRANS2
%%EXEND

\z

Topic doubling involves an initial non-finite predicate, again typically an infinitive, expressed as a contrastive topic, followed by the comment containing the same predicate. This can be used in verum contexts, as illustrated in \xref{ex:cicopi-do-cook}, and in contexts where a contrast is perceived between the mentioned action and possible other, additional actions. This in turn may invite intensive and depreciative interpretations, as described in this book for Kîîtharaka (\chapref{ch:4}), Kirundi (\chapref{ch:5}), Rukiga (\chapref{ch:6}), Kinyakyusa (\chapref{ch:7}), and Cicopi (\chapref{ch:9}). See also \citet{JerrovanderWalFut} for a semantic-pragmatic analysis of topic doubling.

%%EAX
\ea
%%JUDGEMENT
%%LABEL
\label{ex:cicopi-do-cook}
%%CONTEXT
Cicopi (\chapref{ch:9}, \cite{chapters/cicopi})\\
(Context: A mother went out to work and when she returns, she can see that the children are weak. She asks the housemaid ‘Are you cooking for these children?’)\\
%%LINE1
Kubhí:ka hábhî:ka.\\
%%LINE2
\gll
ku-bhik-a  hi-a-bhik-a\\
%%LINE3
15-cook-\FV{}  1\PL.\SM-\DJ{}-cook-\FV{}\\
%%TRANS1
\glt
‘We do cook.’ (but they don’t eat)\\
%%TRANS2
%%EXEND


\z

Finally, in-situ doubling is more variable both in form and in meaning. The non-finite form here follows the inflected form of the same predicate, but this can be an infinitive or a nominalised form. The latter is illustrated for Kirundi with the class 14 derivation \textit{butunge} in \xref{bkm:Ref134546189}. The interpretation for this construction in Kirundi indicates a prototypical, proper way of carrying out the action in the doubled verb. In other languages, in-situ doubling can function to express state-of-affairs focus or an intensive reading, for example.

%%EAX
\ea
%%JUDGEMENT
%%LABEL
\label{bkm:Ref134546189}
%%CONTEXT
Kirundi (database Nshemezimana \& Van der Wal)\\
(Does he keep the cows for someone else, or are they his?)\\
%%LINE1
Inka, arazitunze butunge.\\
%%LINE2
\gll
i-n-ka  a-zi-tung-ye  bu-tung-e\\
%%LINE3
\AUG{}-10-cow  1\SM{}-10\OM{}-rear-\PFV{}  14-rear-\FV{}\\
%%TRANS1
\glt
‘Cows, he keeps them properly.’\\
%%TRANS2
%%EXEND


\z

There is, as expected, a lot of crosslinguistic variation in which of the three predicate doubling constructions are available (if any), in the non-finite form, and in the interpretation of the construction.

Cleft constructions, word order, and predicate doubling are only three of the morphosyntactic strategies in which information structure is expressed in the Bantu languages. We provide more specific detail on these constructions in each chapter and hope that it will be helpful to have already seen their general structure in this section.

\subsection{Interim summary}

Summing up Sections~\ref{sec:3:is} and~\ref{common}, referents have a particular information status as more or less accessible, which influences their expression, and referents may further take one of two broadly-recognised information-structural functions: topic or focus. There are different types of topic and focus, which may require different linguistic strategies of expression. Common strategies used in the Bantu languages are cleft constructions, word order variation, and predicate doubling constructions, but there are many more. In the chapters in this book, we aim to determine which semantic and pragmatic interpretations the various linguistic strategies have in each language, working from form to function. As mentioned, we do this on the basis of a common methodology using diagnostics for information structure. The next sections present this BaSIS methodology (\sectref{bkm:Ref117846286}) and explain its key diagnostics used in this book (\sectref{sec:6}).

\section{Methodology}
\label{bkm:Ref117846286}
For each language, apart from Kirundi where COVID-19 restrictions prevented fieldwork, we gathered data with at least three native speakers of the language, in locations where the language of interest was dominant in the community. Nevertheless, there is always the chance of influence from another language due to the use of a language of common communication for the elicitation process, as well as widespread multilingualism. The metalanguage used in the fieldwork was French for Tunen, Teke-Kukuya, and Kirundi; English for Kîîtharaka and Rukiga; Portuguese for Cicopi and Makhuwa-Enahara (as well as Changana); and Swahili and English for Kinyakyusa. Sessions for elicitation, as well as spontaneous discourse and narratives were recorded with audio (.wav files), and for Tunen also video (.mp4 files). The speakers all gave informed consent for the recordings and data gathering and received financial compensation for their time. Further information about the fieldwork is given in each individual chapter. The databases for all languages and Tunen recordings will soon be available in The Language Archive of the MPI (see the BaSIS collection\footnote{\url{https://hdl.handle.net/1839/2acf92c5-e5db-445b-bcdb-3b13b7e58f3f}}).

The data gathering was conducted in a systematic way for each language, each researcher using the BaSIS methodology document that was developed over the course of the project. This methodology, further elaborated on in \sectref{bkm:Ref117846329} below, draws on the Questionnaire of Information Structure – QUIS \citep{SkopeteasEtAl2006}, the Questionnaire on Focus Semantics \citep{RenansEtAl2010}, and the additional tests gathered in \citet{vanderWal2016}. The methodology document, which is freely available via the project website\footnote{\url{https://bantusyntaxinformationstructure.com/methodology/}} and the Leiden Repository\footnote{\url{https://scholarlypublications.universiteitleiden.nl/handle/1887/3608096}} \citep{vanderWal2021}, grew during the project in three main ways. First, Part I was added, providing a basic introduction to understand the key concepts of information structure such that the methodology can be used by researchers new to information structure. Second, the rather artificial picture stimuli developed by the QUIS were gradually replaced by more natural pictures – compare \figref{fig:quisstimulus} and \figref{fig:basisstimulus}. This change was made as the unnaturalness of the initial stimuli sometimes affected the fieldwork, for example with speakers indicating for \figref{fig:quisstimulus} that ‘there are three melons above William’s head’.

  
\begin{figure}
\includegraphics[width=\textwidth]{figures/fig5-quisstimulus.png}
\caption{QUIS stimulus for corrective focus on numeral}
\label{fig:quisstimulus}
\end{figure}

  
\begin{figure}
\includegraphics[height=.5\textheight]{figures/fig6-basisstimulus-bw-cropped.png}
\caption{BaSIS picture, which can be used as stimulus for corrective focus on numeral (e.g. “Does the woman have four knives?”)}
\label{fig:basisstimulus}
\end{figure}

The third addition to the methodology was Part III, where tests were added for Vergnaud licensing (abstract Case), as proposed by \citet{SheehanvanderWal2018}. This allows for investigation of nominal licensing, another goal of the BaSIS project (although not the focus of the chapters in this book). 

The methodology thus consisted of three parts: 
\largerpage
Part I providing a background on information structure so as to be consistent in conceptualisation and terminology, Part II providing an overview of all the diagnostics to elicit and test different notions of information structure (see \sectref{bkm:Ref117846329}), and Part III providing tests for nominal licensing. The data collected include both (semi-)spontaneous speech and elicitation data. The (semi-)spontaneous speech data took the form of recounted recipes, folk tales, the Frog Story, personal histories, free dialogues, and the QUIS map task. The elicitation data were collected for the purposes of systematic crosslinguistic comparison (as in the comparative work on word order by \citealt{KerrEtAl2023}, and on truth expression by \citealt{KerrvanderWal2023}). Additionally, elicitation tests are useful in providing negative data, which show which constructions and interpretations are \textit{not} possible. Here, the BaSIS elicitation tests provide explicit information-structural context, allowing importantly for both the study of grammaticality (which constructions are well-formed in the language, with the symbol * in the data line indicating ungrammaticality) and felicity (which constructions are appropriate within the given discourse context, with the symbol \textsuperscript{\#} indicating infelicity).

The diagnostics in the methodology work in two directions: \textit{function to form}, and \textit{form to function}. An example of working from function to form is the commonly used question-answer test (also known as \textit{question-answer congruence}), which investigates the expression of focus. As content question words are taken to be inherently focused, the answer to such a question forms the focus as well. Question-answer pairs can therefore be used to investigate focus expression. This test is illustrated for Cicopi in \xref{bkm:Ref117865183}. Here, the translation of the question from Portuguese in \xref{bkm:Ref117865183:a} shows us a strategy that is used to express an interrogative, and the answer to that question in \xref{bkm:Ref117865183:b} also shows a focus strategy. We learn here that focus (function) on the theme argument can be expressed with the theme argument in a postverbal position (form) in Cicopi, both for interrogatives and for their answers.

\ea
\label{bkm:Ref117865183}Cicopi (database Nhantumbo \& van der Wal)
%%EAX
\ea
%%JUDGEMENT
%%LABEL
\label{bkm:Ref117865183:a}
%%CONTEXT
%%LINE1
Utóselá \textbf{cà:nì}?\\
%%LINE2
\gll
u-to-sel-a  \textbf{cani}\\
%%LINE3
2\SG.\SM-\TO{}-drink-\FV{}  what\\
%%TRANS1
\glt
‘[What]\textsubscript{\FOC} did you drink?’\\
%%TRANS2
%%EXEND


%%EAX
\ex
%%JUDGEMENT
%%LABEL
\label{bkm:Ref117865183:b}
%%CONTEXT
%%LINE1
Nitóselá \textbf{sérvhejhá}.\\
%%LINE2
\gll
ni-to-sel-a  \textbf{servhejha}\\
%%LINE3
1\SG.\SM-\TO{}-drink-\FV{}  beer\\
%%TRANS1
\glt
‘I drank [beer]\textsubscript{\FOC}.’\\
%%TRANS2
%%EXEND


\z
\z

Working the other way, from form to function, is useful when a strategy has been identified and the details of its interpretation are of interest. For example, during the fieldwork we knew that Kinyakyusa has a CV nominal prefix, and wanted to investigate the interpretation of this prefix. The form-to-function tests in the methodology allowed us to test for focus and exhaustivity. If the CV prefix were a marker of exhaustive focus, meaning that this is the only referent for which the proposition is true (‘he drank only \textit{the big one}’), then it should be impossible to follow up with a clause stating the truth for another referent too and cancelling the exhaustivity (‘also the small one’). This prediction turned out to be correct \xref{bkm:Ref117864379}, and so we can conclude that, according to this test, the CV prefix (form) here is exhaustively focused (function).
\largerpage[2]

%%EAX
\ea
%%JUDGEMENT
%%LABEL
%%CONTEXT
Kinyakyusa \citep[335]{vanderWalLusekelo2022}\\
%%LINE1
\label{bkm:Ref117864379}Anwile \textbf{ji}nywaamu \textsuperscript{\#}n’ iinaandi\\
%%LINE2
\gll
a-nw-ile  \textbf{ji}{}-nywamu  na  i-nandi\\
%%LINE3
1\SM{}-drink-\PFV{}  \textbf{\EXH{}}{}-9.big  and  \AUG{}-9.small\\
%%TRANS1
\glt
‘He drank (only) the big one \textsuperscript{\#}and also the small one.’\\
%%TRANS2
%%EXEND


\z

All the data collected through these tests, together with the relevant metadata, were entered into Online Language Databases (OLDs) for each language, which were accessed via the graphical user interface Dative\footnote{\url{https://www.dative.ca/}}. The major benefit of Dative is that multiple users can access the database simultaneously and add data. This is crucial in a collaborative project, and this aspect worked very well.\footnote{Other good aspects are the fact that Dative is quite intuitive to work with, and allows for elaborated combined search options and easy tagging of examples for properties of interest (e.g. “focus\_obj” for object focus). A major downside at the time of writing is that it is still an online database, thus requiring internet connection and power.} Access to any of the databases can be granted by contacting Jenneke van der Wal and/or consulting the archival deposit in The Language Archive. An example of the Dative database layout and data/metadata input is given in \figref{fig:dativegloss} and \figref{fig:dativeinput} respectively.

  
\begin{figure}
\includegraphics[width=\textwidth]{figures/fig7-dativegloss.png}
\caption{Screenshot of Dative for the Makhuwa-Enahara Online Language Database, showing transcription of each utterance with interlinear glossing and translation.}
\label{fig:dativegloss}
\end{figure}

\begin{figure}
\includegraphics[height=.5\textheight]{figures/fig8-dativeinput.png}
\caption{Screenshot of Dative for one form in the Makhuwa-Enahara OLD, showing part of the fields for data and metadata.}
\label{fig:dativeinput}
\end{figure}

Having introduced the basic structure of the methodology, let us now turn to consider in more detail how the methodology is applied for the investigation of information-structural concepts as introduced in \sectref{sec:3:is} above. These sections together are intended to form the background to understanding the discussion and diagnostics in each of the individual chapters.

\section{Information structure diagnostics}
\label{bkm:Ref117846329}\label{sec:6}
\largerpage[2]

In determining the form-function mappings in the area of information structure, there are three steps: 

\ea
How to diagnose information structure
\begin{enumerate}
    \item Find out how a function is expressed;
    \item  Test the precise function of each form;
    \item  Test whether the function found is inherent to the form, or associated pragmatically.
\end{enumerate}
\z

The first step, function to form, can for example be a question-answer pair, as illustrated in \xref{bkm:Ref117865183} above: we know that content question words are in focus, and that the part that answers the content question is also in focus, therefore the strategy that is used in the question and the answer is a strategy used to express focus. Note that this does not necessarily mean that it is a ``focus strategy", as at this point we do not know the exact interpretation, nor whether focus is a necessary aspect of the strategy (see \citealt{MatićWedgwood2013} for further nuance on the latter point, and \citealt{Ozerov2022} for a critical discussion of the role of questions in information structure, seeing them as the product of diverse discourse processes).

  Continuing the same example, the second step (from form to function) might involve testing whether the strategy that is used in question-answer pairs is used for simple focus, or also exhaustive focus. If it is found that the interpretation is one of exhaustive focus, then the third step is to check whether this aspect of exhaustivity is necessarily present, or can for example be cancelled (indicating that it would be in the pragmatics and not an inherent, semantic aspect of interpretation associated with this form).

  Regardless of which step or diagnostic is applied, it is essential that examples come with a discourse context. This is automatically the case for spontaneous examples and longer stretches of discourse or narrative, but should also be provided for elicited examples, so that the reader and researcher can identify how the form of the utterance relates to the information structure as visible through the discourse context. We therefore introduced a discourse context during the elicitation, either by verbal means or through use of picture stimuli, rather than working from isolated translation of the metalanguage (in which case it can be difficult to know exactly what discourse context the speaker has in mind).

In what follows, we explain and illustrate some of the diagnostics for topic (in \sectref{topics}), focus (in \sectref{bkm:Ref131673501}), and thetics (in \sectref{bkm:Ref134712116}). The full set of diagnostics and explanations can be found in the methodology document \citep{vanderWal2021}; see also the QUIS \citep{SkopeteasEtAl2006}, as well as \citet{RenansEtAl2010,vanderWal2016} and \citet{Aissen2023}.

\subsection{Topic diagnostics}
\label{topics}

While there is a crosslinguistic tendency for topics to be expressed sentence-initially (\citealt{Gundel1988}, \citealt{Aissen2023}, among others), the identification of topics has received less attention than the identification of focus. 

\citet{Reinhart1981} makes various suggestions for topic tests, among which is the periphrasis test: If a sentence can be paraphrased by ‘as for X’ or ‘I say about X that …’, the referent that X refers to is indeed the topic. As a second test, \citet{Ameka1991} shows that topic-marking strategies are incompatible with focus and should therefore not co-occur with inherently focal question words or (exhaustive) focus markers. 

A third practical test that has turned out to be useful for identifying topics is testing for indefinites (\citealt[66]{Reinhart1981}, referring to \citealt{Firbas1966}). Because the topic is the point to which the information in the comment is anchored, this point must be at least identifiable, and therefore cannot be an indefinite non-specific referent. In other words, if an element is identified as a non-specific indefinite, it cannot also be a topic. Consider as an illustration examples \xref{bkm:Ref134712964} and \xref{bkm:Ref134712974}. In Teke-Kukuya, topical elements tend to occur in the preverbal domain, and the preverbal NP ‘person’ in \xref{bkm:Ref134712964:a} must be interpreted as definite, while the indefinite reading for the context is impossible. In \xref{bkm:Ref134712964:b}, when the word ‘person’ occurs postverbally, either a definite or indefinite reading is possible.\footnote{Note that this is not the immediate before verb focus position, as can be seen in the position of negation (see \chapref{ch:3}, \cite{chapters/teke}, and \citealt{LiFut}).}

\ea
\label{bkm:Ref134712964}
(In)definite test: Teke-Kukuya (\chapref{ch:3}, \cite{chapters/teke})\\
(Context: Your mum and you are entering a dark hall, you are walking in front and your mum asks you from behind if you saw anyone in the hall.)
%%EAX
\ea
%%JUDGEMENT
[\textsuperscript{\#}]{
%%LABEL
\label{bkm:Ref134712964:a}
%%CONTEXT
%%LINE1
%%LINE2
\gll
Me  mbuurú  ka-á-mún-i  ni.\\
%%LINE3
1\SG.\PRO{}  1.person  \NEG-1\SG.\PST{}-see-\PST{}  \NEG{}\\
%%TRANS1
\glt
‘I did not see the person/*anyone.’\\
%%TRANS2
}
%%EXEND

%%EAX
\ex
%%JUDGEMENT
[]{
%%LABEL
\label{bkm:Ref134712964:b}
%%CONTEXT
%%LINE1
%%LINE2
\gll
Me  ka-á-mún-i  mbuurú  ni.\\
%%LINE3
1\SG.\PRO{}  \NEG-1\SG.\PST{}-see-\PST{}  1.person  \NEG{}\\
%%TRANS1
\glt
‘I did not see the person/anyone.’\\
%%TRANS2
}
%%EXEND
\z
\z

In contrast, the preverbal subject ‘person’ in Kinyakyusa \xref{bkm:Ref134712974} can be interpreted as indefinite and non-specific, thereby showing that preverbal constituents do not necessarily refer to topics.

%%EAX
\ea
%%JUDGEMENT
%%LABEL
%%CONTEXT
(In)definite test: Kinyakyusa (\chapref{ch:7}, \cite{chapters/kinyakyusa})\\
%%LINE1
\label{bkm:Ref134712974}Linga siku \textbf{umundu} linga ikukubuula gwinogonengepo.\\
%%LINE2
\gll
linga  siku  u-mu-ndu  linga  i-ku-ku-buul-a  gw-inogon-ang-e=po\\
%%LINE3
\COND{}  9.say  \AUG{}-1-person  \COND{}  1\SM-\PRS-2\SG.\OM{}-tell-\FV{}  2\SG.\SM{}-think-?-\SBJV{}=16\\
%%TRANS1
\glt
‘If during another day someone tells you something, you must think.’\\
%%TRANS2
%%EXEND


\z

Weakly quantified NPs such as ‘few’ in \xref{bkm:Ref135043983}, equally resist topicalisation, as they too behave as indefinites. The example in (\ref{bkm:Ref135043983}) shows that Makhuwa also has a subject position preverbally which can be filled by non-topics.

%%EAX
\ea
%%JUDGEMENT
%%LABEL
\label{bkm:Ref135043983}
%%CONTEXT
Makhuwa-Enahara \citep[176]{vanderWal2009}\\
%%LINE1
%%LINE2
\gll
Epaáwú  vakhaání  yoo-khúúr-íy-a.\\
%%LINE3
9.bread  few  9\SM{}.\PFV{}.\DJ{}-chew-\PASS-\FV{}\\
%%TRANS1
\glt
‘Little bread was eaten.’\\
%%TRANS2
%%EXEND


\z

A final way to test topics is to distinguish subjects from topics through considering not only categorical sentences (with topic-comment) split but also thetic sentences (which do not have a topic). Investigating thetics can thus help to show whether preverbal noun phrases are best described in terms of topicality or subjecthood. For example, the availability of initial NPs in Tunen thetics shows that the word order is best described as Subject-Aux-O-V, rather than Top-Aux-O-V (\chapref{ch:2}, \cite{chapters/tunen}). 

\subsection{Focus diagnostics}
\label{bkm:Ref131673501}
\largerpage
For the precise interpretation of focus, we first want to test whether a construction is about focus at all: if it does express focus, then we would expect it to allow content question words, answers to those questions, and constituents that can associate with the focus-sensitive particle ‘only’. Furthermore, we would expect ``unfocusable" constituents to not be tolerated in this strategy. Such ``unfocusables" are elements for which no alternative set can be generated: in our methodology, we use parts of idioms and cognate objects as tests for such unfocusable elements. The logic here is that there are no alternatives for parts of idioms in their idiomatic interpretation, as this interpretation is dependent on other parts of the idiom. For example, while the English idiom ‘she lost her marbles’ can be interpreted as ‘she has gone crazy’, focusing the object in a cleft ‘it is her marbles that she lost’ only leaves the literal interpretation of someone losing actual glass marbles (as opposed to alternatives such as a handkerchief or pencils), no longer allowing the idiomatic interpretation (for more detail, see \citealt{vanderWal2021}). The same logic holds for cognate objects, such as ‘dance a dance’ or ‘dream a dream’: there are no valid alternatives that could be triggered (what else would you dream if not a dream?) and therefore the cognate object cannot be in focus (\textsuperscript{\#}`It is a dream she dreamed’). To illustrate this, the conjoint verb form in Kirundi cannot be used with a cognate object, as shown in \xref{bkm:Ref118879810:b}, compared with the perfectly acceptable disjoint verb form in \xref{bkm:Ref118879810:a}. This incompatibility shows the exclusive reading of the element following the conjoint verb form. Adding a relative clause to the object, in this case modifying ‘dreams’ by the relative clause ‘that are not pleasant’, makes it possible again to generate alternatives, such as dreams that are pleasant. This enables the generation of alternatives, and therefore allows the use of the conjoint verb form.

\ea
\label{bkm:Ref118879810}Kirundi (\chapref{ch:5}, \cite{chapters/kirundi})
%%EAX
\begin{xlist}
\exi{\DJ}
%%JUDGEMENT
%%LABEL
\label{bkm:Ref118879810:a}
%%CONTEXT
%%LINE1
Naaróose indóoto.\\
%%LINE2
\gll
\N{}-a-a-róot-ye  i-n-róoto\\
%%LINE3
1\SG.\SM-\N.\PST-\DJ{}-dream-\PFV{}  \AUG{}-9-dream\\
%%TRANS1
\glt
‘I dreamt a dream.’\\
%%TRANS2
%%EXEND

%%EAX
\exi{\CJ}
%%JUDGEMENT
%%LABEL
\label{bkm:Ref118879810:b}
%%CONTEXT
%%LINE1
Naroose indoto \textsuperscript{\#}(zitari nziiza).\\
%%LINE2
\gll
\N{}-a-róot-ye  i-n-róoto  zi-ta-ri  n-ziiza\\
%%LINE3
1\SG.\SM-\N.\PST.\CJ{}-dream-\PFV{}  \AUG{}-10-dream  10\SM-\NEG{}-be.\REL{}  10-good\\
%%TRANS1
\glt
‘I dreamt dreams \textsuperscript{\#}(that were not pleasant).’\\
%%TRANS2
%%EXEND

\end{xlist}
\z
\largerpage

In a second step into the investigation of focus interpretation, we can test for exclusivity and exhaustivity. If a strategy is used for exclusive focus, this means that some or all of the alternatives should be excluded. We can test this by using items that do not allow exclusion and checking their acceptability in a focus strategy. There are at least five ways to test exclusivity: 

\ea
Tests for exclusive and exhaustive focus
\begin{enumerate}
    \item alternative questions;
\item  universal quantifiers ‘all’ and ‘every’;
\item  the focus-sensitive scalar particle ‘even’ and additive ‘also’;
\item   indefinite noun-phrases;
\item  a ‘mention some’ context.
\end{enumerate}
\z

We discuss these briefly here. First, when responding to an alternative question, i.e. a question that mentions two or more alternatives, an answer typically not only indicates for which referent the statement is true (e.g. Emily in \xref{bkm:Ref131672872}), but also that alternatives are excluded (in this example, Hamida), thus differing from the answer to a simple content question in which only selection of an alternative is involved (simple focus).

%%EAX
\ea
%%JUDGEMENT
%%LABEL
%%CONTEXT
Rukiga (database Asiimwe \& Van der Wal)\\
%%LINE1
\label{bkm:Ref131672872}Kéék’ ogihííre \'{E}míri nínga Hamída?\\
%%LINE2
\gll
keeki  o-gi-h-ire  Emily  nainga  Hamida\\
%%LINE3
9.cake  2\SG.\SM-9\OM{}-give-\PFV{}  1.Emily  or  1.Hamida\\
%%TRANS1
\glt
‘Did you give Emily or Hamida a cake?’\\
%%TRANS2
%%EXEND

%%EAX
\sn
%%JUDGEMENT
%%LABEL
%%CONTEXT
%%LINE1
Naagihííre \'{E}mirí atárí Hamída.\\
%%LINE2
\gll
n-gi-h-ire  Emily  a-ta-ri  Hamida\\
%%LINE3
1\SG.\SM-9\OM{}-give-\PFV{}  1.Emily  1\SM-\NEG{}-be  1.Hamida\\
%%TRANS1
\glt
‘I gave it to Emily not Hamida.’\\
%%TRANS2
%%EXEND


\z

The second test for exclusivity uses universal quantifiers ‘all’ and ‘every’. These quantifiers are incompatible with exclusive focus, because all referents are included and therefore there can be no exclusion of alternatives from the set. A similar reasoning holds for the additive particle ‘also’, which indicates that more instantiations of the action/state described in the predicate have occurred for different referents, therefore making the referent it modifies non-exhaustive. Similarly, the scalar additive particle ‘even’ presupposes that more instantiations of the action/state described in the predicate have occurred, and in addition expresses that the object modified by ‘even’ is the least likely in the set of contextually relevant alternatives to make the proposition true. Therefore, none of the alternatives are excluded, and a DP modified by ‘even’ is predicted to be infelicitous if a focus strategy is inherently exclusive. The unacceptability of the universal quantifier and the particle ‘even’ in a pseudocleft construction are shown in the examples in \xref{ex:kinyakyusa-kato} and \xref{ex:kinyakyusa-frida}, respectively. This shows that the pseudocleft in Kinyakyusa has an exclusive interpretation.

%%EAX
\ea
%%JUDGEMENT
%%LABEL
\label{ex:kinyakyusa-kato}
%%CONTEXT
Kinyakyusa (\chapref{ch:7}, \cite{chapters/kinyakyusa})\\
%%LINE1
Ifi aagogile Kato fitana (*fyoosa).\footnote{Predication is expressed here by the absence of the augment -- no segmental copula is needed.}\\
%%LINE2
\gll
ifi  a-a-gog-ile  Kato  fi-tana  fi-osa\\
%%LINE3
8.\DEM.\PROX{}  1\SM-\PST{}-kill-\PFV{}  1.Kato  8-cup  8-all\\
%%TRANS1
\glt
‘What Kato broke is (*all) cups.’\\
%%TRANS2
%%EXEND


\z

%%EAX
\ea
%%JUDGEMENT
%%LABEL
\label{ex:kinyakyusa-frida}
%%CONTEXT
%%LINE1
Uju nalyaganiile nagwe jo (*joope) n’uFrida.\\
%%LINE2
\gll
uju  n-ali-aganil-ile  na-gwe  jo  j-oope  na  u-Frida\\
%%LINE3
1.\DEM.\PROX{}  1\SG.\SM-\PST{}-meet-\PFV{}  with-1.\PRO{}  1.\IDCOP{}  1-even  and  \AUG{}-1.Frida\\
%%TRANS1
\glt
‘The one I met is (*even) Frida.’\\
%%TRANS2
%%EXEND


\z

However, the incompatibility can be remedied by specifying a set of alternatives for the universally quantified DP, either by contrasting with another whole set, e.g. ‘he broke all the cups, not all the plates’, or by adding a restrictive relative clause, e.g. ‘he broke all the cups \textit{that are in the cupboard}, not the ones on the table’.

Indefinite non-specific phrases such as ‘someone’, ‘anything’, or ‘nobody’ cannot exclude alternatives: any (or no) referent will satisfy. To illustrate, when saying ‘I need to eat something’, the natural non-specific interpretation is that the speaker is hungry and does not care what they eat: anything will do out of the set of edible alternatives. All alternatives are therefore included. If a strategy expresses exclusive focus, it is thus predicted to be incompatible with the inclusive nature of the non-specific indefinite. This prediction is borne out for the conjoint and disjoint verb forms in Makhuwa-Enahara: the non-specific interpretation is possible after the disjoint verb form as in \xref{bkm:Ref122698002:a}, but unacceptable after the conjoint verb form as in \xref{bkm:Ref122698002:b}. The conjoint verb form is therefore said to express exclusive focus on the element directly following the verb \citep{vanderWal2011}. In order to fit the noun \textit{ntthu} ‘person’ into the exclusive interpretation, it interpreted as a type (a human being), as in \xref{bkm:Ref122698002:c}: other types can now be excluded, and the conjoint form is acceptable.

\ea
Makhuwa-Enahara \citep[1740]{vanderWal2011}

\label{bkm:Ref122698002}
%%EAX
\ea
\begin{xlist}
\exi{\DJ}
%%JUDGEMENT
[]{
%%LABEL
\label{bkm:Ref122698002:a}
%%CONTEXT
%%LINE1
Koḿwéha ńtthu.\\
%%LINE2
\gll
ki-o-n-weh-a  n-tthu\\
%%LINE3
1\SG{}.\SM{}-\PFV{}.\DJ{}-1\OM{}-look-\FV{}  1-person\\
%%TRANS1
\glt
‘I saw someone.’\\
%%TRANS2
}
\end{xlist}
%%EXEND


%%EAX
\ex
\begin{xlist}
\exi{\CJ}
%%JUDGEMENT
[\textsuperscript{\#}]{
%%LABEL
\label{bkm:Ref122698002:b}
%%CONTEXT
%%LINE1
%%LINE2
\gll
Ki-m-weh-alé  n-tthú.\\
%%LINE3
1\SG.\SM{}-1\OM{}-look-\PFV{}.\CJ{}  1-person\\
%%TRANS1
\glt
int.: ‘I saw someone.’\\
%%TRANS2
}
\end{xlist}
%%EXEND


%%EAX
\ex
\begin{xlist}
\exi{\CJ}
%%JUDGEMENT
[]{
%%LABEL
\label{bkm:Ref122698002:c}
%%CONTEXT
%%LINE1
%%LINE2
\gll
Ki-m-weh-alé  n-tthú,  nki-weh-álé  enáma.\\
%%LINE3
1\SG{}.\SM-1\OM{}-look-\PFV{}.\CJ{}  1-person  \NEG{}.1\SG.\SM{}-look-\PFV{}  9.animal\\
%%TRANS1
\glt
‘I saw a person/human being, not an animal.’\\
%%TRANS2
}
\end{xlist}
%%EXEND

\z
\z


A fourth test for exclusivity is the “mention some” context. \Citet[270]{vanderWal2016} writes that “Instances of non-exhaustive focus are found in answers to so-called mention-some questions, where the context of the question does not require, or even allow for an explicit listing of all the true alternatives" (see also \citealt{AbelsMuriungi2008}).

For example, if you can usually buy milk or tomatoes in various places, there is no single correct answer to a question ‘Where can I buy milk?’. An exhaustive focus strategy is thus predicted to be infelicitous in the answer to a mention-some question. The test can therefore be used to discover which strategy is used in non-exhaustive focus (function to form), but can also be used to test suspected exhaustive focus strategies in non-exhaustive contexts (form to function).

There are a number of other tests in our methodology to test exclusivity and exhaustivity, for example the exact reading of numerals under exhaustive focus and the answers to incomplete or overcomplete yes/no questions. We illustrate here with Kinyakyusa numerals for the CV prefix and refer to the methodology document\footnote{\url{https://scholarlypublications.universiteitleiden.nl/handle/1887/3608096}} for further tests and explanations. The meaning of numerals has been taken to have an underspecified interpretation either as the exact amount, or as a lower boundary ‘at least this amount’ \citep{Horn1972,Levinson2000}. However, in exhaustive focus, numerals lose their upward entailing quality and refer only to the exact quantity, because other amounts are excluded. In Kinyakyusa, a numeral in a DP with a CV marker is interpreted as the exact amount, as illustrated by the infelicity of the follow-up ‘maybe more’ in \xref{bkm:Ref131663261}. The CV marker is concluded (also on the basis of other tests) to have an exhaustive interpretation (\citealt{vanderWalLusekelo2022} and \chapref{ch:7}).

%%EAX
\ea
%%JUDGEMENT
%%LABEL
\label{bkm:Ref131663261}
%%CONTEXT
Kinyakyusa \citep{vanderWalLusekelo2022}\\
%%LINE1
\textbf{Si}nguku ntandatu syo isi syalyusigwe (\textsuperscript{\#}pamo n’ iisiingi).\\
%%LINE2
\gll
si-nguku  ntandatu  syo  isi  si-ali-ul-is-igw-e   pamo  na  i-si-ngi  \\
%%LINE3
\EXH{}-10.chicken  10.six  10.\IDCOP{}  10.\DEM{}.\PROX{}  10\SM{}-\PST{}-buy-\CAUS-\PASS{}-\FV{} maybe  and  \AUG{}-10-other\\
%%TRANS1
\glt
‘It’s six chickens (exactly) that were sold (\textsuperscript{\#}maybe more).’\\
%%TRANS2
%%EXEND


\z

A final important focus diagnostic concerns the detection of a presupposition. Whether a presupposition of existence is inherent to a linguistic strategy can be tested by asking a content question using that strategy and establishing whether the question can felicitously be answered by ‘nothing’ or ‘nobody’, i.e. the empty set. We are thus testing the strategy used in the question here. If the strategy presupposes that there is a referent for which the proposition is true, then logically the answer cannot deny that. \Citet{vanderWalNamyalo2016} give the following examples from Luganda, showing that the postverbal focus strategy does not have a presupposition, as the SVO object question in \xref{bkm:Ref134776868} can readily be answered by ‘nobody’, whereas the strategy formed by the question word in initial position followed by the marker \textit{gwe} does have this presupposition, considering the infelicity of the answer in \xref{bkm:Ref134776908}.

\ea
Luganda \citep[360]{vanderWalNamyalo2016}\\
\label{bkm:Ref134776868}
\begin{xlist}
%%EAX
\exi{Q:}
%%JUDGEMENT
%%LABEL
%%CONTEXT
%%LINE1
%%LINE2
\gll
W-á-kúbyé  ání? \\
%%LINE3
2\SG{}.\SM{}-\PST{}-hit.\PFV{}  who\\
%%TRANS1
\glt
‘Who did you hit?’\\
%%TRANS2
%%EXEND

%%EAX
\exi{A:}
%%JUDGEMENT
%%LABEL
%%CONTEXT
%%LINE1
%%LINE2
\gll
Te-wá-lî. \\
%%LINE3
\NEG{}-16\SM{}-be\\
%%TRANS1
\glt
‘Nobody.’, lit. ‘There is not.’
%%TRANS2
%%EXEND

\end{xlist}
\z

\ea
Luganda \citep[360]{vanderWalNamyalo2016}\\
\label{bkm:Ref134776908}
\begin{xlist}
%%EAX
\exi{Q:}
%%JUDGEMENT
[]{
%%LABEL
%%CONTEXT
%%LINE1
%%LINE2
\gll
Aní  gw-e  w-á-kúbyȇ?\\
%%LINE3
who  1-\FOC{}  2\SG.\SM{}-\PST{}-hit.\PFV{}\\
%%TRANS1
\glt
‘Who is it that you hit?’\\
%%TRANS2
}
%%EXEND

%%EAX
\exi{A:}
%%JUDGEMENT
[\textsuperscript{\#}]{
%%LABEL
%%CONTEXT
%%LINE1
%%LINE2
\gll
Te-wá-lî.\\
%%LINE3
\NEG-16\SM{}-be\\
%%TRANS1
\glt
‘Nobody.’, lit. ‘There is not.’
%%TRANS2
}
%%EXEND

\end{xlist}
\z

Each of these tests, as well as additional ones, will come back when necessary in each chapter to show the precise interpretations of the strategies for the expression of information structure.

\subsection{Thetics}
\label{bkm:Ref134712116}
Testing the properties of thetic sentences (sentences without a topic-comment split) is more challenging than testing for topic and focus, as thetics quite often use a strategy that is also used for other interpretations. It is therefore difficult to directly test for their thetic interpretation. For example, the canonical word order can also be felicitous in a thetic context in some languages, and subject inversion constructions are often used for thetics, but may also be used for narrow focus on the subject. The latter example of overlap is illustrated in \xref{bkm:Ref122698455}: default agreement inversion (VS word order with class 17 default agreement) can be used in Changana both in a thetic context out of the blue, as in \xref{bkm:Ref122698455:a}, and also with narrow subject focus, as shown for the subject modified by exhaustive ‘only’ in \xref{bkm:Ref122698455:b}. 

 \ea\label{bkm:Ref122698455}
Changana \citep[236, 238]{vanderWalSimango2022}

%%EAX
\ea
%%JUDGEMENT
%%LABEL
\label{bkm:Ref122698455:a}
%%CONTEXT
%%LINE1
%%LINE2
\gll
Kú-w-é  mú-ya:ki.\\
%%LINE3
17\SM{}-fall-\PFV{}.\CJ{}  1-bricklayer\\
%%TRANS1
\glt
‘A bricklayer fell / There fell a bricklayer.’\\
%%TRANS2
%%EXEND


%%EAX
\ex
%%JUDGEMENT
%%LABEL
\label{bkm:Ref122698455:b}
%%CONTEXT
%%LINE1
%%LINE2
\gll
Kú-lúz-e  ntsé:ná  kókwa:na. \\
%%LINE3
17\SM{}-lose-\PFV{}.\CJ{}  only  1.grandparent\\
%%TRANS1
\glt
‘Only grandpa passed away.’\\
%%TRANS2
%%EXEND


\z

\z


We looked for thetics by using prototypical thetic contexts, asking speakers to describe a picture, describe the weather, or to imagine what would be said in an out of the blue (or \textit{hot news}) context (e.g. “There is a village on a remote part of the river, where hardly any boats come to. One morning, a child comes running from the riverbank, shouting about the arrival of a ship. What does the child shout?”). It is often said that the answer to a question ‘what happened?’ is necessarily thetic, but this is not a watertight diagnostic for theticity. Because some referents are permanently or situationally available, these can be pushed into a topic function \citep{Givón1983,Erteschik-Shir2007}. For example, speech participants and ``always available" referents such as ‘the moon’ or ‘the newspaper’ can always form the topic, even if they have not been mentioned in the previous discourse. Since speakers have a very strong preference to form categorical sentences containing a topic and comment, it is likely that such permanently available referents function as topics even in a context that might otherwise favour a thetic statement. That is, the answer to a question ‘what happened?’ (which does not present any referent as active) typically elicits a thetic answer, but if a referent is permanently available, it may be coded as a topic even in this environment.

\subsection{Section summary}

In this section we have only briefly touched upon the main diagnostics and their logic in application for determining the expression and precise interpretation of information structure in the BaSIS languages. Additional information is provided in each chapter, as needed in order to understand the diagnostics applied for each individual language. We refer the interested reader to the methodology document\footnote{\url{https://scholarlypublications.universiteitleiden.nl/handle/1887/3608096}} for more extensive and detailed information on the diagnostics as well as further examples, hoping that the brief overview here may help to interpret the information in the chapters of this volume. 

\section{Reading guide}
\label{bkm:Ref117846450}
In preparing this book, we have attempted to co-ordinate the structure of each chapter so as to allow for cross-linguistic comparison. At the same time, each researcher has their own research interests, their own set of tests covered in the fieldwork, and was guided by the different nature of each individual language, meaning that the chapters also differ in content and focus.\largerpage

Each of the chapters starts out with some basic information about the language, such as the demography, classification, and geography. As mentioned, we do not provide general grammatical information about the language (e.g. the noun classes or tense system), but straightaway enter into presenting and discussing the main morphosyntactic strategies involved in the expression of information structure, providing references to other sources in which more detailed background information is available on each language.

We structure the presentation for each language from form to function: we take the linguistic (mostly morphosyntactic) expression and describe which function it has. This has some drawbacks, as it does not allow one to easily search ``how does language X express contrastive topics", for example. Nevertheless, this presentation was chosen because strategies can be underspecified or multifunctional in what they express, meaning that providing an as-complete-as-possible description of the strategy will be beneficial (and also considering \citeauthor{MatićWedgwood2013}'s (2013) warning that not every strategy that is used for focus is actually a focus strategy). The chapters are therefore organised with the goal of understanding the functions of each form in a given language, with cross-linguistic comparison a secondary aim.

All descriptions and generalisations presented are illustrated with example sentences, mostly from our own fieldwork and sometimes from existing sources. As in this introductory chapter, the examples that are extracted from the Online Language Databases via Dative generally have 4 lines: the expression in the object language, a morpheme break, a gloss, and a free translation. When the surface morphophonology is transparent enough, we may give only 3 lines by merging the first line and the morpheme break. Surface tone is marked on the first line for all the languages except Kinyakyusa, which does not have tone, and Kirundi, where the standard Kirundi orthography for tone marking is used.

Although the chapters are quite extensive, aiming for as complete a description as possible based on the available data, there will always be interesting phenomena that are not covered. We mention some of these limitations here. Firstly, as the project concentrated on the morphosyntactic expression of information structure (believed to be of key significance in Bantu languages), less attention is paid to phonology and prosody, only mentioning it where relevant and/or known. Secondly, we stick to the level of the sentence and do not discuss what happens on the level of the discourse. In other words, we do not go into discourse particles and the overall organisation of discourse or narratives (with the exception of some discussion of referent expression over discourse, as related to referent accessibility). Finally, as stated before, the aim of the book is descriptive, and so theoretical analyses and modelling have been avoided as far as possible – the intention is for the chapters to form a stepping stone from which such formal analyses may be developed.

Despite the limitations in topics and languages we could cover in this book, we hope that the chapters contribute to a better understanding both of the individual languages and of information structure in general. We also hope that this book and the methodology document may inspire others to describe information structure in other languages too, within and outside of Bantu. This will in turn enable us to better understand the diversity of the Bantu family and make more robust crosslinguistic comparisons regarding how information structure is expressed.  

\section*{Acknowledgements}

This research was supported by an NWO Vidi grant 276-78-001 as part of the “Bantu Syntax and Information Structure” (BaSIS) project at Leiden University, whose support we gratefully acknowledge. We thank Silvio Cruschina, Nancy Hedberg, and Aurore Montebran for their insightful comments on an earlier version of this chapter. Any remaining errors and points of view are ours alone.

\section*{Abbreviations and symbols}\largerpage

Numbers refer to noun classes except when followed by \SG{}/\PL{}, in which case they refer to persons. Commas indicate a pause.

%%% All Leipzig abbreviations are commented out, following the LangSci guidelines of only listing non-Leipzig abbreviations.
\begin{multicols}{2}
\begin{tabbing}
MMMM \= ungrammatical\kill
* \> ungrammatical\\
\textsuperscript{\#} \> infelicitous in the given \\ \> context\\
*(X) \> the presence of X is obligatory \\ \>  and cannot grammatically be  \\ \>  omitted\\
(*X) \> the presence of X would make \\ \>  the sentence ungrammatical\\
(X) \> the presence of X is optional\\
% \APPL{} \> applicative\\
\AUG{} \> augment\\
Aux \> auxiliary\\
% \CAUS{} \> causative\\
\CJ{} \> conjoint\\
\CM{} \> contrastive marker\\
% \COND{} \> conditional\\
\COP{} \> copula\\
\DEM{} \> demonstrative\\
\DEP{} \> dependent conjugation\\
\DISC{} \> discourse\\
\DUR{} \> durative/pluractional\\
\DJ{} \> disjoint\\
% \DUR{} \> durative\\
% \EXCL{} \> exclusive\\
\EMPH{} \> emphatic \\
\EXH{} \> exhaustive\\
% \FOC{} \> focus\\
\FV{} \> final vowel\\
\IDCOP{} \> identificational copula\\
\INCP{} \> inceptive\\
\INT{} \> intensifier\\
% \NEG{} \> negation\\
\N.\PST{} \> near past\\
OLD \> Online Language Database\\
\OM{} \> object marker\\
% \PASS{} \> passive\\
% \PFV{} \> perfective\\
% \PL{} \> plural\\
% \POSS{} \> possessive\\
\PREP{} \> preposition\\ 
\PRL{} \> predicative lowering\\
\PRO{} \> pronoun\\
% \PROX{} \> proximate\\
% \PRS{} \> present\\
\PST{} \> past\\
\PST{}2 \> Tunen second-degree past \\ \>  tense (hodiernal) \\
\REL{} \> relative\\
\RM{} \> relative marker\\
% \SBJV{} \> subjunctive\\
% \SG{} \> singular\\
\SM{} \> subject marker\\
\TO{} \> Cicopi perfective tense\\
Top \> topic\\
\YPST{} \> yesterday past
\end{tabbing}
\end{multicols}\largerpage


\printbibliography[heading=subbibliography,notkeyword=this]
\end{document}
