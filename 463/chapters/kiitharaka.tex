\documentclass[output=paper]{langscibook}
\ChapterDOI{10.5281/zenodo.14833610}
\author{Patrick N.\ Kanampiu\orcid{}\affiliation{Tharaka University College} and Jenneke van der Wal\orcid{}\affiliation{Leiden University}}
\title{The expression of information structure in Kîîtharaka}
\abstract{Kîîtharaka employs a variety of linguistic strategies to structure information. The preverbal domain is preferred for topicalised constituents while non-topical ones tend to be post-verbal. Various pragmatically sanctioned interpretations such as polarity focus, depreciative, contrastive and intensive readings are expressed through topic marking and predicate topic doubling. Focus can be expressed using the basic cleft (exhaustive focus), the pseudocleft and two constructions that at first look like reverse pseudoclefts. The expression of the object is regulated by accessibility, humanness, predicate type, and salience. Finally, the presence or absence of \textit{ni}- correlates with predicate-centred focus or focus on the postverbal constituent.}
\IfFileExists{../localcommands.tex}{
  \addbibresource{../localbibliography.bib}
  % add all extra packages you need to load to this file

\usepackage{tabularx,multicol}
\usepackage{url}
\urlstyle{same}

\usepackage{listings}
\lstset{basicstyle=\ttfamily,tabsize=2,breaklines=true}

\usepackage{langsci-basic}
\usepackage{langsci-optional}
\usepackage{langsci-lgr}
\usepackage{langsci-osl}
% \usepackage{./langsci/styles/langsci-lgr}
% \usepackage{./langsci/styles/langsci-osl}
% \usepackage{langsci-gb4e}

\usepackage{tikz}
\usetikzlibrary{patterns,calc}
\pgfdeclarepatternformonly{south east lines}{\pgfqpoint{-0pt}{-0pt}}{\pgfqpoint{3pt}{3pt}}{\pgfqpoint{3pt}{3pt}}{
    \pgfsetlinewidth{0.6pt}
    \pgfpathmoveto{\pgfqpoint{0pt}{3pt}}
    \pgfpathlineto{\pgfqpoint{3pt}{0pt}}
    \pgfpathmoveto{\pgfqpoint{.2pt}{-.2pt}}
    \pgfpathlineto{\pgfqpoint{-.2pt}{.2pt}}
    \pgfpathmoveto{\pgfqpoint{3.2pt}{2.8pt}}
    \pgfpathlineto{\pgfqpoint{2.8pt}{3.2pt}}
    \pgfusepath{stroke}}
    
\usepackage{stmaryrd}
\usepackage{wasysym}
\usepackage{multirow}
\usepackage{caption}
\usepackage{subcaption}
\usepackage{mathrsfs}
\usepackage{qtree}

\usepackage{linguex}


  %pminos do not split footnotes
% \interfootnotelinepenalty=10000 %Footnote in Laporte chapters has to be split SN


%\DeclareIndexNameFormat{default}{%
%\nameparts{#1}%
%\usebibmacro{index:name}%
%{\index[names]}%
%{\namepartfamily}%
%{\namepartgiveni}%
% {}% L1
% {}% L2
%{\namepartprefix}% generates spurious space L3
%{\namepartsuffix}% generates spurious space L4
%}

%  {\DeclareIndexNameFormat{default}{%
%     \usebibmacro{index:name}{\index[names]}{#1}{#3}{#5}{#7}}}

%\DeclareIndexNameFormat{default}{%
%  \usebibmacro{index:name}{\sindex[nom]}{#1}{#3}{#5}{#7}}

%\DeclareIndexNameFormat{default}{%
%  \usebibmacro{index:name}{\sindex[person]}{#1}{#3}{#5}{#7}}
%\DeclareIndexNameFormat{default}{%
%\nameparts{#1} \usebibmacro{index:name}{\sindex[person]]}{\namepartfamily}{‌​\namepartgiven}{\nam‌​epartprefix}{\namepa‌​rtsuffix}}

%\newcommand{\smiley}{:)}

%\renewbibmacro*{index:name}[5]{%
%\usebibmacro{index:entry}{#1}%
%{\iffieldundef{usera}{}{\thefield{usera}\actualoperator}\mkbibindexname{#2}{#3}{#4}{#5}}}

% \newcommand{\noop}[1]{}

%remove for final
%\overfullrule=1mm

\newcommand{\tobi}[2]}}
\renewcommand{\S}[1]{\tobi{#1}{\textsc{*}}}

% this volume references
% puts: [this volume]
% already defined: \citetv
%\newcommand{\citepv}[1]{(\citeauthor{#1} \citeyear*{#1} [this volume])}
\newcommand{\citealtv}[1]{\citeauthor{#1} \citeyear*{#1} [this volume]}

%parentheses around example number
\newcommand{\pref}[1]{(\ref{#1})}

% in-text examples

\newcommand{\lnex}[1]{\textit{#1}} %target lang word
\newcommand{\lnlit}[1]{(lit.: `#1')} %literal reading
\newcommand{\lnlat}[1]{(#1)} % latinization
\newcommand{\lntrans}[1]{`#1'} %translation
\newcommand{\lnexl}[2]%
{\lnex{#1}{} \lnlat{#2}} % ex with latinization
\newcommand{\lnexlat}[3]{\lnex{#1}{} \lnlat{#2}{} \lntrans{#3}} % ex with latinization and tranl.

%ch01
\newcommand{\co}[1]{\mbox{\textbf{#1}}}

%ch09

\newcommand{\cyrbulg}[1]{\begin{otherlanguage*}{bulgarian}#1\end{otherlanguage*}}


%ch10
\newcommand{\nlp}{{\small NLP}}
\newcommand{\mwe}{{\small MWE}}
\newcommand{\rae}{{\small RAE}}
\newcommand{\lvc}{{\small LVC}}
\newcommand{\pos}{{\small P}o{\small S}}
%\newcommand{\todo}[1]{ \textcolor{red}{#1} }

%\renewcommand{\labelenumi}{\theenumi}
%\ainamefmt{{vv}{ll}{, ff}{, jj}} % fullname

\newcommand{\biberror}[1]{{\color{red}#1}}

\newcommand{\osenovaitem}{--~}
  %% hyphenation points for line breaks
%% Normally, automatic hyphenation in LaTeX is very good
%% If a word is mis-hyphenated, add it to this file
%%
%% add information to TeX file before \begin{document} with:
%% %% hyphenation points for line breaks
%% Normally, automatic hyphenation in LaTeX is very good
%% If a word is mis-hyphenated, add it to this file
%%
%% add information to TeX file before \begin{document} with:
%% %% hyphenation points for line breaks
%% Normally, automatic hyphenation in LaTeX is very good
%% If a word is mis-hyphenated, add it to this file
%%
%% add information to TeX file before \begin{document} with:
%% \include{localhyphenation}
\hyphenation{
    Beck-man
    Ngu-yen
    back-chan-nel
    back-chan-nels
    mo-not-o-nous
    ste-reo-typ-i-cal
}

\hyphenation{
    Beck-man
    Ngu-yen
    back-chan-nel
    back-chan-nels
    mo-not-o-nous
    ste-reo-typ-i-cal
}

\hyphenation{
    Beck-man
    Ngu-yen
    back-chan-nel
    back-chan-nels
    mo-not-o-nous
    ste-reo-typ-i-cal
}

  \togglepaper[4]%%chapternumber
}{}
\begin{document}
\maketitle
\label{ch:4}


\section{Introduction}

This chapter presents an overview of information structure in Kîîtharaka. Kîîtharaka is a Central Bantu language, ISO code [thk] and classified as E54 in \citegen{Maho2009} revised \citet{Guthrie1948} classification. It is spoken by Atharaka people mainly occupying Tharaka Central, Tharaka North and Tharaka South Sub-counties of Tharaka Nithi County of Kenya. Kîîtharaka speakers are also found in eastern parts of Maara and Chuka Sub-counties, and Tharaka Sub-county in Kitui County.\footnote{The speakers are said to speak the Thagicû dialect, which is heavily influenced by the adjacent Kikamba language. Thagicû here refers to a variant of the present-day Kîîtharaka language spoken mainly by inhabitants of Tharaka Sub-county of Kitui County and not the Proto-Eastern Bantu language mentioned in some diachronic literature.} There are approximately 220,000 Atharaka people, according to the 2019 Kenya Population and Housing Census report (2019). Kîîtharaka has approximately 61,000 monolinguals with L1 literacy rate below 15\% as compared to that of L2 that stands at 45\% \citep{EberhardEtAl2022}. Literature on Kîîtharaka morphosyntax includes the following: \citet{Lindblom1914,Harford1991,Harford1997,waMberia1993,Muriungi2005,Muriungi2008,Muriungi2014,AbelsMuriungi2008,Kanampiu2017,KanampiuMuriungi2019}.

The data for this study were collected during a field study in December 2019 and January 2020 with three native speakers (all male) of age bracket 29--47 years. This included translations, judgements, reactions to stimuli as well as traditional folk tales and a recounting of the frog story. Additional data was elicited introspectively by the first author. The data were transcribed and stored in an Online Language Database accessible through the Dative user interface that allows data sharing in a collaborative research. This database will be accessible through The Language Archive. We also refer to the introduction to this book for further information on Dative and for further background on the terms and diagnostics used for information structure.

Kîîtharaka has a seven-vowel system, in which the mid-high vowels [e] and [o] are represented orthographically as 〈î〉 and 〈û〉. Furthermore, Dahl’s Law is active in Kîîtharaka, causing voiceless stops to become voiced when followed by another voiceless stop, for example underlying \textit{a-kû-t-a} `s/he is throwing' is realised as \textit{agûta}.

\section{Word order}
\label{kitwordorder}

Information structure influences Kîîtharaka word order in various ways: in preferring topics in the preverbal domain (\sectref{bkm:Ref115858011}), allowing unmarked focus postverbally only (\sectref{bkm:Ref115857994}), and encoding logical subjects postverbally when not topical (subject inversion, \sectref{bkm:Ref94457814}). Nevertheless, Kîîtharaka does not have a dedicated focus position, and the occurrence of preverbal focus is dependent on further analysis (see also \sectref{bkm:Ref117495353}). See for further discussion on the discourse-configurationality of word order \citet{KerrEtAl2023}.

\subsection{Preverbal domain}
\label{bkm:Ref115858011}
Kîîtharaka prefers topics in the preverbal domain, allowing left-dislocation of subjects, as in \xref{bkm:Ref115858173} and \xref{bkm:Ref115858192}, as well as objects as in \xref{bkm:Ref115858203}. Dislocation of the subject can be identified by an intervening constituent between subject and verb, and/or by a prosodic break after the subject (indicated by a comma).

%%EAX
\ea
%%JUDGEMENT
%%LABEL
\label{bkm:Ref115858173}
%%CONTEXT
(Did Peter eat ugali and stew?)\\
%%LINE1
 Bita i nkima yoonká arîîré.\\
%%LINE2
\gll
Bita  ni  n-kima  i-onka  a-rî-ire\\
%%LINE3
1.Peter \FOC{} 9-ugali  9-only  1\SM{}-eat-\PFV{}\\
%%TRANS1
\glt
‘Peter, it is only ugali that he ate.’
%%TRANS2
%%EXEND

\z

%%EAX
\ea
%%JUDGEMENT
%%LABEL
\label{bkm:Ref115858192}
%%CONTEXT
(How did the teachers do in the performance?)\\
%%LINE1
 Arím\v{û}, í baííníré!\\
%%LINE2
\gll
a-rimû  ni  ba-in-ire\\
%%LINE3
2-teachers \FOC{} 2\SM{}-dance-\PFV{}\\
%%TRANS1
\glt
‘The teachers, they danced (very well)!’
%%TRANS2
%%EXEND

\z

%%EAX
\ea
%%JUDGEMENT
%%LABEL
\label{bkm:Ref115858203}
%%CONTEXT
(Can I buy bananas here?)\\
%%LINE1
 Ndíg\'{û} nwa \'{û}g\'{û}re.\\
%%LINE2
\gll
n-digû  nwa  û-gur-e\\
%%LINE3
10-banana  can 2\SG.\SM{}-buy-\SBJV{}\\
%%TRANS1
\glt
‘Bananas you can buy.’
%%TRANS2
%%EXEND

\z

Whether the dislocated object is resumed by an object marker on the verb depends on the predicate and various other factors, as discussed in \sectref{bkm:Ref115790593}. The dislocated topics can also be interpreted as contrastive, as in \xref{bkm:Ref120101100}, and may then be marked by \NA+\PRO{} (further discussed in \sectref{bkm:Ref115790636}).

%%EAX
\ea
%%JUDGEMENT
%%LABEL
\label{bkm:Ref120101100}
%%CONTEXT
%%LINE1
{Nyóóndǒ, ní yá kûrííng\'{î}ra mîsumáál\'{î}, mûchûménǒ ni wá kûgítáánga mbáo,}{ \textbf{nayo} raandá, ni yá kûmarîîria (mbáo).}\\
%%LINE2
\gll
ny-oondo  ni  y-a  kû-riing-ir-a  mî-sumaarî mû-chûmeno  ni  w-a  kû-gitaanga  mbao na=y-o raanda  ni  y-a  kû-mar-iir-i-a  (mbao)\\
%%LINE3
9-hammer \COP{} 9-\CONN{} 15-hit-\APPL{}-\FV{} 4-nails 3-saw \COP{} 3-\CONN{} 15-cut 9.wood and=9-\PRO 9.plane \COP{} 9-\CONN{} 15-smooth-\APPL-\IC-\FV{} (9.wood)\\
%%TRANS1
\glt
‘The hammer is for hitting nails; the saw is for cutting wood into pieces; and the plane is for smoothening (wood).’
%%TRANS2
%%EXEND

\z

We also find scene-setting topics in the left periphery, which are typically adverbials, as in \xref{bkm:Ref131607959} and \xref{bkm:Ref131607960}.

%%EAX
\ea
%%JUDGEMENT
%%LABEL
\label{bkm:Ref131607959}
%%CONTEXT
%%LINE1
Îîgóró, í kûráárî na matu.\\
%%LINE2
\gll
î-goro  ni  kû-ra-rî  na   ma-tu\\
%%LINE3
5-yesterday \FOC{} 17\SM-\YPST{}-be  with  6-cloud\\
%%TRANS1
\glt
‘Yesterday it was cloudy.’
%%TRANS2
%%EXEND

\z

%%EAX
\ea
%%JUDGEMENT
%%LABEL
\label{bkm:Ref131607960}
%%CONTEXT
%%LINE1
Ai, g\'{û}gûk\'{î}á, ántû bákûy\'{û}kía báúmagarûká, \'{î}ndî baatiga bááthaîká mwarî njá. \\
%%LINE2
\gll
ai kû-kûkia   a-ntu   ba-kû-yûki-a ba-umagar-ûk-a   îndî   ba-a-tig-a   ba-a-thaîk-a   mu-arî  n-ja \\
%%LINE3
\EXCL{} 15-daybreak   2-person 2\SM{}-\PRS{}-take-\FV{} 2\SM{}-get.out-\RECP-\FV{} but 2\SM-\PST{}-leave-\FV{} 2\SM-\PST{}-tie-\FV{}   1-lady  9-compound\\
%%TRANS1
\glt
‘Gosh, when the day broke, they removed the ‘marûa’, and left the girl tethered in the compound.’
%%TRANS2
%%EXEND

\z

Multiple topics can also be found when multiple arguments are given information, for example both `John' and `his father' in \xref{ex:johnfatherdoctor}.

%%EAX
\ea
%%JUDGEMENT
%%LABEL
\label{ex:johnfatherdoctor}
%%CONTEXT
%%LINE1
Jónii, îthé í ndagitáárî.\\
%%LINE2
\gll
Jonii  îthe  ni  n-dagitaarî\\
%%LINE3
1.John  his.father \COP{} 9-doctor\\
%%TRANS1
\glt
‘(As for) John, his father is a doctor.’
%%TRANS2
%%EXEND

\z

This, however, does not mean that the preverbal domain is reserved for topics, as indefinite subjects may appear in preverbal position as in the thetic sentence in \xref{ex:marketnews}:

%%EAX
\ea
%%JUDGEMENT
%%LABEL
\label{ex:marketnews}
%%CONTEXT
(What’s the news from the market?)\\
%%LINE1
M\'{û}ntû n’aathûûng\'{î}\'{î}re ndúkáaní.\\
%%LINE2
\gll
mû-ntû  ni  a-ra-thûûngîr-ire  n-duka=ini\\
%%LINE3
1-person \FOC{}  1\SM-\YPST{}-enter-\PFV{} 9-shop=\LOC{}\\
%%TRANS1
\glt
‘Someone entered the shop.’
%%TRANS2
%%EXEND

\z


Furthermore, SV(O) order is acceptable with idioms in a thetic context, as illustrated in \xref{bkm:Ref131607083}. In this case, the subject of the clause is not an available referent in actual speech context.

%%EAX
\ea
%%JUDGEMENT
%%LABEL
\label{bkm:Ref131607083}
%%CONTEXT
%%LINE1
{Tîîri îgû\'{û}ka.}\\
%%LINE2
\gll
tîîri î-kû-ûk-a\\
%%LINE3
9.dust 9\SM{}-\PRS{}-rise-\FV{}\\
%%TRANS1
\glt
‘Things are finished.’\\ lit. ‘Dust has risen.’
%%TRANS2
%%EXEND

\z

Therefore, not all preverbal referents are topical, but there is a debate about whether preverbal focus is possible in Kîîtharaka. On the one hand, it is clear that focused subjects may not appear unmarked in their canonical preverbal position, as illustrated for interrogatives \xref{bkm:Ref64216844}, answers to content questions \xref{bkm:Ref94180218}, and subjects marked by the exhaustive focus-sensitive particle ‘only’, as in \xref{bkm:Ref90041004} and \xref{bkm:Ref90041005}. While it is correct to use \xref{bkm:Ref90040976}, the context strongly suggests an echo-question interpretation, and we conclude that unmarked focused subjects are not allowed in a canonical preverbal position.

%%EAX
\ea
%%JUDGEMENT
[*]{
%%LABEL
\label{bkm:Ref64216844}
%%CONTEXT
%%LINE1
Ûû áiyiré?\\
%%LINE2
\gll
ûû  a-a-iy-ire\\
%%LINE3
who {1\SM-\N.\PST{}-steal-\PFV{}} \\
%%TRANS1
\glt
‘Who stole?’
%%TRANS2
}
%%EXEND

\z

%%EAX
\ea
%%JUDGEMENT
[]{
%%LABEL
%%CONTEXT
(Context: Peter meets his friends discussing how a certain person stole and was attacked and torched by a mob. He does not get the name of the thief, so he asks):\\
%%LINE1
Ûû aiyá?\\
%%LINE2
\gll
ûû  a-a-iy-a\\
%%LINE3
who 1\SM-\PST{}-steal-\FV{}\\
%%TRANS1
\glt
‘Who stole\label{bkm:Ref90040976}?’
%%TRANS2
}
%%EXEND

\z

%%EAX
\ea
%%JUDGEMENT
[\#]{
%%LABEL
\label{bkm:Ref94180218}
%%CONTEXT
(Who went to the farm?)\\
%%LINE1
Kîmathi n’áthíiré mûndaaní.\\
%%LINE2
\gll
Kîmathi ni  a-thi-ire  mû-nda=ini\\
%%LINE3
1.Kîmathi \FOC{}  1\SM{}-go-\PFV{} 3-farm=\LOC{}\\
%%TRANS1
\glt
‘Kimathi went to the farm.’
%%TRANS2
}
%%EXEND

\z

%%EAX
\ea
%%JUDGEMENT
[*]{
%%LABEL
\label{bkm:Ref90041004}
%%CONTEXT
%%LINE1
Baabá wéenka n’ ákinyiré.\\
%%LINE2
\gll
baaba  we-onka  ni  a-kiny-ire\\
%%LINE3
1.father  1-only \FOC{} 1\SM{}-arrive-\PFV{} \\
%%TRANS1
\glt
‘Only dad arrived.’
%%TRANS2
}
%%EXEND

\z

%%EAX
\ea
%%JUDGEMENT
[]{
%%LABEL
\label{bkm:Ref90041005}
%%CONTEXT
%%LINE1
Jééní wéenká\footnote{When used with animates, \mbox{-\textit{nka}} (\textit{wenka} for human and \mbox{-\textit{onka}} for other animates) is ambiguous between ‘alone’ and ‘only’ but when used with inanimates, it means ‘only’. \textit{Akî} means ‘only’ both with animates and inanimates.} n’aandíkíré baar\'{û}a.\\
%%LINE2
\gll
Jane we-onka  ni  a-andîk-ire baarûa\\
%%LINE3
1.Jane 1-alone \FOC{} 1\SM{}-write-\PFV{} 9.letter\\
%%TRANS1
\glt
*‘Only Jane wrote a letter.’\\
%%TRANS2
‘Jane alone wrote a letter.’
}
%%EXEND

\z

%%EAX
\ea
%%JUDGEMENT
[*]{
%%LABEL
%%CONTEXT
%%LINE1
Jééní ak\'{î} (n’)áandîkiré baar\'{û}a.\\
%%LINE2
\gll
Jane akî  ni  a-andîk-ire baarûa\\
%%LINE3
1.Jane only \FOC{} 1\SM{}-write-\PFV{} 9.letter\\
%%TRANS1
\glt
‘Only Jane wrote a letter.’
%%TRANS2
}
%%EXEND

\z

On the other hand, subjects may be focused as in \xref{bkm:Ref94180265} and \xref{bkm:Ref94180356} when preceded by \textit{ni} (appearing as \textit{i-} before a consonant and \textit{n-} before a vowel).

%%EAX
\ea
%%JUDGEMENT
%%LABEL
\label{bkm:Ref94180265}
%%CONTEXT
%%LINE1
{Ám\'{û}ûríá} ‘Íngukû\'{û}ria mwar\'{î}, \textbf{n’\'{û}û} ágûpéeré r\'{û}ûtha rwá gûtáa rû\'{û}yî?’\\
%%LINE2
\gll
a-mû-ûri-a ni n-kû-kû-ûri-a mû-arî, ni û a-kû-per-ire rûûtha rû-a  kû-taa rû-ûyî?\\
%%LINE3
1\SM-1\OM{}-ask-\FV{}  \FOC{} {1\SG.\SM-\PRS-2\SG.\OM{}-ask-\FV{}} 1-girl \FOC{} who 1\SM-\PRS{}-give-\PFV{} 11.permission 11-\CONN{} 15-fetch 11-water?\\
%%TRANS1
\glt ‘He asked her, ‘let me ask you girl, who gave you permission to fetch water?’'
%%TRANS2
%%EXEND

\z

%%EAX
\ea
%%JUDGEMENT
%%LABEL
\label{bkm:Ref94180356}
%%CONTEXT
(Who went to the farm? cf. \xref{bkm:Ref94180218})\\
%%LINE1
I Kîmáthi ûthííre mûndaaní.\\
%%LINE2
\gll
ni Kîmathi û-thi-ire mû-nda=ini\\
%%LINE3
\FOC{} Kîmathi 1\RM{}-go-\PFV{} 3-farm=\LOC{}\\
%%TRANS1
\glt
‘Kimathi went to the farm.’ / ‘It’s Kimathi who went to the farm.’
%%TRANS2
%%EXEND

\z

A question for analysis is whether this is a preverbal focused element marked additionally by the focus marker \textit{ni}, or whether it is a (biclausal) cleft in which \textit{ni} could be a copula (see brief discussion in \sectref{bkm:Ref117495353}).

We can therefore conclude that the preverbal domain in Kîîtharaka is not restricted to topics, even if topics are preferably placed preverbally. We now turn to the postverbal domain.

\subsection{Postverbal domain}
\label{bkm:Ref115857994}\label{bkm:Ref115798057}
The postverbal domain in Kîîtharaka consists of the non-topical information in the comment, and a right periphery for resumptive/afterthought information. Kîîtharaka does not have a dedicated position for focus; neither the immediate-after-verb (as in Makhuwa-Enahara, \cite{chapters/makhuwa}) nor the final position (as in Kirundi, \cite{chapters/kirundi}) is required or preferred for focused constituents. All internal arguments and non-arguments can be focused in the postverbal domain – only external arguments are excluded, as discussed in \sectref{bkm:Ref94457814} on subject inversion. Note further that the form of the verb also makes a difference in the interpretation of the postverbal constituents: the form prefixed by \textit{ni} is used more naturally in predicate-centred focus and at the end of the clause, whereas the form without this marker goes together with focus on the postverbal constituent. This is discussed in more detail in \sectref{bkm:Ref119919238}.


Both the Recipient and the Theme can be questioned in their canonical position, as shown in \xref{bkm:Ref88333429} and \xref{bkm:Ref89972047}, where we also see a preference to stick with the thematic order Recipient-Theme.


%%EAX
\ea
%%JUDGEMENT
%%LABEL
\label{bkm:Ref88333429}
%%CONTEXT
%%LINE1
Ûy\'{û}\'{û}gú ánénkééré twáana mbí?\\
%%LINE2
\gll
ûyûûgû  a-nenker-ire  tû-ana  m-bi\\
%%LINE3
1.grandmother 1\SM{}-give-\PFV{} 13-child  9-what \\
%%TRANS1
\glt
‘What did grandma give the children?’
%%TRANS2
%%EXEND

\z

\ea
\label{bkm:Ref89972047}
(Context: Grandmother had some mangoes and she gave an unknown person the mangoes.)\\
\judgewidth{\textsuperscript{??}}
%%EAX
\ea
%%JUDGEMENT
[]{
%%LABEL
%%CONTEXT
%%LINE1
Apééré \textsuperscript{!}\'{û}\'{û} méémbé?\\
%%LINE2
\gll
a-per-ire  ûû  ma-embe\\
%%LINE3
1\SM{}-give-\PFV{} who  6-mango \\
%%TRANS1
\glt
‘Who did she give mangoes?’
%%TRANS2
}
%%EXEND

%%EAX
\ex
%%JUDGEMENT
[\textsuperscript{??}]{
%%LABEL
%%CONTEXT
%%LINE1
Apéére méémbe ûû?\\
%%LINE2
\gll
a-per-ire  ma-embe  ûû\\
%%LINE3
1\SM{}-give-\PFV{} 6-mango  who \\
%%TRANS1
\glt
‘Who did she give mangoes?’
%%TRANS2
}
%%EXEND

\z
\z

A question with multiple question words is even possible for both objects, but only in their base order Recipient-Theme, as shown in \xref{bkm:Ref89972049}.

\ea
\label{bkm:Ref89972049}
%%EAX
\ea
%%JUDGEMENT
[]{
%%LABEL
%%CONTEXT
(Context: Grandmother gave something unknown to someone unknown)\\
%%LINE1
Ûy\'{û}gú ánénkéére ûû mbí?\\
%%LINE2
\gll
ûyûgu  a-nenker-ire  ûû  mbi\\
%%LINE3
1.grandmother 1\SM{}-give-\PFV{}  who  what\\
%%TRANS1
\glt
‘What did grandmother give to whom?’
%%TRANS2
}
%%EXEND

%%EAX
\ex
%%JUDGEMENT
[*]{
%%LABEL
%%CONTEXT
%%LINE1
Ûy\'{û}gú ánénkéére mbí ûû?\\
%%LINE2
\gll
ûyûgu  a-nenker-ire  mbi  ûû\\
%%LINE3
1.grandmother 1\SM{}-give-\PFV{}  what  who\\
%%TRANS1
\glt
‘What did grandmother give to whom?’
%%TRANS2
}
%%EXEND

\z
\z

Furthermore, both recipient and theme can be answers to a content question in their canonical position \xref{bkm:Ref89972080}, and a constituent modified by ‘only’ is allowed in either position, as in \xref{bkm:Ref89972089} and \xref{bkm:Ref89972091}.

%%EAX
\ea
%%JUDGEMENT
%%LABEL
\label{bkm:Ref89972080}
%%CONTEXT
(What did the teacher give to the children?\slash Who did the teacher give pens?)\footnote{Note that typically, the subject would be elided because it’s already the topic.}\\
%%LINE1
Mwarímû aránénkéére twaána tûrámu.\\
%%LINE2
\gll
mû-arimû  a-ra-nenker-ire  tû-ana  tû-ramu\\
%%LINE3
1-teacher 1\SM-\YPST{}-give-\PFV{} 13-child  13-pen\\
%%TRANS1
\glt
‘The teacher gave the children pens.’
%%TRANS2
%%EXEND

\z

\ea
\label{bkm:Ref89972089}
%%EAX
\ea
%%JUDGEMENT
%%LABEL
%%CONTEXT
%%LINE1
Mwarimû aránénkéére twaána akî tûramu.\\
%%LINE2
\gll
mû-arimû  a-ra-nenker-ire  tû-ana  aki  tû-ramu\\
%%LINE3
1-teacher  1\SM-\YPST{}-give-\PFV{}  13-child  only  13-pen\\
%%TRANS1
\glt
‘The teacher gave only children some pens.’
%%TRANS2
%%EXEND

%%EAX
\ex
%%JUDGEMENT
%%LABEL
%%CONTEXT
%%LINE1
Mwarimû aránénkéére twaána tûrámu akî.\\
%%LINE2
\gll
mû-arimû  a-ra-nenker-ire  tû-ana  tû-ramu  akî\\
%%LINE3
1-teacher  1\SM-\YPST{}-give-\PFV{} 13-child  13-pen  only\\
%%TRANS1
\glt
‘The teacher gave only pens to the children.’
%%TRANS2
%%EXEND

\z
\z


%%EAX
\ea
%%JUDGEMENT
%%LABEL
\label{bkm:Ref89972091}
%%CONTEXT
%%LINE1
Frída anénkééré Kaw\'{î}\'{î}ra fáánta yóonká.\\
%%LINE2
\gll
Frida  a-nenker-ire  Kawîîra  fanta  î-onka\\
%%LINE3
1.Frida {1\SM{}-give-\PFV{}} 1.Kawîîra  9.fanta  9-only \\
%%TRANS1
\glt
‘Frida gave Kawîîra only Fanta.’
%%TRANS2
%%EXEND

\z

Constituents in the right periphery of the sentence may be interpreted as afterthoughts, as illustrated for a subject in \xref{bkm:Ref119918606} and for an object in \xref{bkm:Ref119918608}. Note that the object referent can also be marked on the verb if it is expressed as a right-dislocated noun phrase (see also section \sectref{bkm:Ref115790593}, and that there must be a pause preceding the noun phrase, indicated by the comma.


%%EAX
\ea
%%JUDGEMENT
%%LABEL
\label{bkm:Ref119918606}
%%CONTEXT
(Context: Hyena roasted the guinea fowl and ate him whole. Where Hare arrived, he checked and found there was no guinea fowl. He asked Hyena.)
%%LINE1
Mbítí, \'{î}rî k\'{û} nkáánga?\\
%%LINE2
\gll
m-biti  î-rî  kû  n-kanga\\
%%LINE3
9-hyena 9\SM{}-be  where  9-guinea.fowl\\
%%TRANS1
\glt
‘Hyena, where is the guinea fowl?’’
%%TRANS2
%%EXEND

\z

\ea
\label{bkm:Ref119918608}
%%EAX
\ea
%%JUDGEMENT
%%LABEL
%%CONTEXT
%%LINE1
Mwarímû agátuona, twaána.\\
%%LINE2
\gll
mû-arimû  a-ka-tû-on-a,  tû-ana\\
%%LINE3
1-teacher 1\SM{}-\FUT-12\OM{}-see-\FV{}  13-child\\
%%TRANS1
\glt
‘The teacher will see them, the children.’
%%TRANS2
%%EXEND

%%EAX
\ex
%%JUDGEMENT
%%LABEL
%%CONTEXT
%%LINE1
Mwarímû akóóna (*,) twaána.\\
%%LINE2
\gll
mû-arimû  a-ka-on-a  tû-ana\\
%%LINE3
1-teacher 1\SM-\FUT{}-see-\FV{}  13-child\\
%%TRANS1
\glt
‘The teacher will see the children.’
%%TRANS2
%%EXEND

\z
\z


As mentioned, the interpretation of postverbal constituents is also dependent on the form of the verb, as discussed in more detail in \sectref{bkm:Ref119919394}. We conclude that the postverbal domain typically hosts non-topical constituents, which may be focal, and which can be focused in their canonical position, i.e. there is no dedicated focus position. This also means that logical subjects can be expected to appear postverbally when non-topical, which is what we turn to next.

\subsection{Inversion constructions}
\label{bkm:Ref94457814}
In subject inversion constructions, the logical subject appears in postverbal position. There are a number of different inversion constructions \citep[see][]{MartenvanderWal2014}, which we discuss for Kîîtharaka in turn. As \citet{BuellMuriungi2008} show, Kîîtharaka does not show patient inversion \xref{monkey}, formal locative inversion (see below), or agreeing inversion \xref{bkm:Ref115793377}, regardless of the form of the verb with or without \textit{ni} (for which see \sectref{bkm:Ref111628808}).

\ea
%%EAX
\label{monkey}
\ea
%%JUDGEMENT
%%LABEL
%%CONTEXT
%%LINE1
Mût\'{î} n’\'{û}g\'{û}tw\'{a} \'{î}g\'{û}na.\\
%%LINE2
\gll
mû-tî  ni  û-kû-tw-a  î-gûna\\
%%LINE3
3-tree \FOC{} 3\SM-\PRS{}-climb-\FV{} 5-monkey\\
%%TRANS1
\glt
*`A/the monkey climbs the tree.’\\
%%TRANS2
 lit. ‘The tree climbs a/the monkey.’
%%EXEND

 %%EAX
\ex
%%JUDGEMENT
%%LABEL
%%CONTEXT
%%LINE1
Mût\'{î} ûgáatwá îg\'{û}na.\\
%%LINE2
\gll
mû-tî  û-kaa-tw-a  î-gûna\\
%%LINE3
3-tree 3\SM{}-\FUT{}-climb-\FV{} 5-monkey\\
%%TRANS1
\glt
*`The/a monkey will climb the tree.’\\
%%TRANS2
lit. ‘The tree will climb the monkey.’
%%EXEND

\z

%%EAX
\ex
%%JUDGEMENT
[*]{
%%LABEL
\label{bkm:Ref115793377}
%%CONTEXT
%%LINE1
N’ yáakúa nyamû. / *Yáakúa nyamû.\\
%%LINE2
\gll
(ni)  î-a-ku-a  n-yamû \\
%%LINE3
(\FOC)  9\SM-\PST{}-die-\FV{} 9-animal\\
%%TRANS1
\glt
‘A/The animal died.’
%%TRANS2
}
%%EXEND
\z

Canonical instrument inversion is not accepted in Kîîtharaka, as illustrated in \xref{bkm:Ref117501764}, with or without \textit{ni} preceding the verb. Even if an applicative were present (which remedies inversion in Zulu, \citealt{Zeller2013}), this is not accepted. Interestingly, however, we do find what looks like instrument inversion in the presence of \textit{na} ‘with’, as in \xref{bkm:Ref117502171:b} and \xref{bkm:Ref117502173:b}. While \xref{bkm:Ref117502173:b} may not in fact be instrument inversion, as indicated in the translation, we wonder how \xref{bkm:Ref117502171:b} may be interpreted. As we currently do not have enough insight into this construction, we leave it to one side here.

%%EAX
\ea
%%JUDGEMENT
[*]{
%%LABEL
\label{bkm:Ref117501764}
%%CONTEXT
%%LINE1
Gîcíko (i)g\'{î}kûr\'{î}a Kanyúá.\\
%%LINE2
\gll
kî-ciko  ni  kî-kû-rî-a  Kanyua\\
%%LINE3
7-spoon \FOC{} 7\SM-\PRS{}-eat-\FV{} 1.Kanyua\\
%%TRANS1
\glt
int. ‘Kanyua has eaten with a spoon.’
%%TRANS2
}
%%EXEND

\z

\ea
\label{bkm:Ref117502171}
%%EAX
\ea
%%JUDGEMENT
%%LABEL
\label{bkm:Ref117502171:a}
%%CONTEXT
(Context: Anyone is welcome to eat the porridge on the table. But I am afraid it’s very thick, so I wonder how you will eat it.)\\
%%LINE1
Kanyúá ákûría na gîciko.\\
%%LINE2
\gll
Kanyua   a-kû-rî-a   na   kî-ciko\\
%%LINE3
1.Kanyua 1\SM-\PRS{}-eat-\FV{} with   7-spoon\\
%%TRANS1
\glt
‘Kanyua has used a spoon to eat.’
%%TRANS2
%%EXEND

 %%EAX
\ex
%%JUDGEMENT
%%LABEL
\label{bkm:Ref117502171:b}
%%CONTEXT
(Context: You’re looking for a spoon to eat and there is one on the table.)\\
%%LINE1
Gîcíko gîkú g\'{î}kûría *(na Kanyúá).\\
%%LINE2
\gll
kî-ciko   kî-ku   kî-kû-rî-a   na   Kanyua\\
%%LINE3
7-spoon   7-\DEM.\MED{}   7\SM{}-\PRS{}-eat-\FV{} with   1.Kanyua\\
%%TRANS1
\glt
‘That spoon has been used by Kanyua to eat (so don’t use it).’
%%TRANS2
%%EXEND

\z
\z


\ea
\label{bkm:Ref117502173}
%%EAX
\ea
%%JUDGEMENT
%%LABEL
\label{bkm:Ref117502173:a}
%%CONTEXT
(Why do you look so happy?)\\
%%LINE1
Kîmáthi \'{a}kûûya na ndek\'{e}.\\
%%LINE2
\gll
Kîmathi   a-kû-j-a   na   n-deke\\
%%LINE3
1.Kîmathi 1\SM-\PRS{}-come-\FV{} with   9-airplane\\
%%TRANS1
\glt
‘Kimathi has come with an airplane.’ / ‘Kimathi has come by airplane.’
%%TRANS2
%%EXEND

 %%EAX
\ex
%%JUDGEMENT
%%LABEL
\label{bkm:Ref117502173:b}
%%CONTEXT
(Why do you look so happy?)\\
%%LINE1
Ndeké \'{î}kûûya na Kîmathi.\\
%%LINE2
\gll
n-deke   î-kû-j-a   na   Kîmathi\\
%%LINE3
9-airplane   9\SM{}-\PRS{}-come-\FV{} with   1.Kîmathi\\
%%TRANS1
\glt
lit. ‘The airplane  has come with/brought Kimathi.’
%%TRANS2
%%EXEND

\z
\z

The more frequent inversion construction is Default Agreement Inversion (DAI), and Semantic Locative Inversion (SLI) is also accepted but not used often. We first discuss the form and then the interpretation of both constructions.

In DAI, the subject marker is the default/expletive \textit{kû}-, the original marker for locative class 17. However, since \textit{kû-} no longer refers to an actual location, and a preverbal locative is not obligatory (see \xref{bkm:Ref115792333}), we analyse this subject marker as default agreement.

%%EAX
\ea
%%JUDGEMENT
%%LABEL
%%CONTEXT
(Context: Reporting on what happened yesterday.)\\
%%LINE1
Í kûrátûûbágá twaána.\\
%%LINE2
\gll
ni  kû-ra-tûûb-ag-a  tû-ana\\
%%LINE3
\FOC{} 17\SM{}-\YPST-\HAB{}-\FV{} 13-child\\
%%TRANS1
\glt
‘The children were jumping.’
%%TRANS2
%%EXEND

\z

%%EAX
\ea
%%JUDGEMENT
%%LABEL
\label{bkm:Ref115792333}
%%CONTEXT
(Mûthít\'{û}\'{û}ní)
%%LINE1
 í kûrágwiire mît\'{î}.\\
%%LINE2
\gll
mû-thitû=ini  ni  kû-ra-gû-ire  mî-tî\\
%%LINE3
3-forest=\LOC{} \FOC{} 17\SM{}-\YPST{}-fall-\PFV{} 4-tree\\
%%TRANS1
\glt
‘(In the forest) there fell trees.’
%%TRANS2
%%EXEND

\z


In SLI, the subject marker is determined by a preverbal semantically locative DP – this is \textit{semantic} locative inversion \citep{Buell2007}, as the preverbal locative is not marked as such by locative morphology. In \xref{bkm:Ref115794012}, the initial noun ‘clinic’ refers to a location but is in the non-locative noun class 7, and not marked by the locative suffix -\textit{ini}. The subject marker on the verb shows agreement with the noun in class 7 rather than a locative class or the default \textit{kû-.}

%%EAX
\ea
%%JUDGEMENT
%%LABEL
\label{bkm:Ref115794012}\label{bkm:Ref117491937}
%%CONTEXT
%%LINE1
%%LINE2
\gll
Kî-rîniki  gî-kî  i  kî-ij-ag-a  a-ekûrû  ba-ingî.\\
%%LINE3
7-clinic  7-\DEM.\PROX{} \FOC{} 7\SM{}-come-\HAB-\FV{} 2-woman  2-many\\
%%TRANS1
\glt
‘To this clinic come many women.’ \\
%%TRANS2
  \citep[7, glosses adapted]{BuellMuriungi2008}
%%EXEND

\z

For SLI to apply, the locative must be an argument of the verb. This is the case in \xref{bkm:Ref117491937} because motion verbs like ‘come’ and ‘go’ take a locative complement, but for other verbs an applicative extension is used, as in \xref{bkm:Ref117492511}. Note that this is still locative inversion (not instrument), as the plate is seen as a location, and the same sentence with ‘spoon’ would not be grammatical.

\ea
\label{bkm:Ref117492511}
\begin{xlist}
%%EAX
\exi{Q:}
%%JUDGEMENT
%%LABEL
%%CONTEXT
%%LINE1
%%LINE2
\gll
Kû-rî  wee  a-rî-îr-a  thaan-iini  îno?\\
%%LINE3
17\SM{}-be  1.\PRO{} 1\SM{}-eat-\APPL-\FV{} 9.plate=\LOC{} 9.\DEM.\PROX{}\\
%%TRANS1
\glt
‘Is there anyone who has eaten from this plate?’
%%TRANS2
%%EXEND

%%EAX
\exi{A:}
%%JUDGEMENT
%%LABEL
%%CONTEXT
%%LINE1
%%LINE2
\gll
Înu  î-kû-rî-îr-a  Kanyúá.\\
%%LINE3
9.\DEM.\PROX{}  9\SM-\PRS{}-eat-\APPL-\FV{} 1.Kanyua\\
%%TRANS1
\glt
‘This one has been eaten from by Kanyua (don’t use it).’
%%TRANS2
%%EXEND

\end{xlist}
\z

Scope interpretations show that the postverbal logical subject is in situ in a position below negation, in both DAI \xref{bkm:Ref94171916} and SLI \xref{bkm:Ref94171922}:

%%EAX
\ea
%%JUDGEMENT
%%LABEL
\label{bkm:Ref94171916}
%%CONTEXT
%%LINE1
Gûtíákwá rwaag\'{î} r\'{û}ûnthe.\\
%%LINE2
\gll
kû-ti-a-ku-a   rû-agî  rû-onthe\\
%%LINE3
17\SM-\NEG-\PST{}-die-\FV{} 11-mosquito  11-all\\
%%TRANS1
\glt
‘Not all mosquitoes died.’ (some are still alive)
%%TRANS2
%%EXEND

\z

%%EAX
\ea
%%JUDGEMENT
%%LABEL
\label{bkm:Ref94171922}
%%CONTEXT
%%LINE1
{Njîrá \'{î}no îtithííyágá antû bóonthe.}\\
%%LINE2
\gll
n-jira  îno  î-ti-thi-ag-a  a-ntû  ba-onthe\\
%%LINE3
9-path  9.\DEM.\PROX{} 9\SM-\NEG{}-go-\HAB-\FV{} 2-person  2-all\\
%%TRANS1
\glt
‘On this path not all people go (but some do).’
%%TRANS2
%%EXEND

\z


In tenses with an optional \textit{ni} marker (see \sectref{bkm:Ref111628808}), the marker is present in DAI for a thetic interpretation:

%%EAX
\ea
%%JUDGEMENT
%%LABEL
%%CONTEXT
(Context: You saw this and report it to the watchman so he can be alert.)\\
%%LINE1
Ndúkáaní í k\'{û}th\'{û}\'{û}ngîîré muntû.\\
%%LINE2
\gll
n-duka=ini  ni  kû-thûûngîr-ire  mu-ntû\\
%%LINE3
9-shop=\LOC{}  \FOC{} 17\SM{}-enter-\PFV{} 1-person\\
%%TRANS1
\glt
‘In the shop there entered somebody.’
%%TRANS2
%%EXEND

\z


When the \textit{ni} marker is absent, the focus is more on the postverbal subject, as seen in the question and answer in (\ref{shop}).

%%EAX
\ea
%%JUDGEMENT
%%LABEL
\label{shop}
%%CONTEXT
(Who entered the shop?)\\
%%LINE1
Ndúkáaní kûth\'{û}\'{û}ng\'{î}îré antû báîr\'{î}.\\
%%LINE2
\gll
n-duka=ini  kû-thûûngîr-ire  a-ntû  ba-îrî\\
%%LINE3
9-shop=\LOC{} 17\SM{}-enter-\PFV{} 2-person  2-two\\
%%TRANS1
\glt
‘In the shop there entered two people.’
%%TRANS2
%%EXEND

\z


Subject inversion is used when the logical subject is not the topic. This can be in thetic contexts, as illustrated in \xref{bkm:Ref90041020} for DAI and \xref{bkm:Ref94171751} for SLI, but also when the subject is in narrow focus, as shown below.

%%EAX
\ea
%%JUDGEMENT
%%LABEL
\label{bkm:Ref90041020}
%%CONTEXT
{(There was an accident on the road and…)}\\
%%LINE1
Í g\'{û}kwíré mûntû.\\
%%LINE2
\gll
ni  kû-ku-ire  mû-ntû\\
%%LINE3
\FOC{}  17\SM{}-die-\PFV{}  1-person \\
%%TRANS1
\glt
‘Somebody died.’
%%TRANS2
%%EXEND

\z

%%EAX
\ea
%%JUDGEMENT
%%LABEL
\label{bkm:Ref94171751}
%%CONTEXT
%%LINE1
Mûthítû n’ûágwîîre m\'{î}t\'{î}.\\
%%LINE2
\gll
mû-thitû  ni  û-a-gû-ire  mî-tî\\
%%LINE3
3-forest \FOC{} 3\SM{}-\PST{}-fall-\PFV{} 4-tree\\
%%TRANS1
\glt
‘In the forest trees fell.’
%%TRANS2
%%EXEND

\z

Three tests for focus all show that the postverbal logical subject in DAI can be focused: the inherently focal wh word is allowed, as in \xref{bkm:Ref90039199}, the answer to a content question too, as in (\ref{bkm:Ref90039957}--\ref{bkm:Ref90039959}), as well as a subject modified by the exhaustive focus particle ‘only’ as in \xref{bkm:Ref90039972} and \xref{bkm:Ref90039973}. As mentioned earlier, in those conjugations that allow an optional preverbal marker \textit{ni}-, the marker cannot be present when the postverbal subject is in focus; compare to the presence of \textit{ni} with a thetic interpretation above (see \sectref{bkm:Ref111628808} on \textit{ni}).

%%EAX
\ea
%%JUDGEMENT
%%LABEL
\label{bkm:Ref90039199}
%%CONTEXT
%%LINE1
Gûkuíré ûû?\\
%%LINE2
\gll
kû-ku-ire  ûû\\
%%LINE3
17\SM{}-die-\PFV{} who \\
%%TRANS1
\glt
‘Who died?’
%%TRANS2
%%EXEND

\z

%%EAX
\ea
%%JUDGEMENT
%%LABEL
\label{bkm:Ref90039957}
%%CONTEXT
(Someone entered this place, was it you? OR Who entered this place?)\\
%%LINE1
K\'{û}th\'{û}\'{û}ng\'{î}\'{î}re kaána.\\
%%LINE2
\gll
kû-thûûngîr-ire  ka-ana\\
%%LINE3
17\SM{}-enter-\PFV{} 12-child \\
%%TRANS1
\glt
‘There entered a child.’
%%TRANS2
%%EXEND

\z

%%EAX
\ea
%%JUDGEMENT
%%LABEL
\label{bkm:Ref90039958}
%%CONTEXT
{(There were visitors here and I can’t see what they brought us; what was brought?)}\\
%%LINE1
Kûréétirwe conda na mîgááté.\\
%%LINE2
\gll
kû-reet-w-ire  conda  na  mî-gaate\\
%%LINE3
17\SM{}-bring-\PASS-\PFV{} 9.soda  and  3-bread\\
%%TRANS1
\glt
‘There was brought soda and bread.’
%%TRANS2
%%EXEND

\z
\pagebreak
%%EAX
\ea
%%JUDGEMENT
%%LABEL
\label{bkm:Ref90039959}
%%CONTEXT
(Who was laughing?)\\
%%LINE1
Í g\'{û}kúthekaga mbiti.\\
%%LINE2
\gll
ni  kû-ku-thek-ag-a  m-biti\\
%%LINE3
\FOC{} 17\SM-\PRS{}-laugh-\HAB-\FV{} 9-hyena\\
%%TRANS1
\glt
‘A/the hyena laughed.’
%%TRANS2
%%EXEND

\z

%%EAX
\ea
%%JUDGEMENT
%%LABEL
\label{bkm:Ref90039972}
%%CONTEXT
%%LINE1
(\textsuperscript{\#}I) Kûthûûngîîre kaána akî.\\
%%LINE2
\gll
{} kû-thûûngîr-ire ka-ana akî\\
%%LINE3
{} 17\SM{}-enter-\PFV{} 12-child only\\
%%TRANS1
\glt
‘There entered a child only.’
%%TRANS2
%%EXEND

\z

%%EAX
\ea
%%JUDGEMENT
%%LABEL
\label{bkm:Ref90039973}
%%CONTEXT
%%LINE1
Kûratóórookire ng\'{û}k\'{û} akî.\\
%%LINE2
\gll
kû-ra-toorook-ire  n-gûkû  akî\\
%%LINE3
17\SM{}-\YPST{}-escape-\PFV{} 10-chicken  only\\
%%TRANS1
\glt
‘There escaped chickens only.’
%%TRANS2
%%EXEND

\z

\citegen{BuellMuriungi2008} manuscript shows that locative inversion is only accepted with passive \xref{bkm:Ref90039958}, unaccusative \xref{bkm:Ref90039199} and unergative intransitive predicates \xref{bkm:Ref90039959}, not with transitives \xref{bkm:Ref94458432}. In our limited data, unergatives are often frowned upon and preferably expressed with a passive.

 
\ea
\label{bkm:Ref94458432} 
\citep[8, glosses adapted]{BuellMuriungi2008}
%%EAX
\ea
%%JUDGEMENT
[]{
%%LABEL
%%CONTEXT
%%LINE1
Arîmi i baendagia nyani thokooni îno.\\
%%LINE2
\gll
a-rîmi  ni  ba-endi-ag-a  nyani  thoko=ini  î-no.\\
%%LINE3
2-farmer \FOC{} 2\SM{}-sell-\HAB-\FV{} 10.vegetable  9.market=\LOC{} 9-\DEM.\PROX{}\\
%%TRANS1
\glt
‘Farmers sell vegetables at this market.’
%%TRANS2
}
%%EXEND

%%EAX
\ex
%%JUDGEMENT
[*]{
%%LABEL
%%CONTEXT
%%LINE1
Thokoni  îno  i  kûendagia  arîmi  nyani.\\
%%LINE2
\gll
thoko=ini  î-no  ni  kû-endi-ag-a  a-rîmi  nyani.\\
%%LINE3
9.market=\LOC{} 9-\DEM.\PROX{} \FOC{} 17\SM{}-sell-\HAB-\FV{} 2-farmer  10.vegetable\\
%%TRANS1
\glt
int. ‘At this market farmers sell vegetables.’
%%TRANS2
}
%%EXEND

%%EAX
\ex
%%JUDGEMENT
[*]{
%%LABEL
%%CONTEXT
%%LINE1
Thoko  îno  n'  îendagia  arîmi  nyani.\\
%%LINE2
\gll
thoko  î-no  ni  î-endi-ag-a  arîmi  nyani.\\
%%LINE3
9.market  9-\DEM.\PROX{} \FOC{} 9\SM{}-sell-\HAB-\FV{} 2-farmer  10.vegetable\\
%%TRANS1
\glt
int. ‘At this market farmers sell vegetables.’
%%TRANS2
}
%%EXEND

\z
\z

\ea
%%LABEL
\citep[7]{BuellMuriungi2008}
\judgewidth{\textsuperscript{??}}
%%EAX
\ea
%%JUDGEMENT
[]{
%%CONTEXT
%%LINE1
%%LINE2
\gll
Tw-ana  tû-kûrû  i  tû-ceth-ag-a  kî-eni=ini.\\
%%LINE3
13-child  13-old \FOC{} 13\SM{}-play-\HAB-\FV{} 7-field=\LOC{}\\
%%TRANS1
\glt
‘Older children play in the field.’
%%TRANS2
}
%%EXEND

%%EAX
\ex
%%JUDGEMENT
[]{
%%LABEL
%%CONTEXT
%%LINE1
%%LINE2
\gll
Kî-eni=ini  i  gû-ceth-ag-a  tw-ana  tû-kûrû.\\
%%LINE3
7-field=\LOC{} \FOC{} 17\SM{}-play-\HAB-\FV{} 13-child  13-old\\
%%TRANS1
\glt
‘In the field play older children.’
%%TRANS2
}
%%EXEND

%%EAX
\ex
%%JUDGEMENT
[\textsuperscript{??}]{
%%LABEL
%%CONTEXT
%%LINE1
%%LINE2
\gll
Kî-eni  i  gî-ceth-ag-a  tw-ana  tû-kûrû.\\
%%LINE3
7-field \FOC{} 7\SM{}-play-\HAB-\FV{} 13-child  13-old\\
%%TRANS1
\glt
‘In the field play older children.’
%%TRANS2
}
%%EXEND

\z
\z

However, Buell and Muriungi report that unergatives fare better with an applicative morpheme. In this case, the intuition is that the applicative encodes an explicit locative interpretation, while the locative sense is implicit in the passive.


%%EAX
\ea
\citep[10]{BuellMuriungi2008}
%%JUDGEMENT
%%LABEL
%%CONTEXT
%%LINE1
\ea
%%LINE2
\gll
Tw-ana  tû-kûrû  i  tû-ceth-ag-îr-a  kî-eni=ini.\\
%%LINE3
13-child  13-old \FOC{} 13\SM{}-play-\HAB-\APPL-\FV{} 7-field=\LOC{}\\
%%TRANS1
\glt
‘Old children play in the field.’
%%TRANS2
%%EXEND


%%EAX
\ex
%%JUDGEMENT
%%LABEL
%%CONTEXT
%%LINE1
%%LINE2
\gll
Kî-eni=ini  i  gû-ceth-ag-îr-a  tw-ana  tû-kûrû.\\
%%LINE3
7-field=\LOC{} \FOC{}  17\SM{}-play-\HAB-\APPL-\FV{} 13-child  13-old\\
%%TRANS1
\glt
‘On the field play older children.’
%%TRANS2
%%EXEND

%%EAX
\ex
%%JUDGEMENT
%%LABEL
%%CONTEXT
%%LINE1
%%LINE2
\gll
Kî-eni  i  gî-ceth-ag-îr-a  tw-ana  tû-kûrû.\\
%%LINE3
7-field \FOC{}  7\SM{}-play-\HAB-\APPL-\FV{} 13-child  13-old\\
%%TRANS1
\glt
lit. ‘The field plays older children.’
%%TRANS2
%%EXEND

\z
\z

The applicative can be seen even with an unaccusative, illustrated for SLI in \xref{bkm:Ref117492221} and DAI in \xref{bkm:Ref117492230}, making the locative into an argument of the verb (see also \xref{bkm:Ref117492511} above).

%%EAX
\ea
%%JUDGEMENT
%%LABEL
\label{bkm:Ref117492221}
%%CONTEXT
%%LINE1
Barabárá n’îrákw*(íír)ííre ndúrú.\\
%%LINE2
\gll
barabara  ni  î-ra-ku-ir-ire  n-duru\\
%%LINE3
9.road \FOC{} 9\SM{}-\YPST{}-die-\APPL-\PFV{} 9-squirrel\\
%%TRANS1
\glt
‘On the road died a squirrel.’\footnote{An alternative non-sensical interpretation has the road as the benefactive ‘A squirrel died for the road’.}
%%TRANS2
%%EXEND

\z

%%EAX
\ea
%%JUDGEMENT
%%LABEL
\label{bkm:Ref117492230}
%%CONTEXT
%%LINE1
Barabáráani í kûráku*(íír)ííre ndúrú.\\
%%LINE2
\gll
barabara=ini  ni  kû-ra-ku-ir-ire  n-duru\\
%%LINE3
9.road=\LOC{} \FOC{} 17\SM{}-\YPST{}-die-\APPL{}-\PFV{} 9-squirrel\\
%%TRANS1
\glt
‘On the road there died a squirrel.’
%%TRANS2
%%EXEND

\z


Similar to inversion constructions, passives also encode the Theme as the preverbal topic and thereby structure the information. Two differences with inversion constructions are that 1) the verb is marked as passive by the suffix -\textit{w}-, and 2) the subject can be omitted but if it is present, it is marked by \textit{ni} (here functioning as ‘by’). The passive is illustrated in a sequence from a folktale in \xref{bkm:Ref131610085}.

%%EAX
\ea
%%JUDGEMENT
%%LABEL
\label{bkm:Ref131610085}
%%CONTEXT
(When Hare came back, he checked and found there was no guinea fowl. He asked Hyena: ‘Hyena, where is the guinea fowl?’)\\
%%LINE1
Mbítí yám\'{î}\'{î}ra: “N\'{î}r\'{î}r\textbf{w}é í mwánki m\'{û}nyáánya. Nkáánga n\'{î}r\'{î}îr\textbf{w}é í mwanki.”\\
%%LINE2
\gll
m-biti  î-a-mî-îr-a  ni  î-rî-w-ire  ni  mû-anki  mû-nyanya n-kanga  ni  î-rî-îr-w-e  ni  mû-anki \\
%%LINE3
9-hyena 9\SM-\PST-9\OM{}-tell-\FV{} \FOC{} 9\SM{}-eat-\PASS-\PFV{} \FOC{} 3-fire  1-friend 9-guinea.fowl \FOC{} 9\SM{}-eat-\APPL-\PASS-\FV{} \FOC{} 3-fire \\
%%TRANS1
\glt
‘Hyena told him, it was eaten by the fire, my friend. The guinea fowl was eaten by the fire.’
%%TRANS2
%%EXEND

\z

In summary, Kîîtharaka shows a preference for topical information to precede the verb and non-topical referents to follow, which accounts for left-dislocations and subject inversion. On the other hand, there is no dedicated focus position, and focused phrases do occur preverbally when preceded by \textit{ni}-, as further discussed in the section on clefts (\sectref{bkm:Ref117495353}). The possible interpretations of the canonical SVO order depend on the conjugation of the verb involving the focus particle \textit{ni}- as discussed in \sectref{bkm:Ref111628808}. For now, we first turn to additional marking of topics.

\section{Topic marking}
\label{bkm:Ref115790636}
There are two types of structures that show evidence of topic marking in Kîîtharaka. One involves a topic marker -\textit{o} that agrees with the topical NP in class, and the other involves a combination of \textit{na} ‘with, and’ and the same -\textit{o}. The two are illustrated in \xref{bkm:Ref94509521} and \xref{bkm:Ref115795536}, respectively. This marker is also found in Kirundi and Rukiga with similar functions, see \textcite{chapters/kirundi} and \textcite{chapters/rukiga}.

%%EAX
\ea
%%JUDGEMENT
%%LABEL
\label{bkm:Ref94509521}
%%CONTEXT
%%LINE1
Ma-búkú   mó   í-má-\'{û}r-iré.\\
%%LINE2
\gll
ma-buku  ma-o  ni  ma-ur-ire\\
%%LINE3
6-book  6-\PRO{} \FOC{}  6\SM{}-lost-\PFV{}\\
%%TRANS1
\glt
‘As for the books, they are lost.’
%%TRANS2
%%EXEND

\z

%%EAX
\ea
%%JUDGEMENT
%%LABEL
\label{bkm:Ref115795536}\label{bkm:Ref119738623}
%%CONTEXT
(Context: The dog would climb to see whether the frog was hiding there.)\\
%%LINE1
Kaána \textbf{nakoó} gácééke mûrinyáani í mbí îrî óogó.\\
%%LINE2
\gll
ka-ana  na=ka-o  ka-ceek-e  mû-rinya=ini  ni  mbii  î-rî  a-ogo\\
%%LINE3
12-child  and=12-\PRO{} 12\SM{}-check-\SBJV{} 3-hole=\LOC{} \FOC{} what  9-be  16-\DEM{}\\
%%TRANS1
\glt
‘The child on the other hand, would check in the hole what is there.’
%%TRANS2
%%EXEND

\z


The topic marker \textit{-o} has four context-driven interpretations, namely; polarity focus/verum, contrastive, intensive, and depreciative. These, together with their contexts are illustrated in \xref{bkm:Ref115795558} to \xref{bkm:Ref115795560}, respectively.

%%EAX
\ea
%%JUDGEMENT
%%LABEL
\label{bkm:Ref115795558}
%%CONTEXT
(Is it really true that Brian lost the books? I don’t trust what Edith says.)\\
%%LINE1
Mabúkú mó\footnote{A reviewer suggested that \textit{mabuku} is a left-dislocated topic and \textit{mo} functions as the argument rather than a topic marker. While this has likely been the case in a previous stage of the language, the fact that a prosodic break (if one is present) would be following \textit{mo} rather than \textit{mabuku} shows that currently \textit{mo} functions as a contrastive topic marker and not a pronoun in this sentence.} imauriré.\\
%%LINE2
\gll
ma-buku  ma-o ni  ma-ur-ire\\
%%LINE3
6-book 6-\PRO{}  \FOC{}  6\SM{}-lost-\PFV{}\\
%%TRANS1
\glt
‘As for the books, they are lost.’   \jambox*{[polarity focus]}
%%TRANS2
%%EXEND

\z

%%EAX
\ea
%%JUDGEMENT
%%LABEL
\label{bkm:Ref115795560}
%%CONTEXT
(Did you water the cows and the sheep? Well, the cows did not drink water…)\\
%%LINE1
\'{Î}ndî cíó ng'oóndu icinyúiré. \\
%%LINE2
\gll
îndî  ci-o  n-g’oondu  ni  ci-nyu-ire\\
%%LINE3
but  10-\PRO{} 10-sheep \FOC{} 10\SM{}-drink-\PFV{}\\
%%TRANS1
\glt
‘…but as for the sheep they drank.’ \jambox*{[contrastive]}
%%TRANS2
%%EXEND

\z

%%EAX
\ea
%%JUDGEMENT
%%LABEL
%%CONTEXT
(Has it rained that much?)\\
%%LINE1
Mbúra yoó îkuúra.\\
%%LINE2
\gll
m-bura  î-o  î-kû-ur-a\\
%%LINE3
9-rain  9-\PRO{} 9\SM-\PRS{}-rain-\FV{}\\
%%TRANS1
\glt
‘It has really rained.’ \jambox*{[intensive]}
%%TRANS2
%%EXEND

\z

%%EAX
\ea
%%JUDGEMENT
%%LABEL
\label{bkm:Ref119738548}
%%CONTEXT
(Context: It has rained but you think it will not help much because there are lots of weeds in the farm that prevent you from planting early.)\\
%%LINE1
  Mbúra yoó îkuúra.\\
%%LINE2
\gll
m-bura  î-o  î-kû-ur-a\\
%%LINE3
9-rain  9-\PRO{} 9\SM-\PRS{}-rain-\FV{}\\
%%TRANS1
\glt
‘It has rained but...’       \jambox*{[depreciative]}
%%TRANS2
%%EXEND

\z

It is possible to move the marker from the pre-verbal to the post-verbal position and still maintain the various interpretations, as in the variant of \xref{bkm:Ref119738548} in \xref{bkm:Ref119738559}.

%%EAX
\ea
%%JUDGEMENT
%%LABEL
\label{bkm:Ref119738559}
%%CONTEXT
%%LINE1
Mbúra îkuúra   yoó.\\
%%LINE2
\gll
m-bura  î-kû-ur-a  î-o\\
%%LINE3
9-rain 9\SM-\PRS{}-rain-\FV{}  9-\PRO{}\\
%%TRANS1
\glt
‘It has rained but...’
%%TRANS2
%%EXEND

\z

It is interesting to note that these interpretations are the exact same as those found in topic doubling, discussed in \sectref{bkm:Ref115797355}.

Second, the combination of \textit{na} ‘with, and’ and the pronominal clitic in -\textit{o} is used as a contrastive or shift topic marker in Kîîtharaka. We first discuss the other uses of \NA+{} \PRO{} before returning to its use as a contrastive/shift topic marker, so that its origin and multifunctionality are clear. We find the combination for pronominalised complements of the preposition \textit{na} ‘with’, for example in the instrumental \xref{bkm:Ref105385725} combination also features as a resumptive pronoun in clefts and relative clauses, as seen in \xref{bkm:Ref105385863}.

%%EAX
\ea
%%JUDGEMENT
%%LABEL
\label{bkm:Ref105385725}
%%CONTEXT
%%LINE1
\label{bkm:Ref94509818}Índaát\'{û}líre ǹk\'{û} baabá ábuá \textbf{nació} mwaanki.\\
%%LINE2
\gll
ni  nda-ra-tul-ire  n-kû  baaba  a-bu-a  na=ci-o  mû-anki \\
%%LINE3
\FOC{} 1\SG.\SM-\YPST{}-split-\PFV{} 10-firewood  1.father  1\SM{}-light-\FV{} with=10-\PRO{} 3-fire \\
%%TRANS1
\glt
‘I split firewood for father to light the fire with it.’
%%TRANS2
%%EXEND

\z

%%EAX
\ea
%%JUDGEMENT
%%LABEL
%%CONTEXT
(Does Thomas have a cat? + QUIS picture of Thomas with a cat and a rabbit.)\\
%%LINE1
Ii ár\'{î} \textbf{nakó} naká wána kayúg\'{u}.\\
%%LINE2
\gll
ii  a-rî  na=ka-o  na=ka  wana  ka-yugu\\
%%LINE3
yes  1\SM{}-be  with=12-\PRO{} and=\POL{} even  12-rabbit\\
%%TRANS1
\glt
‘Yes, he has one, and also a rabbit.’
%%TRANS2
%%EXEND

\z

%%EAX
\ea
%%JUDGEMENT
%%LABEL
\label{bkm:Ref105385863}\label{bkm:Ref94509878}
%%CONTEXT
(Does William have four melons? + QUIS picture of William with three melons.)\\
%%LINE1
Ar\'{î}, i mathátû arî \textbf{namó}.\\
%%LINE2
\gll
ari  ni  ma-thatû  a-rî  na=ma-o\\
%%LINE3
no \FOC{} 6-three  1\SM{}-be  with=6-\PRO{}\\
%%TRANS1
\glt
‘No, he has three.’ (lit. ‘it is three that he is with’)
%%TRANS2
%%EXEND

\z

The pronoun with \textit{na} can also be used with an additive meaning, with a coreferential noun phrase as in \xref{bkm:Ref94510139} or without it as in \xref{bkm:Ref94510233}. The same is found with \textit{wana} ‘even’ + pro, as in \xref{bkm:Ref94510132}.

%%EAX
\ea
%%JUDGEMENT
%%LABEL
\label{bkm:Ref94510139}
%%CONTEXT
%%LINE1
Tóm \textbf{nawé} n’áráíníre.\\
%%LINE2
\gll
Tom  na=we  ni  a-ra-in-ire\\
%%LINE3
1.Tom  and=1.\PRO{} \FOC{} 1\SM{}-\YPST{}-dance-\PFV{}\\
%%TRANS1
\glt
‘Tom also danced.’
%%TRANS2
%%EXEND

\z

%%EAX
\ea
%%JUDGEMENT
%%LABEL
\label{bkm:Ref94510233}
%%CONTEXT
(The gazelle threw the child into the water. The little dog, because he was running, did not see the bank.) \\
%%LINE1
\textbf{Nakó} na kûgwa.\\
%%LINE2
\gll
na=ka-o  na  ku-gûa\\
%%LINE3
with=12-\PRO{} and  15-fall\\
%%TRANS1
\glt
‘He too fell.’
%%TRANS2
%%EXEND

\z

%%EAX
\ea
%%JUDGEMENT
%%LABEL
\label{bkm:Ref94510132}
%%CONTEXT
%%LINE1
Mûtúgí \textbf{wana wé} ár\'{î} na meetho mátuúne.\\
%%LINE2
\gll
Mûtugi  wana  we  a-rî  na  ma-itho  ma-tuune\\
%%LINE3
1.Mûtugi  even  1.\PRO{} 1\SM{}-be  with  6-eyes  6-red\\
%%TRANS1
\glt
‘Mûtugi among others/even he has brown eyes.’
%%TRANS2
%%EXEND

\z


We speculate that this additive use facilitates marking a contrastive topic. In \xref{bkm:Ref94510724}, we see a bridging context in which both interpretations are plausible: as an additive marker (not just the child but also the little dog), or as a shift topic marker (the topic shifting from the child to the little dog).

%%EAX
\ea
%%JUDGEMENT
%%LABEL
\label{bkm:Ref94510724}
%%CONTEXT
%%LINE1
Kaana kaugia, gakurû \textbf{nako} i kaugîîtie.\\
%%LINE2
\gll
ka-ana  ka-ugi-a  ka-kurû  na=ka-o  ni  ka-ugi-îte\\
%%LINE3
12-child  12\SM{}-run-\FV{} 12-dog  with=12-\PRO{} \FOC{} 12\SM{}-run-\STAT.\PFV{}\\
%%TRANS1
\glt
‘The child ran, while the little dog was also running.’
%%TRANS2
%%EXEND

\z


The \NA+{} \PRO{} marker can only mark the second topic (the shift or contrast), as indicated by the felicitous and infelicitous placement of the marker in \xref{bkm:Ref94510956}.

%%EAX
\ea
%%JUDGEMENT
%%LABEL
\label{bkm:Ref94510956}
%%CONTEXT
(What did Souza do with the beans and the carrots?)\\
%%LINE1
Mboócó (*\textbf{nació}) n’ árarugire, kaaráti \textbf{nacíó}, éendíá.\\
%%LINE2
\gll
m-booco  na=ci-o  ni  a-ra-gur-ire  kaarati  na=ci-o  a-endi-a\\
%%LINE3
10-bean  and=10-\PRO{}  \FOC{} 1\SM{}-\YPST{}-cook-\PFV{} 10.carrot  and=10-\PRO{}  1\SM{}-sell-\FV{}\\
%%TRANS1
\glt
‘The beans he cooked and the carrots he sold.’
%%TRANS2
%%EXEND

\z


In \xref{bkm:Ref94510498}, the use of \textit{namo} is only allowed if we have been talking about other fruits before, contrasting the mangoes to bananas and oranges, for example.

%%EAX
\ea
%%JUDGEMENT
%%LABEL
\label{bkm:Ref94510498}
%%CONTEXT
%%LINE1
Méémbé \textbf{namó} mbeendéété márá méérú.\\
%%LINE2
\gll
ma-embe  na=ma-o  n-end-îte  ma-ra  ma-eru\\
%%LINE3
6-mango  and=6-\PRO{} 1\SG.\SM{}-like-\STAT.\PFV{} 6-\RM{} 6-ripe\\
%%TRANS1
\glt
‘And as for mangoes, I like ripe ones.’
%%TRANS2
%%EXEND

\z


Note that in these contrastive topic examples, the marker is syntactically optional, unlike in \xxref{bkm:Ref94509818}{bkm:Ref94509878}; and an optional prosodic break may follow the marker, e.g. after \textit{namo} in \xref{bkm:Ref94510498}  as judged by the first author, and the lengthened syllable (\textit{nakoó}) in \xref{bkm:Ref119738623} above.

In its use as a shift topic marker, the \NA+{} \PRO{} marker is also used as a narrative-structuring device, switching between different referents. This was illustrated in \xref{bkm:Ref94509521} above, and can again be seen in \xref{bkm:Ref94511291}: Hyena and Hare are the protagonists, and the story alternates between the actions of the one and those of the other.

%%EAX
\ea
%%JUDGEMENT
%%LABEL
\label{bkm:Ref94511291}
%%CONTEXT
(Hare got hold of Hyena and beat him. Hyena cries ‘My husband, the black one, leave me! Leave me alone!’)\\
%%LINE1
Kay\'{û}g\'{û} \textbf{nakó} n’wa kûm\'{î}rumia mmá! mmá!\\
%%LINE2
\gll
ka-yûgû  na=ka-o  ni w-a  ku-mî-rum-i-a  mma    mma\\
%%LINE3
12-hare  and=12-\PRO{}  \COP{} 1-\CONN{}  15-9\OM{}-beat-\SC-\FV{} \IDEO{}    \IDEO{}\\
%%TRANS1
\glt
‘But Hare kept on beating him: whack! whack!’
%%TRANS2
%%EXEND

\z


Summarising this section, Kîîtharaka can (but does not need to) mark contrastive topics by the marker -\textit{o}, which follows the topic NP and agrees with it in noun class. The context determines whether the interpretation results in just a contrast on the topical referent, or a polarity focus/verum, intensive, or depreciative interpretation of the clause. When the same proposition applies to a second topic, the marker -\textit{o} is preceded by \textit{na} ‘and’.

\section{Predicate doubling}
\label{bkm:Ref115942967}
\citet{GüldemannFiedler2019} show that across Bantu languages there are three constructions in which the infinitive and the inflected form of the verb can co-occur in the same sentence: the infinitive can appear in a cleft \xref{bkm:Ref93997547}, in situ in a postverbal position \xref{bkm:Ref93997562}, and as a left-peripheral topic \xref{bkm:Ref93997557}.

%%EAX
\ea
%%JUDGEMENT
%%LABEL
\label{bkm:Ref93997547}
%%CONTEXT
%%LINE1
I  kûrííngá  t\'{û}rííngiré  ng’óombé, tûtíracíthaika.\hfill\relax[cleft]\\
%%LINE2
\gll
ni  kû-riinga  tû-riing-ire  ng’-oombe   tû-ti-ra-ci-thaik-a  \\
%%LINE3
\FOC{} 15-hit 1\PL.\SM{}-hit-\PFV{} 10-cow 1\PL.\SM-\NEG-\YPST{}-10\OM{}-tie-\FV{}\\  
%%TRANS1
\glt
‘We \textit{hit} the cows, we didn’t tie them.’
%%TRANS2
%%EXEND

\z

%%EAX
\ea
%%JUDGEMENT
%%LABEL
\label{bkm:Ref93997562}
%%CONTEXT
%%LINE1
Bak\'{î}báthíríá  kûbathíría.\\
%%LINE2
\gll
ba-kî-ba-thiri-a  kû-ba-thiria  \\
%%LINE3
2\SM-\DEP-2\OM{}-finish-\FV{} 15-2\OM{}-finish\\ \jambox*{[in situ]}
%%TRANS1
\glt
‘They completely finished them.’
%%TRANS2
%%EXEND

\z

%%EAX
\ea
%%JUDGEMENT
%%LABEL
\label{bkm:Ref93997557}
%%CONTEXT
%%LINE1
Kûrúgǎ  nkáárúga.\\
%%LINE2
\gll
kû-ruga  n-kaa-rug-a \\
%%LINE3
15-cook 1\SG.\SM-\FUT{}-cook-\FV{}\\ \jambox*{[topic]}
%%TRANS1
\glt
‘I will indeed cook.’
%%TRANS2
%%EXEND

\z


Kîîtharaka shows all three types of predicate doubling, which seems to be unique among the Bantu languages studied in our project. We present the formal and functional properties of each in turn. For completeness, we mention that predicate doubling only occurs with the infinitive, not with other nominalised forms of the predicate (as in Kirundi or Rukiga, for example).

\subsection{Cleft doubling}

In a cleft doubling structure, a preverbal infinitival verb is preceded by a focus marker/copula. The structure looks like a basic cleft with a non-infinitival noun (see \sectref{bkm:Ref117495353}; thus, alternatives are triggered for the clefted constituent, in this case a predicate. The structure has a state-of-affairs focus reading (focus on the verb itself). An object may follow the inflected verb and have the same SoA reading, as in \xref{bkm:Ref115797598}.\largerpage[-1]\pagebreak

%%EAX
\ea
%%JUDGEMENT
%%LABEL
%%CONTEXT
(Context: We were supposed to wash the child, apply lotion, and dress him/her.)\\
%%LINE2a
I  kûtháámbíá  t\'{û}káthaambiiríé.\\
%%LINE2b
\gll
{ni  kû-thaamb-i-a  t\'{û}-ka-thaamb-i-ire}\\
%%LINE3
\FOC{} 15-clean-\IC{}-\FV{} 1\PL.\SM-12\OM{}-clean-\IC{}-\PFV{}\\
%%TRANS1
\glt
‘We only \textit{washed} him/her (the child).’
%%TRANS2
%%EXEND

\z
%%EAX
\ea
%%JUDGEMENT
%%LABEL
\label{bkm:Ref115797598}
%%CONTEXT
%%LINE1
Í kûnywá bánywíiré kawá... \/{î}ndî batírátûûra.\\
%%LINE2
\gll
ni kû-nyua  ba-nyu-ire  kawa  îndî  ba-ti-ra-tûûr-a\\
%%LINE3
\FOC{} 15-drink   2\SM{}-drink-\PFV{} coffee  but 2\SM-\NEG-\YPST{}-pour-\FV{}\\
%%TRANS1
\glt
‘They drank the coffee, but they didn't pour it.’
%%TRANS2
%%EXEND

\z
The infinitive may also contain an object, as in \xref{bkm:Ref115797559}, resulting in VP focus, not SoA focus, as is clear from the following contrasting clause. 

%%EAX
\ea
%%JUDGEMENT
%%LABEL
\label{bkm:Ref115797559}
%%CONTEXT
(Context: We were expected to hit the cows and wash clothes.)\\
%%LINE1
%%LINE2
\gll
{Í  kû-ríínga  ng’óómbé  t\'{û}-rííng-ire;  tû-tí-rá-bûûr-a  n-gúo.}\\
%%LINE3
\FOC{} 15-hit  10-cow 1\PL.\SM{}-hit-\PFV{} 1\PL.\SM-\NEG-\YPST{}-wash-\FV{} 10-cloth\\
%%TRANS1
\glt
‘We hit the cows; we didn’t wash clothes.’
%%TRANS2
%%EXEND

\z



\subsection{In situ doubling}

For in situ doubling, the infinitive follows inflected verb. In tenses with an optional \textit{ni} marker, the marker must be absent, placing the focus on the infinitive (see Sections~\ref{bkm:Ref115798057} and~\ref{bkm:Ref111628808} for postverbal focus and the absence of \textit{ni} marking on the verb marking focus, respectively).

%%EAX
\ea
%%JUDGEMENT
%%LABEL
%%CONTEXT
(Did they sing/dance well?)\\
%%LINE1
(\textsuperscript{\#}Í) báráíníre kwiína.\\
%%LINE2
\gll
ni  ba-ra-in-ire  kû-ina\\
%%LINE3
\FOC{}  2\SM-\YPST{}-dance-\PFV{} 15-dance\\
%%TRANS1
\glt
‘They really danced/sang.’
%%TRANS2
%%EXEND

\z

With a transitive predicate, the infinitive can follow the object but cannot precede it, as in \xref{bkm:Ref115797716}.\largerpage[-1]\pagebreak

%%EAX
\ea
%%JUDGEMENT
%%LABEL
\label{bkm:Ref115797716}
%%CONTEXT
%%LINE1
Twííníré (kîbûco)\footnote{A traditional Tharaka romantic song/dance.} kwíína.\\
%%LINE2
\gll
tû-in-ire  kîbûco  kû-ina\\
%%LINE3
1\PL.\SM{}-sing-\PFV{} kîbûco  15-sing\\
%%TRANS1
\glt
‘We really sang (\textit{kîbuco}).’
%%TRANS2
%%EXEND

\z

In the above examples, the in-situ doubling is used in a polarity-focus context. Another possible interpretation is a high degree of doing the action, with alternatives being lesser ways of doing the action, as in \xref{bkm:Ref118553700}. Some contexts, as in \xref{bkm:Ref118553738}, may also allow state-of-affairs focus.

%%EAX
\ea
%%JUDGEMENT
%%LABEL
\label{bkm:Ref118553700}
%%CONTEXT
(How did Mary do in the dancing competition?)\footnote{The verb \textit{kûina} means both ‘dance’ and ‘sing’; we translate it is as appropriate for the context.}\\
%%LINE1
Araíníre kwíína (kúngwa).\\
%%LINE2
\gll
a-ra-in-ire  kû-ina  (ku-ngwa).\\
%%LINE3
1\SM{}-\YPST{}-sing-\PFV{} 15-sing  {\db}15-self\\
%%TRANS1
\glt
‘She danced amazingly well.’ (nearing professional levels)
%%TRANS2
%%EXEND

\z

%%EAX
\ea
%%JUDGEMENT
%%LABEL
\label{bkm:Ref118553738}
%%CONTEXT
(About Bible verse Numbers 21:3: And what did the Israelites do to the enemy camp when they found them?)\\
%%LINE1
%%LINE2
\gll
Ba-k\'{î}-bá-thírí-á  kû-ba-thírí-a.\\
%%LINE3
2\SM{}-\DEP{}-2\OM{}-finish-\FV{} 15-2\OM{}-finish\\
%%TRANS1
\glt
‘They completely finished them.’
%%TRANS2
%%EXEND

\z

\subsection{Topic doubling}
\label{bkm:Ref115797355}
In topic doubling, the infinitive occurs at the left edge of the sentence. This can be understood the same way as topicalization of a simple noun: it is simply an infinitive (noun) that is placed in the left periphery as a topic.

%%EAX
\ea
%%JUDGEMENT
%%LABEL
%%CONTEXT
(How have I performed?)\\
%%LINE1
Kwííná ûkûíina…\\
%%LINE2
\gll
kû-iina  û-kû-in-a\\
%%LINE3
15-sing 2\SG.\SM{}-sing-\FV{}\\
%%TRANS1
\glt
‘You have sung, but...’
%%TRANS2
%%EXEND

\z

Notably it can involve different predicates as in \xref{bkm:Ref115799130}, providing evidence that the initial infinitive is not a copy of the inflected verb but a simple topicalised phrase.

%%EAX
\ea
%%JUDGEMENT
%%LABEL
\label{bkm:Ref115799130}
%%CONTEXT
(Do you do sports? What kind of sporting activities do you do?)\\
%%LINE1
Gûcéétha,  í  mbúgagía.\\
%%LINE2
\gll
 kû-ceetha  ní  n-ugi-ag-a.\\
%%LINE3
  15-do.sports \FOC{}  1\SG.\SM{}-run-\HAB-\FV{}\\
%%TRANS1
\glt
‘As for sports, I run.’
%%TRANS2
%%EXEND

\z

The predicate being the topic, it is marked as not being the new information, which should therefore be elsewhere. What forms the new information depends on the arguments in the sentence: when an object is present, the focus is typically on the object or the VP as in \xref{bkm:Ref131610727}, and when followed by a subject cleft, the focus is the subject as in \xref{bkm:Ref149295131}. When there is no other constituent, focus on the polarity is a natural interpretation, as in \xref{bkm:Ref115798439}.

%%EAX
\ea
%%JUDGEMENT
%%LABEL
\label{bkm:Ref131610727}
%%CONTEXT
(Context: Someone insisting on catching of goats.)\\
%%LINE1
%%LINE2
\gll
Kû-gwáátá  n-tí-gwaat-a  m-b\'{û}ri.\\
%%LINE3
15-catch 1\SG.\SM-\NEG{}-catch-\FV{} 9-goat\\
%%TRANS1
\glt
‘I won’t catch a goat (but maybe something else).’
%%TRANS2
%%EXEND

\z
%%EAX
\ea
%%JUDGEMENT
%%LABEL
\label{bkm:Ref149295131}
%%CONTEXT
(Who is swimming? + QUIS picture of three people in different activities)\footnote{Note that this may not be the most natural response, which would be simply the cleft. The use of the topical material depends on whether the speaker wants to overtly express the topic or not.}\\
%%LINE1
%%LINE2
\gll
Kû-butîrá  í  mw-aáná  á-kû-butîr-a. \\
%%LINE3
15-swim \FOC{} 1-child 1\SM-\PRS{}-swim-\FV{}\\
%%TRANS1
\glt
‘As for swimming, it’s the child who is swimming.’
%%TRANS2
%%EXEND

\z


%%EAX
\ea
%%JUDGEMENT
%%LABEL
\label{bkm:Ref115798439}
%%CONTEXT
(Context: Someone is doubting whether the teachers danced.)\\
%%LINE1
Kûíná (arím\'{û}) í baííníré!\\
%%LINE2
\gll
kû-ina  a-rimû  ni  ba-in-ire\\
%%LINE3
15-dance  2-teacher \FOC{}  2\SM{}-dance-\PFV{}\\
%%TRANS1
\glt
‘(The teachers) they did dance!’
%%TRANS2
%%EXEND

\z

%%EAX
\ea
%%JUDGEMENT
%%LABEL
%%CONTEXT
(How will they manage to graze the cows and all those sheep?)\\
%%LINE1
%%LINE2
\gll
Kû-r\'{î}\'{î}thía  ba-káá-r\'{î}\'{î}thi-a  ng’-oóndu,  ng’-óómbě  ba-káá-thaik-a.\\
%%LINE3
15-graze 2\SM-\FUT{}-graze-\FV{} 10-sheep  10-cow 2\SM{}-tie-\FV{}\\
%%TRANS1
\glt
‘They will graze the sheep; the cows they will tie.’
%%TRANS2
%%EXEND

\z


Topic doubling is also very naturally used in contexts where the infinitive is contrasted to other actions that are implied not to be carried out.

%%EAX
\ea
%%JUDGEMENT
%%LABEL
%%CONTEXT
(I hear you laughed at him and beat him up.)\\
%%LINE1
Kûthéká í t\'{û}thekiré... \'{î}nd\'{î} tûtírám\`{û}ríínga.\\
%%LINE2
\gll
kû-theka  ni  tû-thek-ire  înî  tû-ti-ra-mû-riing-a\\
%%LINE3
15-laugh \FOC{} 1\PL.\SM{}-laugh-\PFV{} but 1\PL.\SM-\NEG-\YPST-1\OM{}-hit-\FV{}\\
%%TRANS1
\glt
We did laugh (admittedly), but we didn’t hit him.’
%%TRANS2
%%EXEND

\z


Two possible additional flavours of interpretation are possible: a depreciative and an intensive reading. In examples \xref{bkm:Ref115858918} to \xref{bkm:Ref115858998}, we have a depreciative interpretation: although it is true that the actions are carried out, they did not yield much value. In \xref{bkm:Ref119738989}, we have an intensive reading.

%%EAX
\ea
%%JUDGEMENT
%%LABEL
\label{bkm:Ref115858918}
%%CONTEXT
(I saw you weeded quite a large portion!)\footnote{A reviewer asked whether intonation is necessary to derive the intended meaning, as witnessed to some extent in Kikuyu. For Kîîtharaka, however, intonation does not license the intended meaning and a prosodic break at the end of the clause is optional.}\\
%%LINE1
%%LINE2
\gll
Kû-r\'{î}má  í  tû-rîm-iré.\\
%%LINE3
15-dig \FOC{} 1\PL.\SM{}-dig-\PFV{}\\
%%TRANS1
\glt
‘We weeded, but...’\\
%%TRANS2
‘Although we weeded...’ (it’s useless, the weeds will come back soon)
%%EXEND

\z

%%EAX
\ea
%%JUDGEMENT
%%LABEL
%%CONTEXT
(How can one kill a chicken?)\\
%%LINE1
%%LINE2
\gll
Kw-íítá,  nwá  w-iít-e,  \'{î}ndî  î-tí-kw-a  rûa.\\
%%LINE3
15-strangle  can 2\SG.\SM{}-strangle-\SBJV{} but 9\SM{}-\NEG{}-die-\FV{} soon\\
%%TRANS1
\glt
  ‘Well, you can strangle, but it doesn’t die quickly.’
%%TRANS2
%%EXEND

\z

%%EAX
\ea
%%JUDGEMENT
%%LABEL
\label{bkm:Ref115858998}
%%CONTEXT
(I liked your game, you really played!)\\
%%LINE1
%%LINE2
\gll
Gû-céethǎ  i  tû-ceeth-iré,  îndî  n-gúkúm-án-o  y-á-tû-túúny-a  gî-kóómbé. \\
%%LINE3
15-play \FOC{} 1\PL.\SM{}-play-\PFV{}   but  9-corrupt-\RECP-\NMLZ{} 9\SM-\PST-1\PL.\OM{}-snatch-\FV{} 7-cup \\
%%TRANS1
\glt
  ‘We did play well, but corruption snatched the cup from us.’
%%TRANS2
%%EXEND

\z

%%EAX
\ea
%%JUDGEMENT
%%LABEL
\label{bkm:Ref119738989}
%%CONTEXT
(You weeded an incredible two acres in five hours while your father expected that you can only do one acre.)\\
%%LINE1
Kûrîma itûrîmire baaba.\\
%%LINE2
\gll
kû-rîma   ni  tû-rîm-ire   baaba\\
%%LINE3
15-weed \FOC{}  1\PL.\SM{}-weed-\PFV{} dad\\
%%TRANS1
\glt
‘We really weeded (a lot), dad.’
%%TRANS2
%%EXEND

\z

These additional aspects of interpretation are not encoded in the semantics but pragmatics, as they can vary with the context and examples can hence be ambiguous:

%%EAX
\ea
%%JUDGEMENT
%%LABEL
%%CONTEXT
%%LINE1
Kwííná n’ áííniré bai!\\
%%LINE2
\gll
kû-iina  ni  a-iin-ire  bai\\
%%LINE3
15-sing \FOC{} 1\SM{}-sing-\PFV{} buddy\\
%%TRANS1
\glt
intensive: ‘Boy did she sing!’ (we never knew she had such a good voice)
%%TRANS2
depreciative: ‘Well at least she sang...’ (try to see the positive side)
%%EXEND

\z

The infinitive can be analysed as a contrastive topic, where the alternative topics are on a scale of expectation: for the intensive reading the assertion is higher than expected, and for the depreciative reading the assertion is more than zero \citep[see][]{JerrovanderWalFut}. Interestingly, polarity focus/verum, contrast, depreciative and intensive are exactly the same interpretations as those possible for contrastive topics marked by \textit{-o}, as exemplified in \sectref{bkm:Ref115790636}, which supports an analysis of topic doubling as the infinitive functioning as a contrastive topic.

A final interpretation occurs in the future tense, which has an optional \textit{ni} marker on the verb. When the marker is absent, the interpretation is one of verum, as in \xref{bkm:Ref132100667}, but when it is present, a deontic interpretation results.

%%EAX
\ea
%%JUDGEMENT
%%LABEL
\label{bkm:Ref132100667}
%%CONTEXT
(Context: A and B are arguing over cooking. A thinks B is unwilling to cook, then B answers:)\\
%%LINE1
Kûrúgǎ nkáárúga.\\
%%LINE2
\gll
kû-ruga  n-ka-rug-a\\
%%LINE3
15-cook 1\SG.\SM-\FUT{}-cook-\FV{}\\
%%TRANS1
\glt
‘I will cook.’
%%TRANS2
%%EXEND

\z

%%EAX
\ea
%%JUDGEMENT
%%LABEL
%%CONTEXT
(Context: You are organising a function and are told to go and sleep, but you need to prepare. Whatever else I do, ….)\\
%%LINE1
…kûrúga ínkáárúga.\\
%%LINE2
\gll
kû-ruga  ni  n-ka-rug-a\\
%%LINE3
15-cook \FOC{} 1\SG.\SM{}-\FUT{}-cook-\FV{} \\
%%TRANS1
\glt
‘I must cook.’
%%TRANS2
%%EXEND

\z


In summary, Kîîtharaka exceptionally shows three types of predicate doubling with infinitives. These are not special constructions but simply consist of the infinitive occupying a focus or topic position. When the infinitive is clefted (cleft doubling), the resulting interpretation is one of state-of-affairs focus; as a postverbal focus (in situ doubling), the infinitive is used polarity focus/verum context and with an intensive reading; and when the infinitive is in the left-periphery (topic doubling), it functions as a contrastive topic, used with a contrastive, verum, intensive, or depreciative interpretation. It is interesting to note the influence of the pragmatics on the precise interpretation here, specifically the intensive and depreciative aspects of meaning.

\section{Cleft constructions}
\label{kitcleft}

Like other languages in \citet{langsci-current-book}, Kîîtharaka too shows three structures that look like a basic cleft, a pseudocleft, and a reverse pseudocleft (although their underlying structures are subject to debate). All three constructions use the marker \textit{ni}, which surfaces as \textit{i} before a consonant and \textit{n} before a vowel -- the copula has the form \textit{ti} in negation. The marker \textit{ni} functions as a copula in simple predication and identification, illustrated in \xref{bkm:Ref111624670} and \xref{bkm:Ref111624671}, but has been analysed as a focus marker by \citet{AbelsMuriungi2008}. We gloss it as \COP{} in clear copula constructions, and as \FOC{} in cleft(-like) constructions and preceding the verb (see \sectref{bkm:Ref111628808}).

%%EAX
\ea
%%JUDGEMENT
%%LABEL
\label{bkm:Ref111624670}
%%CONTEXT
(What does Jane do for a living?)\\
%%LINE1
%%LINE2
\gll
Jane \textbf{i} mw-arimû.\\
%%LINE3
1.Jane \COP{} 1-teacher\\
%%TRANS1
\glt
‘Jane is a teacher.’ \jambox*{[predicational]}
%%TRANS2
%%EXEND

\z

%%EAX
\ea
%%JUDGEMENT
%%LABEL
\label{bkm:Ref111624671}
%%CONTEXT
(Who is the chef?)\\
%%LINE1
%%LINE2
\gll
Chebu \textbf{i} mw-ekûrû  ûyû.\\
%%LINE3
chef \COP{} 1-woman  1.\DEM.\PROX{}\\
%%TRANS1
\glt
‘The chef is this woman.’ \jambox*{[specificational]}
%%TRANS2
%%EXEND

\z


The debate about the precise status of the copula is briefly summarised in \sectref{bkm:Ref117495353} but an in-depth discussion is part of further research. In what follows, our aim is to describe the information-structural interpretation of the three cleft(-like) constructions. We start with the basic cleft/preverbal focus construction in \sectref{bkm:Ref117495353}, followed by the pseudocleft and reverse pseudocleft in \sectref{bkm:Ref132101501}, and finally a related construction characterised by \textit{ni} followed by a pronoun (\NI-\PRO{}) in \sectref{bkm:Ref116891458}.

\subsection{Preverbal focus construction/basic cleft}
\label{bkm:Ref117495353}
Starting with the basic cleft, there are two conflicting perspectives in the literature for analysing this structure. On the one hand there are authors that advocate for a biclausal cleft analysis (see \citealt{Bergvall1987} for Kikuyu, \citealt{Harford1997} for Kîîtharaka, \citealt{LafkiouiEtAl2016} for Kirundi, and \citealt{Zentz2016} for Shona) with the structure consisting of a copula, clefted constituent, and relative clause. On the other hand it has been argued that the structure is monoclausal, hence, better analysed as a focus construction (see \citealt{Clements1984,Schwarz2003} for Kikuyu, \citealt{Muriungi2005,AbelsMuriungi2008} for Kîîtharaka). In the current chapter we will not go into this syntactic debate (see \citealt{KanampiuvanderWalFut} for additional considerations to not assume Abels \& Muriungi’s monoclausal analysis), and after introducing the basic information about the form of the construction, we rather concentrate on the interpretation of the construction. We refer to the construction as ``preverbal focus" or ``basic cleft" interchangeably without committing to a syntactic analysis.

The basic cleft consists of \textit{ni}/\textit{ti} (the copula or focus particle) followed by a focused constituent, and a relative clause. Apart from argument NPs, as in \xref{bkm:Ref115935812}, the focused constituent can also be a pronoun as in \xref{bkm:Ref111821091} (which will be relevant in the discussion of the \NI-\PRO{} construction in \sectref{bkm:Ref111801991}), a nominalised modifier \xref{bkm:Ref116896235}, or an infinitive as in cleft predicate doubling (see \sectref{bkm:Ref115942967}).

%%EAX
\ea
%%JUDGEMENT
%%LABEL
\label{bkm:Ref115935812}
%%CONTEXT
(Who went to the farm?)\\
%%LINE1
I Kîmáthi ûthííre mûndaaní.\\
%%LINE2
\gll
ni  Kîmathi  û-thi-ire  mû-nda=ini\\
%%LINE3
\FOC{} Kîmathi 1\RM{}-go-\PFV{} 3-farm=\LOC{}\\
%%TRANS1
\glt
‘It is Kîmathi who went to the farm.’
%%TRANS2
%%EXEND

\z

%%EAX
\ea
%%JUDGEMENT
%%LABEL
\label{bkm:Ref111821091}
%%CONTEXT
(If she washes with water from a pool without frogs.)\\
%%LINE1
I r\'{î}o akaabú\'{a}.\\
%%LINE2
\gll
ni  rî-o  a-ka-bu-a\\
%%LINE3
\FOC{} 5-\PRO{} 1\SM-\FUT{}-be.good-\FV{}\\
%%TRANS1
\glt
‘It’s then (that time) that she will heal.’
%%TRANS2
%%EXEND

\z

%%EAX
\ea
%%JUDGEMENT
%%LABEL
\label{bkm:Ref116896235}
%%CONTEXT
(It’s alleged that the Swahili teacher was absent this morning.)\\
%%LINE1
Ar\'{î} t’wa g\'{î}choíri n’wá matháábu.\\
%%LINE2
\gll
arî  ti  w-a  gî-choiri  ni  w-a  ma-thaabu\\
%%LINE3
no \NEG.\FOC{} 1-\CONN{} 7-Kiswahili \FOC{} 1-\CONN{} 6-mathematics\\
%%TRANS1
\glt
‘No, it is not (the one) of Kiswahili but of mathematics.’
%%TRANS2
%%EXEND

\z


Note that there is no presupposition of existence in this construction, as the question in \xref{bkm:Ref111811743:a} can felicitously be answered by the empty set in \xref{bkm:Ref111811743:b}. Here, the construction differs from the pseudocleft (compare in \sectref{bkm:Ref132101940}).

\ea
\label{bkm:Ref111811743}
%%EAX
\ea
%%JUDGEMENT
%%LABEL
\label{bkm:Ref111811743:a}
%%CONTEXT
%%LINE1
N’\'{û}û ûyû\'{û}gú ápéeré meembe?\\
%%LINE2
\gll
ni  ûû  ûyûûgu  a-per-ire  ma-embe\\
%%LINE3
\FOC{} 1.who  1.grandmother 1\SM{}-give-\PFV{} 6-mango\\
%%TRANS1
\glt
‘Who did grandma give mangoes?’
%%TRANS2
%%EXEND

 %%EAX
\ex
%%JUDGEMENT
%%LABEL
\label{bkm:Ref111811743:b}
%%CONTEXT
%%LINE1
Gûtír\'{î} wě.\\
%%LINE2
\gll
kû-ti-rî  we\\
%%LINE3
17\SM-\NEG{}-be  1.\PRO{}\\
%%TRANS1
\glt
‘Nobody.’ lit. ‘There isn’t one.’
%%TRANS2
%%EXEND

\z
\z

\citet{AbelsMuriungi2008} show in detail that the Kîîtharaka basic cleft\slash prenominal focus structure is used to express exhaustive focus on the initial constituent marked by \textit{ni}. We refer to their work for valuable and intricate argumentation, and add the following diagnostics and data to arrive at the same conclusion. First, to show that the preverbal constituent is in focus, consider that questions and answers are naturally given in the preverbal focus construction, especially for subjects (although subjects may also be focused postverbally, see subject inversion in \sectref{bkm:Ref94457814}). Subject clefts are illustrated in \xref{bkm:Ref132102107} and object clefts in \xref{bkm:Ref132102116}.

\ea
\label{bkm:Ref132102107}
\begin{xlist}
%%EAX
\exi{Q:}
%%JUDGEMENT
%%LABEL
%%CONTEXT
%%LINE1
\'{I}baaû báayá?\\
%%LINE2
\gll
ni  ba-û  ba-a-y-a\\
%%LINE3
\FOC{} 2-who 2\SM-\PST{}-come-\FV{}\\
%%TRANS1
\glt
‘Who(pl) came?’
%%TRANS2
%%EXEND

%%EAX
\exi{A:}
%%JUDGEMENT
%%LABEL
%%CONTEXT
%%LINE1
I Mwendé na Baráka (báayá).\\
%%LINE2
\gll
ni  Mwende  na  Baraka  (ba-a-y-a)\\
%%LINE3
\FOC{} 1.Mwende  and  1.Baraka  (2\SM-\N.\PST{}-come-\FV{})\\
%%TRANS1
\glt ‘It’s Mwende and Baraka (who came).’
%%TRANS2
%%EXEND

\end{xlist}
\z

\ea
\label{bkm:Ref132102116}
\begin{xlist}
%%EAX
\exi{Q:}
%%JUDGEMENT
%%LABEL
%%CONTEXT
%%LINE1
I mbi Áshá ágûkáánda?\\
%%LINE2
\gll
ni  mbi  Asha  a-kû-kaand-a\\
%%LINE3
\FOC{} what 1.Asha 1\SM-\PRS{}-bake-\FV{}\\
%%TRANS1
\glt
‘What is Asha baking?’
%%TRANS2
%%EXEND

%%EAX
\exi{A:}
%%JUDGEMENT
%%LABEL
%%CONTEXT
%%LINE1
\'{I}mûgaáté Áshá agûkáánda.\\
%%LINE2
\gll
ni  mû-gaate  Asha  a-kû-kaand-a\\
%%LINE3
\FOC{} 3-bread  1.Asha 1\SM-\PRS{}-bake-\FV{}\\
%%TRANS1
\glt
‘It’s bread that Asha is baking.’
%%TRANS2
%%EXEND

\end{xlist}
\z

If part of an idiom is focused in this construction, it loses its idiomatic interpretation (though see \citealt{vanderWal2021} on types of idioms and focus), as expected if this is a focus construction. This is because no alternatives can be triggered for the object in the idiomatic reading, as this reading is dependent on both the verb and the object, as illustrated for the idiom \textit{kûtwa mûtî} ‘to become pregnant’, lit. ‘to climb a tree’ \xref{bkm:Ref116896542}.

%%EAX
\ea
%%JUDGEMENT
%%LABEL
\label{bkm:Ref116896542}
%%CONTEXT
%%LINE1
Í mût\'{î} mwaarí átwéeté. \\
%%LINE2
\gll
ni  mû-tî  mû-ari  a-tw-îte\\
%%LINE3
\FOC{} 3-tree  1-girl 1\SM{}-climb-\STAT{}.\PFV{}\\
%%TRANS1
\glt
*`The girl became pregnant.’\\
%%TRANS2
‘It’s a tree that the girl climbed.’ \citep[10]{vanderWal2021}
%%EXEND

\z

The preverbal focus is naturally used in a corrective context, as in \xref{bkm:Ref116896557}.

%%EAX
\ea
%%JUDGEMENT
%%LABEL
\label{bkm:Ref116896557}
%%CONTEXT
(Are these people wearing hats? + QUIS picture)\\
%%LINE1
Éék\'{û}r\'{û}  bataíkîrîté (nkoobía), n’ aantû arúme béékîrîté nkoobiá.\\
%%LINE2
\gll
a-ekûrû  ba-ta-ikîr-îte  n-koobia  ni  a-ntû  a-rume  ba-e-kîr-îte  n-koobia\\
%%LINE3
2-woman 2\SM-\NEG{}-dress-\STAT.\PFV{} 10-hat \FOC{} 2-person  2-male 2\SM-\REFL{}-dress-\STAT.\PFV{} 10-hat\\
%%TRANS1
\glt ‘The women don’t wear (hats); it’s the men who wear hats.’
%%TRANS2
%%EXEND

\z


The focus constituent is compatible with the exhaustive focus particle -\textit{onka} (only) as in \xref{bkm:Ref111803027}, but not with the scalar particle \textit{wana} ‘even’ \xref{bkm:Ref111803029:a}.\footnote{The particle \textit{kinya} (a variant of the additive particle) is equally inadmissible in the structure; thanks to a reviewer for asking about this.} The example in \xref{bkm:Ref111803029:a} was improved by placing the particle sentence-initially as in \xref{bkm:Ref111803029:b}, where it has scope over the whole action and cannot be interpreted as ‘even/also tablecloths’. Given the inclusive nature of ‘even’, the incompatibility shows the exclusive nature of the construction.

%%EAX
\ea
%%JUDGEMENT
%%LABEL
\label{bkm:Ref111803027}
%%CONTEXT
(Did the teacher give the children books and pens?)\\
%%LINE1
I tûrámu túunká mwarímû apéeré twaána.\\
%%LINE2
\gll
ni  tû-ramu  tu-onka  mû-arimû  a-per-ire  tû-aana\\
%%LINE3
\FOC{} 13-pens  13-only  1-teacher  1\SM{}-give-\PFV{} 13-child  \\
%%TRANS1
\glt
‘It is pens only that the teacher gave to the children.’
%%TRANS2
%%EXEND

\z
\largerpage[-1]
\pagebreak
\ea
\label{bkm:Ref111803029}
%%EAX
\ea
%%JUDGEMENT
%%LABEL
\label{bkm:Ref111803029:a}
%%CONTEXT
%%LINE1
I (*wana) ítaambáa Naómí áthaambirié.\\
%%LINE2
\gll
ni  wana  i-tambaa  Naomi  a-thaamb-i-ire\\
%%LINE3
\FOC{} even  8-cloth  1.Naomi 1\SM{}-clean-\IC{}-\PFV\\
%%TRANS1
\glt
‘It’s (*even) curtains that Naomi washed.’\\
%%TRANS2
‘It’s the tablecloths that Naomi washed (not the bedding).’
%%EXEND

 %%EAX
\ex
%%JUDGEMENT
%%LABEL
\label{bkm:Ref111803029:b}
%%CONTEXT
(Context: Naomi was supposed to wash the bedding, but …):\\
%%LINE1
Wana n’íítaambaǎ Naómí áthaambirié.\\
%%LINE2
\gll
wana  ni  i-tambaa  Naomi  a-thaamb-i-ire\\
%%LINE3
even \FOC{} 8-cloth  1.Naomi 1\SM{}-clean-\IC{}-\PFV{}\\
%%TRANS1
\glt
‘In addition to that it’s curtains that Naomi washed.'
%%TRANS2
%%EXEND

\z
\z

The universal quantifier ‘all’ is only accepted in this construction in the context of contrasting the universality with other quantities, as indicated in the corrective contexts for \xref{bkm:Ref111806551} and \xref{bkm:Ref111806543:a}. Modification by a relative clause also allows for the generation and exclusion of alternatives and also makes the universal quantifier acceptable in this construction, as in \xref{bkm:Ref111806543:b}: all the animals in this home can be contrasted with the alternative animals outside this home.

%%EAX
\ea
%%JUDGEMENT
%%LABEL
\label{bkm:Ref111806551}
%%CONTEXT
(He tasted this cake. / He tasted two cakes.)\\
%%LINE1
I kéki cíonthé aronchiré.\\
%%LINE2
\gll
ni  keki  ci-onthe  a-ronch-ire\\
%%LINE3
\FOC{} 10.cake  10-all 1\SM{}-taste-\PFV{}\\
%%TRANS1
\glt
‘It is all the cakes that he tasted.’
%%TRANS2
%%EXEND

\z

\ea
\label{bkm:Ref111806543}
\ea
\label{bkm:Ref111806543:a}  (Context 1: You bring a hen, a cow, and a goat to be slaughtered. Someone says we slaughter the hen and the cow only.\\
Context 2: (In a different scenario) someone says all the \textit{people} will die.)\\
Arî, í nyam\'{û} cíonthé igakúá.\\
\gll
arî  ni  n-yamû  ci-onthe  i-ka-ku-a\\
no \FOC{} 10-animal   10-all 10\SM-\FUT{}-die-\FV{}\\
\glt ‘It is all the animals that will die.’

 %%EAX
\ex
%%JUDGEMENT
%%LABEL
\label{bkm:Ref111806543:b}
%%CONTEXT
%%LINE1
 Í nyam\'{û} yoónthé îrî mûciî ûy\'{û} îgakúá.\\
%%LINE2
\gll
ni  n-yamû  y-onthe  î-rî  mû-ciî  û-yû  î-ka-ku-a\\
%%LINE3
\FOC{} 9-animal  9-all  9-be  9-home  9-\DEM.\PROX{} 9\SM-\FUT{}-die-\FV{}\\
%%TRANS1
\glt
‘It is every animal in this home that will die.’
%%TRANS2
%%EXEND

\z
\z

The indefinite quantifier -\textit{mwe} ‘some’ necessarily requires a subset reading in a cleft: in \xref{bkm:Ref111806111} it is not ‘some beans’ but ‘some \textit{of the} beans (not the other)’, as is clear not only from the translation, but also from the impossibility of following up with the whole set. We can again understand this as the cleft construction bringing exhaustive focus.

%%EAX
\ea
%%JUDGEMENT
%%LABEL
\label{bkm:Ref111806111}
%%CONTEXT
%%LINE1
I mbooco ímwé irá irî gîkóómbéení gîk\'{î}\v{î}, ntuúne... \textsuperscript{\#}wana ícíóonthe.\\
%%LINE2
\gll
ni  m-booco  i-mwe  i-ra  i-rî  kî-koombe=ini  kîkî  ni  n-tuune  wana  ni  ci-onthe\\
%%LINE3
\FOC{} 10-bean  10-one  10-\RM{} 10\SM{}-be  7-cup=\LOC{} 7.\DEM.\PROX{} \COP{} 10-red  even \COP{} 10-all\\
%%TRANS1
\glt
‘It’s some of the beans that are in this cup that are red... \textsuperscript{\#}it’s even all of them.’
%%TRANS2
%%EXEND

\z

Numerals lose the lower-boundary reading and instead become the exact amount in the preverbal focus construction. This is illustrated in \xref{bkm:Ref111981473:a}, and can be contrasted with the interpretation of a numeral modifying the object in an SVO sentence – compare \xref{bkm:Ref111981473:b}. The fact that the SVO order allows the follow up ‘and even more’ but the cleft/focus construction does not shows again that this construction excludes alternatives, in this case all amounts higher than 100,000.

\ea
\label{bkm:Ref111981473}
%%EAX
\ea
%%JUDGEMENT
%%LABEL
\label{bkm:Ref111981473:a}
%%CONTEXT
%%LINE1
\'{I}ngirí \'{î}gana ár\'{î}agwá \textsuperscript{\#}wana í nk\'{û}r\'{û}ki.\\
%%LINE2
\gll
ni  n-giri  î-gana  a-rî-ag-w-a  wana  ni  n-kûrûki\\
%%LINE3
\FOC{} 9-thousand  9-hundred 1\SM{}-pay-\HAB-\PASS-\FV{} even \FOC{}  9-more\\
%%TRANS1
\glt
‘It’s 100,000 that he is paid, \textsuperscript{\#}even more.’
%%TRANS2
%%EXEND

%%EAX
\ex
%%JUDGEMENT
%%LABEL
\label{bkm:Ref111981473:b}
%%CONTEXT
%%LINE1
Agwáátága/ arîágwa ngirí {\'{î}gana}, wana í {nk\'{û}r\'{û}ki}.\\
%%LINE2
\gll
a-kû-at-ag-a/  a-rî-ag-w-a  n-giri  î-gana  wana  ni  n-kûrûki\\
%%LINE3
1\SM-\PRS{}-earn-\HAB-\FV{}/  1\SM{}-pay-\HAB-\PASS{}-\FV{} 9-thousand  9-hundred even \FOC{} 9-more\\
%%TRANS1
\glt
‘He earns / is paid 100,000, even more.’
%%TRANS2
%%EXEND

\z
\z

Unexpectedly, an indefinite interpretation of the focused constituent is possible; specifically, the focus construction can be used in a thetic context, as in \xxref{bkm:Ref111803119}{bkm:Ref111803121}.\largerpage[-1]\pagebreak

%%EAX
\ea
%%JUDGEMENT
%%LABEL
\label{bkm:Ref111803119}
%%CONTEXT
(Context: You’re explaining what is happening, what has caused the sadness.)\footnote{The sentence can also be interpreted as narrow focus under a generic reading, for example in a context where people are disagreeing whether it was an animal or a person who died.}\\
%%LINE1
Í múúnt\'{û} ákwííre.\\
%%LINE2
\gll
ni  mû-ntû  a-ku-ire\\
%%LINE3
\FOC{} 1-person 1\SM{}-die-\PFV{}\\
%%TRANS1
\glt
‘Somebody died.’
%%TRANS2
%%EXEND

\z

%%EAX
\ea
%%JUDGEMENT
%%LABEL
%%CONTEXT
(Why were you running away? / Why did you scream?)\\
%%LINE1
I g\'{î}\'{î}nt\'{û} mbónire.\\
%%LINE2
\gll
ni  kî-ntû  n-on-ire\\
%%LINE3
\FOC{} 7-thing 1\SG.\SM{}-see-\PFV{}\\
%%TRANS1
\glt
‘I saw something.’ / ‘It’s something I saw.’
%%TRANS2
%%EXEND

\z

%%EAX
\ea
%%JUDGEMENT
%%LABEL
\label{bkm:Ref111803121}
%%CONTEXT
(Why are you walking in a funny way?)\\
%%LINE1
I m\'{û}goongó \'{û}k\'{û}\'{û}mbaankana.\\
%%LINE2
\gll
ni  mû-goongo  û-kûû-n-aankan-a\\
%%LINE3
\FOC{} 3-back 3\SM-\PRS-1\SG.\OM{}-ache-\FV{}\\
%%TRANS1
\glt
‘It is my back that aches.’
%%TRANS2
%%EXEND

\z


This can perhaps be understood as focus on the whole situation: ‘it is only for this reason that I’m sad’. This interpretation is also possible for the full clefted clause in \xref{bkm:Ref117496479}.

%%EAX
\ea
%%JUDGEMENT
%%LABEL
\label{bkm:Ref117496479}
%%CONTEXT
(Why are you crying?)\\
%%LINE1
N’áatî múntû ágûkua/ n’ákwíré.\\
%%LINE2
\gll
ni  atî   mu-ntû   a-kû-ku-a/  ni  a-a-ku-ire\\
%%LINE3
\FOC{} \COMP{} 1-person 1\SM-\PRS{}-die-\FV{}/ \FOC{}  1\SM{}-\N.\PST{}-die-\PFV{}\\
%%TRANS1
\glt
‘It is that a person has died/died (that I’m crying), and that’s it.’
%%TRANS2
%%EXEND

\z


In summary, the preverbal focus construction/basic cleft shows all the hallmarks of an exhaustive focus interpretation (see again also \citealt{AbelsMuriungi2008}). Its use in thetic contexts may be explained as an exhaustive interpretation of the whole sentence, or interpreted as a widening/bleaching of meaning in the process of grammaticalization.

\subsection{Pseudocleft}
\label{bkm:Ref132101501}\label{bkm:Ref132101940}
The pseudocleft (and the reverse pseudocleft) is a copular clause that links a free relative with a noun phrase. In the pseudocleft, the free relative precedes the copula and the noun phrase follows, as in \xref{bkm:Ref132103821} and \xref{bkm:Ref132103834}.

%%EAX
\ea
%%JUDGEMENT
%%LABEL
\label{bkm:Ref132103821}
%%CONTEXT
(What did Karîmi kick?)\\
%%LINE1
%%LINE2
\gll {}[Kî-rá  Karîmi  á-rííng-iré]  í  mû-bíírá. \\
%%LINE3
{\db}7-\RM{} Karîmi 1\SM{}-kick-\PFV{} \COP{} 3-ball\\
%%TRANS1
\glt
‘What Karîmi kicked is a ball.’
%%TRANS2
%%EXEND

\z

%%EAX
\ea
%%JUDGEMENT
%%LABEL
\label{bkm:Ref132103834}
%%CONTEXT
%%LINE1
[Bará bapéerwé  tûrámu]  i  twáána  tuunka.\\
%%LINE2
\gll
[ba-ra ba-per-w-ire  tû-ramu]  ni  tû-ana  tû-onka.\\
%%LINE3
{\db}2-\RM{} 2\SM{}-give-\PASS-\PFV{} 13-pens \COP{} 13-child  13-only\\
%%TRANS1
\glt
‘Those that were given pens are the children only.’
%%TRANS2
%%EXEND

\z


The free relative is introduced by the definite relative marking, that is, the distal demonstrative ending in -\textit{ra}, taking the noun class of what it refers to. Free relative clauses function as independent NPs, as can be seen in their use as a subject as in \xref{bkm:Ref132114638} or object as in \xref{bkm:Ref132114647}.

%%EAX
\ea
%%JUDGEMENT
%%LABEL
\label{bkm:Ref132114638}
%%CONTEXT
%%LINE1
(Muntû)  [Ûra  gwîtiré]  akáaya.\\
%%LINE2
\gll
mu-ntû  û-ra  û-ît-ire  a-kaa-y-a\\
%%LINE3
{\db}1-person  {\db}1-\RM{} 2\SG.\SM{}-call-\PFV{} 1\SM-\FUT{}-come-\FV{} \\
%%TRANS1
\glt
‘The one you called will come.’
%%TRANS2
%%EXEND

\z

%%EAX
\ea
%%JUDGEMENT
%%LABEL
\label{bkm:Ref132114647}
%%CONTEXT
%%LINE1
\'{I} mbeendeeté  (kîntû)  [kîrá  mbóniré].\\
%%LINE2
\gll
ni n-end-îte  kî-ntû  kî-ra  n-on-ire\\
%%LINE3
\FOC{} 1\SG.\SM{}-like-\STAT{}.\PFV{}  {\db}7-thing  {\db}7-\RM{} 1\SG.\SM{}-see-\PFV{}\\
%%TRANS1
\glt
‘I like what I saw / I like the thing that I saw.’
%%TRANS2
%%EXEND

\z


Kîîtharaka also has the reverse structure where we find a precopular noun phrase and a postcopular free relative, as in \xref{bkm:Ref149295747}.

%%EAX
\ea
%%JUDGEMENT
%%LABEL
\label{bkm:Ref149295747}
%%CONTEXT
(Who is Karîmi in your class?)\\
%%LINE1
Karîmi n’ [ûrá árumiré mwarímû].\\
%%LINE2
\gll
Karîmi  ni  [û-ra  a-rum-ire  mû-arimû\\
%%LINE3
1.Karîmi \COP{} [1-\RM{}  1\SM{}-insult-\PFV{} 1-teacher\\
%%TRANS1
\glt
‘Karîmi is the one who insulted the teacher.’
%%TRANS2
%%EXEND

\z


Structurally, this would count as a reverse pseudocleft, because it is the exact reverse of the pseudocleft, but the term ``reverse pseudocleft" has been applied mostly to constructions in which the preverbal NP is in focus. In both of the constructions above, however, the focus is on the postcopular constituent (bolded in \xref{bkm:Ref111708000} and \xref{bkm:Ref111720154}), as indicated in the accepted and unacceptable context questions for each construction.

%%EAX
\ea
%%JUDGEMENT
%%LABEL
\label{bkm:Ref111708000}
%%CONTEXT
(Context 1: Is Joy washing the dishes?\\
Context 2: Who is washing the dishes?\\
\textsuperscript{\#}Context 3: I am looking for Rob, who is working in a kitchen full of people. I ask ‘Who is Rob?’)\\
%%LINE1
%%LINE2
\gll
{}[Ûrá  û-kû-thaamb-i-a  thaáni]  i  \textbf{Rob}.\\
%%LINE3
{\db}1.\RM{} 1\RM-\PRS{}-clean-\IC{}-\FV{}  10.plate \COP{} 1.Rob\\
%%TRANS1
\glt
‘The one who is washing the dishes is Rob.’
%%TRANS2
%%EXEND

%%EAX
\ex
%%JUDGEMENT
%%LABEL
\label{bkm:Ref111720154}
%%CONTEXT
(\textsuperscript{\#}Context 1: Is Joy doing the dishes?\\
\textsuperscript{\#}Context 2: Who is washing the dishes?\\
Context 3: I am looking for Rob, who is working in a kitchen full of people. I ask ‘Who is Rob?’)\\
%%LINE1
%%LINE2
\gll
Rob  n’  [\textbf{ûrá}  \textbf{û-kû-thaamb-i-a}  \textbf{thaáni}] \\
%%LINE3
1.Rob \COP{} {\db}1.\RM{} 1\RM-\PRS{}-clean-\IC{}-\FV{}  10.plate\\
%%TRANS1
\glt
‘Rob is the one who is washing the dishes.’
%%TRANS2
%%EXEND

\z


We illustrate for the pseudocleft that the postcopular noun phrase is in focus, as seen in the question-answer pairs in \xref{bkm:Ref111708000} above, and that this focus identifies the correct referent among alternative referents. The inverse construction as in \xref{bkm:Ref111720154} we shall refer to as the \NI+\RM{} construction in order to not create confusion between form and function with the term ``reverse pseudocleft", and this is compared to a very similar construction in \sectref{bkm:Ref116898052}.

Examples \xref{bkm:Ref111650296} and \xref{bkm:Ref111650297} illustrate an infinitive and a pronominal demonstrative as the focused constituent in the pseudocleft. The contexts here indicate a choice among alternatives and a correction, respectively.

%%EAX
\ea
%%JUDGEMENT
%%LABEL
\label{bkm:Ref111650296}
%%CONTEXT
(Between getting lost in the forest and encountering wild animals, what don’t you want to experience?)\\
%%LINE1
Bûrá ntakwéénda, í kû\'{û}ra.\\
%%LINE2
\gll
bû-ra  n-ta-kû-end-a  ni  kû-ûra\\
%%LINE3
14-\RM{} 1\SG.\SM-\NEG{}-\PRS{}-like-\FV{} \COP{} 15-get.lost\\
%%TRANS1
\glt
‘What I don’t want, is to get lost.’
%%TRANS2
%%EXEND

%%EAX
\ex
%%JUDGEMENT
%%LABEL
\label{bkm:Ref111650297}
%%CONTEXT
(Is it this one that came late?)\\
%%LINE1
Arî ûrá aceereerwé n’ûûrá (t’\'{û}ûyû).\\
%%LINE2
\gll
arî  û-ra  a-ceererw-e  ni  û-ra  ti  û-yû\\
%%LINE3
no  1-\RM{} 1\SM{}-be.late-\PFV{} \COP{} 1-\DEM.\DIST{} \NEG.\COP{}  \textsc{1-dem.prox}\\
%%TRANS1
\glt ‘No, the one that was late is that one (not this one).’
%%TRANS2
%%EXEND

\z

The free relative establishes a presupposition of existence, which can be seen in the impossibility to answer a pseudocleft question with ‘nobody/nothing’ as illustrated in \xref{bkm:Ref111806764} and \xref{bkm:Ref111806765}. Since it is presupposed that there is a referent for which the predicate is true, this cannot be denied, but instead it is expected that this referent is identified.

\ea
\label{bkm:Ref111806764}
%%EAX
\ea
%%JUDGEMENT
[]{
%%LABEL
%%CONTEXT
%%LINE1
Kîrá ágwáátiré i kîbí?\\
%%LINE2
\gll
kî-ra  a-gwaat-ire  ni  kî-bi\\
%%LINE3
7-\RM{} 1\SM{}-catch-\PFV{} \COP{} 7-what \\
%%TRANS1
\glt
‘What did he catch?’, lit. ‘What he caught is what?’
%%TRANS2
}
%%EXEND

%%EAX
\ex
%%JUDGEMENT
[\textsuperscript{\#}]{
%%LABEL
%%CONTEXT
%%LINE1
Gûtírîkyó.\\
%%LINE2
\gll
kû-ti-rî=kî-o\\
%%LINE3
17\SM-\NEG{}-be=7-\PRO{}\\
%%TRANS1
\glt
‘Nothing.’
%%TRANS2
}
%%EXEND

\z

\ex
\label{bkm:Ref111806765}

%%EAX
\ea
%%JUDGEMENT
[]{
%%LABEL
%%CONTEXT
%%LINE1
Ûrá áceereerwé n’û́û?\\
%%LINE2
\gll
û-ra  a-ceererw-e  ni  ûû\\
%%LINE3
1-\RM{} 1\SM{}-be.late-\PFV{} \COP{} who\\
%%TRANS1
\glt
‘Who is the one who was late?’, lit. ‘Who was late is who?’
%%TRANS2
}
%%EXEND

%%EAX
\ex
%%JUDGEMENT
[\textsuperscript{\#}]{
%%LABEL
%%CONTEXT
%%LINE1
Gûtír\'{î} we.\\
%%LINE2
\gll
kû-ti-rî=we\\
%%LINE3
17\SM-\NEG{}-be=1.\PRO{}\\
%%TRANS1
\glt
‘Nobody / There is no one.’
%%TRANS2
}
%%EXEND

\z
\z

Identification among alternatives entails that alternatives must be present. Since parts of idioms cannot trigger a set of alternatives within the idiomatic reading, only the literal meaning is left under focus. This is what we see in \xref{bkm:Ref120111763}, from the idiom \textit{kûoria nthang’a ûthiû} ‘to make the face of the monkey rot’ which means to worsen the situation: the idiomatic reading is lost when you focus on the constituent \textit{face of the monkey}.\largerpage[-1]\pagebreak

%%EAX
\ea
%%JUDGEMENT
%%LABEL
\label{bkm:Ref120111763}
%%CONTEXT
%%LINE1
Kîrá ûkûóriá n’ûthí\'{û} bwa ntháng’á.\\
%%LINE2
\gll
kî-ra  û-kû-or-i-a  ni  ûthiû  bû-a  n-thang’a\\
%%LINE3
7-\RM{} 2\SG.\SM-\PRS{}-rot-\IC-\FV{} \COP{} 14.face  14-\CONN{} 9.monkey\\
%%TRANS1
\glt
‘What you are causing to rot is the face of the monkey.’\\
%%TRANS2
*`You are worsening the situation.’
%%EXEND

\z


A first diagnostic to check the exclusive reading among alternatives is that the NP can be modified by the exhaustive focus-sensitive particle ‘only’ \xref{bkm:Ref111981556} but not by the inclusive/scalar ‘even’ \xref{bkm:Ref111981565}. As ‘even’ includes all alternatives, no referent is identified to the exclusion of others, and hence this is incompatible with the pseudocleft.

%%EAX
\ea
%%JUDGEMENT
[]{
%%LABEL
\label{bkm:Ref111981556}
%%CONTEXT
%%LINE1
Kîrá gîonká Bita ar\'{î}îré i nkíma.\\
%%LINE2
\gll
kî-ra  kî-onka  Bita  a-rî-ire  ni  n-kima\\
%%LINE3
7-\RM{} 7-only  1.Peter  1\SM{}-eat-\PFV{} \COP{} 9-ugali\\
%%TRANS1
\glt
‘What Peter ate is ugali only.’
%%TRANS2
}
%%EXEND

%%EAX
\ex
%%JUDGEMENT
[*]{
%%LABEL
\label{bkm:Ref111981565}
%%CONTEXT
%%LINE1
Kîrá Bita ágwáátiré í (*wana) mûkûnga.\\
%%LINE2
\gll
kî-ra  Bita  a-gwaat-ire  ni  wana  mû-kûnga\\
%%LINE3
5-\RM{} 1.Peter 1\SM{}-catch-\PFV{}  \COP{} even  3-eel\\
%%TRANS1
\glt
`What Peter caught is (*even) eel.’
%%TRANS2
}
%%EXEND

\z


Secondly, the structures are compatible with universal quantifiers only if they allow an exclusive reading. The pseudocleft in structure \xref{bkm:Ref111709098} is felicitous because it presupposes two subsets; the people who work within the institution and the ones that work elsewhere. Those with hats, therefore, are the ones that work within the institution. The people who work elsewhere are excluded. In structure \xref{bkm:Ref111709106} the focus does not trigger any subsets. It remains inclusive, thus, infelicitous with the universal quantifier \textit{kîra}. The conclusion here is that the pseudocleft structures express exclusive focus, and thus, disallow universal quantifiers because of their inclusive nature. The same holds for the universal quantifier ‘all’ in \xref{bkm:Ref111709046}.

%%EAX
\ea
%%JUDGEMENT
[]{
%%LABEL
\label{bkm:Ref111709098}
%%CONTEXT
%%LINE1
Ûrá gw\'{î}k\'{î}r\'{î}te nkoobíá i kîrá muntû ûrutága ngûgí g\'{û}kû.\\
%%LINE2
\gll
û-ra  û-îkîr-îté  n-koobia  ni  kîra  mu-ntû  a-rût-ag-a  n-gûgî  kû-kû\\
%%LINE3
1-\RM{} 1\SM.\REL{}-wear-\STAT.\PFV{} 9-hat \COP{} each  1-person 1\SM{}-do-\HAB{}-\FV{} 9-work  17-\DEM.\PROX{}\\
%%TRANS1
\glt
‘The one wearing a hat is each/every person who works within.’
%%TRANS2
}

%%EAX
\ex
%%JUDGEMENT
[*]{
%%LABEL
\label{bkm:Ref111709106}
%%CONTEXT
%%LINE1
Ûrá gw\'{î}k\'{î}r\'{î}té nkoobíá i kîrá muntû.\\
%%LINE2
\gll
û-ra  û-îkîr-îte  n-koobia  ni  kîra  mu-ntû\\
%%LINE3
1-\RM{} 1\SM.\REL{}-wear-\STAT.\PFV{} 9-hat \COP{} each  1-person\\
%%TRANS1
\glt
int. ‘The one wearing a hat is everyone.’
%%TRANS2
}
%%EXEND


%%EAX
\ex
%%JUDGEMENT
[]{
%%LABEL
\label{bkm:Ref111709046}
%%CONTEXT
%%LINE1
Bará béékîrîté nkoobíá n’ antû bóonthé *(bará barutagá ngûgî nja).\\
%%LINE2
\gll
ba-ra  ba-îkîr-îte  n-koobia  ni  a-ntû  ba-onthé   ba-ra  ba-rût-ag-a  n-gûg\'{î}  n-ja \\
%%LINE3
2-\RM{} 2-wear-\STAT.\PFV{} 10-hat \COP{}  2-person  2-all 2- \RM{} 2-do-\HAB-\FV{} 9-work  9-outside \\
%%TRANS1
\glt
‘The ones wearing hats are all the people who work outside.’
%%TRANS2
}
%%EXEND

\z

Thirdly, the pseudocleft structures are not compatible with non-specific indefinites, because such expressions do not refer to anything in particular and thus cannot identify or exclude. Infelicity in such contexts thus confirms the exclusive nature of the Kîîtharaka pseudocleft. Instead, they can only be used with a generic interpretation. Structure \xref{ex:kth:seeperson} for instance would only make sense if it were interpreted that the speaker saw a person and not an animal. Proceeding from this premise, structure example \xref{bkm:Ref111647873} is judged infelicitous. The use of the relative marker in class 1 \textit{ûra} (the one/who) already calls for a specific element in the clefted constituent because the speaker already knows it is a person. Identification of ‘a/the person’ requires a specific name and not a generic reference, whose use here amounts to redundancy.

%%EAX
\ea
%%JUDGEMENT
[]{
%%LABEL
\label{ex:kth:seeperson}
%%CONTEXT
(What did you see? OR There was some movement at the door. I guess it was a goat?)\\
%%LINE1
Kîrá mbóniré i muntû (ti mbûri).\\
%%LINE2
\gll
kî-ra  n-on-ire  ni  mu-ntû  (ti  mb-ûri)\\
%%LINE3
7-\RM{} 1\SG.\SM{}-see-\PFV{} \COP{} 1-person {\db}\NEG{} 9-goat\\
%%TRANS1
\glt
‘What I saw is a person.’
%%TRANS2
}
%%EXEND

\z

%%EAX
\ea
%%JUDGEMENT
[\textsuperscript{?}]{
%%LABEL
\label{bkm:Ref111647873}
%%CONTEXT
(Who did you see?)\\
%%LINE1
Ûrá mbóniré i muntû.\\
%%LINE2
\gll
û-ra  n-on-ire  ni  mu-ntû\\
%%LINE3
1-\RM{} 1\SG.\SM{}-see-\PFV{}  \COP{} 1-person\\
%%TRANS1
\glt
\textsuperscript{?}’The one I saw is a person.’
%%TRANS2
}
%%EXEND

\z


Similarly, the NP in a pseudocleft can only be modified by ‘some’ if there is a contrast with ‘all’. Outside of this context, the use of the indefinite quantifier ‘some’ in the clefted constituent fails to achieve the identificational role of the focus.

%%EAX
\ea
%%JUDGEMENT
%%LABEL
%%CONTEXT
(\textsuperscript{\#}Context 1: Did all the chickens escape?) \\
(Context 2: All the animals ran away! Chickens, cows, goats… No, actually…)\\
%%LINE1
Irá irátoorookíre i ngû́kû́ ímwé.\\
%%LINE2
\gll
i-ra  i-ra-toorook-ire  ni  n-gûkû  i-mwe\\
%%LINE3
10-\RM{} 10\SM{}-\YPST{}-escape-\PFV{} \COP{} 10-chicken  10-some\\
%%TRANS1
\glt ‘The ones that ran away are some (of the) chickens.’
%%TRANS2
%%EXEND

\z

%%EAX
\ea
%%JUDGEMENT
%%LABEL
%%CONTEXT
(Context: Did all the chickens escape?)\\
%%LINE1
Irá irátoorookíre n’ímwé.\footnote{A reviewer asked whether it is possible to have \textit{n’imwe ira iratoorookire} to mean ‘it is some that ran away’. In our view, such a structure is not possible in the same context as it results in a different interpretation in Kîîtharaka – ‘it is some; the ones that ran away.’}\\
%%LINE2
\gll
i-ra  i-ra-toorook-ire  ni  i-mwe\\
%%LINE3
10-\RM{} 10-\YPST{}-escape-\PFV{} \COP{} 10-one\\
%%TRANS1
\glt
‘Only some ran away.’ lit: ‘The ones that ran away are some.’
%%TRANS2
%%EXEND

\z


Fourthly, the Kîîtharaka pseudocleft is incompatible with mention-some contexts. This is a non-exhaustive context that asks about entities without excluding alternatives. Instead, the pseudocleft is indicated as exclusive. If for instance, you are in need of a pen and you luckily meet about four people working in an office that have all sorts of pens, you are hopeful to be lent one, so you ask: \textit{who amongst you can lend me a pen?} In the given situation, there is no one correct answer, because there are multiple people who can lend a pen. In Kîîtharaka, an SVO sentence would be acceptable in this context, but a pseudocleft as in \xref{bkm:Ref111648178} is not acceptable. The pseudocleft can only be interpreted exclusively, suggesting that the others do not have pens to lend; \textit{it is Peter that has several pens so he can assist you with one.}

%%EAX
\ea
%%JUDGEMENT
[\textsuperscript{\#}]{
%%LABEL
\label{bkm:Ref111648178}
%%CONTEXT
%%LINE1
Ûrá g\'{û}ûmba kûróómba i Bita.\\
%%LINE2
\gll
û-ra  gû-ûmba  kû-róómba  ni  Bita\\
%%LINE3
1-\RM{} 2\SG.\SM{}-can  15-borrow \COP{} Peter\\
%%TRANS1
\glt
‘The one you can borrow from is Peter.’
%%TRANS2
}
%%EXEND

\z


Similarly, if \xref{bkm:Ref111648598} is given as an answer to the question ‘which places in Kenya can I visit for holiday?’ then it cannot be taken to mean that Mombasa is one of the many places one can visit. Instead, it means Mombasa is the one good place, thus, getting an exclusive indication.

%%EAX
\ea
%%JUDGEMENT
%%LABEL
\label{bkm:Ref111648598}
%%CONTEXT
%%LINE1
Kûra kwééga i Mombasa.\\
%%LINE2
\gll
kû-ra  kû-ega  ni  Mombasa\\
%%LINE3
17-\RM{} 17-good \COP{} 17.Mombasa\\
%%TRANS1
\glt
‘The best place is Mombasa.’\\
%%TRANS2
\textsuperscript{\#}’Where it is good is Mombasa.’
%%EXEND

\z


Fifthly, correction of falsehood is also a natural environment for the pseudocleft, as shown in \xref{bkm:Ref111707394} and \xref{bkm:Ref111707395}.

%%EAX
\ea
%%JUDGEMENT
%%LABEL
\label{bkm:Ref111707394}
%%CONTEXT
(Context: Person A continues to suggest that Karîmi must be in the market because she is a trader there. Person B insists…)\\
%%LINE1
Kûrá Karîmi ár\'{î} i mûndaaní.\\
%%LINE2
\gll
kû-ra  Karîmi  a-rî  ni  mû-nda=ini\\
%%LINE3
17-\RM{} 1.Karîmi  1\SM{}-be \COP{} 3-farm=\LOC{}\\
%%TRANS1
\glt
‘Where Karîmi is is on the farm.’
%%TRANS2
%%EXEND

\z

%%EAX
\ea
%%JUDGEMENT
%%LABEL
\label{bkm:Ref111707395}
%%CONTEXT
(Did the girl steal the ball?)\\
%%LINE1
Arî, kará kaiyiré i ka\'{î}y\'{î}.\\
%%LINE2
\gll
arî  ka-ra  ka-iy-ire  ni  ka-îyî\\
%%LINE3
no  12-\RM{} 12-steal-\PFV{} \COP{} 12-small.boy\\
%%TRANS1
\glt
‘No, the one who stole is a small boy.’
%%TRANS2
%%EXEND

\z


Finally, although it is quite unnatural to ask the question in \xref{bkm:Ref111982718} in the form of a pseudocleft, the best answer to it starts with ‘no’. This negates the exhaustivity that is expressed in the pseudocleft, as it asks for only one of the two referents for which the predicate is true.

%%EAX
\ea
%%JUDGEMENT
%%LABEL
\label{bkm:Ref111982718}
%%CONTEXT
(Context: photo of a woman selling onions and tomatoes)\\
%%LINE1
Kîra mwekûrû akweendia i nyaanya?\\
%%LINE2
\gll
kî-rá  mû-ékûrû  a-kû-endi-a  ni  nyaanya\\
%%LINE3
7-\RM{} 1-lady 1\SM-\PRS{}-sell-\FV{} \COP{} 10-tomato\\
%%TRANS1
\glt
‘What the woman is selling, is it tomatoes?’
%%TRANS2
%%EXEND

\z

%%EAX
\ea
%%JUDGEMENT
%%LABEL
%%CONTEXT
%%LINE1
Arî arîendia nyaanyá na itûngûrû.\\
%%LINE2
\gll
arî  a-rî-endi-a  nyaanya  na  i-tûngûrû\\
%%LINE3
no 1\SM{}-\PRS{}-sell-\FV{} 10.tomato  and  8-onion\\
%%TRANS1
\glt
‘No, she sells tomatoes and onions.’
%%TRANS2
%%EXEND

\z


We conclude that the pseudocleft identifies a referent whose existence is presupposed in the free relative, and does so to the exclusion of alternatives. However, Kîîtharaka speakers commonly use another structure in these contexts with the form [XP + \NI-\PRO{} + V], which is what we turn to now.

\subsection{The \NI-\PRO{} construction}
\label{bkm:Ref116891458}\label{bkm:Ref111801991}\label{bkm:Ref116898052}
There is another construction that differs minimally from the structure we have introduced as the \NI+\RM{} construction, but as we will see it is different in its interpretation. This construction also features an initial NP followed by \textit{ni}, but is then followed not by the relative (demonstrative) pronoun, but by the pronoun in -\textit{o} (see \sectref{bkm:Ref115790636} for other uses) and the verb. This is illustrated in \xref{bkm:Ref111713731}, to be compared to the \NI+\RM{} construction in \xref{bkm:Ref116898341}. We will refer to this as the \NI-\PRO{} construction.

%%EAX
\eanoraggedright
%%JUDGEMENT
%%LABEL
\label{bkm:Ref111713731}
%%CONTEXT
(Is it Mwangi that wrote the letter?)\\
%%LINE1
Bíta n’\textbf{wé} áandîkiré (tí Mwangi).\hfill\hbox{[\NI-\PRO{}]}\\
%%LINE2
\gll
Bita  ni we a-andîk-ire  (ti  Mwangi)\\
%%LINE3
1.Peter \FOC{}  1.\PRO{} 1\SM{}-write-\PFV{} (\NEG.\COP{} 1.Mwangi)\\
%%TRANS1
\glt
‘Peter, it is he who wrote (not Mwangi).’
%%TRANS2
%%EXEND



%%EAX
\ex
%%JUDGEMENT
%%LABEL
\label{bkm:Ref116898341}
%%CONTEXT
(Who is Peter?)\\
%%LINE1
Bíta n’\textbf{ûrá} áandîkiré.\hfill\hbox{[\NI-\RM{}]}\\
%%LINE2
\gll
Bita  ni ûra a-andîk-ire\\
%%LINE3
1.Peter \FOC{} 1.\RM{} 1\SM{}-write-\PFV{}\\
%%TRANS1
\glt
‘Peter is the one who wrote.’
%%TRANS2
%%EXEND

\z


While the structural parallel between the two constructions is striking, they differ on various points. A first point is that, unlike the pseudocleft which can be reversed, the \NI-\PRO{} construction cannot be reversed. That is, the phrase headed by the pronoun cannot precede \textit{ni}, as shown in \xref{bkm:Ref111714205} and \xref{bkm:Ref111714207}.

\ea
\label{bkm:Ref111714205}
%%EAX
\ea
%%JUDGEMENT
[]{
%%LABEL
%%CONTEXT
%%LINE1
[Kîrá  ûkwoóná]  í  [kyó  ûgapewá].\\
%%LINE2
\gll
[kî-ra  û-kû-on-a]  ni  [ki-o  û-ka-per-w-a]\\
%%LINE3
{\db}7-\RM{} 2\SG.\SM-\PRS{}-see-\FV{} \COP{} {\db}7-\PRO{} 2\SG.\SM-\FUT{}-give-\PASS-\FV{}\\
%%TRANS1
\glt
‘What you see is what you get.’
%%TRANS2
}
%%EXEND

%%EAX
\ex
%%JUDGEMENT
[*]{
%%LABEL
%%CONTEXT
%%LINE1
[Ky-o  û-kw-oon-a]  i  [ky-o  û-ga-p-ew-a].\\
%%LINE2
\gll
[ki-o  û-kû-on-a]  ni  [ki-o  û-ka-per-w-a]\\
%%LINE3
{\db}7-\PRO{} 2\SG.\SM-\PRS{}-see-\FV{} \COP{} {\db}7-\PRO{} 2\SG.\SM-\FUT{}-give-\PASS-\FV{}\\
%%TRANS1
\glt 
%%TRANS2
}
%%EXEND

%%EAX
\ex
%%JUDGEMENT
[*]{
%%LABEL
%%CONTEXT
%%LINE1
[Kyo  ûkwoona]  i  [kîra  ûgapewa].\\
%%LINE2
\gll
[ki-o  û-kû-on-a]  ni  [kî-ra  û-ka-per-w-a]\\
%%LINE3
{\db}7-\PRO{} 2\SG.\SM-\PRS{}-see-\FV{} \COP{} {\db}7-\RM{} 2\SG.\SM-\FUT{}-give-\PASS-\FV{}\\
%%TRANS1
\glt 
%%TRANS2
}
%%EXEND

\z
\z

%%EAX
\ea
%%JUDGEMENT
%%LABEL
\label{bkm:Ref111714207}
%%CONTEXT
%%LINE1
%%LINE2
\gll
[Ûrá/  *We  û-kû-thaamb-i-a  thaání]  í  Rób.\\
%%LINE3
{\db}1-\RM{}/  1.\PRO{} 1\RM-\PRS{}-clean-\IC{}-\FV{} 10.plate \COP{} 1.Rob\\
%%TRANS1
\glt
‘The one/he who washes dishes is Rob.’
%%TRANS2
%%EXEND

\z

Secondly, if the copula functions like a predicator, as it does in the pseudocleft and the \NI+\RM{} construction, we expect it to change form to express different tense/aspect in the \NI-\PRO{} construction as well. The copula \textit{ni} is replaced by the verb -\textit{ri} in the past tense, both in copular clauses like \xref{bkm:Ref117604057}, as well as in the \NI+\RM{} construction \xref{bkm:Ref117604067}.

%%EAX
\ea
%%JUDGEMENT
%%LABEL
\label{bkm:Ref117604057}
%%CONTEXT
%%LINE1
Téné  mûnó  mbiti  na  kay\'{û}g\'{û} \textbf{baar\'{î}} acooré.\\
%%LINE2
\gll
tene  mûno  m-biti  na  ka-yûgû ba-a-rî a-coore\\
%%LINE3
long \INT{} 9-hyena  and  12-hare 2\SM-\PST{} -be 2-friend\\
%%TRANS1
\glt
‘A long time ago, Hyena and Hare were friends.’
%%TRANS2
%%EXEND

\ex
\label{bkm:Ref117604067}
\begin{xlist}
\exi{A:} Are you talking about the pen that Peter was given by the teacher?\\
%%EAX
\exi{B:}
%%JUDGEMENT
%%LABEL
%%CONTEXT
No,…\\
%%LINE1
%%LINE2
\gll
Karámu \textbf{ka-a-rî} ka-rá  Kaw\'{î}ra  a-gûr-\'{î}îté.\\
%%LINE3
12-pen 12\SM{}-\PST{}-be 12-\RM{} 1.Kawîra 1\SM{}-buy-\PFV{}\\
%%TRANS1
\glt
‘The pen was the one that Kawîra bought.’
%%TRANS2
%%EXEND
\end{xlist}

\z

Crucially, this tense/aspect-based variation is ungrammatical in the \NI-\PRO{} construction – see \xref{bkm:Ref111791469:b}. This not only shows us that the \NI-\PRO{} construction behaves differently from the \NI+\RM{} construction, but also that \textit{ni} in the \NI-\PRO{} construction does not fully function like a copula between the initial NP and the rest – an important ingredient for the analysis.

\ea
\label{bkm:Ref111791469}
%%EAX
\ea
%%JUDGEMENT
[]{
%%LABEL
\label{bkm:Ref111791469:a}
%%CONTEXT
%%LINE1
%%LINE2
\gll
Ny-oombá  ni-y-ó  î-rá-b\'{î}-íre  (y-óónka).\\
%%LINE3
9-house \COP{}-9-\PRO{} 9\SM-\YPST{}-burn-\PFV{} {\db}9-only\\
%%TRANS1
\glt
‘(Only) the house is what has burnt.’
%%TRANS2
}
%%EXEND

%%EAX
\ex
%%JUDGEMENT
[*]{
%%LABEL
\label{bkm:Ref111791469:b}
%%CONTEXT
%%LINE1
%%LINE2
\gll
Ny-oombá \textbf{y-aa-rî}-yó  î-rá-b\'{î}-íre  (yóónka).\\
%%LINE3
9-house 9\SM-\PST{}-be-\PRO{} 9\SM-\YPST{}-burn-\PFV{} {\db}9-only\\
%%TRANS1
\glt
int. ‘(Only) the house was what has burnt.’
%%TRANS2
}
%%EXEND

\z
\z


Thirdly, in the \NI-\PRO{} construction the focus is on the referent of the preverbal NP, whereas in the \NI+\RM{} construction it cannot be. This is visible in the comparison of the context questions in \xref{bkm:Ref111719190} for the two constructions, as well as the question-answer pair in \xref{bkm:Ref111722601}. We have bolded the focus.\largerpage[2.25]

\ea
\label{bkm:Ref111719190}
%%EAX
\ea
%%JUDGEMENT
%%LABEL
%%CONTEXT
(Context 1: Is Joy doing the dishes?\\
Context 2: Who is washing the dishes?\\
\textsuperscript{\#}Context 3: I am looking for Rob, who is working in a kitchen full of people. I ask ‘Who is Rob?’)\\
%%LINE1
%%LINE2
\gll
\textbf{Rob}  n’-[\textbf{we}  û-kû-thaamb-i-a  thaáni.]\\
%%LINE3
1.Rob \COP-1.\PRO{} 1\RM-\PRS{}-clean-\IC{}-\FV{} 10.plate\\ \jambox*{}
%%TRANS1
\glt
‘Rob, it’s him washing the dishes.’
%%TRANS2
%%EXEND

%%EAX
\ex
%%JUDGEMENT
%%LABEL
%%CONTEXT
(\textsuperscript{\#}Context 1: Is Joy doing the dishes?\\
\textsuperscript{\#}Context 2: Who is washing the dishes?\\
Context 3: I am looking for Rob, who is working in a kitchen full of people. I ask ‘Who is Rob?’)\\
%%LINE1
%%LINE2
\gll
Rob  n’-[\textbf{ûra}  \textbf{û-kû-thaamb-i-a}  \textbf{thaani.]} \\
%%LINE3
1.Rob \COP{}-1.\RM{} 1\RM-\PRS{}-clean-\IC{}-\FV{}  10.plate\\
%%TRANS1
\glt
‘Rob is the one who is washing the dishes.’
%%TRANS2
%%EXEND

\z

%%EAX
\ex
%%JUDGEMENT
%%LABEL
\label{bkm:Ref111722601}
%%CONTEXT
(Where is a good place to go on a holiday? Interpretation: there are no other places that are good.)\\
%%LINE1
%%LINE2
\gll
Nanyukî  i-kû  kw-éégá.\\
%%LINE3
9.Nanyuki \COP{}-17.\PRO{} 17-good\\
%%TRANS1
\glt
‘Nanyuki is the place that is good.’
%%TRANS2
%%EXEND

\z

Concluding that the \NI-\PRO{} construction is not structurally a reverse pseudocleft (and thus not a copular construction), we want to know what the structure and interpretation of this construction are. As we have seen, the initial NP seems to be focused. One analysis is therefore that there is a focus position sentence-initially, in addition to the focus in the basic cleft/focus construction, but marked differently. However, there is also the post-copular independent pronoun that is co-indexed with the initial NP. Instead of taking the preverbal NP as the focus constituent, we will argue that the structure is best analysed as an initial NP followed by a basic cleft/focus construction in which the pronoun is the focused constituent: NP [\textsc{foc/cop} \PRO{} V], translatable as ‘Rob, it is \textit{him} that is doing the dishes’. This analysis fits better with the tendency for post-copular focus (see \xref{bkm:Ref111708000} and \xref{bkm:Ref111720154} above): focus is not necessarily on the initial NP but instead on the post-copular pronoun (the \textit{kyo} in \textit{ikyo}). As they both refer to the same referent, however, we get the impression that the initial NP represents the focus.

Regarding the second part being analysed as a basic cleft/focus construction, it was already shown in \xref{bkm:Ref111821091} above that a pronoun can be focused in the basic cleft/focus construction, with the pronoun referring to a contextually accessible referent – the same is illustrated in \xref{bkm:Ref111724336}.

%%EAX
\ea
%%JUDGEMENT
%%LABEL
\label{bkm:Ref111724336}
%%CONTEXT
%%LINE1
%%LINE2
\gll
I-ky-o  gî-tem-ag-a  n-gûkû    î-kurur-a.\\
%%LINE3
\COP{}-7-\PRO{} 7\SM{}-make-\HAB-\FV{} 9-chicken  9\SM{}-crow-\FV{}\\
%%TRANS1
\glt
‘That’s why Chicken crows.’
%%TRANS2
lit. ‘It is {\itshape that} that makes Chicken crow.’
%%EXEND

\z
\largerpage


Now let us turn to the initial NP in this construction. If it were the focus, we would expect it to be able to host (inherently focused) interrogative phrases. The fact that the initial NP cannot be a true interrogative, as in \xref{bkm:Ref111724189} and \xref{bkm:Ref111724190}, shows us that the initial position is not a normal focus position. The examples in \xref{bkm:Ref111724189} and \xref{bkm:Ref111724190} were indicated by the speakers to be felicitous only in an echo context. Had the initial position been the focus, we would expect it to be able to host a question word and the structure to be used as a true content question.

%%EAX
\ea
%%JUDGEMENT
%%LABEL
\label{bkm:Ref111724189}
%%CONTEXT
(Context: You overhear that someone was tasked to wash the plates but you didn’t get the name right.)\\
%%LINE1
%%LINE2
\gll
Ûû  n’-we  û-kû-thaamb-i-a  thaáni?\\
%%LINE3
1.who \COP{}-1.\PRO{} 1\RM-\PRS{}-clean-\IC{}-\FV{} 10.plate\\
%%TRANS1
\glt
‘\textit{Who} (is the one who) is washing the dishes?’
%%TRANS2
%%EXEND

%%EAX
\ex
%%JUDGEMENT
%%LABEL
\label{bkm:Ref111724190}
%%CONTEXT
(Context: You overhear that something got burnt but you didn’t actually get what it was.)\\
%%LINE1
%%LINE2
\gll
Mbi  ni-y-ó  î-rá-b\'{î}-íre?\\
%%LINE3
9.what \COP{}-9-\PRO{} 9\SM-\YPST{}-burn-\PFV{}\\
%%TRANS1
\glt
‘\textit{What} (is it that) has burnt?’
%%TRANS2
%%EXEND

\z


If the initial NP is independent of the following basic cleft, the possible prosodic break after the NP is also predicted, as in \xref{bkm:Ref111724334} and \xref{bkm:Ref111724336}. This break is unexpected if the initial NP were in a focus position as part of a focus construction.

%%EAX
\ea
%%JUDGEMENT
%%LABEL
\label{bkm:Ref117504186}
%%CONTEXT
(Whose pen did the doctor borrow?)\\
%%LINE1
Karámú  gáakwá,  ikó  ndagitárí  araroombíre.\\
%%LINE2
\gll
ka-ramu  ka-akwa,  ni-ka-o  n-dagitari  a-ra-roomb-ire.\\
%%LINE3
12-pen \textsc{12-1sg.poss} \COP{}-12-\PRO{} 9-doctor 1\SM-\YPST{}-borrow-\PFV{}\\
%%TRANS1
\glt
‘My pen is what the doctor borrowed.’ / ‘My pen, it is \textit{that} that the doctor borrowed.’
%%TRANS2
%%EXEND

%%EAX
\ex
%%JUDGEMENT
%%LABEL
\label{bkm:Ref111724334}
%%CONTEXT
(There is an egg, where did it come from?)\\
%%LINE1
%%LINE2
\gll
N-gûkû   î-nú,  n’  îy-ó  î-ra-ciár-ire.\\
%%LINE3
9-chicken  9-\DEM.\MED{} \FOC{} 9-\PRO{} 9\SM-\YPST{}-give.birth-\PFV{}\\
%%TRANS1
\glt
‘That chicken is who laid it.’ / ‘That chicken, it’s \textit{her} who laid it.’ (pointing at chicken)
%%TRANS2
%%EXEND

\z

Furthermore, the NP may be separated from the basic cleft, for example as in \xref{bkm:Ref117503860}, where the adverb \textit{nûûmba} intervenes between the contextually active \textit{aga} ‘here’ and the cleft/focus construction \textit{n’oo} ‘it is here’.\pagebreak

%%EAX
\ea
%%JUDGEMENT
%%LABEL
\label{bkm:Ref117503860}
%%CONTEXT
(One day, she went to a river, a very calm pool. If you listened you couldn’t hear anything talking.)\\
%%LINE1
\'{A}ugá \'{î}nd\'{î} \textbf{âgá} n\'{û}ûmba n’\textbf{oó} ngaatééthekerá.\\
%%LINE2
\gll
a-ug-a  îndî a-ga nûûmba ni o n-ka-teth-îk-îr-a\\
%%LINE3
1\SM{}-say-\FV{} now 16-\DEM.\PROX{} maybe \textbf{\FOC{}} 16.\PRO{}  1\SG.\SM-\FUT{}-help-\ABL-\APPL-\FV{}\\
%%TRANS1
\glt
‘She said, “Probably, this is where I will get help”.’
%%TRANS2
%%EXEND

\z


So what is the function of the initial NP? Its status seems to be ambiguous, and will need more research for it to become completely clear. We propose that in at least some contexts it is a topic, for example in \xref{bkm:Ref111791469}. The \NI-\PRO{} construction then forms a way to indicate both the highly accessible status of the referent (by the topical NP) and also the focus function (by the clefted pronoun).

On the other hand, in some context we see that the initial NP seems to function as a fragment answer. This is clear in \xref{bkm:Ref117504186} above, for example, and also when we look at the exhaustive focus-sensitive particle ‘only’: this cannot modify a topic NP, but it can be used in a fragment answer, as shown in \xref{bkm:Ref116899546}. In the \NI-\PRO{} construction, we see that it can modify the clefted pronoun, as in the preferred structure \xref{bkm:Ref116899515}, but it can also modify the initial NP as in \xref{bkm:Ref111725571}.

\ea
%%EAX
\ea
%%JUDGEMENT
\label{bkm:Ref111725571}
%%LABEL
%%CONTEXT
(How many students wrote a letter?)\\
%%LINE1
Jééní  wéenká  n’wé  áandîkiré    baar\'{û}a.\\
%%LINE2
\gll
Jeeni  we-onka  ni-wé  a-andîk-ire    baar\'{û}a.\\
%%LINE3
1.Jane  1-only \COP-1\PRO{} 1\SM{}-write-\PFV{} 9.letter\\
%%TRANS1
\glt
‘Only Jane wrote a letter.’
%%TRANS2
%%EXEND

%%EAX
\ex
%%JUDGEMENT
%%LABEL
\label{bkm:Ref116899515}
%%CONTEXT
%%LINE1
Jééní  n'wé  wéenká  áandîkiré  baar\'{û}a.\\
%%LINE2
{
\gll
Jeeni   ni-wé  we-onka a-andîk-ire    baar\'{û}a.\\
%%LINE3
1.Jane \COP-1\PRO{} 1-only  1\SM{}-write-\PFV{} 9.letter\\
}\jambox*{(preferred)}
%%TRANS1
\glt
‘Only Jane wrote a letter.’
%%TRANS2
%%EXEND

%%EAX
\ex
%%JUDGEMENT
%%LABEL
\label{bkm:Ref116899546}
%%CONTEXT
%%LINE1
%%LINE2
\gll
Jééní  wée-nká.\\
%%LINE3
1.Jane  1-only\\
%%TRANS1
\glt
'Only Jane.'\\
\z\z
%%TRANS2
%%EXEND



Topics and fragment answers cannot be the whole story, though. We find a little conundrum when we try the word ‘person’ as the initial NP, as illustrated in \xref{bkm:Ref116899601}. It cannot be interpreted as a (specific or non-specific) indefinite. Instead, a generic interpretation can be forced, i.e. a human being (not a goat), or else ‘Muntu’ has to be interpreted as the name of a person. This is indicated in the three interpretations a-b-c under the example. For interpretation c, it is possible to add a prosodic break after \textit{Muntû}, and this can be an answer to ‘Who did you see this morning?’ (context 3). In contrast, in interpretation b, a prosodic break is not very acceptable, and it can be used in contexts 1 and 2 but not 3.

\ea
\label{bkm:Ref116899601}
(Context 1: Is it a goat or a person that entered your house?\footnote{A basic cleft would also be acceptable as the answer here.}\\Context 2: I heard a goat entered your house this morning. No, …\\Context 3: Who did you see this morning?)\\
Muntû  n’wé  mbóniré.\\
\gll mu-ntû  ni we  n-on-ire\\
1-person \FOC{}  1.\PRO{}  1\SG.\SM{}-see-\PFV{}\\
\glt 

\ea[*]{‘Someone I saw.’}
\ex[]{‘A person/human being is who I saw.’ (not a goat) \\
contexts 1: \cmark{} 2: \cmark{} 3: \xmark}
\ex[]{‘Muntu is who I saw.’\\
context 3: \cmark{}}
\z
\z

If the construction is to be analysed as a fragment answer plus cleft in all three contexts, then we would expect the initial NP to be acceptable as a fragment in all three contexts. The fragment answer in \xref{bkm:Ref111824017}, however, can only be used in context A but not B (and is uninformative in context C – unless interpreted as a proper name). This means that the \NI-\PRO{} construction with interpretation b above can be analysed structurally as a fragment answer + cleft structure in context A, but must be analysed differently in context B.

%%EAX
\ea
%%JUDGEMENT
%%LABEL
\label{bkm:Ref111824017}
%%CONTEXT
%%LINE1
%%LINE2
\gll
Muntû.\\
%%LINE3
1.person\\
%%TRANS1
\glt
‘A person/human being.’\\
%%TRANS2
1:  \cmark{} 2: \xmark{} 3: \xmark{}
%%EXEND

\z

Further data and targeted investigation are needed to reveal the syntactic and interpretational status of the initial NP in this construction. For now, we conclude that in some contexts it functions as a topic and in some as a fragment answer, and move our attention to the cleft-part of the construction.

If the \NI-\PRO{} construction features a basic cleft/focus construction, then we expect the same properties to hold as in that contruction, that is, an exhaustive interpretation. Indeed, the answer to an incomplete question using the \NI-\PRO{} construction as in \xref{bkm:Ref111986751} starts with ‘no’, as in the basic cleft and the pseudocleft. This negates the exhaustivity in the question, showing that the focus is interpreted as exhaustive.

\ea
\label{bkm:Ref111986751}
\begin{xlist}
%%EAX
\exi{Q:}
%%JUDGEMENT
%%LABEL
%%CONTEXT
%%LINE1
Twaar\'{î} itú tûthaambiiríé thaání?\\
%%LINE2
\gll
tû-aarî  ni  tu  tû-thaamb-i-ire  thaani\\
%%LINE3
13-girl \FOC{}  \PRO{}  13\SM{}-clean-\IC{}-\PFV{} 10.plate\\
%%TRANS1
\glt
‘The girls, is it them who washed plates?’
%%TRANS2
%%EXEND

%%EAX
\exi{A:}
%%JUDGEMENT
%%LABEL
%%CONTEXT
%%LINE1
Ar\'{î} ti twaar\'{î} túunká (tûthaambiirie thaani). I twaarí na tw\'{î}y\'{î}.\\
%%LINE2
\gll
arî  ti  tû-aarî  tû-onka  tû-thaamb-i-ire  thaani.  ni  tû-aarî  na  tû-îyî\\
%%LINE3
no \NEG.\COP{} 13-girl  13-only 13\SM{}-clean-\IC{}-\PFV{} 10.plate \FOC{} 13-girl  and  13-boy\\
%%TRANS1
\glt
‘No, it is not girls only (that washed the plates). It is boys and girls.’
%%TRANS2
%%EXEND

\end{xlist}
\z

Furthermore, the unacceptability of the initial NP to be modified by ‘even’ is expected if the \NI-\PRO{} construction is fundamentally the same as the basic cleft/focus construction.\footnote{Note that ‘even’ is grammatical in a fragment answer. For example, a question ‘Did Jane write a letter?’ can be answered by \textit{wana Kanyua} ‘even Kanyua’.} The construction was improved as in \xref{bkm:Ref111987054:b}, where \textit{nawé} marks \textit{wana Kanyua} as a topic (and see \sectref{bkm:Ref115790636} on \textit{nawe}).

\ea
\label{bkm:Ref111987054}
%%EAX
\ea
%%JUDGEMENT
[*]{
%%LABEL
\label{bkm:Ref111987054:a}
%%CONTEXT
%%LINE1
Wana Kanyúá n’wé aandîkiré baar\'{û}a.\\
%%LINE2
\gll
wana  Kanyua  ni  we  a-andîk-ire  baarûa\\
%%LINE3
even  1.Kanyua \FOC{}  1.\PRO{} 1\SM{}-write-\PFV{} 9.letter\\
%%TRANS1
\glt
int. \textsuperscript{?}’Even Kanyua is the one who wrote a letter.’
%%TRANS2
}
%%EXEND

%%EAX
\ex
%%JUDGEMENT
[]{
%%LABEL
\label{bkm:Ref111987054:b}
%%CONTEXT
%%LINE1
Wana Kanyúá nawé n’aandîkiré baar\'{û}a.\\
%%LINE2
\gll
wana  Kanyua  na=we  ni  a-andîk-ire  baarûa\\
%%LINE3
even  1.Kanyua  and=1.\PRO{} \FOC{}  1\SM{}-write-\PFV{} 9.letter\\
%%TRANS1
\glt
‘Even Kanywa (she too) wrote a letter.’
%%TRANS2
} 
%%EXEND

\z
\z


Additionally, checking the indefinite interpretation of \mbox{-\textit{mwe}} ‘some’ in the construction, we find that just as in the basic cleft, a subset reading is necessary, contrasting `some' with `all'.\footnote{Note that ‘some’ can also modify a fragment answer, as in \xref{ex:treescows}.

\ea
\label{ex:treescows}
\begin{xlist}
\exi{Q:} You had trees, cows, goats, maize. What did you sell?\\
%%EAX
\exi{A:}
%%JUDGEMENT
%%LABEL
%%CONTEXT
%%LINE1
%%LINE2
\gll
Mî-t\'{î} î-mwé.\\
%%LINE3
4-tree  4-one\\
%%TRANS1
\glt
‘Some trees.’
%%TRANS2
%%EXEND

\end{xlist}
\zlast
}

%%EAX
\ea
%%JUDGEMENT
%%LABEL
%%CONTEXT
(Did all the chickens escape?)\\
%%LINE1
Ng\'{û}k\'{û} ímwé ició círátooróókire.\\
%%LINE2
\gll
n-gûkû  i-mwe  ni  ci-o  ci-ra-toorook-ire\\
%%LINE3
10-chicken  10-one \FOC{}  10-\PRO{} 10\SM-\YPST{}-escape-\PFV{}\\
%%TRANS1
\glt
‘Some of the chickens are the ones who escaped.’
%%TRANS2
%%EXEND

\z

In summary, there are clear arguments against analysing the \NI-\PRO{} construction as a sort of pseudocleft, and instead we propose that it is an initial NP functioning as a topic or fragment answer, plus a basic cleft/focus construction in which a pronoun coreferential with the initial NP is in exhaustive focus. The results of all the diagnostics for the three tests, where we have data, are summarised in \tabref{tab:thk-foc-diagnostics}.
\largerpage[1]

% This is the original formatting as a sideways table - it's still too wide, so I transposed the rows and columns.

% \begin{sidewaystable}
% \begin{tabularx}{\textwidth}{XXXXXXXXXXXXX}
% \lsptoprule
%  & wh & answer & only & even & mention some & all & some & numeral & idiom & indef & incompl & answer nothing\\
% \midrule
% basic & OK & OK & OK & x & x & x & subset & exact & x & OK & no & OK\\
% pseudo & OK & OK & OK & x & x & x & x &  & x & x & no & x\\
% \NI-\PRO{} & x & OK & OK & x & x & x & subset &  & x & x & no & n.a.\\
% \lspbottomrule
% \end{tabularx}
% \caption{Diagnostics for focus and exhaustivity in Kîîtharaka cleft constructions.}
% \label{tab:thk-foc-diagnostics}
% \end{sidewaystable}

\begin{table}[h]
\begin{tabular}{lccc}
\lsptoprule
    & basic & pseudo & \NI-\PRO{} \\
\midrule
wh & \cmark & \cmark & \xmark \\
answer & \cmark & \cmark & \cmark \\
only & \cmark & \cmark & \cmark \\
even & \xmark & \xmark & \xmark \\
mention some & \xmark & \xmark & \xmark \\
all & \xmark & \xmark & \xmark \\
some & subset & \xmark & subset \\
numeral & exact &       &       \\
idiom & \xmark & \xmark & \xmark \\
indef & \cmark & \xmark & \xmark \\
incompl & no & no & no \\
answer nothing & \cmark & \xmark & N/A \\
\lspbottomrule
\end{tabular}
\caption{Diagnostics for focus and exhaustivity in Kîîtharaka cleft constructions.}
\label{tab:thk-foc-diagnostics}
\end{table}

\section{Object marking and object drop}
\label{bkm:Ref115790593}
The information status of a referent influences the expression used to refer to it. In Kîîtharaka, whether an object is expressed overtly or dropped depends on the interaction of several factors. They include accessibility hierarchy, predicate type and referential properties (whether human or non-human). We will look at each of them with a view to understanding their roles in object drop, drawing on \citet{Kanampiu2023}.

\subsection{Accessibility status}

We hypothesize that the Accessibility Theory by \citet{Ariel2001} (and earlier works), and related works by \citet{Chafe1976,Chafe1987}, \citet{GundelEtAl1993} and \citet{Givón1984a}, can be used to explain the paradigm of object marking and drop in Kîîtharaka. As explained in the introductory chapter, according to \citet{Ariel2001}, referring expressions mark varying degrees of mental accessibility of referents, in an inverse relation between linguistic encoding and accessibility: A new referent is inaccessible, and thus has the most linguistic encoding (full NP, or NP + modifier when being reactivated), whereas a previously mentioned referent is considered accessible, and hence requires less linguistic encoding (with zero anaphora being the lowest in the scale). The degree of accessibility forms a scale referred to as the ``accessibility hierarchy". Accordingly, the Kîîtharaka accessibility hierarchy for object referents can be constructed as follows:

\ea {Lexical NP}+{modifier}  > {Lexical NP} > {DEM} > {Pron/SM/OM/PRO} > {Object Drop}
 \z

This accessibility hierarchy correctly predicts that new referents are expressed with an expression at the far left end of the hierarchy (full lexical NP). This is illustrated in example \xref{bkm:Ref120115686} and \xref{bkm:Ref120115709}, where a new referent (the guinea fowl and the wound, respectively) is introduced into the narrative.

%%EAX
\ea
%%JUDGEMENT
%%LABEL
\label{bkm:Ref120115686}
%%CONTEXT
(Context: Hare and Hyena went to the bush to hunt. They hunted for some time.)\\
%%LINE1
Bakîgw\'{î}mága gwa ûgú, booná nkáánga.\\
%%LINE2
\gll
ba-kî-guîm-ag-a  wa  ûgu  ba-on-a  n-kanga\\
%%LINE3
2\SM-\DEP{}-hunt-\HAB-\FV{} like  that 2\SM{}-saw-\FV{} 9-guinea.fowl\\
%%TRANS1
\glt
‘As they were hunting, they saw a guinea fowl.’
%%TRANS2
%%EXEND

\z
%%EAX
\ea
%%JUDGEMENT
%%LABEL
\label{bkm:Ref120115709}
%%CONTEXT
(Context: A woman was injured by a stick while collecting firewood. The stick was poisonous.)\\
%%LINE1
Aagea na kîronda kînéne mûno kûgûrû.\\
%%LINE2
\gll
a-a-ge-a  na  kî-ronda  kî-nene  mûno  kû-gûrû\\
%%LINE3
1\SM-\PST{}-develop-\FV{} with  7-wound  7-big \INT{} 17-leg\\
%%TRANS1
\glt
‘She developed a very big wound on the leg.’
%%TRANS2
%%EXEND

\z


Apart from introducing a purely new referent, full lexical NPs are used to reactivate referents at paragraph or episode boundaries. If a referent is less accessible because it was mentioned several paragraphs behind, it may be expressed in the form of full NP + relative clause modifier as illustrated in \xref{bkm:Ref120116164}.

%%EAX
\ea
%%JUDGEMENT
%%LABEL
\label{bkm:Ref120116164}
%%CONTEXT
(Context: They saw a guinea fowl. Hyena took the arrows and aimed, shot an killed it. They became very happy.)\\
%%LINE1
Mbítí na Kay\'{û}g\'{û} báákámata \textbf{nkáánga} \textbf{îrá bá\'{û}ragiré}, báathí n’yó bántû kathakáaní.\\
%%LINE2
\gll
mbiti  na  ka-yûgû  ba-a-kamat-a n-kaanga  î-ra  ba-ûrag-ire bá-a-thi  na=yo  ba-ntû  ka-thaka=ini \\
%%LINE3
1.Hyena  and  1-Hare  2\SM{}-\PST{}-carry-\FV{} 9-guinea.fowl  9-\RM{} 2\SM{}-kill-\PFV{} 2\SM-\PST{}-go  with=9.\PRO{} {\db}16-someplace  12-bush=\LOC{} \\
%%TRANS1
\glt
‘Hyena and Hare carried the guinea fowl that they had killed to some place in the bush.’
%%TRANS2
%%EXEND

\z

One step further on the hierarchy, demonstratives can be used to refer to participants, actions or locations that are obviously identifiable by the hearer, either adnominally or pronominally, as illustrated for \textit{aga} ‘here’ in \xref{bkm:Ref116888088}.

%%EAX
\ea
%%JUDGEMENT
%%LABEL
\label{bkm:Ref116888088}
%%CONTEXT
(Context: The lady sees a handsome young man who she thinks is interested in her daughter. She decides to take charge.)\\
%%LINE1
Ntitigá mwaána ûy\'{û} \textbf{áagá}.\\
%%LINE2
\gll
ni  n-ti-tig-a  mû-ana  û-yû aga\\
%%LINE3
\FOC{}  1\SG.\SM-\NEG{}-leave-\FV{} 1-child  1-\DEM.\PROX{} 16.\DEM.\PROX{}\\
%%TRANS1
\glt
‘I can’t leave this young man here.’
%%TRANS2
%%EXEND

\z


Demonstratives may also be used exophorically, as in \xref{bkm:Ref116888125}, where \textit{ûyû} ‘this one’ refers to the person who slapped another person, and \textit{aagá} refers to a place where s/he has been slapped. Both are accompanied by a pointing gesture.

\ea
\label{bkm:Ref116888125}
%%EAX
\ea
%%JUDGEMENT
%%LABEL
%%CONTEXT
(Who slapped you?) \\
%%LINE1
N’\'{û}yû.\\
%%LINE2
\gll
ni  ûyû\\
%%LINE3
\COP{}  1.\DEM.\PROX{}\\
%%TRANS1
\glt
‘It is this one.’
%%TRANS2
%%EXEND


%%EAX
\ex
%%JUDGEMENT
%%LABEL
%%CONTEXT
(Where did he slap you?)\\
%%LINE1
N’áagá.\\
%%LINE2
\gll
ni  aga\\
%%LINE3
\COP{}  16.\DEM.\PROX{}\\
%%TRANS1
\glt
‘It is here.’
%%TRANS2
%%EXEND

\z
\z

In \xref{bkm:Ref120116261}, \textit{ûgu} ‘that’ refers to the swelling and beating of wings by the chicken, ostensibly to frighten the mongooses; this is an example of discourse deixis, referring to the earlier described actions.

%%EAX
\ea
%%JUDGEMENT
%%LABEL
\label{bkm:Ref120116261}
%%CONTEXT
%%LINE1
Mîrûngûûru yóoná ûgu…\\
%%LINE2
\gll
mî-rûngûûru  î-a-on-a  ûgu\\
%%LINE3
4-mongooses 4\SM-\PST{}-see-\FV{} \textsc{14.\DEM.\MED{}}\\
%%TRANS1
\glt
‘When the mongooses saw that…’
%%TRANS2
%%EXEND

\z


Highly active objects can be expressed by an object marker on the verb, a pronoun, or simply dropped (zero expression). This choice, however, seems to be determined by factors other than information status\slash accessiblity, involving syntactic environment (DP object vs. PP or predicate), predicate type, and animacy of the referent. We briefly discuss these here and refer to \citet{Kanampiu2023} for further details.

\subsection{Object marker vs. independent pronoun}

Pronominalising active objects happens in two ways: either by an object marker on the verb, or by a separate pronoun. In example \xref{bkm:Ref116889996}, repeated from above, the referent \textit{nkaanga} is introduced as a full NP, and in the next line \xref{bkm:Ref132196481}, it is referred to in the form of an object marker as it is now active.

%%EAX
\ea
%%JUDGEMENT
%%LABEL
\label{bkm:Ref116889996}
%%CONTEXT
%%LINE1
Bakîgw\'{î}mága gwa ûgú, booná nkáánga.\\
%%LINE2
\gll
ba-kî-guîm-ag-a  gwa  ûgu  ba-on-a  n-kanga\\
%%LINE3
2\SM-\DEP{}-hunt-\HAB-\FV{} like that 2\SM{}-saw-\FV{} 9-guinea.fowl\\
%%TRANS1
\glt
‘As they were hunting, they saw a guinea fowl.’
%%TRANS2
%%EXEND

\z

%%EAX
\ea
%%JUDGEMENT
%%LABEL
\label{bkm:Ref132196481}
%%CONTEXT
%%LINE1
Mbiti yáárutá mûgwí yáugiá bwaa, ya\textbf{mî}rathá, ya\textbf{mî}ûraga.\\
%%LINE2
\gll
m-biti  î-a-rut-a  mu-guî  î-a-ug-i-a  bwaa î-a-mî-rath-a  î-a-mî-u-rag-a \\
%%LINE3
9-hyena 9\SM-\PST{}-remove-\FV{} 3-arrow 9\SM-\PST{}-do-\IC-\FV{} \IDEO{} 9\SM-\PST{}-9\OM{}-shoot-\FV{} 9\SM-\PST{}-9\OM{}-kill-\FV{} \\
%%TRANS1
\glt
‘Hyena took the arrows and aimed, shot and killed it.’
%%TRANS2
%%EXEND

\z

The separate pronoun occurs as an enclitic after the preposition \textit{na} ‘with’, as in \xref{bkm:Ref116889769}, but it also occurs when pronominalising two objects, as in \xref{bkm:Ref116889753}. As there is only one slot for object marking on the verb, the Recipient/Benefactive must be marked there, and a pronominalised Theme object can optionally be expressed as a separate pronoun postverbally.

\ea
\label{bkm:Ref116889769}
%%EAX
\ea
%%JUDGEMENT
%%LABEL
%%CONTEXT
%%LINE1
Kîmathi náayire na kîbanga?\\
%%LINE2
\gll
Kimathi  ni  a-y-ire  na  kî-banga\\
%%LINE3
Kîmathi \FOC{}  1\SM{}-come-\PFV{} with  7-panga\\
%%TRANS1
\glt
‘Did Kimathi bring (lit. come with) a panga?'
%%TRANS2
%%EXEND

%%EAX
\ex
%%JUDGEMENT
%%LABEL
%%CONTEXT
%%LINE1
Yii, náayire na*(kîo).\\
%%LINE2
\gll
yii  ni  a-ya-ire  na=kî-o\\
%%LINE3
Yes, \FOC{}  1\SM{}-come-\PFV{} with=7-\PRO{}\\
%%TRANS1
\glt
‘Yes, he brought (lit. came with) it.’
%%TRANS2
%%EXEND

\z

%%EAX
\ex
%%JUDGEMENT
%%LABEL
\label{bkm:Ref116889753}
%%CONTEXT
(Did you give Mary the carrots?)\\
%%LINE1
Yii i *(mû)néénkeeré (cio).\\
%%LINE2
\gll
yii   ni  n-mû-nenker-ire  ci-o\\
%%LINE3
yes \FOC{}  1.\SG.\SM{}-1\OM{}-give-\PFV{}  10-\PRO{}\\
%%TRANS1
\glt
‘Yes I gave her (them).’
%%TRANS2
%%EXEND

\z

The -\textit{o} pronoun is also used in topic marking to achieve a contrastive reading, as well as a shift topic marker (see \sectref{bkm:Ref115790636} for a description and examples). It is not encountered in our spontaneous texts as a means to mark contrastive focus~– only in the \NI+\PRO{} construction do we find it as such (see \sectref{bkm:Ref116891458}).

\subsection{Object marking vs. object drop}

Basically, transitive verbs are expected to take objects. However, not all active objects that could be expressed as an object marker are actually marked, and objects may simply be dropped. Whether an object marker occurs or not seems to depend on various factors (but more research is necessary to establish the exact interactions and circumstances): dislocation, predicate type, humanness, and something like prototypicality or expectedness.\footnote{We acknowledge the MA thesis by Leiden University student Dominique Loviscach that helped clarify some of the factors.} As these are not directly relevant for information structure, we only touch upon them briefly.

First, an object marker may never be present in the same domain as the coreferring object NP. That is, the object marker functions like a true pronoun, and a coreferring NP must therefore be in a dislocated position to the right or left of the sentence. Second, some predicates seem to require object marking of active referents, here illustrated by left-dislocated objects. Predicates of perception like -\textit{ona} ‘see’, -\textit{enda} ‘like’ \xref{bkm:Ref125199162} and -\textit{îgua} ‘feel’ \xref{bkm:Ref125199180} must be accompanied by object markers when the object is left dislocated, and telic predicates seem to prefer object marking more so than atelic predicates. Object drop is not licensed in such cases, regardless of the properties of the object (noun class, humanness). An overview of which predicates require or allow object marking is still outstanding.\largerpage[2]

%%EAX
\ea
%%JUDGEMENT
%%LABEL
\label{bkm:Ref125199162}
%%CONTEXT
%%LINE1
Kîthére, i n*(kî)endeet\'{e}.\\
%%LINE2
\gll
kî-there  ni  n-kî-end-îte\\
%%LINE3
7-kitheri \FOC{}  1\SG.\SM{}-7\OM{}-like-\PFV{}\\
%%TRANS1
\glt
‘Githeri,\footnote{A meal of a mixture of maize and beans.} I like it.’
%%TRANS2
%%EXEND

\z

%%EAX
\ea
%%JUDGEMENT
%%LABEL
\label{bkm:Ref125199180}
%%CONTEXT
%%LINE1
Mûr\'{î}ó, i ngû*(cú)\'{î}gua.\\
%%LINE2
\gll
mû-rîo  ni    n-kû-cu-îgu-a\\
%%LINE3
3-sweetness \FOC{}   1\SG.\SM-\PRS{}-3\OM{}-feel-\FV{}\\
%%TRANS1
\glt
‘Sweetness, I feel it.’
%%TRANS2
%%EXEND

\z


For other predicates, object marking seems to depend mainly on humanness of the object, and noun class. Transitive verbs require obligatory object marking with dislocated class 1 and 2 human objects, as shown in \xref{bkm:Ref125199202} and \xref{bkm:Ref125199217}. Non-human objects are, however, optionally object marked, as in \xref{bkm:Ref125199236} and \xref{bkm:Ref116893481}, and humans in class 12/13 also receive optional object marking \xref{bkm:Ref116892950}.\largerpage

%%EAX
\ea
%%JUDGEMENT
%%LABEL
\label{bkm:Ref125199202}
%%CONTEXT
%%LINE1
Mwarímû nká*(mû)t\'{û}ma.\\
%%LINE2
\gll
mû-aarimû  n-ka-mû-tûm-a\\
%%LINE3
1-teacher  1\SG.\SM{}-\FUT{}-1\OM{}-send-\FV{}\\
%%TRANS1
\glt
‘The teacher, I will send him.’
%%TRANS2
%%EXEND

\z

%%EAX
\ea
%%JUDGEMENT
%%LABEL
\label{bkm:Ref125199217}
%%CONTEXT
%%LINE1
I *(mû)t\'{û}miré, Précious.\\
%%LINE2
\gll
ni  n-mû-tûm-ire  Precious\\
%%LINE3
\FOC{} 1\SG.\SM{}-\OM{}-sent-\PFV{} 1.Precious\\
%%TRANS1
\glt
‘I sent her, Precious.’
%%TRANS2
%%EXEND

\z

%%EAX
\ea
%%JUDGEMENT
%%LABEL
\label{bkm:Ref116892950}
%%CONTEXT
%%LINE1
Twáána ngá(tû)cereria mabuku.\\
%%LINE2
\gll
tû-ana  n-ka-tû-cer-îr-i-a  ma-buku\\
%%LINE3
12-children 1\SG.\SM-\FUT-12\OM{}-find-\APPL-\IC-\FV{} 6-book\\
%%TRANS1
\glt
‘Children, I will find books for them.’
%%TRANS2
%%EXEND

\z

\ea
\label{bkm:Ref125199236}
%%EAX
\ea
%%JUDGEMENT
%%LABEL
%%CONTEXT
%%LINE1
Gîcíáti ngû(kî)gûra.\\
%%LINE2
\gll
kî-ciati  n-kû-kî-gûr-a\\
%%LINE3
7-broom 1\SG.\SM-\PRS{}-7\OM{}-buy-\FV{}\\
%%TRANS1
\glt
‘The broom, I have bought (it).’
%%TRANS2
%%EXEND

%%EAX
\ex
%%JUDGEMENT
%%LABEL
%%CONTEXT
%%LINE1
Ngû(kî)g\'{û}ra, gîcíáti.\\
%%LINE2
\gll
n-kû-kî-gûr-a,  kî-ciati\\
%%LINE3
1\SG.\SM-\PRS{}-7\OM{}-buy-\FV{} 7-broom\\
%%TRANS1
\glt
‘I have bought (it), the broom.’
%%TRANS2
%%EXEND

\z
\z

%%EAX
\ea
%%JUDGEMENT
%%LABEL
\label{bkm:Ref116893481}
%%CONTEXT
(How can I kill a chicken?)\\
%%LINE1
Noá ûríínge na mûraagí…\\
%%LINE2
\gll
nwa  û-riing-e  na  mû-raagi\\
%%LINE3
can  2\SG.\SM{}-hit-\SBJV{} with  3-stick\\
%%TRANS1
\glt
‘... you can hit (it) with a stick...’
%%TRANS2
%%EXEND

\z


The phenomenon of the predicate type influencing object expression can be related to what is observed for Luguru \citep{MartenRamadhani2001} and Kinyakyusa \citep{LusekeloFut}. For these languages, however, the predicates involved are different from the ones in Kîîtharaka. We refer to \citet{Kanampiu2023} for further discussion on the generalisation of predicate types.

A remaining question is what determines the presence or absence of the object marker when the predicate does not require it, and the object is not a human in class 1/2. In her MA thesis, Dominique Loviscach suggests that the likelihood or predictability of the object referent might play a role, where less expected objects are preferably object marked, and it is easier to drop the object (and the marker) for unsurprising objects, e.g. ‘eat rice’ can go unmarked, whereas ‘eat a cat’ would need marking. Verum is another function that might facilitate object marking. As said, these factors need further systematic research.

\subsection{Salience}

Repetition of a full NP, as well as a combination of a demonstrative and full lexical NP can lead to a salience effect, as illustrated in \xref{bkm:Ref125199305}. The referent \textit{ndia} ‘pool’ is introduced in a conversation, then questioned, and in the answer, referring back to the already active referent, speaker A responds by using the NP+DEM combination, \textit{ndia înu} ‘the/that pool’. The use of the NP+DEM for an already active referent (rather than a pronoun) emphasises the fact that this particular pool is very important to the story. It is not just any other pool that is being referred to.

\ea
\label{bkm:Ref125199305}
(A: ‘This lady, her wound will be healed by water fetched from a pool without voices of frogs.’)\\
\begin{xlist}
%%EAX
\exi{B:}
%%JUDGEMENT
%%LABEL
%%CONTEXT
%%LINE1
Î ndia îtakwaria kîûra?\\
%%LINE2
\gll
î  n-dia  î-ta-kû-ari-a  kî-ûra\\
%%LINE3
\PP 9-pool 9\SM-\NEG{}-\PRS{}-speak-\FV{} 7-frog\\
%%TRANS1
\glt
‘A pool without voices of frogs?’
%%TRANS2
%%EXEND


%%EAX
\exi{A:}
%%JUDGEMENT
%%LABEL
%%CONTEXT
%%LINE1
Îî, wona aathaamba na rûûyî rwa \textbf{ndia înu} îtakwaria kîûra.\\
%%LINE2
\gll
îî  wona  a-a-thaamb-a  na  rû-ûyî  rû-a n-dia  î-nu î-ta-kû-ari-a  kî-ûra\\
%%LINE3
yes  \COND{} 1\SM-\PST{}-wash-\FV{} with  11-water  11-\CONN{} 9-pool  9-\DEM.\MED{} 9\SM{}-\NEG{}-\PRS{}-speak-\FV{} 7-frog\\
%%TRANS1
\glt
‘Yes, if she washes with the water from the pool without voices of frogs.’
%%TRANS2
%%EXEND


\end{xlist}
\z

Similarly, the order within the NP may be reversed so that the demonstrative precedes the noun \citep[see][]{KanampiuMuriungi2019}, as in \xref{bkm:Ref116894087}. With this kind of order, the emphasis is on the particular day – not any other day.

%%EAX
\ea
%%JUDGEMENT
%%LABEL
\label{bkm:Ref116894087}
%%CONTEXT
%%LINE1
Înu ntugû Kanyamû Nkió gakauma gakathi…\\
%%LINE2
\gll
î-nu  n-tugû  Kanyamû  Nkio  ka-ka-uma  ka-ka-thi…\\
%%LINE3
9-\DEM.\MED{} 9-day  12.Kanyamû  Nkio 12\SM{}-\SUBS{}-leave 12\SM{}-\SUBS{}-go\\
%%TRANS1
\glt
‘That particular day, Kanyamû Nkio would leave, and go...’
%%TRANS2
%%EXEND

\z

\subsection{Excursion: Subject drop}

While {we have examined the expression of objects in some detail, the expression of subjects also shows an interesting feature. Subjects are prototypically given information and function as topics in Kîîtharaka (although see \sectref{bkm:Ref115858011} for detailed discussion). They determine subject marking on the verb in most cases, and the subject NP can easily be omitted when the referent is active, leaving the subject marker as the expression of the subject – in fact this is the more natural way to refer to active subjects. This is illustrated in \xref{bkm:Ref132273993}: the daughter of the firstly-introduced man and woman is referred to by a NP+possessive, but subsequently only by the subject marker (with the ``missing" NP subject indicated by $\varnothing$}).

%%EAX
\ea
%%JUDGEMENT
%%LABEL
\label{bkm:Ref132273993}
%%CONTEXT
(A woman has been told that her wound will heal with water from a pool without frogs. She and her husband are wondering how they will find this pool.)\\
%%LINE1
Mwarî wake, augá agaacwa ndia înu îtakwaria kîûra.\\
%%LINE2
\gll
mû-arî  w-ake  a-ûg-a  a-ka-cu-a n-dia  î-nu  î-ta-kû-ari-a  kî-ûra \\
%%LINE3
1-girl  1-\POSS{}.1 1\SM{}-say-\FV{} 1\SM-\FUT{}-find-\FV{} 9-pool  9-\DEM.\MED{} 9\SM-\NEG-\PRS{}-speak-\FV{} 7-frog \\
%%TRANS1
\glt
‘His daughter said that she would find that pool without frog voices.’
%%TRANS2
%%EXEND

%%EAX
\sn
%%JUDGEMENT
%%LABEL
%%CONTEXT
%%LINE1
$\varnothing$ Auma agu, $\varnothing$ aathi muuroni.\\
%%LINE2
\gll
a-um-a  a-gu,  a-a-thi  mû-uro=ini\\
%%LINE3
1\SM{}-move-\FV{} 16-\PRO{} 1\SM-\PST{}-go  3-river=\LOC{}\\
%%TRANS1
\glt
‘She left there and went to the river.’
%%TRANS2
%%EXEND
\pagebreak
%%EAX
\sn
%%JUDGEMENT
%%LABEL
%%CONTEXT
%%LINE1
$\varnothing$ Aathi, keenda $\varnothing$ amenya kana kûrî kîûra kîrîku, $\varnothing$ aina “…”.\\
%%LINE2
\gll
a-a-thi  ka-end-a  a-meny-a  kana  kû-rî  kî-ûra  kî-rî=ku  a-in-a\\
%%LINE3
1\SM-\PST{}-go \FUT{}-like-\FV{} 1\SM{}-know-\FV{} if  17-be  7-frog  7\SM{}-be=17 1\SM{}-sing-\FV{}\\
%%TRANS1
\glt
‘She went, so that she could know whether there was a frog in there, and she sang “…”.’
%%TRANS2
%%EXEND

\z


Interestingly, there seems to be a reduced inflection on subsequent verb forms such as \textit{aina} ‘she sang’. But subject marking is still present and cannot be omitted like we saw for object marking. Nevertheless, in narratives, accessible referents that function as the topic and subject for several subsequent clauses may also be dropped altogether: no NP and no subject marker appears, and the verb in these clauses is an infinitive. This is illustrated in \xref{bkm:Ref118270837}, where the Hare (class 12) is the subject marked on the verbs in the first clause, but the verbs in the second and third clause are infinitives. Note also that these clauses are introduced by \textit{na}, showing that they indicate sequential events that are the culmination of the previously mentioned series of events. For further discussion, we refer to \citet{Kanampiu2023}.

%%EAX
\ea
%%JUDGEMENT
%%LABEL
\label{bkm:Ref118270837}
%%CONTEXT
%%LINE1
Kámîgwatá kám\'{î}ogá bwéega, kám\'{î}oga muumá\\
%%LINE2
\gll
ka-mî-guat-a  ka-mî-og-a  bu-ega  ka-mî-og-a  mu-uma\\
%%LINE3
12\SM-9\OM{}-catch-\FV{} 12\SM-9\OM{}-tie-\FV{} 14-well 12\SM-9\OM{}-tie-\FV{} 3-absolutely\\
%%TRANS1
\glt
‘He got hold (of Hyena) and tied him firmly,’
%%TRANS2
%%EXEND
%%EAX
\sn
%%JUDGEMENT
%%LABEL
%%CONTEXT
%%LINE1
{na \textbf{kû}}{r\'{û}ma mûcíoro}\\
%%LINE2
\gll
na kû-rûma  mû-cioro\\
%%LINE3
and 15-bite 3-stick\\
%%TRANS1
\glt
‘then collected a strong stick,’
%%TRANS2
%%EXEND

%%EAX
\sn
%%JUDGEMENT
%%LABEL
%%CONTEXT
%%LINE1
{na \textbf{kû}}{miúgia mmá! mmá! Kámîb\'{û}\'{û}ra.}\\
%%LINE2
\gll
na kû-mî-ug-i-a  mma  mma  ka-mî-buur-a\\
%%LINE3
and 15-9\OM{}-do-\IC-\FV{} \IDEO{}  \IDEO{} 12\SM{}-9\OM{}-beat-\FV{}\\
%%TRANS1
\glt
‘and went on him whack! whack! He beat him.’
%%TRANS2
%%EXEND

\z

In summary, an object referent is typically introduced as a full NP, reactivated when it is accessible by an NP + modifier, and highly active objects may be expressed as an independent pronoun or an object marker, or they may remain unexpressed (object drop). Whether an object can be dropped depends primarily on the predicate and whether the object is a human in class 1/2 or not. Further research is needed to disentangle the precise factors that determine object marking and object drop. Subject NPs can also easily be dropped when referring to active referents. Typically, the subject is visible as subject marking on the inflected verb, but in sequences of events where the subject stays the same (a familiarity topic), the verb may appear in the infinitive form.

 There is one final aspect of Kîîtharaka grammar that we have referred to in various sections, and that is the marker \textit{ni}- preceding the inflected verb. The marker also has a fundamental influence on the information structure, as we discuss in the next section.

\section{Focus marker \textit{ni} on verbs}
\label{bkm:Ref111628808}\label{bkm:Ref119919394}\label{bkm:Ref119919238}
As \citet{Muriungi2005} indicates, Kîîtharaka has three conjugational categories (tense\slash aspect combinations) with a choice for the presence or absence of the preverbal focus marker \textit{ni}-, and three where the absence or presence is fixed, as summarised in \tabref{tab:thk-ni-tenses}.


\begin{table}[ht]
\begin{tabularx}{\textwidth}{lXl}
\lsptoprule
present perfective & SM-kû-VB-a & a-kû-ring-a ‘he has beaten’\\
\multirow{2}{*}{present progressive} & ni SM-kû-VB-a / & \multirow{2}{*}{n’a-kû-ring-a ‘he is beating’}\\
& SM-ri-VB-a & \\
future & SM-ka-VB-a & a-ka-ringa ‘he will beat’\\
perfective & (ni) SM-VB-ire & (n’)a-ring-ire ‘he beat’\\
yesterday past & (ni) SM-ra-VB-ire & (n’)a-ra-ring-ire ‘he beat’\\
remote past & (ni) SM-a-VB-ire & (n’)a-a-ring-ire ‘he beat’\\
\lspbottomrule
\end{tabularx}
\caption{Kîîtharaka tenses and their use of ni}
\label{tab:thk-ni-tenses}
\end{table}

The question we can ask for the ``optional" conjugations (perfective, yesterday past, and remote past) is thus what determines the presence or absence of the marker. Apart from being restricted to these conjugations, there appear to be three more restrictions at the clausal level:\largerpage

\begin{enumerate}\itemsep=.66\itemsep
\item A verb in sentence-final position of a main clause must have the marker;
\item  Negative verb forms and a verb in the embedded clause cannot take \textit{ni}-;
\item When the marker is absent, the focus falls on the postverbal element.
\end{enumerate}

The first restriction is evident in \citet[46]{Muriungi2005} and illustrated in \xref{bkm:Ref132275141}, where the postverbal element can be an object \xref{bkm:Ref132275141:a} or an adjunct \xref{bkm:Ref132275141:d}, but the verb may not be clause-final without the marker \textit{ni} on the verb \xxref{bkm:Ref132275141:b}{bkm:Ref132275141:c}.

\ea
\label{bkm:Ref132275141}
%%EAX
\ea
%%JUDGEMENT
[]{
%%LABEL
\label{bkm:Ref132275141:a}
%%CONTEXT
%%LINE1
Árár\'{û}míírie îng’ooí.\\
%%LINE2
\gll
a-ra-rûm-i-ire  î-ng’ooi\\
%%LINE3
1\SM-\YPST{}-bite-\IC-\PFV{}  5-donkey\\
%%TRANS1
\glt
‘He fed the donkey.’
%%TRANS2
}
%%EXEND

 %%EAX
\ex
%%JUDGEMENT
[]{
%%LABEL
\label{bkm:Ref132275141:b}
%%CONTEXT
%%LINE1
Wana îng’ooí *(n’)áárûmííríe.\\
%%LINE2
\gll
wana  î-ng’ooi  ni  a-ra-rûm-i-ire\\
%%LINE3
even  5-donkey \FOC{}  1\SM-\YPST{}-bite-\IC-\PFV\\
%%TRANS1
\glt
‘Even the donkey he fed.’
%%TRANS2
}
%%EXEND



%%EAX
\ex
%%JUDGEMENT
[*]{
%%LABEL
\label{bkm:Ref132275141:c}
%%CONTEXT
%%LINE1
(Í) t\'{û}agwííré.\\
%%LINE2
\gll
ni  tû-a-gû-ire\\
%%LINE3
\FOC{}  1\PL.\SM-\PST{}-fall-\PFV{}\\
%%TRANS1
\glt
‘We fell.’
%%TRANS2
}
%%EXEND


%%EAX
\ex
%%JUDGEMENT
[]{
%%LABEL
\label{bkm:Ref132275141:d}
%%CONTEXT
%%LINE1
Ndáráthíír’ \'{î}goro.\\
%%LINE2
\gll
nda-ra-thi-ire  î-goro\\
%%LINE3
1\SG.\SM{}-\YPST{}-go-\PFV{} 5-yesterday\\
%%TRANS1
\glt
‘I went yesterday.’
%%TRANS2
}
%%EXEND

\z
\z

This is reminiscent of the conjoint/disjoint alternation, as \citet[728]{AbelsMuriungi2008} note in comparison with Kirundi; see especially also \citet{Morimoto2017} for a comparison of the marker \textit{n\~\i-} in closely-related Gikuyu to the conjoint\slash disjoint alternation.

The second restriction is shown by \citet[80]{Muriungi2005} in his example \xref{bkm:Ref94515962}: \textit{ni} cannot be used on a negative verb form.

\ea
\citep[80, glosses adapted]{Muriungi2005}

%%EAX
\ea
%%JUDGEMENT
[]{
%%LABEL
%%CONTEXT
%%LINE1
%%LINE2
\gll
Paul  n’ á-rá-rúg-íre  n-kíma.\\
%%LINE3
1.Paul \FOC{}  1\SM-\YPST{}-cook-\PFV{} 9-ugali\\
%%TRANS1
\glt
‘Paul cooked food.’
%%TRANS2
}
%%EXEND

%%EAX
\ex
%%JUDGEMENT
[]{
%%LABEL
%%CONTEXT
%%LINE1
%%LINE2
\gll
Paul  a-tí-ra-rúg-a  n-kíma.\\
%%LINE3
1.Paul 1\SM-\NEG-\YPST{}-cook-\FV{} 9-ugali\\
%%TRANS1
\glt
‘Paul did not cook food.’
%%TRANS2
}
%%EXEND


%%EAX
\ex
%%JUDGEMENT
[*]{
%%LABEL
\label{bkm:Ref94515962}
%%CONTEXT
%%LINE1
%%LINE2
\gll
Paul  n’  a-tí/tá-rúg-a  n-kíma.\\
%%LINE3
1.Paul \FOC{}  1\SM-\NEG-\YPST{}-cook-\FV{} 9-ugali\\
%%TRANS1
\glt
‘Paul did not cook food.’
%%TRANS2
}
%%EXEND

\z
\z


The third restriction comes out in a number of tests, showing that when \textit{ni} is absent, the element following the verb is in focus. The form with \textit{ni} cannot be used in an object question \xref{bkm:Ref125199378}, and neither is it felicitous as an answer to an object question, as shown in \xref{bkm:Ref125199394} and \xref{ex:kimakufriend}.

%%EAX
\ea
%%JUDGEMENT
%%LABEL
\label{bkm:Ref125199378}
%%CONTEXT
%%LINE1
(*N)ûroóníre ûû?\\
%%LINE2
\gll
ni  û-ra-on-ire  ûû\\
%%LINE3
\FOC{}  2\SG.\SM-\YPST{}-see-\PFV{} who\\
%%TRANS1
\glt
‘Who did you see?’
%%TRANS2
%%EXEND

\z

%%EAX
\ea
%%JUDGEMENT
%%LABEL
\label{bkm:Ref125199394}
%%CONTEXT
(What did Asha bake?)\\
%%LINE1
Áshá (\textsuperscript{\#}n’) akáándíré mûgááté.\\
%%LINE2
\gll
asha  ni  a-kaand-ire  mû-gaate\\
%%LINE3
1.Asha \FOC{}  1\SM{}-bake-\PFV{} 3-bread\\
%%TRANS1
\glt ‘Asha baked bread.’
%%TRANS2

%%EXEND

\z

%%EAX
\ea
%%JUDGEMENT
%%LABEL
\label{ex:kimakufriend}
%\label{Ref125199394:b}
%%CONTEXT
(Who did you see at the high school reunion?)\\
%%LINE1
(\textsuperscript{\#}I)mbóniré Kîmákû bái.\\
%%LINE2
\gll
ni  n-on-ire  Kîmakû  bai\\
%%LINE3
\FOC{}  1\SG.\SM{}-see-\PFV{} 1.Kîmakû  buddy\\
%%TRANS1
\glt ‘I saw Kîmakû, friend.’
%%TRANS2
%%EXEND

\z

\citet[footnote~13]{Muriungi2005} indicates that a question word preceded by a \textit{ni}-marked verb may be interpreted as an echo question, but we found that the interpretation is more one of emphasis, as indicated in the context and translation of \xref{emphwhat}. The same interpretation can be obtained by the presence of the polarity particle \textit{ka} as in \xref{bkm:Ref94519626}. Echo questions in our data are asked with a \textit{ni}-less verb, a rising intonation, and possibly a modification of the question word by a following -\textit{o} pronoun, as in \xref{bkm:Ref94519675}.

\ea
\label{bkm:Ref94519675}(Context: Someone got a stomach ache after dinner, so what could they have cooked that made this so?)
%%EAX
\ea
%%JUDGEMENT
%%LABEL
%%CONTEXT
%%LINE1
N’ árárugíré \`{m}bí?\label{emphwhat}\\
%%LINE2
\gll
ni  a-ra-rug-ire  m-bi\\
%%LINE3
\FOC{}  1\SM-\YPST{}-cook-\PFV{} 9-what\\
%%TRANS1
\glt
‘What (on earth) did s/he cook?!’
%%TRANS2
%%EXEND

%%EAX
\ex
%%JUDGEMENT
%%LABEL
%%CONTEXT
%%LINE1
Ká árarúgíré \`{m}bí?\label{bkm:Ref94519626}\\
%%LINE2
\gll
ka  a-ra-rug-ire  m-bi\\
%%LINE3
\Q{} 1\SM-\YPST{}-cook-\PFV{} 9-what\\
%%TRANS1
\glt
‘\textit{What} did s/he cook?!’
%%TRANS2
%%EXEND
\z

%%EAX
\ex
%%JUDGEMENT
%%LABEL
\label{bkm:Ref94519598}
%%CONTEXT
(Context: You haven’t heard well what someone said.)\\
%%LINE1
Arárúgire mbíé?\\
%%LINE2
\gll
a-ra-rug-ire  m-bi-e\\
%%LINE3
1\SM-\YPST{}-cook-\PFV{} 9-what-\ECHO{}\\
%%TRANS1
\glt
‘She cooked what?’
%%TRANS2
%%EXEND
\z


Cognate objects, which cannot be focused, can follow a verb with \textit{ni} \xref{bkm:Ref125199609}, but are awkward when \textit{ni} is absent \xref{bkm:Ref94518433}. Since cognate objects cannot be focused (as there are no realistic alternatives), this shows that the constituent following the \textit{ni}-less form is focused. As expected, when specifying the object with a relative clause, it becomes possible to focus the object (as alternatives can be generated outside this subset, here dreams that are not scary) and hence the verb form without \textit{ni} is acceptable – see \xref{bkm:Ref94518409}.

\ea
\label{bkm:Ref125199609}
%%EAX
%%JUDGEMENT
\ea[]{
%%LABEL
%%CONTEXT
%%LINE1
N’ ááróótíré kîróóto.\\
%%LINE2
\gll
ni  a-a-root-ire  kî-root-o\\
%%LINE3
\FOC{} 1\SM-\PST{}-dream-\PFV{} 7-dream-\NMLZ{}\\
%%TRANS1
\glt
‘He dreamt a dream.’
%%TRANS2
}
%%EXEND

%%EAX
\ex
%%JUDGEMENT
[*]{
%%LABEL
\label{bkm:Ref94518433}
%%CONTEXT
%%LINE1
Aróótíré kîróóto.\\
%%LINE2
\gll
a-a-root-ire  kî-root-o\\
%%LINE3
1\SM-\PST{}-dream-\PFV{} 7-dream-\NMLZ{}\\
%%TRANS1
\glt
‘He dreamt a dream.’
%%TRANS2
}
%%EXEND

%%EAX
%%JUDGEMENT
\ex[]{
%%LABEL
\label{bkm:Ref94518409}
%%CONTEXT
%%LINE1
Aaróótíré kîróóto kîámûmakiá mûnó.\\
%%LINE2
\gll
a-a-root-ire  kî-root-o  kî-a-mû-maki-a  mûno\\
%%LINE3
1\SM-\PST{}-dream-\PFV{} 7-dream-\NMLZ{}  7\SM-\PST{}-1\OM{}-scare-\FV{}  \INT{}\\
%%TRANS1
\glt
‘He dreamt a dream that really scared him.’
%%TRANS2
}
%%EXEND

\z
\z

Similarly, an idiomatic object does not keep its idiomatic reading when \textit{ni} is absent \xref{bkm:Ref125199820:b}, as opposed to when it is present \xref{bkm:Ref125199820:a}, again showing the focus status of the element following the \textit{ni}-less verb.

\ea
\label{bkm:Ref125199820}
%%EAX
\ea
%%JUDGEMENT
%%LABEL
\label{bkm:Ref125199820:a}
%%CONTEXT
%%LINE1
N’áátwéére mûtî́.\\
%%LINE2
\gll
ni  a-ra-tûa-ir-e  mû-tî\\
%%LINE3
\FOC{}  1\SM-\YPST{}-climb-\PFV-\FV{} 3-tree\\
%%TRANS1
\glt
‘She climbed a tree’\\
%%TRANS2
‘She became pregnant.’
%%EXEND
\pagebreak
%%EAX
\ex
%%JUDGEMENT
%%LABEL
\label{bkm:Ref125199820:b}
%%CONTEXT
%%LINE1
Aatwééré mût\'{î}.\\
%%LINE2
\gll
a-ra-tw-ire  mû-tî\\
%%LINE3
1\SM-\YPST{}-climb-\PFV{} 3-tree\\
%%TRANS1
\glt
‘She climbed a \textit{tree}.’ (it’s not a rock)\\
%%TRANS2
*`She became pregnant.’
%%EXEND

\z
\z

For alternative questions, where the focus is on the object, the form without \textit{ni} is required, shown in \xref{bkm:Ref125199864:a}. The form with \textit{ni} (and a variation in constituent order) will be interpreted as a yes/no question, as in \xref{bkm:Ref125199864}.

\ea
\label{bkm:Ref125199864}
%%EAX
\ea
%%JUDGEMENT
%%LABEL
\label{bkm:Ref125199864:a}
%%CONTEXT
%%LINE1
Suúzá arúgííré Yénéké kîîkí kaná Ng\'{î}rûbati?\\
%%LINE2
\gll
Souza  a-rug-ire  Jenneke  kîîki  kana  Wilbert\\
%%LINE3
1.Souza 1\SM{}-cook-\PFV{} 1.Jenneke  9.cake  or  1.Wilbert\\
%%TRANS1
\glt
‘Did Souza bake a cake for Jenneke or for Wilbert?’ (answer: Jenneke/Wilbert)
%%TRANS2
%%EXEND


%%EAX
\ex
%%JUDGEMENT
%%LABEL
\label{bkm:Ref125199864:b}
%%CONTEXT
%%LINE1
Suúzá n’árúgiiré Yénéké kaná Ng\'{î}rûbati kîîkí?\\
%%LINE2
\gll
Souza  ni  a-rug-ir-ire  Jenneke  kana  Wilbert  kîîki\\
%%LINE3
1.Souza \FOC{} 1\SM{}-cook-\APPL-\PFV{} 1.Jenneke  or  1.Wilbert  9.cake\\
%%TRANS1
\glt
‘Did Souza bake a cake for Jenneke or for Wilbert (at all)?’ (answer: yes/no)
%%TRANS2
%%EXEND

\z
\z


The interpretation of the postverbal element as focus also emerges clearly when the postverbal object is modified by ‘some’. Following a verb form without \textit{ni}-, this cannot be interpreted as a lower boundary (some, perhaps all), but must be interpreted as a proper subset as in \xref{bkm:Ref125199899:b}, in line with focus triggering and excluding alternatives. In contrast, the form with \textit{ni} in \xref{bkm:Ref125199899:a} was said to ``focus on the fact that they were slaughtered". The continuation clauses in \xref{bkm:Ref125199899:a} (actually all) and \xref{bkm:Ref125199899:b} (leaving others) cannot be interchanged between the two examples.

\ea
\label{bkm:Ref125199899}
%%EAX
\ea
%%JUDGEMENT
%%LABEL
\label{bkm:Ref125199899:a}
%%CONTEXT
%%LINE1
Í baráurágíre ng\'{û}k\'{û} ímwe... wana ícíónthe.\\
%%LINE2
\gll
ni  ba-ra-urag-ire  n-gûkû  i-mwe  wana  ni  ci-onthe\\
%%LINE3
\FOC{} 2\SM-\YPST{}-kill-\PFV{} 10-chicken  10-one  even \FOC{} 10-all\\
%%TRANS1
\glt
‘They slaughtered some chickens... actually all of them.’
%%TRANS2
%%EXEND

%%EAX
\ex
%%JUDGEMENT
%%LABEL
\label{bkm:Ref125199899:b}
%%CONTEXT
%%LINE1
Baráurágíre ng\'{û}k\'{û} ímwe... baatiga irá ing\'{î}.\\
%%LINE2
\gll
ba-ra-urag-ire  n-gûkû  i-mwe  ba-ra-tig-a i-ra  i-ngî\\
%%LINE3
2\SM-\YPST{}-kill-\PFV{} 10-chicken   10-one 2\SM-\YPST{}-stop-\FV{} 10-\RM{} 10-other\\
%%TRANS1
\glt
‘They slaughtered some chickens... they left others.’
%%TRANS2
%%EXEND

\z
\z

From the interpretation of ‘some’, it already becomes clear that the focus following the \textit{ni}-less verb may be exclusive. This conclusion is also supported by the interpretation of the postverbal word ‘person’ in \xref{bkm:Ref125200200}: when \textit{ni} is present, ‘person’ is interpreted as a non-specific indefinite \xref{bkm:Ref125200200:a}, but in the absence of \textit{ni} as a generic ‘human being’ \xref{bkm:Ref125200200:b}, triggering the speakers’ comment “it was not a dog”. If the focus following a \textit{ni}-less verb is exclusive, we can understand the incompatibility with the indefinite non-specific reading, as there are no alternatives: someone/anyone includes all referents. The generic reading, however, can exclude other species.

\ea
\label{bkm:Ref125200200}
%%EAX
\ea
%%JUDGEMENT
%%LABEL
\label{bkm:Ref125200200:a}
%%CONTEXT
%%LINE1
Í ndaróóníre m\'{û}nt\'{û} îgoro.\\
%%LINE2
\gll
ni  nda-ra-on-ire  mû-ntû  î-goro\\
%%LINE3
\FOC{}  1\SG.\SM-\YPST{}-see-\PFV{} 1-person  5-yesterday\\
%%TRANS1
\glt
‘I saw someone yesterday.’
%%TRANS2
%%EXEND

%%EAX
\ex
%%JUDGEMENT
%%LABEL
\label{bkm:Ref125200200:b}
%%CONTEXT
%%LINE1
Ndaróóníre m\'{û}nt\'{û} îgoro.\\
%%LINE2
\gll
nda-ra-on-ire  mû-ntû  î-goro\\
%%LINE3
1\SG.\SM-\YPST{}-see-\PFV{} 1-person  5-yesterday\\
%%TRANS1
\glt
‘I saw a person yesterday.’
%%TRANS2
%%EXEND

\z
\z

However, the unmarked form is also acceptable if the word ‘person’ is modified, inducing a specific meaning, as in \xref{bkm:Ref119741819}.

%%EAX
\ea
%%JUDGEMENT
%%LABEL
\label{bkm:Ref119741819}
%%CONTEXT
%%LINE1
Ndaróóníre m\'{û}nt\'{û} akííyá.\\
%%LINE2
\gll
nda-ra-on-ire   mû-ntû   a-kî-iy-a\\
%%LINE3
1\SG.\SM-\YPST{}-see-\PFV{}   1-person   1\SM-\DEP{}-steal-\FV{}\\
%%TRANS1
\glt
‘I saw someone stealing.’
%%TRANS2
%%EXEND

\z

This effect seems more general: adding further information can improve the \textit{ni}-less form, as seen in \xref{bkm:Ref125200264}.

%%EAX
\ea
%%JUDGEMENT
%%LABEL
\label{bkm:Ref125200264}
%%CONTEXT
%%LINE1
Gûkuíré ûû?\\
%%LINE2
\gll
kû-ku-ire   ûû\\
%%LINE3
17\SM{}-die-\PFV{}   who\\
%%TRANS1
\glt
‘Who died?’
%%TRANS2
%%EXEND

%%EAX
\sn
%%JUDGEMENT
%%LABEL
%%CONTEXT
%%LINE1
Gûkuíré mûká...  \textsuperscript{\#}(na twaáná tw\'{î}îrí).\\
%%LINE2
\gll
kû-ku-ire   mû-ka   na   tû-ana   tû-îri\\
%%LINE3
17\SM{}-die-\PFV{}   1-woman   and   13-child   13-two\\
%%TRANS1
\glt
‘There died a woman... with two children.’
%%TRANS2
%%EXEND

\z

The (im)possibilties of focus-sensitive particles ‘even’ and ‘only’ also support the exhaustive interpretation. If the object following the verb is modified by ‘even’, the form with \textit{ni} is preferred, as in \xref{bkm:Ref125200361:a}, the absence of \textit{ni} is judged to be degraded -- see \xref{bkm:Ref125200361:b} (see also \citealt[714]{AbelsMuriungi2008}). The same holds for the default agreement inversion in \xref{bkm:Ref94543226}.

\ea
\label{bkm:Ref125200361}(Kimathi feeds the goats grass, sometimes the cows, but yesterday…)
\ea
[]{
\label{bkm:Ref125200361:a}
Kîmááthí n’árár\'{û}míiryé kinyá îng’ooí irió.\\
\gll
Kîmathi  ni  a-ra-rûm-i-ire  kinya  î-ng'ooi  i-rio\\
1.Kîmathi \FOC{}  1\SM-\YPST{}-bite-\IC-\PFV{} even  5-donkey  8-food\\%%TRANS1 
\glt
‘Kîmathi fed even the donkeys food.’
}

%%EAX
\ex
%%JUDGEMENT
[\textsuperscript{?}]{
%%LABEL
\label{bkm:Ref125200361:b}
%%CONTEXT
%%LINE1
Kîmááthí árár\'{û}míiryé wana îng’ooí irió.\\
%%LINE2
\gll
Kîmathi  a-ra-rûm-i-ire  wana  î-ngóói  i-rio\\
%%LINE3
1.Kîmathi 1\SM-\YPST{}-bite-\IC-\PFV{} even  5-donkey  8-food\\
%%TRANS1
\glt
‘Kîmathi fed even the donkeys food.’
%%TRANS2
}
%%EXEND

\z
\z

%%EAX
\ea
%%JUDGEMENT
%%LABEL
\label{bkm:Ref94543226}
%%CONTEXT
%%LINE1
*(Í) gwakamatirwe kinya maíga.\\
%%LINE2
\gll
ni  kû-a-kamat-w-ire  kinya  ma-iga\\
%%LINE3
\FOC{}  17\SM-\PST{}-carry-\PASS-\PFV{} even  6-stones\\
%%TRANS1
\glt
‘Even stones were carried away.’
%%TRANS2
%%EXEND

\z


If the object following the verb is modified by ‘only’, the form without \textit{ni} must be used \xref{bkm:Ref125200457}.

%%EAX
\ea
%%JUDGEMENT
%%LABEL
\label{bkm:Ref125200457}
%%CONTEXT
(Did he wash shirts and sheets?)\\
%%LINE1
Árî’, (\textsuperscript{\#}n)áb\'{û}\'{û}rire shááti cíonká.\\
%%LINE2
\gll
arî  a-bûûr-ire  shaati  ci-onka\\
%%LINE3
no 1\SM{}-wash-\PFV{} 10.shirt  10-only\\
%%TRANS1
\glt
‘No, he washed shirts only.’
%%TRANS2
%%EXEND

\z


The same goes for the inversion construction in \xref{bkm:Ref125200487}. If the verb had \textit{ni}-, it was indicated that this cannot be an answer to a question, but (without \textit{wenka} ‘only’) you would need a list: there died a woman, and children, and her husband.

%%EAX
\ea
%%JUDGEMENT
%%LABEL
\label{bkm:Ref125200487}
%%CONTEXT
%%LINE1
(\textsuperscript{\#}) Gûkuíre mûká wéenka.\\
%%LINE2
\gll
ni  kû-ku-ire  mû-ka  we-onka\\
%%LINE3
\FOC{} 17\SM{}-die-\PFV{} 1-woman  1-only\\
%%TRANS1
\glt
‘Only a woman died.’
%%TRANS2
%%EXEND

\z


Apart from providing a range of tests arguing that the preverbal \textit{ni}-marked noun phrase is in exhaustive focus (see \sectref{bkm:Ref117495353}), \citet{AbelsMuriungi2008} also show that the postverbal constituent is in exhaustive focus when the marker \textit{ni} is absent on the verb. One such test is the entailment test in \xref{bkm:Ref94536081}: If Ruth bought a book and a pen, then the statement in \xref{bkm:Ref94537049} without \textit{ni} cannot be an entailment, because (evidently) this means that Ruth bought \textit{only} a book (which is not true). The statement with \textit{ni} in \xref{bkm:Ref94537037}, on the other hand, is acceptable as an entailment.

\ea
\label{bkm:Ref94536081}\citet[713, tones added]{AbelsMuriungi2008}


%%EAX
\ea
%%JUDGEMENT
%%LABEL
%%CONTEXT
%%LINE1
%%LINE2
\gll
Rúth  a-gûr-íre  î-búkú  na  ka-rámu.\\
%%LINE3
1.Ruth 1\SM{}-buy-\PFV{} 5-book  and  12-pen\\
%%TRANS1
\glt
‘Ruth bought a book and a pen.’
%%TRANS2
%%EXEND

%%EAX
\ex
\begin{xlist}
\exi{$\not\Rightarrow$}
%%JUDGEMENT
%%LABEL
\label{bkm:Ref94537049}
%%CONTEXT
%%LINE1
%%LINE2
\gll
Rúth  a-gûr-íre  î-búkú.\\
%%LINE3
1.Ruth  1\SM{}-buy-\PFV{} 5-book\\
%%TRANS1
\glt
‘Ruth bought \textit{a book}.’
%%TRANS2
\end{xlist}
%%EXEND

%%EAX
\ex
\begin{xlist}
\exi{$\Rightarrow$}
%%JUDGEMENT
%%LABEL
\label{bkm:Ref94537037}
%%CONTEXT
%%LINE1
%%LINE2
\gll
Ruth  n’  á-gûr-iré  î-búkú.\\
%%LINE3
1.Ruth \FOC{}  1\SM{}-buy-\PFV{} 5-book\\
%%TRANS1
\glt
‘Ruth bought a book.’
%%TRANS2
\end{xlist}
%%EXEND

\z
\z



If the absence of \textit{ni} indicates (exhaustive) focus on the constituent following the verb, we may wonder whether its presence is also related to a particular interpretation. \citet[706]{AbelsMuriungi2008} indicate a range of contexts in which the verb can take the marker \textit{ni}-, shown in \xref{bkm:Ref94538368}, as long as the focus is not exhaustively on a term.

%%EAX
\ea
%%JUDGEMENT
%%LABEL
\label{bkm:Ref94538368}
%%CONTEXT
\citep[706, modified]{AbelsMuriungi2008}\\
%%LINE1
%%LINE2
\gll
María  n’  á-gûr-iré  î-búkú.\\
%%LINE3
1.Maria \FOC{}  1\SM{}-buy-\PFV{} 5-book\\
%%TRANS1
\glt
‘Maria bought a book.’
%%TRANS2
%%EXEND

as an answer to:\largerpage

\begin{itemize}
\settowidth\jamwidth{non-exh subject focus}
\item What is the problem?
\jambox*{thetic}
\item What did Maria do?
\jambox*{VP focus}
\item It is there anything that Maria bought?
\jambox*{non-exh object focus}
\item Is there anybody who bought a book?
\jambox*{non-exh subject focus}
\item Did Maria buy a book?
\jambox*{polarity focus}
\item What did Maria do with the book?
\jambox*{state-of-affairs focus}
\item What did Maria buy?
\jambox*{object focus}
\item Who bought the book?
\jambox*{subject focus}
\end{itemize}

\z

We also found that \textit{ni} must be present to express predicate-centred focus in SVO order, illustrated by verum in \xref{bkm:Ref125200571}.

%%EAX
\ea
%%JUDGEMENT
%%LABEL
\label{bkm:Ref125200571}
%%CONTEXT
(Daniel didn’t talk yesterday.)\\
%%LINE1
Ndáníérí \textsuperscript{\#}(n’)ááriirie \'{î}góro.\\
%%LINE2
\gll
Daniel  ni  a-ari-ire  î-goro\\
%%LINE3
1.Daniel \FOC{}  1\SM{}-speak-\PFV 5-yesterday\\
%%TRANS1
\glt
‘Daniel did talk yesterday.’
%%TRANS2
%%EXEND

\z


Again, the same we also find in default agreement inversion, where the form with \textit{ni} is interpreted as a thetic sentence, as in \xref{bkm:Ref94543674:a} and \xref{bkm:Ref117537348}, but this reading is impossible for the \textit{ni}-less form in \xref{bkm:Ref94543674:b} (see also \sectref{bkm:Ref94457814} on subject inversion).

\ea
\label{bkm:Ref94543674}
%%EAX
\ea
%%JUDGEMENT
%%LABEL
\label{bkm:Ref94543674:a}
%%CONTEXT
(What is happening down there?)\footnote{This question could also felicitously be answered by a basic cleft/focus construction; see \sectref{bkm:Ref117495353} for use of the preverbal focus in thetic contexts.}\\
%%LINE1
Í \textsuperscript{!}g\'{û}k\`{û}rííngwa mûbíírá kiéníiní.\\
%%LINE2
\gll
ni  kû-kû-riing-w-a  mû-biira  kî-eni=ini\\
%%LINE3
\FOC{} 17\SM-\PRS{}-hit-\PASS-\FV{} 3-ball  7-field=\LOC{}\\
%%TRANS1
\glt
‘Football is being played at the field.’
%%TRANS2
%%EXEND

%%EAX
\ex
%%JUDGEMENT
%%LABEL
\label{bkm:Ref94543674:b}
%%CONTEXT
(Context: You heard the sound of a ball)\\
%%LINE1
Gûkûrííngwa mûbííra kiéníiní.\\
%%LINE2
\gll
kû-kû-riing-w-a  mû-biira  kî-eni=ini\\
%%LINE3
17\SM-\PRS{}-hit-\PASS-\FV{} 3-ball  7-field=\LOC{}\\
%%TRANS1
\glt
*`Football is being played at the field.’\\
%%TRANS2
‘A ball has been hit on the field.’
%%EXEND

\z
%%EAX
\ex
%%JUDGEMENT
%%LABEL
\label{bkm:Ref117537348}
%%CONTEXT
(How is Marimanti?)\footnote{Note that \textit{ni} can be omitted but only in an echo exclamative content, e.g. A: ‘I heard that the sun is really scorching you guys down there!’ B: \textit{Tiga! Kwarîîte mûno}. ‘Tell me about it (lit. leave it)! It’s damn hot!’.}\\
%%LINE1
I kûáárité mûnó.\\
%%LINE2
\gll
ni  kû-ar-ite  mûno\\
%%LINE3
\FOC{}  17\SM{}-be.hot-\STAT.\PFV{} \INT{}\\
%%TRANS1
\glt
‘It’s very hot.’
%%TRANS2
%%EXEND

\z


Considering this highly underspecified use of the presence of \textit{ni} on the verb, and adding also that \textit{ni} must be present on the verb when it occurs in sentence-final position, we conclude that the \textit{ni}-form is the elsewhere form, and the absence of the marker is the marked case, being associated with exhaustive focus on the postverbal constituent.

The distribution and function of \textit{ni} on the verb in Kîîtharaka closely matches that of Kikuyu. For this language, \citet{Morimoto2017} compares the two verb forms to the conjoint\slash disjoint alternation that is found in other eastern Bantu languages (see for example \textcite{chapters/makhuwa} for Makhuwa and \textcite{chapters/kirundi} for Kirundi).

In conclusion, we have seen that the presence or absence of invariable preverbal \textit{ni}- in Kîîtharaka occurs in various tense-aspect categories, and most importantly, whether the verb or following element is in focus. The absence of \textit{ni}- on the verb indicates that focus falls on the postverbal element; its presence is not associated with a particular interpretation, although it is indeed used with predicate-centred focus.

\section{Summary}

We have shown that Kîîtharaka employs a variety of linguistic strategies to structure information. Such strategies are evident in word order, topic marking, predicate doubling, cleft constructions, object/subject marking and drop, and the use or absence of \textit{ni} on the verb. In \sectref{kitwordorder}, we have shown that topicalised constituents are preferred in pre-verbal position while non-topical ones tend to be post-verbal, thus accounting for left dislocation, and subject inversion. There is evidently no dedicated focus position in Kîîtharaka. Possible interpretations for canonical SVO order depend on tense-aspect conjugation of the verb involving the use of the focus particle ni as discussed in \sectref{bkm:Ref111628808}.

In Sections~\ref{bkm:Ref115790636} and~\ref{bkm:Ref115942967}, we have illustrated how various pragmatically sanctioned interpretations such as polarity focus/verum, depreciative, contrastive and intensive readings are variably encoded through the use of the contrastive topic marker \mbox{-\textit{o}} (topic marking) and co-occurrence of the infinitive and inflected form of the verb (predicate doubling). In \sectref{kitcleft} we have also discussed how various interpretations are encoded using the preverbal focus construction (basic cleft), the pseudocleft and the \NI+\RM{} and \NI+\PRO{} constructions (``reverse pseudocleft"). Particularly, we have indicated that the basic cleft encodes exhaustive focus on the sentence-initial constituent. We have also discussed (reverse) pseudoclefts and a similar \NI-\PRO{} structure and concluded that the two are different grammatical structures which can be analysed differently: While in the \NI+\RM{} construction the postcopular part is in focus, for the \NI-\PRO{} structure, we argue that the post-copula pronoun is in focus (which is co-indexed with the initial NP).

Our discussion has also touched on the parameters that regulate the expression of the object, either in full NP, pronoun, object marker or object drop altogether (see \sectref{bkm:Ref115790593}). We have seen that accessibility, humanness, predicate type, and salience play some role, though more work needs to be done on this. Lastly, we have discussed the role of the invariable \textit{ni}- on the verb and showed that its absence denotes (exclusive) focus on the post-verbal constituent. It will be interesting in further research to see the precise differences in meaning and use between the focus expressed after a \textit{ni}-less verb, in a basic cleft, the pseudocleft, and in the \NI-\PRO{} construction.

\section*{Acknowledgements}

This research was supported by NWO Vidi grant 276-78-001 as part of the BaSIS “Bantu Syntax and Information Structure” project at Leiden University. Patrick Kanampiu was at the University of Edinburgh while preparing this chapter. We thank Dennis Muriuki Katheru, Philip Murithi Nyamu, Onesmus Mugambi Kamwara, Jonah Tajiri, Tabitha Giti and Jane Gacheri for sharing their insights on their language with us; we thank two reviewers for comments on an earlier version, and our BaSIS colleagues as well as Dominique Loviscach for their support and insights. Any remaining errors are ours alone.

\section*{Abbreviations and symbols}

Numbers refer to noun classes, unless followed by \SG{}/\PL{}, in which case the number (1 or 2) refers to first or second person. Tone marking indicates surface tone including intonation; high tones are marked by an acute accent; low tones remain unmarked. The mid-close vowel [o] is realised as 〈û〉 while [e] is written as 〈î〉. What is written as 〈c〉 is pronounced [s], and 〈ng’〉 represents [ŋ].

%%% All Leipzig abbreviations are commented out, following the LangSci guidelines of only listing non-Leipzig abbreviations.

\begin{xltabular}{\textwidth}{@{}lQ@{}}
* & ungrammatical\\
\textsuperscript{?} & degraded grammaticality\\
\textsuperscript{\#} & infelicitous in the given context\\
\textsuperscript{!} & downstep\\
*(X) & the presence of X is obligatory and cannot grammatically be omitted\\
(*X) & the presence of X would make the sentence ungrammatical\\
(X) & the presence of X is optional\\
\ABL{} & able\\
% \APPL{} & applicative\\
\CONN{} & connective\\
% \COP{} & copula\\
DAI & default agreement inversion\\
\DEM.\DIST{} & distal demonstrative\\
\DEM.\MED{} & medial demonstrative\\
\DEM.\PROX{} & proximal demonstrative\\
\DEP{} & dependent conjugation\\
DP & determiner phrase\\
\ECHO{} & echo-question\\
\EXH{} & exhaustive\\
% \FOC{} & focus marker\\
% \FUT{} & future tense\\
\FV{} & final vowel\\
\HAB{} & habitual\\
\IC{} & immediate causative\\
\IDEO{} & ideophone\\
\INT{} & intensifier\\
int. & intended meaning\\
% \LOC{} & locative\\
\NA{} + \PRO{} & combination of ‘and’ with pronoun\\
% \NEG{} & negative\\
\NI-\PRO{} & combination of the copula and a pronoun (as in the ``reverse pseudocleft")\\
\NI{}+\RM{} & construction with a copula followed by a free relative clause which starts with a relative marker\\
% \NMLZ{} & nominalizer\\
\N.\PST{} & near past tense\\
\OM{} & object marker\\
% \PASS{} & passive\\
% \PFV{} & perfective\\
% \PL{} & plural\\
\POL{} & polarity\\
\PP{} & pragmatic particle\\
\PRO{} & pronoun\\
% \PRS{} & present tense\\
% \PST{} & past tense\\
QUIS & Questionnaire on Information Structure \citep{SkopeteasEtAl2006}\\
\RM{} & relative marker\\
% \SBJV{} & subjunctive mood\\
\SC{} & short causative\\
% \SG{} & singular\\
SLI & semantic locative inversion\\
\SM{} & subject marker\\
\STAT{} & stative\\
\SUBS{} & subsecutive\\
XP & a phrase headed by an unspecified category X\\
\YPST{} & yesterday past\\
\end{xltabular}

\sloppy\printbibliography[heading=subbibliography,notkeyword=this]
\end{document}
